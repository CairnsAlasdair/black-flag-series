\documentclass[oneside,polish,12pt,sfheadings]{mwbk}
%polonizacja
\usepackage[T1]{fontenc}
\usepackage[polish]{babel}
\usepackage[utf8]{inputenc}
\usepackage{polski} 
\frenchspacing 
\usepackage{indentfirst} 
% polpauza (nie rozpoczyna nowego wiersza)
\def\ppauza{\ts--\hskip.25em}
%koniec polonizacja
%grafika
\usepackage{graphicx}
%pakiet czcionki
\usepackage{times}
%gwiazdki
%\usepackage{heuristica}
\usepackage{lipsum} 

\newcommand{\triast}{\bigskip\par\noindent\parbox{\linewidth}{\centering\large{*}\\[-4pt]{*}\hskip 0.75em{*}}\bigskip\par}%
\newcommand{\threeast}{\bigskip\par\centerline{*\,*\,*}\medskip\par}%
\newcommand{\oneast}{\bigskip\par{\large\centerline{*\medskip}}\par}%

%pdf anonimize
\pdfsuppressptexinfo=-1 %Suppress PTEX.Fullbanner and info of imported PDFs

%pakiet odnośników i~pdf metadata
\usepackage[unicode, pdftex]{hyperref}
\hypersetup{pdfauthor={Ken Macleod},
            pdftitle={Wieża Kosmonauty},
            pdfsubject={Cosmonaut Keep},
            pdfkeywords={tłum. Jacek Hummel, Creative Commmons, tłumaczenie CC BY 4.0, powieść, science fiction},
            pdfcreator={pdfLaTeX}}

\usepackage[a4paper]{geometry}
\geometry{verbose}
\setcounter{secnumdepth}{-1}

\begin{document}
\title{Wieża Kosmonauty}
\author{Ken Macleod}

%-----titlepage start
\DeclareRobustCommand{\cs}[1]{\texttt{\char`\\#1}}
\newlength{\tpheight}\setlength{\tpheight}{0.9\textheight}
\newlength{\txtheight}\setlength{\txtheight}{0.9\tpheight}
\newlength{\tpwidth}\setlength{\tpwidth}{0.9\textwidth}
\newlength{\txtwidth}\setlength{\txtwidth}{0.9\tpwidth}
\newlength{\drop}
\newcommand*{\titleSI}{\begingroup% Sagas
\drop = 0.13\txtheight
\centering
\vspace*{\drop}
{\Huge WIEŻA KOSMONAUTY}\\[1\baselineskip]
{\huge \textsc{Cosmonaut Keep}}\\[2\baselineskip]
{\huge \textsc{Ken Macleod}}\\[4\baselineskip]

{\large Na podstawie wydania TOR, Nowy Jork, 2001 przetłumaczył i~opracował:}\\[1\baselineskip]
{\large \textsc{Jacek Hummel}}\\[3\baselineskip]
{\normalsize \textit{Tłumaczenie jest dostępne na licencji\\
\href{https://creativecommons.org/licenses/by/4.0/deed.pl}{Creative Commons Uznanie autorstwa 4.0 Międzynarodowe}}\par}

\vfill
{\Large {Warszawa, 2021}}\\
\vspace*{\drop}
\endgroup}
\titleSI
\thispagestyle{empty}
\newpage
%-----titlepage end
\begin{figure}[ht!]
\centering
\includegraphics[width=0.55\paperwidth]{cosmonautKeepCover.jpeg}
\end{figure}


\newpage


Dla Iaina

Niektóre idee w~tej książce były zainspirowane przez stronę Chrisa
Boyce'a\footnote{podany oryginalnie adres \href{http://www.et-presence.ndirect.co.uk}{http://www.et-presence.ndirect.co.uk} jest niedostępny, ostatnia wersja jest dostępna tylko w~archiwum \href{https://web.archive.org/web/20020915000000*/http://www.et-presence.ndirect.co.uk}{Wayback Machine} - przyp.tłum.}.

Wcześniejsza wersja rozdziału drugiego była opublikowana jako
opowiadanie\\ w~\emph{IT@2000}, a~dodatku specjalnym do \emph{Computer
Weekly} z~25 listopada 1999 roku.

Dziękuję Carol, Sharon i~Michaelowi za wszystko, Timowi Holmanowi za
redaktorską cierpliwość i~rozwiązanie problemów z~logiką wątków, Timowi
Adye za spekulacje w~fizyce, Farah Mendlesohn za przeczytanie i~skomentowanie szkicu, Ellis Sharp od której skradłem nazwę statku i~Stacji Biologii Morskiej na Wyspie Cumbrae za szczęśliwy i~pracowity
tydzień dawno temu.

I jednemu z~drzew wodza lub słupów po prawej stronie wejścia do barki,
półtora metra nad ziemią wspaniałymi dużymi literami było wyryte CROATOAN
bez żadnego krzyża czy znaku niedoli.


\chapter{Prolog}


CIEBIE TUTAJ NIE MA. Spróbuj to zapamiętać. Spróbuj zapomnieć, gdzie
jesteś naprawdę.

Znajdujesz się w~pokręconym labiryncie takich samych, ciemnych
korytarzy. Zsuwasz się ostatnim z~nich tak gładko jak tłoczek w~strzykawce i~jesteś wyrzucony w~nagle oszałamiającą, otwartą przestrzeń
wnętrza. Minuty temu, widziałeś kosmos, wszechświat i~to wszystko nie
wyglądało na większe niż to tutaj. Przestrzeń kosmiczna jest,
zasadniczo, znajoma. To tylko nocne niebo, ale bez ziemi pod stopami.

To miejsce jest zasadniczo nieznajome. Trzydzieści kilometrów wzdłuż i~dziesięć wzwyż, większe niż cokolwiek, kiedykolwiek widziałaś. To pokój
ze światem wewnątrz.

Dla nich, to jasny świat. Dla nas, to ciemna, zimna grota. Dla nich,
nasze najdelikatniejsze sondy przypominałyby gigantyczne statki
kosmiczne wiszące na dyszach rakiet ponad miastami, świecące po
wszystkim nieznośnie jasnymi reflektorami. Dlatego spoglądamy ich
oczami, ich narzędziami, w~ich kolorach. Tłumaczenie kolorów ma więcej
wspólnego z~emocjami niż widmem elektromagnetycznym. Dużo namysłu,
naszego i~ich, zostało zawartych w~tej interpretacji.

Zatem to, co widzisz to ciepłe, bogate zielone tło, upstrzone
niezliczonymi małymi, żwawymi kształtami w~znacznie większej liczbie
kolorów, niż potrafiłbyś nazwać. Myślisz o~klejnotach, kolibrach i~rybach tropikalnych. W~istocie, porównanie do lasu deszczowego lub rafy
koralowej jest bardzo bliskie. To jest ekosystem znacznie bardziej
złożony niż całej Ziemi. Gdy punkt widzenia zbliża się do powierzchni,
rozpoznajesz obrazy miast z~lotu ptaka lub wzorce obwodów
elektronicznych. To, także, jest trafne: rozróżnienie pomiędzy
naturalnym i~sztucznym jest tu bez znaczenia.

Punkt widzenia przybliża się i~oddala: od fraktalnych płatków śniegu
zabarwionych na tęczowo, jak w~kalejdoskopie, do ogromnych odległości
zamglonych na fioletowo i~perspektyw habitatu, przedstawia mnogość i~różnorodność miejsca, brak powtarzalności. Wszystko jest unikalne.
Istnieją podobieństwa, ale nie gatunki.

Nie możesz tego wyłączyć. Po cichu, nieustępliwie, punkt widzenia
pokazuje Ci więcej i~więcej, aż nieludzkie, ale zniewalające piękno
ogrodu obcych lub miasta lub maszyny lub umysłu zagarnie Twoje serce.
Nie uwolni Cię, póki go nie pobłogosławisz. Wtedy, jak tylko w~tym się
beznadziejnie zakochasz, wyrzuci Cię z~powrotem do ludzkości, w~ciemność.

\chapter[Statek przybywa]{1 Statek przybywa}

Bóg wisiał wysoko na niebie ponad horyzontem, w~zachodzącym słońcu, jego
długie białe włosy unosiły się na słonecznym wietrze. Później, gdy kolor
nieba zmieniłby się z~zielonego do czarnego, biała łuna sięgnęłaby
prawie zenitu, jego światło przyćmiłoby Spieniony Kilwater, szeroki pas
galaktyki. Przynajmniej tak by było, gdyby ustąpiły do tego czasu chmury
szkwałowe pędzące znad lądu na wschód. Gregor Cairns odwrócił się tyłem do
kilwateru \emph{C.M. Yonge}\footnote{prawdopodobnie Charlotte Mary Yonge (1823–1901) angielska pisarka, która pisała na potrzeby Kościoła Angielskiego, jej książki poruszały sprawy zdrowia publicznego i~higieny, zob.~\url{https://en.wikipedia.org/wiki/Charlotte_Mary_Yonge} - przyp.tłum.} i~spojrzał mimo masztów i~żagli na niebo.
Chmury były coraz ciemniejsze i~coraz bliżej, gdy ostatnim razem
spojrzał, kilka minut wcześniej. Dwóch z~pięcioosobowej załogi lugra już
obracało wielki żagiel, przygotowując się do halsu pod wzmagający się
wiatr.

Choć chciałby pomóc, wiedział z~doświadczenia, że tylko by przeszkadzał.
Z powrotem skupił się na zbiornikach i~sieciach w~których dzienny połów
kłapał, plaskał lub wił się. W~większości trylobity i~bezszczękowce wraz
z srebrno-połyskującymi rybami doskonałokostnymi, śluzowatymi ślimakami
morskimi, oraz skorupiaste grona opancerzonych mięczaków i~koralowców\footnote{odp. trylobity - gromada wymarłych
morskich stawonogów o~owalnym i~spłaszczonym grzbietobrzusznie ciele,
por.~\url{https://pl.wikipedia.org/wiki/Trylobity}, \\ bezszczękowce - takson
prymitywnych, pozbawionych szczęk kręgowców wodnych, por.~\url{https://pl.wikipedia.org/wiki/Bez\%C5\%BCuchwowce}, \\ ryby
doskonałokostne - największa podgromada w~gromadzie ryb
promieniopłetwych, obok przejściowców i~parafiletycznych
kostno-chrzęstnych, por.~\url{https://pl.wikipedia.org/wiki/Doskona\%C5\%82okostne} - przyp.tłum.}.
Dla Gregora taki zbiór zaczynał wyglądać niedorzecznie i~anachronicznie.
Uśmiechnął się na myśl, że wiedział więcej o~życiu morskim oceanów Ziemi
niż o~planecie, której dawno temu pierwsi ludzcy osadnicy nadali miano
Mingulay\footnote{Mingulay - także druga największa
wyspa z~Wysp Barra w~Zewnętrznych Hebrydach - przyp.tłum.}.

Jego wymuszony uśmiech został zauważony przez kolegów, jeden z~nich
odwzajemnił uśmiech. Elizabeth Harkness była grubokościstą, mocno
zbudowaną młodą kobietą, prawie w~jego wieku i~centymetr lub dwa wyższą.
Pod wielkim skórzanym kapeluszem jej niedbale przycięte czarne włosy
były zdmuchiwane na rumiane policzki. Tak jak Gregor miała na sobie
ciężki sweter, sztormiak, gumowe buty i~rękawice. Przykucnęła kilka
metrów dalej na obładowanej rufie, sondując sploty niby-korzenia, fachowo
wrzucając do właściwych zbiorników oddzielone małże, korale i~resztki\footnote{niby-korzeń - struktura podobna do
korzenia, która kotwiczy wodne organizmy takie jak glony, kelp, gąbki i~inne do podłoża - przyp.tłum.}.

-- No chodź -- powiedziała -- robota czeka.

-- Aye -- odpowiedział Gregor, pochylony, by ostrożnie przenieść
dziesięciokilogramowego trylobita, kłapiącego i~poruszającego odnóżami,
do drewnianego koryta z~wodą. -- Im szybciej to posortujemy, tym więcej
czasu na drinki w~porcie.

-- Ta, ale nie ograniczaj się do łatwych rzeczy. -- Rzuciła nadmiarowe
małże morskim nietoperzom, które skrzeczały i~kołowały nad łodzią.

-- Ha -- chrząknął Gregor i~zostawił dość chropowate trylobity, by same
sobie radziły w~sieciach i~koszyczkach, i~zabrał się energicznie do
rozdzielania drobnej fauny skorupiaków. Statek kołysał się, chlupocząc
morską wodą z~koryt i~zbiorników, a~potem słodka woda z~nieba zalała
pokład, gdy napotkali szkwał. On i~Elizabeth pracowali dalej, krzycząc i~śmiejąc się, gdy sortowanie przy ich pośpiechu stawało się coraz
łatwiejsze.

-- Tak długo, aż się wzajemnie nie \emph{zjedzą}\ldots

Trzeci student na łodzi kucał naprzeciwko obojga ludzi, kolana na
wysokości szerokich policzków, nieświadom deszczu bombardującego jego
bezwłosą głowę oraz strumyczków, które spływały z~karku ponad bezszwowym
kołnierzykiem jego matowego szarego kombinezonu izolującego. Błony
migawkowe jego wielkich, czarnych oczu i~sporadyczne parsknięcie małymi
nozdrzami, czy splunięcie z~cienkich, wąskich ust były jedynymi
wskazówkami, że ulewa w~ogóle na niego wpływa. Jego ręce miały po trzy
długie palce i~jeden długi kciuk. Każdy palec miał na końcu pazur, który
sprawiał, że nóż, przynajmniej przy tym zadaniu, był całkiem
niepotrzebny.

Gregor dyskretnie na niego spoglądał, podziwiając niemal automatyczną
łatwość z~którą długie palce sortowały stosy, sploty przed nimi,
starannie podzielone kolumny za nimi, rzeźnicza siła, chirurgiczna
precyzja i~kliniczna czułość kciuka, pazura i~dłoni. Wtedy, w~odpowiedzi
na jakąś ścisłą intuicję, zaur zakołysał się na stopach, umył ręce w~ostatnich kroplach deszczu i~wstał ze swoją częścią skończoną.

Elizabeth i~Gregor spojrzeli na siebie ponad zmniejszającym się pokładem,
na którym nie zostało już nic, prócz plam i~strzępów. Elizabeth zamrugała
mokrymi rzęsami.

-- Zrobione -- powiedziała, wstając i~strząsając deszcz z~kapelusza.

-- Super. -- Gregor podciągnął się do pionu i~też otrząsnął się z~deszczu,
dołączając do pozostałej dwójki przy rufowym relingu. Oparli się o~to,
patrząc na czerwieniejące niebo, na którym bóg błyszczał jaśniej. Chmury
najwyżej na niebie, znacznie wyżej niż chmury szkwału, płonęły
szczególnym zjawiskiem tęczy niczym kolory masy perłowej, rzadki
fenomen, tak że nawet żeglarze mruczeli w~zdumieniu.

Za nimi wielki żagiel zjechał grzechocząc, silnik zakasłał wracając do
życia, gdy sternik zaczął ich wprowadzać do portu. Stumetrowe klify
cypla, ukoronowane skalistym zamkiem, Wieżą Aird, wyrosły na
bakburcie\footnote{Aird -- także obszar w~hrabstwie
Inverness na zachód od miasta Inverness - przyp.tłum.}. Niskie zielone
wzgórza i~pola rozciągały się na sterburcie. Przed nimi pojawiały się
światła w~Kyohvic, głównego portu wybujałej republiki morskiej znanej
jako Herezjarchia Tain\footnote{Tain -- także miasto w~Highlands w~Szkocji - przyp.tłum.}.

-- Dobra robota, Salasso -- powiedział Gregor. Zaur odwrócił się i~poważnie pokiwał głową, jego nozdrza i~usta ledwo drgnęły jako
równoważnik uśmiechu u jego gatunku. Potem wielkie, czarne oczy, ich
krawędzie widoczne w~profilu, znowu zaczęły przyglądać się morzu.

Długie ramię i~długi palec Salassa wskazały.

-- \emph{Teuthys} -- syknął Salasso.

-- Gdzie? -- krzyknęła zachwycona Elizabeth. Gregor przysłonił oczy i~spojrzał się wzdłuż białego kilwateru i~ciemnych fal, tak wiele ich było,
póki nie ujrzał rosnącej ciemniejszej sylwetki, wybrzuszającej wodę
około mili dalej. Przez chwilę, tak pozostawała, wysepka w~głębinach.

-- To może być wieloryb\ldots -- wymamrotał.

-- \emph{Teuthys }-- upierał się zaur.

Garb opadł i~nagle ogromny kształt wyskoczył ponad powierzchnię,
wznosząc się najwyraźniej niemożliwym łukiem na krótkim białym
strumieniu, przelotny widok rozlanych macek za czarnym klinem tego
czegoś, a~potem wielki plusk, gdy wpadło znowu w~wodę. Zrobiło to znowu,
a tym razem nie było czarne, w~trakcie sekundy lotu świeciło i~błyskało
migocącymi kolorami. I~nie było same, drugi kraken dołączył. Skakały
razem, znowu i~potem znowu, skręcając i~się ścigając. Ostatni
zsynchronizowany skok, który trwał dwie sekundy, oraz wielokolorowy
błysk, który rozświetlił wodę jak sztuczne ognie, zakończyły pokaz.

-- Och, bogowie na niebie -- westchnęła Elizabeth. Usta zaura były
wykrzywione w~małe czarne O, a~jego ciało drżało. Gregor gapił się na
miejsce, gdzie bawiły się krakeny, przerażony, ale i~pełny zdumienia. Że
się bawiły, to był pewien, nawet nie wiedząc dlaczego. Niektóre teorie
twierdziły, że takie nieuzasadnione wydatki energii przez krakeny były
jakimś rodzajem gry godowej, a~nawet rytualnej, ale jak większość
biologów, Gregor traktował takie hipotezy jako niewarte rozważenia.

-- \emph{Architeuthys extraterrestris sapiens }-- powiedział wolno. -
Panowie galaktyki. Bawiący się.

Czarny język zaura zatrzepotał, a~potem jego usta znów zmieniły się w~cienką linię.

-- Tego nie wiemy -- odpowiedział, jego słowa może nawet donioślejsze dla
Gregora, niż zamierzał. Ale mężczyzna zdecydował potraktować je lekko,
wychylając się i~uśmiechając się z~bólem i~bezradnie do kobiety.

-- Nie wiemy -- zgodził się Gregor -- ale pewnego dnia się dowiemy. --
Zerknął do góry, gdzie błyski białego rozciągały się na niebie. -- Nawet
bogowie się bawią, jestem o~tym przekonany. Dlaczego by chcieli zostawić
ich\ldots nieskończony spokój między gwiazdami i~nurkować pomiędzy naszymi
światami i~bujać się dookoła słońca?

Wydawało się, że szyja Salassa lekko się poruszyła. Odwrócił wzrok od
nieba, znów drżąc. Elizabeth się zaśmiała, nie zauważając, a~może nie
odczytując delikatnego języka ciała zaurów. 

-- Bogowie na niebie, umiesz,
człowieku, gadać! -- powiedziała Elizabeth. -- Myślisz, że kiedyś się
dowiemy?

-- Tak, tak myślę -- odpowiedział Gregor. -- To \emph{nasza} zabawa.

-- Mów za siebie, Cairns, wiem, jak się bawić po długim ciężkim dniu
pracy i~-- spojrzała szybko przez ramię -- za dziesięć minut zaczynam,
dzisiaj od ciężkiego drinka!

Gregor wzruszył ramionami i~się uśmiechnął, wszyscy się rozluźnili,
patrząc na morze i~rozmawiając. Gdy pierwsze domy miasteczka nad zatoką
przemknęły obok, jeden z~załogantów zaskoczył ich długim, dźwięcznym
okrzykiem:

-- Statek się zbliża!

Wszyscy na łodzi spojrzeli w~niebo.

\threeast

James Cairns stał, skulony w~futrzanym płaszczu, na starożytnych
blankach Zamku i~patrzył na Statek, jak przesuwa się po niebie ze
wschodu, żarzący się zeppelin długi na co najmniej trzysta metrów.
Nadszedł wzdłuż ciemnej doliny, oświetlając zbocza wzgórza, i~ponad
skupiskami domów miasta, jego kurs tak pewny i~stały jak w~kolejce
jednoszynowej. Gdy mijał niemal nad głową na tysiącu metrów, Cairns
przez chwilę był rozbawiony, że pomiędzy wzorami wybranymi do
wyświetlania na bokach były falisty podpis Coca-Coli, podwójne złote
łuki \emph{M}, dzielne kratkowane logo Microsoft, Gwiazdy i~Pasy flagi
Stanów Zjednoczonych oraz flaga Unii Europejskiej, trzynaście gwiazd,
dwanaście małych żółtych gwiazd i~jedna centralna czerwona gwiazda na
niebieskim tle.

Przypuszczał, że celem takiej manifestacji było uspokajanie. U niego
spowodowało to, i~nie miał wątpliwości, że także u innych obserwatorów,
ostry ból dumy i~tęsknoty tak silny, że świecący kształt rozmazał się na
chwilę. Starzec mrugnął i~pociągnął nosem, patrząc za statkiem, którego
droga nieubłaganie schodziła nad morze. Kiedy był około kilometra wgłąb
morza, i~sto metrów nad wodą, seria srebrnych obiektów w~kształcie
soczewek odeszła z~jego boków, wirując, a~potem kierując się z~powrotem
śladami statku. Wchodziły do portu, gdy kadłub długiego statku
delikatnie dotknął fal i~znieruchomiał, jego błyskające światła zmieniły
czarną wodę w~kalejdoskop tęczy. Inne światła, podwodne i~znacznie
mniejsze, ale prawie wcale nie jaśniejsze, dołączyły do statku w~kolorowej śnieżycy.

Cairns przeniósł wzrok ze statku na skify grawitacyjne. Niektóre opadały
na lądowiska w~dokach poniżej, większość potoczyła się ponad i~spływała,
bujając się jak opadające liście, na trawiasty grzbiet długiego wzgórza,
która opadała od strony lądu zamku. James przeszedł na drugą stronę
dachu, żeby popatrzeć. Gdzieś pod jego stopami, zamruczał zapasowy
generator. Zaświeciły się reflektory, podświetlając podejście i~
odbijając się od stalowoszarych boków skifów.

Niemal banalnie po takim brawurowym przylocie, około tuzina skifów
wystawiło i~opadło na teleskopowo wysunięte nogi. Włazy na spodzie
otworzyły się i~pojawiły się schody, po których ludzie i~zaury schodzili
tak zwyczajnie jak pasażerowie w~samolocie. Każda łódź wypuściła dwóch
lub trzech zaurów, dwukrotnie lub trzykrotnie tylu ludzi. Razem około
setki powoli szło w~górę stoku i~na łagodniejszą trawę trawników zamku,
przemierzając je, żeby zostać przywitanymi i~żeby dołączyć do
mieszkańców Zamku. Zaury w~szarych kombinezonach wyglądali na bardziej
wystrojonych niż ludzie, większość z~nich w~kaloszach, sztormiakach, do
szczętu przemoczonych. Ludzie z~tyłu ciągnęli małe wózki na kółkach
wyładowane bagażem.

Poczuł ciepłe ramię wsuwające się przez boczne wcięcie jego płaszcza i~
obejmujące go w~pasie.

-- Nie schodzisz? -- spytała Margaret.

Cairns odwrócił się i~spojrzał w~oczy żony, które błyszczały obramowane
kurzymi łapkami, gdy się uśmiechała i~położył prawe ramię, nagle
ciężkie, na jej ramionach.

-- Za chwilę -- odpowiedział. Westchnął. -- Wiesz, nawet po tylu latach,
ten widok ciągle u mnie powoduje największe zawroty głowy.

Margaret zachichotała mrocznie. 

-- Ta, wiem. Też tak mam.

Cairns wiedział, że gdyby rozwodził się nad osobliwością widoku,
poczucie nierzeczywistości mogłoby przyprawić go o~fizyczne mdłości:
\emph{la nausée}, stara egzystencjalna niepewność Sartre'a, Cairns nie
pierwszy raz zastanawiał się, jak filozof poradziłby sobie z~sytuacją,
która była tak metafizycznie niepokojąca jak ta.

\emph{L'enfer, c'est les autres. }

Odwrócił się stanowczo, zabierając ze sobą Margaret, i~razem zaczęli
schodzić ze szczytu Zamku, żeby z~uśmiechem przywitać burżujów. Pod
lewym łokciem trzymał zwiniętą i~złożoną flagę, sztandar z~kręgiem
gwiazd, który opuścił, jak to miał w~zwyczaju, o~zachodzie słońca. Za
nim, stalowa lina uderzała w~maszt, pusta wobec wietrznej nocy.

Schodzili spiralą schodów, stopnie szerokie na półtora metra i~wysokie
na około trzydzieści centymetrów, przez tysiąclecia każda stopnica
została zużyta w~przerażająco głęboką krzywą dzwonową, jak gdyby kamień
sam się uginał. Żelazna poręcz dookoła centralnej rury miała tylko kilka
wieków, i~była na wysokości właściwej dla człowieka. Elektryczne
światło, choć niewyraźne, zostało dostosowane do ludzkich oczu.

James i~Margaret trzymali się blisko ściany, gdy schodzili. Margaret
szła pierwsza, łomocąc i~rozmawiając radośnie. James podążał za nią,
ledwie słuchając, jego uwaga była zwrócona na liczne skamieniałości
osadzone w~kamieniach wewnętrznej okładziny ściany, niektóre z~nich
zostały wypolerowane aż do połysku mahoniu przez ciekawe lub pełne
szacunku palce pokoleń kolejnych gatunków mieszkańców Zamku. Przesunął
palcami przez częściowe szczątki ryby i~smoków i~morskich potworów i~
innych organizmów w~dziwnym starożytnym potopowym ulepku, którego
porządek nie miał nic do czynienia z~następstwem ewolucyjnym. Jak
zawsze, gdy wspinał lub schodził tymi schodami, przypominał sobie
powiedzenie, z~którego korzystał przy dzieciach i~wnukach: ten Zamek
został zbudowany przez gigantów, wydobyty przez krasnoludy, zdobyty
przez gobliny i~pozostawiony duchom na długo, zanim ludzie z~Ziemi w~
ogóle postawili kamień jeden na drugim.

Dźwięki i~zapachy odbijające lub napływające z~dołu wzmacniały się, gdy
stara para schodziła. Przybycie statku handlowego mogło lub nie mogło
być zaplanowane, ale załoga Wieży rutynowo ćwiczyła na wszelki wypadek.
Pierwszego wieczoru goście oczekiwali jedynie gorącej wody, gorącego
jedzenia, dużo alkoholu i~jakiegoś łóżka, żeby się na nie potem stoczyć:
handlarze prosto ze statku zwykle nie mieli sił na formalne negocjacje
lub świętowanie. Zaury potrzebowały jeszcze mniej.

Mijali wyjścia z~kręconych schodów, ich liczba tak ustalona w~umyśle
Jamesa jak cyfry na wyświetlaczu windy. Razem z~Margaret wyszli na
parterze, same schody prowadziły jeszcze do wielu poziomów poniżej aż do
skały, i~przeszli przez kilka \emph{zygzaków} wąskich korytarzy
obronnych. Antyczne skafandry stały w~pomysłowo zbudowanych wnękach
przeznaczonych na zasadzki.

Korytarz prowadził do głównej sali zamku, przepastnej przestrzeni
wyposażonej w~zmodernizowane światło elektryczne, ściany wysokie na
piętnaście metrów wyłożono arrasami i~gobelinami, obrazami olejnymi
członków rodzin Kosmonautów, głowami i~skórami dinozaurów i~dekoracyjnie
ułożonymi wystawami lekkiej artylerii, dzięki której te gigantyczne
ofiary zostały sportowo ubite.

Szerokie wrota stały otwarte. Płonący ogień w~sali i~bardziej praktyczne
grzejniki elektryczne niewiele pomagały, by odgonić chłodne wpadające
fale wieczornego powietrza. Kupcy, towarzyszący im zaurzy oraz ich
służący już zmieszali się z~powitalnym tłumem, który zgromadził się z~wszystkich komnat Zamku. Przemieszani, ale łatwi do rozróżnienia,
ponieważ tylko tego wieczoru mieszkańcy zamku, Kosmonauci, zarządcy,
seneszalowie i~służący prześcigali kupców w~stylu i~wyglądzie ich szat.
W nadchodzących dniach, najstarsi z~Kosmonautów i~najbogatsi z~kupców w~Kyohvic zostaną łatwo przyćmieni przez najmłodsze dzieci ich gości, czy
ich najniższych służących. Ich obecny plebejski wygląd, choć częściowo
wynikał z~konieczności podróży międzygwiezdnej, był, przez oczywistą
swobodę, częścią protokołu, stawianiem siebie wyraźnie poniżej
gospodarzy, którego niezmienność James Cairns wielokrotnie wcześniej
obserwował.

Teraz nowo przybyli porzucali ochronne ubrania w~niedbałym stosie w~szerokim przedsionku przy wejściu i~krążyli wokół w~wełnianych
skarpetach i~podobnie ciepłych koszulach, podając sobie dłonie lub
całując dłonie, uśmiechając się, śmiejąc się i~poklepując się po
ramionach. Dzieci gnały i~dawały drapaka, gonione przez swoich
służących, grzecznie przekierowywane od wielu wyjść wielkiej komnaty
przez prędko ustawionych kelnerów. W~tym wszystkim zaury dryfowały, ich
okrągłe głowy pojawiały się w~tłumie niczym zabłąkane balony.

Hal Driver, Strażnik, stał w~centrum napierającego tłumu, już głęboko
zaangażowany w~serdecznej rozmowie z~dojrzałym, krzepkim mężczyzną,
który mógłby mieć napis ,,książę handlu'' na sobie, choć był ubrany jak
rybak. Czerwone włosy stały w~szoku na głowie. Piegi pokrywały jego
szerokie policzki i~płaski nos. Jego głęboki głos grzmiał ponad szmerem,
od czasu do czasu na stronie Drivera zmieniając się w~poufne mruczenie.

Margaret szturchnęła Jamesa i~sekretnie wymruczała.

-- Nie zabrało im dużo czasu, żeby dojść, kto tutaj rządzi.

\threeast

-- Nie idziesz do Wieży? -- spytała Elizabeth.

Gregor skończył spłukiwać łuski i~śluz ze swojego sztormiaka i~powiesił
go w~szafce. 

-- Ee tam -- odpowiedział zdejmując buty. -- Będzie dużo
czasu, gdy w~końcu zrobią imprezę. Ale to jeszcze dzień lub dwa. --
Usiadł na niskiej składanej ławce, zdjął z~trudem grube skarpetki i~wepchnął je do butów, potem ostrożnie włożył stopy w~skórzane buty. --
Masz ochotę wpaść?

Elizabeth się zaczerwieniła. 

-- Och, to miłe, dzięki, ale nie wiem, czy
mogę.

-- Cóż, zaproszenie jest aktualne -- powiedział Gregor, nieświadomy jej
chwilowego zażenowania, i~odwrócił się do zaura. -- A~Ty?

Odbicia powoli kołyszących się lamp elektrycznych tworzyły małe łuki w~czarnych oczach Salassa, gdy zaur cierpliwie czekał w~niskim, wąskim
wejściu przerobionej komory. Jego ubrania nie potrzebowały ani
czyszczenia ani poprawek. Ani, Gregor nagle zrozumiał rumieniąc się, nie
potrzebowałby prawdopodobnie drogich wersji na spotkania towarzyskie.

Schylił się, żeby zawiązać buty, gdy Salasso powiedział, że na pewno
przyjdzie.

-- Nie martw się, hm, strojem -- zawołał za Elizabeth, gdy wychodziła na
pokład. -- Wiesz, jacy są kupcy, i~tak by nie wiedzieli, co jest tutaj
modne.

-- Pomyślę o~tym -- odpowiedziała, nie oglądając się.

Na pokładzie troje studentów podziękowało, Renwickowi, szyprowi, który
po raz ostatni sprawdzał łódź przed nocą. Okazy, bezpieczne w~różnych
pojemnikach, miały czekać do ranka, kiedy zostaną zawiezione do Stacji
Biologii Morza.

-- Kufelek w~Bailie's? -- Elizabeth spytała się Renwicka.

Szyper potrząsnął głową. 

-- Nie, umówiłem się na piwko z~załogą.
Widziałem jak idą do Shipwright, chyba. Do jutra, ludziska.

Zatoka była tak stara, że mogła być naturalna, ale była zaledwie
przed-ludzka. Półkilometrowe molo było zbudowane z~tych samych
metamorficznych skał co zewnętrzne ściany zamku i~przypuszczalnie
zostało wyniesione z~basenu zatoki w~czasach starożytnych, albo przy
pomocy brutalnej siły i~prymitywnej pomysłowości wielu, albo przy pomocy
lancy laserowych i~sani grawitacyjnych gości z~kosmosu. Falochron
tworzył zakrzywione ramię obejmujące zatokę, drugie ramię zapewniały
klify przylądka. Poza nim, skaliste odkrywki wydobyły długie białe i~piaszczyste plaże, które w~oddali zamieniały się w~wydmy pokryte
nieregularnymi kępkami traw.

Gregor wspiął się po zardzewiałych szczeblach drabiny na molo. Kilkaset
metrów w~linii prostej nad wodą, w~głównych dokach zatoki, mała grupka
ludzi i~zaurów zebrała się dookoła lądujących skifów. Gregor spojrzał na
nich obojętnie, a~potem zaczął patrzeć w~niebo czekając na pozostałych.

Tęczowe chmury się rozproszyły. Gabriel, gwiazda wieczorna i~poranna,
świeciła się jak lampa nisko na zachodzie, jego blask przyćmiony przez
boga stojącego wyżej na niebie i~przez świecenie zorzowe statku
kosmicznego unoszącego się na morzu pod nimi. Wysoko ponad nimi
błyszczał lodowato Raphael i~jako mała iskierka, jego księżyc Ariel. A~dalej, podążając ekliptyką w~kierunku wschodnim, jasny sierp nowego
księżyca, poza którym Gog i~Magog, gazowe giganty z~własnymi
pierścieniami jarzyły się niczym oczy potwora.

Przemierzając niebo z~północy na południe, samotny na orbicie polarnej,
opuszczony dwieście lat temu, leciał stary statek z~domu. Dla Gregora,
śledzącego go nagle mrugającym okiem, \emph{Jasna Gwiazda} była
dziwniejszym i~sugestywniejszym widokiem niż wszystkie znane
konstelacje, Muszkieter, Kałamarnica, Skrzydło Anioła i~inne, które były
widoczne na niebie po obu stronach Spienionego Kilwateru.

I, ironicznie, trudniejszym dla człowieka do zdobycia.

Dwoje ludzi i~zaur raźnie szli wzdłuż mola do nabrzeżnych ulic Kyohvic,
kierując się w~prawo wzdłuż oświetlonej, brukowanej promenady do Baru
Bailie's, którego znakiem był wesoły, żłopiący hrabia w~kapeluszu z~pióropuszem i~koronkowym krawacie. Wewnątrz, długie, niskie, tynkowane
ściany były udekorowane naiwnymi muralami, półkami i~podpórkami
podpierającymi harpuny i~wypchane ichtiozaury. Połowa stołów i~baru była
zajęta mężczyznami tuż po pracy na łodziach i~statkach. To było tego
typu miejsce, o~tej porze wieczoru. Później, trociny zostałyby
wymienione, stoły przetarte i~czystszy tłum wpłynąłby po
przedstawieniach lub restauracjach. Ale teraz śmierdziało potem i~rybami, drożdżami, tytoniem i~konopiami, przyćmione światło odbijało się
refleksami od szklanek i~kufli oraz od szklistego, zmrużonego wzroku
mężczyzn relaksujących się w~zamyśleniu nad długimi fajkami. Stali
bywalcy przywitali studentów kiwnięciem głowy i~uśmiechem. Inni,
marynarze z~innego portu, wzdrygnęli na widok zaura. Gdy Gregor machnął
na Elizabeth i~Salasso, żeby usiedli i~szedł do baru, usłyszał syczące
mamrotanie o~,,wężach''. Zignorował je. Zaur zwrócił czarny, pusty wzrok w~tym kierunku. I, kątem oka, Gregor zauważył, że najgłośniejszy
przeciwnik był po cichu pilnie pouczany przez jedną z~barmanek.

Gregor zamówił kufle piwa dla siebie i~Elizabeth oraz czajnik gorącego
rosołu rybnego dla Salasso. Gdy czekał na kufle i~zagotowanie cienkiej
zupy, znowu zaczął się martwić o~niechybne dążenie, które tak często
wiązało jego stopę z~ustami. Skrępowanie, był przekonany, nie było
wynikiem różnicy pomiędzy płciami, ale prawdopodobnie większej przepaści
komunikacyjnej pomiędzy klasami. Elizabeth Harkness pochodziła z~lokalnej, w~większości upadającej, aczkolwiek dobrej rodziny, której
niektórzy członkowie wybijali się w~Herezji Szydercy. Jego własny ród,
plus minus kilka egzogamii z~tubylcami, pochodził z~załogi \emph{Jasna
Gwiazda}.

Jego własna twarz zbadała go w~lustrze baru za butelkami: ten sam cienki
nos i~srogie usta i~długie czarne włosy zaczesane do tyłu od wysokiej
linii włosów, które widywał na generacjach portretów. Poczuł ich
obecność jak ciężar na plecach.

-- To będzie pięć szylingów, Greg.

-- Och! -- mrugnął i~potrząsnął głową. Z~lekkim opóźnieniem rozpoznał w~barmance Andree Peden, jedną z~młodych studentów, których pracę czasem
nadzorował. Jej falujące kasztanowe włosy były rozpuszczone na ramionach
zamiast związane z~tyłu, i~dotychczas widywał ją tylko w~fartuchu
laboratoryjnym, ale jednak. 

-- Och, dzięki, Peden. I, pomyślmy, jeszcze
poproszę o~trzy deko konopi. -- Alkohol nie był nałogiem zaur, z~powodu
różnic fizjologicznych. Wchłaniali go bez odurzenia, ale konopie
zdecydowanie podbiły gatunek, wieki temu.

-- To będzie jeszcze sześć pensów -- powiedziała dziewczyna. Rozejrzała
się. -- I~jesteśmy po szkole, więc tylko imiona, ok, Greg?

-- Aha, jasne, Andrea, dzięki.

Wiedział, że gdyby skomentował jej zatrudnienie, to byłoby tylko kolejne
niezdarne przypomnienie różnic pomiędzy jego statusem ekonomicznym i~większości studentów, więc się powstrzymał.

Przy stole, Greg wzniósł na trzeźwo toast do Elizabeth, podczas gdy
Salasso prawie niepostrzeżenie kiwnął głową obojgu i~wyjął z~kieszeni na
udzie cienką aluminiową rurkę i~długą kościaną fajkę, której główka była
misternie wyrzeźbiona. Zaur wyssał rosół rybny przez rurkę i~zaczął
napełniać fajkę z~papierowego rożka.

-- Ha! -- westchnął po minucie, jego usta otwarte, jak u węża,
zadziwiająco szerokie i~ukazujące małe kły. Gregor skrzywił się lekko na
chwilowy powiew oddechu mięsożercy. Cienki mosiężny prostokąt pojawił
się w~długich palcach Salasso niczym magicznej sztuczce, Salasso
przyłożył delikatny płomień do ubitych konopie i~pyknął z~fajki. -- Ha!
Jeszcze lepiej!

-- Jakieś wieści, skąd może być statek? -- spytała Elizabeth, jak gdyby
Gregor mógł wiedzieć. Ale to Salasso odpowiedział.

-- Nova Babylonia -- powiedział i~zaciągnął się z~fajki. Jego powieki bez
rzęs mrugnęły raz, a~jego trzecia powieka, błona migawkowa, poruszała
się znacznie szybciej niż zwykle. -- Z~floty rodziny Tenebre. -- Jego głos
był cienki i~chrapliwy. Podał fajkę Elizabeth.

-- Poznajesz, prawda? -- spytała, zaciągając się z~grzeczności.

Gregor był trochę rozczarowany jej dopytywaniem. Do diabła z~tym, skąd
zaur o~tym wiedział, ważniejsze było pochodzenie statku. Statki z~Nova
Babylonia były rzadkie. W~ciągu jego dwudziestu lat życia, potrafił
sobie przypomnieć tylko dwie takie wizyty.

-- Taaaak -- odpowiedział Salasso, opierając wąskie barki o~wysokie
oparcie krzesła. Znów possał metalową słomkę, czerwień na krótko
pojawiająca się na szarozielonej skórze jego policzków. -- Widziałem ten
statek\ldots wiele razy.

Elizabeth spojrzała się sceptycznie na Gregora, gdy podawała mu fajkę.
Gregor odpowiedział ostrzegawczym spojrzeniem i~tylko pokiwał głową
Salasso, starając się ostrożnie zachować swoją minę, gdy zaciągał się
ostatnim wstrętnym żarem konopi. Wytrząsnął je na podłogę i~znowu
zaczął napełniać fajkę.

-- Zatem może znasz kogoś z~załogi? -- spytał.

Ramiona Salasso się uniosły. 

-- To bardzo możliwe. Jeżeli tak, dowiem
się, kiedy pójdę na powitanie do zamku.-- Jego wielkie oczy zamknęły się
na chwilę. -- Z~rodziny Tenebre, pewnie kilkoro. Ludzkie pokolenia
mijają.

Gregor zapalił zapałkę o~but i~przyłożył do fajki, uderzenie cannabis
zaszumiało mu w~uszach niczym narastający pożar. Usłyszał lekki śmiech
Elizabeth.

-- Muszę to zobaczyć! -- powiedziała. -- Po to, poszłabym na Twoją imprezę
jak stoję, szmaty, dziury, buty i~w ogóle!

-- Ta, zrób tak -- odparł ciepło. -- Przyjdź jako naukowiec.

-- Och, dzięki -- powiedziała Elizabeth z~oburzeniem, a~potem
zachichotała. Gregor oddał fajkę zaurowi, który wypalił resztę, ale już
nic nie powiedział, jego ciało nieruchome, jego umysł wpadający w~typowy
trans zaurów. Oboje ludzi patrzyło w~ciszy przez kilka minut, popijając
małymi łyczkami piwo, aż fajka zagrzechotała wypadając z~palców Salasso
na stół.

Gregor pochylił się nad stołem i~pogłaskał suchą ciepłą skórę na twarzy
zaura. Nie było żadnej reakcji.

-- Jest niedostępny -- zauważył.

-- Jeszcze po piwie, skoro czekamy? -- spytała Elizabeth.

-- Ta, jasne.

Wydawało się, że chwila, albo wieczność, minęła, zanim wrócił.

-- Myślisz, że mówił dosłownie? -- spytała, gdy już usiadł. Głowa zaura,
jego wielkie oczy ciągle otwarte, nagle opadła na bok na jej ramię.
Elizabeth poklepała ją. -- Cóż, oto ja tutaj przez cały czas!

-- Około dwudziestu minut, zdaje się -- powiedział Gregor z~roztargnieniem. -- Um, w~jaki sposób nie mógłby mieć tego dosłownie na
myśli?

Elizabeth oparła się na jednym łokciu, ostrożnie, żeby nie przeszkodzić
Salasso, patrząc Gregorowi w~oczy z~czymś, co wyglądało na wpółzjaraną
pewność siebie. 

-- Wiesz. Metaforycznie. Mogła to być ich rodzina, linia,
lub cokolwiek, wiedziała coś z~tamtego czasu. -- Roześmiała się. -
Naprawdę myślisz, że nasz przyjaciel tutaj jest tak stary? Nie
\emph{zachowuje }się w~taki sposób.

-- Wszyscy wiemy, że zaury żyją bardzo długo.

-- Hipotetycznie.

Gregor spojrzał wprost na nią, oczy zmrużone. 

-- Zaury tak mówią i~nie
mam powodu, żeby w~to wątpić.

Elizabeth skinęła powoli głową. 

-- Ta, cóż, czasem zastanawiam się, czy
zaury są\ldots starożytnym ludem. Tym, czym ludzie by się stali, gdyby żyli
dostatecznie długo.

Gregor się zaśmiał. 

-- To ciekawy pomysł. Ale zaury, są oczywiście
gadami, lub jaszczurami, jeżeli chcesz być precyzyjna!

-- I? Geny gadów mogą być ciągle w~nas, a~tylko wyrażone znacznie później
w życiu niż normalnie żyjących ludzi.

-- Możesz mieć rację -- przyznał Gregor, nie rozważając pomysłu ani przez
chwilę. -- W~końcu, nikt nigdy nie zrobił autopsji zaura, ani nawet nie
ma schematu szkieletów zaurów.

Strząsnął ostatnie pokruszone liście do ozdobnej fajki Salasso. Ta
chwila, kiedy zaury mocno i~naprawdę się najarają konopiami, była
jedyną, kiedy otwierali się na kilka sekund i~pozwalali wymsknąć się
dziwnym informacjom. Ale równie dobrze mógł być to efekt narkotyku.

-- Czy ktokolwiek kiedykolwiek prosił o~taki schemat? -- Elizabeth
zastanawiała się głośno.

Gregor potrząsnął głową. 

-- I~postawię dobre pieniądze, że Salasso nie
odpowie na żadne pytania, które zadamy, kiedy wróci do świata żywych.

-- Hmm, nawet nie próbowałabym. Byłby o~to drażliwy.

-- No i~widzisz.

To było stare zagadnienie. Indywidualizm zaurów i~kolczaste poczucie
prywatności sprawiało, że ludzie wyglądali jak jakiś rodzaj gadatliwego,
plotkującego zwierzęcia, które polowało w~stadach. Zaury mogły wyglądać
tak samo dla zwykłego lub wrogiego obserwatora, choć tutaj, jak odkrył
Gregor, poufałość czyniła to rozróżnienie pozornym, ale ich osobowości
były nieprzewidywalnie różne. Kilka cech, które zaury dzieliły,
dotyczyło niesamowitego pragnienia wiedzy i~wielkiego oporu przy jej
ujawnianiu. Ich język, seksualność, relacje społeczne, polityka i~filozofie, jeżeli istniały, były tak samo zagadkowe jak wtedy, gdy
spotkali pierwszego przerażonego dzikusa, wiele tysiącleci temu w~historii ludzkości.

-- Cóż, -- powiedział Gregor, zapalając fajkę -- najmniejsze, co możemy
zrobić, to wypalić ostatki jego zioła. Pewnie nie będzie go potrzebował.

-- Hmm.-- Elizabeth szybko pyknęła z~fajki i~ją oddała. -- Więcej już mi
nie trzeba, dzięki. Chcę mieć jasne umysł, żeby zapisać notatki.

Bogowie na orbicie, koleżanka zaczynała wyglądać atrakcyjnie, gdy jej
wesołe oczy patrzyły tymi bezdennymi źrenicami prosto w~jego umysł,
nawet jeśli jej twarz i~sylwetka były nieco zbyt\ldots kanciaste niż te,
które mu się zwykle podobały u kobiety\ldots Ta dziewczyna Peden przy
barze, ona była całkiem\ldots

Elizabeth zaśmiała się i~Gregor nagle zaczął podejrzewać, trzeźwiąc jak
oblany zimną falą na pokładzie, że faktycznie powiedział to, o~czym
myślał\ldots ale nie, jego suche usta nadal dotykały ustnika fajki.

Zwilżył usta, przezwyciężając kolejny efekt zioła: nagły głód. 

-- Co?

-- Po Twoim wyglądzie, Gregor, nie radziłabym Ci dzisiaj przejmować się
notatkami. Nie będą miały sensu rano.

Jej głos, lub wibracja śmiechu, spowodowały, że Salasso poruszył się i~nagle usiadł wyprostowany, mrugając mocno i~rozglądając się dookoła.
Elizabeth pogłaskała rękę zaura, lekko chwyciła dłoń Gregora na moment,
który uznał za wynik ponarkotykowej tkliwości i~się podniosła.

-- Dobranoc, chłopaki -- powiedziała i~wyszła, zanim Gregor mógł się jej
zapytać, czy planowała coś \emph{zjeść}.

\chapter[Prawo pobytu: cudzoziemiec]{2 Prawo pobytu: cudzoziemiec}

Budzę się z~dzwonieniem w~uszach i~światłem błyskającym w~prawym oku. Za
pomocą polizanego palca wskazującego odklejam powiekę oka na tyle, żeby
dwoma nieśpiesznymi mrugnięciami przełączyć przychodzącą rozmowę video w~oczekiwanie. Potem dotykam lewego ucha, żeby przyjąć rozmowę audio.

-- Tak? -- pytam drażliwie, siadając. To jakaś bezbożna godzina jak
jedenasta rano. Łóżko jest w~nieładzie i~whisky, którą nieostrożnie
wypiłem podczas słuchania muzyki po wczorajszym powrocie z~pubu, jest
przyczyną bólu głowy.

Gdy tylko słyszę uderzenie monet, wiem, że będą kłopoty. Anglia jest
jedynym miejscem poza Afryką subsaharyjską, która nadal posiada budki
telefoniczne na monety. Moi przyjaciele używają ich nie dlatego, że nie
są na podsłuchu, bo są na cholernym podsłuchu, ale żeby przekazać
niewypowiedzialną wiadomość: ,,Tak, kłopoty''.

-- Cześć, Matt? -- mówi znajomy amerykański głos. -- Tu Jadey. Możemy się
spotkać na Rynku? Powiedzmy około piątej?

Jadey jest naszą lokalną Jankeską. Ma status ,,pobyt: cudzoziemiec'' jako
student na wymianie, lub coś, ale większość czasu spędza prowadząc
operacje dla Oporu na południu. Nigdy nie byłem pewien dla kogo naprawdę
pracuje, ale zawsze byłem chętny zmodyfikować software dla poprawek
hardware, które zabiera ze sobą, gdy odwiedza Londyn. Mój mały hack na
sieciach neuronowych rozpoznawania twarzy był nagłą robotą, ale, jak
powiedziała Jadey, pobił temat.

-- Ta, jasne -- odpowiadam, starając się brzmieć normalnie. Jestem trochę
zauroczony Jadey. Beznadziejnie, biorąc pod uwagę kim prawdopodobnie
jest i~co prawdopodobnie robi, a~zresztą często jej nie ma.

-- Do zobaczenia -- mówi. Opłata telefoniczna kończy się serią
\emph{piknięć?.}

Jeszcze raz mrugam okiem i~przełączam drugą rozmowę do ekranu na
ścianie. Odklejam telefon od policzka, zirytowany, że zasnąłem na nim,
wyrzucam go do śmieci i~patrzę na ekran. To oferta jednej z~agencji i~czytam ją, drapiąc się nieświadomie po skórze podrażnionej telefonem.

Oferta wygląda na kolejną szybką i~brudną robotę dla Europejskiej
Agencji Kosmicznej\footnote{Europejska Agencja
Kosmiczna, ESA -- ang. European Space Agency -- obecnie międzynarodowa
organizacja krajów europejskich, której celem jest eksploracja i~wykorzystanie przestrzeni kosmicznej \url{https://www.esa.int/}{www} - przyp.
tłum.}. Zawiera przekopywanie się przez kilka warstw aplikacji
działających jedna na drugiej, aby znaleźć błąd w~podstawach systemu
operacyjnego, który, znając moje szczęście, zapewne okaże się być
MS-DOS-em. Będę musiał wciągnąć do tego starego programistę, najlepiej
takiego, który nie był opłaconym dozgonnym członkiem dżihadu Linuxa.

Wystawiam ofertę, koszty i~harmonogram, upewniając się, żeby przydzielić
sobie dwa razy tyle czasu, ile zabierze prawidłowe wykonanie zadania.
Agencja natychmiast odpowiada z~harmonogramem, który daje mi około
połowę czasu, którego potrzebuję, żeby prawidłowo wykonać zadanie. Ale
stawka winna pokryć zlecenie dla maniaka komputerowego razem z~moimi
innymi wydatkami i~moją dzienną stawką, więc ją akceptuję.

\threeast

Kierowanie projektami softwarowymi zawsze wyglądało jak wypasanie kotów.
Tak mi mówili, w~każdym razie, starzy kierownicy, pomiędzy kreskami
kokainy w~modnych barach, gdzie przepuszczali swoje dobrze zabezpieczone
emerytury. W~ich czasach, jednak, kotami byli ludzie, lub przynajmniej
rodzaj facetów, który teraz są
geekami\footnote{człowiek, który dąży do pogłębiania
swojej wiedzy i~umiejętności w~jakiejś dziedzinie w~stopniu daleko
wykraczającym poza zwykłe hobby, por.~\url{https://pl.wikipedia.org/wiki/Geek} -
przyp.tłum.} programowania. Obecnie, \emph{programistami} są
programy, podobnie analitykami systemu. Moim zadaniem jako \emph{project
managera} jest zebranie przekonującego zestawu AI, ani nie
niesprawdzonego, ani niezbyt daleko od krzywej, potem wypuszczenie botów
marketingowe, żeby pokazały ich umiejętności w~nieskończonych nudnych
konkursach piękności biznesowych botów agencji, złapanie kontraktu i~poprowadzenie stada przez cały kłótliwy tłum, kiedy dojdzie do umowy.

Potrzebujesz do tego czegoś prawie jak umiejętności społeczne, ale
praktycznie musisz być syndromem Aspergera borderline, żeby rozwinąć te
umiejętności z~AI. A~kiedy potrzebujesz geeków dla rzeczy niskiego
poziomu, musisz jednak być kimś w~rodzaju społecznego zwierzęcia. To
dostatecznie rzadka kombinacja, żeby była warta więcej niż przeciętna
pensja. Jestem artystą, a~nie technikiem. Tak opłacam rachunki.

Ten kontrakt dotyczył interfejsu kontroli produkcji dla projektu
górniczego asteroidy. Asteroida, o~której mowa, to \emph{10049 Lora}, bezpański
kawałek śmiecia pomiędzy orbitami Ziemi i~Marsa, długa na około
trzydzieści kilometrów, z~niskim albedo, obecnie, niejasno pamiętam jej
obserwację, poruszającą się kilka milionów kilometrów od Ziemi, odkryta w~latach dwudziestych dwudziestego pierwszego wieku, dziesięć
lat później zbadana przez sondę ESA, okazała się być chondrytem
węglistym. Potencjalne źródło niezmiernie przydatnych substancji
organicznych dla osadnictwa kosmicznego, jeżeli kiedykolwiek do niego
dojdzie. Eksperymentalna stacja górnicza ESA, zbudowana dookoła statku
\emph{Marszałek Titow}\footnote{prawdopodobnie na cześć
Giermana Titowa (1935-2000) radzieckiego kosmonauty, pilota myśliwskiego
i doświadczalnego, generała pułkownika, Lotnika Kosmonauty ZSRR por.~\url{https://pl.wikipedia.org/wiki/Gierman_Titow} - przyp.
tłum.} działała przez lata wykazując notorycznie złe wyniki: pomyśl o~schemacie Tanganiki\footnote{próba założenia upraw
orzeszków ziemnych w~Tanganika (obecnie Tanzania) przez brytyjski rząd
w latach 1948-1951. Cel ten pochłonął blisko miliard funtów wg cen 2020
zob.~\url{https://en.wikipedia.org/wiki/Tanganika\_groundnut\_scheme} -
przyp.tłum.}, pomyśl o~\emph{Mirze} , pomyśl o~dziurze bez dna.

Patrzę na to przez kilka minut, stukając kciukiem, żeby przejrzeć
kolejne strony: informacje dodatkowe wyglądają na przesadne, ale
(sprawdzam) są zindeksowane i~umożliwiają przeszukanie. Właściwa
specyfikacja jest wielka, ale do ogarnięcia. Mogę sobie z~tym poradzić,
ale nie przed śniadaniem.

Moje mieszkanie jest na dwudziestym piątym piętrze jednego z~nowych
wieżowców Urzędu Mieszkalnictwa na samym końcu Leith
Walk\footnote{obecnie jedna z~najdłuższych ulic
Edynburga - przyp.tłum.}. Struktura budynku pokazuje wiek, pięć lat
czyli około połowy przewidywanej trwałości konstrukcji w~nowej
technologii, ale jest tanio za cztery pokoje. Przechodzę wszystkie z~nich, wędrując z~sypialni przez salon do kuchni przez łazienkę. To w~łazience dochodzę do wniosku, że wyglądam jak gówno.

W kuchni, rozgryzam garść aspiryn, piję kawę i~przegryzam się przez
miskę płatków. Przeglądam bez wielkiego zainteresowania poranne
wiadomości, przerzucając kanały. Patrzę w~roztargnieniu przez południowe
okno kuchni na Zamek i~wysokie wieże South Bridge. Niebo jest błękitne,
chmury białe, przelatujące w~poprzek od lewej do prawej, ze wschodu na
zachód. Ich powolny pochód uspokaja mój umysł.

Projekt zajmuje mi całe popołudnie. Ilekroć potrzebuję patcha spoza UE i~spójrzmy prawdzie w~oczy, potrzebujesz, połączenie staje się nierówne. O
czwartej trzydzieści, z~piekącymi oczami i~bolącymi stawami od ciągłego
przekonywania AI, zachowuję stan plików przez łącze z~satelitą i~wychodzę z~domu.

\threeast

Waverley Market był ślicznym, ekskluzywnym centrum handlowym, aż do
trzeciego tygodnia Wojny Uralsko-Kaspijskiej o~Ropę, kiedy to Edynburg
stał się nieodwracalnie terytorium nieprzyjaciela. Amerykańska rakieta
manewrująca nie trafiła w~węzeł kolejowy i~usunęła wschodni koniec
Princes Street, a~razem z~nim biura rządu Szkocji, które przez cały czas
były prawdopodobnie celem. W~dzisiejszych czasach jest to ładny przykład
roli pchlego targu w~socjaldemokracji. Późnym popołudniem przeglądam
stragany z~elektroniką i~biotechem, słońce końca lata, ramiona zgarbione
w parce -- szkockie sierpnie są nieco chłodne, od kiedy Golfstrom
zaniknął -- łokcie na straży wobec tłumu turystów przepychających się
podróbek zagranicznej technologii. Festiwal w~Edynburgu ciągle jest
największy na świecie i~ściąga turystów z~całej Unii. Widziałem kobietę
z Syberii rzucającą się na pamięć sterowaną nerwowo z~Brazylii, parę
Włochów kłócącą się, czy stać ich na zaślepkę Raytheon, od miesięcy
przestarzałą w~Stanach, lata przed naszymi rzeczami.

Naszymi rzeczami\ldots Mogę wyczuć je na wietrze. Nowa technologia, mokra
technologia: fabryki bioelektroniki z~ich zapachem acetonu i~alkoholu,
znajome technologie browarnictwa, destylacji i~rafinowania Edynburga
rozwinęły się, żeby produkować nowy wachlarz zestawów hardware, tak
tanich, jednorazowych i~recyklingowanych jak papier. Wszystko fajne i~zrównoważone, ale stara technika amerykańskiej ekonomii ropy/metalu
ciągle jest na przodzie.

Jadey odnajduje mnie z~jej zwykłą alarmującą łatwością. Patrzę do góry i~oto ona, opierająca się o~stragan. Przycięte blond włosy, niebieskie
oczy, podkoszulek bez rękawów, długie ciepłe ocieplacze na ramionach i~nylonowa, wojskowo zielona plisowana spódnica. Na plecach ma jedną z~dziewczęcych wersji plecaka. Jej zmęczony uśmiech pasuje do wyglądu
zmęczenia podróżą kolejową z~Londynu.

-- Kłopoty na granicy? -- pytam.

Potrząsa głową, gdy łapie mnie za łokieć i~zaczyna kierować do stoiska z~kawą. 

-- E tam, ale człowieku, miałam kłopoty. -- Rozmowa jest tutaj dość
bezpieczna. Buczenie z~gadżetów na wyprzedaży skutecznie zagłusza
większość podsłuchów. Większość kamer ulicznych i~innych czujników w~Szkocji i~reszta w~UE i~tak jest regularnie rozpieprzana przez hakerów.
Wyścig zbrojeń pomiędzy inwigilacją a~sabotażem jest darwinowskim,
wyścigiem Czerwonej Królowej\footnote{odniesienie do
hipotezy Czerwonej Królowej, gdzie silna konkurencja wymusza stałe
zmiany ewolucyjne o~charakterze kierunkowym por.~\url{https://pl.wikipedia.org/wiki/Hipoteza\_Czerwonej\_Kr\%C3\%B3lowej} -
przyp.tłum.} w~którym hakerzy są zwykle odrobinę szybsi. Na południu
jest trochę trudniej, ponieważ władza używa cięższych, bardziej
złożonych aplikacji, a~hakowanie jest bardziej efektywnie tłumione z~użyciem odwróconego social engineeringu. Stąd moje specjalistyczne
urządzenia dla Jadey.

-- Sprzęt nie zadziałał\ldots -- mówi Jadey.

-- Co?

-- Nie Twoja wina. Coś się zmieniło. O świcie większość aktywistów na
południu została odwiedzona przez władze. Tak jakby, kurde, wszystkie
nasze szyfry zostały złamane czy coś. Wydaje mi się, że i~do mnie
dotarli, gliny na King's Cross po prostu mnie przepuściły z~tym
wszystkowiedzącym uśmieszkiem.

Jadey żyje w~rysach pomiędzy jurysdykcjami: U.S.A., Unii Europejskiej,
Republiki Szkockiej i~Byłego Zjednoczonego Królestwa. W~obrębie Byłego
U.K. Jadey wykorzystuje zazdrostki i~niekompetencje rywalizujących władz
powojennych, Anglików, Ruskich i~niebieskich hełmów.

Kupuję dwa papierowe naparstki espresso i~siadamy, popijając, na kawałku
rozbitej ściany.

-- Mówisz, że, gdy rozmawiamy, ruch oporu jest rozbijany?

Jadey patrzy w~dół, bawi się sznurkiem w~spódnicy. 

-- Chodzi o~rozmiar tego, Matt. Muszę się wydostać.

-- Dobra -- odpowiadam czując ból. -- Czego potrzebujesz?

-- Nowego dowodu osobistego. Och, żadnych zmian w~siatkówce, czy coś,
tylko nowy paszport i~historię. Jeżeli zaczną robić biotesty, znajdą
mnie, zanim miałabym czas, żeby próbować zhakować swoje DNA.

-- Hej, nie bądź taka fatalistyczna. Przygnębiasz mnie. -- Zrywam się. -
Wiesz co. Znajdźmy coś do jedzenia, potem możemy pójść do Darwina i~zobaczymy, co oferują. Zresztą mam tam swoją robotę do załatwienia.

-- Super -- odpowiada. -- McDonald.

-- Co?

Rzuca spojrzenie do tyłu, już idąc ścieżką do ulicy.

-- Ostatnie miejsce, gdzie gliny będą szukać Amerykanki.

\threeast

Gdy przebijamy się przez tłum na Darwin's Arms, sprawdzam odczyty nosowe
w~lewym oku. Dzięki Bogu za bezdymne papierosy, analiza feromonów to
teraz sama radość. Próbujesz czegoś takiego w~Turcji lub Azerbejdżanie i~dostajesz dane botaniczne, a~nie psychologiczne. Atmosfera jest dziwnie
napięta, na podłożu kruchej wesołości. Gdy już zauważyłem to w~powietrzu, mogę to też odczytać ze słuchu. Jadey, idąc za mną i~zostawiając rozszerzający się nurt pożądania (mogę zauważyć mały
czerwony garb na odczycie), też musiała to zauważyć.

-- Nerwowy wieczór, co?

Jej amerykański akcent sprawia, że miękną mi kolana.

-- Nie żartuj. -- Zasadzam łokcie na barze i~przeglądam kartę. -- Co
chcesz?

-- Cally Eighty.

Uśmiecham się doceniając jej dobry smak i~zamawiam dwa kufle piwa. 

-- Pomyślmy spokojnie -- mówię. -- Bez nerwów. Jesteśmy tutaj bezpieczni,
ale\ldots

Jadey rzuca mi spojrzenie nad krawędzią jej podniesionej szklanki. 

-- Ok,
zdrowie.

Opieramy się o~bar i~rozglądamy się po pomieszczeniu, jak gdybyśmy
rozglądali się za siedzeniami.

-- Trochę tłoczno, prawda -- mówi Jadey.

-- Aha -- odpowiadam. -- Dziwne. Jest dopiero osiemnasta i~to miejsce
zwykle nie jest pełne, aż do dwudziestej trzeciej, naszego czasu. Wtedy
na wschodnim wybrzeżu Stanów wybija piąta po południu.

-- Tak. I?

-- Cóż, godziny pracy w~Stanach to godziny szczytu dla problemów z~odziedziczonymi systemami komputerowymi\footnote{system
odziedziczony, lub zastany -- stara technologia, system komputerowy lub
aplikacja, która odnosi się, korzysta lub jest przestarzałą wersją, ale
jest nadal w~użyciu -- por.~\url{https://en.wikipedia.org/wiki/Legacy\_system} - przyp.tłum.}. Nasze starsze chłopaki mają co robić popołudniami i~wieczorami.

-- Myślałam, że programowanie było zabawą młodych -- mówi Jadey z~przekąsem.

-- Tak było dawniej -- odpowiadam, leniwie przyglądając się klienteli
baru. Przynajmniej mam nadzieję, że to tak wygląda. Starsi bywalcy są
znacznie wcześniej niż zwykle, a~także nowi klienci, młodzi managerzy. A
wszystkich jest więcej, niż zwykle widuje w~takich miejscach w~tym samym
czasie. 

-- Ciągle jest, w~pewnym sensie, dla tej roboty, którą robię. Ale
programowanie jako takie jest tak powiązane z~odziedziczonymi, starymi
aplikacjami, że praktycznie jest to gałąź archeologii. Nawet nowe apki
to coś, czemu możesz dotrzymać kroku nawet po trzydziestce. Słyszałaś o~prawie Moore'a?

Jadey potrząsa głową, odpowiadając wyzywającym spojrzeniem jakiemuś
staruszkowi, który patrzył na nią nieco zbyt długo.

-- Nie jestem zdziwiony -- mówię. -- To była obserwacja, że moc
obliczeniowa staje się dwa razy szybsza za połowę ceny co osiemnaście
miesięcy. Ta krzywa spłaszczyła się dawno temu. -- Śmieję się krótko,
zwracając uwagę. -- Równie dobrze, albo ta banda byłaby niczym bogowie.

-- Przerażająca myśl -- zgadza się Jadey. Patrzy na kufel piwa, patrzy na
mnie. -- Możemy porozmawiać?

-- Hmmm. -- Uśmiecham się. Pub jest bezpieczny, to jego przewaga
biznesowa, włożyli zagłuszanie elektroniczne w~kurz, ale nie czuję się
bezpiecznie.

-- Jest jakiś powód, żeby tu być? Oprócz tego, co potrzebuję, znaczy się.

-- Ta, pewnie -- odpowiadam, zaczynając rozumieć, że ona nie jest
paranoikiem. Rzemiosło: Zawsze dysponuj prawdopodobną przykrywką.
Leniwie przeglądam w~myślach szczegóły umowy z~ESA, wtedy\ldots

-- Poczekaj chwilę -- mówię jej. W~końcu znalazłem gościa, którego
szukałem i~wołam go. Jason, wysoki i~szczupły, ubrany na czarno,
najseksowniejszy szuler w~mieście, bierze drinka i~się przysiada. 

--Wejdźmy do gry.

We troje przechodzimy do jedynego wolnego gierstołu i~zakładamy
rękawiczki i~okulary. Stół dostraja się i~nagle staje się znacznie
większy, o~słabej, nieokreślonej szarości. Reszta pubu nagle jest
odległa.

-- Jaką grę chcecie? -- pyta Jadey, palce zawieszone nad klawiaturą.

-- Bilard Kwantowy -- odpowiada Jason.

Jadey wklepuje wybór, a~stół migocze i~zmienia kolor na zielony.
Powietrze staje się przydymione, odkłada się grubą warstwą pod niskim
sufitem. Słabe światło oświetla zielone płótno stołu bilardowego i~kolorowe kule. Poza tym światłem, blisko, w~barze, który wcale nie
przypomina tego w~Darwin's Arms, barmanka rozmawia z~jednym z~mężczyzn,
który pochyla się lub opiera na blacie. Gdzieś dzwoni gierstół, a~w~szafie grającej Jagger śpiewa ,,Sympathy for the Devil''. Nieco dalej,
jeżeli spojrzeć wzdłuż określonych kątów pomiędzy szczelinami w~ścianach
i przegrodach, jest inny bar, kolejny stół, inne maszyn, kobiety i~mężczyźni, miejsce trwa, powtórzone jakby w~lustrach. Żadnych okien, ale
są drzwi. Za jednymi z~nich, jakby patrząc przez zły koniec teleskopu,
jest bar, w~którym realnie jesteśmy. Za innymi są bary, które, mam
nadzieję, są fałszywe, ale dodają autentyczności atmosferze Starego
Świata.

Sięgam pod stół i~wyciągam skrzynkę Schrödingera, w~której wirtualne
życie wirtualnego kota jest na łasce skramblera podłączonego gdzieś tam
w rzeczywistym świecie do rozkładającego się izotopu.

-- Martwy czy żywy?

-- Martwy -- odpowiada Jason.

Kot jest zdecydowanie martwy.

-- Twoje rozbicie -- mówię, zamykając skrzynkę. Wsuwam ją na jej miejsce
pod stołem. Jason kreduje końcówkę kija, pochyla się w~poprzek, patrzy w~wzdłuż i~rozbija bile. Kilka zielonych i~różowych zderza się, a~każda
dzieli się na sześć niebieskich.

Jadey się śmieje. Opiera się o~coś, prawdopodobnie o~oparcie krzesło,
które software wirtualizacji pokazuje jako jaskrawą, mosiężną ladę
barową. Jason prostuje się i~patrzy na nią.

-- A~zatem -- mówi. -- w~czym problem?

Jadey pociera dłonią kark. 

-- Potrzebuję nowego paszportu, dowodu
osobistego i~wizy na wyjazd. Znaczy, szybko.

-- Och -- Jason mruży oczy. -- CIA?

-- Gdybym była -- odpowiada -- myślisz, żebym Ci powiedziała? Albo
potrzebowała Twojej pracy?

Wzrusza ramionami. 

-- Zaprzeczalne nie-zaprzeczenie. To mi wystarczy.

Mnie nie wystarczy. Właściwie całe to przepytywanie bardzo mnie
niepokoi, ale na razie trzymam język za zębami.

Ustalają szczegóły pracy i~specyfikację, podczas gdy ja przygotowuję się
do pierwszego strzału. Poruszam kijem zbyt szybko, prawie tak szybko jak
wolne światło. Skrócenie
Lorentza-Fitzgeralda\footnote{więcej
\url{https://en.wikipedia.org/wiki/Length\_contraction} - przyp.tłum.}
zmniejsza czubek o~kilkanaście centymetrów i~kompletnie chybiam.

-- Cholera.

Jason uderza ponad stołem, zostawiając mnie w~niebezpiecznej pozycji,
ale nie do końca nieodwracalnej.

-- Dlaczego wszyscy są tak wcześnie? -- pytam.

Jason chrząka. 

-- Wszystkie połączenia transatlantyckie były dzisiaj
bardzo zmienne.

-- Ta, zauważyłem -- mówię kwaśno.

-- I~nie było za dużo cholernej roboty.

-- Aha -- odpowiadam, kredując kij. -- Interesujące.

Próbuję zgrabnego uderzenia relatywistycznego: pozwalając na skrócenie,
uderzając bilę mocno, wystrzeliwując jedną z~tych małych, lekkich
ultrafioletowych tak szybko, że ich masa się zwiększa na tyle, by
przesunąć jedną z~zielonych, która okrąża jedną z~małych czarnych dziur
w rogu i~ustawia kilka innych bil, w~które uderza, żeby zmusić Jasona\ldots

Ale udaje mu się wydostać i~kompletnie mnie załatwia.

-- Jeszcze raz? -- sięgam po skrzynkę Schrödingera.

-- Nie -- potrząsa głową. -- Muszę popracować. Masz coś przeciw, jeżeli
zostaniemy tu na chwilę?

-- Żaden problem.

Jadey wyskakuje do świata rzeczywistego po kolejną kolejkę. Jason zgina
palce. Długi niski stół przetacza się przez jedno z~wirtualnych wejść i~zatrzymuje się dokładnie przy nas, gdy Jadey wraca z~naszymi kuflami.

-- Nie stawiaj tutaj. -- Jason przypomina jej w~samą porę. Wielki
stół, wyczarowany z~jego softu, może zatrzymać jego ręce w~datarękawiczkach, ale nasze, i, oczywiście, każdy inny realny obiekt, po
prostu przeleciałby przez stół. Jadey stawia drinki na prawdziwym
gierstole i~obserwujemy pracę Jasona. Odwraca się na chwilę, palcami
obramowuje twarz Jadey, potem kładzie powstały portret na płasko i~zaczyna zmieniać go: z~fotografii paszportowej, przez identyfikator
pracownika, zdjęcie dyplomowe, maturalne, klasowe w~podstawówce,
niemowlęcia\ldots Inne karty i~obrazy pojawiają się na powierzchni dużego
stołu i~tasuje je, przesuwa dookoła z~prędkością eksperta. Przed naszymi
oczami powstaje nowa biografia Jadey, od żłobka do biletu turysty.
Zmiata wszystko w~jeden stos, stuka krawędziami o~stół i~sprawia, że
znikają w~jego rękawie.

Odsuwa stół i~odwraca się do mnie, z~dużym mrugnięciem na Jadey.

-- Czas je urealnić -- mówi. -- Jeden dla geeków.

\threeast

Starzy programiści nie umierają. Po prostu przenoszą się do systemów
odziedziczonych.

Nawet wyglądają w~ten sposób. Pierwsi użytkownicy do końca, nie ruszali
telomerów i~mikserów mitochondrialnych jak reszta z~nas, o~nie, oni
muszą sprawdzić nieprzetestowany biotech, więc wyglądają trochę
niejednolicie, w~rodzaju szara skóra i~gładka broda. Jadey, Jason i~ja
okrążamy ostrożnie krawędź ochrypłego, kilkunastoosobowego zbiegowiska
starych łotrów, wszyscy żłopiący piwo i~gadający pełną piersią.

-- Co z~tymi pierdolonymi wiadomościami? -- ktoś mówi, potrząsając głową i~mocno mrugając. -- Nie mogę złapać CNN, ani nawet Slash-dot\ldots

Ta szczególna klika to nie tylko programiści. Gdzieś półwieku temu, w~latach dziewięćdziesiątych, krąg znajomych programistów nałożył się z~kręgiem szkockiej literackiej inteligencji. Zmysł mody obu grup nie
zmienił się z~czasem. Pisarze nosili różnego rodzaju zniszczone kurtki z~niby-robotniczego jeansu lub sztucznej skóry. Programiści wybierali
kamizelki z~wieloma kieszeniami obładowane hardwarem do naprawy
hardwaru, multitoolami Gerbera i~Leathermana, scyzorykami szwajcarskie
Victorinoxa, latarkami Maglite, oraz spranymi targowymi podkoszulkami: Sun,
Bull, SCO, Oracle, Microsoft\ldots Wcale nie ironicznie, tylko jako
reklamę, nie produktów, czy firm (większość z~nich już znikła), ale
umiejętności, nie do końca zbędnych, hakowania ich odziedziczonego kodu.

Staram się okazać szacunek, jak jakiś fan na konwencji, ale wcale nie
szanuję tej grupy. Rządząca Partia uważa ich za niepewnych, ale na tyle,
na ile jestem zainteresowany, to tylko zwykła KPUE, która połknęła kije.
Mgliście lewicowi, ściśle cyniczni, wpłynęli na wyluzowaną, olewającą
aprobatę tak zwanej ,,importowanej rewolucji'', która była następstwem
naszej porażki w~wojnie. To ten rodzaj ich gównianego nastawienia do
kontroli jakości pozwolił Ruskim w~pierwszym rzędzie ominąć automatyczną
obronę NATO.


Z drugiej strony, jeżeli chcesz zhakować systemy dokumentacji oparte na
Unixie gdzieś w~zakurzonych metalowych obudowach w~szkołach, szpitalach
i departamentach kadrowych w~całych kontynentalnych Stanach
Zjednoczonych, wezmą się do roboty bez zadawania pytań, szczególnie
jeżeli płacisz w~dolarach. Kieruję się na Alasdaira Currana, wysokiego
dziewięćdziesięciolatka z~długimi białymi włosami i~chełpliwymi czarnymi
bokobrodami.

-- Facet, który mnie wytrenował, pracował na LEO -- Alasdair głośno się
przechwala -- i~był uczony przez pewnego agenta, który był w~Bletchley
Park, więc myślę, że\ldots

-- Ta, Alec, i~ciągle jesteś gówniany -- ktoś inny krzyczy.

Gdy kołysze się w~ogólnym śmiechu, Jadey łapie jego wzrok, a~ja
korzystam z~możliwości, żeby powiedzieć mu coś do ucha. 

-- Masz chwilę?

-- Och, jasne, Matt. Co teraz robisz?

-- Cóż, szukam kogoś do MS-DOSa\ldots

Curran rzuca gniewne spojrzenie, potem pokazuje kciukiem jednego ze
swoich kumpli. 

-- Szukasz Tony'ego.

-- \ldots a~Jason potrzebuje kogoś do wczesnego dialektu Oracle.

-- Ach! -- Curran uśmiecha się. -- To ja, poradzę sobie.

-- Potrzebujemy, jakby, teraz -- mówi mu Jadey.

-- Teraz? -- patrzy ze smutkiem na piwo, a~potem na Jadey. Jadey uśmiecha
się swoim najlepszym uśmiechem, przed którym Curran nie zna obrony. Hej,
nawet ja czuję ciepło na twarzy, a~nawet nie jestem głównym celem.

\threeast

Z powrotem w~pokoju bilardu kwantowego, ale tym razem nawet nie udajemy
grania. Curran uruchamia jakiś niezdarny manipulator bazy w~wirtualnej
rzeczywistości, Jason znowu ustawia swój stół z~kartami, a~ja mobilizuję
niektórych programy, żeby zajęły się protokołami interfejsów i~złamały
amerykańskie firewalle.

Mam niesamowite poczucie otwierania otwartych drzwi. W~ciągu kilku
chwil, Curran jest po łokcie w~amerykańskich bazach administracyjnych,
Jason wrzuca nielogowane aktualizacje w~historii Jadey, a~ja utrzymuję
jeden i~tylko jeden rejestr zmian i~moje AI rezerwują nowej tożsamości
bilet lotniczy.

Wycofujemy się.

Jason podaje Jadey plastikową kartę.

-- To wszystko -- mówi. -- Zabierz to do dowolnej drukarni, wydrukują i~obłożą je dla Ciebie. Nawet będą właściwe ślady grupy krwi.

Potrząsam głową. 

-- Zbyt cholernie łatwo. Tak jakby wszystkie
amerykańskie kody były złamane\ldots

-- Ożeeeesz -- mówi Jadey.

Nagle przypominam sobie. Angielska sieć oporu, odkryta.

-- Aha.

Curran patrzy na nas przenikliwie, gdy wracamy do jego części pokoju. 

-- Co jest?

-- Och, nic -- mówię prędko.

Wtedy zauważam, że całe miejsce jest bardzo ciche i~wszyscy obserwują
monitor na ścianie. Słychać trochę podkręconych fragmentów z~,,Ody do
Radości'', która poprzedza oficjalne ogłoszenia i~pojawia się twarz
Wielkiego Brata. Pierwszy Sekretarz KPUE Giennadij Jefrimowicz zwykle
wygląda odpowiednio dobrodusznie, jowialnie, ale poważnie. Teraz wygląda
po prostu nieznośnie kołtuńsko.

-- Towarzysze i~przyjaciele -- zaczyna, tłumaczenie i~oprogramowanie
synchronizujące ruch ust zwiększa jego wiarygodność jak zwykle dla
każdego z~języków Wspólnoty. Dla tego określonego narodu i~regionu,
zaczyna mówić angielskim z~ciężkim akcentem centralnej Szkocji, o~którym
skądinąd wiem, że został stworzony ze starych taśm lidera
komunistycznego związku zawodowego i~autentycznego bohatera Micka
MacGahey\footnote{zob. \url{https://en.wikipedia.org/wiki/Mick_McGahey} - przyp.tłum.}. -- chciałbym oficjalnie ogłosić. Stacja badawcza Europejskiej
Agencji Kosmicznej \emph{Marszałek Titov} nawiązała kontakt z~pozaziemskim inteligentnym życiem wewnątrz asteroidy 10049 Lora.

Przerywa na chwilę, żeby ta wiadomość dotarła. Jak gdyby z~dużej
odległości, słyszę dźwięk rozbijanych kilkunastu szklanek w~różnych
miejscach w~pubie. Pierwszy sekretarz się uśmiecha.

-- Zapewniam was wszystkich, że nie ma powodu do niepokoju. Obca
inteligencja nie jest zagrożeniem dla ludzkości. Te organizmy są
niezwykle delikatne i~byłyby bardzo podatne na atak lub wyzysk. Dla nich
i dla nas to szczęśliwy zbieg okoliczności, że ich pierwsze spotkanie z~ludzkością odbyło się z~pokojowymi badaczami demokracji
socjalistycznych, a~nie z~komercyjnymi kompaniami lub siłami zbrojnymi.

Coś w~ironicznym skrzywieniu jego krzaczastych brwi niesie informację,
że ostrożnie nie jest wymieniane nic, co mogłoby obrazić
imperialistycznych wyzyskiwaczy, ale wszyscy wiemy, które firmy i~siły ma
na myśli.

-- Nie trzeba dodawać, że ciepło zapraszamy do bliższej współpracy
agencje naukowe całego świata, w~tym Stany Zjednoczone. To niesamowite
odkrycie otwiera wspaniałe widoki współpracy. Teraz skieruję waszą uwagę
do zwykłych wiadomości przekazujących dalsze szczegóły, i~życzę wam
dobrze w~ten historyczny wieczór.

Wycofanie kamery, z~kolejnymi fanfarami trąbek.

Zaciemnienie ekranu, z~\emph{czymś } na środku ekranu\ldots i~wtedy
rozpoznaję część tego i~skala reszty uderza mnie.

Mam dreszcze, jakby woda spływała mi po plecach. Każdy włos na moim
ciele stoi i~myślę, że jest to największa informacja w~historii
ludzkości, ten dzień będzie zapamiętany na zawsze. Patrzę się na ekran,
sparaliżowany jak wszyscy obrazami z~kosmosu. 10049 Lora wygląda jak
kawał żużlu, stacja kosmiczna to drobna ażurowa siateczka z~jej boku.
Mężnie powstrzymuję narastające pragnienie nerwowego chichotu.

-- Obcy? -- słyszę swój pisk. Jadey odwraca się, prawie rozlewając piwo,
gdy wszyscy zaczynają wrzeszczeć na raz. Ciągnie mnie od stołu, obok
starych geeków krzyczących i~wiwatujących lub, w~niektórych przypadkach,
po prostu patrzących przed siebie z~otwartymi ustami z~łzami w~oczach.

-- Co?

Wciska się koło mnie w~narożną ławkę i~wygląda jakby jedynie ta
wzajemnie dziwna pozycja wstrzymywała ją od spoliczkowania mnie. 

-- Tak, tak -- mówi niecierpliwie -- wielki krok i~takie tam, ale\ldots

W części ekranu, najładniejsza prezenterka ruskich szczebiocze żywo o~,,naszych bohaterskich kosmonautach ESA'' i~,,genialnych naukowcach''. W~innej części, w~znacznie mniejszym oknie, reporter na zewnątrz
Parlamentu Europejskiego brzęczy o~nowym skandalu, jakiś poseł trockista
ujawniony jako biorący większą gażę z~Waszyngtonu, lub prawdopodobnie
Langley. W~innym okresie, to byłby najgorętszy temat. Teraz wygląda na
trywialny i~tandetny, dosłownie doczesny. Dlaczego w~ogóle to wszystko
puszczają?

-- Czy mógłbyś mnie, kurwa, \emph{posłuchać}? -- syczy Jadey. Mrugam i~otrząsam się z~hipnotycznego wirtualnego chwytu ekranu. Skupiam się na
jej twarzy, napiętej i~bladej w~ciepłym świetle.

-- Ok. Sorry. -- Jezus Maria. Czuję pragnienie, by spojrzeć znów na ekran,
jak grawitacja.

-- Matt, ta stacja ESA, to ta sama z~Twojej dzisiejszej roboty.

-- Tak! -- odpowiadam. -- To jest to, co mnie tak zszokowało, oprócz
tego\ldots hm -- powtarzam jej cyniczne słowa -- \emph{wielkiego kroku}.

-- ,,Towarzysze, to nie przypadek'' -- odpowiada szybko. Jadey rzuca szybkie
pogardliwe spojrzenie na ekran. Traktuję to jako pozwolenie. W~dolnym
rogu jest zdjęcie policyjne Webera, potem przebitka na niego, gdy jest
wrzucany policyjnej suki w~Brukseli, następnie szybka sonda zszokowanych
wyborców na jakiejś ulicy, gdzie słońce odbija się od blaszanych szop i~wieżowców i~palm.

-- Gdzie to jest? -- pyta Jadey.

-- Gujana Francuska -- odpowiadam, bezmyślnie.

-- Jak w, sławna Kourou? To miejsce, gdzie mają \emph{drugi } kompleks
startowy ESA?

-- Ta, miał duże poparcie wśród pracowników kosmicznych\ldots

Gapię się na nią, z~dzikim domysłem.

-- Kolejna nie-koincydencja -- mówi.

Oboje patrzymy trzeźwo na uzupełnienie reportażu, wywiad ze zrozumiale
defensywną kadrą partii Webera, \emph{Ligue Prolétarienne
Révolutionnaire. } Telewizor jest prawie zagłuszony przez jednego ze
starych programistów.

-- Pomyślcie o~fizyce niskich temperatur, o~kondensacie Bosego-Einstein,
pomyślcie o~komputerach kwantowych -- wyjaśnia pełną piersią. -- Do
zobaczenia kryptografio, witaj panoptykonie. Przywitajmy kody startowe
amerykańskich rakiet i~otwórzmy drzwi dla Czwartej Wojny Światowej. A~to
nawet nie jest koniec tego. To jest tylko to, co \emph{Ruscy } mogą
zrobić. Czego chcą \emph{obcy }?

Mówca jest solidnie zbudowany, siedzi na stołku i~rozrzuca zakręcone
czarne włosy dookoła jak uszkodzony materac. Widzi nas, i~poszerzając
krąg innych, słuchających, macha rękoma i~kontynuuje entuzjastycznie: 

-- Ludzie, ta stacja ESA jest węzłem internetowym! Kto wie, co mogli już
zhakować? Hej, jeżeli było tutaj tak długo, sobie też nie możemy ufać.
Mogli ustawić pułapki w~naszym pierdolonym DNA jeszcze w~prekambrze\ldots

I ktoś inny mówi: 

-- Karol H. Marks na rowerze, Charlie, obcy już nie
\emph{wystarczą}! -- I~wszyscy znów się śmieją.

Muszę to powiedzieć o~tych gościach, dostosowują się do końca świata,
jaki zawsze znaliśmy, szybciej niż ja. Widzieli wiele w~swoim czasie:
Upadek Muru, Kryzys Millenijny, Boom Wieku, przepełnienie Unixa, Wojnę,
Rewolucję\ldots Kiedy Jadey i~ja zaczynam się przygotowywać do wyjścia,
dyskutują o~tym, jak obcy tak starzy muszą mieć \emph{naprawdę } stare
systemy zastane, oraz, że muszą potrzebować wykonawców, i~jaka duża
powinna być ich dzienna stawka\ldots Gnojki.



\chapter[Pieśni Nietoperzy]{3 Pieśni Nietoperzy}

Esias De Tenebre, magnat i~rejestrowany członek elektoratu Republiki
Nova Babylonia, wypuścił dym ze skręta i~skromnie, ale nie całkowicie
przypadkowo kaszlnął, gdy podawał ten raczej niezdrowy obiekt do damy na
lewo.

-- Moja niewielka znajomość terrologii -- powiedział do wielkiego lorda po
prawej -- jest oczywiście -- pomachał dłonią, nie tyle, żeby rozwiać
żywiczne powietrze przed ustami, o~ile podkreślić wypowiedź -- ściśle
amatorska. -- (To było, dokładnie, kłamstwo. Jego zainteresowanie
wiadomościami było obsesyjne i~lukratywne.) -- Jednakże, mogę Pana
zapewnić, że około sto lat temu, lub wcześniej dzisiaj, jeżeli o~mnie
chodzi (Wielki Zeusie, te rzeczy sprawiają, że staje się gadatliwy, musi
z nimi uważać!)\ldots żaden podróżny, lub w~istocie, nieumyślny przylot
nie pojawił się z~twierdzeniem późniejszego wylotu niż wasz.

Zastanawiał się, czy ta informacja była warta zatrzymania.
Prawdopodobnie nie, i~tak byłoby trudno w~każdym przypadku, jeżeli w~ogóle chciał wymienić się informacjami. Ten lokalny lord był bystry.
Krępy, twardy mężczyzna z~poobijanym rzymskim nosem, każda płaszczyzna
jego twarzy i~szyi zbiegająca się do płaskiej, z~krótkimi włosami,
okrągłej głowy. Jego ojczysta mowa była dialektem angielskiego,
gramatycznie zdegenerowanego w~porównaniu do tych, których Tenebre
słyszał wcześniej, na Croatan. Nawet nieznane wyrażenia naukowego w~jego
słowniku często mogły być zrozumiane dzięki dedukcji z~ich klasycznych
korzeni. Ale przy tej okazji, on, jak większość obecnych, mówił łaciną
handlową, de facto lingua franca Drugiej Sfery.

Szerokie ramiona Hala Drivera lekko opadły, a~jego poza stwardniała po
chwili, którą Tenebre zinterpretował jako smutek, może rozczarowanie.

-- Nic po 2049, co? Och, dobra, zdaje się, że są dwa sposoby spojrzenia
na to. -- Podciągnął niebieskie jedwabne rękawy koszuli, położył gołe
łokcie na zalanej drinkami powierzchni długiego drewnianego stołu,
przyłożył dłonie do ust i~krzyknął po więcej brandy. Jakaś dziewczyna
wstała od stołu i~pośpieszyła do odległej spiżarni.

-- Albo zdmuchnęliśmy się daleko stamtąd -- kontynuował, kierując na
Tenebre niewyraźnie jowialny wzrok -- albo rozszerzamy się w~inną sferę,
Drugą Sferę naszej własnej \emph{Pierwszej }Sfery, bliżej do rodzinnej
gwiazdy.

-- Tak równie dobrze mogłoby być -- powiedział kupiec. Tenebre
dyplomatycznie nie wskazał, że istnieje kilka innych możliwości, żadna z~nich szczególnie radosna nawet w~porównaniu do \emph{wojny nuklearnej },
procedury, która, jak wnioskował, była słabą aproksymacją znacznie
straszliwszych katastrof, które mogły w~tym momencie dotykać Ziemi.

-- Ale na tę chwilę -- Tenebre kontynuował -- bez wątpienia jesteście
ostatnimi przedstawicielami Ziemi, którzy przybyli do Drugiej Sfery.
Więc, naturalnie, jesteśmy bardzo zainteresowani robieniem interesów z~Wami.

Brzmiał zbyt gorliwie, ale nie dbał o~to. Właściwa strategia dla tej
szczególnej okazji nie polegała na targowaniu z~tubylcami, graniu na
spokojnie, trzymaniu kart przy sobie, raczej na położeniu rąk na tak
wielu technologiach i~wiedzy jak to możliwe, zanim pojawi się jakiś
konkurent.

Jeden z~doradców lorda Sternika, ten, którego nazywali Nawigatorem, lord
Cairns, natrętna osoba, która wcześniej testowała jego cierpliwość
uporczywymi i~ledwie zrozumiałymi pytania o, ze wszystkich rzeczy,
\emph{maszyny liczące}, pochylił się ku dyskusji ze swojego miejsca na
prawej jego wysokości lorda, wulgarnie machając widelcem na który był
nabity kawałek przegrzebka. Choć widocznie stary, jego policzki
posrebrzone białym zarostem, był energiczny i~uważny. Szczupły
muskularny mężczyzna z~łysiną, za którą białe włosy zwisały do tyłu.

-- Nadal nie jest dla mnie jasne -- powiedział, w~angielskim z~ciężkim
akcentem -- gdzie byś umieścił Nova Babylonia w~porównaniu z~tym, co
wiesz o~Ziemi, jaką była, gdy opuszczali ją Twoi przodkowie. W~znaczeniu, wiesz, wiedzy, technologii i~tak dalej. Standardy życia ludu
i takie tam. -- Włożył widelec do ust zamknął oczy w~ekstazie, przeżuł i~połknął, a~potem znowu potrząsnął widelcem. -- Zgodnie z~naszymi zapisami
i wspomnieniami, wasza dwunastka, lub coś, poprzedników przez ostatnie
kilkaset lat \emph{bełkotała} -- zachichotał się na myśl o~słabej grze
słów -- o~wielkich błyszczących miastach, pięknych parkach, widowiskowych
puszczach i, wiecie, morzach, w~których kąpał się księżyc \ldots -
Kolejny chichot samozadowolenia. - \ldots i~tak dalej i~tym podobne, oraz
sprawiedliwości i~stabilności potężnej i~starożytnej Republiki. Wszystko
to bardzo dobrze i~doskonale, ale poprzednicy byli tak milkliwi jak
każdy zaur o~rzeczach typu, cóż, jakimi maszynami władacie, standard
życia mas i~tak dalej.

Cairns zauważył widocznie, że zaczyna brzmieć powtarzalnie, ponieważ w~tym momencie litościwie się zamknął. Ale ciągle patrzył wytrwale na
Tenebre, brwi uniesione w~grzecznym pytaniu, jednocześnie w~roztargnieniu nabijając kolejny organizm morski i~pożerając go z~lubieżnym apetytem.

-- Ach -- odpowiedział Tenebre, czując, że jest na mocniejszym gruncie -
to mogę wyjaśnić. -- Spojrzał dookoła, wdzięcznie akceptując rozproszenie
uwagi dolewką brandy i~cygarem, które miał pobożną nadzieję było czystym
tytoniem, potrzebował jasnego umysłu. Ten osobliwy naprędce przygotowany
obiad był wydany nie w~sali bankietowego zamku, ale w~jadalni służby, a~służący dołączali do jedzenia różnych dań, jak tylko skończyli je
serwować. Przy stole, przy którym Tenebre siedział, było kilkanaście
miejsc, i~jak inne, był zajęty mieszanką mężczyzn i~kobiet, panów i~służących, gości i~gospodarzy. W~całej sali było kilka setek osób,
licząc zaury, większość z~których była już dobrze poza nią patrząc
nieprzytomnie w~umysłowe przestrzenie, które konopie otworzyły dla nich.
Hałas rozmów, z~tłumu i~akustyki niskiego, drewnianego sufitu oraz
wszystkich garnków i~patelni zawieszonych na ścianach był straszliwy.

Na stole po drugiej stronie przejścia, Tenebre widział gromadę kobiet
wraz z~jego pierwszą i~trzecią żoną, kilku kobiet służących i~dwie damy
zamku, jedną z~nich była Margaret Cairns, która najwidoczniej była
pierwszą i~jedyną żoną lorda Cairnsa. Wyglądało na to, że wszystkie
mówiły w~tym samym czasie, prócz chwil, gdy głośno się razem śmiały.
Taki rodzaj wymiany informacji był jednym z~tych, których Tenebre nie do
końca akceptował, ale z~trudnością mógł im zapobiec. A~czasem,
rozmawiając swobodnie, jego żony odkrywały sprawy, których nigdy sam by
nie odkrył.

Kiwnął głową podziękowanie dziewczynie, która przyniosła mu brandy.
Podała od niechcenia butelkę Sternikowi i~przyciągnęła taboret, patrząc
cała w~napięciu na Tenebre. Tenebre zaniepokojony podejrzewał, że w~każdej chwili może, jak jakieś ciemne chłopskie dziecko, sięgnąć i~dotknąć jego włosów.

-- Mówiłeś? -- podpowiedział Sternik.

-- A~tak, nasi przodkowie. To proste. Nie wiedzieli o~waszej obecności, w~ogóle się nie spodziewali, i~naturalnie prywatnie byli dość tym
zaniepokojeni. Statek z~ludzką załogą z~Ziemi jest dość bezprecedensowy,
i mogło to zapowiadać różne kłopoty.

-- Masz na myśli, że mogła to być awangarda inwazji z~Ziemi -- powiedział
Sternik. Ta myśl widocznie go rozbawiła.

Tenebre ostro kiwnął głową. 

-- Dokładnie to, w~tak wielu słowach.

-- Chwila, moment -- powiedział James Cairns, rysując diagram ostrzem noża
po stole. -- Mieliśmy statki z~Nova Babylonia z~najbliższych kolonii,
które zdecydowanie o~nas \emph{słyszały}, i~oni byli tak samo milkliwi
jak\ldots Och, racja -- wbił nóż w~stół. -- Teraz rozumiem. Wygląda na to, że
Nova Babylonia nie słyszała o~nas, kiedy tamci wyruszyli, a~jakakolwiek
odpowiedź, która by nadeszła w~międzyczasie, nie miała jak do nich
dotrzeć\ldots .

-- Dokładnie -- powiedział Tenebre. -- I~stąd Elektorat nie mógł tego
przedyskutować. Teraz, po dyskusji, cieszę się, że będę mógł
odpowiedzieć na Wasze pytania\ldots

\threeast

To był składany scyzoryk, słowa\emph{Opinel } i~\emph{France } ciągle
czytelne na ostrzu, ale już dawno starte z~rękojeści z~tworzywa
imitującego drewno. James Cairns bawił blokadą ostrza i~ponuro gryzmolił
czubkiem ostrza. Czy Francja ciągle istniała? Jako fizyczne miejsce? Ile
francuskiej kultury dostatecznie zaawansowana inteligencja mogłaby
odtworzyć z~analizy eleganckiego projektu tego prostego narzędzia?

Cairns oderwał uwagę od tych próżnych, ale kuszących myśli i~wrócił do
teraźniejszości. Część uwagi poświęcił na słuchanie odpowiedzi magnata
na pytania, które były zadawane z~każdej strony stołu. Częścią uwagi
natomiast badał długie trójkąty, które narysował na stole, odpowiadające
latom przeszłym wobec lat świetlnych.

Nova Babylonia, na planecie Nova Terra krążącej dookoła przewidywanie
nazwanego Nowego Słońca, była w~odległości około stu lat świetlnych od
Mingulay. \emph{Jasna Gwiazda} przybyła na Mingulay około dwustu lat
temu, a~pierwszy gwiezdny statek kupców odwiedził kilka lat później,
więc\ldots

Wiadomości o~\emph{Jasnej Gwieździe} niewątpliwie dotarły do Croatan,
pięć lat świetlnych, w~ciągu mniej niż sześciu lat, a~do bardziej
odległych kolonii z~proporcjonalnym opóźnieniem. Ale statek w~drodze nie
mógł odbierać ani nadawać informacji, podróżując z~prędkością światła,
gwiezdny statek uczestniczył w~bezczasowej, bez masowej wieczności
fotonu, co sprawiało, że podróż była subiektywnie natychmiastowa. Zatem
statki pierwotnie z~Nova Babylonia, ale odwiedzające różne porty na
kupieckiej trasie, dowiedziałyby się o~nowych osadnikach Mingulay na
długo przed Nova Babylonia, lub przed dowolnym statkiem wysłanym
bezpośrednio stamtąd.

Ta rodzina Tenebre była w~ten sposób pierwszymi kupcami z~Nova
Babylonia, która miała jakieś pojęcie czego oczekiwać, gdy dotrą do
Mingulay. Interesujące. Zastanawiał się czego chcieli, i~co mogli
zaproponować w~zamian.

Cairns czasem czuł, że, głęboko w~najbardziej niedojrzałych zakątkach
swojego siedemdziesięcioletniego mózgu, żywi urazę do kosmosu za to, że
nie okazał się takim, jakim jego przodkowie oczekiwali. Mógłby żyć we
wszechświecie, którego międzygwiezdne przestrzenie mogłyby być pokonane
przy pomocy statków pokoleniowych, w~hibernacji lub w~rakietach
strumieniowych\footnote{oryg. ramscoop czyli technologia
podróży kosmicznej wykorzystująca rozległe pole elektromagnetyczne do
zebrania i~kompresji wodoru w~celu uruchomienia reakcji termojądrowej
napędzającej pojazd por.~\url{https://en.wikipedia.org/wiki/Bussard_ramjet} - przyp.tłum.} Byłby całkowicie zajebiście
\emph{zachwycony } wszechświatem, który mógłby być przemierzany jakimś
rodzajem napędu warp lub tunelami czasoprzestrzennymi lub innym podobnie
zmyślnym sposobem. W~podobny sposób, byłby całkiem metafizycznie
zadowolony z~kosmosu bez boga, lub, gdyby kiedyś poznał przekonujące
wyjaśnienia, szczęśliwy, by potwierdzić, że ten jest wynikiem pracy
Boga.

Zamiast tego żył w~kosmosie, gdzie bogowie roili się bilionami, normalna
chmura Oorta\footnote{ por.~\url{https://pl.wikipedia.org/wiki/Obłok_Oorta} - przyp.tłum.} dookoła każdej z~gwiazd, przy czym większość bogów, na tyle
na ile wiadomo, była przekonanymi ateistami. Jedyną rzeczą, którą
bogowie kiedykolwiek stworzyli dla korzyści innych był napęd gwiezdny.
Napęd gwiezdny mógł zabrać do gwiazd, w~mgnieniu czasu subiektywnego. Z~prędkością światła.

Czasem nadchodziły momenty, kiedy czuł się, jakby mówił bogom
\emph{wielki dzięki.}

-- \ldots zakup dużej części naszych podstawowych środków spożywczych i~mechanizmów od zaurów, naturalnie -- mówił Tenebre. -- Większość naszego
dostatku, jak możecie się spodziewać, pochodzi z~korzyści handlu.
Większość handlu pomiędzy starszymi gatunkami jest prowadzona przez
rodziny Babylonii, które w~ten sposób mogą wspierać pospólstwo poprzez
zakup różnych usług. Klasy rolnicze i~fabryczne mają tendencje do
specjalizacji w~produktach luksusowych na rynek zaurów. Pośród zaurów
jest moda na ludzkie rękodzieło, smak określonych owoców, warzyw,
przypraw i, ehe, ziół\ldots

Wszyscy się roześmiali.

Kupiec przechylił się do tyłu, poklepując brzuch z~zadowoleniem. 

-- Wszystkie oznaki są takie, że masy ludowe są całkiem zadowolone z~ich
losu. Ci, którzy nie są, mogą przenieść się do młodszych kolonii, i~przez kilka stuleci w~małym stopniu taka emigracja istniała.

James uśmiechnął się wewnątrz, zauważając wolne, trzeźwe kiwanie głową
Sternika. Nova Babylonia wcale nie brzmiała jak technicznie dynamiczne
społeczeństwo. Tyle podejrzewali, ale było to zachęcające, w~pewien
sposób, w~końcu to potwierdzić. \emph{Moglibyśmy ich już podbić, }
pomyślał, zastanawiając się, czy Sternik, nie mówiąc nic Tenebre, myślał
podobnie. Nie, żeby starsze gatunki pozwoliłyby im to zrobić, ale
ciągle. \ldots

-- Więc -- powiedział Sternik -- choć to nie czas na poważne targi, ciągle
się zastanawiam, co moglibyście chcieć od naszego\ldots nieco izolowanego
i zacofanego społeczeństwa -- wzruszył ramionami, rozkładając ręce. --
Wasi poprzednicy w~większości handlowali tutaj z~innymi gatunkami,
kupując tylko trochę naszych słabych lokalnych kopii gadżetów i~świecidełek wyprodukowanych oryginalnie w~Układzie Słonecznym. Trudno mi
przyrównywać naszą technologię z~zauryjską.

Cairns zauważył marsy na czołach dookoła stołu, u osób nieco bardziej
zorientowanych biznesowo, oczywiście myślących, że Sternik nie jest
zbytnim sprzedawcą. Cairns miał nadzieję, że miało to jakieś tajne
znaczenie, przykładowo, \emph{doceńmy go za tę odrobinę przebiegłości},
potem zrozumiał, że te zmieszane twarze na jakimś poziomie mogły być
częścią przebiegłego planu Sternika.

-- Och, kiedy będziemy mieli możliwość zbadania, co moja rodzina i~wasi
kupcy mogą sobie zaoferować, myślę, że wszyscy przyjemnie się zaskoczymy
-- odparł gładko Tenebre. -- Produkty do użytku ludzi są najlepiej
opracowane, jeżeli niekoniecznie wyprodukowane, przez ludzkie istoty. I~mamy, jak zwykle, wiele interesów do załatwienia z~zaurami i~naszymi
kuzynami w~kopalniach i~lasach: farmaceutyki, pewne rzadkie minerały,
drewno i~tak dalej. -- Pomachał dłonią. -- Zwyczajny handel. Ale szczerze
Wam powiem, że najbardziej nas interesuje to co przywieźliście z~Ziemi.
Sztuka, nauka, technologia, historia, filozofia, cała wiedza ojczystego
świata. Nova Babylonia jej pożąda!

-- Ale to \emph{informacja!} -- powiedział Matt. -- A jak mawiamy,
informacja chce \ldots

Głowa Sternik obróciła się jak u atakującego węża. Jego szybkie
spojrzenie spowodowało, że Cairns się zawahał.

-- Informacja chce być opłacona? -- powiedział Tenebre. 

Uśmiechnął się dookoła do rozmówców. 

-- My -- powiedział dumnie -- też mamy takie
powiedzenie.

\threeast

Goście poszli odpocząć, służba i~młodsi członkowie rodziny sprzątnęli
większość wieczornych resztek i~poszli do swoich łóżek. Kilku starych
członków Rodzin Kosmonautów, którzy ciągle żyli w~zamku, przeszli przez
frontową salę i~usiedli w~fotelach w~luźnym łuku dookoła kominka.
Towarzyszył im jeden zaur, stary Tharovar, który przywitał ich przodków,
oryginalną załogę, gdy pierwszy raz przybyli. W~ciągu długiej znajomości
z ludźmi, nabył lepszej odporności na konopie niż większość jego
rodzaju, i~teraz był zrelaksowany, raczej niż w~śpiączce w~sali służby
jak inni.

Cairns doglądał brandy i~cygara, w~fotelu najbliżej gasnącego ognia.
Margaret siedziała na podłodze, opierając się o~ramię fotela,
wygrzewając się w~cieple żaru. Tharovar przykucnął na jego innych
dłoniach. Inni patrzyli niewidzącym wzrokiem w~ogień: Driver, Andriej
Wolkow, Larysa Telesnikowa i~Jean-Pierre Lemieux. Wszyscy oni mieli
partnerów lub kochanków, którzy byli spoza pojęciowej, dziedzicznej
załogi, spoza kadry kosmonautów, oraz taktownie zostawili ich im
prywatnym myślom i~rozmowie.

Driver rozejrzał się po zubożałej grupce i~chrząknął, splunął w~ogień.
Ślina skwierczała okropnie przez kilka chwil.

-- Doskonale -- powiedział -- otrzymałem interesującą ofertę od Tenebre.

-- Inną niż ta, którą przedstawił przy stole? -- spytał Wolkow.

Driver przytaknął. 

-- Tenebre wybrał cichszy moment\ldots Zaoferował nam
zapłatę, cóż, za transport. Mógłby nam zapewnić korzystne warunku,
gdybyśmy weszli w~biznes transportowy.

Odpowiedział mu cichy i~gorzki śmiech.

Cairns poczuł, że Margaret ściska jego kostkę i~rozluźnił swój chwyt,
nieświadomie wzmocniony, na jej ramieniu.

-- A~więc co odpowiedziałeś?

Driver wzruszył ramionami, ruch wyolbrzymiony przez watowane ramiona
luźnego kubraka. 

-- Grałem\ldots na zwłokę, ale dałem mu do zrozumienia, że
bylibyśmy zainteresowani.

-- \emph{ Co? } -- prawie wrzasnął Cairns. Inni wyprostowali się na
siedziskach, równie poruszeni. Driver uśmiechnął się do nich cynicznie.

-- Wszyscy wiedzieliśmy, że w~końcu do tego dojdzie -- powiedział
łagodnie. -- Przygotowaliśmy się na to. -- Wstrzymał Cairnsa karcącym
spojrzeniem. -- Do pewnego stopnia. Zatem, jakie postępy
\emph{poczyniłeś}, Nawigatorze?

James poczekał chwilę. Margaret natarczywie gładziła mu stopę i~delikatny dotyk trochę, niedużo, go uspokoił. Tharovar siadł koło niego
napięty i~sztywny. Ścięgna na cienkim karku zaura były jak naprężony
liny, a~jego usta były, jeżeli to możliwe, cieńsze niż zwykle.

-- Daj spokój, Hal -- powiedział Cairns. -- Przez ostatnie dekady to stało
się tylko przeklętym \emph{hobby}, jak dobrze wiesz. Nie jest łatwo
zainteresować młodszych członków rodziny -- wykrzywił usta -- \emph{Wielką
Pracą}, a~staje się ona jeszcze bardziej nudna, gdy kolejny komputer się
psuje i~nie może być naprawiony. Co jakiś czas ktoś wstydliwie
przekazuje kilka stron logiki lub matematyki. Chrystusie wszechmogący,
niekiedy mógłbym przysiąc, że papier ma na sobie ślady \emph{łez}, jak
jakiś zeszyt ćwiczeń dla dzieci. Porządkuję kartki, odkładam do
szuflady, wysyłam kilka kolejnych zadań i~te wracają po jeszcze dłuższym
czasie. Ludzie mają inne priorytety, inne możliwości i~tak dalej, gdy
czas płynie.

Tylko poczucie, jak patetycznie, jak kiepsko by to zabrzmiało,
powstrzymało go od dodania \emph{Co jeszcze mogę zrobić? } Nienawidził
usprawiedliwiającego się siebie. To nie była maniera, w~ogóle nie jego
styl, ani część programu. \emph{W trakcie, stary kumplu. } Ale to była
prawda i~Driver wiedział, że taka była prawda, a~Cairns wiedział, że on
wiedział.

Więc podsumował powiedzeniem, pewnie i~agresywnie, jego najstarszą
wymówką z~wszystkich, żartem rodziny Nawigatorów:

-- Jestem artystą, a~nie technikiem.

To wywołało śmiech, nawet Driver się uśmiechnął, i~napięcie zelżało.
Larysa Telesnikowa pochyliła się do przodu i~powiedziała dyplomatycznie.

-- Ok, towarzysze -- zaczęła, jak zwykle, gdy mówiła poważnie do
zgromadzenia większego od dwóch osób -- czyli znaczy to tyle, że
\emph{nie wiemy}, jak duży mógłby być postęp. Dlaczego by nie wykorzystać
formalnego przyjęcia kupców, by zaprosić tak wielu jak to możliwe
członków rodziny Nawigatora i~poprosić ich, żeby przynieśli swoje
ostatnie wyniki, nawet ostatnie prace?

-- Lepsze to niż nic -- zgodził się Driver.

-- To wszystko brzmi dobrze -- powiedział Cairns -- ale nie mam dużych
nadziei. -- Spojrzał na Drivera. -- Jak wszyscy dobrze wiecie. A~co
powiesz swojemu nowemu przyjacielowi Tenebre, kiedy stanie się
oczywiste, że nie możemy przewieźć towarów?

Driver zachichotał złowieszczo, drapiąc się po brzuchu przez batyst
koszuli.

-- To właśnie jest piękne -- odpowiedział. -- Powiedziałem mu, że mamy
problemy techniczne, zażądałem istotnego wynajmu, w~ciemno przysiągłem,
że nie dogadamy się innymi kupcami, którzy mogliby się pojawić, i~poprosiłem, by zapytał się o~to, w~trakcie następnej podróży. Dla niego,
to może oznaczać oczekiwanie rzędu kilku miesięcy, może roku. Dla nas\ldots
cóż, w~jedną czy w~drugą stronę, to nie będzie nasz problem.

Cairns zarechotał. Inni też się roześmiali, mniej serdecznie. Wszyscy
mieli po siedemdziesiąt, osiemdziesiąt lat i, nawet z~wiedzą medyczną,
którą już dawno zaury podzieliły się gatunkiem ludzkim, żadne z~nich nie
oczekiwało żyć dłużej niż kilka dekad. Chyba, że w~międzyczasie, rzecz
jasna, sekrety starożytnych Kosmonautów nie zostały na nowo odkryte, ale
to była nadzieja, nie oczekiwanie.

Tharovar wstał, podszedł do kominka i~stanął na wprost w~palenisku. Jego
sylwetka wywołała u Cairnsa atawistyczny niepokój, jak dziecinna reakcja
na znajomą osobę w~przerażającej masce.

-- Czy zastanawialiście się -- powiedział zaur swoim niskim syczącym
głosem -- nad \emph{towarzyszeniem} rodzinie Tenebre do Nova Babylonia i~z powrotem? Moglibyście użyć ich statku jako maszyny czasu w~przyszłość
tej kolonii, przyszłość w~której, może, wasze problemy matematyczne
zostałyby rozwiązane, a~wasze życia przedłużone jeszcze dalej.

-- Tak, rozważaliśmy to -- powiedział Driver, zaskakując Cairnsa, który o~tym nie pomyślał. -- \emph{Nie }mam, kurwa, zamiaru odrywać się od mojego
życia, moich potomków i~mojej zdolności nadążania, w~celu zamiany w~obcego w~obcym czasie.

Cairns dołączył do pomruków zgody.

-- To moglibyście polecieć do Croatan -- nalegał zaur -- i~z powrotem,
wracając tutaj co dziesięć lat. To z~pewnością byłoby wystarczające.

Margaret przemówiła. 

-- Naprawdę nie umiesz zbyt dobrze radzić sobie z~tym całym ,,postępem'', co Tharovar?

Uśmiech w~jej głosie zaprzeczał krytycyzmowi słów i~zaur odpowiedział z~humorem swojego typu.

-- Pewnie nie -- powiedział. -- Jestem tylko jajem.

\threeast

Gregor zwlekł się metr od brzegu materaca, przeczołgał się po całym
dywanie i~uderzył w~przycisk budzika, żeby przerwać głośne dzwonienie.
Wczesne słońce szalało przez wąskie okna jego pokoju. Leżał trochę w~łóżku, trochę na podłodze minutę lub dwie, policzek przyciśnięty do
szorstkich włókien bezlitosnym ciążeniem planety, podczas których
sprawdzał ostrożnie wszystkie podsystemy. Na szczęście, różne bóle w~kończynach i~plecach były wynikiem pracy poprzedniego dnia na łodzi.
Niewielkie ruchy głowy nie spowodowały żadnych eksplozji. Uczucia w~żołądku pochodziły z~pustego brzucha i~pełnego pęcherza. Nie wyczuwał
mdłości. Wzwód, do którego dłoń odruchowo wróciła, był pocieszająco
twardy. Jego usta były suche, ale smakowały nie gorzej niż neutralnie.

Wynikało z~tego, że nie miał kaca i~nie wypił zbyt dużo poprzedniego
wieczora. Pamięć wróciła, wstydliwie przyznając się do kilku dziur w~zapisie, ale wszystko wydawało się być spójne z~fajką dzieloną z~Salasso, potem powrotem do pokoju, zaśnięciem w~ubraniu, obudzeniem się
około północy z~kolorowych snów, czytaniu przez godzinę czy coś, i~ostatecznym powrocie do łóżka, jedynie pięć godzin lub coś koło tego
wcześniej.

Poruszając się wolno, częściowo z~powodu prawdziwych (do pewnego
stopnia) bólów mięśni, a~częściowo z~powodu możliwości ukrytego kaca,
takiego, który czai się tuż za świadomością i~nagle wyskakuje jak kot z~drzewa w~reakcji na pierwszy nagły ruch, Gregor się przetoczył i~wstał.
Skoro wszystko nadal było w~porządku, owinął się szlafrokiem i~poszedł
korytarzem do wspólnej toalety, żeby sobie ulżyć. Gdy wracał do pokoju,
przeciągał się i~schylał wykonując sekwencję Powitania Słońca, i~skończył ją ożywiony. Potem włączył czajnik elektryczny i~zrobił sobie
dzbaneczek herbaty.

Pokój był na tyle duży, żeby pomieścić materac, stół i~krzesło, setki
stron notatek i~kilkaset książek. Nie był to gest niezależności, biorąc
pod uwagę, że ojciec dawał mu pieniądze potrzebne na czynsz, opłaty
uniwersyteckie i~utrzymanie, ale było to lepsze niż życie w~domu. Pokój,
na szczycie budynku w~starszej części miasta Kyohvic, wybudowanego,
zanim statek jego przodków przybył, zapewniał mu spokój i~prywatność, a~także, jeżeli potrzebował tego, towarzystwo innych studentów lub
starszych ekscentryków, którzy żyli w~pozostałych około dwudziestu
pojedynczych pokojach i~współdzielili zużyte urządzenia.

Jak miał to zwyczaju, otworzył dar od ojca, ciężki obity skórą tom
Dobrych Ksiąg, słów filozofów: Fragmenty Heraklita, Powiedzenia
Epikteta, Nauki Epikura, Poematy Lukrecjusza. Angielskie parafrazy były
jednymi z~ulubionych prac literatury Mingulay. Niektórzy twierdzili, że
były lepsze niż oryginały. Jego wzrok spoczął na jednym z~Fragmentów.

\begin{quotation}

Tego świata, jednego i~tego samego świata wszechrzeczy, nie
stworzył ni żaden z~bogów, ani żaden z~ludzi, lecz był on, jest i~będzie
wiecznie żyjącym ogniem zapalającym się według miary i~według
miary gasnącym.\\
\end{quotation}
Książka otwarła się znowu na kolejnej znajomej stronie, z~Nauk:

\begin{quotation}
Dookoła świata brzmi przyjaźni zaproszenie do tańca, połączmy
dłonie w~szczęściu wszystkiego i~wszystkich.
\end{quotation} 

To wystarczy, pomyślał, jako poświęcenie dnia. Dopił herbatę, ubrał się
i poszedł do pracy, jedząc śniadanie po drodze.

\threeast

Elizabeth wyskoczyła z~brzęczącego tramwaju na Harbor Hat i~raźno poszła
na nabrzeże. Zaparkowane skify świeciły się w~porannym świetle na
pomarańczowo, ich kręcone nogi i~soczewkowate kadłuby rzucały długie
cienie na wodę niczym wysokie, trójnożne maszyny kroczące. Statek
kupców, przykucnięty w~cieśninie, ciągle uderzał ją w~oczy widokiem
zaskakującym, natrętnym, w~oczywisty sposób obcym mocno ,,nie z~tego
miejsca''. Wysoko ponad nim, sterowiec z~lotniska na wzgórzach za Kyohvic
wznosił się, żeby spotkać południowy prąd powietrzny, a~małe brzęczące
samoloty wykonywały turystyczne kółka dookoła zatoki i~jego potężnego
gościa. Przez chwilę, statek powietrzny i~samolot wyglądały jak żałosne
podróbki statku gwiezdnego i~skifów grawitacyjnych.

Wzdłuż nabrzeża zaskoczone morskie nietoperze sprzeczały się dookoła
dobrze zabezpieczonych zbiorników, które Renwick i~jego załoga już
podnosili z~pokładu przy pomocy skrzypiącego żurawia i~piszczącej
wciągarki. Elizabeth włączyła się na ile mogła, pomagając manewrować
zbiornika i~skrzyniami, ładując zaparkowaną wzdłuż ciężarówkę
departamentu. Po chwili zauważyła śpieszącego się Gregora i~jej serce
podskoczyło jak ryba.

-- Dzień dobry -- powiedział. -- Przepraszam za spóźnienie.

-- Jesteś prawie na czas -- odpowiedziała Elizabeth. -- Byliśmy wcześniej.

Elizabeth uśmiechnęła się do niego, patrząc i~próbując nie robić tego
zbyt długo, z~nadzieją, że zauważy, że patrzy zbyt długo. Ale Gregor po
prostu się uśmiechnął, kiwnął głową i~złapał linę. Jego dłoń dotknęła
przypadkowo jej, gdy razem ciągnęli. Prawie odskoczyła.

Sprawy mogłyby wyglądać zupełnie inaczej, gdyby Gregor jej nie
\emph{polubił}, gdyby poznali się na jakiejś studenckiej imprezie
zamiast laboratorium, gdyby nie pracowali razem i~zostali kolegami i~dobrymi kumplami, zanim zrozumiała co naprawdę czuła do niego i~co czuła
od początku. Teraz czuła się kompletnie uwikłana w~tę prostą przyjaźń i~bliską współpracę, zmrożona przez strach utraty tego w~galimatiasie
zażenowania i~nieporozumienia.

Siedział koło niej, gdy prowadziła ciężarówkę, elektryczny silnik
jęczący pod obciążeniem, wzdłuż nabrzeżnej drogi do Wydziału Biologia
Morza po zachodniej, nadmorskiej stronie miasta. Tam oddali próbki
dozorcy akwariów morskiej wody i~ruszyli do laboratoriów, żeby zacząć
kolejny dzień badań. Częste wycieczki rybackie po nowe okazy były prawie
wakacjami. Tutaj odbywała się ich prawdziwa praca.

Gregor, Elizabeth i~Salasso pracowali nad mapowaniem systemu nerwowego
kałamarnicy. Jej prosta struktura, stosunkowo wielkie neurony, i,
dosadnie rzecz biorąc, brak twardego szkieletu tworzyły z~niej idealne
zwierzę laboratoryjne do badania neurofizjologii w~ogóle, ale
koncentrowali się na osobliwościach morfologii nerwowej głowonogów.
Ściany były wytapetowane rysunkami, wykresami, odczytami skali pH i~potencjałów elektrycznych.

Salasso, jak zwykle, już tam był, skulony nad głębokim szklanym
naczyniem, w~którym unosiła się mała kałamarnica, nieświadoma delikatnej
igły elektrody, którą zaur powoli wprowadzał.

-- No, chodź, maleńka -- zanucił przy ledwie otwartych ustach. -- To Twój
szczęśliwy dzień.

\threeast

Tenebre obudził się koło żony numer trzy przy świetle poranka i~chórze
nietoperzy. Gdzieś w~przestrzeni poddasza nad sufitem, ptaki świergotały
i drapały, gdy układały się w~grzędach na całodzienny sen. Przez kilka
minut, jak one, kulił się w~współdzielonym cieple, obserwując mgiełkę
swojego oddechu. Wieża Aird, jak zamki wszędzie w~znanym wszechświecie,
nie miała centralnego ogrzewania.

Tenebre chrząknął i~wysunął się z~niskiego łoża, opatulił się w~jedną z~pikowanych szat, którą gospodarze przemyślnie przygotowali i~założył
wełniane skarpetki, które zrzucił poprzedniej nocy. W~ten sposób
wzmocniony, przeszedł do południowego okna, przynajmniej było
przeszklone nawet jeżeli nie podwójną szybą, oparł się o~parapet i~spojrzał nad zatoką na miasto.

Oglądanie Kyohvic ze skifu było jego pierwszym szokiem. Dzienne światło
nie zmniejszało ukazywanej niespodzianki. Patrzył się przez dłuższy czas
na budynki, ostre w~długich cieniach i~różowe od światła jesiennego
świtu. Kiedy ostatni raz je widział, cztery wieki temu, pięć miesięcy
dla niego, miasto było porozrzucanymi niskimi domami wzdłuż brzegu,
zatoka pełna kutrów rybackich oraz rozproszone farmy w~tle. Zamek był
pusty, przesądnie opuszczony. Teraz budynki wznosiły się na pięć, sześć
pięter i~rozciągały się na kilometry po obu stronach doliny. Kutry nadal
kłębiły się w~zatoce, ale były przyćmione przez znacznie większe statki
pełnomorskie, najeżone wysokimi masztami. Pola były rozplanowane według
gęstego wzoru, niektóre zaorane na czarno, inne brązowe od rżyska, inne
zazielenione pędami ozimej pszenicy. Na szczycie wzgórza, sterowce
przesuwały się i~podskakiwały pośród pylonów kotwicznych, a~latające
maszyny (przerażająco rachityczne pojazdy powietrzne. Według niego,
niewiele więcej niż latawce z~motorem) startowały do lekkomyślnych lub
fatalnych lotów.

Tenebre był przyzwyczajony do przyśpieszonych, skompresowanych zmian.
Była to jedna z~zalet życia kupca, dawało to długi wgląd w~historię,
najbliższy, prawdopodobnie, do którego ludzki umysł mógł dojść wobec
tysiącletniej perspektywy zaurów. W~ciągu czterdziestu lat życia i~pięciu stuleci czasu obiektywnego, widział kolonię rodzicielską dla
Mingulay, Croatan, gwałtownie rozwijającą się i~rozszerzającą pomimo
mało obiecujących początków. Widział Nova Babylonia upadającą w~płomieniach i~wstającą z~popiołów \ldots ale to było coś innego, to było
coś nowego pod słońcami.

Ci ludzie, których gościnność (społecznie ciepła, jednak fizycznie
chłodna) doznawał i~którą się cieszył, byli potomkami niezależnych
ludzkich badaczy kosmosu, ,,Kosmonautami'', jak nazywali siebie. Smakował
to słowo, z~pewną buntowniczą próżnością ludzkiego gatunku, o~której nie
przypuszczał, że będzie ją czuł. W~wielkim drabinie bytów, ludzkość
miała szanowane, ale ograniczone miejsce: ograniczone nie przez siłę,
ale przez okoliczność.

Bogowie poruszali się na swych milionletnich orbitach, obojętni i~nietknięci w~przestrzeniach pomiędzy światami, prawie jak zakładał
ziemski filozof Epikur i~jak opisywał to poeta Lukrecjusz. Krakeny
działały w~handlu pomiędzy gwiazdami, nawigując statkami o~prędkości
światła. Zaury dążyli w~kierunku krótszych kursów, pilotując ich skify
grawitacyjne i~pracując w~tropikalnych i~podzwrotnikowych fabrykach
biologicznych, swoich \emph{zakładach produkcyjnych}.

Ludzie\ldots och, tak, ludzie mieli miejsce: wynajdując i~produkując,
przewożąc i~handlując, uprawiając rolę i~łowiąc ryby, wszystko to na
powierzchni ziemi czy morza, lub jako pasażerowie pojazdów starszych
ras. Jedyne rozumne gatunki o~skromniejszej roli były kuzynami
ludzkości, małe człekokształtne kopiące w~kopalniach i~wysokie
człekokształtne dbające o~lasy umiarkowane. Tak było, w~różnych
proporcjach, na wszystkich światach Drugiej Sfery, w~promieniu stu lat
świetlnych dookoła Nova Sol. Taka była hojna granica podróży, które
statki gwiezdne krakenów były gotowe zabrać ludzi.

Hojna, ale jednak granica.

Rzeczy były inaczej kierowane na Ziemi, rodzinnej planecie. I~może także
na tej, Mingulay, na którą przybyli ludzie z~Ziemi z~własnej inicjatywy
i własnym statkiem.

Tuż przed pójściem spać, jedna z~towarzyszących zaurów, Bishlayan,
przekazała mu informację, którą pozyskała od lokalnych zaurów. Wierzono,
że niektórzy z~pierwszej załogi, oryginalni kosmonauci, ciągle żyli,
gdzieś w~dziczy. Ten statek przywiózł sekret długowieczności, jak
również sekret długich podróży. \emph{Jasna Gwiazda} w~istocie, pomyślał
Tenebre, zwracając się z~uśmiechem, by przywitać mamrotania i~jęki kaca
jego trzeciej żony. Pozostałe dwie ciągle spały koło niej.



\chapter[Systemy odziedziczone]{4 Systemy odziedziczone}

Na zewnątrz, Princes Street falowała zwykłym tłumem Festiwalu, ale
ludzie nie zachowywali się w~zwykły festiwalowy sposób. Zadziwiająca
liczba osób patrzyła do góry, jakby oczekiwali, że błyszczący
statek-matka może w~każdej chwili nadlecieć. Inni stali rozmawiając, lub
zatrzymując przechodniów i~przekazując wieści: liczba osób dyskutujących
lub obserwujących niebo wzrastała z~minuty na minutę. Nie widziałem
czegoś takiego od Rewolucji, kiedy byłem małym dzieciakiem, kiedy
wynurzyliśmy się ze schronów, piwnic i~ruin, żeby przywitać Ruskich
żołnierzy na ulicach. Pamiętam dźwięk radosnych klaksonów. Teraz,
porównując, szum głosów, stóp, rowerów i~trolejbusów brzmiał upiornie
cicho.

Jadey złapała mnie za łokieć, gdy zatrzymałem się przed przekroczeniem
jezdni.

-- Dokąd idziesz?

Szarpnąłem wskazując głową na prawo. 

-- Waverley, wydrukować Twoje
rzeczy w~drukarni na stacji, potem złapać pociąg na lotnisko?

-- Nie, nie, nie, nie. Musimy to przemyśleć. Bez pośpiechu, to bilet
open, prawda?

-- Ta, jasne, ale im szybciej się wydostaniesz\ldots

Jadey spojrzała na mnie ostro. 

-- Hej, kto tu jest ekspertem? Czy
opowiadam Ci o~trikach programistycznych? Więc, zamknij się i~chodź ze
mną.

Na takie coś niewiele mogłem odpowiedzieć. Skręciła w~lewo i~poszliśmy
wzdłuż Leith Walk, koło budynków new-tech w~rejonie bombardowania, gdzie
żyłem i~dalej do starszej części ulicy. Tłumy tutaj były mniejsze, też
mniej rowerów. Trolejbusy zjeżdżały środkiem jezdni na dół. Na północy,
kierunku, gdzie mniej więcej zmierzaliśmy, niebo pozostawało zauważalnie
jasne: kilkaset kilometrów bliżej biegunowi słońce nadal świeciło.

Po kilku minutach cichego pędzenia obok sklepów z~softwarem, delikatesów
i restauracji, Jadey skręciła w~lewo w~jedną z~bocznych ulic w~rejonie
Broughton, kanionie kamienic z~piaskowca. Zatrzymała się przy drzwiach
obok tandetnego butiku.

-- Czy to miejsce nie będzie obserwowane, skoro\ldots?

Kolejne spojrzenie. 

-- Tak jak mówiłam.

Jadey wbiła numer, spojrzała na skaner siatkówki i~drzwi się otworzyły.
Wsunąłem się za nią, obok splątanych rowerów i~stosów poczty, potem w~górę kamiennej klatki schodowej. Na trzecim piętrze, otworzyła drzwi
mieszkania przy użyciu metalowych kluczy. Hardware.

Wewnątrz, było chłodno i~ciemno. Przespacerowała się pstrykając
włącznikami do świateł. Okna, ujrzałem gdy weszliśmy do głównego pokoju,
były zasłonięte aluminiowymi żaluzjami weneckimi. Była też tam sofa,
ekran, stół i~niewiele więcej. Plakaty ścienne były ustawione na
zeszłoroczne pasma. Wyglądało to jak pusty studencki kwadrat, i~prawdopodobnie było.

-- Kawy?

-- Dzięki. Czarną, bez cukru.

-- Po prostu dobrze -- odpowiedziała Jadey.

Do czasu, gdy wróciła z~kuchni, uruchomiłem ekran, z~wyłączonym
dźwiękiem. Większość kanałów z~wiadomościami emitowała gadające głowy.
Jadey usiadła na drugim końcu sofy, kiwnęła na ekran.

-- Profilaktyka -- powiedziała. -- Wbudowana. Możemy rozmawiać.

-- Więc\ldots jesteś z~CIA? -- spytałem. Niezbyt dyplomatyczne pytanie na
początek, ale miałem je ciągle w~głowie.

-- Nie, oczywiście, że nie jestem z~cholernego CIA! -- odpowiedziała,
prawie rozlewając kawę. -- Etatystyczne skurwysyny! Są prawie tak źli jak
przeklęte komuchy, kiedy nie robią z~nimi interesów.

-- W~porządku. Tylko pytałem? Więc \emph{skąd} jesteś?

Poważnie zmarszczyła brwi. 

-- Naprawdę chcesz wiedzieć?

-- Cóż, tak. Nazwijmy to czystą ciekawością.

-- Ha! Dobrze. Pracuję dla pewnej organizacji politycznej, która
realizuje to, co \emph{winno} robić CIA: podburzać do przewrotu w~UE.

-- Domyśliłem się tego -- powiedziałem wolno. -- Ale zaskoczyło mnie w~pewnym sensie coś, co powiedziałaś wcześniej. Jak to działa?
Kontrrewolucja dla zabawy i~zysku?

-- Ani jedno ani drugie -- odpowiedziała. -- Pieniądze pochodzą z~\ldots
cóż, zasadniczo ze spadków i~funduszy powierniczych utworzonych przez
przedsiębiorców branży informatycznej, którzy wzbogacili się na Boomie
Wieku i~którzy pomyśleli, że to dobry pomysł, by, hm, zainwestować w~przyszłość wolnego rynku. Co do zabawy\ldots

Jadey odstawiła kubek. Jej dłonie drżały. 

-- Przez chwilę było fajnie,
tam w~starej Anglii. Zbieranie kontaktów, organizowanie, podstawowy
agitprop\footnote{skrót od słów: agitacja i~propaganda
por.~\url{https://pl.wikipedia.org/wiki/Agitprop} - przyp.tłum.}. Ale
ostatnio krajobraz stał się trudniejszy. Wiesz, pseudo-gangi?

-- Co?

-- Grupy oporu utworzone przez\ldots kogokolwiek, chyba Ruskich, ale równie
dobrze Brytoli, żeby skompromitować prawdziwą opozycję przy pomocy
dziwnych skandali. Czarna propaganda, która sprawia, że zarzucają nam
faszyzm, rozsiewanie plotek, że \emph{prawdziwy} ruch oporu to
pseudo-gangi, że najlepsi aktywiści to prowokatorzy policyjni -- pomachała dłonią. -- Znasz zasady.

-- Problem ,,Bezkrytycznego zaufania''\footnote{problem ,,Trusting Trust'' -- opisuje metodę wprowadzania wirusów do wykonywanego programu nie poprzez kod
źródłowy, ale poprzez odpowiednio zmieniony kompilator, czyli dopiero w procesie kompilacji kodu źródłowego są wprowadzane instrukcje wirusa, stąd nazwa
problemu, która odnosi się do granic zaufania do
kodu wykonawczego, zob.~\url{https://www.cs.cmu.edu/\textasciitilde{}rdriley/487/papers/Thompson\_1984\_ReflectionsonTrustingTrust.pdf}
- przyp.tłum.}? -- spytałem, tłumacząc na geekowy.

-- Dokładnie!

Znów zmarszczyła brwi, spojrzała na paznokcie. Ten na kciuku był
obgryziony. 

-- Kurde, myślałam, że pozbyłam się tego nawyku\ldots --
spojrzała do góry.\emph{} -- Opowiem Ci o~ostatniej nocy.

\threeast

Jest taka scena w~filmie \emph{Bitwa o~Algier\footnote{por.~
\url{https://pl.wikipedia.org/wiki/Bitwa\_o\_Algier} - przyp.tłum.}}, gdzie
muzułmanki z~FLN przygotowują się do podłożenia bomb w~dzielnicy
europejskiej i~stroją się w~nieprzyzwoite, europejskie ubrania, po raz
pierwszy w~życiu malują się, a~kiedy przeglądają się uroczyście w~lustrach, ścieżka dźwiękowa staje się bezlitosnym wojennym bębnieniem.

Jadey słyszy taki rytm, gdy przygotowuje się do nocnego zadania. Zawsze
lubiła swoją karnację. Kremowa gładkość naturalnej blondynki pasowała do
jasnego czoła i~bladych warg, ale teraz Jadey zakrywa to wszystko różem
do policzków, maskarą, cieniami do powiek i~krwiście czerwoną szminką.
Żel do farbowania włosów zmienia jej włosy w~czarne i~kolczaste, brudzi
wodę w~odpływie, gdy płucze dłonie pod kranem.

Przygotowania zakończone, czeka jednak kilka minut obserwując zegarek.
Czas jest najważniejszy. Dwie minuty do kontaktu. Już czas.

Ogląda się w~lustrze: koronkowa biała bluzka, mała czarna spódniczka ze
sztucznej skóry, kabaretki, wysokie obcasy. Jej gra to nie Subtelność.
Uśmiecha się do nieznanego wyglądu i~wesoło zarzuca czerwoną skórzaną
torbę na ramię. Wcześniej, przed włożeniem do torby, sprawdziła broń.

-- Naprzód, dziewczyno -- mówi do samej siebie. -- Idź i~powal ich
widokiem.

Powietrze jest wilgotne i~światło żółte. To martwa godzina przed świtem,
ale nie za późno dla prostytutek na ich fach. Jadey unika ich oczu,
patrzy bezpośrednio w~oczy ukrytych alfonsów i~klientów. Przed nią,
widzi plecy człowieka, którego szuka, w~mundurze wojsk rosyjskich.
Hardware jest miękki i~ciepły przez rękawiczkę, jak mała bryłka tej
rzeczy, którą bawią się dzieci, albo może plastiku, i~tak samo
niebezpieczny: Głupi Semtex. Przykleja go klapiąc w~słup i~odchodzi
energicznie w~York Way, około trzydziestu metrów za mężczyzną. Dziesięć
po cichu odliczonych i~wolnych sekund później, Ruski skręca z~głównej
drogi w~alejkę. Jadey podąża za nim bez patrzenia za siebie.

Dziesięć sekund to wystarczający czas, żeby hardware zadziałał: żeby
bryłka przyczepiła się i~popłynęła jak złośliwa szybka szlamo-pleśń,
rozwijająca swoje wici w~kablach kamery ulicznej na lampie i~żeby
wprowadziła swój program do strumienia danych. W~tym momencie, jakość
obrazów powinna subtelnie się pogarszać, aż do momentu, gdy każda twarz
na York Way równie dobrze może mieć założoną kominiarkę. Przy odrobinie
szczęścia, hardware wkopie się głębiej i~przeedytuje pamięć tak dobrze
jak wejście, szyfrując software rozpoznawania twarzy, ale na to nie
można liczyć. Maskowanie strumienia wejściowego, jednakże, jest możliwe,
ale nadal Jadey nie rozgląda się dookoła, nie daje im nawet tej odrobiny
szansy.

Idzie prosto koło Ruskiego, który udaje, że ogląda wystawę zakurzonej
hydrauliki. Jej wzrok spotyka odbicie jego przez ułamek sekundy, potem
Jadey go mija. Ruski wygląda w~tym mundurze elegancko, choć w~porównaniu
do niej jest w~mniejszym stopniu żołnierzem. Wszyscy urzędnicy okupacji
muszą nosić mundur wojskowy. To jeden z~tych ruskich pomysłów.

Ruski czeka, aż jej szpilki wybiją metronomiczne pięć sekund, wtedy
gwiżdże na nią. Jadey się odwraca, obracając ramiona i~biodra, ściskając
miękką skórę jej torby. Uśmiecha się do niego i~rzuca okiem na coraz
węższą alejkę. Ruski kiwa niepostrzeżenie i~wkracza w~alejkę. Ona podąża
za nim, jej chód raczej biznesowy niż profesjonalny.

Patrzy na Josifa z~pewnym ciepłem i~uznaniem, któremu nie zapobiegnie
żadna ostrożność. W~ciągu tych miesięcy Jadey polubiła tego Ruskiego.
Mimo tego, Jadey jest zaskoczona, kiedy zamiast ich zwykłego udawanego
targowania, Josif łapie ją za pas i~przyciąga ją do siebie. Jego usta
zbliżają się do jej. Następuje moment kontaktu ust i~języka, wtedy
słyszy, podobnie jak czuje, mały metalowy obiekt wpychany delikatnie w~jej usta. Prawie połyka, co może nie byłoby takim złym pomysłem, gdyby
nie konsekwencje w~dłuższym czasie. Udaje jej się to wepchnąć
językiem pomiędzy zęby a~policzek. Jest to wielkości i~kształtu małej
monety z~zaokrąglonym brzegiem.

Josif cofa głowę cicho oddychając. Jadey kiwa głową, prawie
niepostrzeżenie. Wtedy jego pierś uderza w~jej, Jadey słyszy walnięcie,
czuje uderzenie i~słyszy straszny zgrzytliwy dźwięk. Gdy Jadey robi
wymuszony krok do tyłu, Josif staje się ciężarem w~jej ramionach i~musi
go puścić. Jego usta otwierają się jak do krzyku. Wypływa tylko krew, a~potem leży na mokrym bruku, głowa i~stopy stukają, krew wypływa i~tryska, jelita się opróżniają.

A ona stoi dwa kroki do tyłu, w~pozycji strzeleckiej, plastikowy
jednostrzałowy Liberator przed nią w~złożonych dłoniach.

Młody mężczyzna stoi naprzeciw niej, wygląda na zszokowanego z~wielkim
zakrwawionym nożem zaciśniętym w~prawej dłoni. Kamizelka, dżins i~rękawiczka, otoczony od głowy do stóp w~błonie polimeru izolacyjnego:
ubiór dla lekarza lub mordercy.

Sama mogłaby wykorzystać taki strój. Na jej bluzce jest pełno krwi.

Na twarzy młodego mężczyzny pojawia się irytacja i~zdziwienie. 

-- Nie jesteś\ldots

-- Czego się spodziewałeś? -- Jadey ma sucho w~ustach, próbuje nie ugryźć
tej rzeczy w~jej ustach, ani swojego języka. -- Na pewno nie mnie.

Nie odwraca wzroku ani od jego oczu ani od ostrza. Josif umiera. Jeszcze
niekoniecznie martwy, ale w~ciągu kilku następnych minut musiałby się tu
pojawić cały zespół medyczny, żeby go uratować, a~Jadey wątpi, czy
mogłaby to załatwić.

-- Odsuń się -- ostrzega. On się wzdryga, a~Jadey stara się utrzymać swój
głos na wodzy. 

-- Nie, nie Ciebie się cholera spodziewałem. Dlaczego
miałbym?

Wtedy Jadey zaczyna rozumieć, co ten facet myśli się tu odbywało.

-- Och, Chryste, myślałeś, że jestem po Twojej stronie?

Facet potakuje. W~niewyraźnym świetle, Jadey może jedynie dostrzec
krótkie włosy, szczupłą twarz, zakryte oczy. Wątpi, czy facet jest
dorosły. Dokładnie ten typ dzieciaka, który jest angażowany w~komórkach
nacjonalistycznych terrorystów, którzy myślą, że wciąganie rosyjskich
żołnierzy w~ciemne alejki, a~potem zabijanie ich, jest dobrą metodą,
żeby stworzyć arsenał na Wielki Dzień. Dzieciaki takie jak te są zmorą
jej pierdolonego życia.

-- Myślałem, że to była ustawka -- mamrocze dzieciak. Jego londyński
akcent jest tak ciężki, że z~trudem go rozumie. -- O, gówno.

Na razie pozwala konsekwencjom uciec.

-- Biegnij -- mówi, wskazując pistoletem.

\threeast

-- A~potem co zrobiłaś?

-- Pobiegłam w~przeciwnym kierunku, wzdłuż innej alejki. Minutę później
okrążyłam i~przeszłam York Way trochę dalej, gliny krążyły jak muchy na
gównie dookoła skrzyżowania, gdzie byłam. Dlatego właśnie uważam, że
Twój hardware jest spieprzony. Zdecydowałam się nie wracać do
mieszkania, poszłam do bezpiecznego domu i~okazało się, że już nie jest.
Drzwi właśnie były wyważane, gdy wychodziłam zza rogu, więc dość ostro
poszłam w~innym kierunku. W~dzielnicy czerwonych latarni, miałam kilka
skrytek z~zapasowymi ubraniami, dokumentami i~tak dalej. Szczelne
pojemniki przyklejone w~koszach na śmieci, w~podobnych miejscach. Więc
użyłam jednego z~nich, żeby się przebrać i~wywalić ubranie dziwki.
Poszłam do mojej pracy przykrywki, to była specjalnie podejrzanie
wyglądająca praca, amerykańska firma importująca książki,
zatelefonowałam do Ciebie w~przerwie na kawie i~złapałam pociąg w~środku
dnia. I, jak mówiłam, gliny dziwnie się na mnie patrzyły.

Patrzyłem się na nią, nieco wstrząśnięty historią, z~powodu ukrytego
seksu oraz otwartej przemocy. Ile czasu ją znałem? Kilka lat, co
najwyżej. Wpadła do sklepu przy Leith Walk, gdzie pracowałem pomiędzy
kontraktami, śpieszę dodać, że to było zanim zdobyłem obecną wysoką
pozycję jako kierownik, rzuciła parę okularów Calvin Kleina na ladę i~zapytała o~kilka interesujących modyfikacji. \emph{Poważne} naruszenie
warunków gwarancji i~praw autorskich, tego rodzaju rzeczy, które nie
miały prawnego uzasadnienia: rażące i~nielegalne jak skrócenie lufy
strzelby. Wziąłem to zlecenie, nie zadając pytań, i~następnego dnia
oddałem jej zestaw. Zapewniło to kolejne interesy, a~ona zaczęła
pojawiać się w~różnych miejscach, gdzie przesiadywałem. Gadaliśmy,
czasem nad kawą, czasem z~narkotykami, ale nic więcej z~tego nie wyszło.
Nie byłem zbyt pewien, teraz, kiedy uświadomiłem sobie, że była związana
z angielskim ruchem oporu. Musiała mi to jasno powiedzieć w~którymś
momencie, ale nie potrafiłem sobie przypomnieć kiedy. Właściwie nigdy o~tym nie rozmawialiśmy.

-- Masz tę rzecz, którą dał Ci ten rosyjski gość?

-- Oczywiście. -- Przedmiot pojawił się na jej rozłożonej dłoni jak w~czarach. Wziąłem go i~obróciłem.

-- To dysk danych. -- powiedziałem, nie zaskoczony, ale niejasno
rozczarowany, tak jakbym podświadomie oczekiwał, że będzie to nowa tajna
broń.

-- Powiedz mi coś, czego nie wiem!

-- Może mógłbym -- odparłem. -- Mogę Ci powiedzieć co \emph{na} nim jest.

Potrząsnęła głową. 

-- Próbowałam na swoim czytniku w~pociągu. To śmieci lub jest zaszyfrowany.

-- Ech. -- Wyjąłem swój własny czytnik z~kieszeni, podłączyłem go do
telefonu i~włożyłem dysk. -- To prawdopodobnie nie jest jeden z~tych
komercyjnych kodów, ale wątpię, żeby to był najnowszy wojskowy, bo nie
miałoby sensu przekazywanie dalej, prawda?

-- Masz rację. -- Jej głos brzmiał smutno. -- Josif i~tak nie miałby
dostępu do prawdziwych tajemnic, to musi być ważne, ponieważ jest teraz
ważne, a~nie że utajnione.

Wywołałem swój plik tysiąca kluczy i~uruchomiłem program testujący je na
dysku. To nie było właściwie łamanie kodów, raczej dopasowywanie szyfru
do kluczy, których, prawnie rzecz biorąc, nie miałem powodu mieć,
dlatego też trzymałem je ukryte na serwerze daleko stąd. Z~boku ekranu
mała czerwona linia powoli malała, gdy program pracował.

-- Nie wygląda na coś, za co warto umierać.

-- Nie wiedział, że tak ryzykuje -- powiedziała Jadey.

-- Lub za co zabijać.

-- Myślisz, że był wrobiony? -- Zrobiła kwaśną minę. -- To możliwe.

\emph{Ding} zadzwonił czytnik.

-- O hej!

Zacząłem przeglądać odszyfrowany tekst. Jadey nachyliła się, żeby
spojrzeć, mamrocząc zainteresowanie i~uznanie. Przyśpieszyłem
przeglądanie, nagle obawiając się, że to wszystko wyglądało zbyt
znajomo. Było.

-- To specyfikacja ESA. Pamiętasz, ta sprawa ze stacją kosmiczną?

-- Możesz to sprawdzić?

-- Ta, pewnie. Niedużo mogę \emph{zrobić} na tej rzeczy, ale \ldots --
Wrzuciłem strony na wielki ekran, tak, że mogliśmy komfortowo je
przeglądać i~wróciłem do klikania przez strony, mniej lub więcej losowo.
Niektóre strony były tekstem, z~rozwiniętą numeracją dokumentacji
technicznej, niektóre linie w~obrębie akapitów miały własne numery
wersji, a~niektóre, które zajmowały nieco więcej czasu do wyświetlenia,
były trójwymiarowymi wykresami i~schematami zoptymalizowanymi do
patrzenia przez okulary VR lub soczewki. Większość pasowała albo do
fotografii albo nieodróżnialnej superrealistycznej wizualizacji
końcowego przedmiotu.

-- Wiesz -- Jadey głośno dumała -- nie uważam się za eksperta w~górnictwie
asteroid, ani tym bardziej w~sprawie sprzętu, którego potrzebujesz, żeby
porozumieć się z~obcy umysłem-rojem, czy coś, ale to zdecydowanie nie
wygląda jak sprzęt górniczy, ani tym bardziej jako naukowa placówka
badawcza.

Parsknąłem. 

-- Masz rację. Potrzeba czegoś jak rafineria w~górnictwie
asteroid, ale to jest coś znacznie większego. Wygląda mi to na jakiś
rodzaj automatycznej fabryki powstałej z~części, które mogły być zebrane
albo dla tego albo dla stacji badawczej. \emph{Budują} tam coś, lub
planują wybudować.

-- Jakaś wskazówka co to mogłoby być?

-- Może być w~tym -- powiedziałem wzruszając ramionami -- ale byłoby
strasznie trudno to stwierdzić. W~tym jest więcej danych niż w~\emph{Encyclopedia Britannica}.

-- Dobra. -- Jadey spojrzała się na mnie dziwnie. -- Nie wydaje mi się, że
to my powinniśmy próbować się tego dowiedzieć. -- Podrapała się po
głowie. -- Znaczy, myślę, że nie powinniśmy na to nawet patrzeć.

-- Och. -- Wyłączyłem ekran. -- Szczególnie nie ja, co?

-- Szczególnie nie Ty. Dla własnego spokoju ducha.

-- Można patrzeć w~ten sposób. -- W~jakiś sposób czułem się urażony myślą,
że miałem mniej powodów do szperania w~możliwych państwowych tajemnicach
niż ona, ale doszedłem do wniosku, że jest to nieracjonalne, UE nie
było bardziej ,,moim'' państwem niż jej, a~ja na swój sposób byłem jego
wrogiem. Tylko nie tak dobrze wytrenowanym wrogiem. Nie, żebyśmy się
martwili, to była Szkocja, szczęśliwa mała socjaldemokracja, nie jakieś
państwo-kolonia w~Trzecim Świecie którejś ze stron, ale oboje
wiedzieliśmy co narkotyki prawdy Federalnego Biura Bezpieczeństwa
potrafiły, a~im mniej wiedzieliśmy, prawdopodobnie mieliśmy mniej
kłopotów, gdybyśmy zostali złapani.

Jadey przyniosła więcej kawy i~siedzieliśmy przez kilka minut ogrzewając
dłonie na kubkach i~nic nie mówiąc.

-- Więc -- powiedziałem w~końcu -- co robimy? Nadal chcesz jechać do
Ameryki?

Jadey przygryzła dolną wargę zębami, odstawiła kubek i~włożyła dłonie
pod pachy, lekko kołysząc się do przodu i~do tyłu. 

-- Och, kurde, nie wiem -- odpowiedziała. -- Gdybym nie podejrzewała, że wszystkie nasze szyfry są łamane w~czasie rzeczywistym, po prostu wrzuciłabym to na
szybkie łącze i~wróciła do domu czysta. Cholera, gdyby zatrzymali mnie
na lotnisku, czy gdzieś, wyszłabym za kilka tygodni.

-- Wymiana agentów? Ale Ty nie\ldots

-- Och, rodzaj podobnej umowy. Sektor prywatny ma swoje sposoby, ok? Ale
tak jak to jest, muszę zabrać tę rzecz fizycznie, i~to musi polecieć
oddzielnie. I~wiesz co? Myślę, że nie możemy polegać na poczcie.

-- Zabierz to do konsulatu Stanów? -- zasugerowałem bystro. -- Paczka
dyplomatyczna?

-- Nie wiem, czy chcę, żeby etatyści to dostali -- powiedziała ponuro. --
Myślę, że moi przyjaciele w~organach Ruskich na południu by tego nie
chcieli. Chcieli wysłać to ludziom w~Stanach, którzy mogliby to
maksymalnie wykorzystać. A~to oznacza \emph{naszych} ludzi, a~nie
wywiad.

Wstrzymałem się z~dalszymi pytaniami, kim są ,,nasi ludzie'', nie tylko z~powodu bezpieczeństwa. Nie urodziłem się wczoraj, już miałem pewne
podejrzenie. Jedną z~dyskusyjnych zalet życia w~prawicowo-komunistycznej
wersji kapitalizmu państwowego było to, że oficjalne media przedstawiały
całkiem rozsądne analizy materialistyczne, w~każdym razie, spraw
wewnętrznych innych krajów. Rozłam w~klasie amerykańskich kapitalistów
był klamrą mądrości \emph{Prawdy Europy}. Na samym dnie, pod tym całym
gadaniem o~Jankesach i~Kowbojach, Globalistach i~Izolacjonistach,
Starymi Fortunami i~Nuworyszami znajdowała się korzyść materialna, którą
była niemal żenująco ropa naftowa.

Amerykański krajowi producenci ropy zasilali siły Nuworyszy, a~zagraniczni inwestorzy w~ropę wspierali Stare Fortuny. Ci ostatni
dostali baty w~Uralsko-Kaspijskiej Wojnie Naftowej i~Izolacjoniści
świętowali krótki triumf przy pośpiesznym wycofywaniu żołnierzy i~oportunistycznym rozliczaniu starych urazów w~polowaniu na czarownice
pod tytułem ,,kto stracił Europę?''/ ,,Zieloni w~Maszynie'' w~pierwszej
połowie lat trzydziestych XXI wieku, ale dużo czasu nie minęło, by Stare
Fortuny znów znalazły się na szczycie. Od dawna ropiejące nienawiści i~świeże żale były obecnie rozgrywane w~cichej wojnie domowej, zabójstwo
polityczne tu, bomba tam, gniewna demonstracja przeciwko pośmiertnej
rehabilitacji Janet Reno gdzieś indziej.

Jadey prawie przyznała się, że jest dziewczyną od Nuworyszy, ale
prawdopodobnie nie miała większego pojęcia ode mnie, kto, za wieloma
fasadami funduszy, pośredników i~fundacji, pociąga za jej sznurki

Wstała, jak gdyby podjęła jakąś decyzję.

-- Ok -- powiedziała. -- Zróbmy to tutaj.

-- Jak to, \emph{tutaj}? -- rozejrzałem się po pustym pokoju, zaskoczony.

-- Tutaj w~\emph{Szkocji. } Masz sprzęt i~powiązania, by rozpracować tę
rzecz, cokolwiek to jest, prawda?

-- Cóż, może -- powiedziałem z~powątpiewaniem. -- Musiałbym w~temat
wprowadzić innych ludzi\ldots

-- Tak jak mówiłam, powiązania. Podejrzewam, że część z~tego, co się
dzieje to walka pomiędzy frakcjami gdzieś wysoko w~aparacie, co oznacza,
że jedna ze stron osłania nas, przynajmniej taktycznie. Posunięcia i~kontrposunięcia, wiesz? Nie wiem, czy ktokolwiek, kto za tym stoi, wie,
że jesteśmy powiązani, lub czy ktoś po prostu nie szarpie nas licząc na
wyciągnięcie większej ryby. Ale mam wrażenie, że panika i~ucieczka po
prostu zaprowadziłyby nas, i~dane, w~ręce FBB, które w~pewien sposób
jest pewnie po złej stronie, z~naszego punktu widzenia.

-- A~skąd wiesz, że \emph{moje} powiązania nie są większą rybą, którą
próbują złapać?

Camila się roześmiała. 

-- Nie wiem. Ale, no weź. Chłopaki. Mogliby po
prostu zrobić łapankę w~Darwin's Arms, jeżeli chcieli poznać Twoje
kontakty.

-- Hah! -- odparłem. -- To tylko stare geeki i~kilku szulerów. \emph{Ja}
mówię o~Websach.

-- Psach? -- spytała, zaniepokojona.

-- Nie, Związku. Robotnicy Informacji WWW, RIWWW\footnote{w oryg. IWWWW -- Information Workers of World Wide Web, nazwa wzorowana na IWW -- Industrial Workers of World zob.~\url{https://pl.wikipedia.org/wiki/Robotnicy_Przemys\%C5\%82owi_\%C5\%9Awiata}. Podobnie \emph{websy} naśladuje oryg. nazwę \emph{webblies}, która powstała ze zwyczajowej nazwy aktywistów IWW -- \emph{wobblies} - przyp. tłum.}.


Wyglądała niezdecydowanie. 

-- Słyszałam o~tym. To było coś wielkiego w~latach dwudziestych.

-- Globalny Strajk, tak, dni chwały. Dwa tysiące dwudziesty szósty, i~tak
dalej. Powinnaś posłuchać starych sióstr i~braci, jak opowiadają, że
prawie doprowadzili do upadku korporacje wielkich liter. --
zachichotałem. -- I~posłuchasz.

-- Chcesz powiedzieć, że są ciągle aktywni?

-- Nie tak bardzo, jak kiedyś, ale tak. Aż do twardego jądra starzejących
się anarchistów, rewolucyjnych socjalistów i~kilku młodych wariatów. Jak
ja.

-- Och! -- Jadey spojrzała się na mnie zabawnie. -- Więc \emph{stamtąd}
pochodzisz!

-- A~co myślałaś, że jestem patriotą?

-- To, lub kompletnie niemoralnym przestępcą.

-- Wielkie dzięki.

Uśmiechnęła się, wyglądając na szczęśliwszą w~porównaniu z~ostatnimi
czasami. 

-- Obie opcje wyglądały trochę mało prawdopodobnie w~miarę
upływu czasu, ale nie chciałam się wtrącać, na wszelki wypadek.

Oparłem się o~sofę, patrząc na nią. 

-- Teraz powinnaś mi jeszcze mniej
ufać, wiesz. Nie ma miłości pomiędzy nami Websami, a~Twoimi tak zwanymi
libertariańskimi kapitalistami.

Pomachała dłonią. 

-- Ach, to. -- Roześmiała się. -- W~ogóle Ci nie ufam,
tak właściwie, ale zaufanie nie ma nic do rzeczy, natomiast
przewidywanie tak. Wiem teraz, którą stronę prawdopodobnie wybierzesz.

-- Zobaczymy.

Energicznie wstaję, zaskakując ją i~siebie dziwnym jednoramiennym
przytuleniem i~wychodzę przez główne drzwi. Dramat zostaje zniszczony
przez moją nieudaną próbę otwarcia drzwi, Jadey zamknęła je od środka.
Staję z~boku i~pozwalam jej otworzyć.

Patrzy na mnie tuż zanim drzwi się otworzą.

-- Więc gdzie teraz idziemy?

-- Gdzieś, gdzie bezpieczniej, cieplej i~milej -- odpowiadam. -- Biuro
Związku.

-- Ojej -- Jadey mówi. -- Wiesz, jak dobrze zabawić dziewczynę.

-- Byłabyś zaskoczona -- odpowiadam.

\threeast

Budynek RIWWW na Picardie Place, naprzeciw Playhouse na końcu Leith
Walk, był nieco zgrzybiały, ale ciągle imponujący: siedem pięter betonu
i szkła, powojenny, ale nie nowa technologia. Nie wątpiłem, że był
obserwowany, ale nie sądzę, żeby obserwacja wykraczała poza rutynę.
Oficjalnie zaklasyfikowani jako ,,wrogi i~zniesławiający wobec państwa i~systemu społecznego Demokracji Socjalistycznej'', RIWWW był oficjalnie
tolerowany jako książkowy (i, co ważne, \emph{telewizyjny}) przykład jak
tolerancyjna i~pluralistyczna naprawdę jest socjaldemokracja.

Gdy przesunąłem kartę związkową nad zamkiem, Jadey wskazała oczami
slogan wyciosany rzymskimi kapitalikami nad wejściem: KLASA PRACUJĄCA I~KLASA PRACODAWCÓW NIE MAJĄ NIC WSPÓLNEGO\footnote{cytat
z preambuły konstytucji Industrial Workers of World z~1905, por.~\url{https://iww.org.uk/preamble/} - przyp.tłum.}.

-- Hmmm -- powiedziała Jadey, gdy drzwi się dla nas uchyliły. -- Co ze
wspólnym człowieczeństwem?

-- Po trzech wojnach światowych? Nie rozśmieszaj mnie.

Hol wejściowy był pusty, prócz faceta przy recepcji, który spojrzał na
nas, a~potem wrócił do książki. Wciągnąłem głęboko znajomy zapach
miejsca, gumowa wykładzina, delikatny podmuch potu i~chloru z~sali
gimnastycznej i~basenu w~piwnicy, zapach alkoholu i~dymu ziołowego z~baru na pierwszym piętrze razem z~cieplejszymi, wilgotnymi zapachami
stołówki, oraz pod nimi wszystkimi ostre nuty przewodów, plastików i~świeżego cementu z~trwającego remontu elektroniki.

Jadey również wąchała, oglądając mężczyznę w~średnim wieku i~kilka
młodych kobiet, wszyscy z~ręcznikami i~napojami schodzili na dół.

-- Nic, czego się nie spodziewałam -- przyznała, gdy szliśmy do windy. --
Przypomina mi schronisko dla młodych, lub YMCA -- Jadey uśmiechnęła się,
gdy winda dojechała. -- Młodzi Bojownicy Towarzystwa Wojny
Klasowej\footnote{w oryg. Young Militants' Class-War
Association jako wariacja akronimu Young Men's Christian Association -
przyp.tłum.}?.

-- Wystarczająco blisko -- przyznaję, uderzając w~przycisk czwartego
piętra. -- Jesteśmy bardzo towarzyscy. Możesz nawet spędzić tutaj noc.

Jadey uśmiecha się, nieobecna, do jakiegoś punktu za moim ramieniem.
Drzwi windy \emph{z trzaskiem }się zamykają. Stoimy przez chwilę w~nieskończonych odbiciach i~zmiennej sile ciążenia, potem wychodzimy.
Czwarte piętro nie jest takie życzliwe i~zwyczajne jak te, które
widzieliśmy. Długie korytarze wyłożone dywanami, ciężkie drzwi, wszystko
pod nadzorem kamer. Zapach elektryczności jest silny.

Poszedłem spokojnie korytarzem, Jadey szła ostrożnie za mną, aż
doszedłem do drzwi oznaczonych 413. Kolejne przyłożenie karty i~byliśmy
w środku. Pokój miał dziesięć metrów na pięć, brak okien, świetlówki,
wypełniony kilkoma długimi stołami, fotelami obrotowymi, klawiaturami i~ekranami. Wyglądał jak sala w~szkole albo szkolne laboratorium. Nikogo
nie było, co było ulgą. Podszedłem do klawiatury wmontowanej w~ścianę i~prędko zarezerwowałem pokój do północy: pazernie, ale nieprawdopodobne
do zakwestionowania.

-- Dobrze -- powiedziałem, siadając i~zapraszając Jadey, odpowiednio
wylewnym gestem, do zajęcia miejsca. Usiadła, podwijając nogi pod
siedzenia fotela obrotowego i~wprawiając go w~obrót.

-- Boże, co za nudne miejsce -- zauważyła. -- Nawet ściany są puste.
Żadnych obrazów, żadnych ekranów.

-- Ta, cóż, ma to swoje wytłumaczenie -- powiedziałem. -- To
\emph{firewalle} -- uśmiechnąłem się ze swojego słabego żartu.

Wyjąłem czytnik, rozwinąłem kabel i~podłączyłem do tyłu najbliższego
ekranu. 

-- Mogłabyś dać mi swój dysk?

Jadey podała.

Włożyłem go do czytnika, włączyłem ekran i~klawiaturę, wprowadziłem
hasło. Pojawiło się znajome logo Microsoft Windows 2045, żeby
natychmiast być zastąpione przez demonicznie śmiejącego się pingwina,
który zostawił słowa ,,SERIO\ldots'' znikające na ekranie, zanim pojawił się
podstawowy interfejs. Połączyłem się z~serwerem satelitarnym z~wnioskiem
o natychmiastowe połączenie download. Zabrało to około minuty, biuro
naturalnie miało antenę na dachu i~przepustowość do zużycia, po czym
oparłem się na krześle, ręce za głową, nogi na tyle blisko na ile moja
niepewna równowaga pozwoliłaby mi oderwać stopy od ziemi.

-- To załatwia problem -- powiedziałem.

-- Cudnie -- powiedziała Jadey. -- Zainteresowany wytłumaczeniem dlaczego
powinnam czuć ulgę?

-- Zależy jak dużo chcesz się dowiedzieć o~komputerach -- odpowiedziałem,
kładąc stopy na podłodze, łokcie na kolanach i~pochylając się poważnie
do przodu. Jak zwykle czułem się trochę dziwnie rozmawiając na ten
temat. Jeżeli nie byłbym ostrożny, mógłby łatwo skończyć brzmiąc jak
stary geek.

Jadey hojnie zamachała dłonią. 

-- Powiem, kiedy opuścić trudne kawałki.

-- Dobrze\ldots. Cóż, właściwie, my, to jest biznes, staliśmy się zależni
od tego, co nazywamy modelem pustych dłoń. -- Pomachałem czytnikiem. --
Jak ten sprzęt. To bezprzewodowy terminal, całkiem głupi w~standardach
systemów do których ma dostęp, które zwykle są na hardwarze daleko stąd.
Stajesz się zbyt zależna od szyfrowania, po pierwsze, oraz od dobrej
woli właścicieli serwerów, po drugie. To dokładnie ten rodzaj problemów,
które my Websy lubimy unikać. Zależność ma tendencję, powiedzmy to
sobie, do osłabiania Twojej pozycji negocjacyjnej. Zawsze byliśmy za
pracownikami kontrolującymi środki produkcji. Wynik: ten budynek ma tyle
wpakowanej mocy obliczeniowej, że nie musisz w~ogóle łączyć się na
zewnątrz z~dowolnym programem, który może być wykonywalny. -- Podrapałem
się w~głowę. -- W~każdym razie oprócz tych, które potrzebują dedykowanych
procesorów rozproszonych. Oznacza to tyle, że właśnie skopiowałem
wszystko, co mam i~wszystko, co masz, na maszyny w~tym budynku. A
najlepsza rzecz o~tym jest taka, że dane nie są dostępne z~zewnątrz. Ten
download musi przejść przez równowartość serii śluz powietrznych i~pryszniców, zanim zostanie zapisany. Nikt nie może się do nich dostać.

-- Znaczy, to miejsce jest rajem serwerowym? -- rozejrzała się dookoła, z~większym szacunkiem.

-- Nie do końca -- powiedziałem. -- Fizycznie, nie jest strasznie
bezpieczne przeciwko atakowi reverse social engineering, ale poza tym,
ta, jest całkiem bezpieczne. Teraz możemy pracować na danych całkowicie
pewni, że nikt nas nie podsłuchuje.

Jadey przechyliła głowę. 

-- A~wewnętrznie?

-- Hej, -- odparłem. -- To jest Związek. Mamy zasady przeciwko takim
rzeczom.

-- Ok -- powiedziała. -- Co dalej?

-- Zbiorę małą grupkę, żeby to zbadać. -- Obróciłem się do ekranu. -- Mam
bardzo dużo znajomości do takiej pracy.

-- Może tak -- powiedziała -- ale nie dzisiaj.

-- Co?

Jadey spojrzała się na mnie, sięgnęła i~złapała moją dłoń. 

-- No chodź. Miałam \emph{morderczo }długi dzień. Chodźmy do baru, potem
przyjmę Twoją ofertę noclegu.

Pierwszy raz dowiedziałem się o~takiej ofercie, ale nie odmówiłem.

\chapter[Wieża Kosmonauty]{5 Wieża Kosmonauty}

Elizabeth Harkness poszukiwała sukni, której potrzebowała, z~tyłu
,,Starożytnego Gałganka'', starego sklepu na Starym Mieście, popularnego
wśród studentów, ale dotąd poza terenem jej poszukiwań. Kyohvic, Tain i~właściwie generalnie Mingulay, miało przemysł tekstylno-ubraniowy, ale
nic takiego jak samonapędzający przemysł modowy. Pozostawieni sobie
samym, nikt nie wątpił, że style zmieniałyby się tak wolno w~miastach
jak zmieniały się w~wioskach wzdłuż wybrzeża. Moda stałaby się
kostiumem, ze zwykłymi osobistymi i~lokalnymi wariantami kroju i~ozdób.
Gwiezdne statki wszystko to zmieniły, ich nieregularne, ale częste
przybycia narzucały nierówną interpunkcję tej tendencji do równowagi.
Mody i~fantazje, od lat lub dekad martwe na planetach powstania,
dziwacznie kwitły na tej prowincji, aż do następnego przybycia nowych
pojęć z~nieba. Ta relacja, Elizabeth była zadowolona ze zrozumienia,
była dokładnym przeciwieństwem tego, co Gregor myślał, że było: jeżeli
kupcy nie wiedzieli jaka była moda w~Kyohvic, działo się tak, ponieważ
tworzyli ją (lub cofali w~czasie) z~każdym przybyciem.

To nie była jedyna relacja, którą Gregor błędnie postrzegał. Elizabeth
nie była pewna, czy był wyniosły, czy po prostu \emph{głupi} czy raczej
nie uważał jej za atrakcyjną czy (miejmy nadzieję) błędnie interpretował
każde jej spojrzenie i~gesty jako część ich relacji
przyjacielsko-koleżeńskiej.

Ale nie miała lepszego pomysłu niż on, co będzie Następnym Objawieniem
na balu kupców i~nie stać jej było na to, co lepiej urodzone damy miasta
mogłyby założyć, podczas gdy przyglądałyby się na to co pokazywały damy
z ostatniego statku. Stąd najlepsze co mogła zrobić, to wybrać coś tak
przebrzmiałego, że nie byłoby to niemodne. ,,Starożytny Gałganek'',
najlepszy z~takich produktów ubocznych zewnętrznie sterowanych cykli
mody na Mingulay, był miejscem do szukania.

Tak więc, kilka dni po przybyciu Statku, Elizabeth wyszła wcześniej z~laboratorium, zresztą i~tak to była sobota, poszła do domu jej rodziców
w Nowym Mieście, prędko przebrała się i~wzięła prysznic, żeby usunąć
zastarzały zapach martwego życia morskiego z~jej włosów i, szczególnie,
jej dłoni, potem wsiadła w~autobus elektryczny do wyłożonych kostką ulic
starego Kyohvic poniżej warowni Uniwersytetu.

Para obtłuczonych manekinów stała na straży drzwi sklepu. Męski manekin
był olśniewający w~starym galonowym mundurze straży z~mniejszych
majątków na Croatan, damski manekin w~lekko ryzykownej chińskiej koszuli
wzorowanej na przedostatnim statku gwiezdnym z~jakiegoś oddalonego
miejsca. Elizabeth myślała o~tym przez brawurową chwilę, potem
uśmiechnęła się do siebie i~weszła do przepastnego, jasno oświetlonego
sklepu. Sufit był przynajmniej na wysokości pierwszego piętra, w~odstępach były zawieszone grupy długich sukien. Na ścianach wisiały dwa
rzędy, jeden nad drugim, sukien i~płaszczy, mniejsze i~krótsze ubrania
były na półkach lub wisiały na przenośnych wieszakach stojących na
podłodze. Powietrze w~tym miejscu było cudowną mieszanką zapachu
starych, ale czystych ubrań, płynu do czyszczenia, dawniejszych
zapachów, mieszanek ziół i~paczuli, tlących się kadzidełek i~sporadycznego papierosa ukradkiem wypalonego przez dziewczynę na kasie.

Elizabeth wzięła głęboki wdech i~zanurzyła się szczęśliwa.
Niezauważalnie minęła godzina. Niezależnie od, dla niej, rzadko
pobłażanej kobiecej frywolnej zabawy, coś w~warstwie antyczności zapasów
sklepu odwoływało się do jej naukowej duszy. Była tutaj historia, nawet
astronomia, prawie niepojmowane szczegółowe drobiazgi rodzaju dowodów,
które mógłbyś znaleźć w~skamielinach lub powłokach wybuchających gwiazd.
Wiązki tkanin i~echa idei, które poruszały się z~prędkością światła\ldots
 Myślała o~tym częścią umysłu, podczas gdy inna część kierowała
szperaniem, oceną i~odrzucaniem.

Nic tutaj nie było starsze niż przybycie \emph{Jasnej Gwiazdy} lub
kulturalnej eksplozji, która potem nastąpiła. Wszystko tak historyczne
było w~muzeach, nie w~sklepach z~ubraniami z~drugiej ręki. Ale niektóre
z rzeczy pokazywały ślady wpływu, promieniującego z~Mingulay i~odbijającego się, podróże później, dekady, mody z~dwudziestopierwszowiecznej Ziemi: głupie detaliki, które rozpoznawała z~obrazów ze statku, jak sznurek do ściągania czy guzik do marynarki, te
trywialne niepraktyczności zdradzające swoje pochodzenie jak śmieciowe
DNA.

Historia, jak moda, była z~konieczności rozłącznym procesem w~Drugiej
Sferze. Nowe przyloty z~Ziemi były rzadkie, migracje pomiędzy planetami
w obrębie Sfery relatywnie częste. Każde z~nich mogłoby pchnąć
społeczeństwo do przodu, lub przynajmniej wytrącić z~poprzedniego kursu,
jak \emph{Jasna Gwiazda} zrobiła to z~Mingulay.

Przypuszczalnie sześćset pięćdziesiąt lat temu przodkowie Elizabeth
przybyli na Croatan. Około tysiąca osób: niektórzy pochodzenia
angielskiego, inni Indianie, ładunek Afrykanów ze statku, przy tym nie
wszyscy z~tego samego miejsca (z tego co ustalili później historycy z~częściowych dokumentów lub tradycji), czy nawet tego samego czasu. Inni,
rybacy, żeglarze i~niewolnicy uratowani z~dzikiego Atlantyku przez byty,
które niektórzy z~nich postrzegali jako aniołów, inni jako diabły,
przybywali, w~miarę upływu czasu, w~niewielkich, zdezorientowanych
grupkach. Daty ich zabrania niekoniecznie były w~takim samym porządku
jak daty ich przybycia. Po dwóch i~pół wieku życia na tym świecie,
nowszym niż Nowy Świat z~którego przybyła większość przodków, i~prób
zrozumienia tego i~innych światów z~którymi stopniowo zaczęli się
kontaktować, powstała sekta, którą większość ludzkiej społeczności
Croatan nazwała Szydercami, nazwa, którą w~końcu z~dumą przyjęli. Ich
prorokini, Joanna Tain, nauczała, że większy wszechświat został im
objawiony poprzez ich przemieszczenie i~dziwną naturę pozostałych
mieszkańców, stawiając święte pisma w~najlepszym przypadku jako
nieistotne (,,Objawienie wyłącznie dla Ludu Ziemi, jak Zakon Mojżesza dla
ludu Izraela, a~nie powszechne, jak nawet Pismo samo rzecze''), w~najgorszym -- jako fałszywe. Wpływ filozofii stoickich i~epikurejskich w~Nova Babylonia był oczywisty w~jej doktrynach, i~opłakiwany.

Krew została rozlana, i~w wyniku nagłych apeli z~obu stron, szary lud
szybko wkroczył i~ewakuował kilka tysięcy zwolenników Tain na inną
planetę, którą Szydercy nazwali Mingulay. Żyli tam przez dwa wieki,
kiedy \emph{Jasna Gwiazda} pojawiła się na ich niebie i~sprowadziła
herezje wykraczające poza najdziksze tyrady Joanny i~dowód, że kosmos
jest jeszcze dziwniejszy niż prorokini przypuszczała. Biblioteka statku
stała się fundamentem uniwersytetu i~większości nauk i~technologii, oraz
większej części kultury i~sztuki, w~rękach ludzi na Mingulay i~w~rozszerzającym się promieniu innych światów.

I stąd, Elizabeth przypuszczała, suknia, na którą w~końcu się
zdecydowała. Znalazła ją z~tyłu sklepu, pośród pęku sukienek na
drucianym wieszaku zawieszonym na tym samym haku. Zdjęła je ostrożnie,
wzięła wyglądającą na najstarszą i~odłożyła pozostałe już mniej
ostrożnie.

Suknia miała gorset z~wyszywanych liści, na satynowym jedwabiu i~w
kolorach jesiennych. Długa, szeroka spódnica z~szyfonu na sieci
ciemniejszego klosza i~sztywniejszej struktury, zanikającej i~strzępiącej się w~kierunku szwów. A~do tego krótka kurta z~długimi
rękawami ze złotymi koronkami. Zabrała ją z~triumfem za zasłonę
przymierzalni, wyszła kilka minut później, żeby zakręcić się przed
wolno stojącym lustrem. Elizabeth polubiła wygląd tak mocno, że musiała
świadomie zdecydować się nie jechać w~niej autobusem do domu.

\threeast

Jeżeli wysoki służący tuż na wejściu do Wieży Kosmonautów pomyślał, że
jej sztormiak z~kapturem był zabawny w~swojej praktyczności i~w~kontraście do jej sukni, jego wyraz twarzy, gdy odbierał pelerynę, nie
dał temu żadnego znaku. Elizabeth uśmiechnęła się do niego, podziękowała
i weszła szybko do holu wejściowego. Nie było formalnych powitań,
zapowiedzi czy przedstawiania. To nie był ten rodzaj imprezy. Zakładano,
przynajmniej było to uprzejma iluzja, że każdy, kto się pojawi, już znał
wystarczająco dużo osób, że takie powitanie było niepotrzebne.

Elizabeth nie była tego taka pewna. Czuła się zostawiona samej sobie,
znacznie bardziej wolałaby jakąś witającą osobę, żeby ją wprowadzić, gdy
przeszła z~cieni korytarza do jasnego światła głównej sali. Czuła się
niepewnie na nogach, i~nie tylko dlatego, że szpilki pantofli
(pożyczonych na tę okazję od siostry) były wyższe od tych, do których
była przyzwyczajona. Jedynym dźwiękiem, który wydawał się głośniejszy od
bicia jej serca był szelest jej sukni, jak dźwięk chodzenia po stosie
suchych liści.

Wielka sala była zaprojektowana na zimny bufet i~na tańce, z~długimi
stołami, krzesłami i~ławami wzdłuż dwóch ze ścian i~stołami zastawionymi
jedzeniem i~piciem wzdłuż trzeciej, z~muzykami, w~tym momencie
strojącymi instrumenty, w~rogu stworzonym przez tę ścianę i~ścianę na
której stał kominek. Przez kilka sekund, Elizabeth zatrzymała się w~drzwiach wejściowych, gapiąc się na gigantyczne dekoracje na ścianach.
Skala wszystkiego, ścian, dywanów, łbów zwierzęcych, była dwa razy
większa od ludzkiej skali. Nawet portrety były wielkie i~wysoko. Potem
stanowczo weszła dalej, z~ulgą zauważając, że około setka osób już
przybyła, więc nie była żenująco wcześnie lub późno. Kilku z~nich,
starego Kosmonauta Cairns, rektora Uniwersytetu, właściciela Mueller's
Mill, Elizabeth rozpoznała z~daleka, a~jednego lub dwóch znała
osobiście. Gregora, jak dotąd, nie było pośród nich, co też było ulgą, w~pewien sposób.

-- Och, witaj, Harkness. -- Mark Garnet, kierownik wydziału biologii morza
zawołał do niej i~przyzwał ją do zgromadzenia akademików przy stole z~drinkami. Niski, raczej grubas z~przylizanymi, odrzuconymi do tyłu,
ciemnymi włosami, nieodparcie przypominał jej fokę, Garnet był
prawdopodobnie jej najlepszym przyjacielem wśród pracowników, zawsze
pomocnym z~zawiłym problemem statystycznym czy niejasnym odnośnikiem.

-- Na co masz ochotę? -- spytał Garnet.

Elizabeth spojrzała na szereg szklanek. 

-- Ach, na razie, myślę, białe wino.

-- Dobrze, bardzo dobrze -- podał jej szklankę i~pomachał do szczupłej,
prawie chudej kobiety przy jego boku. -- Dobrze, Harkness, to moja żona,
Judith.

-- Miło mi poznać. -- Obie kobiety dygnęły uprzejmie. Suknia Judith była
dopasowana i~elegancka, nie nowa, ale też nie z~drugiej ręki, jej włosy
zwinięte i~ułożone.

-- Muszę Cię ostrzec -- kontynuował Garnet -- że Judith kompletnie nie jest
zainteresowana biologią, więc nie rozmawiajmy o~pracy.

Zabrali szklanki i~usiedli w~rogu jednego ze stołów, plotkując leniwie o~polityce Wydziału i~obserwując rosnący tłum. Coraz więcej osób
przybywało: mistrzowie gildii i~czeladnicy, przemysłowcy i~inżynierowie,
herezjarchowie w~wysokich czarnych czapkach i~Szydercy w~czarnych
garniturach z~szerokimi białymi kołnierzami, z~mieczami u boku.
Międzygwiezdny kupiec, jego rodzina i~świta weszli szczególnie
imponująco. Mężczyźni mieli na sobie długie lniane płaszcze z~wyszywanymi kamizelkami na koszulach i~bryczesami, kobiety luźne suknie
w różnych perłowych lub pastelowych odcieniach satyny. Konstrukcja
sukien była prosta, dekoracja wyszukana, z~symbolami wieku i~statusu
sugerowanymi przez różnice w~długości, braku lub posiadaniu kołnierzy i~rękawów.

-- Nieźle -- zauważyła Judith, sotto voce. -- Wygląda, że Nova Babylonia
ciągle tworzy swój własny styl. Przynajmniej na razie. Nasi przodkowie
naprawdę nie powinni wypuszczać z~rąk tych podręczników historii mody.
Niebiosa wiedzą tylko jakie mikstury są teraz wymyślane ze starych pudeł
na ubrania Ziemi.

Elizabeth uśmiechnęła się do niej. 

-- Zabawne, myślałam bardzo podobnie, kiedy szukałam sukienki na dzisiejszą imprezę.

-- Hmmm, no nie wiem -- powiedział Mark. -- Nie wydaje mi się, żeby Nova
Babylonia miała czas na bycie pod wpływem. To znaczy, jaki jest czas
zwrotu?

-- Krótszy niż dziwactwo, tam -- odparła Judith. Brzmiała nieco
potępiająco. Nova Babylonia miała fantastyczną reputację, trwającą od
czasu pierwszych kontaktów z~Croatan, jako ziemia luksusów, prawie
dekadencji, wizerunek zintensyfikowany przez względną rzadkość kontaktów
i wiarygodnych informacji. Stuletnie wiadomości byłyby chętnie podjęte
na tej imprezie, i~w negocjacjach handlowych i~omawiane przez lata, lub
póki nie przybędzie następny statek. Było wątpliwe, żeby ta
częstotliwość wzrosła, teraz gdy kupcy Nova Babylonia aktywnie
odpowiadali na nowe wypadki w~kolonii, raczej niż odkrywali je po raz
pierwszy. Ale nieważne jak często by przybywali, wiadomości zawsze
byłyby stuletnie\ldots.

-- W~każdym razie -- powiedziała Judith -- wyciągnęłaś fajną sukienkę.
Powiedz mi, gdzie ją znalazłaś?

Elizabeth uśmiechnęła się z~boku do Marka. 

-- Teraz \emph{porozmawiamy} o~pracy -- odpowiedziała Elizabeth. Mark machnął na nie ironicznie i~odszedł po jedzenie i~informacje.

\threeast

Gdy Gregor wszedł do sali, poczuł się lekko nieśmiały w~odziedziczonym
ubiorze czarnej aksamitnej kurtce, białej koszuli i~wąskich czarnych
spodniach, ale nikt mu się nie przyglądał. W~wewnętrznej kieszeni
kurtki, raczej psując fałdy, tkwiła wiązka złożonych papierów nad
którymi męczył się cały dzień i~większość poprzedniej nocy. Jego
dziadek, James, dołączył swoje żądanie, tak władczo jak zwrot podatku,
do zaproszenia.

Gregor znał salę i~zamek z~wakacyjnych wspomnień z~dzieciństwa i,
znacznie rzadziej, dojrzewania. W~tych dniach, choć ciągle lubiąc
dziadków, odwiedzał zamek znacznie częściej w~snach. Jego ojciec, pomyślał Gregor,
mądrze zrobił to, co większość potomków oryginalnej
załogi zrobiła i~wyprowadził się, wykreślając się samodzielnie, jako
nastolatek, kilka lat, zanim Gregor się urodził. Teraz, jako właściciel
poważnej floty rybackiej, Frederick Cairns był zadowolony, że jego
pierworodny syn podzielał jego entuzjazm do firmy i~szorstko tolerował
bardziej akademickie zainteresowania życiem morskim u drugiego syna. Dla
znacznie bardziej abstrakcyjnych zainteresowań i~motywacji swojego ojca,
Frederick okazywał tylko pogardę.

Przy bufecie, Gregor wybrał mały stosik mięczaków i~skorupiaków,
odrobinę przypraw, trochę warzyw i~łyżkę ryżu. Wziął także szklankę
czerwonego wina zgodnie z~rozsądną teorią, że wino było szybszym
sposobem na odurzenie i~rozejrzał się raczej ogólnie za kimś, kogo znał
lub miejscem, żeby usiąść. Niektóre z~lokalnych dzieci, wyglądając
sztywniej i~dziwniej w~swoich formalnych ubraniach niż Gregor, gapiły
się i~popychały w~odpowiedzi na widok zaurów. Rzadkością było, w~Kyohvic, zobaczyć ich prawie trzydziestu w~jednym miejscu, a~możliwość
gapienia się i~szeptania chętnie była wykorzystywana, bardziej
dyskretnie, przez dorosłych.

Gregor odwrócił się, zażenowany tym wiejskim zwyczajem i~stanął twarzą w~twarz z~młodą kobietą, którą natychmiast ocenił jako najpiękniejszą
osobę, jaką widział w~swoim życiu. Jej skóra miała kolor bursztynu, jej
długie, falujące włosy, wielkie ciemne oczy jasnego mahoniu, wszystko to
podkreślone przez promienny róż prostej, miękkiej sukni, która
delikatnie się marszczyła i~subtelnie podkreślała kształy jej ciała. Z~pełnymi dłońmi, tak jak on, talerza, szklanki i~ostrożnie ułożonymi
sztućcami, patrzyła na niego z~czarującą bezradnym, ale jakoś
samozwańczym urokiem. Wyglądało jakby miała coś powiedzieć, ale się
wahała. Gregor czuł się tak samo. Coś, przypuszczalnie jego serce,
skakało w~środku.

Gregor uśmiechnął się, przerażające szczęki, i~wziął szybki łyk wina,
żeby jego usta przestały być kompletnie suche. 

-- Dobry wieczór -
powiedział. -- Zastanawiasz się, gdzie usiąść?

-- Tak, racja -- odpowiedziała. -- Właściwie -- roześmiała się lekko i~wzięła łyk swojego wina, białe -- bardzo stanowczo powiedziano mi, żebym
wmieszała się pomiędzy tu\ldots, hm, tubylców i~ćwiczyła język, a~nie jestem
pewna z~kim najpierw rozmawiać.

-- Cóż -- powiedział Gregor -- może chciałabyś usiąść i~porozmawiać ze mną,
przez chwilę?

-- Och, tak -- odpowiedziała, nagle brzmiąc pewnie. -- To byłoby bardzo
utylitarne.

Gregor wskazał oczami puste miejsce przy stole po drugiej stronie sali i~podążył za nią, zdumiony, że nikt ze zgromadzonych nie patrzy na jej
każdy krok i~migotanie tak bardzo jak on. Ale wtedy, jak jakaś
mamrocząca, naukowa, oderwana część jego mózgu, małpa na plecach,
podkreśliła, właściwie nie \emph{widział} reszty imprezy, więc problem
był dyskusyjny.

Usiedli, częściowo zwróceni do siebie na ławce. Kolejny moment braku
pojęcia co powiedzieć. Gregor wskazał kciukiem na swoją pierś. 

-- Gregor Cairns.

-- Nazywam się Lydia de Tenebre -- uroczyście powiedziała dziewczyna, a~potem bardziej od niechcenia -- kupca Esias de Tenebre siódma córka,
trzecie dziecko jego drugiej żony. Mam dziewiętnaście lat i~urodziłam
się -- machnęła dłonią -- och, setki, setki lat temu.

-- Muszę powiedzieć, że nie wyglądasz. -- Jak tylko to powiedział, Gregor
był pewien, że zabrzmiał głupio, pierwsza myśl, która wpadła mu do
głowy, ale Lydia się roześmiała. Lydia odrzuciła włosy do tyłu i~w tym
momencie z~opóźnieniem Gregor zrozumiał, że został porażony, ,,trafiony
strzałą Erosa'' jak mawia poeta, i~że jej obiektywny wiek był nie tylko
najdziwniejszą, ale także najważniejszą informacją o~niej: tą, która
wisiałaby nad wszystkim, co mogłoby się zdarzyć pomiędzy nimi.

-- Teraz, czy mógłbyś powiedzieć mi o~sobie i~swojej rodzinie? -- spytała
Lydia, jak gdyby to był kolejny punkt w~protokole.

-- Uczę się, żeby zostać biologiem morskim -- odpowiedział Gregor. -- Mój
ojciec jest rybakiem, a~moja matka uczy dzieci. Mój dziadek, tam dalej,
ma mniej więcej dziedziczną pozycję Nawigatora.

-- Co te stare osoby robią, te, które nazywają się Kosmonautami? Jak
żyją? -- Lydia rozejrzała się dookoła, po ścianach i~suficie. -- Jak mogą
sobie pozwolić na\ldots to wszystko? Nie są kupcami. Czy są władcami?

Gregor potarł się po karku, z~jakimś poczuciem bycia obrońcą. 

-- Nie, nie
do końca, prawdziwymi władcami tutaj są Herezjarchowie, choć Driver, ten
wielki facet rozmawiający z~moim dziadkiem, jest bardzo wpływowym
człowiekiem. A~Ochrona --  Podrapał się po głowie, gdy Lydia zmarszczyła brwi. -- On kierują
policją. Początkowo była tylko dla zamku i~uniwersytetu, ale teraz jest
dla całego miasta.

-- Dobrze -- odpowiedziała, jak gdyby rozumiała, ale prawdopodobnie nie
zrozumiała -- a~co z~resztą?

-- Są potomkami załogi starego statku gwiezdnego. Każde pokolenie ma
kilka osób, które nominalnie zajmuje pozycję, które ich przodkowie mieli
w oryginalnej załodze. To tradycja. Pierwsza załoga zajęła ten zamek,
ponieważ lokalni mieszkańcy opuścili go, i~ponieważ jest to, cóż, zamek!
Łatwy do obrony. Kiedy tu przybyli, mogli sprzedać istniejącej populacji
wiedzę i~technologię, potem zaurom i~potem kupcom z~innych światów. W~końcu założyli uniwersytet, który stał się centrum badawczym dla
lokalnego przemysłu, rolnictwa i~rybołówstwa. I~nie tylko lokalnego
przemysłu, mamy studentów i~naukowców z~najbliższych światów,
szczególnie Croatan. Tytularna załoga ciągle z~tego uzyskuje dochód, i~z~innych inwestycji, oraz je nadzoruje. Dziedziczna synekura. Niezbyt
wymagająca praca, ale ją robią, żeby podtrzymać ciągłość ze statkiem i~z~Ziemią.

-- Ciągłość, która teraz została zagubiona, tak?

Przebiegłe pytanie, i~nie takie na które byłoby polityczne dać pełną
odpowiedź.

-- Tak, została zagubiona. Mój ojciec nie ma zamiaru zostać Nawigatorem,
ja też nie. Ale mój dziadek, mam nadzieję, ma wiele lat przed sobą, a~jeżeli będę miał synów, jeden z~nich może zechcieć zająć jego miejsce.

-- Lub córki?

Policzki Gregora zapłonęły. 

-- Oczywiście, tak.

Ciemne oczy Lydii się zaiskrzyły, może rozbawione jego widoczną porażką.

-- A~co pociągałoby za sobą bycie Nawigatorem?

-- Och, pokażę Ci -- powiedział Gregor. Wyciągnął złożone papiery z~kieszeni i~rozłożył je na stole. Dokumenty były pokryte zawiłymi
symbolami i~pracowitymi wykresami.

-- Kilka miesięcy temu James, mój dziadek, przysłał mi ten problem.
Dotyczy logiki i~matematyki, w~żadnym nie jestem zbyt dobry. Ale
pracowałem nad tym, z~przerwami, aż do wczorajszego wieczoru, gdy James
przysłał mi wiadomość z~prośbą o~dostarczenie rozwiązanie dzisiaj
wieczorem. -- Gregor smutno popatrzył na papiery. -- Straciłem cały dzień,
ale przynajmniej jest to zrobione.

Lydia szturchnęła krawędź papierów swoim doskonale owalnym paznokciem. 

-- Nie macie do tego maszyn liczących?

\emph{Ostrożnie, teraz ostrożnie!}, pomyślał
Gregor.  

-- Mamy, oczywiście, mamy maszyny
liczące. Ale nie wszystko obliczenia można wykonać na maszynach. -- To
był etyczny komunał w~całej Drugiej Sferze.

-- Rzeczywiście nie -- odpowiedziała uroczyście Lydia. -- Ale jeżeli nie
radzisz sobie z~tym, dlaczego nie dać tego problemu osobom, które sobie
poradzą?

\emph{Wskazałaś swoim ładnym małym palcem istotę problemu}. 

-- Ach, cóż, celem ćwiczenia jest utrzymanie pewnych umiejętności
aktywnych w~obrębie rodziny.

Co było prawdą, choć z~innymi naciskiem dla osoby, której tłumaczył.
Wyglądało, że wyjaśnienie ją satysfakcjonuje, przyglądała się stronom
przez moment i~właśnie miała sięgnąć po nie, gdy śpiesznie podszedł
James Cairns. Gregor wstał i~uściskał go.

-- Nie pokazuj tych rzeczy tutaj! -- wysyczał w~ucho Gregora James.

-- Nie pokazywałem! -- zaprotestował Gregor w~ramię dziadka. Odwrócił się
do Lydii.

-- Lydio, to jest James Cairns, Nawigator, mój dziadek.

Gdy Gregor przedstawił dziadka, starszy pan ukłonił się i~pocałował dłoń
młodej dziewczyny, po omacku sięgnął wolną rękę ponad stołem i~wcisnął
obliczenia do kieszeni. Lydia, patrząc prosto na Gregora nad głową
Jamesa, obserwowała ten manewr z~wymuszonym uśmiechem. Gregor pokrył
swoje zażenowanie mrugnięciem oka.

James usiadł, jedno miejsce od Lydii. Przynajmniej nie usiadł pomiędzy
nimi.

-- Zatem, Gregor, jak sobie radzisz tam, na dzikim oceanie?

-- Och, dobrze -- Gregor zaczął opowiadać ostatnie przygody, i~choć były
drobne, sprawiły, że Lydia patrzyła na niego z~rozchylonymi ustami, a~jej brwi od czasu do czasu drgały. James słuchał i~obserwował coraz
bardziej dziwacznym spojrzeniem.

-- Jakieś nowe informacje o~kałamarnicach? -- spytał, gdy Gregor zakończył
historią ostatniego wieczory, gdy statek, statek rodziny Lydii,
nadleciał.

Gregor wzruszył ramionami. 

-- Same obserwacje, jak ta. Opiszę to, cóż,
jedno z~nas, które tam było, opisze. Na tyle, na ile jest warte.

-- W~rzeczy samej. -- James zastanawiał się nad małą szklanką alkoholu,
którą ze sobą przyniósł i~zapalił skręta. Oczy Lydii rozszerzyły się,
gdy jej go przekazał. Lydia sączyła, raczej niż wdychała, dym i~podała
to prędko Gregorowi. Gregor zaciągnął się głęboko i~przekazał skręta z~powrotem dziadkowi.

-- Właściwie -- uśmiechnął się James -- byłem ciekaw Twoich badań tych
maleństw. Nie krakenów! Jak stary Matt zwykle mówił, one są ciągle
jebanym fortowskim fenomenem.

Gregor spojrzał na niego, a~starzec odwrócił się przepraszająco do
Lydii. 

-- Wybacz mój język -- znowu podał jej skręta i~znowu Lydia tylko
wciągnęła dym do ust, wydmuchała i~podała Gregorowi.

-- Rozumiem ,,jebany'' i~''fenomen'' -- powiedziała. -- Ale co to jest
,,fortowski''?

-- Miałem zadać to samo pytanie -- powiedział Gregor, dobre wychowanie
przeważyło nad irytacją z~powodu wulgarności starca.

-- Ach -- odpowiedział zadowolony James, kiwając się do przodu i~tyłu gdy
kończył teraz przygasającego jointa. -- Zgodnie z~zapisami naszego statku
gwiezdnego \emph{Jasnej Gwiazdy}, ludzie na Ziemi doświadczali wielu
fenomenów, których nie potrafili wyjaśnić, których nie potrafili, ktoś
mógłby powiedzieć, zrozumieć, które były zebrane przez Charlesa Forta, i~stąd takie zjawiska zaczęły być nazywane ,,fortowskimi''. Te zjawiska
obejmowały, pozwolę sobie dodać, naszych przyjaciół zaurów i~ich łodzie
grawitacyjne i~z tego co wiem, ich przeciwieństwo, ogniste statki
gwiezdne. Wybacz, Lydio. Oraz dziwne potwory morskie. Obecnie, na tyle
na ile wiemy, przeklęte krakeny lub kałamarnice lub
\emph{Archi}pieprzone\emph{teuthys} nadal są dziwnymi potworami
morskimi. Ale może są bardziej znane Tobie i~Twojej rodzinie, co?

Lydia złożyła policzek na splecionych palcach i~przeniosła spojrzenie od
Jamesa do Gregora i~z powrotem. Jedynym efektem zioła na nią było to, że
widziała śmieszną stronę tej oczywistej próby wyciągnięcia z~niej
informacji.

-- Och, tak -- odpowiedziała. -- Jesteśmy z~nimi\ldots w~komunii. Zaury są z~nami do tłumaczenia, oczywiście, ale wierzymy, że rozmawiamy z~krakenami
-- uśmiechnęła się złośliwie do Jamesa. -- Są naszymi prawdziwymi
,,nawigatorami''. Mój ojciec był pod wrażenie, że udało wam się dorównać
ich dokonaniom.

-- Ale nie stopom!-- Gra słów Jamesa nie trafiła do niej, ku uldze
Gregora, i~w tym momencie orkiestra w~końcu zaczęła grać. Gregor wstał i~wyciągnął dłoń do Lydii.

-- Czy mogę mieć ten zaszczyt?

-- Oczywiście, dobrze. -- Lydia wstała, przeszła nad ławką w~różowym
trzepocie i~dygnęła, gest z~którym Gregor nie był obeznany, ale który
uznał za czarujący.

-- Bawcie się dobrze -- powiedział James dobrodusznie zamroczonym głosem.
Rzucił Gregorowie ostre spojrzenie, całkowicie trzeźwy. -- Pogadamy
później.

\threeast

Pierwszy taniec był umiarkowany i~formalny, bardziej typowy dla tego
rodzaju przyjęć niż rodzime tradycje Mingulay, a~Gregor zauważył, że
błędnie tańczy. Ale jeżeli Lydia zauważyła, to nie jej to nie
przeszkadzało, a~po paru minutach stateczny rytm tańca umożliwił im
kontynuowanie rozmowy.

-- Czy Ty i~rodzina rzeczywiście mieszkacie na Statku?

-- Och, nie. -- Obrót. -- Wynajmujemy willę na Nova Babylonia. -- Dwa kroki
do tyłu. -- Na innych światach polegamy na gościnności, lub na hotelach
kupieckich. -- Krok do przodu, półobrót, podnieść dłoń, wziąć dłoń,
odwrócić się. -- Statek jest zaprojektowany dla krakenów, nie dla nas.
Większość wnętrza jest permanentnie zalana. -- Uśmiech, zmarszczenie
nosa. -- I~to pachnie. -- Puścić, dwa kroki do tyłu. Podać obie dłonie,
dwa kroki do przodu. -- Rybami. -- Podanie dłoni.

Ciepłe dłonie, delikatne, kościste, trzepoczące i~zatrwożone jak małe
nietoperze śpiewające. Muzyka ucichła. Ona dygnęła, on się ukłonił.

-- Więc gdzie na statku -- spytał Gregor głosem, który dla niego brzmiał
raczej nienaturalnie -- właściwie podróżujecie?

-- W~skifach, oczywiście. Lecimy nimi do statku, który może być na morzu
lub w~kosmosie, zależnie od rejonu, a~potem ze statku, tak jak widziałeś
tamtej nocy. Czasem musimy brnąć dookoła statku, żeby sprawdzić i~zabezpieczyć skrzynie. Więc na podróże ubieramy się roboczo, nie tak jak
teraz. -- Lydia skubnęła spódnicę, uśmiechnięta. -- Ale to wszystko przed
i po. Podróż nie zabiera czasu. -- Pstryka palcami. -- Tak po prostu.

Usiedli znowu koło ich porzuconych talerzy. James wywędrował w~tłum,
może taktycznie zostawiając ich samych. Gregor był niekomfortowo
świadomy, że mógłby wrócić, lub mógłby wziąć go na osobności gdzieś i~podzielić się jakąś ważną rodzinną wieścią. Lydia zaczęła jeść, szybko i~zręcznie, przeplatając jedzenie rozmową. Gregor żuł znacznie wolniej.

Mniej przyzwyczajony to socjalizowanie się tego typu, przez większość
czasu był zredukowany do gestów, kiwnięć i~pomruków, gdy słuchał Lydii
opowiadającej o~innych światach Sfery. Nigdy wcześniej nie uderzyło go
tak mocno jak dużo podobieństw było podporą ich różnorodności, wszystkie
te różne słowa zapisane tym samym alfabetem DNA. Nie byli bardziej
różni, ostatecznie, niż kontynenty na jednej planecie, lub raczej ich
różnice były rozwinięciem izolacji reprodukcyjnej, jak gdyby kontynenty
i oceany były cechami jednego olbrzymiego świata. Modelem, i~wspólnym
pochodzeniem organizmów, była odległa, niedostępna Ziemia.

Kiedykolwiek jadł owoce morza, czyli często, myśl, że jest coś
delikatnie złego w~jedzeniu kałamarnic, nawet takich jak te leżące na
talerzu, małe, nieświadome krewne lub przodkinie krakenów, pojawiała się
w jego głowie. To było prawie tak złe jak jedzenie małp. Ale w~cieplejszych klimatach czy światach, ludzie jadali małpy. Również
jaszczurki, jeżeli o~to chodzi. Doszedł do wniosku, że była to myśl,
której wypowiedzenie byłoby niegrzeczne.

Lydia musnęła usta chusteczką i~spojrzała na pusty kieliszek.

-- Jeszcze raz?

-- Tak, proszę. Białe.

Gregor ruszył przez teraz znacznie gęstszy tłum, koło znacznie żywszych
tańców, na krawędzi których odbywały poważne dyskusje pomiędzy
odwiedzających kupcami a~ich lokalnymi odpowiednikami. Czuł drżenie
trwale promieniujące na zewnątrz ze splotu słonecznego, cel strzały.
Odurzony medytował, że się zakochał, najbardziej niefortunne
doświadczenie, ale takie, które jak każda inna choroba mogło być tylko
albo przetrwane albo pokonane. Tak nauczali Szydercy, ale w~tej chwili,
ale to właściwie też był symptom, Gregor nie był w~stanie wyobrażać
sobie siebie jak tylko jako wyznawcę.

Gdy mijał jeden ze stołów, zauważył trzecią kuzynkę Clarissę, która
siedziała nieświadomie karmiąc najnowsze dziecko, więc zatrzymał się,
żeby powitać i~jej pogratulować. Rozmawiali przez kilka chwil,
wymieniając rodzinne plotki, podczas gdy Gregor podziwiał niemowlę.

-- Jakie ona ma cudowne małe paluszki -- powiedział łaskocząc je pod
koronkowym brzegiem.

-- Dlaczego wszyscy mężczyźni to mówią? -- spytała się, uśmiechając,
Clarissa. -- I~to chłopiec. Owen. To jego szatka na nawilżenie. -
Spojrzała na niego z~nadzieją. -- Ceremonia jest jutro, w~Domu Spotkań
przy North Street. Chciałbyś przyjść?

-- Zrobię co mogę, Clarisso -- odpowiedział, i~po chwili pogawędki, Gregor
udał się do stołu z~drinkami.

Właśnie podnosił wypełnione kieliszki, kiedy jego łokieć został złapany.

-- Chwilka, Gregor.

-- Och, witam ponownie, Dziadku.

Stary człowiek się uśmiechnął. 

-- Nie rób ze mnie starca. Jesteś
dostatecznie dorosły, by mówić mi James.

Gregor skłonił głowę. 

-- Będę pamiętał.

-- Widzę, że się śpieszysz -- powiedział Nawigator, ciągle trzymając
łokieć Gregora jak Stary
Marynarz\footnote{prawdopodobnie odniesienie do poematu
Samuela Taylora Coleridge Rymy o~Starym Marynarzu, zob.~\url{https://pl.wikipedia.org/wiki/Rymy\_o\_starym\_marynarzu\_(poemat)} -
przyp.tłum.}. -- Więc powiem to szybko.

Tak jak podejrzewał, Gregor odkrył, że stoi plecami w~rogu.

-- Tak, proszę -- powiedział, trzymając dwa kieliszki i~wymownie nie
pijąc.

-- Ta dziewczyna z~którą rozmawiasz, na litość boską, tylko nie napomknij
jej o~rodzinnej sprawie, czym jest właściwie Wielka Praca.

-- Nic nie powiedziałem. -- Gregor przez chwilę się zastanawiał. -- A~poza
tym i~tak nic nie wiem.

-- Dobrze. -- James chytrze się uśmiechnął. -- Spojrzałem na to, co mi
oddałeś, Gregor i~muszę powiedzieć, że to dobra robota. Cholernie lepsza
niż cokolwiek, co przynieśli Twoi wujowie i~kuzyni. Teraz, musimy pilnie
z tym ruszyć. Wiem, że masz własne badania i~obowiązki, ale czy jest
szansa, żebyś poświęcił trochę czasu, żeby mi pomóc?

-- Och, myślę, że tak -- odpowiedział ostrożnie Gregor. James oczywiście
uznał zgodę za bardziej wiążącą, niż była.

-- Dziękuję -- powiedział. -- Spróbujmy jutro. -- Mimowolne przerażenie w~oczach Gregora napotkało kolejny chytry uśmiech Jamesa. -- A~to dałoby Ci
powód, żeby przyjść do Zamku.

-- Cóż, skoro stawiasz sprawę w~ten sposób\ldots -- Gregor zmarszczył brwi
na chwilę. -- Wiesz co. Zaaranżuję spotkanie z~Lydią, jeżeli będzie
chciała mnie widzieć, i~wtedy przyjdę kilka godzin wcześniej, żeby się z~Tobą spotkać. Prawdopodobnie późnym rankiem, idę na namoczenie
ostatniego dziecka Clarissy, o ile jestem na to gotowy.

-- Doskonale! -- James w~końcu puścił jego ramię i~Gregorowi udało się
uciec.

\threeast

Elizabeth, po taktownym i~uprzejmym wyplątaniu się z, kolejno,
towarzystwa Garnetów, głębokiej rozmowy z~Tharovarem i~Salasso, i~tańcem
z jednym z~kuzynów Gregora, z~którym dwa lata temu miała przygodą
miłosną, w~końcu zobaczyła przejście Gregora bokiem sali. Udała się za
nim, ale gdy w~końcu go znowu zobaczyła, Gregor z~powrotem był przy
stole i~rozmawiał z~Lydią, a~nie zabrało jej to nawet dwóch sekund
obserwacji obojga twarzy, żeby zrozumieć, że kompletnie się spóźniła.

Odwróciła się, zanim którekolwiek mogło ją zobaczyć. Nie żeby było to
duże ryzyko. Nie widząc reszty tłumu, tak jak oni, Elizabeth wyszła, na
początku wolno, potem przyśpieszając po tym, jak złapała swój płaszcz od
służącego przy drzwiach. Nie padało i~nie była to zimna noc, jak na
wiosnę, ale naciągnęła sztormiak, zapinając guziki i~obejmując się
ramionami, gdy oddalała się od światła i~muzyki, ciężar płaszcza
miażdżył pod nim delikatną suknię, ale nie dbała o~to. Jej stopy bolały
w butach, gdy klekotała wzdłuż długiego podjazdu Zamku, ale nie dbała o~to. Do piekła z~tym wszystkim, buty i~bryczesy byłyby wystarczająco
wygodne. W~niczym innym nie spotkałaby już Gregor.

Statek gwiezdny świecił się w~wodzie niczym zniekształcony księżyc. Nie
nienawidziła tej dziewczyny, zbyt eleganckiego, delikatnego i~niewinnego
stworzenia do nielubienia. Nie, to Gregor, niewidzący, nieczuły bękart,
jej zainteresowanie każdego dnia prosto w~jego twarz, a~on odpowiedział
na to przyjacielską poufałością, jak gdyby była jednym z~kolegów.
Nienawidziła go.



\chapter[Zaufane Osoby Trzecie]{6 Zaufane Osoby Trzecie}


-- Wiesz -- powiedział Jason, żując smażony plasterek i~rozglądając się po
stołówce Związku -- to miejsce nie jest strasznie zabezpieczone. W~sensie, fizycznie. -- Pomachał dłonią w~kierunku szerokich okien, jedno z~nich było podparte i~lekko uchylone.

-- Mów mi jeszcze -- odparła z~przekąsem Jadey. -- W~domu, biuro związku,
lub cokolwiek opozycyjnego jak to, byłoby bardziej \emph{obronne}.

Zorganizowaliśmy na godzinę jedenastą rano spotkanie biznesowe jako
śniadanie w~pracy, na które zaprosiłem Jasona, Tony'ego i~Aleca Currana,
wszyscy mieli użyteczne umiejętności, było mało prawdopodobne, że nas
zdradzą i~sami byli Websami. Nie, że istnieje jakaś konieczna korelacja
pomiędzy tymi trzema faktami. Przedyskutowałem najpierw wybór dwóch
starych programistów z~Jasonem, przez telefon, ostrożnie, i~zapewnił
mnie, że to rozsądny wybór.

Stołówka była całkiem głośna o~tej porze ranka, w~większości przez
pracowników wsparcia i~ochotników, Związek był dumny, że nie miał nikogo
na etacie ani oficjalnie nie płacił. Ci, którzy nie byli pochłonięci
własnymi dyskusjami, obserwowali ekrany na ścianie, gdzie jakiś
prowadzący dziennej telewizji zorganizował debatę pomiędzy Papieżem, z~jego domu w~Rzymie, a~Przewodniczącą Głównego Zgromadzenia Kościoła
Szkocji z~jej domu w~Harare, Zimbabwe. Próby gospodarza, by podważyć
pewne teologiczne różnice w~kwestii życia kosmitów odbijały się od
chwalebnie stanowczego zjednoczonego frontu chrześcijan. Kościół, raz
zgromadzony, zawsze wierzył, że nadludzkie, ale nie boskie, inteligencje
żyją w~niebiosach ponad nami.

Curran wykonał niebezpieczny gest widelcem, żeby utrzymać naszą uwagę,
zanim udało mu się przełknąć. Odwrócił się do Jadey. 

-- To jest tak -- powiedział irytującym tonem cierpliwego wytłumaczenia. --
\emph{Moglibyśmy} zamienić to miejsce w~fortecę, ale co by to dobrego
dało? Jeżeli państwo kiedykolwiek chciałoby nas złamać, sprowadziłoby
przeważające siły. Nie istnieje sposób, żebyśmy mogli pokonać państwo w~grze w~przemoc. Przemoc jest tym, w~czym są dobrzy. To, w~czym nie są
dobrzy, to rozpowszechnianie idei, a~to ostatecznie idee w~głowach ludzi
powodują, że decydują, czy użyć broni w~ręku. Państwo jest dobre w~użyciu siły, ale nie w~uzasadnianiu siły. Więc jak długo większość ludzi
wierzy, że nie robimy krzywdy i, że powinniśmy być zostawieni w~spokoju,
mamy duże szanse na pozostawienie w~spokoju. Zamiana w~uzbrojony obóz
uderzyłaby w~nas, nawet zakładając, że oni by nam na to pozwolili.

-- Myślałem bardziej o~fizycznej infiltracji -- odparł łagodnie Jason. -- Inteligentny pył i~takie tam.

-- Dodatnie ciśnienie -- odpowiedziałem. -- Okna dmuchają, ale nie
wpuszczają. Poza tym to miejsce ma urządzenia zagłuszające, nie
najlepsze, ale też nie takie złe. -- Uśmiechnąłem się do Jadey. Jej palce
badały górę mojej stopy pod stołem. -- Dostatecznie dobre dla prac
rządowych. Nie, naprawdę, myślę, że naszym jedynym problemem jest
odwrócony social engineering, a~jak Alec mówi, to problem polityczny, a~nie fizyczny. Jesteśmy tu bezpieczni jak gdzie indziej.

Jadey wyglądała na niezdecydowaną. 

-- Na pewno są miejsca, gdzie możesz
mieć trochę więcej prywatności? Na północy w~Highlands\footnote{górzysty region północnej Szkocji,
por.~\url{https://pl.wikipedia.org/wiki/Highlands\_(Szkocja)} - przyp.tłum.},
może?

Curran prawie się zakrztusił. Reszta po prostu się uśmiechnęła. 

-- Highlands są najgorsze -- powiedział Curran, kiedy odzyskał oddech. --
Reforma ziemska kupiła Partii naprawdę dużo poparcia.

-- Och, dobra. -- Jadey oddaliła sprawę. -- Zdaje się, że pokój komputerowy
wygląda dostatecznie bezpiecznie. Robimy to stamtąd, prawda?

-- Prawda.

-- Więc w~końcu co robimy stamtąd? -- spytał Tony.

-- Właściwie -- powiedziałem, bardzo ostrożnie -- robimy pracę w~ramach
kontraktu, który dostałem wczoraj. Potrzebuję trochę pomocy od was
trzech i~wszyscy jesteście na pod-zleceniu na standardowej stawce, jeżeli
chcecie. -- Pomachałem dłonią. -- Później ustalimy zakres prac. Na razie
ok?

Kiwnięcia głową dookoła.

-- Doskonale. -- ciągnąłem dalej. -- Niemniej jednak, Jadey tutaj dostała
zbiór plików z~ESA. I~wszyscy wiecie, jak ważne nagle to się stało i~jak\ldots Cóż, pamiętacie, o~czym wczoraj wieczorem mówił Charlie.

Zdecydowanie ich zainteresowałem.

-- Chciałbym wiedzieć, co to jest. Pasuje wam to?

Dwóch starych geeków odpowiedziało pirackim uśmiechem. Jason trzeźwo
pokiwał głową.

Rozejrzałem się dookoła. 

-- Ok. Wszyscy gotowi?

Zabraliśmy nasze kawy i~poszliśmy schodami do miejsca, które Jadey
nazwała pokojem komputerowym. Alec zauważył, jakie to było śmieszne, jak
przypomniało mu to stare dzieje\ldots ale to był całkowicie nowoczesny
zestaw czytników, okularów VR i~kontaktów, które wszyscy uruchomiliśmy,
gdy zebraliśmy się dookoła staromodnych klawiatur i~ekranów. Już
załadowałem mój zestaw Sztucznych Inteligencji, miałem kopie ich
wszystkich i~moje typowe biblioteki programistyczne bezpieczne zachowane
na własnych rdzeniach budynku. Dwa razy sprawdziłem w~agencji rano
(ponad mamrotanymi protestami Jadey, ponad jej nogami\ldots) i~potwierdziłem, że kontrakt ESA był ciągle ważny, nawet po historycznym i~zaskakującym ogłoszeniu z~ostatniej nocy.

Sprowadziłem starych geeków, Tony i~Aleca, na wypadek, gdybyśmy musieli
poradzić sobie bezpośrednio z~przestarzałym oprogramowaniem stanowiącym
bazę dla systemów ESA. Specjalnością Jasona, szlifowaną w~różnych
pracach na boku przy fałszowaniu dokumentów, była praca w~wirtualnej
rzeczywistości VR i~systemy bezpieczeństwa. Praca w~VR wygląda na
prostą, ale bez doświadczenia w~indeksowaniu, skrótach i~metodach
szukania staje się fizycznym szukaniem małego obiektu na wielkiej
przestrzeni, nie jak igły w~stogu siana, ale jak igły na prerii.

Spauzowałem, gogle na czubku nosa. Jadey siedziała na ławce, machając
nogami jak znudzony dzieciak i~robiąc coś delikatnego paznokciom przy
pomocy nieproporcjonalnie wielkiego scyzoryka.

-- Chcesz do nas dołączyć?

Potrząsnęła głową. 

-- Będę na czujce.

Prawie wcale konieczne, pomyślałem, ale jeżeli chciała w~ten sposób
działać\ldots

-- Nie ma sprawy -- powiedziałem. -- Okay, koledzy, podążajcie za mną.

Zsunąłem gogle ostatnie pół centymetra i~dopasowałem je ciasno do oczu.
Mrugnięcie, i~byłem w~środku, mój punkt widzenia unoszący się
naprzeciwko abstrakcyjnego projektu jako renderowanego jednolitej
zamkniętej książki. Pozostali wisieli nad moimi ramionami. Czułem ich z~tyłu głowy, choć, gdybym spojrzał do tyłu, nie zobaczyłbym ich, dziwne
niesamowite uczucie, jak bycie obserwowanym przez ducha. Moje AI runęły
dookoła nas jak podniecone ptaki, gdy otworzyłem księgę. Kaskady domino
indeksów rozwinęły się w~całej wirtualnej przestrzeni.

Realna informacja potrzebna dla systemu zarządzania zapotrzebowaniem
surowcami była ledwo wykrywalnym ułamkiem tego, co było dostępne. To, co
najpilniej chciałem ustalić, było, czy produkt końcowy procesu był
gdzieś opisany lub zdefiniowany i, jeżeli tak, co to było. Szczęśliwie,
to był ten rodzaj rzeczy, na które oprogramowanie zarządzania projektami
i umiejętności było gotowe, więc poprowadziłem moich ludzi w~dżunglę z~okrzykiem \emph{Banzai!} w~myślach.

Wymagania kompleksu rafinerii, które Jadey i~ja widzieliśmy wcześniej,
były oczywistym miejscem rozpoczęcia, więc tam zaczęliśmy. W~międzyczasie, posłałem AI na głębokie przeglądanie dokumentacji,
używając jako kryterium koncepcyjnego szukania odniesień do wyjścia lub
zakończenia.

Pierwsza rzecz, jaką zrozumiałem, teraz, gdy miałem czas właściwie
przejrzeć, była taka, że byłem w~błędzie co do skali, rafineria w~rzeczywistości pomieszczeniem pełnym niesamowicie delikatnych
mechanizmów. Natychmiast zacząłem myśleć w~kategoriach maszyn i~materiałów wymaganych do produkcji takich mechanizmów, do wiercenia rur
o takiej średnicy przy koniecznej tolerancji, koszcie dostarczenia nawet
kilku molekuł stabilnych izotopów transplutonowców, rzadkich sztucznych
atomów na ,,wyspie stabilności'' z~wagą atomową dobrze w~dolnych setkach,
oraz uruchomieniu procesu produkcji, lub przynajmniej terminarza dostaw,
których produktem byłaby ta maszyna, której ostatecznego celu nie
znałem. Z~bliska, mechanizmy wyglądały prawie organicznie, miały tę
ewolucyjną złożoność, nieplanowaną i~nieoczekiwaną, którą można dostrzec
w obrazie mikroskopu elektronowego komórek i~wykresach mitochondrii.

-- Wygląda jak jakiś pieprzony cykl
Kerbsa\footnote{cykliczny szereg reakcji biochemicznych,
który stanowi końcowy etap metabolizmu aerobów, czyli organizmów
oddychających tlenem, zob.~\url{https://pl.wikipedia.org/wiki/Cykl\_kwasu\_cytrynowego} - przyp.tłum.} --
wymamrotał Alec, gdzieś daleko nad moim prawym uchem. W~tym samym
momencie poruszenie jednej z~AI przyciągnęło moją uwagę. Przybliżyłem
obraz. Moi towarzysze i~inne AI podążyły za mną. Podniecona AI wykonała
ekwiwalent machania plikiem papierów, a~ja je złapałem.

Tytuł na stronie był: \emph{Projekt konstrukcji 1 i~2} --- \emph{Przegląd
i rekomendacje }. Napis był podbity pieczęcią z~napisem ESA ,,Tylko do
wglądu ALPHA'' i~datą: 24 lipca 2048.

-- Bingo! -- powiedziałem. -- Spójrz na to na ekranie, Jadey.

-- Ok. -- Jej głos dobiegał z~daleka.

Zacząłem przeglądać strony. Wkrótce poczułem ból w~klatce i~napięcie w~gardle, a~moje ręce drżały. Plan dla rafinerii, lub jednostki
produkcyjnej, lub cokolwiek to było, pochodził z~obcej inteligencji na
asteroidzie. Metoda pozyskania pozostawała niewyjaśniona. Wymienione
były dwa produkty końcowe.

Wynik Projektu konstrukcji numer 1 był opisywany jako \emph{Silnik} a~projektu numer 2 jako \emph{Statek }. Pierwsze wystąpienia słów były
podświetlonym linkiem. Dotknąłem ich i~odniesienia rozwinęły się w~obrazy, które świeciły jak urządzenia widziane w~snach.

\emph{Silnik} wyglądał jak model silnika odrzutowego lub rakietowego
zrobionego na tokarce, gładkie żłobione powierzchnie, ale brak
widocznego wlotu lub wylotu, tylko specyficzny skręt na powierzchni,
niezłamany, ale, gdy obróciłem widok, jakoś dający złudzenie, że gdzieś
tam jest niewidziane otwarcie, jak w~butelce Kleina.

\emph{Statek} był, w~szalony, straszny sposób, rozpoznawalny. Były to
błyszczące soczewki metalu, z, tylko do wewnątrz od krawędzi, małymi
okrągłymi wypukłościami, które w~złym świetle mogłyby być pomylone z~nitami. Rozwinięty widok pokazywał ukryty właz, teleskopowe nogi,
wewnętrzne stery i~siedzenia zakrzywione w~środku, a~w rdzeniu coś, co
maszyna nazwała \emph{Silnikiem}, ale w~innych proporcjach i~połączony w~coś, co powierzchownie mogłoby być nazwane kadłubem \emph{Statku }. Był
to ewidentnie, żenująco, niewątpliwie latający spodek.

\threeast

Wszyscy wyszliśmy z~VR, siedzieliśmy lub staliśmy patrząc na siebie i~gadając gwałtownie. Alec Curran przerwał to uderzając pięścią w~stół.

-- To jest \emph{to} -- powiedział. -- Kamień z~Rosetty. Święty Graal. To
jak dokumenty Majestic 12\footnote{rzekoma organizacja,
która pojawia się w~teoriach spiskowych dotyczących UFO, więcej
\url{https://pl.wikipedia.org/wiki/Majestic\_12} - przyp.tłum.}.

Ku mojemu zaskoczeniu, Jadey roześmiała się i~powiedziała: 

-- Pamiętaj, że MJ-12 było dezinformacją!

Przez minutę sprzeczali się gwałtownie, przerzucając się odniesieniami.
Nie miałem pojęcia, o~czym rozmawiają. Idea, że latające spodki zostały
zbudowane przez obcych, a~nie Amerykanów, należała do dwudziestego
stulecia tak jak węże morskie do dziewiętnastego. Przez ostatnie dekady
nawet liczba obserwacji spadła, cały kult UFO przeniósł się w~matecznik
białej biedoty i~opustoszałe pustkowia Internetu.

-- Czy to nie jest zabawne -- ciężkim tonem podsumował Alec -- że mamy
pierwszy dowód tajnych kontaktów rządu z~obcymi dzień po tym, jak rząd
to obwieszcza?

-- Cóż, dostaliśmy go dzień \emph{wcześniej} -- zwróciła uwagę Jadey,
wystarczająco rozsądnie, ale bez większego wpływu na Aleca. Właściwie
nie dyskutowali, zrozumiałem, oboje byli tak podekscytowani tym, co
znaleźliśmy, że każde chciało przetestować krytykę tej prawie nieznośnie
zdumiewającej możliwości, że dokumenty są prawdziwe. Oboje widocznie brali
mity UFO poważniej niż ja, coś co nieładnie przypisałem wiekowi Aleca i~możliwemu wychowaniu Jadey. Ludowa podstawa frakcji Nuworyszy, matecznik
białej biedoty, żeby być dosadnym, była notorycznie skłonna do teorii
konspiracyjnych, entuzjastycznych religii i~takich ekscentryczności,
według \emph{Europa Prawda}.

-- Panowie -- powiedziała ostatecznie Jadey, sięgając jak gdyby chciała
zderzyć razem ich głowy -- to nas nigdzie nie prowadzi, prawda? Mam na
myśli, kształt dysku jest dość logiczny, w~pewnym sensie, dla jakiś
maszyn latających. Cholera, używali takiego w~czasie wojny. To nic nie
znaczy dla starych bzdur o~UFO, w~ten czy inny sposób. Jeżeli te rzeczy
pojawiły się jako prawdziwe dane ESA, uważam, że powinniśmy założyć, że
był jakiś powód.

-- To ciągle może być dezinformacja, nawet jeżeli projekt jest prawdziwy
-- nalegała Jadey. -- Ale wiecie, nie sądzę, żeby to miało znaczenie.
Jeżeli \emph{przykrywką} jest to, że to jest projekt obcych dla jakiejś
technologii statków kosmicznych, to, co \emph{naprawdę} jest, musi być
całkiem ważne.

Dostrzegałem kilka problemów z~tą teorią, ale dyskutowanie byłoby
marnowaniem czasu. Można skręcić kabel paranoi tylko tyle razy, póki coś
nie puści, i~niekoniecznie kabel.

Tony, drugi stary geek, którego wciągnąłem dla jego doświadczenia z~MS-DOS, żuł gumę z~półotwartymi ustami, jego żółte palce przeczesywały
kosmyki białej brody, a~jego paznokcie wydawały niemiły zgrzytliwy
dźwięk na jego policzku. Odniosłem wrażenie, że był nieco spięty.

Wytarł usta o~nadgarstek.

-- Więc, co planujecie z~tym zrobić? -- spytał, patrząc tam i~z powrotem
pomiędzy mną a~Jadey, a~potem rzucając spojrzenie na Aleca. -- Sprzedać
to Jankesom?

-- Nie, oczywiście, że nie -- powiedziałem z~oburzeniem i~może zbyt
szybko. -- Myśleliśmy o\ldots rozpowszechnieniu tego.

-- Prawdopodobnie nie sądzisz, że UE powinna być jedyną stroną, która ma
dostęp do takiej technologii, cokolwiek to jest -- dodała Jadey.

Tony potrząsnął głową. 

-- Nie, nie, ale nie wiecie, że tak będzie.

Jefrimowicz powiedział wczoraj, że chcą współpracy naukowej. Skąd
wiecie, że to nie uwzględnia tych projektów?

-- Nie wiemy. -- Jadey wzruszyła ramionami. -- Ale pewne okoliczności, w~jakich otrzymaliśmy te dokumenty, sugerują przeciwnie.

-- Hmmm -- powiedział Alec. -- To brzmi dostatecznie rozsądnie, tak.
Informacja chce być wolna i~tak dalej. -- Wstał i~uśmiechnął się do nas
raczej nieśmiało. -- Wybaczcie mi na chwilę. Natura wzywa. Wracam za
kilka minut, ok?

-- Pewnie -- odpowiedziałem. -- Czekamy.

Alec uciekł. Była dwunasta trzydzieści, ku mojemu zdziwieniu, czas szybko biegnie,
gdy jesteś w~VR. Rozmawialiśmy przez chwilę. Po dziesięciu minutach
Jadey się rozejrzała.

-- Ile czasu \emph{zabiera} skorzystanie z~toalety?

Mój telefon zadzwonił. Stuknąłem odebranie.

-- Halo?

-- Tu Alec. Hm, Matt, jestem w~barze i~stąd wygląda na to, że gliny
poważnie się kłócą na recepcji. Sądzę, że znajdą się w~windzie za około
minutę.

Rozłączył się.

-- Alec mówi, że gliny będą tutaj za minutę!

Jason spokojnie pochylił się i~uderzył awaryjny DELETE. Każdy ślad
naszej porannej pracy i~dane, które ściągnąłem poprzedniej nocy,
zostanie wyczyszczony z~rdzeni. Twarz Tony'ego pokazywała poruszenie
przeciwstawnych min, potem wzruszył ramionami.

-- Nie będę uciekał -- powiedział.

Jadey energicznie wstała. 

-- Przyszli po mnie -- powiedziała. 

Złapała mnie za rękę i~pociągnęła za sobą. 

-- Idź. -- Pacnęła dyskiem danych w~moją dłoń. Jej usta drasnęły moje, na ułamek sekundy. -- Idź już! Dam sobie
radę.

Jason był już w~drzwiach, patrząc na mnie z~niecierpliwością. Dołączyłem
do niego w~jednej chwili, potem spojrzałem do tyłu.

-- Do zobaczenia w~Ameryce -- powiedziała Jadey.

-- \emph{Gdzie?}

-- Wrota Krainy Marzeń -- odparła.

Jason wyciągnął mnie energicznie.

\threeast

Jason znał budynek lepiej niż ja. Rzucił się biegiem wzdłuż korytarza,
otworzył coś, co wyglądało jak drzwi do szafy i~wskoczył. Podążałem za
nim i~znalazłem się w~jakiejś windzie dla kelnerów, która od razu
zaczęła się z~wielką szybkością opuszczać. Zaparłem się rękoma o~sufit
tuż przed tym, gdy winda się gwałtownie zatrzymała, że prawie zwichnąłem
sobie kolana.

Nadal sprawdzałem mój kark pod kątem naciągnięć, gdy wyszliśmy do
niskiej, piwnicy z~betonową podłogą. Zwisające tuby fluorescencyjne
syczały i~błyskały. Stojące powietrze pachniało delikatnie olejem
silnikowym i~cementem.

-- Kiedyś to był garaż -- powiedział Jason z~przekąsem. -- Teraz to wyjście
awaryjne.

Pobiegliśmy wzdłuż rampy, która poprowadziła nas w~dół i~na bok do
szerokich metalowych drzwi, najwyraźniej zaspawanych. Jason odciągnął
rygiel i~mniejsze drzwi lub właz poziomo, otworzyły się, wyszliśmy przez
nie, żeby znaleźć się na Leith Walk, w~deszczu. Pół minuty później
siedzieliśmy z~tyłu trolejbusu jadącego w~dół w~kierunku Leith.

-- \emph{Nie} oglądaj się -- powiedział Jason.

Rozluźniłem ramiona, potem wyciągnąłem czytnik i~poruszałem kciukami po
przyciskach. Większość kanałów była szumem. Jason rzucił okiem, potem
zesztywniał. Wyjął swój telefon, popatrzył na niego, pochylił się i~położył go na podłodze pod butem. Wyprostował się. Usłyszałem chrzęst i~trochę szurania.

-- Dobrze -- powiedział. Patrzył szalenie spokojnie przed siebie.

-- Co?

-- Człowieku, spójrz na zewnątrz. To jak pierdolona \emph{Inwazja
porywaczy ciał}\footnote{film z~gatunku horror science
fiction z~1956 roku - przyp.tłum.}. -- Jego głos był cichy, choć
jedynymi osobami, które były w~tym autobusie była para starszych kobiet
siedząca z~przodu.

Spojrzałem oczami na boki, przeszukując ulicę. Trolejbus dotarł gdzieś w~połowie dwukilometrowej ulicy. Rzędy frontów sklepowych występowały na
zmianę z~rzędami kamienic mieszkalnych. Chodnik był zajęty, ale nie
zapchany.

-- Wszystko wszystko normalnie -- powiedziałem.

-- To właśnie problem -- odparł Jason. -- To Leith, nie pierdolona
Morningside. Przyjrzyj się.

I nagle sposób, w~jaki patrzył, także do mnie dotarł. Nikt się nie
wałkonił na ulicach, żadnych spacerowiczów czy żebraków, czy
straganiarzy. Wszyscy szli, jak gdyby musieli w~każdej chwili wyjaśnić,
dlaczego ich pobyt na ulicy był naprawdę konieczny. Dwójka policjantów
szła jak gdyby nie musieli się martwić o~swoje bezpieczeństwo. Gdy
trolejbus szarpał i~dźwięczał od przystanku do przystanku, całość stała
się jeszcze bardziej absurdalna, Constitution Street wyglądała jak gdyby
jej osławione skwery i~bulwary zostały wyczyszczone przez jakiś
szczególnie purytański samorząd (czym Rada Leith, bezczelnie od samego
początku, nie była).

Nawet wtedy, nie byłem pewien, czy nie ulegamy paranoi. Może to był
cichy okres dnia. Spojrzałem na zegarek. Była 13:10.

Tylko 13:10. Przerwa na lunch. Ale ciągle\ldots

Przeszedłem tyle szoków w~ciągu ostatnich dwudziestu sześciu, lub coś
koło tego, godzin (Chryste, tylko?), że mogłem sobie wybaczyć
odrobinę paranoi. Podobnie Jason. W~istocie, to samo dotyczyło ludzi na
ulicach, którzy byli na tyle zdolni do wyciągnięcia kłopotliwych
konkluzji z~wiadomości rządu jak każdy zorientowany geek w~Darwin's
Arms. Odkrycie, że supermocarstwo, w~którego przyjemnie zepsutych
objęciach żyliśmy, miało najwyraźniej przyjacielskie stosunki z~\emph{obcymi z~kosmosu} byłoby całkiem wystarczające, by ludzie stali
się bardziej lękliwi, by nie wylądować po złej stronie. Może bali się
własnego cienia, tak jak my, ale my baliśmy się własnego cienia jeszcze
bardziej, ponieważ wiedzieliśmy więcej.

I ponieważ już zostaliśmy zdradzeni. Mocno podejrzewałem, że Curran miał
napad patriotycznego stracha na myśl, że nasze odkrycie dostałoby się
Jankesom, lub na myśl, że jest wplątany w~taką sytuację, i~zdecydował
się wykorzystać szansę korzystania z~toalety, żeby zadzwonić po gliny.
To, że natychmiast potem dał nam szansę uciec, było całkowicie w~jego
stylu.

-- Nigdy nie powinniśmy ufać starym geekom -- powiedziałem pod nosem.

-- Pierdolona racja -- powiedział Jason. -- Teraz zamknij mordę.

Trolejbus szarpnął, błysnął iskrami w~lewo w~Great Junction Road i~zadzwonił, by zatrzymać się na przystanku.

-- Teraz -- powiedział Jason.

Padał deszcz. Zapiąłem kurtkę i~dogoniłem Jasonem, gdy przechodził
ostrożnie na światłach i~szedł raźno, ale od niechcenia południową
stroną ulicy, potem długo jakimiś tylnymi uliczkami i~w końcu zanurkował
do nabrzeżnego baru ,,Deil and Excisemen''. Miejsce było pełne ludzi, jak
gdyby wszyscy nieporządni ludzie, którzy nie mogli być widziani na
ulicach, zebrali się tutaj. Niewątpliwie każda knajpa w~Leith była
podobnie pełna. To było tego rodzaju miejsce, gdzie oczy i~soczewki
zwracały się ku drzwiom, kiedykolwiek ktoś wchodził, ale widocznie
rozpoznając Jasona, wszyscy się odwrócili. Przepchnęliśmy się przez parę
i zapach mokrych płaszczy w~ciepłej, zadymionej mgle do baru.

-- Co chcesz? -- spytał Jason.

-- Belhaven Export.

Nagle głodny, zamówiłem kilka pasztecików. Mikrofala \emph{zadzwoniła} w~tej samej chwili, gdy dostaliśmy piwo.

-- Boże, to mile widziane -- powiedziałem.

Odeszliśmy od baru i~stanęliśmy w~rogu, gdzie była półka na nasze łokcie
i piwa. Muzyka była dostarczająca głośna, by utrudnić rozmowy i~by
bardzo utrudnić podsłuchiwanie. Jednak pochyliłem się i~mówiłem cicho.

-- Czy to miejsce jest bezpieczne?

Jason zachichotał się mrocznie. 

-- Jest bezpieczne dla nas.

Nie byłem całkiem pewny zapewnień Jasona, jak byłbym jeszcze rano, ale
ciągle nie miałem niczego innego, na czym mógłbym polegać.

-- Co teraz zrobimy?

Jason wzruszył ramionami. 

-- Wyślemy Cię do Ameryki, chyba.

-- Co? -- Zapomniałem mówić szeptem.

-- Pewnie. Czy to nie to, co powiedziała dama?

-- Tak, ale myślałem, że miała na myśli to jako ostatnią deskę ratunku.
No weź. Możemy coś zrobić, mogę załatwić prawnika, pójść do mediów i~ambasad, zobaczyć czy jej nie wyciągną, upewnić się, że ktoś o~mnie
zapyta, gdy zniknę na ulicy. Może nawet nie byłbym, hm, poszukiwany.

Jason patrzył się na mnie. 

-- Nie rozumiesz tego, prawda? Jadey może o~siebie zadbać. To koniec gry. To \emph{jest } pierdolona ostatnia deska
ratunku.


\chapter[Wielka Praca]{7 Wielka Praca}

Niektóre sekty Szyderców ciągle lgnęły do starych sposobów, Biblii,
przynajmniej według interpretacji Joanny, i~do jej wczesnoprzemysłowego
materializmu, łącznie z~takimi sakramentami jak namaszczenie olejem,
które symbolizowało przekonanie, że człowiek był maszyną zbudowaną przez
Twórcę. Inni przyjęli dialektyczny materializm Engelsa i~Haldane\footnote{prawdopodobnie mowa o~J.B.S. Haldane,
sławnym biologu, marksiście i~popularyzatorze nauki, zob.~\url{https://en.wikipedia.org/wiki/J.\_B.\_S.\_Haldane} - przyp.tłum.} (i
odpowiednio, dialektycznie, podzielili się na kolejne kłótliwe frakcje).
Większość, w~tym sekta, w~której Gregor był wychowany, zajęła to, co
uważali za umiarkowaną pozycję, czcząc starożytnych materialistów
znacznie bardziej niż współczesnych proroków religijnego lub
politycznego mesjanizmu, oczywiście doceniając ich wkład (jak mówił
tolerancyjny stereotyp).

Wnętrze Domu Spotkań przy North Street było ciemne od drewna, jasne od
kolorowego szkła. Idąc, jak gdyby balansował z~książkami na głowie,
Gregor doszedł nawą do ławki jego rodziny i~usiadł na obrzeżu koło
Anthony'ego, swojego młodszego brata. Jego rodzice pochylili się do
przodu, jego matka z~jej zwykłym lękliwym uśmiechem, ojciec z~jego
zwyczajowym szorstkim skinieniem, potem oparli się z~powrotem. Bez
wątpienia oboje byli wdzięczni za przyjście. Jego wizyty w~Domu
Filozofii stawały się tym rzadsze, im stawał się starszy.

W rzeczywistości przyszedł z~powodu mieszanki motywów, w~których
pragnienie materialnego oświecenia było ostatnie. Niejasno obiecał
Clarissie, która teraz siedziała z~przodu, jej mąż z~jednej strony, a~po
drugiej stronie komicznie regularna seria sukcesywnie starszych i~wyższych dzieci. Ciągle miał swój dobry garnitur, relatywnie
nie-pomarszczony. A~jego kac był zbyt delikatnie gotów na niego, by
nawet myśleć o~śniadaniu. Więc był tutaj, zamiast w~łóżku, o~dziesiątej
rano w~niedzielny poranek.

Szyderca wszedł na mównicę i~uśmiechnął się do większej niż zwykle
kongregacji. Była prawdopodobnie zdublowana przez młodszych członków
rodziny Cairns i~ich dalszych kuzynów. Podniósł ramiona i~dźwięcznym
głosem zaintonował przywołanie:\\

-- \emph{Samoporuszająca się materio, matko-twórczyni wszystkiego, porusz
mnie małego!}\\

-- \emph{Użycz wagi moim słowom, energii mojemu głosowi, pędu moimi
instrukcjom!}

Schodząc, stanął przy zbiorniku morskiej wody i~czekał, podczas gdy
Clarissa przyniosła dziecko. Delikatnie wziął dziecko w~ramiona i~spytał:

-- Kto nazywa to dziecko?

-- Ja, Clarissa Louise Cairns, jego matka.

-- Jakie imię mu nadajesz?

-- Owen John James Matthew Cairns.

Szyderca zanurzył palec wskazujący w~morskiej wodzie, sprawdził na
języku, że rzeczywiście jest słona, zanurzył palec środkowy w~ślinie ze
swoich ust, potem znów zamoczył palec wskazujący w~morskiej wodzie i~dwiema wodami życia narysował okrąg na czole dziecka.

-- Witaj -- powiedział.

Podniósł niemowlę do góry, żeby wszyscy zobaczyli: mała, szczęśliwie
śpiąca głowa wyglądała na nawet mniejszą nad białą szatą nawilżenia,
której ciągnący się tren symbolizował braterstwo dziecka z~bogami na
niebie. Potem zwrócił niemowlę Clarissie, która usiadła na miejscu i~wysłuchała błogosławieństwa, formalnie skierowanego do nowego przybysza.

-- Owen, przybyłeś do nas przez śmierć gwiazd i~do ich narodzin
powrócisz. Wcześniej nic nie wiedziałeś i~nic nie będziesz wiedział
potem. Przez moment pomiędzy, będziesz cieszył się darem życia. Twoje
życie jest bronione przez nas wszystkich. -- Na chwilę wysunął miecz,
gładko schował go do pochwy. -- Twoja krew jest naszą krwią. Twoje życie
jest Twoje. Ciesz się nim przez wszystkie swoje dni, a~kiedy musisz,
zostaw życie bez lęku. Twoje potrzeby są nieliczne i~łatwe do
zaspokojenia. Zrozum to, a~Twoje życie będzie szczęśliwe, warte bogów.
Żyj długo, żyj radośnie, żyj szczęśliwie!

Błogosławieństwo zakończone, Szyderca wrócił do mównicy, otworzył Dobre
Księgi i~zaczął mowę. Było to całkowicie nieagresywne i~banalne kazanie
o dobrym życiu, etyce ilustrowanej przez naciągane metafory z~biologii i~fizyki, ożywione krótkimi bajkami dla dzieci, i~bez wątpienia, wszystko
co najlepsze dla wszystkich. Po około pięciu minutach uwaga Gregora
odbiegła do wysokich witraży, w~których kwiaty kwitły, plątały się
liście, dinozaury biegały, latały nietoperze, męczennicy płonęli, pary
uprawiały seks, naukowcy badali, a~inne budujące fenomeny natury,
społeczeństwa i~myśli były pokazywane w~wielkiej obfitości. To
prawdopodobnie był jego pech, że jeden panel, z~bogowie wiedzą jakiego
powodu, przedstawiał ciemnowłosą pannę w~różowej sukni. Ból w~sercu
sprowadził ból w~głowie i~był niezmiernie wdzięczny, kiedy Dyskurs się
zakończył.

Pod osłoną końcowego hymnu uciekł, unikając jakiejkolwiek rozmowy ze
swoją rodziną. Dzień był piękny i~przenikliwy. Ciepłe słońce i~chłodne
podmuchy zaczęły łagodzić kac Gregora, gdy szedł na dół North Street i~w
górę High Street, na drodze z~miasta i~do Zamku. Po drodze kupił arkusz
wiadomości. Gdy płacił i~wymieniał uprzejmości, zauważył trzeźwo po raz
pierwszy w~życiu powieści romantyczne dyskretnie wystawione w~prostych
okładkach na dolnej półce z~tyłu kiosku, poniżej linii oczu dzieci i~znacznie niżej niż stojaki książek i~broszur z~erotycznymi obrazami i~fantazjami, których kolorowe okładki były tak barwne i~jawne i~tak
wesoło wyraźne, jak witraże w~Domu Spotkań.

Krótko rozważał nabycie jednej z~tych powieści miłosnych spod lady,
potem zdecydował, że byłoby to zbyt żenujące.

\threeast

-- Dobrze Cię widzieć, Greg. Wejdź.

James cofnął się, otwierając szerzej masywne drzwi, a~Gregor wszedł do
studia. Kurz tańczył w~promieniach słońca z~okien, które zajmowały
większą część ściany szerokiego, wysokiego pokoju. Gregor wiedział z~dziecięcych wspomnień gdzie był. Nawet dziecięce wspomnienie nie
przejaskrawiło liczby schodów do pokonania lub długości i~mroku
korytarzy do przejścia, by dostać się do tego pokoju, wysoko w~ledwie
zajętym skrzydle Wieży.

Ale teraz, półki wyglądały na niższe, stół na szerszy, stosy papieru
wyższe i~bardziej nieuporządkowane, maszyny liczące bardziej
ekscentryczne i~przestarzałe. Powietrze było ostre od kurzu.
Powstrzymując kichanie, Gregor zaakceptował powitalny kubek kawy, którą
Nawigator nalał z~termosu i~usiadł na najczyściej wyglądającym dostępnym
krześle. Dziadek zrelaksował się na starej skórzanej sofie, w~której
przebijały się sprężyny i~końskie włosie, machnął ręką na otaczający
bałagan.

-- Cóż, oto i~ona -- powiedział. -- Wielka Praca, jak dotąd. Chciałbym,
żebyś mi pomógł\ldots ją skończyć.

Niedowierzanie Gregora musiało być widoczne na jego twarzy. Wielka Praca
trwała już tak długo, że jej skończenie w~realistycznej perspektywie
nigdy nie przyszło mu na myśl. Zadanie, które zaproponował James,
wydawało się majaczyć przed nim jak niemożliwy klif.

-- Och, nie martw się -- dodał pośpiesznie James. -- To nie będzie wymagać
wiele Twojego czasu. Po prostu potrzebuję kogoś młodszego i~bystrzejszego niż ja, szczerze, by zintegrować na wysokim poziomie to,
co mamy i~zobaczyć, czy to w~ogóle ma sens.

-- Dobra -- powiedział Gregor. Łyknął teraz chłodnej kawy. -- Jedno
pytanie. Czy mógłbyś powiedzieć, w~tajemnicy, jeżeli trzeba, po prostu
czym jest właściwie Wielka Praca?

-- Pewnie -- odparł James. -- W~tajemnicy, tak, w~największej tajemnicy.
Próbujemy wyznaczyć kurs, żeby polecieć \emph{Jasną Gwiazdą} na Croatan.

Gregor prawie upuścił kubek. Szczerze myślał, że celem tych ćwiczeń było
\emph{ćwiczenie}, przedłużona i~płonna walka, by utrzymać umiejętności
programistyczne żywe i~w obrębie rodziny.

-- Cały czas robimy to \emph{ręcznie}?

James pokiwał głową.

-- Dlaczego w~imię bogów nie używamy maszyn liczących lub nawet\ldots
komputerów?

-- Komputery, które pierwsza załoga zabrała ze sobą ze statku -- odparł
James -- były częściowo organiczne, nazywali je biotech, mokrą
technologią, i~w większości pogniły lub stały się zawodne. A~co do
maszyn liczących, mechanicznych czy elektronicznych, cóż\ldots

Postawił ostrożnie kubek na oparciu sofy, rozłożył ręce i~uśmiechnął się
rozbrajająco. Potem niejasno i~lekceważąco pomachał na maszyny, lśniący
lub pordzewiałe, grube od oleju i~kurzu. 

-- Możesz je użyć do obliczeń,
ale nie możesz zaprogramować komputera \emph{innym} komputerem.

-- Z~całą pewnością możesz! -- zaprotestował Gregor. -- Nawet ja to wiem.

-- A~więc przejrzałeś książki informatyczne w~rodzinnej bibliotece -
powiedział James tonem jednocześnie pochwały i~drwiny. -- Cóż,
przejrzałem znacznie więcej książek niż Ty i~pracowałem ze starymi
komputerami biotech, o~tak!, i~mogę cię zapewnić, że te zręczne skróty
są pośród rzeczy, które w~większości straciliśmy. We wczesnych latach,
dwa lub trzy pokolenia temu, moi poprzednicy \emph{mogli} tak robić, a~wtedy praca szła znacznie szybciej. Teraz, z~pracą wysłaną do wszystkich
wiejskich kuzynów klanu\ldots -- James wzruszył ramionami. -- To jest, jak
widać. Nie, żebyśmy byli całkowicie zdegenerowani. Dobra robota jest
wykonywana na uniwersytecie. Pewnego dnia zbudujemy nasze własne
komputery, takie, które poradzą sobie z~tego rodzaju zadaniami, tutaj na
Kyohvic. Ale nie szybko i~z pewnością niedostatecznie szybko.

-- Dostatecznie szybko na co?

-- Pomyśl o~tym -- powiedział James. Wstał, podszedł do okna i~stał,
patrząc na zewnątrz, ręce złożone z~tyłu.

-- Widziałeś tego początki -- kontynuował bez odwracania. -- Tam jest
pierwszy statek z~Nova Babylonia, która już jest świadoma naszej tutaj
obecności. Za kilka lat, kiedy ich podróż zaprowadzi ich do Croatan i~innych bliskich światów, zobaczą nasz wpływ na nie wszystkie. W~porównaniu z~Nova Babylonia, jesteśmy nowością w~Drugiej Sferze. Wszyscy
ludzie w~tym sektorze byli\ldots dostarczeni\ldots z~Ziemi lub Układu
Słonecznego po powstaniu kapitalizmu. Większość przodków ludu w~sektorze
Nova Terra przybyli ze starożytnego świata, wyciągnięci z~utraconych
legionów, umierających miast dławiących się w~dżungli lub na pustyni,
plemion koczowników. Stali się wielką imperialną republiką, miejscem
bardzo zaawansowanym i~oświeconym z~każdej strony, ale my nie jesteśmy
tacy jak oni. Jesteśmy nowi.

James odwrócił się ostro, gwałtownie. 

-- I~jesteśmy słabi. Jeżeli nie
zdobędziemy jakiejś \emph{rozstrzygającej} przewagi, zostaniemy
wchłonięci pod życzliwym wpływem Nova Babylonia. Nasze pisma wypełnią
ich biblioteki, nasze myśli zafascynują ich filozofów, nasze sztuki
dodadzą nowych kolorów do ich palet. Ktoś może to nazwać swego rodzaju
zwycięstwem. Ale \emph{oni się nie zmienią}, a~my tak. Co czyni nas
unikalnymi, co czyni nas nami, zostanie utracone.

-- Co to -- Gregor zmarszczył brwi -- jest, co czyni nas ,,unikalnymi''?

Stary człowiek się uśmiechnął.

-- Niestabilność -- odpowiedział James. -- Nova Babylonia wchłaniała nowe
idee i~ludzi, a~także tworzyła własne, przez setki, jeżeli nie tysiące
lat. I~jest miejscem bardzo stabilnym. My chłoniemy idee od nich,
niektóre pochodzą z~Ziemi, ale spójrz, co z~nimi robimy! Świeckie
chrześcijaństwo Szyderców jest bardzo różne od raczej pasywnych
filozofii głoszonych przez starożytnych materialistów w~Dobrych
Książkach, mimo wszystko ciężko jest sprawić, by herezjarchowie to
zauważyli. Zmieniamy się cały czas, a~nie chcę, byśmy zmienili się w~nich i~przestali się zmieniać. Co, jak mówię, się zdarzy, gdy coraz
więcej ich statków przybędzie, rok po roku, może miesiąc po miesiącu.
Chyba że coś z~tym zrobimy.

-- Co \emph{możemy} zrobić?

-- Możemy zbudować własne statki -- powiedział James. -- Statki, które nie
będą zależeć od krakenów i~zaurów. Możemy stać się \emph{handlarzami}
Drugiej Sfery i~poza nią. Z~taką mocą, możemy utrzymać naszą
niezależność.

Gregor spojrzał się na niego ze zdumieniem. 

-- Teraz to -- powiedział w~końcu -- brzmi jak \emph{wielka }praca.

-- Więc zabierzmy się do niej -- powiedział James. Zrobił krok i~wyciągnął
dłoń. -- Witamy w~kadrze Kosmonautów.

\threeast

Gregor był wstrząśnięty mimochodem przyznanym honorem. Kadra była
rdzeniem Rodzin, frakcją, która, przez członkostwo w~nominalnej załodze
\emph{Jasna Gwiazda}, podtrzymywała mistyczną ciągłości z~Ziemią, z,
naprawdę, jego najpotężniejszym i~najwspanialszym imperium, Unią
Europejską. Niektórzy w~Rodzinach wzbogacili się na Mingulay, inni byli
biedni. Ale najbiedniejszy rybak lub drobny właściciel pochodzący z~oryginalnej załogi czuł przynajmniej dotyk odziedziczonej wyższości
ponad jej lub jego lokalnymi sąsiadami, a~którą tylko uzasadniała
ciągłość kadry z~wielką unią socjalistycznych republik. W~opinii
członków Rodziny, którzy doszli do czegoś własnymi zasługami, jak ojciec
Gregora, cały styl był pustą tradycją.

James przeczesał palcami chude białe włosy i~związał je w~kucyk przy
pomocy gumki. Potem podszedł spokojnie do półki, wyciągnął pakiet
papierów i~rozłożył je na stole.

-- Dobrze -- powiedział, opierając się na dłoniach i~patrząc z~góry na
papiery -- stąd zaczynamy. Opis problemu.

Najstarsze dokumenty, gdzie zadanie się zaczynało, były nawet fizycznie
trudne do odczytania, wyblakłe i~pożółkłe, ich pismo wystarczająco
antyczne, żeby czytanie wymagało wysiłku.

-- Założenia zaczynają się tutaj -- Nawigator powiedział Gregorowi,
wskazując prążkowanym paznokciem linię nabazgranych liczb -- z~tymi
obserwacjami paralaksy gwiezdnej. Musieli najpierw wyznaczyć odległość
do Croatan, oczywiście. Dryf w~kolejnych wiekach jest, hm, w~zakresie
marginesu błędu. \emph{Jednak } praca Napędu jest czule zależna od
rozkładu masy w~otaczającej przestrzeni kosmosu, aż do kilku lat
świetlnych, powiedzmy dziesięciu, żeby być bezpiecznym. Więc proces musi
być powtórzony dla tuzina najbliższych gwiazd.

Postukał kolejną stronę.

-- Następnie, to są odczyty z~instrumentów \emph{Jasnej Gwiazdy}. To
tylko pierwsze strony, przy okazji. Reszta jest tam.

Gest jego dłoni alarmująco objął kilka regałów, drewnianych i~uginających się pod ciężarem stosów papieru.

-- Te dwa zbiory informacji są zasadniczo wejściem. Istotą programu,
,,algorytmu'', jak to jest nazywane, dla obliczenia z~danych wejściowych
ustawień dla Napędu, co zabierze statek do Croatan, a~nie do, powiedzmy,
pierdolonego środka jakiejś pierdolonej międzygalaktycznej otchłani
miliard lat świetlnych stąd, jest, \emph{jak myślimy}, ten zbiór równań
tutaj. Wyprowadzenie praktycznego programu z~nich, żeby przeliczyć dane,
jest ogromnym zadaniem w~sobie samym, które\ldots

I tak to szło. James spędził kolejną godzinę lub coś koło tego,
pokazując Gregorowi czysty zarys zadania integracji i~interpretacji, w~której miał pomagać. To ciągle wyglądało przytłaczająco, jak gdyby był
wyznaczonym wykonawcą jakieś straszliwej skumulowanej woli, otrzymał
pracę przejrzenia spraw pokoleń prokrastynatorów. Kiedy wyjaśnienia
były, na tyle na ile mogły być, skończone, przyszła kolej na Gregora, by
stanąć w~oknie i~spojrzeć nastrojowo na zewnątrz.

-- Dlaczego nie możemy \emph{kupić} komputerów? -- spytał w~końcu. -- Zaury
sprzedają nam instrumenty i~automaty dla fabryk. Dlaczego nie do tego?

-- Zaury są bardzo ostrożne w~sprawie tego, co nam sprzedają -- powiedział
James, ciągle patrząc na papiery rozłożone na stole. -- Nie sprzedali nam
nigdy komputera ogólnego przeznaczenia. Mam na myśli, próbowaliśmy
kanibalizować i~odtworzyć rzeczy, które nam sprzedają, ale to jest jak
próby na żywych organizmach, zanim nawet masz pojęcie o~genetyce, nie
mówiąc już o~inżynierii genetycznej. Pierdolona niemożliwość. Coś
twardego i~błyszczącego zamienia się w~śmierdzącą kałużę.

-- Dlaczego nie sprzedają nam komputerów?

James westchnął. 

-- Z~tego, co stary Tharovar raczy powiedzieć w~tej
sprawie, a~nawet on jest ostrożny, wygląda na to, że bogowie tego by nie
zaakceptowali. A~zaury są bogobojni, na sposób, który my nie jesteśmy.
Bogowie mogli być wplątani w~jakąś katastrofę w~ich przeszłości\ldots
Możemy spekulować o~tych bytach pamiętanych w~tradycji, nawet ,,pamięci
rasy'', coś w~genach, ale to wszystko. Nie chcą o~tym rozmawiać.

-- Zauważyłem coś takiego -- powiedział Gregor. -- Z~Salasso.

-- W~każdym razie, opieranie się w~nawigacji na komputerach od zaurów\ldots
mijałoby się z~celem, nie sądzisz? -- James porzucił zadanie i~dołączył
do Gregora przy oknie.

-- Tak -- odpowiedział Gregor. -- Rozumiem.

-- Doskonale! -- James uśmiechnął się do niego i~klepnął go w~plecy. --
Teraz idź zobaczyć swoją dziewczynę.

\threeast

Powoli przechadzał się ciemnymi korytarzami, schodził powoli długimi
prostymi lub kręconymi schodami. Anachroniczne zbiory skamieniałości w~okładzinach ścian ze skał osadowych były odbiciem dezorientacji w~jego
umyśle. Ciągle wstrząśnięty ogromem zadania, które zrealizowali jego
przodkowie i~żyjący krewni, ciągle zbulwersowany skalą i~złożonością
zadania do zrobienia, już drżał na myśl o~kolejnym spotkaniu Lydii.

To nie była normalna i~naturalna pasja seksualnego pożądania lub proste
uczucie, które wynika ze wzajemnej satysfakcji, nawet, czasami,
wzajemnego przyjacielskiego poznania. To było szaleństwo zaślepienia,
zdolnego zawiesić rozum, zniszczyć życia. Jego nagła nieumyślna obsesja
Lydii stała się tylko bardziej intensywna przez nieprawdopodobieństwo
spełnienia kiedykolwiek bez nieszczęśliwych konsekwencji. Jeżeli byliby
razem dłużej niż kilka krótkich tygodni statku na Mingulay, jedno lub
drugie zostałoby oddzielone przez lata świetlne i~życia od tego, co
dotąd kochali.

Przelotne romanse seksualne pomiędzy wolnymi podróżnikami gwiezdnymi a~tubylcami były oczekiwane i~były, rzeczywiście, mile widziane po obu
stronach ze względu nowe, w~ten sposób wymienione, geny. Każda wizyta
skutkowała w~małej ulewie ciąż, a~nawet chwilowo złamanych serc.
Prawdziwe złamanie serca, wykluczająca pasja, szalone pożądanie jednej i~tylko tej jednej osoby, ani nie było popierane ani częste. Ale to było
to, co czuł.

Ostatniej nocy nie powiedział Lydii jak się czuje. Ale ona musi
wiedzieć! Rozmawiali, rozmawiali, rozmawiali, aż zauważyli, że ich ciche
głosy rozbrzmiewały echem, rozejrzeli się i~zauważyli, że są ostatnimi
spośród ostatnich kilku osób w~sali. I, tuż przed tym, zanim się
odwróciła, Lydia położyła dłonie w~jego dłoniach, tak jak zrobiła to w~trakcie tańca. A~potem odeszła tanecznym krokiem.

\threeast

Lydia siedziała na ławce oparta o~ścianę, plecami ku morzu, jednego z~niższych pięter Zamku, która wychodziła na ogród otoczony murem: zielony
trawnik otoczony przez grządki, w~których rododendrony, hortensje i~niskie sosny rosły dziko. Wiciokrzewy i~bluszcze dawno temu wbiły swoje
haczyki w~tę ścianę Zamku i~wdrapały się prawie do szczytu. Z~jej oczami
przymrużonymi od odległego blasku i~ciągłej bryzy, Lydia patrzyła na
popołudniowe morze, w~którego wzburzonych wodach statek gwiezdny jej
rodziny nie pływał, ale unosił się nad nim, bucząca energia jego
silników wysyłała widoczne wzorce zniekształceń na otaczającej
powierzchni. Lichtugi na morzu, łodzie grawitacyjne w~powietrzu,
spieszyły się do i~z, ładując i~rozładowując. Niewidoczne z~tego kąta i~odległości, większe łodzie podwodne robiłyby to samo, załatwiając dla
statku prawdziwe interesy, które były prowadzone pomiędzy krakenami. W~porównaniu z~tym, handel zaurów i~ludzi był w~każdym znaczeniu
powierzchowny.

Gregor podszedł do niej z~boku, przez trawę, ciesząc się z~niestrzeżonej
chwili, zanim go zauważyła. Jej włosy były zdmuchnięte na twarz wiatrem,
dla którego jej sukienka, do kolan, fałdowana i~złożona podobnie do
rzeźbionej muszli z~ciemnoniebieskiego materiału, była najwyraźniej
nieprzenikliwa. Gdy wszedł w~jej pole widzenia peryferyjnego, odwróciła
głowę ostro, zobaczyła go, wstała, uśmiechając się. Zatrzymał się kilka
kroków wcześniej, nie chcąc się zatrzymać, pragnąc podejść bliżej.

-- Dzień dobry -- powiedziała Lydia.

-- Dzień dobry -- odpowiedział Gregor.

Stali przez chwilę patrząc się na siebie.

-- Czy chciałbyś mnie zabrać na spacer? -- spytała Lydia.

-- Dobry pomysł -- odparł Gregor, w~myślach przeklinając się za banalny
dobór słów.

Przespacerowali przez trawę, do najdalszego kąta ogrodu, gdzie brama
otwierała się na otwarte powietrze. W~słońcu jej włosy, tak faliste, że
aż trochę kędzierzawe, wyglądały inaczej tak jak jej skóra na
nieskończone fascynujące sposoby. Zapach, który płynął z~szerokiej
wysokiej fałdy kołnierza jej sukienki, konkurował z~tym flory ogrodu:
było w~nim coś zwierzęcego jak również roślinnego.

Na szczycie schodów zatrzymała się, patrząc w~dół na dzikie trawy
przylądka dwadzieścia metrów poniżej. Stopni były wąskie, kamienne,
zużyte i~mokre, schodziły w~jednym, długim biegu aż do zewnętrznej
ściany. Położyła rękę na poręczy i~delikatnie ją sprawdziła.

-- Jest bezpiecznie -- zapewnił ją Gregor.

-- Wygląda jak dodatek.

-- Prawda. Przykręcona tysiące lat po schodach. -- Gregor wzruszył
ramionami. -- Które same są dodatkiem do oryginalnej budowli. Kiedy oni
je zbudowali, bezpieczeństwo nie było ważne. -- Wskazał gestem na nawis
ściany, teraz trochę powyżej ich linii wzroku. -- Widzisz szczeliny tam,
gdzie światło przebija? Na olej. Stopnie musiały być wygodne, może, dla
kogokolwiek, kto był w~zamku i~śmiertelną pułapką dla atakujących,
których byli skuszeni ich wykorzystaniem.

-- To zachęcające.

-- Pójdę pierwszy -- powiedział. Podszedł i~wyciągnął dłoń. Wzięła ją,
zarumieniła się i~spojrzała w~dół.

Z jedną ręką na poręczy i~drugą podtrzymującą ją z~tyłu, zaczął zejście.
Jej buty były płaskie i~elastyczne, zrobione z~czegoś, co nie było skórą
i co się trzymało, kilka szybkich spojrzeń do tyłu, lepiej niż jego.

Gdzieś w~połowie, coś wielkiego i~białego wyskoczyło, pohukując ze
ściany metr naprzeciw twarzy. Jego nieumyślne szarpnięcie do tyłu
spowodowało, że uderzył tyłem głowy w~brzuch Lydii. Ich krzyki,
chwytanie i~potknięcia były jednoczesne.

Ryzykowna chwila minęła. Spojrzał w~górę na bladą twarz Lydii. Jego
własna paliła. Wzajemnie puścili się części ciała drugiego, które
złapali.

-- Wszystko w~porządku? -- spytał.

-- Tak. .. -- odpowiedziała. Jej głos lekko drżał. -- Co to, do cholery,
było?

Wskazał. Kilkadziesiąt metrów w~powietrzu, biały kształt z~metrowymi
skrzydłami i~czarnymi pazurami kołował na upustach. Gdy zawrócił, jego
wielkie oczy wyglądały, jakby patrzyły na niego.

-- Nocny nietoperz -- powiedział Gregor. -- Polują na małe nocne ssaki.

Odwrócił się, zauważając czarną pustkę pomiędzy blokami, z~których
nietoperz się pojawił. Ciche, oburzone głosy doszły ze środka.

-- O! -- powiedział, onieśmielony pomimo wszystkiego. -- Gniazdo.

-- Możemy do niego zajrzeć?

Popatrzył na nią, pod wrażeniem, i~potrząsnął głową.

-- Nie, niestety. Rodzic może \emph{naprawdę} się zdenerwować. A~my tego
nie chcemy.

Spojrzała na kołującego czujnego drapieżnika, potem z~powrotem na niego
z czymś, co wyglądało na prawdziwy żal. 

-- Nie -- powiedziała -- zdaje się,
że byłoby to niemądre.

Trzymała jego dłoń ciaśniej, póki nie dotarli do gruntu. Nie puścił jej,
gdy odwrócił się, by spojrzeć w~twarz. Sięgnął drugą ręką i~ona ją
wzięła.

-- To było ekscytujące -- powiedziała ze śmiechem. -- Nie róbmy tego
więcej.

-- Przepraszam za\ldots

-- Nie, wszystko w~porządku. Nie mogłeś wiedzieć o~gnieździe.

-- Nie wiedziałem. Lata minęły, gdy ostatni raz schodziłem tym zejściem.

Lydia uśmiechnęła się i~puściła jego dłonie, ocieniła oczy i~spojrzała w~górę na ścianę widoczną nad nimi. Nietoperz nocny wrócił do swojego
gniazda. Dookoła ścian stada znacznie mniejszych nietoperzy, z~długimi,
ostrymi skrzydłami szybowały i~spadały z~głośnymi ćwierkami, łapiąc
owady w~locie.

-- Te są nazywane połykaczami -- powiedział Gregor. Przez mijające chmury
poczuł, że ściana upada. Spojrzał w~innym kierunku, na jej ciągle
podniesioną twarz i~delikatny drżenie w~jej gardle.

-- To niesamowita sprawa -- powiedziała. -- Ten zamek. -- Pochyliła się do
przodu i~wyciągnęła dłoń do górnej krawędzi najniższego rzędu bloków. --
Takie wielkie, takie\ldots przed-ludzkie. Ale ludzie też.

-- Zbudowane przez gigantów -- zgodził się Gregor.

On i~Lydia, bez mówienia, zaczęli iść wzdłuż ścieżki, która prowadziła
na cypel kilkaset metrów do brzegu klifów. 

-- Macie takie wieże na Nova
Babylonia?

-- Nova Terra -- skorygowała go Lydia. -- Tak, kilka, na dzikich brzegach
jak to. Miasto, niektóre stare świątynie tak wyglądają, ale wiemy, że
zostały zbudowane przez ludzi, którzy chcieli poczuć się mali.

-- Och. -- Gregor nie pomyślał o~tym. -- Jacy są bogowie w~starych
świątyniach?

Lydia nagle zadrżała. 

-- Byłam zabrana do jednej, kiedy byłam dzieckiem,
na naukę. Wielka pusta przestrzeń, ponura, oświetlona pachnącymi lampami
olejnymi. Kamienne rzeźby w~niszach tak wysokie, jak ta ściana,
dwadzieścia, trzydzieści metrów. Ale te przedstawiały wielkich królów i~uskrzydlone cherubiny, nie bogów. Posąg boga był w~północnej części
świątyni i~był całkiem mały jak głaz, na wysokość mężczyzny. Był
wyrzeźbiony, rzeźbiony!, z~żelaznego meteorytu. Ciężko zapamiętać
kształt, ale był bardzo, bardzo brzydki i~wydawało się, że jest pełno
oczu. Nie ludzkich czy zwierzęcych oczu. Ciężko wytłumaczyć dlaczego,
ale wiedziałam, że to są oczy. A~to, co wyglądało jak rdza, było
antycznymi śladami krwi.

Lydia zaśmiała się i~zamachała dłonią, jak gdyby chciała rozwiać
ciemność jej słów. 

-- Wydostałam się ze świątyni tak szybko, jak moje
nóżki pozwoliły mi iść!

-- A~od tamtej pory dziękowałaś bogom za Epikura?

-- Tak! -- Lydia podniosła ramię w~geście retorycznym i~wyrecytowała:

-- \emph{,,Kto uderzył w~płonące mury świata i~zabobony zrzucił w~ruiny''}

Gregor spojrzał na nią z~boku, zaskoczony i~zadowolony. 

-- Znasz Dobre
Księgi?

-- Och tak, używamy tłumaczeń Mingulayan, żeby nauczyć się angielskiego.

-- Nic dziwnego, że czasem brzmisz osobliwie -- dokuczył Gregor, potem
ustąpił. -- Nie, naprawdę, Twój angielski jest zadziwiająco dobry.

-- Och, wiem -- odpowiedziała Lydia. -- Mam nadzieję używać go często.

-- Już to robisz.

Potraktowała to jako komplement zgodnie z~intencją. Gregor był
zadowolony, że nie zauważyła bólu, który się za tym krył.

Poszli cyplem, aż dotarli do końca, który wznosił się jak dziób statku
wyżej nawet niż Zamek. Ścieżka zakręciła kilka metrów od samej krawędzi
klifu. Oboje spojrzeli na krawędź, potem na siebie, roześmiali się.

-- Nie mogę -- powiedział Gregor.

-- Ja też.

Lydia przyklękła i~poczołgała się na czworaka do przodu. Po chwili
wahania, Gregor zrobił to samo. Było coś śmiesznie uspokajającego w~tym,
że ostatnie metry były nachylone w~górę. Racjonalnie wiedział, że był
bezpieczny, klif był zbudowany z~solidnej skały metamorficznej, bez
skłonności do kruszenia. Irracjonalnie, żywo wyobrażał sobie oderwanie
szczytu.

Dotarli na skraj, posuwając się centymetrami na palcach i~łokciach, i~spojrzeli, czarne skały, biała woda, a~w ogromnej masie mieszającego się
powietrza, tyły morskich nietoperzy i~nietoperzy nurkujących wynoszonych
na pływach. Paznokcie Gregora kopały w~cienkiej warstwie gleby.
Przemyślanym wysiłkiem woli oderwał jedną dłoń i~położył ją na plecach
Lydii. Jej ciepło ciała uderzyło w~niego przez suchą, papierową teksturę
materiału. Usłyszał szelest i~okazało się, że głaszcze ją, wyżej i~niżej. Lydia zamknęła oczy. Poczuł, że mięśnie jej pleców się
rozluźniają.

-- Mmm, -- powiedziała -- to miłe.

Otworzyła oczy, ciągle patrząc w~dół i~przesunęła się w~taki sposób, że
opierała się bokiem o~niego. 

-- Czuć się zagrożoną, a~w tym samym czasie
czuć się trzymaną i~bezpieczną.

Jego całe ramię leżało teraz w~poprzek niej, jego dłoń pomiędzy jej
ramieniem a~jej piersią. Ich ramiona były nad samą krawędzią klifu, ich
twarze patrzyły się w~morze. Leżeli tak przez długi, w~ich poczuciu,
czas, ryk krwi w~jego uszach i~bicia serca zagłuszał dźwięki fal
przybrzeżnych i~ostre krzyki nietoperzy.

Odwrócili się do siebie. Ich twarze, teraz centymetry od siebie,
nieubłaganie ciągnęły do siebie jakby przez grawitację. Jej oczy się
zamknęły, jej usta otworzyły do jego. Całowali się ponad przestrzenią
przez długą minutę, potem ona się oderwała.

-- To nie jest bezpieczne -- powiedziała, potem, w~odpowiedzi na jego
uśmiech dodała -- możemy nie przestać na całowaniu i~możemy być
obserwowani. To byłoby żenujące dla naszych rodzin.

Lydia przetoczyła się na plecy, siadła i~wstała w~jednym ciągłym,
płynnym ruchu. Kilka szybkich muśnięć jej dłoni przywróciło sukience
idealny kształt, nie zostawiając śladu trawy, wilgoci czy zmarszczki.

Szedł za nią z~powrotem ścieżką i~poszli razem z~powrotem do Wieży.

\threeast

W niedzielne wieczory Bailie's Bar był stosunkowo czysty i~cichy. Zwykła
klientela rybaków i~marynarzy, przygotowując się do wczesnej pracy w~poniedziałek, zostawiła bar studentom i~byłym studentom, którzy osiedli
w dorywczych pracach i~życiu studenckim, zanim, w~wielu przypadkach,
znaleźli prawdziwy zawód.

Gregor pił i~palił z~Salasso i~Elizabeth, z~bratem Anthonym oraz dwoma
jego przyjaciółmi, Muir i~Gunn. Smutno opowiedział im dyskretnie
zredagowaną wersję niepowodzeń.

-- Nie mogę znieść bycia bez niej -- podsumował.

-- Więc dlaczego teraz z~nią nie jesteś? -- spytała Gunn. Gunn była bystrą
studentką z~kręconymi, czerwonymi włosami.

-- Musi pomóc w~rodzinnych interesach -- wyjaśnił nieszczęśliwy Gregor. -
Może jutro ją zobaczę. Ona i~jej ojciec są zainteresowani naszą pracą w~stacji morskiej.

-- Czy to dozwolone? -- spytał Anthony.

-- Oczywiście, że jest cholernie dozwolone. Nie badamy niczego tajnego.

\emph{W każdym razie nie na stacji.}

-- Nie miałem na myśli dzielenia się wynikami badań -- powiedział
Anthony, złośliwie się uśmiechając. -- Miałem na myśli Ciebie i~nią
miziających się w~pracy. Te wszystkie feromony w~powietrzu,
prawdopodobnie zepsułoby to eksperymenty.

-- Och, zamknij mordę!

Jego brat traktował to jako bezwstydną rozrywkę. Anthony nie widział
Gregora robiącego z~siebie głupka od czasu, gdy Gregor w~wieku ośmiu lat
złamał nogę, spadając z~drzewa, więc teraz maksymalnie wykorzystywał sytuację.

-- Trafiło Cię mocno -- powiedział Anthony.

-- Zdecydowanie -- powiedziała Elizabeth. 

Spojrzała na Gregora. 

-- No weź -- powiedziała. -- Wiesz, co robić. Lukrecjusz, księga czwarta\footnote{odniesienie do~De Rerum Natura
Lukrecjusza, zob.~\url{https://pl.wikipedia.org/wiki/De_rerum_natura} - przyp.tłum.}, wersy od 1063 do 1065 czyli

\begin{verse} \itshape
Trzeba więc strawę miłości obrzydzać wyobraźnią,\\
Umysł zwracać gdzie indziej, a~nadmiar nasion płynnych\\ 
W~różnych ciałach przygodnie umieszczać, coraz innych\ldots \\
\end{verse}


Ten pomocny cytat z~Dobrych Ksiąg był zwykle proponowany w~przyjacielski, kojący sposób zakochanym, ale Elizabeth powiedziała go
gorzkim tonem, którego Gregor nie rozumiał.



\chapter[Wrota Krainy Marzeń]{8 Wrota Krainy Marzeń}


Idąc w~deszczu po mokrym asfalcie, czułem się wystawiony. Moje obcasy
unosiły się z~trudnością, a~w głowie czułem ciężar, jakby po śnie.
Przede mną, prawie dwadzieścia metrów dalej, wielka prostokątna skrzynia
majaczyła się w~syczących kroplach. Znacznie dalej, ponad dwieście
metrów, żurawie i~kontenerowiec gromadziły się jak stacja kosmiczna.
Jason załatwił mi agenta podróżnego lub szmuglera ludzi, wg
\emph{Pravda}, i~ten przedsiębiorca wziął cztery tysiące euro w~gotówce
za miejsce w~barce w~Leith Docks. Resztę dnia i~całą noc zabrało
dotarcie barki do Firth of Forth, by minęła ruchliwe miejsce przy
rafinerii ropy w~Grangemouth i~opuszczone stanowisko elektrowni w~Longannet, potem przez kanały Forth i~Clyde i~dalej wzdłuż Clyde.
Dotarliśmy do Greenock i~terminala kontenerów na Atlantyk.

Poszedłem do wielkiej skrzyni i~obszedłem dookoła, przenikliwie świadom
sterowca Harbor Patrol wiszącego pod chmurami, kilkaset metrów dalej nad
Firth of Clyde. W~najdalszej części kontenera, która przypadkiem była
odwrócona tyłem do wody i~skierowana do innych kontenerów, więc trudno
byłoby ją zobaczyć z~odległości, były żelazne drzwi. Obróciłem klamkę.
Ze środka dobiegł hałas szurania, buszowania. Gdy drzwi się otworzyły,
małe światełko padło do środka i~natychmiast odbiło się od oczu
zgromadzenia może tuzina ludzi z~tyłu, poza tym pustego, kontenera. Nie
całkiem pustego, ich małe paczki własności leżały rozrzucone na
podłodze. Nie mogłem wymyślić niczego do powiedzenia, więc podniosłem do
nich dłoń, wszedłem, zamykając drzwi za mną. Ciemność opadła na mnie jak
kaptur z~filcu. Stopy ruszały się niewidzialne. Założyłem gogle i~włączyłem podczerwień: ludzie z~tyłu zaczęli się rozkładać wzdłuż boków,
stopy na podłodze, plecy na ścianie. Poruszali się jakby nie mogli
widzieć po omacku, potykając się. Ustawiłem się w~podobny sposób jak
oni.

Minęło około półgodziny. Raz lub dwa dziecko zaszeptało, a~dorosły
syknął uciszająco. Ktoś mamrotał coś o~papierosie. Potem pojawił się
dźwięk silnika i~grubych opon na mokrej powierzchni, brzęki,
grzechotanie łańcuchów i~uderzenia i~drapania, gdy jakieś połączenie
było robione na zewnątrz skrzyni. Po chwili narastającego wysiłku,
kontener został uniesiony z~ziemi i~poniesiony. Więcej szarpania i~brzęków, krzyki, a~potem bujanie, teraz nieuciszany płacz dzieci. Mogłem
poczuć, że kontener podnosi się jak w~windzie, i~próbowałem nie
wyobrażać sobie wysokości. Znowu opadł na dół i~był zwolniony, w~końcu,
delikatnie, na swoje miejsce.

Po tak wielu ruchach, ten z~początku wyglądał na bezruch, ale po minucie
cichej uwagi było oczywiste, że powierzchnia, na której stał kontener,
sama się poruszała, w~lekkim i~subtelnym rytmie. Byliśmy na statku. Po
kilku minutach mogliśmy poczuć pod stopami pulsowanie silnika, a~kołysanie się podłogi lekko wzrosło.

Czekaliśmy tam, w~ciemności, przez kolejne sześć godzin, w~których
jedyną rozrywką było długie nagromadzenie narastający podnieconych
szeptów, zanim jedno z~dzieci się odlało. Blask ciepła kałuży powoli
blaknął. Ten cykl został powtórzony jeszcze kilka razy.

Ktoś zapukał w~drzwi, niezbyt mocno. Dźwięk ciągle brzęczał i~nas
pobudził.

-- W~porządku -- powiedział męski głos z~zewnątrz -- możecie już wyjść.
Otworzę bardzo wolno drzwi, ok?

Wachlarz światła od drzwi stopniowo się rozszerzał, dając oczom czas do
dopasowania. Schowałem do kieszeni gogle i~czekałem, pozwalając innym,
,,nielegalnym'', jak ciągle myślałem o~nich, protekcjonalnie \emph{nie}
włączając siebie, by wyszli na pokład. Wszyscy szli naprzód. Rodzina,
mężczyzna, kobieta, dwoje małych dzieci, oraz piątka nastolatków i~mężczyzna, który wyglądał nieco starzej niż ja. Poszedłem za nim.

Statek był tak kurwa wielki, że kiedy wyszedłem gdzieś pośrodku pokładu,
z trudem czułem, że jestem w~ogóle na statku. Poza statkiem tak daleko,
jak widziałem, nie było nic, prócz stalowo-białej szarości aż do
horyzontu. Horyzont trochę ruszał się do góry i~do dołu, to wszystko.
Pokład był niską, otwartą przestrzenią pomiędzy superstrukturami dziobu
i rufy oraz labiryntem ciasno upakowanych i~przymocowanych kontenerów.

Człowiek, który otworzył drzwi dla nas, był niskim krępym czarnym
Amerykaninem ubranym w~dżinsy i~podkoszulkę oraz imponującą kolekcję
świecącego hardware dookoła palców, nadgarstków i~szyi. Pojedynczy
obsydian zakrzywiony w~pasmo na jego owiniętych goglach stał się
przezroczysty i~tak nieodbijający, że prawie niewidoczny. Uśmiechnął się
do nas, gdy staliśmy, mrugając w~południowych słońcu i~oddychając
świeżym powietrzem.

-- Cześć -- powiedział -- witamy w~wolnym świecie i~tak dalej. Jesteście
poza wodami terytorialnymi komuchów, więc możecie robić, co do cholery
chcecie. -- Wskazał kciukiem do tyłu. -- Tak długo, jak nie wchodzicie w~drogę i~kapitan nic nie mówi, oczywiście.

Ludzie stłoczyli się dookoła niego, przytulając się, całując, płacząc.
Starszy facet tak właściwie ucałował pokład. Patrzyłem oszołomiony.
Poczułem ulgę, oczywiście, byłem bezpieczny, bezpieczny do Państwa i~nie
wpadłem w~gorsze ręce, ale zachowanie innych uderzyło mnie jako
przesadne i~zimne.

\threeast

Przez kilka kolejnych dni zrozumiałem, że byłem w~błędzie. Ich reakcja w~ogóle nie była przesadna. Pierwsze, marynarz miał, dosłownie, rację,
udało nam się dostać do Stanów, ponieważ nie było kontroli imigracyjnej
dla wjeżdżających z~Unii.

Załoga tego wielkiego statku była tej samej wielkości jak barki, która
zabrała mnie ze Szkocji. Mężczyźni byli bardziej technikami niż
żeglarzami. Nie pamiętam ich imion, a~ich twarze mieszają mi się w~pamięci, ale ich szybkie nieopatrzne miny i~głośne nieironiczne głosy
ciągle brzmią w~mojej pamięci. Nawet sposób poruszania był ekspansywny,
nieskrępowany. Na wachcie i~poza nią, ich uwaga przelatywała i~przeskakiwała pomiędzy światem realnym i~światami wirtualnymi tak
szybko, że wydawało się czasem, że ich gogle błyskały jak stroboskopy
pomiędzy przezroczystym a~nieprzezroczystym. Ich ręce, kiedy nie były
zajęte, zginały się w~seriach pięciopalczastych akordów alfabetu
wirtualnych klawiatur, a~ich usta synchronizowały się w~cichych
rozmowach.

Nie wszyscy byli wysocy, ale każdy z~nich wyglądał, jakby zawsze był
wysoki. To nie było tak (zauważyłem, że ja, też, zaczynam się
prostować), że mieliśmy ciężar na plecach, ale że żyliśmy całe życie pod
niskim sufitem.

Nawet mój ręczny czytnik i~gogle biotech wyglądały, jakby się
rozjaśniły, a~mój dostęp do źródeł amerykańskich stał się łatwiejszy,
ale to równie dobrze mogła być iluzja. Używałem intensywnie sprzętu
komputerowego i~komunikacji. Ceny na statku, jedzenie, koje, ponieważ
nasze tysiące, jak nam powiedziano, ledwo opłaciły nasz przejazd,
spowodowały, że nasza gotówka szybko znikała. Byłem bardziej szczęśliwy
niż moi koledzy podróżnicy w~tym, że mogłem zacząć nową pracę w~wspaniałym Nowym Świecie od razu na statku, tymczasem oni zwiększali
zadłużenie. Moje poważne konta zawsze były zagranicą i~ciągle były
ważne. Pogrążyłem się w~rynku pracy Nowego Jorku, szukając prac w~systemach odziedziczonych. Wydawało się, że starzy geeki wypadli z~biegu, więc byłem w~zapotrzebowaniu. Róg kantyny statku stworzył
odpowiednie biuro, z~wygodnym kubkiem kawy bez dna.

Zadzwoniłem do matki. Jej obraz pojawił się przestrzeni danych. Miała
trzydzieści pięć lat w~tym czasie i~mogła uchodzić za młodszą ode mnie,
prócz tego ostrożnego, zmęczonego wyglądu.

-- Wpod żeś w~problemy\footnote{oryginalnie wypowiedzi
matki protagonisty są zapisane w~szkockiej odmianie języka angielskiego. Aby zachować
zamierzoną przez autora odmienność, wykorzystany został automatyczny
translator polsko-śląski
https://silling.org/translator/?dir=pol-szl\#translation - przyp.tłum.}
-- powiedziała tonem ponurej satysfakcji. -- Kaj je żeś?

-- W~drodze do Ameryki.

Wyglądała na lekko zaskoczoną. Była najbardziej konserwatywną osobą,
jaką znałem. Wierzyła w~Rewolucję.

-- Dycki Ci godałach, iże bydom same problemy z~tymi anarchistami i~ta
jankeską.

Po chwili złagodniała. 

-- Uwożej na siebie, synku.

-- Tak, tak zrobię, mamo.

Uważałem na siebie, pomyślałem po tym, jak się rozłączyła, już dość
długo.

W minutach pomiędzy pracami kontraktowanymi na godziny, przejrzałem
pocztę i~wiadomości, próbując dowiedzieć się, co się stało z~Jadey, co
się działo w~domu i~o Krainie Marzeń. Odpowiedzi były powiązane, co mnie
nie zaskoczyło.

\threeast

Ciągle miałem ślad oryginalnych rekordów Jadey, które Jason i~Curran tak
fachowo zniekształcili. Pracując wstecznie, trafiłem\ldots

-- Co Pan do cholery kombinujesz?

Kobieta wsadziła twarz w~moje pole, zmuszając mnie do lekkiego odskoku.
Rozdzielczość nie była na tyle dobra, by określić, czy była realna czy
reprobotem, ale z~pewnością wyglądała na wystarczająco oburzoną, jej w~innych sytuacjach życzliwe, czterdziestokilkuletnie cechy rozognione jak
henna na jej niemodnych włosach.

-- Szukam informacji na temat Jadey Ericson -- wymamrotałem do
laryngofonu. Kobieta odsunęła się trochę i~skonsultowała coś poza polem
widzenia.

-- A~co \emph{Ty} o~niej możesz wiedzieć? -- zażądała odpowiedzi.

-- Ostatnim razem, gdy ją widziałem, miała być aresztowana przez
komuchów.

-- Och! -- Wpatrzyła się we mnie. Arkusze danych lśniły między nami jak
gorące powietrze. -- Mówisz, że \emph{tam } byłeś?

-- Ta, w~Edynburgu -- odpowiedziałem. -- Co Ci do tego?

Jej oczy zwęziły się, a~jej mina stała się spokojniejsza.

-- Spinguję Cię -- powiedziała. -- I~lepiej bądź tym, kim myślę, że jesteś,
inaczej cię wyrzucę.

Mogłem poczuć, że mój sprzęt był przesłuchiwany. Uczucie było
nieprzyjemnie przerażające. Delikatna linia światła zeskanowała moje
oczy, zanim miałem szansę mrugnąć. Nie, żeby w~tych czasach skan
siatkówki był coś warty, ale niewykonanie takiego byłoby z~jej
niedbałością. W~międzyczasie wysłałem tłum agentów AI ryjących
połączenie, które stworzyła, zaczynając ze mną rozmowę. Wróciły w~sekundę, błyskając danych organizacyjnymi. Szybki rzut oka przed
zapisaniem dał mi powidok organizacji zwanej Federacją Praw Człowieka,
ze stosem firmowych sponsorów, każdy z~ciągiem imponująco wyglądających
liter po jej lub jego nazwisku: ludzie biznesu, kilku związkowców,
akademicy, inżynierowie, standardowy Nuworyszowski think-tank.

-- Człowieku -- powiedziała kobieta, wyglądając na bardziej zrelaksowaną -- ten biodegradowalny sprzęt komuchów to dno.

-- Nie lekceważ go -- powiedziałem zadowolony, drapiąc przód szyi.
Laryngofon miał już kilka dni i~powodował wysypkę jak tępa żyletka. --
Czyli to FPC wysłała Jadey do Europy?

-- Ha, bystry -- odparła niechętnie kobieta. -- Ta, jesteśmy jej
zwolennikami. A~Pan, panie Cairns, musi być Cienkim Czerwonym. Jej
źródłem hardware'u.

-- Czy to -- spytałem -- wykracza poza ,,musi wiedzieć'' na tym etapie?

Kobieta wzruszyła ramionami. 

-- Och, przeklęte komuchy mają to teraz
wszystko ujawnione jak w~książce. Ale ta, możesz mieć rację. Daj mi
pozycję, a~spotkamy się na poważną rozmowę.

-- Wrota Krainy Marzeń -- powiedziałem.

-- Aha. Nieźle. Ok, do zobaczenia tam.

-- Gdzie to jest?

-- Znajdziesz.

Wymigała się, zostawiając mnie wpatrzonego w~strukturę danych, którą
moje AI w~międzyczasie cierpliwie złożyły. To było zbyt proste dla
paranoika, prawie zbyt proste do zignorowania dla profesjonalnego
paranoika jak aparatczyk bezpieczeństwa i~szło tak:

Kompleks startowy ESA w~Kourou we Francuskiej Gujanie był, kilka lat
temu, miejscem małego skandalu, rozegranego przez skrzydło
protekcjonistów w~Partii w~Europie. Jeden z~fabrykantów w~Kourou kupował
komponenty rakiety nośnej, nie od należycie dotowanej fabryki w~jakimś
zapomnianym miejscu w~Angoli czy gdzieś, ale od firmy amerykańskiej. Ta
firma, Nevada Orbital Dynamics, miała wiceprezesa wśród założycieli  FPC i~fabrykę w~Groom Lake w~Nevadzie.

Miejsce znane również jako Strefa~51 i~jako Kraina
Marzeń\footnote{dosł. Dreamland, patrz.
https://pl.wikipedia.org/wiki/Strefa\_51 - przyp.tłum.}.

Poseł Parlamentu Europejskiego z~obwodu Kourou, Weber, stanowczo bronił
fabrykantów, a~po przedstawieniu istotnych bilansów i~dowodów jakości w~debacie parlamentarnej, krytycy umowy, mamrocząc, się wycofali.

Pewnie, myślałem, wsparcie Webera, jeżeli to właśnie się zdarzyło, nie
było takie proste i~oczywiste. Z~trudem mogło być podstawą jego krytyki,
Weber i~fabrykanci robili to, co spodziewano się po nich, zarówno
komercyjnie i, w~kontekście pokojowej koegzystencji (linia du jour
Partii), politycznie, a~nawet mógłby wykorzystać to jako obronę
przeciwko zmyślonych zarzutów ,,dywersji'', jak niezyskowne umowy zwykle
były nazywane, gdy jakiekolwiek oskarżenia musiały być postawione
retrospektywnie.

Kolejna część mojego umysłu pomyślała, \emph{Bingo!}

\emph{Wróciłem} z~nową nadzieją do szukania innych informacji o~Jadey.
Była w~wiadomościach, w~ostrożnie niejasny sposób: schowana w~dolnym
rogu wewnętrznej strony edycji online \emph{Europa Pravda} i~raziła na
pierwszych stronach całkowicie nieoficjalnych amerykańskich kanałów
informacyjnych. DLACZEGO, żądały, NASZ TAK ZWANY RZĄD NIC nie robi, by
uwolnić tę NIEWINNĄ AMERYKANKĘ?

Amerykański rząd, w~małych akapitach w~\emph{New York Times} i~\emph{Washington Post} postulował niejasno i~pośrednio użycie
,,właściwych kanałów''. Każdy mógł próbować zgadnąć, czy oznaczało to, że
rząd zapamiętale poruszał ten problem w~najwyższych kręgach dyplomacji
czy przesyłał pytania z~konsulatu do komendy policji w~Edynburgu.

Cały problem z~Jadey, który w~innych przypadkach mógłby łatwo być
napompowany do cause celebre, był kompletnie przesłonięty zamieszaniem
wobec aresztu Webera (ton rządu U.S. o~zarzutach przeciwko niemu brzmiał
jak zraniona niewinność) i~znacznie większym zamieszaniem wobec kontaktu
ESA z~obcymi kosmitami. Ponadto, świat kolektywnie i~przewidywalnie
tracił głowę. Przeglądając wiadomości z~ostatnich kilku dni, wyglądało,
że każdy naukowiec, filozof, ksiądz, generał, polityk i~kabareciarz na
planecie, i~poza nią, był zachęcony do wypowiedzi. Przekazałem powstałą
kakofonię partii świeżo powstałym AI, żeby zamieniły to w~jakiś strawny
format i~odwróciłem się z~pewną ulgą do kolejnego kontraktu ze szczytu
mojej listy.

Kojąca ulga rutynowej pracy hakerskiej nie trwała długo. Dwadzieścia
pięć minut później, AI zaczęły błyskać pilnie, a~kucharz statku Pan
Nguyen pośpiesznie wyszedł z~kambuza i~uderzył w~stół. Zapisałem pracę
na serwerze i~przełączyłem się na oba przerwania, człowiek pierwszy.

-- Wielkie wiadomości dla Ciebie -- powiedział mi. -- Dla nas wszystkich.
Sprawdź CNN.

 -- Dzięki -- odparłem. AI ponaglały mnie, żebym zrobił to
samo. Podążyłem za ich radą. Globalne media nie miały wątpliwości, co
było najważniejszą globalną wiadomością, spychając debatę o~kontakcie z~obcymi bezceremonialnie w~dół:

\textbf{BUNT NA STACJI ESA.}\\
Naukowcy i~kosmonauci stacji naukowej ESA \emph{Titow} zaapelowali
dzisiaj do światowej społeczności o~zapobieżenie ,,militaryzacji'' ich
historycznego kontaktu z~obcą inteligencją. Najwidoczniej bezkrwawa
walka wyparła pięciu wojskowych reprezentantów z~komitetu zarządzającego
stacji. Były szef ochrony stacji, Colin Driver, dotychczas uważany za
całkowicie solidnego komisarza Komunistycznej FSB, przeprowadził ten
ruch. W~osobistym oświadczeniu Driver zapowiedział:\\
\emph{Kliknij klip}.

Driver zajął ekran. Dla mnie to wyglądało, jakby siedział naprzeciwko przy
stole. Za nim, w~tle pola, kilku ludzi czepiało się pod różnymi kątami
podpór i~uśmiechało się dziko do kamery. Wyglądali jak naukowcy. Driver
był krępym, muskularnym mężczyzną w~mundurze z~wieloma medalami. Jego
twarz mogła być słowiańska, ale głos i~akcent (w nieprzetłumaczonej
wersji, którą dostałem) był bez wątpienia południowoangielski.

-- Nie jestem przyzwyczajony do publicznych wystąpień, więc będę mówił
krótko. Trzy dni temu, Sekretarz Główny Jefrimowicz przedstawił
oświadczenie, które wstrząsnęło światem. Czas tego oświadczenia, po
którym wielu słusznie wydedukowało o~latach tajemnicy, wzmógł szerokie i~alarmujące spekulacje. Moi przyjaciele, muszę wam powiedzieć, że
niektóre z~tych spekulacji są częściowo usprawiedliwione. Prawie na
pewno nieznane Sekretarzowi Głównemu i~przewodniej partii braterskich
krajów, złośliwe i~reakcyjne elementy w\ldots

Driver przestał mówić i~potem powiedział: 

-- Och, do cholery z~tym gównem komuchów! -- Konwulsyjnie zerwał wstążki i~odznakę z~kurtki i~wziął
głęboki wdech.

-- Ok -- kontynuował. -- Ludzie, powiem to prosto. Niektórzy z~twardogłowych generałów w~Ludowej Armii Europy myślą, że mogą
wykorzystać to, co nauczyliśmy się od obcych, żeby mocno trafić
Amerykanów, wygrać Czwartą Wojnę Światową i~dokończyć światową rewolucję
za, co uważają za koszt do przyjęcia, życie kilku milionów. Wcześniej
lub później, i~lepiej wcześniej. Policzcie to sobie. Ale zapewniam was,
nie mają wszystkich informacji, których potrzebują. Mają niektóre,
większości waszych szyfrów jest złamanych, jak pewnie zgadliście. Ale
jeszcze nie złamali amerykańskich kodów startowych. Oświadczenie było, z~ich punktu widzenia, przedwczesne, ale to może nie zatrzymać pewnych
pochopnych działań.

-- Więc my, naukowcy, kosmonauci, ochrona \emph{Marszałka Titowa}
zdecydowaliśmy się zrobić, co możemy, żeby temu zapobiec. Zatrzymaliśmy
naszych militarystów w~areszcie ochronnym i~namawiamy rząd, siły
zbrojne, Partię i~lud UE, by zrobiły tak samo. Dopóki nie jest to
zrobione, żaden bajt nie opuści tej stacji bez równoczesnego
opublikowania w~sieci publicznej.

-- Jesteśmy gotowi wypuścić reprezentantów wojska pod jednym warunkiem,
że wszystkie zarzuty przeciwko Henri Weber zostaną oddalone, że zostanie
bezwarunkowo zwolniony i~będzie mógł negocjować pomiędzy nami a~rządem
UE.

Driver uśmiechnął się lekko. 

-- W~końcu, \emph{jest} posłem reprezentującym
tę stację poprzez kompleks startowy w~Kourou. Jako oficer, \emph{były}
oficer, Federalnego Biura Bezpieczeństwa, jestem całkowicie pewny, że
jest niewinny. Nie jest agentem CIA. Ten zarzut został zmyślony, żeby go
skompromitować, i~nas i~całkiem możliwe, że i~FBB też. Wiem to bardzo
dobrze, ponieważ\ldots

Kolejna przerwa, kolejny głęboki wdech.

- \ldots przez ostatnie pięć lat on i~ja współpracowaliśmy blisko w~dezinformacji CIA i~w izolowaniu prawdziwego agenta CIA na tej stacji,
majora Ivana Suchanowa, który jest teraz z~kolegami w~areszcie.

\threeast

Ruch uliczny Nowego Jorku mnie zaskoczył. Wziąłem taksówkę z~portu na
lotnisko JFK i~usiadłem zdziwiony i~przerażony, gdy taksówka ruszała i~zatrzymywała się pomiędzy i~pośród największych, najgłośniejszych,
ultrabłyszczących, superśmierdzących pojazdach, które widziałem w~moim
życiu. Znaczenie wojen paliwowych nagle stało się jasne tak jak różnica
pomiędzy USA a~UE. W~USA silniki spalania wewnętrznego ciągle
dominowały. W~UE spalanie frakcji ropy było mniej lub bardziej
ograniczone do lotnictwa i~wojska. Reszta przemysłu poszła w~nowe
technologie. Większość samolotów cywilnych w~Europie to sterowce lub
pojazdy hybrydowe. W~USA to odrzutowce. Podróże były szybsze, ale
znacznie mniej komfortowe, na sposób, o~którym właściwie nie chcesz
wiedzieć.

Las Vegas było ćwiczeniem w~dowodzeniu, że przerost architekturalny i~behawioralny w~rzeczywistości ciągle mógł rywalizować z~wirtualną
rzeczywistością. Ale mój najdziwniejszy moment nie pojawił się przy
patrzeniu przez ogromne szklane okna terminala Lotniska McCarran na
jeszcze większe szklano-plastikowe budowle. Wiedziałem już, że Kraina
Marzeń jest miejscem. Stojąc w~terminalu Janet Airlines i~patrząc na
tablicę odlotów, poczułem przewrotne podniecenie i~nierealność, gdy
pierwszy raz zobaczyłem tę nazwę wyświetloną w~kolumnie \textbf{CEL}.

\threeast

Zszedłem kładką z~małego pięćdziesięcioosobowego samolotu i~rozglądałem
się z~pewną obawą, gdy maszerowałem do budynku terminala, stali
pracownicy Bazy żwawo kroczyli przede mną. Lot z~Lotniska McCarran w~Las
Vegas zabrał około pół godziny. We wczesnoporannym słońcu suche dno
Jeziora Groom sprawiło, że dwa pasy startowe lotniska rozmyły się w~olśniewającej plamie światła i~cienia. Pośpieszna praca kciuków na
wirtualach w~moich nowych wyprodukowanych w~Ameryce goglach-szkłach, tak
były tu nazywane, zmniejszyła jasność, a~zwiększyła kolor i~kontrast.
Miejsce nadal wyglądało dziwnie. Płaska równina otoczona górami, na
której produkty ludzkiej technologii stały jak dopiero co porzucone
artefakty obcych.

Była baza Sił Powietrznych była centrum całego źródła mitów, tajemnic i~podejrzeń. Pomiędzy Drugą a~Trzecią Wojną Światową region był używany do
tajnych testów bomb atomowych, rakiet, legendarnego jądrowego silnika
rakietowego NERVA i~najbardziej zaawansowanych i~tajnych samolotów USA:
czarne projekty od U2 przez Blackbird do serii myśliwców stealth
zakończonych niesławnym EDSF. 

Myśliwiec EDSF tj. Electro Dynamic Stealth Fighter, 
był konstrukcją niezmiernie udaną dla wysokich i~szybkich
lotów, unikania wykrycia przez radar, unikania inteligentnych rakiet i~generowania fal podnieconych raportów o~UFO. Jednakże, jak pokazał teatr
wschodnioeuropejski, nie był tak niewrażliwy na staromodne wykrycie
wzrokowe i~ogień przeciwlotniczy od tych, którzy odważyli się wyłączyć
komputer celowniczy, spojrzeć i~zaufać Mocy.

Po katastrofach wojny i~oskarżeniach w~polowaniach na czarownice, cała
sprawa została zamknięta, projekty anulowane, tajemnice wywiezione do
głębokich i~odległych magazynów, a~pozostałe obiekty zmieniły się w~bogactwo prywatnych przedsiębiorstw.

Te uwzględniały to, co było, pewnego rodzaju, organizacjami wariackich
kultów, które straciły lata, przeszukując pustynię i~opuszczone budynku
dla stopów nie z~tej Ziemi, fragmentów dokumentacji, dowodów, że
marynowane ciała obcych z~Roswell kiedyś tu były. Do dzisiaj mam stosy
tego rodzaju gówna na czytniku. Zacząłem się zastanawiać, jak ludzie
\emph{mogą być głupi! } Jeżeli niezidentyfikowane obiekty latające były
widziane niedaleko baz badawczych lotnictwa wojskowego, oczywistym
wnioskiem jest to, że były tajnymi samolotami wojskowymi. \emph{Ale
tajne samoloty powstały z~badań nad odzyskanymi pojazdami obcych\ldots
} Nie, ktoś mógłby pomyśleć, ta szczególna hipoteza pomocnicza nie
pomoże. William Ockham już dawno temu je obalił, ale ciągle, takie
hipotezy się pojawiały, zawierające byty obcych niekoniecznie pomnożone.


Ważniejsze i~w długim okresie wytrzymalsze, okazały się kompanie
kosmiczne, te, których osiągnięcia w~tym momencie wykonywały dodatkowe
manewry nad głowami. Latające trójkąty i~latające dyski ślizgały się po
niebie, ruszały do góry i~znikały w~niebie.

Wszedłem do terminala drzwiami koło zwykłego wzrokowego i, bez
wątpienia, niewidzialnego skanowania przez gości w~kamuflażu i~przeszedłem do chłodu klimatyzacji. Naprzeciwko hali zobaczyłem mój cel,
otwarty bar na lotnisku, którego nazwa wyświetlała się w~błyskającym,
ironicznym neonie:

\emph{Wrota Krainy Marzeń}

W środku przetrwał nurt postmodernistyczny. Ściany były wytapetowane
plakatami science-fiction i~ufologicznymi, i~dodatkowo udekorowane
zniszczonymi zardzewiałymi znakami z~różnych sektorów starego
ogrodzenia, większość z~nich kończyła się słowami: UŻYCIE ŚMIERTELNEJ
BRONI DOZWOLONE. Model szarego obcego stał w~kącie, a~obsesyjnie
szczegółowy polistyrenowy model latających spodków wisiał na czarnych
niewidzialnych niciach spod sufitu, bujając się chaotycznie w~podmuchach
wentylatora. Dziewczyna przy barze nosiła fałszywy aluminiowy skafander
kosmiczny i~chłopak miał znaczek Wackenhut Security przyczepiony do
ubrania. Za nimi, różne smaki i~kolory wódki były ustawione na półkach w~wielkich butlach, w~których mdło realistyczne płody szarych obcych
pływały jak gdyby w~formaldehydzie.

Usiadłem przy stoliku w~rogu z~piwem i~bajglem na śniadanie, zacieniłem
szkła i~rozejrzałem się po barze. Około połowa tłumu, która zajmowała
miejsce, wyglądała na pracowników pożerających szybkie, zaabsorbowane
śniadania nad widokami z~gogli. Spokojniejsi klienci, rozmawiający
głośno lub cicho i~konspiracyjnie, byli najwidoczniej turystami i~starzejącymi się pasjonatami, z~dosypką dziennikarzy, którzy przyjechali
po łatwą zabawę ich kosztem. Prawdziwy kontakt z~obcymi zdmuchnął kurz z~wszystkich starych historii wyobrażonych kontaktów, odświeżył je i~reaktywował, by przetoczyć się raz jeszcze w~mediach jak zombie oblany
dezodorantem, a~Kraina Marzeń stała się raz jeszcze mekką żałosnych i~wariatów i, żeby być sprawiedliwym, niewygodnie dociekliwych.

-- Można się dosiąść?

Kobieta pojawiła się nade mną z~tacą i~sztucznym uśmiechem. Była to
kobieta, która mnie przechwyciła w~danych Jadey, wyglądała dokładnie tak
samo, w~niestosownie wzorzystej bluzce z~miękkim kołnierzem. Nie
widziałem, jak nadchodziła. Usiadła koło mnie, przesuwając mnie wzdłuż
ławki i~zgrabnie blokując mnie przy narożnej ścianie. Dalsze wtargnięcia
terytorialne były zapewnione przez jej pokaźny wybór żywności. Jej męski
towarzysz, wysoki i~ciężki w~ciemnym garniturze, białej koszuli i~przyciemnionych szkłach, usiadł naprzeciw, stawiając obraźliwie kubek
Coca Coli ostrożnie przed nim.

-- Cóż, cześć -- powiedziała kobieta. Wyciągnęła prawą dłoń do mnie, do
uściśnięcia tak dziwnego, że musiało wyglądać jak u wolnomularzy. --
Nazywam się Mary-Jo Greenberg. -- Brew drgnęła. -- A~to jest Al.

Wielki mężczyzna przechylił lekko głowę. 

-- Z~Nevada Orbital Dynamics.

-- Matt Cairns -- powiedziałem. -- Miło mi was poznać.

-- Zakładam, że sprawdziliśmy wzajemnie poświadczenia -- powiedziała
Mary-Jo. -- Wiemy, kim jesteś i~wiesz, kim jesteśmy.

Kiwnąłem głową i~rozejrzałem się. Pomyślałem o~nieuniknionym pytaniu. 

-- Czy możemy tutaj rozmawiać?

Mary-Jo się zaśmiała. 

-- W~miarę, Cienka Czerwona. Bezpieczniej niż
jesteś przyzwyczajony. Oprócz naszych praw o~prywatności i~tak dalej,
tak dużo śmieci jest tutaj omawianych, że wymagałoby to cholernego
specjalizowanego komputera, żeby oddzielić ziarno od plew.

-- Zaufam Twojemu osądowi. -- Wzruszyłem ramionami.

-- Jakieś wieści o~Jadey?

-- Pracujemy nad tym -- odpowiedziała Mary-Jo. -- Mam na myśli, mieliśmy
trochę bezpośredniej komunikacji. Amerykański konsulat w~Edynburgu
pracuje na tyle, o~ile jest to warte. Jadey jest w~porządku. Praktycznie
wszystko, co musi być zrobione to umowa o~wymianę. Powinna wyjść za parę
dni, żaden problem.

-- Och, to wspaniale -- powiedziałem. Te dobre wiadomości połączyły się z~obezwładniającym wybuchem ulgi i~przyjemności z~możliwości porozmawiania
z kimś w~końcu.

-- Więc -- kontynuowałem -- jak bardzo chcecie latający spodek?

-- Ach -- powiedział Al. -- Wydaje mi się, że to miejsce nie jest
dostatecznie bezpieczne, żeby rozmawiać o~\emph{tym}.

\threeast

Biura Nevada Orbital Dynamics znajdowały się w~długim, niskim i,
najważniejsze, klimatyzowanym budynku. Krótki spacer trochę mnie
wycieńczył. Pot parował, zanim miał czas zwilżyć skórę. Potem, jak tylko
wszedłem do wnętrza, stał się lepki i~zimny. Usiadłem w~skórzano-aluminiowym krześle, przełknąłem piwo, żeby zastąpić utracone
płyny, i~popijałem kawę, żeby się rozgrzać.

Biuro, w~którym byliśmy, wydawało się należeć do AI, na drzwiach było
ALAN ARMSTRONG i~był obeznany ze wszystkim w~środku, ale nie przedstawił
się przesadnie. Usiadł ze stopami na biurku, odchylając się i~ssąc
bezdymnego papierosa. Mary-Jo stała przy oknie. Pomalowane betonowe
ściany były nagie, prócz kilku dyskretnych plakatów pokazujących
przekroje niezrozumiałych maszyn i~komponentów, przeskakujących w~swoich
niezrozumiałych czynnościach w~rozpraszająco chwytający uwagę sposób.

Opowiedziałem im swoją historię. Wykazali się mniejszym wrażeniem niż
moi koledzy geekowie. Może już wcześniej słyszeli tego rodzaju historie.
Kiedy skończyłem, położyłem mały dysk danych na biurku Alana. Zebrali
się dookoła, patrząc na niego. Przez chwilę nikt nic nie mówił.

-- Masz co, co może to \emph{odczytać}? -- spytał Alan.

-- Ta, pewnie -- odpowiedziałem rozbawiony. Nie spodziewałem się problemów
z kompatybilnością sprzętu, choć powinienem był. Wyjąłem czytnik,
włożyłem dysk, nawiązałem połączenie i~sięgnąłem po terminal Alana,
potem spojrzałem na niego.

-- Masz coś przeciwko?

Wyciągnął skądś wtyczkę, bystry gość, potem machnął ręką.

-- Nie ma sprawy.

Połączyłem je, podałem mu czytnik i~cofnąłem się.

-- Sprawdź to.

Przez kolejną godzinę czy coś, Alan rył przez dokumentację. Gdy jego
podejrzenia zostały uśmierzone, dodał swoje szkła do interfejsu,
wpisując komendy w~powietrzu i~mamrocząc. Mary-Jo podążała za nim, ale
co jakiś czas przerywała, żeby posłać mi uspokajający uśmiech lub
poprosić o~kolejną kawę. W~końcu Alan zakończył sesję i~odepchnął
sprzęt, zdjął szkła i~spojrzał na mnie. Jego oczy były niebieskie i~łagodne. Skóra dookoła oczu zdradzała zmęczenie ze sprawdzania
specyfikacji.

Przerwałem połączenie, wziąłem czytnik i~go schowałem.

-- A~zatem? -- spytałem.

Alan kiwnął głową wolno z~zaciśniętymi ustami.

-- Cokolwiek to jest -- powiedział -- wygląda cholernie autentycznie.
Fragmenty, które rozumiem, są rozsądne, a~fragmenty, które nie rozumiem,
są\ldots, cóż \emph{obce} w~sposób, który trudno byłoby podrobić. -- Zaśmiał
się krótko. -- \emph{Widziałem} fałszywe dokumentacje obcych latających
spodków, które były naprawdę dobre, stara dezinformacja, sprzed wojny. W~ogóle nie były podobne do tego. Zawsze trafiasz na jakąś bzdurę i~unikanie tematu, kiedy dostatecznie dokładnie czytasz.

-- Boron -- powiedziała Mary-Jo. Z~jakiegoś powodu uważali to za śmieszne.

-- Ta, boron -- westchnął Alan. -- Wiele gadania o~boronie, magnetyzmie i~Tesli. Uważasz, nie mieli wtedy transplutoników, żeby o~nich ględzić,
może to wtedy było równoważnikiem. Nie ma żadnego
,,unobtanium\footnote{wsp. określenie stosowane w~naukach
inżynieryjnych, science fiction i~eksperymentach myślowych, opisujące
bardzo rzadki, drogi lub nawet fizycznie niemożliwy do wykonania
materiał, którego właściwości są niezbędne do uzyskania zakładanego
rezultatu. zob.~https://pl.wikipedia.org/wiki/Unobtainium - przyp.
tłum.}'', ale jedynym sposobem, żeby ta rzecz latała, jest, jeżeli
bylibyśmy wielkimi ignorantami właściwości elementów z~,,wyspy
stabilności''. Kim, jak się zdarza, jesteśmy, więc\ldots -- Rozłożył ręce.

-- Mówimy o~AG? -- spytała Mary-Jo.

-- Coś takiego. -- odpowiedział Alan. Potarł nos. -- Mam na myśli, czym
jest latający spodek bez antygrawitacji? -- Szarpnął głowę do tyłu,
mgliście wskazując na okno. -- Oprócz tych, które latają nad nami,
prawda? \emph{A} ta rzecz, którą dokumentacja nazwy Silnikiem wygląda mi
na napęd gwiezdny.

-- Mówimy o~FTL\footnote{FTL -- ang. faster than light,
szybsze niż światło - przyp.tłum.}? -- spytałem, naśladując Mary-Jo.

Alan potrząsnął głową. 

-- Nie. Ale szybkie.

-- Możemy to zbudować? -- spytała Mary-Jo.

-- Tak, ale tylko w~kosmosie. Proces wymaga środowiska mikrograwitacji. A
jeżeli chodzi o~transplutoniki, kurczę, i~tak są produkowane tylko w~przestrzeni kosmicznej. Przez ESA, żeby być dokładnym. Wydaje mi się, że
moglibyśmy grzecznie poprosić.

Wstałem, czując się niespokojnie. Rozciągnąłem ramiona i~roztarłem
ramiona. 

-- Zastanawiam się -- powiedziałem leniwie -- czy już to
zbudowali, wiecie tam.

Mary-Jo i~Alan spojrzeli na siebie. Ruch ramion Alana nie był
dostrzegalny. Mary-Jo odwróciła się do mnie.

-- Nie -- powiedziała. -- Nie zbudowali.

-- Skąd\ldots Och. Jesteście w~kontakcie z~nimi.

-- Z~powodu buntu. Ta. To nie tajemnica. Robią wszystko jawnie. Chyba że
nie odgrywają jakiś naprawdę skomplikowany podwójny blef, który, biorąc
pod uwagę historię intryg komuchów, nie może być wykluczony. Przesłali
nam ogólny obraz tego, co się nauczyli. Widziałeś coś z~tego, przy
okazji?

-- Nie, byłem zajęty martwieniem się o~Jadey i~polityką w~domu.

-- Opowiedz mi o~tym. -- Mary-Jo się uśmiechnęła. -- Wygląda na to, że
rozpętało się piekło w~Czerwonej Europie, co? \emph{W każdym razie},
poczytaj dane naukowe, w~miarę szybko, są fascynujące. Człowieku, oni
gadają z~\emph{bogami }. Ale załoga \emph{Titowa} nic o~tym nie mówi.
Obcy napędy kosmiczne, mój Boże, pomyślałbyś, żeby o~tym wspomnieli.

-- Och. -- Poczułem zimny prysznic rozczarowania, jakby oczywisty wniosek
wreszcie do mnie trafił. -- Myślisz, że te rzeczy mogłyby być jakąś
,,dezinformacją'', o~której ten facet Driver mówił w~trakcie komunikatu?

Alan potrząsnął głową. 

-- Wątpię -- powiedział. -- Spójrz na znacznik
czasu, zeszły rok. Cokolwiek to jest, było przetwarzane gdzieś w~ESA
przez jakiś czas, a~gdyby rząd amerykański lub nawet CIA dostało w~swoje
ręce coś takiego jak to, na pewno bym o~tym usłyszał. Bardzo pracowicie
podtrzymywałem kontakty przez ostatnie kilka dni i~jeżeli jestem czegoś
pewien, to, że nikt po naszej stronie nie wiedział w~ogóle o~obcych
umysłach. Dostawali stosy czegoś, o~czym myśleli, że były fragmentami
wartościowych danych naukowych, przede wszystkim informatyki i~fizyki
niskich temperatur, i~wszystko było prawdziwe, ale nic, mam na myśli
\emph{nic}, żadnej wskazówki o~prawdzie. Jezu, ci faceci są świetni w~swojej pracy.

-- Driver i~Weber?

-- Tak. -- Alan potarł się po karku. -- Zastanawia mnie, jak zatrzymali
prawdziwego agenta CIA, majora Suchanowa, od zdradzenia tajemnicy, kiedy
nie był na stacji. Miał co najmniej dwa urlopy na Ziemi przez ostatnie
dziesięć lat i~musiał mieć \emph{jakieś} kontakty na dole.

-- Och, to -- powiedziałem. -- Wydaje mi się, że Suchanow jest całkowicie
niewinny, a~Driver oskarżył go o~podburzanie w~Armii i~zdjął uwagę z~FBB.

-- Więc kim \emph{był} CIA\ldots? -- Alan spojrzał na mnie. -- Żartujesz.

-- Driver -- powiedziałem. -- Musiał być. Lub to jest to, co CIA myśli! On
i Weber są podwójnymi agentami. Dlatego istniały poważne dowody
przeciwko Weberowi.

-- Jeżeli ,,dowody'' znaczą cokolwiek w~tym kontekście -- powiedziała
Mary-Jo. Usiadła na brzegu biurka. -- Zastanawiam się, jednak, w~jaki
sposób informacje, które dostałeś, pojawiły się na Ziemi, a~nie na
\emph{Titowie}.

-- Ponieważ powstały na Ziemi? -- zasugerowałem. -- Może informacje o~projekcie przeszły bez analizy i~cała dodatkowa praca została wykonana
tutaj, z~intencją nieinformowania stacji o~niej, póki nie będą pewni
sytuacji bezpieczeństwa, może, żeby uruchomić proces i~zbudować rzecz.

-- Minimum informacji? Ta, wydaje się to rozsądne. -- Alan energicznie
wstał. -- W~tym przypadku najlepsze co możemy zrobić to wysłać to na
stację. Zabrać ten dysk lub kopię, tam, fizycznie.

-- Dlaczego nie nadać?

-- Ponieważ nie chcemy, żeby UE wiedziało, co robimy -- powiedziała Mary-Jo. -- Potrzebujemy kogoś, kto rozumie system i~programy. Jak, ten program kontroli produkcji? Kogoś, kto poradzi sobie
z interfejsem pomiędzy twardą technologią a~biotechem, kto jest
fizycznie zdatny, politycznie bystry i~politycznie pewny -- uśmiechnęła
się do mnie. -- Przykładowo, kogoś takiego jak Ty.
 

\chapter[Utracone lata świetlne]{9 Utracone lata świetlne}


Elizabeth siedziała na stołku przy ławie laboratoryjnej, sącząc dzienną
pierwszą kawę i~patrzyła się na diagramy na ścianie. Komentowane,
poprawione i~pokreślone przebiegi systemu nerwowego kałamarnic wyglądały
tak bez sensu, jak kępa korzeni w~losowej grudzie ziemi. Dookoła niej
akwaria słonowodne generowały ciągły syk, gdy bąbelki sączyły się z~bloczków pumeksu na krańcach tub aeratorów. Mała elektryczna pompa,
które je wszystkie napędzała, mruczała schowana w~kącie laboratorium,
niezawodna jak serce.

Wstanie z~łóżka było aktem odwagi, ubranie się jak założenie zbroi,
jazda tramwajem jak jazda na pole bitwy. Powiedzenia mądrego stoika było
małym pocieszeniem, gdy przyjemność i~ból były wszystkim, co chciałabyś
czuć: cokolwiek prócz tego odrętwiałego smutku. Jedynym pocieszeniem, i~było to niemiłe, było to, że Gregor wkrótce będzie się tak samo czuł.
Wątpiła, by Lydia została, lub Gregor odleciał z~nią, oboje mieli zbyt
dużo powiązań. Każde milcząco oceniłoby, że wyrywanie się z~ich domu
byłoby bardziej bolesne niż rozstanie. Ale to rozstanie byłoby
dostatecznie bolesne. Była lekko zszokowana życzeniem bólu Gregora i~nadzieją, że zwróci się do niej w~chwili słabości.

Bardziej prawdopodobne, ten głupiec cierpiałby miesiącami. Całkowita
złośliwość miłości romantycznej nie mogła być bardziej oczywista w~tym
przypadku. Lub w~jej. Bogowie jedynie wiedzą ile okazji dobrych
jednonocnych przygód lub zdrowych, satysfakcjonujących związków
przepuściła, gdy traciła czas na obsesję o~tego sukinsyna. A~ponieważ
nikt nawet nie wiedział o~tym, podstępnie zdobyła reputację raczej
zimnej osoby, niezainteresowanej seksem. Istnieli tacy ludzie, których
zainteresowanie zajęciami intelektualnymi, umiejętnościami fizycznymi
lub nawet w~biznesie, polityce, nie zostawiało czasu lub energii na
ludzką intymność. Był to godny szacunku, jeżeli nie szanowany, styl
życie, nie tyle podziwiany, ile zastanawiający.

Nie pragnęła być jedną z~takich osób, ale czasem bała się, że jest.
Oczywiście, gdyby była normalna, a~jej pragnienia tak naglące, jak
wydawały się u większości ludzi, przełamałaby skrępowanie sytuacją,
zaryzykowała odrzucenie i~zażenowanie czy nawet ich wzajemną przyjaźń,
tylko po to, aby go dorwać choć raz, zaskoczyć go gorącym, szczerym
słowem lub złaknionym pocałunkiem.

Usłyszała lekkie, szybkie kroki Salasso w~korytarzu i~szybko zmieniła
wyraz twarzy, patrząc z~uśmiechem, gdy wszedł zaur.

-- Dzień dobry -- powiedział. Zdjął fartuch laboratoryjny z~haczyka i~go
założył, nieświadom jak zawsze jak śmiesznie wyglądał: zbyt długi do
jego wzrostu, zbyt krótki dla ramion, zbyt luźny na klatce. Salasso
sięgnął po czajnik i~go włączył, krusząc kostkę bulionu rybnego do
kubka. -- Jesteś wcześnie.

-- Chciałem przemyśleć, co zrobiliśmy.

Trudno było dostrzec, gdzie patrzy, kąciki jego oczu sięgały tak daleko.
Wlał wodę i~zamieszał.

-- Hmmm. Och, już lepiej. -- Salasso sączył bulion i~widoczne się
zrelaksował. Jego gatunek uwielbiał ryby i~nienawidził rybactwa.
Przybycie ludzi na Mingulay przeniosło ryby i~produkty rybie z~rzadkich
luksusów do podstawy w~diecie zaura. Nic nie mogło zmienić ich niechęci
do morza i~strachu do rybactwa pełnomorskiego. Na tyle, na ile Elizabeth
wiedziała, Salasso był pierwszym zaur, o~którym ktokolwiek słyszał,
który postawił stopę na łodzi. Wcale mu to nie przeszkadzało.

-- Tak, możemy dzisiaj oczekiwać gości -- kontynuował. -- Czy dlatego
ubrałaś się inaczej?

Pod jej fartuchem, Elizabeth miała białą jedwabną bluzkę z~wysokim
kołnierzem oraz czarną lnianą spódnicę do połowy łydki, z~czarnym
pończochami i~lekkimi skórzanymi butami. Salasso nigdy wcześniej nie dał
najmniejszego znaku, że zauważa ubrania kogokolwiek.

-- Tak, właśnie tak -- odparła. -- By wyglądać mądrze przed handlarzem. --
\emph{I przed jego córką. } -- Nigdy nie wiesz, kupiec może myśleć o~inwestycji.

-- Lub będzie chciał podzielić się wiedzą -- powiedział zaur, raczej
wyszukanie. -- Mamy tyle do nauczenia się o~oceanach na innych światach.

-- Tak -- powiedziała apatycznie.

Głowa Salasso przechyliła się odrobinę. Mógł być nieczuły na niuanse
ludzkich ekspresji na twarzy, ale szybko łapał ton głosu.

-- Jesteś zmartwiona -- powiedział.

-- To nic, co mogłabym wyjaśnić.

-- Masz na myśli, że nie oczekujesz po mnie zrozumienia. Myślę, że
umiałbym. -- Wielkie oczy zaura spojrzały na chwilę na podłogę, potem na
nią. -- Też mamy takie kłopoty. Ale są znacznie bardziej długotrwałe.

Elizabeth spojrzała się. Zdanie to było najbliższe temu, co zaur
kiedykolwiek powiedział, o~swoim osobistym życiu, lub o~związkach w~obrębie gatunku.

Potem jego wąskie ramiona poruszyły się i~dodał: 

-- Może to sprawia, że
są zbyt inne, żeby dyskusja była korzystna.

Zanim mogła wymyślić cokolwiek do powiedzenia, drzwi na zewnątrz otwarły
się z~uderzeniem i~głosy, potem kroki, się zbliżyły. Gregor otworzył
drzwi z~samozamykaczem do laboratorium i~trzymał je, gdy kupiec i~jego
córka weszli, mijając go. Ubrani byli w~kurtki, bluzy i~dżinsy, jak
gdyby właśnie zeszli z~łodzi. Widok, jak Lydia się nosiła, sprawił, że
Elizabeth poczuła się jednocześnie bez gustu i~zbyt dobrze ubrana.

Szeroka piegowata twarz De Tenebre uśmiechnęła się, jego głos zagrzmiał.

-- Dzień dobry -- powiedział, wystawiając rękę. -- Wydaje mi się, że
poznaliśmy się na imprezie.

-- Jak się masz.

-- A~to moja córka, Lydia. Wątpię, żebyście były sobie przedstawione.
Elizabeth Harkness.

Dobra pamięć do nazwisk. Elizabeth potrząsnęła dłonią Lydii tak lekko,
jak to możliwe. W~tym samym czasie, Tenebre powiedział, lub zaśpiewał,
coś, co sprawiło, że Salasso rzucił się do przodu i~ukłonił się nad
dłonią, odpowiadając czymś, co brzmiało jak te same słowa/tony, ale
szybciej. Po dalszej takiej wymianie, Salasso kiwnął głową i~powiedział:

-- Jestem zaszczycony tą znajomością.

-- I~ja także.

Wyglądając na zadowolonego z~siebie tym wyczynem poligloty, kupiec
odsunął się trochę i~spojrzał na rysunku na ścianach i~wokół zbiorników,
tacek i~sprzętu.

-- Interesujące -- powiedział. -- Fascynujące. Widziałem coś takiego w~domu\ldots -- possał wargę i~kilka razy strzelił palcami. -- Och, tak, Muzeum
Morskie! Pamiętasz, Lydio?

-- Och, tak -- odpowiedziała. -- Zabrałeś mnie tam, kiedy byłam mała. Była
tam taka duża skrzynka ze szkła, a~wewnątrz była kopia mózgu i~systemu
nerwowego krakena, z~czarnego szkła. To \emph{rzeczywiście} wyglądało
jak ten rysunek, ale większe.

-- Bogowie na niebie -- powiedział Salasso -- ktoś zrobił sekcję
\emph{Teuthysa} 

-- Jestem przekonany, że był martwy, wyrzucony na plażę -- powiedział Tenebre, ciągle się rozglądając. -- Naukowcom udało się
zachować, zanim miał czas się rozłożyć, a~potem rozpuścili w~jakimś
płynie, który pozostawił nerwy i~mózg nienaruszone, zafarbowali, i~odlali w~żywicy, a~potem wyciągnęli model w~szkle. Bardzo pomysłowa
technika. Oczywiście, była pomysłowa jeszcze kilkaset lat temu.

-- Oczywiście -- powtórzyła rozczarowana Elizabeth. -- A~czego się z~tego
nauczyli?

-- Och, niewiele, droga pani. Wówczas dominowało podejście naturalnej
historii. Obserwacja i~spekulacja. Metody eksperymentalne jeszcze się
nie upowszechniły. Ciągle\ldots

Jego uśmiech przeniósł się z~Elizabeth na Lydię. 

-- Zapewniło to mojej
dziewczynce zainteresowanie historią naturalną, którą ciągle się
zajmuje.

\emph{Jasne, że tak} pomyślała Elizabeth. \emph{pewnie zbiera } motyle
\emph{ i~} kwiatki \emph{i } piórka.

-- To \emph{było } interesujące -- powiedziała Lydia. -- Ten potężny,
skomplikowany mózg, tak różny od naszego, z~pniami nerwowymi grubymi jak
liny, jak korzenie wyrastające z~pnia. Oczywiście, Muzeum było
całkowicie zawalone interesującymi stworzeniami, ale -- Lydia się
zaśmiała -- to ten mózg zmusił mnie do myślenia.

-- O czym zaczęłaś myśleć? -- To było wszystko, co Elizabeth mogła zrobić,
by zatrzymać truciznę w~głosie.

-- Języki -- odpowiedziała Lydia. -- czy sposób komunikacji głowonogów
przez chromatofory jest nieodłączny od ich neuroanatomii? Czy różni się
w obrębie gatunku jak języki ludzi? Czy jest abstrakcyjnie symboliczny,
czy raczej zasadniczo ideograficzny i~quasi-obrazkowy? W~jaki sposób
jest możliwe tłumaczenie pomiędzy nim a~dźwiękowymi i~gestykularnymi
językami człowiekowatych i~zaurów? Tego rodzaju rzeczy.

-- Aha. -- Ten minimalnie komunikatywny szum był całą odpowiedzią
Elizabeth, na jaką się zdobyła.

-- Głębokie pytania -- powiedział Salasso. -- Nasze podejście do takich
problemów jest skromne i, jak Twój ojciec zasugerował, eksperymentalne.

-- Z~pewnością, \emph{nie} siekacie krakenów? -- spytał Tenebre.

-- Bogowie, nie -- powiedział Gregor, dotykając łokcia Lydii i~popychają
ją w~kierunku ławki w~laboratorium. -- Siekamy niewinne, małe
kałamarnice.

-- Aha! -- powiedział Tenebre. -- Według hipotezy o~wspólny pochodzeniu!
Cóż, można powiedzieć, że to początek.

-- Można -- odpowiedział Salasso. W~jego głosie pojawiła się jakaś napięta
wibracja. -- Ale\ldots wspólne pochodzenie nie jest \emph{hipotezą. } Jest
obserwacją.

Tenebre zaczął kroczyć wzdłuż pokoju, patrząc na wykresy jak gość w~galerii sztuki, który wie, co lubi.

-- Dla waszego gatunku, pewnie tak, Salasso -- zażartował. -- Dla mojego,
jednakże, pozostanie hipotezą, póki nie zaczniemy żyć tak długo jak wy.

Salasso odrobinę się zaśmiał, Elizabeth nie potrafiła powiedzieć, czy z~autentycznej rozrywki czy ze służalczości. Rozbawienia, domyśliła się.
Pochlebstwa nie były wadą zaurów. Salasso dołączył do kupca i~zaczął
gorliwie pokazywać istotne lub problematyczne cechy mapowania nerwowego.
Gregor i~Lydia już pochylali się nad preparatem na stole, głowy prawie
się dotykały, rozmawiając cicho.

Elizabeth przypomniała sobie jak ona i~Gregor się poznali. Na pokazie
laboratorium, gdzie studenci byli losowo dobrani w~pary, żeby
przeprowadzić klasyczne ćwiczenie: sekcja czaszki rekina
kolenia\footnote{na Ziemi gatunek ryby
chrzęstnoszkieletowej z~rodziny koleniowatych, zob.~\url{https://pl.wikipedia.org/wiki/Kole\%C5\%84\_pospolity} - przyp.tłum.}.
Ryba strasznie śmierdziała, trzeba było użyć masy kremu do rąk i~założyć
gumowe rękawice, jeżeli nie chciało się pachnieć martwym rekinem przez
tydzień. Facet koło niej dzielnie zgłosił się na ochotnika do wykonania
cięć, pozwalając jej skoncentrować się na szkicowaniu mózgu, nerwów
optycznych i~gałek ocznych, które były przedmiotem ćwiczenia. Pamiętała
jego duże palce trzymające skalpel, precyzyjny i~pewny siebie sposób, w~jaki przeciął chrzęstną czaszkę i~ją otworzył, jego zorientowane
komentarze. To nie był pierwszy rekin, którego dobrze oglądał: siekał
je, na przynętę, z~ciekawości, na pokładzie łodzi jego ojca.

Ledwo na siebie patrzyli, cóż, on ledwo na nią patrzył, a, po kilku
pierwszych spojrzeniach z~boku, ledwo się odważała spojrzeć na niego, i~to widoczne, skierowane na zewnątrz, łatwe koleżeństwo od tego czasu
nadało ton ich relacjom.

Przeszła energicznie do kolejnego stołu i~zabrała się do pracy
kalibrując elektrodę czytnika, nudną, drobiazgową pracę, którą musiała
być powtarzana każdego ranka z~powodu nocnych zmian w~temperaturze i~wilgotności. Zajęło ją to, pozwalając przestać słuchać lekkiej rozmowy
Gregora i~Lydii. Zaur i~kupiec kontynuowali swoją przechadzkę dookoła
laboratorium. Elizabeth mogła podsłuchać ich rozmowę przeskakującą z~angielskiego na handlową łacinę i~z powrotem, oraz fragmenty mowy
zaurów. Nie obraziła się, że Tenebre wybrał Salasso na rozmowę o~pracy
zespołu. Wyższa inteligencja zaura i~jego uczciwość dawałyby, według
każdego handlarza tak doświadczonego jak ten, niskie szanse na brednie.
(Salasso kiedyś jej wytłumaczył, z~doskonałą pewnością siebie, że cechy
inteligencji i~uczciwości są połączone: każdy wystarczająco inteligentny
mógł zrozumieć rozgałęzione konsekwencje kłamstwa, zwykły koszt
procedowania umysłowego, mocy, żeby je podtrzymać, i~po prostu wycofać
się z~niego. 

-- Może ta relacja nie występuje wśród człowiekowatych --
dodał z~brutalnym taktem.)

Cisza sprawiła, że się rozejrzała. Kupiec stał z~przodu jak wykładowca,
Salasso trochę z~boku, Lydia i~Gregor ciągle siedzieli razem.

-- Doskonale, moi przyjaciele -- zaczął Tenebre -- to było bardzo
interesujące. Fascynujące. Muszę powiedzieć, że jest to najbardziej
zaawansowane badanie biologiczne, na jakie natrafiłem. Jestem pewien, że
wasi przodkowie je prześcignęli, ale moi na pewno nie. Ani moi
współcześni -- uśmiechnął się rozbrajająco. -- Oczywiście, chyba że,
akademie Nova Babylonia zmieniły swoje podejście w~ciągu ostatniego
wieku.

Podkradł się do stołu i~podparł się jego krawędzi, pochylając się do
przodu poufnie.

-- Tak, jestem człowiekiem praktyki i~nie mam pojęcia, jakim praktycznym
celom mogłoby służyć to badanie. Ale nie mam wątpliwości, że kiedy tutaj
wrócę, pojawią się na pewno jakieś użyteczne zastosowania, w~medycynie,
przemyśle, bogowie wiedzą w~czym. Może nawet w~obliczeniach, rozumiem,
że Kosmonauta lord Cairns jest zainteresowany czymś, co nazywa ,,sieci
neuronowe'' i~popierał waszą pracę w~tym celu.

Gregor zerknął ponad ramieniem Tenebre na Elizabeth, podnosząc brew na
pół sekundy. Elizabeth pozwoliła sobie na prawie niezauważalne
wzruszenie ramion i~poruszenie głową. Salasso, zauważyła, wybrał ten
moment do spojrzenia przez okno.

Jeżeli Tenebre zauważył ten krótki epizod, to nie dał po sobie poznać,
kontynuując: 

-- To nie jest ważne. Co jest ważne, to, że mogą być z~tego
pieniądze i~byłbym zachwycony, mogąc wpłacić pieniądze teraz za jakieś
udziały w~zwrotach później.

-- Dziękuję -- odpowiedziała Elizabeth, zanim ktokolwiek mógł coś
powiedzieć. -- Bylibyśmy tym bardzo zainteresowani. Rozumiem, że kolejnym
krokiem byłoby przedyskutowanie proponowanej inwestycji z~syndykiem.

Salasso pokiwał głową energicznie. Gregor odwrócił się znów, patrząc
ciągle z~zaskoczeniem, ale zadowolony. Potem odwrócił się i~podziękował
handlarzowi za jego zaufanie.

-- Dobrze -- powiedział Tenebre. -- Oczywiście są szczegóły do
wypracowania, pytania o~własność intelektualną, informacja chce być
opłacona, i~tak dalej. I~na pewno chcecie zabezpieczyć, że wy lub wasi
następcy nie będziecie mieć związanych rąk co do kierunku badań. --
Pokazał otwarte dłonie. -- Żadne z~tych nie powinno być problemem,
naprawdę chcę i~oczekuję, że obie strony odniosą korzyści. Moi prawnicy
mają standardowe umowy i~nigdy nie mieliśmy skarg.

-- To dla nas całkowicie w~porządku -- powiedział Gregor, brzmiąc
ostrożnie. -- Również chcielibyśmy uczestniczyć w~dyskusjach.

-- Oczywiście. Ale naprawdę, jeżeli właściwie to opiszemy, nic z~tego nie
wpłynie na to, co robicie, będziecie mieć więcej zasobów do pracy, a~za
sto lat, wy lub wasi następcy zapłacą mi dość rozsądną porcję
czegokolwiek, co może być otrzymane z~tych badań.

Gregor stanął i~potrząsnął dłonią Tenebre. Salasso i, po chwili,
Elizabeth się dołączyli.

-- Wspaniale, wspaniale -- powiedział kupiec. Wyjął zegarek z~kieszeni i~spojrzał na niego. -- Doskonale, jestem pewien, że macie pracę do
wykonania, podobnie i~ja. Niektórzy moi służący są zajęci na
uniwersytecie, kupując duże liczby książek i~instrumentów. Wyjeżdżamy
pojutrze, do Nowej Lizbony, okazało się, że w~tym roku rynek mięsa
zaczął się wcześniej niż zwykle. Spotkam się z~doradcą wieczorem i\ldots

Lydia zerwała się z~siedzenia przy stole, z~głośnym szlochem, i~wybiegła
z pokoju.

-- Przepraszam -- powiedział Gregor i~zniknął za nią.

Tenebre stał, patrząc na obrotowe drzwi przez kilka sekund. Potem,
zarumieniony, marszcząc brwi, wyszedł.

\threeast

Gregor znalazł ją za głównymi drzwiami, skierowaną we wnękę w~grubo otynkowanej ścianie, jej ramię zasłaniało oczy.

Położył rękę na jej ramieniu i~odwrócił ją. Ukryła zalaną łzami twarz w~jego barku i~trzęsła się przez minutę.

-- Wiedziałam, że nie mamy dużo czasu -- powiedziała, stłumionym głosem,
pociągając nosem -- ale to nie \emph{fair}.

Gregor usłyszał otwierające się drzwi i~ciężkie, pośpieszne kroki jej
ojca.

-- Och, w~imię Zeusa! -- powiedział Tenebre. -- Proszę. Lydia. Przestań
płakać, chodź, usiądźmy i~porozmawiajmy o~tym\ldots cokolwiek to jest.

Za blokami laboratorium był teren z~drewnianymi stołami i~ławami,
skierowanymi ku brzegowi, wypolerowanymi przez kłęby powietrza, które
dominujący wiatr uderzał w~ściany, tak rzadko używany zgodnie z~zamierzonym celem spożywania posiłków na powietrzu. Przeszli do stołu,
Gregor i~Lydia siedli po jednej stronie, a~de Tenebre naprzeciwko nich.
W końcu ramiona dziewczyny przestały drżeć i~pochyliła się do przodu,
łokcie na stole, podpierając się i~patrząc na ojca.

-- Nie \emph{możemy} po prostu wyjechać za dwa dni! -- powiedziała.

De Tenebre podrapał się po karku.

-- Przepraszam -- odpowiedział. -- Widzę, co się wydarzyło. Nie mogę
powiedzieć, że was obwiniam. Jestem rozsądnym człowiekiem i~naprawdę
dbam o~wasze dobro. Szczególnie Twoje, Lydio, jesteś moją córką. Nie
zrobiłbym niczego, żeby cię skrzywdzić, przecież wiesz. -- Rzucił
Gregorowi mroczne spojrzenie. -- I~nie \emph{pozwolę} nikomu cię
skrzywdzić. Mam nadzieję, że ten mężczyzna nie dawał, lub żądał,
jakichkolwiek obietnic.

-- Nie! -- Oboje odpowiedzieli oburzonym głosem.

Kupiec westchnął głośno. 

-- Cóż, nie jest tak źle. Serca się wyleczą, ale
słowa już nie, co?

Lekceważące, filistyńskie powiedzenie zszokowało Gregora. Próbował
powstrzymać swój gniew, o~którym wiedział, że nikomu nie pomoże. W~tym
czasie, Lydia objęła go ramieniem, i~mocno go trzymała. To ośmieliło go
do zabrania głosu.

-- Kocham ją -- powiedział. -- Mógłbym kochać ją wiecznie.

Ramię Lydii uścisnęło ją mocniej, a~ona uśmiechnęła się do niego.

-- Bez wątpienia tak czujesz -- powiedział De Tenebre z~pewnym rodzajem
chłodnej sympatii. -- I~uwierz mi, rozumiem. Ale, nie mogę pozwolić temu
wpłynąć na moje działania. \emph{Musimy} wyjechać. -- Westchnął. -- A~ja
mam dzisiaj jeszcze inne spotkania.

Wczesne światło słoneczne padało ukośnie na nich, bryza znad morza ich
szarpała. Niedaleko nad wodą, pola wielkiego statku trzaskały i~mruczały. Lydia spojrzała w~dół, strząsnęła płatki z~rękawa, skrzywiła
się i~pociągnęła nosem.

-- Nie mogłabym zostać na jakiś czas? -- spytała. -- Mogłabym dołączyć do
Ciebie w~Nowej Lizbonie. W~końcu mają tutaj transport lotniczy!

-- Och, Lydio -- powiedział jej ojciec z~mieszanką zniecierpliwienia i~czułości. -- Nie powierzyłbym służącego tym pęcherzom z~gazem i~latawcom,
a co dopiero Ciebie. Pomijając wypadki, są niepewne i~niepunktualne.

To była prawda, a~Gregor wiedział, że nie mógłby zaprotestować.

-- Pojadę z~Tobą -- powiedział.

De Tenebre bujnął się do tyłu i~prychnął. 

-- Wlec się przez trzy tygodnie
z nami? Nie mogę sobie wyobrazić gorszego sposobu na przedłużanie
Twojego bólu, i~Lydii.

-- Nie -- powiedział Gregor nagle oszołomiony decyzją. -- Chciałem\ldots

De Tenebre podniósł rękę, potrząsnął głową.

-- Nie! -- ostro powiedział. -- Nie będę tego słuchał. Nie pozwolę ci tego
powiedzieć. Podróżowanie to nie jest życie dla kogoś, kto się nie
urodził do tego, i~z pewnością nie dla Ciebie. Człowieku, masz inne
powołanie. Nie lekceważ darów, które bogowie ci dali. A~nie dali ci
mojej córki\ldots

Zatrzymał się, marszcząc brwi w zamyśleniu.

-\ldots lub, jeżeli ci dali -- kontynuował -- to tylko dzięki Twojej
własnej pracy i~Twoim darom, będziesz mógł ją zdobyć.

Gregor zacisnął na kilka sekund oczy. Bał się, że w~każdej chwili mógłby
zacząć płakać gorzej od Lydii. Powoli słowa kupca dotarły do niego.
Spojrzał na niego.

-- Co masz na myśli? -- spytał Gregor.

De Tenebre wstał i~pochylił się oparty na pięściach.

-- Zamierzamy zostać na Croatan pół roku. Mogę Ci zostawić harmonogram
naszej podróży od tego czasu, każdy port przez całą drogę do Nova Terra.
Twój naczelnik, lord Driver, powiedział mi o~waszej rodzinnej Wielkiej
Pracy. Lord Cairns, Twój dziadek, to potwierdził. Pokładają w~Tobie
wielkie nadzieje. Jeżeli je spełnisz, możesz nas dogonić, i~spotkać
Lydię w~ciągu kilku miesięcy lub lat Twojego życia i~nawet mniej w~jej
życiu. Załatw mi statek. Jeżeli to zrobisz, Gregor Cairns, możesz wziąć
moją córkę, a~ja będę wiecznie ci dłużnikiem.

Gregor poczuł, że ramię Lydia odpada. Świat stał się, na chwilę,
czarno-biały i~wypełniony białym szumem. Gregor odetchnął głęboko kilka
razy. Jego pierwszą myślą był obraza za to wyzwanie, tę ofertę
\emph{wymiany} Lydii za statek lub korzystanie z~niego. Potem\ldots

Myśli toczyły się, \emph{klik-klak-klik} jak bramki logiczne. Jeżeli Hal
Driver i~James Cairns powiedzieli kupcowi, że Wielka Praca mogła być
ukończona w~realnym czasie, to wyjaśniałoby ponaglenia Jamesa, i~w tym
samym czasie sprawiło trudność, żeby odrzucić sugestie De Tenebre jako
nierozsądne. A~jeżeli James był zainteresowany w~badaniach zespołu jako
mogących jakoś wnieść wkład do obliczeń, coś z~sieciami neuronowymi, to
istniało powiązanie pomiędzy tym, co robili, a~Wielką Pracą\ldots

Przeszedł go dreszcz, gdy zrozumiał, co mogło być takim powiązaniem.
Pojawiło się przed jego oczami, mapa systemu nerwowego kałamarnic
nałożona nad strukturami danych problemu nawigacyjnego. Zrozumiał
architekturę umysłu, który mógł zrozumieć problem, a~w tym rozumieniu,
zrozumiał też i~on. Mógł ujrzeć, w~zasadzie, jak ten problem mógłby
zostać rozwiązany.

Mrugnął oczami, a~świat wrócił, w~pełnym kolorze i~wysokiej
rozdzielczości. Lydia i~jej ojciec patrzyli na niego bardzo dziwnie.

-- Zdziwiony jestem, że wyglądasz na tak zadowolonego -- powiedział de
Tenebre. Wyprostował się i~zrobił krok do tyłu. -- I~zachęconego.
Martwiłem się, że Twoi starsi blefowali przy targowaniu.

\emph{Targ}, Gregora uderzyły dalsze konsekwencje jego toku myśli.
Umowa, którą zaoferował kupiec zespołowi, dałaby mu udział we wszystkich
przyszłych wynikach i~zastosowaniach, a~jeżeli ich badania byłyby
połączone z~Wielką Pracą, to dałoby mu udział w~wielkim przedsięwzięciu
gwiezdnych statków, które nakreślił James. Stałe, powiedzmy w~ich
wszystkich przyszłościach, a~w przyszłości Mingulay, która by w~ten
sposób była powiązana z~Nova Babylonia.

Przerzucił nogi nad ławką i~wstał, patrząc na kupca.

-- W~ogóle nie jestem zadowolony -- powiedział. -- Nie daję Ci żadnych
obietnic i~nie akceptuję zaoferowania Twojej córki, ponieważ to jest jej
decyzja -- stanął za Lydią i~położył delikatnie czubki palców na jej
ramionach. -- Ona może być Twoją córką, ale jej życie jest jej, nie
czymś, co może być targiem pomiędzy rodzinami. Kocham ją zbyt mocno.
Wiedziałem od początku, że to było beznadziejne, ale miłość bez nadziei
jest możliwa.

Lydia sięgnęła ręką do góry i~złapała jego dłoń.

-- Masz rację -- Gregor kontynuował -- że nie mogę i~nie pójdę za Wami.
Jeżeli Lydia czuje do mnie to samo, co ja czuję do niej, zostanie tutaj.
Jeżeli nie\ldots zrobię, co mogę, żeby podążyć za Wami. Ale co Lydia
zrobi potem lub teraz, to jej wybór. Co \emph{ja} zrobię teraz, to wrócę
do laboratorium i~przekonam współpracowników, żeby Twoja oferta
inwestycji w~nasz zespół została grzecznie odrzucona.

Uchwyt Lydii na jego palcach zaczął boleć.

Wtedy go puściła, wygramoliła się zza stołu i~stanęła, patrząc na niego,
jej oczy mokre.

-- Nie! -- powiedziała. -- Nie rozumiesz! Przybycie po mnie w~swoim własnym
statku wygląda, tak jak powinno! Kiedy ojciec tak powiedział, poczułam
po raz pierwszy jakąś nadzieję dla nas! Musisz to zrobić! Musisz dokonać
czegoś własnego, żeby zdobyć kobietę, tak właśnie u nas jest. Nie
czułabym się w~ogóle przehandlowana! Jeżeli mnie kochasz, zrobisz to!

\emph{Jak różne są nasze światy}, pomyślał Gregor. \emph{I jak podobne.
} Mogła go mieć. Ciągle mogła.

-- Kocham Cię -- powiedział, odwrócił się i~odszedł. Nie spojrzał do tyłu,
ale przy każdym kroku miał nadzieję, przez rosnącą trawę, skrzypiące
kamyki, a~potem dalej wyłożonym kafelkami korytarzem do laboratorium, że
Lydia przyjdzie, biegnąc za nim.

Nie przyszła.

\threeast

-- Startuje -- powiedziała Elizabeth przy oknie laboratorium.

Gregor podniósł wzrok od stołu pokrytego stronami papieru. Biały papier
i czarny atrament, zabazgrany kształtami, upstrzony numerami. Poczuł
całkowitą nudę taką, jaką miał przez ostatnie czterdzieści osiem godzin.
Nie mogąc wyjaśnić powodów swojego sprzeciwu wobec finansowania kupca
nikomu, prócz Jamesa, który je zaakceptował i~prędko spotkał się z~syndykiem, żeby potwierdzić sprzeciw, pozostawał w~złych stosunkach z~zespołem i~całym wydziałem. Wszyscy myśleli, że za tym kryje się jakiś
skandal, jakaś zniewaga dana lub otrzymana, jakiś cień.

Ale ciągle, zwabiony prymitywnym pędem naczelnych do stymulacji
wizualnej, zmusił się do wstania i~podejścia do okna. Kolejny słoneczny,
przenikliwy dzień. Ostatnie skify wlatywały do wnęk w~kadłubie statku
jak morskie nietoperze do gniazd na klifach. Barki i~statki turystyczne
oraz małe brzęczące samoloty stały lub krążyły z~daleka.

Boki statku pokazywały kolorowe światła, które wypisywały nazwiska,
logo, flagi i~symbole. Wnęki i~włazy zamknęły się bez pozostawiania
spoin. Dookoła statku, woda odgięła się spod niego, póki nie było
oczywiste, że statek nie pływa, ale unosi się nieco ponad rozległą,
płytką depresją. Statek zaczął powoli się unosić. Ognie Kastora i~Polluksa trzaskały na masztach kilometr dalej. Menisk wody wzniósł się
pod statkiem, póki morze nie wybrzuszyło dwa metry ponad swój poziom.

Wtedy woda osunęła się z~powrotem, tworząc falę, która ścigała się, żeby
pokołysać odległe łodzie, a~Statek uniósł się szybciej, jak gdyby został
zwolniony. Zaczął poruszać się do przodu, tak jak kontynuował
przyśpieszenie, a~w ciągu minuty znikł z~widoku na tle błyszczącego
niepojętego błękitu nieba.

Gregor pojął, że wykręcał się, żeby obserwować, jego policzek
przyciśnięty do szyby. Oderwał się od szyby, zrobił krok do tyłu i~odwrócił się tyłem do horyzontu. Elizabeth i~Salasso spojrzeli na niego,
zaur bez dostrzegalnego wyrazu twarzy, kobieta z~niepewnym uśmiechem.

-- Cóż, stało się -- powiedział. -- Odlecieli.

W tym momencie emocje wróciły do niego, zalewając jego żyły i~nerwy
radosną ulgą. Ból rozstania z~Lydią i~ból braku pewności, co rozstanie
oznaczało, złamały uścisk anhedonicznej depresji. Poczuł się o~tyle
lepiej, że się uśmiechnął.

Lydia odeszła, ale ciągle miał przyjaciół, ciągle miał pracę, i~nagle
stało się oczywiste, w~jaki sposób jego przyjaciele i~jego praca mogłyby
pozwolić zobaczyć znów Lydię.

-- Tak, cóż\ldots -- zaczęła mówić Elizabeth. Gregor podszedł i~złapał ją z~uśmiechem za ramionami. Prawie się cofnęła, ale uśmiechnęła się szerzej.

-- Muszę Ci coś powiedzieć -- powiedział Gregor. Jej ramiona, pod szorstką
wełną swetra, drżały w~sposób, który boleśnie przypominał mu dotyk
ramion Lydii. Oderwał jedną rękę i~złapał również ramię zaura. -- Chodzi
o nawigację statku gwiezdnego.

-- Och -- powiedziała Elizabeth. Jej twarzą przez chwilę straciła wyraz,
potem spojrzała w~bok i~spojrzała z~powrotem na niego, zainteresowana. -
Więc nie trzymaj nas w~niepewności.

\threeast

Ale trzymał ich w~niepewności, całą długą drogę do Zamku i~przez jego
korytarze i~schody. Niosąc pod ramieniem zrolowane papiery, nad którymi
pracował, wmaszerował do pokoju Nawigatora. Był niezajęty.

Machnął ręką na sofę. 

-- Czujcie się jak u siebie\ldots nie, nie tam, kawa
się trochę rozlała.

Elizabeth przysiadła na boczku kanapy, Salasso znalazł jakieś miejsce,
które ani nie było poplamione, ani niezajęte przez książki i~dokumenty,
potem położył dłonie za głową i~oparł się z~wyciągniętymi nogami, ludzka
postawa, którą naśladował, ale której jego wzrost, lub brak, nie pomagał
zająć.

-- Więc powiedz nam -- powiedział.

-- Moja rodzina przez pokolenia pracowała nad problemem nawigacyjnym -
powiedział Gregor. -- Tyle jestem pewien, że wiecie. To nie tajemnica.
Zrozumiałem, że faktyczne rozwiązanie tego problemu wymaga nie-ludzkiego
umysłu, konkretnie umysłu kałamarnicy, a~nasze badanie neurologii
głowonogów może wnieść wkład do symulacji takiego umysłu, jego nagiego
zarysu, oczywiście, ale zarysu, struktury, \emph{architektury}, jeżeli
chcecie, która się liczy.

Salasso szarpnięciem usiadł i~teraz pochylił się do przodu, spięty.

-- Rozumiem, rozumiem -- powiedział. -- Potencjały elektryczne, zgrubna i~dokładna anatomia, tak! Tak! Ale gdzie byś to symulował?

-- W~maszynie liczącej, oczywiście -- powiedział Gregor. -- Mózg jest
komputerem, a~dowolny komputer może symulować inny komputer.

Elizabeth rozejrzała się po maszynach liczących.

-- W~\emph{tych} stosach mechanicznego złomu?

-- Jeżeli to będzie konieczne -- odparł Gregor. -- Tak. Ale mam nadzieję, z~nimi, i~wieloma innymi, pracującymi równolegle.

-- To ciągle zajęłoby wieczność.

Oczy Gregora się zwęziły. 

-- Och, wiesz to?

-- Cóż, jako naukowiec mogę spróbować zgadnąć!

Gregor wstał. 

-- Widzę to wszystko w~głowie, widzę, jak mogłoby być to
zrobione. Struktura problemu i~struktura mózgu pasują tak dokładnie, że
jest niesamowite, to jakby były stworzone dla siebie.

Zdał sobie sprawę z~tego, co właśnie powiedział i~dodał:

-- Może były.

Salasso nic nie powiedział, ale wyglądało, że jego usta stały się,
jeżeli to możliwe, cieńsze.

-- Ale masz rację -- Gregor kontynuował. -- W~teorii, tak, jeden komputer
może symulować inny. Ale po prostu nie jest możliwe, by zrobić to
szybko, bez znacznie lepszych komputerów niż mamy. -- Jego pięści się
zacisnęły. -- Tak, jeżeli mielibyśmy ciągle te komputery, które pierwsza
załoga zabrała ze statku\ldots

-- To mogłoby być możliwe -- powiedział Salasso.

\chapter[Start bez ostrzeżenia]{10 Start bez ostrzeżenia}


Ktoś szarpał mnie za ramię. Zmagałem się z~ciężką drzemką w~środku dnia,
żeby odnaleźć się na sofie w~biurze Alana. Alan patrzył się z~góry na
mnie z~zaniepokojeniem.

-- Przepraszam -- powiedziałem. -- Nie miałem\ldots

-- W~porządku, po prostu miałeś kilka trudnych dni do nadrobienia -
powiedział Alan. -- Pozwolilibyśmy Ci spać, tylko\ldots

Wskazał gestem na ścianę, a~z~tym ruchem ręki, plakaty znikły i~przekonfigurowały się w~szachownicę ekranów z~wiadomościami. Większość z~nich pokazywała tę samą twarz: Jadey. Od razu stałem się przytomny.

Zdjęcie było aktualne, nie, to było na żywo, kamera śledziła jej ponurą
twarz, gdy była eskortowana przez dwie policjantki z~Sądu w~Edynburgu do
policyjnej furgonetki. Załapałem wypowiadane frazy. ,,Przetrzymywana w~areszcie'', ,,Rozprawa w~sprawie ekstradycji''.

-- Ekstradycja?

-- Do B-U-K -- powiedziała Mary-jo, literując złośliwie skrót. -- Została
tam oskarżona o~morderstwo ruskiego oficera, z~którym według nich miała
romans.

-- To pierdolone kłamstwo!

-- Ktoś mógłby tak założyć -- powiedział Alan -- z~pierwszych zasad. Ale
skąd Ty wiesz?

Powiedziałem im. Palec Mary-Jo przez cały czas poruszał się w~powietrzu.

-- Racja, dobrze -- powiedziała, kiedy skończyłem. -- Zakładając, że
powiedziała ci prawdę, a~myślę, że tak zrobiła. To tłumaczy, dlaczego
mają próbki ubrania z~krwią tego biednego skurwysyna i~jej komórek
skóry. I~zakrwawiony nóż bez przeklętych odcisków czy coś. A~nawet
jeżeli Twój sprzęt nie zadziałał, a~cała sprawa była nagrana, ciągle
byłyby dowody, że odbył się atak hakerski na kamery uliczne, co
sprawiłoby, że każde nagranie, pokazujące co \emph{właściwie} się
zdarzyło, byłoby niedopuszczalne w~sądzie. Gówno.

-- Będzie gorzej -- powiedział Alan płasko. Przekopał kilka warstw
wiadomości, aż do szczegółów za nagłówkami. Na tym poziomie to był
praktycznie niesformatowany protokół sądowy, nikt nie martwił się
podsumowaniem nieinteresujących drobnych detali o~zwykłym Brytolu, a~wszystko, co zauważyłem, to było moje nazwisko, i~wiele zdjęć mnie,
większość ziarnistych zrzutów z~obserwacji, ale wystarczająco
czytelnych.

-- Chcą Ciebie. -- Mary-Jo przetłumaczyła z~prawniczego. -- Jeszcze nie są
pewni, czy chcę cię wezwać jako świadka, czy zażądać ekstradycji jako
wspólnika po fakcie. Cokolwiek, rząd Stanów Zjednoczonych prawdopodobnie
będzie współpracował, a~nawet jeżeli to nie skończy się w~sądzie, w~najlepszym przypadku możesz oczekiwać długiej prawnej męczarni. W~najgorszym przypadku Departament Sprawiedliwości wsadzi Twoją dupę na
samolot z~powrotem do komuchów, zanim zdążysz powiedzieć ,,uchodźca''.

-- Chwila -- powiedziałem, próbując zatrzymać kamyk w~lawinie złych
wiadomości -- myślałem, że ludzie z~Europy dostają automatycznie status
uchodźcy, czy coś.

-- Nie tam. -- Potrząsnęła głową. -- To wszystko de facto. Rząd odwraca
wzrok, i~to jest, wiesz, polityka, rezygnacja z~kontroli imigrantów, ale
prawo ciągle jest w~kodeksie. To przywilej, nie prawo, i~może być
wycofane w~każdej chwili i~podważone w~sądzie po fakcie. Legalnie rzecz
biorąc, jesteś ciągle nielegalnym imigrantem.

-- Dobrze -- zapadłem się w~sofie, czując się wyczerpany. -- Mogę sobie z~tym poradzić. Co z~Jadey?

-- Przepraszam, -- powiedziała Mary-Jo -- ale nie \emph{możesz} sobie z~tym
poradzić. Sprawa Jadey, \emph{możemy} ją załatwić. To jest to, co
\emph{robimy. } Nieważne, jakie są zarzuty, to ciągle sprawa polityczna,
możemy ciągle wynegocjować umowę. Przecież to nie tak, że wy tam macie
niezależne sądownictwo czy co. Tymczasem tutaj, my mamy, mniej więcej.
Twoim problemem \emph{tutaj} jest grupa super prawniczych urzędników i~działaczy z~wolnej ręki, czyli ludzie, którzy mogliby cię ścigać. Kurde,
dowolny z~tych gości przy bramie może zdecydować się na dodatkowy
zarobek i~donieść na Ciebie. Wierz mi, nie chcesz być przedmiotem
postępowania administracyjnego, \emph{ani} utkwić w~sądach.

Wstała, przeszła do okna, jak gdyby wypatrując ochroniarzy lub czarnych
helikopterów. 

-- Wiesz -- powiedziała -- Twoja wiarygodność jako obywatela
UE który jest opozycjonistą, właśnie przekroczyła skalę. Mógłbyś
naprawdę pogadać tam z~buntownikami. I~jesteś w~głębokim gównie tutaj na
Ziemi.

Odwróciła się do mnie, spoglądając badawczo. 

-- Ciągle myślisz, że lot w~kosmos to szalony pomysł?

-- Tak -- odparłem. Dałem im mocno do zrozumienia, co myślę na ten temat,
gdy taka sugestia się pojawiła. -- Ale\ldots

-- Bardzo dobrze! -- powiedział Alan. -- Wiedziałem, że się zgodzisz.

\threeast

-- To ma być \emph{statek kosmiczny!}

Byłem przyzwyczajony do obrazów startów z~Bajkonuru, z~Kourou, i~także z~Canaveral. Nawet z~jednym stopniem na orbitę, wznoszeniem się jakby
pchła skakała po wielkiej patelni, wszystkie przypominały rakiety V2 w~trakcie pionowego wznoszenia. Ten czarny obiekt, w~blasku słońca na
płaskim gruncie, nie przypominał niczego. Wyglądał bardziej obco niż
jakikolwiek realny czy wyobrażony latający spodek. Wyglądał jak rzeźba
jakiegoś zwierzęcia dopasowanego do próżni. Samo próbowanie objęcia
perspektywicznie tej rzeczy, stworzenia obrazu jako całości, sprawiało,
że oczy mi łzawiły, a~głowa zaczynała boleć.

-- Nevada Orbital Dynamics JSG -- powiedział Alan Armstrong. (W końcu
przyznał, że to było jego nazwisko. Był tylko głównym inżynierem,
notorycznie skromnym, według informacji, które moje szkła ściągały.) --
Od Jeden Stopień Gdziekolwiek. Jest to udoskonalenie starego
Electro-Dynamic Stealth Fighter Sił Powietrznych. Naszego sławnego
latającego spodka, którego tyle zostało strąconych przez Ruskich w~czasie wojny. Ten może zjonizować otaczające powietrze i~elektromagnetycznie wydrgać je, żeby osiągnąć prędkość ucieczki \emph{w~atmosferze}, potem wygenerować własny żagiel plazmowy i~dalej
przyśpieszać. Mógłbyś na nim dotrzeć do Plutona. Oczywiście, wcześniej
byś umarł z~głodu, ale Twoje ciało by tam dotarło.

Żagiel plazmowy, sprawdziłem odnośnik, znalazłem elaboraty o~takim
systemie, rozległym polu elektromagnetycznym otaczającym elipsoidę
zjonizowanego gazu, która wzajemnie oddziaływała z~pływem fotonów jak w~żaglu słonecznym. Statek mógł dotrzeć do \emph{Marszałka Titowa} w~ciągu
kilku dni, z~czego połowę czasu spędziłby na zwalnianiu i~halsowaniu.
Ostateczne korekty kursu rakietami termojądrowymi.

-- \emph{Termojądrowymi? } Więc dlaczego\ldots

-- Nie użyli ich do wszystkiego? Są drogie, to dlatego. Ten statek jest
wart miliardy dolarów.

Gapiłem się na niego. 

-- W~jaki sposób macie zamiar na tym zarobić?

Wzruszył ramionami. 

-- Amerykańskie wojsko nieco poparzyło się przy
latających spodkach -- przyznał -- a~prywatni operatorzy nie potrzebują
niczego takiego, jeszcze. Moglibyśmy sprzedać to pewnego dnia do NASA,
jeżeli w~końcu się zbiorą, ale na razie jest to otwarty sprzęt,
finansowany przez fundację, na potrzeby eksploracji kosmosu, co do
której jesteśmy pewni, że ożyje. Nawet bardziej pewni teraz niż
kiedykolwiek, jeżeli o~tym pomyśleć.

-- Zaczynam się zastanawiać, dlaczego zawracamy sobie głowę obcą
technologią -- powiedziałem.

-- Warto byłoby mieć antygrawitację -- odparł łagodnie Armstrong. -- Tak
czy inaczej, wkrótce się dowiesz.

-- Jeny. -- Przeszedł mnie dreszcz. Już, ciężarówki podjeżdżały, we mgle
kurzu i~wycieków ze zbiorników gazu, żeby przygotować tę rzecz do startu
za kilka godzin.

-- Weź pod uwagę, że możesz tym wrócić tak samo szybko -- powiedział Alan.
-- To nie tak, że opuszczasz Ziemię na miesiące, czy coś.

\threeast

Powinienem się domyślić czego oczekiwać, gdy badanie zawierało kilka
minut wicia się przez długą wąską, ciemną rurę z~czujnikami
sprawdzającymi objawy, żeby być pewnym, że nie cierpię na klaustrofobię.
Było to coś, o~czym nigdy wcześniej nie myślałem. Psycholog misji
powiedział, że byłbym bardzo dobrym speleologiem. Odpowiedziałem mu, że
będę o~tym pamiętał, jeżeli kiedyś zapragnę bezpieczniejszego hobby.

Skafander przeciążeniowy wypakowany żelem był ostatnim krzykiem
techniki. Co każdy nosił pod nim, było rodzajem obcisłej miękkiej
kombinacji ubrania, a~należało założyć pod\ldots

-- Dlaczego pielucha? -- zapytałem z~oburzeniem.

-- Na wypadek, gdyby ta tabletka nie zadziałała.

-- Mogę dwie?

Powiedziano mi, żebym zostawił sobie szkła i~załatwiono mi nowy
wyprodukowany w~Ameryce komputer, do którego ściągnąłem wszystkie AI
oraz programy zarządzania systemami. Komputer, razem z~moim czytnikiem
biotech i~dyskiem danych, trafił do kieszeni na udzie.

Pilotem była Camila Hernandez. Była kilkanaście centymetrów niższa i~lata młodsza ode mnie, wahałem się, żeby ocenić ile kilogramów lżejsza.
Jej twarz byłaby ładna, gdyby nie była taka chuda, prawie anorektyczna,
oraz gdyby nie była ostrzyżona na krótkie pięć milimetrów. Potrząsnęła
moją ręką, gdy usiedliśmy twarzami do siebie z~tyłu przyczepy, która
przetoczyła się do statku. Poza tym nie miała dużo do powiedzenia. Miała
na twarzy wyraz okrutnej koncentracji i~sądziłem, że w~głowie układała
sobie plan testów, więc siedziałem cicho.

Statek \emph{Bluźniercze Geometrie} miał jakieś dwa metry wysokości w~centrum. Właz wejściowy był w~centrum. Kokpit nie był. Camila wspięła
się, potem ja, żeby wić się wzdłuż długiej, ciemnej i~wąskiej rury.
Wyłoniłem się w~przestrzeni na mniej niż metr głębokiej i~na dwa metry
długiej i~na trochę poniżej dwóch metrów szerokiej. Camila już leżała
twarzą do dołu na łożu po prawej. Wciągnąłem się na drugie łoże koło
niej, a~na końcu włożyłem głowę do bańki hełmu, który sam się uszczelnił
dookoła szyi. Za mną, mogłem usłyszeć, jak węże i~dysze wysuwają się i~wślizgują się w~mój kombinezon.

Z przodu hełmu, zakrzywiając się z~przodu, powyżej i~poniżej nas obojga,
było coś, co wyglądało jak bańka okna. Z~całą pewnością z~zewnątrz nie
było widocznego okna, więc to musiała być jakaś szczelna wersja szkieł,
ale iluzja mojej głowy na początku pojazdu była doskonała. Powyżej,
błękitne niebo. Poniżej, zakurzona ziemia, technicy wykonujący
ostateczne zadania. Przed nami lśniące ciepłem i~mirażami Groom Lake,
cień statku rozciągający się przed nami w~późnym popołudniowym słońcu.

Ręce Camili oparły się z~przodu jej głowy na panelu kontrolnym. Panel
kontrolny przede mną był zakryty plastikową pokrywą i~uszczelniony.
Oparłem ramiona na nim i~spojrzałem do przodu. Absurdalnie pomyślałem o~sylwetce lecącego Supermana i~stłumiłem nerwowy śmiech.

-- Wszystko w~porządku?

-- Ta, w~porządku.

-- Dobrze. Wpuszczanie żelu.

Camila rozsunęła sworzeń. Z~głośnym sykiem przestrzeń dookoła naszych
nóg i~torsów wypełniła się pianką jakiegoś rodzaju, która natychmiast
zmieniła się w~rodzaj gumy, która ustępowała jedynie tyle, żeby móc
oddychać. Naciskała mocno na moje stopy, nogi i~na przeponę.

-- Zakończono -- powiedziała. Potem, innym tonem: -- \emph{Bluźniercze
Geometrie}, gotowi!

Dźwięk czegoś się odłączającego i~syreny na zewnątrz ryczały przez dobre
dwie minuty. Powietrze przed nami zaczęło trzaskać i~wyginać. Głębokie
buczenie pojawiło się w~moich stopach, potem w~nogach, rezonując w~każdej kości i~każdym zębie na innej częstotliwości. Razem z~buczeniem
narastało poczucie nacisku, ucisku i~niesamowite uczucie, że moje ręce i~nogi są szarpane i~rozciągane.

Statek zaczął się poruszać, powoli z~początku, kurz i~jaskrawe plamki
płynące dookoła wykrzywionego okna, ziemia poruszająca się pod nami,
szybciej i~szybciej, aż stała się plamą prostych linii. Spojrzałem do
góry. Góry otaczające dno jeziora zbliżały się w~strasznym tempie, a~potem\ldots

\ldots znikły.

Widok dookoła nas zrobił się czerwony, a~potem się przejaśnił.
Domyśliłem się, że wizjer korygował niezliczoną liczbę zniekształceń i~załamań. Pomimo skafandra żelowego i~żelu, czułem jak gdyby moja waga
tak szybko wzrastała, że moje kości mogłyby pęknąć. Każdy staw bolał.
Poniżej, krajobraz zmieniał się jak w~panoramie, potem stał się
zamazany, gdy minęliśmy pas startowy, zwalniając trochę, gdy się
wznosiliśmy. W~szokująco krótkim czasie byliśmy nad Atlantykiem. A
krótko potem, ponad atmosferą.

Nie miałem dużo czasu, żeby docenić błękitny widok Ziemi. Przyśpieszenie
spadło, skończyło się, a~potem, gdy moje kości powoli wracały do swojej
długości, wróciło, łagodniejsze, ale bardziej natarczywe. Dookoła statku
coś rozkwitło jak neon, potem wyblakło z~widoku, gdy oprogramowanie
korekcyjne zareagował jak zmiana tęczówki. Ziemia na horyzoncie rzuciła
się na nas i~znikła.

Patrzyłem na przestrzeń i~gwiazdy na niebie, które nas otoczyły.

-- To wszystko na razie -- powiedziała Camila spokojniejszym głosem niż
kiedykolwiek od niej słyszałem. -- Chcesz zobaczyć, jak wyglądamy z~Ziemi?

-- Pewnie -- odpowiedziałem niechętnie, odrywając wzrok od gwiazd.

Camila uderzyła kilka klawiszy i~część wizjera na wprost stało się
nieprzezroczyste i~pokazało, trochę dezorientująco, pasmo niebieskiego
nieba. Wzdłuż tego pasa pędziła przyśpieszająca czerwona kula ognia. A
potem, w~kolejnych klatkach, widok przeskakiwał do nawet krótszych
rzutów naszej coraz mniejszej smugi bolidu przez coraz ciemniejsze
niebo.

-- Dwadzieścia siedem alertów stacji przeciwrakietowych -- odczytała skądś
-- i~dwieście osiemdziesiąt raportów spotkania UFO do tej pory. Nieźle.

-- Jak teraz wyglądamy? -- spytałem.

Rzuciła mi spojrzenie z~boku, podwójne zniekształcone przez nasze oba
okrągłe akwaria. 

-- Hm -- powiedziała -- nie chciałam cię martwić, wiesz?
Ale skoro pytasz, ponieważ dookoła nas są tysiące kilometrów
sześciennych zjonizowanego gazu, jesteśmy, hm, widoczni gołym okiem\ldots

Widok na niebie. Zgodnie z~napisami w~cyrylicy na dole, to był program z~lokalnej stacji w~Mińsku realizującej program na dworze w~wyniku raportu
o UFO lub strachu przed wojną. Ponad nierównym horyzontem słupów domów w~nowej technologii, błyszczeliśmy wyraźnie, jasna gwiazda.

\threeast

Nie usłyszałem alarmu zbliżeniowego, który był ustawiony na uszy pilota.
Pierwsze, czego się dowiedziałem, było, gdy Camila odwróciła się do mnie
i spytała nagle: 

-- Matt, jesteś religijny?

-- Nie -- odparłem zaskoczony pytaniem.

-- Ok -- powiedziała. -- Święta Maryjo, Matko Boża, módl się za nami
grzesznymi teraz i~w godzinę śmierci naszej. Amen. -- Potem innym tonem -
Dobrze, to jest ten moment, kiedy to staje się nudne.

Jej palce grały po tablicy kontrolnej szybciej niż przy pisaniu
dotykowym. Natychmiast, w~miłosiernie trafnym odruchu, położyłem dłonie
z tyłu hełmu i~wcisnąłem go tak mocno, jak mogłem w~ustępliwą
powierzchnię łoża żelowego. Wizjer stał się czarny. Chwilę później
trzasnęliśmy bokiem. Było to pierwsze z~wielu takich zmian,
przyprawiających o~mdłości chwil nieważkości, po których następowały
brutalne impulsy przyśpieszenia w~nieprzewidywalnych kierunkach.
Rzeczywiście to stało się nudne, jak przejażdżka cyrkową kolejką górską,
która trwa zbyt długo, powtarzające się dreszczyki emocji aż do momentu,
gdy łączą się w~ciągły nudny strach i~pragnienie \emph{jej} opuszczenia.

Po tym, co wydawało się długim czasem, ale, zgodnie z~zegarem w~szkłach,
było tylko godziną, gwałtowne ruchy się skończyły. Cały czas
przyśpieszenie narastało, ciągle daleko od jednego g, ale już
zauważalne.

-- Wyprzedzimy je -- powiedziała Camila. -- Każdy pobliski impuls daje nam
więcej gazu i~więcej pływu, i~pozwala nam lecieć szybciej. Naprawdę
fajne -- uśmiechnęła się do mnie, pot wysychający na jej twarzy. -- Oraz
naprawdę \emph{wchłania} uderzenia laserów -- dodała niepokojąco.

Poczułem się absurdalnie przepraszający.

-- Nie wiedziałem, że Europejskie Siły Powietrzne próbowałyby nas
zestrzelić.

Camila się roześmiała. 

-- Chłopie, to był przyjacielski ogień. Nasza
strona. Obrona Orbitalna USAF.

\emph{-- Co?}

-- Nic zaskakującego. Nieautoryzowany start, przelot nad terytorium UE\ldots musieli \emph{coś} zrobić, żeby przekonać komuchów, że nie zamierzamy
ich zbombardować jądrówkami. Pewnie też inne cele polityczne. Federalni
są naprawdę przejęci nowymi stronami korzystającym na przetasowaniach w~aparacie obrony i~bezpieczeństwa UE, jeżeli o~tym mowa. -- Uśmiechnęła
się dziko. -- Zburzmy cały śmierdzący blok komuchów.

-- Boże, brzmisz jak Jadey -- powiedziałem. Myśl o~niej wróciła jak rana,
która nie bolała, póki trwała walka.

-- Jadey Ericson? Znaczy \emph{znasz} ją? O! Opowiedz mi wszystko!

-- Czy nie powinnaś sterować tego pojazdu?

-- Przez następne kilka dni będzie lecieć samo, póki nie będę musiała
zacząć halsować -- zmarszczyła brwi. -- No, \emph{to } jest nudne. W~międzyczasie nie ma nic do roboty. Przy okazji, możemy zdjąć nasze
hełmy. Przekręć je \emph{tak}, w~prawo, potem w~lewo.

Oboje odetchnęliśmy głęboko, powietrzem, które było nieodróżnialne od
tego, które było w~hełmach, oprócz może lekkiego uspokojenia wzajemnie
ludzkim zapachem.

-- Cóż, oto jesteśmy. Możemy ssać glop z~tych tub, wodę z~tamtych. Możemy
sikać, ale nie możemy się wypróżniać. Możemy spać. -- Wskazała kciukiem
do tyłu. -- Te rzeczy zostają dookoła nas, póki nie zadokujemy. Równie
dobrze moglibyśmy być w~pierdolonym gipsowym odlewie. Najlepiej o~tym
nie myśleć, co? -- Camila się rozpogodziła. -- Ale możemy rozmawiać.
Możemy śledzić wiadomości, dostajemy je laserem, jeżeli chcesz sprawdzić
szkła, zobaczysz je wszystkie. -- Wzięła łyk wody z~rurki. -- Mów do mnie
albo oszaleję.

-- Albo ja będę mówić. -- Chciałem zobaczyć najnowsze wiadomości, ale
wiedziałem, co miała na myśli.

\threeast

Opowiedziałem jej swoje przygody, pomijając jedynie naturę informacji,
które przewoziłem.

-- To, czego nie rozumiem -- powiedziała, gdy skończyłem, patrząc na
niezmieniający się widok gwiazd, żadna nasza prędkość nie mogła
widocznie zmienić -- w~jaki sposób z~tym \emph{żyjecie}. Cała ta
korupcja, władza i~gówno.

-- Nie jest tak źle -- powiedziałem. -- Państwo jest trochę większe niż w~Ameryce, pewnie, ale spójrzmy prawdzie w~oczy, robi trochę więcej
rzeczy. Więcej szkół, mniej zanieczyszczeń, żadnych żebraków\ldots -
zaśmiałem się. -- I~eksploracja kosmosu, nie zapominajmy o~eksploracji
kosmosu.

-- Ale \emph{Partia!} -- odparła. -- Jak możecie to znieść? Mam na myśli,
nikt nie wierzy już w~komunizm, nawet komuchy.

-- Och, tak, oni wierzą -- powiedziałem. -- Po prostu tego tak nie
nazywają. Nazywają to ,,zrównoważone społeczeństwo'', co ekonomiści
nazwali stanem stacjonarnym\footnote{ang. stationary
state -- stan gospodarki przy stałej populacji i~stałym kapitale, więcej~\url{https://en.wikipedia.org/wiki/Steady-state\_economy\#Concept\_of\_the\_stationary\_state\_in\_classical\_economics}
- przyp.tłum.}. I~myślą, że nas tam prowadzą, że wszyscy w~końcu tam
się zjawią, nawet Amerykanie.

-- Nigdy! -- odpowiedziała Camila. -- Może liberałowie ze Wschodniego
Wybrzeża mogliby iść na to, ale nie reszta.

Westchnąłem. 

-- To nie ma nic wspólnego z~tym, co każdy \emph{wierzy}.
Spadająca stopa zysku w~końcu was dopadnie. Możecie tego unikać przez
jakiś czas, eksportując kapitał, i~podążać za spadającą ratą jak za
gwiazdą znikającą za horyzontem, co, przy dostatecznie szybkim
nadążaniu, może wam przynieść wzrost na chwilę, ale wszystko, co
dostaniecie to w~pełni kapitalistyczny, i~całkowicie skapitalizowany,
świat z~niskimi stopami zysku, a~wtedy nigdzie nie możecie pójść, tylko
w gospodarkę stanu ustalonego, ekonomię, która cicho tyka, zamiast się
rozwijać. W~stanie stacjonarnym łatwo jest pracownikom skończyć z~kapitałem zatrudniającym, socjalizm, prawie bez różnicy.

Spojrzała się na mnie podejrzliwie.

-- To Marks, prawda?

-- Błąd -- powiedziałem. -- To John Stuart Mill.

-- Bez różnicy -- odparła. -- Pieprzone liberały. -- Markotna cisza, potem:

-- Tak czy siak, to wszystko brednie, ponieważ mamy teraz kosmos do
ekspansji, na zawsze!

-- \emph{Jakiej } ekspansji? Nie ma \emph{zysku} w~przestrzeni
kosmicznej. Nikt nie jest tak zdesperowany, by chcieć tutaj \emph{żyć}.
Ta grupa libertarian, którzy tego próbowali, nie mogli tego znieść, nie
mogli znieść siebie\ldots

-- Ta, ta -- powiedziała. -- Wiem o~Hell-Five. Ale na dłuższą metę\ldots

-- Na dłuższą metę -- odparłem, cytując innego podejrzanego i~zmarłego
ekonomistę -- wszyscy będziemy \emph{martwi}\footnote{cytat z A
Tract on Monetary Reform, w: The Collected Writings of John Maynard
Keynes, t. 4, Palgrave Macmillan, Londyn 1971. - przyp.tłum.}.

\threeast

Konwersacja podtrzymywała nas na długich odcinkach, skończyliśmy,
wiedząc o~sobie znacznie więcej, jak kochankowie, a~pomiędzy tym, snem,
patrzeniem w~gwiazdy, był kanał informacyjny.

Mary-Jo przesadziła, gdy powiedziała, że rozpętuje się piekło nad
Czerwoną Europą. Bez wątpienia wiadomość Drivera wrzuciła klucz w~maszyny państw robotniczych. Partia, Federalne Biuro Bezpieczeństwa,
Europejska Armia Ludowa manewrowały przeciwko sobie w~bezprecedensowo
jawny sposób: budżety militarne kwestionowane, uruchamiane śledztwa
wobec naruszeń prawa FBB, nagłe awanse, degradacje, ćwiczenia wojskowe
bez zgody, niezapowiedziane mobilizacje rezerwistów (które, zdaje się,
dodały unikanie poboru do moich przestępstw).

W tym tempie, to była tylko kwestia czasu, zanim ludność, również,
będzie chciała zabrać głos. Czy rozpęta się piekło, w~tym momencie było
ciężko przewidzieć. W~całej Europie ruch pokojowy, dotychczas konający
dodatek do oficjalnej polityki zagranicznej, organizował masowe
demonstracje, na których sztandary Websów były dobrze widoczne na
froncie. To, przynajmniej, było coś przeciwko (również narastającym)
,,patriotycznym'' demonstracjom antyamerykańskim, które były, z~nachalnym
naruszenie prawa, wspierane przez frakcje w~Armii. Rzekomą przyczyną
tych demonstracji był ciągłe pojawianie się ,,odkryć'' na temat powiązań
USA z~,,angielskimi faszystowskimi terrorystami'', którzy zostali
aresztowani po złamaniu kodów. Kamery pokazywały obciążające skrytki
broni.

-- To jest właśnie wasz problem. -- Camila poinformowała mnie w~zaufaniu.
-- Nie macie prawa posiadać broni. To dlatego Ruscy was podbili i~dlatego
nie możecie ich wyrzucić.

Patrzyłem na nią z~otwartymi ustami.

-- Skąd to bierzesz? -- spytałem w~końcu. -- \emph{Wszyscy} w~Europie mają
broń. W~każdym razie od rewolucji. Ruscy byli zszokowanie brakiem
gotowości, kiedy wkroczyli i~zdecydowali się być cholernie pewni, że
nigdy to się nie zdarzy. Póki nie jesteś pacyfistą sumienia, wiesz,
kwakrem lub coś, \emph{obowiązkowe} jest posiadanie Makarowa, AK i~amunicji w~domu. Byłem najlepszy w~klasie w~strzelaniu z~pistoletu w~\emph{podstawówce}, jeżeli chcesz wiedzieć. Odsłużyłem rok wojska w~wieku osiemnastu lat, tuż po urodzinach. Mógłbym dalej trenować, gdybym
dołączył do OCO, Obrony Cywilnej i~Oporu, ale nigdy się tym nie
zająłem. Jednak ciągle jestem w~rezerwie.

Teraz ona się gapiła. 

-- Więc dlaczego nie powstaniecie i~nie obalicie
komunistów?

-- Ponieważ mało kto, do cholery, \emph{chce }, oto dlaczego! Partia
naprawdę została wybrana! Wszystko, co musimy zrobić to zagłosować
przeciwko niej!

Camila nadal była przekonana, że był to rodzaj oszustwa. Broń zapewniana
przez rząd nie miała znaczenia i~wybory z~zakazem kupowania polityków
prze bogaczy nie mogły być wolne.

\threeast

Nie było żadnych wiadomości o~Jadey.

-- \emph{No dobra} -- powiedziała Camila, pięćdziesiąt godzin po starcie -
hełmy włóż. Czas zacząć pracę. -- Brzmiała znacznie bardziej gorliwie niż
jej wcześniejsze uwagi o~nudzie mogłyby sugerować. Byliśmy już daleko za
orbitą asteroidy i~mieliśmy zacząć halsować, w~kierunku słońca, żeby ją
przeciąć.

Wizjer zmienił się z~czarnego na biały, ze słońcem, przysięgam,
wskazywanym przez znak gwiazdki w~środku, a~asteroida jako stale
zmieniający się ciąg liczb. Wyglądało to jak te prymitywne gry ASCII,
które można znaleźć pochowane, jako wewnętrzne żarty programistów, w~najbardziej niejasnych rejestrach adresowych systemów operacyjnych.

Jeżeli ekran był grą, rzeczywiste podejście, które rejestrowało, było
kolejną jazdą z~zaciśniętymi pięściami, gwałtowniejszą niż uniki przed
rakietami. Camila wymieniała krótkie transmisje głosowe ze stacją
pomiędzy każdą nagłą zmianą kursu. Trwało to godzinami i~skończyło się
swobodnym spadaniem. Szerokim gestem Camila przełączyła kontrolę
wizjera, a~asteroida zmieniła się w~trójwymiarowy kolorowy widok, taki
jak widzieliśmy pierwszy raz, tak realny jak w~telewizji. Stacja
rozwinęła się w~widoku, jej wygląd zmieniał się od czegoś małego i~krętego, jak obwód elektryczny, to czegoś wielkiego i~krętego, jak
fabryka. W~tym samym czasie, ogólny widok zmienił się z~lotu ku
asteroidzie na przelot nad nią. Ostateczne drobne korekty były coraz
bardziej delikatne, od uczucia przepychania się w~tłumie, poprzez
szarpnięcie żebraka, aż do podejrzenia kradzieży kieszonkowej.

Z ostatnimi podmuchami rakiet wstecznych \emph{Bluźniercze Geometrie}
osiadła w~chwytakach, które klikały dookoła krawędzi statku. Pojawiły
się dźwięki uderzeń i~skrzypienia.

-- Co to?

-- Podłączanie śluzy powietrznej. -- Uśmiechnęła się. -- Nie dostanę nawet
dziękuję?

Przybiłem jej piątkę. 

-- Ta, dzięki!

-- Nie zdejmuj na razie hełmu -- powiedziała. Sięgnęła po przełącznik. -
Usuwanie żelu.

Niebieskawy płyn rozprysnął się z~dysz po bokach wizjera, zlewając nasze
ramiona i, gdy przestrzeń się przejaśniła, inne dysze syczały na naszych
plecach i~bokach. Camila podniosła się na rękach, otwierając przestrzeń
pomiędzy nią a~łożem, a~ja postąpiłem podobnie. Płyn spowodował
wyschnięcie żelu, w~gumowe paski, potem odparował w~podmuchu ciepłego
powietrza.

-- O! -- poruszyłem nogami, obróciłem biodrami. -- Jest super.

Nagła fizyczna wolność sprawiła, że bardziej się bać zniknięcia
pozostałych ograniczeń. Po raz pierwszy, rzeczywiście czułem coś jak
klaustrofobię: przestrzeń wokół mnie była zbyt mała, a~powietrze nie
docierało do moich płuc.

Oboje wyglądaliśmy jakbyśmy byli pokryci poszarpanymi bandażami, roboty
z grobu mumii. Camila otarła lepkie kawałki suchego żelu. Wisiały
denerwująco w~powietrzu, unosząc się.

-- W~następnym modelu wentylator usuwający!

Coś stuknęło i~zrobiło dźwięki ścierania. Camila przekrzywiła głowę,
słuchając głosu w~jej uchu.

-- To jest to -- powiedziała. -- Śluza jest hermetyczna. -- Machnęła ręką na
mnie. -- Idź pierwszy. Muszę wyłączyć światła.

Palcami rąk i~nóg pchnąłem się do tyłu wzdłuż wąskiej rury do włazu.
Tuż, zanim moje stopy dotknęły go, płyta gładko odsunęła się na bok.
Nawet w~kombinezonie i~hełmie, mogłem poczuć zmianę ciśnienia i~powiew.
Światło zaświeciło się od dołu, lub zza, moich stóp. Kontynuowałem ruch
w kierunku, który teraz był ,,dołem'', choć nie z~powodu mikrograwitacji
asteroidy, której nie w~ogóle nie czułem. W~dół, potem: koło pierścieni
uszczelniających, potem znów w~dół długą, ale szerszą rurą, potem na
zewnątrz.

Przez chwilę wisiałem, ręce trzymały się na wyjściu tuby, stopy trochę
powyżej podłogi z~metalowej siatki. Port, w~której statek zadokował, był
wielki, około trzy metry wysokości poniżej rury śluzy, dwadzieścia długi
i na dziesięć szeroki. Skrzynie i~elementy ekwipunku były przywiązane
lub przyklejone do stojaków i~belek dwuteowych. Fluorescencyjne światło
błyskało nad krzykliwymi kodami. Naprzeciw mnie były wielkie drzwi,
przed którymi unosił się mały, krótkoobcięty mężczyzna w~mundurze z~wieloma naszywkami różnych misji, jedną ręką trzymając się poprzecznej
belki, drugą opierając na rękojeści standardowego Aerospatiale Officier
9mm, wepchniętego w~pas biodrowy.

Puścił pistolet i~gestem nakazał zdjęcie hełmu. Puściłem się krawędzi
śluzy i~tak zrobiłem, zaczynając opadać w~powietrzu. Camila pojawiła się
w szybie i~zrobiła to samo bardziej wdzięcznie.

Miejsce śmierdziało: ostre, organiczne zapachy człowieka, rośliny i~zwierząt połączone z~przykrym fetorem gorącego metalu, spalonego
plastiku, starego oleju maszynowego. Prawie się zadławiłem. Camila tylko
zmarszczyła nos. Mężczyzna obdarzył nas krzywym uśmiechem.

-- Przyzwyczaicie się -- powiedział. Wyciągnął prawą rękę, pustą. -- Jestem
Paul Lemieux. Witamy w~Rewolucji.

Nawet patrząc do góry na niego, jedną ręką trzymającego się kratki
podłogi, gdzie zdryfowałem, mogłem uchwycić sens, lub w~tym momencie
jego dłoń, i~pomyśleć:

\emph{O cholera!}



\chapter[Zakład Produkcyjny]{11 Zakład Produkcyjny}

Tutaj, na szczycie wzgórza, wiatr od morza był nieunikniony i~stały.
Dawał samolotom nieco dodatkowej siły nośnej, gdy podskakiwały i~szarpały się wzdłuż pasa startowego startującego, oraz pozwalał na
niższą prędkość podejścia, gdy lądowały, również podskakując i~szarpiąc.
Jeżeli pilot zbliżył się za mocno do prędkości przeciągnięcia, to kiedy
ustawał wiatr, samolot spadał.

Wiatr jęczał w~wysokich bambusowych słupach, sprawiał, że się wyginały,
a sterowce kołysały się na cumach jak przynęta na haku. Inne, puszczone
z cumy, ale przywiązane do ziemi, napinały się i~falowały. Grupy
pracowników używały lin, kołowrotów i~głównie siły, by przywlec i~utrzymać pojazd dla pasażerów i~załogi, żeby weszli lub wyszli. Cysterny
wody na balast i~nafty jako paliwa (te wcześniejsze poręcznie pracowały
też jako wozy strażackie) śpieszyły się tam i~z powrotem.

Gregor siedział w~przeszklonej poczekalni i~obserwował, zafascynowany
wszystkim, co zobaczył. Nie odwiedzał lotniska od czasów dzieciństwa, a~było to w~pewien sposób irytujące, że cała ta ekscytująca aktywność,
którą zapamiętał z~tamtych czasów, nigdy nie była w~takim bliskim
zasięgu.

Koło niego siedział Salasso, dalej Elizabeth, każde, jak on, z~torbą
podróżną przy nogach. Torba Salasso była mała i~zrobiona z~czegoś, co
wyglądało jak elastyczne aluminium. Ta Gregora była masywniejsza i~niewiele cięższa, ze skóry dinozaura. A~Elizabeth miała laminowaną
walizkę, prawie na granicy limitu wagi. Była ubrana raczej dobrze na
lot, bluzka, spódnica i~długi skórzany płaszcz. Gregor i~zaur byli w~ich
zwykłych ubraniach roboczych, choć w~przypadku Gregora, z~powodu
nalegania matki, świeżo wyczyszczonego.

Elizabeth zobaczyła, że patrzy i~się uśmiechnęła.

-- To jest wspaniałe! -- powiedziała. -- Jakie niesamowite miejsce i~jak
blisko.

-- Właśnie tak sobie pomyślałem -- odpowiedział Gregor. Uśmiechnął się. -
Może ludzie, którzy tutaj pracują, nigdy nie odwiedzają portu.

Salasso nic nie powiedział. Jego cienkie usta były wykrzywione do dołu,
a jego ramiona wisiały. Jego powieki mrugnęły, jego błona migawkowa
mrugała częściej niż zwykle. Jego długie palce były zaciśnięte na
kościstych kolanach.

-- Czy ktoś mógłby nabić mi fajkę? -- powiedział.

-- Nie chcesz teraz\ldots zasnąć -- zbeształa go Elizabeth.

-- Chcę -- odparł Salasso. -- Mam nadzieję, że będę mógł. Chciałbym,
żebyście mnie \emph{wnieśli} nieprzytomnego na pokład. Ale zdaję sobie
sprawę, że dla wszystkich zainteresowanych byłoby to niegodne. Więc
proszę, przygotujcie mi fajkę. Nie ufam swoim dłoniom, a~chcę ją mieć
gotową przynajmniej gdy już wejdziemy na pokład.

-- Czy palenie jest dozwolone? Zastanawiam się. -- Gregor spytał, gdy
Salasso podał mu fajkę i~woreczek widzialnie drżącymi dłońmi.

-- Tak -- odpowiedział Salasso. -- Sprawdziłem to bardzo dokładnie. Tylko
silne narkotyki mogą mnie uspokoić w~czasie podróży jednym z~tych
urządzeń.

-- Och, powinniśmy wynająć łódź grawitacyjna -- powiedział Gregor. -- Cóż,
czemu o~tym nie pomyślałem?

Oczy Salasso obróciły się, źrenice nieco czarniejsze w~czarnych
tęczówkach.

-- Pomyślałem o~tym -- odparł. -- Ale nie stać nas na taki wydatek.

-- Prawdę powiedziawszy -- powiedział Gregor -- nawet nie przyszła mi na
myśl taka opcja. Gdzie je można wynająć?

-- Och, niektórzy zaurowie je wynajmą, jeżeli wiesz, kogo pytać. Ale
zwykłe tylko innym zaurom. Jest to, jak powiedziałem, kosztowne w~najlepszych przypadkach, a~w tym momencie kompletnie wykluczone,
ponieważ wszyscy właściciele łodzi na planecie zarabiają fortunę na
południu, kierując dinozaury na rynek mięsa.

-- To musi być niezły widok -- powiedziała Elizabeth.

-- Prawda. Choć trochę niepokojący.

-- To tylko bydło -- powiedział Gregor. Oczywiście, \emph{olbrzymie}
bydło.

Salasso potrząsnął głową. 

-- Nie jestem, jak wy ludzie to nazywacie,
sentymentalny -- powiedział. -- Niektórzy z~nas są! Ale wszyscy czujemy\ldots pewien szacunek, pokrewieństwo z~tymi szlachetnymi, choć nieświadomymi
bestiami, na które polujemy. Potrzeby muszą\ldots Jesteśmy rodzajem
mięsożernym, znacznie bardziej niż wy. Ale ciągle, byliśmy myśliwymi,
zanim staliśmy się hodowcami, a~przed tym byliśmy ofiarą polowań.
Zachowujemy coś z~naszej przeszłej natury.

-- I~chcesz powiedzieć, że zachowujecie coraz mniej?

Salasso odchylił się. 

-- Tego nie mówię.

Ostrzegawcze spojrzenie od Elizabeth wstrzymało chęć Gregora, by
kontynuować temat. Nie powstrzymało go to od rozważań. Generalnie było
to akceptowane, nie był pewien, na jakiej podstawie, że zaur rozwijały
cywilizację przez miliony, jeżeli nie dziesiątki milionów lat. Nawet
uwzględniając ich dłuższe życia i~większą inteligencję, było coś
nieprawidłowego i~wstrząsającego w~sugestii, że zachowali niektóre
tradycje ze stanu dzikości.

A dla ludzi, Nova Babylonia przedstawiała ciągłość i~starożytność! Nawet
Wieża i~zatoka, chociaż były przedludzkie, stały płytko na głębinach
czasu ewolucyjnego, który dla zaur był \emph{historią}.

Skończył nabijać fajkę i~ją oddał.

\threeast

Zabrzmiał dzwonek, wiadomość pojawiła się na ekranie, obwieszczając
nadchodzący odlot ich rejsu. Wzięli torby i~poszli na pole wraz z~pozostałymi około pięćdziesięcioma pasażerami, w~cień stumetrowego
półsztywnego sterowca. Salasso wspiął się po schodkach do gondoli bez
dalszych wahań, kierując się do sekcji dla palących z~tyłu i~stracił
przytomność, zanim statek wystartował.

-- Cóż, tyle za supercywilizowanych -- zauważył Gregor, gdy on i~Elizabeth
schowali swoje torby i~usiedli po przeciwnej stronie stolika przy oknie
kilka rzędów z~przodu od miejsca, gdzie osunął się Salasso.

-- Prawdopodobnie dla niego jest to jak, nie wiem, dla nas wypłynięcie na
ocean rozklekotaną tratwą. A~niektórzy ludzie też się boją latać.

Sterowiec szarpnął i~podłoga podniosła się do góry, gdy kabel na dziobie
został zwolniony sekundę wcześniej od tego na rufie. Gregor złapał brzeg
stołu, Elizabeth wyrzuciła ramiona w~poprzek. Silniki zaryczały, by
sprowadzić statek do poziomu.

-- Strach latania, co? Kto by pomyślał?

Zaśmiała się i~oboje zaczęli obserwować pejzaż, gdy sterowiec się
wznosił. Jak szerokie miasta, jak małe statki.

-- O! -- powiedziała Elizabeth -- możemy zobaczyć nasz cień.

-- Gdzie? \ldots Och, racja. -- Był tam, falując przez ulice, pola, rzekę
jak czarny płatek, poznaczony błyszczącą iskrą odbitego słońca na
basenie lub wodzie. Potem linia brzegowa jak na mapie. Chmury z~boku i~z~góry.

Ale gdy sterowiec osiągnął wysokość rejsową i~podążał jakiś czas
wybrzeżem na południe, nawet ta fascynacja zblakła. Gregor i~Elizabeth
odwrócili się od okna i~usiedli jak doświadczeni podróżnicy. Salasso
ciągle odsypiał efekty wypalenia całej fajki samemu.

-- Ciągle zainteresowana? -- spytał Gregor.

-- Och, tak. Powinno być interesująco, nawet jeżeli nic nie znajdziemy.
Mam na myśli, wiem, że byłoby to dla Ciebie rozczarowaniem, ale\ldots

-- Ta, wiem, co masz na myśli -- westchnął Gregor, rozglądając się po
kabinie. Większość pasażerów wyglądała na biznesmenów, i~albo w~duchu
popijania w~raczej wymuszonej próbie relaksu, albo pracując nad
dokumentami z~ręcznymi kalkulatorami.

-- Wiesz, co potrzebuje ten świat? -- spytał. -- Podróż \emph{Beagle}. Ktoś
powinien zasponsorować długodystansową podróż tylko dla badań i~próbek.

-- By wrócić z~teorią ewolucji?

Uśmiechnęła się, Gregor się roześmiał.

-- Sprawozdanie z~niej -- powiedział Gregor. -- Już mamy teorię. To, czego
potrzebujemy, to lepsza idea jak to się stało tutaj. Historię planety.

Elizabeth spojrzała na chwilę w~dół na pełzający brzeg, a~potem z~powrotem na niego.

-- To byłoby trudne -- powiedziała. -- Ile najazdów tutaj było? Nawet nie
możemy zgadnąć liczby większych celowych osiedleń gatunków, nie mówiąc o~wypadkach. Przysięgam, że musi się to zdarzać za każdym razem przybywa
statek gwiezdny.

Nagle się zaśmiała.

-- Co?

-- Przypomniałam sobie. Rozmawiałam z~szyprem barki w~Bailie's.
Powiedział, że jeżeli podejdziesz blisko do statku, możesz zobaczyć na
nich \emph{wąsonogi}\footnote{wsp. gromada skorupiaków
wyłącznie morskich. Jedyna grupa, która obejmuje stawonogi osiadłe.
Zamieszkują przeważnie płytkie, przybrzeżne wody, osiadając na obiektach
podwodnych, skałach, koralowcach, muszlach mięczaków, pancerzach
skorupiaków, na portowych urządzeniach, czy zanurzonych w~wodzie
częściach statków zob.~\url{https://pl.wikipedia.org/wiki/W\%C4\%85sonogi} -
przyp.tłum.}.

-- Twarde małe stworzonka te wąsonogi.

-- No. -- Ale te muszą być w~stanie przetrwać co najmniej godziny, czasem
dni, w~\emph{kosmosie}.

-- Skorupiaki próżniowe!

Przytaknęła. 

-- O proszę bardzo, to właśnie nowy gatunek. I~założę się,
że tu też się rozpowszechniły.

-- Dobrze -- powiedział. -- Nazwiemy statek \emph{Wąsonóg} i~będziemy
sponsorowani przez producenta farby przeciw porostom.

Potarła oko. 

-- Zdaje się, że to produkt zaurów. Ale, ta, coś w~tym
rodzaju. Właściciele statków. Rybacy.

-- Hej, nie patrz tak na mnie. Mój ojciec nie jest zainteresowany
badaniami.

-- A~co z~Zamkiem?

-- Mają dostatecznie dużo roboty przy Wielkiej Pracy.

Kontynuowali tę rozmowę, nie biorąc jej na poważnie, aż do tego momentu.

-- Co?

-- Jeżeli nam się uda -- powiedział Gregor -- Wielka Praca będzie
ukończona. Koniec. A~Zamek, Załoga i~Rodziny będą bogate. Więc równie
dobrze mogą dać więcej pieniędzy na badania.

Spojrzał na zewnątrz, w~dół. Linia brzegowa przesunęła się do wewnątrz,
właśnie przekraczali wschodnio-zachodnią nogę odcinka oceanu w~kształcie
L, Cargill's Sound, które oddzielało północnozachodni subkontynent od
jego większego sąsiada. Kontynent był krajem zaurów, krajem dinozaurów.

-- Wiesz, mógłbym się tym podekscytować -- powiedział. -- Żeglować dookoła
świata tylko by \emph{dowiedzieć się }więcej o~tym miejscu.

-- To byłoby cudowne -- powiedziała Elizabeth niespotykanym marzycielskim
tonem.

-- Nawet gdybyśmy nigdy nie ustalili historii planety?

-- Ta, nawet wtedy.

Wyglądało na to, że Elizabeth dalej nie chce rozmawiać, wyciągając
książkę z~kieszeni płaszcza. Gregor wyglądał przez okno przez jakiś
czas, potem zdecydował się ponaśladować podróżnych biznesmenów i~popracować.

Papier był najcięższą częścią jego bagażu. Wyciągnął dwie garści
notatek, wysupłał pióro z~wewnętrznej kieszeni i~zaczął je jeszcze raz
przeglądać. Dokumenty przedstawiały najlepszą część tygodniowej pracy
jego i~jego towarzyszy. Nawet nie próbował myśleć, jak bardzo cenna
praca leżała przed nim. Jednak, była prawdopodobnie niewystarczająca.
\emph{Jeżeli} by znaleźli jedną lub więcej osób z~oryginalnej załogi i~ciągle mieli w~posiadaniu ciągle działające stare komputery (jak
twierdził Salasso) i~\emph{jeżeli one } chciałyby się podzielić i~współpracować, wówczas \emph{może } dane i~struktury podsumowane tutaj
mogłyby być podstawą zarysu modelu, który sobie wyobraził. A~nawet
wtedy, problem nawigacji sam w~sobie musiałby być właściwie sformułowany
i sformatowany, zanim mógłby być wprowadzony.

Jeżeli, rzeczywiście, mógłby. Jego idee pojemności tych starożytnych
maszyn były oparte na niewiele więcej niż rodzinnych legendach. Nawet
jeżeli najbardziej entuzjastyczne i~niesamowite opisy były prawdziwe,
było bardzo prawdopodobne, że przez ten czas maszyny się zdegenerowały.

Wystarczy. Mógł zrobić tylko to, co mógł. Pracował równo kilka godzin,
przerywając jedynie na kawę roznoszoną przez stewarda, a~potem zauważył,
że Salasso się poruszał. Wstał i~dołączył do niego, zostawiając
Elizabeth przysypiającą nad książką.

-- Lepiej?

-- Tak -- odpowiedział Salasso. Wyjrzał przez okno i~mrugnął. -- Każdy
przyzwyczaja się do niesamowitych rzeczy. Myślę, że sam będę mógł nabić
sobie fajkę.

Zrobił to, zapalił, zaciągnął się i~podał Gregorowi. Po wypaleniu dwóch
fajek i~gdy zaur raz jeszcze stracił przytomność, Gregor wrócił na swoje
miejsce i~stwierdził, że liczby nie mają już sensu. Lub raczej, były
sensowne w~zupełnie odmienny sposób. Zaczęły przypominać fizyczną
strukturę mózgu kałamarnicy, który modelowały matematycznie.

Jakiś czas potem, po obudzeniu, odkrył, że oparł się o~stół, bokiem o~okno podobnie jak zasnęła Elizabeth. Oboje mieli położone przedramiona w~poprzek stołu, a~dłoń Elizabeth leżała na jego dłoni. Delikatnie odsunął
głowę, uważając na skurcze w~szyi. Jego i~Elizabeth włosy splątały się.
Gdy uwalniał pasma, obudziła się. Mrugnęła, spojrzała na niego,
oszołomiona, ale zaczynając się uśmiechać. Wtedy obudziła się do końca i~szarpnęła do tyłu.

-- Au! Przepraszam. -- Przeczesała włosy palcami, uwalniając je w~końcu i~usiadła prosto. -- Zasnąłeś zjarany, a~ja zasnęłam przy czytaniu.
Dokładnie na Tobie!

-- Och, w~porządku -- powiedział Gregor. -- Od czego są przyjaciele?

\threeast

-- Dobranoc -- powiedziała Elizabeth.

Gregor spojrzał znad stosu papierów i~filiżanki kawy.

-- Dobranoc.

Wyciągnęła nocną torebkę z~torby, podniosła książkę i~poszła na kraniec
kabiny, gdzie małe schody wiły się do wnętrza kadłuba. Salasso wrócił do
patrzenia się w~noc lub na odbicia, i~uniósł zwiotczałą rękę. Większość
z reszty pasażerów poszła już spać.

Po schodach, dwa zwroty. Czy grawitacja \emph{rośnie} gdy wznosisz się
nad powierzchnię. Wyglądało na to, że tak. Wewnątrz trzeszczącego,
mocnego kadłuba z~tkanin, mieściły się wąskie, słabo oświetlone
przejścia pomiędzy półprzezroczystymi plastikowymi wybrzuszeniami worków
z gazem. Użyła małej umywalni i~przeszła do kabiny, małej nieco większej
niż obudowana koja, z~laminowanego drewna. Balsa na ścianach, aluminium
jako łóżko. Cienki piankowy materac i~puchowe pokrycie. Miejsce było do
stania, rozebrania się i~odwieszenia ubrań. Koja była zbyt nisko, żeby
siedzieć, tuląc się do kolan. Leżała na boku, tuląc się do kolan.

Gregor wydawał się lekko zaskoczony i~zadowolony, że chciała w~ogóle
jechać. Naprawdę, nalegał, to nie było konieczne. Salasso musiał jechać,
ponieważ musieli wyśledzić plotki zaurów, plotki tak niejasne, że nawet
stary Tharovar nigdy ich nie słyszał. Gregor musiał jechać, ponieważ
musieli potem wyśledzić ludzi, może w~dziwnych miejscach. Nie musiała
wystawiać się na ten kłopot i~możliwe ryzyko. Dlaczego nie kontynuować
pracy w~laboratorium, gdy byli na wyjeździe?

Powiedziała mu, że nie ma zamiaru w~ogóle zostawać w~tyle. Nie ominęłaby
tego za nic. Zapłaciłaby swoją część, gdyby musiała. James zapewnił ją,
że Zamek zapłaci za jej podróż, to nie problem. Wydatki na badania.

Więc była tutaj: skulona w~wąskiej koi, kilka metrów od Gregora, w~podróży, by znaleźć coś, co mogłoby pomóc Gregorowi dogonić, i~zdobyć,
Lydię. Jej jedynym pocieszeniem było to, że sukces był mało
prawdopodobny.

Cudownie.

\threeast

Podróż na dwa tysiące kilometrów zabrała im jeszcze dwa dni i~dwie noce.
Spędzali dni tak jak ten pierwszy. Noce w~oddzielnych kabinach w~kadłubie, pomiędzy plastikowymi kulami zbiorników z~gazem.

Trzeciego ranka, przy śniadaniu, Salasso zwrócił się do Elizabeth i~Gregora.

-- Spójrzcie w~dół -- powiedział.

Sterowiec opadał na wysokość, która dla Gregora wydawała się kilkuset
metrami. Poniżej, w~jaskrawym niskim słońcu, leżały północne peryferia
Zakładu Produkcyjnego. Tutaj, były ciągle rzadkie, ale jednak
zadziwiające. Kępy i~stanowiska drzew, ich zielone gałęzie, wychylone,
potem prostopadłe do ziemi, jak w~tych drzewach decyzyjnych lub
schematach, kładły długie cienie. Niektóre miały liście w~kształcie
odwróconych parasoli, inne kształtu diamentu.

-- Jak wielkie kaktusy -- powiedziała Elizabeth.

Zaur podszedł i~pochylił się koło nich, jego oddech pachnący wędzonymi
rybami.

-- Dokładnie -- powiedział. -- Kaktus był jednym ze źródeł oryginalnych
genów. Oczywiście, były od tego czasu zmieniane.

-- Tak jak rurociągi pomiędzy nimi -- powiedział Gregor. Właśnie zaczynał
rozumieć skalę tego, co widział. Niektóre z~tych rzeczy były wysokie na
setki metrów. Sterowiec ciągle opadał i~mógł zobaczyć kropki poruszające
się na ziemi. Z~początku myślał, że to były zaury, ale teraz mógł
dostrzec, że to były pojazdy.

Przełknął, by zmniejszyć ciśnienie w~uszach. Skala znowu się zmieniła,
pojazdy były cysternami chemikaliów. Zobaczył drogę i~jego wzrok pobiegł
na południe\ldots do nawet wyższych struktur, widocznych ponad horyzontem i~zbliżających się szybko.

-- Dokowanie w~Mieście Zaurów Jeden za dwadzieścia minut -- podał steward.

Salasso coś powiedział.

-- Co?

Usta zaura drgnęły. 

-- Jego prawdziwa nazwa -- powiedział. -- Niektóre
sylaby są poza zasięgiem ludzkiego słuchu.

-- Ach -- powiedział Gregor, który w~głowie próbował przeliterować. --
Dobrze. Zatem Miasto Zaurów Jeden.

Gdy zdryfowali bliżej, stało się jasne, że miasto było z~tego samego
materiału i~tych samych kształtów co fabryka, ale rozszerzonych i~wykrzywionych w~wieże, żurawie, platformy, place, drogi powietrzne i~chodniki. Te struktury były obłożone i~laminowane gęstymi, dekoracyjnymi
przerostami z~mniejszych, bardziej kolorowych, bardziej roślinnych
wersji rośliny.

Jedna wysoka wieża pojawiła się, spiralna konstrukcja trzech pni
podpierająca platformę, która była najeżona masztami.

-- Dokowanie za dwie minuty -- powiedział steward. -- Proszę wrócić na
miejsca.

\threeast

Gregor podążył za innymi drabiną, wykonaną, prawie uspokajająco, z~czegoś jak zmutowany bambus. Platforma także była drewniana, ale bez
desek i~innych spojeń, lekko się kołysała. Dwa zaury siedziały siodle za
krawędzią platformy, operując dźwigniami, które kontrolowały wijące się
liny sterowca. Stały wylot utlenionego i~chłodnego powietrza ze szparki
w ścianie na dalszym końcu platformy przynosił niewiele ulgi od gorąca i~wilgotności.

-- Tutaj -- powiedział Salasso. Inni pasażerowie, wyładowując się lub
tylko rozprostowując nogi przed kolejną częścią podróży na południe,
przeszli przez drzwi z~podwójnego szkła. Gregor i~Elizabeth ponieśli
bagaże za Salasso, przez łukowe drzwi na lewo. Wewnątrz, znaleźli się w~zielonym, gładkim korytarzu, który rozszerzył się do okrągłego pokoju,
ściany wypunktowane zielonożółtymi światłami i~kolejnymi drzwiami.
Pierścień niskich siedzeń, jak jakiś grzybowy rozrost kory, zajmował
centrum pokoju.

-- Czekamy tutaj -- powiedział Salasso.

-- Przynajmniej jest chłodniej -- powiedziała Elizabeth, siadając. -- Na co
czekamy?

-- Winda -- odparł Salasso.

Inne zaury wchodziły i~wychodziły, większość zajętych i~niezainteresowanych, kilka wymieniających słowa i~gesty z~Salasso.

-- Dlaczego tutaj, a~nie z~innymi pasażerami? -- spytał Gregor.

Salasso wzruszył ramionami. 

-- Zostaną tutaj, by zawrzeć umowy, w~ludzkich kwaterach. Bardziej komfortowe warunki. Udajemy się głęboko w~miasto. Może głębiej. -- Zawahał się. -- Wybaczcie mi. Pragnęliście zostać
z innymi ludźmi? Mogę równie dobrze udać się sam.

-- Nie ma szans. -- Elizabeth i~Gregor odpowiedzieli.

-- Dobrze -- powiedział. -- Możecie, jednak, zostawić tutaj bagaż.

W jednym z~otworów coś zatrzymało się z~uderzeniem, wypełniając
przestrzeń. Potem to, również, otworzyło się, ukazując małą, jasną
komnatę, dostatecznie dużą, żeby kilka osób mogło stać.

Salasso wstał i~wszedł do niej. Pośpieszyli za nim i~weszli do środka.

-- Co to?

-- Jak powiedziałem -- wyjaśnił Salasso. -- Winda.

Sekcja ściany zasłoniła wejście. Potem, bez ostrzeżenia, spadli. Gregor
poczuł, że jego ciało na kilka sekund stało się lżejsze, a~potem krótko
cięższe. Przesuwne drzwi otworzyły się na otwarte powietrze.

Wyszli na trawę, a~potem Gregor i~Elizabeth zatrzymali się i~wytrzeszczyli oczy. U stóp wszystkich wieży i~silosów, wyglądali jak
myszy w~lesie. Słońce, świecąc pomiędzy pniami i~odbijając się od ich
wypolerowanych powierzchni, nie dawało cienia, przefiltrowane do
zieleni. Przez wszystkie trawiaste przestrzenie, które mogli dostrzec,
pomiędzy wieżami zaurowie wędrowali, lub siedzieli, lub biegali w~trawie. Większość zaurów wyglądała całkiem inaczej od tych, które
wcześniej widywał Gregor. Nosili różnorodne ubrania i~ozdoby: luźne,
trzepoczące spodnie i~kurty, szaty i~suknie, peleryny i~sztylety, w~różnych, żywych kolorach. Ich wysokość także zmieniała się od wyższych
niż Salasso do tak niskich, że musieli być dziećmi. Krzyki o~wysokim
tonie, wypełniały powietrze jak dźwięki nietoperzy.

Salasso spojrzał do tyłu po kilku krokach i~wrócił do nich.

-- Zapomniałem -- powiedział. -- Nie było tu nigdy człowieka.

Skinął i~poszli za nim tam, gdzie siedziały trzy zaury na małym pagórku,
trochę dalej od ścieżki. Dwa były normalnej wysokości, jeden w~czarnej
pidżamie, drugi w~luźnej szacie. Trzeci, półmetrowy, klęczący na trawie
wpatrzony w~drewniane pudełko na kółkach, które pchał obiema rękami, był
ubrany w~coś, co Gregor z~początku wziął za puszyste jednoczęściowe
ubranie.

Potem zauważył, jak ubranie jest rzadsze na karku i~zrozumiał, że to
były własne puchate pióra dziecka. Elizabeth zrozumiała to w~tym samym
momencie i~natychmiast przykucnęła kilka metrów od dziecka, patrząc
intensywnie. Salasso rozmawiał z~dorosłymi. Gregor czekał, nie chcąc
przestraszyć nikogo. Jego kolana się trzęsły. Żaden człowiek nie widział
nigdy wcześniej dziecka zaurów, nawet na fotografii. Słyszał głupie
żarty, nerwowe spekulacje, że dzieci zaurów wykluwały się z~jaj w~gorących piaskach i~żyły jak bestie, bez rodziców, póki nie udowodniły
swojej umiejętności przetrwania, że zaur nawet nie miały potomstwa, że
były tak sterylne jak starożytne, że wszystkie były skonstruowane jak
roboty w~fabrykach\ldots

Żałował, że nie pomyślał, żeby zabrać aparat fotograficzny, żeby
udokumentować to bliskie spotkanie.

Dziecko odwróciło się, by spojrzeć na Elizabeth, potem wstało. Głowa
była bardziej nieproporcjonalnie większa niż u ludzkiego dziecka. Po
pewnych uspokajających dźwiękach dorosłych, młody zaur pobiegł przez
trawę i~wpadł w~ramiona Elizabeth. Zanuciła mu, łaskocząc i~głaszcząc.
Dziecko sięgnęło pazurami do jej włosów i~zagwizdało.

-- Nazywa się Blathora -- powiedział Salasso. -- Ma dwa lata.

-- Ona jest taka urocza -- powiedziała Elizabeth. -- Super fajna. Jakie
duże oczy. Kto jest najsłodszy. O! Jakie małe usta.

-- Ostre zęby -- ostrzegł Salasso, gdy palce Elizabeth dotykały wysokich
kości policzkowych dziecka. -- I~apetyt na krew ssaków.

-- Och. -- Elizabeth trochę się wycofała. -- Gregor? Chcesz ją potrzymać?

-- Och tak.

Kołysząc zauryjskie dziecko, tonąc w~czarnych oczach tak głębokich jak
przeszłość jej gatunku, Gregor miał jedną z~tych chwil, gdy czas się
zatrzymuje. Rzadkość tego momentu, ten przywilej, ogłuszył go. Jak wielu
ludzkich rodziców zawierzyłoby ich dziecko rękom innego człowiekowatego,
jednemu z~wysokich owłosionych ludzi z~zaśnieżonych gór? Odnalazł się,
odpowiadając na zaufanie zaurów, w~napływie czułości i~poczucia ochrony
wobec tego małego, ale ważnego życia.

Oddał Blathorę dorosłym i~odkrył, że zostawiła odrobinę smarkatego,
kredowego guano na jego udzie.

\threeast

-- Dokąd idziemy? -- spytał Gregor, po półgodzinie spaceru po mieście.

Salasso spojrzał do tyłu.

-- To jest bardzo bezpieczny obszar, poza procesami przemysłowymi. Jak
park, ale bezpieczniejszy. Jest używany do odpoczynku, zabawy i~nauki.
Oczekuję spotkać jednego z~moich starych nauczycieli.

-- Nauczyciele!

-- Dlaczego się śmiejecie?

-- To zabawna myśl, że zaury mają nauczycieli.

-- Myśleliście, że się wykluwamy wiedząc wszystko?

-- Niektórzy ludzie tak myślą -- powiedziała Elizabeth. -- Naprawdę.

-- Nie, to nie to, to bardziej\ldots -- Gregor wzruszył ramionami. -- Zdaje
się, że wyobrażałem sobie, że jesteście nauczani przez maszyny, lub coś.

-- Maszyny! -- zagrzmiał Salasso. Poczekał na nich, żeby szli koło niego i~kontynuował ciszej:

-- Mamy zasady, umowę. Nie jest narzucona siłą, ale\ldots prawie wszyscy z~nas dostrzegają jej mądrość, by nie dzielić się z~wami informacjami, a~w
szczególności nie o~nas. Jesteśmy bardzo prywatnym ludem i~ostrożnym.
Ale kiedy słyszę takie rzeczy, chciałbym, żebyśmy mogli być bardziej
przygotowani.

Syknął przez zęby, westchnienie. 

-- Ale nie możemy. Nasze społeczeństwa
stałyby się mniej odrębne, a~silniejsze wchłonęłoby słabsze.

Gregor rozejrzał się dookoła, ponownie spojrzał w~górę na kompleks
biotechnologiczny, błyszczący i~obcy. Choć było to widowiskowe, nie
przyciągało go w~ogóle.

-- Nie sądzę, żebyście musieli się martwić wchłonięciem nas -- powiedział.

-- Tym -- odparł Salasso -- się nie martwimy.

\emph{Och.}

\threeast

-- Ilu tu jest ludzi? -- spytała Elizabeth.

-- Zaurów?

-- Tak. W~Miasto Zaurów Jeden.

Salasso pomachał ramionami. Szedł znowu przed nimi, pośpiesznie,
chętnie. Cienie się zmniejszyły, a~światło i~ciepło stało się bardziej
intensywne.

-- Około miliona. Na oko. Nie liczymy się wzajemnie.

-- Hmm -- powiedziała Elizabeth. Patrzyła na boki, gdy szła, jej usta się
ruszały, jej kciuki poruszały przy czubkach palców.

-- Co robisz? -- spytał szeptem Gregor.

-- Statystyki populacji -- spojrzała na niego z~boku. -- Wydaje się, że
jest dużo dzieciaków\ldots -- Roześmiała się. -- Znaczy, kurczaków? piskląt?
tak czy inaczej\ldots młodych zaurów. Ale jeżeli je rzeczywiście
policzywszy, na hektar, i~jak każdy z~nich otrzymuje prawdziwą uwagę
dorosłych\ldots wygląda na bardzo niski współczynnik reprodukcji. Strategia
wysokiego K/R.

-- Wynika z~długiego życia -- lekko odparł Gregor.

Elizabeth mocno potrząsnęła głową. 

-- Nie! Niekoniecznie. Ograniczają
swoją populację, może utrzymują nieco ponad. Nie wzrost.

-- Podczas gdy, my\ldots -- Uśmiechnęła się krzywo. -- My ludzie. Tak. Ilu masz kuzynów?

Pomachał palcami przed nią. 

-- Musiałbym zdjąć buty, by ich policzyć,
chyba że masz ręczny licznik.

Salasso zatrzymał się i~spojrzał do tyłu. Jego usta rozciągnęły się na
boki.

-- Ona jest tam! -- krzyknął pokazując. Pięćdziesiąt metrów dalej, zaur w~długim niebieskim płaszczu siedział na korkowych krześle przed tuzinem
młodszych i~mniejszych osób, w~cieniu metalowego parasola.

Wtedy Salasso zrobił coś, czego nigdy nie widzieli, żeby zrobił. Pobiegł
do dorosłego zaura, który wstał i~go przytulił.

-- Bogowie na niebie -- powiedział Gregor. -- Ten gnojek ma uczucia.

-- Wiedziałam \emph{o tym} -- odparła Elizabeth.

Coś w~jej tonie sprawiło, że Gregor się do niej odwrócił, ale już
patrzyła w~bok. Salasso zamachał do nich i~dołączyli do niego.

Z bliska ani starszy wiek, ani inna płeć drugiego saura nie były
oczywiste, przynajmniej dla nich. Wygląd był tak gładki jak Salasso,
ciało, nagie, gdzie płaszcz się rozchylał, tak neutralne i~nijakie w~zewnętrznej anatomii jak osobnik męski, którego raz lub dwa widzieli
pływającego.

-- Moi przyjaciele, poznajcie Athranal, moją czcigodną nauczycielkę -
powiedział Salasso.

Przedstawili się.

-- Dzień dobry wam -- powiedziała w~nierównej, grzecznej łacinie. -- Witamy
was tutaj.

-- Jesteśmy zaszczyceni spotkaniem.

Zaśmiała się, świszcząc, po raz pierwszy brzmiąc staro.

-- Na pewno jesteście. Byłam młoda, kiedy ten język był nowy.

Gregor poczuł, że szczypią go włosy na karku. Nagle poczuli się sztywni
jak piórko u dziecka zaur, z~którym się bawili. Wolał myśleć, że nie
zrozumiała, albo, że jej znajomość języka była zapomniana. Pokiwał głową
grzecznie.

-- Szukamy naszych własnych\ldots starszych.

-- Tak mi powiedział mój najlepszy uczeń -- odparła. Położyła rękę na
ramieniu Salasso. -- Salasso, Salasso! Jak dobrze Cię znowu widzieć!

Zwracając się młodszych zaurów, dorastających, jak przypuszczał Gregor,
którzy stali grzecznie z~tyłu, zaczęła deklamację w~mowie zaurów, która
sprawiła, że uczniowie przysiedli na swoich zadach, patrząc na Salasso,
który po chwili przyciskał swoje ramiona i~patrzył w~dół. Czy nauczył
się języka ciała zażenowania od ludzkości? Gregor się zastanawiał.

Po około pięciu minutach, Athranal przestała. Jej uczniowie uderzali w~ziemię stopami.

-- I~tak mówimy do widzenia -- podsumowała, klapiąc Salasso w~plecy i~odwracając się do innych.

-- Do widzenia i~szczęścia -- powiedział Salasso.

Gregor i~Elizabeth wyjąkali to samo pozdrowienie.

-- Musimy iść -- powiedział Salasso.

Prawie ich złapał, prawie pchnął w~ramionach, gdy przeszedł koło nich,
potem pobiegł do podstawy najbliższego szybu bez patrzenia do tyłu. Z~jednym szybkim rzutem oka do tyłu na starą nauczycielkę, teraz spokojnie
wracającą na miejsce, Gregor pobiegł za zaurem, z~dudniącymi krokami
Elizabeth na darni za nim.

Przewrócili się wzajemnie w~windzie. Drzwi się zasunęły. Ich kolana
ugięły, gdy winda się wznosiła, potem zostali szarpnięci w~bok, gdy
przyśpieszała horyzontalnie.

-- Co się dzieje? -- spytała Elizabeth. -- Dlaczego nie poczekałeś na nią,
żeby Ci powiedziała?

-- Powiedziała mi -- odparł Salasso. Jego oddech stał się głęboki i~szybki. -- W~tej\ldots oracji, zawarła informację, gdzie oni są. Stara
załoga. Nie, żeby młodzi mogli zrozumieć, ale zrozumiałem.

-- Jak?

-- Kod. -- Winda minęła jakiś róg i~wystartowała znowu w~górę. -- Coś, co
moglibyście nazwać akrostychem.

-- Czym?

-- Pierwsze litery słów dają słowo -- wyjaśniła Elizabeth.

-- Wiele słów -- powiedział Salasso.

-- Zrobiłeś to wszystko w~swojej głowie? -- spytał Gregor. -- A~ona
musiała\ldots

-- Improwizować od razu. Ona jest mądra.

-- Dlaczego po prostu nam nie powiedziała?

Salasso zasłonił ramieniem czoło, w~ciekawie nieludzkim geście, jak
mucha czyszcząca oczy.

-- Jeżeli dzieci zrozumiałyby, mogłyby powiedzieć rodzicom, a~niektórzy
mogliby pomyśleć, że bogowie nie akceptują.

\emph{Cóż, to wszystko wyjaśnia}, pomyślał Gregor.

-- A~gdzie oni są? -- spytał.

-- W~i~wokół Nowej Lizbony -- powiedział Salasso. -- Na spędzie dinozaurów
i targu mięsnym. Musimy złapać lot. Nie ma kolejnego przez kilka dni, a~wtedy będzie po wszystkim.

Winda trzasnęła i~uderzyła ich, wypluła z~powrotem na okrągły hol.
Przebiegli, ledwie pamiętając, by złapać torby, a~potem na platformę w~sam raz, by zobaczyć znikający sterowiec za najdalszymi wieżami miasta.



\chapter[Orbitalne piekło komuchów]{12 Orbitalne piekło komuchów}



Stacja radośnie poinformowała nas, gdy byliśmy prowadzeni przez jej
zatłoczone, zagracone korytarze, że nie nazywa się już \emph{Marszałek
Titow}. Teraz nazywała się (uwaga, wdech) \emph{Ciemniejsza Noc,
Jaśniejsza Gwiazda}, zdaje się, po jakimś niezrozumiałej tomie biografii
Trockiego. Chciałem zasugerować, że \emph{Zjarany Prorok} byłby lepszy,
ja również trafniejszy, ludzie, których mijaliśmy byli zdecydowanie na
czymś, może na przenikającym acetonie i~oparach alkoholu z~rozpadającego
się biotech, ale się powstrzymałem.

Zacząłem łapać poruszanie się w~mikrograwitacji, przy pomocy
sporadycznych pchnięć od Camili, żeby skorygować błędy. Za ciężkimi
uszczelnionymi drzwiami przestrzeni dokującej, zapachy miejsca były
silniejsze i~bardziej różne. System wentylacji generował dużo białego
szumu, ale wydawało się, że powietrze nie jest świeższe. Rzadko które
źródło światła nie miało swojej kupki hydroponiki. Króliki i~kurczaki,
ssaki lepiej przygotowane do nieważkości niż ptaki, co jest dziwne,
pływały, nurkowały w~przestrzeniach ogrodzone drobną plastikową
siateczką, która zatrzymywała ich odchody, ale nie ich smród. Ludzie w~brudnych kombinezonach lub dziwnych, skąpych wyborach ubrań i~narzędzi
pracowali pod każdym kątem przy każdej możliwie rysie. Gdy spoglądali do
góry lub do dołu, w~poprzek, na nas, wyglądali na zadowolonych z~naszego
widoku, ale wracali bez wstrętu do pracy.

W rogu dwóch korytarzy Camila wykorzystała chwilową kolizję i~splątanie,
by zapytać, bystro pod nosem: 

-- Dlaczego oni nic nie \emph{mówią} do
nas?

Lemieux spojrzał się na nas. 

-- Ponieważ nie wiedzą kto słucha.

-- Tak nam powiedziano -- wymamrotała Camila, gdy mnie odpychała.

Nie wierzyłem ani przez chwilę wyjaśnieniom Lemieux. Ci ludzie nie
wyglądali, jak gdyby byli zmartwieni, że powiedzenie czegoś mogłoby być
ich oskarżeniem przez kogokolwiek, kto pojawił się na szczycie władzy.
Wyglądali jak ludzie, którzy mieli coś lepszego do przemyślenia.

\threeast

Driver wyglądał prawie tak samo, gdy go widziałem w~jego ogłoszeniu, ale
chudszy, z~kilkudniowym zarostem, oczy czerwone i~bezsenne. Ślady na
jego policzkach były podrapane prawie do żywego od swędzenia
nadużywanego biotechu. Obrócił się pomiędzy płaszczyzną pracy a~ścianą,
jakby siedział przy biurku. Siatka przypięta do powierzchni wypełniona
była zmiażdżonymi filtrami bezdymnych i~zmiętymi zestawami szkieł i~słuchawek. Półki przezroczystych kontenerów, stojaki na narzędzia,
sprzęt obliczeniowy i~do inwigilacji zajmowały ściany za nim oraz sufit.
Pocztówki krajobrazów i~wybrzeży, w~tych okolicznościach prawie
pornograficzne, były dekoracją.

Lemieux przykucnął w~górnym rogu nieużywanej przestrzeni, patrząc na nas
pod niepokojącym kątem. Camila i~ja wpięliśmy ramiona w~jakieś siatki
naprzeciw biurka Driver i~rozluźniliśmy się w~powietrzu.

-- Doskonale -- powiedział Driver -- dostałem ostrzeżenie ze Strefy~51, że
latający spodek jest w~drodze. Niby zabawne. Przy tym, nie było
informacji, dlaczego przylatujecie lub kim jesteście. -- Skrzywił się. -
Ogłaszaliśmy wasze przybycie jako pierwszych amerykańskich naukowców,
którzy przyjęli uprzejme zaproszenie Wielkiego Wujka, by do nas
dołączyć. Więc kim właściwie jesteście?

Camila w~tych warunkach prawie stanęła na baczność.

-- Camila Hernandez, pilot doświadczalny dla Nevada Orbital Dynamics. To
jest Matt Cairns, kierownik systemów ze Szkocji, który ma dla was pewne
informacje. To wszystko, co wiem.

Uwaga Colina Drivera przeniosła się na mnie. 

-- Hej, widziałem cię gdzieś
w wiadomościach. Dezerter, prawda?

-- Można tak powiedzieć.

-- Cóż, co to za informacje?

-- Przywiozłem pewne informacje, które ściągnęliście dla ESA -
powiedziałem. -- Było to spakowane w~projektach fabrycznych. Zawiera
kompletną dokumentację produkcyjną, gdy nad tym jeszcze trochę
popracuję.

-- Dokumentacja czego?

Spojrzałem na Camilę i~Lemieux, pomyślałem, do diabła z~tym, to jest
coś, co każdy powinien wiedzieć.

-- Wiecie, dla pojazdu antygrawitacyjnego i~napędu kosmicznego.

Camila odwróciła się i~wgapiła się na mnie.

-- \emph{Czego?}

Lemieux zarechotał ze swojej małpiej grzędy. Driver wcale lepiej nie
ukrył swojego rozbawienia.

-- Pierwszy, słyszałem o~tym -- powiedział. -- Chcesz mi powiedzieć, że to
wyszło \emph{stąd?}?

Przytaknąłem. 

-- Przez system planowania ESA, dokładnie tak.

-- Cóż, nie \emph{ mogło} -- upierał się Driver. -- Wszystko, prawdziwe
info do ESA, dezinfo do, hm, drugiej strony, przechodzi przez moje
biurko. -- Uderzył w~nie, podkreślając dźwięcznie.

-- Może -- zasugerowała Camila -- niektórzy naukowcy przesłali to
niezależnie?

Driver podniósł i~machnął rękoma. 

-- Tak, to możliwe. Nic nie przeszkadza
komuś ustawić gdzieś pirackiego radia. Ale nie ma możliwości, że
zhakowali system planowania, lub moje biurko, prawdę powiedziawszy.

-- Nawet z~obcą matematyką łamiącą kody, którą macie? -- Zaryzykowałem.

-- Ehe -- odparł. -- Ta matematyka pojawiła się tutaj, racja, ale zasoby,
żeby ją zastosować, są znacznie poza tym, co każdy ma tutaj, lub może
mieć dostęp. Muszą wyrosnąć, żeby być szczerym, \emph{lasy} nowej
technologii, żeby stworzyć taką pojemność komputerową na Ziemi. Poza tym
nie opieramy się tylko na szyfrowaniu. Podstawowe zasady
bezpieczeństwa\ldots -- Driver wzruszył ramionami. -- Pewnie je znasz.

Przytaknąłem. 

-- Znasz Alana Armstronga?

-- Nie.

-- Słyszałem o~nim -- powiedział Lemieux.

-- Pogadaj z~nim, kiedyś -- powiedziałem. -- Zadzwoń, jeżeli chcesz. Nie
musisz wchodzić w~szczegóły. Tylko zapytaj go, co myśli o~tym, co
przywiozłem.

Driver grzebał sobie w~uchu. 

-- Dobrze -- powiedział. -- Zobaczmy, co masz.

-- Czy jesteś właściwą osobą, żeby to ocenić? -- spytała Camila.

-- Nie, ale Paul jest. A~jeżeli cokolwiek z~tego rzeczywiście wyszło
stąd, mogę znaleźć dowody w~plikach audytu.

-- To chwilę zabierze -- powiedziałem. -- Jesteśmy wykończeni, potrzebujemy
porządnego jedzenia, snu i~prysznica.

Driver się wściekł. 

-- A~my nie? \emph{Wszyscy} tutaj jesteśmy
wykończeni. -- Przesunął palce po zamkniętych powiekach. -- Och, do
cholery z~tym. Masz rację. Cokolwiek to jest, może poczekać kolejne kilka godzin.
Dobrze, Paul, zabierz ich stąd, a~ja się tutaj trochę zdrzemnę.

Lemieux wysunął się z~kąta przy suficie, przepłynął koło nas i~otworzył
drzwi. 

-- No chodźcie. Pozwólcie mi okazać wam odrobinę gościnności.

\threeast

Kurczak, ziemniaki puree, fasolka szparagowa, wszystko przyczepione do
papierowego talerza lepkim sosem. Do tego plastikowa bulwa do wyciskania
z sokiem pomarańczowym. To był najlepszy posiłek, jaki mógłbym
zapamiętać. Kiedy się najadłem na tyle, żeby znów zacząć myśleć,
zwolniłem i~rozejrzałem się dookoła. Mesa była wąskim pokojem z~jednym
długim aluminiowym stołem, przy którym jedzący mogli orientacyjnie
siedzieć zaczepiając kolana o~poręcz pół metra niżej krawędzi stołu.
Wolna kolejka manewrowała przez właz na dalszym końcu pokoju. Iluzja
bycia w~normalnej grawitacji musiała być ważna dla projektantów i~użytkowników pokoju, nikt nie opuszczał stołu, przelatując nad nim, choć
to nikogo by nie kłopotało.

Ludzie jedli szybko, rozmawiali i~mieli swoje szkła włączone i~zacienione. Rzadko kto nosił gogle w~nowej technologii. W~każdej chwili
było około dwudziestu osób dookoła stołu.

-- Ile jest osób na stacji? -- spytałem.

Lemieux spojrzał znad czepliwego deseru szarlotki z~melasą. 

-- Dwadzieścia jeden po treningu kosmonautów, w~tym dziesięciu oficerów
naukowych i~technicznych, w~tym ja, pięć ochrony, trzech oficerów
łącznikowych, z~tego dwóch obecnie w~areszcie strzeżonym przez
trzeciego, plus piętnastu administratorów cywilnych i~dwieście
siedemdziesięciu dwoje naukowców i~techników.

-- To \emph{dużo} -- powiedziała Camila.

-- Jest dużo do zrobienia -- odparł Lemieux. -- Co powinniście wiedzieć,
jeżeli spojrzeliście na dane naukowe, które opublikowaliśmy.

Camila potrząsnęła głową. 

-- Widziałam jakieś coś w~wiadomościach, to
wszystko.

-- Ja też nie -- przyznałem. -- Byłem zbyt zajęty ucieczką.

-- Cóż -- powiedział Lemieux. -- Mamy kontakt ponad pięć lat. Jest dużo
pracy. Naukowcy -- machnął rękę ku nieświadomej grupce -- są kompletnie
zafascynowani szczególnie, teraz gdy mogą swobodnie badać, dzielić się i~publikować. Było bardzo trudno, kiedy to była wielka tajemnica.

-- Trudno utrzymać tajemnicę -- powiedziałem -- z~tak wieloma osobami tutaj,
a zdaje się z~setką na Ziemi również o~tym wiedzących.

Oboje, Camila i~Lemieux, się roześmiali.

-- Twoja przyjaciółka się śmieje -- powiedział Lemieux. -- Ona ma rację.
Wielkie tajemnice mogą być długo utrzymywane w~tajemnicy przez wielu
ludzi, a~im większy sekret, tym jest to łatwiej. Jak pokazała Strefa~51,
Projekt Manhattan i~nasza Operacja Wyzwolenie.

-- Jacy są obcy? -- spytałem, uśmiechając się, żeby pokazać, że zdaję
sobie sprawę, jak głupie jest moje pytanie.

Lemieux pochylił się do przodu, łokcie na stole, pozwalając przelecieć
widelcowi z~jednej dłoni do drugiej, rzucając i~łapiąc go precyzyjnie
palcem i~kciukiem.

-- Oni są \emph{jak} te mikroorganizmy, które produkują wapienne maty, które
tworzą stromatolity\footnote{formacje skalne złożone z~cienkich lamin węglanu wapnia wytrąconego z~wody morskiej jako efekt
uboczny życia sinic, por. \url{https://pl.wikipedia.org/wiki/Stromatolity} -
przyp.tłum.}. Z~tym że, to co budują, to nie stosy kamienia, ale coś
pomiędzy większym organizmem a~komputerem, mówiąc z~grubsza. Mówiąc
delikatnie, budują quasiorganiczne mechanizmy niesamowitego piękna i~różnorodności. Podstawowa jednostka, budowniczy, jest czymś \emph{jak }
nanobakterią ekstremofilną. Oczywiście to nie jest miejsce świadomości,
nie bardziej niż nasze neurony. Wspólnie jednak, budują coś większego od
siebie -- uśmiechnął się i~dodał -- jak nasz bystry angielski towarzysz
Haldane błędnie powiedział o~mrówkach, są one ,,najmniejszymi
komunistami''.

Camila prychnęła głośno. 

-- Komunizm pasuje \emph{bakteriom} -- powiedziała miażdżąco.

-- Nie sądzę, że to takie zabawne, kiedy zobaczysz, co osiągnęli. Nie są
kolektywnym umysłem jako całość, jest znacznie więcej oddzielnych
umysłów na tej asteroidzie, niż byłoby, powiedzmy, w~ludzkim Imperium
Galaktycznym, gdyby taka rzecz istniała.

-- I~macie z~nimi łączność? -- spytałem, chcąc uniknąć dyskusji
politycznej.

Oczy Lemieux się zwęziły, usta zaciśnięte, jak gdyby w~chwilowym bólu. 

-- Mamy -- odpowiedział. -- To teoretycznie skandal, ale tak jest.

-- Jaki skandal? -- spytałem. -- Że możesz\ldots tłumaczyć? Czy że oni mogą?

-- Jest gorzej -- powiedział Lemieux. Podrapał się w~głowę. -- Nie ma takiej
potrzeby. Znają nasze języki.

-- Bo nauczyli się z~telewizji? -- spytała Camila.

-- \emph{Niemożliwe} -- powiedział Lemieux. Brzmiało to bardziej
zdecydowanie, po francusku. -- Teoretycznie nie jest możliwe, aby
prawdziwy kosmita nauczył się języka z~programów telewizyjnych. Język
nie może być poznany bez\ldots interakcji.

Czułem ograniczony szacunek dla teorii lingwistycznej, szczególnie
przedstawianej z~francuskim akcentem.

-- Jesteś tego pewien? -- spytałem. -- Może, nie wiem, coś w~ich łamaczach
kodu, coś, czego nie jeszcze nie rozumiemy, jakaś struktura
matematyczna, gramatyka głęboka Chomskiego\ldots

Lemieux złapał widelec po obu stronach i~zaczął go wyginać.

-- To jest wyobrażalne -- powiedział. -- Ledwo. Możemy być w~błędzie co do
teorii. To nie jest skandal. Skandalem jest to, że rozumieją języki,
które nigdy nie były transmitowane czy w~telewizji, ale które nie miały
żywych przedstawicieli od początku istnienia pisma. \emph{To} jest
skandal.

Widelec pękł.

\threeast

Umyliśmy się mokrymi gąbkami w~cylindrycznych budkach, przez które było
pompowane powietrze i~woda się unosiła. Wysuszyliśmy się ciepłym
powietrzem w~tych samych boksach. Dostaliśmy wypraną bieliznę i~sztywne,
świeże niebieskie kombinezony robocze oraz miękkie plastikowe buty.
Lemieux zostawił nas razem w~schowku za zasłonką, z~ostrożnymi
przeprosinami, że to jest wszystko, co może zrobić. Ściany tej małej
przestrzeni były zawieszone jak wiele ścian w~tym miejscu na luźnej
sieci. Popatrzyliśmy na siebie i~się roześmialiśmy. Nie goliłem się od
trzech dni, nasze twarze były zarumienione i~podpuchnięte od krwi, która
nie była już pchana w~kierunku stóp. Zaczepiliśmy łokcie i~kostki przez
właściwe otwory w~sieci i~zasnęliśmy oparci o~górne i~dolne przegrody,
pół metra od siebie twarzami ku sobie.

Śniłem. Słowa i~zmartwienia Lemieux połączyły się z~fragmentami obrazów,
które widziałem wewnątrz asteroidy, obcym miastem, komputerem lub
ogrodem, fraktalny, krystalicznymi i~kwietnie organicznym. Spadałem na
to z~dużej wysokości jak w~sterowcu opadającym na oświetlone miasto, jak
widok spadochroniarza na zbliżającą się trawę, a~każdy kwiat to zegar.
Wewnątrz, zielone ludziki wielkości mrówek oglądały telewizję, śmiejąc
się cienkimi wysokimi głosami i~pisały notatki do siebie w~klinowym i~Liniowym B.

Upadłem na coś i~obudziłem się, żeby uświadomić sobie, że przesunąłem
się i~zostałem złapany przez sieci. Camila chrapała z~otwartymi ustami,
kilkanaście centymetrów przed moją twarzą. Szarpnięciami wróciłem na
swoje miejsce i~znów zasnąłem.

I znowu spadłem, tym razem delikatniej, na łóżko. Twarz Jadey,
zakłopotana, taka, gdy ją ostatni raz widziałem, unosiła się nade mną.
Potem się uśmiechnęła, tak jak to zrobiła ostatnim razem, gdy była w~tej
pozycji, i~nasze twarze i~usta się spotkały.

Jakiś czas potem usłyszałem głos mówiący mi, żebym się obudził, i~poczułem, przez kilka długich sekund, szczęśliwy i~przyjemnie, zanim
obudziłem się i~okazało się, że Camila i~mnie udało się wyciągnąć
kończyny z~sieci i~teraz kuliliśmy się i~tuliliśmy się jak przestraszone
małpy, a~ja miałem całkiem oczywistą erekcję, która na nią naciskała.

Camila odłączyła się i~uśmiechnęła się na przyjazny sposób
\emph{jesteśmy przecież dorośli}, odsunęła z~grzechotem zasłonę i~wypchnęła się na zewnątrz. Dziwnie się pochylając, podążyłem za nią.

\threeast

W biurze Drivera zajęliśmy wszyscy miejsca tak jak wcześniej. Driver
wyglądał tylko nieco lepiej.

-- Dobrze -- powiedział. -- Jesteście gotowi pokazać mi, co macie?

Było łatwiej przedstawić materiały Driverowi i~Lemieux niż Amerykanom.
Technologia była kompatybilna, protokoły znajome im obu, te swoiste dla
ESA nawet bardziej im niż mnie. Przeskakiwali po nich, znajdując linki i~ścieżki, która przeoczyłem, przechodząc po nich szybko, z~szybkimi,
tajemniczymi uwagami wymienianymi pomiędzy nimi. Camila wisiała na
krawędzi naszej współdzielonej przestrzeni danych, nie komentując, prócz
okazjonalnego mamrotania ,,O kurde!''.

Wycofaliśmy się i~spojrzeliśmy po sobie, mrugając.

-- Hmmm -- powiedział Driver. Spojrzał na Lemieux, brwi uniesione.

-- Interesujące -- powiedział Lemieux.

Driver bawił się dyskiem danych, potem włożył do swojego biurka. -
Spróbuję sprawdzić logi audytowe.

Znowu założył gogle i~zaciągnął się bezdymnym. Lemieux zrelaksował się,
przyjmując pozycję medytacyjną, myśmy się wiercili.

-- Cholera -- powiedział Driver. Zdarł gogle i~pstryknął niedopałkiem w~worek. -- Cholera -- spojrzał na Lemieux, potem na nas.

-- To tam jest -- powiedział. -- Wszystko opisane. Nie ma możliwości, żebym
to ominął, nie wtedy. Zeszłego roku, kurde, Paul, pamiętasz. Wszystko
dwa razy sprawdzaliśmy. Ty i~ja.

-- Nie sprawdzałem dezinformacji -- powiedział Lemieux. -- Oczywiście.

-- Ta -- powiedział Driver niebezpiecznie spokojnym głosem. -- Ale
wiedziałbym o~dezinfo, ponieważ cokolwiek, co nie było całkowicie
niewinnym, mdłym, ocenzurowanym gównem o~niskotemperaturowej chemii i~tak dalej, \emph{sam kurwa zmyśliłem.} Nie w~całości, pewnie, ale
możesz się założyć, że przeczytałem każdą pieprzoną linię z~prawdziwymi
rzeczami, która tam się pojawiła. I~\emph{nie} wysłałem dezinfo do ESA!
Chłopie, widziałbyś.

-- To -- powiedział Lemieux -- jest niewątpliwe.

Dwaj mężczyźni spojrzeli na siebie. Jakieś niewypowiedziane porozumienie
zostało znowu potwierdzone.

-- Dobrze -- powiedział Driver. -- Więc co \emph{ jest} wątpliwe?

-- Kto i~jak -- powiedział Lemieux. -- Którym z~cywili udało się zhakować
te dane i~jak to zrobili?

-- Nie wierzę w~to ani przez chwilę -- powiedział Driver. -- Nie mają
takich umiejętności, nie mają takich jaj. -- Rozważał to przez chwilę. --
Właściwie -- dodał -- nie mają motywacji. Mam na myśli, każdy kto, by to
znalazł, byłby bardzo gorliwy, żeby nam o~tym powiedzieć. I~dlaczego
wysyłać do ESA, jeżeli nie byli lojalni? Dlaczego nie wysłać tego do
drugiej strony, wiedzieli o~kontrolowanej transmisji idącej na Zachód, a~to byłoby znacznie prostsze, wepchnąć to tam.

Zamknął oczy i~podrapał brwi. 

-- A~może nie. Może potrzebuję więcej snu.

-- Tak, Colin -- powiedział Lemieux z~dziwną czułą, osobistą nutą, której
wcześniej nie słyszałem w~jego głosie. -- Ale masz rację. To nie ma
sensu.

-- Co z~waszym szpiegiem CIA? -- spytała Camila.

Driver odrzucił to ruchem ręki.

-- To była bzdura. Sorry. -- Spojrzał na nas zaciekle. -- Przeprosiłem
Suchanowa za oszczerstwa. Były konieczne.

\emph{ Aha! } pomyślałem. 

-- Więc CIA musiało\ldots

Spojrzał na mnie, rzucił okiem na Camilę. 

-- Zostaw to.

-- Dobrze -- powiedziałem. -- Wszyscy zakładacie, że ktoś na stacji złamał
transmisję i~wrzucił dane o~,,latającym spodku'' do ESA.

-- Cóż, nie -- powiedział Driver. -- Właśnie znalazłem \emph{ dowód}, że
ktoś\ldots -- mrugnął, potem uśmiechnął się do mnie nieprzyjemnie. -- E tam!
No weź.

-- Co jest w~tym niewiarygodnego?

-- W~\emph{czym!} -- spytała Camila, brzmiąc zdenerwowanie.

\emph{Dość dramatu}, pomyślałem, i~powiedziałem płasko: 

-- Strumień
danych został zhakowany przez obcych.

Wszyscy na raz próbowali odpowiedzieć. Driver uderzył w~biurko.

-- Paul? Co o~tym myślisz?

-- To możliwe. To znaczy, to nie jest teoretycznie niemożliwe i~nie po
prostu nieprawdopodobne, podczas gdy naukowcy próbujący czegoś takiego
byliby. Więc, tak, powinniśmy to rozważyć.

Driver siedział w~ciszy przez chwilę.

-- Cholera -- powiedział. -- Jeżeli to zrobili, to jest to coś, co nigdy
wcześniej się nie zdarzyło. -- Podrapał policzek. -- Interwencja.

Słowo zawisło w~powietrzu jak kawałki złamanego widelca Lemieux.

-- A~te informacje, którymi was karmią, \emph{nie są}? -- spytała Camila.

-- Myślę, że naukowcy nie nazwaliby tego ,,karmieniem'' -- powiedział
Driver. -- Obcy są dość wybiórczy co do pytań, na które odpowiadają.

-- Och? A~jak \emph{nazwiesz} dawanie UE poważnej przewagi wywiadowczej?

-- Niekoniecznie wiedzą o~tym -- Driver zaprotestował. Spojrzał na Paula
Lemieux jak gdyby po wsparcie. -- Nie ma dowodu, że obcy rozumieją
politykę Ziemi, nie mówiąc już o~braniu stron w~niej. Ani nie ma dowodu,
że nasza obecność jest rozpatrywana na jakimkolwiek wyższym poziomie w\ldots społeczności obcych. Z~tego, co wiemy, możemy kontaktować się z~niczym więcej jak ich odpowiednikiem \emph{Encyclopedia Britannica}, a~właściwie wersją dla dzieci.

Lemieux kręcił głową. 

-- Wiem, że to jest to, co chciałbyś sobie myśleć,
Colin -- powiedział -- ale wiesz, że każdy naukowiec na tej stacji oraz ja
byśmy to zakwestionowali. -- Poruszył zębami, jak gdyby skubał dolną
wargę. -- Czy to nie czas, żebyśmy wyjaśnili sytuację?

-- Och, chyba tak -- powiedział Driver i~nagle wyglądał bardziej radośnie.
-- Albo to albo wsadzić ich do aresztu, a~tego nie chciałbym robić. Opinia
publiczna mogłaby być zła. -- Uśmiechnął się do nas. -- ,,Amerykańscy
zakładnicy w~czerwonym orbitalnym piekle'' i~tak dalej.

Zaśmialiśmy się z~nerwowej grzeczności.

-- Poczekaj chwilę -- powiedziałem. -- Macie pomysł, wiecie, czy zamierzamy
\emph{zbudować} tę rzecz? Ponieważ to dlatego tu przybyliśmy.

Driver odłączył się od biurka i~sięgnął po drzwi.

-- Najpierw ważne rzeczy -- powiedział. -- Czas poznać obcych.

-- Ma na myśli naukowców -- powiedział Lemieux przy wyjściu. Driver go
usłyszał.

-- Bez różnicy -- chrząknął.

\threeast

Naukowiec zdjął szkła i~mrugnął do nas. W~dosłownie wyluzowanej pozycji,
oraz pozie, wisiał w~poprzek małego korytarza lub długiego schowka,
otoczony przez więcej przewodów niż pacjent na intensywnej terapii.
Niektóre z~nich były światłowodami, inne izolowanymi metalami, większość
miała ten włóknisty, prawie organiczny wygląd Nowej Technologii. Żaden z~nich, na tyle o~ile widziałem, nie był rzeczywiście podłączony do jego
ciała, ale większość z~nich kończyła się w~sprzęcie wiszącym dookoła.
Jego sprane dresy i~wyciągana podkoszulka ledwo zakrywała piwny brzuch,
a jego włosy i~broda wyglądały tak dziko i~pozwijanie jak okablowanie.

Pomachał wszystkim powietrznym, pojęciowym powitaniem.

-- Cześć, chłopaki. Nazywam się Armen Avakian. A~wasze nazwiska są
wszędzie w~intranecie statku. Witamy na pokładzie. Czy nasi dwaj
politicos już was wprowadzili?

-- Zdecydowaliśmy, żebyś Ty to zrobił -- powiedział Driver. -- Włączając do
tego \emph{politykę. } Na razie rozmawialiśmy o~bezpieczeństwie.

-- Więc nie jestem podejrzany, kimkolwiek by nie byli -- powiedział
Avakian. Zaśmiał się w~taki sposób, że chciałem zasłonić uszy. --
Wspaniale! Więc, macie szkła? Tak, prawda, dobrze. Teraz przez chwilę,
gdy będę je synchronizował i~otwierał przestrzeń współdzieloną\ldots

Otoczyła nas pustka, bez cienia, perłowo jasna. Głos Avakian mamrotał,
gdzieś za nami:

-- Gotowi?

To było coś oczywistego i~niepokojącego, że pytanie nie było skierowane
do żadnego z~nas.

Jasna monochromatyczna bańka wybuchła, wrzucając nas w~kolor i~złożoność. Więcej kolorów, więcej złożoności niż kiedykolwiek widziałem,
wyobrażałem sobie lub śniłem. Obrazy pokazywane w~kanałach
informacyjnych były dość niewystarczającym przygotowaniem na realne
zjawiska. Wisieliśmy w~olbrzymiej wewnętrznej przestrzeni. Odległość i~perspektywa były niemożliwe do ocenienia, kształty trudne do zobaczenia
gołym okiem. W~pewnym momencie miało to sens jako wnętrze ogromnie
powiększonego nieludzkiego mózgu, w~następnym, jako widok miasta z~góry,
lub katedry wykonanej w~całości z~witraży, potem znowu jako
wielopłaszczyznowego ogrodu botanicznego, w~którego olbrzymich
szklarniach byliśmy tylko muszkami owocówkami.

Przez długi czas jedyną możliwą odpowiedzią była cisza. Miejsce
wypełniało umysł, oczy i~wewnętrzne oko.

Mój moment oczarowania i~medytacji został zniszczony śmiechem Avakiana.
Widok zmniejszył się, z~powrotem do białego światła, jak obudzenie się
zimną wodą po ciepłym, barwnym śnie.

-- Tyle, jeżeli chodzi o~cały obraz -- powiedział. -- Obecny interfejs ma
nieco mniejszą przepustowość.

Wisiałem w~powietrzu, drżąc, mrugając, żeby powstrzymać łzy za szkłami.

-- Po prostu dobrze -- Avakian kontynuował -- ponieważ interfejs jest
dostatecznie uzależniający. Gdybyśmy pracowali na tej perspektywie, nic
byśmy nie zrobili, tylko patrzyli z~szeroko otwartymi ustami.

Znowu się zaśmiał, dźwięk bardziej maniakalny i~brzydki niż wcześniej, a~nawet gdy mnie uraził i~wzbraniałem się, nie \emph{skuliłem} się od
dźwięku, zrozumiałem, że robił to specjalnie i~w naszym interesie: bez
tego obrazoburstwa stracilibyśmy się w~bałwochwalstwie. Lub gorzej,
nasze uwielbienie tego gwiezdnego miasta mogłoby być najbliższe dla nas
temu, co byłoby kultem prawdziwych bogów.

Ze słyszalnym \emph{klik} pojawił się interfejs, tym razem szeroki
zawijający się ekran zamiast widoku pełnej immersji. Gdybyśmy nie
widzieli tego widoku, bylibyśmy tak samo oszołomieni ekranem,
zatłoczonym nieruchomymi i~poruszającymi się obrazami, głębią i~\emph{tekstem}.

-- To jest to, co można nazwać ,,bogaty w~opcje'' -- powiedział sucho
Avakian. -- To jest, w~czym większość z~nas, naukowców, przebywa, w~każdej chwili, gdy możemy.

Obraz znikł i~staliśmy się znowu czwórką ludzi wiszących w~oczyszczonym
powietrzu śmierdzącej, zamkniętej i~przeładowanej przestrzeni.

-- Więc -- powiedział -- co mogę dla was zrobić?

Miałem zamiar mu powiedzieć, gdy włączył się Lemieux.

-- Nie, nie! -- powiedział. -- Po pierwsze, możesz powiedzieć im, co Ty i~inni naukowcy odnaleźli, i~jaki jest wasz konsensus w~sensie, co musi
być zrobione, polityka, jeżeli można.

-- Aha, tak, to -- powiedział Avakian. Przeczesał rękami swoje elastyczne
włosy, nie wprowadzając żadnych dostrzegalnych zmian. -- Cóż, to miejsce
tutaj jest unikatem, racja, ale nie jest \emph{jedyne.} Wiecie, kiedy
Wielki Wujek ogłosił? To był lekkie niedopowiedzenie, towarzysze i~przyjaciele. Stary Jefrimowicz był odrobinę oszczędny.

Jego palce znowu przeleciały przez włosy.

-- Prawda jest taka, że tych gnojków są \emph{miliardy}. W~pasie asteroid,
Kuipera i~Oorta. Jest więcej\ldots społeczności\ldots jak ta dookoła Układu
Słonecznego niż ludzi na Ziemi. A~każda z~nich zawiera więcej
oddzielnych umysłów niż, niż\ldots

-- Imperium Galaktyczne -- dopowiedział Lemieux.

-- Tak! Tak! Dokładnie! -- Avakian rozpromienił się.

-- Skąd o~tym wiecie? -- spytała Camila.

Avakian machnął ręką wskazując za ramię.

-- Obcy nam powiedzieli i~powiedzieli nam, jak znaleźć ich łączność. Ich
emisje elektromagnetyczne są bardzo nikłe, ale istnieją, dobra, a~źródła
wypełniają niebo jak mikrofalowe promieniowanie tła, echo Wielkiego
Wybuchu.

-- Na pewno to nie jest po prostu część tego?

-- E tam, to komunikacja, dobra. -- Avakian possał dolną wargę. -- O czym
należy pamiętać to, że nasza zewnętrzna powłoka chmury kometarnej
przecina się z~tą z~systemu Centauri, i~cóż\ldots

-- Są wszędzie?

Wzruszył ramionami. 

-- Dookoła wielu gwiazd, ta, całkiem możliwe.
Handlując, rozmawiając, może nawet podróżując. Kontrolują świadomie
własne odgazowania, mają niewyobrażalną moc obliczeniową, a~zmiana orbity
wymaga tych szturchnięć. Pomiędzy gwiazdami może zabrać miliony lat,
pewnie, ale ci gościa mają \emph{długi} okres koncentracji uwagi.

-- A~co właściwie robią?

-- Z~naszego punktu widzenia, małych zajętych naczelnych, nie robią dużo.
Spędzają wolny czas i~zmieniają widoki. Podróżują dookoła słońca co
kilka milionów lat. Może polecą do innego słońca i~okrążą tamto kilka
razy. Nuuuda. -- powiedział dziecięcym, marudzącym głosem. -- ,,Czy
\emph{już} dolecieliśmy? On mnie \emph{uderza}. Chcę do \emph{toalety}.''

Zaśmiał się, tym razem prawdziwy, zabawny śmiech, i~kontynuował
energicznie: 

-- Ale z~ich punktu widzenia, oni się świetnie bawią.
Niekończąca, absorbująca, ekstatyczna i~z tego, co wiem,
\emph{orgazmiczna} zabawa. Dyskurs, stosunek, na ich poziomie to pewnie
ta sama pieprzona rzecz. -- Podkreślił oczywiste z~chichotem. -- Są jak
bogowie, człowieku, i~są dosłownie w~niebie. A~w~całej ich
nieskończoności, cóż, dobrze \emph{nieograniczonej}, różnorodności,
rozumiemy, całkiem jednomyślne zdanie o~jednej rzeczy. Nie lubią spamu.

Spojrzał w~trzy zaintrygowane twarze i~na moją.

-- Spam -- powiedział. -- Powiedz tym dobrym ludziom o~spamie.

Zamachałem, dosłownie i~metaforycznie. To był niezrozumiały problem,
trudny do wyjaśnienia ludziom spoza biznesu. Ale spróbowałem.

-- Spam to, hm, rodzaj bezmyślnie powtarzanych reklam i~gówna. Śmieci.
Część pochodzi z~oszustw i~startupów, część z~programów nazywanych
spambotami, które uwolniły się jakieś pięćdziesiąt lat temu i~które od
tamtego czasu bobrują. Nie zauważacie, ponieważ bardzo mało się
przedostaje, że moglibyście myśleć, że to legalna reklama. Ale to
dlatego, że na samym dole, mamy programy, które usuwają te śmieci, a~one
też nad tym pracują. -- Wzruszyłem ramionami. -- Spam i~antyspam marnuje
zasoby, to ostatecznie gra o~zerowej sumie, ale co można zrobić? Trzeba
z tym żyć. Antyspam jest jak system odpornościowy. Nie musisz o~nim
wiedzieć, ale bez niego byś umarł. Toczy się cała wojna, które jest
kompletnie nieistotna dla tego, co naprawdę chcecie zrobić.

-- \emph{Exactamundo } -- powiedział Avakian. -- Podobnie
ET\footnote{ang. extra-terrestrial pozaziemskie - przyp.
tłum.} traktują te sprawy. A~na tyle o~ile są zainteresowani, jesteśmy
wielkimi, niezdarnymi spambotami, zepsutymi serwerami, skłonnymi w~każdym momencie, albo w~ciągu miliona lat zamienić się w~miliony
bezsensownych, nieco zmienionych replik samych siebie. Większość z~tego,
co chcielibyśmy zrobić, jeżeli poważnie zaczniemy kolonizację kosmosu,
to spam. Przemysł kosmiczny to spam. Uploady Moraveca to spam na
talerzu\footnote{procedura przeniesienia mózgu do
symulacji komputerowej przy pomocy robota badającego kolejne warstwy
komórek nerwowych, więcej
\url{https://www.ibiblio.org/jstrout/uploading/moravec.html}}. Maszyny Von
Neumanna\footnote{prawdopodobnie autor ma na myśli
samoreplikujące się mechanizmy, zob.~\url{https://en.wikipedia.org/wiki/Self-replicating\_machine\#Von\_Neumann's\_kinematic\_model}},
spam i~frytki. Kolonie kosmiczne, spam, spam, spam, jaja, spam.

-- A co z~kopalniami na asteroidach i~wykorzystaniem komet? -- Nie udało mi
się zachować poważnej miny, ale Avakian wyglądał ponuro.

-- Nawet o~tym nie \emph{myśl} -- odpowiedział. -- Hmm\ldots Odpowiedź na
pytanie, dlaczego to stało się sprawą polityczną, nie zajęło nam dużo
czasu, gdy zrozumieliśmy, że ostatecznym silnikiem spamu jest
kapitalizm. Nieskończona ekspansja jest mokrym snem wielkich
kapitalistów, a~to jest kompletnie niekompatybilne z~tym jaki naprawdę
jest kosmos. Zdecydowanie jest niekompatybilny z~tym, co przeważająco
dominująca forma inteligentnego życia w~kosmosie może chcieć
zaakceptować. Całkiem szczerze, nie jestem partyjniakiem, ale faktem
jest, że kierunek Partii ku społeczeństwu stanu ustalonego z~odrobiną
zrównoważonej, ostrożnej nieinwazyjnej eksploracji kosmosu jest jedynym
rodzajem społeczeństwa, które obcy są w~stanie zaakceptować. -- Avakian
zrobił ironicznie smutną minę ku Camili. -- Sen, który wy macie, o~traktowaniu Systemu Słonecznego jako surowego materiału na orbitalne
domy, broń i~puszki piwa jest \emph{nieaktualny}.

-- A~co -- spytała Camila -- oni mogą z~tym oni zrobić?

Grube brwi Avakian drgnęły. 

-- Kontrolując orbity komet i~asteroid, mogą,
hm, zaprojektować masowe wymieranie. -- Rozłożył ręce. -- Tylko pomysł.

-- Poczekaj chwilunię! -- zawołała Camila. -- Z~takim zagrożeniem z~kosmosu,
cholera, moglibyśmy wszyscy zacząć współpracować. Byłoby prawdziwe
wsparcie dla wielkiego sprzętu, laserów, jądrówek, stacji bojowych,
wreszcie prawdziwego systemu obrony kosmicznej. Hej, moglibyśmy nawet
wciągnąć komuchów, gdy zrozumieją, przeciwko czemu występujemy. A~z~taką
mocą polityczną, moglibyśmy naprawdę szybko załatwić ciężki sprzęt na
orbicie! Te twoje kosmity nie zdążyłyby zareagować, żeby nas zatrzymać. A
gdyby spróbowali, zobaczyliby kilka małych wymierań nadlatujących w~\emph{ich} stronę. Cholera, te gostki wybrali zły gatunek, żeby zamknąć
go w~beczce.

Avakian spojrzał na mnie, Lemieux i~Drivera, słuchających tej małej
tyrady z~niedowierzaniem i~rozrywką.

-- Ach -- powiedział. -- Zaczynasz rozumieć problem.

-- Nie traktuj \emph{mnie} protekcjonalnie -- powiedziała Camila, wsuwając
twarz przed niego i~zmuszając do spojrzenia na nią. -- Zresztą nieważne,
jeżeli obcy nie chcą nas, byśmy polecieli w~kosmos, to dlaczego do
cholery dali nam plany latającego spodka i~napędu gwiezdnego?

Avakian powoli mrugnął. 

-- Powiedzcie mi więcej -- powiedział.

\chapter[Skif grawitacyjny]{13 Skif grawitacyjny}





Gregor patrzył za odlatującym sterowcem, jego pięści zaciśnięte w~gwałtownej frustracji i~irytacji. Odwrócił się do Elizabeth i~Salasso,
którzy jak on wychylali się niebezpiecznie nad balustradą platformy, jak
gdyby to miało pomóc.

-- Nie może poprosić wieży, żeby ich zawrócili?

-- Nie ma szans.

-- Więc dlaczego do cholery nie\ldots

-- Posłuchaj -- powiedział cierpliwie Salasso -- nie spodziewałem się, że
ludzie, których szukamy, będą w~Nowej Lizbonie. Ostatnim razem, gdy
słyszałem, plotki mówiły o~innym miejscu. -- Machnął ręką na inne
sterowce dryfujące w~doku. -- Gdzieś dostępnym przez jeden z~tamtych
lotów. Tylko oczekiwałem potwierdzenia od mojej starej nauczycielki, a~także lepszych szczegółów co do lokalizacji. Gdybym przez moment uważał,
że mogą być w~Nowej Lizbonie, nie spacerowalibyśmy.

Ich przechadzka po mieście, spotkanie z~dzieckiem zaurów wyglądała teraz
jak bezcelowa strata czasu, do wspominania z~żalem, zamiast z~zachwytem.
W tej samej chwili, wspomnienie o~Nowej Lizbonie spowodowało skok serca
Gregora. Lydia miała tam być.

-- Chwila -- powiedziała Elizabeth. -- Powiedziałeś Nowa Lizbona, gdzie są
teraz skify grawitacyjne? Nie mogą pracować \emph{cały} czas. Czemu by
nie zadzwonić i~nie zobaczyć, czy jakiś pilot nie chciałby skoczyć tutaj
i z~powrotem z~nami?

-- To jest\ldots -- Język zaura drgał pomiędzy ustami. -- To jest bardzo dobry
pomysł. Powinienem sam o~nim pomyśleć.

Gregor upuścił torbę i~objął ramieniem Elizabeth.

-- Tak! -- krzyknął. -- Cudownie!

-- Przewidujesz -- powiedział Salasso. -- Zobaczmy.

Poszli za nim do okrągłego pokoju. Salasso podszedł do płaskiej, szarej
płyty na ścianie i~zaczął wciskać małe prostokąty naszkicowane wzdłuż
dolnej krawędzi. Po chwili, płyta zaświeciła się słabo, a~Salasso zaczął
żywą konwersację gestykulując. Gregor patrzył z~boku z~poczuciem
niejasnego resentymentu. Telewizja lub jej brak, było pewnym bolesnym
punktem.

Ekrany zaurów pracowało poza zakresem widzenia ludzi, a~nawet po
korekcie, nie miały żadnego sensu, większość obrazu zgubiona w~śnieżycy
dodatkowych informacji, które system optyczny zaurów filtrował odmiennie
od ludzkiego mózgu. Możliwości przemysłowe Mingulay uniemożliwiały
masową produkcję monitorów kineskopowych, nie mówiąc już o~czymkolwiek
bardziej zaawansowanym, i~ten niedostatek był tym, którego zaurowie nie
śpieszyli się zmniejszyć zaopatrzeniem z~ich zakładu produkcyjnego.

To, co sprawiło, że ta sprawa bolała, było wyraźne wrażenie ich kupców,
że ich decyzja jest całkowicie korzystna dla ludzi.

Słabe światło zgasło. Salasso odwrócił się.

-- Zaplanowałem to -- powiedział. -- Łódź będzie tutaj za godzinę.

\threeast

Automatyczne szklane drzwi \emph{otworzyły się} gdy szli do wejścia.
Patrząc raczej ostrożnie z~jednej strony na drugą, Gregor i~Elizabeth
przeszli. Salasso wahał się przez moment, drzwi zaczęły się zamykać,
potem otworzyły się, gdy skoczył przez nie. Spojrzał na nie
podejrzliwie.

Pomieszczenie było całkiem duże, z~ladą wzdłuż jednej ze ścian, tubami
fluorescencyjnymi zawieszonymi z~wysokiego sufitu. Monitory kineskopowe
wyświetlające informacje o~lotach były zamontowane na ścianach tak jak
długie pionowe głośniki grające niewyraźną i~wyróżnialną muzykę.
Urozmaicone siedzenia plastikowe i~laminowane plastikowe stoły były
rozrzucone i~zgrupowane na przestrzennej podłodze, część z~polerowanego
drewna, reszta z~wykładziny. Kilku biznesmenów siedziało, niektórzy
ciągle negocjujący z~zaurami, inni sączący drinki i~patrzący bez wyrazu.

-- To jest dziwaczne -- powiedział Gregor. -- Jak coś z~przestrzeni
kosmicznej.

Salasso wspiął się na krzesło przy stoliku. 

-- W~każdym razie, luksusowe
-- powiedział, machając nogami. -- Znacznie lepsze niż nasze udogodnienia.

-- Mówiąc o~udogodnieniach\ldots -- powiedziała Elizabeth.

Salasso wskazał. Elizabeth spojrzała na znak i~potrząsnęła głową.

-- To dobrze, że nie noszę spodni. \ldots Przy okazji, jesteście głodni?

Dziewczyna przy ladzie z~napojami była ubrana w~różowobiałą sukienkę w~pasy, która wydawało się, że nie pasuje, i~fartuszek, który wyglądał
ładniej i~drożej niż sukienka, którą miał chronić, oraz opaskę z~podobnymi falbankami, która nie chroniła jej oczu przed włosami. Gregor
zmartwił się niejasnym zakłopotaniem tym wszystkim, krępującym uczuciem,
że wszystko tutaj było kopią czegoś, co samo nie było oryginalne i~w
ogóle nie do końca właściwe. Zapłacił za kawy, kanapki i~bulion rybny
monetami z~Kyohvic, niechętnie odebrał resztę garścią lir Nowej Lizbony
z otworami w~środku, i~wrócił z~tacą.

-- Nie wiedziałem, że tutaj pracują ludzie -- zauważył. -- Musi być nudno
jak nie wiem co.

Usta Salasso drgnęły.

-- Mają tendencję do krótkiego pobytu.

-- Hmm -- powiedziała Elizabeth. -- Uważam to za interesujące. Fascynujące.
Praca w~Mieście Zaurów!

-- Nowość się przejada -- odparł Salasso.

Gdy Gregor pił kawę, uderzenie stymulantu sprawiło, że uświadomił sobie
jasno delikatne trzepotanie w~żołądku. Zrozumiał, że w~rzeczywistości
był zdenerwowany i~podniecony perspektywą podróży w~łodzi grawitacyjnej.
Właściwie, bardziej zdenerwowany niż podniecony.

-- Salasso -- powiedział. -- Nie miałbyś nic przeciwko podzieleniu się
fajką?

-- Proszę uprzejmie.

Elizabeth wypiła kawę i~wstała, zabierając na wpół zjedzoną kanapkę ze
sobą.

-- Wybaczcie mi, chłopaki -- powiedziała. -- Wolałabym na razie nie
uczestniczyć. Pójdę i~pooglądam sterowce.

-- Dobrze -- powiedział Gregor, zezując na nią spoza ognia zapalniczki. -
Do zobaczenia.

Elizabeth ruszyła energicznie.

-- Co \emph{ją} gryzie? -- spytał Gregor.

Salasso wzruszył ramionami. Człowiek i~zaur palili w~towarzyskiej ciszy
przez chwilę, zaur biorąc znacznie mniej niż połowę zaciągnięć. Salasso
zaakceptował ostateczne zaciągnięcie. Potem odłożył fajkę i~spojrzał się
wprost na Gregora.

-- Może nie powinienem mówić tego -- powiedział ledwo słyszalnym głosem,
pochylając się do przodu. -- Zioło relaksuje i~pokazuje emocje. Elizabeth
życzy sobie trzymać na wodzy, co jej emocje wyrażają.

-- Och. Rozumiem. -- Gregor zmarszczył brwi. -- Martwi się o~coś?

Niepokojące możliwości pojawiły się w~myślach, chory rodzic, ranne
rodzeństwo, dług studencki, jakieś własne problemy medyczne\ldots

-- Czy to coś, w~czym mogę pomóc?

Oczywiście ona nie chciałaby \emph{prosić}. Jej własna godność by nie
pozwoliła.

Drzwi otworzyły się, gdy kolejna grupa pasażerów weszła. Salasso bawił
się fajką, wystukując popiół do ozdobnego szklanego talerza na stole,
zanim Gregor mógł mu przypomnieć, żeby użył podłogi.

-- Nie wiem, czy mógłbyś pomóc -- odpowiedział. -- Ale\ldots

Zamknął oczy na sekundę, a~potem spojrzał na Gregora.

-- Musisz to wiedzieć. Elizabeth kocha cię i~jest nieszczęśliwa, że
kochasz kogoś innego.

Gregor poczuł chłód idący przez brzuch niczym zimne ostrze. Niewyraźna
chmura zioła natychmiast się rozproszyła, pozwalając wszystkiemu innemu
wpaść. Nigdy nie czuł się tak zaskoczony, tak zażenowany, tak w~złym
miejscu. A~w~tej samej chwili tak boleśnie zadowolony i~tak spełniony,
gdy wszystko, co Elizabeth powiedziała lub zrobiła w~jego obecności,
nabrało prawdziwego, a~teraz oczywistego, znaczenia.

To osłupienie pojawiło się w~jego głosie.

-- Och, bogowie nad nami. Nie wiedziałem.

-- Przepraszam, że musiałem ci powiedzieć -- powiedział Salasso
dwuznaczność słów grzeczna jak zawsze. -- Ale ta wiedza może być ważna w~naszej wyprawie. Byłoby dobrze, gdybyś wziął pod uwagę jak się czuje i~gdybyś dopilnował nie dać jej żadnej możliwości, w~niebezpiecznej
sytuacji, do lekkomyślności na Twoją korzyść. -- W~jego wyrażeniu pojawił
się humor. -- Lub przeciwko Tobie, jeżeli do tego dojdzie.

-- Och, bogowie, tak -- powiedział Gregor. Zabrzmiało to jak jęk. -- Myślę,
że potrzebuję kolejnej kawy, żeby się zebrać w~sobie.

-- Przynieś dwie -- powiedział Salasso, wyglądając przez szklane drzwi. --
Elizabeth wraca.

Przyniesienie napoi posłużyło jako małe odwrócenie uwagi dla umysłu
Gregora i~umożliwiło mu okazanie Elizabeth pewnego spokoju ducha, gdy
wrócił do stołu.

-- Och, dzięki -- powiedziała.

-- Jak Ci się podobały sterowce? A~może łódź przyleciała?

-- Mamy jeszcze około dziesięciu minut -- powiedział Salasso.

-- Nie oglądałam dużo sterowców -- powiedziała Elizabeth. -- Miasto jest
tak właściwie bardziej interesujące. Więcej się dzieje, cały czas. I~ciągle obierałam złą perspektywę, to jest\ldots

-- Fraktalne?

-- Tak. 

-- Jak fale, gdy jesteśmy nad morzem. Nie umiesz powiedzieć, jak
wysoko jesteś, tylko patrząc na nie.

-- Ja umiem -- powiedział Salasso. Przewrócił oczami z~boku na boku i~wszyscy się roześmiali, wypili napoje i~wyszli poczekać na łódź.

\threeast

Salasso zauważył ją pierwszy, wskazał na południe, do góry i~śledził.
Gregor nic nie widział, prócz niebieskiego nieba przez pół minuty, a~potem zauważył punkcik światła śpieszący się do zenitu, gdzie się
zatrzymał. Srebrna plamka powiększała się nad nimi, póki nie stało się
oczywiste, że to opadający dysk. Inni ludzie zaczęli patrzeć i~pokazywać, zbierając się w~mały podniecony tłum. Trzysta metrów nad
nimi, dysk przeszedł w~brawurowy pokaz ruchu opadającego liścia, w~końcu
pikując dookoła platformy, póki nie zatrzymał się kilka metrów nad nią i~metr od poręczy bezpieczeństwa. Z~bliska, srebrna powierzchnia była
poplamiona i~spryskana brązowym błotem, który wyglądał jak suchy muł,
odchody i~krew. Kawałki odpadły, gdy właz się otworzył i~pochylnia
przeciągnęła się do platformy.

Salasso podniósł swoją walizkę.

-- Nie pozwólmy im czekać.

Gregor pokazał gestem Elizabeth. 

-- Wezmę Twoją torbę. 

-- Och, dzięki!

Weszła po drabinie raczej wspaniale, podnosząc rąbek spódnicy, marszcząc
nos na słaby smrodek farmy. Gregor szedł za nią, ściskając ich torby i~starając się nie patrzeć w~dół, gdy pochylnia minęła krawędź platformy.
Gdy tylko wszedł do środka, schody się zwinęły, a~właz się zamknął.

Wnętrze było w~pozornym świetle dzienny, z~okna, które biegło dookoła i~które było niewidzialne z~zewnątrz, ekran jak się domyślał. Kołowe
siedzenie otaczało centralne oprofilowanie, na dalekim końcu siedzenia
siedział zaur naprzeciwko panelu, patrząc na nich ponad ramieniem, jego
dłonie spoczywające na pochyłym panelu poniżej ekranu-okna.

Wymienił powitania z~Salasso i~powiedział po angielsku 

-- Cześć,
siadajcie. Nie ma znaczenie, gdzie siadacie. Lub gdzie stoicie, jeżeli
wolicie.

Pod wpływem impulsu uniknięcia bycia rzuconym przez przyśpieszenia,
Elizabeth, a~potem Gregor usiedli obok Salasso, który zajął miejsce koło
pilota.

-- Włączyłem widok do waszej wizji -- powiedział pilot. -- Mniemam, że
dobrałem dobrze kolory.

Gregor się rozejrzał. Tam, gdzie ekran owijał się za nim, mógł zobaczyć
ludzi machających z~platformy, cofających się.

-- Wygląda doskonale -- powiedział.

-- Dobrze -- powiedział pilot.

Odwrócił się do przodu, a~jego palce zafalowały na panelu. Widok
przechylił się od szczytów wież miasta do nieba i~chmur. Przez chwilę
wyglądały nieruchomo, potem chmury zaczęły stawać się widocznie większe.
Gregor spojrzał do tyłu i~zobaczył miasto przechylone pod szalonym
kątem, malejące za nim. Nie było absolutnie żadnego poczucia ruchu lub
że pojazd był w~innej pozycji niż horyzontalna. Poczuł, że Elizabeth
chwyta jego ramię, gdy oparła się o~niego. Wystrzelili przez chmury w~białym migotaniu, a~potem widok przechylił się znowu, pokazując tylko
bardzo ciemnoniebieskie niebo.

Ciągle nie mogąc zmusić swoje odruchy, by uwierzyć w~to, co widzi,
Gregor wstał. Elizabeth, wisząc na jego ramieniu, również wstała.
Patrząc w~dół z~górnej części ekranu mogli zobaczyć ziemię, lub raczej
powierzchnię planety, horyzont dostrzegalnie zakrzywiający się w~każdą
stronę.

-- Och, bogowie nad nami! -- sapnęła Elizabeth, puszczając jego ramię i~opierając się na plastycznym materiale parapetu ekranu, patrząc w~dół, a~potem przekrzywiając głowę na bok i~patrząc do góry. -- Możesz zobaczyć
\emph{ gwiazdy! } Jesteśmy praktycznie w~\emph{kosmosie!}

-- Witamy w~stratosferze -- powiedział pilot. Odchylił się do tyłu i~zabrał ręce z~panelu, odkrywając coś, co wyglądało jak płytkie odciski
jego dłoni. -- Też przyjemna przerwa dla mnie, muszę powiedzieć. Unikałem
gówna zauropodów tygodniami.

-- Nie zawsze unikając -- powiedziała Elizabeth.

-- Nic, czego dobry tropikalny deszcz nie zmyje. To na ich ogony musisz
uważać, nawet w~tej skrzynce.

Założył ręce za głowę i~wyprostował nogi. Gregor domyślił się, że, tak
jak Salasso, pilot już jakiś czas miał do czynienia z~ludźmi.

-- Przede wszystkim, dlaczego musisz podlatywać blisko ich ogonów? -
spytał Gregor, wracając na miejsce.

Pilot się roześmiał. 

-- Może zrobię wam pokaz, po drodze do Nowej
Lizbony. Nie martwcie się, nigdy nie straciłem skifa.

Na powierzchni dookoła odcisków dłoni, pojawiły się małe kwadraty
słabego światła, migocząc na przemian. Pilot pochylił się i~zbadał
szczegółowo.

-- Och wspaniale, przed nami burza Coriolisa. Czas na myjnię.

Znowu położył ręce na panelu i~widok opadł do powierzchni niebieskiego
oceanu i, nieco przed nimi, wirująca masa chmur. W~ciągu sekund statek
pogrążył się w~nich, wylatując z~nurkowania na lot poziomy. Ciemność
biczującego deszczu, kawałek czystego nieba, kolejna ciemność i~deszcze,
potem wylot z~drugiej strony i~znowu do stratosfery.

-- Jeżeli zrobisz coś takiego raz jeszcze -- powiedziała Elizabeth -
będziesz musiał posprzątać w~\emph{środku}.

-- Jeżeli będę musiał tak zrobić, wpierw poproszę was o~zamknięcie oczu -
odparował pilot. -- To dysonans pomiędzy okiem a~wewnętrznym uchem,
który\ldots

Salasso syknął coś i~pilot się zamknął.

\threeast

Gregor, teraz gdy wszyscy zrozumieli, że żaden ruch statku nie rzuci
nimi, z~uwagą usiadł z~dala od Elizabeth. Przesuwał się po siedzeniu, aż
patrzył bezpośrednio za, na północ i~wschód. Oparcie i~lekko wibrujący,
cicho szumiący, lekko świecący ścięty stożek silnika były pomiędzy nim,
a Elizabeth, znacznie bardziej niebezpiecznym ciałem. Huragan, typowy
dla pasma oceanu równikowego pomiędzy Mainland a~Southland, ustąpił za
horyzont.

Gdy lot z~kolei zmienił się z~magicznego na znajomy, to, co Salasso mu
powiedział, wróciło do niego w~całej swojej nowości i~sile. Od razu
poczuł się zaszczycony i~zbulwersowany. Gdy przypominał i~ponownie
oceniał ich trzyletnią znajomość i~przyjaźń, żałował szczerze, że
Elizabeth nie okazała swoich uczuć już na początku. Nie wiedział, czy by
je odwzajemnił, ale przynajmniej sprawy byłyby zdecydowane. Nigdy nie
myślał o~niej w~kontekście seksualnym, oprócz podświadomej aprobaty jej
jako przystojnej, zdrowej kobiety, ledwie bardziej erotycznej niż ten
rodzaj zachwytu, który czuł dla człowieka ładnego, inteligentnego, w~dobrej formie. Możliwe było, jak myślał, że w~czasie ich pracy, relacja,
niekiedy bliska i~czasem fizyczna, kiedy pocili się razem na łodzi,
podświadomie zamaskował takie myśli jako nie bardziej właściwe dla
kolegi naukowca niż dla kolegi żołnierza.

Teraz, przebijająca się do świadomości, wiedza, że kochała go, była sama
w sobie wystarczająca, by nagle ona stała się niezwykle atrakcyjna i~podniecająca, w~sposób, który był równocześnie naturalny i~przewrotny.

Lydia ciągle błyszczała w~jego głowie w~sposób, który sprawiał, że czuł
się winny, nawet myśląc o~Elizabeth\ldots a~jednak Lydia odwróciła się od
rzucenia się w~los razem z~nim. Zamiast tego stawiając warunek
spełnienia przesadnego wymogu jej ojca. Jego własne oczekiwanie wobec
niej, wiedział, było może bardziej wygórowane. Pozostawał fakt, że
Elizabeth wybrała jego towarzystwo, a~Lydia nie.

Marzył, aby Elizabeth, a~nie zaur, powiedziała mu o~swoich uczuciach,
ale nie mógł za to winić Salasso. Myśl o~tym, jakie nieszczęścia i~gafy,
nawet niebezpieczeństwa, mogła sprowadzić jego nieświadomość sytuacji,
sprawiła, że poczuł zimny pot.

Morze za nim zmieniło się w~długie białe plaże północnego wybrzeża
Southland, a~potem szeroki skraj zakładów produkcyjnych, które z~kolei
niezauważalnie scalił się z~naturalnym lasem deszczowym. Gregor wstał i~przesunął się znowu do przodu. Kiedy Elizabeth spojrzała, gdy siadał,
odwzajemnił jej uśmiech delikatniej, bardziej badawczo, niż może
zamierzał.

\threeast

Znowu krajobraz poniżej przechylił się do góry, prawie wypełniając ekran
i runęli w~dół, gdzie las deszczowy przechodził w~sawannę. Przerywane
tylko przez odkrywki i~góry, morze traw rozciągało się szeroko aż do
wiecznej zmarzliny poniżej pokrywy lodowej bieguna. Olbrzymie stada
gigantycznych bestii przemierzały prerię. Z~tej wysokości, wyglądały jak
plamy na ziemi, nieregularne łaty wielkości powiatów. Bez żadnych
zmartwień, prócz stad drapieżników i~hord pasożytów, które prześladowały
je w~każdej chwili, dinozaury północnych krańców nie zwracały na
wysokolecącą łódź większej uwagi niż na muchę.

Pewnego dnia, może nauczyłyby się innej odpowiedzi.

Radośnie krzycząc, pilot znowu skierował łódź do góry, przeskakując
ponad pasmem dziesięciotysięczników, jak spodek posłany wirem nad wodą.
I znowu w~dół, na południową równinę.

Tutaj, stada nie były plamami, ale rzekami, płynąc na północ w~ich
dorocznej jesiennej migracji przed nadchodzącymi śniegami ku
zbliżającym się deszczom i~bujnemu wzrostowi. Większość strumieni
kierowała się w~kierunku przejść i~przełęczy pasma gór. Inne były
skierowane na ścieżkę, która zabierze je równolegle ku południu od gór,
w kierunku zachodniego wybrzeża.

Głos zaura, różny od tych dwóch w~statku, zabrzmiał w~powietrzu. Pilot
wydał długi niski dźwięk w~odpowiedzi, wzniósł znowu statek i~posłał go
jeszcze dalej na południe, potem w~dół w~kolejnym pikowaniu, póki nie
ślizgał się na wysokości stu metrów. Gregor poczuł, że jego palce
wbijają się w~parapet ekranu, ruch do przodu na tej wysokości był
przerażający, trawa pod nimi rozmazała się w~zielony-brąz.

Plama na horyzoncie okazała się po sekundach wielką falą zwierząt
poruszającą się w~chmurze kurzu i~mgle insektów i~nietoperzy. Statek
poleciał prosto na stado, a~potem, pięćset metrów przed jego prowadzącą
krawędzią, zatrzymał się.

I opadł, aż wisiał metr nad trawą. Stado nadchodziło jak kroczący las,
poruszające się szyje dorosłych sięgały piętnaście metrów w~górę. Gregor
widział, że ziemia przed nimi rzeczywiście drga, cząstki pyłu skaczące
ponad twarde łodygi szorstkiej trawy. Mógł zobaczyć cętkowane wzory skór
dinozaurów, brązowe i~zielone w~większej części, ich spody żółte i~białe. Młodsze i~mniejsze zwierzęta sprawiały wrażenie prawie
tańczących, uskakując dookoła wielkich nóg ich starszych. A~pomiędzy
nimi rzucające się, ciemne ślizgające cienie odważniejszych
drapieżników. Kiedy zwarta grupa przodowników była kilka kroków, kilka
sekund dalej, pilot wzniósł statek przeciwko nim. Głowa, która wyglądała
na większą niż sam statek, pojawiła się na ekranie, posyłając Gregora i~Elizabeth w~daremne, instynktowne szarpnięcie do tyłu.

W ostatniej chwili, pilot zboczył na lewo, uniknął ich i~znowu zbliżył
się do liderów stada, tym razem z~boku. Zanurkowali prosto w~duże
przewracające oko, potem w~górę i~ponad wymachem biernie podrzucanej
głowy. Przelecieli do góry i~dalej, znacznie bardziej na lewo, na
wschód, i~minęli tę krawędź stada, bestie spłoszone i~wycofujące się,
gówno latające w~parujących tonach, gdy szarpnęły się na boki, uderzając
w swoich towarzyszy.

Potem znowu przed grupą na początku, tym razem nadlatując do nich z~tyłu
i z~lewej. A~potem znowu z~boku, a~potem od przodu, póki, za czwartym
zbliżeniem, prowadzące byki i~krowy zaczęły biec, skręcając na kurs,
który teraz skłaniał się na zachód. Wtedy pilot odsunął łódź, wznosząc
się do góry i~zatrzymując się na kilku tysiącach metrów, oglądając
sukces swojego odchylenia całego wielokilometrowego strumienia. Choć
wyglądało to nieznacznie, pilot wyglądał na zadowolonego.

-- To na razie wystarczy -- powiedział. -- Wrócę. Postawiłem dużo na to
stado.

Pochylił się nad wciętym panelem i~skierował się na kurs prosto na
zachód. W~ciągu kilku kolejnych minut minęli kilka stad brnących na
południe w~tym samym kierunku co statek. Inne łodzie działały przy nich,
goniąc je, blokując jakiekolwiek ruchy w~kierunku północnym. Takie próby
ucieczek stały się częstsze, gdy pasmo gór zmieniło się w~oddzielne
szczyty i~pogórza. Długoszyje padlinożerne nietoperze, krążąc na prądach
termicznych, zostały rzucone we wrzeszczące spirale przez kilwater
łodzi. Drapieżniki i~padlinożercy ucztowały na ciałach bestii, które
odmówiły zawrócenia i~uległy, jak tłumaczył pilot, dekapitacji przez
łodzie przelatujące przez ich karki.

-- Czy to jest bezpieczne? -- Twarz Elizabeth była biała, jej ręce
zaciśnięte, usta wąskie jak u zaura.

-- Och tak -- powiedział pilot. -- Tylko krawędzią, to nie konkurs. To
ogonów uderzających z~dołu lub z~góry trzeba się obawiać. Właściwie
także, zmiażdżenia pod stopami. Trzeba być przeklęcie głupim lub
niefartownym, ale to się zdarza.

Przed nimi zachodnie morze pojawiło się na horyzoncie i~jak lecieli ku
niemu, mogli zobaczyć coraz bardziej intensywne gromadzenie stad, łodzie
brzęczące przy wielkich bestiach jak pszczoły, aż do momentu, gdy całe
stada panikowały i~pędziły na oślep ostatnich kilka kilometrów i~minut
ich życia.

Ku rzeźniczym klifom.

\threeast

Łódź unosiła się, wisząc w~powietrzu sto metrów nad szczytem klifów i~kilkaset metrów nad plażą. Same klify były częścią zachodniego wybrzeża
Southland, szczyt, zwany przez ludzi Gadara, wznosił się około dwustu
metrów nad plaże.

Każdy zauropod, który spłoszył się na krańcu, był nieubłaganie
zepchnięty przez te za nim. Więc dziesiątkami, według wyniku, wielkie
zwierzęta spadały ku śmierci. Te, które przeżyły chwilę, ich uderzenie
zmiękczone przez już spadłe ciała, były szybko miażdżone przez kolejne
upadki.

Ciałom nie dawano czasu, żeby się gromadziły. Na kilometry wokół tej
prymitywnej masowej rzezi działał przemysłowy proces rzeźnictwa i~przechowywania. Specjalistyczne pojazdy brodziły w~krwawych falach,
ciągnąc i~ciąć, pompując i~zlewając. Morze blisko brzegu było zatłoczone
wielkimi żelaznymi statkami, z~których były rozwijane haki i~kable, żeby
przyciągnąć porąbane mięso. Kilometr na południe stała na palach w~morzu
instalacja wielkości małego miasta. Chmury pary i~dymu dryfowały ponad
nią.

Ponad wszystkim, ponad plażą, przybrzeżną flotą i~przetwórnią, powietrze
było wypełnione białoszarymi skrzydłami morskich nietoperzy,
powierzchnia morza rozbita przez ich nurkowanie w~milionie miejsc jak
krople deszczu na jeziorze.

Gregor patrzył się na proces z~rodzajem mdlącej fascynacji. Bardzo się
cieszył, że nie czuł zapachu niczego. Mógł się tylko zmusić do spytania:

-- Czy to nie trochę nieefektywne? Czy nie zanieczyszcza mięsa?

-- I~-- dodała Elizabeth podobnie sparaliżowana -- czy to nie jest
\emph{okrutne}?

-- Bogowie, nie -- odpowiedział Salasso. -- Śmierć jest szybka, może nawet
szybsza niż jakakolwiek inna metoda zabijania tak wielkich bestii. Co do
wydajności, używamy każdej części zwierzęcia. Zanieczyszczenia, cóż,
właściwie nie jest trudno spłukać odchody.

-- Poza tym -- powiedział pilot -- to tradycja. Robiliśmy tak na większą
skalę w~dawnych czasach i~mniej wydajnie.

Kolejny nieprzyjemny skok zabrał ich dalej i~wyżej, by pokazać szeroki
krajobraz czarnych klifów wybrzeża i~kilometrów białego piasku.

-- Obejrzyjcie plaże, których już nie używamy -- powiedział. -- Piasek tam
to odłamki kości.

\threeast

Statek znów się obrócił i~poleciał jeszcze kilka kilometrów na południe.
Klify zmniejszyły się do kamienistych brzegów graniczących z~trawiastymi
równinami, a~prosta droga i~linia kolejowa przecinające prerię i~kończące się kilometrowej długości groblą pojawiły się na ekranie. Na
końcu grobli, nad morzem garbiła się Nowa Lizbona, kamienista wyspa
wysadzana ulicami i~obrębiona nabrzeżami. Jej porty były zatłoczone
statkami i~łodziami. Kilometr w~morze, obsługiwany przez zwykłą flotę
małych łodzi, statek gwiezdny wisiał na wodzie.

Pilot wysadził ich na końcu jednego z~moli i~odleciał, z~powrotem do
prowadzenia dinozaurów. Gregor stał na chodniku, obserwując znikanie, i~wziął głęboki wdech. Bryza nie niosła zapachu straszliwej pracy
wykonywanej pod klifami. Mięso podawane tutaj, przewiezione z~wahadłowych statków-fabryk do długodystansowych tankowców-mroźni, było
już przetworzone, spakowane do mrożenia, lub posolone i~uwędzone,
ugotowanie i~w puszkach. Część z~niego byłoby wysłane na dłuższy rejs
niż przebycie oceanu. Jakie przysmaki, Gregor daremnie się zastanawiał,
(oczy? języki? nerki?) byłyby warte przewiezienia do gwiazd?

Odwrócił się do miasta na wyspie i~swoich przyjaciół. Nowa Lizbona
ukazała się przed nimi na jej wulkanicznej zatyczce, budynki gęste i~wysokie, ulice wąskie i~strome, stóg igieł.

-- Więc dokąd teraz? -- spytał.

Salasso podniósł walizkę i~zaczął iść wzdłuż płyt nieustraszonym
krokiem. Gregor i~Elizabeth pośpieszyli za nim.

-- Szukamy noclegu -- powiedział zaur. -- Potem się dzielimy i~szukamy. Mam
listę miejsc. Proste.



\chapter[Rewolucyjna Platforma]{14 Rewolucyjna Platforma}

Delegaci wymachiwali bronią. Siedzieliśmy na skraju chwiejnej sceny.
Driver i~Lemieux, ich awatary nędzne, ubrane jak Bucharin i~Zinowiew,
siedzieli za stołem, marszcząc brwi nad danymi w~ich iluzyjnych
głębokościach. Camila, jej głowa płynnie dodana do ciała długonogiej
amazonki, podtrzymywała kordelas na ramieniu jej płaszczu z~jaka i~patrzyła na tłum zebrania z~nerwową zabawą.

-- Na czym to jest oparte?

Avakian, ubrany jak mułła, podniósł wzrok znad brudnego notatnika, jego
wirtualnego ekranu.

-- Kongres Ludów Wschodu w~Baku\footnote{rzeczywiście
odbył się w~1920 roku, zob.~\url{https://en.wikipedia.org/wiki/Congress\_of\_the\_Peoples\_of\_the\_East} - przyp.tłum.}
-- powiedział. -- Tysiąc dziewięćset dziewiętnasty, chyba. Ściągnęliśmy
szczegóły ze starego filmu o~Johnie Reedzie, prawdę powiedziawszy. W~zeszłym tygodniu mieliśmy Sowiety Piotrogrodu, ale to była drobna zabawa
z zabłoconymi płaszczami i~spodniami Lenina.

\emph{Pamiętajcie Dwudziestu Sześciu Komisarzy Baku}, pomyślałem ponuro,
gdy moja wirtualna skórzana kurtka i~spodnie, bolszewicki szyk,
skrzypiały. Moja dłoń zacisnęła się na realistycznie zardzewiałej
afgańskiej kopii Lee-Enfielda, która reprezentowała mój głos w~obradach.
Pasztuni, Mongołowie, Turcy, Ormianie, Kazachowie, Kałmucy i~wiele
innych narodowości, wszyscy w~tradycyjnych kostiumach z~mieczami,
karabinami i~okrutnymi minami, wypełniali i~zajmowali miejsce w~półokrągłym audytorium pod trzepoczącym namiotem. Scena i~nasze
wyposażenie mogło być dowodem pokręconego poczucia humoru Avakiana, ale
spotkanie było tak buntownicze i~tak brzemienne w~konsekwencje, jak
oryginał.

Nawet jeżeli żaden z~nas nie skończył zastrzelony przez Brytyjczyków.

Był to dzień po naszym przybyciu, dzień po tym jak pokazaliśmy dane
projektu Avakianowi. \emph{Nie} potrzebował mojej pomocy. Po przeleceniu
przez całość szybciej nawet niż Driver, zaczął bredzić i~pędzić do
wyjścia, gestykulując dzikie spekulacje ukrytej fizyki AG, z~czego
mogłem jedynie wyciągnąć, że gęstość jądra transplutonowca generowało
plazmę kwarkowo-gluonową w~jądrze, które, kiedy wprawione w~cykliczny
ruch, oddziaływało bezpośrednio z~kwantową pianą rozmaitości
czasoprzestrzennej, po czym sprawy stały się skomplikowane i~tajemne.
Miał także kilka pomysłów o~Silniku, co do których był jeszcze bardziej
podekscytowany i~znacznie mniej zrozumiały.

-- Pomyśl o~bombach termojądrowych w~porównaniu z~bombami jądrowymi -
powiedział. -- To ta sama fizyka, ale podniesiona kilka rzędów wielkości
do czegoś, co nie ma \emph{praktycznego limitu. } Nie \emph{kontrolujesz
tylko masy. } \emph{Stajesz się światłem.}

A potem jego przerażający śmiech.

Driver powiedział mu, żeby rozgłosił w~intranecie stacji, a~póki tym się
zajmował, Driver i~Lemieux wyjaśnili błyskotliwy plan buntowników o~Rewolucji.

Byli zajęci ujawnianiem matematyki obcych, podstawy dla kompletnego
ujawnienia dowolnego szyfrowania opartego o~liczby pierwsze, tak wielu
węzłom Internetu do ilu potężne i~bardzo precyzyjne nadajniki stacji
mogły dotrzeć. Potem nie zabrałoby dużo czasu dla rozproszonej mocy
obliczeniowej i~pomysłowej współpracy światowych hakerów i~geeków, by
otworzyć tajemnice każdej placówki wojskowej i~bezpieczeństwa, na Ziemi.
Ludzie wkrótce mieliby dorobki wszystkich oszustw praktykowanych przez
władze po obu stronach i\ldots

I w~ten sposób zrobilibyśmy Rewolucję!

Uderzyło mnie to jako rodzaj programu, który mógłby być wymyślony przez
naukowców, rozczarowanych bezpieczniaków, geeków. Dokładnie tego rodzaju
naiwne, apolityczne sugestie młodych informatyków były wyśmiewane na ich
pierwszym spotkaniu w~RIWWW. Wiedza to jedno, władza to coś innego.

Ale nie powiedziałem, że tak myślałem. Ich program nie obaliłby żadnych
państw, może kilka rządów, ale nie raczej nie wyrządziłoby szkody.

\threeast

Driver wstał i~przemówił do zebrania.

-- Towarzysze i~przyjaciele -- powiedział z~odrobiną ironii w~głosie --
wiem, że jesteście bardzo zajęci i~wiem, że wszyscy wiecie, o~czym jest
to spotkanie, więc nie będę tracił czasu. Zdecydowaliśmy o~naszej
rewolucyjnej strategii\ldots

Przerwał mu pomruk aprobaty i~szuranie opozycji, które software VR
Avakiana, w~duchu kongresu, przetłumaczył na uderzanie kolb i~tumult
krzyków \emph{''Allah-hu Akbar!''}. Uśmiechnął się i~kontynuował:

-- \ldots więc musimy teraz zdecydować czy włączyć Projekty Nevada, jak je
nazywamy, do programu. Widzę argumenty za i~przeciw. Oczywistym
argumentem za jest to, że jeżeli te urządzenia będą pracować tak jak
niektórzy z~nas myślą, otrzymamy nadzwyczajne efektywne środki obrony.
Przeciwko, oczywisty punkt jest taki, że produkowanie tych maszyn może
być dywersją czasu i~zasobów od znacznie ważniejszych zadań. Nie uważam
siebie w~żadnym stopniu za kompetentnego do podjęcia decyzji, to
częściowo pytanie techniczne. Teraz Ty.

Usiadł w~brzęczącej ciszy. Jeden z~naukowców natychmiast wstał. Nazywała
się Aleksandra Czumakowa, drobna kobieta o~intensywnym spojrzeniu.
Musiała zhakować swojego awatara, ponieważ pojawiła się, nosząc
zwyczajne ubranie.

-- To jest farsa! -- powiedziała. -- Przestańmy grać w~rewolucjonistów,
jesteśmy naukowcami. Odkryłeś pewne nielegalne działania Majora
Suchanowa i~jego powiązania z~Ziemią, dobrze i~wspaniale. Działałeś
zdecydowanie, odwołując się do ludzi i~właściwych władz, zostały podjęte
działania\ldots Doskonale! A~teraz zrujnowałeś swoją moralną pozycję,
rozpoczynając tę kampanię przewrotu\ldots

Około dwóch trzecich zgromadzenia podniosła broń i~pokazała zęby.
Czumakowa spojrzała na nich i~kontynuowała:

-- \ldots nad którą wielu z~nas ubolewa, a~które łatwo mogą destabilizować
wojskową równowagę na Ziemi\ldots

Driver podniósł dłoń.

-- Mieliśmy tę dyskusję w~zeszłym tygodniu -- powiedział. -- Podjęliśmy
decyzję i~ponowne jej rozważanie \emph{nie} jest w~programie tego
spotkania.

-- Bardzo dobrze -- powiedziała Czumakowa. -- Spójrzmy na to, co jest w~programie. Sugerujesz zbudowanie w~oczywisty sposób niebezpiecznych
urządzeń, dla których plany zostały tutaj przywiezione przez
amerykańskiego agenta i~dezertera! Dlaczego po prostu nie wysadzić nas
wszystkich i~skończyć z~tym?

Jeden na trzech, który wydawało się zgadzać z~nią, prawdopodobnie
pokiwał głową i~krzyknął w~rzeczywistości, ale VR przetłumaczyła ich
odpowiedź tak jak wcześniej.

-- To może się szybko znudzić -- powiedziała Camila pod nosem. Avakian
spojrzał na nią i~zmarszczył brwi, ale późniejsze odpowiedzi były mniej
hałaśliwe.

Mężczyzna nazwiskiem Angel Pestaña wstał, opierając się na kolbie
długiego karabinu. Czarny burnus jego awatara wskazywał na północną
Afrykę.

-- Myślę, że moja koleżanka Czumakowa nieco przesadza. Towarzysz Driver
nie wzywa nas do budowania tych maszyn, prosi o~przedyskutowanie. Bardzo
dobrze, zatem, dyskutujmy! Pierwsze, problem bezpieczeństwa jest
dywersją, jak to, że jest to jakiś amerykański sabotaż. Ci z~nas, którzy
kłopotali się sprawdzić, potwierdzili, że dane naukowe pochodzą z~interfejsu z~obcymi. Mogę dać ci bibliografię Aleksandro!

-- To, co jest znacznie ważniejsze dla nas, to zrozumienie, dlaczego obcy
dali nam te informacje i~w jakim celu nas pominęli. Gdy to zrozumiemy,
możemy decydować czy nadać priorytet budowie tych maszyn.

-- To może opóźnić konstrukcję o~kilka milionów lat. -- powiedział Driver.
-- Zakładam, że tego nie proponujesz?

Pestaña potrząsnął głową. 

-- Sugeruję, by się ich zapytać.

Śmiech.

-- Hmm, tak, zawsze jest to -- powiedział Driver. Rozejrzał się dookoła za
kimś chcącym przemówić. Louis Sembat wstał następny.

-- Prowadziłem nocą pewne obliczenia -- ogłosił, machając czym, co mogłoby
wyglądać dla wszystkich innych jako zwój pokryty zawijasami kaligrafią,
ale prawdopodobnie był ręcznym czytnikiem -- na podstawie prac
rozpoczętych przez pana Cairnsa. Wygląda na wykonalne, w~końcu nie używamy
zbytnio fabrykatorów w~kampanii informacyjnej. Podstawowe materiały są
dostępne. Mamy wystarczająco transplutoników w~laboratoriach, by
zbudować przynajmniej prototypy Statku i~Silnika. To naprawdę nie jest
wielka sprawa. Spróbujmy. Według dokumentacji Cairnsa, pierwsza
konstrukcja powinna zająć tylko kilka tygodni. Silnik może zabrać trochę
więcej, ale doświadczenie przy budowie Statku może nawet przyśpieszyć.

Sygnalizowałem gorączkowo do Avakiana, który dał mi głos. Kilka szybko
wprowadzonych komend makro sprawiło, że mój awatar stukając wysokimi
butami do jazdy, stanął przed stołem.

-- Chciałbym tylko zauważyć, nawiązując do ostatniego mówcy. Nie jestem
naukowcem ani technikiem, jestem menedżerem systemów i~wiem doskonale,
że jakakolwiek ocena czasu, którą znajdziecie w~dokumentacji, jest
bezwartościowa. Więc proszę, nie liczcie, że dostaniecie latającą
maszynę w~czternaście dni.

Mój awatar przedstawił ten skromny wkład jak gdyby przemawiał do
żołnierzy z~pociągu pancernego. Gdy stał w~rozkroku lub cofał się,
powiedziałem Avakianowi, żeby przestał się bawić. Tylko się uśmiechnął.

Pojawiło się więcej argumentów, niektóre całkiem techniczne, a~pod
koniec dwie oczywiste grupy na spotkaniu same były podzielone: ci,
którzy poparliby Czumakową, chcieli iść dalej, może dlatego, że to było
coś innego niż wojna informacyjna. Niektórzy ze zwolenników Drivera
wyraźnie się przeciwstawiali, z~tego samego powodu. Rozłam wyglądał na
bardzo bliski, gdy Driver skinął na Camilę. Rzuciła mi zaskoczone
spojrzenie i~przeszła do mównicy przy stole, wyglądając szczupło,
wrażliwie i~srogo.

-- Przyjaciele -- powiedziała -- jestem tylko komercyjnym pilotem
doświadczalnym. Nie wiem zbyt dużo o~polityce, ale znam się na
samolotach. Jeżeli te maszyny są tym, czym pan Avakian myśli, że są, to
możecie zmienić stąd świat. Możecie udostępnić je, tym kogo wybierzecie,
lub nikomu. Ponadto, możecie zdobyć szacunek, jeżeli będziecie widocznie
operować najbardziej zaawansowaną technologią, jaką widzieliśmy. To
mogłoby złamać wiele zacięć. Mogłoby sprawić, że kosmos byłby atrakcyjną
możliwością, nawet gwiazdy! Moglibyście podtrzymać obietnicę gwiazd!
Dobrze, może umysły komet już posiadają większość nieruchomości, ale
możemy coś wymyślić. Mam na myśli, no chłopaki, jesteście naukowcami!
Możecie to zrobić, i~\ldots wiecie\ldots myślę, że powinniście.

Wróciła na skraj sceny w~niesamowitej ciszy. Avakian, mistrz
manipulacji, pozwalał tej odpowiedzi wybrzmieć.

-- Boże -- Camila wyszeptała -- zepsułam, wepchnęłam to piach.

-- Byłaś wspaniała -- powiedziałem jej.

Głosowanie było trzy do jednego na korzyść.

-- Bóg jest wielki -- powiedziała Camila.

Delegaci wymachiwali bronią.

\threeast

Gdy tylko się rozwiali w~mgiełce, Avakian wyłączył ekran i~raz jeszcze
zaczepiałem się nogą do wspornika w~zatłoczonym biurze Drivera. On,
Lemieux, Avakian, Camila i~ja patrzyliśmy po sobie, mrugając i~potrząsając głową, zwykłe niepewne chwile po wyjściu z~VR w~pełnej
immersji.

-- Poszło dobrze -- powiedział Driver. -- Nie dzięki Tobie, Armen. Jestem
zaskoczony, że zwolennicy Aleksandry nie strajkowali przeciwko Twojemu
pogrywaniu.

-- Hej, zwolniłem w~trakcie i~będę stąpał lekko w~przyszłości, dobrze?

-- Dobrze. Matt, jak dużo czasu potrzebujesz, zanim zamienisz ten plan
projektu w~produkcję?

Próbowałem dać uczciwą ocenę. 

-- Najdalej kilka dni. Ale nie mogę
obiecać, że nie będzie wyskakujących błędów, gdy będziemy rzeczywiście
próbować.

-- Ta, pewnie, żaden plan nigdy nie przetrwał\ldots et cetera. Dobra.
Ruszaj. Przygotuj listę materiałów, nawet zgrubne sumowania, tak szybko,
jak możesz. Jeżeli brakuje nam istotnych \emph{elementów}, mamy problem,
ale komponenty mogą być zrobione w~fabrykatorach, i~ciągle mamy trochę
surowych rzeczy, od lotnych, aż do żelaza, które możemy wykopać i~oczyścić, jeżeli będziemy musieli. Nie chcę znaleźć wąskiego gardła, gdy
ruszymy. Jeżeli są jakieś, chcę wiedzieć o~nich ASAP.

Zwrócił się Lemieux. 

-- Paul, zbierz zespół, współpracuj z~Mattem i~naukowcami, wciągnij Sembat, informuj Czumakową, pogadaj z~Wolkowem i~Telesnikowem, żeby kosmonauci byli przygotowani i~upewnij się, że znasz
wejścia, Statek nie będzie żadnym cholernym pożytkiem, że jeżeli stery
są dla, nie wiem, \emph{macek} czy coś.

Avakian zaśmiał się i~wtrącił: 

-- Jest w~porządku, widziałem panele, są
dla dłoni. Może nie \emph{naszych}, och, zapomnij. Ale zdecydowanie
żadnych macek.

-- Dobrze -- powiedział Driver, jak gdyby zastanawiał się, czy jego żart
nie został potraktowany zbyt dosłownie. -- Może mógłbyś pomóc Mattowi z~integracją systemów i~danych naukowych, dobrze?

-- Ok.

-- Zgadzasz się z~tym, Paul?

Lemieux kiwnął głową.

-- Dobrze. Camila\ldots-- Driver patrzył na nią przez chwilę, marszcząc
brwi. -- Czy możesz skontaktować nas z~Nevadą i~może popracować z~naszymi
inżynierami nad Twoim, och, latającym spodkiem? Jestem pewien, że wymaga
konserwacji. Żadnych kłopotów z~bezpieczeństwem?

-- To otwarta technologia -- powiedziała Camila. -- Ta, to miłe.

-- Super! -- Driver uśmiechnął się w~najbardziej nieosobliwy sposób i~podniósł zaciśniętą pięść. -- Zatem \emph{per ardua ad astra }. To w~łacinie ,,zabierać dupy''. 

-- Tak zrobiliśmy.

\threeast

Moje pierwsze zadanie było świeckie i~nudne, przeniesienie większości
mojego softwaru z~pamięci w~amerykańskiej technologii do implementacji w~europejskim biotechu i~zintegrowanie całości z~moim czytnikiem i~amerykańskimi szkłami. Szczęśliwie, intranet stacji miał całą bibliotekę
haków i~tipów dla takich realizacji, ale mogłem to zrobić ze starym
geekiem z~boku. Kiedy przeniesienie było zakończone, moje poczucie
triumfu zostało zagłuszone myślą, że już pół dnia poślizgu zostało
dodane do naszego harmonogramu. Ale martwienie się tym zabrałoby kolejne
minuty, więc\ldots

Possałem kawy, włączyłem szkła i~zająłem się prawdziwą pracą.

W scenariuszu Kongresu Avakiana, moje odruchy, już przystosowujące się
do mikrograwitacji, zostały wytrącone z~równowagi przez środowisko,
gdzie obiekty wirtualne zachowywały się jak gdyby miały wagę, a~moje
rzeczywiste ciało nie miało. To była przyjemność wrócić do własnego VR,
gdzie wszystko się zazębiało. Wcześniej o~tym nie myślałem, ale
przestrzeń danych, w~której zwykle pracowałem, realizowała wirtualną
fizykę zero-g. Mój punkt widzenie rzucał się w~niej jak rybka, teraz gdy
podświadomie nie rywalizował z~uchem wewnętrznym.

Ponadto, cały projekt był teraz osadzony w~oryginalnym kontekście, z~którego była wyciągnięta dokumentacja ESA. Pojęcia, szczegóły i~powody,
które za pierwszym razem były trudne i~bezbarwne, teraz brzęczały w~rezonansie. Rozumiejąc znacznie więcej tego, co robiłem, robiłem to
szybciej. AI także zmądrzało, ich sugerowane rozwiązania często mnie
zaskakiwały, wychodząc poza granice ich pojęciowych słowników. Nie
zmieniło to głupoty niektórych sugestii, oczywiście, ale przeglądanie
takich było miejscem, gdzie się włączałem i~robiłem to lepiej. Możesz
zrobić dużo z~dobrym zbiorem danych, ale nic nie może pobić praktycznego
rozwiązywania problemów na miejscu.

Gdy skończyłem przygotowywać pierwszą wersję listy potrzebnych surowców
i wysłałem ją do właściwych departamentów, zawołałem Avakiana, który
połączył swój pulpit z~moim i~wtedy, wzrastająco, z~naukowcami, który
byli zainteresowani projektem. Godziny mijały, szybko, przypominało to
moje wcześniejsze doświadczenie przeniesienia się ze spętanego
środowiska UE do USA, ale z~większą wolnością, ponieważ każdy mógł się
dzielić. To był styl wspólnej pracy, którego wcześniej nie
doświadczyłem, a~który był uzależniający jak dobrze zaprojektowana gra.

W końcu Avakian zauważył, że nasz wskaźnik błędów, nieporozumień i~sporów zaczął narastać.

-- Koniec -- powiedział -- koniec dziennej zmiany. Widzimy się za osiem
godzin.

Gdy wyszliśmy z~wspólnej przestrzeni, zostałem jeszcze na chwilę w~VR,
przeskakując kanały wiadomości stacji. W~Europie kryzys wydawał się
łagodnieć, uliczne demonstracje ciągnęły się długo, podczas gdy mieszane
komisje i~komitety starały się wypracować kompromisy. W~tym samym czasie
skandale na wysokim szczeblu i~protesty na ulicach wybuchły w~Stanach
Zjednoczonych. Rządy Indii i~Chin złożyły w~ONZ pewne nieokreślone
skargi o~rzekomo nieprzyjaznych postępowaniach UE, nadwyrężających
wielki antyimperialistyczny sojusz. W~tle tego wszystkiego, w~obu
blokach i~niepodległych, rozprzestrzeniały się przecieki, plotki i~spekulacje na temat skali obecności obcych w~Układzie Słonecznym.

O sytuacji Jadey nie było żadnych wiadomości. Wysłałem jej wiadomość,
przez Sąd Szeryfa, ale bez większej nadziei, że ją szybko zobaczy. Potem
poszedłem do jadalni i~zjadłem tłuczone ziemniaki, grilowane marchewki i~królika w~curry.

Mniam.

\threeast

Odsunąłem kurtynę miejsca, gdzie spaliśmy, żeby napotkać wzrok Camili,
która wycierała nagie ciało mokrą szmatką.

-- Jeju! Co \emph{to} jest, Dworzec Centralny?

-- Przepraszam -- powiedziałem, ruszając, żeby zasłonić kurtynę.

Skinęła. 

-- Hej, jest ciepło, wejdź.

Wsunąłem się i~zawisłem naprzeciwko niej, gdy wisiała, żeby wyschnąć.
Nie wyglądała tak chudo w~mikrograwitacji, jak wyglądała w~skafandrze
kosmicznym, jej piersi pełne ponad klatką piersiową. Spojrzała się na
mnie w~ten bezwstydny bezpośredni, amerykański sposób, który jak akcent
zawsze mnie podniecał.

-- Chcesz, żebym znalazł coś innego?

Potrząsnęła głową. 

-- Wszędzie jest źle. I~tak musiałabym dzielić z~kimś,
a wolę dzielić z~Tobą niż z~jednym z~tych komuchów. Szczególnie kobietą.

-- One nie są komunistami.

-- Ta, ta, wiem. Ruskie, francuzy, to samo. Dziwaki. Świry. Nie to, co
my.

-- My Anglos, panno Hernandez?

-- Jak mówiłam. Wiesz, co mam na myśli. -- Nie odpowiedziała na pytanie. --
Wyglądasz na zmęczonego.

-- Wykończonego. Ale robimy postępy. -- Opowiedziałem jej o~tym, krótko. --
Co robiłaś?

-- Och, rozmawiałam z~inżynierami. Wędrowałam w~rzeczywistości.
Sprawdzałam starą \emph{Bluźniercze. } Napadałam komisarza.

Pogrzebała w~bałaganie za nią.

-- Załatwiłam trochę zioła -- powiedziała. -- Chcesz spróbować?

-- Jak dostałaś \emph{to} na stacji kosmicznej?

-- Hydroponika i~w ogóle, ktoś musiał ją dobrze wykorzystać. No i~to
prawie nie jest ryzyko pożarowe.

Zajęła się dziesięciocentymetrowym zwojem plastiku i~fajką na baterie.

Och, dobrze, dzięki. I~tak brzęczy mi w~głowie.

Possała aromatycznej pary i~przekazała mi aparat. Wciągnąłem z~wdzięcznością i~oddałem, czując odlot. Jej ciemne oczy błyszczały.

-- Brzęczy czym?

-- Kiedy zamykam oczy -- powiedziałem, zamykając je -- ciągle widzę
struktury danych, ścieżki krytyczne, rozwinięte widoki, dane, Silnik,
Silnik na końcu tego wszystkiego, jakby świeciły się w~ciemnościach.

Otworzyłem oczy. 

-- I~wiadomości z~domu.

-- Ach. -- Dopompowała parnik aż znowu zaczął syczeć i~bulgotać. -- Jadey.
Żadnych wiadomości?

-- Nie.

-- Przykro mi. -- Jej źrenice się rozszerzyły, powieki zmniejszyły. -- Naprawdę. Ty i~ona. Gówno. Pech dla\ldots

Zaśmiała się i~podała mi fajkę. Wentylacja wyła mi w~uszach.

-- Pech dla kogo?

-- Ach, dla niej -- zamknęła oczy, dryfując. -- Nie, uczciwie, miałam na
myśli Ciebie i~mnie.

Widziałem, w~jakim kierunku to zmierza, nie urodziłem się wczoraj.

-- Dlaczego? -- spytałem, tak jakbym nie wiedział.

Spojrzała się na mnie, dryfując bliżej, jej piersi i~oczy ukazywały się
jak zbliżające się statki.

-- W~locie byliśmy bardzo blisko -- powiedziała. -- Rozmawialiśmy o~wszystkim. Nigdy nie rozmawiałam z~nikim w~taki sposób.

Z zawrotami głowy przypomniałem sobie rozmowy. Nie wyglądały na takie
intymne dla mnie, raczej\ldots nie wiem\ldots przyjaźń z~kimś z~kim można
porozmawiać o~wszystkim, co język przyniesie. Mówiła o~dzieciństwie,
podróży na dętce jej dziadków z~Kuby, szkole, treningu. Mówiła o~facetach, z~nostalgią, sentymentem, a~czasem surowo. Oboje ubrani w~żelu, wydawało się to przerwą od seksualności.

Samotność jej rzadkiego talentu i~brawurowa odwaga dotknęły mnie.

-- Nie wiedziałem -- powiedziałem.

-- Och cholera, Ty, po prostu słuchałeś. Kurde, kiedy spałeś, robiłam
wszystko, żeby Cię nie pocałować.

-- \emph{Kocham} Jadey -- powiedziałem. -- Chryste, \emph{tęsknię} za nią.

-- Nie mam z~tym problemu -- powiedziała. -- I~myślę, że Ty też.

Nasze usta się spotkały, zanim mogłem odpowiedzieć.

\threeast

-- Chryste, człowieku, miałeś się trochę przespać -- powiedział Avakian. -
Co do cholery robiłeś, parowałeś jakieś zioło całą noc?

-- Coś w~tym rodzaju -- wymamrotałem. Potarłem swoje powieki i~założyłem
szkła. -- Trudno zasnąć po tym wszystkim.

-- Tak, nie tak po prostu! -- zgodził się, gdy jedno po drugim reszta
zespołu się włączała.

Puściliśmy zapytanie na listach dostępnych i~potrzebnych surowców.
Zabrało to chwilę, drzewa akceptowanych substytutów były rozgałęzione,
przeplatane, bliskie kombinatorycznej eksplozji, wyzwanie nawet dla AI.

-- raz, raz, raz -- mamrotał Avakian.

Wyświetlacz pokazał zieleń.

-- Super!

-- Dobrze, teraz dla matryc Leontiewa\ldots

Program, który był w~stanie kierować ekonomią małej republiki
socjalistycznej lub dużej międzynarodowej korporacji, przestukał
iteracjami i~wydrukował kompletny plan produkcji. Zatrzymaliśmy się na
chwilę, po prostu patrząc. W~tym momencie, wydawało się to
wystarczającym osiągnięciem. Gdybym był w~domu, zabrałbym teraz cały
zespół do restauracji.

-- To wszystko -- powiedział Michail Telesnikow, kosmonauta. Jego widmowa
obecność promieniowała zniecierpliwieniem. -- Puśćmy to w~symulatorze.

Symulowana produkcja odkryła wystarczająco dużo błędów, byśmy byli zajęcie
przez kolejne godziny, poprawiając i~ponownie uruchamiając. W~końcu
wirtualne modele fabrykatorów wykonały swoją pajęczą pracę i~wypluły
dysk.

Unosił się w~centrum naszej przestrzeni, srebrne spodki, które
skoncentrowały naszą uwagę. Podwójnie nierzeczywiste, symulacja, której
w sercu nie mogliśmy uwierzyć. Oryginał zbyt tani dzięki niekończącym
się fałszywym reprodukcjom i~fałszywym raportom, żeby zrobić efekt,
który musiał wywrzeć na ich pierwszych widzach, lub w~intencji jego
twórców.

\emph{Nieprawdopodobne Geometrie} pomyślałem, mentalnie nazywając
urządzenie. Telesnikow przełączył się na pełnego awatara i~stanął przed
nim, patrząc na nasze, dla niego, niewidoczne profile.

-- Cóż, chodźcie -- powiedział. -- To tylko \emph{statek.}

Avakian, cichy przez chwilę, przepiął nas wszystkich i~podeszliśmy, gdy
Telesnikow sięgnął i~dotknął błyszczącej okrągłej krawędzi tej rzeczy.
Przypomniałem sobie rekolekcje na Festivalu, kaznodzieja w~Princes
Street Gardens mówiący o~jakimś biblijnym widgecie, nazywała, zdaje się,
arka przymierza, która zabijała, jeżeli się jej dotknęło.

Ale wszystko, co się zdarzyło, to z~bezszwowej struktury wysunęły się
trzy nogi, otworzył się właz i~pochylnia z~małymi dziecięcymi stopniami
się rozwinęła. Telesnikow odważnie wszedł, potem Avakian. Oszukałem
kolejność i~wszedłem jako trzeci. Inni, nie tak szybcy do startu,
przeszli na nieimersyjne punkty widzenia i~przeszli przez kadłub.

Wewnątrz, Statek był prawie znajomy, z~początku rozczarowująco, potem
niesamowicie. Gładka centralna obudowa silnika uformowała oparcie
kolistej ławy, skierowanej ku ekranom, które podobnie obiegały cały
Statek. Poniżej ekranów była opadająca półka, jedna sekcja, które
zawierała niezrozumiały monitor nieczytelnych instrumentów i~płytkę, w~którą były wpuszczone kształty dwóch małych i~długich dłoni, jak gdyby
ktoś z~trzema palcami i~długim kciukiem wcisnął swoje dłonie w~materiał,
zanim ten zastygł.

Widziałem podobne rozmieszczenia w~dokumentach i~dowodach w~starych
śmieciach, które skanowałem z~Krainy Marzeń. Tak jak wszyscy, którzy
kiedykolwiek twierdzili, że byli zabranie przez UFO, lub że odtworzyli z~jednego z~rozbitych okazów, mogliby wymyślić coś takiego.

-- Niech to diabeł -- powiedział Telesnikow. -- Śmieją się z~nas.

-- Może nie są zbyt jaśni w~zakresie \emph{palców} -- powiedział Avakian.

-- Nie, nie to miałem na myśli. To jest śmieszne! To jest kopia jakiejś
tandetnej dezinformacji USAF.

Jego słowa uruchomiły podniecone paplanie naszych kolegów, wirujących
dookoła kokpitu jak niewidzialne, ale wściekłe pszczoły. Telesnikow i~ja
wydawaliśmy się jedynymi osobami obecnymi, które miały więcej niż ogólne
pojęcie o~szczegółach mitów UFO. Inni skłaniali się do miłosierniejszej
interpretacji Avakiana, że to był zwykły błąd w~rozumieniu ludzkiej
anatomii przez obcych, sugestia, którą ci z~najdłuższym stażem łączenia
z nimi traktowali jako znacznie wiarygodniejszą niż ja.

-- Myślą i~widzą w~innej skali niż my -- nalegał Louis Sembat. -- Istnieją
luki w~ich wiedzy, martwe punkty. Wyobraź sobie nas rozmawiających z~bakteriami! Skąd moglibyśmy wiedzieć, że ta rzęska jest istotna?

Avakian przerwał bezceremonialnie dyskusję, wrzucając nas wszystkich z~powrotem do abstrakcyjnego obszaru roboczego, w~którym siedzieliśmy
dookoła stołu.

-- Wystarczy -- powiedział. -- Jakiekolwiek są powody tego błędu, wiemy, że
nasi przyjaciele są zdolni dostarczyć nam odpowiedni interfejs, ponieważ
wiemy, że raz już tak zrobili. To tylko kwestia wykorzystania
ograniczonego widoku i~poinformowania ich o~naszych wymaganiach.

Z komentarzy i~śmiechów, które powitały to stwierdzenie, zrozumiałem, że
to nie było prawdopodobnie tak proste, jak o~tym mówił.

-- Możemy także to ominąć -- kontynuował niewzruszenie -- hakując jakąś
formę kontrolnego interfejsu. W~pewnym stopniu rozumiemy fizykę tej
rzeczy. Sterowanie nie powinno być niezrozumiałe. W~międzyczasie,
uderzamy z~budową, analizą projektu i~tak dalej dla urządzenia numer dwa Napęd Gwiezdny.

-- Chwila -- powiedziałem -- jeżeli szukamy odmiennego wyniku, nawet jeżeli
to tylko stery i~monitory, zmiany mogą wymagać korekt gdziekolwiek w~trakcie produkcji.

Avakian spojrzał na mnie. 

-- Tak -- powiedział. -- W~istocie może tak być.
Ale to jest ten rodzaj rzeczy, w~których Ty i~Twoja menażeria AI winna
być pomocna.

-- Och dzięki -- powiedziałem sarkastycznie. -- Myślałem, że będę miał
trochę wolnego przez kilka następnych dni.

-- Nie martw się -- powiedział. -- Pomogę Ci, możemy też poprosić innych o~pomoc. -- Pomachał ręką na innych dookoła stołu. -- Jeżeli nam się nie uda,
szczerze wątpię, czy komukolwiek się uda, poza \ldots Hej!

Teatralnie uderzył się w~czoło.

-- \emph{I} jesteśmy w~kontakcie z~jedynym miejscem, które może by sobie
poradziło, praktycznie. Inżynierowie Twojego pana Armstronga w~Nevadzie.
Zatem to \emph{prawdziwy } Projekt Nevada, co? Zdaje się, że to oznacza,
że musimy mieć nie-towarzyszkę Hernandez w~zespole. Może mógłbyś ją
przekonać?

Jego straszny śmiech odbił się echem w~wystarczającej liczbie chichotów,
żebym zrozumiał, że w~takim miejscu bez prywatności, pewne wiadomości
podróżują szybko.

\threeast

Pod koniec wieczornej zmiany, Driver wezwał mnie do biura. Zapisałem
pracę i~przybyłem do biura, gdzie już byli Lemieux, Camila i~Avakian.
Ciągle wyglądało na to, że jesteśmy samowybranym komitetem projektu.

-- Całkiem dobra robota dzisiaj -- powiedział Driver. Przeglądał raporty
zgarnięte z~aktywności naszych VR. -- Wydaje mi, że mówiłeś coś wczoraj o~\emph{dłoniach}, Armen. Dlaczego nie zgłosiłeś tego jako problemu?

Avakian wzruszył ramionami. 

-- Miałem tylko podejrzenie na podstawie
kilku niejasnych schematów, które mogły nie być ostateczne. Poza tym
chciałem zobaczyć, co w~końcu wyjdzie niż ugrzęznąć wcześniej w~dyskusji.

-- W~porządku, chyba -- pozwolił Driver. -- Jednak, coś takiego się
pojawia, przekazujesz mi to absolutnie jasno, dobrze?

-- Skoro o~tym wspominasz -- powiedział Avakian -- wygląda, że \emph{w
ogóle} nie ma żadnego interfejsu kontrolnego w~Silniku. Tym dużym
silniku. Napędzie gwiezdnym.

-- Hmm. -- Oczy Drivera prawie się zamknęły. -- To może być problem.
Powinniśmy dodać to do listy rzeczy, które chcemy, żeby obcy nam
wyjaśnili. Jeżeli możemy lub oni mogą.

-- Co to za problem z~pozyskaniem informacji od obcych? -- spytała Camila. -- Myślałem, że chłopaki już dużo uzyskaliście.

-- Ta, zgadza się -- powiedział Avakian. -- Problem w~tym, to w~większości
rzeczy wysokiego poziomu: matematyka, algorytmy kwantowe, i~tak dalej. Niezbyt dużo konkretów, jak to widzimy. Nic o~historii Ziemi lub Układu
Słonecznego, choć się pytaliśmy.

-- Istnieją rzeczy, których Człowiek nie powinien wiedzieć --
powiedziałem.

-- Nie tak bardzo -- odparł Driver. -- Moje wrażenie z~zewnątrz tutejszego
cyrku nauki jest takie, że istnieją sprawy, które Człowiek winien
cholernie dobrze odkryć dla samego siebie.

Spoczywał w~ciszy przez chwilę. 

-- Skoro o~tym mowa\ldots Kiedy uznasz za
stosowne, Armen, myślę, że tymi, którzy zadadzą pierwsze pytania powinni
być Ty i~Matt.

-- Ja? -- spytałem. -- Ale nie mam doświadczenia\ldots

-- Doświadczenie z~interfejsem jest cenne -- powiedział Lemieux. -- Ale nie
jest niezbędną wartością w~przygotowywaniu zapytań i~rozumieniu
odpowiedzi. Przynajmniej wiesz, jaki rodzaj odpowiedzi będzie
pożyteczny. A~i~tak jest to coś, z~czym powinieneś się zaznajomić.
Jesteś bardzo dobry w~integracji pomiędzy platformami, a~to jest chyba
najwyższy poziom.

-- Nie mogę się doczekać -- powiedziałem.

Podejrzewałem, że po prostu chcieli, żebym to zrobił, ponieważ bali się
wystawiania się bardziej niż to konieczne na uwodzicielski,
uzależniający urok interfejsu obcych, a~także nie do końca ufali
naukowcom, którzy tak zrobili, ale nie wrócili z~niczym znaczącym.

Załatwiliśmy jeszcze jakieś nudne szczegóły jutrzejszej pracy zespołu i~przygotowaliśmy się do wyjścia.

-- Zanim pójdziemy\ldots

Lemieux, wysoko na narożnej rurce, narysował coś w~papierowym notatniku,
oderwał kartkę i~pozwolił jej opadać w~powietrzu pomiędzy nami.
Zauważyłem w~przelocie: to był owal, z~jedną poziomą linią nieco powyżej
ostrego końca i~dwoma przechylonymi elipsami na małej osi, ikoniczny,
ironiczny ideogram mitycznego Szarego Obcego.

-- Ledwie śmie to powiedzieć -- powiedział. -- Ale jako rozwiązanie
problemu skąd znają nasze języki, i~dziwnego projektu sterowania
statkiem, zastanawiam się, Camilo, może jest coś, co mogłabyś wiedzieć,
nawet plotka o\ldots starych plotkach?

Ale Camila już się śmiała, w~chichotach wyjaśniając ciągle zaskoczonemu
Avakianowi. Sięgnęła i~złapała szkic, zgniotła w~kulkę i~włożyła do
kosza Drivera.

Potrząsnęła głową. 

-- Przepraszam, że was rozczaruję, chłopaki, ale to
przeszłam. Rozmawiałam z~ludźmi, którzy wiedzieli, jeżeli ktokolwiek
wiedziałby. Jedyny szary lud, który istniał lub który się pojawił, to
ten w~waszych głowach.

Uśmiechnęła się do nas. 

-- No weź -- powiedziała. Przez chwilę wyglądała
na zamyśloną, jak gdyby zaskoczoną pewną myślą, potem potrząsnęła głową
nawet silniej.

-- E tam.



\chapter[Kosmiczny brzeg]{15 Kosmiczny brzeg}


To miejsce było mniejsze niż Kyohvic, ale wyglądało jak miasto, lub jak
sobie wyobrażała, z~tego, co przeczytała i~usłyszała, jak wygląda
miasto. Kyohvic, dla wszystkich pół miliona mieszkańców, uniwersytetu,
domów filozofii, statków i~handlu, miało w~genach wpisane ,,miasteczko''.
Nowa Lizbona mogła mieć jedną dziesiątą populacji jej rodzinnego miasta,
ale ludzie tutaj byli znacznie bardziej różnorodni. Miasto leżało na
wybrzeżu, nie tylko morza, ale także kosmosu. Inne światy były w~powietrzu, w~zapachach, niespodziankach za każdym rogiem. W~nastawieniu,
że wszyscy wszystko wiedzieli, ale nikt nie znał nikogo.

Szła raźno, ale ostrożnie wzdłuż nachylonej wybrukowanej ulicy, jeżeli
mógłbyś nazwać coś o~szerokości trzech metrów ulicą. Gregor szedł koło
niej, stanowczo odrzucają argument Salasso, że będą bardziej efektywni,
jeżeli będą szukać oddzielnie. To było dziwne być z~nim samej. Nie
rozumiała jeszcze rozmiaru obecności Salasso, do którego się
przyzwyczaiła, gdy byli razem. Nie, żeby ją to męczyło. Jakiekolwiek
hamulce, które czuła, były jej własnym działaniem. Ale ciągle.

Budynki rosły na trzy, cztery piętra z~obu stron, czarne i~wąskie jak w~domino i~tak samo zależne od podparcia innych. Ponad głowami, kolejka
linowa syczała i~iskrzyła, pracując w~górę pochyłości na dokładnie
takiej wysokości, że ledwo unikała zrzucenia wysokiego kapelusza
mężczyzna na koniu. (Regulacje miejskie, jak jej powiedziano). Stado
małych, niebiesko-czerwonych marmurkowanych dinozaurów, które dzieliły
wielkość, kształt, chód i~prawdopodobnie przeznaczenie gęsi,
prześlizgnęło się i~potoczyło koło nich, trąbiąc w~proteście przeciwko
zaganianiu przez obdartą dziewczynkę z~dużym kijem. Ulica była tak
stroma, że Elizabeth mogła zobaczyć ocean, jeżeli spojrzała prosto przed
siebie. Co nie było dobrym pomysłem, ponieważ dinozaury, ponieważ bruk
był nierówny, ponieważ tam na morzu przykucnął statek gwiezdny.

Tak, to statek De Tenebre. Wszyscy to wiedzieli. Żywiła nadzieję, że
jednak nie był. Gregor niedużo powiedział o~perspektywie zobaczenia
znowu Lydii. Właściwie oczekiwała, że będzie o~tym szczebiotał, ale
wydawał się skupiać uwagę i~podniecenie na możliwości odnalezienia
niektórych ze starej Załogi. Co było, jak się zdaje, bardzo dobre.

Salasso napisał listę trzynastu miejsc na nabrzeżu (,,na początek'') i~narysował elegancką i~dokładną mapę ulic stosownej dzielnicy.

-- Dostałeś to wszystko od swojej starej nauczycielki?

Salasso spojrzał na Gregora, jakby był głupi.

-- Oczywiście, że nie -- odpowiedział. -- To moja własna dedukcja z~tego,
co wiem o~tym miejscu. Byłem już tutaj i~dużo się nie zmieniło.

Salasso zostawił dla siebie myszkowanie po miejscach zaurów, od barów
dla pilotów skifów do bardziej eleganckich spelunek dla przedsiębiorców
w biznesie rzeźniczym. Słuchając plotek, jak to wyjaśnił. Był niechętny,
by rzeczywiście pytać, a~także poradził im ostrożności i~zwykły ubiór. I~żeby pamiętać, że byli biologami morskimi, tutaj szukającymi możliwych
kierunków badania, może wynajęcia łodzi do obserwacji krakena, coś w~tym
rodzaju. Dostatecznie blisko prawdy.

-- Wiesz -- powiedziała Elizabeth, gdy małe dinozaury odeszły kaczkowato w~boczną alejkę -- nasza przykrywka może być jedyną rzeczą, która nam
zostanie z~tej wyprawy. To właściwie \emph{dobry pomysł}.  To jest
znacznie lepsze miejsce do badania krakenów niż Mingulay.

-- Nie wierzysz, że możemy przydybać jednego z~tych starych gnojków, co?

-- Nie wiemy nawet jak teraz \emph{wyglądają}!

-- Ja wiem -- powiedział Gregor. -- Lub cholera powinienem, dostatecznie
często widziałem ich portrety. Cairns, Lemieux, Wolkow, Telesnikow,
Driver\ldots

Włochaty gigant zatoczył się dookoła rogu i~potem w~ulicę, prawie ją
blokując, gdy kołysał się do tyłu i~przodu, śpiewając pijanym basso
profondo, którego słodycz wprawiła ją w~dreszcze. Przycisnęli plecy do
ściany, schylili się pod ramieniem, gdy ich mijał.

-- Ale czy będą wyglądać tak samo?

Gregor spojrzał na nią z~boku. 

-- Tak mówi historia.

\threeast

Na dole ulicy skręcili w~lewo, minęli koniec kolejki, potem weszli na
główną ulicę, która biegła dookoła całej wyspy i~prowadziła na groblę od
strony brzegu. Wybudowana wzdłuż wału szerokiego na trzydzieści metrów,
czasem przechodziła za słupami eleganckich esplanad, czasem nurkowała w~odkrywki skał, czasem wisiała nad morzem. Co kilkaset metrów wychodziły
z niej mola, na palach lub kamieniach.

Większość ruchu składała się z~wózków wyładowanych mięsem lub rybami,
ciągniętych na dworzec traktorami z~silnikami spalinowymi lub przez
potężne, ciężkie czworonogi. Ich kierowcy oraz piesi, którzy zajmowali
chodniki, byli z grubsza równym podziałem zaurów i~trzech najbardziej
rozpowszechnionych gatunków człowiekowatych: ludzi, gigantów i~piktów.
Elizabeth widziała wcześniej kilku przedstawicieli ostatnich dwu, ale z~początku ciężko było się nie gapić. Giganty miały trzy metry wysokości,
nagie, prócz kudłatego, czerwonego owłosienia i~pasów z~narzędziami i~bronią. Pikty, smukłe i~gibkie, około półtora metra, nosiły ubrania w~ludzkim stylu na ich srebrnym lub złotym futrze, które zakrywało
wszystko prócz ich ostrych twarzy.

-- Myślałam, że pikty są bardziej masywne niż to -- półszeptem zauważyła
do Gregora. -- Wszystkie, które wcześniej widziałam były, wiesz, mocno
umięśnione.

-- To dlatego, że wszystkie były \emph{górnikami } -- odparł. -- Ale
kopalnia jest dla nich tak samo nienaturalna jak dla nas.

-- Więc w~jaki sposób stała się ich specjalnością, co?

-- Może zaury dały im prawa do minerałów -- powiedział. -- Lub bogowie -- 
jego spojrzenie wskazało boga, już widocznego na niebie wczesnego
wieczoru. -- Kto wie?

-- Ale pewnego dnia się dowiemy?

Odwrócił się do niej z~ciepłym uśmiechem.

-- Tak!

Jego ramię poruszyło się do góry i~na boki, jak gdyby chciało objąć jej
ramiona, jednak opadło. Dziwnie łamiąc krok, sięgnął do kieszeni po
listę Salasso, całkiem niepotrzebnie, pomyślała. Oboje doskonale znali
pierwsze kilka nazw.

-- Oto i~jest -- powiedział, wskazując na znak tawerny dziesięć metrów
dalej. -- Bezgłowy Kurczozaur.

-- \emph{ Zły } smak -- powiedziała.

-- Nie, nie -- odparł. -- Smakuje jak kurczak.

Jej skowyt podążył za nim przez drzwi.

\threeast 

Tawerna była wysoka, jasna i~przewiewna z~wysokimi oknami, sceny
żeglarskie w~witrażach. Może kiedyś był to dom filozofii, potem
desekularyzowany. Właścicielem był gigant, barmankami pikty, tłum w~większości ludzki i~odpoczywający. Dla wielu ludzi praca tutaj ciągnęła
się cały wieczór. A~o tej porze roku, także i~nocą.

-- Czego się napijesz? -- spytał Gregor.

-- Może na razie sok z~gujawy.

-- Ta, chyba tak. -- Niechętnie pozwolił.

Usiedli na stołkach przy barze, sącząc mrożone drinki i~rozmawiając
leniwie, gdy Gregor przyglądał się tłumowi. Większość z~nich była
tubylcami: opaleni mężczyźni w~ubraniach roboczych, ciągle brudni od
wózka czy łodzi, kilku żeglarzy z~Kyohvic zidentyfikowanych po
jaśniejszej skórze i~miększym akcencie, jeden lub dwu skinęło mu głową.
Zakładał, że rozpoznali go w~takim stopniu jak on ich.

Ale był tam jeden człowiek, siedział przy oknie, rozmawiając ze starymi
marynarzami lub dokerami, który wyglądał znajomo. Gregor nie potrafił
sobie przypomnieć skąd. Czerwone włosy, blady piegowaty jak mieszkaniec
Północy, bardzo zrelaksowany. Bardzo otwarty, po kilku minutach Gregor
zobaczył jak macha i~kiwa po następną rundę i~płaci z~tacę wypełnioną
wysokimi i~niskimi szklankami.

-- Co robisz? -- spytała Elizabeth. -- Gapisz się na piktyjki?

-- Wyglądają dość sexy -- przyznał Gregor, uśmiechając się. -- Lisie damy\ldots

Elizabeth kopnęła go w~kostkę, niezbyt mocno.

-- Nie rozglądaj się -- powiedział Gregor, stoicko ignorując ostry ból. -- Policz do trzydziestu w~głowie i~spójrz w~lustro barowe na tego młodego
kolesia ze starymi przy oknie.

Kiedy odwróciła się do lustra, całkiem sprytnie spojrzała wprost przed
siebie, jak gdyby na siebie, strząsając niesforny lok włosów.

-- Widziałam go wcześniej -- powiedziała, odwracając się od lustra,
Elizabeth.

-- Ja też, ale gdzie?

Elizabeth wzruszyła ramionami. 

-- Jakiś facet, którego widzimy codziennie
bez zauważania, doker, ktoś na uniwersytecie\ldots

Gregor kręcił głową. 

-- Eh, pamiętałbym. Musieliśmy widzieć go choć raz
razem\ldots

-- Impreza! -- powiedziała Elizabeth. -- Na Zamku. Pamiętasz?

Gregor pamiętał go, w~podobnej pozie, ale wspaniale ubranego,
słuchającego jakichś kupców Kyohvic.

-- Och, racja. Ta, to jest to. To kupiec. I~tyle z~zagadki. -- Patrzył na
nią, zafrapowany. -- Byłaś w~końcu na tej imprezie?

Gdy tylko to powiedział, zrozumiał, że nie powinien. Mógł zbyt wyraźnie
sobie wyobrazić, jak impreza przebiegła dla Elizabeth. Zdał sobie sprawę
z tego, że nie może dać do zrozumienia, że rozumie, ponieważ nie
wiedziała, że on wiedział.

Spojrzała w~bok, jej policzki poczerwieniały, ostro i~nagle jak gdyby
uderzone. Potem spojrzała na niego z~wymuszonym radosnym uśmiechem,
który chwycił go za serce.

-- Tak, byłam! -- powiedziała. -- Zdaje się, że minęliśmy się w~tłumie.
Wątpię, czy w~ogóle byś mnie zauważył, to wtedy poznałeś Lydię,
nieprawdaż?

-- Tak -- powiedział. Dopił i~wstał.

-- Ruszamy dalej? I~moglibyśmy coś zjeść na następnym, jeżeli nazwa ma
jakiekolwiek znaczenie.

-- Gorąca Kałamarnica? Ta, dobra, umieram z~głodu.

Na ulicy tłum zgęstniał. Następne miejsce było kilkadziesiąt metrów
dalej, znakiem była ponura scena parzenia się głowonogów, interpretacja
artysty kluczowej anatomii miała się nijak do biologii morza. Był to,
jednak, prawdziwy bar z~grillem, bardziej zadymiony i~głośniejszy niż
Kurczę. I~większy, z~więcej niż jedną salą, niemożliwy do obejrzenia z~pojedynczego miejsca.

Wypchany miecznik i~morskie gady wisiały pomiędzy lampami na niskich
krokwiach. Owoce morza syczały na szerokiej płycie grzejnej. Małże,
kalmary, przegrzebki i~rybie filety był przewracane, oblewane sosem i~posypywane ziołami w~sekundy lub minuty przez giganta, którego długie
mocne ramiona wydawały się wstępnie przygotowane do tej pracy. Do
smażenia było użyte bardzo mało tłuszczu, więc powietrze było pachnące
raczej niż ciężkie, dym pochodził z~konopi, a~nie od spalonego oleju, a~taka kombinacja sprawiła, że Gregor poczuł ślinę w~ustach i~ból brzucha
od głodu. Kelnerki i~kelnerzy piktyjscy wykrzykiwali zamówienia w~szybkiej kontraltowej łacinie lub angielskim. Niski kucharz dudnił
odpowiedzi i~narzekał. W~podniesionej wnęce z~tyłu, zaur kierownik lub
właściciel ubrany nieprawdopodobnie w~czarny biznesowy garnitur i~białą
koszulę, klekotał i~martwił się nad kalkulatorem, wyglądając, jakby miał
wyrwać sobie włosy z~głowy, gdyby miał jakieś. Tłum był podobnie
zmiksowany, zaury i~człowiekowate dotykający się ramionami, pijący i~rozmawiający głośno, niektórzy na wpół słuchając sopranu piktyjki, jej
srebrna satynowa suknia spływająca po jej srebrnym futrze, jej pieśń w~łacinie unosząca się ponad paplaniną.

-- Założę się, że ona \emph{prowokuje } kilka przypadków beznadziejnej
miłości. -- powiedziała Elizabeth z~pewną gwałtowną nonszalancją. Gregor,
wsuwając się na siedzenie przy małym stoliku pokrytym lepkim plastikiem,
zdecydował się potraktować jej uwagę dosłownie.

-- Bogowie, naprawdę myślisz\ldots -- powtrząsnął głową z~przesadnym
dreszczem.

-- To nie bardziej szalone niż to, co robią naprawdę zamroczeni ludzie -- powiedziała Elizabeth, odpowiadając śmiało, potem odwracając się, by
machnąć na kelnerkę.

-- Myślisz, że możemy zaryzykować piwo lub dwa do tego? -- spytał Gregor.

-- Nie marzyłabym o~piciu czegokolwiek innego.

-- Może białe wino.

Uśmiechnęła się. 

-- Tak, proszę. Małe.

Po złożeniu zamówień nastała chwila dziwnej ciszy. Kelnerka wróciła z~kilkoma butelkami piwa.

-- Myślisz, że mamy szansę? -- spytała Elizabeth. -- Odnalezienia.

Gregor rysował po wilgotnej naklejce na piwie, potem przestał, jak gdyby
złapał się na robieniu czegoś obsesyjnego.

-- Salasso wydaje się pewny, a~ja myślę\ldots

-- Co?

-- On w~to tak po prostu nie zabłądził. Przykładowo, nasze badanie
kałamarnic. Musiałbym sprawdzić to przez administrację uniwersytetu,
kiedy wrócimy, ale podejrzewam, że miał coś wspólnego z~uruchomieniem
tego projektu. I~jest trochę dziwny jak na zaura.

Elizabeth roześmiała się. 

-- Oni wszyscy są dziwni.

-- Tak, ale on jest znacznie bardziej otwarty na ludzi niż większość.
Może te na statku i~stary Tharovar w~Zamku są tak samo przyjaźni. Ale
nie wielu.

-- Hmm -- powiedziała Elizabeth. -- Wydaje się, że wie dużo o~Pierwszej
Załodze, jak udali się do zaurów po pomoc w~ukryciu.

-- Może był tam cały czas -- powiedział Gregor. -- Dlaczego nie?

Kelnerka przyszła z~tacą jedzenia i~butelką wina. Elizabeth odłożyła jej
w połowie skończone piwo, gdy kieliszki były napełniane.

-- Spekulacja -- powiedziała. -- Jedzmy.

Jedli i~pili przez chwilę, zbyt głodni, żeby rozmawiać.

-- Dlaczego w~ogóle -- spytała Elizabeth -- stara załoga wyruszyła, by żyć
incognito?

-- Nie wiem. Domyślam się, że nie chcieli kręcić się w~pobliżu i~stać się
celem oburzenia lub niezasłużonego szacunku. Niedużo jest zabawy w~byciu
ponadczasowym, gdy każdy albo zazdrości, albo wielbi.

-- Lub jeżeli musisz patrzeć na dzieci starzejące się i~umierające\ldots ale
dlaczego nie mogli użyć tego czegoś, co to było, na reszcie z~nas?

-- Może nie mieli technologii do odtworzenia tego dla kogokolwiek.

-- Mogli nam zostawić jakieś kierunku badań!

Gregor wzruszył ramionami. 

-- Może tak zrobili. W~końcu jesteśmy na
drodze rozwoju światowego przemysłu biotechnologicznego.

-- Tak, w~końcu! A~zaurowie już taki mają! Dlaczego nie zachęcić ich do
pracy nad tym?

-- Ach -- powiedział Gregor. -- To inne pytanie: co zaurowie chcą lub nie
chcą zrobić dla nas i~podzielić się z~nami. Jestem pewien, że gdyby
zaurowie chcieli, mogliby dać nam wszystko, co mają, od lekarstwa na
starość, jeżeli je mają, do skifów grawitacyjnych. Ale nie chcą.

-- To może mieć coś wspólnego z~tym, co powiedział Salasso, że nie chcą
połączyć naszych społeczeństw.

Spekulacja zdawała się bezowocna, i~Gregor nie miał ochoty prowadzić
rozmowy dalej w~tym kierunku, dobrze świadom, że tak jak Elizabeth,
unika tego, o~czym naprawdę chcieli porozmawiać.

-- Skończyłaś?

-- Tak. -- Elizabeth westchnęła zadowolona i~wytarła usta. -- Przejdźmy
się.

Wstali.

-- Razem, czy osobno? -- spytała.

-- Och, razem -- odpowiedział Gregor. -- Ludzie w~ten sposób chętniej z~nami porozmawiają.

Elizabeth uśmiechnęła się do niego wyzywająco. 

-- Możemy udawać, że
jesteśmy parą.

-- Jestem pewien, że wszyscy i~tak to założą.

\threeast

Dotarli do trzeciego pokoju i~rozmawiali z~kilkoma osobami z~łodzi o~ich
pomyśle naukowej ekspedycji, ale bez przyciągania większego
zainteresowania.

-- To jak łowienie ryb, ale bez skubania -- narzekał Gregor, gdy ktoś
uderzył go w~plecy.

-- Cześć, Matt, co porabiasz?

Gregor odwrócił się i~zobaczył wysokiego mężczyznę w~stroju rybaka,
uśmiech powoli znikał z~jego rumianej twarzy.

-- Przepraszam, chłopie -- powiedział mężczyzna. -- Wziąłem Cię za kogoś
innego. -- Zmarszczył brwi, potrząsnął głową, uśmiechnął się
przepraszająco i~odszedł przez tłum do innej lady.

Elizabeth złapała ramię Gregora. 

-- Jego zapytajmy!

Gregor potrząsnął głową. 

-- Poczekaj chwilę. Nie chcemy ich ostrzec.

Zabrało mu minutę lub dwie, by skończyć małe piwo, potem podniósł pustą
szklankę do Elizabeth.

-- Jeszcze jedno proszę?

-- Dobra.

-- Wracam za sekundę. -- Odwróciła się, żeby zamówić rundę, jej usta się
zacisnęły.

Gregor przepchnął się przez kołyszące się ciała i~pełne, stabilne drinki
i poszedł, mrugając do jaśniejszego światła i~gęstszego dymu kolejnej
sali. Mężczyzna, który go zaczepił, był z~tyłu przy stole z~jakimiś
kolegami, zdecydowanie ze statku, z~trzema młodymi kobietami
wklinowanymi pomiędzy nich. Wszyscy rozmawiali głośno i~byli
wysłuchiwani przez kupca, którego Gregor wcześniej poznał.

Jednakże to nie ta identyfikacja sprawiła, że Gregor zatrzymał się i~odwrócił się, żeby oprzeć się na łokciach o~bar i~przyjrzeć się lustru
pod słabym pozorem przyglądania się odwróconym butelkom alkoholu
ustawionym powyżej. Rozpoznał jednego z~marynarzy.

O ile nie zrobił tego samego błędu, co zrobił ten facet wcześniej,
patrzył bokiem na odbicie w~lustrze członka Załogi i~Kosmonauty Grigory
Wolkowa. Szerokie cechy mogły być podobieństwem rodzinnym, ale
krótko ścięte blond włosy były zbyt charakterystyczne. Twarz mężczyzny
miała kilka zmarszczek i~wiele bladych blizn, ale poza tym wyglądała tak
jak na obrazach oraz na fotografiach w~starych książkach.

Gregor poczuł się jak gdyby potrzebował mocnego drinka z~jednej z~tych
butelek na stojakach. Jego kolana były miękkie. Biorąc głęboki wdech,
uspokoił się i~wrócił do Elizabeth. Patrzyła na niego pytająco, trochę
kwaśno, i~pchnęła ku niemu małe piwo, gdy siadał koło niej.

-- Znalazłem jednego -- powiedział. -- Jednego z~załogi.

-- Wygląda, że trochę Tobą wstrząsnęło.

-- Tak. -- Odłożył kufel, ostrożniej, i~wytarł krople piwa z~tyłu dłoni. -
Grigory Wolkow. Po nim otrzymałem imię. Sławny kosmonauta sam w~sobie.
Są o~nim napisane \emph{książki}.

-- Nigdy o~nim nie słyszałam.

-- Och, cóż. -- Gregor się uśmiechnął. -- Bycie pierwszym człowiekiem na
Wenus prawdopodobnie nie wydaje się taką dużą sprawą po przylocie tutaj.
Tak czy inaczej, oto on, rozmawia z~kupcem, którego widzieliśmy
wcześniej.

-- Jakieś pomysły co robimy dalej?

-- Nie. Nie umiem sobie wyobrazić sposobu podejścia do niego, przy tym
udawania nierozpoznania.

-- Dobra, ja mogę! No weź.

Wzięła swój drink i~wysunęła się ze stołka. Gregor zdecydował, że równie
dobrze może iść za nią. Znowu spacer pijanych, przeplatany subtelnymi
ruchami i~etykietą, jak rozbudowany taniec. Gdy tylko Elizabeth była w~rozsądnej odległości od docelowego stołu, zamachała wolną ręką i~zawołała przyjaźnie. Gregor schylił się i~schował za nią, gdy wszyscy ludzie przy stole zwrócili wzrok na nich.

Elizabeth poszła prosto do kupca, pochyliła się nad stołem i~potrząsnęła
jego ręką, uśmiechając się w~jego zaskoczoną twarz.

-- Hej, \emph{cześć} -- powiedziała. -- Jestem \emph{bardzo} zadowolona, że
znowu cię widzę! Wcześniej nie miałam szansy porozmawiać z~żadnym z~was.

Kupiec mrugnął i~trochę wstał, trochę pochylił się nad jej dłonią. Jego
mina dezorientacji gładko została zastąpiona zaskoczonym, ale uprzejmym
uśmiechem.

-- Przepraszam?

-- Spotkanie w~Zamku w~Kyohvic, pamiętasz?

-- Och, oczywiście -- kiwnął energicznie, zamiatając dłonią, by wskazać,
że powinni usiąść i~inni powinni zrobić im miejsce. -- Twoja suknia,
włosy, wtedy tak eleganckie, nie rozpoznałem Cię. Wybacz mi.

Gregor nie był pewien, czy to wspomnienie było prawdziwe, ale gdy
przysiadł na końcu ławki, podziwiał szybkie myślenie i~opanowanie
mężczyzny tak jak u Elizabeth. Siedziała koło kupca na końcu przeciwnej
ławki, dotykając włosów i~wygładzając brudne dżinsy.

-- Marcus de Tenebre -- powiedział kupiec. -- A~teraz masz przewagę nade
mną.

-- Elizabeth Harkness. A~to jest Gregor Cairns, mój\ldots mm\ldots przyjaciel.
Jesteśmy biologami morskimi.

Człowiek, którego rozpoznał jako Wolkowa, był wciśnięty w~róg po drugiej
stronie stołu i~patrzył na Gregora cały czas z~lekkim marsem na czole.
Słysząc nazwisko Gregora wzdrygnął się, patrząc na jedną z~kobiet
naprzeciw niego i~rozpoczynając, lub podtrzymując, jakąś cichą
konwersację.

Gregor miał nadzieję, że jego własna reakcja na nazwisko kupca nie była
oczywista. Ten człowiek nie dał po sobie żadnego znaku rozpoznania. Może
był dostatecznie odległym, lub zajętym, kuzynem Lydii nie słuchającym
rodzinnych plotek.

-- Dotarłaś tutaj bardzo szybko -- powiedział Marcus.

-- Och tak, wzięliśmy łódź -- powiedziała Elizabeth, jak gdyby to była
całkiem normalna rzecz. -- Chcieliśmy odwiedzić, póki działa rynek mięsa.

-- Dlaczego, jeżeli mogę zapytać?

-- Och, chodzi o~naukę -- powiedziała Elizabeth. -- Zastanawialiśmy się jak
całe przetwarzanie mięsa, statki-fabryki i~tak dalej wpływa na lokalne
życie morskie i~może rozpocząć pewne przyszłe badania. Zobaczyć krakeny
w ich własnych wodach, takie rzeczy.

Spojrzała dookoła stołu. 

-- Ktokolwiek zainteresowany w~wynajęciu łodzi
po sezonie?

Wiele kręcenia głowami i~wzruszeń ramionami.

-- Nie ma po sezonie -- powiedział jeden z~mężczyzn. -- Przetwarzanie mięsa
zajmuje nas na jesieni, wożenie mięsa zajmuje nas przez zimę, potem
wielorybnictwo na wiosnę, kiedy lód polarny pęka na południu, a~resztę
czasu, łowienie ryb. Nie znaczy, że nie mogłabyś gdzieś się wcisnąć,
albo dostać koję na trawlerze lub wielorybniku. Musiałabyś porozmawiać z~szyprem w~dokach albo z~biurem kompanii.

-- Dużo krakenów jest widzianych przy wielorybnictwie -- dodał ktoś inny.

Elizabeth lekko się uśmiechnęła. 

-- Nie trafiliście jakiegoś przez
przypadek?

To spowodowało trochę śmiechu.

-- Nie ma szans -- powiedział pierwszy mężczyzna. -- Bystre z~nich gnojki.
Mądre, wiesz?

-- Mądre na tyle, by latać statkami gwiezdnymi -- powiedział Gregor.

-- Ta, ale to mogłoby nie wystarczyć, by ich trzymać z~daleka od
harpunów. Krakeny polują na wieloryby. Widziałem jednego, który uciekł.
Ślady po przyssawkach tej wielkości na bokach.

Rozłożył ręce na metr i~wszyscy się roześmiali, prócz Gregora,
Elizabeth, Marcusa i~mężczyzny, który mógł być Wolkowem.

Gregor spytał Marcusa: 

-- A~czym się tutaj zajmujesz?

Kupiec uśmiechnął się rozbrajająco do wszystkich. 

-- Och, tylko
relaksuję, dobrze się bawię w~towarzystwie. Miałem długi dzień. A
szczerze powiedziawszy, korzystne jest dla nas poznanie ludu.

Odwrócił się do Elizabeth. 

-- A~Twoim zainteresowaniem we mnie było ..?

-- Och! Cóż, handlarze zawsze są interesujący! Ale mam szybkie pytanie,
zastanawiałam się, czy kiedykolwiek zauważyłeś. Czy statki w~ogóle\ldots zmieniają pilotów, kiedy są nad oceanem planet?

-- Ach -- Marcus wyglądał na zaskoczonego pytaniem. -- Jestem przekonany,
że tak, choć niezbyt często. Rozumiemy, że pilot wypoczywa poza
statkiem. Zakładamy, że ten sam wpływa z~powrotem. Szczerze
powiedziawszy, byłoby ciężko to stwierdzić.

Gregor zauważył powtórzenie frazy i~zastanawiał się, czy kupiec
\emph{jest} całkowicie szczery. Również zauważył, że większość szklanek
przy stole była wyczerpana i~wstał, by zaoferować kolejkę. Marcus
odmówił, Gregor naciskał. Poszedł, gdy Elizabeth rozpoczęła szczegółowe
badanie o~wąsonogi w~próżni.

Przy barze dołączył do niego ktoś ze stołu.

-- Pomogę Ci je nieść -- powiedział mężczyzna. Jego akcent ciężko było
umiejscowić.

-- Dzięki\ldots Och, nie usłyszałem Twojego imienia.

Patrzyli na siebie z~boku, gdy piktyjska barmanka realizowała zamówienie
z nadludzką skutecznością.

-- Grigory -- powiedział mężczyzna. Jego głos opadł, ledwie słyszalny
ponad muzyką. -- A~pomiędzy Tobą a~mną, Gregorze Cairns, moje nazwisko
jest takie, jak myślisz, że jest, ale Antonow jest tym, które noszę.
Czego szukasz?

Gregor wygrzebał nieznajome banknoty i~monety, zawahał się przed
podniesieniem szklanek ze strachu, że je upuści.

-- Szukamy członków starej Załogi -- powiedział. -- Szczególnie Matta.

-- Tak jak nasz przyjaciel Marcus -- powiedział szorstko Wolkow. -- Tylko
nadstawia uszy, bezczynnie bada. Podejrzewa, ale nie myślę, że mnie
wykrył, więc uważaj na gębę.

-- Będę -- obiecał Gregor.

-- Czego chcesz od nas?

Gregor miał nadzieję, że ich rozmowa mogła ujść za lekkie żartowanie
przy barze. Przyjął tacę i~zaczął ją ładować, podając lżejsze drinki
Wolkowowi, żeby miał co robić.

-- Technologia nawigacyjna -- powiedział. -- Kompy.

-- Ach -- brwi Wolkowa zadrgały. -- Interesujące.

Wrócili do stołu i~rozdali drinki. Ludzie siedzący po stronie Elizabeth
zajęli miejsce, gdzie siedział Wolkow. Ten usiadł na miejscu Gregora, a~Gregor wcisnął się koło Elizabeth, nagle świadomy jej naciskającego
ciepłego ciała. Jej rozmowa o~wąsonogach rozwinęła się w~dyskusję każdy
na każdego na temat gatunków inwazyjnych, co do których każdy miał
głośną opinię.

Gregor napotkał ironiczne spojrzenie Wolkowa.

-- Jesteś rybakiem, Grigory?

Wolkow potrząsnął głową. 

-- Inżynierem na statkach-fabrykach przez
większość czasu. Przychodzę i~odchodzę.

Marcus pochylił się koło Elizabeth, jego twarz interesująco przejęta.

-- Grigory Antonov, zanim mi to wypadnie z~głowy, może moglibyśmy jutro
prywatnie pogadać? Jesteśmy zainteresowani silnikami morskimi, mamy
pewne materiały i~techniki, które mogą być warte obejrzenia. Wysokiej
jakości smary i~tym podobne.

-- Jasne, jasne -- powiedział Wolkow. -- Wpadnij do biura firmy, trzeci
budynek, Nabrzeże cztery. Spytaj o~Ferman i~synów. Otwierają o~dziewiątej. Będę tam.

Rozmowa toczyła się dalej. Ludzie przychodzili i~odchodzili z~drinkami,
zmieniając miejsca, póki pół godziny później Elizabeth i~Gregor znaleźli
się oboje koło ściany. Miejsce stało się bardziej zatłoczone, muzyka
głośniejsza. Gigant teraz śpiewał, głosem głębokim, ale zdecydowanie
kobiecym, dziwne. Elizabeth zaczęła się martwić o~inne miejsca, które
mieli odwiedzić.

-- Myślisz, że idzie nam dobrze, czy wolałbyś ruszyć dalej?

Gregor zastanawiał się przez chwilę.

-- Myślę, że znaleźliśmy\ldots ludzi, których szukamy -- powiedział. Fraza ją
uderzyła. Gregor wskazał Wolkowa wzrokiem. -- Przy okazji, potwierdzone.

-- Och. Dobrze.

Spojrzała na swoją szklankę. 

-- Ale powinniśmy iść, ponieważ nadal nie
znaleźliśmy nikogo, kto by wynajął nam łódź.

-- Możemy to zrobić rano -- powiedział Gregor. -- Biura firm lub przy
dokach, powiedział tamten człowiek, pogadać z~szyprem, pamiętasz?

-- Och. Pewnie. Ale nadal chciałabym już iść.

Odwróciła się do niego. Jego twarz była blisko jej, zarumienionej od
drinka i~gorąca, jego oczy nieco szkliste od zioła, które dzielili. Jego
włosy były nieporządne i~sznurowate po kilku dniach bez mycia. Ich
biodra zderzyły się razem. Gdy się odwróciła, jej ramię wsunęło się pod
jego i~położyła je w~jego pasie w~nagłej brawurowej chwili. Myśl
pojawiła się w~jej głowie, że taka szansa mogłaby nigdy się nie pojawić.
Gdyby nie odpowiedział w~sposób, o~jakim marzyła, później przedstawiłaby
to jako część udawania, że są parą.

Więc przeniosła ramię do jego barków i~położyła drugą dłoń na jego
policzku.

-- Chodź, Gregor -- powiedziała, uśmiechając się. -- Chodźmy w~cichsze
miejsce.

Jego oczy rozszerzyły się i~usta otwarły. Dotknął jej policzka bardzo
delikatnie. Jej palce, nagle zauważyła, badały tył jego głowy, jego
włosy opinały jej nadgarstek. Czy pociągnęła go, nie była pewna.
Całowali się, zanim wiedzieli, co się zdarzyło, gorąco i~bez pamięci,
ich języki ślizgające i~plątające jak u parzących się delfinów.

Potem odsunęli się od siebie i~spojrzeli na siebie. Gregor trzymał jej
ramiona jak gdyby mogła pęknąć.

-- Chciałam to zrobić, -- powiedziała -- od pierwszego momentu, gdy cię
zobaczyłam.

Wyglądał na szczęśliwego, ale bardziej zmieszanego niż zaskoczonego.
Może, pomyślała z~nadzieją, może podejrzewał.

-- Szkoda, że tego nie zrobiłaś -- powiedział.

-- Nigdy się nie odważyłam.

-- Teraz się odważyłaś.

-- Tak!

Zanim którekolwiek mogło powiedzieć, przy stole pojawiło się małe
zamieszanie, gdy Marcus de Tenebre wydostał się ze środka ławki i~wstał,
by przywitać Lydię.

\threeast

Gregor, jego ręce ciągle na Elizabeth, spojrzał na Lydię, życząc sobie
mocno, żeby ziemia otwarła się i~go pochłonęła. Spojrzała na niego z~bardzo dziwną miną, niezgorszoną lub zszokowaną, ale zatroskaną. Jej
twarz błyszczała się od potu. Jej długie czarne włosy były związane z~tyłu fioletową wstążką, miała na sobie te samą mądrze złożoną sukienkę,
którą nosiła na ich spacerze.

Powiedziała coś ważnego do Marcusa i~minęła go i~stanęła na końcu stołu,
ciągle wyglądając na zmartwioną.

-- Gregor\ldots Elizabeth. Tak się cieszę, że was znalazłam tak szybko.
Możecie ze mną iść, proszę? \ldots ze mną i~moim kuzynem. Wszystko będzie
dobrze, możemy iść\ldots

Zatrzymała się jak gdyby z~braku oddechu.

-- Co się stało? -- spytał Gregor.

-- Wasz zaur. Och, Salasso. Ma kłopoty.

Gregor stanął przed Lydią i~obok Elizabeth, nie do końca wiedząc, jak to
się stało.

-- Jakie kłopoty?

Lydia położyła na chwilę dłonie na swoich uszach i~spojrzała się na
niego karcąco.

-- Z~innymi zaurami. Musicie natychmiast iść.

-- Oczywiście, natychmiast. Elizabeth, czy mogłabyś\ldots

\emph{pójść i~powiedzieć Salasso} chciał powiedzieć.

-- Idę z~wami -- powiedziała Elizabeth.

Gregor mrugnął i~pokręcił głową.

-- Tak. Dzięki. Dobrze.

Całkowicie sobą rozczarowany, podążył za innymi. Przedostanie się przez
tłum było brodzeniem w~gęstym błocie. Spojrzał ponad ramieniem i~napotkał uważny wzrok Wolkowa. Kosmonauta podniósł rękę jak gdyby do
pomachania, potem bardzo nieśpiesznie zacisnął ją w~pięść.


\chapter[Fajne towary]{16 Fajne towary}



-- Gotów? 

-- Tak -- odpowiedziałem, ale, tak jak wcześniej, pytanie nie
było skierowane do mnie. Avakian poruszył palcem w~rękawicy i~otoczył
nas ekran. Mogliśmy zobaczyć siebie, innych, interfejs i~nic więcej. Ze
szkłami i~rękawicami, mogłem zobaczyć i~dotknąć ekranu w~komfortowej
odległości. Śledziło moje spojrzenia, właściwości rozjaśniały się i~powiększały gdziekolwiek spojrzałem.

-- Jesteśmy zdania, że jest to zindeksowane -- powiedział Avakian. -- Niestety, w~nieznanym alfabecie. Użyj wyszukiwarki. To jest, miejsce na
lewej.

Złapałem schematy, podświetliłem system sterowania, wbiłem złożone
zapytanie Boole'a, nad którym pracowaliśmy ostatnie kilka godzin i~wepchnąłem w~slot. Otaczający ekran natychmiast zamigotał. Wszystkie
obrazki i~słowa, które były ich ikonami, zniknęły zastąpione czarnym
tłem, na którym błyszczały latające spodki. Szeregi dysków rozciągały
się w~nieskończoność w~każdym kierunku. Patrzyłem na nie zafascynowany
ich niekończącymi się subtelnymi wariacjami. Skupiając się na kolumnie,
mogłem szybować wzdłuż, eksplorując możliwości ścieżek projektowych aż
do ich granic i~dalej\ldots .

-- To jak bycie w~środku floty inwazyjnej -- powiedział Avakian. -- Scena
początkowa \emph{Mars Atakuje!} z~dodatkowymi lustrami.

Jego rechot wypchnął mnie z~transu.

-- Co?

-- Zapomnij. Popatrz na te rzeczy \emph{krytycznie}, do cholery! Dla mnie
wyglądało to jak najmniej pomocna odpowiedź, jaką widziałem od mojego
pierwszego nieumyślnego outer joina.

-- Może właśnie to zrobiliśmy.

To bardzo powszechny i~łatwy błąd zapytania, które zwróci
znacznie więcej niż jesteś zainteresowany, które w~istocie zwróci
wszystko \emph{prócz} tego, co chciałeś. Jeżeli jesteś bystry lub
wystarczająco głupi, możesz odpalić kwerendę, którego odpowiedź zwróci
joina, złączenie, wszystkiego w~bazie z~wszystkim innym i~zeżre
wszystkie zasoby systemu, gdy będzie się wykonywać. Wskazówką będą gasnące
światła.

-- Nie -- powiedział Avakian. -- Składnia jest poprawna, sprawdziłem to
pierwsze.

Oczywiście, że sprawdził, tak jak ja.

-- Dobra, to na pewno nie była odpowiedź na pytanie, które zadaliśmy.

-- Lub szukaliśmy w~nieprawidłowy sposób\ldots Czy możesz przywrócić to do
stanu, w~jakim było, zanim udałeś się na małą wyprawę?

-- Co?

Avakian rzucił mi spojrzenie zza szkieł.

-- Byłeś w~tym przez \emph{dziesięć minut}, człowieku. Myślałem, że
\emph{znalazłeś} coś, ale zrezygnowałem, kiedy zaczęło się ślinienie i~ciężkie oddychanie.

-- O, gówno.

Rozejrzałem się tablicy, rozumiejąc, że beznadziejnie się zgubiłem.

-- Spróbujmy je uruchomić jeszcze raz -- powiedziałem.

Wyciągnąłem schematy i~kwerendę z~pola wyszukiwarki jak zatykający włos
ze wtyczki i~wepchnąłem je jeszcze raz. Tym razem starałem się mocno nie
ruszać i~nie patrzeć na nic prócz najbliższego dysku, wprost przed moimi
oczami. Sięgnąłem i~dotknąłem go. Odpowiedź dotykowa była chłodna i~gładka. Ten dysk rozwinął się, reszta zniknęła.

-- To już lepiej -- powiedział Avakian. -- Przejrzyjmy go.

Rozejrzeliśmy się.

-- To się staje coraz bardziej znajome -- powiedziałem.

-- Lepsze renderowanie -- odparł Avakian. -- Ale patrz tutaj.

Panel sterowania został wyrwany, jak gdyby przy odpalaniu na krótko, a~setki wijących się kabli opisano. Przejrzałem nalepki, potem wyciągnąłem
kilka książek inżynierskich z~czytnika Camili.

-- Cholera -- powiedziałem. -- Zrobili to według amerykańskiej dokumentacji
wojskowej.

-- Dotąd -- powiedział Avakian -- mogłem wierzyć, że napisali to w~pierwszej kolejności.

-- Kraina Marzeń, co?

Roześmialiśmy się, zapisaliśmy system w~naszych własnych systemach i~się
wylogowaliśmy.

-- Żebym miał jasność -- powiedział Driver. -- Mówicie mi, że możemy po
prostu wyrwać panel i~podpiąć drążek sterowniczy?

-- Hm, nie -- powiedziałem. -- Cały system sterowania w~tym dysku jest
inny, od tego, do którego mamy plan produkcji. Nie jest to tak
oczywiste, jak je połączyć.

-- Ktokolwiek patrzył na system sterowania w~tym pierwszym?

-- Ta -- powiedziała Camila. -- Ja. Jakiś milimetr pod tym czymś z~odciskami palców tworzywo jest w~stanie stałym aż do końca. Przyjrzałam
się tej grubości milimetra pod wysoką rozdzielczością, moje najlepsze
przypuszczenie jest takie, że to jakiś rodzaj czułej na nacisk poduszki
z czymś, co odpowiada za zmiany w~przewodności w~górnej części. Z~tego,
co wiem, to może być dopasowane do wzorców ciepła i~potu.

-- Obcy z~zapoconymi palcami -- powiedział Avakian. -- Co za przerażająca
myśl.

-- A~stąd -- Camila kontynuowała, rozkładając ramiona na zewnątrz i~do
góry -- rozgałęzia się na cały statek, szczególnie do silnika. Przy tym
nic tak surowego jak przewody. To jest kompletnie inne od tego, co Matt
i Armen wyciągnęli.

-- Ale możesz dodać drążek i~ekran na tym?

-- Och, pewnie -- Camila energicznie pokiwała głową. -- Bez spiny.

Wyglądała na zaintrygowaną, kiedy się roześmialiśmy, potem dołączyła
się.

-- Jedyny problem jest taki, że nie mamy planów budowy \emph{tego
}statku?

-- Może dałoby się cofnąć do projektu z~tego? -- spytał Lemieux.

-- Daj mi kilka lat -- powiedziałem. -- Pamiętaj, łączenie planów mogłoby
zabrać więcej czasu.

-- Co skłania mnie do zapytania -- powiedział Driver -- dlaczego od razu
nie dali nam planów statku ze sterowaniem kompatybilnym z~człowiekiem.

-- Moglibyśmy zapytać o~taki -- powiedział Avakian.

-- Warte próby -- dodałem.

Driver patrzył ponuro na nas.

-- Nie obijajcie się -- powiedział.

Wypchnęliśmy się z~jego biura i~zanurkowaliśmy do sześcianu Avakiana. Po
dziesięciu minutach dyskutowania szczegółów kwerendy, wrzuciliśmy ją do
interfejsu, odpaliliśmy i~dostaliśmy pusty ekran za nasz wysiłek.

-- Hmm -- powiedział Driver, kiedy mu to zgłosiliśmy. -- Czemu mnie to nie
zaskakuje?

-- Masz na myśli, że to jakaś próba umiejętności? -- pyta Camila.

-- Nie -- odpowiedział Driver. -- Oni nie grają w~gry. Muszą myśleć, że
dali nam odpowiedź.

Camila szturcha w~powietrzu przez szkła, badając nasze wyniki.

-- Coś mnie zastanawia -- mówi. -- Konwencja pochodzi z~amerykańskich
dokumentów.

-- I?

Pstryka palcami i~patrzy.

-- Wy, mam na myśli, możecie mi powiedzieć, tak? One nie są właściwie
tajne, są w~przeklętej domenie publicznej. Więc może to Wy
przekazaliście je obcym?

Driver i~Lemieux patrzą na siebie, marszcząc brwi.

-- Nikt niczego nie przekazywał -- mówi Lemieux. -- Nie
\emph{wprowadzaliśmy} informacji w~interfejs obcych. Dobra, możemy, ale
nie ma to żadnego celu.

-- Więc skąd do cholery to wiedzą?

-- To wydaje się strasznie trywialne pytanie -- mówię. -- Wiedząc, że nie
mamy nawet wskazówki, skąd znają nasz język.

-- To nie jest trywialne -- mówi Lemieux. Pociera kilkudniową brodę. -- I~to nie jest coś, co mogli ściągnąć z~naszych łączności, ponieważ używamy
konwencji ESA i~nie mieliśmy okazji odnieść się do waszej.

-- Jestem gotowa się założyć, -- mówi Camila -- że jedynym miejsce na tej
stacji, która literuje amerykańską konwencję wojskową to te podręczniki
przechowywane tutaj.

Podnosi swój czytnik.

-- A~jedyną rzeczą zbudowaną według nich -- kontynuuje -- to systemy
\emph{Bluźnierczych Geometrii}.

-- A~co z~Twoimi szkłami? -- pytam. -- Mam na myśli, wszyscy ich tutaj
używają.

Driver kręci głową.

-- Wszystkie cywilne -- mówi. -- Komercyjne

-- Amerykańskie wojsko ich używa!

-- To -- wyjaśnia cierpliwie Camila -- dlatego, że ten typ szkieł możesz
kupić w~każdym amerykańskim sklepie lub centralach telefonicznych baz
wojskowych, z~tego powodu, jest lepszy niż pieprzone złomy, których
używa Armia. Nawet wasze czerwone biodegradowalne są lepsze niż\ldots

-- Dokąd zmierzasz? -- spytałem, niecierpliwie.

-- Zmierzam do tego -- powiedziała nam -- że obcy mogą przeczytać każdy bit
i bajt danych na każdym komputerze na tej stacji.

-- Ach -- powiedział Lemieux. -- Skoro wykryliśmy wcześniejsze hakowanie
strumienie, to było nasze wyjściowe założenie.

-- Dobra, w~ten sposób wyjaśniliśmy małą tajemnicę -- powiedział Driver.
-- Teraz, jak mówiliśmy\ldots

-- Nie! -- krzyknęła Camila. -- Chwila.

-- Czekam -- odparł Driver.

Camila, Armen i~ja wszyscy zaczęliśmy mówić o~tym samym na raz. Driver
podniósł dłoń.

-- Camila.

-- Miałeś rację minutę temu -- powiedziała. -- Oni myślą, że dają nam
odpowiedź, i~oni, oni mówią nam, żebyśmy wbudowali sterowanie i~silnik w~\emph{Bluźniercze Geometrie}.

Chwila ciszy.

-- Dobrze -- mówi Driver. -- Fajny pomysł. Ale jeżeli ta rzecz jest
dostatecznie modułowa, żeby tak zrobić, dlaczego nie jest dostatecznie
modułowa, żeby złączyć dwa dyski obcych?

Kręcę głową. 

-- Nie, nie, to kompletnie inny problem. Chwila. Camila,
możesz mi podesłać dokumentację do \emph{Bluźnierczych!}

Wyciąga kabel z~czytnika i~wpina go w~port moich szkieł.

-- Wszystko Twoje. Pamiętaj nie dzielić się tymi informacjami z~nikim z~krajów komunistycznych.

-- Będą o~tym pamiętał -- mówię, zanurzając się.

Pierwsze, sprawdziłem, czy sterowanie na nowym dysku było kompatybilne z~tymi na naszym statku. Były tak jak oprzyrządowanie. Potem nałożyłem
obrazy obu dysków i~puściłem znaczniki na kablach w~nowym. Rzeczywiście
pasowały do siebie z~jasno określonymi węzłami Silnik pierwszego. Kiedy
wyizolowałem Silnik i~cofnąłem się do planu produkcji, odkryłem, że plan
miał ukrytą modułowość,  możliwe było niezależne zbudowanie Silnika.
Wymagało to znacznie większej pracy, ale widziałem jak ją można zrobić.

Kiedy spróbowałem zrobić coś podobnego na obu statkach, ugrzęzłem z~problem braku wiedzy, które części były zbędne, system sterowania, a~które nie były. Ten, jednakże, pasował razem doskonale.

-- Więc, spróbujemy -- powiedział Driver.

Jedyny problem, który mnie męczył, gdy wylogowałem się na koniec zmiany
tego długiego dnia, było pytanie, które Driver zadał wcześniej:
dlaczego obcy najpierw dali nam plan Statku, którym nie mogliśmy latać,
statku zaprojektowanego dla innego gatunku. Czy to była ich odpowiedź,
zastanawiałem się, na pytanie, którego nie zadaliśmy?

\threeast

-- Myślisz, że ci dwaj są kochankami?

-- Kogo?

Camila patrzy na mnie, jak gdyby z~większej odległości niż pół metra
pomiędzy naszymi twarzami, gdy wisimy, każde zaczepione piętą o~tyłek
drugiego, w~naszym towarzyskim sześcianie. Wtedy kładzie łokcie na moich
kolanach i~pochyla się, żeby mówić cicho.

-- Driver i~Lemieux.

-- Co? -- Śmieję się. -- Nie powiem, że zauważyłam pewne kwieciste
manieryzmy u obu.

-- Lemieux\ldots

-- \ldots jest Francuzem. Oni wszyscy tak gadają, prócz gejów, i~\emph{brzmią} jak Amerykanie. \emph{Très, très } modne, jak mi mówiono.

-- Hm -- nalegała -- pomiędzy nimi coś się dzieje. Jestem tego pewna.

-- Dobra, a~nawet jeśli? -- spytałem. -- To nie tak, że to jakaś wielka
sprawa. Zresztą, nie w~wyrafinowanej socjalistycznej Europie.

-- Dobrze, dobrze -- powiedziała, brzmiąc nieco obronnie. -- Mam na myśli,
jeżeli nie są, to co zamierzają?

-- A~to jest dobre pytanie. Ale daj spokój. Są konspiratorami, którzy
mogli być w~tym latami. Właście przeprowadzili zamach stanu, który nie
cieszy się stuprocentową popularnością. Grupa Czumakowej bez wątpienia
knuje przeciwko nim. Kiedy sprawy się uspokoją tam w~domu, w~ten czy
inny sposób będą musieli się mocno tłumaczyć. Driver był traktowany
przez CIA jako aktywa, a~teraz twierdzi, że przez cały czas był
podwójnym agentem, ale w~takich sytuacjach, księgowość jest zawsze
otwarta.

-- Ta, mów mi o~tym -- powiedziała ponuro. -- Czym jesteśmy?

-- W~jakim kontekście?

Pocałowała czubek mojego nosa. 

-- Politycznie.

-- Och. -- Myślałem o~tym, pocierając brodę, prawie zaskoczony jego
gładkością. Camila przyniosła elektryczną maszynkę od komisarzy i~mocno
nalegała, żebym jej używał.

-- Hm, jestem dobrym Europejczykiem, a~Ty jesteś dobrą Amerykanką, ale
nie wszyscy w~domu mogą tak to widzieć.

-- Ty to powiedziałeś. Nie chcę zaczynać listy wszystkich praw, które
złamałam będąc po prostu tutaj, eksport technologii, handel z~wrogiem i~tak dalej, a~Ty jesteś nazwany dezerterem. I\ldots

Westchnęła i~sięgnęła w~bok po fajkę i~paczkę zioła.

-- I?

-- I~powinniśmy zacząć na siebie uważać. Upewnić się, że nie zostaniemy
zmieleni, kiedy to wszystko się uporządkuje. Ofiarowani jako poświęcenie
mocom, które będą, wiesz.

Wzdrygnąłem w~wilgotnym cieple. Fraza ,,mocom, które będą'' brzmiały
dziwnie niespecjalnie dla rządów, teraz gdy wiedzieliśmy, jakie inne
moce były. Ale wiedziałem, że nie o~to jej chodziło.

-- Nie wydaje mi się prawdopodobne, by Driver nas zdradził --
powiedziałem. -- Ani Twoi szefowie.

-- Może, później, nie zależeć od nich.

Spieniła fajkę, possała i~mi podała. Zaciągnąłem się, patrząc dookoła
naszego legowiska w~nagłym ataku paranoi.

-- Czy to rozmowa tutaj jest bezpieczna?

-- Pewnie. -- Wzruszyła ramionami, sięgnęła za siebie i~pomachała małym
urządzeniem jak latarką. -- Gdy przyszliśmy, były zwykłe podsłuchy, ale
je wyczyściłam.

-- Co to za rzecz?

-- Tajne. -- Uśmiechnęła się, chowając je. -- Jednak według mnie, wykańcza
milimetrowy biotech.

-- Dobrze -- powiedziałem. -- Co zatem powinniśmy zrobić?

-- Trochę szpiegowania. Zdobyć informacje, które możemy przehandlować,
coś, co obie strony mogą wykorzystać. Na początek, odkryć, do czego
zmierzają Driver i~Lemieux.

-- Och, wspaniale. -- Oddałem jej fajkę. -- A~jak proponujesz działać?

Uśmiechnęła się dziko do mnie.

-- Podsłuchamy ich -- powiedziała. -- przez interfejs obcych.

\threeast

Obudziłem się i~okazało się, że to był poranek w~cyklu dniowym stacji,
nie tylko z~powodu mojego zegarka, zawieszonego na pasku kilka
centymetrów dalej na siatce, ale z~mocniejszego światła przy skrajach
kurtyny oraz z~głośniejszych dźwięków pracy w~korytarzu. Słuchając
dalej, domyśliłem się, że mogło nawet mnie obudzić pianie koguta. Ktoś
napełniał podajniki jedzenia w~najbliższym wybiegu dla kurczaków.

Camila ciągle spała i~byliśmy ciągle wzajemnie owinięci wokół siebie.
Jedną z~zalet mikrograwitacji jest to, że możesz zasnąć w~objęciach bez
przebudzenia się, by dowiedzieć się, że jedno z~ramion jest uwięzione
pod kochanką i~całe zdrętwiało. Potarłem jej ramię policzkiem, znowu
szorstkim i~pogłaskałem jej krótkie czarne włosy, które wydłużyły się o~milimetr lub dwa od startu i~teraz stały się przyjemnym, futrzanym
meszkiem. Camila poruszyła się, mamrotała i~przytuliła się bardziej. Tej
nocy więcej spaliśmy niż noc wcześniej, choć nie z~powodu utraty
zainteresowania sobą, uprawiając seks przed i~po naszej rozmowie i~budząc się do jakiejś wzajemnej dawki wspólnej stymulacji w~środku nocy.
Teraz, jeżeli jej senne głaskanie było czymkolwiek, Camila rozgrzewała
się przed kolejną sesją przed śniadaniem.

Gdy unosiłem się w~jej ramionach, cała ta erotyczna intymność stanęły
żywe i~realne w~mojej pamięci, a~tylko nasza rozmowa wydawała się snem.
Ale potem, gdy rozdzieliliśmy się lepcy i~poszliśmy wykąpać się,
wysuszyć i~ubrać, znowu to uderzyło we mnie jak zimny prysznic. Jej
ocena sytuacji była bardziej realistyczna lub w~każdym razie bardziej
przemyślanie niż moja, nadgoniła, gdy byłem zafascynowany pracą.

Camila przedstawiała się jako spokojna i~trzeźwo myśląca, jak niewiele
osób, które znałem, Charlie, może, pośród starych geeków, Jason, jeden
lub dwóch Websów, i~Jadey. Myśl o~Jadey zabolała, ale bez poczucia winy.
Praktycznie wypracowywałem sposób na powrót do niej, najszybszą drogą,
jaką mogłem. Tak jak ją kochałem, i~naprawdę, nie miałem złudzeń, że nie
zrobiłaby tego samego. Cokolwiek koniecznego, żeby się przedostać.

A żeby dostać się tutaj, i~z powrotem na Ziemię, i~uwolnić Jadey,
potrzebowałem Camili. A~potrzebowałem myśleć jako ona, myśleć jak
szpieg. Gdy włożyłem kombinezon, poczułem znajomy kształt ręcznego
czytnika w~kieszeni, a~obok niego, dysk danych.

W tym momencie zrobiłem pierwszy krok w~myśleniu jak szpieg, a~moją
myślą było: \emph{Coś się tutaj nie zgadza. } Odpiąłem kieszeń i~przesunąłem palcami dookoła krawędzi dysku, i~gdy szedłem do biura
Drivera, wyjąłem go i~obejrzałem, i~zrozumiałem, że to było element
puzzli, który nie pasował. Nie było miejsca na niego w~obrazie, który mi
pokazano.

Prawie krzyknąłem, gdy dookoła tego anormalnego przedmiotu elementy
całkiem odmiennego obrazu wpadły na swoje miejsce.

\threeast

-- Gotów? -- spytałem.

Słowo przepłynęło przez moje szkła:

\emph{Tak.}

Interfejs mnie otoczył.

Spędziłem większość dnia, kończąc zmodyfikowany plan produkcji i~przekazując go ludziom obsługującym fabrykatory, oraz współpracując z~Camilą, która pracowała z~inżynierami, Wolkowem i~innymi, nad
\emph{Bluźnierczymi Geometriami. } W~trakcie jednej z~przerw wpadłem do
biura Armena i~poprosiłem o~dostęp do interfejsu obcych. Jak gdyby
zaskoczony, że jeszcze go nie mam, wysłał mi kody wejściowe do moich
szkieł. W~międzyczasie, sprawdziłem wiadomości, zmusiłem się do
zignorowania ich i~pracowałem nad kwerendą.

Teraz gdy moja praca skończyła się pół godziny przed zwyczajowym
wieczornym spotkaniem w~biurze Drivera, miałem czas na mały eksperyment.

Zwalczając hipnotyczne rozproszenia uwagi interfejsu, uruchomiłem
kwerendę w~wyszukiwarce. To było bardzo proste zapytanie, o~zbiór
danych, który wiedziałem, że był unikalny dla mojego czytnika, ponieważ
sam go przygotowałem, bardzo pracowicie, dane testowe dla pracy, którą
robiłem kilka miesięcy temu. Rodzaj niskopoziomowego programowania,
które naprawdę powinno być poniżej mnie, i~przeklinałem ograniczony
budżet, który zmusił mnie do pracy nad tym. ,,Artysta, nie technik'' i~tak
dalej.

Ale teraz byłem z~niego zadowolony. Musiałem powstrzymać okrzyk radości,
gdy ekran wrócił pusty, niemal natychmiast, gdy mój kciuk włączył
wirtualny przełącznik, który odpalił kwerendę.

Następnie przeskanowałem port wejścia i~znalazłem jeden, ekscentrycznie,
ale odpowiednio, 180 stopni dookoła od pola wyszukiwarki. Wrzuciłem dane
testowe, odwróciłem swój widok i~powtórzyłem zapytanie.

Dane, które właśnie wprowadziłem, przesunęły się przede mną, jak kolejny
nudny rozdział Księgi Liczb.

Widok przyprawił mnie o~dreszcze.

W poczuciu satysfakcji stopionym z~pewnym smutkiem, powiedziałem: 

-- Koniec? -- i~interfejs powiedział tak i~się wyłączył.

Dołączyłem do Camili w~drodze do biura Drivera. Jej dłoń dotknęła mojej
jak skrzydło w~locie.

-- Cześć Matt. -- Ciepły uśmiech. -- Miałeś okazję\ldots?

-- Tak -- powiedziałem całkiem szczerze. 

-- Znalazłeś przy tym coś. 

-- Och. Gówno. Jednak warte spróbowania. Zdaje się byli bardzo ostrożni. Bystre
chłopaki.

-- Tak -- odparłem. -- Musieli być.

\emph{Ale nie tak bystre jak Ty, Camilo} czego już nie powiedziałem.

\threeast

-- Zatem to wszystko -- powiedział Driver po odebraniu raportów. -- Możemy
zacząć produkcję jutro.

-- Cholera, możemy zacząć już teraz -- powiedział Avakian. -- Dla tej
roboty, byłbym szczęśliwy, mogąc zarwać noc.

To spotkanie było większe niż nasze nieoficjalna koteria. Różni liderzy
zespołów dołączyli przez szkła, wypełniając zatłoczony pokój nierealnym
tłumem i~zmieniając grafikę w~surrealizm. Driver, prawdopodobnie
niechętny pozwoleniu Avakianowi na pokazanie jego możliwości w~kolejny
lekkomyślny i~manipulacyjny sposób, odrzucił ofertę konferencji w~pełnej
immersji.

Nakładające się kształty fantomowe Sembata, Telesnikowa i~Czumakowej
równocześnie podnieciły się na uwagę Avakiana.

-- Nie możemy tego zrobić -- powiedział Sembat. -- Bądź realistyczny.
Zespół jest zmęczony, przygotowywaliśmy fabrykatory cały dzień\ldots

-- A~my byliśmy w~EVA\footnote{ang. extravehicular
activity, aktywność pozapojazdem, tu praca kosmonauty w~otwartej przestrzeni kosmicznej - przyp.tłum.}, przenosząc materiały -- powiedział Telesnikow w~imieniu kosmonautów. -- Jeszcze trochę, a~zaczniemy mieć wypadki. A~tam,
to oznacza możliwe ofiary śmiertelne.

Driver przyjął ostatnią wypowiedź ze znudzonym sceptycyzmem kierownika
słuchającego przedstawiciela związkowego, ale podniósł dłoń i~kiwnął
głową, patrząc przez chwilę na Avakiana.

-- Dobrze, dobrze, Michail, nie ma wątpliwości co do dalszej pracy. To
nie jest pilna sprawa. Paul.

Lemieux, ogolony i~wymuskany, uśmiechnął się do nas z~góry.

-- Jednakże -- zaczął. -- Wzrasta priorytet dla całego projektu, czym
chciałbym na was zrobić wrażenie i~przekonać was do przedstawienia jej
waszym zespołom. Wszyscy słyszeliście dzisiejsze wiadomości, chyba że
byliście bardziej zajęci niż to wyglądało.

Poważne potakiwania dookoła. Czumakowa wyglądała, jak gdyby miała zamiar
coś powiedzieć, ale potem zmieniła zdanie.

-- Muszę podziękować wam wszystkim za waszą dyscyplinę w~kontynuowaniu
pracy, niezależnie od\ldots rozproszenia uwagi, lęku i~w istocie oburzenia,
które te wiadomości bez wątpienia sprowokowały. Musimy mieć nadzieję, że
nasza polityczna interwencja pomoże rozwiązaniu politycznemu, a~w~międzyczasie musimy pracować mocniej, żeby pokazać, że większość
konfliktu politycznego i~wojskowego jest obecnie przestarzała, jak mówi
Camila.

To wydawało się złagodzić i~wpłynąć na większość ludzi obecnych i~teleobecnych, ale sprawiło, że zacząłem się zastanawiać, w~jaką grę gra.
Dzisiejsze wydarzenia były brutalnym przypomnieniem, że nie gramy w~gry,
że strategia publikowania łamaczy kodów i~zalewania świata tajemnicami
miała konsekwencje. Ludzie dłużej nie wiedzieli w~co wierzyć i~żałośnie
wielka liczba była gotowa uwierzyć w~cokolwiek.

Wiadomości, która stanowczo ignorowałem przez cały dzień, przewinęły mi
się w~jednym wspomnieniu. Jeszcze wczoraj wydawało się, że kryzys
polityczny w~UE uspokajał się przy negocjacjach. Pomimo, a~może dlatego,
że, wysypka zamieszek wybuchła w~całej Zachodniej Europie. Większość w~biedniejszych rejonach, tych gdzie mafia miała więcej wpływu niż Partia.
(Części Leith, zauważyłem, były dosłownie w~płomieniach). Apolityczne,
apokaliptyczne slogany towarzyszyły niszczeniu i~grabieży. Wielu ludzi
wydawało się przekonanych, że rządy, wszystkie rządy, były jakoś w~jednej lidze z~obcymi. Nie tylko naszymi obcymi, ale także obcymi z~koszmarów popkultury, złowrogimi, satanistycznymi
Szarymi\footnote{rasa rzekomych istot pozaziemskich.
Szary kosmita stał się archetypem inteligentnego stworzenia niebędącego
człowiekiem i~ogólnie życia pozaziemskiego w~kulturze popularnej doby
eksploracji kosmosu, zob.~\url{https://pl.wikipedia.org/wiki/Szaraki} - przyp.tłum.}.

-- Matt? Jesteś z~nami?

Szturchnięcie Avakiana przywróciło mnie do teraźniejszości. Inni już
poszli, a~my byliśmy znów w~małej koterii. Zdjąłem szkła i~potarłem
oczy, patrząc dookoła na Armena, Camilę, Drivera i~Lemieux. Nie
wyglądaliśmy już na przyjemną małą klikę, teraz gdy wiedziałem nieco
więcej o~tym, co się działo.

-- Jesteś bardzo zmęczony -- powiedział Driver.

-- Tak -- powiedziałem. -- I~zmartwiony. Znam wielu ludzi w~rejonie
zamieszek w~Edynburgu, a~Jadey jest ciągle w~więzieniu kilka kilometrów
obok.

Driver przytaknął. 

-- Wszyscy jesteśmy zmartwieni, wszyscy znamy ludzi
tam, w~domu. Nie możemy nic zrobić, prócz naszej pracy.

Zastanowiłem się nad konfrontacją z~nim wtedy i~tam, ale zdecydowałem
odwrotnie. Była jeszcze Camila, a~nie do końca zrozumiałem jej punktu
widzenia.

-- Jasne -- powiedziałem. -- Chodźmy spać.

\threeast

Sen nie był w~moich myślach, choć był w~moim mózgu. Gdy tylko
zabezpieczaliśmy kurtynę, Camila zaczęła się wysuwać z~ubrań tak jak i~ja. Uderzyliśmy się i~potoczyliśmy, śmiejąc. Złapała mnie i~przytrzymała.

-- Potrzebuję tego -- powiedziała. -- Potrzebuję Cię. Inaczej robię się
bardzo napięta.

-- Hm, dziękuję -- wymamrotałem. -- Ja też.

Przez jakiś czas zapomniałem o~przyczynach, dla których mogła być
spięta. Potem gdy wisieliśmy w~naszej własnej zadowolonej, połączonej
orbicie wokół słońca, pytanie powróciło.

-- Sprawdziłaś podsłuchy? -- wyszeptałem.

-- Robię to tak regularnie jak mycie zębów -- odparła. -- Dlaczego?

Odsunąłem twarz od jej ramienia.

-- Mogłabyś włączyć jakąś muzykę?

Wyłowiła odtwarzacz i~dopasowałem ostrożnie głośność, żeby ukryła nasze
głosy wobec bezpośredniego podsłuchiwania.

-- To było ryzykowne -- powiedziałem -- grać tę gierkę z~dokumentacją
wojskową.

Jej ramię napięły się, jej nogi na chwilę się zacisnęły, potem znowu
rozluźniły. Zmarszczyła brwi.

-- Jaką ,,gierkę''?

-- Sama wgrałaś specyfikacje, wrzuciłaś podręczniki do interfejsu, racja?

Zmrużyła oczy i~potrząsnęła głową.

-- Dlaczego tak myślisz?

-- Odkryłem dzisiaj, że interfejs właściwie nie ma dostępu do wszystkich
danych trzymanych na stacji.

Odepchnęła mnie, sama odleciała do tyłu. Zatrzymaliśmy się w~przeciwnych
częściach przestrzeni, patrząc na siebie.

-- Cholera -- powiedziała. -- To jest poważne. Nie ufasz mi?

-- Nie, \emph{ufam} Ci -- powiedziałem. -- Ale nie oczekuję, że zawsze mi
wszystko powiesz. Po prostu daję Ci znać, że odkryłem, co robisz i~by
cię ostrzec, ponieważ ci ufam, że przynajmniej jeden z~naszych
przyjaciół odkryje to samo. Driver lub Lemieux wiedzą.

Zamknęła znowu oczy, potem spojrzała na mnie.

-- Zacznijmy od początku, dobrze? -- powiedziała. -- Skąd wiesz, że
Interfejs nie ma dostępu do każdego komputera na stacji?

Opowiedziałem jej o~moim małym eksperymencie.

-- I~stąd wniosłeś, że musiałam sama wrzucić te dane?

-- No tak.

-- Dobra, nie zrobiłam tego! Naprawdę nie wierzę w~kłamanie, Matt. Nie w~ten sposób. Zresztą, dlaczego miałabym to robić?

-- Żeby móc pierwsza testować silnik AG, i~może\ldots zabrać go do domu?

Camila się roześmiała. 

-- Całkiem miły pomysł. Szkoda, że o~nim nie pomyślałam.

-- Dobrze, więc jak wytłumaczysz, że Interfejs znał amerykańskie
konwencje militarne dla opisywania wykresów?

-- Nie mam pojęcia -- odparła. -- Jestem tak samo zdumiona co Ty. W~każdym
razie, co sprawiło, że się nad tym zastanawiasz? Może dlatego, że
odkryłeś, że nie możesz podsłuchiwać przez Interfejs?

-- Nie. -- pominąłem informację, że nawet nie próbowałem jej sugestii. -- Nie, ponieważ zrozumiałem, że Driver lub Lemieux, lub obaj, opowiadali
bzdury od pierwszego dnia naszego przybycia. Mówili nam, że nie ma
możliwości, żeby dane projektu były przesłane do ESA bez ich wiedzy, i~myślałem, że to znaczyło, że obcy zhakowali komunikację. Ale nie
myślałem prawidłowo. Jest coś jeszcze, czego nie wziąłem pod uwagę.

-- Co?

Pogrzebałem się za siatką i~w kieszeni mojego ubrania i~wyciągnąłem dysk
danych, który rosyjski oficer dał Jadey.

-- To -- powiedziałem. -- Jadey dostała to w~bardzo niebezpiecznych
okolicznościach. Cóż, mogę uwierzyć, że jest to wynik informacji dodanej
do wyjściowych danych z~tej stacji, z~dodanym adresem ESA, i~że to jakby
grzechotało po różnych zautomatyzowanych systemach. Ale wydostanie tych
danych wymagałoby rozwagi, decyzji, organizacji. To nie był wypadek, jak
mówi powiedzenie komunistów.

-- Ok -- powiedziała. -- Kontynuuj.

-- Co raczej mocno sugeruje, że to było opublikowane celowo stąd, a~nie
przez obcych. Przez Drivera, Lemieux lub obu, w~porozumieniu z~dowolną
organizacją, z~którą współpracują na Ziemi, prawdopodobnie tą samą,
która dostarczyła dysk do Jadey.

Uśmiechnąłem się do niej w~poprzek półtorametrowej czeluści.

-- A~Jadey jest powiązana z~organizacją finansowaną, pośród innych, przez
Nevada Orbital Dynamics. Twoim pracodawcą. Co oznacza Ty i~ja, moja
droga, byliśmy powiązani cały czas. No, czy to nie słodkie?

Camila na to odwzajemniła uśmiech.

-- I~oczywiście firma nas tutaj wysłała -- powiedziała. Narysowała palcem
kółko. -- To wielki łańcuch i~wszystko wraca tutaj.

-- Tak -- odparłem. -- I~wiemy, kto jest na Twoim końcu, amerykańskim, ale
nie wiemy, co jest na europejskim końcu, tym końcu. Nie wiemy, kto to
zorganizował. Nie wiemy, kto jest zamieszany.

\threeast

Następnego ranka wiadomości były nieco lepsze, jeżeli zdjęcia
zniszczonych budynków, strzelaniny, aresztowania i~ofiary liczą się jako
,,lepsze''. Straty wynosiły już miliardy. Uczestnicy zamieszek byli
odpowiednio potępieni lub ostrożnie, \emph{niepotępieni}, a~analizy
niuansów takich oświadczeń zajmowały dużo czasu antenowego. Camila i~ja
zostaliśmy wezwani do biura Drivera na spotkanie przed pracą.

-- Po prostu bądźcie w~gotowości -- powiedział mi -- i~zachowajcie otwarte
kanały dla fabrykatorów. Bez wątpienia będą usterki, gdy zacznie się
właściwa praca. Resztę czasu nie przeszkadzajcie, może zacznijcie ten
plan Silnika, o~ile wam się uda. A~Camila, trzymaj się grupy pracującej
na Twoim statku, upewnij się, że wiedzą co wymontować, a~co zostawić.
Armen, trzymaj się blisko zespołów produkcyjnych. Daj im wszystko, czego
potrzebują od strony naukowej i~śledź postępy w~drugim projekcie.

-- Dobrze -- powiedziała Camila. -- I~tak zamierzaliśmy to robić.

-- Zanim pójdziecie -- powiedział Lemieux. -- Ty też Armen, zostań proszę.

Spojrzał na Drivera.

-- Mamy wam coś do powiedzenia.

Camila klasnęła w~dłonie, chichocząc.

Obaj mężczyźni wyglądali tak poważnie i~zawstydzeni, mogłem przez chwilę
uwierzyć, że zamierzają obwieścić ich długotrwałą wzajemną miłość.

-- Podsłuchiwaliśmy was -- powiedział Driver. -- Przepraszamy.

-- Jak?

-- Camila -- powiedział Lemieux -- wiem, że nie jesteś szpiegiem, ponieważ
gdybyś była, wiedziałabyś, że Twoje urządzenie antypodsłuchowe działa
bardzo dobrze przeciwko urządzeniom biotech UE, ale nie, niestety,
przeciwko ostatnim amerykańskim mikrobotom.

-- Federalne Biuro Bezpieczeństwa -- powiedział Driver -- nie używa niczego
innego.

-- Hm, mam nadzieję, że mieliście zabawę -- powiedziałem.

Wymienili zawstydzone spojrzenie.

-- Przykro nam, że naruszyliśmy waszą prywatność -- powiedział Lemieux. --
Ale to elementy polityczne, a~nie osobiste w~rozmowach były przedmiotem
naszego zainteresowania. Myślimy, że mogły powstać nieporozumienia,
jeżeli ci nie zaufamy, ale nie możemy na to pozwolić.

Przeniósł wzrok na Armena. 

-- A~Ty, również, jesteś dostatecznie bystry,
żeby w~końcu ogarnąć te rzeczy i~dostatecznie bystry, żeby zrobić to
źle. Musimy sobie wszyscy zaufać, ponieważ kilka następnych dni będzie w~istocie bardzo niebezpiecznych. Matt, powiedziałeś o~łańcuchu powiązań i~masz rację. Powiedziałeś, że nie wiesz, kto jest po naszej stronie tego
łańcucha. Czas, żebyś się dowiedział.


\chapter[Sąd Krakenów]{17 Sąd Krakenów}





Gdy byli jeszcze w~barze, zapadła noc z~prędkością charakterystyczną dla
tej szerokości. Światła wskazywały długą ulicę dookoła brzegu. Gregor
śpieszył się przez gęsty tłum chodnikiem esplanady, obok Lydii.
Elizabeth i~Marcus szli szybko przed nimi.

Lydia uśmiechnęła się i~złapała go za rękę, machając nią, gdy
maszerowała obok.

-- Dobrze jest znowu Ciebie zobaczyć -- powiedziała. -- Nawet w~tak trudnej
sytuacji dla Twojego przyjaciela Salasso.

-- W~jaki sposób się o~tym dowiedziałaś?

Wyjęła małą prostokątną skrzyneczkę z~głębokiej kieszeni z~boku sukni, a~potem wsunęła ją powrotem.

-- Radio. Na brzegu, większość z~nas je nosi. Byłam na zakupach, gdy
dostałam wiadomość od Bishlayan, jednego z~naszych zaurów. Ona zna
Salasso, rozmawiali, gdy zaczęły się kłopoty. Pożyczyła radio Salasso,
który podał mi listę miejsc, gdzie mogłabym cię znaleźć.

-- W~jakiego rodzaju kłopoty wpadł Salasso?

-- Nic agresywnego. Zaury nie są tacy jak my, nie urządzają
\emph{awantur. } Czym to jest, najlepiej, żebyś sam zobaczył. To jest
bardzo intensywne. Naprawdę się cieszę, że znalazłam cię tak szybko. To
była taka ulga zobaczyć ciebie i~Elizabeth.

Gregor nie słyszał żadnej wskazówki ironii lub nagany.

-- Och\ldots Co do Elizabeth, ona i~ja\ldots

-- Tak -- powiedziała Lydia -- widzę, że się lubicie.

Znowu ta sama nieskomplikowana uwaga. Gregor zmarszczył brwi.

-- Nie jesteś\ldots zdenerwowana?

Mocniej ścisnęła jego dłoń. 

-- Dlaczego miałabym? Widziałam, że Cię lubi
jeszcze w~Kyohvic, tego dnia w~laboratorium. Jestem zadowolona, że masz
z kim być.

-- Ciągle nie rozumiem.

-- Jest możliwe kochać więcej niż jedną osobę na raz -- powiedziała
szczerze Lydia. -- Mój ojciec kocha.

-- Tak, ale to jest \emph{inne}\ldots

Spojrzała na niego z~boku. 

-- Nie bądź taki naiwny.

Zanim Gregor mógł zebrać zmieszane myśli, choćby powiedzieć cokolwiek,
Marcus ostro skręcił w~lewo, prowadząc ich wąskimi schodami ze zużytego
mokrego kamienia. Drzwi do małych tawern lub sklepów pojawiały się na
omszałych ścianach co około dziesięć kroków. Gregor skoncentrował się na
zachowywaniu równowagi i~podążaniu za obcasami Lydii. Spojrzenie w~górę,
niebezpieczne, ale interesujące, wykazało, że ściany sięgają w~górę jak
ściany kanionu, światło z~ulicy i~okien przesłaniało jakikolwiek kawałek
nieba nad nimi.

Po około stu krokach dotarli do końcowej kondygnacji, szerszej, mniej
śliskiej, ale nie mniej zużytej, która skończyła się ulicy. W~połowie,
Marcus się zatrzymał. Wskazał drzwi po ich lewej stronie, kilka kroków
dalej.

-- To jest to miejsce -- powiedział. Drzwi były częściowo przeszklone,
okna przyciemnione, znak jasno oświetlony, ale nie do odczytania. Mógł
mieć sens, dla innych oczu. Gregor widział tylko zawijasy i~bloki.

-- Co to za miejsce? -- spytała Elizabeth.

Marcus skrzywił się. 

-- Odpowiednik baru dla zaurów lub\ldots miejsce
przydziału. Byliście kiedyś w~takim?

Gregor i~Elizabeth nie byli.

-- Wejście jest całkiem bezpieczne, ale bardzo ważne jest być uprzejmym,
nie gapić się i~nie robić głośnych dźwięków lub nagłych ruchów. W~przeciwnym wypadku, możemy być wyrzuceni. Jasne? Dobrze. Elizabeth, może
zostań koło mnie, a~Gregor z~Lydią. Jeżeli będą kłopoty, pozwólcie nam
na ochronę. Chodźcie za mną.

Marcus trzymał drzwi otwarte, póki nie stłoczyli się za nim, a~Gregor
pozwolił im się zamknąć, gdy weszli. Zajęło mu chwilę, by oczy
dostosowały się do słabego oświetlenia. Powietrze śmierdziało rybą i~mięsem. Słodki powiew konopi tylko powodował większe mdłości.

Pierwsze zobaczył oczy, pochyłe elipsy obsydianu odbijające słabe
jarzenie się zawieszonych lamp. Potem rozpoznał zacienione kształty
zaurów, siedzących w~krzesłach przy szerokich okrągłych stołach. Nie
było baru, tylko ciemniejszy otwór z~tyłu, źródło klekoczących hałasów i~mocnych zapachów. Przy wejściu dwa zaury w~szatach z~paskiem stały
naprzeciwko siebie, ręce uniesione, w~krzywej kanciastej postawie.
Poruszali się bardzo wolno, jak gdyby w~tańcu lub rytualnej walce. Na
stołach leżały talerze i~kubki. Żarzące się kominy fajek poruszały się w~górę i~w dół jak tajemnicze światła na ciemnym niebie. Rozmowy zaurów
toczyły się cicho, raczej syki w~tle niż mamrotanie. Ponad tym i~poza
tym, jakieś rytmiczne dźwięki, których nie mógł całkiem usłyszeć,
cisnęły na gniazda jego zębów.

Lydia chwyciła jego dłoń. Razem podążali za Elizabeth i~Marcusem,
również trzymających się za dłonie, jak zauważył, w~kierunku narożnego
stołu z~tyłu. Zza stołu, pięć par oczu obserwowało ich zbliżanie. Gdy
podszedł bliżej, rozpoznał Salasso, który kiwnął głową. Siedział koło
zaura, który nosił ciemną, ale błyszczącą suknię. Jeden z~innych zaurów
wstał i~wskazał na cztery puste krzesła z~boku stołu. Podążając za
Marcusem, ludzie usiedli. Krzesła były z~tej samej korkowej substancji,
którą widzieli w~Miasto Zaurów Jeden. Stół, wykonany z~jednego bloku
tego samego materiału, miał ostrą skierowaną do wnętrza krzywiznę od
góry do podstawy, żeby zrobić miejsce dla kolan, choć nie dość dużo dla
ludzkich kolan.

-- Bishlayan -- powiedział Marcus. Zaur w~ciemnej sukni krótko pochylił
głowę.

-- Salasso -- powiedział Gregor. -- Wszystko w~porządku?

-- Na razie.

Z innych zaurów, dwa na prawo od Salasso były ubrane w~znajome
jednoczęściowe kombinezony, ten na lewo od Bishlayan w~coś, co wyglądało
podejrzanie jak jedna z~tych masywnych, skórzanych kurtek preferowanych
przez pilotów samolotów. Prawie komiczne na jego, lub jej, delikatnej
sylwetce. Futrzany kołnierz dodawał do wrażenia i~nieodpowiedniości.

-- Przedstawmy się wzajemnie -- powiedział zaur z~przeciwnego końca grupy.
-- Gregor, Lydia, Elizabeth, Marcus, wasze imiona, płcie i~zawody są nam
znane. Salasso i~Bishlayan znacie. Ten koło nich to Delavar, jest on,
jak możecie się domyślać lokalnym pilotem łodzi. Moje imię to Tharanack
i jestem płci męskiej. Moja żeńska towarzyszka jest nazywana Mavikson.
Jesteśmy obywatelami Nowej Lizbony i~jesteśmy zatrudnieni jako ktoś,
kogo ludzie nazywając ,,rozjemcami'', a~co my nazywamy ,,wojownikami''.

Rozłożył ręce, rozkładając cztery palce na każdej. 

-- Możecie prosić o~nasze dokumenty lub, jeżeli chcecie, możecie wezwać rozjemców z~innego
gatunku. Nie? Bardzo dobrze. Nie mogę zapytać Marcusa i~Lydii, ale muszę
zapytać was, Gregorze i~Elizabeth, czy jesteście uzbrojeni?

Gregor odwrócił się do Elizabeth i~poczuł otuchę na jej skrzywiony
uśmiech.

-- Nie -- powiedziała. -- Oczywiście, prócz naszych noży.

-- Jesteście dobrze uzbrojeni -- powiedział Tharanack. -- To dobrze. Nie
chcielibyśmy, byście poczuli się zastraszeni.

Gregor nie wierzył przez sekundę, że wytrzymały, ostry scyzoryk w~jego
kieszeni zrobiłby dużą różnicę w~walce z~zaurami, ale przypuszczał, że
znaczenie pytań było i~tak symboliczne. Ta wstępna paplanina
prawdopodobnie nawet nie była zwyczajem zaurów, tylko procedurą
policyjną w~wielogatunkowym mieście Nowa Lizbona. Zauważył, że jego
towarzysze patrzyli na niego, czekając, żeby coś powiedział.

Położył ręce na stole i~odwrócił je dłoniami do góry. Jako gest pokoju i~otwartości umysłu, prawdopodobnie wyglądało to teatralnie dla zaurów,
ale był mocno świadom potrzeby błędu z~ostrożności. Jeżeli to było
równoważne rzuceniu się na kolana i~odsłonięcia piersi, niech tak
będzie.

-- Co zdaje się problemem? -- spytał. Kątem oka zauważył lekkie aprobujące
skinięcie Marcusa. Pilot Delavar pochylił się ostro do przodu, sycząc
jakiś epitet. Mavikson uciszyła go spojrzeniem.

-- Problem jest taki -- powiedziała rozjemczyni. -- Delavar, Salasso i~Bishlayan są w~długoterminowej relacji. Salasso i~Bishlayan, oczywiście,
ostatnio spotkali się w~Kyohvic. Narosło pewne napięcie, gdy Salasso
pojawił się tutaj, podczas gdy Bishlayan była z~Delavar. W~celu
zapewnienia Delavara, że nie jest tutaj w~celu ubiegania się o~uwagę
Bishlayan, lub z~jakiegoś innego ukrytego motywu, Salasso ujawnił
prawdziwy cel. Delavar i~inni, którzy, hm, szybko stali się świadomymi
tej rozmowy, byli jeszcze bardziej wstrząśnięci tym, niż wcześniejszym
podejrzeniem o~zazdrość. Została wezwana pomoc. Zatem tutaj jesteśmy.

Irytujący rytm w~tle znikł. Za nim, Gregor słyszał cztery gołe nogi
zaurów przesuwające się na inną pozycję. Inny, ale ciągle lekko
irytujący, rytm się rozpoczął. Z~boku, widział wiele czarnych oczu
obserwujących go, i~kilka najwidoczniej, na tancerzach. Widok
przypomniał mu jego zaskoczenie pomysłem związku, lub rywalizacji, który
trwał wiekami.

-- Aha -- powiedział, skupiając wzrok na Mavikson. -- A~co według was jest
prawdziwym celem Salasso?

-- Wiesz to, tak dobrze jak ja, Kosmonauto Gregorze Cairns.

Gregor lekko się ukłonił, akceptując swój błąd. Zaurowie nie oddawali
się grom słownym.

-- Bardzo dobrze -- powiedział. -- Ale to, czego szczerze nie jestem
pewien, a~chciałbym prosić o~wyrozumiałość, jest, jaki jest zarzut wobec
tego celu?

Prawa ręka Delavar wystrzeliła do przodu i~w dół, wbijając pazury w~stół. Dwóch rozjemców syknęło ostro. Bishlayan położyła rękę na jego
przedramieniu i~pogłaskała je, powiedziała coś w~jego ucho. Powoli,
wyrażając w~języku ciała złość, która łatwo przeskoczyła granice
międzygatunkowe, pilot usiadł.

-- On rozumie wasz język -- powiedziała Bishlayan, ciągle głaszcząc ramię
Delavara. -- Ale jest zbyt rozzłoszczony, żeby nim mówić. Będę mówić za
niego, choć nie mam własnej opinii w~tej kwestii.

Jej druga ręka skubała i~głaskała jej pierś, pazury zaczepiały co chwilę
tkaninę sukni. Gregor miał wrażenie, że dla zaura oznaczało to wprost
niewypowiedziane roztargnienie i~zmartwienie. Podniósł ręce, dłonie
otwarte i~odchylone do tyłu, jak gdyby oferował swoje nadgarstki do
podcięcia.

-- Proszę -- powiedział.

Wydawało się, że odzyskała odrobinę spokoju.

-- Mój kochanek Salasso rozgniewał mojego kochanka Delavar, i~innych
tutaj, pomysłem pomocy wam\ldots. -- powiedziała coś, czego nie mógł złapać.

-- Człowiekowatym -- powiedział Salasso.

-- ,,Małpojebcom'' -- przetłumaczyła Mavikson tonem znużonej uczciwości.

-- \ldots stać się nawigatorami -- kontynuowała Bishlayan. -- Delavar wierzy,
że to rozgniewa bogów. Salasso był zaskoczony jego opinią, którą
określił jako\ldots

Kolejna fraza zauryjskiego.

-- ,,Może irracjonalnie konserwatywną''

-- ,,Parująca kupa śmierdzącego gówna dinozaurów''

-- \ldots ponieważ długo byli przyjaciółmi i~myślał, że mają podobne
poglądy. Kłótnia stała się niesamowicie gorąca. Obaj robili \emph{to}.

Zastukała palcami na stole, potem zrobiła szybkie trzepoczące gesty
dłońmi, żeby pokazać, że nie miała tego naprawdę na myśli.

-- Kiedy to się zdarzyło, poprosiłam opiekuna domu, by zadzwonił po
opiekunów pokoju, oraz wywołałam koleżankę ze statku Lydii, i~pozwoliłam
Salasso porozmawiać z~nią przez radio, żeby was sprowadzić.

Oparła się o~krzesło, wsuwając dłonie w~jej szerokie rękawy, Gregor mógł
zobaczyć pod materiałem, że jej ręce mocno zacisnęły na przeciwnym
łokciu.

Salasso pochylił się do przodu i~zwrócił się do Mavikson.

-- Chciałbym -- powiedział -- abyś nie tłumaczyła idiomów tak dosłownie. I~nalegam wobec naszych ludzkich przyjaciół, by nie czuli się obrażeni.

-- Nie czujemy się -- powiedziała Lydia, odzywając się po raz pierwszy. -- Prócz tego, czy obaj możecie potwierdzić, że Bishlayan zdała prawdziwą
relację waszej kłótni?

Salasso i~Delavar spojrzeli się na siebie, ostro odwrócili się i~przytaknęli.

-- Dobrze -- powiedziała Lydia. -- Gregor, Elizabeth, mam sugestię. Mogę?

Elizabeth wzruszyła ramionami. Gregor, który nie miał pojęcia co dalej
robić, kiwnął głową. Lydia uśmiechnęła się do nich obojga i~odwróciła
się do zaurów. Przesunęła głowę jeszcze dalej, obracając ramiona razem z~nią, potem włożyła dłoń pod jej kucyk i~podniosła go, żeby pokazać zaurom
kark. Gregor patrzył, zafascynowany delikatnymi drobnymi lokami na jej
karku. Zaurowie wszyscy nabrali powietrza w~tej samej chwili.

-- Jak widzicie -- powiedziała znów na nich patrząc -- jestem bardzo młoda.
Nie mam doświadczenia ani mądrości. Skąd mogłabym wiedzieć co może lub
nie może, rozgniewać, lub zadowolić bogów? A~widzę, że wy sami, którzy
jesteście znacznie starsi i~mądrzejsi niż ja, nie możecie się zgodzić.
Więc proszę was o~rozważenie przez chwilę przedstawienia waszej niezgody
komuś, kto był stary i~mądry, gdy wszyscy tutaj byli mniej niż jajem.
Komuś, kto rozmawiał z~bogami. Czy zaakceptujecie taką opinię i~pozostaniecie przyjaciółmi niezależnie od tego osądu?

Jak gdyby nieświadomie, naiwnie, jej ręka przesunęła się znowu do karku,
podnosząc pęczek kucyka do góry głowy. Stała w~tej pozie przez chwilę.

-- Tylko proszę -- powiedziała.

Jej włosy opadły.

Gregor zauważył, że jego paznokcie wbijały się w~stół. Szybko je
rozluźnił i~obrócił dłonie, ale nikt nie zauważył. Wszyscy patrzyli się
na Lydię.

Delavar sięgnął przed Bishlayan i~wziął rękę Salasso, początkowo
próbnie, potem we wspólnym pewnym uchwycie.

-- Zrobimy tak -- powiedział.

-- Dobrze -- powiedziała energicznie Lydia. -- Chodźmy zatem do statku i~poradźmy się Nawigatora.

Przez zawrotną chwilę Gregor jej nie zrozumiał, potem uświadomił sobie,
że mówiła o~krakenie.

\threeast

Delavar był gotów zaakceptować jednego z~rozjemców jako świadka i~był w~każdym razie niespokojny o~wznowienie przerwanej schadzki z~Bishlayan,
więc tylko Salasso i~Tharanack wyszli razem z~czwórką ludzi. Salasso,
bardzo starannie, trzymał się blisko rozjemcy i~nic nie mówił. Lydia i~Marcus prowadzili. Raczej niż z~powrotem stopniami, odwrócili się do
ulicy powyżej, która, jak większość ulic w~mieście, opadała do nabrzeża
i kończyła się wygodnym molem. Gregor i~Elizabeth zostali na tyłach.

Gregor wciągnął głęboko powietrze, starając się wyrzucić zapach knajpy
zaurów z~nosa. Ludzie płynęli w~dół i~w górę ulicy, wolnej od pojazdów
prócz kolejki nad głowami.

-- Dobrze jest być tutaj -- powiedział.

-- Co do cholery się dzieje? -- powiedziała Elizabeth.

-- Ciągle nie mam pojęcia, dlaczego Marcus był\ldots

-- Nie o~tym mówię. Co jest pomiędzy Tobą a~Lydią?

-- Nie wiem.

-- Widziałam ją z~Tobą, trzymającą dłoń i~beztrosko rozmawiającą jak
gdyby w~ogóle nas nie zobaczyła tam w~barze. Nie polubiłam tego, to
było, jak gdybym nie istniała.

-- Racja, zobaczyła nas. Nie ma nic przeciwko, właściwie wydawało się, że
jest całkiem zadowolona z~tego.

-- Czy ona rzeczywiście? Jak bardzo racjonalnie z~jej strony. Jak
filozoficznie. Jestem pewna, że jest bardzo szczęśliwa, że masz kogoś,
żeby ulżył ci w~cierpieniach miłości, póki Ty i~Twoja rodzina nie
przygotujecie tego cholernego statku, by móc za nią popędzić.

Patrzyła się wprost przed siebie, gdy to mówiła. Gregor poczuł się
roztrzęsiony, gdy szedł koło niej. Ich pocałunek w~barze, sprawił, że
to, co czuła Elizabeth stało się dla niego realne na sposób, którego
raport Salasso nie zrobił. Wstrząsnął nim, a~przybycie Lydii, zanim
mieli czas porozmawiać, pozostawiło jego własne uczucia wstrząśnięte.
Jego rozmowa z~Lydia tylko to pogorszyła. Napięte minuty w~zajeździe
zaurów były ulgą i~rozproszenie uwagi.

Tam, podziwiał Lydię bezstronnie zaciekawiony, bez wpływu adoracji. Jej
taktyczne, taktowne umiejętności rozmawiania z~zaurami były, może, tym,
co powinien oczekiwać od córki gwiezdnego kupca, ale niemniej jednak go
zadziwiły. Przypomniało mu to nieoczekiwane zrozumienie, jakie okazała
trudnym pytaniom o~krakeny, kiedy zwiedzała laboratorium, choć wtedy
podejrzewał, że popisywała się. Tym razem pokazała się zdolna do
myślenia.

-- Elizabeth\ldots

-- Co?

Ciągle patrząc wprost i~idąc szybko.

-- Czy możemy się na chwilę zatrzymać?

Zatrzymała się i~spojrzała na niego. Przez sekundę widział ją ostro i~wyraźnie, nagła suma jego wiedzy o~niej. Była wyższa, mocniejsza i~starsza niż Lydia, nie tak piękna, ale w~tym momencie wyglądała na
znacznie wrażliwszą i~znacznie piękniejszą. To go zabolało i~oszołomiło
go, że nie widział jej takiej jak teraz od początku.

Trzymał jej ramiona, jak wcześniej.

-- Kocham Cię -- powiedział. I~gdy to powiedział, stało się to prawdą,
całe jego napięcie i~zamieszanie rozwiązane, stało się jasne, proste,
śpiewające, cięciwa łuku, która ciągle brzmiała po wysłaniu strzały w~lot.

-- Zawsze Cię kochałam -- odpowiedziała.

Kiedy przestali się ściskać, ciągle się trząsł, i~musieli biec.

\threeast

Odbicie świateł statku gwiezdnego rozmazało się przez wodę jak rozlana
benzyna w~kałuży. Z~bliska, był zbyt ogromny, by być dziwny. Mógł być
jednym ze statków-fabryk lub masowców w~zatoce, prócz jego wielkości,
która pomniejszała je wszystkie. Woda dotykała boków, ale zdecydowanie
statek się nie unosił. Gdy tak było, Gregor niejasno myślał, musiał być
niżej, z~większą wypornością. Pola wygładzały morze dookoła, zastępując
fale i~kołysanie ze złożonymi drobnymi zmarszczkami, oraz sprawiały, że
włosy stały i~w uszach szumiało.

Powyżej nawisu kadłuba, przypadkowe skify wpadały i~wypadały przez
długie wąskie prostokątne otwarcia, ich soczewkowaty kształt błyskał
odbiciami świateł w~środku. Na jednym końcu, Gregor nie mógł domyślić
się, czy rufie, czy dziobie, okrągłe pochyłe otwarcie w~dolnej części
gapiło się jak usta, częściowo w~morzu, częściowo ponad nim. Woda
poniżej i~powyżej tego wejścia była jasno oświetlona, zielonkawa,
wirowała od krakenów, których pełnowidmowa komunikacja chromatoforowa
wysyłała migocące tęczą błyski przez górne warstwy wody.

Ich wejście było skromniejsze: szerokie drzwi w~dolnej krzywej kadłuba z~pontonem z~drewna i~starych opon, z~rurami przymocowanymi do parapetu.
Przewoźnik zdławił silnik benzynowy, przybił do wejścia, a~dwa zaury i~czworo ludzi zeszło z~łodzi.

-- Poczekasz na nas? -- powiedział Marcus, jak zapłacił za podróż.

-- Nigdzie się nie wybieram -- zapewnił go przewoźnik, siadając na rufie i~zapalając papierosa.

Poszli po kołyszących się deskach, Elizabeth i~Gregor pewniej niż reszta
i weszli przez próg do statku.

Elizabeth spojrzała w~dół gdy wchodzili i~trąciła Gregora.

-- Wąsonogi -- powiedziała. Uśmiechnął się do niej.

Młody załogant siedząc na cumach, czytając książkę, spojrzał się i~kiwnął głową, gdy go mijali. Za nim, wielki magazyn odbiorczy, wyłożony
deskami i~chlupoczący wodą morską był prawie wypełniony skrzyniami.
Marcus poprowadził go koło członka załogi i~skręcił w~prawo w~korytarz
wzdłuż boku statku, w~kierunku okrągłego otworu, który mijali w~łodzi.

-- Wszyscy jesteśmy spokrewnieni -- wyjaśnił ponad ramieniem. -- Nie
wymagamy ceremonii. Tędy.

Nie było innej drogi. Korytarz ciągnął się i~ciągnął, przez setki
metrów, lub tak to odczuli. Biało pomalowane płyty metalu z~dużymi
nitami, nad głowami światła elektryczne w~klatkach, sporadyczny właz po
ich lewej i~przegrody co około dziesięć metrów. Równie dobrze mogli być
w środku dowolnego dużego statku. Lub sterowcu zbudowanym ze stali,
myślał Gregor, ten korytarz prowadzący przez przestrzeń pomiędzy
zewnętrznym a~wewnętrznym kadłubem.

Po około pięciu minutach dotarli do końca korytarza i~wyszli na szeroką,
mokrą półkę, która drgała pod ich stopami. Trzech zaurów stało przy
balustradzie około dziesięć metrów przed nimi. Poza tym otwór do morza
wyglądał jak małe jezioro, około stu metrów w~poprzek, podświetlone od
dołu i~z boków jak gdyby dla jakiejś ekstrawaganckiej uroczystości. Dwa
krakeny unosiły się, ich dwudziestometrowe macki rozprostowane. Z~tego
jeziora, po ich lewej, szeroki na piętnaście metrów kanał biegł z~powrotem do środka statku. Boki jednostki zakrzywiały się dookoła
basenu, aby wysoko w~górze spotkać się na wypukłej piętrze ze szkła.
Ponad szkłem świeciły inne światła, a~dwa inne krakeny pływały wśród
poruszających się ławic ryb i~dryfujących roślin. W~odległej części
gigantycznego akwarium, szklana kolumna wyrastała z~dalszej części
basenu. Wewnątrz kolumny, winda, albo tłok pompy, przesuwał się powoli
do góry, niosąc krakena w~pozycji pionowej, jego macki zwinięty przy
głowie, jego płaszcz falujący w~mocnym rytmie.

-- To -- powiedział Marcus, wskazując do góry -- jest kabina Nawigatora i~mostek, a~to jest jego prywatna sala, gdzie spotyka i~zabawia swoich
gości. Kanały i~śluzy morskiej wody łączą to z~innymi częściami statku.

Wskazał kanał za nimi i~potem poprowadził ich do balustrady. Pochylając
się nad nią, Gregor odkrył, że patrzy w~największą parę oczu, jaką
kiedykolwiek widział. Nawet trzydzieści metrów dalej, nadal wyglądały
niekomfortowo blisko. Myśl o~wielkości i~złożoności mózgu, który musi
znajdować się za nimi, był zbyt niepokojący do rozważania. Oprócz bogów,
\emph{Architeuthys extra-terrestris sapiens } był największym
inteligentnym gatunkiem i~prawie na pewno największą inteligencją, z~którą ludzkość miała kontakt.

Był także, traktowany zaledwie jako zwierzę, przerażający. Myśl, że to
był mięczak, nie była szczególnie pocieszająca.

-- Skonsultujmy się z~naszym Nawigatorem -- powiedziała Lydia.

-- Skąd wiemy, który nim jest? -- spytał Gregor.

-- Musimy zapytać -- odparła Lydia. Porozmawiała z~jednym z~zaurów statku,
który poprowadził ich do rogu pomiędzy głównym basenem a~kanałem, gdzie
pochyły ekran i~panel sterowania były zamontowane na balustradzie. Jego
długie palce zatańczyły na panelu i~złożone wzorce światła zabłysły na
ekranie.

Gdy zaur to robił, Marcus pochylił się nad balustradą i~wskazał w~dół.
Kiedy Gregor i~Elizabeth również się pochylili, mogli zobaczyć znacznie
większą wersję ekranu, około czterech metrów na siedem, lśniącą
dokładnie pod nimi w~wodzie i~oczywiście powtarzającą wzory wyświetlane
na ekranie powyżej. Jeden z~krakenów zanurkował pod powierzchnią, i~po
minucie lub dwóch się wynurzył, skierowany w~przeciwnym kierunku, jego
macki daleko od niego, a~szeroki tył w~ich kierunku. Oczy przyglądały
się im tak jak wcześniej.

Wzory światła krótko zagrały na jego tyle.

Zaur przy ekranie odwrócił się do nich.

-- To jest nasz Nawigator.

-- Doskonale, to szczęście -- powiedziała Lydia. Wskazała na Salasso. -- Proszę, zadaj pytanie, tak jak pragniesz, w~swoim własnym języku.
Tharanack przetłumaczy je i~odpowiedzi, na nasz, a~Voronar tutaj
przetłumaczy je na język światła.

Salasso podszedł i~zadał pytanie. Varonar lekko się cofnął, patrząc na
Lydię i~Marcusa jak gdyby prosząc o~pomoc. Oboje stanowczo skinęli. Zaur
znowu się pochylił nad panelem, jego palce niepewne w~ich zadaniu.

-- Salasso zapytał -- powiedział Tharanack -- czy nawigatorzy wskazani
przez bogów wiedzą, czy bogowie rozgniewaliby się, i~czy oni sami by się
obrazili, gdyby pewni, hm, człowiekowaci wzięliby na siebie kierowanie
statkami pomiędzy gwiazdami.

Efektem pytania, gdy Varonar je przetłumaczył w~kolorowe ideoglify i~wyświetlił na podwodnym ekranie, było jakby podpalenie lontu przy
ogniach sztucznych. Krakeny w~basenie i~inne teraz widoczne w~morzu
poniżej i~te w~akwarium ponad nimi, niemal jednocześnie wpadły w~szybką
wymianę błyskawicznych, błyskających kolorowych świateł.

Gregor poczuł ramię Elizabeth obejmujące jego talię i~objął ją w~odpowiedzi, ale bardziej stanowczo. Poczuł, że wszyscy potrzebowali
przylgnąć do innych, żeby ustać na nogach. Lydia, Marcus i~zaury
patrzyły na ekran w~prawie takim samym zdumieniu.

-- Rzadko można zobaczyć coś takiego -- powiedziała Lydia. -- Tak długo i~tak intensywnie. Ilość wymienianych informacji musi być kolosalna.

W końcu, po około pięciu minutach, światła zgasły i~ciało nawigatora
ściemniało. Potem, całkiem wolno, znacznie prostsza seria wzorów
przesunęła się po jego tyle. Varonar zaczął mówić, a~Tharanack tłumaczyć
na angielski.

-- Bogowie są dookoła nas i~nie dbają o~takie rzeczy. Ich szczęściem jest
kontemplacja kosmosu takim, jaki jest. Nic, co nie zagraża różnorodności
i pięknie, które widzą w~nim, nie może rozgniewać bogów. Inni, ale nie
bogowie, wydźwignęli naszych przodków z~mórz Ziemi dawno temu. Ci Inni
wzniecili gniew bogów i~unieśliśmy przodków i~kuzynów zaurów z~powierzchni Ziemi, aby uciec przed gniewem, który zniszczył Innych. Zaur
wydźwignęli człowiekowate i~inne gatunki. Ostatnio niektórzy
człowiekowaci wydźwignęli samych siebie i~przybyli tu bez nas i~bez
zaurów. Musimy przyjąć, że bogowie zaakceptowali ich przybycie i~zaakceptują ich dalsze podróże.

Co do nas, jesteśmy szczęśliwi, będąc nawigatorami, ale bylibyśmy równie
szczęśliwi jako pasażerowie. Naszym domem jest wielki ocean, który
obejmuje światy. Jeżeli stracimy jedną specjalizację, znajdziemy inną.
Gatunki się zmieniają, nisze pozostają. Jeżeli człowiekowaci wypełnią
naszą niszę po niższej cenie, tylko skorzystamy na tym, tak jak
skorzysta każdy inny inteligentny gatunek. Pokój i~handel Wam.

Salasso okręcił się dookoła i~objął dwoje przyjaciół.

-- Wiedziałem! -- powiedział. -- Wiedziałem!

-- To nie takie proste -- powiedział Varonar, tłumacz. -- Nawigator tylko
powiedział wam, że ani on, ani jego rodzaj, nie będą z~wami walczyć, ani
nie będą bogowie. Ale będą konkurować. Tak jak i~my.

Gregor uśmiechnął się do niego nad głową Salasso.

-- Pokój i~handel -- powiedział.

Delikatnie odłączył się od zaura i~od Elizabeth i~spojrzał na Marcusa,
Lydię i~ich załogantów.

-- Musimy znaleźć nawigatora -- powiedział.

Marcus pożegnał się szybkim podaniem ręki i~cienkim uśmiechem, Lydia
nagłym pocałunkiem. Potem poszli z~Tharanackiem długim korytarzem do
pływającej przystani i~oczekującej łodzi.

\threeast

Tharanack odłączył się od nich na końcu nabrzeża.

-- Muszę przedstawić osąd nawigatorowi Delavarowi -- powiedział. -- Rano
będzie to znane w~całym mieście. W~południe na całym świecie. Nic się
nie zmieni. Ludzie ciągle muszą sami sobie poradzić.

-- Oczywiście -- powiedział Salasso. -- Ale przynajmniej nie będą stawiać
czoła nieświadomemu sprzeciwowi.

-- Miejmy nadzieję, że tak -- powiedział Tharanack i~odszedł.

Salasso czekał, aż rozjemca zniknie w~tłumie, wtedy przyjął pozę jak te
przybierane przez tancerzy zaurów. Po chwili znów stał prosto i~spojrzał
w bok jak gdyby zażenowany.

-- To było niegodne -- powiedział. -- Ale nadal to są dobre wiadomości.
Lepsze niż sobie wyobraża Tharanack, ale wkrótce zrozumie. Powtórzy osąd
słowo w~słowo i~inni, którzy nie są tak przejęci pytaniem o~ludzi,
usłyszą inną wiadomość w~odpowiedzi, wiadomość o~naszej przeszłości.

-- Jaką wiadomość? -- spytał Gregor.

Błona migawkowa Salasso zamigotała.

-- Że bogowie nie byli rozgniewani na nas w~odległej przeszłości. Nigdy
nie gniewali się na nas, ale na Innych. To są bardzo dobre wiadomości.
Mam ochotę wejść na dach i~wykrzyczeć ją. Powiem ją każdemu, kogo
poznam.

-- Nie mów -- powiedziała Elizabeth. -- Chyba że chcesz skończyć przybity
na krzyżu.

-- Przepraszam?

-- Zrzucony z~klifu -- powiedział Gregor, odgadując bardziej prawdopodobny
sposób męczeństwa zaurów.

-- Taka rzecz nie wydarzyła się od wielu tysięcy lat.

\emph{Aha.}

-- Ale rozważę, co mówicie -- Salasso oddalił temat. -- W~międzyczasie
musimy zdecydować co dalej. Czy znaleźliście kogoś ze starej załogi?

Powiedzieli mu o~Wolkowie.

Oczy zaur się zwęziły.

-- Więc Marcus i~prawdopodobnie inni ze statku, też ich poszukują. To
alarmujące.

-- W~istocie -- powiedział Gregor. -- W~jaki sposób kupcy wiedzą w~ogóle o~Pierwszej Załodze?

-- Powiedziałem Bishlayan, w~Kyohvic, że niektórzy z~nich żyją.
Wiedziała, że Athranal, nasza stara nauczycielka będzie wiedziała, gdzie
są. Więc wzięła łódź do Miasta Zaurów Jeden i~się jej spytała.

-- Czy Athranal powiedziała Ci o~tym?

-- Nie -- powiedział Salasso. -- Bishlayan powiedziała mi wieczorem.

Gregor popatrzył na zaura, potem wzruszył ramionami.

-- Prawdopodobnie mają nadzieję na jakąś umowę. W~końcu, oryginalna
załoga musi wiedzieć jak nawigować.

-- Umówić się z~nimi i~nas wywalić? -- spytała Elizabeth.

-- Całkiem możliwe -- powiedział Salasso. -- Myślę, że mogą też chcieć
czegoś cenniejszego, wiedzy o~długim życiu.

-- Mogą jej nie mieć -- powiedział Gregor. -- Mogą długo żyć, jasne, ale to
nie znaczy, że wiedzą jak przekazać to komuś.

-- Nie muszą wiedzieć -- odparł Salasso. -- Mają tę informację w~swoich
ciałach. A~jeżeli istnieje jedno miejsce w~ludzkich społeczeństwach,
które mogłoby taką informację wyciągnąć, to tylko akademie Nova
Babylonia.

Gregor zaczynał się niecierpliwić wałęsaniem.

-- Wątpię -- powiedział. -- Pamiętasz, co powiedział Esias de Tenebre? Że
nasze laboratorium jest bardziej zaawansowane niż akademie Nowej
Babylonii? Wróćmy po prostu do Gorącej Kałamarnicy i~znajdźmy Wolkowa.

-- Tak -- odpowiedział Salasso -- i~to tak szybko, jak to możliwe. Mówiłeś,
że Wolkow umówił się na spotkanie z~Marcusem jutro o~dziewiątej. Musimy
się z~nim spotkać wcześniej albo zostaniemy w~ciemnym lesie.

-- Czy Marcus mógłby zaoferować zachętę starej załodze, która byłaby
wystarczająca, żeby nas przebić?

-- Och tak -- odpowiedział Salasso. -- W~istocie mógłby.

Ale w~Gorącej Kałamarnicy, nie było nigdzie Wolkowa i~jego towarzyszy.
Gdy skończyli sprawdzać inne podobne miejsca od frontu, było już dobrze
po północy.

-- Spróbujmy go spotkać rano -- powiedział Salasso. -- W~międzyczasie,
wróćmy do zajazdu i~chodźmy do łóżka.

Elizabeth i~Gregor spojrzeli na niego, a~potem na siebie.

-- Co za dobry pomysł -- powiedział Gregor.

-- Tak -- odpowiedział Salasso. -- Wszyscy potrzebujemy snu.

-- Tak, -- wymamrotała Elizabeth, gdy poszli za nim -- ale nie wszyscy z~nas go dostaną.



\chapter[Inżynieria Społeczna]{18 Inżynieria Społeczna}

Unosiłem się w~słabo oświetlonym korytarzu, odpychając się sporadycznymi
dotknięciami rąk lub stóp od boków. Zielone liście roślin co chwila się
o mnie ocierały. W~szkłach miałem widok ciągle zmieniający się pomiędzy
rzeczywistością przede mną a~trójwymiarowym schematem planu znalezionym
w bibliotece stacji. Jedynymi dźwiękami, jakie mogłem usłyszeć, były
ciągły szept wentylacji i~mój własny oddech.

Przez ostatnie dwa dni badałem stację niczym nurek sondujący podwodny
system jaskiń. Nie wyjawiłem tego, kiedykolwiek ktoś mnie spotykał,
byłem wiarygodnie w~drodze dokądś lub wiarygodnie zgubiony. Byłem cały
czas pod telefonem i~często musiałem odwiedzić fabrykatory, w~rzeczywistości lub wirtualach, pomóc rozwiązać sprzeczności pomiędzy
planem a~praktycznymi aspektami konstrukcji. Reszta mojego czasu pracy
zawierała powtórzenie procedury, którą przeszliśmy dla Statku, tym razem
dla drugiego projektu: Silnika.

W pewien sposób, drugi projekt był łatwiejszy. Znalazłem większość bugów
w planie w~czasie pracy nad detalami dla Statku, a~Silnik sam w~sobie
był prostszą konstrukcją, bezpośrednią i~solidną, mniej wybredną, nawet
niż okrojona wersja Statku właśnie nabierająca kształtów w~fabrykatorach. Wymagał więcej konkretnych materiałów, włączając takie
egzotyki jak atomy czarnodziurowe, ale budowa mogła zabrać mniej czasu.
Konsultowanie interfejsu stało się łatwe i~zwykłe, a~to także wyjaśniło
sprawy i~przyśpieszyło prace.

Na końcu korytarza usłyszałem głosy. Złapałem stojak i~pozwoliłem
ramionom przyjąć wysiłek nagłego zatrzymania. Słuchając bliżej, mogłem
usłyszeć dwa głosy rozmawiające po rosyjsku, zbyt cicho, by zrozumieć i~zbyt szybko dla mnie, żebym nadążył. Jeden z~nich brzmiał męsko, drugi -
kobieco. Skok do mapy stacji pokazał wielki magazyn na prawo, choć pod
ciśnieniem, większość potrzebnej obsługi była zrobotyzowana i~nie
wydawało się to być prawdopodobnym miejscem dla przebywania ludzi.
Szczególnie że jedną z~cech magazynu było to, że jest wielkim metalowym
pudłem, klatką Faradaya, nieprzenikliwym dla promieniowania
elektromagnetycznego, a~stąd dla komunikacji szkieł.

Znowu odepchnąłem się kopnięciem, celując w~drzwi. Drzwi były, z~dobrych
powodów bezpieczeństwa, nie do zablokowania. Przesunąłem dźwignię, by je
otworzyć i~udałem, co wydawało mi się przekonujące, niezdarne wejście,
ramiona machające, gdy dryfowałem przez kilka metrów sześciennych
powietrza, zanim złapałem nogą górną krawędź przymocowanej plastikowej
skrzyni.

Ciasno zaczepieni stopami o~rzędy skrzynek, twarzą w~twarz wisieli
Aleksandra Czumakowa i~Grigory Wolkow. Spojrzeli na mnie karygodnie, jak
gdybym złapał ich na jakiś tajnych rozmowach, potem natychmiast
odzyskali spokój, pokrywając zmieszanie pobłażliwymi uśmiechami, gdy ja
ukrywałem swoją sytuację kolejnymi wymachami nowicjusza.

Aleksandrę widziałem wcześniej, prowadzącą opozycję na spotkaniach i~później przemawiającą w~imieniu zespołu na spotkaniach Drivera. Nigdy
nie widziałem Wolkowa, ale rozpoznałem go od razu. Jego słowiańskie
kości policzkowe i~krótkie jasne włosy sprawiły, że był najbardziej
fotogenicznym kosmonautą od czasu Gagarina. Pierwszy i~ostatni,
człowiek na Wenus, który ryzykował swoje życie dla chwały lądowania,
które było tylko chwałą, i~oczywiście, członek KPUE, jeden z~rosyjskich
lojalistów KP i~patriota UE.

-- Cześć Matt -- powiedział po angielsku z~doskonałym akcentem Głosu
Ameryki. -- Zgubiłeś się, czy szukałeś odrobiny pokoju i~ciszy?

Czumakowa wachlowała ręką przy uchu i~potrząsała głową. 

-- Wiem, jak to
jest. Czasem tam nie możesz usłyszeć własnej myśli.

Złapałem krawędź i~wymanewrowałem siebie w~lepszą pozycję poza zasięgiem
i nieco ponad nimi.

-- Tak, to właśnie to -- powiedziałem. -- Ale jak to się zdarza, jestem
zadowolony, że was znalazłem.

-- Problem w~fabrykatorach? -- powiedział Wolkow, zacieniając szkła, potem
je rozjaśniając. -- Ach, widzę Twoją trudność. Pracowaliśmy offline.

Moje szkła stały się offline, gdy tylko wszedłem do pokoju. Jedyny
sposób, żebyś \emph{mógł} pracować w~środku tej metalowej puszki to
offline.

-- Ach, to nie jest ten rodzaj problemu -- powiedziałem, układając się
wygodniej. -- Myślałem o~tym, co powiedziałaś na spotkaniu, Aleksandro.
Pamiętasz, Baku jeden?

-- \emph{Ten} cyrk? Pamiętam bardzo dobrze.

-- Cóż, -- westchnąłem -- wydajesz się mieć rację w~niektórych sprawach. Ta
tak zwana kampania informacyjna każdego dnia kosztuje życie ludzkie w~domu.

Czumakowa pokiwała głową. 

-- \emph{Oczywiście} ludzie wzniecają
zamieszki, gdy pojawia się plotka jako właśnie złamany sekret państwowy.

-- Tak -- powiedział poważnie Wolkow. -- Nawet tam gdzie historie są
prawdziwe, też wprowadzają w~błąd, kiedy są przedstawiane poza ich
właściwym kontekstem.

-- Prowokacje -- powiedziałem. -- Widziałem, co zrobili z~moim własnym
miastem, Edynburgiem. Ale oprócz tych, wiecie, osobistych wątpliwości,
to co mnie martwi, to, że niepokoje w~istocie wzmocnią militarystów po
naszej stronie i~ekstremistów po stronie amerykańskiej.

Wolkow kiwał głową i~się uśmiechał. 

-- Oczywiście, oczywiście -
powiedział. -- Można się spodziewać, że ekscesy tak zwanej ,,lewicy''
działają na korzyść prawicy, zarówno w~Partii i~w świecie kapitalistów.
Nie zrozum mnie źle, Matt, całkowicie zgadzam się z~eksponowaniem
militarystycznych konspiratorów, ale ta anarchistyczna kampania jest
tylko rodzajem wymówki dla zwolenników twardej linii potrzebujących jej
do represji i~może zagranicznej przygody\ldots pewnych konfrontacji, które
mogą być początkowo symboliczne, koncesji syberyjskich, może, ale takie
rzeczy mogą się wymknąć spod kontroli i~stać się naprawdę szybko realne
i brzydkie.

Czumakowa posłała mi rodzaj przyjacielskiego spojrzenia. 

-- Ale Matt -
powiedziała -- to dla Ciebie raczej nagłe nawrócenie, czyż nie? Jeżeli
zrozumiałam, jesteś członkiem związku anarchosyndykalistycznego.

-- Och, nie zmieniłem swoich poglądów -- powiedziałem. -- Wiem, że nie są
takie same jak Partii. Wiecie, jak to jest, w~mojej pracy ciągle mam
nosa przycieranego w~tych kilku obszarach, gdzie amerykańska technologia
jest ciągle przed nami. Nie jest możliwe, by nie krytykować trochę
oficjalnej polityki.

-- To bardzo zrozumiałe -- powiedział Wolkow. Zdjął swoje szkła i~uśmiechnął się krzywo. -- Wiemy, jak musisz się czuć. Dobry pracownik
docenia dobre narzędzia.

-- Właśnie -- powiedziałem. -- Ale, wiecie, dobrze jest przegadać
zmartwienia z~ludźmi,którzy, wiecie\ldots

Oboje pokiwali i~uśmiechnęli się. Jak wielu Rosjan, byli niezachwianie
przekonani, że większość prawdziwego, zwykłego ludu pracującego była
lojalna wobec socjalistycznych bratnich narodów, nawet jeżeli niektórzy
głosowali na partie inne niż Partia lub chodzili do kościoła, lub
farbowali oczy na zabawne kolory.

Ale Czumakowa obstawała przy swojej ostrożności, ciągle mnie sondując.

-- Wydaje się, że masz dużo do gadania ze swoim jankeskim pilotem -- powiedziała. -- Oczywiście, to Twoja sprawa, że tak powiem. A~według
kanałów informacyjnych, byłeś w~jakimś związku z~amerykańskim szpiegiem.

-- Tak -- odparłem, nieco się wijąc -- czuję się winny o~Jadey. Nie z~powodu Camili, ona jest\ldots przyjacielem, i~nie martwiłbym się o~nią, nie
ma jednej politycznej kości w~ciele.

-- Jestem pewien, żebyś wiedział -- powiedział Wolkow.

Zaśmialiśmy się.

-- Więc dlaczego czujesz się winny -- kontynuował Wolkow -- jeżeli w~tej
sprawie nie jesteś moralizatorski?

-- To\ldots Hm, mniemam, że to \emph{jest} moralne, lub może polityczne.
Jadey Ericson jest w~więzieniu z~mojego powodu. Nie dlatego, że została
aresztowana, a~mnie udało się uciec, a~musicie pamiętać, że mieliśmy
dobre powody do strachu, ale ponieważ jest przetrzymywana na wymyślonych
powodach. Już jest nakaz dla mnie, obraza sądu, ponieważ nie pojawiłem
się jako świadek, i~nie mogę przestać się martwić, czy nie zostanie
wykorzystana w~którymś momencie jako źródło nacisku wobec mnie.

-- Żeby co zrobić?

Wzruszyłem ramionami. 

-- Nie wiem, to mnie właśnie martwi. Tak czy siak,
zostałem zapewniony, że fakcja Reformy robi co może, żeby ją wydostać,
więc w~tym momencie nie mogę pozwolić sobie na antagonizowanie Paula.

-- Lemieux jest w~fakcji Reformy? -- spytał Wolkow.

-- Och, jasne -- odpowiedziałem. -- Nie wiedziałem, że chce trzymać to w~sekrecie. Gówno. Nie mówcie mu, że ja wam powiedziałem!

-- Nie, nie, oczywiście, że nie -- powiedział Wolkow.

-- Aha! -- powiedziała Czumakowa. -- Więc \emph{dlatego} Driver robił taki
cyrk przy tym bękarcie Weberze.

-- Kim? -- spytałem.

-- Trockistowski europarlamentarzysta, ten, który został aresztowany\ldots

-- Och, ta, racja. Pamiętam, ale, przepraszam, nie widzę związku.

-- Fakcja Reformy to grupa trockistów, w~zasadzie\ldots prawicy udającej
lewicę -- powiedział Wolkow z~pewnością człowieka potwierdzającego długo
trzymane uprzedzenia. -- Spójrz, jak przemianowali stację
\emph{Ciemniejsza Noc, Jaśniejsza Gwiazda } Po książce o~Trockim!
Śmieszne.

-- Wydaje się, że to zirytowało wielu ludzi -- powiedziałem. -- W~końcu,
Marszałek Titow był prawdziwym sowieckim bohaterem kosmosu.

-- Pierwszy spacer w~kosmosie, tak -- powiedziała Czumakowa, spoglądając z~boku na Wolkowa. -- Nie mogą nam tego zabrać.

-- Nie -- powiedziałem. -- Nie mogą. I~ciągle możemy tutaj dokonać wielkich
rzeczy.

-- Już to robimy -- powiedziała Czumakowa. -- Pierwszy Kontakt, mój Boże! I~budowa pojazdu antygrawitacyjnego! Co Jankesi za to by dali.

Odepchnąłem się i~obróciłem.

-- Och, cóż, pieprzyć politykę, ten projekt ciągle jest wart pracy.
Lepiej wrócę do niego, zanim Driver da mi reprymendę. Do zobaczenia.

-- Ta, na razie -- powiedział Wolkow, gdy wypływałem przez drzwi. Ważne
wiadomości zabłysły, gdy tylko moja głowa minęła framugę.

\threeast

-- Gdzie kurwa byłeś?

Przypiąłem pasek do taśmy i~poprawiłem szkła.

-- Och, potrzebowałem chwilę spokoju -- powiedziałem Avakianowi. -- Czasem
w kwaterach jest trochę zbyt tłoczno.

-- Ta, zdaje się, że niektórym to przeszkadza -- powiedział tolerancyjnym,
ale nie rozumiejącym tonem. -- Musisz, chłopie, na to uważać, może jakieś
leki.

-- Nie, teraz jest w~porządku -- powiedziałem. -- Teraz wiem, że są na
stacji miejsca, gdzie nie ma komunikacji.

-- Cóż, nie chodź do nich bez zostawienia wiadomości -- powiedział.

-- Dobra, dobra, to było trochę nieodpowiedzialne, powiem Ci na
przyszłość. A~teraz, gdzie jesteśmy?

-- Spójrz na to -- powiedział Avakian.

Zalogowaliśmy się do wspólnej wirtualki.

-- O! -- powiedziałem.

-- W~istocie pieprzone O! -- odparł Avakian. -- Zrobiłem to. Cóż, szczerze,
myśmy to zrobili, ale zdałem sobie sprawę, co właściwie zrobiłem kończąc
to, i~chciałem, żebyś był pierwszym widzem.

To był Silnik. Tylko w~wirtualnej rzeczywistości, oczywiście, ale to
znaczyło, że cały proces produkcji w~symulacji przebiegł poprawnie.
Lśnił na swoim gładko zintegrowanej podstawie jak kowadło z~innego
wymiaru albo zamontowany rakietowy motor z~jakiegoś muzeum w~odległej
przyszłości. Widziałem szkice, schematy 3D w~dokumentacji, ale to było
inne: superrealistyczne renderowanie jak by wyglądał po zbudowaniu. Miał
około czterech metrów długości, mniej niż metr szerokości w~najgrubszym
miejscu i~maksymalną wysokości około dwóch metrów. Mogłem sięgnąć i~dotknąć, i~tak zrobiłem.

-- Dzięki, Armen -- powiedziałem. -- Co za widok.

-- Ta -- odparł. -- Zasadniczo to jest dziwniejsze niż Statek. Widzisz te
małe cztery otwory w~rogach podstawy? Zdaje się, że powinieneś po prostu
przykręcić pieprzonymi \emph{śrubami do podłogi. } Chociaż mamy jeden
mały problem.

-- System Sterowania? -- zaryzykowałem, myśląc: \emph{Znowu!}

-- W~tym, nie ma żadnego.

-- Poczekaj chwilę -- powiedziałem. -- Był w~planach.

Przeleciałem strony. 

-- Tutaj, ta płytka, to oczywiście jest system
sterowania, ma przyciski, nawet jeżeli nie możemy użyć go bez\ldots

-- Ta, spójrz na to, jak to wyszło.

Odwrócił widok i~przybliżył na kompletnie zwyczajny czarny prostokąt na
przyczółku.

-- O, gówno.

-- Z~tego, co wiemy, -- mówił dalej -- to mogłaby być przeklęta plakietka z~nazwą, a~to, co wyglądało na przyciski w~planie mogło być równoważnikiem
nazwy firmy wygrawerowanej w~mosiądzu.

-- Dobra -- odparłem -- nie istnieje powód, dlaczego obcy daliby nam to tak
sobie. Może jeżeli zadamy pytania Interfejsowi, dostaniemy coś, co
będziemy mogli użyć.

Przygotowanie kwerendy zabrało nam resztę tego dnia. To, co wyszło, nie
było odpowiedzią, ale obrazem i~zbiorem współrzędnych na trzech osiach,
który wskazywały z~dokładnością do jednego centymetra miejsce w~obrębie
asteroidy.

-- Zdaje się, że mówią nam: idź i~otrzymaj tam odpowiedzi -- powiedziałem.

-- Ty pierwszy -- powiedziałem Avakian.

-- Och kurwa -- powiedziałem hojnie. -- Musi być ktoś, kto byłby znacznie
lepszy niż ja w~poruszaniu się w~dużym obrazie.

-- Nie o~tym mówiłem -- powiedział Avakian. -- Chodziło mi o~to, że możesz
być pierwszy, żeby powiedzieć Driverowi.

\threeast

Driver był zbyt zmęczony, żeby wybuchnąć. Nawet nie wyglądał na
szczególnie zmartwionego.

-- Nigdy nie oczekiwaliśmy natychmiastowych testów Silnika -- powiedział.
-- To w~Statku pokładamy nadzieję, że możemy go użyć. Nawet nieużyta, ale
niewątpliwie prawdziwa wersja Silnika byłaby wystarczająca, żeby ludzie
się podekscytowali. Mam na myśli, nie zrozum mnie źle, to wspaniale, że
dotarliście tak daleko i~możecie rozwiązać problem z~systemem
sterowania, jeżeli chcecie, ale niech to nie opóźnia innych tematów.

-- Dobra -- odpowiedziałem, z~ulgą i~lekkim rozczarowaniem.

-- Jutro jest wielki dzień -- powiedział. -- Przenosimy mały Silnik z~fabrykatorów do doku, a~potem montujemy go w~\emph{Bluźniercze. } To
będzie wymagać trochę spacerowania w~kosmosie. Michaił, jak Twoje
dziewczyny i~chłopcy?

Telesnikow, fizycznie obecny, pokazał kciuki do góry.

-- Jesteśmy gotowi -- powiedział. -- Prawdę powiedziawszy, jesteśmy dość
skłonni wykonać całą zmianę jako EVA, wyjąć silnik z~drzwi fabrykatora i~przetaszczyć go dookoła do \emph{Bluźnierczych}, zamiast poruszać się
korytarzami. Będzie to bardziej bezpośrednie, z~jednej strony, a~z
drugiej, wiemy, że silnik radzi sobie z~próżnią, jest już w~próżni w~fabrykatorach, ale nie wiem, jak sobie poradzi z~ekspozycją na
biologiczne.

-- To nie taki zły pomysł -- powiedział Driver.

Telesnikow wyszczerzył zęby. 

-- Ta, tak oczywisty, że wolałbym sam o~tym
pomyśleć.

-- Kto to wymyślił? -- spytałem.

-- Grigory Wolkow.

Przełknąłem mocno.

-- Hm, możemy przedyskutować to dalej przez minutę?

Driver spojrzał spod brwi. 

-- Minuta.

-- Dobra, -- powiedziałem -- wiem, że nie jestem specjalistą od prac w~kosmosie, ale znam tę maszynę, którą zbudowaliśmy na tyle, o~ile
ktokolwiek zna bez jej zrozumienia, i~przysięgam, że jest całkowicie
odporna na skażenie biologiczne. Znaczy, no, każda ruchoma część jest
hermetyczna. System sterowania jest nasz własny i~wiemy, jak jest
odporny. Natomiast, hm, bez obrażania Twojego zespołu, Michaił, ale im
dłużej trwa operacja w~EVA, większa szansa na wypadek. Jedno potknięcie
i moglibyśmy wysłać tę rzecz w~kosmos i~stracić ją na dobre.

Telesnikow pomachał dłonią.

-- Będzie w~siatce, cały czas przymocowana -- powiedział. -- Nie ma kwestii
niebezpieczeństwa.

-- Liny mogą pęknąć -- powiedziałem.

-- Nie te liny -- powiedział Telesnikow. Uśmiechnął się do mnie
uspokajająco. -- Specyfikacja NASA. I~mamy najbardziej doświadczonego
operatora EVA w~Układzie Słonecznym, o~ile wiemy!

-- Kogo?

-- Grigory, oczywiście. -- Jego oczy nagle się rozszerzyły. -- Och,
rozumiem! Mogłeś słyszeć, że ESA przydzieliła go tutaj z~powodu
prestiżu, ale to jest rodzaj zazdrosnej plotki, która się rozpowszechnia
przez biurokrację. Żaden kosmonauta w~to nie wierzy. Grigory to nie
tylko ładna twarz.

-- Ale\ldots

Driver uniósł rękę. 

-- Miałeś swoją minutę, Matt. Robimy to w~przestrzeni
kosmicznej, całą drogę. Następna sprawa.

Camila szarpała mnie za ramiona.

-- Matt! Obudź się!

-- Co?

-- Twój \emph{bip} działa. Nie słyszysz? Wszyscy wokół słyszą to
cholerstwo!

Obudziłem się i~wygrzebałem czytnik i~szkła. Kiedy wyłączyłem
\emph{bipa} stało się jasne, że nie był jedynym grającym w~pobliżu.
Ucichały jeden po drugim, gdy zakładałem szkła na zaspane oczy.

-- Włączysz mnie? -- spytała Camila.

-- Ta, jasne -- otworzyłem kanał do jej szkiel, podczas gdy raport unosił
się przede mną.\\
\\
WIADOMOŚĆ OSOBISTA:\\
Otwierający kadr ktoś pakowany po schodach do United Airlines 777.
Odejście i~śledzenie: dwie szkockie policjantki trzymają Jadey. Wydaje
się walczyć, ale w~teatralny, pro forma sposób. Na szczycie schodów
puszczają ją i~tylko popychają. Potyka się, łapie za framugę i~odwraca.\\
Podnosi wyprostowane ramię z~wyprostowanym palcami wskazującym i~środkowymi oraz kciukiem, obecna wersja wyzywającego salutu.

-- Śmierć komunizmowi! -- krzyczy i~wycofuje się do samolotu.

POWIERZCHNIA:\\
Amerykański szpieg Jadey Ericson została zwolniona o~północy i~jest
teraz na pokładzie samolotu do Stanów Zjednoczonych. Zarzut morderstwa
został anulowany w~świetle nowych dowodów. Pani Ericson przedstawiła
wyjaśnienia ujawniające antyeuropejski i~antysocjalistyczny spisek, w~którym była pionkiem.\\
GŁĘBIA:\\
Wywrotowa Federacja Praw Człowieka, finansowana częściowo przez firmę
przemysłu zbrojeniowego Nevada Orbital Dynamics, która ostatnio wysłała
pomoc dla zbuntowanej stacji kosmicznej \emph{Marszałek Titow}, jest
powiązana z~faszystowskimi i~nihilistycznym grupami kryjącymi się za
przemocą ostatnich kilku dni oraz z~buntownikiem i~agentem CIA, Colinem
Driverem. Partnerem Drivera w~koterii, która tymczasowo przejęła stację,
ku przerażeniu jej uczciwych naukowców i~kosmonautów jest Paul Lemieux,
zidentyfikowany jako przedstawiciel grupy ,,Reformy'' w~KPUE, który był
członkiem trockistowskiej LPR w~okresie studiów w~Lausanne. Dalsze
wpływy CIA na elementy fakcjonalistyczne w~KPUE i~w trockistowskiej LPR
zostały jasno przedstawione w~związku ze śledztwem powiązań byłego Posła
Parlamentu Europejskiego Henri Webera. Motywacja obecnej kampanii
dezinformacji i~roszczeń dostępu do ,,technologii okrętów obcych'' wydaje
się wzmacniać fakcję ,,Reformy'' w~KPUE, która przedstawia się w~UE jako
popularny, demokratyczny prąd oraz międzynarodowo jako jedyni Komuniści
gotowi i~zdolni ,,robić interesy'' ze Stanami Zjednoczonymi. Demagogiczna
i sprzeczna natura tej ,,platformy'' winna być oczywista.\\
ANALIZA:\\

W tym momencie wyłączyłem. Ramię Camili obejmowało mnie.

-- To wspaniałe informacje, Matt! Jadey jest wolna! Super!

-- Ta, dzięki -- powiedziałem. -- To wspaniale, duża ulga. Ale, kurde,
mówią, że wyznała wszystkie te rzeczy\ldots

-- Ach, bzdury -- powiedziała Camila. -- Nikt w~to nie uwierzy! Szczególnie
kiedy sami twierdzą, że była pionkiem. Skąd mogłaby o~tym wiedzieć?

-- Nie mogłaby -- przyznałem. -- A~o tyle o~ile wiem, nie była.

-- To wszystko i~tak komuszy paranoidalny szwargot.

Zdjąłem szkła, potarłem oczy i~popatrzyłem na nią w~słabym świetle
naszej alkowy.

-- Nieprawda -- powiedziałem. Modulo komiczna proza prasy partyjnej, to
jest dokładnie to, co Driver i~Lemieux powiedzieli nam tamtej nocy.

-- Tak?

-- Przyzwyczailiśmy się do robienia tłumaczeń.

Złapałem ją i~przytuliłem, tylko i~wyłącznie dla pocieszenia.

-- Nie wyglądasz na szczęśliwego.

-- Jestem szczęśliwy -- powiedziałem. -- Boże, czuję taką ulgę, że mógłbym
płakać. Ale ciągle gramy w~niebezpieczną grę.

-- Ta, masz rację -- potarła moje plecy. -- Wracaj spać. Jadey powinna być
rano w~domu. Obudzę Cię, kiedy nadejdą nowe wiadomości.

Wsunąłem szkła z~powrotem i~rozprostowałem palce w~ich podczerwonej
wizji.

-- Nie ma potrzeby -- powiedziałem. -- Ta rzecz wystarczy.

Zanim mogłem je zdjąć i~wrócić do snu, błysnęła przychodząca wiadomość.
Zaakceptowałem i~przystojne twarz Grigory Wolkowa zajęła pole widzenia,
jak plakat na ścianie sypialni nastolatka.

-- Myślę, że gratulacje są w~porządku, Matt, -- powiedział, uśmiechając
się. To było sprytne zdanie, nikt podsłuchujący nie zgadłby, że on się
pytał, raczej niż oferował.

-- Tak -- odparłem. -- Gratulacje dla wszystkich. Dzięki Grigory.

\threeast

Kiedy czytnik obudził mnie wiadomościami o~wylądowaniu Jadey, na
Lotnisku McCarran, Las Vegas, Camili już nie było. Była zajęta w~doku
pod \emph{Bluźnierczymi} od początku dziennej zmiany. Lekko urażony, że
nawet nie powiedziała ,,do widzenia'', ani nie zbudziła mnie porannym
tuleniem, umyłem się, ubrałem i~udałem się do jadalni. Nad śniadaniem
(solone mięso zająca i~kanapka z~jajkiem, nie rekomenduję) sprawdziłem
wiadomości, myśląc w~tyle głowy o~tym, że przyzwyczaiłem się porannego
tulenia Camili. Ale to Jadey miałem w~myślach. Gdy tylko wydostała się z~przestrzeni powietrznej UE, wydała oświadczenie odrzucające zeznania,
zaprzeczając im i~ośmieszając ich zawartość, jak to podsumowano.

Większość komentarzy, które śledziłem, wydawała się z~nią zgadzać, ale
nie zgadzać w~kwestii czy była to kompletna fałszywka, czy raczej ta
sprawa była oparta o~jakieś zaszyfrowane wiadomości, które FBB (lub
jakiś haker) złamało, a~które były ujawniane w~ten sposób, żeby nie
narazić prawdziwych źródeł. Dodając do wykrętnego zamieszania całej
labiryntowej sprawy, najbystrzejsi analitycy, czy w~\emph{Europa Prawda}
czy \emph{Daily Web}, wskazywali, że FBB bez wątpienia preferowała grupę
Reformy i~że CIA zwykle trzymała się z~dala od popierania gwałtownej
opozycji w~socjaldemokracjach, znacznie częściej chcąc wywrzeć wpływ na
FBB\ldots która, oczywiście, sama\ldots

Wyłączyłem. Świat stał się wielkim trawiastym pagórkiem, czołgając się
za strzelcem, który myślał, że tamci są Komisją Warrena. Moje własne
podejście do sprawy było takie, że moja mocna wskazówka Grigoriemu
poprzedniego dnia poprowadziła go do przekonania, że Jadey jest
przetrzymywana jako karta przetargowa przez fakcję Reformy, i~że
wypuszczenie jej pomogłoby jego sprawie, tej prostej centrystycznej
fakcji, konserwatywnej, ale nie wprost reakcyjnej jak militarystyczni
zwolennicy twardej linii. Zobaczymy.

Przesłałem wiadomość telefoniczną Jadey przez skrzynkę stacji. Nie była
online, ale wiadomość poszła do Nevady. Po dobrym łyku kawy, ze
śniadaniem układającym się do dłuższego pobytu w~żołądku, udałem się do
fabrykatorów.

\threeast

Jednostka fabrykatorów zajmowała oddzielne skrzydło stacji. To była moja
pierwsza wizyta poza wirtualną rzeczywistością. Za którą, po obijaniu
się przez tuzin śluz, drzwi i~pomieszczeń dekontaminacyjnych, byłem
całkiem wdzięczny.

Pokój kontrolny był pełen przynajmniej dwudziestu osób, oprócz pięciu
operatorów, którzy litościwie byli w~stanie ignorować to wszystko w~ich
szkłach i~rynsztunku całego ciała. Większość ludzi była mi znana tylko
jako nazwiska, które pojawiały się w~mojej przestrzeni roboczej, lub
wywoływały mnie do ich, przy problemach konstrukcyjnych. Driver i~Lemieux byli na przodzie, Czumakowa koło nich. Avakian unosił się z~tyłu. Przepchnąłem się i~udało mi się znaleźć miejsce z~dobrym widokiem
do przodu.

Jednostka fabrykatora stała za grubą przegrodą ze szkła laminowanego
diamentami. Wielokrotne, wielorako podzielone, ramiona robotyczne
fabrykatorów, najbliższa realizacja robota gąszczu Moraveca, jeżyły się
i strzelały iskrami. W~końcówkach ich najmocniejszych palców trzymały
silnik i~system sterowania Statku.

Pomimo mojej znajomości z~nimi w~VR, było coś poruszającego w~zobaczeniu
go w~rzeczywistości, z~realnymi fotonami, które po prostu odbijały się
od niego przed wpadnięciem w~moje oczy. Chciwie wchłaniałem widok, który
naprawdę był niczym innym jak bulwą z~gładkiego metalu, płaską podstawą,
połączonym trzymetrowymi przewodami do bloku styropianowej okładziny, o~której wiedziałem, że zawiera panel i~tablicę przekładni.

Zewnętrzne drzwi fabrykatora już rozsuwały się, by ukazać prostokąt,
dziesięć metrów na pięć metrów, czerni. Ten obraz szybko został
zapełniony dwoma kosmonautami w~kombinezonach EVA, rozstawiającymi ledwo
widoczną sieć dookoła framugi. Kable ciągnęły się za nimi. Ruchy lin i~sieci w~próżni i~mikrograwitacji były dostatecznie inne od znanych, że
dały mi poczucie niepokoju.

Lub by zapewnić skupienie dla niepokoju, który już czułem. Włączyłem
szkła do kanałów komunikacyjnych i~słuchałem kosmonautów i~załogi pokoju
kontrolnego. Rozmawiali po rosyjsku i~ani moje umiejętności językowe,
ani te szkieł, nie były w~stanie dużo wydobyć.

Mechaniczne palce drgnęły i~rzecz popłynęła, by zostać złapana w~sieć.
Otwór sieci został zamknięty prostą liną i~sieć została wyciągnięta na
prawo, poza widok.

Przełączyłem się do zewnętrznej kamery i~obserwowałem siatkę i~jej
zawartość przywiązywaną do prostych sani, odpowiednik wózka widłowego w~nieważkości. Sanie składały się z~kilku metrów sześciennych kraty ze
zbiornikiem paliwa pod nim. Na każdym końcu, dziób i~rufa, były
zamontowane cztery silniki odrzutowe, sterowanie i~stopień dla pilota, z~dyszami silników bezpiecznie za nim. Torba była teraz wolna, tylko
przypięta do sani. Sanie, z~rodzajem zabezpieczenia pasami i~szelkami, o~których mówił Telesnikow, same były na uwięzi. Długi luźny kabel z~jednej strony zamocowany na fabrykatorze z~drugiej na, w~odległości
trochę poniżej pięciuset metrów, \emph{Bluźnierczych Geometriach},
przechodził przez dwa mocne metalowe półpierścienie wystające z~boku
sanek. Pięciu kosmonautów, rury osobistych pakietów rakietowych
zakrzywione z~ich ramion jak zarysy skrzydeł anielskich, było
rozmieszczonych wzdłuż ścieżki sanek.

Pilot sanek odpalił na krótko silniki i~holownik ruszył z~małą
prędkością do przodu w~prostej linii. Minął dwóch kosmonautów i~był w~połowie drogi do celu, gdy coś poszło źle.

Lina się splątała i~przestała przesuwać się przez pierścienie. Nagle
zatrzymane sanki ruszyły do przodu i~w tym samym momencie przednie
silnik zaczęły błyszczeć, znacznie bardziej intensywnie niż te z~tyłu.
Sanki wystrzeliły do tyłu i~od powierzchni asteroidy, natychmiast
rozciągając linę do spłaszczonego V. Gdy zrobiłem zbliżenie, stało się
jasne, że lina utkwiła nie tylko wzdłuż boku, ale także z~przodu
skrzyni. Tak nagle, jak utknęła, lina pękła po obu stronach sań, które
uniosły się pod kątem, silniki uruchomione jeszcze przez kilka sekund.
Kiedy przestały pracować, sanie były poza nawet powiększeniem szybko
śledzącej kamery.

Wszyscy w~pokoju albo krzyczeli, albo milczeli w~szoku. Kanał
komunikacyjny kosmonautów był spokojny. Dyscyplina została utrzymana.
Usłyszałem głos pilota sanek w~trzeszczącym, wstrzymywanym rosyjskim:

-- Paliwo sanek wyczerpane, sanki koziołkują.

Wolkow powiedział: 

-- Mamy Cię na radarze. Odskocz, ustabilizuj własnymi
rakietami, zmniejsz prędkość odejścia na tyle, o~ile się da, a~my cię
podejmiemy.

-- \emph{Niet.}

-- Na litość boską, Andrea! Zostaw to!

Odpowiedź przyszła, ciągle trzeszcząc, po angielsku.

-- Tu nie Andrea, tu Camila, i~nie zamierzam tego zostawić.

Wrzasnąłem, słysząc to, kompletnie daremny ryk, ponieważ nie mogłem go
wysłać do kanału komunikacyjnego. Dyscyplina komunikacja wydawał się
łamać, w~nagłym bełkocie. Przez kamery widziałem kosmonautów
przelatujących, do siebie, lub od siebie.

Głos Camili przebił się raz jeszcze, cichszy.

-- Czekajcie -- powiedziała. -- Wracam z~tym.

Przełączałem gorączkowo pomiędzy kamerami, aż znalazłem tę skierowaną na
zewnątrz. Na tle gwiazd, punkt błyszczał jak niebieska nowa, wolno
jaśniejąc i~stając się niewyraźny. W~ciągu sekund, był w~pełnym obrazie,
pędząc prosto na nas. Kamera zmniejszyła przybliżenie i~ustabilizowała
obraz, mogłem zobaczyć sanki i~ich pilota w~niebieskiej chmurze.

Camila doprowadziła to aż do drzwi, a~potem się zatrzymała. W~jej rękach
był panel sterowania silnika. Kawałki zniszczonego styropianu wirowały
dookoła niej jak gdyby w~atmosferze, widok tak rażąco nie możliwy, jak
jej widowiskowy nie-Newtonowski powrót. Zamachała, przesunęła sanki na
bok, by doprowadzić do gwałtownego zatrzymania koło jej własnego statku.

-- Przelot EVA zakończony -- powiedziała. -- Nieplanowany test lotu
zakończony. Silnik i~sterowanie nominalne.

W tym samym czasie, aresztowania, także, były zakończone.

\threeast


Lemieux przykucnął zwyczajowo w~górnym rogu, ćwicząc nową, irytującą i~niebezpieczną sztuczkę. Ustawiał swojego Aerospatiale 9mm w~powietrzu,
potem uderzał w~koniec lufy, wprawiając broń w~obrót i~pozwalając jej
nieco odpłynąć. Potem łapał ją z~orbity. Raz za razem. Wydawało się, że
nie zwraca uwagi na nic innego. Bez przycinania paznokci nożem bojowym i~gwizdaniem przez zęby, nie mógłby mniej subtelnie przedstawić
niestabilności i~groźby.

Driver, w~międzyczasie, nieudanie grał rolę dobrego policjanta, łącznie
ze sporadycznymi zatroskanymi spojrzeniami na Lemieux. Spoczywał za
biurkiem, przed którym Czumakowa i~Wolkow wyglądali jak gdyby stali na
baczność, ich ramiona zaczepione o~taśmy. Nie byli związani, to wszystko
było bardzo cywilizowane, prócz rutyny Lemieux.

Wisiałem po jednej stronie, zablokowany o~jakieś półki, Camila unosiła
się przy drzwiach. Prawie wszyscy na stacji, oprócz czterdziestu siedmiu
w areszcie w~różnych improwizowanych miejscach, oglądali przedstawienie
w szkłach.

-- No dalej towarzysze -- powiedział Driver. -- Gdyby to było przeklęte
śledztwo powypadkowe NASA, mógłbym uwierzyć, że to, co się zdarzyło,
było przesadzonym systemem bezpieczeństwa, który błędnie zadziałał.
Jakaś chemiczna degradacja w~kablu, która spowodowała jego lepkość i~kruchość, nieprzewidywalne chlupotanie w~silniku sanek, wypalenie
paliwa. Takie rzecz się zdarzają, prawda?

Wolkow wzruszył ramionami. 

-- Tak nam mówią. Czasem jest błędem opierać
się na technologii amerykańskiej i~procedurach NASA zamiast na naszych
własnych umiejętnościach.

-- Tak -- powiedział Driver. -- Ale to był Twój pomysł, prawda?

-- Pomysł nie był zły -- powiedział Wolkow. -- Jeżeli sugerujesz sabotaż,
to jest śmieszne. Myślałem, że na sankach była Andrea Barsowa. Nie
zaryzykowałbym życia kosmonauty. Wiesz o~tym, Colin.

-- Ale nie ryzykowałeś życiem kogoś -- powiedział Driver. -- Barsowa jest
doświadczonym operatorem sanek, mogłeś oczekiwać po niej odskoczenia na
silnikach przy pierwszej oznace kłopotów.

-- To wszystko spekulacje -- powiedziała Czumakowa.

-- Nie -- powiedział Driver. -- Wiemy, że jesteście w~kontakcie z~osobami w~Partii i~w rządzie. Po waszej rozmowie z~Mattem, przekazaliście
informacje kontaktowi w~Brukseli. Komuś całkiem wysoko w~administracji.
W ciągu godzin, Jadey Ericson została zwolniona i~fałszywe zeznanie
zostało ujawnione, nazywające mnie agentem CIA i~tak dalej. Nie sądzę,
żeby to był przypadek i~nie sądzę, że ludzie, z~którymi natychmiast się
skontaktowałeś w~obrębie stacji, tylko zdarzyli się być kolegami.

-- Skąd o~tym\ldots? -- Wolkow przerwał i~spojrzał na mnie.

-- Dobrze -- powiedział. -- Więc Matt powiedział wam, a~wtedy wy
wyśledziliście wszystkie kontakty, jakie potem mieliśmy. Co z~tego? To
nie zbrodnia.

-- Niektórzy ludzie zaczęli mówić i~przyznali, że to była więcej niż
rozmowa -- powiedział Driver.

Wolkow roześmiał się.

-- Nie złapiesz mnie czymś takim.

-- Może nie -- przyznał Driver. -- Ale złapiemy cię nagraniami.

Czumakowa poruszyła się konwulsyjnie. Lemieux przestał obracać pistolet
i go przeładował. Driver spojrzał na niego z~lękiem.

-- Spokojnie, Paul -- powiedział. -- Aleksandra, mówiłaś coś?

-- Nic, co zrobiliśmy, nie jest przestępstwem! Cenimy naszą pracę i~nie
pozwolimy na przekazanie jej Amerykanom! Jesteś szpiegiem i~ohydnym
zdrajcą, Colinie Driver, a~kiedy porządek zostanie przywrócony,
zostaniesz rozstrzelany.

-- Zaryzykuję -- powiedział Driver. -- Teraz, proszę was o~wyjście i~towarzyszenie innym w~areszcie.

Wolkow spojrzał na mnie ze wstrętem, potem wzruszył ramionami i~kiwnął
głową.

-- Bardzo dobrze -- powiedział. -- To honor. Nie będziemy zbyt długo się
nim cieszyć.

-- Co masz na myśli? -- powiedział Lemieux.

-- Wszyscy, spójrzcie na wiadomości -- powiedziała Czumakowa nad
ramieniem. -- Porządek jest przywracany.

\chapter[Pierwszy Nawigator]{19 Pierwszy Nawigator}



Elizabeth, siedząc okrakiem na jego biodrach, pochyliła się do przodu,
włosy kołyszące, policzek odbijający słabe światło, i~pogładziła palcami
jego sutek. -- Z~czego się śmiejesz?

Sięgnął i~odwzajemnił przysługę. Piersi były miękkie i~gładkie, sutki
tak twarde i~szorstkie i~większe niż jego dwa małe czubki. Zastanawiał
się, w~jakoś oderwany sposób, czy jej przyjemność przy tej manipulacji
była większa w~podobnym stopniu. Jeżeli była, zazdrościł jej.

-- Śmieję się ze mnie -- powiedział. -- Byłem głupcem.

Jej włosy opadły szerokim wachlarzem na jego pierś.

-- Byłeś, Cairns, ale nie tak wielkim głupcem jak ja.

Jego dłoń była w~jej włosach, kolejny cud. Miał nadzieję, że jego
odpowiedź będzie tak niewyczerpana, jak bodźce, tak ich wiele, tak dużo
dżungli i~oceanu, gór i~wzgórz, białe plaże jej pleców, cała niekończąca
się planeta jej ciała, płonące ciemnie niebo jej umysłu. Świat, który
badał godzinami, i~przez który był badany.

-- Nie wiem, czy to tym razem zadziała -- powiedział.

Jej język zrobił coś szokująco sprytnego z~jego napletkiem w~ramach
odpowiedzi. Eksperyment, który odrzucił jego hipotezę zerową. Była
biologiem i~dobrze znała swój przedmiot.

\threeast

Trzeci blok, Nabrzeże 4, Ferman i~synowie. O ósmej rano, nabrzeże było
ohydnym miejscem, wiatr znad morza niósł smród od klifów, a~bliżej,
chemiczne fetory z~zużytych chłodziw i~ostrych odkażalników
statków-fabryk. Pod nogami odłamki kości i~śliska mieszanka mineralnych
i zwierzęcych olei. Pojazdy transportowe zgrzytały i~dudniły na bruku.
Pośród dokerów i~marynarzy zaur i~dwoje ludzi byli niepozorni. Znaleźli
kawiarnię nadwodną naprzeciwko wejścia do biura i~czaili się nad stołem
przy zaparowanym oknie. Elizabeth i~Gregor przegryzali się przez kanapkę
z wędzoną rybą. Salasso wybrał paski wołowiny. Gregor pilnował, ciągle
wycierając rękawem okno.

-- Lipidy koloidalne zawieszone w~kroplach wody uformowanych wokół
cząsteczek dymu -- powiedział. -- Można napisać pracę magisterską o~tym
miejscu nawet bez wchodzenia w~biologię.

-- Weź kolejną kawę -- powiedział Salasso. -- Twój mózg jest w~trakcie
wczesnych konsekwencji deprywacji snu.

Gregor ziewnął i~przytaknął, uśmiechając się do Elizabeth, gdy Salasso
pokazał trzy palce kelnerce. Kawiarnia była pełna pracowników fizycznych
jedzących późne śniadanie i~pracowników biurowych lub biznesmenów
jedzących wczesne. Większość z~nich była ludźmi, prócz dokera giganta i~kilku zaurów.

-- Ten człowiek Wolkow -- powiedział Salasso, gdy kelnerka przyniosła im
dolewki. -- Odniosłeś wrażenie, że znał Matta Cairnsa?

-- Och, zdecydowanie. Zresztą, był z~człowiekiem, który mnie pomylił, od
tyłu, z~Mattem.

-- Więc wiemy, że Twój przodek ma podobne włosy do Twoich i~może podobną
sylwetkę i~postawę -- powiedział Salasso. -- To może być pomocne, ale
wolałbym, żebyś mógł więcej porozmawiać z~tym człowiekiem.

-- Prawdę powiedziawszy -- powiedział Gregor -- byłem tak wstrząśnięty
spotkaniem z~Wolkowem, że ten inny mężczyzna wydawał się mniej ważny. I~byłem ostrożny, ponieważ wiemy, że \emph{oni} są ostrożni. Nie chciałem
zasypać go pytaniami.

-- Nawet jeśli\ldots

-- Spójrz -- powiedziała Elizabeth, uśmiechając się nad stołem do obu --
fakt jest taki, że odwróciłam uwagę Gregora od jego badań. Nie bądź dla
niego zbyt surowy.

-- Jestem z~was zadowolony -- powiedział Salasso -- ale ten romans zdarzył
się w~dziwnym momencie. A~teraz oboje cierpicie przez deprywację snu.

Gregor nie przestawał uważać na zamazany widok zza okna. Wspomnienie
nocy z~Elizabeth wydawało się wdrukowane w~każdą część jego skóry, a~wszystkie jej krzywizny i~kąty zapamiętane w~jego dłoniach.

-- Nie nazwałbym tego ,,cierpieniem'' -- powiedział. -- A~skoro jesteśmy w~sprawie dziwnych czasów, Ty sam byłeś\ldots

-- To jest to -- powiedział Salasso. -- Ale konsekwencje moich osobistych
związków były \emph{szczęśliwe}.

Gregorowi to zabrzmiało nietypowo obronnie. Jakiekolwiek emocje były
wplątane w~najwyraźniej wiekowym romansie Salasso mogły być tylko
intensywne. Zdecydował się nie naciskać w~tej kwestii.

-- W~każdym razie, Wolkow -- powiedział. -- Nie był tak chętny powiedzieć
Marcusowi, kim jest, więc nie sądzę, że sprzeda jakiekolwiek sekrety
kupcom.

-- To dlaczego tutaj przychodzi? -- powiedziała Elizabeth.

-- Zakładając, że przyjdzie\ldots Nie powiedział jasno, że przyjdzie. Może
po prostu chce załatwić umowę na smary do silników morskich.

-- Dzieje się więcej niż tylko to -- powiedział Salasso. -- Jestem
irracjonalnie tego pewny.

Gregor oparł policzek o~wilgotne szkło, nie dla zmysłów, tłusty dotyk
był dość niemiły, ale żeby zobaczyć ulicę dalej do końca nabrzeża. Zegar
na ścianie kawiarni pokazał wpół do dziewiątej.

-- To jest jedna z~rzeczy, które lubię o~was -- powiedział leniwie. --
Ludzie nie nazywają swoich pewników ,,irracjonalnymi'' szczególnie kiedy
są.

-- Racjonalność jest warta ambicji -- powiedział Salasso. -- Dla waszego
gatunku.

Gregor ciągle się chichotał, kiedy rozpoznał człowieka idącego powoli,
nieco dalej po drugiej stronie nabrzeża, zatrzymującego się od czasu do
czasu, by spojrzeć na drzwi i~znaki.

-- Nie wstawajcie -- powiedział -- ale właśnie zauważyłem Matta Cairnsa.
Poczekajcie tutaj.

Gregor wstał i~był już za drzwiami, zanim ktokolwiek mógł się
sprzeciwić, ledwo pamiętał, by spojrzeć w~obie strony przed przejściem
drogi.

\threeast

Mężczyzna stał na chodniku przy drzwiach trzeciego bloku, patrząc na
nazwy firm wypisane koło przycisków dzwonków. Właśnie podnosił niepewny
palec do nich, gdy zauważył nadchodzenie Gregora i~się odwrócił.

Gregor wpatrzył się w~niego sparaliżowany. Jedyną rzeczą w~nim, która
wyglądała staro była jego kurtka, skóra dinozaura znoszona tak miękko,
że zwisała jak tkanina. Pomimo tego, co wiedział, podświadomie
oczekiwał, że jego przodek będzie wyglądał staro, obraz młodego
mężczyzny w~Zamku ciążący do linii cechujących Jamesa. Nawet poznanie
Wolkowa nie wyparło tych założeń. Twarz tego mężczyzny wyglądała
młodziej niż ta, którą Gregor widział niewyraźnie w~lusterku do golenia
kilka godzin temu. Nie zdradzała ani rozpoznania ani zaskoczenia.

-- Mogę w~czymś pomóc? -- powiedział mężczyzna.

Gregor wypalił pierwsze pytanie z~głowy. 

-- Czy Wolkow przysłał cię
tutaj?

-- Wolkow? Gówno!

Mężczyzna natychmiast się odwrócił i~odszedł, w~górę nabrzeża ku ulicy.
Gregor pośpieszył się, żeby go dogonić.

-- Przepraszam -- powiedział. -- Nazywam się Gregor Cairns\ldots

-- Wiem, jak się nazywasz -- powiedział mężczyzna. -- I~podziękuję ci,
jeżeli nie wypowiesz mojego nazwiska.

Gregor prawie się potknął. 

-- Co?

-- Zamknij się i~idź dalej, a~może wydostaniemy się z~tej pułapki.

Dotarli do skrzyżowania nabrzeża i~ulicy, zanim mężczyzna się trochę
uspokoił. Stanął tyłem do rogu Bloku 1, gdzie mógł obserwować wszystkie
trzy możliwe podejścia.

-- Dobrze -- powiedział. -- O co chodzi?

-- Miałem właśnie o~to zapytać\ldots

-- W~porządku. Zeszłej nocy usłyszałem o~Twoim śledztwie i~kupców. -- Jego
wzrok ciągle się przesuwał, gdy mówił, dając niepokojący efekt. -- I~usłyszałem, że kupcy spotykają kogoś u Fermana około dziewiątej. Nie
wiedziałem, że Wolkow stał za tym, żebym się tego dowiedział. Ktoś
otrzyma niezłe baty za te małe przeoczenie.

-- Wolkow\ldots

-- Cholernie mnie nienawidzi. Nie, żeby chciał wsadzić mi nóż, ale
cokolwiek, na co mnie chce wsadzić, jest wątpliwe, żeby było zabawą. --
Spojrzał na Gregora po raz pierwszy. -- Czego szukasz?

-- Mieliśmy nadzieję, że masz trochę starego sprzętu ze statku.

-- Po co?

-- Nawigacja.

Odpowiedzią był grubiański śmiech.

-- Co jest takiego śmiesznego? -- Gregor uważał sposób bycia mężczyzny tak
irytującym jak jego zmienne spojrzenie i~zaczynał samemu się
niespokojnie rozglądać. Ulica była nieprzyjazna w~świetle dnia, ruch
lekki, chodnik zastawiony trzepoczącymi zadaszeniami, zwykłymi stolikami
i pozostałościami po zamknięciu rynku. Nabrzeże było głośne od pisków
metalu i~syku gumy na bruku.

-- Ja będę się rozglądał -- powiedział mężczyzna. -- Ty patrz na mnie i~powiedz mi, co widzisz.

-- Widzę Matta C\ldots

-- Jak mówiłam. Zamknij kurwa \emph{ryja}. Drugie. Tak, jestem Matt. Matt
Spencer. Odnoga rodziny. Interesujące podobieństwo, nieprawdaż?

-- Znaczy \emph{nie} jesteś\ldots

-- Tak, oczywiście, że jestem przeklętym nawigatorem. To jest znacznie
bardziej wartościowe dla kupców niż cokolwiek z~nawigacji. Mają
nawigację. Nie mają tego.

-- Ach -- powiedział Gregor. -- To właśnie powiedział Salasso.

-- Twój zaurzy kumpel złapał to, prawda? Dobrze dla niego. Jak znam
Wolkowa, pomyślał o~tym samym i~cholernie się upewnił, że jeżeli ktoś
pojawi się na spotkaniu z~kupcami, to nie będzie on.

-- Czy spotkanie kupców może być takie niebezpieczne?

Wzrok Matta znowu się na nim skoncentrował.

-- Chciałbyś się dowiedzieć?

\threeast

Gregor poszedł z~powrotem nabrzeżem w~kurtce ze skóry dinozaura, z~której kieszeni usunięto interesującą kolekcję broni. Wyobrażenie
dodatkowego kilograma wagi na ramionach pomogło mu przyjąć krok i~pozę
Matta mniej więcej poprawnie. Oparł się pragnieniu spojrzenia na
kawiarnię.

Blaszane drzwi bloku stały otwarte na betonowy korytarz kończący się
schodami kręconymi. Poza schodami kolejne drzwi stały otwarte na wąski
brzeg nabrzeża. Sprawdził zblakłe etykiety przyklejone koło dzwonków:

\emph{Ferman i~synowie, trzecie piętro. Inż. Morscy.}

Wspiął się po trzech biegach schodowych i~dotarł lekko oszołomiony.
Wielkie drzwi z~nazwą firmy na mosiężnej tabliczce były lekko uchylone.
Otworzyły się po delikatnym pchnięciu. Za kilkoma metrami poplamionego
dywanu stało ogromne drewniane biurko. Piktyjka siedząca przy nim
spojrzała i~uśmiechnęła się.

-- Dzień dobry -- powiedziała. Spojrzała na otwarty dziennik. -- Jesteś
umówiony?

Głowa Gregora opadła, a~ramiona bezwiednie zgarbiły się, gdy wszedł do
biura, przerobionego magazynu, na otwartym planie, podzielonego do
wysokości głowy. Klawiatury stukały, rozmowy szumiały. Wąskie okna od
podłogi do sufitu wyglądały na zatokę. Nikt nie czekał po drugiej
stronie drzwi.

Ciągle się rozglądając, stanął przed biurkiem.

-- Dzień dobry -- powiedział. -- Nie jestem umówiony, ale jestem tutaj na
spotkanie z~Grigorym \emph{Antonowem}.

-- Inżynier Antonow powinien się pojawić lada chwila -- powiedziała
recepcjonistka. Wzięła długopis. -- A~Twoje nazwisko?

-- Cairns.

Zanotowała nazwisko, potem pofalowała dłonią, długie palce, długie
paznokcie, w~kierunku skórzanej sofy na jej lewo.

-- Proszę, usiądź.

-- Dziękuję.

Usiadł na krawędzi, pięści w~pustych kieszeniach, a~potem zmusił się do
rozwalenia na sofie, jeżeli nie relaksu. Po minucie wszedł Wolkow.
Właśnie go mijał, kiedy zauważył Gregora kątem oka, wtedy odwrócił się
bystro. Krawędzie dłoni pojawiły się przed nim jak noże, kolana ugięte.
Potem wyprostował się i~wycofał. Gregor zerwał się, żeby przyjąć
oczekiwany atak, tylko tyle dobrego mógł zrobił.

Wolkow zaśmiał się i~podszedł bliżej, ręka wystawiona przed nim. Gregor
potrząsnął ją delikatnie.

-- Dzień dobry -- powiedział Wolkow. -- Przepraszam, przez chwilę pomyliłem
cię z~moim przyjacielem Mattem -- wskazał wymownie na kurtkę. -- Widzę, że
go poznałeś.

-- Tak -- odpowiedział Gregor. -- A~gdybyś się nie mylił?

Wolkow wzruszył ramionami i~uśmiechnął się. 

-- Mógł próbować mnie
zaatakować. Jest trochę paranoikiem, jak mogłeś zauważyć.

-- Aha -- odparł Gregor, tak obojętnie, jak tylko mógł.

-- Mniemam, że podejrzewał, że ludzie ze statku porwą go na statek lub
coś, i~że w~jakiś sposób wystawiłem go. -- Wolkow potrząsnął głową. -- A
dlaczego przyszedłeś tutaj najpierw, zanim wpadłeś na Matta?

Gregor się rozejrzał.

-- Hm, możemy porozmawiać prywatnie?

-- Oczywiście -- odpowiedział Wolkow. -- Tędy.

Poza labiryntem podziałów było biuro w~rogu na betonowym podium z~dwoma
szklanymi ścianami. Z~tej wygodnej przewagi nadzorcy, Gregor mógł
zobaczyć około tuzina osób w~obrębie podzielonej powierzchni,
pracujących przy tablicach kreślarskich, klawiaturach lub maszynach
liczących. Wolkow przesunął zużyte krzesło na rolkach ku Gregorowi i~siadł za biurkiem.

-- Kiedy oczekujesz kupców? -- spytał Gregor.

-- W~każdej chwili, więc mów szybko.

-- Jestem tutaj, ponieważ szukamy starej technologii komputerowej do
nawigacji, tak jak wspominałem, i~całkiem szczerze myślimy, że kupcy też
tego szukają. Jesteśmy także zatroskani tym, że kupcy mogą być skuszeni
zabraniem jednego z~was z~nimi, aby wyciągnąć technologię przedłużania
życia w~ten lub inny sposób. Z~tego powodu, mogliby złożyć ci ofertę,
której nie mógłbyś odrzucić.

-- A~Matt myślał, że go wystawiam na to? No, no. -- Wolkow znowu pokręcił
głową. -- Co do Twoich zainteresowań, wątpię, czy ktokolwiek ma
jakikolwiek sprzęt ze statku. Ja z~pewnością nie mam.

Wstał i~podszedł do szklanej ściany. 

-- Gdybym miał, używałbym jej,
oczywiście w~tajemnicy, żeby przechytrzyć moich rywali, zamiast płacić
ludziom do wykręcania obliczeń na tych tam stukających potworach.

-- Firma jest Twoja?

-- Nie, nie. Mam tutaj biuro, różne umowy z~pracownikami, większość
mojej pracy jest na morzu. Szczerze jestem zainteresowany tym, co De
Tenebre mają do zaoferowania. O wilku mowa, już tutaj są.

Wyszedł, żeby wrócić minutę później z~Marcusem de Tenebre i~jednym z~członków załogi. Marcus uniósł brew na widok Gregora, a~Gregor ruszył do
wyjścia.

Wolkow podniósł dłoń. 

-- Gregor, chciałbym, żebyś został. To nie jest
poufne. Chciałbym, żebyś opowiedział o~tym spotkaniu Mattowi i~swoim
kolegom oraz Rodzinom. -- Wzruszył ramionami. -- I~papierowym wiadomościom
czy radio, jeżeli chcesz.

Marcus zajął krzesło, które opuścił Gregor, Wolkow usiadł za biurkiem, a~Gregor podążył za przykładem załoganta i~oparł się o~ścianę.

-- Panowie -- powiedział Wolkow -- czy dobrze myślę, że nie jesteście
tutaj, by sprzedać smary wysokiej jakości?

Marcus pokiwał głową.

-- Dobrze. Więc nie traćmy czasu. Rozumiem, że zamierzacie niedługo
wylatywać. Chciałbym z~wami polecieć. W~zamian za podróż, i~oczywiście
gościnność oraz początkową pomoc na Nova Babylonia, oferuję moją pełną
współpracę w~odkryciu procedur medycznych, które umożliwiły mi tak
długie życie.

Twarz Marcusa pozostała obojętna. Załogant po prostu się gapił.

-- To hojna oferta -- odparł Marcus. -- Wydaje się zbyt hojna. Oferujesz
długie życie w~zamian za przejazd? dom? pomoc przy szukaniu
\emph{pracy}?

-- Proszę o~więcej niż to -- powiedział Wolkow. -- Proszę o~gwarancję mojej
wolności. -- Machnął ręką. -- Nie martwię się pobieraniem próbek w~laboratorium, przez lata poznałem dostatecznie wielu Nowaterrańczyków i~emigrantów, by wiedzieć, że nie mam się czego obawiać. Ale nie chcę być
związany z~waszą rodziną, lub statkiem, choć oczywiście będziecie
pierwszymi beneficjentami sukcesu. I~oferuję mniej, przy okazji, nie
gwarantuję, że badania zakończą się sukcesem.

-- To rozsądne -- powiedział Marcus. -- W~jaki sposób oczekujesz, że
dotrzymamy obietnic?

Wolkow przesunął kawałek papieru przez biurko. 

-- Mam umowę. Oczywiście,
nie jest sprecyzowana co do natury wiedzy, ale jest dostatecznie
szczegółowa. Wiem, że w~waszym interesie jest dotrzymywanie umów,
ponieważ wasze interesy zależą od \emph{bardzo} dobrej długoterminowej
reputacji. Kopie zostały złożone u moich prawników, a~młody Gregor może
poświadczyć i~również jedną zabrać.

Marcus przejrzał dokument i~przytaknął.

-- Podpiszę -- powiedział.

Wolkow podpisał, Gregor poświadczył. Potem wszyscy podpisali kopie.

-- Nie masz nikogo, kogo chciałbyś zabrać ze sobą? -- spytał załogant.

Usta Wolkowa się zacisnęły. 

-- Nie -- powiedział. -- Długie życie może być
samotnym biznesem.

-- A~Twoja praktyka? -- Marcus rozejrzał się po zajętym biurze, widocznie
pod wrażeniem.

-- Cieszę się, że ją zostawiam -- Wolkow wstał. -- Jesteśmy gotowi,
panowie?

-- Chwilę -- powiedział Marcus. Wstał, oparł się o~krawędź biurka i~odwrócił się do Gregora. -- Jesteś specjalistą w~biologii, może bardziej
niż nasi filozofowie. Mógłbyś pomóc nam w~badaniach. Na Nova Babylonia,
mógłbyś stać się wielkim naukowcem, człowiekiem sławnym. Wiem o~Twojej
rozmowie z~moim wujkiem. Mogę Cię zapewnić, że doceniłby to zarówno jako
właściwe użycie Twoich talentów, jak i~jako dar wart jego córki.

Gregor nie wątpił w~ani jedno słowo. Mógł wyobrazić to wszystko, jasne i~wyraźne, błyszczące i~chwalebne. Potrząsnął głową.

-- To, co chcę, jest tutaj.

Marcus wyciągnął otwartą dłoń.

-- Twoja przyjaciółka równie może jechać, jeżeli chce. Lub, jeżeli wolisz
odlecieć bez pożegnań, nasza łódź jest na nabrzeżu z~tyłu.

-- Nie -- odparł Gregor. Czuł się lekko oszołomiony. -- Nie, dziękuję.

Wziął dokument, poczekał chwilę, aż jego usta nie były już suche.

-- Może byłoby lepiej, żebym wyszedł przed wami, panowie. Przekaż Lydii
moje pozdrowienia, mimo wszystko. Jeżeli zobaczę ją znowu, będzie to w~jednym z~naszych własnych statków.

Marcus pokiwał głową, Wolkow uśmiechnął się sceptycznie, członek załogi
stanął z~boku.

To wyglądało na długi spacer przez biuro. Gdy wyszedł z~biura do
recepcji, zobaczył Elizabeth i~Salasso siedzących na sofie. Elizabeth
zerwała się.

-- Wszystko w~porządku?

-- Wszystko dobrze -- odpowiedział.

-- Pilnowaliśmy Ciebie -- powiedział Salasso. -- Matt powiedział nam, że to
zły pomysł, ale się nie zgodziliśmy.

Gregor klepnął oboje w~ramiona. 

-- Dzięki. Nie było to konieczne, ale
dzięki. Gdzie Matt?

-- Ciągle w~kawiarni, mam nadzieję.

-- Dobrze -- powiedział Gregor. -- Mam do niego kilka pytań.

\threeast

-- Cóż, to wszystko -- powiedział Matt. -- Nic już nie można zrobić.

Wyszli z~kawiarni i~poszli na koniec nabrzeża i~przeprowadzili większość
rozmowy siedząc dookoła pachołków, poza zasięgiem słuchu. Statek
przygotowywał się do startu. Ich włosy szczypały. Dziwne prądy wiatru
unosiły kawałki papieru i~folii w~małe wiry.

-- Nic nie można zrobić w~sprawie? -- Elizabeth brzmiała zaczepnie.

Matt wskazał na statek unoszący się ponad bąblem wody. Ostatnie łodzie
wlatywały do długich rozcięć na bokach.

-- Wolkow -- powiedział. -- Puszczając go, nie wyświadczyłeś Nova Terra
żadnych przysług. Ani jego przyjaciołom kupcom, jeżeli o~to chodzi. Ani
nam, w~długi okresie.

-- Wyglądał dostatecznie rozsądnie dla mnie -- powiedział Gregor.

-- Oczywiście, że kurde wyglądał! Kiedy żyjesz tak długi jak ja, dowiesz
się, że każdy wygląda rozsądnie, jeżeli chce.

\emph{ Tak, ale dlaczego ktokolwiek chciałby wyglądać jak paranoiczny
głupek? }, miał ochotę zapytać Gregor.

-- Ty -- powiedział Salasso -- nie wyglądasz na rozsądnego. Czy to dlatego,
że nie chcesz?

-- Może będę chciał, jeżeli będę żył tak długo jak Ty -- odparł Matt. -- A
może nie. My małpy nie robimy się lepsze od dłuższego życia. Nic się nie
uczymy i~nic nie zapominamy. Stajemy się gorsi. Mój sposób bycia gorszym
jest znacznie lepszy niż Wolkowa, uwierzcie mi.

Statek uniósł się do góry, jak sterowiec, którym tak rażąco nie był.
Wysoko na niebie, zaczął poruszać się do przodu, linią poziomą, która
wkrótce zabierze go z~atmosfery, a~która znacznie wcześniej zabrała go z~widoku.

-- W~drodze do Croatan -- powiedział Salasso. -- Znam kurs.

-- Co masz na myśli, mówiąc, że jesteś lepszy niż Wolkow? -- spytał Gregor
w nagłej burzy rozczarowania i~gniewu na swojego przodka. -- Wolkow jest
uznanym biznesmenem. Ty jesteś włóczęgą.

-- Wolkow w~swoim czasie był włóczęgą -- powiedział Matt. -- A~ja byłem
bogaty. \emph{C'est la} jebane \emph{vie}

Wstał, ciągle patrząc za statkiem. 

-- Chodzi o~to, że Wolkow mógłby być
udanym politykiem. Co człowiek, który się nie starzeje, mógłby zrobić
polityce Nova Babylonia może martwić. Ciągle. Co się stało, to się nie
odstanie.

Odwrócił się. 

-- Dobrze, co mogę dla was zrobić?

-- Na początek -- powiedziała Elizabeth -- możesz nam powiedzieć, czy masz
jakąkolwiek starą technologię.

-- Tak -- odparł Matt. Pogrzebał w~głębokich kieszeniach i~wyciągnął
aluminiową obudowę, którą Gregor już widział pośród noży, pistoletów i~breloków. -- Spójrzcie na to.

Zgromadzili się dookoła pachołka, na którym siedział.

Otworzył obudowę i~podał Gregorowi parę zawiniętych okularów
przeciwsłonecznych.

-- No weź, spróbuj.

Ręka Gregora drżała lekko, gdy je otwierał. Zauszniki miały małe
siateczki głośników na ich zaokrąglonych końcach i~ciągle jasną miedź i~połączenia optyczne na zawiasach. Założył okulary. Kiedy spojrzał na
morze, skrzyło małymi, doskonałymi odbiciami słońca.

-- O! -- powiedział. -- Naprawdę zmniejszają blask.

-- Dokładnie -- powiedział Matt. Wyciągnął po nie rękę. -- I~to wszystko,
co robią. Ktoś jeszcze chce spróbować?

-- Co się z~nimi stało?

Matt wzruszył ramionami, gdy składał okulary do pudełka.

-- Narastające błędy, uszkodzenie promieniowaniem, ogólne uszkodzenie
katalogów\ldots w~skrócie, wszystko to, co mi się nie przytrafiło.

Wstał. 

-- Hej, nie wiedzieliśmy -- powiedział, brzmiąc obronnie. -- Nie
wiedzieliśmy, jak dobrze zadziałają te cholerne kuracje. Nie istniały
zbyt długo, mam na myśli, pewnie, kompanie biotechnologiczne strasznie
się przechwalały, ale one tak zawsze. Poprawka telomerów była
jednorazową sprawą, prawda, większość ludzi robiła je we wczesnych
dwudziestych. Zrób i~zapomnij. Nie mieliśmy ich na statku i~nie mieliśmy
dokumentacji do nich. To nie jest tak, że cokolwiek przed wami ukrywamy.

Jego twarz sposępniała.

-- W~porządku -- powiedziała Elizabeth. -- Sami tam trafimy.

Matt uśmiechnął się do niej. 

-- Oto postawa. Mówiąc o~trafianiu
gdziekolwiek, kiedy mogę zobaczyć to wasze rozwiązanie nawigacji?

Poszli wzdłuż nabrzeża z~powrotem do miasta.

\threeast

To było małe okno, a~światło wpadało przez nie wąskim promieniem.
Podążali za gorącym ciepłym polem dookoła podłogi, przesuwając
nieświadomie pozycje, gdy Gregor prowadził Matta przez obliczenia, które
podsumowywały Wielką Pracę. Elizabeth i~Salasso uzupełniali szczegóły
modelu systemu nerwowego kałamarnic.

Ostatnia kartka papieru leżała na podłodze: spód stosu. Gregor nakreślił
linię ołówkiem pod ostatnią wierszem wzorów i~zakołysał się na piętach.

-- To wszystko -- powiedział.

Wstał. Kolana trochę bolały. Matt wstał szybciej i~podszedł do okna.
Słońce, niskie i~pomarańczowe, rzuciło jego ciemny cień.

-- A~więc -- powiedziała Elizabeth -- co o~tym myślisz? Rozwiązaliśmy
problem?

-- Nie wiem.

-- Co? -- Gregor usłyszał swój łamiący głos. Salasso cicho podał chłodną
butelkę piwa, lokalny produkt. Smakowało chemicznie. Łyknął z~wdzięcznością.

-- Musisz wiedzieć -- powiedział Gregor. -- Jesteś pierwszym Nawigatorem.
Prowadziłeś statek przez dziesięć tysięcy pierdolonych lat świetlnych.
Stworzyłeś Wielką Pracę. Musisz wiedzieć, czy ją rozwiązaliśmy.

Matt odsunął się od okna i~usiadł na łóżku. Było ciągle w~nieładzie, tak
jak je zostawili Gregor i~Elizabeth. Hotel Jedna Gwiazda, trafnie
nazwany, nie miał obsługi pokojów. Sięgnął do kurtki i~wyszperał
woreczek i~trochę papieru.

-- Dzięki bogom, że to macie -- powiedział. -- Inaczej nigdy bym nie
przetrwał. Oszalałbym.

Jego dłonie lekko drżały, gdy zawijał liście.

-- Wiedząc, że dziecko w~Twoich ramionach zestarzeje się i~umrze przed
Tobą. Wiedząc, że Twoje wnuki umrą przed Tobą. Wybraliśmy, wiecie.
Byliśmy naukowcami, ogólnie, byliśmy cywilizowanymi ludźmi. Nie
chcieliśmy stać się bogami lub królami. Więc musieliśmy zniknąć i~ciągle
znikać, pokolenie po pokolenie, dekadę po dekadę. Niektórzy z~nas
polecieli do innych słońc. Reszta z~nas\ldots cóż. Dosyć użalania się nad
sobą. Po prostu powiedzmy, było ciężko, a~zioło i~alkohol pomagają,
nawet nas nie zabijają, choć powinny.

Głęboko się zaciągnął. Gregor oparł się impulsowi uderzenia go i~przyjął
skręta. Uspokajający dym rozwiał jego gniew.

-- Dobrze -- powiedział po tym, gdy Elizabeth i~Salasso wzięli udział. --
Wszystkich nas uspokoiłeś, Matt Cairnsie. Teraz powiedz nam, czego nie
wiemy.

-- Jestem artystą, a~nie technikiem -- odpowiedział Matt. -- Jestem
matematykiem, kierownikiem systemów, programistą. Sprawdziłem każdy krok
Twojego rozumowania i~muszę powiedzieć, że wydaje mi się prawdziwe.
Stworzyłem ten problem do rozwiązania przez moich następców, tak. Jestem
w tym dobry. Myślę, że go rozwiązałeś, ale nie jestem tego pewny,
ponieważ\ldots

Spojrzał w~dół, potem w~górę. 

-- Nie jestem Pierwszym Nawigatorem.

-- Więc kto \emph{był} Pierwszym Nawigatorem?

-- Nie było Pierwszego Nawigatora -- powiedział Matt. -- Ale teraz jest. Ty
jesteś Pierwszym Nawigatorem.



\chapter[Bluźniercze Geometrie]{20 Bluźniercze Geometrie}

Czumakowa miała rację. Istotnie porządek był przywracany w~Unii
Europejskiej. Kiedy radziliśmy sobie z~nieudaną konspiracją Wolkowa,
raczej lepiej zaplanowany zamach stanu był przeprowadzany w~Brukseli i~stolicach regionalnych. Oskar Jilek, major-generał w~Europejskiej Armii
Ludowej pojawił się na ekranach, goglach i~biurkach, by ogłosić
powstanie Komitetu Nadzwyczajnego i~o honorowej rezygnacji Pierwszego
Sekretarza Giennadija Jefrimowicza. Zdecydowanie działanie zostanie
podjęte przeciwko uczestnikom zamieszek, prowokatorom, awanturnikom
wojskowym, rewizjonizmie, dogmatyzmie i~korupcji w~obrębie Partii i~aparacie państwowym oraz agentom imperializmu w~obrębie organów
bezpieczeństwa państwowego. Uruchomione zostaną pilne negocjacje ze
Stanami Zjednoczonymi nad \emph{prawdziwie } przyjaznym dostępem do
ostatnich osiągnięć w~eksploracji kosmosu.

Raczej mądrze, Komitet Nadzwyczajny odwołał wszystkie ,,środki
administracyjne'' przeciwko członkom wybranych ciał. Weber, inni
parlamentarzyści i~doradcy, którzy byli aresztowani, zostali natychmiast
zwolnieni. To wyeliminowało jedną demokratyczną pretensję i~natychmiast
zapchało prace parlamentów procedurami zainicjowanymi przez Partię, by
pozbyć się ich legalnymi sposobami. To również odwróciło uwagę od
szybkiej łapanki mniej powiązanych obywateli, w~większości za
wykroczenia, które dawniej byłyby puszczane płazem. Kontrola importu i~regulacje bezpieczeństwa zamknęły bazary hardware i~software jak
Waverley Market w~ciągu kilku godzin. Skorumpowani urzędnicy, którzy
otwierali swoje kieszenie, pozwalając czarnorynkowym oszustom zagrażać
utrzymaniu i~życiu obywateli UE, byli ujawniani i~aresztowani w~wielkich
przedstawieniach szoku i~oburzenia.

-- Będą obaleni za kilka dni -- powiedziała Camila. -- To 1991 po raz
kolejny, zobaczysz.

-- Nie ten sierpień -- powiedziałem.

Jeszcze jedna wiadomość dała dużo do myślenia: w~Bajkonurze był
przygotowywany statek, żeby uratować naukowców i~kosmonautów
\emph{Marszałek Titow} od małej rebelianckiej koterii obecnie
przetrzymującej ich jako zakładników.

Ognie wielu boosterów wynoszących ciężki sprzęt i~dużą liczbę personelu
na orbitalne spotkanie z~wielkim statkiem. Musiał być duży, ponieważ
utrzymywał komplet około stu osób.

Dwóch było kosmonautami ESA. Reszta była Siłami Specjalnymi EAL:
żołnierzami kosmosu.

\threeast

Spojrzałem znad mojego talerza, żeby zobaczyć Drivera poruszającego się
bokiem za długim stołem. Wcisnął się naprzeciwko nas z~talerzem
dzisiejszego lepkiej mikstury ryżu-mięsa i~litrową plastikową butelką
czerwonego. To był pierwszy raz, gdy widziałem go w~jadalni. Jak o~tym
pomyśleć, pierwszy raz, gdy go widziałem na zewnątrz biura.

Przycisnął talerz do stołu i~posłał w~poprzek wino.

-- Częstujcie się -- powiedział.

Jedliśmy przez chwilę, przerywając co jakiś czas na tryśnięcie wina.
Driver pił raczej więcej niż my.

-- Wyglądasz na bardzo zrelaksowanego -- powiedziała Camila.

-- Och, jestem -- powiedział Driver. -- Antygrawitacja rzeczywiście działa!
Ciężar z~umysłu, wiecie, co mam na myśli?

Grzecznie się roześmialiśmy.

-- E tam, właściwie to rozbicie małej konspiracji Wolkow mi pomogło -
powiedział. -- Chryste, to się robi nudne, gdy wiesz, że ludzie spiskują
przeciwko Tobie, ale nie wiesz, kiedy zrobią swój ruch. Jutro zamierzam
oddać odpowiedzialność za prowadzenie tej stacji w~ręce dowolnego
komitetu, który naukowcy uznają za wybrany. Niech ktoś inny męczy się
przez chwilę.

-- Ciągle będziesz kierował projektami? -- spytałem.

Wzruszył ramionami. -- Jeżeli ciągle będą chcieli.

-- Mam nadzieję -- powiedziałem.

Camila patrzyła to na jednego to na drugiego.

-- Nie \emph{wierzę}, chłopaki -- powiedziała. -- Właśnie odbył się
przeklęty \emph{wojskowy zamach stanu} w~waszych krajach, a~wy to
znosicie, jakby zdarzyło się coś dobrego.

Driver zmiażdżył talerz i~zapalił bezdymnego.

-- Nie jest dobrze -- powiedział. -- Ale nie jest tak źle, jak to wygląda.
To ciągle Partia jest u władzy, nie Armia i~nie FBB, dzięki Bogu. A~to
ciągle są centryści Partii, nie jacyś ideologiczni szaleńcy.

-- Ha! -- powiedziała. -- Te komunistyczne brednie, z~którymi wyszedł
Jilek, brzmiały dla mnie wystarczająco ideologicznie.

Driver i~ja się roześmialiśmy.

-- To nie ideologia -- powiedział Driver.

-- Dobra, a~co?

-- Czy wierzysz -- spytał -- że ludzie są obdarzeni przez ich stwórcę lub
ze swojej natury pewnymi niezbywalnymi prawami?

-- Oczywiście, że tak!

-- Dlaczego?

-- To jest, hm, jak ktoś powiedział, to jest oczywiste. Albo to
rozumiesz, albo po prostu nie rozumiesz.

-- Dobrze -- powiedział Driver. -- No, \emph{to jest} ideologia.
Major-General odwołał się w~jego poważnej małej mowie do \emph{słownika.
} To tylko struktura idei, symboli i~organizacji, która pomogła ruskim
zebrać się pokolenie temu i~pomóc Europejczykom zjednoczyć się nieco
później. W~co nasi ludzie naprawdę wierzą to nie mikrofalowy
breżniewizm, ale \emph{prawdziwa} ideologia partii, która jest znacznie
bardziej podstępna.

-- A~jest nią?

Wzruszył ramionami. 

-- Chyba protekcjonizm. Tak czy siak, walić to.
Zamach jest ulgą. To jakbyśmy nie czekali, żeby spadł drugi but.

-- To wy spadniecie -- powiedziała Camila -- kiedy tutaj przybędą.

Driver pokręcił głową, zwęził oczy jak gdyby w~wyobrażonym dymie.

-- E tam -- powiedział. -- Nie wieszamy już. Ani nie rozstrzeliwujemy,
nawet szpiegów i~zdrajców, pomimo tego, co mówi Aleksandra. Ani nie
smażymy obrzydliwie elektrycznością jak wy Jankesi. -- Zrobił gest
siekania. -- Gilotyna. Szybka, ludzka, przynajmniej po niej nikt nigdy
się nie skarżył.

Złożyłem talerz z~resztkami obiadu i~wziąłem szybki trysk wina.
Asteroida 2048. Szorstka, zdecydowanie niedobry rocznik.

-- Naprawdę myślisz\ldots

-- Nie oszukujmy się -- powiedział Driver. -- Wy powinniście wyjść w~porządku. Camila, jesteś Amerykanką wykonującą swoją robotę, nie mogę
cię dotknąć. Matt, cóż, może gdyby rzucili paragrafami, ale nie sądzę.
Emigracja nie jest przestępstwem, nawet jeżeli uciekłeś nielegalnie. A
co do reszty załogi\ldots

Oparł się i~w tym momencie zauważyłem, że wszyscy inni w~jadalni
przestali jeść i~zaczęli słuchać. Bez wątpienia wszystko to było
wysyłane przez więcej niż jedne szkła. Driver udawał, że nie zauważa.

-- Większość naukowców powinna być w~porządku, mogą powiedzieć, że nie
mieli innego wyboru. W~najgorszym wypadku zostaną zabrani ze stacji i~skierowani do innej pracy. Nawet mój dobry przyjaciel Paul, cóż, hej,
nie chcą tworzyć męczenników fakcji Reformy. Pięć lat, góra, w~umiarkowanym klimacie. -- Wyszczerzył zęby do mnie i~mrugnął okiem. --
Słyszałem, że obozy leśne w~Highland nie są takie złe, prócz muszek.
Widziałem ludzi, którzy przerobili tam dziesiątkę, bez kłopotu. Paul ma
powiązania. Ja nie.

-- Czy to dlatego, że jesteś Anglikiem? -- spytała Camila.

-- Ta. Moi rodzice byli angielskimi lewicowcami. Byliśmy na wakacjach w~południowej Francji tego lata, gdy Rosjanie rozjechali Jankesów na Uralu
i po prostu wciąż nadchodzili. Nie było powodu wracać do Londynu. Miałem
dobrą edukację i~dobrze pracowałem w~FBB, ale, wiecie jak to idzie.
Jeżeli kiedykolwiek był ktoś, kto byłby idealnym kandydatem do ścięcia,
to ja.

Wstał i~się przeciągnął. 

-- Jednak, to była dobra gra. Żadnych skarg.
Widzimy się jutro, pominiemy dzisiejsze raportowanie i~zrobimy generalne
spotkanie rano. Upewnijcie się, że zespoły wiedzą, dobra?

Wyszedł, zanim którekolwiek z~nas mogło powiedzieć słowo.

\threeast

Niezbyt dużo pracy wykonano tego wieczoru. Ludzie siedzieli w~intranecie
stacji, rozmawiając lub oglądając wiadomości. Komitet Nadzwyczajny
działał nieprzytomnie, żeby naprawić relacje: z~własną populacją, z~Chinami i~Indiami, ze Stanami Zjednoczonymi. Wykonali wrogie gesty wobec
Japonii. Jak Avakian wskazał, irytowanie kraju, który wszyscy olewali,
było dość bezpieczne.

Obserwowałem rozmowy w~dość bezładny sposób, gdy pojawiła się informacja
o połączeniu z~Nevady.

-- Cóż, cześć tam, Matt -- powiedziała Jadey. Uśmiechnęła się jasno przez
sekundy świetlne. Poczułem przypływ radości i~ostry ból winy, którego
się nie spodziewałem.

Podparłem czytnik tak, żeby kamera mnie widziała i~odpowiedziałem.

-- Hej -- powiedziałem. -- Wspaniale Cię widzieć! Wszystko w~porządku?
Wyglądasz dobrze.

Po czterdziestu kilku sekundach opóźnienia nadeszła jej odpowiedź.

-- Ta, w~porządku. Fajnie jest cię widzieć, Matt! Traktowali mnie dobrze,
prócz tego steku kłamstw, który włożyli w~usta, ciągle jestem na to
wnerwiona. Dzięki za Twoje wiadomości, mimo, docierały i~były dużą
pomocą. Wiesz, to bardziej jak czat niż telefonowanie? Bo, \emph{nie}
jest natychmiastowy? Więc będę mówić, a~potem pozwolę ci odpowiedzieć,
inaczej będziemy ciągle wchodzić sobie w~słowo. Nie wyglądasz zbyt
dobrze, Matt.

-- Ach, wszystko w~porządku, tylko wykończony. Byliśmy zajęci. Słyszałaś,
że mamy latający spodek do testów?

Obserwowałem jej czekanie.

-- No. Camila Hernandez była w~kontakcie z~Alanem. Ale zobacz, Matt, nie
łapiesz tego, musisz mówić dłużej niż to, inaczej spędzimy połowę czasu
czekając. Więc podaj mi swoje wiadomości, co myślisz o~tym zamachu
komuchów i~tak dalej. A~kiedy nad tym będziesz myślał, mogę ci
powiedzieć o~sprawach, które stają się trochę lewe tutaj, wszystkie
rodzaje prawnych kłopotów. Federalni oskarżają nas o~kradzież
technologii obcych od komuchów bez autoryzacji \emph{i} o~podarowanie
jej komuchom. To jakby nas złapali wychodzących i~przychodzących.

Kiedy przestała, ciągle myślałem o~implikacjach kontaktów Camili z~jej
bazą. Nie, żeby było coś złego z~tym, ale mi nie powiedziała o~nich, nie
żeby musiała, ale\ldots

-- Tęskniłem za Tobą -- powiedziałem. -- Myślę, że to ja mogłem pomóc ci
się wydostać.

Przedstawiłem moje społeczne inżynierowanie z~Wolkowem i~co zamierzał.

-- I~-- kontynuowałem -- Twoje tak zwane zeznanie było fałszywe, pewnie,
ale w~zasadzie myślę, że było mniej lub więcej prawdziwe. Wasi
federalni, CIA, czy kto, i~pewne fakcje w~UE używali nas i~muszą mieć
jakiś plan, żeby to zakończyć.

-- Ta, ta -- przytaknęła niecierpliwie. -- Mają ten sam sen, stabilne
społeczeństwo z~nimi na górze. Etatystyczne gówno! Najwyższa ocena dla
was chłopaki za wrzucenie tej matematyki do dekodowania do Netu, ale oni
już myślą o~sposobach na jej ominięcie: tajemnica przechowywać na
papierze, używać kurierów zamiast elektroniki. Mogę Wam powiedzieć, broń
nuklearna jest niewrażliwa na łamanie kodu, tam weszli naprawdę szybko,
co zdaje się, jest dobre, chyba. Ale będą próbować zachować technologię
obcych w~bezpiecznym miejscu\ldots dla nich! Pomyśl o~tym, Matt. O, muszę już iść, to jeden z~kilku bezpiecznych
kanałów, które nam zostały i~jest kolejka. Ale wysyłaj mi wiadomości
głosowe, a~ja odpowiem, kiedy będę mogła. I~dzięki za wyciągnięcie mnie.
Spróbuj i~wróć w~jednym kawałku, dobra? Tylko dla mnie?

-- Zrobię tak -- obiecałem. -- A~Ty uważaj na siebie. Uważaj na czarne
helikoptery.

\threeast

-- Jesteś spięty -- powiedziała Camila. Masowała moje ramiona, jej uda
naciskały na moje biodra.

-- Tak -- zaśmiałem się. -- Rozmawiałem z~Jadey.

-- Jezu, o~\emph{to} chodzi? Hej, no weź. To jej nie rani, a~ci pomaga. I~mnie, muszę powiedzieć, nic się nie zmieniło. Więc daj spokój sumieniu,
dobra?

-- To nie tylko to -- powiedziałem. Obróciliśmy się i~zacząłem oddawać
przysługę. -- Chodzi o~te rzeczy, które nadchodzą. Nie, żebym
kiedykolwiek oczekiwać, że kampania informacyjna zmieni świat, ale
oczekiwałem, że zrobi coś więcej niż tylko zmieni rządy w~bardziej
paranoiczne niż były. I, Boże, to brzmi dziecinnie, nigdy nie sądziłem,
że będziemy mieć kłopoty, za to co robimy.

-- Ach -- powiedziała. -- Niech to Cię nie męczy. To gra końcowa, jasne,
ale jeszcze nie przegraliśmy, nie za daleko. I~hej, Matt, dzisiaj prawie
poleciałam przeklętym latającym spodkiem? \emph{Nic} mnie nie zdołuje po
czymś takim! -- Przesunęła ramiona pod moimi dłońmi i~westchnęła. -- Tylko
rób tak jeszcze trochę, a~ja zobaczę, czy nie mogę sprawić, żebyś się
poczuł lepiej już wkrótce.

\threeast

Amfiteatr w~Efezie, dobre miejsce na zebranie. Tym razem Avakian
powstrzymał się od bawienia się wyglądem. Skan był aktualny, ruiny,
zarośla, śmieci i~jaszczurki. Wszyscy pojawiali się w~swoich awatarach,
mały tłum w~przestrzeni zbudowanej dla wielkiego tłumu. Poza tym, i~podprogowymi implikacjami elitarnej demokracji, wirtualne miejsce
spotkania wydawało się dostatecznie neutralne.

-- To jak gra w~piłkę nożną -- powiedziała Camila, siadając koło mnie i~gestykulując na ludzi zajmujących miejsca na zniszczonych poziomach.

-- Nie, to byłby \emph{stadion}.

Uderzyła mnie, pięść jej awatara przesunęła się przez pierś mojego
awatara.

-- Więcej tego nie zrobię -- powiedziała. -- Mdli mnie.

-- Mnie też -- powiedziałem. Zamknąłem oczy. To było coś jak choroba
lokomocyjna.

Kiedy znowu spojrzałem, skoncentrowałem się na scenie na dole, gdzie
Driver zajął pozycję. Kiedy spojrzał do góry, wyciągnął jedną dłoń,
palce podniesione jak u klasycznego mówcy. Nie potrafiłem zgadnąć, czy
naśladowanie było celowe.

-- Dobrze, towarzysze -- powiedział. -- Wszyscy wiemy, dlaczego tutaj
jesteśmy. -- Rozejrzał się dookoła. -- Dobrze, niektórych jeszcze nie ma.
Dałem, hm, towarzyszom w~karcerze możliwość dołączenia. Żadne z~nich nie
skorzystało.

-- Dobrze, możemy założyć, że żołnierze są w~drodze. Te wczorajsze
zdjęcia przygotowywanych statków nie zostałyby opublikowane przed
startem. Problem jest taki, nie wiemy, kiedy dokładnie zostali wysłani,
ale mamy minimum osiem dni, maksimum trzynaście. Musimy zdecydować
teraz, co zamierzamy zrobić, ponieważ mamy całkiem duży zakres opcji,
zaczynając od bezwarunkowego poddania i~kolejnych dalej.

-- Dotąd wszyscy macie możliwość twierdzenie, że robiliście, to co wam
kazałem, i~byliście posłuszni z~dowolnego powodu, przymus, przekonanie,
że mam jakąś konstytucyjną władzę lub z~braku alternatyw. Ta opcja
kończy się tutaj. Od teraz, rezygnuję z~tymczasowego dowództwa tej
stacji. Co zdecydujecie na temat, hm, misji ratunkowej, i~co zrobić ze
mną, zależy tylko od was.

Wtedy dosłownie zszedł, żeby usiąść kilka rzędów wyżej od sceny. Przez
chwilę, wszyscy patrzyliśmy na siebie, niepewni co robić dalej.
Spojrzałem na Avakiana, pochylonego nad wirtualną klawiaturą. Wzruszył
ramionami i~pokręcił głową. To nie było w~stylu Drivera, by coś takiego
zostawić szansie. Byłem przekonany, że przekonał kogoś innego do
wkroczenia w~tym punkcie. Lemieux, również siedzący blisko frontu, wstał
bez zajmowania sceny.

-- Z~innych powodów, stoję na tej samej pozycji co Colin.

Z krępującej ciszy, która nastąpiła, naukowiec Louis Sembat zerwał się i~podszedł do podium dla mówców.

-- Od tej chwili, jeżeli nie ma sprzeciwu, będę prowadził to zebranie.

Brak sprzeciwu.

-- Bardzo dobrze -- wskazał. -- Angel, chcesz przemówić?

Pestaña wstał, obrócił się, jak gdyby zwracał się do tłumu raczej niż do
podium, choć zachował formalności.

-- Colin musi wiedzieć -- powiedział -- że jego komentarze zeszłej nocy
były szeroko przedyskutowane. Nikt, prócz niektórych w~areszcie, nie
zamierza pozwolić mu ponieść odpowiedzialność. Jesteśmy pewni, że nie
zamierzamy widzieć jego egzekucji lub uwięzienia Paula. Ponadto,
pozwólcie, że będę szczery, nie zamierzamy pozwolić odebrać sobie naszej
pracy.

-- Możecie dostać wybór -- powiedział Driver ze swojego miejsca. -- Dajcie
im mnie i~kogokolwiek wskazanego jako przywódcę i~zachowajcie pracę.

Pestaña potrząsnął głową. 

-- Nie byłoby politycznie możliwe, by Ciebie
potępić za podżeganie do czegoś, co nadal moglibyśmy robić. Do tego, nie
zrobiłeś niczego, czego myśmy nie zrobili. Zastąpienie nas byłoby
trudne. Myślę, że jest przestrzeń dla poważnych negocjacji.

Matematyczka Ramona Gracia przemawiała następna.

-- Nie byłabym tego zbyt pewna -- powiedziała. -- Mogliby zaangażować
kolegów, którzy nie są obecni, i~oni mogliby wytrenować dopływ nowych
naukowców. Część pracy byłaby utracona, ale to może być uznane za warte
w zamian za zaufaną załogę.

Jon Letonmyaki, fiński kosmonauta: 

-- Może mógłbym zapytać co my, lub w~istocie Colin, lub Paul, zrobiliśmy nielegalnego? Wiemy, że
powstrzymaliśmy bardzo niebezpieczny ruch militarystów. Opublikowaliśmy
pewne informacje w~domenie publicznej, ale Pierwszy Sekretarz, były
Pierwszy Sekretarz, wezwał przecież do międzynarodowej współpracy. Colin
zrezygnował z~FBB i, jak sądzę, z~Partii w~sposób raczej wylewny, ale to
nie jest zbrodnia! Więc co zrobiliśmy?

Driver uniósł rękę. 

-- Mogę?

Sembat pokiwał głową.

-- Dobra -- powiedział Driver, stając na nogi. -- Powiem wam, co zrobiłem i~co wy zrobiliście. Świadomie przesłałem informacje o~kontakcie z~obcymi
i plany kryptowojny Amerykanom. To było świadoma i~celowa zdrada, co do
której powziąłem kroki, żeby ją ukryć przed moim przyjacielem Paulem
Lemieux. On jest tylko odpowiedzialny za próby wykorzystania tego
politycznie. Co do was, wszyscy współpracowaliście w~opublikowaniu
narzędzi matematycznych, które sprawiły, że większość istniejących metod
szyfrowania stała się bezużyteczna, oraz destabilizowała większość
rządów świata. Za to może was nie zamkną, ale możecie być pewni, że na
pewno odsunie to was to od pracy tego rodzaju, lub jakiejkolwiek
potencjalnie związanej z~bezpieczeństwem, na resztę waszych przeklętych
karier. Jeżeli władze UE odzyskają kontrolę tej stacji, wasze kariery są
skończone.

Nikt się śpieszył do kolejnych przemów. Większość osób tutaj, domyślałem
się, już to zrozumiało, ale na pewnym poziomie ciągle trzymała się tego
rodzaju nadziei, która została naiwnie przedstawione przez Letonmyaki,
i, mniej naiwnie, przez Angela Pestañę. Wielu z~nich wydawało się nagle
zainteresowanych w~realistycznie renderowanej zgniecioną puszką po coli
i stwardniałej gumie do żucia na stopniach, albo mgłą i~cyprysami w~pewnej odległości.

Telesnikow, którego już dawno temu oznaczyłem jako człowiek Drivera
wśród kadry kosmonautów, wstał i~przeszedł na front.

-- Dość tego! -- powiedział. -- Nie jesteśmy bezradni, nie musimy siedzieć
i czekać na żołnierzy. Możemy wybrać komitet reprezentujący nas w~negocjacjach z~ESA i~jeżeli to możliwe z~tym nowym Komitetem
Nadzwyczajnym. Możemy odwołać się do opinii publicznej, raczej według
tego, co Jon Letonmyaki sugerował: zrobić wszystko, by brzmieć rozsądnie
i niewinnie. A~robiąc to wszystko, możemy przygotować się na najgorsze.

-- Mamy najbardziej zaawansowany pojazd kosmiczny na świecie, tylko
czekający na dokończenie. Wiemy, że działa, i~mamy tylko wstępną ideę
jego możliwości. Avakian powiedział mi, że jest przekonany, że statek
może dotrzeć do Ziemi w~ciągu godzin. Jest bardzo dużo możliwości z~maszyną jak ta!

-- Mamy również napęd gwiezdny, który możemy skonstruować w~ciągu może
dziesięciu dni. Jest możliwe, że możemy go skończyć przed przybyciem
żołnierzy. A~jeżeli tak\ldots

Zawiesił głos, rozejrzał się dookoła wyzywająco.

-\ldots możemy go użyć i~kiedy przybędą, będziemy gdzieś indziej!

Tym razem, tak wiele osób chciało mówić, że Avakian ledwo mógł
powstrzymać natłok.

\threeast

Bengalski astrofizyk Roxanne Khan przedstawiała najsilniejsze zarzuty,
po najbardziej oczywistych co-jeśli, które zostały odrzucone. Jeżeli ta
rzecz nie zadziałała, byliśmy w~niezgorszej sytuacji. Było mało możliwe,
żeby wybuchło, lub w~inny sposób nas zniszczyło, chyba że obce
inteligencje w~istocie miały mocno skrzywione powody. A~jeżeli tak było,
martwi jesteśmy bezpieczniejsi niż na łasce morderczych i~samobójczych
bogów. 

-- Problemem jest, jak to rozumiem, kwestia nawigacji -
powiedziała Khan. -- Michaił mówi o~zrobieniu małych, kontrolowanych
skoków kilka minut świetlnych, dowód, który w~istocie byłby całkiem
wystarczający, byśmy mieli argumenty w~negocjacjach. Ale już wiemy, że
informacje nam przekazywane są czasami dwuznaczne, trudne do
interpretacji. Co jeżeli zrobimy błąd? Zostawmy na boku patologiczne
myśli o~skończeniu skoku w~słońcu. Co jeżeli znajdziemy się w~połowie
drogi w~galaktyce? Lub w~połowie w~poprzek kosmosu?

Telesnikow miał na to odpowiedź. Podejrzewałem, że myślał o~tym już
jakiś czas.

-- Jest nas około trzysta, dość równo podzieleni pomiędzy płcie. To dość
dobra liczba do rozpoczęcia kolonii.

-- Używając czego? -- ktoś krzyknął przez wrzawę bardziej rubasznych
komentarzy. -- Zasobów komet?

-- Podkopać bogów -- powiedział Avakian, jak gdyby pod nosem, ale tak, że
wszyscy usłyszeli i~większość się roześmiała.

-- Nie wszystkie małe ciała w~Układzie Słonecznym są\ldots zamieszkałe -
kontynuował Telesnikow niewzruszony. -- Nie mamy powodu sądzić, że to
wyjątek. Więc w~zasadzie, tak. Moglibyśmy.

To właściwie nie zakończyło dyskusji, ale jakoś zmniejszyło napięcie. Z~tego, co usłyszałem potem, większość z~nas była zadowolona, że miała coś
innego do roboty niż czekanie na aresztowania, a~wiedza, że jeżeli
negocjacje nie powiodą się, mamy dodatkową szansę ucieczki.

Istniało także dziwnie pocieszająca uwaga, że w~najgorszym wypadku
możemy umrzeć, ale nie \emph{wymrzemy}.

\threeast

-- Jestem gotów -- powiedziałem.

Mój oddech brzmiał głośno w~hełmie. Dziesięć metrów od pokrywy twarzy, w~kierunku, któremu odmawiałem myślenia jako ,,dół'', była powierzchnia
asteroidy, strona nocna w~tym momencie, jej ceglany detal ledwo widoczny
w słabym świetle mojego skafandra. W~każdym innym kierunku były gwiazdy,
ostrzejsze niż mróz. Nie mogłem myśleć o~nich jako o~celu podróży.
Hipoteza Kopernikańska wydawała się absurdem. Te rozrzucone punkty
światła nie mogły być słońcami.

-- Wyłącz światła -- powiedział Armen.

Przede mną teraz, nic prócz czerni.

-- Podciągnij się bardzo delikatnie na linach i~zatrzymaj się, gdy to
zobaczysz.

Para lin, jedna na każdą dłoń, były rozciągnięte pomiędzy dwoma
wysokimi masztami sto metrów od siebie. Poruszałem się wzdłuż nich około
czterdziestu pięciu metrów, przesuwając karabińczyk mojej uwięzi pod
jedną ręką. Pociągnąłem się dalej, przyglądając się czerni. Pode mną, na
wprost ode mnie, zobaczyłem aparaturę obcych. Żarzyła się na tyle, żeby
być widoczna. Wyglądała jak busz lub robot buszu, na tyle duża, żeby
wpaść do niej.

-- Teraz -- powiedział Armen -- tylko pociągnij liny do siebie, puść i~pozwól sobie na dryf ku temu. Nie martw się nietrafieniem, trafisz,
pamiętaj, że ciągle jesteś przypięty do lin. Wyłącz radio.

Wyłączyłem, potem pociągnąłem tak delikatnie, jak to możliwe, puściłem i~poleciałem do przodu pomiędzy linami. W~sekundach, które zabrało mi
przemierzenie kilku metrów, aparatura wyglądała jeszcze bardziej
krystalicznie i~delikatnie, jak gdyby mogłem ją zmiażdżyć bardzo wolno w~żyrandol śniegu.

Kiedy Avakian pokazał nam coś, co nazwał ,,dużym obrazem'' wnętrza
asteroidy, założyłem, że widzieliśmy bezpośredni jej widok, coś, co było
przekazane nam w~bardzo ograniczonej formie przyjaznej użytkownikowi,
przez interfejs. W~istocie, to, co zobaczyliśmy, było nagraniem z~poprzednich kontaktów, jak to, które miałem właśnie mieć. Oprócz
Interfejsu nie istniały trwające bezpośrednie widoki wnętrza. Interfejs
był zasilany przez światłowód gruby jak moje ramię wystający z~buszu z~boku asteroidy do stacji, ale dla akcji w~czasie rzeczywistym musieliśmy
używać tego drugiego interfejsu, skonstruowanego, lub narosłego, przez
samych obcych. Z~jakich powodów, tylko bogowie wiedzieli. I~choć raz ten
zwrot był dosłowną prawdą.

Aparatura nie roztrzaskała się, gdy się z~nią zderzyłem. Niektóre z~jej
gałęzi przesunęły się i~podzieliły, inne zebrały, żeby zazębić swoje
końcówki w~kształty jak wielkie płatki. Zaabsorbowało to mój pęd i~podtrzymało mnie. Jeden z~płaskich kształtów pokrył panel. Przez sekundę
kompletnej ciemności poczułem, całkiem irracjonalnie, jak gdyby pokrycie
mogło mnie udusić. Wtedy odkryłem, że mogłem zobaczyć odczyty
koordynatów na płycie wyświetlane w~górnym lewym rogu w~delikatnych
czerwonych cyfrach. Ta manifestacja, wejście w~hełmie i~sterowanie
orientacją pod moimi palcami, były jedynymi interakcjami z~aparaturą, na
którą obcy raczyli zezwolić.

Narastało światło lub moje oczy się zaadaptowały. Zobaczyłem obsydianowe
ściany mijane po obu stronach, potem szybciej i~szybciej punkt widzenia
poruszał się szybko po nieskończonych rozgałęziających się korytarzach,
każdy nieco szerszy niż poprzedni. Czerwone liczby na odczycie mrugały.
Przyszło mi do głowy, że może widzę gałęzie krystalicznego buszu od
środka. Poczucie ruchu było nieuniknione. Zamknąłem oczy i~odkryłem, że
nadal widzę czarne korytarze, po których bezradnie pędzę. Metodami, o~którym mogę tylko zgadywać, ta scena była wyświetlana bezpośrednio na
siatkówce. Tylko odczyt zniknął z~pola widzenia. Kiedy otworzyłem oczy,
zobaczyłem go znowu, liczby były czerwoną plamą.

Na końcu ostatniej prostej, wszedłem w~gładki szyb i~zostałem wysłany z~niego w~środek przestrzeni asteroidy. Piękno zalało mój mózg. Jeżeli
zamknięcie oczu mogłoby powstrzymać od widzenia, żałowałbym mrugnięcia.

Tym razem nie miałem chłodnego uderzenia śmiechu Avakiana, żeby mnie
ocalił. Moja cała mentalna siła został zużyta, by zwrócić uwagę na trzy
długie numery na wyświetlaczu pozycji i~wcisnąć palce w~aparaturę, żeby
przejęła kontrolę nad wirtualnym lotem. A~kiedy mi się to udało,
pragnienie, by użyć tego do zabawy, lotu i~szybowania, było prawie
nieodparte, ale nie do końca. Ruszyłem, aż numery zgadzały się z~koordynatami, które nam dano i~znalazłem mój punkt widzenia unoszący się
cale ponad zawiłym, kwietnym, fraktalnym wzorem jak wał mchu.

Mój punkt widzenia otrzymał lodowaty chip falujący molekularnie,
przesyłający informacje z~powrotem do aparatury i~stamtąd do wyjścia.
Więc tak to miało być. Moja twarz pogrążyła się w~tej minucie i~doskonałym ogrodzie, a~jakaś roślina przede mną została wyrwana z~korzeniami. Poczucie zrobionego zniszczenia wypełniło moje oczy łzami
tak szybko, jak struktura sama się naprawiła. Mrugnąłem i~widok zniknął.
Aparatura wypchnęła mnie na zewnątrz z~taką samą siłą, jaką dostarczyło
moje przybycie. Gdy dryfowałem do tyłu, wszystko, co mogłem zrobić to
złapać się lin.

\threeast

-- To jest to -- powiedział Avakian. -- Chcesz rzucić okiem?

Kręcę głową. 

-- Po prostu mi powiedz.

-- Pokażę Ci -- mówi, machając od szkieł do ekranu na ścianie i~stukając
kciukiem.

Godziny od mojego spotkania, byłem niechętny, prawdopodobnie niezdolny
wejść w~VR. Nawet interfejs, we wspomnieniu, uderzał mnie jako nie do
zniesienia niezdarny. Avakian zapewnił mnie, że efekt minie: 

-- To jak
wypalenie endorfin, lub coś. Odbijesz się.

Z nudnym pozorem zainteresowania oglądałem schemat i~arkusze danych
pojawiające się na płaskim ekranie. Wysoka rozdzielczość, ale dla mojego
steranego oka, były surowe, jak gdybym mógł zobaczyć piksele. Silnik był
wyświetlony, lekko zmieniony. Nie mogłem z~początku zauważyć różnicy.
Avakian wyświetlił laserową kropkę na ekranie.

-- Tam -- powiedział. -- System sterowania i~wygląda jak interfejs
przygotowany dla człowieka. Kolumny danych pod nim to \emph{ustawienia.
} Jesteśmy w~domu, człowieku. Wszystko, co musimy zrobić, to go
zbudować.

Wszystko, o~czym mogłem pomyśleć to nuda przebijania się przez projekt
raz jeszcze, zmieniając kawałek po kawałku, żeby wprowadzić zmieniony
wynik.

-- Dobrze -- powiedziałem. -- Zajmę się tym.

-- Nie zrobisz tego -- powiedział Avakian. -- Możemy udać się do gwiazd,
ale \emph{Ty} udajesz się spać.

-- Czy to pomoże? -- Pomysł brzmiał niejasno intrygująco, ale nieistotnie.

-- Zaufaj mi -- powiedział. -- Jestem doktorem.

\threeast

-- Obudziłam Cię?

Camila wydostała się z~kombinezonu.

-- Ta -- powiedziałem -- musiałaś we mnie uderzyć. Która godzina?

-- Północ -- odparła.

-- Więc opuściłem spotkanie.

Zaczepiła stopą o~taśmę i~przyciągnęła mnie do siebie, owijając ramiona
wokół mnie.

-- Nie straciłeś dużo -- powiedziała. -- Nowy komitet ciągle jest
ignorowany przez ESA, nie mówiąc o~juncie. Dobra wiadomość jest taka,
mamy przygotowane \emph{Bluźniercze}. Biorę je jutro na pierwszy
doświadczalny lot.

-- Hej, super! To wspaniale!

Złapała moje ramiona. Jej własne zatrzęsły się od powstrzymywanego
śmiechu.

-- Matt, obudź się poprawnie! To była rutyna, oboje wiedzieliśmy, że to
zrobimy. Wszyscy mówili o~tym, co zrobiłeś.

-- Co ja zrobiłem\ldots Och!

Wspomnienie tego, co zrobiłem, wróciło, ale to było jak wspomnienie snu
zapamiętanego nie przy obudzeniu, ale później rankiem, już dzielącym się
na nieuchwytne, kolorowe części. W~tym samym momencie, poczułem falę
dobrostanu. Moje endorfiny, lub cokolwiek, znowu były we krwi.

-- Dane, który wyciągnąłeś, które Armen nazywa ustawieniami, Roxanne i~Michaił sprawdzili je i~zdecydowanie są na krótki skok, tak jak
prosiłeś. A~wszedłeś pomiędzy obcych, żeby je dostać!

-- Tak, zrobiłem to. Z~trudem w~to wierzę. I, och cholera\ldots

Położyłem na chwilę czoło na jej ramieniu.

-- Co się dzieje?

Przełknąłem. 

-- Rozumiem teraz, dlaczego byli tacy chętni, żeby zgłosił
się na ochotnika. Jeżeli zrobiłeś to raz, \ldots nigdy nie będziesz chciał
tego zrobić.

Dreszcz wstrząsnął jej ciepłym ciałem. Odepchnęła mnie i~spojrzała
prosto w~oczy.

-- Czy to jest \emph{aż tak} straszne?

-- Nie! Nie, to jest piękne. To jest najpiękniejsza rzecz, jaką
widziałem.

-- Piękniejsza ode mnie?

-- Tak -- powiedziałem, bez wahania. -- To tylko zachwyca umysł jak burza
pakietów przeładowująca bufor.

-- O jej, jakie sugestywne słownictwo.

Musiałem się roześmiać.

-- Ale zapominam to -- powiedziałem. -- I~chcę zapomnieć. Piękno, które
teraz widzę, jest znacznie prawdziwsze.

-- Teraz gadasz -- powiedziała.

Jakaś synergia podniecenia Camili na lot \emph{Bluźniercze Geometrie} w~końcu z~nowymi silnikiem, mojego przeregulowania endorfin i~może
ostatnich fragmentów wspomnienia ogrodu inteligentnych maszyn, wypełniła
nas energią, inwencją i~tkliwością, oraz nie dawała zasnąć większą część
nocy.

\threeast

Nie czułem się w~ogóle zaspany rankiem. W~istocie czułem się niezmiernie
odświeżony. Poszedłem z~Camilą do doku, gdzie pierwszy raz przybyliśmy,
tak kilka dni i~tak dawno temu.

Mała załoga techników kosmonautów czekała na nią. Roxanne Khan była
nominalnie wybrana jako przewodniczącą ostatnio wybranego komitetu.
Colin Driver trzymał się niedaleko w~pozycji czysto doradczej.

Camila wyciągnęła swój kombinezon przeciwprzeciążeniowy ze skrzyni.

-- Czy to konieczne? -- spytał Driver.

-- Może nie -- powiedziała Camila, wślizgując się w~skafander ładnym
saltem. -- Ale na wszelki wypadek.

Hełm pod ramieniem, jak u astronauty pozującego do fotografii przed
startem, zdryfowała do mnie i~zawiesiła szklany bąbel w~powietrzu, zanim
mocno się do mnie przytuliła.

-- Życz mi szczęścia -- wyszeptała.

-- Wszystko będzie dobrze -- powiedziałem jej. -- Jesteś najlepsza.
Powodzenia.

-- Tobie też -- powiedziała i~odwróciła się jak ryba. Zniknęła we włazie.
Jej rutynowe testy i~komunikaty zaczęły dochodzić przez radio, póki nie
były skończone.

-- Wszystkie systemy w~normie.

Driver spojrzał na technika przy ścianie. 

-- Zamknij śluzę -- poradził. -
Dobra, wszyscy wychodzą z~doku.

-- Dlaczego? -- spytała Kahn. -- Jest wystarczająco bezpiecznie.

-- Nie wiemy tego -- podrapał się po gardle, robiąc szumy w~mikrofonie. --
Mogą wystąpić pewne, hm, fenomeny elektromagnetyczne.

-- Dlaczego tak sądzisz? -- spytałem.

Gdyby nie stał w~powietrzu pod kątem, mógłby przysiąc, że zaszurał
nogami i~spojrzał w~dół.

-- Jeżeli, hm, to, co się słyszy o~bliskich spotkaniach z~tego rodzaju
maszynami, jest jakąkolwiek wskazówką.

\emph{Och.}

Wszyscy wyszliśmy z~doku i~przełączyliśmy się na kamery. Statek
zewnętrznie nie był zmieniony, jego wygląda tak nieprawdopodobny jak
zawsze. Dźwięki odłączenia kliknęły i~uderzyły przez ściany i~podłogi.

-- Statek może odejść.

-- Odpalenie dodatkowych silników -- powiedziała Camila.

Dwusekundowe uruchomienie odsunęło statek, kolejne ustabilizowało jego
pozycję kilometr od asteroidy.

-- Uruchamiam AG.

Tym razem nie było widać niebieskiej aureoli ani żadnej zmiany w~statku.

-- Dobrze -- powiedziała Camila wesołym głosem. -- Tak wygląda uruchomione
w pozycji neutralnej. Zamierzam przesunąć naprzód.

Statek ruszył. W~jednej chwili był tam, w~kolejnej zatrzymał się
nieruchomo kilometr dalej. Nawet ci z~nas, którzy widzieli
przedstawienie z~sankami mogli z~trudem w~to uwierzyć. Roxanne Khan,
która nie widziała, właściwie zasłoniła oczy na sekundę. Zobaczyła, że
patrzę i~jej krótko jaśniejsze policzki zaczerwieniły się.

-- Spoczywaj w~pokoju, Sir Isaac -- powiedziała pod nosem. Potem, głośno:

-- Kosmonautko Hernandez, zabierz to.

-- Dziękuję, proszę pani -- odpowiedziała Camila. -- Uruchamiam ruch
naprzód.

\emph{Bluźniercze Geometrie} znikły.

Szybkie odtworzenie zapisu kamer, potem tego na radarach, pokazało tylko
krótkie malejące przebłyski, zanim zniknęła z~obu.

Driver głośno odetchnął.

-- Jak nietoperz z~piekła -- powiedział.

Odwrócił się do techników.

-- Możemy ją wywołać?

-- Pewnie.

Kiwnął na Kahn.

Bardzo formalnie, powiedziała: 

-- Stacja \emph{Ciemniejsza Noc,
Jaśniejsza Gwiazda} wzywa \emph{Bluźniercze Geometrie } Proszę o~raport.

Brak odpowiedzi. Kahn powtórzyła wywołanie.

Po kolejnej sekundzie, głos Camila odpowiedział.

-- \emph{Bluźniercze Geometrie} do \emph{Jasna Gwiazda } Działanie
pojazdu w~normie, systemy w~normie.

-- Bardzo dobrze -- powiedziała Roxanne. -- Wyłącz ruch do przodu, odwróć
kierunek i~wracaj na stację.

-- \emph{Bluźniercze} do \emph{Jasna Gwiazda}, hm, to negatywne.

-- Jest jakiś problem? -- spytała Roxanne.

Tym razem opóźnienie było około dwóch sekund. Nagle olśniło mnie, że
statek jest już sekundę świetlną stąd: trzysta tysięcy kilometrów.

-- Żaden problem -- powiedziała Camila. -- Wracam do bazy, Jezioro Groom,
Strefa~51.

\threeast

Czasami tylko wtedy, gdy założenie jest sfalsyfikowane, rozumiemy, jakie
było, lub, że je zmyśliliśmy. Założyłem, że jeżeli Camila wracałaby do
domu, zabrałaby mnie ze sobą. Również założyłem, że ponieważ kochałem
Jadey, nie mogłem jednocześnie kochać Camili.

Na zmianę gniewałem się Camilę i~miałem nadzieję, że wróci. To był sen.
Camila i~Jadey były obie, bardzo zdecydowanie i~w przewidywalnej
przyszłości, w~prawdziwej Krainie Marzeń. Wiadomość głosowa od Jadey
pojawiła się kilka minut po wylądowaniu Camili.

-- Och, Matt, jest coś, co muszę Ci powiedzieć. Dysk, który ci dałam, nie
był tym od Josifa. \emph{Tamten} dysk zawierał informację, które Driver
wypuścił do jednego z~naszych agentów w~ESA, o~kontakcie z~obcymi i~matematyce obcych i~co to znaczyło dla kryptografii. Przesłałam go do
Nevady z~mojego biura w~pracy tamtego ranka.

Nie wiedziałam, co było na nim, ale wątpię, czy był nawet zaszyfrowany.
Nie ma sensu, prawda? A~FBB musiał odczytać go, co rozpoczęło całą
lawinę, zmusiło UE do ogłoszenia. Dane na dysku, który ci dałam, są na
stronach całej Sieci, i~były tam przez ponad rok. Ściągnęłam je z~jednej
z nich.

-- Myślimy, że obcy zaspamowali informacją o~napędzie gwiezdnym szeroko
rozrzuconym stronom, bez wiedzy nikogo na stacji. Było to w~formacie,
które systemy ESA na stacji sprowadziły do dokumentacji produkcyjnej.
Nevada Orbital Dynamics ma ludzi, którzy przeszukują sieć pod kątem
spraw latających spodków. To dlatego, że firma wykryła kilka, hm,
anomalii w~zapisach. Wiesz, co mam na myśli. Wystarczająco, żeby zacząć
te trzymać te rzeczy przynajmniej częściowo pod uwagą. Oczywiście
większość z~tego to kompletne śmieci. Znaleźli dane w~całym tym
bałaganie i~sprawdzili je, wyglądały interesująco, ale nie mieli
potrzebnych umiejętności, żeby poradzić sobie z~konwencją ESA i~przeprowadzić analizę systemów i~planowanie produkcji, ponieważ, jak
wiesz, te właściwie wymagały problematycznych kombinacji z~technologiami
UE i~amerykańską.

-- Jednakże, znali kogoś, kto to umiał, Ciebie, przeze mnie. A~zawsze
wiedzieliśmy, że możemy na Ciebie liczyć, politycznie. Ociągali się,
jednak, póki te rzeczy nie mogły być autentykowane. Ściągnęłam je już,
lekko zaszyfrowane, nie wiedziałam, co to było. Kiedy wysłałam Driverowi
wiadomość potwierdzającą kontakt z~obcymi, odpowiedzieli wcześniej
ustaloną frazą, która znaczyła, że powinnam wziąć dysk danych z~latającym spodkiem do Ciebie i~gdy już cię przekonam, zabrać cię do
Stanów. Dane nie były ważne, \emph{Ty} byłeś ważny. Gdyby nie było to,
że Josif zostanie zabity, co było po prostu pechem, i~represji,
pojechałabym z~Tobą. Tak jak to było, moje aresztowanie przynajmniej
posłużyło jako odwrócenie uwagi.

-- Ponieważ to Ciebie potrzebowaliśmy, i~to Ciebie gliny powinny szukać.
Dane były tam cały czas.

Wisiałem na siatce przez chwilę, z~boku ruchliwego korytarza, obserwując
ludzi mijających jak ryby, ich usta ruszające się prawie bezgłośnie, gdy
rozmawiali w~innej, niewidzialnej sieci. Wyjąłem mój czytnik,
wyciągnąłem kompletne plany produkcji i~wysłałem je przez nadajnik
stacji do tak wielu węzłów, do ilu się udało.

Nie było to naprawdę konieczne, ale dało mi małą satysfakcję.

Nie byłem jedyny, który zrobił fałszywe założenia. Cały statek
obserwował w~napięciu, gdy oskarżenia były rzucane w~komitecie naukowym.

-- Żadne z~nas nie wyobrażało sobie, że Hernandez zabrałaby statek na
Ziemię -- powiedziała Roxanne. -- Ponieważ \emph{założyliśmy}, że nasz
ekspert ochrony ma dobre powody, by ufać Hernandez, inaczej nigdy by nie
pozwolił jej na test!

-- Och, wierzyłem Camili, prawda -- odparł Driver. -- Byłem absolutnie
przekonany, że zniknie przy pierwszej możliwości. Jak nietoperz z~piekła.

-- Więc dlaczego na to pozwoliłeś?

-- Ponieważ to jest to, co chciałem.

Gdy hałas ucichł, Lemieux powiedział:

-- Colin, przyjacielu, proszę, powiedz mi, teraz kiedy nie ma nic do
stracenia lub zyskania, czy jesteś, w~końcu, amerykańskim agentem?

-- Nie -- odpowiedział Driver. -- Ręka na sercu, kumplu, nie jestem. Nie
jestem teraz ani nigdy nie byłem.

-- Więc \emph{kim} jesteś?

-- Jestem Anglikiem -- powiedział Driver.

\threeast

Biuletyny CNN pokazujące trzęsące amatorskie video z~Jeziora Groom ledwo
znikły, gdy major generał Oskar Jilek pojawił się w~transmisji na całe
UE.

-- Poważna sytuacja powstała w~odniesieniu do utrzymywanej przez
rebeliantów stacji kosmicznej \emph{Marszałek Titow }. Wiedza naukowa
otrzymana przez jej historyczne osiągnięcia, która winna być użyta dla
korzyści całej ludzkości, została przywłaszczona przez obcych agentów i~jednostronnie zastosowana, by zagrozić Pokojowi. Komitet Nadzwyczajny
Unii Europejskiej z~żalem ogłasza, że jego cierpliwość wobec rebeliantów
się wyczerpała. Ich rosnące prowokacje i~bezczelne żądania przekroczyły
granicę. Od tej chwili, Unia Europejska jest z~nimi w~stanie wojny. Ich
działania również zagrażają Stanom Zjednoczonym i~nalegamy, żeby rząd
tego narodu przyjął sposób działania odpowiedni do ciężaru sytuacji.

-- Nie mamy o~czym negocjować z~rebeliantami. Jakakolwiek dalsza
komunikacja z~ESA, przez kogokolwiek na stacji prócz Majora Suchanowa
będzie traktowana jako kolejny wrogi akt. Major Suchanow oraz inni
zakładnicy mają być bezwarunkowo zwolnieni, władza nad stacją ma być
przekazana majorowi Suchanowi w~ciągu godziny. W~przeciwnym przypadku
Siły Specjalne Europejskich Ludowych Sił Powietrznych odpowiedzą z~konieczną siłą i~bez dalszych ostrzeżeń.

Driver, także, nie tracił czasu. Zignorował komitet naukowy. Jego twarz
i głos wypełnił statek.

-- Jilek blefuje -- powiedział. -- Już wiemy, kiedy ekspedycja opuściła
orbitę ziemską. Astronom w~Kazachstanie zrobił zdjęcie, a~jakiś haker w~Sydney właśnie przesłał je do nas. Ciąg był siedem dni temu. Mamy pięć
dni, by zbudować silnik i~odłączyć stację od asteroidy. A~potem, ludzie,
\emph{skoczmy}.



\chapter[Ciemniejsza Noc, Jaśniejsza Gwiazda]{21 Ciemniejsza Noc, Jaśniejsza Gwiazda}


-- Ona nigdy nie wróciła? -- głos Elizabeth brzmiał smutno.

Godzina jest późna, nawet w~Nowa Lizbona. Pikt i~gigant przy barze
prawie śpią, ale są dumni, że przetrwają klientów. Pub jest pusty, prócz
ich, nas i~kilku zaurów, i~kogo to obchodzi, co usłyszą?

Opowiedziałem im moją historię, w~długiej wędrówce, która zabrała nas o~Hotelu Jedna Gwiazda przez kolejne bary. W~którymś momencie zjedliśmy.
Zachowywałem części historii, której nie chciałem, by ludzie usłyszeli
przy naszym zataczaniu się po ulicach lub knajpy siostrzanych gatunków.

-- Oczywiście, że kurwa nigdy nie wróciła -- odpowiadam. Lekko się wzdryga
i łagodzę to. -- Skontaktowała się ze mną. Rozmawialiśmy. Kochała mnie,
tak myślę, ale nie było sposobu, żeby wyleciała na przeklętym,
cholernych \emph{Bluźnierczych Geometriach } Amerykańskie Siły
Powietrzne były wszędzie jak muchy na gównie.

-- Ciągle nam nie powiedziałeś -- mówi Gregor -- co poszło nie tak z~nawigacją? Źle obliczyłeś, czy co?

Patrzę na niego. Czasem się zastanawiam, naprawdę. Mit naszej nawigacji
służył nam dobrze, ale musiał też służyć tubylcom. Musiał spełnić jakieś
głęboko ukryte potrzeby, żeby przetrwać tak długo w~obliczu takiego
bezczelnego nieprawdopodobieństwa, żeby nie powiedzieć fałszerstwa.

-- Nigdzie nie nawigowaliśmy -- mówię im. -- Dane, które odzyskałem, mogły
być autentyczne, z~tego, co wiem, lub może tylko krakeny, lub wasze
sztuczne kałamarnice, mogłyby je zrozumieć. Lub mógł być to kompletny
śmietnik, świadomie lub nie. Cokolwiek. Podejrzewam, że silnik miał
wstępne instrukcji, by tu przybyć. Z~tego, co wiem, ustawiliśmy coś, o~czym myśleliśmy, że jest skokiem przez Układ Słoneczny i~okazało się, że
jesteśmy na orbicie polarnej nad Mingulay. Właśnie zrozumieliśmy, że
zdecydowanie to nie była Ziemia i~zdecydowanie to nie był Układ
Słoneczny, gdy pojawił się skif Tharovara. Łódź, w~ogóle nie była
przerażająca, jakby spodziewaliśmy się tego. Przerażające było, gdy on
wypadł ze śluzy powietrznej.

Patrzę na Salasso, czymś, co mam nadzieję, jest twardym spojrzeniem, a~pewnie jest po prostu mętnym patrzeniem.

-- Wy ludzie macie dużo do wyjaśnienia.

Zaur rozkłada długie ręce. 

-- W~tym nie mogę pomóc. Żaden z~nas nie wie,
co jakieś zaury w~Układzie Słonecznym mogły zrobić w~czasach
historycznych.

\emph{Zastanawiam się, ile byś powiedział, gdyby wcisnął próbnik w~Twoją
dupę}. 

Mam nadzieję, że tego nie powiedziałem.

-- Zastanawiam się -- mówi Elizabeth -- czy komputery na statku jeszcze
działają?

-- Prawdopodobnie tak -- mówię. -- Osłony przeciwpromienne, wiesz? Cały
sprzęt do restartu. Kurde, mógłby to zrobić teraz. Prócz tego, że
Tharovar i~jego kumple zabrali nas z~tego statku w~wielkim pośpiechu,
zrobili wielkie zamieszanie, i~nigdy, nigdy nie pozwolili wrócić.

-- To nie powinien być problem -- mówi Salasso -- teraz gdy wiemy, że
możecie nawigować. Nigdy nie wierzyliśmy, że znaleźliście sami Mingulay,
choć nigdy nie zaprzeczaliśmy. Wierzyliśmy, że bogowie was wysłali i~chcieli, byście zostali. Może tak było. Mieli jakiś cel w~utworzeniu tej
Drugiej Sfery, ale ani my, ani krakeny, nie wiedzą, jaki to cel.
Jednakże, teraz gdy krakeny przedstawiły swój sąd, nie ma powodu, żeby
was zatrzymywać.

Gregor łapie ramię zaura.

-- Chcesz powiedzieć, że możemy spróbować nawigacji z~\emph{Jasnej
Gwiazdy}?

-- Tak.

Uśmiecha się Elizabeth i~nawet w~moim pijanym, zjaranym stanie, widzę,
że ona, może, nie do końca podziela jego radość.

Odwraca się do mnie.

-- Jak to jest?

-- Chodźmy na zewnątrz -- mówię.

Wyciągamy się i~objęci ramionami, kołyszemy się na ulice. Prowadzę ich
daleko od lamp ulicznych, na plac, gdzie nie ma światła. Patrzymy na
Spieniony Kilwater, na płonącego boga, i~czekamy chwilę, póki nie
widzimy lecącej iskry przez niebo od północy na południe \emph{Jasna
Gwiazda } 

-- Chodźmy tam -- mówię. \\


Przepychasz się długimi korytarzami
statku z~niczym w~dłoniach prócz numerów. Twoi koledzy, towarzysze,
marynarze klepią cię po plecach, gratulują i~zachęcają, z~lękiem w~oczach, którego masz nadzieję, że nie pokazujesz.

Zbliżasz się do Silnika, zbliżasz się do podstawy, masz nadzieję, że to,
co robisz, to wprowadzanie numerów w~jego obcy umysł. Potwierdzasz, że
wszystko jest gotowe.

Naciskasz to, co masz nadzieję, jest właściwym przyciskiem i\ldots

\begin{center}
---
\end{center}

skaczesz, stając się światłem.

\chapter{Posłowie od tłumacza}

Ken Macleod urodził się w 1954 roku, obecnie mieszka w Gourock. Jego powieści zdobyły lub były nominowane do nagród HUGO, Nebula, BSFA,  Clarke Award, Prometheus. W Polsce wydano dotychczas tylko \emph{Dywizja Cassini}, trzeci tom tetralogii Jesienna Rewolucja.

Dlaczego w ogóle tłumaczyć twórczość Kena Macleoda? Takie pytanie na pewno sobie zada wiele czytelniczek. 

Moja odpowiedź jest następująca. Autor jest wart poznania z powodu wątków politycznych przedstawianych w powieściach, z żywej i interesującej wyobraźni oraz dobrego rzemiosła pisarskiego. 

Czegóż to bowiem nie mamy w \emph{Wieży Kosmonauty}: krakeny nawigujące statkami FTL, w ogóle loty FTL, potomkowie dinozaurów operujący latającymi spodkami, latające spodki wykorzystujące antygrawitację, nanokomputery o niemal boskich możliwościach żyjące w asteroidach, IWWWW czyli anarchosyndykalistyczny związek informatyków (tu tłumaczone jako Robotnicy Informacji WWW) i wiele innych wątków, pomysłów, idei splecionych w~interesującą historię. 

Mieszanka taka, składająca się z interesujących postaci, konceptów politycznych, konceptów science fiction i ciekawej fabuły, jest moim wzorcem dobrej powieści.

Choć \emph{Wieża Kosmonauty} jest pierwszym tomem trylogii, wybrałem tę powieść do przetłumaczenia, ponieważ chciałem się przygotować do przetłumaczenia tetralogii Jesiennej Rewolucji, debiutanckiej serii tegoż autora odpowiadającej na pytanie ,,Co jeśli kapitalizm jest nie do utrzymania, a socjalizm jest niemożliwy?''. Spoiler:~Autor proponuje odpowiedź opierając się na anarchizmie.\\

W celu maksymalnego przybliżenia tekstu, dodałem przypisy może odnoszące się do koncepcji znanych czytelnikowi. Może tych przypisów jest za wiele, ale uważam, że warto poznać każdą znaczącą aluzję autora.

Jestem przekonany, że uważny czytelnik odnajdzie wiele błędów w tym tłumaczeniu. Ponoszę za to całkowitą odpowiedzialność.\\

Jacek Hummel

Warszawa, sierpień 2020 - kwiecień 2021



\tableofcontents{}
\end{document}