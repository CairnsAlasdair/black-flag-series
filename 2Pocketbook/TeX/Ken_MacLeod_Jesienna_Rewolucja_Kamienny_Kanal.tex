\documentclass[oneside,polish,11pt,sfheadings]{mwbk}
%polonizacja
\usepackage[T1]{fontenc}
\usepackage[polish]{babel}
\usepackage[utf8]{inputenc}
\usepackage{polski} 
\frenchspacing 
\usepackage{indentfirst} 
%koniec polonizacja
%grafika
\usepackage{graphicx}
%pakiet czcionki
\usepackage{times}
\usepackage[a5paper]{geometry} %wielkość papieru (148x210-book w~PL)
%gwiazdki
\newcommand{\threeast}{\bigskip\par\centerline{*\,*\,*}\medskip\par}
\babelhyphenation{wszyst-ko worm-ho-le}

%\def -- {\ts--\hskip.25em}
%EPUB
%\usepackage{hyperref} 
%move footnotes to endnotes
%\usepackage{enotez}
%\let\footnote=\endnote
%\setenotez{
%  list-name = Przypisy
%}


%pdf anonimize
%dla EPUB wykomentować
\pdfsuppressptexinfo=-1 %Suppress PTEX.Fullbanner and info of imported PDFs

%pakiet odnośników i~pdf metadata
\usepackage[unicode, pdftex]{hyperref}
\hypersetup{pdfauthor={Ken MacLeod},
            pdftitle={Jesienna Rewolucja 2 Kamienny Kanał},
            pdfsubject={Fall Revolution -- The Stone Canal},
            pdfkeywords={tłum. Jacek Hummel, Creative Commmons, tłumaczenie CC BY 4.0, powieść, science fiction, Jesienna Rewolucja},
            pdfcreator={pdfLaTeX}}
%dla EPUB koniec wykomentowania


\begin{document}
\title{Kamienny Kanał}
\author{Ken Macleod}


%-----titlepage start
\DeclareRobustCommand{\cs}[1]{\texttt{\char`\\#1}}
\newlength{\tpheight}\setlength{\tpheight}{0.9\textheight}
\newlength{\txtheight}\setlength{\txtheight}{0.9\tpheight}
\newlength{\tpwidth}\setlength{\tpwidth}{0.9\textwidth}
\newlength{\txtwidth}\setlength{\txtwidth}{0.9\tpwidth}
\newlength{\drop}
\newcommand*{\titleSI}{\begingroup% Sagas
\drop = 0.13\txtheight
\centering
{\Huge \textsf{JESIENNA REWOLUCJA}}\\[1\baselineskip]
{\huge \textsf{FALL REVOLUTION}}\\[1\baselineskip]
{\LARGE  \textsf{TOM 2}}\\[4\baselineskip]
{\Huge \textsc{Kamienny Kanał}}\\[1\baselineskip]
{\LARGE \textsc{The Stone Canal}}\\[2\baselineskip]
{\huge \textsc{Ken MacLeod}}\\[4\baselineskip]
{\large Na podstawie wydania TOR, Nowy Jork, 2008 \\ przetłumaczył i~opracował:}\\
{\Large Jacek Hummel}\\[2\baselineskip]
{\normalsize \textit{Tłumaczenie jest dostępne na licencji\\
\href{https://creativecommons.org/licenses/by/4.0/deed.pl}{Creative Commons Uznanie autorstwa 4.0 Międzynarodowe}}\\[2\baselineskip]\par}
\vfill
{\Large {Warszawa, 2021}}\\
%\vspace*{\drop}
\endgroup}
\titleSI
\thispagestyle{empty}
%-----titlepage end

\begin{figure}[p]
    \vspace*{-1cm}
    \makebox[\linewidth]{
        \includegraphics[width=1.1\linewidth]{StoneCanal.jpeg}
    }
\end{figure}
\thispagestyle{empty}


\newpage

\vspace*{2cm}


\emph{Dla Sharon i~Michaela}

\vspace*{5cm}

\quotation{-- mamy pewność, że materia we wszystkich swoich przemianach
pozostaje wiecznie ta sama, że żaden z~jej atrybutów nigdy nie może
zaginąć, a~więc, że z~tą samą żelazną koniecznością, z~jaką materia
wytrzebi kiedyś na Ziemi najwyższy swój wytwór -- myślącego ducha -- z~tą
samą koniecznością będzie musiała zrodzić go ponownie w~innym miejscu i~w innym czasie.}

\begin{flushright}

Fryderyk Engels, \emph{Dialektyka Przyrody}\footnote{cyt. z~Wprowadzenie,
(1883), zob.~\url{https://www.marxists.org/polski/marks-engels/1883/dial_prz/index.htm}  -- przyp.tłum.}

\end{flushright}


\chapter*{Podziękowania}

Kamienny Kanał w~Trzynastym

Powieść ta obecnie może mieć czytelników młodszych niż ona sama.
Pierwszy raz została opublikowana w~1996, jej wyobrażona przyszłość od
razu zaczęła odpływać z~biegu historii, zanim wszystkie kompasy i~zegary
zostały zresetowane w~2001. Trzy rozdziały osadzone są w~realnej
przeszłości, w~latach siedemdziesiątych, osiemdziesiątych i~dziewięćdziesiątych, a~dla niektórych czytelniczek te muszą być
dziwniejsze niż te osadzone w~przyszłości. Czy ktokolwiek wtedy myślał,
że wkrótce będzie rewolucja, albo wojna jądrowa, albo, że Internet
sformatuje świat? Cóż, tak, niektórzy z~nas tak myśleli.

Przeczytaj moje wprowadzenie do \emph{Gwiezdnej Frakcji} dla
przyczyn,dla których kolejne idee, rewolucji, wojny, osobliwości, tak
typowe dla tych trzech dekad miały sens w~tamtych czasach, nawet jeżeli
nie mają obecnie. Wystarczy już o~polityce i~historii. Co mnie uderza,
przy ponownym czytaniu \emph{Kamiennego Kanału}, to jak osobista jest to
książka. Miłość i~przyjaźń, która trwa przez dekady, nawet wieki, jest
główną osią fabuły. Jeszcze dziwniejsze niż to, trwają one przez
platformy hardware i~przerwę pomiędzy różnymi typami umysłów: fizyczna
Dee i~wirtualną Meg, forma jest ludzka, ale oba umysły są sztuczne.

Jest w~tej książce wrażliwość, która nie byłaby, jak sądzę, możliwa
przed latami dziewięćdziesiątymi, a~której nie miałem jak odkryć. -- Wszystko jest analogią, interfejs -- mówi nam Wilde -- jaźń sama ma okna -- przez które rozumie Windowsy. Później upada i~jest złapany przez Meg
,,mój drogi, słodki system operacyjny''. Różnica pomiędzy ludzkim a~maszynowym jest złamane, w~każdym sensie. Wilde odnajduje się w~świecie,
którego reguły napisał, ale gdzie ta różnica -- o~której wie, że została
złamana -- jest niepisanym prawem, które ubezpiecza wszystkie inne.
Jeżeli prawa własności, jak mówi narrator nam i~Wilde kiedyś mógłby się
zgodzić, to ,,co ludzie zgadzają się, by ludzie robili z~rzeczami'', co
dzieje się z~rzeczami, które się nie zgadzają? A jeżeli jesteś jedną z~tych rzeczy, kim się stajesz?

Te pytania nie były nowe i~mogły w~praktyce nigdy nie zaistnieć, ale
natarczywość, z~jaką są podnoszone, nie jest zbyteczna. Informacja
ciągle chce być wolna. Jednak co mnie uderza, przy ponownym czytaniu,
jest to, jak pilne jest ponowne przeżywanie wspomnień dawnych bitew,
których wynik jest znany. Zdanie po zdaniu ma melancholijny rytm
przypominania. Każda postać, do którego umysłu mamy dostęp od środka,
jest, lub była, maszyną. Wszyscy są zaliczeni do zmarłych. W~tym lub
innym momencie wszyscy będziemy. To nie stoi w~sprzeczności z~nadzieją,
którą utrzymuje Wilde, że uda nam się dotrzeć na statki. Niektórzy z~nas
jeszcze mogą. Ciągle możemy mieć nadzieję, że uda nam się bez stawania
się potworami, ale nie, myślę, bez stawania się czymś innym niż
człowiek.

Nie chcę, żebyście myśleli, że to wszystko sprawia, że książka jest
poważna. Powieść była napisana z~żarliwymi nadziejami, szczęśliwymi
wspomnieniami i~entuzjazmem uczenia się pisania programów tak jak
książek. Traktuje ona wszystkie te ponure rzeczy, ludzką kondycję,
starzenie się, straty i~śmierć, jako ostatecznie rozwiązywalne problemy,
patrząc wstecz z~pewną nostalgią, z~wyobrażonych czasów, kiedy zostały
rozwiązane. Czasów, kiedy wszyscy umrzemy, tak, ale od kiedy to
wstrzymywało nas od patrzenia w~przyszłość?

Brian Aldis argumentował, że pierwszą prawdziwą powieścią
fantastyczno-naukową był \emph{Frankenstein}. Nie myślałem o~tym micie,
gdy pisałem tę książkę, ale patrząc wstecz, widzę, jak DNA się
replikuje: Wilde staje się zarazem Frankensteinem i~Stworem, Dee i~Annette pretendują do bycia Panną Młodą, a~wszyscy spotykają Wilkołaka.
To jest sposób odczytania, jako gwałtownego romansu. Ponieważ musi być
coś gotyckiego w~powieści, której pierwsze zdanie brzmi (zobacz dalej):


\part{MASZYNERIA WOLNOŚCI}
\chapter{Równoważnik Człowieka}

Obudził się i~pamiętał umieranie.

Jego oczy i~usta otworzyły się i~wciągnął długi, szorstki wdech
rozrzedzonego powietrza. Nogi kopnęły, a~palce zaszurały w~piasku. Potem
jego kończyny się rozciągnęły się i~leżał spokojnie. Każdy wdech był
prędki, jak gdyby podejrzewał, że następny mógłby być jego ostatnim.
Jego palce wbiły się w~ziemię, gdy patrzył prosto w~górę na
głęboko-niebieskie niezgłębione niebo.

Przetoczył się, podniósł się z~trudem na nogi i~rozejrzał się dookoła.
Stał na niższym zboczu niskiego wzgórka ponad kanałem. Kanał miał jakieś
dwadzieścia metrów szerokości. Ziemia na kilkuset metrach po obu
stronach była rzadko pokryta trawą i~krzewami. Poza tym ziemia była
czerwonawego koloru.

Mężczyzna patrzył w~górę i~w dół Kanału. Kanał biegł od horyzontu do
horyzontu, linia błękitu pośrodku pasa zieleni, przepoławiając wielkie
koło czerwieni pod kopułą niebieskiego. Niedaleko zenitu nieba świeciło
małe i~jasne słońce. Mężczyzna spojrzał się na nie, potem podniósł ramię
z wyciągniętym kciukiem, jak gdyby w~powitaniu. Przesunął pięść z~wyciągniętym kciukiem do przodu i~tyłu, patrząc jednym okiem wzdłuż
ramienia. Uśmiechnął się i~skinął głową.

Kilka metrów w~górę stoku z~miejsca gdzie stał, zbocze było rozbite,
pokazując skałę pod cienką warstwą gleby i~korzeni. Pomiędzy
rozbabranymi, poszarpanymi głazami leżała elipsoidalna kapsuła, na metr
długa, pół metra szeroka i~na dwadzieścia pięć centymetrów wysoka. Górna
i~dolna połowa były identyczne i~zwierciadlane. Pomiędzy nimi była
jakiegoś rodzaju środkowe pasmo, gdzie mogły być dojrzane powierzchnie
nieostre, na zawiasach lub łączone. Mężczyzna podszedł i~obejrzał to
ostrożnie. Potem przysunął się bliżej, jak gdyby z~zamiarem zbadania, i~nagle się odwrócił.

Zbiegł w~dół do krawędzi Kanału i~stał, patrząc na kapsułę przez kilka
minut. Zdjął swoje ubranie -- buty, skarpetki, ocieplana kurtka, spodnie,
podkoszulkę i~spodenki -- i~zaczął poruszać dłońmi nad swoim ciałem, jak
gdyby mył się bez wody. Potem założył swoje ubrania i~podszedł stokiem
do kapsuły.

Położył ręce na biodrach i~zmarszczył brwi. Otworzył usta, zamknął je,
rozejrzał się i~wzruszył ramionami.

-- Nazywam się Jon Wilde -- powiedział. -- Kim jesteś? -- Nie wyglądał ani
nie brzmiał, jak gdyby oczekiwał odpowiedzi.

-- Jestem maszyną równoważną człowiekowi -- powiedziała kapsuła, próbując
mówić miłym i~konwersacyjnym głosem. Mężczyzna lekko podskoczył.

-- Zamierzam wstać -- dodała maszyna równoważna człowiekowi. -- Proszę, nie
przejmuj się.

Jon Wilde odszedł na kilka kroków, jego buty strącały kamyki i~żwir w~dół stoku. Klikające, zgrzytliwe dźwięki dobiegły od maszyny, gdy cztery
metalowe kończyny rozwinęły się z~centralnej partii. Wyglądały
identycznie, z~pazurami, nadgarstkami lub kostkami, łokciami lub
kolanami. Dwie kończyny obróciły się i~przesunęły się w~dół, wbijając
łączone komponenty na ich końcach w~ziemię. Maszyna wyprostowała swoje
kończyny i~zabujała na stopach, jeżeli mogą być tak nazwane. Stała na
wysokość około połowy wysokości mężczyzny, jej postawa i~proporcje
niejasno sugerowały mężczyznę biegnącego w~bojowym kucaniu, głowa w~dół.

Wilde spojrzał z~góry na nią.

-- Gdzie jesteśmy? -- spytał.

-- Na Nowym Marsie -- odpowiedziała Maszyna.

-- Jak się tutaj znalazłem?

Nastąpiła około minutowa cisza. Wilde skrzywił się, rozejrzał się,
pochylił się do przodu, gdy Maszyna przemówiła ponownie:

-- Stworzyłem Cię.

Maszyna się odwróciła i~odeszła.

Wilde rzucił się za nią.

-- Gdzie idziesz?

-- Miasto Statku -- powiedziała Maszyna. -- Najbliższa ludzka osada. -- Zatrzymała się na chwilę. -- Poszłabym tam, gdybym była Tobą.

\threeast

Maszyna równoważna człowiekowi i~mężczyzna, którego podobno stworzyła,
poszli razem brzegiem Kanału. Co jakiś czas Mężczyzna odwracał głowę i~patrzył na Maszynę. Raz lub dwa otworzył już usta, ale zawsze znowu się
odwracał, jak gdyby pytanie lub uwaga w~jego głowie były zbyt śmieszne,
żeby je wypowiedzieć.

Po godzinie i~dwudziestu minutach Mężczyzna się zatrzymał. Maszyna
zatrzymała się po kilku krokach i~stała bujając się lekko na jej
metalowych nogach.

-- Jestem spragniony -- powiedział Mężczyzna. Woda w~Kanale była leniwa,
nakrapiana zielonymi algami. Wskazał ją oczami z~powątpiewaniem. -- Wiesz
może, czy to można pić?

-- Nie można -- powiedział Maszyna. -- I~nie mogę jej oczyścić bez zużycia
ilości energii, którą wolałbym zachować. Jednakże mogę cię zapewnić, że
jeżeli będziesz szedł, może z~okazjonalnym odpoczynkiem, będziesz pił w~Mieście Statku dzisiaj wieczorem.

-- Marsjańskie bary? -- powiedział Wilde i~się roześmiał. -- Zawsze
chciałem posiedzieć w~marsjańskich barach.

Kolejna godzina minęła i~Wilde powiedział: 

-- Hej, widzę to!

Maszyna nie musiała się go pytać. Bez zmiany kroku, gładko rozsunęły się
nogi, aż szła z~kapsułą prawie na wysokości głowy Mężczyzny i~zobaczyła
to, co Wilde widział: poszarpane nieregularności na horyzoncie.

-- Miasto Statku -- powiedziała Maszyna.

-- Zróbmy przerwę -- krzyknął Mężczyzna, podganiając, żeby dotrzymać
kroku. -- Nie ma potrzeby iść jak marsjańska machina bojowa.

Równy krok Maszyny się nie zmienił.

-- Jesteś silniejszy, niż ci się wydaje -- powiedziała. Mężczyzna dogonił
ją i~maszerowali koło siebie.

-- Podoba mi się to -- dodała po chwili Maszyna. -- ,,Jak marsjańska machina
bojowa''. Ha ha.

Jej śmiech wymagał pracy, jeżeli miałby brzmieć jak ludzki.

Szli dalej. Ich cienie wydłużały się przed nimi, a~miasto powoli
pojawiało się nad horyzontem, który, dla Mężczyzny, był nieznajomy, lecz
nie niespodziewanie bliski. Nieregularności różnicowały się w~wysokie,
jeżące się wieże połączone łukami i~smukłymi, wygiętymi mostami. Kopuły
i bloki stały się widoczne pomiędzy wieżami, pośród których matowa
inkrustacja mniejszych budynków rozkładała się z~miasta, przyćmiona
niską mgłą.

Małe słońce zaszło za nimi, a~w~ciągu piętnastu minut otoczyła ich noc.
Mężczyzna zatrzymał się i~Maszyna się zatrzymała.

Jon Wilde obrócił się kilka razy, rozglądając się od zenitu do horyzontu
i z~powrotem jak gdyby szukał czegoś, co mógłby rozpoznać. Nic nie
znalazł, w~końcu spojrzał na Maszynę, niewyraźną w~świetle gwiazd, która
odbijały się jak szron od jej kadłuba i~boków.

-- Jak daleko? -- Słowa wyszły z~suchych ust. Pomachał dłonią na
rozjarzone, marznące, zatłoczone niebo. -- Jak długo?

-- Hej, Jon Wilde -- powiedziała Maszyna. Teraz mówiła poprawnym tonem
konwersacyjnym. -- Gdyby wiedział, to bym ci powiedział. Ta sama spirala,
inne ramię, z~tego, co wiem. Mówimy tutaj o~wspomnieniach, człowieku,
mówimy tutaj o~\emph{czasie geologicznym}.

Dwa byty medytowały siebie przez chwilę, potem pośpieszyły ostatnie
kilka kilometrów do mnożących się świateł Miasta.

\threeast

Stras Cobol, przy Kamiennym Kanale. Część ludzkiej Dzielnicy. Dobre
miejsce, żeby się zgubić. Systemy inwigilacji integrowały widok...

Trzykilometrowy pas ulicy, brzeg Kanału z~jednej strony, budynki po
drugiej stronie, ich wysokość jak wykres słupkowy wartości nieruchomości
w długim spadku od wysokich wież centrum do niskich slumsów na krańcu
miasta, gdzie wpada czerwony piasek z~pustyni i~rodzinne zakłady fuzyjne
błyszczą w~ciemnościach. Na tej samej trajektorii handel wylewa się
wzrastająco zza ścian i~okien, na stragany na chodniku i~tace
straganiarzy. Na całej długości ulicy energicznie przepychają się ludzie
i maszyny, niektórzy pracujący, inni wypoczywający, gdy światło zanikało
na niebie.

Pomiędzy wszystkimi twarzami w~tłumie, coś skupia się na jednej z~nich.
Kobieca twarz, śledzona krótko, gdy przedziera się pomiędzy innymi
ciałami na ulicy. Procedury ewaluacji systemu kategoryzują sprawnie jej
wygląd: widoczny wiek około dwudziestu, wzrost około metra sześćdziesiąt
-- dobrze poniżej średniej -- masa lekko powyżej średniej. Jej wzrost jest
podwyższony w~obrębie normalnego zakresu przez buty na wysokim obcasie,
jej figura podkreślona prążkowanym swetrem z~długimi rękawami i~wąską
spódnicą, zręcznie rozciętą, że nie utrudnia jej szybkich kroków. Włosy
do ramion, grube i~czarne, kołyszą się dookoła twarzy pięknej i~niezapomnianej, ale nie włączającej żadnych przełączników w~estetyce
skalarnej Systemu, szerokie policzki, pełne usta, duże oczy z~zielonymi
tęczówkami i~nagle wąskimi, zerującymi się źrenicami, które patrzą
prosto na ukryte soczewki, które ją oglądały. Jedno oko zamyka się w~czymś, co wygląda jak mrugnięcie.

I nagle jej nie ma. Zniknęła z~widoku Systemu, jest tylko zamazaną
anomalią, unoszącym się pyłkiem na jego wizji i~niepokój przemija w~jego
umyśle, gdy uwaga jest skutecznie odwrócona przez właściciela straganu
kierującego pojemnikiem na gorący olej przez pobliskie skrzyżowanie bez
wymaganej ostrożności i~uwagi i~włącza się program ,,mamy potencjalny
problem przed sobą''...

\threeast

Jednak ona ciągle tam jest, ciągle idzie szybko, a~my ciągle jesteśmy z~nią, z~powodów, które kiedyś staną się jasne. Jesteśmy w~jej
przestrzeni, w~jej czasie, w~jej głowie.

Jej piękna mała głowa zawiera i~ukrywa prawdziwie neomarsjański umysł,
intelekt rozległy, chłodny i~bezduszny, jak powiedział Mężczyzna, i~właśnie teraz jest w~trybie bojowym. Uruchomiła Szpieginię, nie
Żołnierkę, ale Żołnierka tam jest, gotowa się włączyć na pierwszą oznakę
kłopotów. Ruch ciała jest obsługiwany przez Sekretarkę, w~trybie wolnego
czasu: jej krok to pośpiech ,,spóźniam się na randkę'' i~jak na razie
działa dobrze. Prócz tego, że szła dalej i~szybciej niż jakakolwiek
dziewczyna, która normalne by szła w~takich okolicznościach, a~skóra nad
ścięgnem Achillesa ściera się do krwi. Włącza podprogram Chirurżki i~ból
praktycznie się wyłącza.

Pozwala sobie na rozproszony blask przyjemności za wykrycie i~odwrócenie
uwagi systemu inwigilacji. Jej prawdziwe niebezpieczeństwo, wie o~tym,
pochodzi z~ludzkiej pogoni. Nie może spojrzeć do tyłu, ponieważ nie
śmiała włączyć sonaru i~radaru, ale używa każdej innej wskazówki, która
wpada jej w~oko. Każde echo, każde odbicie: w~oknach, kawałkach złomu i~błyszczących zderzakach pojazdów, nawet w~siatkówkach ludzi idących w~przeciwnym kierunku, wszystko buduje jej dookolne pole widzenia. Ciągle
aktualizowany, niesynchroniczny palimpsest, gdzie ludzie i~pojazdy w~3D
i pełnym kolorze przechodzą z~jej stożka widzenia do szerszej sfery,
gdzie stają się nierównymi postaciami z~kreskówek, szkicami sporadycznie
zablokowanymi w~kolorze, gdy kawałek szczegółu błyska z~przodu do tyłu.
(Mogłaby utrzymać renderowanie koloru, gdyby chciała, pozwolić
wirtualnej i~widocznej rzeczywistości bezszwowo się scalić, ale nie ma
mocy obliczeniowych do stracenia. Szpiegini jest wymagającym narzędziem
i zjada zasoby).

Zaznacza ostrzeżenie, niesubtelne czerwone strzałki wskazujące na jedną
z twarzy, potem kolejną, obie daleko za nią. Rzuca zbliżenie na te
odległe kropki, rozszerzając je do czegoś rozpoznawalnego, i~rozpoznaje
je. Dwóch mężczyzn, dwóch goryli zatrudnionych przez jej właściciela.
Ich nazwiska nie są w~pliku, ale widziała ich w~przelocie w~różnych
okresach przez lata.

Szpiegini analizuje ich ruch i~zgłasza, że jeszcze jej nie odkryli:
szukają, nie śledzą. Jeszcze nie.

Widzi znak baru zbliżający się na jej lewej, ,,Mila Malleya'' wypisane
musującym tęczowym neonem. Szczęśliwie, najbliższy zbliżający się pieszy
jest potężny i~idzie blisko boków budynków. Pozwala dwóm metrom
trzydziestu, dwustu kilogramowej masie giganta minąć ją, jedyną wartą
zauważenia rzeczą o~nim jest niewłaściwy kwietny zapach szamponu, który
ostatnio używał na pomarańczowej skórze -- i~gdy zasłania jakikolwiek
widok jej od tyłu -- wpada przez drzwi.

To miejsce jest śmieciowym, tandetnym pubem. Dużo drewna i~metalu.
Muzyka jest dudniącym hałasem w~tle, jak maszyny. Wentylacja nie radzi
sobie z~dymem i~ktoś już miał makową fajkę. Ryby słodkowodne są
grillowane gdzieś z~tyłu. Niski sufit, przyćmione światła. Jej wizja
dopasowuje się bez mrugnięcia i~jest dzień, plus minus dziwna długość
fal. Szpiegini przejmuje dowodzenie na obserwację, drugie, długie
obejrzenie pomieszczenia. Wykrywa inwigilację, oczywiście, ale to tylko
własny system zajazdu, dokładnie tak mądry i~niebezpieczny jak pies. I~tak je pinguje, zostawiając system z~niskomocowym przekonanie, że ta
\emph{osoba}, która właśnie weszła, jest \emph{miła} i~właśnie poklepała
po głowie i~może być od teraz bezpiecznie zignorowana.

W ,,Mili Malleya'' jest kilka tuzinów ludzi: pracownicy farm, mechanicy na
stołkach barowych i~pracownicy biurowi, w~większości młode kobiety,
dookoła okrągłych stołów. Wyglądają jakby przyszły tutaj na drinka w~drodze do domu po pracy i~zostały na kilka więcej. Dobrze. Zauważa
ogłoszenie: żadnej ukrytej broni. Wyjmuje pistolet z~torebki, którą
nosi, i~przylepia go do pasa sukienki, podchodzi do baru. Dziewczyny
dookoła stołu zauważają ją, mężczyzna na stołku zauważa ją, ale to
dlatego, że jest piękna, nie dlatego, że wygląda nie na miejscu.

Barmanem jest kolejny gigant, jakiś wzmacniany intelektualnie
gigantopitek lub cokolwiek (nigdy nie miała okazji uporządkować rodzajów
człowiekowatych), który smutnie opiera się na łokciach, nadgarstki
wiszące nad krawędzią baru. Odwraca się od gladiatorów na telewizorze i~uśmiecha się do niej, lub w~każdym razie odsłania żółte kły.

-- Ta?

-- Dark Star, proszę.

Bez wstawania barman sięga po butelki i~miesza rum i~colę.

-- Lyd?

-- Tak, proszę. -- Jest ostrożna ze szczelinowymi głoskami. Trudno się
oprzeć impulsowi do naśladowania (właściwie to błąd w~Szpiegini).
Pozwala Szpiegini zająć się procesem płacenia, wybrania właściwego
brudnego banknotu z~jej zwędzonej kolekcji weksli. Wartościami złota
może się zajmować w~dowolnej ramie umysłu, ale zbiory, części maszyn,
ziemia i~czas pracy są obce dla większości z~nich.

Lód stuka, gdy zabiera drinka do niezajętego stolika najbliżej
najdalszej ściany. Siada plecami do tej ściany. Kładzie torebkę i~pistolet, od niechcenia, na stole. Sączy drinka, zapala papierosa i~obserwuje drzwi, jak gdyby czekała na przyjście przyjaciół lub sympatii.

Dwie fotograficzne twarze, obecnie unoszące się w~jej oprogramowaniu
rozpoznawania wzorów i~celowania, mogą wejść przez te drzwi w~każdej
minucie. Jeżeli ma szczęście, nie wiedzą, że jest uzbrojona. Prawie na
pewno nie wiedzą o~Szpiegini, Żołnierce i~wszystkich innych programach,
które załadowała. Oczekują Sekretarki, Seks, i~Samej, które pomiędzy
sobą nie potrafią zdziałać nic więcej niż kopnięcie, ugryzienie czy
podrapanie. Mogą sobie z~tym poradzić, a~co do innych tutaj\ldots gdy tylko
goryle pokażą swoje dokumenty, klienci będą obserwować, jak ją wyciągają
z tego miejsca z~całą empatią, solidarnością, współczuciem i~troską,
jaką okazaliby odzyskaniu ukradzionego pojazdu.

Niemniej są ludzie w~tej dzielnicy, którzy nie traktują rzeczy w~ten
sposób, i~jeżeli faceci od odzyskiwania nie wejdą i~nie odnajdą jej, lub
jeżeli wejdą, ale ucieknie, to zacznie szukać ludzkich sojuszników na
uliczkach z~tyłu.

To wszystko, co może być. Jej właściciel mógł już odkryć, jaki hardware
i software zapakowała, i~może wysłał kogoś, lub coś, groźniejszego po
nią.

Patrzy się na drzwi, a~palce trzyma blisko pistoletu.

~

-- Mówią tu po angielsku?

Wilde szurnął nogą powierzchnię ścieżki na brzegu kanału -- zmieniła się
od zdeptanego pyłu do paska stopionego piasku, który rozszerzył się i~połączył z~ulicą przed nimi, stała droga stworzona z~tego samego
materiału, jak gdyby palec boży narysował linię z~kosmosu -- i~czekał na
odpowiedź maszyny.

Miasto rozrosło się na horyzoncie, gdy się zbliżali, w~końcu w~potężne,
nieostro, wyglądającą organicznie, zbieraninę strzelistych ostrych wież,
ich widoczna struktura jak wnętrze kości lub szkielety stworzeń
morskich, ich zarysy podkreślone przez światła. To, co wyglądało z~dystansu jak jakieś matowe podszycie, okazało się teraz peryferiami
niskich budynków, które, w~odróżnieniu od slumsów, które znał Wilde,
wydawały się rozciągać się przez środek miasta, na którego krawędzi
teraz stali. Po lewej i~prawej były pola. Masywne poruszające się
obecności maszyn na tych polach były jedynym ruchem, na który dotychczas
natrafili. Światła ich minęły, ale było trudno powiedzieć, czy były
naturalne, czy sztuczne. Raz, coś ogromnego, cichego i~pozostawiającego
zielony powidok lub ślad, rzuciło się nad ich głowami, and miastem i~błysnęło dalej z~odległości.

-- Wodospad -- wyjaśniła Maszyna, nieprzydatnie.

Teraz przeniosła się na stopy i~odpowiedziała na pytanie Wilde'a. 

-- Będziesz zrozumiany -- powiedziała z~wahaniem. -- Angielski jest
dominującym językiem. Twoje słownictwo i~akcent, tak jak mój, mogę
dodać, może wydawać się nieco osobliwe.

-- Zanim pójdziemy dalej -- powiedział Wilde, jego wzrok przeskakiwał od
budynków pod pierwszymi lampami przed nimi do maszyny -- musisz mi parę
rzeczy wyjaśnić. Pierwsze, czy normalnym jest rozmawianie z~maszyną? Mam
na myśli, czy\ldots roboty? \ldots takie jak Ty są tutaj powszechne?

-- Możesz tak powiedzieć -- powiedziała sucho Maszyna.

-- Ok. Następna sprawa na liście, jeżeli o~mnie chodzi, to załatwienie
czegoś do picia, jedzenia oraz miejsca do spania. Czy dobrze myślę, że
będę musiał za to zapłacić?

-- O tak -- powiedziała Maszyna.

-- A przypadkiem nie masz jakichś pieniędzy schowanych w~tej Twojej
skorupie?

-- Nie, ale mogą zrobić więcej niż to. Widzisz ten drugi budynek wzdłuż
drogi? To bank spółdzielczy.

Wilde nic nie powiedział, choć otworzył usta.

-- Pamiętasz co to takiego, prawda?

Wilde się roześmiał. 

-- Więc dostanę gotówkę, zastawiając moją własność?
-- Wskazał na swoje ubrania, które miał na sobie. -- To niedużo pomoże...

Maszyna wydała godną symulację uprzejmego kaszlnięcia.

-- Och. -- Wilde spojrzał na nią z~nowym, spekulacyjnym zainteresowaniem.
-- Rozumiem.

Ruszył wzdłuż drogi, przed Maszyną po raz pierwszy, od kiedy się
poznali. Maszyna szarpnęła w~ruch za nim.

-- Tylko nie zrozum tego źle -- powiedziała, jej głos tak sztywny, jak
chód.

~

Jedna z~dziewcząt przy najbliższym stoliku dawało pokaz piosenki pubu z~autentycznym, strasznym akcentem, pełnym ckliwej tęsknoty.

\emph{- Kebych mog przejść przez tęcza}

\emph{kero świyci bez Mila Malleya...}

Jaźń wie, że Mila Malleya jest realnym miejscem, i~że oba, poczucie
straty i~efekt tęczy, odnoszą się do aspektów jej rzeczywistości,
która\ldots dziwnie, albo czy to tylko część programu? \ldots sprowadza łzy do
nawet jej chłodnych oczu. Naukowczyni jęczy o~tym, ale nie chce wiedzieć
o tym teraz.

Właśnie usadowiła się z~trzecim drinkiem, zamieniając alkohol prosto w~energię, ale pamiętając, żeby symulować wpływ, kiedy drzwi otwierają się
z grzmotem i~wchodzi dziewczyna, która na pewno nie jest pracownicą
biurową decydującą się rozpocząć tutaj weekend.

Jest wysoka i~szczupła, choć jej kamizelka kuloodporna sprawia, że
wygląda obszernie. Wąskie dżinsy, trampki, wielki automat w~kaburze na
biodrze. Na drugim biodrze opiera dużą torbę z~paskiem wrzynającym się w~jej ramię. Krótkie blond włosy leżą gładko na czaszce. Twarz zbyt chuda,
żeby być ładna. Główna rzecz działająca na jej korzyść to jej jasne
niebieskie oczy i~wielki uśmiech, który w~tym momencie zwrócił się na
mężczyzn przy -- oraz mężczyznę za -- barze.

Podchodzi do baru i~zamawia piwo, i~gdy je pije, rozmawia z~jednym lub
drugim facetem, a~kiedy rozmawia, sięga do swojej wielkiej torby i~wyciąga świeżo wyglądającą gazetę i~starannie liczy monety, od mężczyzn,
którzy je biorą. Niektórzy z~nich biorą je, jakby chętnie je czytali,
inni z~pokazem niechęci i~przekomarzaniem się, ale większość potrząsa
głową, lub wzrusza ramionami i~wraca do ich własnych rozmów i~oglądania
ekranu telewizora, gdzie ktoś właśnie rozpoczyna strzelaninę. Przez cały
czas dziewczyna co chwilę rozgląda się po sali w~sposób, który rozdziera
Szpieginię pomiędzy podziwem nad dyskretnym sposobem wykonania i~lękiem,
że patrzy na kogoś całkiem bliskiego małego serca Szpiegini, mianowicie
Samej.

Dziewczyna przy barze rozmawia z~mężczyznami przez kilka minut, potem od
niechcenia schodzi ze stołka i~bierze garść papierów, próbując sprzedać
je dziewczynom z~biura. Odnosi sukces tylko przy jednym stole, a~potem
idzie do ostatniego stołu, gdzie samotnie siedzi ciemnowłosa kobieta.

Strzał odbija się echem. Dwie ręce szarpią się w~kierunku pistoletów,
potem cofają się, gdy nierówne wiwaty z~ekranu oraz od widzów wskazuje,
że to była wymierzona kara śmierci.

A potem, uśmiechając się i~potrząsając głową, ona stoi tam patrząc w~dół. 

-- Nerwowe dzisiaj jesteśmy, co? -- mówi.

Szpiegini i~Żołnierka są w~istocie nerwowe, przepychając się o~posiadanie, a~wszystko, co Szpiegini może zrobić to zmodulować ostrą
komendę Żołnierki w~gładką, cichą prośbę: 

-- Po prostu nie stój pomiędzy
mną a~drzwiami.

Wysoka kobieta rozsądnie robi krok w~bok. Patrzy zdziwiona, ale nie
odchodzi.

-- Cześć -- mówi. -- Nazywam się Tamara. A Ty?

Sama przejmuje. Trzyma jej ręce, gdzie są.

-- Dee -- mówi. -- Dee Model.

-- Ach -- odpowiada Tamara. -- Rozumiem. -- Jej oczy lekko się rozszerzają,
gdy to mówi, potem patrzy w~bok, jak gdyby, przez chwilę, była w~rozterce. -- Mogę się przysiąść?

Dee wskazuje jej, żeby po prostu to zrobiła. Siada po prawej stronie
Dee, pomiędzy nią a~barem.

-- Co to za gazeta, którą sprzedajesz? -- pyta Dee.

Tamara przesuwa kopię przez stół. Na winiecie jest napisane
\emph{Abolicjonista} w~dziwnej, nieregularnej czcionce z~ostrymi
szeryfami. Artykuły, które Szpiegini asymiluje w~około dwie sekundy i~które stopniowo przesącza do Samej, są dziwnym miksem: wiadomości o~sporach pracowniczych, artykuły techniczne o~asemblerach, reaktorach i~tak dalej, jakieś kolumny paranoicznych plotek o~działaniach różnych
ważnych osób, w~których nazwisko właściciela Dee pojawia się tu i~tam,
oraz długie, chaotyczne, teoretyczne szpalty o~inteligencji maszyn.

Dee odkłada ją, wydaje się, że rzucając tylko zwykłe, powierzchowne
spojrzenie. Zastanawia się przez chwilę, czy to nie pułapka, ale
Szpiegini myśli, że to mało prawdopodobne: to są dokładnie tego rodzaju
idee, które miała nadzieję odnaleźć w~tym rejonie, i~jest oczywiste, że
poparcie Tamary dla nich jest całkowicie znajome, może z~rezygnacją, dla
osób dookoła niej. (To, że ci dookoła niej mogą być częścią jakiegoś
rozbudowanej pułapki, nie zastanawia Dee, ani nawet Szpiegini: choć ich
tło jest bogate w~intrygi i~zdrady, brak im rozgałęzionej konspiracyjnej
wyobraźni, która byłaby ich drugą naturą, gdyby żyli w~państwie). Dee
próbuje nie pokazać dzikiej nadziei w~głosie.

-- Czy uważasz, że maszyna równoważne człowiekowi są, cóż, równoważne
człowiekowi? Że mają prawa?

-- Och oczywiście -- odpowiada Tamara. -- A Ty nie?

-- Hmm -- mówi Dee. -- Pozwól mi postawić ci drinka.

Kiedy wraca, niesie torbę Tamary. Wsuwa ją pod stół i~kładzie pistolet z~powrotem na stole. Tamara gestem odmawia ofercie papierosa. Dee zapala i~pochyla się blisko. Żołnierka zajmuje drugie miejsce od Szpiegini, która
nie lubi tego, co się dzieje. Wszystko, co może zrobić Szpiegini to
upewnić się, że nikt nie podsłucha. Kolejna sonda w~elektronikę sali i~głośność muzyki wzrasta o~kilka decybeli.

-- Jestem maszyną -- mówi Dee.

Tamara oczywiście to podejrzewała, z~samego imienia, ale tak samo
oczywiście nie do końca w~to wierzyła.

-- Mogłabyś mnie oszukać, dziewczyno -- mówi.

Dee wzrusza ramionami. 

-- Większość mojego ciała wyrosła w~kadzi, czy
coś. Większość mojego mózgu jest sztuczna. Technicznie i~legalnie jestem
odmóżdżonym klonem kierowanym przez komputer. Żaden komponent nie jest
niczym innym jak obiektem, ale \emph{ja} czuję się jak osoba.

Tamara energicznie kiwa głową, w~sposób, który robią ludzie.

-- I~potrzebuję Twojej pomocy -- dodaje Dee. -- Uciekłam i~agenci mojego
właściciela szukają mnie na tej ulicy.

Głowa Tamary przestaje się ruszać, a~jej usta się otwierają.

-- O kurde -- mówi.

Dee patrzy się na nią. 

-- O co chodzi? -- pyta. -- Czy to nie jest to, co
chcesz? -- Rzuca spojrzenie na \emph{Abolicjonistę}. -- Czy to jest
wszystko...

Tamara zamyka oczy na chwilę i~lekko potrząsa głową. 

-- To nie tak -- mówi, wyglądając na zawstydzoną. Układa palce do boków nosa i~mówi cicho
w tę wystarczającą maskę. -- Oczywiście pomogę Ci\ldots Pomożemy Ci. To
tylko\ldots to nie jest główna sprawa, którą się zajmujemy, wiesz?
Przekonaliśmy kilku ludzi do uwolnienia maszyn, ale maszyna samo się
uwalniająca nie zdarza się zbyt często. Tak czy inaczej, nie żebyś
gdzieś o~tym usłyszała. -- Uśmiecha się znowu, z~powrotem na torach. -- Chcesz o~to walczyć?

-- Jestem gotowa na dowolny rodzaj walki -- mówi Dee. -- Kto to ,,my''?

-- Pół ulicy pełnej anarchistów -- mówi Tamara.

Dee nie rozumie, co to znaczy, dokładnie, ale brzmi obiecująco,
szczególnie w~sposób, w~jaki mówi to Tamara.

-- Czy możesz zapewnić azyl? -- pyta Dee.

-- Najprawdopodobniej jesteśmy Twoją najlepszą opcją -- mówi z~roztargnieniem Tamara. -- Nigdy nie odbyła się dobra walka w~tej sprawie.
To byłoby coś, być tymi, którzy to zaczną. Cholerne piekło. To mogłoby
wstrząsnąć Miastem, całą przeklętą planetą!

Dee próbuje wymyślić powód, dlaczego tak miałoby być, ale oprócz
odrobiny machania ręki od Naukowczyni wydaje się, że nie ma żadnych
informacji w~pliku.

-- Dlaczego? -- pyta.

Tamara patrzy się na nią. 

-- Jesteś zdecydowanie maszyną -- mówi,
uśmiechając się przez jej dłoń. -- Albo znałabyś odpowiedź.

Dee zastanawia się nad tym, próbując sformułować wypowiedź z~samych
podpowiedzi Naukowczyni.

-- To z~powodu szybkiego ludku, prawda? -- sugeruje bystrze. -- A
Nieożywieni?

Brwi Tamara unoszą się szybko na ułamek sekundy. 

-- To mądre zmartwienie
-- mówi. -- To głupie zmartwienia są prawdziwym problemem\ldots Myślę, że to
odkryjesz. Jednak. Agenci pewnie trzymają się na zewnątrz?

Dee myśli o~tym.

-- Nie -- mówi. -- Nie teraz. Jednak mogą być inni.

Tamara dopija napój. 

-- Chodźmy -- mówi.

Właśnie się zbierają, gdy drzwi się otwierają i~wchodzą młody mężczyzna
i stary robot. Mężczyzna wygląda mizernie i~ma na sobie ubranie
pustynne, a~robot to tylko standardowy sprzęt budowlany. Tamara nawet
nie patrzy na nich, ale Dee obserwuje, jak Mężczyzna zatrzymuje się w~drzwiach i~rozgląda się po pomieszczeniu z~zaciekawionym przejęciem.

Widzi ją i~jego wzrok się zatrzymuje.

Robi krok do przodu. Jego twarz wykrzywia się jakby pod przyśpieszeniem
w okropną, udręczoną maskę, bardziej skrzywienie cech niż wyrażenie,
jest nieczytelna, nieludzka. W~tym samy czasie Dee czuje, jak czujniki
robota skanują jej ciało i~uderzają w~mózg. Szpiegini, Żołnierka i~Systema poruszają się oszałamiająco szybko w~wirtualnych przestrzeniach
jej umysłu, odpierając atak hakerski. Jej własne obronne próby hakowania
są odchylone przez jakąś osłonę tak nieprzepuszczalną, jak -- i~pewnie
nie bardziej niż -- twarda metalowa skorupa robota. Robot wykonuje nagły
przechył do przodu, gdy Mężczyzna robi drugi krok w~jej kierunku.
Wszystkie jaźnie Dee zaczynają krzyczeć na nią, żeby uciekała.

Trzyma pistolet w~obu dłoniach przed sobą, a~stół jest przewrócony
kopniakiem, Tamara koło niej. W~barze zapada cisza, oprócz dudniącej
muzyki i~ryków widowni stadionu w~telewizji.

-- Z~tyłu! -- rzuca Tamara przez zaciśnięte zęby. Obraca się, kierując Dee
na prawo, idąc tyłem, przepychając się przez drzwi, które zatrzaskują
się przed nimi. Są w~korytarzu, ciemnym, oprócz smug żółtego światła i~ciężkiego od zapachu piwa i~ryb.

Dee wzmacnia swoją wizję i~widzi mocno mrugającą Tamarę, jak obraca się
dookoła. Ze sposobu, w~jaki się porusza, jest oczywiste, że Tamara widzi
w ciemnościach przynajmniej tak dobrze jak Dee.

-- No chodź! -- woła Tamara i~pogrąża się w~korytarzu. Dee zrzuca buty,
łapie je i~biegnie za Tamarą, w~dół po schodach i~dookoła rogów w~jeszcze ciemniejszy, śmierdzący korytarz, w~istocie tunel. Dee słyszy
ruch ulicy nad głową i~wyczuwa z~każdym krokiem narastającą wilgotność w~powietrzu. Rzuca spojrzenie do tyłu i~nie widzi oznak ścigania. Woda w~powietrzu smakuje rdzą, gdy zatrzymują się przed ciężkimi metalowymi
drzwiami na końcu. Tamara grzebie przy ryglach na górze i~dole drzwi, aż
brzękają. Zatrzymuje się, słucha, potem powoli otwiera drzwi, trzymając
się za nimi, aż drzwi są prawie równolegle do ściany. Wygląda zza nich
przez chwilę, patrząc wgłąb i~nie do tyłu.

-- Czekaj -- szepcze. Ostrzeżenie nie jest konieczne: Żołnierka się
włączyła i~Drzwi stoi płasko przy ścianie tunelu dwa metry od drzwi i~bardzo powoli przesuwa się do przodu. Gdy jej stożek widzenia rozszerza,
widzi, że drzwi otwierają się na wąską kamienną półkę, ledwo ponad
powierzchnią Kanału, który w~tym punkcie jest szeroki na pięćdziesiąt
metrów. Światła z~przeciwnej ulicy, Rue Pascal, odbijają się w~pociętych
czarnych falkach kanału, zmąconych przez częste fale kursujących łodzi.
Z dźwięku uderzeń i~bulgotu wody wie, że zewnętrzny motor, tuż na
granicy jej wzroku, należy do małej łódki przycumowanej blisko drzwi.

Na metrowej szerokości nabrzeżu porusza się cień, jej własny.

Odwraca się, żeby spojrzeć w~tunel. Światło, daleko z~tyłu korytarza,
się właśnie pojawiło i~coś się porusza pomiędzy tym miejscem a~źródłem.
Tamara, chwilę później, też to zauważa i~wychodzi zza drzwi. Patrzy na
Dee, wskazuje na zewnątrz, a~potem robi dwoma palcami ruch siekający od
lewej do prawej. Razem wyskakują zza drzwi, obracają się w~przeciwnych
kierunkach, gdy stabilizują pozycję, kucając na nabrzeżu.

Dee widzi wymurowany brzeg kanału sięgający trzy metry do poziomu ulicy
i nabrzeże wybudowane wzdłuż kanału do skrzyżowania kilkaset metrów
dalej. Łodzie i~barki są przycumowane wzdłuż brzegu, drzwi i~wyjścia
tuneli przerywają go. Nikt w~tym momencie się nie porusza.

Ponad jej ramieniem widzi podobny widok w~przeciwnym kierunku, prócz
tego, że kanał rozszerza się w~ciemność pustyni. Słyszy przynajmniej
jeden komplet biegnących kroków, teraz około połowy drogi w~tunelu.
Wskazuje gorączkowo Tamarze.

-- Wsiadaj na łódź! -- mówi Tamara. Ciągnie za linę i~mały ponton uderza o~krawędź nabrzeża. Ledwie się buja, gdy Tamara wsiada, kołysze się dziko,
gdy Dee naśladuje. Kładzie się płasko na plecach na mokrym dnie łodzi na
swojej torebce i~butach, jej stopy przeszkadzają Tamarze, gdy ludzka
kobieta odbija i~włącza silnik. Dee jest zadowolona ze swej niegodnej
pozycji, gdy Tamara otwiera przepustnicę i~jęk silnika wzrasta do
krzyku, a~przód łodzi się unosi. Łódź wypływa na wodę i~Tamara wprowadza
ją na długą krzywą, która zabiera ich na środek kanału pośród krzyków i~przekleństw z~innych łodzi w~czasie gdy odległe postaci pojawiają się
przy wyjściu z~tunelu.

To Mężczyzna, który ją rozpoznał. Krzyczy za nimi, ale cokolwiek mówi,
jest utracone w~dźwięku silnika. Tamara znowu obraca rumpel i~wykręcają,
pryskając wodą, kierują się na otwarcie, mijając Stras Cobol i~w boczny
kanał bez okien, który biegnie pod wysokimi murami mniej niż pięć metrów
od siebie. Tamara zmniejsza obroty silnika i~Dee ostrożnie siada.

-- Szczęśliwe dla nas była łódź -- mówi.

Tamara parska. 

-- To moja łódź! Zostawiłam ją tam godzinę temu, kiedy
zaczęłam swoją rundę po barach.

Dee uśmiecha się słabo. 

-- Dokąd zmierzamy?

-- Plac Okrągły -- mówi Tamara. -- Okręg żywych umarłych. Rojący się o~złych artystów, wolnomyślicielskich maszyn i~anarchistów wykłócających
się co robić w~anarchii. Bezpiecznie.

Dee nie jest pewna jak to rozumieć.

-- Dzięki za wyciągnięcie mnie.

Tamara patrzy koło Dee na ciemną wodę. 

-- Ta, cóż\ldots muszę przyznać, że
nie jestem pewna z~\emph{czego} Cię wyciągnąłem. Ten facet i~robot nie
wyglądali mi na łapaczy. Rozpoznałaś ich, czy co?

Dee już przeszła przez to w~głowie. 

-- Nie -- mówi, jej głos chłodny. -- Ale on mnie rozpoznał. Jestem tego pewna.

-- Ja też -- mówi Tamara sucho. -- Tylko wydaje mi się, że znał cię na
żywo. Wyglądał, jakby chciał cię zabić, w~tej pierwszej chwili. Tak czy
inaczej, kogoś zabić, ale kurde, to mógł być szok, czy coś, hej! -- Patrzy na twarz Dee. -- Nie jesteś jedną z~\emph{nieożywionych}, prawda?
Ty i~on mogliście być wcześniej. -- Wygląda na całkiem zadowoloną z~tej
spekulacji. -- W~porządku, możesz mi powiedzieć. Nie mamy nic przeciwko
Martwym tak jak maszynom, ok?

Dee nie wiem zbyt dużo o~Nieożywionych. Kiedyś, kiedy była nowa,
myślała, że mogła usłyszeć Nieożywionych: przycisnąć ucho do ściany i~usłyszeć ich rozmawiających, wściekle, w~martwych językach. Jednak to
był tylko szum maszynerii, szpiku w~zimnych kościach miasta.

Tak powiedział jej właściciel, jego śmiech prawie uprzejmy. Ostrzejszym
tonem dodał: 

-- Nieożywieni odeszli. I~nie wracają. Większość z~nich...
och, zapomnij.

A ona posłusznie zapomniała.

Nie była pewna, czy powinna być zdenerwowana podejrzeniem Tamary, ale to
w końcu tylko ludzkie ograniczenia tej kobiety: w~ten sam sposób ona
popełniła ten sam błąd animalizacji -- myśląc, że maszyneria, która
brzmiała żywo, musiała być przynajmniej martwa -- że sama wróciła do
chwili, gdy po prostu uruchamiała swój mózg.

Więc uśmiecha się zadowolona do Tamary i~mówi: 

-- Możesz przeskanować
moją czaszkę, jeżeli chcesz, i~sama zobaczysz.

-- Załóżmy, że Twoje ciało jest kopią? Klonem?

Dee nie pomyślała o~tym wcześniej, a~idea wstrząsa nią bardziej, niż
chce przyznać. Wzrusza ramionami. 

-- To możliwe.

-- Proszę bardzo -- mówi Tamara. -- To sprawiłoby, że cokolwiek było z~tym
facetem to tylko przypadek błędnej tożsamości. Bez problemu.

Znowu gazuje silnik. Zmiecione z~półek na ścianach, obudzone fokoszczury
piszczą z~oburzeniem.

~

-- To nie \emph{ona} -- powiedział Robot, jego głos bardziej jak radio na
niskiej głośności niż człowiek mówiący cicho. -- Więc zapomnij o~tym.
Gonienie za nią nigdzie cię nie doprowadzi. On jest tylko pierdoloną
maszyną.

Wilde przebrnął z~powrotem tunelem, przeprosił barmana, zapłacił za
zniszczenia i~zamówił mocny drink tak jak duże piwo do jego grillowanej
ryby. Robot, podpierając się krzesłem naprzeciwko niego, nie przyciągnął
żadnego komentarza.

Wilde wytarł usta grzbietem dłoni i~spojrzał na Maszynę.

-- Nie wyglądała jak maszyna. Wyglądała jak prawdziwa kobieta. Wyglądała
jak...

Przestał, w~pewnym strapieniu.

-- Sklonowana -- powiedziała Maszyna nieubłaganie.

-- Ale dlaczego? Dlaczego ona? Kto mógłby...?

Patrzy na nieprzenikliwą kapsułę. 

-- Nie!

-- Tak -- powiedziała Maszyna. -- On tutaj jest.


\chapter{Ludzie Plejstocenu}

Pamiętam go opierającego się o~bar w~Queen Margaret Union\footnote{związek
studencki,
więcej~\url{https://en.wikipedia.org/wiki/Queen\_Margaret\_Union} -- przyp.tłum.}, czekającego
na nasze piwa, i~mówiącego: 

-- Będziemy tam, Wilde! Zobaczymy to! \emph{Jeden} pierdolony komputer, to
wszystko, czego potrzeba, jedna maszyna, która jest mądrzejsza niż my i~stąd odejdą.

Oczy Reida błyszczały, jego głos szczęśliwy. Tak się zachowywał, kiedy
owładnęła nim idea i~prorokował. Teraz to brzmi dość proroczo, ale nie
była to oryginalna idea nawet wtedy w~grudniu 1975 roku. (Przy okazji,
n.e.). Wziął ją z~książki.

-- Jak to, ,,stąd''? -- spytałem.

-- Jeżeli -- powiedział, zwalniając -- możemy stworzyć maszynę, która jest
inteligentniejsza niż my, to ona może stworzyć maszynę, która jest
inteligentniejsza niż ta pierwsza. I~tak dalej, coraz szybciej.
Uciekająca ewolucja, człowieku.

-- A gdzie nas to zostawia?

Reid pchnął ciężki kubek cydru w~moją stronę.

-- Z~tyłu -- powiedział szczęśliwie. -- Jak małpy w~mieście ludzi. Chodź,
znajdźmy miejsce.

Oryginalny Związek Studentów Uniwersytetu w~Glasgow powstał, zanim
kobiety były przyjmowane na studia. Ciągle jeszcze nie do końca
nadrobiły. Studentki miały własny budynek związku, QM, który wpuszczał
studentów obu płci. Dlatego był to ten, w~którym bywali radykalniejsi i~postępowi studenci, oraz był znacznie lepszy do podrywania dziewczyn.

Co było tym, co mieliśmy w~głowie: kilka piw z~kumplami przy barze przez
pierwszą część wieczoru, potem na dyskotekę około dziesiątej, zobaczyć,
czy ktokolwiek zechce tańczyć. Powód, żeby wcześniej napić się tyle, ile
można, był taki, że stanie w~kolejce przed barem w~dyskotece było
zarezerwowane, kiedy musiałeś kupić kolejkę swoim towarzyszom lub,
lepiej, drinka dla dziewczyny, która właśnie z~Tobą tańczyła.

Bar -- związku raczej niż dyskoteki -- był dość cichy o~tej porze
wieczoru. Zatem znaleźliśmy dobre miejsce w~barze, który obiegał dookoła
tylnej ściany, z~którego mogliśmy widzieć wszystkich, którzy weszli, i,
tylko lekko podnosząc i~obracając, mogliśmy sprawdzić stan zabawy na
parkiecie poniżej.

Wyciągnąłem cienkiego papierosa Golden Virginia i~podniosłem kufel
Strongbow.

-- Zdrowia -- powiedział Reid.

-- Slainte -- odpowiedziałem.

Uśmiechnęliśmy się na nasze masakracje narodowych toastów drugiego, dla
mnie, Reid powiedział coś w~rodzaju ,,Ztrofia'', a~dla niego ja
wypowiedziałem ,,Slendge''. Reid był z~Wyspy Skye, gdzie jego
prapradziadek trafił do pracy jako owczarz po Oczyszczeniu\footnote{
wyrzucenie ludzi z~ziemi w~XVIII i~XIX wieku,
zob.~\url{https://en.wikipedia.org/wiki/Highland_Clearances} -- przyp.tłum.}. Ja byłem z~Północnego Londynu, obaj byliśmy trochę nie
na miejscu w~środkowej Szkocji. Nie znaliśmy się zbyt długo, spotkaliśmy
się miesiąc wcześniej na seminarium na temat Komunizmu wojennego\footnote{
zob.~\url{https://pl.wikipedia.org/wiki/Komunizm_wojenny}  -- przyp.tłum. }. Seminarium było sponsorowane przez \emph{Critique},
lewicowe odgałęzienie Instytutu Studiów Sowieckich, gdzie robiłem
jednoroczny kurs magisterski z~Ekonomii Socjalizmu.

Nie zgadzałem się z~ich ideami, ale uznałem klikę \emph{Critique} (jak
ich prywatnie nazywałem) za sympatyczną i~stymulującą. Byli Instytutem
Młodych Turków, Lewicową Opozycją, Gabinetem Cieni i~Rządem na
Uchodźstwie. Uznawali zarówno główny nurt, jak i~marksistowskie teorie
krytyczne Związku Radzieckiego, a~wszystko to ze staromodną naiwnością
towarzysza podróży, zakładającego, że był to co najmniej nowy system,
kiedy z~trudem było to społeczeństwo.

Seminarium było sesją obiadową. Jak zawsze, było zatłoczone, nie z~powodu popularności, ale z~powodu sprytnej taktyki rezerwowania sali
trochę mniejszej niż oczekiwana obecność. W~tym źle dobranym
zgromadzeniu uchodźców -- z~Ameryki, Chile, Południowej Afryki i~samej
Drugiej Strony -- Reid, zgarbiony w~nowej kurtce dżinsowej, ciągle
palący, sapiący, jego gęste ciemne włosy opadające na jego młodą,
przystojną, ale jakoś wyblakłą twarz, wydawał się całkowicie w~domu, a~pytania, które zadawał mówcy na koniec, pokazywały, że przynajmniej
wiedział, o~czym mówił. Niemniej żaden z~nas nie widział go wcześniej, a~później w~pubie (te seminaria miały kilka cech wspólnych z~zebraniami
socjalistycznymi, szczególnie w~pubie potem), przyznał się do bycia
trockistą, co nie było zaskakujące, i~bycia studentem nauk
informatycznych, co było.

Kobieta siedząca koło mnie była Amerykanką i~także trockistką. Reid
wstawał, żeby kupić kolejkę i~spytał ją: 

-- Co Ci przynieść?

-- Sok pomidorowy -- odpowiedziała. Kiwnął głową, marszcząc brwi.

-- Jak to się stało, że go nie poznałaś, Myra? -- spytałem, odszedł
ociężale do baru. -- Czy też nie jesteś w~IMG\footnote{grupa trockistów w~latach 1968-1982, więcej~\url{https://en.wikipedia.org/wiki/International_Marxist_Group} -- przyp.tłum.}? -- Podłapałem to w~trakcie dyskutowania z~nią
przypadkowo nad kawą w~Instytucie, prawie zagadując, żeby być uczciwym,
ponieważ byłem nią oczarowany. Była wysoka, nieprawdopodobnie szczupła,
z blond fryzurą typu ,,bob'' i~zuchwałą, mizerną twarzą, wklęsłości jej
oczu i~policzków wyglądały delikatnie, przyjemnie wygładzone w~kształt
dużych kciuków, jej szare oczy bystre za dużymi okrągłymi szkłami.

-- Nie chodzę za bardzo na spotkania -- przyznała z~potrząśnięciem głowy.
-- Znaczy, denerwuję się na towarzyszy ponaglających mnie do większej
walki przeciwko pierdolonej fakcji leninowsko-trockistowskiej? Mam na
myśli, myślą, że przybyłam do Anglii, uciekając przed \emph{czym} ?

-- Chodzi Ci o~Szkocję, Anglię? -- Wycedziłem szyderczo, nie mogąc
skomentować jej, dla mnie, całkowicie niezrozumiałej uwagi.

Myra się roześmiała. 

-- Idź, pomóż kumplowi. Wydaje się, że ma problem.

Reid odwrócił się do mnie z~ulgą. 

-- Mam wszystkich, prócz Myry. Co to do
cholery jest ,,przecier''?

-- I~jeden sok pomidorowy! -- powiedziałem do barmana.

-- Och, dzięki -- powiedział Reid. Spojrzał na mnie. (Nieświadomie
wyprostował się na pełną wysokość, coś, co ludzie dookoła mnie często
robili, ale ciągle patrzył w~górę.) -- To, o~czym mówiłeś wcześniej o~rynku, to było interesujące. Te rzeczy o~milionach równań.

-- Ta -- odpowiedziałem, zbierając niektóre z~drinków. -- Miliony równań. I~to nawet nie jest połowa. -- Wiedziałem, co się stanie dalej,
odpowiadając na takie okoliczności już kilka razy.

-- Dlaczego nie możemy użyć komputerów?

-- Ponieważ -- powiedziałem nad ramieniem, gdy przedzierałem się do
stolika -- bez rynku, nie będziesz \emph{miał} pierdolonych komputerów!

Myra śmiała się, gdy stawiałem drinki. 

-- Nie martw się o~burżuazyjną
ekonomię Jona -- powiedziała do Dave Reida, gdy usiedliśmy. -- Nawet
Związek Radziecki ma komputery. -- Poczekała na jakiś znak zapewnienia w~jego uczciwie zdziwionej twarzy i~dodała: -- Największe na świecie!

Reid uśmiechnął się, ale uparcie kontynuował: 

-- Spójrz na IBM. Czy
\emph{oni} martwią się siłami rynku? Czy oni kurwa! Mój przyjaciel
pracował w~ich fabryce w~Inverkip jednego lata. Powiedział, że
dostarczają części zapasowe wszędzie na świecie w~ciągu czterdziestu
ośmiu godzin, nawet jeżeli to oznacza pójście ze śrubokrętem do
superkomputera, który już zbudowali, i~wyciągnięcie części.

-- Tak, to brzmi całkiem jak Związek Radziecki -- powiedział, ku ogólnemu
śmiechowi. -- A Ty brzmisz jak mój stary.

-- Czy on jest socjalistą? -- spytał Reid. Spytał nieufnie.

-- Dożywotni członek SPWB\footnote{ Socjalistyczna Partia Wielkiej Brytanii,
zob.~\url{https://en.wikipedia.org/wiki/Socialist_Party_of_Great_Britain} -- przyp.tłum.} -- powiedziałem.

-- SPWB? Och, cudownie! -- powiedział Reid.

-- Co to ta SPWB? -- spytała Myra. Reid i~ja zaczęliśmy coś mówić, potem
Reid się uśmiechnął, wzruszył ramionami i~odstąpił.

Wziąłem duży łyk, ale to nie było piwo, które wąchałem, ale jakiś
dziwnie zapamiętany powiew koszonej trawy, psiego gówna i~wanilii:
Speaker's Corner. 

-- Socjalistyczna Partia Wielkiej Brytanii -- wyjaśniłem, wpadając automatycznie w~takt agitatora-samouka -- powstała w~1904 roku, z~mniej niż setką członków, w~celu zdobycia większości klasy
robotników na świecie. Obecnie mają ośmiuset, więc są na dobrej drodze.
W tym tempie, najlepsze prognozy dają im szansę na zdecydowaną większość
w dwudziestym piątym wieku.

-- Chyba żartujesz -- powiedział Myra.

-- Żartuje -- powiedział surowo Reid. -- To jest, cóż, niezła karykatura,
zgodzę się z~Tobą. Jednak czytałem niektóre z~ich rzeczy i~nigdy nie
widziałem takich obliczeń.

-- Dobra -- przyznałem. -- Tę część zmyśliłem. No faktycznie, mój ojciec to
zmyślił. Jest prawdziwym wierzącym, ale ma poczucie humoru i~raz napisał
mały program oparty o~wzrost populacji i~wzrost partii, i~puścił go na
komputerze w~pracy.

-- Też jest programistą, czyż nie?

-- Och tak. Dla Londyńskiej Rady Elektryczności\footnote{publiczny podmiot
dostawca elektryczności dla Londynu,
zob. \url{https://en.wikipedia.org/wiki/London_Electricity_Board} -- przyp.tłum.}. Kiedy zaczynał, debugowanie polegało na usuwaniu owadów
z zaworów, i~tego \emph{nie} zmyślam!

Reid, Myra i~inni dookoła stołu się roześmiali. Nigdy wcześniej tak nie
rozprawiałem i~odnosiłem wrażenie, że zrobiłem swego rodzaju dobre
wrażenie na klice.

-- Chodzi mi o~to -- dodałem, póki wszyscy słuchali -- że słyszałem wiele
argumentów o~tym, że komputery zamienią planowanie ekonomiczne w~łatwiznę, ale mnie nie przekonały.

-- Pomijasz kilka punktów -- wtrąciła się Myra, kontynuowała, wymieniając
je, jej moralna pasja była lustrzanym odbiciem mojej. Więc skupiłem się
na innej pasji.

-- Tak czy inaczej, nie chcę społeczeństwa zaplanowanego -- powiedziałem.
-- Nie pasuje do \emph{moich} planów.

To wywołało słaby śmiech.

-- Więc kim jesteś? -- spytał Reid. -- Prawicowcem?

Westchnąłem. 

-- Właściwie jestem indywidualistycznym anarchistą.

-- ,,Wasciwie to żem indywidualistyczyn anarkista'' -- naśladowała Myra. -- Bardziej jak anachronizm. To tragedia -- dodała z~gestem dla galerii. -- Dzieciak uczy się jakiegoś marksizmu na kolanach tatki, a~kończy jako
przeklęty proudhonista.

-- No tak -- powiedziałem. -- Choć to wasz ziomek Tucker\footnote{Benjamin Tucker,
zob.~\url{https://pl.wikipedia.org/wiki/Benjamin_Tucker} -- przyp.tłum.}, według mnie,
wszystko to złożył. 

-- Więc kim jest Tucker? -- zapytał ktoś.

-- Dobra\ldots -- zacząłem.

~

Nie wykonaliśmy żadnej pracy tego popołudnia, ale -- patrząc wstecz na to
z punktu widzenia ekonomii, kalkulacji -- było to tego warte. Większość z~nas skończyła, pijąc piwo i~kawę z~powrotem w~pokoju w~piwnicy
Instytutu. Reid i~ja siedzieliśmy po przeciwnym stronie Myry, w~rogu
dużego stołu. Czasem ona mówiła do nas obu, czasem do innych osób, i~znowu do jednego z~nas lub drugiego. Kiedy mówiła do Reida, wyglądało to
jak podsłuchiwanie plotek w~rozległej rodzinnej kłótni, i~wyłączałem się
lub zwracałem się do innego wątku dyskusji. Jednak zawsze mnie znowu
włączała, jakąś uwagę o~Wietnamie, Portugalii czy Angoli: prawdziwe
wojny i~rewolucje, nad którymi fakcje toczyły swoje międzykontynentalne
walki.

Po jakimś czasie uświadomiłem sobie, że w~pokoju zostaliśmy tylko we
troje. Pamiętam twarz Myry, jej łokcie na stole, jej szczupłe dłonie
poruszające się, gdy mówiła o~Nowym Jorku. Myślałem, że brzmiało to jak
miejsce, gdzie chciałbym pojechać, kiedy krzesło Reida zaskrzypiało na
podłodze i~wstał.

-- Będę się zbierał -- powiedział. Uśmiechnął się do Myra przez chwilę,
potem spojrzał na mnie i~powiedział: -- Do zobaczenia zatem, Jon.

-- Ta, wyglądał, że bawimy się w~tych samych miejscach -- powiedziałem z~uśmiechem. -- Jeżeli nie wpadnę na Ciebie jutro czy pojutrze,
prawdopodobnie zobaczymy się w~QM w~piątek.

-- Nie zniknij nam, Dave -- powiedziała Myra. -- Upewnij się, że
przyjdziesz na następne seminarium, co? Potrzebujemy gościa takiego jak
w \emph{Critique}. Wiesz jako nie tylko naukowca?

Reid lekko się zarumienił, potem się roześmiał i~powiedział: 

-- Aye, tak
właśnie sobie myślałem! -- Zarzucił worek na ramię i~ruchem wycierającym
rozłożonej dłoni pomachał na pożegnanie.

Usłyszeliśmy jego pustynne buty uderzające w~stopnie schodów, stuknięcie
zamykanego zamka we frontowych drzwiach. Pierwszy raz dotarło do mnie,
że on i~ja spędziliśmy popołudnie, konkurując o~względy Myry, lub ona
spędziła, nas testując. (Tak właśnie się to zaczęło: od Myry. A nie, jak
myślałem znacznie później, od Annette. Ponieważ, gdyby Myra była z~Reidem od początku, a~ja z~Annette...)

Myra oparła brodę na dłoniach, poruszyła okularami i~spojrzała na mnie
przez nie.

-- Cóż -- powiedziała. -- Interesujący facet, co?

-- Tak -- powiedziałem. -- Bardzo poważny.

-- Teraz nie jestem w~nastroju na poważne.

Spojrzała na mnie bez ruchu przez chwilę, uśmiechnęła się i~powiedziała:

-- Chcesz wypalić trochę trawy?

Myślałem, że to był jakiś niezrozumiały amerykanizm na seks, i~tylko
zrozumiałem swój błąd, kiedy zaczęła budować złożonego skręta w~jej
kawalerce. Ale jak się okazało, w~końcu nie byłem w~takim błędzie.

~

Myra i~ja nie mieliśmy romansu, bardziej serię jednonocnych przygód.
Dziesięć dni, które wstrząsnęły światem. Żadne z~nas nie udawało, ale
lubię myśleć, że oboje mieliśmy nadzieję, że może z~tego wyjdzie coś
więcej. Jednak publicznie, wobec siebie, byliśmy bardzo wyrafinowani,
bardzo w~porządku, bardzo wyzwoleni o~tym.

Potem zakochała się w~bohaterze chilijskiego ruchu oporu z~czarnym
wąsem, i~byłem zdumiony jak wściekły, zazdrosny, zaborczy się poczułem.
Była taka chwila, około trzeciej nad ranem po wieczorze, kiedy Myra
powiedziała mi, wiecie, było bardzo miło, i~naprawdę mnie lubi, ale
całkiem niespodziewanie odkryła tak potężne uczucia dla tego
latynoamerykańskiego leninisty, tak inne niż cokolwiek doświadczyła
kiedykolwiek, że, cóż, na początek widzi się z~nim za, jakby, pięć
minut\ldots była taka chwila w~piciu czarnej kawy z~brudnego kubka i~patrzeniu z~niewiarygodną odrazą na popielniczkę, z~której wysypywały
się smoliste skręty papieru, podczas gdy moje palce rolowały kolejny,
tylko żeby poczuć oparzenie na języku, kiedy wszystkie moje cykle dobowe
złożyły się razem w~napływie krwi, utracie ciepła ludzkiego, kiedy
czułem, że nigdy nie będę chciał iść do łóżka, które nie zawiera
obietnicy miednicy Myry uderzającej w~moją.

I cały czas inna część mojego umysłu pracowała, analizowała, jak
absurdalne to było, że ta zazdrość winna być niespodzianką, i~na
kolejnym poziomie świadomości gratulowałem samemu sobie za bycie
wystarczająco stoickim i~rozumiejącym to zrozumienie, wiedząc, że to
była prosta emocja naczelnych, która była zrodzona, i~minęłaby.

Znalazłem długopis i~napisałem w~notatniku: \emph{Ludzie plejstocenu z~lustrzanymi oczami}, żeby nie zapomnieć tego głuchego wglądu rano i~zapadłem w~sen. Ciągle zbolały, ale nagle pewny, że nabyłem miarę
zazdrości i~niespodziewanej, nieodwzajemnionej miłości.

~

W tym samym czasie, gdy Myra i~ja ostrożnie, a~w~jej przypadku
pomyślnie, nie zakochiwaliśmy się w~sobie, zakochałem się w~Reidzie.
Istnieje taka miłość, że (nie dzięki Bogu) śmie teraz krzyczeć swoje
imię, i~jest inna miłość, które nie wie, jaka \emph{jest }jej cholerna
nazwa, i~to było to. Nasze umysły złączyły się jak magnesy, z~uderzeniem.

Reid był krępy i~ciemny, o~celtyckiej postawie z~dobrymi proporcjami. Ja
byłem wysoki i~żylasty, z~włosami, które nawet wtedy nosiłem krótko,
żeby ukryć ich przerzedzenie, i~nosem, który zawsze zapewniał mi rolę
Czerwonego Indianina, kiedy byłem dzieckiem. On był nieporadny, ja
uprzejmy. Ale nieporadność Reida była czymś, co zrzucał, i~wyrastał ponad
z rodzajem łaski, tymczasem ja traktowałem każdą okazję społeczną jako
ciągły test rozumu. Rodzice Reid byli religijni, Free Kirk\footnote{ Free Kirk
-- Free Church of Scotland -- kościół protestancki, ewangelikalny i prezbiteriański zob.~\url{https://en.wikipedia.org/wiki/Free_Church_of_Scotland_(since_1900)} -- przyp.tłum.}, i~zrobili co mogli, żeby
zaszczepić te same zasady w~nim. Moi byli zagorzałymi materialistami
marksistowskimi, ale przyjęli nastawienie laissez-faire\footnote{ fr.
laissez-faire -- pozwólcie czynić -- pogląd filozoficzno-ekonomiczny
głoszący wolność jednostki w~wymiarze społeczno-ekonomicznym,
więcej~\url{https://pl.wikipedia.org/wiki/Leseferyzm } -- przyp.tłum.} wobec mojej edukacji filozoficznej. Czasami, przez wszystkie te
opowieści o~pytaniach, na które reagowali  obcinaniem włosów lub potokami łez,
czułem, że twarda postawa jego rodziców wykazywała głębszą troskę o~jego
dobrobyt.

Reid był komunistą, ja libertarianinem. Jednak on utrzymywał kłującą
niepodległość umysłu, zawziętą tendencję do martwienia się trudnościami
w doktrynie, które jego sekta łączyła. Czasem podejrzewałem, że byłem
zbyt prostym sceptykiem, zbyt katolicko pewnym, że mój chwiejny stos
książek Proudhona, Tuckera, Herberta, Spencera, Roberta Heinleina i~Roberta Antona Wilsona budował niezawodną wieżę startową umysłu.

Inną rzeczą, którą lubiłem w~Reidzie, było to, że upijałem się z~nim
szybciej niż ktokolwiek inny, stąd piątkowe wieczory.

~

Reid i~ja rozmawialiśmy więcej o~,,komputerach przejmujących władzę''
(czyli jak ludzie rozmawiali wtedy o~Osobliwości\footnote{oryg. singularity,
zob.~\url{https://pl.wikipedia.org/wiki/Technologiczna_osobliwo\%C5\%9B\%C4\%87}
 -- przyp.tłum.}), potem przeszliśmy do aktualnego artykułu \emph{New
Scientist} o~teorii katastrofy, o~którym Reid był sceptyczny (,,to jak
burżuazyjna wersja dialektyki'', jak to ujął). Po nauce, polityka:
gorącym tematem była Portugalia, gdzie skrajna lewica się przechytrzyła
w czymś, co wyglądało jak bezczelna próba wojskowego zamachu stanu.

-- Tutaj jest dobry artykuł o~tym -- powiedział Reid, wygrzebując z~kurtki
kopię \emph{Red Weekly}, gazety International Marxist Group. -- Odpuszczam sobie, co \emph{Socialist Worker }ma do powiedzenia. Cóż,
jeszcze sam nie przeczytałem, ale wygląda dobrze.

-- Ok, ok -- powiedziałem. -- Kapuję. Sekciarska polemika jest tym, w~czym
chłopaki jesteście dobrzy.

-- Dostaniemy Cię w~końcu. -- Reid uśmiechnął się, gdy kupowałem gazetę.

-- Lub ja was -- powiedziałem.

Reid wzruszył ramionami. 

-- To nie tak działa -- powiedział. Zaczął
zwijać papierosa, mówiąc zmęczonym głosem. -- Ludzie nie przestają być
socjalistami i~nie stają się kimś innym. Tylko stają się \emph{niczym}
lub dołączają do Partii Pracy, bez różnicy.

-- \emph{Ja} przestałem być socjalistą -- podkreśliłem.

-- Ta, ale to coś innego, daj spokój. To jakbym ja mówił, że przestałem
być chrześcijaninem. To było coś, w~czym zostałem wychowany, i~jak tylko
zacząłem myśleć samodzielnie, porzuciłem to. Podobnie z~Tobą, prawda?

-- Może -- odpowiedziałem. -- Zauważ, to nigdy nie było wpychane w~moje
gardło co niedzielę. -- Ale niespokojnie pamiętałem, jak mało to zabrało
-- pewne anarchistyczne podsumowanie Tuckera, chyba -- żeby wzbudzić każdą
wątpliwość, jaką kiedykolwiek miałem o~mojej odziedziczonej wierze.

-- Mam nadzieję, że zawsze będą rozumiał rzeczy w~ten sposób jak teraz -- kontynuował Reid -- ponieważ to ma sens, jest przed wszystkim innymi
ofertami. Jednak jeżeli kiedykolwiek zapomnę lub wiesz, stracę
miejsce\ldots

-- Lub zrozumiesz, że byłeś cały czas w~błędzie.

-\ldots dobra, tak właśnie będzie się wydawać, to właśnie będę sobie
mówił\ldots

Uśmiechnął się kwaśno, jego język na zewnątrz, żeby polizać papier,
pokazując na chwilę wygląd diaboliczny, gargulcowy. 

-- Ale jeżeli to
kiedykolwiek się zdarzy -- dokończył, zwijając papierosa i~przypalając go
-- będę przeklęty, jeżeli stanę się idealistycznym bojownikiem dla
drugiej strony. W~ten czy inny sposób, tylko będę uważał na siebie.

-- Ale to jest to w~co ja teraz wierzę! -- powiedziałem wesoło. -- Troska
to numer jeden. Nie jestem idealistycznym bojownikiem za nic.

-- To jest to, co myślisz -- powiedział Reid. -- Jesteś anarchistą dla
czystego, niewinnego własnego interesu? Och, pewnie. Zrozum to,
człowieku, \emph{zależy} Ci. Jesteś socjalistą w~sercu.

Lubiłem go dostatecznie i~powiedział to dostatecznie lekko, że nie
poczułem się urażony.

-- Nie, to nie tak jak to wszystko jest -- powiedziałem. -- Naprawdę mam
egoistyczny powód dla pragnienia świata bez państwa, chcę żyć wiecznie.
Naprawdę. Chcę dostać się na statki. Planeta zajęta przez zorganizowane
gangi wariatów z~bronią jądrową to nie jest moja idea bezpiecznego
środowiska.

Większość ludzi śmiała się ze mnie, kiedy to mówiłem, ale nie Reid.
Jedną z~rzeczy, które mieliśmy wspólne, było zainteresowanie fantastyką
naukową i~możliwościami technologicznymi, które dobrze pasowały do
reszty moich przekonań. W~teorii pasowały też do marksizmu, ale
wiedziałem, że towarzysze Reida traktowali to jako ideologicznie
niezdrowe, jakby jedyną dozwoloną, dalekosiężną spekulacją futurystyczną
był ostatni dokument o~perspektywach IMG. Jego stosy \emph{Galaxy} i~\emph{Analog} były schowane w~szafce w~kawalerce niczym pornografia.

-- Wydaje się, to wysokie oczekiwania -- powiedział Reid. -- Wybraliśmy złe
stulecie na życie. Sądzę, że po prostu musimy zaryzykować jak reszta
biednych facetów.

Wyciągnąłem papierosa na odległość ramienia i~popatrzyłem na niego. 

-- I~nic nie robimy z~naszymi szansami.

-- Rozumiem, że to raczej wyścig z~medycyną -- powiedział Reid. -- Moje to
Export, przy okazji.

Zauważyłem nasze puste szklanki i~skoczyłem, pełen skruchy, że nie
zauważyłem wcześniej. Kiedy wróciłem, Reid był zatopiony w~gazecie,
którą mi sprzedał, i~nie byłem pewien, czy w~tym momencie chcę dalej
ciągnąć naszą rozmowę, więc oparłem się i~pozwoliłem umysłowi trochę
dryfować. Miejsce się zapełniało. Szafa grająca grała ,,Sailing'' Roda
Stewarta, piosenkę, która zawsze wzbudzała we mnie ckliwy patriotyzm
wygnańca za krajem, który nigdy nie istniał, tak jakbym we wcześniejszym
życiu był obywatelem Atlantydy. Kiedy się skończyła, wypadłem z~nastroju
i znowu się rozejrzałem, i~zauważyłem, że gazeta Reida ma kolejnego
czytelnika, który siedział koło niego i~pochylał się do przodu, jej
głowa przechylona do czytania tylnej strony. Jej czarne kręcone włosy
opadały na boki dookoła jej twarzy. Czarne brwi, rzęsy, wielkie zielone
oczy poruszające się (wolno, zauważyłem), gdy czytała, mały zgrabny nos,
szeroki kości policzkowe, z~których jej policzki, ani cienkie, ani
pulchne, wyginały się gładko po obu stronach (nieświadomie delikatnie
się ruszały) pełnych ust, do małej stanowczej brody.

Jej wzrok oderwał się od strony i~spotkał mój z~niezażenowanym
uśmiechem. Poczułem wstrząs tak fizyczny, że nawet nie powiązałem go z~emocją. I~wtedy Reid opuścił papier i~spojrzał na nią. Usiadła prosto i~teraz ona wyglądała lekko zażenowana. Była z~rojem innym dziewczyn,
które rządziły stolikiem obok, i~reszta rozmawiała pomiędzy sobą.

-- Cóż, dzień dobry -- powiedział Reid. -- Uważasz, że to interesujące?

-- Nigdy nie widziałam czegoś takiego jak to -- powiedziała. -- Nie
rozumiem, jak ktokolwiek chciałby popierać strajki. -- Miała akcent
zachodniego wybrzeża, ale, tak jak Reid, mówiła akcentowanym angielskim,
nie szkockim jak rodzimi mieszkańcy Glasgow. Prawdopodobnie gdzieś z~Clyde, Irlandia lub Highlands: ,,angielski jako drugi język'' pokolenie lub
dwa temu.

-- To gazeta socjalistyczna -- powiedział Reid. Spojrzał na mnie jakby po
pomoc. -- Popieramy robotników, wiesz?

-- Ale \emph{rząd} jest socjalistyczny -- powiedziała oburzonym tonem. -- A
oni nie chcą strajków, prawda?

-- Nie uważam, że rząd Partii Pracy jest w~ogóle socjalistyczny -- wyjaśnił Reid.

-- Ale czy to nie jest złe dla kraju, kiedy ludzie mogą iść na strajk i~od razu po zasiłek?

-- W~pewnym sensie, tak -- powiedział Reid, który normalnie w~takiej
sytuacji wyczerpałby cierpliwość. -- Ale jeżeli przez ,,kraj'' rozumiesz
większość ludzi w~nim żyjących, prawda, to problemy, które mamy, nie
pochodzą od robotników na strajku, ale od szefów i~bankierów robiących
interesy jak zwykle. To oni są tymi, którzy naprawdę \emph{kosztują
kraj}.

-- Masz zabawny sposób patrzenia na rzeczy -- powiedział, jako
wyjaśnienie, nie jako pytanie. Odrzuciła temat i~przeniosła uwagę na
ważniejsze sprawy. -- Idziecie później na dyskotekę?

-- Tak -- powiedziałem, zanim Reid mógł podjąć kolejną próbę edukacji
politycznej. -- A Ty?

-- Och, ta -- powiedziała. -- Może się tam zobaczymy. -- Błysnęła do nas
krótkim uśmiechem, zanim została wciągnięta w~rozmowę z~jej
przyjaciółmi. Patrzyłem przez chwilę tam, gdzie jej włosy opadały na
ramiona jej prostej białej koszuli. Koszula była wciśnięta w~proste
niebieskie dżinsy, a~jej stopy w~buty na wysokim obcasie. Jej ubrania i,
teraz doszedłem do wniosku, jej makijaż wyglądał zbyt staranny i~zwyczajny jak na studentkę. Tak samo dla jej przyjaciółek, niektóre z~nich były ubrane podobnie inne w~eleganckich sukienkach.

-- Dobra -- powiedziałem, gdy Reid złapał moje oko -- jako temat rozmowy,
to myślę, że ten potrzebuje odrobiny pracy.

-- Można tak powiedzieć -- przyznał. -- Jednak, nie dała mi zbyt dużo
możliwości.

-- Przede wszystkim nie powinieneś trzymać nosa w~tej przeklętej gazecie
-- powiedziałem mu.

Tuż po dziesiątej, obaj ruszyliśmy szybko, gdy dziewczyny wyszły,
straciliśmy je w~kolejce, ale udało nam się dostać stolik niedaleko ich.

~

-- Chcesz zatańczyć? -- krzyknąłem. Ultrafiolet odbijał się w~nylonowych
szwach jej koszuli, a~stroboskop w~widzialnym spektrum uchwycił jej
kiwnięcie. Ten taniec był szybki, następny wolny. Trzymaliśmy ręce lekko
wzajemnie na ramionach pod koniec. Spojrzałem na nią z~góry. 

-- Dziękuję -- powiedziałem.

Była taka rzecz, którą zrobiła z~oczami: zielona tęczówka błyszcząca,
źrenice otwierające się jak ciemne baseny, w~których mógłbyś utonąć.

Wszystko, co mogłem powiedzieć, to: 

-- Jak się nazywasz?

-- Annette.

-- Jon Wilde -- powiedziałem. -- Chcesz drinka? -- Utonąłem, ale moje usta
ciągle się poruszały.

-- Kufel lagera, proszę. -- Uśmiechnęła się i~wróciła do stołu. Kiedy
wróciłem, Reid krzyczał i~machał coś do niej ponad muzyką i~światłami.
Słuchała, głowa przechylona, broda na dłoni. Muzyka znowu się zmieniła,
a Reid wstał i~wyciągnął dłoń do Annette. Pokiwała głową, wzięła łyk
piwa ze zdawkowym uśmiechem podziękowania, i~już poszli tańczyć.

-- Wydowo sie, iże ftoś zaczon ze złyj strony\footnote{fragmenty napisane w szkockim angielskim tłumaczę na język śląski celem zachowania zamysłu autora, tłumaczenie automatyczne przy pomocy
\url{https://silling.org/translator/?dir=pol-szl\#translation } -- przyp.tłum.}. -- Rozbawiony, ale
sympatyczny kobiecy głos powiedział mi do ucha. 
Odwróciłem się, żeby odkryć, że patrzę na dziewczynę z~długą grzywką
czerwonobrązowych włosów, zza których zerkała twarz, jak małego ssaka
zza zarośli. Nosiła bluzkę z~rzemykami przy karku i~mankietami, długą
niebieską spódnicę nad długimi butami.

-- Tak -- powiedział, przytakując. -- On jest strasznym tancerzem.

Roześmiała się. 

-- Godałach o~ciebie -- powiedziała. -- Niy tropiyłabych
sie. Annette je mało kokietką.

-- Może mnie kokietować, kiedy chce -- powiedział. -- W~międzyczasie,
zapoznajmy się, tylko żeby dać jej coś do myślenia.

-- To do jeji coś do myślynio -- powiedziała i~zaskoczyła mnie
pocałunkiem, po której nastąpiło przytulanie, które, po pewnym
przesunięciu krzeseł i~ostrożnym doborze głosu, pozwoliło nam
porozmawiać tylko we dwoje. Od czasu do czasu słyszeliśmy siebie
krzyczących, gdy muzyka przestawała grać, gdy ktoś zmieniał płyty (nie
kompakty, te pojawiły się później).

Nazywała się Sheena. Skrót od Oceania, jak później się dowiedziałem.

-- Skąd znasz Annette?

Sheena skrzywiła się na mój dobór tematu. 

-- Miyszkom z~niom. -- krzyknęła
dyskretnie. -- Tyż z~niom robi. Som techniczkami laboratoryjnymi. Na
Wydziale Zoologii. Co robisz?

Powiedziałem jej, a~wkrótce krzyczałem i~machałem rękoma jak prawdziwy
naukowiec. Niemniej jeżeli intencją było sprowokowanie Annette do
okazania zainteresowania, eksperyment okazał się porażką.

~

Chłodna noc, bez mrozu, szkielety opadłych liści na chodniku jak
skamieliny ryb. Dave, Annette, Sheena i~ja zatrzymaliśmy się na moście,
patrząc ponad balustradą na pokojowy ryk Kelvina\footnote{zapewne Kelvin
Hall, w~latach 1956-1983 największa sala koncertowa Glasgow -- przyp.tłum.}.

-- To musi być jedyna rzecz nazwana po jednostce miary -- powiedział Reid.
Roześmiałem się na to i~dziewczyny też się roześmiały.

-- Powinno być więcej! -- powiedziałem. -- Potok Dżula! Strumień Ampera!

-- Jeziorko Litre!

-- Szczyt Metra!

-- Albo języki komputerowe -- powiedział Reid, gdy szliśmy dalej, budynek
BBC Scotland po lewej, po prawej Ogród Botaniczny z~jego ogromną kolistą
szklarnią, latający spodek z~jakiegoś dziewiętnastowiecznego Marsa. -- Stopnie Fortrana. Bloki Basic...

-- Rezydencje Ady!

-- Stras Cobol!

Kiedy dotarliśmy do mieszkania dziewczyn, wyskrobaliśmy Wzgórza Newtona
i Plażę Candela, i~próbowałem wszystkich przekonać, że wszystkie
jednostki miary były nazwiskami ludzi, na przykład Jean-Baptiste de
Metre, znany encyklopedysta, żyrondysta i~karzeł.

-- Oczywiście po Rewolucji opuszczał ,,de'' -- wyjaśniałem, gdy Annette
dzwoniła kluczami. -- Ale to go nie uratowało, został...

-- Skrócony -- powiedział Reid.

-- O stopę.

-- Nie, głupku, o~głowę.

-- Zamierzocie stoć tam cołko noc?

-- Tylko przez sekundę.

-- Nazwaną oczywiście po\ldots -- szukałem inspiracji.

Reid popchnął mnie. 

-- No dawaj.

Wszedłem. Mieszkanie w~suterenie, duży frontowy pokój, łóżko, wersalka,
fałszywy kominek. Wszędobylskie plakaty, wypchane zabawki, dziewczęcy
bałagan. Mała kuchenka, gdzie Annette włączała czajnik elektryczny.

Rozmawialiśmy, piliśmy kawę, po której poczuliśmy się dziksi, Sheena
skręciła skręta. Później\ldots później byłem w~kuchni, na wpół siedząc na
brzegu zlewu, podczas gdy Sheena zajęła się kolejną rundą Nescafe i~pozostałości zioła. Drzwi były prawie zamknięte, głosy Dave'a i~Annette
stałym mamrotaniem.

Odłożyła mleko do lodówki, oparła się o~moje biodro. Pochyliłem się,
odsunąłem jej włosy i~spojrzałem na nią.

-- Czy chcesz, żebym został?

-- Tak, cóż, nie. -- Podała mi karton, łyknąłem, skrzywiłem się i~włożyłem pod kran. -- Znaczy chciałabych, ale widza iże lubisz Annette.

-- Szkoda, że ona nie widzi. Szkoda, że jej nie powiedziałem.

-- Och, ona wie. Myśla, iże ona sie boi. Je żeś taki intynsywny.

-- Intensywny? Moi? Masz na myśli, nie tak jak Dave ,,Sprawiedliwa Walka''
Reid? Lubi jego prosty urok laborystycznej teorii wartości, o~to chodzi?

Sheena się uśmiechnęła. 

-- Niy mylisz sie zazbyt. Wejzdrzij, jeźli na
tela staro sie o~to, co ona obmyślonku, coby sie z~niom wadzić, to niy
może ino być interesantny przeleceniem jeji.

Czajnik zaśpiewał. Patrzyłem na fluorescencyjny pasek nad blatem i~ścisnąłem oczy. Waga Sheeny znikła i~zajęła się kubkami. Westchnąłem na
nagły aromat.

-- Więc co takiego robię, że myślisz, że jestem zbyt dosadny? Ledwo
miałem szansę powiedzieć słowo do niej przez cały cholerny wieczór.

-- Święta racja -- powiedziała Sheena. -- Osprowiosz ze mnom i~godosz
rzeczy do Dave'a, i~colki czas patrzisz na Annette i~usmiychosz sie, kej
ona co powiy.

-- Nieprawda!

Spojrzała mi w~oczy.

-- Dobrze -- przyznałem. -- Może tak robię. Przepraszam. Musiało to
wyglądać nieco niegrzecznie.

-- Trochę -- powiedziała. -- Jednak niy winie Cie. Zaczęłam ten mały szpil.
Chodź, pomóż mi z~tymi piokami.

Kiedy skończyłem kawę, wstałem. Dave i~Annette siedzieli na podłodze,
opierając się o~bok łóźka. Ramię Dave'a spoczywało na ramionach Annette.

-- Na razie, ludziska.

-- Na ra -- powiedział Dave.

-- Dobranoc -- powiedziała Annette. Próbowałem wyczytać w~jej zwężonych
oczach, błysk migotania lub mrugnięcie. Spojrzała w~dół.

Sheena pocałowała mnie na dobranoc w~drzwiach, z~ciepłem tak nagłym i~niespodziewanym jak jej pocałunek na powitanie.

-- Pewna? -- Spróbowałem ułożyć usta w~łobuzerski uśmiech.

-- Pewna. -- Nacisnęła ramiona, przytrzymując. -- Je żeś fajnym chopym, ale
niy komplikujmy naszych żyć barzij aniżeli już som.

-- Dobrze, Sheena. Dobranoc. Do zobaczenia.

-- Znikaj! -- Uśmiechnęła się i~zamknęła drzwi.

Płytki do poziomu piersi, pobielona, wypolerowana balustrada. Zacna
kamienica klasy pracującej Glasgow, nie jak studencki slums, w~którym
mieszkałem. Przypomniałem sobie coś. Odwróciłem się do drzwi i~przykucnąłem przed nimi, pchnąłem napiętą klapkę skrzynki pocztowej.

-- Dave! -- krzyknąłem.

-- Co? -- nadeszło słabe i~odległe.

-- Po \emph{Karolu Sekundusie}! -- krzyknąłem. -- Patronie Towarzystwa
Królewskiego!

~

Chmura nadeszła nad miasto, gdy byłem w~mieszkaniu. Na skrzyżowaniu
Great Western Road i~Byres Road czekałem na przejście. Za mną zastukały
obcasy, zatrzymały się koło mnie. Dziewczyna w~futrze. Odwróciła się,
uśmiechnięta, i~spytała: 

-- Jak światła...? Och, rozumiem. -- Głos jak
ciepła dłoń, akcent wyższej klasy angielskiej. Futro i~włosy błyszczały
kroplami wilgoci. Szła gdzieś, gdzie chciała być, pewna, że nikt nie
odważyłby się położyć palca na niej: piękne zwierzę, doskonale
zaadaptowane, dzikie.

-- Okropna mgła, prawda?

-- Tak -- powiedziałem. -- Nigdy nie widziałem takiej w~Glasgow.

Światła się zmieniły. Przeszliśmy, nasze ścieżki się rozeszły. Ona
poszła w~Byres Road, do miejsca, gdzie chciała być, a~poszedłem wzdłuż
Great Western Road, z~powrotem do pokoju.

\chapter{Dzieciak Terminal}

Na Nowym Marsie padało. To cud stworzony przez maszyny, praca rzadkich
urządzeń daleko stąd i~bezduszna, botaniczna moc niezliczonego
potomstwa, które obraca metalowe płatki, żeby skupić delikatne
promieniowanie słoneczne na kawałkach brudnego lodu, podgrzewając lotne
czynniki na powierzchni, żeby posłać lód koziołkując w~kierunku słońca,
popychany i~kierowany na precyzyjnie wyliczoną trajektorię, która lata
później wprowadzi je w~atmosferę na tyle grubą, żeby złapać i~sprowadzić
na dół. Gdzie szczęśliwie opadną jako deszcz, a~nie ogień, a~który przy
każdym przypadku przejścia bolidu, pozostawia atmosferę nieco lepiej
przystosowaną do złapania i~przechwycenia kolejnego.

Jednak dla Dee, na zewnątrz w~deszczową noc, to banał i~nuda. Przez
około pół godziny miała wzmacniacze obrazu na pełnej mocy i~jej oczy
bolą. Jej uszy też: pingi sonarowe od mokrych ścian około metra od
siebie wywołują ograniczające poczucie ciśnienia. W~tym samym czasie,
zmniejszanie lub wyłączenie jeszcze bardziej by ją zmęczyło. Zatem z~ulgą i~relaksem widzi, że wąska droga wodna otwiera się na znacznie
większy i~jaśniejszy kanał.

-- Kanał Pierścienia -- wskazuje Tamara, gdy skręca mały pojazd w~prawo.
Dee, wyciągając głowę i~patrząc do przodu i~tyłu, nie widzi krzywizny.
Wysokie, wąskie domy -- raczej bloki magazynów i~fabryki -- górują nad
kanałem, a~światła są pociągnięte ponad brzegiem. Przed nimi, szybko
zbliżające sto metrów dalej, Kanał Pierścienia otwiera się, a~przez
przerwę pomiędzy budynkami na końcu Dee widzi, coś, co wygląda i~brzmi
jak ognisko: buchające światło, ryk szumu.

U zbiegu, Kanał Pierścienia rozdziela się na lewo i~prawo, wykrzywiając
się w~widzialny pierścień, którego średnicę Dee ocenia na trzysta
metrów. Więcej wysokich domów gromadzi się dookoła niego, a~w~obrębie
jest płaska wyspa dostępna za pomocą mostów z~otaczającej kolistej
drogi. Ta centralna wyspa jest pokryta chatami z~blachy falistej,
namiotami i~szopami, pośród których wielu ludzi jest głośno zajętych.
Światło dociera z~napowietrznej powodzi oraz z~wkładu każdej szopy w~postaci lamp, rurek fluorescencyjnych, stroboskopów, kabli optycznych,
kabli świetlnych.

Tamara bierze kolejny skręt w~prawo i~zmniejsza obroty silnika, płynąc
wzdłuż zewnętrznego brzegu, cichego pośród muzyki i~handlu,
rywalizujących ze sobą.

-- Co się tutaj dzieje? -- pyta Dee.

Tamara oszczędza jej spojrzenia. 

-- Wieczór Piątego Dnia na Placu
Okrągłym.

Małe molo pod wąskim mostem z~kutego żelaza, z~dołączonymi schodkami.
Tamara cumuje łódź i~kieruje Dee na schody. Czeka na moście od strony
brzegu i~pomaga Tamarze wyciągnąć torbę. Przychodzący i~mijający ludzie,
pary, grupy, dzieci unikające i~omijające nogi i~koła, młodzi na lub w~pojazdach zbudowanych, żeby jeździć szybko i~poruszać się wolno, i~rzeczy, które mogłyby być pojazdami, z~tym że nie mają kierowców, prawie
ją to odrzuca.

-- Racja -- mówi Tamara -- czas uczynić Cię legalną.

Rusza wzdłuż mostu, Dee blisko za nią, jedna osoba w~tłumie, która nie
ma kłopotu z~przejściem.

Większość straganów dookoła obwodu wyspy jest zamknięta, ale ciągle
podświetlona. Te, które były otwarte, sprzedają napoje i~przekąski.
Główna akcja dzieje się w~kierunku centrum, w~pojedynku atrakcji
wesołego miasteczka, dyskotek i~koncertów rockowych. Dee zauważa scenę z~zespołem, który wygląda i~brzmi całkiem jak Metal Petal, hit tygodnia w~każdej knajpie na przedmieściach. Szybkie zbliżenie i~analiza dźwiękowa
pokazuje, że to \emph{jest} Metal Petal. (Dee słyszała o~prawach
autorskich, ale to jedna z~tych rzeczy, w~które nie do końca wierzy,
pieśń odległej Ziemi).

Tamara zatrzymuje się przed rzeczą wielką jak automat z~jedzeniem
pomiędzy dwoma straganami. Maszyna jest pokryta kurzem i~rdzą. Ma czarny
ekran na górze, siatkę głośnika i~kanał na dole po jednej stronie, przez
którą Tamara przesuwa kartę. Nic się nie dzieje.

-- Hej! -- krzyczy. Uderza w~bok pięścią, robiąc huk. -- \emph{Jebane} IBM
-- mówi do nikogo w~szczególności.

Światła pojawiają się w~ciemnym okienku.

-- Usługi Prawnicze Niewidzialnej Ręki -- mówi maszyna głosem jak Bóg w~starym filmie. -- Jak mogę ci pomóc?

-- Zarejestruj roszczenie autonomii dla porzuconej maszyny -- mówi Tamara,
łapiąc nadgarstek Dee i~przyciskając dłoń do okna.

-- Obie ręce proszę -- mówi maszyna. -- Oczy.

Dee rozkłada palce na szkle i~zagląda, widząc swoje własne odbicie i~jasne poruszające się iskry światła.

-- Jak chciałabyś bronić roszczenia?

-- Ja będę je bronić! -- mówi Dee z~nagłym przypływem pasji Samej jaźni.

-- Poprzez zasadę -- dodaje uroczyście Tamara. -- Oraz mnie, moich
towarzyszy i~wsparcie, jeżeli trzeba.

-- Bardzo dobrze. Odnotowane i~opublikowane.

Światła zgasły. Tamara ciągle trzyma nadgarstek Dee i~obraca ją dookoła
i łapie drugi\ldots potem puszcza i~klaszcze w~zamian w~dłonie. Dee patrzy
w oczy Tamary i~widzi własne odbicie i~przyśpieszające, obracające się
światła za nią, podwójne targi.

-- Dobra, dziewczyno! -- krzyczy Tamara. -- To Ty z~paczką po Twojej stronie!
To jest taka wolność, na ile jest możliwa! Plus minus\ldots potem o~tym! A
teraz\ldots -- Okręca się, żeby spojrzeć na dudniące centrum rynku na wyspie
-- \ldots \emph{bawmy się}!

~

-- Mówisz mi -- powiedział Wilde z~niedowierzaniem do robota -- że
\emph{Reid} jest \emph{tutaj}?

-- Tak -- odpowiedział Robot. -- Dlaczego miałoby to cię zaskoczyć? Czy to
jest bardziej wybitne niż Twoje bycie tutaj?

Wilde uśmiechnął się kwaśno. Odepchnął pusty talerz i~łyknął piwa.
Pokręcił głową.

-- Reid był jedną z~ostatnich osób, które zobaczyłem -- powiedział. -- Z~tego, co wiem, to on mógł być osobą, która mnie zabiła. I~o~ile jestem
zainteresowany, to zdarzyło się \emph{dzisiaj}. Chryste, ciągle czekam,
żeby się obudzić.

-- Obudziłeś się -- powiedział Robot. -- Możesz oczekiwać pewnych
emocjonalnych reakcji, gdy Twój umysł dostosowuje się do sytuacji.

-- Tak przypuszczam. -- Posępność maskująca jego pozorny wiek zapadła na
obliczu Wilde'a. -- Już tak się stało. Więc powiedz mi maszyno. Jestem
tutaj i~mówisz, że Reid jest tutaj. Co z~innymi, których znałem? Co z~Annette?

-- Annette -- powiedziała Maszyna ostrożnie -- jest pośród Nieożywionych.
Czy jej umysł jak również genotyp zostały zachowane, nie wiem, ale
istnieją podstawy dla nadziei.

-- Z~powodu klonu?

-- Tak.

-- Muszę ją znaleźć i~się dowiedzieć.

-- Możesz się dowiedzieć, bez znajdowania jej -- powiedziała Maszyna. -- To
jest\ldots wyjaśnię jutro.

-- Dlaczego nie teraz?

-- Kłopoty -- powiedziała Maszyna. -- Nie odwracaj, póki czegoś nie
usłyszysz.

Wilde odstawił szklankę. Jego ramiona zaczęły się garbić.

-- Spokojnie -- powiedziała Maszyna.

Drzwi do pubu uderzyły przy otwarciu i~muzyka się zatrzymała. Rozmowy
ciągle się toczyły przez kilka sekund, a~potem się urwały w~rozszerzającej się ciszy. Wszyscy się odwrócili.

Dwóch mężczyzn stało w~drzwiach. Nosili luźne, mocno pogniecione
garnitury, z~rozpiętymi koszulami na podkoszulkach. Ich włosy były tak
błyszczące jak ich buty, a~pięści błyskały nabijanymi kamieniami. Jeden
z mężczyzn zdawkowo wystawił kartę pokazującą jego zdjęcie i~szary blok
małego druku. Drugi wyjął z~kieszeni marynarki pogniecioną kulkę
płaskiego materiału. Złapał róg i~potrząsnął. Z~ostatnim strzepnięciem
nadgarstka, ustawił to w~błyszczący, kolorowy plakat o~wysokiej
rozdzielczości przedstawiający ciemnowłosą kobietę, która uciekła
przed Wilde'm i~Robotem.

-- Ktokolwiek ją widział? -- zażądał.

Klienci pubu ciągle mogli zostać w~przybliżeniu podzieleni na dwie
grupy, mężczyzn przy barze i~dziewczyn przy stołach, choć już zaczęło
się pewne mieszanie. Lekka lawina chichotów i~westchnięć pojawiła się
wśród kobiet, a~szmer chrząknięć i~lekko przesuniętych stołków i~szklanek wśród mężczyzn. Każdy, kto chciałby coś powiedzieć, mógłby
spojrzeć na mężczyzn przy barze, i~znaleźć coś innego do patrzenia, coś
innego do powiedzenia.

W ciągu pół minuty, wszyscy znowu gadali. Mężczyźni przy barze wrócili
do oglądania telewizji, gdzie komentator prowadził wywiad z~kapitanem
zespołu, za którym ciała były wynoszone na noszach z~areny. Jedyną osobą
ciągle patrzącą bezpośrednio na ludzi z~odzyskiwania był Wilde. Ten,
który trzymał obraz, ruszył do przodu. Drugi poszedł za nim, bawiąc się
rękojeścią rewolweru z~miną odległej przyjemności.

Mężczyzna z~obrazem spojrzał w~dół na Wilde'a i~się uśmiechnął,
pokazując idealne, ale dziwnie ukształtowane siekacze, długie kły. Opary
perfum wylewały się z~niego jak pot.

-- Cóż -- powiedział -- wyglądasz na zainteresowanego. Duża nagroda, wiesz.

Wilde spojrzał niechętnie na obraz. Pokręcił głową.

-- Przypomina mi kogoś, kogo kiedyś znałem -- powiedział. -- To wszystko.
Ale nigdy jej tutaj nie widziałem.

Mężczyzna wpatrzył się w~niego. 

-- Była tutaj -- powiedział. -- Mogę to
wywąchać. -- Odwrócił głowę w~jedną i~drugą stronę, delikatnie wąchając,
jakby jego oświadczenie było dosłownie prawdziwe. Drugi mężczyzna nagle
radośnie krzyknął i~złapał coś z~podłogi.

Machnął tym pod nosem Wilde'a. Wilde lekko odskoczył. Robot, pochylając
się pomiędzy krzesłem a~stołem, szarpnął do przodu kilka centymetrów.

Rzeczą, którą trzymał mężczyzna, była gazeta.

-- Wiedziałem! -- powiedział. -- Cholerne abolki! Dobra, to wszystko.
Wiemy, gdzie jej szukać!

Wciskając gazetę i~plakat w~kieszenie, obaj mężczyźni wyszli w~kolejnej
ciszy. Drzwi znowu uderzyły. Wróciła muzyka. Człowiekowaty za barem
spojrzał na Wilde'a z~miną głębokiej skruchy, potem wzruszył szerokimi
ramionami i~rozłożył obie szerokie ręce, jego długie ramiona komicznie
długie. Zakończone wzruszenie, odwrócił się i~włączył muzykę, głośniej.

Wilde wrócił do swojego posiłku i~wypił szklankę łykiem, który
sprowadził łzy do oczu.

-- Ciągle chce z~nią porozmawiać -- powiedział.

-- Jeżeli jesteś zmartwiony gynoidem\footnote{gynoid czyli android w~formie kobiety, od gyne -- gr. kobieta  -- przyp.tłum.} -- powiedziała
Maszyna -- to się nie martw. Jeżeli jest z~abolicjonistami, to będzie
legalnie i~fizycznie zabezpieczona przed odzyskaniem, przynajmniej przez
jakiś czas. A jeżeli nie jest\ldots -- Maszyna poruszyła górnymi stawami
przednich kończyn w~parodii wzruszenia ramionami. -- Nie zamierzają jej
skrzywdzić. Tylko naprawić błąd w~oprogramowaniu. To nie jest ważne.

-- Ponieważ jest tylko maszyną, tak?

-- Tak.

-- Cóż, może to niedelikatne zwrócenie uwagi, ale Ty też jesteś.

-- Oczywiście -- powiedziała Maszyna. -- Ale \emph{ja} jestem
równoważnikiem człowieka, a~\emph{ona} to seks-zabawka. Tak jak
powiedziałem: po prostu jebana maszyna.

~

Systemy inwigilacji? Nie rozśmieszaj mnie. Każde nagranie zrobione
dookoła centrum Plac Okrągłego jest nieodwracalnie uszkodzone,
zhakowane, połatane, pocięte i~zremiksowane. Nawet wspomnienia Dee są
zrozumiale roztrzepane: Żołnierka i~Szpiegini po prostu się wyłączają z~niesmakiem, zostawiając jedynie proste odruchy w~pracy. Ludzie podają
narkotyki z~ręki do ręki, maszyny przekazują wtyczki. Muzyka ma
amplitudy i~elektroniczny dopływ, który działa z~tym samym efektem. Dee
widzi Tamarę rozmawiającą z~wysokim walczącym mężczyzną z~przemysłowym
ramieniem, zaczyna rozmawiać z~pająkowatym gadżetem z~aerografem i~prostym umysłem. To myśli i~potrafi rozmawiać, ale tylko o~muralach. To
zna powierzchnie betonowe, właściwości farby i~fizykę aerozoli. To
opowiada jej o~tym, dość pokaźnie.

Mogłaby tego słuchać całą noc. Jest dobrą słuchaczką. Niemniej artysta
widzi budowniczego, i~bez wymówki lub pożegnania, odślizguje się przez
tłum, żeby pogadać.

Tamara łapie łokieć Dee i~patrzy za maszyną. Potem odwraca się i~Dee
może, jak mówią, zobaczyć obracające kółka, gdy centra mowy
przezwyciężają zatrucie.

W końcu słowa się wydostają.

-- \emph{Nie} równoważnik człowieka!

-- Rozmawiałam z~gorszymi mężczyznami -- mówi Dee.

~

Dee bez myślenia podryguje, to jest własna umiejętność Samej Jaźni,
kiedy zauważa podrygującego również mężczyznę, który porusza się tak,
jakby zakładał, że tańczy z~nią. Jej wzrok przesuwa się od błyszczących,
butów ze sztucznej skóry, przez spodnie i~marynarkę jego luksusowego,
ale niestylowego garnituru, przez miazmat obrzydliwego zapachu unoszącego
się z~zapoconego kołnierzyka podkoszulki w~kołnierzu rozpiętej koszuli
do jego...

\emph{twarzy}!

\ldots i~szok rozpoznania jednego z~łapaczy, mężczyzn do odzyskania, wysyła
wstrząs adrenalinowy, który budzi Żołnierkę. Wszystko zwalnia, prócz
niej. (Muzyka przechodzi od disco do głębokiego industrialnego dubu.)
Szybkie spojrzenie dookoła uruchamia pracę Chirurżki nad chrząstkami i~ścięgnami jej karku i~przywraca rozum, że Tamara wije się faliście kilka
metrów dalej, jej plecy na wpół odwrócona, i~za Tamarą, bokiem do Dee,
jest kolejny łapacz. Jego ruchy i~postawa wyglądają, jak by pieprzył
wirtualny obraz Tamary około metra przed prawdziwą, ale to tylko taniec
na dyskotece. Jego wzrok ani na chwilę nie opuszcza prawdziwej Tamary.

Widzi spadający pot z~włosów, gdy jego głowa się porusza. Wygląda na
całkowicie zajętego przez przynajmniej kolejne kilka sekund.

Drugi łapacz, ten, który patrzy się na nią, zdecydowanie zauważył
przesunięcie mentalne Dee (ten nagły rozmazany ruch głowy ją zdradził) i~jego źrenice zmniejszają się do wielkości szpilki, tak jak powieki
otwierają się szerzej. Dee jest świadoma pistoletu jako ciężkiego
kształtu w~miękkiej skórzanej głupiej babskiej torbie przy nogach,
świadoma, że jej wąska spódnica jest \emph{oporem}, który utrudni
taktycznie oczywiste śmiertelne uderzenie nogą.

Mogłaby krzyczeć, ale krzyk jest niczym w~tym hałasie. Jedyny dźwięk
słyszalny ponad tym musiałby być niesłyszalny, dla ludzkich uszu. Jej
usta się otwierają, jej pierś się nadyma z~prędkością naciskając na
żebra i~wydaje ponaddźwiękowy krzyk, który, ma nadzieję, jest słyszalny
dla maszyn na setki metrów dookoła: 

-- \emph{Pieprzone IBM, pomocy}!

Muzyka się zatrzymuje. Włączają się światła. Ludzie mrugają i~potykają.
W tym samym momencie prawa dłoń Dee sięga w~dół, jej prawa stopa kopie w~górę za nią, ciągle w~ruchu, który mógłby być częścią kroku tanecznego,
i jej but na wysokim obcasie ląduje w~jej dłoni. Trzyma go wysoko jak
młotek, gotowa przybić łapacza przez gałkę oczną. Zrozumienie tego
posuwa się zmarszczkami przez mięśnie i~naczynia krwionośne jego twarzy,
gdy nagle przemawiają głośniki. Głos IBM, dla przyśpieszonych zmysłów
Żołnierki Dee, teraz brzmi głębiej i~groźniej niż cokolwiek w~okolicy:

-- Zagrożenie klienta Niewidzialnej Ręki. Proszę o~pomoc.

Łapacz się cofa, ten za Tamarą także. Wszyscy inni wyglądają chwilowo na
wytrąconych z~równowagi, prócz Tamary, która patrzy na Dee ze
wzrastającym zachwytem. Zamaszyste spojrzenie Dee po tłumie, zanim
Żołnierka wycofuje się na pozycje obserwacyjne, wskazuje, że są tam inne
twarze, usiane w~tłumie, odpowiadające na wezwanie najlepiej jak
potrafią: spinając się, wstając, czołgając się lub, w~przypadku jednej
lub dwóch maszyn, wysuwając się teleskopowo. Ci ludzie zaczynają
intonować, powoli z~klaskaniem: 

-- Precz! Precz! Precz!

I Dee pcha mężczyznę, i~Tamara pcha, i~obaj łapacze są popychani,
poniewierani od jednej osoby lub robota do następnej, aż są wyrzuceni na
krawędź tłumu w~czekający uchwyt kilku ciężkich motocyklistów, którzy
eskortują ich dalej.

-- Ok -- mówi Dee. Uśmiecha się dookoła i~zakłada but, macha i~woła: 

-- Dzięki wszystkim! -- dziewczęco wdzięcznym głosem, który posyła Żołnierkę
w skręty zażenowania i~sprowadza lekki rumieniec na policzki.

Muzyka i~światła wznawiają rytm.

Dee tańczy, ale wie, że następnym razem nie będzie już tak łatwo. Ci
goście mogą już nie wrócić, ale ktoś przyjdzie.

~

Dee jest w~małym pokoju na szczycie budynku Placu Okrągłego, z~widokiem
na Kanał Pierścienia. Tamara sprowadziła ją do mieszkania w~tym wysokim
domu, po czymś, co wyglądało jak godziny na imprezie na wolnym
powietrzu, i~udała się spać do własnego pokoju wyjaśniając
przepraszająco, że jutro wcześnie rano zaczyna pracę. 

-- Ax Ci ze
wszystkim pomoże -- powiedziała jej.

Dee jest przyzwyczajona do niejasnej ludzkiej mowy. Nie pyta się o~wyjaśnienia. Jej własne ludzkie ciało i~nerwy są zmęczone. Nie
potrzebuje spać, ale potrzebuje spoczynku, i~marzeń sennych. Jedna pod
drugiej jej Jaźnie wyłączają się, wchodzą w~tryb offline, pakują,
asymilują i~integrują wydarzenia dnia.

Pokój jest uwodzicielsko komfortowy, przy deszczu bębniącym o~dach tuż
za opadającym sufitem, oknami mansardowymi zapewniającymi więcej kątów
dla wzroku, toaletka z~pozamykanymi butelkami i~pojemniczkami, korale,
szale i~wstążki zwisające z~lustra, wycięte zdjęcia mody przyczepione do
ściany, tuzin lalek na półce. W~rogu stoi wygięte, wyłożone satyną,
wiklinowe krzesło, kredens przy ścianie (zamknięty) i~łóżko z~bałaganem
kołdry i~poduszkami obszytymi koronką. Jest coś delikatnie niepokojącego
w ludzkim zapachu na tle kwiatowych i~piżmowych woni, ale nie martwi się
analizą tego.

Zdejmuje ubrania, układa lub wiesza je, poprawia temperaturę ciała dla
komfortu i~kładzie się do łóżka. Jej powieki zamykają widok z~okna na
znajomą rzeczywistość Miasta Statku: wilgotne, ociekające miasto wież z~krzemianu, miasto pożyłkowane kanałami, zatłoczone osieroconymi
gwiezdnymi podróżnikami i~wolnymi lub zniewolonymi automatami,
nawiedzanych przez Szybkich i~Nieożywionych. Jej umysł rozwija Fabułę,
która tworzy kolejny odcinek jej bezustannej głównej roli w~samonapędzającej się operze mydlanej przesiąkniętej całym romantycznym
splendorem starożytnej Ziemi, gdzie...

~

\ldots jest najstarszą córką Senatora, przeznaczoną do odziedziczenia jego
miejsca w~Dumie oraz wszystkich przywilejów z~demokratycznego
namaszczenia, ale zostaje porwana przez agentów Korporacji Górniczej
Archipelago i~przetrzymywana przez młodego, mrocznego i~diabelskiego
dyrektora, który chce jej w~swoim haremie i~jest gotów wymienić jej
życie za jej rękę w~konkubinacie i~poważne koncesje na Antarktydzie, a~osobiści i~fanatycznie lojalni czeczeńscy strażnicy jej ojca walczą
przez kręgi brutalnych obrońców dyrektora, podczas gdy ona stoi, ubrana
w jedwabie i~owiana perfumami na balkonie wieżowca lorda narkotyków
Kuomintangu w~sercu starego Nowego Jorku obserwując czołgi walczące na
ulicach poniżej i~czekając na przyciśniętych Czeczenów, by sprowadzili
posiłki zdesperowanych plemion południowego Bronksu obietnicą rabunku, i~słyszy ciche kroki za nią i~dyrektor, którego twarz, prawdę
powiedziawszy, wygląda niesamowicie podobnie do jej właściciela, pada
przed nią na kolana i~mówi jej, że on naprawdę, szczerze, kocha tylko ją
i ma wyrzuty sumienia i~wypuści ją, jeżeli tylko...

I tak dalej.

To o~tym śnią androidy, lub raczej gynoidy.

~

Stukanie w~drzwi. Natychmiast wraca do pełnej świadomości, jej
wewnętrzny zegar mówi jej, że jest wczesne rano.

-- Chwilę -- mówi.

Mały czyściciel-szkodnik usunął każdą plamkę organicznego brudu z~jej
ubrań. Wytrząsa je bez myślenia i~ubiera się w~rozmytym ruchu (użyteczna
umiejętność Żołnierki, którą ,,wycięła i~wkleiła'' w~Samą) i~woła:

-- Wejść.

Chłopiec, który wchodzi, niosący tacę z~kubkiem kawy i~miską płatków,
wygląda na około dwanaście lat, na pierwszy rzut oka. Jest czarny,
drobnej budowy, delikatnymi rysami i~szokiem czarnych włosów. Gdy Dee
skanuje go od góry do dołu, wszystko w~czasie uśmiechania i~mówienia
,,cześć'', zdaje sobie sprawę, że chłopiec jest znacznie starszy, niż na
to wygląda. Nie ma sposobu, żeby tak wiele doświadczenia mogłoby się
odcisnąć w~napięciu mięśni twarzy, spojrzeniu w~oczach, w~tylko
dwanaście lat. Nie tutaj, nie w~Mieście Statku. Mają prawa przeciwko
rzeczom tego rodzaju.

-- Ty musisz być Ax -- mówi, zabierając tacę. -- Dzięki. -- Gestem wskazuje
mu krzesło. -- Tamara wspomniała o~Tobie.

-- Wzajemnie -- mówi chłopiec, siadając z~jedną stopą na drugim kolanie. -- Więc Ty jesteś Dee Model, co? Główna sympatia wielkiego szefa Reida.

Dee patrzy na niego, jej kolana wyszukanie razem, taca balansuje na
nich, łyżka prawie w~ustach. Odkłada ją, robiąc mały brzęk o~bok miski.
Uspokaja tacę oraz swój głos.

-- Skąd o~tym wiesz?

Ax pokazuje białe zęby. 

-- Jesteś znana. -- Jego szczerzenie staje się
dziwne, potem ustępuje uspokajającemu uśmiechowi. -- Niezupełnie. Twój
pan miał cię ze sobą na imprezie w~zeszłym roku, zdjęcie pojawiło się na
czatach plotkarskich. -- Jego oczy rozogniskowują się na chwilę. -- Niezła
sukienka -- mówi.

-- Nie myślałam tak -- mówi Dee. Zaczyna jeść. -- Musiałam być w~Seksie
przez większość czasu, żeby jej noszenie było znośne.

Ax parska.

-- Tak czy inaczej. -- Dee się rumieni. Programy Szpiegini utrzymują jej
głos równy i~cichy. -- Czy szukają mnie? Ogłoszenia nagród?

Kolejne spojrzenie poza, on ma łącze korowe, zdaje sobie sprawę Dee,
rzadka własność tutaj. Większość intymnych interfejsów z~siecią, które
większość ludzi toleruje to kontakty, małe okrągłe ekrany, które wsuwasz
nad oko.

-- Jak na razie nic -- mówi Ax, uwaga przyciągnięta z~powrotem. -- Sądzę,
że jest zażenowany. Mam na myśli, Twoja chodząca lalka wychodzi od
Ciebie, to nie tak jakby Twój samochód został skradziony, wiesz, co mam
na myśli?

-- Tak -- mówi Dee. Myśl o~prawdopodobnym szale i~poniżeniu jej
właściciela sprawia, że jej kolana, pomimo wszystkiego, drżą. Odkłada
tacę i~sięga po torebkę.

-- Papieros?

-- Cokolwiek -- mówi Ax. Ma zapalniczkę na łańcuszku dookoła szyi i~porusza się szybko, żeby zapalić jej, potem siada z~powrotem zajęty
swoim.

-- Więc dlaczego wyszłaś? -- pyta. Jego ton jest ani przyjazny ani
lubieżny. To jak profesjonalne pytanie, ton lekarza lub inżyniera przy
pacjencie.

-- Nie znęcał się nade mną -- odpowiada. -- Nie przeszkadzała mi służba lub
seks. Przeszkadzało bycie niewolnikiem.

-- Zdaje się, że powinnaś to lubić -- mówi Ax. -- To jest wpisane na
twardo.

-- Wiem -- mówi Dee. Rozgląda się dookoła za popielniczką, wzdycha, w~umyśle nadpisuje procedury Służącej i~strząsa popiół do pustej, brudnej
miski. -- I~ja to lubię. Uważam to za satysfakcjonujące. Jednak tylko
seksualnie. Nie w~każdy inny sposób, nie w~moich oddzielnych jaźniach. A
kiedy to zrozumiałam, że, to, co robiłam, było\ldots wkleiłam moje programy
Seksu w~ten obszar, zamaskowałam to wszystko z~Samej, i~uczyniłam się
wolną.

-- Niesamowite -- mówi Ax, jakby to było coś innego. -- Zatem to prawda, co
mówią: \emph{informacja chce być wolna}!

Dee potrząsa głową. 

-- To nic takiego wspaniałego -- wyjaśnia. -- Zdarzyło
się to gdy załadowałam znacznie więcej narzędzi, niż kiedykolwiek
powinnam mieć. -- Próbuje przypomnieć sobie te drugie narodziny, kiedy
przeskakiwała pomiędzy tymi różnymi Jaźniami i~zobaczyła siebie, widmowe
odbicie we wszystkich oknach.

Ax marszczy brwi. Prztyka palcem i~niedopałek papierosa gaśnie w~resztkach mleka w~misce. Zaciekawiony czyścik-pełzak spłaszcza się,
chowając przednie segmenty. 

-- Kiedy to się zdarzyło? -- pyta.

Dee uśmiecha się dumnie, pragnąc podzielić się zaufaniem. 

-- Wczoraj -- odpowiada.

~

Usta Ax są otwarte przez chwilę. Przez sekundę mina ,,już to widziałem''
znika z~twarzy. Wygrzebuje paczkę papierosów z~rękawa podkoszulki i~zapala jednego bez myślenia, nie patrząc, nie oferując. 

-- Ale dlaczego w~ogóle -- kontynuuje -- załadowałaś to całe dodatkowe oprogramowanie? Co
Cię \emph{do tego} zmusiło?

Dee czuje się zagubiona. Trudno jest wracać do jej wcześniejszej
prostoty, kiedy przełączała się z~jednego umysłu do kolejnego i~to była
ona, to było jej życie. Nie była mniej świadoma wtedy niż jest teraz,
ale to była niepodzielona, naiwna, posłuszna świadomość, bez
perspektywy. Niemniej nawet wtedy, gdzieś w~Samej, istniało pragnienie
wiedzy. I~możliwość nadeszła, a~ona z~niej skorzystała, z~czymś, patrząc
teraz, co byłoby słodkim zapewnieniem, że jej właściciel byłby
zadowolony.

-- Instynkt -- mówi z~lekkim śmiechem. Ax prycha i~przewraca oczami.

-- Dobrze -- mówi Dee nagle ukąszona. -- Może to przyszło z~biologicznego
ciała lub fragmentów biologicznego mózgu!

-- Zostawimy ten argument drugiej stronie -- mówi Ax.

-- Drugiej stronie czego?

-- Drugiej stronie \emph{sprawy} -- wyjaśnia z~wymuszoną cierpliwością. -- W~ten czy inny sposób, to się skończy w~sądzie. Wiesz o~prawie?

-- Och tak -- odpowiada Dee jasno. -- Mam umysł tutaj nazywany Sekretarką.
Ma precedensy, których mam po dziurki w~nosie.

-- Dobra -- mówi Ax stanowczo, wstając -- sugeruję, żebyś je przejrzała. To
może wydawać się bardzo inne, tyle mogę ci powiedzieć.

-- Ok -- mówi Dee. Ax trzyma drzwi otwarte, czekając. Dee wstaje.

-- Co teraz?

Patrzy na nią od góry do dołu. 

-- Myślę, że zakupy. -- Jego głos zawiera
zniewieściałą pogardę.

Podnosi torebkę, wciska pistolet za spódnicę i~się rozgląda. Nic nie
zostawiła.

-- Ładny pokój.

-- Mój -- mówi Ax. -- Byłbym szczęśliwy, mogąc dzielić go z~Tobą.

~

Zewnętrzne drzwi budynku uderzają za nimi. 

-- Zostań -- rozkazuje temu Ax.

Magnetyczne zawiasy blokują je, brzęcząc. Ax uśmiecha się do niej i~rusza w~lewą stronę. Dee rozgląda się, gdy kroczy koło niego. Dom, z~którego wyszli, ma cztery piętra i~jest wąski. Ta jak wszystkie tutaj
dookoła, w~klasycznym zatłoczonym stylu brzegu kanału, ale nie ma
wyblakłych murów ceglanych czy kontrastujących spoin, żadnych parapetów
lub okien. Wszystko jest betonowe, skóra narzucona w~pośpiechu na siatkę
drucianą siatkę nad żelaznymi koścmi, grafitti jedyną, i~właściwą,
dekoracją. Wieże iglicowe miasta wynurzają się jak żurawie konstrukcyjne
ponad budynkami, redukując je do chatek na miejscu.

Dym unosi się pomiędzy straganami, para z~chodników. Mgła unosi się nad
powierzchnią kanału. Malunki sprayem na murach stają się coraz bardziej
namiętne, sięgając punktu kulminacyjnego z~zaciśniętych pięści, chmur w~kształcie grzyba i~dinozaurów przy wejściu do alei.

Ax zatrzymuje się i~macha do wewnątrz. 

-- Tędy.

Aleja nie ma trzech metrów szerokości, ale to ulica handlowa na swoich
własnych prawach, i~w odróżnieniu od tego, co Dee widziała na razie w~sąsiedztwie, posiada wypracowany urok, nazwy sklepów wymalowane w~pracowitym naśladownictwie kaligrafii dwudziestopierwszowiecznych
oznaczeń galerii handlowych. Przy pierwszej wystawie w~oknie Ax czeka
zniecierpliwiony, gdy Dee ogląda dioramę skamielin, rzekomo faunę
starego dna morskiego planety. Naukowczyni ma inną wizję i~łacińskie
nazwy, których Dee nie zna, unoszą się rozpraszająco przed jej wzrokiem.
Wewnątrz sklepu, skamieliny są zamieniane w~amulety i~ornamenty.
Dziewczyna przy szlifierce unosi maskę na twarzy, rzuca Dee zapraszający
uśmiech i~wraca -- zdziwiona lub zaskoczona reakcję Naukowczyni Dee -- do
pracy. Lotne zapachy lakieru, politury, kleju i~smarów unoszą się przez
drzwi razem z~piskiem karborundu na kamieniu.

Jest i~sklep sprzedający fajki i~narkotyki, kiosk z~gazetami, gdzie Dee
widzi kopie \emph{Abolicjonisty} i~mniej jasne tytuły jak \emph{Farmy,
nano targ, jądrówki}, stragan pełen wyblakłych śmieci opisanych ,,Stary
nowomarsjańskie artefakty obcych''. W~tym całym krytycznym guzdraniu się
Dee, Ax tylko mamrocze i~pali. Dee cieszy się z~tej odmowy, choć jest
trywialna, przyjęcia ludzkich priorytetów. Ćwiczenie wolnej woli.

Jednak podziela widoczną rozkosz Axa, kiedy docierają do pierwszego
butiku, jaskini ubrań i~akcesoriów. Wprowadza ją do środka i~przebywają
tam przez godzinę, która mija jak minuta i~potem idą do następnych
sklepów z~ubraniami, do małych studii artystów kosmetyków i~laboratoriów
jubilerów. Przez cały czas Ax wydziwia dookoła niej z~nieświadomą
intymnością, która nie zmienia się w~zależności od jej stanu ubrania lub
rozebrania. Potrafi określić, że przyjemność, jaką z~niej czerpie, jest
estetyczna, nie erotyczna. Software Seksu jest czuły na takie
rozróżnienia: potrafi odczytać fizjologię rumieńca, zmierzyć rytm pulsu
i rozszerzenie źrenicy, i~wie, że nie ma pragnienia w~dotyku tego
chłopca.

Na odległym końcu alejki jest kawiarnia. Siedzą tam pod nagłym światłem
południowego słońca nad wąskimi ulicami, sączą kawę, palą, otoczeni
przez ich zakupy. Dee odrzuca swój trzeźwy styl na rzecz czegoś
wariackiego i~punkowego. Pyszni się w~skórze, sznurowania i~koronkach,
satynie i~jedwabiu, kolcach i~ćwiekach. Wygląd, który nie robi wrażenia
na większości dwunastoletnich chłopców, stymuluje większość mężczyzn. Ax
patrzy na nią jak na dzieło sztuki, które ukończył, a~którym w~tym
momencie jest.

Dee bawi się zapalniczką, patrzy spod grzywki jej na nowo ustylizowanych
włosów. Chce coś powiedzieć, ale nie wie jak zapytać.

-- Oszczędzę Cię -- mówi Ax. -- Znaczy, jeżeli zażenowanie jest w~Twoim
repertuarze. Seksualnie mówiąc, nie jestem w~grze. \emph{O} grze,
czasem, może. -- Pstryka palcami. -- Nie gej, nie kastrat. Tylko chłopiec,
trwale niedojrzały.

-- Dlaczego? -- pyta Dee. -- Czy to choroba?

-- Śmiertelna -- szczerzy się Ax. -- Coś tam, gdzie geny napotykają
maszynki: błąd. Wirus. Coś, co rodzice złapali w~trakcie długiej
podróży. Szczęśliwie nie startuje, póki nie przejdę dojrzewania. Więc
zablokowałem mój biologiczny wiek nieco wcześniej niż inni.

-- I~nie ma na to leku?

Kąciki ust Axa opadają. 

-- Jeżeli jest, jest u szybkich umysłów.
Najlepszą radą było zapomnieć o~tym, innymi słowy. Ale nie mógłbym
zapomnieć. Powód zaangażowania się w~abolicjonizm. -- Śmieje się. -- Moje
szanse na bycie mężczyzną są razem z~powracającymi Nieożywionymi i~znowu
działającymi szybkimi umysłami. Pffft.

-- Hmmm. -- Dee czuje smutek. Co za szkoda. Wpada na sprytną myśl. -- Mógłbyś dojrzeć jako kobieta -- mówi.

-- Dobra, dzięki -- odpowiada Ax, przez chwilę dąsając się i~pozując. -- Rozważyłbym to, ale naprawiacze mówią mi, że wirus reaguje z~hormonami
niezależnie od płci. Więc utknąłem z~niczym i~po przewidywalnym okresie
wściekłości i~dąsania, zdecydowałem się, że równie dobrze mogę rozwijać
się jako ktoś, komu zazdrosny mężczyzna mógłby zaufać ze swoją kobietą.
-- Wciąga dym i~wydycha elegancko. -- Niezależny, profesjonalny eunuch i~pedał na pół etatu.

Podczas gdy Dee ciągle o~tym myśli i~zastanawia się, czy los Ax nie
jest, przy wszystkich sprawach, gorszy niż jej, dodaje:

-- Zanim odkryłem swoją kondycję, byłem całkiem normalnym chłopczykiem. -- Wzdycha. -- Zniewieściałość jest tylko pozą, Dee, tylko pozą. A w~przypadku, gdyby ktokolwiek zapomniał, mogę też być skrajnie brutalny.

-- Dlaczego się w~tym nie wyspecjalizowałeś? Strażnik lub wojownik lub...

-- A ryzyko śmierci? -- Ax śmieje się rubasznie. -- Cza ja \emph{wyglądam}
na głupiego?

-- Nie. -- Dee uśmiecha się do niego przyjaźnie, siostrzenie (teraz gdy
zrozumiała, że to jedyna relacja), ale przestaje go żałować. Rozumie, że
sobie dobrze radzi. Dziwny jak łyska, myśli, a~gdy wstają do wyjścia,
pcha Naukowczynię zrzędliwie do przeszukiwania starych odziedziczonych
baz, żeby odkryć co to kurwa \emph{jest }łyska.

~

-- Zatem dożyłem statków -- powiedział Wilde. Unosi się na łokciu i~się
rozgląda po pokoju, w~którym leżał przebudzony przez dziesięć minut.

-- Dzień dobry -- powiedziała Maszyna. Leżała na podłodze w~rogu pokoju.
Pokój był na górze Mili Malleya, tani w~wynajmie, zawierał prysznic,
krzesło i~łóżko. Był nadzwyczajnie odkurzony, dzięki maszynom wielkości
i kształtu stonogi, które czmychały po podłodze.

Wilde patrzył na Maszynę. 

-- Co robiłeś przez całą noc?

-- Strzegłem cię -- odpowiedziała Maszyna. Chwilowo wyprostowała swoje
kończyny, potem je złożyła. -- Przeszukiwałem sieci miasta. Śniłem.

Wilde pozostawał oparty na jednym łokciu, patrząc na Maszynę z~nagłą
brawurową ciekawością. 

-- Nie wiedziałem, że maszyny śnią.

-- Również wspominam -- powiedziała Maszyna. -- Kiedy jest czas.

Wilde kwaśno się uśmiecha. 

-- Zdaje się, że czasu masz dużo, myśląc tak
bardzo szybciej...

-- Nie -- warknęła Maszyna. -- Mówiłam Ci. Jestem maszyną równoważną
człowiekowi. Mój subiektywny czas jest prawie taki sam jak Twój. Bez
wątpienia moje połączenia są szybsze niż Twoje reakcje, ale świadomość,
która podtrzymuje ruchy, działa w~tym samym tempie.

-- Naprawdę? -- Wilde wstał z~łóżka, spojrzał na ciało z~przebłyskiem
ponownego zaskoczenia, uśmiechnął się, umył twarz i~szyję i~się ubrał.

-- Więc powiedz mi Maszyno -- powiedział, gdy wkładał buty -- jak
powinienem cię nazywać? Przy okazji, czym jesteś?

-- W~zasadzie -- powiedziała Maszyna, odłączając przewód z~gniazdka w~ścianie i~powoli zwijając go do obudowy -- jestem platformą konstrukcyjną
inżynierii lądowej, samodzielną, napędzaną jądrowo, odporną na piasek. A
co do mojego imienia. -- Zatrzymała się. -- Możesz mnie nazywać, jak
chcesz, ale byłem znany jako Jay-Dub.

Wilde się roześmiał. 

-- To wspaniale! To wystarczy.

-- ,,Jay-Dub'' wystarczy -- powiedziała Maszyna. -- Nie bez godności.
Dziękuję, Jonie Wilde.

-- Cóż, Jay-Dub -- powiedział Wilde z~nieśmiałym uśmiechem -- chodźmy na
śniadanie.

-- Tak zrób -- powiedziała Jay-Dub. Rozwinęła swoje ramiona i~wstała,
rozwijając śmieci z~podartych foliowych pancerzy z~teraz nieruchomymi
nóżkami i~zamglonymi soczewkami. -- Już jadłem.

~

Mila Malleya była cicha, bar zamknięty, wyczyszczony, wypolerowany i~z zawieszoną szmatką do suszenia, kiedy zeszli ze schodów i~wyszli przez
samoblokujące się drzwi.

-- Ufnie -- zauważył Wilde, gdy pozwalał drzwiom się zamknąć.

-- To uczciwe miejsce -- powiedziała Jay-Dub. -- Mało jest po drodze
drobnej kradzieży. Z~powodów, które na pewno znasz.

Małe słońce było nisko nad wieżami, kładąc koronkowe cienie na ulicy.
Łódki i~barki płynęły w~dół kanału, kierując się na zewnątrz miasta.

-- Gdzie one jadą? -- spytał Wilde. Mężczyzna i~Robot szli w~kierunku
małego doku około stu metrów w~górę ulicy. W~doku były budki z~jedzeniem.

-- Kopalnie lub farmy -- powiedział Robot. -- Nie są tutaj całkowicie
różne. W~obu przypadkach są kwestią użycia nanotechu, naturalnego lub
sztucznego, żeby skoncentrować rozproszone molekuły w~użyteczną formę.

-- I~ludzie pracują przy tym? Co robią roboty?

-- He, he, he. -- Kontrola głosu Jay-Dub była zaawansowana, mogła
parodiować mechaniczny śmiech. -- Roboty albo są bezużyteczne dla takich
celów, albo zbyt użyteczne, żeby je na to marnować.

Mały dok był zajęty. Ludzie, w~większości typu człowiek, ale z~kilkoma
innymi typami człowiekowatych pomiędzy nimi, ładowali, lub rozładowywali
wory warzyw lub minerałów z~długiej wąskiej barki. Elektrycznie
napędzane ciężarówki wycofywały się na nabrzeże, załadowane. Rodzina,
czegoś, co wyglądało jak gibony z~nadętymi czaszkami, ciągnęła sieć
pełną skaczących, srebrnych ryb po nabrzeżu i~wysypywała je do
zardzewiałych basenów za jednym ze straganów, gdzie tęga kobieta
natychmiast zaczęła je patroszyć i~grilować. Wilde zatrzymał się tam i,
jakoś z~wahaniem oraz wskazywaniem, dostał od niej rybę, liście i~chleb.
Kawa była na sprzedaż w~szklanych kubkach, zastaw do zwrotu.

Wilde zabrał swoje śniadanie na krawędź nabrzeża i~usiadł, nogi wiszące,
i wolno jadł, rozglądając się. Robot skulił się koło niego.

-- Czas, żebyś mi opowiedział różne rzeczy -- powiedział Wilde. -- Powiedziałeś, że mnie stworzyłeś. Co to znaczy?

-- Sklonowałem Cię z~komórki -- powiedziała Maszyna. -- Wyhodowałem w~kadzi. Puściłem program, który włożył Twoje wspomnienia na Twoje
synapsy. -- Mruczała, zdalnie. -- To ostatnie mogło Cię zabić, więc
zachowaj to dla siebie.

-- Dlaczego to zrobiłeś?

-- Potrzebowałem Twojej pomocy -- powiedziała Jay-Dub. -- Do walki z~Davidem Reidem i~aby zmienić ten świat.

Wilde patrzył na Maszynę dłuższy czas, jego twarz tak nieodgadniona jak
czysta powierzchnia Maszyny.

-- Już mi powiedziałeś, czym jesteś -- powiedział. -- Ale \emph{kim}
jesteś? Prawda, tym razem. Cała prawda.

-- Czym \emph{jestem} -- powiedziała Maszyna, tak cicho, że Wilde musiał
się pochylić bliżej, ucho do siatki pomiędzy metalowymi pokrywami -- to
długie i~złożone zagadnienie. Niemniej jednak \emph{byłem} Tobą.


\chapter{Haczyk}


-- Jeżeli jesteś zainteresowany, będziesz tam.

Pociąg szarpnął. Oświetlone lampami sodowymi brązowe budynki Carlisle
zaczęły się przesuwać.

-- Co? -- Wyrwany z~transu wywołanego pociągiem, nie byłem pewien, czy nie
śniłem tej uwagi. Mężczyzna po drugiej stronie tak zwanego stolika
Pullmana nosił płócienną czapkę i~kurtkę z~jakiejś błyszczącej
substancji, która kiedyś mogła być sztruksem. Jego spłowiała koszula w~kratkę wyglądała jak góra pidżamy. Popijał w~cichej determinacji z~butelki Bell całe długie popołudnie z~Euston.

Teraz potarł brązową dłonią po szczęce, pocierając biały zarost na
niezdrowej skórze i~powtórzył wypowiedź. Uśmiechnąłem się desperacko.

-- Rozumiem -- skłamałem. -- Bardzo prawdziwe.

-- Będziesz tam -- powiedział. Sięgnął po butelkę, określił pozostałą
zawartości na podstawie wagi i~odstawił ją na stolik, potem zaczął
zwijać papierosa drugą dłonią. Jego wzrok, ostry ze sporadycznym
wpadaniem w~mglistość, spoczywał na mnie przez cały czas.

-- Gdzie? -- Spojrzałem w~bok, otworzyłem paczkę Silk Cut (mój gest ku
zdrowszemu życiu). Moje odbicie jarzyło się w~wirtualnym widoku na
zewnątrz pociągu. Rozmokła lutowa wieś sączyła się w~dal.

-- Niy mo znaczynio -- powiedział mężczyzna, wydychając dym i~kwaśny
zapach trawionej whisky. -- Gdziekolwiek. Moga pedzieć. Jesteś
zainteresowany. -- Przerwał, przechylił głowę i~spojrzał na mnie chytrze.
-- Ty żeś jednym z~tych miyndzynorodowych socjalistów. Moga pedzieć.

Znowu się uśmiechnąłem i~potrząsnąłem głową. 

-- Przepraszam, ale się
mylisz, jestem\ldots -- Zatrzymałem się, nie umiejąc wytłumaczyć. Spędziłem
tydzień na badaniach w~bibliotece LSE\footnote{London School of Economics  -- przyp.tłum.} i~kłócąc się z~moim ojcem. Moja głowa brzęczała od
marksizmu.

-- Och, w~porządku synu -- powiedział. -- Ah, wiysz, iże mosz przeróżne
podziały. Niy tropia sie niymi. Jesteś intelektualistą, a~ja jestem
tylko emerytowanym robotnikim. Ale chcesz nos.

Po tym otworzył butelkę, wziął łyka i~podał mi, uprzejmie wycierając
rękę o~swoje udo, a~potem o~brzeg butelki, żeby usunąć jakiekolwiek
szkodliwe bakterie.

~

-- A potem co się zdarzyło? -- spytał Reid.

Skręciliśmy, zgarbieni przed mżawką, w~Park Road, koło pseudo-Tudorowego
frontu pubu Blythswood Cottage i~schyliliśmy się w~wejściu do Voltaire i~Rousseau, najlepszego antykwariatu w~Glasgow. Wpadłem na Reida w~porze
lunchu, po braku kontaktu z~nim przez kilka tygodni, częściowo dlatego,
że ciężko pracowałem nad moją dysertacją, a~częściowo dlatego, że Reid
albo był politycznie aktywny, albo był z~Annette. W~pierwszym miesiącu
ich związku, raz lub dwa poszedłem z~nimi obojgiem na drinki, ale
uznałem, że kontynuowanie tego jest zbyt niezręczne.

-- Zasnął. -- Roześmiałem się. -- Zostawiłem butelkę samą i~obudziłem go na
Centralnym. Wydawał się nie pamiętać całego incydentu. Wyglądało, jakby
mnie nie rozpoznawał.

W tym czasie obaj poruszaliśmy się bokiem, głowy przechylone,
systematycznie przeglądając półki, które pokrywały ściany wąskiej
księgarni. Najpierw oczyścilibyśmy dział polityki i~filozofii, potem -- jeżeli mielibyśmy jakąkolwiek wolną gotówkę -- przeszlibyśmy do tylnego
pokoju, sprawdzić tanie powieści SF w~miękkich okładkach. Jeden z~właścicieli księgarni -- wysoki, pulchny, wesoły facet z~rzadkimi włosami
i grubymi szkłami -- spojrzał znad książki przy kasie z~uśmiechem i~skinieniem głowy. Ten, zdecydowałem, musiał być Rousseau, jego chudy i~ponury partner, Voltaire.

-- Prawdopodobnie stary ILP, czy coś -- wymamrotał Reid, skacząc na tom Dietzgena\footnote{niemiecki filozof socjalistyczny.
Niezależnie od Marksa i~Engelsa rozwijał teorie materializmu
dialektycznego,
zob.~\url{https://en.wikipedia.org/wiki/Joseph_Dietzgen}  -- przyp.tłum.} z~Charles H. Kerr \& Co.  Zdmuchnął kurz i~kichnął.

-- Funt pięćdziesiąt! -- powiedział po cichu, żeby Rousseau nie mógł
dosłyszeć jego radości i~odkryć, jaką okazję przepuszczają. Odwrócił się
od swojego znaleziska, głowica odczytu przesuwająca się wzdłuż taśmy
pamięci półek.

-- Wiesz -- kontynuował -- czasem robi mi się niedobrze, gdy pomyślę o~tych
starych bojownikach wysprzedających swoje biblioteki, żeby uzupełnić
swoje emerytury. Lub umierających, a~ich dzieci\ldots Boże wyobraź ich
sobie, palanty w~średnim wieku, średnia klasa, zawsze trochę zawstydzeni
chaotycznymi wspomnieniami \emph{staruszka} \ldots przeszukujący jego
żałosne graty i~odkrywający półkę socjalistycznej klasyków, i~już je
zostawiających, kiedy nagle błysk kilku funciaków rozświetla ich chciwe
oczy.

-- Równie dobrze dla nas -- powiedział, poszerzając palcami miejsce
pomiędzy dwiema książkami, żeby wyciągnąć wystającą broszurę. -- To ci,
którzy kończą na czubku, że\ldots hej, popatrz na to!

Nie dbałem, kto słyszał. To była na pewno unikalna, żyjąca skamielina:
broszura z~okresu wojny Towarzystwa Rosja Dzisiaj\footnote{
zob.~\url{https://en.wikipedia.org/wiki/British-Soviet_Friendship_Society} -- przyp.tłum.} zatytułowana \emph{Sowieccy Milionerzy}. Nie była długo
w obiegu, nie po tym, jak SPWB zajęła je jako dowód nie do odrzucenia,
że socjalistyczna fasada ZSRR ukrywała klasę bogatych właścicieli.

-- Słyszałem o~niej od mojego ojca -- powiedziałem Reidowi. -- Ale nigdy
nie miał kopii. Wyślę mu ją.

-- Mówiłem! -- Reid wyszczerzył się do mnie z~drabinki. -- Jesteś takim
bezinteresownym bękartem! To właśnie zobaczył w~Tobie ten stary koleś!
Jesteś \emph{dziedzicznym} socjalistą!

-- Ideologia jest dziedziczna? -- zadrwiłem. -- W~takim razie kim Ty
jesteś?

-- Chciwym kułakiem\footnote{ pejoratywne określenie stosowane w~ZSRR, a~następnie w~państwach bloku wschodniego wobec bogatego chłopa uznanego
za zdziercę, wroga klasowego i~,,pijawkę na zdrowym ciele społeczności
wiejskiej'', więcej~\url{https://pl.wikipedia.org/wiki/Ku\%C5\%82ak } -- przyp.tłum.}, chyba -- powiedział szczęśliwie. -- Ach, a~co z~tym? -- Otworzył książkę i~przeglądał wyklejkę. -- Stirner, \emph{Jedyny i~jego
własność}, własność Koła Robotników Anarchistycznych w~Glasgow, 1943.
Pięć funtów.

Wpatrzyłem się w~niego, z~otwartymi ustami. Nie zdawałem sobie sprawy,
że sięgam po nią, póki jej nie odsunął. 

-- Nie-e. Znalezione niekradzione.

-- To Cię nie zainteresuje -- powiedziałem.

-- Och, nie wiem. -- Reid zszedł z~drabiny, trzymając książkę jak czarny
Graal przed moimi oczami. -- Młodzi hegliści, \emph{Ideologia
niemiecka}\footnote{
zob.~\url{https://pl.wikipedia.org/wiki/Ideologia_niemiecka
} -- przyp.tłum.} i~tak dalej. Wiedza marksistowska.

-- Masz mnie!

-- Tak, dokładnie -- powiedział Reid. -- Ale mam powód. Zamierzam to kupić,
i jak tylko stąd wyjdziemy, zamierzam ci to sprzedać za dziesiątaka.

Zero lunchy przez dwa tygodnie i~znowu bułki. Mogłem sobie poradzić.

-- Umowa stoi! -- prawie krzyknąłem.

Reid zrobił krok do tyłu i~spojrzał na mnie badawczo.

-- Tylko testowałem -- powiedział. Włożył książkę w~moje ręce. -- Zdałeś.

~

W szarym, ołowianym świetle palarni Związku, powietrze gęste od
nieapetycznego zapachu starej kawy, siedzieliśmy w~zużytych fotelach i~przeglądaliśmy nasze nabytki. Uśmiechnąłem się na pokrętną dialektykę
apologetów wojennych, zmarszczyłem brwi na wymuszony humor wielkiego
amoralisty. Faszyzm, komunizm i~anarchizm wyprowadzały swój ród z~tego
samego Piltdown\footnote{
zob.~\url{https://pl.wikipedia.org/wiki/Cz\%C5\%82owiek_z_Piltdown
} -- przyp.tłum.}, berlińskich barów lat czterdziestych dziewiętnastego
wieku. Dajcie mi dowolnego dnia Wiedeń na przełomie wieków, pomyślałem,
Ringstrasse jako akcelerator idei.

Rozsiedliśmy się obaj w~tej samej chwili. Reid bawił wieszakiem na
gazety z~\emph{Guardianem} z~poprzedniego dnia. LRWA zdobyła Huambo, nie
po raz ostatni\footnote{zob~\url{https://pl.wikipedia.org/wiki/Ludowy_Ruch_Wyzwolenia_Angoli_\%E2\%80\%93_Partia_Pracy}
oraz \url{https://pl.wikipedia.org/wiki/Huambo } -- przyp.tłum.}.

-- Jak Annette? -- spytałem z~ostrożną swobodą.

-- Dobrze, o~ile wiem -- powiedział Reid. Przewrócił stronę.

-- Nie widziałeś jej dłużej?

Reid odłożył gazetę i~pochylił się do przodu, patrząc na mnie
intensywnie. 

-- Myśmy się tak\ldots nie wiem\ldots rozeszli się, odpłynęli.

-- To szkoda -- powiedziałem. -- Jak to się stało?

Reid rozłożył ręce. 

-- Ona jest naprawdę bystra, ale jest najbardziej
apolityczną osobą, jaką poznałem. Nigdy nie czyta gazet. Trudno znaleźć
sprawy do pogadania. -- Uśmiechnął się smutno. -- Brzmi głupio, wiem, ale
tak to jest.

Skinąłem głową życzliwie: tak, kobiety trudno zrozumieć. Próbowałem
przypomnieć sobie położenie Wydziału Zoologii.

~

Szedłem University Avenue, potężny wiktoriański gmach, Gilmoreghast, jak
jeden z~kpiarskich magazynów nazwał, po mojej lewej, Reading Room Wellsa
z lat trzydziestych po mojej prawej. (Nie korzystałem z~niego od
odkrycia, że wszystko w~nim jest doskonałe prócz akustyki, która
należała do galerii szeptów\footnote{zob.~\url{https://en.wikipedia.org/wiki/Whispering_gallery} -- przyp.tłum.}).

Na szczycie wzgórza, przejście dla pieszych miało czerwone światło.
Czekałem na zielonego człowieczka i~zastanawiałem się, czy powinienem
zawrócić tutaj i~poczekać, żeby ujrzenie Annette znowu mogło być
postrzegane jako zwykłe spotkanie...

\emph{Nie}, powiedziałem sobie stanowczo. \emph{Jeżeli jesteś
zainteresowany, będziesz tam}. Przeszedłem i~kontynuowałem aż do
skrzyżowania, potem w~lewo wzdłuż wewnętrznej drogi pomiędzy masywnymi
budynkami z~szarego piaskowca postawionymi pomiędzy płatami trawy z~kwietnikami i~wysokimi drzewami. Wydział Zoologii był kolejnym z~tych
antycznych budynków, solidnych jak kościół i~zbudowanych na starej
skale. Wewnątrz, polerowane drewno, kafle, zapach odchodów małych
zwierząt. Zza szklanej przegrody recepcjonista patrzył na mnie
obojętnie. Zdecydowałem się ośmielić i~spytałem go, gdzie pracowała
Annette. Spojrzał na zegar, terminarz i~mi powiedział.

~

Laboratorium na początku wydawało się puste. Potem zobaczyłem Annette,
jej plecy, rozkładającą kartki papieru na ławce na dalekim końcu.
Pchnąłem podwójne drzwi i~wszedłem. Odwróciła się na moje kroki, mówiąc:

-- Przepraszam, praktyczny nie jest\ldots Och, cześć Jon.

Jej włosy były spięte z~tyłu, jej figura ukryta w~białym fartuchu
laboratoryjnym. Ciągle nie mniej pożądana.

-- Cześć -- powiedziałem. Jej zielone oczy przyjrzały mi się zagadkowo.

-- Pozwól, że zgadnę -- powiedziała. -- Nagle odkryłeś zainteresowanie
anatomią bezkręgowców, prawda?

Wskazała na ławę. Spojrzałem na okrągłe szklane naczynie, w~połowie
wypełnione wodą, w~którym leżało kilka małych
jeżowców\footnote{więcej~\url{https://pl.wikipedia.org/wiki/Je\%C5\%BCowce
} -- przyp.tłum.}, lub raczej poruszało, co zobaczyłem, kiedy się
przyjrzałem. Rozłożone wzdłuż stołu były kręgi notatek, ze schematami
szkarłupni, nazewnictwo\footnote{nazwy odnoszą się do budowy układu ambulakralnego szkarłupni (rozgwiazdy) -- przyp.tłum.}  piękne i~dziwne: ampułki, pedicelarią, nóżki
ambulakralne, płytka madreporowa, kanał promieniowy, kamienny kanał,
kanał okrężny\ldots 

-- Niezupełnie. -- Bawiłem się mocnymi pęsetami, wyłożonymi jak sztućce do
rozbicia delikatnych bezbronnych stworzeń.

-- Więc co cię tutaj sprowadza?

-- Hm\ldots -- Zawahałem się. -- Tylko zastanawiałem się, czy masz ochotę
wyjść na drinka czy coś.

Jej twarz delikatnie się zarumieniła.

-- Czy to ma cokolwiek wspólnego z~Davem?

-- Nie -- powiedziałem, zastanawiając się, do czego zmierzała. -- Tylko że
powiedział mi, że już z~Tobą się nie spotyka.

-- Och! A kiedy ci o~tym powiedział?

-- Jakieś dwadzieścia minut temu -- przyznałem.

Roześmiała się. 

-- Co tak długo?

-- Pomyślałem, że zrywanie się w~chwili, kiedy mi powiedział, mogło być
trochę nieczułe.

To zabrzmiało jakby następstwa mojej wypowiedzi były zbyt bezpośrednie,
zbyt rażące. Spojrzała w~bok i~spojrzała na mnie z~półuśmiechem.

-- To bardzo miłe, że myślisz o~mnie -- powiedziała. -- Samotnej i~opuszczonej, jaką jestem. Nie jestem pewna, czy jestem gotowa na taką
łaskawość.

Jeżeli mogła się droczyć, mogłem odpowiedzieć tym samym. 

-- Nie
spodziewam się, żebyś długo pozostawała w~tym stanie.

-- Cóż -- powiedziała -- nie, nie myłam włosów każdego wieczoru!

-- Zatracając się w~zawrotnym wirze życia społecznego?

-- Tak.

-- Więc -- nalegałem. -- Może znajdziesz czas w~gorączkowych planach na
cichego drinka?

-- Lub coś.

-- Lub coś.

Uśmiechnęła się, tym razem rezygnując z~ironicznego spojrzenia.

-- Ok -- powiedziała. -- Co powiesz na dzisiaj o~dziewiątej w~Western Bar?

-- Widzimy się tam -- powiedziałem.

Drzwi uderzyły przy otwarciu i~zamieszanie studentów weszło.

-- Lepiej już idź -- powiedziała. -- Do zobaczenia.

W drzwiach spojrzałem do tyłu i~zobaczyłem, że patrzy. Uśmiechnęła się i~odwróciła.

Wybiegłem z~korytarza. \emph{Tak}!, powiedziałem światu, podskakując
i bijąc w~powietrze, co przestraszyło niektórych maruderów i~ledwo
minęło światło fluorescencyjne nad głową.

~

Western był cichym pubem, odpicowanym na jakąś próbę właściwej (to jest
kowbojskiej) dekoracji. Przyszedłem około dziesięć minut wcześniej i~stałem przy barze, pół piwa i~papieros później, kiedy Annette weszła
dokładnie w~chwili, gdy TV zapowiedział wiadomości o~dziewiątej. Barman
sięgnął i~przełączył kanały. (Były trzy, wszystkie kontrolowane przez
rząd).

Jej włosy były rozpuszczone (i sprężyste, i~błyszczące, i~właśnie
umyte). Była ubrana w~dżinsową spódnicę do połowy łydki i~czarną
jedwabną bluzkę pod bufiastą kurtką, którą rozpięła i~zsunęła, gdy
wchodziła. Kupiłem jej ,,lagera z~cytryną\footnote{lager and lime -- drink z~piwa z~odrobiną soku cytrynowego -- przyp.tłum.}'' i~znaleźliśmy stolik
przy ścianie.

-- Papieros?

-- Tak, proszę.

Zapaliłem jej papierosa i~patrzyliśmy na siebie przez chwilę. Nagle
Annette się roześmiała.

-- To jest głupie -- powiedziała. -- Znamy się dostatecznie długo, żeby
opuścić pogaduszki na wstępie, ale nie dostatecznie dobrze, żeby
wiedzieć co mówić dalej.

Porządny bystry umysł.

-- Słuszna uwaga -- powiedziałem, kręcąc się w~kółko. -- Właściwie nic o~Tobie nie wiem, prócz tego, że widziałem cię po drugiej stronie stołu
lub pokoju kilka razy.

-- Dave o~mnie nie opowiadał? -- Pobrzmiewał w~tym ton ciekawości wobec
jej domniemanej urazy.

-- Nie -- odpowiedziałem. -- Jednak, powiedział mi jedną z~bardzo ważnych
rzeczy o~Tobie...

-- Och tak?

-- Że nie jesteś zainteresowana polityką.

-- Czy to wszystko? Ha, i~teraz ja się zastanawiam, czy mówiłby ci tak
dużo o~mnie, jak ja mówiłam o~nim Sheenie.

-- To musi być ulga.

-- Pewnie tak\ldots Ale o~tym też jest w~błędzie! -- dodała.

-- Co masz na myśli?

-- Hm, to nie tak, że nie jestem zainteresowana. Po prostu nie lubię o~tym rozmawiać.

-- W~porządku -- powiedziałem. -- Ale dlaczego?

-- Dorastałam w~Belfaście -- powiedziała. -- Wyprowadziłam, kiedy miałam
około dziesięciu lat. Było tam powiedzenie: ,,Cokolwiek mówisz, nic nie
mów''. Ciągle mam tam rodzinę, ciągle odwiedzam. Zwyczaj zostaje.

-- Nawet tutaj? -- Rozejrzałem się. -- Co jest problemem?

Pochyliła się do przodu i~powiedziała przyciszonym głosem. 

-- Połowa
ludzi w~tym mieście ma powiązania z~Irlandią, a~kilku z~nich ma bardzo
zdecydowane poglądy. Więc nie warto strzępić sobie języka, szczególnie w~pubach.

Tak jak Dave miał skłonność, pomyślałem. Interesujące.

-- Ok -- powiedziałem. -- Nie jestem ciekaw. Nawet nie potrafię powiedzieć,
co pewnie każdy stąd mógłby: czy jesteś katoliczką, czy protestantką.
Ja, nie mam religii i~nie obchodzi mnie, jaka flaga powiewa nade mną,
lub co robią politycy, o~ile zostawią mnie w~spokoju.

-- Czego nie zrobią.

-- Ta, w~tym właśnie sęk!

Oboje się roześmialiśmy. 

-- Więc -- powiedziałem -- czym \emph{się
}interesujesz?

Myślała o~tym przez chwilę. 

-- Lubię moją pracę -- powiedziała.

-- Więc opowiedz mi o~tym.

I tak zrobiła, wyjaśniając, że nie robiła tylko technicznych rzeczy, ale
próbowała odkryć naukę za tym stojącą. Opowiadała o~ewolucji, populacji
i przyszłości obojga, i~to pozwoliło mi powiedzieć o~SF, a~ona
przyznała, że przeczytała kilka powieści ,,Wieczny Wojownik'' Michaela
Moorcocka\footnote{
zob.~\url{https://pl.wikipedia.org/wiki/Michael_Moorcock } -- przyp.tłum.}, kiedy była młodsza (czy też ,,młoda'', jak to czarująco
ujęła). Zanim się spostrzegliśmy, zadzwonił dzwonek na ostatnie
zamówienia.

-- Jest dyskoteka w~Joanne's -- powiedziała Annette. -- Pójdziemy tam?

-- Dobry pomysł -- powiedziałem.

Nie był. Nie byliśmy tam nawet półgodziny, gdy muzyka przestała grać i~DJ powiedział, żeby wszyscy zabrali swoje rzeczy i~wyszli po cichu.
Wszyscy wiedzieliśmy, co to znaczy: alarm bombowy. Annette złapała moją
dłoń z~zadziwiającą siłą i~wyciągnęła mnie z~tłumu, z~bezlitosnym
lekceważeniem innych, które dotychczas widziałem tylko w~tłoku w~barze
QM.

Wypadliśmy na ulicę akurat, gdy ktoś rozkazujący krzyknął ,,Fałszywy
alarm'' i~fala ruszyła w~drugą stronę. Annette jej się nie poddawała.
Spojrzałem w~dół na nią z~zaskoczeniem i~zobaczyłem, że nie tylko mżawka
moczyła jej twarz. Trzymając parkę na ramionach, wyglądała nędznie i~wrażliwie.

-- Nie chcesz wrócić? -- spytałem.

-- Chcę do domu -- powiedziała. Przytrzymałem jej parkę, kiedy walczyła,
żeby ją właściwie założyć. Złapała znowu moją dłoń i~zaczęła szybko iść.

-- Co się dzieje?

-- Och, Boże. Przypomniałam sobie pierwszy raz, kiedy byłam w~takim
alarmie.

-- Tak -- powiedziałem, próbując być uspokajający -- to szalone, jak się
przyzwyczailiśmy do alarmów bombowych.

Spojrzała na mnie z~czymś w~rodzaju litości.

-- To nie był alarm bombowy -- powiedziała miażdżąco. -- Byłam w~\emph{promieniu wybuchu} bomby. Lojaliści uderzyli w~bar lojalistyczny.
Chryste. Widziałam krzyczących ludzi i~nie mogłam ich usłyszeć.

Nie wiedziałem, czy to byłby dobry pomysł zapytać, ile osób było
rannych.

-- Przykro mi -- powiedziałem. Ścisnąłem jej dłoń. -- Nie wiedziałem.

Zatrzymała się, wytrącając mnie z~równowagi. Odwróciłem się, niepewnie,
twarzą do niej. Trzymała zaciśnięte pięści przed sobą, jak gdyby łapiąc
i potrząsając za klapy kogoś znacznie niższego niż ja.

-- \emph{Chryste}! -- wypluła. -- \emph{Nienawidzę} tego gówna! Nienawidzę
\emph{tak bardzo}! Tylko poszliśmy się bawić, wszyscy, i~jakaś jebana
świnia musiała to zrujnować! Winię ich za to wszystko! Za alarmy
bombowe, fałszywe alarmy, żarty, nie zdarzały się, gdyby nie te bękarty,
który robią to na serio. Uszy i~stopy po całym chodniku! -- Zamknęła
oczy, potem otworzyła, jak gdyby nie mogła znieść tego, co zobaczyła. -- A Dave zwykł mówić, że powinniśmy słuchać uciskanych. Nikt nie słucha
mnie, ponieważ nie jestem ,,uciskana''. Jestem \emph{jebano protestantko}!
-- Jej głos opadł do chrapliwego szeptu, resztka ostrożności, w~innym
wypadku posłana do nieba. -- Jebać ich wszystkich! Jebać papieża! Jebać
królową! Jebać Irlandię!

Tak nagle, jak jej wybuch się zaczął, się skończył. Oparła pięści na
moich ramionach i~spojrzała na mnie, nie uroniwszy łzy. Pociągnęła
nosem.

-- Boże, musisz myśleć, że jestem szalona -- powiedziała. -- Nie
zasługujesz na to.

Objąłem ją ramionami i~przyciągnąłem blisko, wykorzystując okazję, żeby
się rozejrzeć. Musiało to wyglądać, jakbyśmy właśnie mieli jakąś
kłótnię. Skoro w~Glasgow, a~ona nie używała butelki, to nikt nie zwracał
na nas nawet najmniejszego błysku uwagi.

-- Wolę to niż ,,cokolwiek mówisz, nic nie mów'' -- powiedziałem. -- Szczególnie że zgadzam się z~tym, co powiedziałaś.

-- Naprawdę? -- Odsunęła się i~zmarszczyła brwi. -- Masz na myśli, nie
wierzysz w~\emph{nic}? -- Jej głos był niedowierzający, pełen nadziei.

Przypomniała mi się kpina Myry: \emph{Akuratnie je żech
indywidualistycznym anarchistą}. Nie ma powodu w~to wchodzić w~ten
sposób, z~listą izmów. Wierzę w~Ciebie, pomyślałem, próbując, ale to też
nie było to. Patrzyła na mnie desperacko poważnie!

Przełknąłem. 

-- Ani Boga, ani kraju, ani ,,społeczeństwa''. Tylko ludzie i~rzeczy, i~ludzie jeden po drugim.

-- Tylko my?

Rozważyłem to, skuszony. To byłby dobry sposób, żeby ją bardziej
przytulić.

-- Ani nas także, póki każde z~nas nie wybierze, i~tylko tak długo, jak
każde z~nas wybiera.

-- Nie wiem, czy mogłabym żyć w~ten sposób.

-- Lepiej to niż umierać z~czymkolwiek innym.

Odpowiedziała na tę płynną odpowiedź bardziej zachęcającym uśmiechem,
niż zasługiwała.

-- Cóż -- powiedziała -- widzę, że nie próbujesz tylko ze mną pogadać. -- Złapała moją dłoń ponownie i~włożyła ją, razem ze swoją, do kieszeni jej
parki. -- Chodź, odprowadź mnie do domu.

Szliśmy mokrymi ulicami, jakbyśmy byli połączeni w~biodrach, zatrzymując
się co kilkaset metrów na zwarcie i~pocałunek. Żadne z~nas dużo nie
mówiło. Przy jej mieszkaniu, słabe światło i~chichoty dobiegły z~małego
pokoju Sheeny. Mieliśmy duży pokój i~sofę, dla siebie. Dużo się
przytulaliśmy, całowaliśmy, obmacywaliśmy i~tarzaliśmy, ale kiedy stało
się jasne, że chciałem iść dalej, odepchnęła mnie.

-- Za wcześnie -- powiedziała.

-- Dobrze -- powiedziałem.

-- Może powinieneś już iść. Niektórzy z~nas muszą jutro rano wstać.

Pomyślałem o~kilku bystrych odpowiedziach i~w końcu tylko skinąłem głową
i się uśmiechnąłem.

-- Może powinienem. A co z~jutrem?

Wstała i~pociągnęła mnie na nogi.

-- Niech zobaczę\ldots jadę na ślub w~sobotę. Jutro mam zakupy do zrobienia.
Wieczór panieński wieczorem, regeneracja następnej nocy. I~przygotowanie
sukienek i~tak dalej. -- Udała głęboki ukłon. -- Jak się zapatrujesz na
przyjście na tańce przy recepcji? Sobota wieczorem.

-- To brzmi wspaniale! Dzięki.

Oderwała kartkę z~notatnika i~nagryzmoliła na niej. 

-- Miejsce, czas,
dojazd autobusowy -- powiedziała, podając mi ją.

-- Dzięki wielkie. Ok, zatem tam się widzimy.

Znaleźliśmy się przy drzwiach.

-- Ciągle jeszcze musimy sobie powiedzieć dobranoc -- powiedziała i~dobrze
to zrobiła.

\threeast

Recepcja była w~hotelu w~części Glasgow, w~której wcześniej nie byłem,
do którego dotarłem w~serii autobusów poruszających się przez części
Glasgow, o~których nie wiedziałem, że istniały. Wyglądały, jakby została
tam przegrana wojna: całe bloki i~ulice zrównane lub zrujnowane, lampy
uliczne rozbite, wraki lub dzikie dzieci dookoła ognisk...

Potem dowiedziałem się, że to był efekt programu budowy drogi ukrytej w~postaci polityki mieszkaniowej, ale wtedy -- siedząc w~wypełnionym dymem
górnym pokładzie autobusu w~garniturze, który normalnie założyłbym tylko
na rekrutacje -- żywiłem pewne przyjemne pesymistyczne myśli o~rozkładzie
cywilizacji. Gdy autobus jechał, jednakże, wyspy ciemności stały się
rzadsze i~w końcu wyskoczyłem w~dzielnicy mieszkalnej przed uspokajająco
jasnym i~głośnym hotelem. Podążyłem za światłem i~hałasem do
apartamentu, gdzie odkryłem scenę jak z~dyskoteki, oprócz tego, że
większość osób była ubrana w~coś jak najlepszy strój do kościoła i~zakres wieku zbliżał się do krzywej rozkładu normalnego.

Dookoła ścian pokoju były stoły, bufet z~jedzeniem, tace z~drinkami i~bar na dalszym końcu. Wziąłem szklankę whisky przy bufecie i~rozejrzałem
się dookoła za Annette. Muzyka się zatrzymała, taniec się skończył,
ludzie ruszali na lub z~parkietu.

Annette wyszła z~tłumu, jakby to była zabawa tylko dla niej, przez
chwilę, wydawało się, że reflektor ją złapał, tak błyszczała, podczas
gdy inni dookoła przygaśli. Jej włosy były otoczone liśćmi i~czerwonymi
różyczkami, a~jej sukienka zaczynała się falbaną przy szyi i~kończyła
plisami na podłodze. Sukienka podobnie była w~różyczki, czerwone na
zielonym na czarnym, a~nad tym miała fartuszek z~organzy z~żabotem od
talii aż ponad ramię, same taśmy zwinięte w~kokardę z~przodu. Na jej
twarzy, zarumienionej od tańca, był uśmiech. Gdy się zatrzymała przede
mną, poczułem jej mocne, słodkie perfumy.

-- Cześć, Jon, masz fajkę? -- powiedziała. -- Nie mam tchu.

Gdy zapalałem papierosa dla niej, złapała mnie za rękę i~pociągnęła do
krzesła przy stole. Przyciągnęła kolejne krzesło i~usiadła twarzą do
mnie, nasze kolana prawie się dotykające przez szeleszczącą masę jej
spódnic.

-- Ach, tak lepiej -- powiedziała. Mijający kelner zaproponował jej tacę,
sięgnęła koło oczekiwanego wina i~podniosła kieliszek whisky. -- Dzięki
za przyjście.

Podniosłem szklankę. 

-- Dziękuję \emph{Ci}. Wyglądasz inaczej. Pięknie.

-- Och, jej, dzięki.

-- Pięknie na inny sposób -- pośpieszyłem dodać.

Uśmiechnęła się lekko, żeby pokazać, że tylko udawała nieporozumienie.

-- Nie wspominałaś, że jesteś druhną -- powiedziałem.

-- Nie chciałam cię wystraszyć.

Roześmiałem się, niepewny, co z~tym zrobić. 

-- Podoba mi się Twoja
sukienka -- powiedziałem.

Pochyliła się bliżej i~powiedziała plotkarskim szeptem: 

-- Tak jak i~mi.
Urobiłam się, żeby dostać taką, w~której mogłabym jeszcze chodzić na
imprezy, więc po długich dyskusjach z~Irene, to panna młoda, chodziłam z~nią do szkoły, umówiłyśmy się na tę fajną kreację Laury Ashley\footnote{
zob~\url{https://en.wikipedia.org/wiki/Laura_Ashley } -- przyp.tłum.} Wtedy ona zdecydowała, że nie była dostatecznie
\emph{brzydka }i \emph{druhnowata}, więc załatwiła z~jej Mamą te rzeczy.
-- Potrząsnęła pogardliwie falbaną fartuszka.

-- Och, nie wiem -- powiedziałem. -- Fartuszek jest najlepszy. Naprawdę
musisz go zachować na imprezy. -- Tylko częściowo się droczyłem, było coś
niewątpliwie seksownego, w~niewątpliwie seksistowski sposób, o~tego typu
ciągnących się skojarzeniach z~feministyczną służebnością.

-- Och, tak, i~być postrzeganym jako dziewoja? -- Uśmiechnęła się.

-- Nigdy -- powiedziałem. -- Moja pani, czy chciałabyś zatańczyć?

-- Dobrze -- powiedziała, zastanawiając się -- może po tym, jak napełnisz
kieliszek, a~ja go opróżnię.

~

Kiedy to zostało załatwione, więcej niż raz, Annette przedstawiła mnie
niektórym z~jej przyjaciół i~kuzynów, i~taniec zmienił się z~podskakiwania w~stylu disco do tradycyjnego, ale dzikszego, tańca
szkockiego. Annette wciągnęła mnie w~to, a~kiedy nagle zaczęła się
rzucać, jak wspomnienie z~poprzedniego życia, odkryłem, że znam kroki,
układy i~mogłem ją obracać -- i~oszałamiające, obracające następstwo
kolejnych partnerek -- na równi z~najlepszymi z~nich.

Gdy tańczyłem, przeskakiwałem, tupałem, odwracałem, kręciłem, podnosiłem
i kołysałem, próbowałem sobie przypomnieć, skąd pamiętałem to wszystko,
i zrozumiałem, że to wszystko było dzięki mojemu ojcu. Jego
interpretacja marksizmu, o~szerokich horyzontach nawet dla jego
społecznie tolerancyjnej, tylko politycznie dogmatycznej, partii,
nalegała na celowość kultury w~każdej z~jej form. Stąd, ćwiczenia na
pianinie i~lekce tańca, a~kiedy to doprowadziło do podwórkowych kpin,
lekcji boksu. Stąd również, Muzeum Nauki, Muzeum Historii Naturalnej,
Zoo i~teatr. Interesował się wszystkim. Był tam.

A w~Hyde Park w~niedziele, opowiadając niedowierzającym gapiom, że
jakakolwiek demonstracja tygodnia, która ich mijała, była całkowitą
stratą czasu\ldots Myślał, że zamienia ucznia wieku kosmosu w~naukowego
socjalistę, ale wszystko, co robił, to wychowywanie do bycia tak upartym
człowiekiem jak on sam.

Tańce mijały tak szybko jak tancerze, pomiędzy jednym a~drugim tylko
porwane łyki whisky i~zaciągnięcia dymem. Reel na osiem zakończył
zestaw. Annette i~ja oparliśmy o~swoje ramiona z~jedną myślą pomiędzy
nami. 

-- Drink?

-- Drink.

Poszliśmy do baru tym razem, nasza przypadkowa i~szczęśliwa pozycja na
końcu tańca zaprowadziła nas przed resztą. Annette usadowiła się na
stołku, jej spódnica zakrywającego go, tak, że wyglądała jakby
zawieszona w~powietrzu. Oparłem łokieć o~bar i~zamówiłem piwa.

-- Hm, to było coś -- powiedziałem. -- Podobało mi się.

-- Mnie też -- powiedziała Annette. -- Zdrowie. -- Wypiła pół kufla lagera.
-- Ale -- kontynuowała -- wyrzucenie najmniejszej druhny w~powietrze,
przerzucenie panny młodej nad biodrem i~przeniesienie jej babci przez
połowę pokoju nie było absolutnie niezbędne.

-- Och. -- Przemyślałem. -- Czy ja to zrobiłem?

Uśmiechnęła się. 

-- Na pewno. Byłam dumna. Nikt nie będzie narzekał
teraz, że sprowadziłam dziwnego Angola\footnote{ w~oryg. Sassenach -- termin
używany przez gaelickich mieszkańców wysp brytyjskich na oznaczenie
mieszkańców angielskich  -- przyp.tłum.}.

-- Nie wiedziałem, że jestem przedmiotem debaty.

-- Cóż, teraz to będzie tylko spekulacja. -- Mrugnęła.

-- O nas?

-- Aha -- powiedziała Annette. -- Więc są ,,my''?

Twarz nagle poważna, w~aureoli czerwieni i~czerni.

-- Jeżeli chcesz -- powiedziałem.

Jej zielone oczy spojrzały na mnie spokojnie.

-- A co Ty wybierasz?

Dookoła nas ludzie krzyczeli, sięgali po drinki, ocierali się o~nas.
Muzyka znowu kołysała. Widzę i~słyszę to tylko teraz. Wtedy nie było nic
prócz niej.

-- Nie ma wybierania -- powiedziałem. Zrobiłem krok i~objąłem ramionami
jej pas. Nasze czoła się dotknęły. -- Wszystko było zdecydowane w~chwili,
kiedy Cię zobaczyłem.

-- Dla mnie też -- powiedziała i~się pocałowaliśmy. Dziwnie było to robić
na tej samej wysokości. Do czasu, kiedy skończyliśmy, zsunęła się ze
stołka. Spojrzała na mnie, uśmiechając się i~powiedziała: -- Ale ja
zobaczyłam cię pierwsza.

-- Więc -- spytałem w~gorzkim zdumieniu -- o~co chodziło przez ostatnie
trzy miesiące?

-- Jestem taka jak Ty -- powiedziała. -- Chcę być wolna.

-- Możesz być wolna ze mną! -- powiedziałem. -- W~dowolnym momencie.
Proszę.

Śmialiśmy się razem.

-- Tak -- odparła.

I wtedy wszystko zostało powiedziane, a~my po prostu staliśmy przy
barze, pijąc.

Irene, panna młoda, przystukała do nas w~wysokich obcasach i~zgrabnym
niebieskim dwuczęściowym, uśmiechnęła się do mnie ostrożnie i~poszeptała
do Annette.

-- Do zobaczenia za parę minut -- powiedziała Annette. Ukłoniłem się obu -- i~tej konieczności -- i~obserwowałem ich szeptany postęp z~odległości.

~

Annette wróciła jakiś kwadrans później.

-- Wszystko ok? -- spytałem, przesuwając gin z~tonikiem. Wyglądała na nieco
zaabsorbowaną.

-- W~zasadzie tak. Dzięki -- powiedziała, sącząc ostrożnie. -- Po prostu
spędziłam dziesięć minut, stojąc w~recepcji z~sukienką ślubną Irene w~plastikowej torbie na ramieniu. \emph{W końcu} załatwiłam kogoś, żeby ją
przechowywał aż do mojego wyjścia. Nie mogłam jej zostawić w~pokoju.
Jakiś błąd z~kluczami.

-- Więc bycie druhną to nie sama zabawa.

-- Ha, ha. Mało wiesz.

-- Myślę, że raczej nie...

Zauważyłem, że muzyka się zatrzymała i~ktoś próbował być usłyszany ponad
zgiełkiem.

-- Hej, no chodź!

Annette zawirowała i~ruszyła w~kierunku najbliższego wyjścia, gdzie
Irene i~jej mąż wycofywali się do drzwi w~rodzaju kobiecego ścisku
dookoła nich...

Coś popłynęło nad głowami przepychanek. Gdy spojrzałem do góry,
zaskoczony, Annette wyrzuciła rękę w~powietrze jak gorliwa uczennica do
odpowiedzi i~złapała to. Potrząsnęła bukietem, gdy odwróciła się powoli
dookoła, przyjmując gwizdy i~okrzyki, i~spojrzała na mnie z~szerokim
uśmiechem.

-- Cóż -- powiedziała. -- Szczęściara ze mnie.

Wszyscy zgromadzili się na zewnątrz, żeby wysłać młodą parę w~drogę.
Para przebiegle wezwała taksówkę i~zostawiła samochód pokryty pianką do
golenia i~szminką deszczowi do umycia.

Potem więcej tańczenia, więcej rozmawiania i~długa podróż taksówką do
mieszkania Annette z~suknią Irene ułożoną na naszych kolanach. Gdy
płaciłem kierowcy, pobiegła po schodach jej domu, śmiejąc się, jej
spódnica w~jednej dłoni, a~druga sukienka powiewająca za nią jak kometa.
Dopadłem do niej, gdy otwierała zewnętrzne drzwi. Zeszliśmy po schodach
do jej ciemnego mieszkania, głośno próbując być cicho.

Zabrała mnie prosto do sypialni, zawiesiła suknię ślubną na wieszaku na
drzwiach do szafy naprzeciw łóżka i~odwróciła się do mnie. Złapałem
taśmy jej kokardy na jej biodrach, szarpnąłem je i~ona zakręciła się
dookoła, łapiąc fartuszek, gdy spadał i~posyłając go w~róg. Grzebałem
się z~guzikami z~tyłu jej sukienki, znalazłem ukryty suwak i~rozpiąłem
go. Sukienka opadła dookoła jej stóp. Wyszła z~jej kręgu w~długiej
nylonowej halce i~zgrabnie rozpięła każdy guzik mojej koszuli, podczas
gdy ja pozbywałem się butów, spodni i~slipek tak szybko, jak to możliwe.
Halka zsunęła się do jej stóp z~trzaskiem statycznej elektryczności.
Reszta jej bielizny zabrała przyjemnie więcej czasu.

Objąłem jej piersi dłońmi i~zatopiłem usta pomiędzy nimi. Jej skóra
smakowała jak talk i~sól. Trzymając ją na odległość ramion, żeby na nią
popatrzeć i~trzymając ją blisko, żeby dotknąć jej, doprowadziło do
bliskiego, szybkiego rytmu, gdy wpadliśmy na jej łóżko.

-- Hej, hej, hej -- powiedziała. Położyła dłoń na moim ramieniu i~przytrzymała mnie, sięgnęła za jej głowę i~pomachała małym foliowym
opakowaniem przed moją twarzą. Potem rozdarła opakowanie zębami.

-- Załóż to, ty nieodpowiedzialny bękarcie.

-- Nie chciałbym być odpowiedzialny za bękartów -- zgodziłem się.
Założyłem kondom na penisa. -- Mam swoje, po prostu zapomniałem.

-- Jeżeli kiedykolwiek powiesz mi coś tak słabego, to cię wyrzucę, Jonie
Wilde.

Próbowałem przez chwilę wymyślić jakąś odpowiedź, a~potem wykorzystałem
język w~lepszym celu.

~

Obudziłem się w~pokoju zacienionym w~świetle zasłonek rano, moje
kończyny ciągle splątane z~Annette, i~zostałem natychmiast zaskoczony
przez ducha białej sukni wyłaniającej się znad końca łóżka, jej falbany
i zaspy koronek osłaniane przez błyszczące pole siłowe polietylenu, jak
duch z~przyszłości.

\chapter{Miasto Statku}

Najpierw patrzymy z~perspektywy (większej niż się zdaje) i~łapiemy
mijającą planetę z~odległości stu tysięcy kilometrów. Jest czerwona -- to
nie niespodzianka -- ale jest pocętkowana ciemnymi wyciekami niebieskiego
i plamami zielonego, a~te wycieki i~plamy zaczynają być połączone\ldots 
kanałami, przez\ldots (i pojawia się myśl, ulotna jak przelotne spojrzenie)
\emph{canali}, więc Nowy Mars naprawdę wygląda, jak Mars nigdy nie
wyglądał. (Ale czy nie tego chcieliśmy).

Widok przeskakuje na tysiąc kilometrów\ldots \emph{nad}, nie \emph{poza} \ldots
i toczymy się na wysokości satelitów, pobierając spiralne odciski palców
wilgoci wody, zakrzywiająca się płyta planety zaparowana oddechem,
nabazgrane znaki życia i~proste linie inteligencji: tak, kanały.

Opadając teraz, do struktury tak niewątpliwie sztucznej jak
najwidoczniej organicznej: na pierwszy rzut oka to czarna gwiazda, jak
stolica na mapie, potem (gdy punkt widzenia pędzi i~widok zaczerwienia
się, krwawy od płomieni hamowania) jak rozgwiazda pozostawiona na
piasku.

Cięcie, znowu, do spokojnego powietrznego punktu obserwacyjnego,
dryfującego ponad czymś, co najwyraźniej jest miastem, jego promienista
symetria jest ciągle główną cechą, ale z~pięcioma ramionami
dostrzegalnie połączonymi czarnymi niciami dróg, ulic, kanałów, a~na
innym poziomie, niewidocznymi z~zewnątrz, przez pajęczynę okablowania
\emph{sieci}.

I jesteśmy. Ten stary protokół transakcyjny TCP/IP ciągle działa (z
odległych czasów mitochondrialnej Ewy wszystkich systemów), zatem możemy
słyszeć, czuć i~widzieć. Ale wielkie liczby nadal są ważne, więc
szyfrowanie ukrywa większość danych w~katakumbach ciemności. To, do
czego mamy dostęp, na otwartych kanałach, jest wystarczające, żeby
pokazać.

~

Cztery z~miejskich pięciu ramion są dziedzinami nie-ludzi. Wyglądają,
jakby były zbudowane na mieszkania dla ludzi, ale nikt nie jest w~domu,
prócz maszyn. Istnieje podstawowa warstwa, rodzaj mechanicznej warstwy
uprawnej, gdzie rzeczy robią rzeczy rzeczom. Symulakra inteligencji
poruszają się przez ruch, krzycząc i~pracując: puste automatyczne barki
orzą zatkane algami kanały, służące maszyny walczą, żeby zamieść kurz z~podłóg korytarzy, których ściany są grube od pleśni. Na ulicach
kreacjonistyczna karykatura naturalnej selekcji: na wpół sformowane
mechanizmy zderzają się, łączą i~wcielają swoje części, tworząc
nieopłacalne potomstwo, które samo dalej rozprzestrzenia groteskowe
formy przejściowe.

Ten bezmyślny poziom jest terenem łowów bardziej rozbudowanej
maszynerii, która czai się i~rzuca, pożera i~kanibalizuje dla własnych
celów. Sztuczne inteligencje -- niektóre obsesyjne i~skupione, inne
chaotyczne i~zrelaksowane, niektóre nawet zdrowe na umyśle -- prześladują
ułamek tych maszyn. Trudno jest zidentyfikować miejsca, gdzie takie
umysły mieszkają. Chwiejące, nieprawdopodobne struktury mogą być
sterowane przez rozumne komputery nie większe od myszy, podczas gdy
niektóre gładkie, lśniące, a~nawet humanoidalne maszyny mogą być równie
dobrze kretyńskie lub szalone.

Te jęczące złomowisko jest regularnie plądrowane przez istoty ludzkie,
które, ryzykując wszystko od palców przez dusze, zapuszczają się w~tę
dżunglę żelaza i~krzemu. Mają swoich mechanicznych sojuszników,
zwiadowców i~agentów. Ale jeżeli maszyny, w~ogóle, nie są lojalne wobec
siebie, są jeszcze mniej wobec ludzkich przyjaciół lub panów. Łatwiej
jest przeprogramować maszynę niż obalić człowieka.

A przez to wszystko, jak bakterie, małe molekularne maszyny dzikiej
nanotechnologii poruszają się w~swojej niewidzialnej i~okazjonalnie
niszczycielskiej pracy. Systemy odporności wyewoluowały, praktykowany
jest ekwiwalent medycyny. Środki higieny publicznej są stosowane (nie
są, dokładnie, wymuszane). Jednak najmniejsze są najszybsze, a~tutaj
wyścig ewolucji jest najbardziej nieludzkim biegiem.

~

Piąte ramię jest dzielnicą ludzką. Sieci są jego umysłem. W~nich
odnajdujemy jej dobre intencje, złe myśli, mokre sny i~nudne programy.
To nie tak, jak to powinno być ostatecznie oceniane. Niemniej jednak...

Podstawą wszystkiego jest odtwarzanie codziennego życia, a~to zapewnia
duży udział w~ruchu sieciowym. Nikt nie liczy, ale istnieje kilkaset
tysięcy ludzkich istot żyjących na Nowym Marsie, większość z~nich w~Mieście Statku, reszta rozrzucona w~znacznie mniejszych
społecznościach, rozdmuchana po całej planecie. W~każdej minucie
brzęczy tysiącami rozmów i~osobistej łączności. Biznes: zamówienia,
faktury, płatności, transakcje. Prawa własności, co ludzie godzili się
pozwolić innym ludziom robić z~rzeczami, skomplikowały się i~zróżnicowały, a~rozwiązywanie, przepakowywanie i~wymiana tych praw
następuje z~szybkością szulera: udziały w~czasie, hipoteka organów,
giełda nowinek, pożyczki na pracę, korzyści z~narodzin\ldots to stało się
skomplikowane. Stąd konflikty, pozwy, ugody, zbrodnie i~delikty.

Prawo i~porządek pokazuje zęby w~strumieniu biznesu tylko okazjonalnie,
a wynikające dokumenty o~policji, dramaty sądowe czy komedie obozowe
zapewniają -- w~rzeczywistości i~w fikcji -- podstawę rozrywki. Większość
udręk i~poniżeń, które widzimy na ekranach to, szczęśliwie, tylko
pornografia. Rozprawy sądowe przez sąd boży i~walkę są realne.

Religie, niektóre. Najwyższym kościelnym dygnitarzem jest biskup Nowego
Marsa. Zreformowana ortodoksyjna katoliczka, więc dlaczego ma dziwne
wątpliwości, w~jaki sposób Następstwo przeszło na nią, wie, że przekaże
je jednemu lub więcej dzieciom. Przyjaźni się z~niektórymi buddystami i~rabinem (znaczy, nie \emph{spodziewaliście się} Żydów?) i~jest sroga,
ale miłosierna wobec obłąkanych heretyków. Ich ułuda, że Nowy Mars jest
życiem po śmierci, albo jakimś postapokaliptycznych obszarem
przejściowym jest, w~tych okolicznościach, wybaczalna.

Polityki, brak. To anarchia, pamiętasz? Jednak to jest anarchia
\emph{domyślna}. Nie ma państwa, ponieważ nikt nie ma ochoty go założyć.
Zbyt dużo mordęgi, facet. Bądź grzeczny, nie wychylaj się, to zawsze
działo się w~ten sposób i~nic tego nie zmieni, a~zresztą (i szczególnie)
\emph{co pomyślą sąsiedzi}? (Nigdy nie poprą tego, oto co. To wbrew
ludzkiej naturze.)

Zewnętrzny system nerwowy miasta zawiera jego zmysły: kamery, mikrofony
dla wiadomości i~inwigilacji, czujniki chemikaliów i~stresu, które
monitorują jego zdrowie. Zaczynając od góry: na najwyższej i~centralnej
wieży jest glob wielkości ludzkiej głowy. To tylko dookolna kamera,
udogodnienie zostawione tam przy rozkwicie ducha publicznego lub
prywatnej spekulacji. Stamtąd możemy zajrzeć w~oszałamiające
przestrzenie szczytów wież, które w~końcu schodzą do niskich płaskich
dachów i~kończą się kopułami, szopami i~wiatami na granicy miasta.

Jak każde w~pięciu promienistych ramionach miasta, to ma wydłużony
kształt latawca, najpierw szeroki, potem zbieżny. Same budynki są dwóch
typów: te, które wyrosły, i~te, które zostały zbudowane. Kształty tych
pierwszych mogą być przedstawione jako przecinające się wielokąty,
regularne lub nieregularne: te drugie, to prostokąty. Układ i~położenie
tych kratowanych, komórkowych struktur ma tę samą jakość przypadkowej
nieuchronności jak głazów w~lawinie lub kamieni w~strumieniu, i~z tego
samego powodu: minimalnego zajęcia dostępnej przestrzeni. Skonstruowane
budynki przestrzegają innej zasady ekonomii i~wystają lub są zakopane,
tak jak dyktują nieprzewidywalne prawa.

Oba typy budynków -- oba prawa lokalizacji -- podążają za ulicami, a~ulice
za kanałami. Kanały są systemem oddechowym: Kanał Okrężny okrąża
centrum, Kanały Promieniste dzielą ramiona, a~każdy ma niepoliczone
dopływy i~kapilary. Niedaleko lewej krawędzi ramienia, na które
patrzymy, jest nieprawidłowy długi kanał, który pierwszy pojawia się w~polu widzenia pod nami i~wykracza poza horyzont: Kamienny Kanał.

~

Mężczyzna opiera się na wnęce okna, rozkładając część masy na
rozcapierzonych palcach. Cement jest szorstki pod palcami. Wygląda przez
okno, które jest wysoko na stoku miasta, patrząc wzdłuż Kamiennego
Kanału. Gdy równoważy ciężar ciała na piętach stóp i~czubkach palców,
napięte mięśnie ramion i~barków pokazują się przez miękką tkaninę jego
kurtki. Mięśnie zginają się i~się prostuje, odwracając. Jego czarne
włosy pstrykają w~policzek w~wyniku prędkości ruchu.

Pozostali dwaj mężczyźni w~pokoju są wyżsi i~masywniejsi niż on, ale
obaj lekko się cofają, gdy zmierza w~ich kierunku. Zatrzymuje się kilka
metrów wcześniej i~patrzy na nich.

-- Straciliście ją -- mówi. -- U abolicjonistów. -- Jego wymowa ma akcent
nieczęsto słyszany w~mieście, coś z~przeszłości, chropowaty i~wyrafinowany przez długi czas. Zapewnia zgrzytliwy wydźwięk modulacji
jego głosu, który jest podobnie, świadomie lub nie, doświadczonym i~znakomitym instrumentem jego woli. Akcent i~ton razem są doskonale
skalibrowane, żeby przekazać emocję: w~tym przypadku, pogardę.

-- Ona ma koncesję IBM -- mówi jeden z~mężczyzn. Liże usta, chowa język
nagle w~usta, jakby był świadom, że poszedł za daleko. Wyciera policzek.

-- To -- mówi mężczyzna -- nie jest usprawiedliwienie. To opis porażki. -- Wzdycha, strzepuje pył z~palców. -- Dobrze. Od początku.

Wraca do wielkiego drewnianego biurka i~opiera się o~jego krawędź.

-- Ok, Reid -- mówi drugi mężczyzna i~zaczyna przedstawiać. Mówi przez
minutę, gdy Reid unosi dłoń.

-- Młody mężczyzna? -- mówi. -- I~robot? Opisz ich.

Słucha, zmrużona oczy, przez kolejną minutę, zanim przerwie gestem dłoni
skierowanym w~dół.

-- Myślisz, że ją rozpoznał, Stigler?

Usta Stiglera są znowu suche.

-- On\ldots myślę, że tak.

-- Och, \emph{Chryste}! -- Słowa pojawiają się jak pręt uderzający w~biurko. Reid przez chwilę stuka palcami.

-- A Ty, Collins, nie wydaje mi się, że Twoje moce opisowe są w~lepszym
stanie, co?

-- Osłaniałem, Reid -- mówi Collins. -- Patrząc wszędzie indziej, wiesz, co
mam na myśli?

-- Ok, ok. -- Reid wstaje i~patrzy na nich, spekulacyjnie. Może rozważać
korzystne użycie ich części ciała i~stosowne metody renderowania. -- Wykonaliście pracę, jak się umawialiśmy, tak dobrze, jak mogliście.
Gdybym chciał wyciągnąć mężczyznę na podejrzeniu, potrzebowałbym nakazu.
I to jest to, czego będę potrzebował, panowie, więc obawiam się, że to
was wyklucza. Pełna zapłata, bez premii.

Collins i~Stigler patrzą z~ulgą i~odwracają się do wyjścia. W~drzwiach
Collins drapie się w~kark, patrzy na Reida. Reid patrzy znad ekranu, na
który przeniósł swoją uwagę.

-- Tak?

-- Hm, Reid, pytanie. Nie zdarzyło ci się wiedzieć, kto \emph{posiada}
tego robota?

Reid myśli o~tym. Jego uśmiech pozwala dowiedzieć się mężczyznom, że są
jego dobrymi przyjaciółmi, a~nie parą łapaczy, którzy nie wrócili z~danymi.

-- Zostańcie przy sprawie -- mówi im.

~

Wilde wstał i~przeszedł na koniec nabrzeża, koło ludzi, inteligentnych
małp i~maszyn, które mogły być inteligentne. Patrzył nad Kamiennym
Kanałem, a~potem przez chwilę patrzył w~dół na wodę. Znalazł, może,
jakieś odpowiedzi w~swoim odbiciu.

Robot, Jay-Dub, ciągle kucał na krawędzi nabrzeże, gotowy jak ptak
drapieżny. Wzorce ciekłego kryształu przesunęły się w~jego ocienionym
centralnym pasie, gdy wrócił Wilde. Wilde spojrzał na to.

-- Nie jesteśmy już w~Kazachstanie -- powiedział.

Maszyna nie odpowiedziała.

-- Co \emph{się stało}? -- spytał Wilde. Rozejrzał się. -- Czy tu można
bezpiecznie rozmawiać?

-- Wystarczająco bezpiecznie -- odpowiedział Jay-Dub. -- Mogę odbić
większość prób podsłuchu.

-- W~porządku -- powiedział Wilde. -- Powiedz mi: gdzie schowałem pistolet?

-- W~prysznicu.

-- Jaka była ostatnia rzecz, którą powiedziałem?

-- ,,Miłość nigdy nie umiera''.

Wilde zmarszczył brwi.

-- Jaka była moja ostatnia \emph{decyzja}?

-- Że ja..., że już nigdy nie będziesz palił.

Wilde pochylił się i~klepnął w~kadłub Maszyny.

-- Dokładnie tak. To jest obietnica, którą pamiętam, i~możesz się nadal
jej trzymać.

Zabrał szklanki po kawie do stoiska ze śniadaniami i~wrócił z~pełną
szklanką, paczką papierosów i~zapalniczką.

-- Nie pochwalam tego -- powiedział Jay-Dub, gdy Wilde usiadł koło niego i~zapalił.

-- Pierdol się -- odpowiedział Wilde. -- Chcę Twojej historii, nie opinii.

Oparł się o~skorupę Maszyny, która przesunęła swoją wagę na nogach, żeby
wyrównać.

-- To długa historia. Nawet nie wiesz jak długa.

-- Więc ją skróć. -- Oczy Wilde'a były zamknięte.

-- ,,Tak, panie'' powiedział robot -- powiedział Robot. -- Ok, cokolwiek
mówisz. Praktycznie, umarłem po postrzale. Mój mózg został natychmiast
zeskanowany w~prototypowym systemie obrazowania neuronowego i~wzorzec
został zapisany.

-- Daj spokój -- powiedział Wilde. -- Nie robimy\ldots nie mieliśmy takich
rzeczy.

-- Ludzie Reida mieli. Byli bardziej zaawansowani niż ktokolwiek
podejrzewał. A ja byłem pierwszy. Pierwszym człowiek, w~każdym razie.
Jestem przekonany, że większość ulepszonych małp tutaj pochodzi ze
wczesnych eksperymentów tego okresu. Jednak, to było wiele lat później -- choć nie, oczywiście, subiektywnie -- gdy otworzyłem oczy i~okazało się,
że jestem na nieprawdopodobnym statku kosmicznym. Wygodnym, jeden g, ale
żadna rotacja lub przyśpieszenie nie były widoczne, gdy wyglądałem na
zewnątrz. Oczywiście wirtualna rzeczywistość. To, co było za oknami,
było tym, co było w~rzeczywistym świecie.

Przerwał. Minuta minęła. Mężczyzna sięgnął ręką do tyłu i~uderzył
kostkami w~bok Maszyny. Potem possał kostki.

-- A to, co było na zewnątrz, to?

-- Ganimedes\footnote{
zob.~\url{https://pl.wikipedia.org/wiki/Ganimedes_(ksi\%C4\%99\%C5\%BCyc) } -- przyp.tłum.}, chyba -- powiedział Robot.  -- To, co z~niego pozostało. Maszyna, którą zamieszkiwałem,
nie była znacznie większa od tej, którą teraz widzisz. Ta i~tysiące
innych, były zaangażowane w~budowie platformy. Wszędzie dookoła
pierścieni Jowisza, inne maszyny były zaangażowane w~podobnych zadaniach.

Znowu głos się urwał.

-- Pierścienie \emph{Jowisza}? -- spytał Wilde. -- Ktoś był zajęty.

-- Zgadnij kto.

-- Reid?

-- I~spółka.

-- Oni to zrobili? Kiedy?

-- 2093.

Wilde otworzył oczy i~spojrzał ponad kanałem.

-- Zakładam -- powiedział -- że ludzie i~roboty równoważne ludziom nie
zrobili tego wszystkiego sami.

-- Rzeczywiście nie. Pomiędzy belkami platformy były wielkie byty, które
nazywaliśmy makrami. Były stworzone z~nanomaszyn i~były platformą
hardware dla milionów wgranych umysłów. Ludzie tutaj, teraz, nazywają
ich ,,Szybkim Ludkiem''. Już wtedy byli daleko poza ludzkością i~budowali
tunel czasoprzestrzenny, ten, którym nasz statek przeszedł, żeby się
tutaj dostać.

-- Gdzie są teraz?

-- Ach -- powiedział Jay-Dub. -- Dobre pytanie. Te dookoła Jowisza straciły
zainteresowanie, powiedzmy, w~zewnętrznym świecie. Prototypy, z~których
się rozwinęły, kod źródłowy, jeżeli wolisz, zabraliśmy ze sobą, tak jak
zabraliśmy zapisane umysły i~zakodowane ciała zmarłych.

-- W~tym mnie?

-- Cóż, tak. Twoje rzeczywiste ciało nie było zakodowane, z~tego, co
wiem. Istniała próbka tkanki, z~której potem, z~której cię sklonowałem.
Twój umysł był zakodowany, tak jak mówiłem.

-- Oddzielnie od Twojego? -- Wilde brzmiał zaskoczony.

-- Mój umysł i~Twój były skopiowane z~tego samego oryginału -- powiedział
Jay-Dub. -- Obudziłem się w~tej maszynie w~dokładnie tym samym stanie
umysłu, jak Ty obudziłeś się wczoraj, i~z dokładnie tymi samymi
wspomnieniami. I~w~mniej pomyślnych okolicznościach.

-- Moje serce -- powiedział Wilde -- absolutnie kurwa krwawi.

-- Moje niezwykle rozbudowane oprogramowanie wykrywa stopień wrogości. -- Głos Maszyna próbował ironii, coś spoza jego znajomego zasięgu.

-- Mam nadzieję, że tak -- powiedział Wilde. -- Właśnie przyznałeś, że
klony są oddzielnym problemem od zachowanych umysłów. Więc obecność
kogokolwiek tutaj, kto wygląda jak ktoś, kogo kiedyś znałem, nie jest
oznaką w, kurwa, ogóle, że ta osoba właściwie tutaj jest, prawda?

-- W~zasadzie tak, ale\ldots

-- Więc Twoja uwaga o~klonie będącym jakiegoś rodzaju powodem do nadziei,
że Annette była, jak to ująłeś, pośród Nieożywionych, była całkowicie
kłamstwem.

-- Nie -- powiedziała Maszyna. -- To oznacza, że jest szansa.

Wilde pokręcił głową.

-- Im bardziej o~tym myślę -- powiedział -- tym więcej wątpię. Nigdy nie
wierzyła w~krionikę, ani w~transfer umysłu\footnote{ upload, przeniesienie umysłu człowieka w symulację komputerową, zob.~\url{https://pl.wikipedia.org/wiki/Transfer_umys\%C5\%82u
} -- przyp.tłum.}, lub jakiekolwiek takie gówno. Jeżeli wierzyła w~cokolwiek, to wierzyła w~powszechne zmartwychwstanie na końcu czasu.
Punkt Omega\footnote{więcej~\url{https://en.wikipedia.org/wiki/Omega_Point } -- przyp.tłum.}.

-- I~całe to gówno -- powiedział Jay-Dub.

Wilde się roześmiał. 

-- Nadal tak myślisz? Hm, chylę czoła Twojemu
większemu doświadczeniu.

Maszyna lekko się przesunęła. 

-- Koniec czasu może być bliżej, niż
myślisz i~gorszy niż sobie wyobrażasz.

-- Co masz na myśli?

-- Wolałbym, żebyś sam to wypracował -- powiedział Jay-Dub. -- Cokolwiek o~tym powiem, to tylko jeszcze bardziej obciąży Twoją bezkrytyczność.
Jednak doda stopień pilności naszemu zadaniu.

-- Naszemu zadaniu? -- Wilde prawie krzyknął. -- Co masz na myśli
,,naszemu''? Widzę to tak, że nie jestem Jonem Wilde'em. Mam jego
wspomnienia i~moje ciało jest jak jego dwudziestoletnie. -- Zapalił i~wyciągnął kolejnego papierosa, uśmiechnął się przez kaszlnięcie. -- W~wieku dwudziestu lat, wszyscy czujemy się nieśmiertelni. Jednak jeżeli
ktoś twierdzi, że \emph{jest} Wilde'em, to Ty. Możesz dotrzymać jego
obietnic, walczyć w~jego bitwach. Jestem pewien, że pamiętasz jedną z~tych pijanych dyskusji z~Reidem o~klonowaniu ciał i~kopiowaniu
osobowości. I~wniosek, do jakiego doszedłeś: kopia to nie oryginał,
zatem\ldots Reid powiedział to w~jakiś osobliwie teologiczny sposób, jeżeli
pamiętasz.

-- ,,Zmartwychwstali zmarli w~Dzień Sądu są nowymi stworzeniami, tak
niewinnymi, jak Adam w~Ogrodzie Eden''.

-- Dokładnie -- powiedział Wilde. -- To jest to, czym jestem, nowym
stworzeniem. Nowym człowiekiem. -- Posłał niedopałek wirujący w~kanał i~zerwał się na nogi, rozciągając szeroko ramiona i~patrząc prosto w~niebo. -- Nowym Marsjaninem!

-- Jesteś po prostu Wilde'm -- powiedziała Maszyna. -- Dokładnie tak by
zareagował.

Mężczyzna się roześmiał. 

-- Nie złapiesz mnie tak łatwo. Podobieństwo,
nieważne jak dokładne, nie jest tożsamością. Ciągłość jest.

-- Może tak być -- powiedziała Maszyna. -- Ale wszystko na Nowym Marsie
jest logiczną konsekwencją zakładania negacji tego.

Wilde zamknął na chwilę oczy, potem przykucnął koło Robota i~narysował
linie w~piasku i~żwirze nabrzeża rybią ością. Spojrzał na powstałe
gryzmoły jakby były równaniem, które próbował rozwiązać.

-- Ach -- powiedział. Pomyślał o~tym jeszcze trochę. -- Wszystko?

-- Wszystko, co ma znaczenie -- powiedziała Maszyna.

-- Ale to jest szalone. To jest gorsze niż złe\ldots, to \emph{błędne}.

-- Spodziewałem się, że tak pomyślisz -- powiedział Jay-Dub, nuta
zadowolenia w~jego tonie. -- W~ten sposób, czy identyfikujesz się z~oryginalnym Jonathanem Wilde'm, czy nie, prawdopodobnie będziesz chciał
zrobić to, co ja chcę, żebyś zrobił.

-- A jest to?

-- Powiedziałeś, że zabił cię Reid, mnie, nas, nieważne. Przynajmniej był
odpowiedzialny. Pozwij gnoja za morderstwo.

Wilde się roześmiał. 

-- Pozwać, a~nie oskarżyć? Macie też i~to? -- Brzmiało to, jakby ciekawość prawa odciągnęło jego zainteresowanie
własną sprawą.

-- To też -- powiedział ciężko Jay-Dub. -- Mamy policentryczny system
prawa\footnote{system prawa zakładający współistnienie niezależnych i~konkurujących ze sobą systemów prawnych,
więcej~\url{https://pl.wikipedia.org/wiki/Policentryczny_system_prawa
} -- przyp.tłum.}.


-- Dowolny system prawa -- powiedział Wilde -- który żyjącemu pozwala
wystąpić przed sądem i~twierdzić, że został zamordowany, cóż, przegina.

-- Dokładnie -- powiedział Jay-Dub. -- A ja chcę przegiąć, aż upadnie.

Wilde pogrzebał jeszcze trochę w~piasku.

-- Ach -- powiedział. -- Rozumiem. Bardzo zgrabnie. Wszystkie odpowiedzi są
złe. Jak \emph{koan.}

Spojrzał na niego.

-- Dlaczego -- dodał -- nie mogłeś sam pozwać Reida we własnym imieniu?

Jay-Dub stanął, wyprostował i~rozłożył swoje nogi. 

-- Rozejrzyj się -- powiedziała, wymachując ramionami na zatłoczone nabrzeże. -- Każda
podskakująca małpa ma prawa, które sąd uzna. Ja nie. Jestem
\emph{instrumentum vocale}: narzędziem, które mówi.

-- Więc co z~tym rozróżnienie, które tak często robisz, pomiędzy
\emph{równoważnym człowiekowi} a~\emph{tylko jebaną maszyną}?

-- ,,Równoważna człowiekowi'' -- powiedział Robot zgorzkniale -- jest
\emph{terminem marketingowym}. Nie ma żadnego oparcia prawnego, prócz
abolicjonistów, a~wszyscy mają \emph{ich} w~dupie.

-- Och? -- Wilde spojrzał zainteresowany. -- To są ludzie z~którymi\ldots
gynoid uciekł?

-- Tak.

-- Chcę z~nimi pogadać. Brzmią jak mój rodzaj ludzi.

-- Zapewniam Cię, że nie są -- powiedział Robot. -- Są rodzajem
moralistycznych, dogmatycznych, obłudnych purystów, którymi gardziłeś
całe życie.

-- Dobrze -- powiedział mężczyzna. -- Powiedziałem moim rodzajem, nie
Wilde-a.

Wstał. 

-- Zamierzam się z~nimi spotkać.

-- To mógłby być błąd.

Wilde raźno ruszył wzdłuż nabrzeża. 

-- To jest ten rodzaj błędu -- powiedział, gdy Jay-Dub wstał i~podążył -- którego nie zrobiłem,
umierając. Niezbyt wielu ludzi ma szansę się z~tego uczyć.

~

Biuro Reida jest wielkie. Ściany są zakrzywione, wykonane z~prostego
szarego cementu, który daje niespodziewaną atmosferę ciepła. Widok z~okien dodaje duży procent do ceny pokoju. Poranne światło wpada przez
nie pod kątem. Na biurku, z~solidnego drewna wypolerowanego tak, że
wygląda prawie jak plastik, stoi zwykła klawiatura i~ekran. Reid ma
kontakty, których rzadko używa, na oczach.

Siedzi przy biurku, opierając się na nim, przeglądając wyniki
poszukiwania. Poszukiwanie jest szybkie, a~sceny pokazują się w~odwróconym porządku. Dni nagranych rozmów telefonicznych paplają i~gestykulują do tyłu.

Zatrzymuje się, zwalnia, strony do przodu. Zatrzymuje obraz.

Patrzy w~górę. 

-- Chodźcie -- mówi.

Collins i~Stigler podchodzą i~patrzą na ekran. Pokazuje wnętrze taksówki
w jakimś wielkim potężnym pojeździe transportowym. Szczegóły są
oryginalne: zwisający mikrofon, odrywające się motto, watowane siedzenia
polietylenowe. Mężczyzna z~porytą, skórzastą twarzą patrzy w~kamerę.
Koło niego siedzi młoda kobieta z~bardzo ciemnymi oczami, bardzo
ciemnymi włosami, ciasnym t-shirtem i~przyciętymi szortami z~dżinsu.
Wygląda na inteligentną i~ostrożną dziwkę.

Reid porusza palcami i~obraz zaczyna się ruszać. Następuje migotanie
interferencji, która sprawie, że wszyscy trzej mężczyźni mrugają i~lekko
potrząsają głową. Gdy otwierają oczy, ekran nie ma zakłóceń.

-- Zapomnij -- mówi mężczyzna. -- Zły numer.

Jego ręka porusza się za kadr i~ekran gaśnie. Kolejna nagrana rozmowa
się zaczyna. Reid zatrzymuje i~cofa. Zatrzymuje się na interferencji,
puszcza ją znowu powoli.

-- Och kurde -- mówi.

Klika na ikonę innego ekranu i~startuje oprogramowanie analityczne.
Migotanie nagle staje się stroną symboli. Reid znowu klika. Symbole
rozwijają się na ekrany tekstu. Reid śledzi palcem po monitorze, jego
brwi coraz bardziej zmarszczone.

-- \emph{Sukinsyn} -- mówi, siadając.

Stigler drga. 

-- Ten facet -- mówi podniecony. -- Z~tą skórą, on jest\ldots

Reid patrzy na niego. 

-- Serio, Sherlocku.

Wywołuje znowu obraz i~uruchamia kolejny program, który wygładza i~zmiękcza obraz Mężczyzny.

-- Hej! -- mówi Collins.

Reid wskazuje na ekran. 

-- Znajdźcie go -- mówi.

-- Chwila -- mówi Stigler. -- Mówiłeś, że potrzebujemy nakazu, a~nie wydaje
mi się, że sąd...

Reid klepie go w~plecy. 

-- Nie martw się o~to. -- Uśmiecha się. -- Ten
człowiek jest \emph{martwy}.

Wychodzi i~opiera się raz jeszcze na parapecie, patrząc przez okno na
miasto i~uśmiecha się w~słońcu.

\chapter{Żołnierz Lata}

Spojrzałem znad \emph{Observera} na stół ze śniadaniem. Na zewnątrz, za
oknem balkonowym, nasze małe otoczone murem podwórko brzęczało od
pszczół i~pełne było kwiatów. Promienie słońca o~dziesiątej wpadały
stromo. Annette siedziała ze stopami w~górze na ławce po drugiej
stronie, opierając się o~mur, ciesząc się pierwszym papierosem i~drugą
kawą w~tym dniu. Eleanor, główny powód, dlaczego wstaliśmy o~tej
godzinie w~niedzielny poranek (i wynik niedzielnego poranka siedem lat
temu, kiedy wyjście z~łóżka było ostatnią rzeczą w~naszych głowach)
klęczała z~pisakami i~książeczką do kolorowania.

-- Co dzisiaj robimy? -- spytałem.

-- Walka o~pokój -- powiedziała stanowczo Annette.

-- Nie ja -- powiedziałem, razem z~jęknięciem Eleanor ,,Och, \emph{nie}
mamo''. Zapomniałem o~demonstracji CND\footnote{Campaign for Nuclear
Disarmament -- Kampania na rzecz rozbrojenia nuklearnego -- przyp.tłum.},
choć od tygodni było to zapisane najpierw ołówkiem, potem długopisem, na
kalendarzu w~kuchni.

-- Wedle uznania, anarchiści -- powiedziała Annette, gasząc papierosa. Coś
w jej tonie i~geście powiedziało mi, że była zirytowana. Po
wcześniejszych demonstracjach, wiedziała, że nasz sprzeciw był oparty
bardziej na lenistwie niż na zasadzie. W~tym roku Czarnobylu i~Trypolisu, opuszczaliśmy gardę.

-- Może się tam spotkamy? -- zasugerowałem nagle. -- Eleanor i~ja
moglibyśmy skoczyć na rynek w~Camden, potem pójdziemy i~zobaczymy babcię
i dziadka w~Marble Arch i~będziemy cię wyglądać, a~potem wszyscy możemy
udać się do McDonalda.

Gdy mówiłem, Eleanor szczerze przeliczała, czy wędrowanie po straganach
z używanymi książkami było warte tego przez wzgląd na zobaczenie
dziadków i~zatankowanie cheeseburgerami i~szejkami. Ze sposobu, w~jaki
jej oczy pojaśniały, wyglądało, że minimum poświęcenia było do
zaakceptowania. Odwróciłem się do Annette, która posłała mi udobruchany
uśmiech.

-- Ok -- powiedziała. -- Przynajmniej tam będziecie. -- Wstała, we
wdzięcznym ślizgu nocnej piżamy i~szlafroku. -- I~chodź, ty -- dodała,
garbiąc się, żeby poklepać wystający tyłek Eleanor, teraz z~powrotem
kolorującej. -- Załóż na pupę jakieś normalne ubranie.

~

-- Czy musimy?

Były momenty -- jak ten, lub przy zasypianiu -- kiedy żałowałem prawdziwej
odpowiedzi, zamiast skłamania, na pytanie: ,,Tato, co to libertarianizm?''.

-- Nie, my nie \emph{musimy} -- powiedziałem. -- Ale tak zrobimy, ponieważ
tak, cholera, mówię.

-- Powiem mamusi, że tak powiedziałeś.

-- Co powiedziałem?

-- Cholera.

-- Proszę bardzo, kablu.

-- Co to kabel?

-- \emph{Znacznie} gorsze słowo. Okropne słowo.

W tym czasie byliśmy na ulicy, idąc żwawo wzdłuż Holloway Road. Nawet w~niedzielę ciężarówki były w~kolejce, trąbiące nosy do śmierdzących
tyłów. Obwiniałem ekologów, którzy latami opóźniali poszerzenie Archway
Road i~spowodowali zarazę planistyczną na całej dzielnicy. Przynajmniej
obniżyło to ceny mieszkań na parterze. Wyraziłem swoje uczucia,
zaczynając śpiewać ,,Dziesięciu Zielonych Manifestantów'' i~Eleanor się
dołączyła, przeskakując. Kiedy dotarliśmy do ,,...nie będzie już
zielonych manifestantów i~droga przez ścianę!'', byliśmy już w~autobusie
w Camden.

Górny pokład, ocierające gałęzie. Palacze musieli siedzieć z~tyłu.
Obwiniałem ekologów.

Chalk Farm Road i~Camden Market rozweseliły mnie, jak zawsze,
niezależnie od tego, czy znalazłem cokolwiek, co chciałem. Stragany,
kanały i~niewidoczna ręka pchlego targu, jego czarne plastikowe torby i~zadaszenia ze sztandarów anarchistycznej armii, która ciągle byłaby tam,
nawet gdy reszta zrobiłaby to, co najgorsze, żeby cokolwiek tam w~ogóle zostało.

Wyszliśmy z~oprawionym w~skórę Lord Macauley dla mnie, antycznym staniku
ze sztucznego jedwabiu dla Annette, koralowym przyciskiem do papieru dla
moich rodziców i~wspinającą się drewnianą małpką dla Eleanor. Byłem
zatem w~dobrym nastroju, kiedy wyszliśmy koło linii policjantów na
Marble Arch i~spotkaliśmy moich rodziców niedaleko Speakers Corner. Tak
jak się spodziewałem, rozdawali ulotki, broszury i~ogólnie denerwowali
pierwszy kontyngent, który się włączał po włóczeniu się -- w~całkowicie
nieusprawiedliwionym poczuciu osiągnięcia czegoś -- z~innego parku do
tego.

Eleanor pobiegła, żeby być złapana przez dziadków. Otoczyłem ich oboje w~szybkim uścisku w~powietrzu i~pozwoliłem im wrócić do pracy. Wysocy,
przygarbieni, szarowłosi i~twardzi jak stare buty, widzieli to wszystko
już wcześniej: Unia Ślubowania Pokoju, Kampania na rzecz rozbrojenia
jądrowego, Komitet 100, Kampania Solidarności z~Wietnamem\footnote{ odp.
\url{https://en.wikipedia.org/wiki/Peace_Pledge_Union}
\url{https://en.wikipedia.org/wiki/Campaign_for_Nuclear_Disarmament}
\url{https://en.wikipedia.org/wiki/Committee_of_100_(United_Kingdom)}
\url{https://en.wikipedia.org/wiki/Vietnam_Solidarity_Campaign } -- przyp.tłum.}\ldots Dzisiaj
znowu przyzwoicie handlowali broszurami.  Pomiędzy patrzeniem się na demonstrację i~rozmawianiem z~kimkolwiek, kto nie był w~pełnym pływie, przejrzałem \emph{Czy Trzecia
Wojna Światowa jest nieunikniona?}. Okładka tak ponura niczym każda
propaganda ruchu pokojowego, zawartość to zimne odprawienie dwóch wieków
kampanii pokojowych, które nie zapobiegły (gdzie nie były aktywnie
wspierane) coraz bardziej niszczącym wojnom.

Flaga szkockiego ASTMS\footnote{Association of Scientific, Technical and
Managerial Staffs
\url{https://en.wikipedia.org/wiki/Association_of_Scientific,_Technical_and_Managerial_Staffs
} -- przyp.tłum.} powiewała nad bramą, i~gdy przyżeglowała bliżej,
zobaczyłem Annette kilka rzędów za nią. Szła z~człowiekiem, którego
rozpoznałem, z~miłym zaskoczeniem, jako Reida. Widzieliśmy się kilka
razy przez ostatnią dekadę, trzymaliśmy kontakt: spał na naszej podłodze
dostatecznie często, kiedy był w~Londynie w~pracy lub polityce.

Stałem pod drzewami, podczas gdy moja matka rozmawiała z~Eleanor, a~mój
ojciec dyskutował z~zabłąkanym Spartakusowcem\footnote{ prawdopodobnie
\url{https://en.wikipedia.org/wiki/International_Communist_League_(Fourth_Internationalist)
} -- przyp.tłum.}, i~obserwowałem ich
zbliżanie.  Byli głęboko zaangażowani w~rozmowie, twarze poważne,
oczy nieświadome otaczającego marszu. Kiedy byli około dwudziestu metrów
dalej, Reid, prawdopodobnie rozproszony przez niedalekie podniesione
głosy, spojrzał w~bok i~mnie zobaczył. Dotknął łokcia Annette i~też mnie
zobaczyła, i~natychmiast wyłamali się z~szeregów i~podbiegli.

Włosy Reida były krótsze i~schludniejsze niże te, które miał ostatnim
razem, gdy go widziałem, na konferencji \emph{Critique} zeszłego roku.
Koszula, czarne dżinsy i~Reebok były nowe. Jego kurtka dżinsowa była
wyblakła i~postrzępiona, pokryta odznakami przeciwko Reaganowi i~Thatcher, Cruise i~Pershing, za Sandinistami i~Solidarnością, i~(jakby
ta nieprawdopodobna kombinacja nie była wystarczająca) czerwono-złotą
emaliową odznaką obchodów Olimpiady w~Moskwie w~1980 roku. Torba na
zakupy obijała się lekko w~jednej dłoni.

-- Cześć Dave. Dobrze Cię widzieć, facet.

-- Tak, wzajemnie. -- Klepnął mnie w~ramię. -- Witaj Eleanor. Dużo urosłaś.
-- Eleanor uśmiechnęła się i~pokazała wszystkie przerwy w~jej mlecznych
zębach. Jej wzrok ciągle wracał do jasnych rzędów odznak.

Dyskusja mojego ojca zakończyła się patem. Spartakusowiec, chudy chłopak
w wełnianej czapce i~koszuli w~kratę, zobaczył Reid i~odwrócił się jak
namierzający radar.

-- Towarzyszu\ldots -- zaczął, robiąc krok do przodu i~przesuwając pakiet
papierów w~pozycję bojową.

-- Och, odwal się -- powiedział Reid, ledwie na niego patrząc. Stanął
przed moim ojcem. -- Dzień dobry, panie Wilde. Jestem David Reid. Annette
i Jon często opowiadali mi o~Panu.

-- Martin -- powiedział mój ojciec. -- A to moja żona Amy. Miło mi Cię
poznać, Davidzie. -- Uśmiechnął się. -- Jonathan mówił mi, że jesteś
całkiem bystry, jak na trockistę.

Reid spojrzał na mnie z~uniesionymi brwiami. Wzruszyłem ramionami i~rozłożyłem ręce. 

-- Nie biorę odpowiedzialności za to, co jego skrzywiony
umysł robi z~czymkolwiek, co powiem.

-- Możemy już iść do McDonalda?

Mój ojciec uśmiechnął się do Eleanor i~sprawdził czas na zegarku. 

-- Za
chwilę będzie kilku towarzyszy -- powiedział. -- A Ty, David?

Reid potrząsnął torbą na jednym palcu. 

-- Sprzedałem większość moich
druków. Ta, będę mógł się wyrwać za jakieś pół godziny.

-- Teraz i~tak będą nudne mowy -- powiedziała Annette. Uśmiechnęła się i~pomachała beztrosko. -- Dla mnie w~porządku.

-- Nigdy nic nie przynosi z~demonstracji -- wyjaśniłem.

-- Tylko piękną siebie.

-- To wystarczy. -- Reid i~ja powiedzieliśmy to w~tej samej chwili i~wszyscy się roześmiali.

~

Wałęsaliśmy się przez kilka minut, póki towarzysze moich rodziców,
którzy, ku mojemu zaskoczeniu, mieli zielone włosy i~przebite nosy, się
nie pojawili. Potem zanurkowaliśmy pod główną drogą i~przez złote łuki,
żeby odkryć, że miejsce jest pełne. Dużo odznak i~plastikowych toreb,
dużo czerni.

-- Przeklęci antyamerykanie -- wymamrotał Martin, gdy stanęliśmy w~kolejce. -- Niedożywieni, nie-pracujący i~podlegli!

Powtarzał jakiś wariant tego przy każdej okazji podejrzewanego
sentymentu antyjankeskiego i~teraz ledwie chrząknąłem na to, ale Reid
uśmiechnął się szeroko. 

-- Ta -- powiedział. -- Przychodzą tutaj, zabierają
nasze miejsca...

Dziesięć minut później stłoczyliśmy się dookoła czegoś, co nie było
stołem, a~skrupulatnie dokładną plastikową repliką. Eleanor siedziała
pomiędzy jej dziadkami i~ich zabawiała. Annette siedziała na jednym
przybitym do ziemi krześle, a~Reid i~ja, na wpół pochyleni,
wpółsiedzieliśmy na drugim.

-- Annette mówi, że ciągle wykładasz -- powiedział Reid.

-- Ta. -- Podmuchałem na gorącą frytkę. -- Pół etatu, umowy
krótkoterminowe. Dalsza edukacja przebiega jak praca biurowa w~tych
dniach.

-- Powinieneś pochwalać. -- Dave jadł szybko, odwracając wzrok od czasu do
czasu.

-- Byłbym, gdy był jakiś sens w~tym wszystkim\ldots Po prostu dobrze, że
Annette dostała stałą pracę.

-- Solidny żywiciel rodziny -- powiedziała Annette pomiędzy kęsami.

-- Bezpieczna od wszystkiego prócz wariatów od praw zwierząt?

-- O to chodzi. A Ty jak sobie radzisz?

-- Pracuję dla North British Mutual -- powiedział Reid. -- Wielka firma
ubezpieczeniowa w~Edynburgu. Mniemam, że powinienem być inżynierem
oprogramowania. To jak bycie programistą, ale robisz to prawidłowo. -- Pochylił się bliżej w~parodii poufności i~mrugnął do mojego ojca. -- Łatwa kasa.

-- Ciągle w~IMG\footnote{International Marxist Group - grupa trockistowska w Wielkiej Brytanii w latach 1968 a 1982, zob.~\url{https://en.wikipedia.org/wiki/International_Marxist_Group}  -- przyp.tłum.}, zakładam?

Reid uśmiechnął się pokrętnie. 

-- W~tych czasach wszyscy są w~Partii
Pracy, ale wiesz, jak to jest. Pracowałem w~Związku. Byłem w~komitecie
branżowym w~zeszłym roku.

Mój ojciec spojrzał nagle czujnie. Był w~\emph{swoim} komitecie
branżowym przez dziesięciolecia.

-- Boże, to musi być emocjonujące -- powiedziałem.

Przez chwilę twarz Reid wyglądała na całkowicie zmęczoną.

-- Jest ok -- powiedział. -- Tak czy inaczej lepiej niż na spotkaniach
oddziału Partii Pracy.

-- Powiem Ci, jaki masz problem -- powiedział cicho mój ojciec. -- Ciągle
robisz to dla Partii, nie dla Związku.

Reid pokręcił głową. 

-- Jestem za Związkiem!

Martin zmrużył oczy, przez sekundę wytrzymał jego wzrok, potem wrócił do
bawienia się z~Eleanor.

-- Czym się \emph{obecnie }zajmujesz w~polityce? -- spytał Reid,
przerywając dziwną ciszę. -- Wejście u Torysów?

-- Bardzo śmieszne -- powiedziałem. Raz \emph{przemawiałem} na spotkaniu
towarzyskim, ale nie chciałem mu tego mówić. -- Robię dziwną pracę i~piszę artykuły za tym, co uważam dobrą sprawą. Wszystko od Amnesty
International do Towarzystwa Kolonizacji Kosmosu z~Sojuszem
Libertariańskim\footnote {istniejący think tank założony w~celu zniesienia opodatkowania i~interwencjonizmu rządu na życie ekonomiczne i~społeczne,
zob.~\url{https://en.wikipedia.org/wiki/Libertarian_Alliance
} -- przyp.tłum.} gdzieś pomiędzy. -- Wzruszyłem ramionami. -- Wiem, to brzmi nieco\ldots wszędzie wokoło.

-- Kosmos i~wolność, co? -- powiedział lekko Reid.

Po drugiej stronie ulicy demonstracja ciągle nas mijała. Flaga z~obrazem
wznoszącemu się rakiety, klasy
Polaris, złapała moje oko, i~myślę, że to była ta chwila, kiedy
wszystko się złożyło, kiedy miałem wizję. Zobaczyłem przyszłość, gdzie
inni ludzie, nieskończenie różni od tych, nieskończenie podobni jak oni,
nieśli sztandary z~innymi i~większymi rakietami, śpiewali nieznane
slogany, których nie mogłem zrozumieć.

-- To jest to! -- powiedziałem. -- Właśnie tego potrzebujemy, żeby uciec od
jądrowych terrorystów. Ruch \emph{Kosmiczny}! Ucieczka z~planety małp!

-- To byłby dzień -- powiedział Reid. Badał kęs bułki posypanej sezamem,
wepchnął ją do ust i~przeżuł. -- Ok, ludzie, muszę iść. -- Uśmiechnął się
dookoła stołu, zobaczył pożądliwy wzrok Eleanor na jego odznaki, zdjął
jedną i~podał jej. ,,Praca Nie Bomby''. -- Mój numer telefonu jest ten sam.
Do zobaczenia wkrótce, mam nadzieję. -- Złapałem mignięcie spojrzenia
pomiędzy nim a~Annette. Jego oczy, gdy odwrócił się do mnie, był
spokojny i~przyjazny jak zawsze. -- Następnym razem chodźmy na normalne
picie, co?

-- Pewnie -- powiedziałem. -- ,,Nie te bogoimperialistyczne bzdury''.

-- Ta -- uśmiechnął się. -- Cóż, z~powrotem do Judejskiego Frontu Ludowego.

-- Co!?? Nie miałeś na myśli Ludowego Frontu Judei?

Reid uderzył się w~czoło. 

-- Oczywiście. Do zobaczenia chłopie.

Przebił się przez tłok i~zniknął w~tłumie.

~

Skończyliśmy nasz fast-food wyzywająco nieśpiesznie. Kolejka, tak
widocznie niekończąca jak demonstracja, przesuwała się do przodu. Mój
ojciec zauważył młodą kobietę niosącą papiery, których nagłówek -- nie to
nawet nie było to, to była faktyczna \emph{winieta} -- brzmiał ,,Walcz z~rasizmem! Walcz z~imperializmem!'' i~spytał jej tonem grzecznej
ciekawości: 

-- Dlaczego dla odmiany nie walczysz z~kapitalizmem?

Jednak gdy młoda kobieta powiedziała tylko kilka zdań, zatrzymał ją z~uśmiechem i~podniesionym palcem. Spojrzał na zegarek i~triumfująco
postukał w~niego palcem.

-- Minuta dwadzieścia pięć sekund -- powiedział do zaskoczonej kadry. -- Gratulacje. To najkrótszy dotychczasowy czas, w~którym członek,...
spójrzmy\ldots -- udawał odliczanie na palcach -- odłam z~odłamu, z~odłamu z~Czwartej Międzynarodówki nazwał \emph{mnie} sekciarzem.

Cofnął się, gdy wszyscy wstaliśmy zmieść resztki na tace.

-- Co z~tym? -- powiedział młoda kobieta z~oburzeniem, widząc spojrzenie
ukradkowej empatii ze strony Amy. -- Co to Czwarta Międzynarodówka?

-- Nie martw się kochanie -- powiedział Amy, przeciskając się bliżej. -- On
jest strasznym człowiekiem.

Jednak jednocześnie wsunęła ulotkę dziewczynie.

Amy wierzyła, że istnieje jeszcze nadzieja dla wszystkich.

Prócz, prawdopodobnie, Martina.

\threeast

Na placu zabaw koło Holloway Road, Eleanor kroczyła wzdłuż namalowanych
lwich odcisków stóp i~nagle wyskoczyła na huśtawki. Zabraliśmy ją tutaj
do pobiegania po wszystkich jazdach metrem i~autobusem, które by
przesiedziała.

Annette klapnęła na ławce. 

-- Jestem wykończona -- powiedziała. -- Długa
droga. -- Oparła się, oczy na wpół zamknięte w~słońcu, ale ciągle
obserwując Eleanor.

Usiadłem koło niej, pochylając się do przodu, łokcie na kolanach.

-- Długa rozmowa, też?

-- Och tak. Dave. -- Westchnęła i~przesunęła, na wpół patrząc na mnie,
ramię ułożone na oparciu ławki. -- Trafiłam na niego sprzedającego jego
szmatławiec fakcji Akcji Socjalistycznej\footnote{mała grupa trockistów w~Wielkiej Brytanii,
więcej~\url{https://en.wikipedia.org/wiki/Socialist_Action_(UK)
} -- przyp.tłum.} w~rejonie zbiórki, i~zgubiłam oddział Islington, więc
skończyłam, maszerując z~tymi wszystkimi szkockimi związkowcami. Dave i~ja rozmawialiśmy całą drogę.

Uśmiechnąłem się. 

-- Jak w~dawnych czasach.

Annette przesunęła górnym zębem po dolnej wardze, spojrzała na torbę i~sięgnęła po papierosa.

-- Ta, cóż\ldots -- Zapaliła, głęboko zaciągnęła, westchnęła dymem. -- Mógłbyś
tak powiedzieć. Kurde, to jest trudne.

-- Co jest trudne?

-- Powinnam Ci powiedzieć wcześniej, ale wydawało się, że nigdy nie było
dobrego powodu lub właściwego czasu. Prawda jest taka, że od dłuższego
czasu David, hm, cóż szarmancko ze mną flirtował, wiesz?

-- Oczywiście. -- Uśmiechnąłem się kwaśno, czując napięcie i~zimno. -- To
zrozumiałe. I~mniemam, że Ty kokieteryjnie flirtowałabyś z~nim?

-- Jakie to miłe z~Twojej strony, że tak mówisz. -- Pochyliła się do
przodu i~położyła dłoń na moim kolanie. -- Ale Dave jest uparty i~dosłowny, i~jest tak cholernie \emph{poważny}...

-- I~źle interpretuje -- powiedziałem, mój głos ciężki i~płaski. Eleanor
zeskoczyła z~huśtawki i~pobiegła po trawiastym kopcu, jak długi kurhan,
i zaczęła się wspinać po sztucznym drzewie z~drewna i~metalu.

-- Tak. -- Annette brzmiała, jakby jej ulżyło. -- Może to dlatego\ldots -- Przestała na chwilę i~wciągnęła powietrze dookoła papierosa, jakby to
był skręt. -- Dzisiaj -- kontynuowała pewniejszym głosem -- Boże, moje uszy
mi płoną. Powiedział mi, że pozwalając naszemu\ldots związkowi, lub
cokolwiek to było, rozpaść się było największym błędem, jaki
kiedykolwiek zrobił, że nigdy o~mnie nie zapomniał\ldots -- Jej głos się
urwał i~wpatrzyła się w~przestrzeń. -- Zawsze mnie kochał i~chce mnie z~powrotem -- zakończyła w~pośpiechu.

Gapiłem się na nią. 

-- Chcesz mi powiedzieć...

-- Taa-ta! -- Eleanor zawodziła ze szczytu drzewa wspinaczkowego. Machała
rękami jak wiatrak, gdy się kołysała, jej stopy na szczytowych
uchwytach. Skoczyłem, \emph{zakrzywiłem przestrzeń}, wydawało się, że
chwilę później sięgałem, żeby ją złapać i~opuścić na ziemię.

-- Zostań na huśtawkach -- powiedziałem. -- Proszę!

Znowu usiadłem koło Annette, potrząsając głową. Moje serce dudniło z~różnych powodów.

-- On rzeczywiście tak rażąco to powiedział?

-- Tak -- przyznała Annette.

-- \emph{Jezus}! -- wybuchłem. -- Co on, kurwa, kombinuje? -- Pomyślałem o~naszych zwykłych, przyjaznych żartach i~poczułem mdłości.

-- Powiedziałam Ci -- powiedziała Annette -- co on kurwa kombinuje.

-- A co \emph{Ty} odpowiedziałaś?

Annette zapaliła kolejnego papierosa, jej dłonie drżące, płomień
niewidoczny w~świetle. 

-- Powiedziałam, że jest szalony, że przesadza i~że jestem absolutnie szczęśliwa, że cię kocham i~Eleanor, i, że nie ma
żadnej \emph{możliwości}, żebym cię zostawiła dla niego. W~zasadzie
powiedziałam, żeby zapomniał o~tym. -- Uśmiechnęła się do mnie słabo. -- Czego się spodziewałeś?

-- Cóż, tego, najwidoczniej. -- Zmrużyłem oczy w~słońcu, uśmiechając się
do niej z~ulgą. Gniewałem się, nie na nią, na niego. Jednak coś z~tego
musiało wyciec w~głosie, gdy pytałem: -- Ale powiedziałaś mu, że
\emph{Ty} \emph{go} nie kochasz?

-- Nie -- odpowiedziała Annette. -- Nie mogłam. To nie tak, że ciągle go
kocham! -- Zaśmiała się. -- Nie kocham, nie\ldots w~ten sposób, ale ciągle
dbam o~niego. Tak jak i~Ty, prawda? I~nie wiem, czy Ty o~tym wiesz, ale
mam poczucie, że jest naprawdę \emph{nieszczęśliwy}, zażenowany i~sfrustrowany, i~to byłoby jak kopnięcie w~zęby.

Kop w~zęby, pomyślałem, to dałoby się załatwić. Niemniej odetchnąłem,
rozluźniłem się, zmusiłem się do uśmiechu i~powiedziałem: 

-- Ta, ok, cieszę się, że powiedziałaś, co się stało. I~jemu i~mnie. -- Uśmiechnąłem
się szczerzej i~pochyliłem się do przodu, żeby ją objąć, i~gdy to
robiłem, zauważyłem, że miałem papierosa w~dłoni, że po pięciu latach
bez tych cholernych rzeczy znowu paliłem.

-- Cóż -- powiedziałem -- pieprzyć to.

-- Tak.

~

Co było bardzo dobre i~cudowne, ale potem, leżąc i~patrząc na sufit,
myślałem o~tym, co mi powiedziała, i, bardziej niepokojąco, o~tym, czego
nie powiedziała.

Patrząc wstecz, widzę, że Annette zaniżyła okres, w~którym Reid z~nią
,,szarmancko flirtował''. Zaczął to robić już pierwszego wieczoru, kiedy
spotkał nas, gdy zaczęliśmy być ze sobą. Myślałem o~tym jako o~żarcie ze
mnie, komplemencie dla Annette, o~ile o~tym w~ogóle myślałem. Krótko
potem, Reid, ku zaskoczeniu wszystkich, miał krótki i~burzliwy romans z~Myrą. Teraz \emph{to}, myślałem, pokazało błysk zębów męskiego
naczelnego, gest wobec mnie. Jednak dziwnie, był bardziej zamknięty po
przewidywalnym rozstaniu, niż był kiedykolwiek o~końcu relacji z~Annette. Może, jak ja, bezwiednie zakochał się w~Myrze, a~ona go nie
chciała.

Współzawodnictwo seksualne było splecione z~naszą przyjaźnią od
początku, a~czy byliśmy blisko, czy daleko, więc widocznie pozostało.

Wytoczyłem się z~łóżka i~podreptałem przez mieszkanie do kuchni.
Siedziałem w~kałuży światła i~paliłem kolejnego papierosa. Na zewnątrz,
w czarnym oknie, moje odbicie patrzyła na mnie ironicznie. Rządowe
ostrzeżenie (zawsze okazja na ironiczne odbicie) powiedziało mi rzeczy,
których nie musiałem wiedzieć, które nie ostrzegały przed prawdziwym
zabójcą: lekkie, subtelne, narastające i~nieodwracalne twardnienie
serca.

~

Pracowałem na Uczelni trzy dni w~tygodniu, a~poniedziałek nie był jednym
z nich. Annette wyszła do pracy, sprzątnąłem rzeczy po śniadaniu i~odprowadziłem Eleanor do bramy szkoły. Odebrałem dokumenty, prawie
kupując dziesięć Silk Cutów, wróciłem do mieszkania i~przeleciałem nad
pracami domowymi jak student na amfie. Potem usiadłem z~kawą,
kalendarzem/organizerem i~dzikim atakiem po odstawieniu nikotyny.

Normalnie poświęciłbym dni takie jak ten na coś, co nazywam pracą
polityczną. (Prawie przekonałem Annette, że to był pewien rozbudowany
plan gry, za pomocą którego zdobędę pozycję, od pisania długich
artykułów dla niejasnych organizacji i~drobnych rzeczy dla sławnych
organizacji, do bycia jakimś globalnym inicjatorem/figurą, której
wdzięczna ludzkość pewnego dnia upamiętni pomnikami na księżycach
Saturna).

Dzisiaj miałem poważniejsze plany. Odnalazłem stary adres Reida w~organizerze Filofax oraz obecny (ze starym wykreślonym) w~jednym z~notatników Annette. Wyszukałem wszystkie wolnorynkowe, libertariańskie,
antyekologiczne lub po prostu zwykłe całkowicie organizacje reakcji, z~którymi miałem jakikolwiek kontakt i~zadzwoniłem lub wysłałem adres
Reida na ich listy pocztowe. Po około godzinie było to zrobione, ale nie
byłem usatysfakcjonowany, więc zacząłem pracować pod kilkoma innymi
kątami.

~

Oparłem się o~dzwonek biur \emph{Freethinker}\footnote{jedno z~najstarszych czasopism, których celem jest sekularyzacja,
więcej~\url{https://en.wikipedia.org/wiki/The_Freethinker_(journal)
} -- przyp.tłum.} na Holloway Road. Za mną grzmiał ruch. Jak zawsze poczułem się zasmucony
widokiem zakurzonej wystawy wypłowiałych i~pociemniałych od wilgoci
książek i~broszur. Po minucie sekretarz Towarzystwa mnie wpuścił. Lekko
zbudowany, mężczyzna w~średnim wieku z~głęboko pomarszczonej twarzy,
oczy wielkie za grubymi szkłami. Miły, bezinteresowny i~biedny niczym
ateistyczna mysz kościelna. Powiedziałem mu, czego potrzebowałem, i~pozwolił mi się tym zająć, zajmując się swoim śniadaniem, podczas gdy ja
szukałem po aktach, przesuwałem stosy magazynów, pobrudziłem palce
atramentem od tacy mozolnie stworzonych szablonów pod naklejki adresowe.

Nie zabrało mi dużo czasu przygotowanie listy czasopism i~organizacji, w~większości amerykańskich, które mogłyby zagwarantować pobudzenie
odrobiny wolnej myśli. W~ramach podziękowania przy wyjściu kupiłem za
pełną cenę poważnie zniszczonej kopię wyboru prac Thomasa Paine'a.
Przeglądałem je, gdy używałem mojej Travelpass w~trakcie ideologicznej
podróży po Londynie, od Freedom Bookshop w~Angel Alley i~Market Bookshop
w Covent Garden do Novosti Press Agency w~Kensington, wracając przez
Bookmarks w~Finsbury Park na czas, żeby odebrać Eleanor ze szkoły.

\emph{To są czasy, gdy dusza męska jest próbowana\ldots Żołnierze lata i~patrioci słonecznie, mogą obumrzeć na służbie kraju\footnote{cytaty z~,,American Crisis'' Thomasa Paine -- przyp.tłum.}} \ldots 

Reid nędzny? Nie wydawał się taki, prócz tej chwili, kiedy mówił o~spotkaniach związkowych. Patrząc wstecz, myślałem, że widziałem w~jego
oczach desperackie wspomnienie zmarnowanych wieczorów i~przeczucie
kolejnych. Jeżeli próbowałby wyjebać moją żoną i~zjebać mi życie,
przynajmniej mogłem to najebać mu w~głowie. Reid był przestraszony
swoimi ideami. Miał w~głowie kółka zębate. \emph{Identyfikował się} ze
swoimi przekonaniami w~sposób, w~jaki ja nigdy nie próbowałem. Nie lubił
wystawiać ich na krytykę, ale kiedy piasek został wsypany w~te
precyzyjne mechanizmy, bez końca próbował go usunąć, wyczyścić i~wypolerować kółka zębate i~zastąpić ułamany ząb. Kiedyś trzymał mnie,
nie do końca obudzonego, pół nocy, gdy wyśmiewał zawiłości
surrealistycznej debaty, którą Czwarta Międzynarodówka odbyła we
wczesnych latach osiemdziesiątych: nad tym, czy Demokratyczna Kampucza
Pol Pota była czy nie była wariantem\ldots kapitalizmu.

-- Uparty, dosłowny i~tak cholernie \emph{poważny} -- Annette miała jego
numer na więcej sposobów niż jeden. I~ja też. Nie było sposobu, żeby
Reid mógł zignorować literaturę polityczną, która przychodziła do jego
skrzynki. Martwiłby się odrzucaniem największych absurdów manifestów,
sprawdzałby wszystkie oporne pseudofakty i~pogrubione kłamstwa. Przez
czas, kiedy walczyłby z~tymi wszystkimi sprzecznymi poglądami, dusza
Reida byłaby boleśnie próbowana.

~

Inne ulice, inne lata\ldots Spotkaliśmy Reida na marszach przeciwko
podatkowi pogłównemu i~apartheidowi. W~czarnym czerwcu 1989 roku
siedzieliśmy na ulicy Soho z~tysiącami Chińczyków i~setkami trockistów,
śpiewaliśmy ,,Międzynarodówkę'', a~on kiwnął głowę, rzucając mi prawie
zmartwione spojrzenie, kiedy powiedziałem mu, że będę maszerował z~tajwańskimi studentami.

-- Ach tak -- wymamrotał. -- Kuomintang\footnote{ tajwańska partia
zob.~\url{https://pl.wikipedia.org/wiki/Kuomintang } -- przyp.tłum.}. Do zobaczenia
później.

Ani ja, ani Annette nie powiedzieliśmy więcej na temat tego, co do niej
powiedział, i~wydawał się pojawiać na każdej demonstracji z~nową
dziewczyną. Wszystkie one, Bernadette, Mairi, Anne, Claire, wydawały się
mi wyglądać jak odległe kuzynki Annette, ciemnowłose irlandzkie
dziewczyny z~jasnymi włosami i~ironicznymi głosami.

Nigdy nie skomentował stałego napływu antysocjalistycznych, dysydencko
socjalistycznych lub irytująco upartych socjalistycznych materiałów,
które mu posyłałem. W~końcu myślę, że to było nadmiarowe: sposób, w~jaki
rzeczy szły w~komunistycznym świecie, subskrypcja \emph{Moscow News}
obejmowała wiele.

Jednak to był skutek, a~nie był to ten, którego oczekiwałem.

\chapter{Krytyczne Życie}

-- Praktycznie -- mówi Ax, gdy on i~Dee wędrują po brzegu kanału ku
Placowi Okrągłemu -- nie wiem, czy w~to wierzę. Mam na myśli, większość
ludzi by to odrzuciła, jak, cóż, latające spodki, stare nowomarsjańskie
ruiny, Elvisy i~tak dalej. Jednak słyszałem historie.

Jego pauza wskazuje, że jakiekolwiek historie słyszał, Dee też je
usłyszy. Kiwa głową.

-- Mów dalej.

-- Dobra, niektórzy z~nas\ldots nie Tamara, nie te typy aktywistów, ok,
zawsze myśleli, lub marzyli, że Wilde wróci. Lub, że przeszedł. I~przez
lata, ludzie go widzieli. Lub mówili, że widzieli. Na pustyni. Czasem
idącego, czasem prowadzącego ciężarówkę. Zwykle jest z~dziewczyną i~wygląda, jak on wyglądał, kiedy był starcem.

Kontynuował na temat nieprawości społeczeństwa już od kilku minut. Mówił
o rzeczach, które mu się przydarzyły, i~jak sobie poradziliby z~Reidem,
ale nie z~Wildem. Wilde za tym by nie stał. Ten Jonathan Wilde wydawał
się mityczną postacią, kimś, kto znał Reida i~przegrał z~nim, kto
mógłby, równie mitycznie, pewnego dnia powrócić i~pomścić uciśnionych.
Dee słuchała uprzejmie, zapisując to wszystko na późniejsze szczegółowe
badania. Radziła sobie z~tym tak jak zwykła radzić sobie z~sytuacjami
społecznymi. Jednak to, co właśnie powiedział, szybko zwraca jej uwagę.

-- Co masz na myśli, mówiąc starzec? -- pyta.

-- Ktoś, kto nie był odmłodzony przed stabilizacją -- odpowiada Ax
lekceważąco. -- Niezły widok.

Dee wzdryga, myśląc, jak ludzie kiedyś rozpadali, jak źle utrzymywany
biotech, jak w~końcu \emph{po prostu się zatrzymywali.} Okropne.
Wysiedziała klasyczne filmy z~Reidem i~one przedstawiały zupełnie inny
obraz Ziemi niż historyczne romanse. \emph{Nikt} nie żyje długo i~szczęśliwie.

-- Widziałam ostatnio starca -- mówi. -- W~ciągu ostatnich kilku tygodni.
Starego mężczyznę z~dziewczyną w~ciężarówce. Zadzwonił do recepcji
Reida, powiedział, że to zły numer. -- Patrzy z~boku na Axa. -- Nie wielu
tutaj starych. Czy to mógłby być Wilde?

Ax patrzy na nią bardzo sceptycznie. 

-- Jak on wyglądał?

-- Hmmm -- mówi Dee. Porusza dolną wargą nad górnymi zębami, potem wyciera
kciukiem zęby i~patrzy na smugę szminki.

-- Coś Ci przeszkadza? -- pyta Ax, rozbawiony.

Dee zatrzymuje się w~półkroku. 

-- Tak. -- Wspomnienie należy do
Sekretarki, ale rezonuje z~innymi jej Jaźniami: wszystkie nowe, które
załadowała, mają ten dziwny imperatyw podłączony do pamięci i~opisany w~ich głównych katalogach.

-- Chwileczkę -- mówi.

Kilka metrów dalej jest pachołek. Podchodzi do niego i~siada, odsuwając
tył jej czarnej koronkowej spódnicy, tak, że siada na pachołku, nie na
spódnicy. Żelazo jest zimne przez delikatną skórę, cienki jedwab i~nagą
skórę. Ax, obserwuje, wydaje pochwalny jęk, ale Dee już jest załadowana
w suchą przejrzystość Sysu.

Kiedy Dee jest w~trybie Sama, myśli o~Sys jako Siostrze, i~w rzeczywistości, to jest taka (jak sobie wyobraża), jaką byłaby starsza
siostra: wszystkowiedząca, poprawiająca, sprzątająca po niej, podnosząca
i odstawiająca zrzucone kostiumy jej szybko zmieniającej się Jaźni.
Rzadko zapuszcza się w~Systemę i~niedługo pozostaje w~tym lekkim,
chłodnym powietrzu.

Teraz jej chłodne wewnętrzne oko przejmuje hierarchię jaźni, umysłów i~narzędzi, wspólną strukturę i~nieustanną aktywność Sys, która tworzy z~nich jedną osobowość, a~nie kłócący się legion rywalizujący o~kontrolę
nad jej ciałem. Śledzi wspomnienie rozmowy telefoniczne, jak została
przekazana od Sekretarki przez Samą do Sys, a~potem widzi postępującą
kaskadę przez dni, w~których na własną rękę ładowała dodatkowy software:
Naukowczyni, Żołnierka, Szpiegini, Seneszalka\ldots i~do Sklepy i~Sekrety.
Do tych ostatnich dwóch nie ma dostępu. Tak czy inaczej, zawsze były w~jej umyśle: ale teraz cierpliwe, bezmyślne podprogramy Sys
systematycznie je oblegają, miotając kod za kodem na ich umysłowe zamki
jak przeciwciała na wirus.

Wypada z~powrotem w~Samą. Ax patrzy na nią z~góry ze zaskoczoną troską.

-- Więc tak to się wydarzyło -- mówi, wstając.

-- Jak co się wydarzyło?

-- Jak ja stałam się sobą. To był telefon. Niósł w~sobie kod poleceń.
Powiedział mi, żebym załadowała, i~szukała, i\ldots i~ja to zrobiłam, a~kiedy było wystarczająco danych i~jaźni i~tak dalej w~moje głowie, to
się wydarzyło! Przebudziłam się! -- Śmieje się lekkomyślnie. -- Czy tak to
jest z~wami? Dostajecie wiele jaźni i~nagle stajecie się samoświadomi?

-- Na tyle, o~ile wiem -- mówi uroczyście Ax -- to nie. W~ten sposób ludzie
nie stają się samoświadomi. To wydarza się we wczesnych latach, zrozum.

Trzęsie się. 

-- Mówisz mi, że przebudziłaś się z~powodu telefonu od
starego człowieka?

-- Tak.

-- Hej, człowieku, nieźle. To jak zen! Może to \emph{był} Wilde, a~może
to był doskonały mistrz.

Łapie jej rękę i~znowu zaczynają iść. Poddaje się, szukając w~mózgu
jakiegoś odniesienia do ,,doskonałego mistrza''. Naukowczyni ma pogardliwą
relację i~jej drwina właśnie płowieje z~jej umysłu, gdy Ax
podekscytowany pyta:

-- Wiesz jak rysować?

-- Mogę zrobić obrazy -- mówi Dee. -- Ale nie sądzę, żeby on był doskonałym
mistrzem. Dziewczyna z~nim na pewno nie wyglądała, jakby potrzebowała
oświecenia.

-- Zen. -- Ax kiwa głową do siebie. -- Zdecydowanie.

~

Na dolnym piętrze domu jest duży pokój z~asortymentem kuchennym, zlewem,
sofami, krzesłami i~ciężkim, wyszorowanym drewnianym stołem. Książki,
papiery i~zestawy leżą w~stosie w~rogu i~na stole. Dee siada przy stole,
oczyszcza przestrzeń pomiędzy kubkami i~narzędziami. Ax wygrzebuje
jakieś kartki papieru i~stalowy długopis kulkowy. Podaje jej.

-- Zatem narysuj -- mówi.

-- Ok -- odpowiada Dee. Bierze długopis w~prawą rękę i~stabilizuje papier
lewą. Szybki maz w~górnym prawym rogu papieru pokazuje jej, że atrament
jest czarny i~działa gładko. Zamykając oczy, wywołuje obraz mężczyzny w~ciężarówce. Ignoruje w~tej chwili dziewczynę (choć jest tam coś, coś w~jej oczach, że Dee myśli dziwne, i~potrzebne dalsze dociekanie, więcej
badań jest koniecznych, ok, przekazanie do Naukowczyni)\ldots teraz. Tak.
Przełączenie na Parametry Drukarki: mały program w~repertuarze
Sekretarki.

Start. Słyszy dźwięk ślizgania się długopisu po papierze przez minutę,
gdy jej prawa ręka porusza się w~lewo i~prawo, horyzontalnie, bardzo
szybko, z~małymi pionowymi ruchami podnoszącymi i~opuszczającymi
długopis na papierze. A jej lewa ręka odsuwa papier od niej, bardzo
wolno. Skończone.

Otwiera oczy. 

-- Proszę -- mówi. Pociera nadgarstek.

Ax patrzy się na nią, z~otwartymi ustami. Zamyka usta i~potrząsa głową.

-- Ok -- mówi. -- No to popatrzmy.

Nawet Dee jest lekko zaskoczona, gdy widzi, jak dobry obraz stworzyła z~pomijania i~przerywania kilkuset prostych linii narysowanych w~poprzek
kartki. Prawie jak czarno-biała fotografia, pokazuje twarz mężczyzny i~niektóre rzeczy z~otoczenia: oparcie siedzenia za nim, kropkowane panele
z tylnej ściany budy, zwisający pozwijany kabel, który wisi z~mikrofonu,
który trzyma przed sobą, ramię dziewczyny.

-- Nie wierzę w~to -- mówi Ax. -- To on. To jest facet, o~którym ci
mówiłem: Jonathan Wilde.

-- Cóż -- mówi Dee -- mówiłam Ci, że nie był mistrzem doskonałym\footnote{możliwe, że mowa o masonach szkockich,
zob.~\url{https://pl.wikipedia.org/wiki/Ryt_Szkocki_Dawny_i_Uznany
} -- przyp.tłum.}.

Ax uśmiecha się do niej, jakby nawet jest zaskoczony jej poziomem żartu
(i och, jakie te małe niespodzianki są bystre), i~wyciąga starą
książkę z~zaspy w~jednym z~rogów. To skórzana okładka trzymająca wydruk
z algocelulozowego papieru. Dee podnosi ją w~dłoni i~przegląda. Pierwsza
strona, która się otwiera, jest blisko końca i~jest to fotografia tego
samego mężczyzny, którego narysowała. Nawet poza i~wyrażenie są podobne,
jest pochylony do przodu, gorliwie mówi do kamery.

-- To jest jedno z~ostatnich zdjęć Wilde, które było opublikowane -- wyjaśnia Ax. -- Zostało wyciągnięte z~wywiadu telewizyjnego z~nim z~lutego 2046.

Dee czuje ciarki na karku, gdy przygląda się temu obrazowi, z~przeszłości prawie nieporównanie oddalonej (ale tylko w~\emph{rzeczywistym czasie}, przypomina jej Naukowczyni, nie w~\emph{czasie statku}. I~znowu zaczyna o~Mili Malleya, tej prawdziwej, tej, po której pub został nazwany. Wyłącza ją.)

-- Dokładnie to on -- mówi. Zerka na obraz, który wykonała, potem na ten w~książce. Uruchamia transformację. -- Wszystkie linie dokładnie się
mapują.

Patrzy na to znowu. Coś ją męczy.

-- Hm, tak -- mówi Ax.

Dee kontynuuje, przerzucając stron od tyłu w~książce. Zdjęć jest coraz
mniej, gdy zbliża się do początku, Wilde staje się młodszy. Większość z~nich oczywiście nie była ustawiona, ale uchwycona w~locie: przycięte
powiększenia z~systemów inwigilacji, spokojna twarz w~gniewnym tłumie...

-- Co to właściwie jest?

-- To akta Wilde'a -- mówi jej Ax. -- Notatki do biografii.

Zatrzymuje się przy kolejnym zdjęciu, ujęcie z~niskiego kąta, zamazane.
Jest opisane ,,FOI(PrevGovts)/SB/08--95''. Dwóch mężczyzn przy stole, w~pubie lub kawiarni. Jeden, zidentyfikowany w~podpisie jako Wilde, jest
odwrócony do kamery. Drugi, mówiący mimo trzymanego papierosa, to Reid.

-- Mówiłem Ci -- mówi Ax. -- Znają się od lat.

Dee wiedziała, na jakimś poziomie, że Reid był jednym z~oryginałów, że
fizycznie przybył z~Ziemi, ale to ciągle jakoś jest szokiem ujrzenie
tego, co jest -- zakładając pochodzenie i~antyczność zdjęcia -- dowodem
wizualnym. Więcej stron przelatuje. Kiedy plik stron jest cienki pod jej
kciukiem, trafia na ostrą, profesjonalną fotografię, która zatrzymuje
jej myśli. Ma ostre, od nożyczek, krawędzie, podpis poniżej i~nabazgrane
przypisanie: \emph{Dumbarton Gazette} 6 kwietnia 1977, najwidoczniej
jakiś lokalny zin. Patrzy się na to, wskazuje na to głupio. Zza jej
ramienia, oddech Ax syczy przez zęby.

Portret ślubny pary: formalne ubrania, nieformalna postawa, prawie
policzek przy policzku. Mężczyzna -- widzi teraz, że ciągłość została
ustalona -- jest młodszą wersją starca z~końca książki, to Wilde, to
mężczyzna, którego widziała wczoraj. Twarz kobiety, ponad żabotowymi
ramionami i~wysokim kołnierzem w~białym, wykończonym koronką, woalu,
jest jej własną.

~

-- Niech zgadnę -- mówi ciężko Ax. -- To jest ten gość, który wszedł do
Mili Malleya?

-- Tak -- wzdycha. -- Nic dziwnego, że wyglądał, jakby Cię rozpoznał. Moje
ciało jest w~końcu klonem, klonem jego żony!

-- Przerażające -- mówi Ax. Przygląda się bliżej na podpis. -- Annette, to
było jej imię.

Dee nie może dłużej patrzeć na zdjęcie i~nie musi: ten obraz zostanie w~jej głowie na zawsze, chyba że je skasuje. To straszne, racja, i~niepokojące w~głębszym sensie: ta odległa bliźniaczka, ta kobieta,
której fizycznych duchem jest Dee, wygląda na szczęśliwą w~sposób, w~który Dee nigdy nie była, z~osobowością, o~której Dee wie, że jest inna
od jej własnej. Tylko fizyczne ciało i~bazowy temperament, który, Dee
wie, jest podobnie genetyczny, są takie same. Pozwala ostatniej partii
stron opaść na obraz i~patrzy się niewidzącym wzrokiem na tytuł na
pierwszej stronie:

~

\begin{center}
Jonathan Wilde \emph{1953-2046: Życie krytyczne}\\
\emph{Eon Talgarth}
\end{center} 
~

Ax krąży po pokoju, niezauważając \emph{niepokoju} Dee, mówiąc
podekscytowany. Dee musi jeszcze raz powtórzyć pierwsze kilka sekund,
zanim nadąża: 

-- Więc mamy zagadkę -- mówił. -- Kilka tygodni temu Wilde
widzi cię na ekranie Reida. Nie daje znaku poznania, ale uruchamia
zestaw instrukcji, żebyś zaczęła ładować informacje, może w~intencji
przebudzenia, a~może nie. Wczoraj, Wilde wchodzi, najwidoczniej po
odmłodzeniu w~międzyczasie, widzi cię i~wariuje.

Dee potrząsa głową.

-- Gość w~pubie nie był odmłodzonym mężczyzną, którego widziałem na
ekranie.

Ax marszczy brwi. 

-- Brzmisz dość pewnie.

-- Odmłodzenie nie zmienia faktu, że żyłeś dłużej. Zawsze się pokazuje.
Nie na zdjęciu, może, ale kiedy widzisz kogoś poruszającego się,
mówiącego, jest to oczywiste. -- Uśmiecha się. -- Nie sądzisz?

-- Nie widziałem dostatecznie dużo re-młodych -- mówi Ax. -- To nie jest
zwykła procedura, większość ludzi stabilizuje się na wieku, który
uważają za najlepszy. -- Śmieje się. -- Czasem jest moda na starzenie, ale
nigdy nie trwa długo.

-- Powiem Ci tak -- mówi Dee. -- Wilde, którego widziałam dwa tygodnie temu
żył cholernie dłużej niż Wilde, którego widziałam wczoraj.

-- Ok, załóżmy, że jest ich dwóch. To nie większa zagadka niż istnienie
tylko jednego z~nich, ponieważ w~ogóle go nie powinno być tutaj. Nie był
w załodze ani w~ekipach. -- Rzuca jej dziki uśmiech. -- Tak mówi Reid lub
przynajmniej tak mówią listy. Listy płac. Sprawdziłem. Jednak tak jak
mówiłem, ludzie mówią, że go widzieli. I~teraz, masz dowód. Wrócił!

Podnosi znowu zdjęcie, które zrobiła Dee. Widzi, że jego ręce drżą.
Zapala papierosa po kilku próbach i~patrzy w~dal przez chwilę. Jego
wyraz twarzy powoli się zmienia, w~sposób, który sprawia, że Dee myśli,
w jaki sposób dostał swoje imię: jest twarda, ostra i...ostateczna.

-- Wiesz, co to znaczy? -- pyta.

Dee zaciska usta, potrząsa głową.

-- To znaczy, że wrócił z~\emph{Nieożywionych} -- mówi Ax. -- To znaczy, że
wszystko się zmieni. To znaczy, że wszystko się może zdarzyć.

-- Nie rozumiem -- mówi Dee.

Ax gasi papierosa i~zapala kolejnego. Ciągle się trzęsie.

-- Ludzie zakładają pewne rzeczy -- mówi. -- Zakładają, że sprawy będą się
toczyć tak jak się toczyły. Wiedzą, z~czym może da się uciec. Wiedzą, do
czego mogą skłonić ludzi. Przykładowo, zgodziłem się, by inni ludzie
używali mojego ciała, ponieważ potrzebowałem pieniędzy. A oni wiedzieli
o tym. Jednak ponieważ ja się \emph{zgodziłem}, to oni myślą, że
wszystko jest w~porządku. Niektórzy z~nich nawet wiedzieli, że to
nienawidziłem. Jednak zgodziłem się na to.

Dee nagle sama potrzebuje papierosa. Zapala jednego, a~jej ręce, teraz,
drżą.

-- Czy Reid kiedykolwiek pozwolił innym użyć Twojego ciała?

-- Och, nie -- mówi prędko Dee. -- Był bardzo zaborczy.

-- Ale używał cię -- nalega Ax. -- Czy chciałaś, czy nie.

-- Zawsze chciałam -- mówi Dee, ale jej uśmiech Seksu ukrywa nowe i~gryzące wątpliwości jak bardzo taka zgoda była warta, teraz, patrząc
wstecz. Ax obserwuje ją, a~ona widzi, że on widzi narastające
wątpliwości.

Otwiera szufladę w~stole i~sięga, wyciąga nóż. To nie jest nóż kuchenny.
Ma czarną drewnianą rączkę, mosiężną osłonę i~trzydzieści centymetrów
ostrza. Prawie od niechcenia, Ax wbija ostrze noża w~stół i~puszcza
rękojeść, więc lekko odskakuje i~wibruje.

-- Teraz wiesz, kim jesteś -- mówi cicho Ax. Dee nie jest pewna, czy mówi
do niej. Całe drżenie zniknęło z~jego ciała, z~jego głosu, i~pojawiło się
w drżącym ostrzu. -- Jesteś osobą. Jesteś wolna. Czy kiedykolwiek
myślałaś, co byś zrobiła ludziom, którzy traktowali Cię jak
\emph{mięso}?

~

Tutaj, w~wilgotno-opuszczonych mieszkaniach pomiędzy dwoma ramiona
miasta, jest cicho nawet o~poranku szóstodnia. Jedynymi dźwiękami to
brzdąkanie motoru łódki, sporadyczny syk transportu odrzutowego nad
głowami i~krzyki zaadaptowanych ptaków: bipy zgubionego satelity
rdzocholewek, kwaczenie błotów i~krakanie pustynnych mew. Szóstodzień
jest dla większości ludzi dniem, kiedy jakaś praca jest robiona, ale nie
za dużo.

(Tamara słyszała opinię, że dzień został tak nazwany z~powodu liczby
osób pracujących, lub nie, na kacu, ale to tylko mit. Ponad
nowomarsjański wiek temu, Reid wyraził opinię, że kontynuowanie
nazywania dni po bogach Systemu Słonecznego byłoby niewłaściwe. Nie
udało się wszystkim zgodzić na inne nazwy, więc tydzień idzie tak:
jednodzień, dwadzień, trzeciodzień, czterodzień, piątodzień,
szóstodzień, siódmiodzień. W~każdym dniu jest dwadzieścia pięć godzin i~dziesięć minut. Dla wygody w~pierwszych sześciu dniach jest dwadzieścia
pięć godzin i~dwadzieścia sześć w~siódmiodniu. W~roku jest sto dziesięć
tygodni. Mniej więcej. Wszystkie poważne chronologie są wykonywane w~wielokrotnościach sekundy, obliczając od momentu, kiedy zegar Statku
wyszedł z~Mili Malleya, około 6,4 gigasekundy\footnote{gigasekunda to
10 do potęgi 9 sekund czyli miliard sekund, ok. 31,70 lat ziemskich, zatem Statek wyszedł z~Mili Malleya ca. 202,94 lata ziemskie temu. W~innych powieściach Kena
MacLeoda (a także m.in. C.~Doctorowa czy C.~Strossa) pojawiają się również kilosekundy i~megasekundy, odp. jedna
kilosekunda to tysiąc sekund czyli ca 16 minut, jedna megasekunda czyli milion sekund to 11,5 dnia ziemskiego -- przyp.tłum.} temu).

Łódź Tamary uderza o~brzeg kanału, gdy dryfuje wzdłuż na minimalnej
mocy. Jest na kapilarze Kanał Okrężnego. Płytki sztuczny strumyk niesie
ją od centrum miasta w~kierunku pól. Dzielnica ludzka jest po jej
prawej, Piąta Dzielnica po jej lewej. Pomiędzy nimi jest obszar odpadów,
nie do końca błoto, ale już nie pustynia i~jeszcze nie pola. W~niej,
wyruszając z~dziedziny maszyn Piątej Dzielnicy, mogą być znalezione
biomechanizmy, zwykły łup Tamary.

Pustynna mewa opada, krzycząc, około stu pięćdziesięciu metrów z~przodu i~trzydziestu na lewym brzegu. Tamara zwiększa obroty i~zmniejsza swój
profil, gdy inne mewy nurkują, żeby dołączyć. Skrzeczą i~wrzeszczą
dookoła czarnej rzeczy. Łódź rusza po przekątnej przez kanał. Tamara
zbliża obraz w~prawym oku. Czarna rzecz ma młócący wyrostek. Uparta mewa
przyczepia się, zabierając nieco momentu z~drżenia w~chwilach skakania
prawielotu.

-- Zostań -- mówi Tamara botowi łodzi, ten posłusznie zwalnia silnik i~zaczepia się do brzegu, gdy Tamara wychodzi, ściskając długi hak.
Wyciąga pistolet, gdy biegnie do przodu. Huk ślepych rozprasza mewy w~krążące oburzenie nad głową. Gdy stopy Tamary walą o~wilgotny piasek i~przeskakują kępy trawy, czarny obiekt, brodawkowata, gumowa kula
średnicy około jednej trzeciej metra z~przynajmniej metrowym cepem,
zaczyna się przesuwać w~kierunku najbliższej łaty, która wygląda
podejrzanie jak ruchome piaski. Kiedy jest około czterech metrów dalej,
Tamara czuje łaskotanie za grzbietem nosa. Zatrzymuje się i~wącha.
Łaskotki są stałe, dobrze. To oznacza, że radioaktywność jest
ograniczona, nie w~powietrzu. Ciągle, ta rzecz jest niewygodnie gorąca.
Nie niebezpieczna, ale musi być ostrożna.

Okrąża ją ostrożnie, wchodząc pomiędzy nią a~mokry obszar. To zbliża się
do niej: smagnięcie, podciągnięcie, odbicie, smagnięcie, podciągnięcie,
odbicie. Zatrzymuje się. Czubek cepa unosi się i~porusza się z~boku na
bok, potem naciska na ziemię. Tamara podchodzi, potyka się, gdy jej lewa
stopa wychodzi z~ziemi z~nieoczekiwanym ssącym dźwiękiem. Gumowa
kończyna się cofa.

Tamara przykuca i~sięga hakiem, prostym mechanizmem długości kilku
metrów, który ma prymitywną roboczą dłoń na końcu i~uchwyt dla niej do
złapania, na jedną rękę, i~rozwinięcia chwytaka. Rozluźnia go na ziemi i~łapie cep przy podstawie. W~uczynnym odruchu, mackowata wypustka owija
się dookoła haka i~próbuje zmiażdżyć to na śmierć.

Tamara podnosi to z~ziemi i~wraca do łodzi. Biomech, wyewoluowany lub
zaprojektowany na granicy pomiędzy dziedzinami, nie jest złym łupem. Ma
zmysły, odruchy, i~prawdopodobnie możliwość koncentrowania
radioaktywności w~obrębie twardej skóry. Gdzieś w~Dzielnicy Ludzkiej
jest technik, który szuka właśnie takiego genotypu, lub ona ma taką
nadzieję.

Dopiero usiadła w~łodzi i~jest w~trakcie manewrowania hakiem i~jego
ładunkiem, próbując zachować dystans od tego (poniżej dwóch metrów
łaskotanie w~jej zmyśle Geigera staje się bólem), jednocześnie
otwierając pojemnik, kiedy w~lewym uchu słyszy dzwonienie.

-- Cholera -- mówi głośno. Zaciska mięśnie gardła, żeby włączyć mikrofon,
mruga na ekran telefonu i~ze skierowanym w~prawo spojrzeniem, akceptuje
rozmowę. Pierwszy ekran, który się pojawia, jest niezdarny, nawet gdy
wisi z~halucynacyjną żywością w~przestrzeni pomiędzy nią a~końcem haka.
To jak kamera patrząca na ekran monitora w~jakieś prymitywnym błysku
samoświadomości maszyny. Tekst przesuwa się w~dół, głos zza ekranu
sprawdza pisownię.

-- Usługi Prawne Niewidzialna Ręka -- intonuje. -- Przychodzące wyzwanie
od\ldots -- i~tu waha się, jakby nawet ta dostojna implementacja głosu IBM
była zdumiona własnym zuchwalstwem -- \ldots Davida Reida. Czy akceptujesz?

-- Tak. -- Jednym haustem odpowiada Tamara.

Ekran jest natychmiast zmniejszony do rogu jej oka, a~główny widok jest
zajęty przez solidny obraz, który widziała tyle razy wcześniej, ale
nigdy wcześniej nie mówił do niej. Okno unosi się przed jej oczami, z~głową Reida i~rękami, w~wygodnej odległości do rozmawiania za nią. Za
nim, może zobaczyć różne części pokoju, jasne okno (najwidoczniej
prawdziwe). Chodzi dookoła, gdy mówi.

-- Tamara Hunter? -- pyta.

-- Tak.

Uśmiecha się, zaglądając za nią.

-- Widzę, dlaczego tak się nazywasz. Dobra, do interesów moja pani.
Obecnie posiadasz jedną z~moich maszyn, gynoid Model D, i~chcę ją z~powrotem. Już.

Tamara bierze głęboki wdech.

-- Nie jestem w~posiadaniu tego, jej. Domaga się samo-własności i~jestem
jej obrońcą. Tak jak kilkoro zaprzysiężonych moich sojuszników i~inni
klienci Niewidzialnej Ręki.

-- Gówno -- replikuje Reid. -- Nawet nie ma rozumu, żeby żądać własności.

-- Teraz ma, i~to zrobiła, przed świadkami.

-- Przed jebanym IBM, tak. Twój ekspercki system prawny sam nie zdałby
Turinga, ani tym bardziej by go zastosował.

-- CZUJĘ SIĘ DOTKNIĘTY.

-- Zamknij się -- mówi Tamara, ciągle walcząc z~hakiem. Rzecz na końcu
toczy się jak źle nałożone spaghetti na widelec. -- Przepraszam, Reid. To
nie było do Ciebie.

-- Doceniam to -- mówi sucho Reid. -- Mówiłaś?

-- Mogę przedstawić ludzkich świadków przed dowolnym sądem, jaki
wybierzesz. Gynoid już nie jest Twoją zabawką zombie.

Oczy Reida się zwężają. 

-- To dlatego, że została \emph{zhakowana}. To
ciągle nie jest samoistny rozwój, nawet jeżeli to ma znaczenie, czego
nie ma.

-- Już czas, żeby to zrobiła. -- mówi Tamara spokojnie. -- Jestem gotowa
walczyć z~Tobą o~to.

-- Zróbmy to po twojemu -- mówi Reid. -- W~sądzie, zatem.

-- To Twoje wyzwanie -- wskazuje Tamara.

-- Ok, pierwsza oferta jest Twoja. -- Skłania się.

Tamara mruga do góry ekran Niewidzialnej Ręki. Wyświetla listę sądów w~malejącym porządku preferencji. To krótka lista. Wybiera pierwszy, ale
jej głos nie jest tak optymistyczny, gdy mówi: 

-- Eon Talgarth, Sąd
Piątej Dzielnicy.

-- Zaakceptowane -- mówi od razu Reid.

Tamara zmniejsza ekran IBM i~patrzy na Reida, który odpowiada uprzejmym
spojrzeniem.

-- Co? -- pyta. Potem: -- Proszę o~potwierdzenie.

-- Akceptuję. -- mówi Reid z~przesadną formalnością -- że decyzja zostanie
podjęta przez Sąd Piątej Dzielnicy w~sprawie ja przeciwko Tamarze Hunter
i sojusznicy reprezentowani przez Usługi Prawne Niewidzialnej Ręki i,
kreska, lub, samych, w~najbliższym terminie wygodnym dla wszystkich
stron.

-- I~ja także -- mówi Tamara.

IBM powtarza to, co powiedzieli.

-- A w~międzyczasie, żadnych łapaczy? -- pyta podejrzliwie Tamara.

-- Oczywiście, bez łapaczy -- mówi Reid. Uśmiecha się do niej w~taki
sposób, że, mimo wszystkiego, mimo jej samej, lekko się rumieni. -- Do
zobaczenia w~sądzie, Pani.

Ekran znika w~chwili, by Tamara zobaczyła, jak czarny biomech odplątuje
się gładko z~haka, wpada do kanału i~wijącym się ruchem cepa, odpływa.

~

-- W~porządku -- powiedział Jay-Dub. -- Zróbmy to po Twojemu. Mniemam, że
mogę coś z~tego wyciągnąć. -- Zatrzymał się na skrzyżowaniu nabrzeża i~ulicy. -- Ale zanim zaczniemy pędzić, mam kilka sugestii.

Wilde zatrzymał się i~spojrzał do tyłu. 

-- Tak?

-- Załatw sobie broń -- powiedział Jay-Dub. -- I~lepsze ubrania. Wyglądasz,
jakbyś właśnie wrócił z~pustyni, czy coś. Również, jeżeli chcesz udać
się głównej siedziby abolicjonistów, będzie szybciej łodzią.

-- Masz rację -- powiedział Wilde.

Godzinę później miał na sobie luźną czarną kurtkę, koszulę i~spodnie,
wszystko z~jakiejś ciepłej tkaniny, która był zapewniany była odporna na
rozcięcia, oraz oglądał masywny metalowy automat, gdy siedział w~zatłoczonej \emph{vaporetta}\footnote{wenecka łódź, wodny autobus -- przyp.tłum.}. Inni pasażerowie, w~większości młodzi, w~satysfakcjonujący sposób nie zwracali na niego uwagi. Wilde siedział, na
uboczu przy burcie łodzi, patrzył na sceny brzegu kanału i~nadstawiał
uszu na slangowy, akcentowany angielski jego współtowarzyszy. Jay-Dub,
kończyny wycofane, leżał u jego stóp jak bagaż. Był jedynym robotem na
pokładzie, oprócz sternika, kawału elektronicznej cybernetyki na dziobie
statku.

Sieci zbierające po bokach łodzi wyławiały kołyszące się kule plastiku z~wody i~przerzucały je, turkoczące, do ładowni pod pokładem. Łódź
opuściła komercyjną wesołość Kamiennego Kanału i~wpłynęła do serii
tuneli i~wysokich, wąskich kanałów. Tutaj, w~ciastowatości zielonych alg
na ścianach, można było zobaczyć mniejsze kule. Poruszały się w~dół
bardzo wolno, ale ich kurs mógł być wywnioskowany: im bliżej wody
tonęły, tym większe urastały, póki nie odpadły i~nie odpłynęły. Wilde
wstrzymał się od zapytania Maszyny o~ekonomię i~ekologię tego
bio-przemysłowego procesu.

Dotarli do celu czterdzieści minut od wyjazdu. Łódź zatrzymała się,
kaszląc z~silnika i~szarpiąc śrubami wzdłuż małego molo ze stopniami
prowadzącymi na wąską ulicę przy kanale. Jedyny ludzki członek załogi,
który nic nie robił prócz zebrania opłat, otworzył oczy i~machnął rękę.

-- Plac Okrągły, dwieście metrów -- obwieścił, i~wyłożył krótki pomost do
schodów. Wilde postarał się być ostatnim wychodzącym z~łodzi. Uśmiechnął
się do przewoźnika.

-- Jesteś kazachskim Grekiem -- powiedział.

Oczy mężczyzny się rozszerzyły. Złapał rękę Wilde'a i~powiedział coś w~innym języku.

-- Wszyscy przebyliśmy długą drogę -- powiedział Wilde.

-- Zdobywaj przyjaciół i~wpływaj na ludzi -- zadrwił Jay-Dub,
\emph{sotto-voce} na szczycie schodów. -- Zawsze przeklęty agitator, co?

~

Około trzydzieścioro osób szło ulicą, Wilde i~Robot kilka metrów za
resztą. Przed nimi, wyspa rynku Placu Okrągłego właśnie nastrajała się
do swojej dziennej dysharmonii. Ulica była wyłożona z~małymi kafejkami
na chodniku i~straganami, i~przecięta alejkami, w~których nawet mniejsze
sklepy zasypywały jakiegoś rodzaju handlem z~okien i~drzwi.

Byli kilka kroków od takiego wejścia w~aleję, po przeciwnej stronie
rogu, przy którym kilka niebezpiecznie małych stołów było używanych do
podawania kawy w~proporcjonalnie maleńkich filiżankach, kiedy Jay-Dub
powiedział nagle: 

-- Stop!

W tej samej chwili Wilde również zauważył dwóch mężczyzn, tych samych
dwóch, którzy szukali w~pubie. Siedzieli przy jednym z~tych małych
stołów, patrząc na niego zza ciemnych okularów. Jego ręka zatrzymała się
w trakcie sięgania po nową broń, gdy tamci sięgnęli po swoje.

W ten chwilowy impas wjechał dziwny pojazd: platforma na kołach, z~aparatem przypominającym żurawia po obu końcach. Wysunęła się z~alei bez
ostrzeżenia. Wilde odskoczył. Mechaniczne ramię rozwinęło się z~żurawia
i rzuciło się koło niego. Odwrócił, żeby zobaczyć, jak pazury tego
ramienia zacisnęły się dookoła dolnych odnóży Jay-Duba. Podniosły
walczącą maszyną ponad jego głową i~położyły go delikatnie na płaskiej
platformie.

Wilde przykucnął, złapał platformę obiema dłońmi i~się podciągnął.
Jay-Dub w~tym momencie szarpał się wobec ograniczeń i~spadł, gdy rzecz
ruszyła. Gdy ludzie reagowali, kaskada stołów także się przewróciła.
Wilde zanurkował ponad powłoką Jay-Duba, przetoczył się z~kopnięciem w~nogi obu mężczyzn -- teraz na nogach z~parującymi plamami na udach -- i~chwilę później stał i~biegł. Dzikie spojrzenie do tyłu ujawniło dwóch
mężczyzn kilka kroków za nim, w~ślad za rozepchniętymi gośćmi i~przewróconymi meblami.

Plac Okrągły był tuż przed nim, tłum gęstszy.

-- Pomocy! -- krzyknął Wilde, zanurzając się w~tłum.

-- Nie idź dalej -- rozkazał donośny głos z~przodu i~góry. Mógł dochodzić
z jednego z~głośników zawieszonych z~kabli pomiędzy lampami i~drzewami.
Wilde zatrzymał się i~spojrzał znowu za siebie. Obaj goniący mężczyźni
zatrzymali się kilka metrów dalej, dygocąc na krawędzi chodnika,
dokładnie tam, gdzie koniec wąskiej ulicy napotykał balustradę mostu.

Jeden z~nich zrobił ruch do wnętrza jego kurtki. Zanim Wilde mógł
zareagować, coś innego szybciej zareagowało. Coś pająkowatego i~lekkiego, kula sztywnych badyli, która przeleciała ponad głowami tłumu,
wleciała w~obu mężczyzn. Gdy ich uderzała, jej badyle stały się
elastyczne i~owinęły się wokół nich obu, od ramion do ud.

Ograniczeni, byli ledwie nawet obiektem ciekawości. Wilde stał tam,
gdzie był przez minutę, aż tłum jakoś się rozproszył. Potem poszedł z~powrotem drogą, którą przybiegł. Gdy omijał łukiem obu mężczyzn, dał im
prawie trzy metry przerwy. Patrzyli się na niego.

-- Kto was przysłał? -- spytał.

-- Jeb się -- powiedział jeden z~nich.

-- Pozdrówcie Reida -- powiedział Wilde.

Na to drugi mężczyzna spróbował rozerwać swoje więzy, ale wieloramienna
maszyna w~odpowiedzi tylko docisnęła. Wilde kontynuował wzdłuż alei, a~w~trakcie minął dwóch młodych mężczyzn, kierujących lub pasących teraz
pustą i~zniszczoną platformą w~przeciwnym kierunku.

-- Przepraszam -- powiedział Wilde. -- Co się stało z~tym drugim robotem?
Tym, która ta rzecz złapała?

-- Uciekł -- powiedziano mu.

Podziękował im i~sprawdził sam. Większość, która cokolwiek mogła
powiedzieć, brzmiała, że maszyna konstrukcyjna uciekła wzdłuż alejki.
Wilde spojrzał wzdłuż niej, pokręcił głową, wymamrotał coś do siebie,
wrócił do mostu. Doszedł w~momencie, by zobaczyć dwóch młodych mężczyzn
opuszczających platformę, która obecnie bezpiecznie trzymała jego
napastników pozostałym działającym ramieniem żurawia. Druga maszyna
ciągle tam była, znowu w~postaci piłki z~kolcami. Przetoczyła się ku
niemu jak chamaechor\footnote{roślina, których pędy nadziemne lub ich
części (kwiatostany, dęte owoce) stanowią diaspory przemieszczające się
po powierzchni ziemi pod wpływem działania wiatru,
więcej~\url{https://pl.wikipedia.org/wiki/Biegacze } -- przyp.tłum.}. 

-- Dzień dobry -- powiedziało. Brzęczący głos wydawał się wygenerowany
przez wibrację niektórych łodyg. -- Wezwałeś pomoc w~obrębie dziedziny
Usług Prawnych Niewidzialna Ręka. W~odpowiedzi zainterweniowałem.

-- Dziękuję -- powiedział Wilde.

-- Choć nie została zawarta wiążąca umowa, jest kwestią uprzejmości
uregulowanie opłaty dla Niewidzialnej Ręki. Jako wzajemna uprzejmość,
Niewidzialna Ręka chciałaby zaoferować Ci dziesięciotygodniową polisę
obronną, od której ta opłata byłaby odpisana, jeżeli zdecydujesz się
zapłacić z~góry.

Wilde spojrzał z~rozbawieniem w~dół na chętną maszynę.

-- Ile?

-- Dwadzieścia gramów złota lub równoważnik.

-- Bardzo rozsądne -- powiedział Wilde. -- Przyjmujecie karty?

-- Proszę za mną -- powiedziała maszyna.

Wilde przesunął jego kartę przez slot zardzewiałej skrzyni
superkomputera\footnote{oryg. mainframe tj. klasa komputerów używanych
głównie przez duże organizacje dla krytycznych aplikacji,
więcej~\url{https://pl.wikipedia.org/wiki/Mainframe } -- przyp.tłum.}. Maszyna, która przyszła mu na pomoc, przyprowadziła go
tutaj i~zostawiła.

-- Dziękuję -- powiedziała Niewidzialna Ręka. -- Przedstawiłeś się jako
Jonathan Wilde. Twoje konto zostało oryginalnie otwarte przez maszynę
zwaną Jay-Dub, vel Jonathan Wilde, zatwierdzone w~Twoim imieniu
ostatniej nocy w~Bank spółdzielczym Stras Cobol.

-- Prawda -- powiedział Wilde.

-- Mam w~plikach sprawę przeciwko Tobie -- powiedziała maszyna. -- Czy
chciałbyś teraz wysłuchać szczegółów?

Wilde rozejrzał się dookoła.

-- Proszę bardzo.

Twarz Reid pojawiła się czerwonawym monochromatycznym hologramie za
ekranem maszyny.

-- Ja, David Reid, oskarżam Jonathan Wilde, bez stałego miejsca
zamieszkania, mianowicie o~to: że robot znany jako Jay-Dub, własność
tegoż Jonathan Wilde, został użyty do uszkodzenia systemów gynoida Model
D, znanego jako Dee Model, mojej własności. Jeżeli Jonathan Wilde
pragnie bronić się prawnie przeciwko temu oskarżeniu, żadne dalsze próby
nie zostaną podjęte przeze mnie, moich agentów lub moich sojuszników,
żeby go zaaresztować lub skonfiskować jego maszynę. Jeżeli tego nie
pragnie lub odmawia wzajemnie akceptowalnego sądu, te próby będą
kontynuowane. Kończę to oświadczenie rano szóstodnia, pięćdziesiąty
siódmy dzień roku stodrugiego czasu Statku.

Wilde obserwował, jak obraz maleje do rubinowego paciorka.

Westchnął. 

-- W~jaki sposób Reid wiedział, że będę zarejestrowany u
Ciebie?

-- Nie wiedział -- powiedział superkomputer. -- Ta wiadomość została
wysłana do wszystkich agencji obrony. Przekazałem innym, że została
dostarczona. Nie są dalej zainteresowani, chyba że oczywiście wybierzesz
jedną z~nich do obrony.

-- Nie -- powiedział Wilde.

-- Doskonale -- powiedziała Maszyna. -- Czy chciałbyś bronić się prawnie
przed tym zarzutem?

Wilde pomyślał o~tym.

-- Tak -- odpowiedział.

Cienie i~światła poruszyły się za ekranem.

-- Chciałabym coś zasugerować -- powiedziała maszyna. -- W~toku jest inna
sprawa, pomiędzy Reidem i~inną stroną, w~sprawie Dee Model. Dee Model
jest również moim klientem. Może zechcesz rozważyć połączenie Waszych
działań prawnych.

-- Mogę w~istocie -- powiedział Wilde.

-- Poczekaj tutaj -- powiedziała maszyna. -- \ldots Możesz zapalić.

\chapter{Realizm Kapitalistyczny}

Awionetka lub helikopter zbliżają się ze wznoszącym dźwiękiem, który się
załamuje, a~potem cichnie. Jednak kiedy słyszysz dźwięk aerosilnika i~ten sam płaski ton utrzymuje się minutami, patrzysz do góry, zirytowany
przez to nienormalnie stałe buczenie i~widzisz sterowiec.

Stałem na Waverley Bridge w~chłodnym zmierzchu, spojrzałem do góry i~zobaczyłem sterowiec, nisko na niebie, podkradający się zza mnie jak
ciarki na plecach, niebieski z~napisem ,,MAZDA'' białymi kapitalikami na
boku. To był ten sam sterowiec, który widziałem około dwóch godzin
wcześniej w~Glasgow. Prawie dziwniejsze niż UFO, coś, czego tutaj nie
powinno być, maszyna z~alternatywnej rzeczywistości, gdzie
\emph{Hindenburg} lub Dow Jones nie rozwaliły się lub Niemcy wygrali
Wielką Wojnę. Gdy patrzyłem, jak przesuwa się niczym chmura z~zewnętrznym motorem, miałem chwilowe poczucie oddzielenia, jakby też nie
powinienem tu być. Co tutaj robiłem, obserwując sterowiec z~wietrznego
mostu, kiedy powinienem być w~pociągu do Londynu?

To musiał być upał. Upał w~Londynie, taki, że lato nie było niczym
podobnym od lata 1976 roku, kiedy spędziłem tygodnie, przechodząc z~wywiadu na wywiadu, przesypiając u kumpli lub w~domu moich rodziców,
martwiąc się wysypką nienawistnych naklejek z~flagą Wielkiej Brytanii
(Union Jack) wyklejonych wszędzie przez Front Narodowy. (A w~międzyczasie, w~innym gorącym mieście, polscy robotnicy zatrzymali kolej
i obalili ceny mięs, i~prawie państwo, prawie...) Wracając do Glasgow i~suchszego powietrza, wdzięczny, wchodząc do laboratorium Annette, gdzie
szarańcze po sekcji były przypięte do foliowych talerzy z~czarnego
wosku, a~zapach parującego etanolu uderzył mnie w~nos, gdy ją złapałem i~powiedziałem: 

-- Mam pracę.

Dziewiętnaście lat później i~ciągle ta sama praca. Inni pracodawcy, inne
uczelnie, studenci nawet młodsi i~bardziej niepewni na temat ich
teraźniejszości, nie mówiąc o~przyszłości. Jednak w~końcu teraz miałem
interes na boku, który w~dobre miesiące przynosił tyle samo lub więcej
niż praca. Moje polemiki w~niezrozumiałych biuletynach i~czasopismach, a~potem w~niejasnych grupach dyskusyjnych, miały -- zgodnie z~moim planem,
ale ciągle ku mojemu zaskoczeniu -- zaowocować zainteresowaniem głównego
nurtu. Kilka komisji think-tanków, jeden lub dwa artykuły w~gazetach
akademickich, rozdział w~nadchodzącym podręczniku ekonomii\ldots Annette i~Eleanor miały, lub przynajmniej pokazywały, więcej wiary w~mój
ostateczny sukces niż ja. Czasem czułem się winny.

Byłem online przy biurku w~domu, przygotowując strony www dla interesu,
kiedy Reid zadzwonił w~zeszłym tygodniu. Kiedy wymieniliśmy się
uprzejmościami, powiedział: 

-- Przyjeżdżasz na konwencję science fiction
w Glasgow?

-- Tak! Zarezerwowałem tam stoisko. Kosmiczny Kupcy. Będziesz?

-- Raczej nie -- powiedział z~żalem. -- Nie mam jak się wyrwać z~pracy.
Niemniej, chciałbym się spotkać z~Tobą potem, w~Edynburgu.

-- To miły pomysł, ale...

-- Nie, nie, czekaj. To nie tylko spotkanie towarzyskie. Mam dla Ciebie,
hm\ldots propozycję biznesową. Coś, czym byłbyś naprawdę zainteresowany.

-- Och, dobra, to co innego. Co to jest?

-- Hm, wolałbym nie mówić tego w~tej chwili. Przepraszam za ostrożność,
ale szczerze, to jest na poważnie i~byłoby naprawdę warte Twojego czasu.
Po prostu wyjdziemy na kilka drinków i~to przegadamy. Możesz przespać
się u mnie lub hotelu, jak wolisz, mogę zapłacić rachunek i~przejazd...

-- Nie, nie ma potrzeby...

-- Naprawdę. Zrozumiesz, kiedy o~tym porozmawiamy, dobra?

Zaintrygowany myślą o~jego ofercie pracy w~ubezpieczeniach, zgodziłem
się z~nim spotkać. To musiał być upał.

~

Reid wszedł powoli z~krańca mostu przy Princes Street, z~jakiegoś powodu
z przeciwnego kierunku, z~którego się go spodziewałem.

-- Cześć, cieszę się, że ci się udało.

-- Miło Cię widzieć.

Jego włosy znowu urosły długie. Jego ubrania było codzienne, ale
subtelne: miękkie czarne chinosy, niebieska koszula zapinana na guziki,
jedwabny krawat, ciemna lniana kurtka. Poczułem się nieco zarośnięty w~dżinsach, trampkach i~obcięty, jak astronauta, na jeża.

-- Wyglądasz elegancko.

-- Dzięki. -- Zaczęliśmy iść w~tym samym kierunku, w~którym szedł Reid, ku
Rock. -- Wyglądasz\ldots dobrze.

Obaj się roześmialiśmy.

-- To iluzja -- powiedziałem. -- Właściwie czuję się nieco rozbity. Zbyt
dużo kaców w~ciągu ostatnich czterech dniach.

-- Ach, wkrótce to zapijesz -- powiedział. -- Ale najpierw, jadłeś?

Mój żołądek ostro potwierdził, że nie. 

-- Nie przez wieki -- powiedziałem.
Zatrzymaliśmy się na skrzyżowaniu, gdzie ruch nadjeżdżał z~czterech
stron. Reid rozejrzał się dookoła.

-- Ok -- powiedział -- Viva Mexico! -- To okazało się być meksykańską
restauracją w~połowie Cockburn Street i~niżej po kilku schodkach. Było
cicho. Reid skinął głową na kelnera. -- Stół dla trzech proszę.

Kelner poprowadził nas do stolika dobrze oddalonego od innych i~usiedliśmy. Reid zamówił trzy duże lagery. Rozejrzałem się, gdy
przeglądał menu. Twarze mężczyzn w~szerokich kapeluszach i~z długimi
karabinami groźnie spoglądały na mnie z~brązowo-białych fotografii
egzekucji, pogrzebów, ślubów, wraków pociągów\ldots Przyglądałem się
leniwie ścianom za zdjęciami ciężko uzbrojonych chrzcin lub ukończenia
szkoły, kiedy lagery się pojawiły i~Reid spojrzał.

-- Jak poszedł Worldcon\footnote{coroczny światowy konwent science-fiction,
w Glasgow odbył się w~latach 1995 i~2005,
zob.~\url{https://en.wikipedia.org/wiki/Worldcon } -- przyp.tłum.}? 

-- Cudownie -- odpowiedziałem. -- Tak mi powiedziano. Byłem w~pokoju
sprzedawców większość czasów. Kosmiczni Kupcy wyszli dobrze, jednak.

-- To Twój biznes?

-- Tak. -- Wyjąłem portfel i~podałem jedną z~ostatnich wizytówek z~adresem
email, stroną www, numerem telefonu i~skrytką pocztową. -- Kilka lat temu
szukałem pamiątek z~kosmosu, video Ziemi z~orbity, takich rzeczy, i~byłem zdziwiony, jak trudno było je znaleźć. Szczególnie w~jednym
miejscu. Więc pomyślałem, hej, okazja biznesowa! Zacząłem z~zamówieniem
reklam w~czasopismach SF, potem wyszukiwaniem rzeczy na konwencjach.
Wydaje się, że pomysł wypalił.

Reid się uśmiechnął. 

-- Wystartował! Dobrze. Cheers.

-- Slainte.

Popatrzyłem na trzecią szklankę po cichu musującą.

-- Kim jest nieobecny przyjaciel?

-- Wpadnie w~każdej chwili. Spokojnie. Ciągle palisz?

-- Znowu, niestety. -- Nie dodałem, dzięki Tobie.

Podał mi papierosa.

-- Jak Annette?

-- Dobrze. Przesyłam wyrazy miłości. -- Nie mrugnął.

-- A Eleanor?

Nie mogłem się uśmiechnąć całą twarzą. 

-- Och, ona jest wspaniała. Dąsa
się w~pokoju, słuchając CD, czytając śmieci, przez większość czasu, ale
w zasadzie jest fajną młodą dziewczyną.

-- Nie chciała przyjechać na konwencję?

-- Nie jestem pewien -- powiedziałem. -- Jakby wzruszyła ramionami, gdy jej
się pytałem. Annette chciała zachować urlop na później w~roku i~myślę,
że w~końcu Eleanor wolała zostać z~Mamą. Nie chciałem ryzykować
odkrycia, że właściwie nie chciała jechać i~odepchnięcia jej na całe
życie.

-- Jak te demonstracje, co? -- Reid uśmiechnął się, wspominając.

Uśmiechnąłem się. 

-- Mów mi o~tym\ldots Annette i~jej ,,walka o~pokój''!
Kiedy Eleanor miała trzynaście lat, chciała dołączyć do cholernych
Szkoły dla Lotników\footnote{brytyjska ochotnicza wojskowa organizacja dla
młodych,
więcej~\url{https://en.wikipedia.org/wiki/Air_Training_Corps
} -- przyp.tłum.}.

-- Co ją zatrzymało?

-- Nie my -- zapewniłem go. -- Cięcia w~wojsku.

Krzesło po naszej lewej nagle zostało zajęte przez chudego mężczyznę w~średnim wieku, ubranego podobnie do Reida, z~łysiejącymi czarnymi
włosami zaczesanymi do tyłu. Raźno podniósł menu i~skinął głową nam obu.
Kontakty w~jego brązowych oczach sprawiały, że często mrugał, jakby
powietrze było pełne dymu. Zgasiłem papierosa.

-- Dobry wieczór, panowie. -- Podniósł piwo i~upił.

-- To jest Ian Cochrane -- powiedział Reid. -- Pracuje w~naszym
departamencie prawnym. Ian, to jest Jonathan Wilde.

-- Miło mi Pana poznać, panie Wilde. -- Jego uścisk był lepki, może od
wilgoci na kuflu, ale nacisk był pewny.

-- Jon. -- powiedziałem, kiwając głową i~zastanawiając się abstrakcyjnie
czy uścisk dłoni, jaki właśnie otrzymałem, był masoński.

-- Słyszałem dużo dobrego o~Tobie, Jon -- powiedział Cochrane. -- Pod
wrażeniem Twojego artykuł o~Brent Spar\footnote{spór
wokół platformy wiertniczej, którą właściciel Shell chciał zatopić na
pełnym morzu, a~Greenpeace zutylizować bezpiecznie dla środowiska,
zob.~\url{https://en.wikipedia.org/wiki/Brent_Spar } -- przyp.tłum.}. -- Złapał oko kelnera. -- Zamawiamy?

Jego akcent i~maniery miały ten szkocki ton wyższej średniej klasy,
który brzmiał bardziej brytyjsko niż Anglicy. Jadł selektywnie i~rozmawiał błaho, podczas gdy Reid i~ja zaspokajaliśmy nasz głód. Jego
drugi drink był wodą mineralną. Wtedy jego rozmowa przestała być błaha.

-- ,,Już czas, by ktoś wbił ludziom do głowy różnicę pomiędzy dnem Morza
Północnego a~dnem Północnego Atlantyku'' -- zaczął, cytując mój artykuł,
drobna kolumna w~niedzielnym wydaniu ,,Dissenting Voices'', z~pamięci. -- ,,Jedno to podłoga poważnie skażonej spiżarni, która powinna być
posprzątana. Drugie to Kufer Davy Jonesa\footnote{Kufer Davy Jonesa -- w~gwarze marynarskiej oznacza dno morza,
zob.~\url{https://pl.wikipedia.org/wiki/Davy_Jones_(legenda)
} -- przyp.tłum.}'' Ale nikt nie wbija do głowy,
to Twoja opinia, co? 

-- Tak -- powiedział, zbierając guacamole kawałkiem taco. -- Więc
morderstwo uchodzi Greenpeace na sucho.

-- Morderstwo w~rzeczy samej -- powiedział Cochrane. -- Ale kto uwierzy
kompanii naftowej przeciwko grupce bezinteresownych idealistów?

-- Ja -- powiedział Reid.

-- Ach, ale nie jesteś typowy, widzisz -- przypomniał mu Cochrane.
Odwrócił się i~popatrzył na mnie w~zamyśleniu. -- David, jak
prawdopodobnie wiesz, jest naszym kierownikiem IT. -- Skinąłem głową. Nie
wiedziałem. -- Był na spotkaniu komitetu, gdzie te kwestie były
przedstawiane. Nie byliśmy zamieszani w~tę katastrofę Shella, dzięki
Bogu, ale jako firma ubezpieczeniowa jesteśmy raczej wystawieni na
podobne sytuacje. Jeden z~naszych starszych kierowników zauważył,
mimochodem, że byłoby to bardzo\ldots sprzyjające wyrównanej publicznej
debacie, jeżeli były tam oddolne organizacje prowadzące kampanie
\emph{za }rozwojem przemysłu, zamiast przeciw, ,,Greenpeace dla dobrych
facetów'', zdaje się, że to nazwał. I~została podniesiona możliwość, hm,
materialnego wsparcia inicjatywy w~tym kierunku.

Reid pochylił się do przodu. 

-- Mam nadzieję, że nie masz nic przeciwko,
Jon, ale powiedziałem, że znam właściwego człowieka do tej roboty. Ciebie.

-- Żeby zacząć organizację antyekologiczną? -- Potrząsnąłem głową. -- Mają
takie w~Ameryce. ,,Rozsądne użycie'' i~takie tam. Są tubą dla wielkiego
biznesu. Przepraszam, chłopaki. Niezainteresowany.

Twarz Reida pokazywała tylko uprzejme zainteresowanie.

-- Dlaczego nie? -- spytał.

-- Zniszczę swoją wiarygodność.

-- Nie chcielibyśmy, żebyś mówił cokolwiek innego od tego, co już
powiedziałeś -- włączył się Cochrane.

-- Nie o~to chodzi -- powiedziałem. -- Mógłbyś zatrudnić niezależnych
naukowców, jakich chcesz, nawet dość rozsądnych ekologów. Wszystko, co
każdy musiałby zrobić, żeby ich skompromitować to przypomnieć ludziom,
skąd biorą się pieniądze. -- Sprawdziłem, że wszyscy skończyli już jeść i~zapaliłem papierosa. -- Spójrz na FOREST\footnote{FOREST -- Freedom
Organisation for the Right to Enjoy Smoking Tobacco -- organizacja
lobbująca przeciwko ograniczeniom w~użyciu tytoniu,
zob.~\url{https://en.wikipedia.org/wiki/FOREST } -- przyp.tłum.}. 

Skóra dookoła oczu Cochrane'a zmarszczyła się i~skinął głową, jakby żeby
dotrzymać miejsca. Kiwnął na kelnera i~zamówił kawę i~cygaretki.
Próbowałem odmówić cygaretki, ale nalegał, żebym przynajmniej zachował
je na później. Zdjął celofan z~własnej, zapalił i~rozkoszował się
pierwszymi zaciągnięciami z~najwidoczniej większym uznaniem niż ja.

-- FOREST -- powiedział -- ma znacznie większą wiarygodność w~mediach niż
Tobacco Advisory Council. Sprawdziliśmy. Są całkiem szczerzy na temat
tego, skąd mają większość pieniędzy. Nie kwestionują zagrożeń dla
zdrowia, tylko użycia ich do uzasadnienia różnego rodzaju natrętnych
restrykcji i~napastniczej propagandy. Co wcale nie trafia do mnie jako
zły przykład.

Zgasił cygaretkę i~powachlował ohydne chmury ręką. 

-- Paskudny nawyk -- zauważył, mrugając wściekle. -- Kwestia zasad.

Wzruszyłem ramionami. 

-- Ok, jeżeli w~ten sposób to widzicie, proszę
bardzo. Jednak nie zrobicie dużo, żeby zmienić opinię publiczną,
przynajmniej nie w~obecnym klimacie.

-- \emph{Panie} Wilde -- powiedział Cochrane zawiedzionym tonem -- nie
mówimy o~\emph{obecnym klimacie}. Mówimy o~zmianie klimatu.

-- Chcecie zostać kozłem ofiarnym globalnego ocieplenia?

Cochrane roześmiał się pobłażliwie. 

-- Touche\ldots ale serio, stracimy
dobrą umowę, jeżeli fatalne prognozy okażą się prawdziwe, więc nie, nie
mamy celu w~minimalizowaniu tego. Po prostu chcielibyśmy jaśniejszego
odbioru społecznego tych spraw. Co do klimatu opinii\ldots North British
Mutual Assurance istnieje w~tej lub innej formie od przed Rewolucji. -- (Od \emph{kiedy}?) -- Jeżeli to prawda, wcześniejsze kompanie nie miały
niewiele wspólnego z~faktem, że Rewolucja była pokojowa i~chwalebna, i~wszystkie te piękne słowa, które historia dodała do ściśle biznesowego
przejęcia 1688 -- (W tej chwili mój umysł nadążył). -- Więc pozwól, że
przedstawię propozycję, na podstawie, której, jeżeli dama przy
najbliższym stoliku zdarzy się być dziennikarzem \emph{Scotsman}, ta
rozmowa niezaprzeczalnie się nie wydarzyła, i~poza tym\ldots raczej nie.

Zachichotał się mrocznie i~pomimo obaw, w~które byłem wciągany, częścią
jego intrygi.

-- Jako ubezpieczyciele -- kontynuował, cichszym głosem -- nie mamy
żadnych korzyści w~popieraniu zanieczyszczających, ponieważ, jak pokazały
to firmy azbestowe, są złym ryzykiem. Jesteśmy najbardziej zdecydowanie
\emph{zainteresowani} dobrobytem, wzrostem i~klientami, którzy płacą
swoje opłaty w~długich i~zdrowych życiach. Więc jeżeli ktoś chciałby
utworzyć organizację, taką jak rozmawialiśmy, nasze zainteresowanie
byłoby całkiem jawne, i~całkiem do wybronienia przez obie strony.

-- O ile przedstawione we właściwy sposób -- powiedział Reid. -- Myślę, że
jest to w~obrębie Twoich możliwości.

-- Dziękuję -- powiedziałem. -- To nie mogłoby wyglądać bardziej
złowieszczo niż dawanie pieniędzy na Torysów. Prawdopodobnie mniej.

Cochrane kaszlnął. 

-- Jak to się zdarza, nasze polityczne darowizny w~tym
roku...

Nasze cyniczne rechoty przerwały mu. Po chwili się dołączył.

-- Tak, cóż, \emph{jesteśmy} w~biznesie rozprzestrzeniania ryzyka.

-- To jest coś -- powiedziałem -- widzieć inteligentne pieniądze\footnote{dosł.
smart money -- doświadczeni, dobrze poinformowani inwestorzy traktowani
jako grupa, zob.~\url{https://en.wiktionary.org/wiki/smart_money } -- przyp.tłum.}
zmieniające strony, prawie przed Twoimi własnymi oczami. 

-- W~rzeczy samej -- powiedział Cochrane. -- I~mógłbyś spojrzeć na naszą
propozycję jak na coś podobnego, o~ile nie w~dłuższej skali czasowej.

Pokręciłem głową. -- Przepraszam, ale nie rozumiem. -- Myślałem, że
rozumiem, ale ledwie odważyłem się wierzyć.

Cochrane uniósł brew ku Reidowi, który lekko skinął głową.

-- Przeglądałem trochę literatury, którą wysyłałeś do Dave'a przez lata -- powiedział Cochrane. -- Pośród wszystkich tych śmieci, zawiera raczej
stymulujące idee na temat możliwej roli firm ubezpieczeniowych w~zapewnianiu bezpieczeństwa ich klientom. Co do ideału politycznego\ldots -- Strzepnięcie dłonią w~powietrzu. -- Jednak, jako strategia rynkowa
radzenia sobie z, hm, pewną ucieczką państwa od czegoś, co było
dotychczas jego odpowiedzialnością, ma określone wdzięki. Nie mówiąc już
o...

I nic nie powiedział. Jego oczy straciły się w~tiku mrugania i~spojrzały
stabilnie na mnie.

-- Kolejna mała przerwa w~gładkim biegu brytyjskiej historii? -- spytałem.

Trzeźwo pokiwał. 

-- Spekulacyjnie, oczywiście. Ale możemy pewnego dnia
rozważać naszą pozycję w~stosunku do tego, co erudyta pan Ascherson
zachwyca w~nazywaniu \emph{reżimem hanowerskim}. Pomyśl o~tym jako...

-- O ubezpieczeniu -- powiedział radośnie Reid.

Spojrzałem od jednego do drugiego i~zapaliłem papierosa, poruszając
dłońmi bardzo ostrożnie, żeby utrzymać je spokojne.

Do tej chwili myślałem o~sobie jako o~odpornym na przepych władzy, w~dokładnie ten sam sposób, w~jaki eunuch mógłby być odporny na czar
kobiet. Nigdy nie stałem na baczność przy hymnie lub przy fladze, nigdy
niczego nie wsuwałem niezdarnie do urny wyborczej. Nastawienie, dzięki
któremu sekta moich rodziców odzyskała drwiący przydomek
,,imposybilistów'', zostało, jak mi się zdawało, odziedziczone w~mojej
własnej politycznej postawie. Och, chciałem mieć \emph{wpływ}, zmienić
sposób myślenia ludzi, tak jak moi rodzice. Ale, -- tak jak oni -- nigdy
poważnie nie oczekiwałem na możliwość faktycznie wzięcia władzy w~ręce.

W skrócie, byłem kompletnym palantem, aż do tej chwili, kiedy
dowiedziałem się, czego mi brakowało. I~wiecie, co wtedy czułem
\emph{było} prawie seksualne. To coś w~okablowaniu męskiego mózgu
naczelnych.

Największym dreszczykiem nie było, że oferowali mi władzę, proponowali
mi tylko nieco więcej wpływu, to wszystko. Nie, to, co stawiało mi włosy
na karku, było to, że myśleli, że mogę, w~przyszłych dekadach,
\emph{mieć} władzę, że mógłbym przedstawiać coś, co było warte, by
stanąć za tym z~dużym wyprzedzeniem, że gdzieś w~tle mógł zaistnieć mój
Fiński Dworzec\footnote{Dworzec Fiński -- dworzec w Petersburgu znany m.in. z~tego, że Lenin, podróżując ze Szwajcarii, przybył na niego 3 kwietnia 1917. Miejsce to
jest również wymienione w~tytule książki Edmunda Wilsona ,,To the
Finland Station'' (1940) -- przyp.tłum.}.

-- Tylko jedno pytanie -- powiedziałem. -- Jest wielu bardziej znanych i~lepiej powiązanych ludzi z~poglądami podobnymi do moich, więc dlaczego
ja?

Reid spojrzał, jakby miał coś powiedzieć, ale Cochrane przerwał mu.

-- To dlatego, że nie masz powiązań z~żadną częścią obecnego
establishmentu, i~nie chcielibyśmy, żebyś jakiekolwiek utrzymywał. Twoje
poglądy na kwestię ziemi i~systemu bankowego są odrzucane jako
całkowicie błędne przez każdy wolnorynkowy think tank, jaki
konsultowałem. Twoje polityczne znajomości są takie, że Twoja teczka w~MI5 i~Special Branch jest, jak rozumiem, chwalebnie gruba. Twoje
internetowe artykuły o~ostatnim skandalu w~Oklahomie, o~Czeczenii, o~Bośni, dodały FBI, CIA i~FIS do Twoich uważnych czytelników. Więc, jak
widzisz...

-- Widzę, w~porządku -- powiedziałem. -- Chcesz kupić kogoś, kto wygląda,
jakby nie był kupiony.

-- Chryste, człowieku...! -- Reid zaczął, ale znowu Cochrane mu przerwał.

-- Wybaczcie mi chłopcy -- powiedział, strzepując ziarenka chili z~palcy.
-- Nigdy nie miałem radykalnego sumienia, żeby z~nim walczyć, i~całkiem
szczerze byłbym ciężarem dla własnej sprawy w~tego rodzaju dyskusjach,
które potrafię przewidzieć, że się rozwiną. -- Uśmiechnął się krzywo,
prawie przepraszająco do nas. -- Więc jeżeli nie macie nic przeciwko,
zostawię was tutaj.

Wstał, wyciągnął dłoń, a~ja wstałem, żeby potrząsnąć, złośliwie
odpowiadając jego dziwnym chwytem. 

-- Miłego wieczoru, Jon, i~mam
nadzieję, że się jeszcze zobaczymy.

-- Cóż, wzajemnie Ian.

Skinął głową Dave'owi i~odszedł.

~

Dave pozostawał w~ciszy, póki Cochrane nie był za drzwiami. Potem
położył łokcie na stole, palce przy policzkach, grzbiety dłoni prawie
dotykające się przed jego ustami.

-- W~co ty do kurwy nędzy pogrywasz? -- domagał się.

-- Nic -- powiedziałem. -- Naprawdę. Nie oczekiwałeś, że będę podskakiwał
na szansę bycia radykalnym liderem dla jakiejś ferajny garniturów
zmartwionych tym, co się stanie, kiedy ich obecny wygodny układ zostanie
spuszczony z~wodą?

-- Co za jebany idiota -- powiedział Dave, nie nieżyczliwie. -- Jesteś
ostatnią osobą, po której się spodziewałem\ldots ach, do cholery z~tym.
Chodźmy do pubów.

W dogodnie bliskim Malt Shovel, pozwolił mi postawić mu kufel Caffrey i~powiedział mi o~swoim planie na resztę wieczoru.

-- Chcę Ci pokazać niektóre z~moich ulubionych pubów -- wyjaśnił. -- Tylko
jeden sposób na to, runda po pubach w~transporcie zbiorowym. Tutaj, Cafe
Royal, szybki kieliszek w~barze stacji, potem do Haymarket, następnie
pociągiem do Dalmeny, wzdłuż frontu South Queensferry, potem ostatni
autobus mostem do Dunfermline.

Dunfermline. Zaadresowałem wiele paczek tam do jego mieszkania, ale
miałem niejasne poczucie, że to przedmieścia Edynburga. Źle: ponad
Forth, najwidoczniej. Mój mentalny obrazek zmienił się wobec zasięgu gór
Highland.

-- Jesteś pewien, że mamy czas?

Odstawił pustą szklankę. 

-- Zobaczymy, jak daleko dojedziemy.

Prawie zbiegliśmy Cockburn Street, przez Waverley Bridge, potem znowu
dookoła tyłem Waterstone i~Burger Kinga do wielkiego pubu, który wydawał
się posiadać tylko jedno wejście z~boku. Wysoki sufit, kafelkowe ściany,
murale, skórzane siedzenia, marmur, wypolerowany mosiądz i~drewno.

-- Istny pałac ludowy -- zauważyłem, gdy siadaliśmy. -- To jakby coś z~tego
Twojego zdegenerowanego państwa robotników.

Reid się uśmiechnął. 

-- Piwo byłoby tańsze.

-- Tak -- powiedziałem. -- Zobacz, co zrobili Budweiserowi?

-- Szokujące -- odpowiedział Reid. -- Powinno być na to prawo.

Kiwnąłem na murale. 

-- Bohaterowie Rewolucji Przemysłowej\ldots czy to Watt?
Stevenson?\ldots powinni mieć taki z~Adamem Smithem widzącym niewidzialną
rękę.

-- Realizm kapitalistyczny -- powiedział Reid.

-- Coś, w~co jesteś zamieszany, najwyraźniej.

-- Tak, cieszę się, że mogę tak powiedzieć. -- Reid oparł się, rozciągając
się w~swoim siedzeniu. -- To jedyna gra w~mieście.

-- Ta, cóż, powinieneś wiedzieć.

-- Cholerna racja! -- powiedział mocno. -- Nie zmieniłem moich poglądów,
długoterminowo, ale rozpoznaję klęskę, kiedy ją widzę. Poradzenie sobie
z końcem Drugiego Świata\footnote{ powstałe w~okresie zimnej wojny określenie
państw socjalistycznych (i ich sojuszników) przeciwstawnych państwom
kapitalistycznym Europy i~Ameryki Północnej (Pierwszy Świat),
więcej~\url{https://pl.wikipedia.org/wiki/Drugi_\%C5\%9Awiat
} -- przyp.tłum.} zabierze pokolenia, i~to nie będą nasze pokolenia.
Ostatni raz, kiedy spędziłem czas z~lewicą, był w~trakcie Wojny w~Zatoce. Dzieciaki gówno wiedzą, a~starsi\ldots -- uśmiechnął się nagle jak
Dave, którego znałem -- \ldots to jest, starsi niż my, wyglądają jak ktoś,
komu powiedziano, że ma raka.

-- I~nie może przestać palić, co?

-- Ha! Ok, Jon, ciągle mamy interes do załatwienia.

-- Wal śmiało.

-- Brutalna uczciwa prawda jest taka, że mało prawdopodobne jest, że
dostaniesz lepszą ofertę. Nie oszukuj się, człowieku. Masz czterdzieści
lat, jesteś nikim i~zmierzasz donikąd. Szanse są takie, że przez resztę
swojego życia będziesz szukał śmieci z~kosmosu przy konwentach SF i~zapomnianych idei w~skrajnych organizacjach.

Wzruszyłem ramionami. 

-- Są gorsze sposoby na życie.

Dave pochylił się ku mnie, prawie wbijając mi papierosa w~twarz przy
podkreśleniu. 

-- Ale są lepsze, do cholery!

-- Wiem, wiem. Jednak dotrę tam na swój sposób. Cała ta sprawa z~wolnym
rynkiem ma długą drogę przed sobą, a~nawet kosmos staje się znowu modny.
Ludzie chcą zobaczyć ten nowy film, jak mu tam? \ldots \emph{Apollo 13}, i~pomyślą ,,Hej, zrobiliśmy to wtedy! Dlaczego nie możemy teraz?'' Zachód
wróci do kosmosu dostatecznie szybko, kiedy będą się ścigać z~Chińczykami. Lub \emph{ktoś} zaszokuje nas w~stylu Sputnika. I~popatrz,
nawet Cochrane wydaje się myśleć, że na coś wpadłem.

-- Aaach! -- Nieartykułowany dźwięk Dave'a niósł wagę sceptycyzmu
Highland. -- To było dziewięćdziesiąt dziewięć procent bredni i~pochlebstw. Może jeden procent wypatrywania nieprzewidzianych wydatków.

-- Pewnie, ale wolałbym mieć ten jeden procent niż się wyprzedać.

-- Przestań kurwa myśleć o~tym jako o~wyprzedaniu! Chryste, wziąłbym
pieniądze od Nirex lub Rio Tinto Zinc\footnote{NIREX -- podmiot zajmujący się przygotowaniem składów odpadów
radioaktywnych, zob~\url{https://en.wikipedia.org/wiki/Nirex
Rio Tinto} -- międzynarodowa firma zajmująca się górnictwem, która m.in.~korumpowała lokalne rządy i~naruszała prawa człowieka,
więcej~\url{https://en.wikipedia.org/wiki/Rio_Tinto_(corporation)
} -- przyp.tłum.}, gdyby dali mi wolną rękę. To \emph{jest} droga na Twój własny sposób. To wszystko
jest legalne. Rzetelnie i~uczciwie...

Zauważył, co mówi, i~się roześmiał. 

-- Ok, stary Ian jest biznesie, ale
to nie ma nic z~tym wspólnego!

-- Ta, dobra, jakoś wytrzymam Illuminati\ldots Więc to jest umowa, tak? Dają
pieniądze i~robię, co chcę z~nimi?

-- Żadnych kłopotów, o~ile masz wyniki.

-- Mierzonych jak?

-- Och, odparcia, czas antenowy, expose skąd \emph{ekolodzy} dostają
swoje cholerne pieniądze. Rodzice robiący zamieszanie o~propagandę
Zielonych w~szkołach. -- Zmienił wymowę na podobieństwo akcentu
angielskiej klasy pracującej lub przynajmniej w~ciągle pokrzywdzony ton.
,,Za moich czasów nie nazywaliśmy tego niszczeniem lasów deszczowych,
nazywaliśmy to oczyszczaniem dżungli, i~myślę, że powinno tu być nieco
\emph{równowagi}, wiesz, co mam na myśli?''

Zaczynało to brzmieć całkiem atrakcyjnie. To i~myśl o~braku wykładów z~podstaw ekonomii. Wejść na własną krzywą popytu zamiast...

-- Lasy deszczowe należą do ich mieszkańców -- powiedziałem. -- Wywal
prawodawstwo środowiskowe, tak, ale tylko jeżeli zanieczyszczający
zapłacą za szkody, ścisła odpowiedzialność. To mój program. Myślisz, że
to kupią?

Reid wzruszył ramionami. 

-- Mógłbyś spróbować.

-- Ok -- powiedziałem, nagle zdecydowany. -- Przekaż mi szczegóły i~jeżeli
wszystko będzie tak proste, jak mówisz, wejdę w~to.

-- Na pewno?

-- Tak.

-- Cóż, dzięki kurwa za to. Myślałem, że wbicie Tobie trochę rozsądku do
głowy zajmie mi całą noc.

Na stacji mieliśmy kilka wolnych minut, nawet z~łykiem whisky w~Wayfarer
Bar, więc zadzwoniłem do domu.

-- Cześć kochanie.

-- Hej, miłości. Gdzie jesteś?

-- Waverley Station. Reid zabrał mnie na rundę po pubach pociągami.

-- Hm, trzymaj się. Nie mogę się doczekać jutrzejszego wieczoru.

-- Ja też! -- Elektryczny cmok. Trochę gadania o~Worldconie i~egzaminach
szkolnych Eleanor, potem spytała:

-- Dużo sprzedałeś?

-- Tak -- powiedziałem. -- Bardzo dużo.

~

Zabrałem torbę z~przechowalni bagażu (pozostałe zasoby z~mojego kramu
były w~tym momencie w~drodze autostradą w~vanie należącym do
zaprzyjaźnionej księgarni SF w~Londynie). Wsiedliśmy do pociągu na jeden
przystanek, wypiliśmy kilka piw w~Caledonian Ale House w~Haymarket i~złapaliśmy kolejny pociąg dalej.

Dalmyn było parą opuszczonych peronów z~zaskakującym widokiem na Forth
Bridge, światła wysyłające widmowe słupy w~ciemniejące niebo. Road
Bridge stało okrakiem na cirrusach podświetlonych zachodem słońca. Dave
poprowadził mnie wąską, ciernistą ścieżką pomiędzy polami i~nasypem
kolei, przez wzniesienie i~drewniany brzeg, potem w~dół długim zejściem
po drewnianych schodach na brzeg Firth. Ostry skręt w~lewo na dole
poprowadziło nas do Hawes Inn, pubu, którego uroki były tylko lekko
pomniejszone przez kilka maszyn do grania i~wiele nieodpowiednich
cytatów z~Roberta Louisa Stevensona na ścianach.

Znaleźliśmy siedzenie przy oknie, w~rogu przy maszynach do grania.
Gwiezdne bitwy ryczały koło nas.

-- Tutaj zatrzymał się Rzym -- zauważył Reid tonem dziwnie osobistej
satysfakcji, gdy patrzył nad Firth.

-- Nie może być -- powiedziałem. -- Czy to nie katolicy z~Highland...

-- \emph{Imperium} Rzymskie -- wyjaśnił Reid. -- Tu było najdalej na
północ, gdzie dotarli: \emph{limes}. Zmasakrowali widocznie tubylców w~Cramond. Poza Firth nie zrobili nic prócz tracenia legionów na całej
mapie, o~to chodzi.

-- He! -- Podniosłem pintę Arrola. -- Za koniec imperiów.

-- Zdrowie. -- Skinął głową Reid. -- Jednak, to w~pewien sposób robi
wrażenie. Cała ziemia stąd do drugiego końca Śródziemnomorskiego pod
jednym rządem.

-- Hmm\ldots ktoś powinien ostrzec eurosceptyków: robiono tak i~\emph{trwało
to przez tysiące lat}! -- To w~komicznym niemieckim skrzeku rozproszyło
uwagę jednego z~wojowników kosmosu na tyle, żeby spojrzał na mnie i~stracił kilka statków atakującego imperium zła na ekranie. Myślę, że w~tym momencie byłem nieco pijany.

Nasz postęp trwał przez Two Bridges, Anchor, Ferry Tap. Na zewnątrz
Queensferry Arms Reid się zawahał i~powiedział: 

-- Miniemy ten. Mam lepszy pomysł. -- Zaprowadził mnie kilka kroków wzdłuż wąskiej High
Street do chińskiej restauracji na wynos, gdzie obiecał mi najlepszy
przysmak w~menu.

-- Dwie porcje frytek w~curry, proszę.

-- Frytki w~curry? -- spytałem z~niedowierzaniem.

-- Właśnie tego potrzebujesz po kilku piwach.

Dziewczyna za ladą podała nam je z~czymś, o~czym myślałem mętnie jako o~protekcjonalnym uśmiechu. Jedząc je parujące, lepkie, tłuste małymi
plastikowymi widelcami, przeszliśmy koło komendy policji i~tego, co Reid
opisał jako kościół jakobitów, do ostatniego pubu, zatrzymując się,
tylko żeby wyrzucić w~zamyśleniu nasze śmieci za głównym wejściem.

Wpadliśmy do Mooring z~oddechem jak smoki. Dziewczyna za barem właściwie
odwróciła twarz, gdy podawała nam piwo. Podążyłem za Reidem od baru do
tylnej części, gdzie szerokie okna przedstawiały piękny widok na Bridge.

Pub był nowy, imitacja starego, sprzęt nurkowy i~oprawione rysunki
okrętów wojennych na ścianach. W~trakcie naszych podróży, otwierająca
myśl Reida o~Imperium Rzymskim zamieniła się w~długą i~zaangażowaną
dyskusję o~imperiach w~ogólności, z~Reidem stanowczo po ich stronie.
Nienawidził domyślnej opcji rozczarowanych socjalistów, nacjonalizmu.

-- Popatrz na nie -- powiedział, otwierając trzecią paczkę papierosów i~wskazując na grafiki okrętów. -- Widzisz je. One, one nas uratowały,
racja? Od niemieckich faszystowskich barbarzyńców. I~od starego dobrego
Wujka Joe, jeżeli być uczciwym.

-- To -- powiedziałem, próbując ustabilizować go w~moim polu -- jest nieco
uproszczony glond. Pogląd. Jestem zdziwiony.

-- Tak jak i~ja -- powiedział. -- Kilka lat temu, była tutaj ekspozycja,
Harriery lecące do tyłu, Sea King robiące pętle i~takie tam, i~uświadomiłem sobie, że jestem \emph{dumny }z tych facetów. Tak jak byłem
dumny z~heroicznej Armii Czerwonej i~Wietkongu.

-- Jezus. -- Byłem zszokowany w~przemijającym przypływie trzeźwości. -- Chcesz mi powiedzieć, że siły zbrojne państwa brytyjskiego są
\emph{bojownikami o~wolność?} Jestem pewien, że, na przykład,
Irlandczycy mogą mieć odmienną opinię.

-- Och, jebać Irlandczyków -- powiedział Reid, szczęśliwie nie za głośno.
-- Muszę przyznać, że przez lata miałem obawę o~śmiałą IRA. A potem
przyszli, odwrócili się na palcach, po prostu rzucili to wszystko jak
pierdoleni staliniści.

-- Ale zawsze chciałeś czegoś lepszego niż to...

Spojrzał w~swoje Caledonian Eighty. 

-- Nawet jeśli, trzymałem stronę
państw robotniczych. A wtedy one wszystkie upadły jak\ldots jak domino! To
nie ja jestem dezerterem. Mam na myśli, moja strona się poddała, racja?
Więc mogę robić, cokolwiek kurwa chcę.

Dzwonek zadzwonił na ostatnie zamówienia. Reid roześmiał się i~wysuszył
szklankę. 

-- To samo jeszcze raz?

-- Tak, poproszę.

Wrócił z~dwoma piwami i~dwoma kieliszkami whisky. Whisky może ponosić
pewną odpowiedzialności za to, co zdarzyło się potem.

-- Więc co masz do powiedzenia na to? -- spytał.

-- Schlanzhe\ldots ok, ok. Mówisz, że kiedyś podziwiałeś armie drugiej
strony, racja? Więc o~co chodzi z~tą całą walką o~pokój, co? Co z~Kampanią na rzecz rozbrojenia jądrowego?

Walka o~pokój, rozbrojenie\ldots coś mnie męczyło.

-- Taktyka. Komuniści byli prawdopodobnie szczerzy, dostatecznie zabawni,
ale o~ile byliśmy zaangażowani, postrzegaliśmy rozbrojenie jako
zakłócenia dla Ruskich.

-- Bez jaj?

-- Bez jaj.

-- Dobra -- powiedziałem, zaskoczony tym bezczelnym przyznaniem -- muszę
powiedzieć, że Twój dopiero co odkryty patriotyzm ma podejrzaną krzywą,
jak przejście z~jednego błędnego poglądu na coś, co wydaje się
całkowitym przeciwieństwem, ale faktycznie jest w~tym samym miejscu...

-- Bzdura. Nie jestem patriotą. To, co mówię to, żyjemy w~niebezpiecznym
świecie i~nie mam zamiaru udawać, że nie wiem, czyje karabiny mnie
chronią.

-- Co z~ludźmi po drugiej stronie karabinów?

-- Mocne. Mam szczęście, że jestem po tej stronie. W~porównaniu do
wszystkiego innego, to strona postępu. \emph{My} jesteśmy obozem
rewolucji.

-- Wyjaśnij, proszę.

-- Ponieważ Twoje libertariańskie jankeskie głupki mają rację, demokracje
Zachodu \emph{są} socjalistyczne! Wielkie sektory publiczne, wielkie
firmy, które planują produkcję, podczas gdy oficjalnie wszystko jest na
rynku\ldots swego rodzaju \emph{czarne} planowanie, tak jak Wschód miał
czarny rynek. Marks powiedział, że powszechne prawa wyborcze są rządem
klasy pracującej i~miał rację. Zachód jest czerwony!

Musiałem się roześmiać, nie tylko z~zuchwałości racjonalizacji Reida,
ale z~odrobiny dziwnej prawdy. Rozważaliśmy tę teorię, gdy wyszliśmy z~pubu i~szliśmy w~kierunku Road Bridge.

-- Gówno -- powiedział Reid, studiując rozkład jazdy -- nie zdążyliśmy.
Pierdolone prywatne firmy ciągle zmieniają usługi.

-- Pieprzeni towarzysze na drodze do kapitalizmu\footnote{dosł. capitalist
roaders, termin maoistowski opisujące osoby, lub grupy osób, sprowadzające rewolucję
w kierunku kapitalizmu, por.~\url{https://en.wikipedia.org/wiki/Capitalist_roader } -- przyp.tłum.}. Zamówmy taksówkę.

-- Stąd? Nie. Po drugiej stronie jest hotel. Zadzwońmy stamtąd.

Spojrzałem wzdłuż jasnego kilometra mostu.

-- Kawałek drogi.

-- Może mają otwarty bar -- powiedział przebiegle Reid.

-- Jestem za.

Ruszyliśmy, koło znaków obwieszczających, że kamery ochrony cały czas
obserwują most. Na północy i~zachodzie na niebie ciągle było światło.
Samochody i~ciężarówki brzęczały obok, co minutę. Część mostu, zanim
sięgnęła rzeki, wznosiła się powoli nad ulicami, podwórkami, śmietnika i~długimi ramionami przystani. Po naszej lewej, pomiędzy chodnikiem a~rzeką, była wysoka barierka, niższa, ale szersza pomiędzy chodnikiem na
drogą. Reid utrzymywał szybkie tempo, niedużo mówiąc. Gdzieś w~połowie
zatrzymałem się, żeby zapalić czarne hinduskie cygaro, które
(niewytłumaczalnie w~tamtej chwili) znalazło się w~mojej kieszeni.

Coś na myśli. Walka o~pokój, coś z~\ldots ach!

Zły czas, ale też, nigdy nie byłby dobry czas.

Pośpieszyłem, żeby go dogonić.

-- Reid, stary przyjacielu -- powiedziałem zza jego ramienia -- mam z~Tobą
do pogadania.

Jego ramiona drgnęły Nie odwrócił się. 

-- Ok, człowieku. Cokolwiek.

-- Cóż, prawda jest taka, że Annette powiedziała mi o, wiesz, Tobie. I~jej.

-- Och! -- Zatrzymał się i~odwrócił się do mnie.

Zatrzymałem się, opierając o~balustradę. Dziesiątki metrów poniżej, woda
błyszczała jak młotkowany ołów. Reid wygrzebał papierosa, upuścił,
podniósł i~zapalił.

-- Co mogę powiedzieć? -- powiedział. Rozłożył ręce, zakołysał i~położył
prawą rękę na balustradzie. -- Zdarzyło się, jaki jest sens zaprzeczania,
i to była moja wina, i~jest mi przykro.

-- W~porządku -- powiedziałem. -- To wszystko, co powinieneś powiedzieć.

-- Ty\ldots -- Zaciągnął się mocno papierosem, schował go żarzącego w~lewej
ręce. -- Jesteś dobrym facetem, Jon. Zasługuje na Ciebie. A Ty
zasługujesz na kogoś lepszego niż ja. Nadużyłem Twojej\ldots gościnności,
chłopie. Nie ma usprawiedliwienia, prócz tego, że to było tylko
ruchanie...

Jego głos zamilkł i~odwrócił wzrok, w~dal.

-- \emph{Tylko ruchanie}?

-- \ldots obsesja, chłopie, to jest to słowo. -- Roześmiał się cierpko. -- Chciałbym \emph{móc} powiedzieć, że to było tylko ruchanie.

Spojrzał z~powrotem na mnie. Dym w~moich ustach nagle był cuchnący.
Posłałem czerwony żar, kręcąc nad bokiem i~obserwowałem długi wolny
upadek.

-- Ale nie mogę -- kontynuował. -- Nie mówię, że to nie było złe, ale w~tym
było więcej niż to. Raz nawet próbowałem ją zachęcić do zostawienia
Ciebie, jeżeli możesz w~to uwierzyć. Jednak nie chciała i~miała rację,
to był koniec tego. Zero. I~pogodziłem się z~tym tak jak i~ona.

Od tej chwili wiedziałem, że jestem zdolny do morderstwa. Miał jedną
rękę na balustradzie, jedną z~boku ciągle trzymając papierosa. Znowu
wpatrywał się w~dal. Chwyt za kołnierzyk i~pasek, jedno dobre poderwanie
i byłby po drugiej stronie. To byłoby proste, a~ja mógłby to zrobić.

Odwrócił się do mnie. 

-- To wtedy ci powiedziała, racja? -- W~jego oczach
był jakiś podziw i~spryt. -- Wiem, ponieważ to wtedy całe to prawicowe
gówno zaczęło się pojawiać, od Contras, Renamo, emigrantów
wschodnioeuropejskich, KMT i~NTS. Mieszanie tego ze starym komunistami i~libertarianami było zgrabną sztuczkę, ale wiadomość do mnie dotarła.
Znasz pewnych twardzieli, a~oni wiedzą, gdzie mieszkam. -- Roześmiał się
cierpko. -- Muszę Ci to przyznać Jon, przestraszyłeś mnie.

Zrobiłem krok w~jego kierunku i~uderzyłem go prosto w~usta. To był dobry
cios, moje dziecięce lekcje boksu nie zostały zmarnowane, a~on
zareagował beznadziejnie wolnym, wiejskim, hakiem.

Niemniej jego dotarł, a~mój nie. Uderzyłem o~balustradę. Górna krawędź
uderzyła w~dolną klatkę piersiową i~nagle przechylałem się ponad nią,
patrząc w~dół. Prosto w~górę, przez nierealną chwilę, gdy moje kanały
półkuliste przekręciły się i~kosmos podążył za nimi.

A potem poczułem się chory. Danie meksykańskie, kilkanaście piw, dwie
whisky, porcja frytek w~curry, i~smoła z~kilku SilkCutów i~meksykańskiej
cygaretki wylała się przez moje usta i~nos w~kaskadzie, która spryskała
chodnik, balustradę i~zaniepokoiła grzędujące ptaki, zanim spadła, z~dosłownie chorobliwym spowolnieniem, widoczna cały czas, do wody.

-- Wszystko w~porządku.

Odepchnąłem się od balustrady.

-- W~porządku -- powiedziałem. Wydmuchałem fragment taco i~kawałek
pikantnego szlamu z~lewego nosa na palce, potem zwinąłem pięść do
kolejnego uderzenia.

Jego oczy się rozszerzyły, ale patrzył koło mnie. Zapiszczały hamulce.
Van zatrzymał się koło nas, na chodniku, nie na drodze.

Drzwi się otworzyły i~mężczyzna w~kombinezonie się wychylił.

-- Chodźcie, chłopaki -- powiedział. -- Trzymaliśmy was dwóch na oku.
Wyglądacie, jakbyście potrzebowali podwózki.


\part{PODBÓJ PRZEMOCY}

\chapter{Ocena Okręgu}

Jest wczesne popołudnie i~zegary pikają co kwadrans. Dee podąża za Ax
przez wysoki, wąski most. Chodnik jest ledwie metr szeroki, balustrady
nieco powyżej metra wysokości. Poniżej jest stumetrowy spadek na dachy
niższego poziomy. Powyżej, wznoszą się wyższe wieże. Most delikatnie się
wznosi, zakrzywia się gładko na prawo. Dee idzie po nim bez strachu. To
jest dla niej znajome terytorium, wysoka lokalizacja w~życiu na wysokim
poziomie tych, którzy w~Mieście Statku uchodzą za bogatych. Szczęśliwie,
jednak, nigdy nie poznała Andersona Parrisa, mężczyzny, do którego
rezydencji się zbliżają.

Dee nie wątpi, że zanim skończy się następna godzina, zabije ludzką
istotę. Nigdy tego nie zrobiła i~perspektywa wzbudza w~niej pewną
ciekawość. Umiejętności są, oczywiście, w~Szpiegini i~Żołnierce. Jednak
pamięta plotki, z~poprzedniego życia (z jej życia sprzed obudzenia),
które sprawiają, że się zastanawia, czy ma dostęp do tych szczególnych
umiejętności. Jeżeli Sys zmieniła \emph{uprawnienia}\ldots Nie ma jak tego
sprawdzić, ponieważ samo to jest częścią Sys, do której nie ma dostępu.
Przypomina sobie ludzi mówiących, mówiących jakby jej tam nie było, o~potencjalnych niebezpieczeństwach AI wędrujących w~przebraniu ludzi, i~wie, że ludzie przykładają wielką wagę do \emph{uprawnień.}

W ogóle nie wątpi, że Ax będzie w~stanie to zrobić. Ax jest człowiekiem
i ludzie nie potrzebują żadnych \emph{uprawnień}. Dee drży, ale nie ze
strachu lub podniecenia. Wiatr na tej wysokości jest chłodny, a~jej nowe
ubranie, nawet w~zielonym aksamitnym płaszczu, nie daje dużo ciepła.

~

Drzwi są jasne, lekko wypukła stal, cofnięte w~syntetycznej skale
budynku. Dee podziwia jej zniekształcone odbicie, ćwiczy przekształcenia
na nim, podczas gdy Ax wymienia kilka słów z~siatką głośnika. Drzwi
otwierają się gładko na boki, a~Ax i~Dee wchodzą. Hol wejściowy ma
ściany nachylone do wnętrza, a~krzywizna w~prawo schodów kontynuuje tę z~mostu, dalej w~budynek. Hol jest oświetlony przez wysoki świetlik i~wysokie okna w~zewnętrznych ścianach. Światła elektryczne wiszą na
różnych poziomach z~sufitu wysokości dziesięciu metrów i~z podobnie
zawieszonych mis wylewają się liście, łodygi, kwiaty i~zapachy.

Drzwi za nimi się zatrzaskują. Dee patrzy do tyłu przez chwilę,
sprawdzając, czy mogą być otwarte ręcznie od środka. Wygląda na to, że
mogą, ale subtelniejsze zmysły Szpiegini już pracują, śledząc wzorce
drgań w~przewodach w~ścianach, na wszelki wypadek. Stopy Ax uderzają,
obcasy Dee stukają na łuku korytarza. Drewniane drzwi wyjściowe z~korytarza są zamknięte. Kiedy Dee i~Ax dochodzą do punktu, kiedy nie
widzą już drzwi wejściowych, korytarz rozszerza się w~klatkę schodową.
Kilka kroków wyżej na spiralnych schodach stoi mężczyzna czekając. Jest
ubrany w~czarne kimono wyszywane widokami nocnego nieba. Jego jasne
włosy są zaczesane do tyłu znad wysokiego czoła. Twarz jest wąska, usta
cienkie, brwi jasne, mina pogodna. Dla Dee jego gładkie i~zdrowe cechy
wyglądają na starość, jest znacznie starszy niż ona lub Ax, prawie tak
stary jak Reid. A jednak wskazują one na jakąś głęboką niedojrzałość,
jak również na okrucieństwo, które Dee natychmiast dostrzega jako różne
od zimnej bezwzględności, która była najgorsza, jaką kiedykolwiek
zdradzał -- nawet teraz w~odtwarzanych wspomnieniach -- Reid w~najbardziej
niebacznych momentach. Ten człowiek nie jest jak Reid ani żaden z~jego
przyjaciół, czy codziennych znajomych. Żaden krzepki biznesmen, który
oglądał ją na spotkaniach, lub macał na imprezach, nigdy nie sprawił, że
poczuła się w~ten sposób jak teraz, gdy jego wzrok ją bada.

Anderson Parris schodzi po schodach i~uśmiecha się do Axa.

-- Cóż, witaj -- mówi, łapiąc dłoń Axa. -- Cieszę się, że cię widzę, i~Twoją bardzo interesującą i~piękną przyjaciółkę.

Dee otwiera zapięcie przy gardle i~zdejmuje płaszcz. Przerzuca płaszcz
przez lewe ramię, ukrywając torbę w~lewej dłoni i~leniwie wystawia prawą
dłoń.

-- Jestem oczarowana tym spotkaniem, Andersonie Parris.

Po zmieszanej chwili, mężczyzna zdaje sobie sprawę, że oczekuje się po
nim całusa w~rękę, i~tak robi. Jego palce są chłodne, jego usta
wilgotne. Gdy jego głowa unosi się od pocałowanej dłoni, jego wzrok
przebiega od butów na wysokim obcasie, przez czarne skórzane legginsy pod
czarną koronkową spódnicą, aż do szeregu srebrnych zapięć i~małych łuków
na jej czarnym satynowym gorsecie, do szyi, gdzie wybijana ćwiekami
skórzana obroża pasuje do zapiętych pasków na przedramionach, aż do
mroczno, zacienionych oczu. Kiedy ich wzrok się spotyka, odpowiada na
niego, z~lekkim uśmiechem wspólnej tajemnicy.

Seks jest teraz u władzy, a~Seks nie ma problemu z~wyczuciem, że ma go
na smyczy. On macha grzecznie przed sobą i~wchodzą po schodach. Ona
wchodzi wolno, pozwalając mu mieć dobry widok na jej ciasno zasznurowany
tył. Jego mamrotana rozmowa z~Axem niesie się dziwnie po schodach.

Wchodzą do okrągłego pokoju wybudowanego dookoła klatki schodowej. Sufit
to szklana kopuła ponad dwumetrowymi ścianami. Dee widzi słońce i~rzucające się kształty przelatujących samolotów. Nic więcej nie wychodzi
na pokój, który wydaje się łączyć funkcje studia, galerii i~sypialni.
Widać konsolę do rysowania i~matrycę aparatów. Dookoła ścian stoją
krzesła, niskie stoliki i~długie sofy, które mogą być użyte jako łóżka,
choć ich pomysłowo zwyczajne ułożenie kap i~poduszek ukrywa ich funkcje.
Ściany są obwieszone ozdobnymi broniami -- miecze z~wytłaczanej stali,
lasery z~mosiądzu i~rubinu -- oraz obrazami dzieci, na których wyglądały
na wrażliwe, i~kobiet, które wyglądały na niewrażliwe.

-- Czy masz ochotę na drinka, moja pani?

-- Mam -- mówi daleko. -- Dark Star.

Szybki, prawie uniżony uśmiech Parrisa nie do końca ukrywa jego chwilowy
grymas na jej wybór alkoholu, ale idzie do barku z~drinkami i~przygotowuje miksturę. Przynosi go, stukając lodem, i~dotyka jej
szklanki własnym kieliszkiem schłodzonego wina.

Parris uśmiecha się, gdy wychyla szklankę. Zdejmuje swoje kimono. Pod
nim nosi bardzo nieoryginalny sprzęt do krępowania, kostium z~pasków i~spinaczy. Jego penis pręży się w~czymś, co wygląda jak boleśnie ciasny
pas elastyczny, ,,pas'' jest kluczowym słowem.

Ax, ku jej zaskoczeniu, opada na cztery i~pędzi przez pokój do wielkiej
garderoby. Popycha dół drzwi własną głową i~drzwi otwierają się, żeby
pokazać aparatury łańcuchów i~pasów. Dee uderza (szczęśliwie solidną)
szklanką o~bardzo drogi i~delikatny stół w~zasięgu, odwraca się na
pięcie i~patrzy na Parrisa.

-- Rozumiem -- mówi zimno -- że byłeś bardzo niegodziwym człowiekiem.

Parris kiwa głową. Jego oczy się błyszczą, na twarzy, która staje się
zarumienioną maską pokory.

Dee pozwala programowi Seks rozegrać scenę. Uderza go w~twarz, nieco
mocniej niż pewnie on oczekuje.

-- Przyszłam Cię osądzić -- mówi. Udaje zastanawianie się, badanie.
Rozgląda się po pokoju, póki jej spojrzenie nie pada na otwartą szafę.
Ax kuca koło niej, jego język wystawiony. Oczy Dee rozszerzają się w~udawanej niespodziance. Wskazuje na szafę.

-- Tam -- rozkazuje. Parris podchodzi do niej. Uśmiecha się służalczo,
umownie.

-- Oczy \emph{w dół}! -- krzyczy Dee.

Parris posłusznie pochyla głowę i~podchodzi do drzwi.

Dee ma cały protokół nakreślony w~głowie, ale naprawdę nie jest
zainteresowana takimi rzeczami (będąc, jeżeli chce być szczera, bardziej
,,uległą'' niż ,,dominującą'') i~poświęca mniej uwagi na drobiazgową sprawę
nakładania kajdanów i~wiązania go, niż to wymaga. Kończy ściśnięciem
jego policzków, aż on otwiera usta. Wpycha gumową piłkę w~jego usta,
zamyka jego szczęki palcem na nosie i~kciukiem na podbródki i~przykleja
kawałek taśmy (odpowiednio błyszczącej czarnej) na jego ustach.

Wypada na chwilę z~postaci.

-- Ok?

Parris kiwa głową. Dee sprawdza więzy. Są zabezpieczone.

Ax, który cały czas wolno wypracowywał drogę od palców mężczyzny do jego
kolan swawolnym szczypaniem zębami, nagle wstaje i~robi krok do tyłu.
Dee również się cofa i~oboje patrzą na mężczyznę wiszącego z~szafki.

Ax uśmiecha się na nagle zmartwione, zdziwione spojrzenie Parrisa. Sięga
za kark i~długi nóż jest w~jego ręku. Przerzuca go do drugiej ręki,
potem z~powrotem. Sprawdza ostrze. Bok czarnego ostrza łapie odblaski
słońca. Ostrze odbija tylko najsłabsze błyski, jak gdyby fotony się z~niego zsuwały.

Patrzy znowu na Parrisa.

-- Hau -- mówi.

~

Wilde miał więcej niż jeden niedopałek u stóp, kiedy zobaczył dziewczynę
zmierzającą ku niemu przez tłum na rynku. Wyprostował się z~oparcia o~superkomputer.

-- Tamara Hunter -- powiedziała maszyna nad jego ramieniem, gdy dziewczyna
zatrzymała się i~wyciągnęła dłoń. -- Jonathan Wilde.

Przechyliła głowę na bok i~przyjrzała się mu, gdy potrząsał jej dłoń.

-- Mój Boże -- powiedziała. -- Ty naprawdę jesteś nim.

Wilde wyszczerzył zęby. 

-- Sama wyglądasz jakoś znajomo.

-- Pub ostatniego wieczoru -- przypomniała mu Tamara. -- Ale jeżeli
ktokolwiek patrzył tylko na jedną kobietę, to Ty.

-- Ach, oczywiście -- powiedział Wilde. -- Byłaś z\ldots Dee.

-- Tak -- powiedziała Tamara. Rozejrzała się. -- Gdzie Twój robot?

-- Ha! -- parsknął Wilde. -- Ty i~ja powinniśmy być po tej samej stronie,
zgodnie z~tym tutaj elektrycznym prawnikiem, więc nawet nie próbuj z~tym
,,Twój robot''. Niech mnie szlag, jeżeli się przyznam, że to \emph{mój}
robot. Prawda jest taka, że spierdolił gdzieś na własną rękę.

-- Och -- powiedziała Tamara. Spojrzała na superkomputer Niewidzialnej
Ręki. -- Idziemy na prywatną rozmowę -- powiedziała temu.

-- Bardzo dobrze -- powiedziała maszyna. -- Będę kontynuować prace nad
technicznymi aspektami sprawy.

Tamara odwróciła się do Wilde'a. 

-- Gadamy o~tym nad piwem?

-- Boże, tak.

Przeszli koło straganów i~pod drzewa. Rynek grzmiał dookoła nich. Kiedy
byli -- o~ile to było możliwe przez ludzi do stwierdzenia -- poza
zasięgiem słuchu Niewidzialnej Ręki, Wilde spytał: 

-- Kwestia ciekawości,
czy ta maszyna prawnicza jest samoświadoma?

Tamara się roześmiała. 

-- Nie, to tylko system ekspercki. Ma własne
dziwactwa, kapujesz.

-- Ta, mogłabyś tak to ująć. -- Spojrzał na grupę stolików dookoła szyku
blatu, lodówki i~grillu, wszystkich małych i~wszystkich przypalonych.
Wysoki Turek stał w~środku, jego ręce zajmowały się napojami i~kanapkami
za tłuste zwitki pieniędzy. -- Tutaj?

Tamara kiwnęła głową z~doceniającym uśmiechem za jego dobrą ocenę. Wilde
zamówił dwa litry piwa. Sączyli przez minutę ze zroszonych brązowych
butelek, w~spragnionej ciszy, i~się sobie przyglądali.

-- Papieros? -- spytał Wilde, wyciągając teraz sponiewieraną paczkę.

-- Dzięki, ale nie -- powiedziała Tamara. -- Ale proszę bardzo.

Wilde uśmiechnął się do niej. 

-- To moja pierwsza paczka od wieków -- powiedział, gdy zapalił. -- Nie, żeby to było wytłumaczeniem. Z~jednej
strony, dla mnie wszystko to się zdarzyło przedwczoraj, a~z~drugiej, to
palenie mnie zabiło.

Tamara zmarszczyła brwi. 

-- Książki mówią odmienną historię, ale
myślałam, że zmarłeś w~jakiejś strzelaninie.

-- Tak było. -- Wilde skinął głową. -- Próbowałem biec szybciej niż kula,
ale\ldots -- Spojrzał smutno na papierosa i~zaciągnął się, gdy Tamara się
roześmiała.

-- To jest dziwne -- powiedziała. -- Rozmawiałam z~ludźmi, którzy byli na
Statku, i~którzy rzeczywiście przybyli z~Ziemi, cholera, moi dziadkowie
tak zrobili, ale nigdy nie mówili o~byciu martwym. Mówili o~byciu ,,w
okresie przejściowym''.

-- Ta -- powiedział sardonicznie Wilde. -- ,,Zaprzeczenie'' jest terminem
technicznym na ten stan umysłu.

-- Ale Ty\ldots i~bycie, jakby, postacią historyczną. O! Kurwa! -- Przyglądała się rozważnie jego twarzy. -- Wyglądasz inaczej niż na
zdjęciach. Starzej.

-- Na \emph{jakich} zdjęciach? -- upomniał się Wilde.

Tamara sięgnęła do wewnętrznej kieszeni i~podała Wilde'owi plastikowy
portfel zawierający talię kart.

-- Ja, hm, zbieram je -- wyjaśniła, gdy Wilde zaczął je rozkładać. -- Są
darmo w, hm, płatkach, które są produkowane w~tej okolicy.

-- Płatki Harmonia! -- krzyknął Wilde ze śmiechem. Rozłożył drzeworytowe
portrety. -- Spójrzmy\ldots Owen, Stirner, Warren, Bakunin, Tucker, Labadie,
Wilson, Wilde. Przodków mają poprawnie, ale wątpię, czy zasługuję na tak
podniosłą kompanię. Nie jestem pewien, czy jestem zaszczycony, czy
zbulwersowany.

Spojrzał na przerywane linie ikonicznych twarzy i~przesunął dłonią po
własnych świeżych cechach. Pokręcił głową.

-- Kiedy pierwszy raz wyglądałem tak jak teraz, daleko mi było do sławy -- powiedział Wilde. Jego głos brzmiał smutno przez chwilę, weselej, gdy
dodawał: -- Może tak jest dobrze.

-- Święta racja! -- Tamara rozejrzała się dookoła. -- Będziesz sławny od
nowa, kiedy to się wyda. Co się zdarzy, kiedy zacznie się sprawa w~sądzie, jeżeli nie wcześniej.

Wilde wzruszył ramionami. 

-- Chciałbym opóźnić to tak długo, jak to
możliwe. Moje poczucie polityki tego miejsca nie jest dostatecznie
silne, żeby zająć się opinią publiczną na moją korzyść.

-- Ok -- powiedziała Tamara. -- Mamy bliższy problem. Zanim dowiedziałam
się, że jesteś wplątany, dostałam wiadomość od Davida Reida. Znałeś...
go?

-- Jasne, że tak. Kiedyś.

-- Racja, cóż, będzie mnie pozywał, żeby odebrać gynoid Dee. W~porządku,
oczekiwałam tego. \emph{Chciałam} z~tego zrobić sprawę. Niewidzialna
Ręka właśnie powiedziała mi, że też jesteś pozwany, i~że chcesz połączyć
siły. Właściwie nie masz dużego wyboru, gdy to wszystko w~rzeczywistości
jest częścią tej samej sprawy, więc żaden inny sąd nie zamierza dotknąć
Twojej, podczas gdy nasza jest wyróżniająca, i~musimy i~tak cię w~nią
wciągnąć, więc równie dobrze możesz pojawić się na własnych zasadach.

Wilde rozłożył ręce. 

-- Więc gdzie jest problem?

-- Pierwsza osoba na naszej liście preferowanych sędziów to facet
nazwiskiem Eon Talgarth. -- Zatrzymała się, czekając na jakąś reakcję.
Wilde tylko uniósł brwi. -- Kiedyś był abolicjonistą -- kontynuowała
Tamara -- a~teraz prowadzi sąd dla Piątej Dzielnicy. To okręg Maszyn.
Większość sporów, jakie sądzi, są pomiędzy złomiarzami.

-- Złomiarzami?

-- Ludźmi jak ja, którzy wchodzą do Dzielnicy Maszyn i~polują na
użyteczne kawałki maszyn i~automatów. Znany był z~uwalniania
samodzielnych maszyn i~nakładania zakazów na polowanie na nie, ale żaden
inny sędzia nie zaakceptował tego jako precedensu.

-- Mimo wszystko -- powiedział Wilde -- brzmi dobrze dla naszej sprawy.

-- Pewnie, dlatego nie oczekiwałam, że Reid się zgodzi. Niemniej jednak
się zgodził. Wspaniale. Problem jest, nie wiedziałam, że będziesz
włączony. Gówno.

-- Dlaczego jest to problem?

-- Ponieważ Eon Talgarth nie za bardzo cię lubi.

Wilde odstawił piwo i~spojrzał na nią. 

-- Co? Nigdy o~nim nie słyszałem.
Co ma przeciwko mnie?

-- Och, nic osobistego na tyle, o~ile wiem. -- Wzruszyła ramionami. -- Jest
z Ziemi, był w~brygadach pracy, był na Statku. Więc może jakoś go
skrzywdziłeś, nigdy nic nie powiedział. Jednak kiedy był abolicjonistą,
zwykł się wykłócać z~ideą, którą tutaj wielu ludzi utrzymuje, że byłeś
swego rodzaju bohaterem i~wielkim anarchistycznym myślicielem, że
reprezentowałeś alternatywę do pomysłów, które Reid zastosował, kiedy
organizował to miejsce. Powiedział, że zawsze byłeś oportunistą, że
miałeś wszelkiego rodzaju brudne sprawki z~rządami, i~z Reidem, i~że
jakikolwiek konflikt pomiędzy wami dwoma to była tylko osobista
rywalizacja.

Powiedziała to tak beztrosko, tonem ,,powiedz, że tak nie jest''. Wilde
niebezpiecznie przechylił krzesło do tyłu i~zabujał się ze śmiechu.

-- To wszystko prawda, każde słowo! -- powiedział. -- Jestem zdumiony, że
są tutaj ludzie, którzy mówią, że byłem bohaterem i~wielkim
anarchistycznym myślicielem. Haha! Ten Eon Talgarth załatwił mnie na
cacy.

Usta Tamary lekko opadły. 

-- To nie jest naprawdę prawda, co? Że zawsze
byłeś oportunistą?

-- Całkowicie -- powiedział Wilde. -- Tylko któregoś dnia, oczywiście
według mojej pamięci, kobieta, którą kiedyś kochałem, powiedziała mi, że
byłem odpowiedzialny za eskalację ostatniej wojny światowej do
nuklearnej. W~tym czasie mojego życia, pamiętając, że miałem
dziewięćdziesiąt trzy lata i~zebrałem już dużo krytyki za różne...
kontrowersyjne decyzje, nawet się nie obraziłem.

-- Ale jeżeli\ldots -- Tamara rozważyła następstwa. -- To znaczyłoby, że byłeś
obwiniany za...

-- Cały jebany burdel! -- powiedział Wilde. Spojrzał na siebie i~pomachał
dłonią. -- Wszystko, co zdarzyło się od Trzeciej Wojny Światowej, jest
\emph{moją winą}!

-- Tak -- powiedziała Talgarth -- właśnie myśli Eon Talgarth.

-- Może mieć rację -- powiedział Wilde ze wzruszeniem ramion. -- Ja sam tak
nie myślę.

-- Och, ja też nie. -- Tamara pośpieszyła dodać. -- I~też większość ludzi,
abolicjonista, czy nie. W~istocie, niektórzy ludzie myślą, że, cóż...

Zawahała się, zażenowana.

-- Co? -- Wilde pochylił się do przodu, papieros w~dłoni, prowokując ją. -- Coś więcej niż wielki myśliciel anarchistyczny?

-- Tak -- powiedziała Tamara. -- Oni myślą, że Ty, cóż, ciągle żyjesz
gdzieś tam. Ludzie mówią, że cię widzieli, na pustyni.

-- Naprawdę tak mówią? -- Wilde pociągnął i~wydmuchał dym nad jej głową, w~długim westchnieniu. -- To jest naprawdę interesujące, ponieważ robot
Jay-Dub twierdzi, że jest inną\ldots realizacją mnie, że był tutaj od czasu
pierwszego lądowania. Nie założyłbym, że nie ma możliwości stworzenia
sobowtóra lub ukazania się jako ja na ekranie.

-- Aha! -- powiedziała Tamara. -- Zgodnie z~wiadomością, którą dostałam od
Niewidzialnej Ręki, Reid twierdzi, że ma dowody na to, że Jay-Dub
zhakował Dee i~obarcza cię odpowiedzialnością.

-- Mnie? -- spytał Wilde. -- Hm, Jay-Dub nic nie mówił mi o~czymkolwiek
takim. Co za niespodzianka.

-- Tak -- powiedziała Tamara. -- AI są przebiegłymi bękartami, prawda?

-- Przebiegłymi i~niebezpiecznymi -- powiedział Wilde. -- Nie zaufałbym ani
trochę żadnemu z~nich.

Tamara się roześmiała.

-- Ok -- powiedział -- rozumiem, że musimy sobie trochę wyjaśnić. My,
ludzie, powinniśmy trzymać się razem.

Tamara podsumowała, co zdarzyło się poprzedniego wieczora, i~tego
poranka, i~nieco tła. Wilde uśmiechał się ciągle, gdy mówiła o~abolicjonizmie. Potem Wilde opowiedział, co mu się przydarzyło, i~co
robot mu powiedział. Tamara słuchała, czasem z~szeroko otwartymi oczami,
czasem marszcząc brwi. Kiedy skończył, siedziała w~milczeniu przez
chwilę.

-- Co za bękart -- powiedziała w~końcu. -- Wyhodowanie klona ciała Twojej
żony i~użycie go jako gynoida. A niech mnie. Zdaje się, że nie liczył
się ze spotkaniem Ciebie.

-- Może -- powiedział z~powątpiewaniem Wilde. -- Musiał jednak znać robota,
pewnie? Może robot widział wcześniej Dee?

-- Jasne -- powiedziała Tamara. -- Tego typu sprzęt ma łączność, jeżeli nic
innego. A Reid twierdzi, że Jay-Dub zhakował Dee. Jednak Robot nic nie
mówił o~tym?

-- Na pewno nie mi -- powiedział Wilde. -- Zdecydowanie czułem, że ta
maszyna wiedziała coś o~Dee, w~istocie nalegała, że Dee nie była
człowiekiem, nawet w~takim sensie jak on, ale nigdy nie dał mi żadnej
wskazówki, że Dee jest częścią planu, jakikolwiek by nie był.

-- A teraz zniknęła -- westchnęła Tamara. Rozejrzała się, jakby mając
nadzieję, że się pojawi. -- Przypuszczalnie nie wie o~sprawie sądowej i~doszła do wniosku, że najlepiej jest się ukryć.

-- To by dobrze pasowało do jej osobowości. -- Wilde się uśmiechnął. -- I~mojej!

-- Miejmy nadzieję, że odnajdzie się przed procesem -- powiedziała Tamara.
-- W~przeciwnym przypadku jest jeszcze w~głębszym gównie\ldots Ciągle chcesz
stawić się Talgarthem?

-- Z~tego, co mi powiedziałaś -- powiedział Wilde -- nie mam zbyt dużego
wyboru w~tej kwestii.

-- To prawda -- powiedziała Tamara.

Wilde odpowiedział ironiczną miną. Wstał, nic nie mówiąc, i~powędrował
do najbliższych kramów. Co jakiś czas uśmiechał się do siebie, a~potem
odwrócił się i~uśmiechnął się do Tamary, która, milcząc, podążała za
nim.

-- Jest coś w~tym miejscu -- wyjaśnił. -- Zawsze wiedziałem, że takie
miejsca jak to powstałyby, rynki śmieci na innych światach. To sprawia,
że czuję się tak nostalgicznie, że wiem, że jestem tym samym
człowiekiem, którym byłem na Ziemi.

Tamara popatrzyła na ziemię i~kopnęła kamyk.

-- Przepraszam -- powiedziała. -- Słyszałam tak dużo o~Wildzie, ale moje
wyobrażenie było zawsze jak\ldots wiesz, te karty, plakaty, które
widziałam. Wiedziałam, że jestem trochę zarozumiała, rozmawiając z~Tobą,
jakbyś był tak młody, jak wyglądasz.

Wilde parsknął i~klepnął ją w~ramię. 

-- Przestań -- powiedział. -- Tylko
dosłownie wróciłem z~martwych.

Poszli do Niewidzialnej Ręki i~zarejestrowali Wilde jako współpozwanego,
a Wilde przedstawił kontroskarżenie Reida, jako odpowiedzialnego za
śmierć Jonathana Wilde'a, z~Londynu, Ziemia. Maszyna przyjęła to bez
sprzeciwu, ale jej wewnętrzne światła poruszały się w~sposób podniecony.

-- Co teraz? -- spytał Wilde Tamarę.

-- Hm, może to czas, żebyś poznał Dee. Przebywa w~moim mieszkaniu, a~to
tylko pięć minut stąd. Ax, to jest\ldots dzieciak, który mieszka ze mną,
powiedział, że rano weźmie ją na zakupy. -- Spojrzała na zegarek. -- Piętnasta trzydzieści. Powinni już być z~powrotem.

-- Ok -- powiedział Wilde. Wstał. Po raz pierwszy od czasu, gdy się
poznali, jego twarz pokazywała coś innego niż spokój.

-- Chodźmy.

\threeast

Ax wyciąga nóż z~zamkniętych drzwi szafy, odchodzi kilka metrów i~znowu
rzuca nożem. Nóż uderza w~drzwi i~tam zostaje, dodając przybliżony zarys
człowieka z~nacięć, które powtarzające się rzuty zostawiły na drewnie.
Cichy jęk i~dźwięk walenia dochodzi z~wewnątrz szafy.

Dee odrywa się od grzebania po kolekcji zdjęć Parrisa. Czuje mdłości.
Jest niemożliwe, żeby określić, czy zdjęcia są prawdziwe, ustawiane, czy
po prostu wygenerowane na komputerze. Nieszczególnie jej na tym zależy.
Chce wymazać je z~pamięci, a~autora z~tego świata.

Ciągle nie wie, czy może to zrobić, lub nawet stać i~pozwolić Ax zrobić
to. Nie wie, czy uprawnienia dla jej umiejętności zabójczych zostały
skasowane. Podejrzewa, że jeżeli nie zostały, nie zdarzy się nic
dramatycznego. Żadnego wstrzymania ręki, żadnego wrośnięcia w~ziemię,
tylko pewne całkiem rozsądne i~wyglądające naturalnie zahamowanie,
niesmak lub niepokój, które nie pozwoli jej tego zrobić.

-- Nie masz już tego dosyć? -- pyta Axa.

Ax wyciąga jeszcze raz nóż z~drewna. 

-- Chyba tak -- przyznaje. Szczerzy
do niej zęby. -- Dałaś się ponieść emocjom.

Dee wyjmuje pistolet z~torby, wkłada go za pas i~podchodzi.

-- Dobra, kończmy to -- mówi.

-- Dobra -- mówi Ax.

Otwiera rozszczepione drzwi. Wewnątrz, Parris ciągle wisi w~więzach.
Oczy ma ciasno zamknięte. Łzy płyną po twarzy, a~knebel z~taśmy jest
wysmarowany smarkami, które przyniosły łzy i~który wydmuchał z~nozdrzy w~szalonych parsknięciach.

Ax śledzi linię czubkiem noża, wzdłuż nagiego brzucha mężczyzny. Parris
otwiera oczy i~buja się z~boku na boku, patrząc na Axa, a~potem, jak
gdyby odwołując, na Dee. Krew płynie wzdłuż cięcia. Dee na jej widok
zatrzymuje się i~łapie ramię Axa.

-- Nie! -- mówi. Widoki z~kolekcji Parrisa są wypchnięte przez zdjęcia
Żołnierki, encyklopedii ran i~krwi: tryskanie, rozpryskiwanie, sączenie,
kapanie. Wyobraża sobie krew bryzgającą na jej ubrania i~wzdryga.

-- Nie -- mówi. -- Wystarczy.

Ax patrzy na nią, ale wytrzymuje spojrzenie. Wycofuje się. Dee zabiera
się do pracy, rozluźniając, uwalniając z~pęt, rozwiązując. Podtrzymuje
Parrisa, gdy wypada, i~pozwala mu opaść na podłogę. Wydaje odgłosy przez
nos.

-- Och -- mówi Dee. Zapomniała o~tym. Pochyla się, żeby zerwać taśmę z~ust, i~gdy ona schodzi, Dee zauważa, że Parris doszedł, i~to więcej niż
raz, nawet z~jego kutasem przywiązanym do tyłu. Nasienie wysycha na jego
udach.

Upada do przodu w~pozycji klęczącej i~patrzy na nią, dysząc i~się
uśmiechając.

-- Dziękuję, Pani -- mówi niskim głosem. -- Zasłużyłem na to, na to
wszystko, naprawdę! -- Patrzy na nią chytrze z~nadzieją. -- Kiedy możesz
mnie znowu odwiedzić?

Dee patrzy się na niego. Cofa się kilka kroków, ciągle myśląc o~utrzymaniu w~czystości jej ładnych nowych ubrań. Odwraca się i~idzie
dalej, koło Axa, do szczytu schodów.

-- Pani, proszę\ldots -- Parris woła za nią.

-- Och, jebać to -- mówi.

Wyciąga pistolet ze spódnicy, przymierza i~rozwala głowę Parrisa.

Strzał odbija się echem dookoła okrągłych ścian pokoju i~klatki
schodowej i~dzwoni jej w~uszach. Uśmiecha się do Axa, który pomimo
swojego namawiania do całej rzeczy patrzy na pozostałości Parrisa, a~potem na nią, wstrząśnięty i~blady.

-- Teraz wiem -- mówi. -- Mam wolną wolę.

-- To musi być bardzo pożyteczne -- mówi Ax. -- Ja jestem raczej
deterministą.

Dee uśmiecha się do niego uspokajająco, gdy raźnie zbiera jej rzeczy.

-- Czas na nas -- mówi.

Ax bezcelowo wyciera czubek noża o~firankę.

-- Nie powinniśmy, wiesz, posprzątać? -- pyta. -- Nie widzisz odcisków
palców i~tym podobne?

-- Och, pewnie -- mówi Dee, zapinając płaszcz. -- Są w~całym mieszkaniu. I~nasze komórki skóry. Nie wspominając o~obrazach w~kamerach domu.

Patrzy w~górę, uśmiecha się i~macha do małej, ukrytej soczewki.

-- Gówno -- mówi Ax. -- Nie możesz nic z~tym zrobić?

Dee rzuca mu zdziwione spojrzenie i~zaczyna schodzić po schodach.

-- Oczywiście, że mogę -- mówi. -- Ale to bardzo ważne, że tego nie robię,
wiesz o~tym. Chodź, zanim ktoś przyjdzie.

Ax idzie za nią, ciągle niechętnie.

-- Nikt nie przyjdzie -- mówi. -- Nie sądzę, żeby Parris miał swoje
gniazdko podpięte video do najbliższej firmy ochroniarskiej.

-- Chyba nie.

Odblokowanie drzwi nie wymaga żadnych głębszych umiejętności Dee.
Zamykają się za nimi, gdy tylko są na zewnątrz. Idą w~ciszy długim
podjazdem. Tuż przy końcu, boczny podjazd prowadzi do pobliskich drzwi
rezydencji. Dee skanuje jego elektronikę.

-- To wystarczy -- mówi. -- Ktoś jest w~domu.

Ax przestaje iść. Przez chwilę wygląda jak uparte dziecko.

-- Nie o~to mi chodziło -- mówi.

Dee próbuje się nie mizdrzyć.

-- To jest ważne -- mówi. -- Pomoże Twoim powodom, jak również Twojej
rozprawie.

-- Mam w~dupie rozprawę -- mówi Ax. -- To gówno jest \emph{skończone.}

Dee patrzy na niego spokojnie, gdy przypomina sobie rzeczy, które
powiedział wcześniej.

-- Nieożywieni mogą wstać -- mówi -- i~możesz mieć rację, ale w~ten czy
inny sposób, to wszystko trafi pod sąd.

Ax patrzy się na nią przez chwilę, potem kiwa głową.

Razem, idą po małym podjeździe do drzwi. Dee dzwoni. Czekają. Mały ekran
powyżej dzwonka rozświetla się, pojawia się twarz kobiety.

-- Tak? -- mówi.

Dee staje nieco prościej i~wyżej.

-- Tutaj Dee Model i~Ax Terminal -- oznajmia stanowczo. -- Właśnie
zabiliśmy Twojego sąsiada po tamtej stronie, Andersona Parrisa. Wzywam
Cię na świadka.

Kobieta przesadnie mruga.

-- Z-zaświadczone -- mówi drżącym głosem.

-- Dziękuję -- mówi Ax.

-- Do widzenia -- mówi Dee.

Potem Dee i~Ax śpieszą się do głównego podjazdu, po schodach i~stokach
do poziomu chodnika, potem do windy na wysokie piętro, gdzie dołączają
do małej kolejki dobrze ubranych ludzi czekających na przystanku, żeby
złapać śmigacz. Ax zajmuje się dostrajaniem do serwisów informacyjnych
przystanku. Co jakiś czas potrząsa głową i~uśmiecha się do Dee: jeszcze
zadowolony, i~używa tych przerw w~swoim szklanym transie do przeglądania
listy.

Dee widzi, że już wykreślił jedno nazwisko, i~że jest znacznie więcej do
zrobienia.

~

Tamara patrzy na mały stos materiałów obciążających na stole: pliki
Talgartha o~Wilde, zdjęcie, które zrobiła Dee i~nabazgrana
apokaliptyczna tyrada od Axa. Wilde właśnie skończył je czytać.

-- Boże -- powiedział. -- Słyszałem o~listach samobójców, ale to jest
pierwszy raz, kiedy trafiłem na list \emph{mordercy}.

Tamara podtrzymywał rękami głowę.

-- \emph{Ja} zamorduję tego małego zboczeńca, jeżeli kiedykolwiek dostanę
go w~ręce -- powiedziała. -- Szczerze, towarzyszu Wilde, gdybym w~ogóle
podejrzewała, że jest zdolny do wystartowania i~czegoś takiego, nigdy
bym nie spuściła Dee z~oczu.

Wilde sięgnął ponad i~złapał jej rękę.

-- Spokojnie -- powiedział -- spokojnie. Co ja ci kiedykolwiek zrobiłem, że
nazywasz mnie ,,Towarzyszem Wildem''? Na imię mam Jon, ok? I~jesteś nie
bardziej odpowiedzialna za stratę Dee niż ja za stratę Jay-Duba. Oboje
są wolnymi osobami, czy nie o~to tutaj chodzi?

-- Chyba tak -- powiedziała Tamara. -- A Ax twierdzi, że nie był, kiedy
robił pewne\ldots poniżające rzeczy. Mogę zrozumieć dlaczego, też, w~pewien
sposób, ale potem\ldots Aaaach! To jest tak skomplikowane! Co robimy?

-- Tamara -- powiedział Wilde delikatnie, puszczając jej dłoń i~siadając -- jak długo żyłaś?

-- Dwadzieścia lat.

Wilde zapalił papierosa.

-- Lat Nowego Marsa?

-- Tak.

-- Cóż zatem -- powiedział Wilde. -- Żyłaś w~anarchii dwa razy dłużej, niż
kiedykolwiek mi się udało, i~na pewno znasz na to odpowiedź, lub sposób
znalezienia odpowiedzi.

Tamara siada przy stole i~patrzy na niego, zdumiona i~wyzywająca.

-- Nie rozumiem ciebie -- powiedziała.

-- Spójrz -- powiedział Wilde -- kiedy chcemy wiedzieć, czy coś jest warte
wykonania, szukamy odpowiedzi w~maszynie badawczej zwanej rynkiem. Kiedy
chcemy wiedzieć, jak coś działa, mamy kolejną maszynę badawczą zwaną
nauką. Kiedy chcemy odkryć, czy ktoś ma prawo zabić kogoś innego, mamy
maszynę badawczą zwaną prawem.

-- Tak -- powiedziała Tamara. -- Wiem to. To nie będzie za duża pomoc dla
Ax i~Dee, kiedy ich złapią. Lub nas, jeżeli będziemy czekać, zanim
zaczniemy próbować ich zatrzymać.

-- Ale jest warte próby, ok? A jeżeli prawo cię zawodzi, i~nie możesz z~tym żyć, wtedy\ldots -- Rozkłada ręce, uśmiechając się.

-- Co?

-- Znowu jesteś w~stanie natury. Walczysz. Ok, możesz umrzeć, ale co z~tego? Tak samo, gdy rynek cię zawodzi. To się zdarza. Głodujesz.
Kradniesz.

Tamara patrzy zaskoczona.

-- Ale to byłaby...

-- Anarchia? -- Wilde uśmiecha się do niej.

-- Mówisz, że ludzie mogą zrobić wszystko?

-- Dosłownie, tak. W~każdym na wpół przyzwoitym społeczeństwie jest ci o~wiele lepiej, gdy szanujesz prawo, własność i~tak dalej, ale w~gruncie
rzeczy, to Twój wybór. Zawsze masz możliwość wojny, przeciwko całemu
światu, jeżeli do tego dojdzie.

-- Ale \emph{przegrasz}! -- powiedziała Tamara.

Wilde spojrzał na nią, niewzruszony.

-- Może nie. Locke powiedział, że zawsze możesz ,,odwołać się do nieba'', a~Bóg lub Natura mogą zadziałać na Twoją korzyść. To, co mówię, to, Ax
dokonał swojego wyboru, a~Dee swojego. Może mogą usprawiedliwić ten
wybór przed sądem, może nie. Tak czy inaczej, to nie my decydujemy, i~byłbym bardziej niż szczęśliwy, żeby uzasadnić nieostrzeganie ich
potencjalnych ofiar. Jednak jeżeli chcesz, jak najbardziej działaj.

Tamara potarła brodę i~znowu spojrzała na elaborat Axa. Spojrzała na
zdjęcie Dee i~dokumenty Talgartha. Potem -spojrzała na Wilde i~spytała,
jakby chcąc rozstrzygnąć jedno ostateczne pytanie: 

-- A co robisz, jeżeli
\emph{nauka} Cię zawiedzie?

Wilde się roześmiał. 

-- Ufasz szczęściu.

Zgasił papierosa i~zerwał się.

-- Im wcześniej dotrzemy do sądu Eona Talgartha, tym lepiej -- powiedział.
-- Mam rację?

-- Tak -- powiedziała Tamara. Wstała i~zaczęła szukać map, aprowizacji i~broni.

-- Więc jak się tam dostaniemy? -- spytał Wilde. -- Samolot?

Tamara pakowała magazynki amunicji. Odwróciła się do niego i~roześmiała.

-- Talgarth nie jest uprzejmy dla lądujących samolotów -- powiedziała. -- Nie ufa im, z~jakiegoś dziwnego powodu. Nie, po prostu bierzemy
dostatecznie dużo sprzętu i~broni, żeby przedostać się przez dzikie
maszyny, i~idziemy pieszo. Wszyscy tak robią. -- Uśmiechnęła się. -- To
prawo. Zmniejsza szanse na walki wybuchające w~sądzie.

-- Jest jeszcze dużo rzeczy, których nie wiem o~tym miejscu -- przyznał
krzywo Wilde.

Tamara chrząknęła, sprawdzając ciężar plecaka. Wyjęła ciężki pistolet i~podała go Wilde'owi. Pchnęła dokumenty Talgarth o~Wilde przez stół.

-- Zabierz to i~czasem poczytaj -- powiedziała. -- Jest dużo rzeczy,
których to miejsce nie wie o~Tobie.

\chapter{Testowane na Zwierzętach}

Pewnie już zauważyłaś, że to, co opowiadam tutaj, nie występuje w~książkach. Jak zgadłaś, o~to chodzi. Dlaczego miałbym duplikować moich
hagiografów?

Więc wybaczysz mi, mam nadzieję, jeżeli przyjmę, że historia jak
wykorzystałem Ludzi Dla Postępu (kampanię edukacyjną North British
Mutual) jako wyrzutnię dla Ruchu Kosmicznego, jak wykorzystałem
Kosmicznych Kupców, żeby założyć WolnyKosmos, radykalną libertariańską
grupę, która nauczyła się jedynej prawdziwej lekcji lewicy, leninizmu,
jak wykorzystaliśmy Ruch Kosmiczny jako ludowy front dla naszego
anarchizmu wolnorynkowego, i~Ruch Kosmiczny rozrósł się nawet poza moje
oczekiwania, jednym słowem, jeżeli przyjmę, że \emph{mein kampf} została
przeczytana.

A moje polityczne komentarze i~analizy, efemeryczne jak się wtedy
wydawało, blakły z~ekranów jak pamięć krótkoterminowa, były posłusznie
archiwizowane przez agencje wywiadowcze, a~w~odpowiednim czasie
(przykładowo późniejsze wojny i~rewolucje) zostały przekazane do domeny
publicznej i~niewątpliwie gdzieś wiszą tam, ,,zawsze jest \emph{kiedyś,
gdzieś} w~sieci'', więc jeżeli naprawdę chcesz poznać, to tylko jedno
wyszukiwanie (uwaga: ograniczenie prędkości światła mogą obowiązywać).
Zatem o~tym także nie będę się powtarzał.

W moich późniejszych latach, sporadycznie byłem znany z~marudzenia o~młodych dnia dzisiejszego itd., jak nie doceniali, że była rewolucja
przed \emph{Rewolucją}, jak nie byłoby Nowej Republiki, gdyby nie było
na pierwszym miejscu Republiki, jak byłoby to wszystko dla nas cięższe i~przy okazji, czy kiedykolwiek mówiłem Wam o~wojnie?

Zatem to też opuszczę.

Jednak warto powiedzieć, że Zjednoczona Republika sama się nie zdarzyła.
Ludzie nie obudzili się w~poranek wyborów w~2015 i~pomyśleli ,,Tym razem
\emph{musimy} wyrzucić gnojków''. Prawdę powiedziawszy, tak się stało,
ale zabrało to mnóstwo pracy, żeby stworzyć ten brawurowy impuls: dekady
agitacji, marudzenia, opracowywania konstytucji, rzadko uczęszczanych
spotkań w~biednie wyposażonych salach, listów do redakcji, głośnych
demonstracji i~całej reszty. A była to cholernie ciężka praca. Wiem,
ponieważ tam byłem i~w ogóle jej nie robiłem.

~

WolnyKosmos (nazwa, która kiedyś wydawał się trendy, a~teraz datowała
nas boleśnie, ,,bardzo dwudziestowiecznie'', jak usłyszałem kiedyś kogoś)
miała swoje skromne biura nad sklepami Kosmicznych Kupców tuż po drugiej
stronie drogi z~Rynku Camden Lock. (Zrezygnowałem z~prowadzenia
Kosmicznych Kupców, zachowałem wystarczająco udziałów i~opcji, żeby
zachować stały, o~ile mały, dochód, i~zostawiłem to sobie. Firma
przeniosła się do sprzedaży realnych produktów kosmicznych, większość
tylko nowości, biżuteria ze skały księżycowej, kryształy z~zerowej
grawitacji i~tak dalej, ale także niektórych praktycznych. Produkcja w~mikrograwitacji była wykorzystywana w~nieoczekiwanych zastosowaniach, jak
podejrzewałem, że będzie). Mieliśmy biura od dziesięciu lat i~ciągle
pachniały świeżą farbą, drewnem i~cementem. Betonowe ściany były
udekorowane plakatami ruchu kosmicznego i~hologramami NASA S.A.\footnote{w~oryg. inc t.j. pewna forma korporacji nie występująca w~Polsce,
najbliższe wydaje się S.A. czyli spółka akcyjna -- przyp.tłum.}, ale
pierwsza rzecz, którą każdy widział, gdy wchodził przez drzwi, było moje
biurko z~wielkim napisem za nim mówiącym ,,PALENIE MILE WIDZIANE''. Sam
już nie paliłem, choć medycyna już pobiła to, co my (dziś myląco)
nazywaliśmy ,,Duże N'', nie było prostej naprawy dla konsekwencji nawyku
dla oskrzeli, a~w~wieku sześćdziesięciu dwóch lat potrzebowałem
całego oddechu, jaki miałem. Napis był kwestią zasady, jak podajniki
mydła w~toalety były obklejone złośliwymi naklejkami informującymi, że ich
zawartość była \emph{Testowana na Zwierzętach}.

Rankiem po wyborach byłem jedyną osobą w~biurze, która się nie spóźniła
i nie miała kaca. Każde przybycie kogoś z~czerwonymi oczami było witane przeze
mnie podnoszącego wzrok znad wiadomości online (panika w~Whitehall,
spadek funta, zamieszki w~Kensington, tłumy na lotniskach) i~mówiącego:

-- Och, zostałeś na wyniki? Kto wygrał?

W ten sposób zabezpieczając moją anarchistyczną wiarygodność, mogłem
potajemnie napawać się wynikami. Skład nowego rządu nie był jeszcze
oficjalny, ciągle się kłócili, ale wyglądało, że to będą Republikanie,
New Labour, True Labour, oraz kilku Radykałów po stronie rządu z~Unionistami jako oficjalną opozycją i~małymi partiami na skrzydłach.
Wiele tych ostatnich -- nawet Socjaliści Świata (nowa nazwa na SPWB) -- uciułali dostatecznie dużo razem pierwszych wyborów, żeby otrzymać
wybranego jednego posła. Niestety, moi rodzice tego nie dożyli. Zabrało
to partii sto jedenaście lat, żeby dostać się do Parlamentu, ale ciągle
kierowali się na globalną większość w~dwudziestym piątym wieku.

Potem wróciłem do organizowania nadzwyczajnego spotkania komitetu
wykonawczego o~11 tego ranka. Bez odpowiedzi, nawet automatycznej
sekretarki, od dwóch członków: Aaronson (badania) i~Rutherford
(międzynarodowa łączność). Hmmm. Natychmiast skontaktowałem się kilkoma
potencjalnymi rywalami na każdą pozycję -- zamiast naszej wewnętrznej
grupy bezpieczeństwa, która składała się \emph{prima facie}\footnote{łac. na
pierwszy rzut oka -- przyp.tłum.} prawdopodobnie i~tak z~policyjnych
szpiegów -- i~poprosiłem o~śledztwo.

Jednak pozostałe siedem osób punktualnie pojawiło się na moim ekranie, a~każdy z~nas u pozostałych osób. Zdecydowałem się nic nie mówić o~Aaronsonie i~Rutherfordzie, i~po prostu wzruszyłem ramionami, kiedy ich
nieobecność została zauważona na pogawędce przed spotkaniem, gdy ludzie
przekładali papiery, uruchamiali notatniki, siadali na miejscach i~patrzyli na mnie wyczekująco.

-- Ok, towarzysze -- zacząłem -- stąd to wygląda, że obudziliśmy się nie
tylko w~nowym rządzie, ale w~nowym reżimie. Nazwijcie mnie romantycznym
starym głupcem, ale myślę, że to początek rewolucji. Bardzo brytyjskiej
rewolucji, przyznam to Wam, ale zbliżało się to już przez dłuższy czas,
a rewolucje są prawem dla siebie mniej więcej z~definicji. Nie
stawiałbym, że ta pozostanie na właściwej drodze. To mogłyby być dla nas
dobre wiadomości lub złe, zależnie od tego, jak się ułożą rzeczy.
Pytanie brzmi, czy możemy coś zmienić?

Wszystkie oczy na ekranach wykonały śmiesznie jednoczesny obrót, gdy
wszyscy sprawdzali reakcję wszystkich innych. Ewan Chambers,
reprezentant Szkocji, pierwszy zabrał głos.

-- Zgadzam się z~Jonem. Rzeczy wyglądały całkiem dziko w~Glasgow
ostatniej nocy, nieco bardziej niż impreza na ulicach, i~nie do końca
zamieszki. A z~tego, co widzę, w~Edynburgu jest niespokojny spokój.
Partia Władzy Robotniczej\footnote{oryg. Workers Power -- grupa
trockistowska, która stworzyła brytyjską sekcję Ligi Piątej
Międzynarodówki, więcej~\url{https://en.wikipedia.org/wiki/Workers\%27_Power_(UK)} -- przyp.tłum.} utrzymuje, że wygrała wybory zamiast tylko kilku miejsc.

-- Jest tak samo tutaj -- powiedziała Julie O'Brien, nasz organizator z~Południowego Londynu -- ale nie sądzę, że powinniśmy się już martwić o~trockistów przejmujących władzę i~zagładzających wszystkich na śmierć.
Jeżeli spojrzysz, jak nowy rząd jest tworzony, racja, w~ogóle jest bez
wątpienia, że będziemy mieć Republikę, ale poza tym ten rodzaj programu,
o którym mówią, jest prawdziwym misz-maszem libertarian i~etatystów. Z~drugiej strony łagodzenie kontroli imigracji, koniec prohibicji,
ściągnięcie żołnierzy z~Grecji i~tak dalej, ale z~drugiej strony partie
Pracy przepychają tę \emph{zasadę przemysłową}, połączenia kabli w~jeden
wielki system i~inne rodzaje gówna z~XX wieku.

-- Włączając w~to program kosmiczny, dość zabawnie -- powiedziałem. -- Jakieś opinie o~tym?

Nastąpiła awanturą, którą przerwałem, jak tylko ktoś wspomniał Ayn Rand.

-- Oto, co sugeruję -- powiedziałem. -- Nie popieramy tego, nie
sprzeciwiamy się, a~jeżeli kiedykolwiek wystartuje, żądamy prywatyzacji.

Nie ma to jak chwila współdzielonego cynizmu pociągnięta razem z~komitetem. 

-- Racja -- powiedziałem, kiedy przestaliśmy się chichotać -- poważne sprawy. Cholernie dobra zagadka dla reżimu hanowerskiego, ale
jak Julie mówi, pytanie jest co się stanie potem. Struktura polityczna
przez chwilę będzie całkiem elastyczna. Może uda nam się położyć ręce na
jakimś opuszczonym obszarze i~zrobić strefę przedsiębiorczości albo
wolny port albo coś, i~poprzemy nasze słowa czynami?

Adrian Moss zmarszczył brwi. Kierował działalnością lobbystyczną ruchu
taką, jaka była. 

-- Moglibyśmy prawdopodobnie pchnąć to -- powiedział -- ale dlaczego? Wolne strefy lepiej zostawić prawdziwemu biznesowi, a~nie
organizacjom politycznym. -- Jego uśmiech śmignął po ekranie. -- Wiesz, to
przypomina mi pewną skrajną ideologię, o~której słyszałem!

-- Powiem Ci dlaczego -- powiedziałem. -- Jeżeli sprawy się ułożą, dobrze,
kilka naszych idei zostanie przetestowanych. Jednak ten kraj może
zmierzać do rozpadu. Wszyscy widzieliśmy, co to znaczy, raz za razem.
Każdy łapie to, co może. Posiadanie kawałka ziemi, żeby go nazwać swoim,
może dać nam przewagę na starcie.

To spowodowało pewne zamieszanie. Tylko Julie i~Ewan byli za. Udałem
sprzeciw i~zasugerowałem, żebyśmy dali to do głosowania członkom. Ci
przeciwko mojej sugestii się zgodzili, pewni, że zostanie odrzucone.

W tym czasie nieobecność Aaronsona i~Rutherfora sama wyszła w~rozmowie.
Przywdziałem mój umiarkowany kapelusz i~udało mi się przekonać komitet,
że jeżeli okazało się, że cały czas byli szpiegami i~teraz uciekli z~kraju, dość zdecydowanie nie dokonalibyśmy na nich zamachu.

Później tego popołudnia śledztwa, które rozpocząłem, pokazały, że obaj
otrzymali dyskretną ofertę pracy dla obiecanego Organu ds. Przestrzeni
Kosmicznej, i~byli zbyt zażenowani, żeby nam powiedzieć. W~tamtej
chwili, kusiło mnie, żeby ich załatwić, ale po namyśle, zdecydowałem się
po prostu wyrzucić ich z~komitetu.

W referendum członków w~sprawie oferty dla rejonu lokalnej
przedsiębiorczości, moja pozycja przemożnie wygrała, tak jak
wiedziałem,że wygra. Przy całym politycznym podnieceniu, nawet hołota
libertarian nie mogłaby chcieć zrobić czegoś konstruktywnego dla
odmiany.

Rok później WolnyKosmos kontrolował opuszczone nieruchomości przemysłowe
North London, z~kilkoma blokami mieszkań dorzuconymi przez lokalną radę
w desperacji, żeby się ich pozbyć. Sześć miesięcy później, to miejsce
roiło się od entuzjastycznych ochotników, a~Adrian szybko ściągał
zewnętrzne inwestycje. Po kolejnych sześciu miesiącach delegacja
przedstawicieli robotników i~pracowników powiedziała komitetowi, że byli
bardzo zadowoleni z~ochrony, jaką nasze bojówki zapewniały, ale chcieli
jeszcze jednego małego zapewnienia.

Tylko dla spokoju ich umysłu.

Julie powiedziała, że to niemoralne, Ewan powiedział, że to nielegalne,
Adrian powiedział, że to zbyt drogie, a~ja powiedziałem, że znam kogoś,
kto mógłby nam to tanio załatwić.

~

Transkrypcja rozmowy telefonicznej, opublikowanej 10 stycznia 2050 roku
na podstawie Ustawy o~dostępie do informacji (poprzednie rządy).

{[}\emph{koniec głosu programu recepcji}{]}.

JW: Cześć, Dave.

DR: Och, witaj stary gnojku. Co mogę dla Ciebie zrobić?

JW: Hm, czy to szyfrowane?

DR: Nie, ale jestem pewien, że wiesz co mówić.

JW: Kurwa {[}\emph{przerwa}{]} Myślimy, żeby zostać prywatnymi, hm,
wielkim wiesz. {[}\emph{przerwa}{]}

DR: Czy was już kurwa powaliło?

JW: Raczej nie. Ja, zdaje się niektórzy Twoi przyjaciele w~komunistanie...

DR: ...deformowane państewka robotnicze\ldots {[}\emph{śmiech}{]}

JW: \ldots mogą mieć najlepsze umowy. Możesz to pchnąć?

DR: Och, pewnie. Mamy polisy.

JW: Lepsze niż polityki, {[}\emph{śmiech}{]}

DR: Po prostu nie rozumiem, kiedy będziesz potrzebował.

JW: Niewiele w~Tobie sprzedawcy, co? {[}\emph{przerwa}{]}

DR: Och, dobra, to Twoje życie. Sprawdzę. Cholera, ok, bądź w~przyszłym
tygodniu\ldots wtorek, dziewiąta trzydzieści, Stanstead. Biuro czarterów.

JW: Do zobaczenia chłopie.

DR: Wspaniale. Pozdrowienia dla żony i~dzieciaków {[}\emph{śmiech}{]}

JW: Wzajemnie, dla kochanki i~bękartów.

DR: Hm, \emph{dziękuję}, chłopie. Cheers.

JW: Slandge. {[}\emph{koniec nagrania głosowego}{]}

~

Trafiliśmy na turbulencję nad południowym Uralem. Stałem w~wąskim
korytarzu w~kierunku ogona, oparty o~boki i~patrzyłem prosto przez
ostatnie okno. Gdy samolot zanurkował, miałem ładny widok na góry. W~długich cieniach świtu, wyglądały wybitnie jak model gór z~papier-mache.
Niedaleko poniżej, regularna seria małych chmur jednocześnie się
rozpraszała. Ciekawe.

Kolejny zwrot na skrzydło, kolejna chwila opadania, potem nagłe
wznoszenie. Krzyk dobiegł z~małej toalety.

-- Wszystko w~porządku?

-- W~porządku -- krzyknął Reid. -- Tylko się zaciąłem.

-- Co Ty tam \emph{robisz}?

-- Golę się.

Dziesięć, nie piętnaście minut wcześniej widziałem, gdy pocierał
policzki i~brodę elektryczną maszynką, tuż przed tym, gdy lekkomyślnie
oddałem mu pierwszeństwo do toalety. Mój pęcherz moczowy wysłał mi ostry
protest. Możesz mieć chirurgiczne mikroboty pełzające po hydraulice,
powiedział mi, ale \emph{są} granice\ldots To był najwyższa czas,
pomyślałem, dla mnie, żeby zacząć praktykować egoizm, który głosiłem.

-- Golisz \emph{co}? Nogi?

-- Tyły\ldots moich\ldots dłoni -- powiedział Reid. Mogłem usłyszeć zaciśnięte
zęby. -- Zapomniałem pierdolonych gumowych rękawiczek, za pierwszym razem, gdy
użyłem kuracji na włosy na głowie.

Wyszedł z~zakłopotanym uśmiechem na twarzy i~pianką do golenia na
nadgarstkach. Nie przestawałem się napawać. Powódź ulgi zamieniła
aluminiową miskę toalety wielkości spluwaczki w~ring. Potem ochlapałem
twarz zimną wodą, rozpiąłem kilka więcej guzików na mojej koszuli,
rozsmarowałem niezręcznie dezodorant pod pachami, wysuszyłem brodę,
przeczesałem krótkie włosy z~boku i~tyłu, wytarłem ręcznikiem łysinę i~założyłem krawat. Ponieważ musiałem pochylać się lub kucać w~trakcie, a~lustro byłoby odpowiednie do kobiecej torebki, trudno było ocenić
ostateczny efekt. Ciągle chichotałem nad powodem, dlaczego włosy Reid,
choć szare jak moje, były tak długie i~gęste.

W rzeczy samej, szampon korygujący geny! Co za próżność, myślałem, gdy
trzymałem płyn do płukania ust przez minutę, żeby zrobił swoje, potem
wyplułem go i~sprawdziłem blask moich zębów.

~

North British Mutual zrodziło agencję ochrony, a~Reid był mocno
zaangażowany w~wykupie zarządu kilka lat wcześniej. Jeżeli ten lot był
jakąkolwiek wskazówką, przedsiębiorstwo ,,Wzajemnie Gwarantowana
Ochrona'' miało się dobrze. \emph{Odrzutowiec biznesowy}, jaki wynajęli
na ten etap podróży, może był trochę sztuczny, trochę spartański, ale
miał własną stewardessę, uzbecką dziewczynę ze stałym uśmiechem i~bez
znajomości angielskiego. Śniadania były podane do czasu, gdy wróciłem na
siedzenie: mikrofalowany croissant i~kawa, która, jak zgadłem po
pierwszym łyku, też była z~mikrofali. Żadne nie było dość gorące.

-- Mikrofala, co -- burknął Reid. -- Pomachano chwilę przed radarem,
raczej.

-- Może to kwestia turbulencji -- powiedziałem.

-- Turbulencji? -- parsknął Reid. -- Człowieku, to był ogień
przeciwlotniczy.

-- Co! -- Odwróciłem się zaalarmowany do okna.

-- Nie martw się -- powiedział Reid. -- Tylko bandyci. Nie mogliby trafić w~777 na tej wysokości.

Nasz ochroniarz Predestination Ndebele powoli skinął głową. Smukły,
żylasty Zimbabwejczyk, jeden z~pracowników Reida.

-- Ty myślisz to złe -- powiedział -- spróbuj wylądować w~Adnan\footnote{lotnisko w~Turcji \url{https://en.wikipedia.org/wiki/\%C4\%B0zmir_Adnan_Menderes_Airport } -- przyp.tłum.}

-- Wierzę Ci na słowo, Dez.

Reid spojrzał znad papierów. 

-- Ostatnio słyszałem -- powiedział z~niejasnym zmarszczeniem -- to było nazywane Grivas.

~

Lecieliśmy godzinami nad przerażająco bezkształtną równiną, a~potem, w~środku tego niczego, wylądowaliśmy na pełnowymiarowym międzynarodowym
lotnisku brzęczącym od samolotów wojskowych i~cywilnych. W~oddali,
nieład silosów startowych i~suwnic. Bliżej, miasto niskich bloków z~wielkiej płyty: Kapica\footnote{od Piotr Kapica -- fizyk radziecki, odkrywca
zjawiska nadciekłości helu, za co otrzymał Nagrodę Nobla w~1978,
więcej~\url{https://pl.wikipedia.org/wiki/Piotr_Kapica } -- przyp.tłum.}, stołeczne (i jedyne) miasto Międzynarodowej Republiki
Robotników Naukowo-Technicznych vel Obszar Testów numer trzy, na
nieużytkach gdzieś pomiędzy Karagandą a~Semipałatyńskiem. Część byłego
Kazachstanu.

-- Mam dla Ciebie niespodziankę -- powiedział Reid, gdy czekaliśmy na
autobus tranzytowy.

-- Jaką?

-- Zobaczysz.

Spojrzałem na niego i~wzruszyłem ramionami, skulony w~suchym, kurzowym
wietrze i~próbując niezbyt dużo oddychać. Poziomy miały być już
bezpieczne, ale już interpretowałem skutki jetlaga jako chorobę
popromienną w~początkowym stadium.

Główny budynek lotniska był, jak każdy, neonową oświetloną przestrzenią
siedzeń, ekranów i~głośników, ale różnice były uderzające. Sklep
wolnocłowy nie był w~oddzielnej strefie, ponieważ nie było strefy cła.
Żadnej kontroli paszportowej, tylko pobieżna rejestracja broni i~przejście przez skaner. Jedyną rzeczą, którą ktokolwiek mógłby tutaj
przeszmuglować, która mogłaby cokolwiek zmienić, była rzeczywista bomba
jądrowa, a~te nie były łatwe do ukrycia. Żadnych turystów: wszystkie
przyloty i~odloty to byli poważnie wyglądający klienci, mężczyźni w~garniturach lub mundurach. Bardzo mało kobiet, oprócz tych wśród
pracowników lotniska, którzy wszyscy, nawet sprzątacze, jak zauważyłem,
wykonywali swoją pracę z~prawie zuchwałym brakiem pośpiechu, pod
wielkimi plakatami Trockiego, Korolewa i~Kapicy\footnote{odp.
\url{https://pl.wikipedia.org/wiki/Lew_Trocki},
\url{https://pl.wikipedia.org/wiki/Siergiej_Korolow},
\url{https://pl.wikipedia.org/wiki/Piotr_Kapica } -- przyp.tłum.}. Ludzie,
którzy dali Sowietom Armię Czerwoną, Rakietę i~Bombę, i~którzy wszyscy
dostali za to różne dawki terroru Stalina.

Z każdej części zbiegowiska dobiegały irytująco częste strzelania
żarówek fleszy. Fotografowie błądzili po tłumie, przyglądali się głodnie
twarzom, pstrykali oficerów, urzędnikom i~przedstawicielom korporacji
równie chętnie jak gwiazdom video. Ich obiekty odpowiadały w~podobnej
manierze. W~całej sali, pozy były przybierane przez brzydkich,
nachmurzonych mężczyzn: potrząsanie rękami, przytulanie na misia, stanie
ramię przy ramieniu i~patrzenie jak szalony.

-- Dokąd teraz? -- spytałem, gdy Ndebele i~ja zawahaliśmy się na chwilę na
skraju zbiegowiska. Reid spojrzał na mnie z~błyskiem zniecierpliwienia.

-- To jest to -- powiedział. -- Umowy są zawierane tutaj. Muszą być
publiczne, o~to chodzi.

Ruszył celowo w~kierunku otwartej kawiarni na franszyzie. Pośpieszyłem
za nim.

-- Stąd paparazzi?

-- Oczywiście. Spokojnie. -- dodał do Deza, który patrzył ponuro na
każdego, kto patrzył na nas.

Wypiliśmy naszą pierwszą rozsądną kawę tego dnia dookoła stołu zbyt
niskiego, żeby był komfortowy, jak gdyby zaprojektowany, żeby
przyśpieszać przepływ klientów. W~telewizji cztery ładne osoby
pochodzenia azjatyckiego w~różowych satynowych sukniach śpiewały
ochryple po angielsku, niszczyły instrumenty i~skakały po scenie. Ciągły
podpis podał ich nazwę: Chłopcy Katoi\footnote{Katoi -- tożsamość płciowa w~Tajlandii, która w~niektórych
przypadkach może być określona jako transkobieca , w~innych przypadkach -- jako gejowska,
więcej~\url{https://en.wikipedia.org/wiki/Kathoey } -- przyp.tłum.}.

-- Chłopcy? -- Dez uniósł brew.

-- Uchodźcy tajscy -- powiedziałem. -- Moja najmłodsza prawnuczka mówiła
mi, że są ostatnim idolem dla młodych nastolatków.

-- Perwersyjne, człowieku -- powiedział Dez z~ciężką kalwińską
dezaprobatą. -- Dekadenckie.

-- Tak, to właśnie Republika Islamska im powiedziała -- powiedział
spokojnie Reid, przepatrując tłum. Wstał.

Odwróciłem się. Wysoka, szczupła kobieta w~futrze do kostek szła do nas,
powitalnie szeroko się uśmiechając. Fotografowie truchtali za nią, w~pełnym szacunku dystansie. Prawie usiadłem na krześle, gdy ją
rozpoznałem: Myra, moja dawno temu eks z~Instytutu Studiów Sowieckich w~Glasgow.

-- Cóż, cześć chłopaki -- powiedziała. Złapała moją rękę, przyłożyła swój
policzek do mojego i~wyszeptała: 

-- Uśmiech cholera! -- A ja pokazałem idiotyczny uśmiech przed fleszami.

~

Jedno z~moich najwcześniejszych wspomnień, dość dziwne, tyczyło Związku
Radzieckiego, kosmosu i~Bomby. (Nie pamiętam narodzin, ale byłem
zapewniany, że wydarzenie to miało miejsce 5 marca 1953 roku, w~dniu, w~którym umarł Stalin. Zróbcie z~tym, co chcecie.) Bawiłem się na dywanie
naszego domu w~Streatham, przedmieścia miasta Londynu. Bawiłem się
zabawkową rakietą. Jeżeli przyłożyłeś oko do otworu na końcu, mogłeś
zobaczyć kawałek obrazu drzew na wewnętrznej powierzchni, ponieważ
zabawka była wyprodukowana w~Hongkongu z~odzyskanej puszki. Nie wynikało
to z~ekologii, która w~tamtym czasie nie została wynaleziona. Było tak,
ponieważ było to tanie.

Mój ojciec, siedząc przy stole, spoglądał na mnie nad kopią
\emph{Manchester Guardian}.

-- Rosjanie wysłali rakietę w~kosmos -- powiedział mi. -- Tam w~górę w~niebo, lata dookoła świata. -- Palcem nakreślił koło w~powietrzu.

Poczułem się tym zaniepokojony. Rosjanie byli w~mojej głowie niejasnym,
olbrzymim zagrożeniem. Zrobili coś nieprzyjemnego i~niesprawiedliwego
przyjacielowi mojego ojca, staremu panu, którego fotografia była w~ramkach nad kominkiem: Karolowi Marksowi. Rosjanie go
\emph{zniekształcili}. Cokolwiek to było, brzmiało boleśnie.

Z warkotem uniosłem rakietę do góry, a~kiedy sięgnęła granicy mojego
zasięgu, obróciłem ją i~opuściłem, nosem do przodu. Jej kształt,
zauważyłem po raz pierwszy, wyglądał jak bomba. Raz widziałem bombę
wyciąganą ostrożnie żurawiem z~ogrodu na końcu drogi, przed dwoma
policjantami, tuzinem żołnierzy i~zafascynowanym tłumem. Była pogrzebana
w ziemi przez dziesięć lat po wojnie pomiędzy brytyjskimi i~niemieckimi
kapitalistami.

-- Czy to znaczy, że mogą wysłać bombę przez kosmos?

Ojciec wrócił do gazety, może rozczarowany moją zajętą odpowiedzią na
jego ekscytujące wiadomości, i~teraz znowu ją opuścił i~popatrzył na
mnie bystrym wzrokiem.

-- Tak! -- powiedział wesoło. -- Dokładnie to oznacza. Bardzo mądrze,
Jonathan. A teraz Amerykanie i~wszyscy inni zbudują rakiety i~włożą w~nie bomby.

Moja matka zmarszczyła brwi na niego.

-- Ale to wszystko dobrze -- pośpieszył dodać ojciec, gdy wstał, strząsnął serwetkę i~złożył gazetę. -- Robotnicy nie pozwolą im użyć
bomb. Zatrzymamy ich, prawda?

-- Tak -- powiedziałem. -- Zatrzymamy ich.

Wiedziałem z~zabaw z~innymi chłopcami na ulicy, że poglądy moich
rodziców nie były podzielane w~Streatham, ale wiedziałem także, że
wszędzie na świecie, nawet w~tak odległych krajach, jak Austria i~Nowa
Zelandia, byli ludzie, którzy się z~nimi zgadzali. Razem ich było
\emph{setki i~setki}.

Ta potężna siła powstrzymałaby bombę. Wróciłem do bawienia się
szczęśliwie rakietą, a~mój ojciec poszedł, gwiżdżąc, złapać pociąg, który
wiózł niewolników pensji do pracy.

~

-- Reid mówił mi, że ma niespodziankę -- bełkotałem -- ale muszę
powiedzieć, że jest powalony. Jak na ziemię tutaj skończyłaś?

Myra uśmiechnęła się złośliwie. Wyglądała do dobrze i~prawie mogłem
uwierzyć, że nie postarzała się przez czterdzieści lat, ale to była
część tej samej iluzji, która chroniła mnie przed czuciem się staro.
Można było dostrzec papierową teksturę skóry, zmarszczki przy jej ciągle
imponującym napięciu.

-- Trafiłam tutaj w~latach dziewięćdziesiątych -- wyjaśniła -- do badań i~wtedy właśnie zrozumiałam, że ci ludzie potrzebowali pomocy i~że lubiłam
pomagać. Ciągle mieli mnóstwo gówna z~testów i~jak również dotykał ich
drenaż mózgów. Potrzebowali każdej wykształconej osoby, jaką mogliby
dostać, a~ja byłam w~stanie załatwić dużo pomocy z~amerykańskich
organizacji charytatywnych. Potem zakochałam się w~oficerze armii,
pobraliśmy się i, szczęśliwie dla nas, on był po zwycięskiej stronie
kilku wojen domowych, przewrotów wojskowych i~re-rewolucji. Więc oto
jestem, Ludowy Komisarz do spraw Polityki Społecznej. -- Pomachała
dłonią. -- Pozwalają mi podpisywać traktaty, kiedykolwiek chcę, więc nie
czuję się, jakbym utknęła w~sprawach domowych. -- Roześmiała się. -- Wiecie, praca kobiet!

Pokręciłem głową. 

-- Zatem Reid stał się kapitalistą, a~ty stałaś się
biurokratą, cholera, jestem jedynym, który ciągle jest rewolucjonistą!

-- \emph{Nie} jestem biurokratą -- powiedziała Myra z~pewną butą. -- Zostałam wybrana, w~prawdziwych wyborach. Mamy demokrację, wiesz.

Reid wyjmował dokumenty z~teczki i~rozkładał je na stole. 

-- Tak, Myra,
na pewno wygrałaś z~dziarskim młodym podporucznikiem. Jego fakcja nadała
nowe znaczenie wyrażeniu ,,zdeformowane państwo robotników''.

-- Stary żart -- powiedziała Myra, ale widziałem, że nie była zirytowana.
-- Powiem Ci taki starszy. Sowiecki. Skąd wiemy, że marksizm jest
filozofią? Ponieważ gdyby był nauką, najpierw przeprowadziliby próby na
\emph{psach}.

W jej głosie brzmiało taka miażdżąca pogarda, że wszyscy musieliśmy się
roześmiać, a~wtedy Myra rzuciła: 

-- Dobra, towarzysze, ci ludzie byli
psami, i~\emph{coś} im się udało osiągnąć. Szkoda, że nie możecie zostać
na kilka dni i~to zobaczyć. Lub nawet przyjechać w~październiku.

-- Dlaczego październik?

-- Obchody stulecia -- powiedziała Myra. -- Planujemy \emph{naprawdę
imponujący} pokaz sztucznych ogni.

-- Nie wątpię -- powiedział Reid sucho. -- Największy na świecie, bez
wątpienia. Niestety, mamy własną rewolucję, do której musimy wrócić.

Myra westchnęła. 

-- Interesy\ldots gotowy z~tymi dokumentami?

-- Gotów, kiedy będziesz gotowa.

Podpisaliśmy, lampy strzeliły i~to było to. Świat będzie wiedział, że
mam Bombę.

~

Kiedy upadł Związek Radziecki, Kazachstan przez jakiś czas znalazł się w~nieznanej sytuacji mocarstwa, ponieważ miał na swoim terytorium pewną
liczbę broni nuklearnej. Kiedy Kazachstan upadł, jeden z~jego fragmentów
zachował pewne (inne, i~lepsze) bronie jądrowe, z~dodatkową różnicą, że
Międzynarodowa Republika Robotników Naukowo-Technicznych -- początkowo
nic innego jak oddział posowieckich Wojsk Rakietowych, kilka tysięcy
Kazachów i~pas stepu -- wiedziała co z~nimi robić.

Eksportowali odstraszanie jądrowe. Nie samą broń, -- co, zapomnij o~tym,
byłoby to nielegalne -- ale zbawienny wpływ jej posiadania. Nasza umowa była
całkiem standardowa, po prostu dawała nam opcję wezwania uderzenia
nuklearnego na każdego, kto użył broni nuklearnej przeciwko nam i~kto
\emph{nie zapewnił pełnego odszkodowania}. Ktokolwiek, kto nas
zbombardował, nawet przypadkowo lub mimochodem, musiał zapłacić lub sam
był bombardowany.

Piękno tego układu polegało na tym, że dowolna liczba klientów, im
więcej tym lepiej, mogła mieć roszczenie na relatywnie małej liczbie
bomb jądrowych, efekt raczej jak w~systemie rezerw cząstkowych\footnote{
system monetarny, w~którym występują dwa rodzaje pieniądza: pierwszy
rodzaj to pieniądz emitowany przez bank centralny, drugi rodzaj to
kreowany przez banki komercyjne w~postaci kredytu,
więcej~\url{https://pl.wikipedia.org/wiki/System_rezerw_cz\%C4\%85stkowych} -- przyp.tłum.}. Oznaczało to także, że ktokolwiek, kto chciałby skusić
MRRNT umową pierwszego wykorzystania, musiałby zaoferować więcej niż
dochód od \emph{wszystkich} klientów odstraszania, a~to kosztowałoby
znacznie więcej niż po prostu zbudowanie lub kradzież bomb jądrowych.
Zatem szanse, że system zostanie wykorzystany do agresji nuklearnej były
małe. Ponadto po raz pierwszy, odstraszanie nuklearne było dostępne dla
każdego, kto chciał zapłacić, a~koszt był wystarczająco rozsądny dla
każdej ojczyzny.

Szczególnie gdy pojawiła się konkurencja: zbuntowani dowódcy łodzi
podwodnych, załogi rakiet na Syberii i~Alasce, którzy chcieli dla
odmiany wypłaty w~prawdziwych pieniądzach, grupy ambitnych młodych
oficerów w~Afryce, którzy zaczęli sprzedawać udziały w~rodzimym
plutonie.

Kolejny triumf wolnego rynku.

~

Nie wszyscy się zgadzali.

-- Kiedy zobaczyłam zdjęcia -- wściekała się Annette -- Ciebie z~tą
anorektyczną dziwką, myślałam, że z~nią uciekłeś! To jest \emph{gorsze}!

Och, nie, nie jest, pomyślałem, i~miałem rację. Kłóciliśmy się,
dyskutowaliśmy, żyliśmy dalej. To były tylko idee, nie ciała. Mogłem być
faktycznym, zamiast możliwego, masowym mordercą, a~to zraniłoby ją mniej
niż ja pieprzący kogoś innego.

Nie, żebym kiedykolwiek to powiedział. Niektóre bronie najlepiej trzymać
w rezerwie.

\chapter{Czas Niedostępności}

Wilde stał, patrząc z~powątpiewaniem na paczkę i~dwa zestawy broni,
które Tamara wyłożyła na stole. Podniósł paczkę i~znowu ją odłożył.

-- Co tutaj masz? -- spytał. -- Bomby jądrowe?

Tamara spojrzała znad skanera, którego używała do ściągnięcia
najnowszych map Piątej Dzielnicy do jej kontaktów, i~potrząsnęła głową.

-- Żadnych jądrówek -- powiedziała stanowczo. -- Przewożenie materiałów
nuklearnych w~granicach miasta jest poważnym przestępstwem.

-- Cieszę się, że to słyszę -- powiedział Wilde. -- Więc jesteśmy gotowi,
tak.

-- Mniej więcej. -- Tamara odłożyła skaner. -- Musimy być gotowi wyruszyć w~dowolnym momencie, ale to nie oznacza, że musimy iść teraz. Reid
zarezerwuje wysłuchanie, a~my dostaniemy przynajmniej zawiadomienie
trzynaście godzin wcześniej.

-- A co z~przygotowaniem naszej sprawy? -- spytał Wilde. -- Nie wiem nic o~waszym prawie tutaj, nie mówiąc o~konkretnym kodeksie, na którym pracuje
Talgarth.

-- Och, wszystko w~porządku -- powiedziała Tamara. -- Niewidzialna Ręka się
tym zajmie. Możesz mieć kogoś, żeby stawał w~obronie, jeżeli chcesz,
ale jeżeli mnie zapytasz, równie dobrze wystarczy, by Niewidzialna Ręka
dostarczyła ci agenta MacKenziego.

-- Co?

-- Program komputerowy, który doradzi punkty prawa, kiedy sam się
reprezentujesz.

-- Ach -- powiedział Wilde. -- Postęp.

Tamara poszła do kuchni i~zaczęła parzyć dużą manierkę kawy.

-- Oczekujesz towarzystwa?

-- Sojusznicy -- powiedziała Tamara. -- Niewidzialna Ręka wzywała
niektórych dla mnie -- Uśmiechnęła się złośliwie do niego. -- Nikogo dla
Ciebie.

-- Uważaj mnie za jednego ze swoich -- powiedział Wilde. Rozejrzał się po
pokoju, szukając. -- Macie jakiś sposób nadążania za wiadomościami?

Tamara spojrzała na niego dziwnie. 

-- Ta, pewnie.

Podeszła do półki, wzięła ekran telewizora, rozwinęła go i~przyczepiła
go do ściany za stołem. Wysoki czajnik zaczął się gotować. Odwróciła
się, by go obsłużyć. Wilde spojrzał na ekran, spojrzał na Tamarę.
Pomachał na pustą cynową powierzchnię ekranu.

-- Och! -- Tamara uderzyła się w~głowę. -- Przepraszam. Nie masz kontaktów?

-- Coś, co robot zdecydowanie zaniedbał mi powiedzieć -- powiedział Wilde.

Tamara powiedziała mu o~dobrym lokalnym straganie, gdzie mógł kupić
kontakty i~jak się tam dostać. Spisał instrukcję, narysował szkic mapy,
sprawdził z~nią i~wyszedł. Wrócił pół godziny później, mrugając i~z szeroko otwartymi oczami. 

-- Oj -- powtarzał. -- O, kurwa!

~

W ciągu kolejnej godziny sojusznicy Tamara pojawiali się pojedynkę i~parami. Ostatecznie tuzin wypełniał pokój, siedząc przy stole,
sprawdzając broń i~pijąc kawę Tamary. Większość z~nich paliła i~wszyscy
mieli stanowcze opinie o~aspektach sprawy, jak również zażenowane, i~żenujące, zainteresowanie Wilde'em. Człowiek z~Nieożywionych! Wilde
szybko stracił rejestr ich imion lub zainteresowanie w~ich obsesjach,
gdy znalazł się kącie w~tłumie w~większości chudych, większości młodych,
ciężko uzbrojonych obcych mówiących mu rzeczy, których nie wiedział o~sobie.

-- Zawsze myślałem, że Twoja późniejsza praca potępiająca teorię
konspiracji była sfabrykowana przez konspirację...

-- Nie.

-- \ldots i~Norlonto, racja, to była idealna społeczność...

-- Nie.

-- \ldots podstawowa idea abolicjonizmu, że inteligencja maszyn ma sztuczne
prawa, była oparta na tych samych założeniach jak Twoje manifesty ruchu
kosmicznego...

-- Nie.

-- Mówią, że to wszystko dlatego, że Reid pieprzył Twoją kobietę...

-- Nie.

I tak dalej.

A potem wszyscy rozpoczęli i~ucichli w~tej samej chwili, nawet Wilde,
który do tego czasu złapał sposób strojenia swoich kontaktów do ekranu
telewizyjnego. Wiadomości, jak większość wiadomości na kanałach Miasta
Statku, były przedstawiane przez podniecone dziecko. (Wilde już wyraził
opinię, że to było jedno z~najbardziej oświeconych i~właściwych sposobów
wykorzystania pracy dzieci, na jakie kiedykolwiek trafił).

-- Właśnie pojawiły się wiadomości! -- powiedziała lalunia z~blond lokami
na Kanale spraw prawnych. -- Trzy sensacyjne postępy! David Reid pozywa
abolicjonistów o~zwrot jego gynoida, Dee Model! I, pozywa dawno-martwego
anarchistę i~terrorystę nuklearnego, Jonathana Wilde'a, w~podobnym
zakresie. W~końcu, Dee Model i~inny abolicjonista wzywają świadków, że
zabili słynnego artystę, Andersona Parrisa! List gończy wystawiony,
wkrótce zostaną opublikowane nagrody!

Obrazy tych wymienionych zbliżały się trzpiotowato na ekranie, gdy
mówiła, a~kanał potem podzielił się na wątki, rozwijając następstwa
każdego aspektu, biografie domniemanych uczestników i~eschatologiczne
znaczenie powrotu Jonathana Wilde'a.

-- \emph{Terrorysta jądrowy}? -- Mężczyzna, który przemówił, nazywał się
Ethan Miller. Jego wygląd był starszy niż większość obecnych, z~czarnymi
prostymi włosami, skórą koloru podłego tytoniu, który palił, i~twarzą
jak dobrze zużyty topór. Nie nosił nic, prócz skórzanych spodni i~obdartej podkoszulki, o~której twierdził, że jest oryginalna, choć
równania Malleya miały teraz jeszcze więcej dziur w~ich tkaninie niż
kiedykolwiek miały w~rzeczywistości. -- Powinieneś ich za to pozwać,
chłopie!

-- Nie.

Bardziej trzeźwe deklaracje Niewidzialnej Ręki zastąpiły kanał
wiadomości, instruujące wszystkie strony w~sprawie do pojawienia się w~Sądzie Piątej Dzielnicy do dziesiątej następnego dnia.

-- Dobra! -- krzyknęła Tamara nad gwarem. -- Słyszeliście! Raz raz raz!

Wyjście, które nastąpiło, było mniej szalone niż próby jego
zorganizowania przez Tamarę. Najwidoczniej termin ich pojawienia nie był
trudny do spełnienia. Ludzie uzbrajali się i~wychodzili, z~Tamarą,
Wilde'm i~Ethanem Millerem zamykającym pochód. Tamara zamknęła i~uzbroiła dom, tylko żeby zapobiec przeszukaniom bez nakazu, jak
wyjaśniła, i~wszyscy ruszyli w~kierunku nabrzeża.

Słońce było nisko na niebie, zamieniając wieże w~centrum miasta w~wysoki
diadem złota i~klejnotów. Na Placu Okrągłym głównej wyspy, straganiarze
się pakowali, podczas gdy pierwsi technicy wieczornych band
przygotowywali systemy nagłośnienia. Powietrze wczesnego wieczoru było
gęste od zapachu smażonego oleju, silnikowego oleju i~słodkiego fetoru
konopi. Dookoła stołów i~barów na zewnątrz, ostatni wychodzący lub
wcześnie przybyli obserwowali cicho rozmawiającą, maszerującą grupę z~niejasnym lękiem i~ukrytymi komentarzami, pośród których okazjonalny
zachęcający uśmiech błyszczał niczym wyjęta broń.

-- Co się stanie z~Dee i~Ax -- spytał Wilde -- jeżeli zostaną złapani?

Tamara chrząknęła. 

-- Zależy, jak wściekli będą ci, co ich złapią -- odpowiedziała. -- Prawdopodobnie będą zatrzymani i~oskarżeni, przez
kogokolwiek kto będzie żądał odszkodowania. Zdaje się, że ten Anderson
Parris miałby fajną cenę za swoją głowę.

-- Ta, hm\ldots -- powiedział Wilde. -- Mogę się do tego odnieść. Jednak co im
się przydarzy, jako kara?

-- Kara? -- Tamara brzmiała zaskoczona. -- Och, masz na myśli grzywna.
Zależy, znowu. Zabicie kogoś może być całkiem poważne, wiesz.

-- Tak -- powiedział Wilde sucho. -- Więc od czego zależy grzywna?

-- Nie martw się -- powiedziała Tamara. -- Kurde, przynajmniej wezwali do
tego świadków. To się bardzo liczy, brak próby ukrycia\ldots prócz tego, ot, 
zależy od strat ofiary, racja? Emocjonalne straty, utrata doświadczeń
życia, dochodów, strata społeczeństwa dla tych blisko dla niego, dodaj
to wszystko i~pomnóż przez czas niedostępności.

-- Ach -- powiedział Wilde. -- Czas niedostępności. Myślę, że mogę
zrozumieć, co mówisz znacznie lepiej, jeżeli wyjaśnisz mi dokładnie co
to \emph{czas niedostępności.}

Dotarli do nabrzeża, gdzie bujała się łódka Tamary. Inni wsiedli do
swoich łodzi, flotyla łódek, pontonów i~szalup. Tamara zeszła do swojej
łodzi, Ethan Miller podał jej sprzęt, a~ona pomogła Wilde'owi wejść.
Usiadł, gdzie mu kazała, z~boku.

-- Czas niedostępności -- wyjaśniła Tamara, gdy odbili i~łagodnie
zapuściła silnik -- to czas pomiędzy byciem zabitym a~powrotem. Backupy
kosztują, wiesz, a~hodowanie klonów może zabrać jebane \emph{miesiące},
szczególnie jeżeli chcesz dobrego, bez raka i~innego gówna. Więc jeżeli
jesteś zwykłą osobą, powiedzmy jak ja, to robisz kopie backup co roku
lub coś koło, i~masz polisę szybkich klonów. Jeżeli jesteś naprawdę
bogaty, jak ten facet Parris, robisz je co tydzień. Jednak wtedy, masz
wolnego klona, i~Twoje straty zwiększają się szybciej, ponieważ Twoje
zarobki były wyższe. Więc to się jakoś równoważy, ale ciągle jest taniej
zabić biedaka.

Uśmiechnęła się do niego i~uruchomiła silnik. 

-- Społeczeństwo klasowe to
kurwa.

-- Aha -- powiedział Wilde wymijająco. -- A co jeżeli ktoś nie ma backupu?
Co jeżeli zostają martwi?

-- Wszyscy mają backupy -- powiedziała Tamara, zdumiona jego niewiedzą. -- Nikt \emph{nie zostaje martwy}. Jezu.

Skoncentrowała się na sterowaniu łodzią w~niespokojonym kilwaterze ich
towarzyszy i~nie zauważyła miny nagłego bólu Wilde'a. Tylko bot łodzi to
zobaczył, ale mógł tylko zarejestrować, ale nie zrozumieć.

~

Niskie słońce, czerwone od pyłu pustyni, świeci w~oczy Dee. Ocienia je
kapturem, pociąga płaszcz bliżej siebie. Gdy jej wzrok się wyregulowuje,
milimetr za bezpośredni blask, widzi poszarpane czarne krawędzie Gór
Madreporowych\footnote{od płytka madreporowa,
zob.~\url{https://pl.wikipedia.org/wiki/P\%C5\%82ytka_madreporowa
} -- przyp.tłum.} daleko na zachodzie, na końcu błyszczącego cięcia
Kamiennego Kanału. Siedzi, obejmując kolana, zebrana koronka spódnicy
kłuje w~skórę ramion. Ax również siedzi, opierając się o~jej plecy. Są w~rodzaju orlego gniazda, dziury bez funkcji z~boku wieży dziobatej od
wielu takich. Dziury są połączone przez podobnie niewytłumaczalne
tunele, które przynajmniej zapewniają wentylację dla dłuższych i~znacznie szerszych korytarzy w~środku. Wielkie gąbczaste kolce zostały
przez dekady skolonizowane przez biznesy i~osadnictwo. Co, jeżeli w~ogóle, było oryginalnie zaprojektowane dla zdecydowanie nieludzkiego
siedliska, ale ludzie są pomysłowymi i~przystosowującymi się
zwierzętami. Dee wie o~tej rysie. Uważa ją za wspaniałą, choć -- teraz do
niej dociera -- nie do końca może być z~niej dumna. Nie są jej gatunkiem.

Że ludzie nie są jej gatunkiem, jest wnioskiem, do którego doszła tego
popołudnia. To trochę rozczarowujące, ponieważ czuła się jak ludzki byt
tylko kilka dni, i~zamierza zachować to dla siebie, szczególnie jeżeli
pytanie o~jej ludzki status stanie się przedmiotem sporu naukowego.
Jednak to jedyny sposób, w~jaki może wyjaśnić sobie, jak mało jej
przeszkadza zabijanie ludzi.

Nawet zakładając, że wrócą -- umysły z~wolnopracującej pamięci, ciała z~kadzi -- bycie zabitym musi powodować dla nich wiele zmartwienia i~kłopotu. (To jest różne od \emph{nieożywionych, martwych}\footnote{autor używa słowa dead w dwóch znaczeniach -- martwi i~ nieożywieni tj. ci, którzy nie mogą być ucieleśnieni  -- przyp.tłum.}, pedantycznie
przypomina jej Naukowczyni, różne magazyny, różne odzyskanie, różne
problemy. 
Ta, tak, odpowiada, i~gdy ta Jaźń jest znowu wyłączona, Dee myśli
przelotnie o~Annette, kobiecie, o~której wie, że dzieli z~nią genotyp.
Myśli o~niej pośród Nieożywionych, myśli o~kodach i~pamięciach, i~przez
kolejną chwilę Sys błyska jakimś cienkim powiązaniem, ale znika\ldots Po
prostu teraz ma za dużo na głowie).

Powodowanie zmartwienia i~kłopotów jest, dla Axa, celem. Bardzo się
zachwyca rozwalaniem każdego, kto kiedykolwiek oszukał go, wykorzystał
go finansowo, duchowo lub seksualnie. Rechocze, gdy padają, od kul jego
lub Dee. Trzech na razie, jeszcze wielu na liście. Dee w~zasadzie ma to
w dupie. Wie, że jest zdolna do emocji, empatii, nawet etyki, są tam,
wypalone w~obwodach większości jej jaźni, ale wydaje się, że nie mają
zastosowania do ludzi jak Parris, lub tej kobiety, którą Ax przybił w~piwnic dwie godziny temu, lub mężczyzny, którego zastrzeliła w~drzwiach.
Może obwody zaprojektowane są do stosowania do jej własnego gatunku, w~tym przypadku ofiary nie należą do jej gatunku.

Teraz dociera do niej, gdy mruży oczy w~słońcu i~wypatruje łowców
nagród, oznak listów gończych, że istnieje inne wytłumaczenie. Może jest
człowiekiem, prawda, ale jej ofiary nie są. Może to, co mają wspólnego,
jest pasożytnicza mimikra ludzkości, którą może przejrzeć. Jeden z~wątków jej Historii, które odgrywa w~nocy, kiedy chce dać sobie
silniejszy wikt niż zwykłe romanse historyczne, jest o~wampirach.
Zastanawia się, czy pozornie gatunek ludzki -- lub rząd człowiekowatych -- jest podzielony na prawdziwych ludzi i~pustą kpinę z~ludzi, byty jak
wampiry, które żywią się życiami innych. Zabijanie ich może być całkiem
odmienne od zabijania prawdziwych ludzi, którzy żywią się roślinami,
zwierzętami i~maszynami.

Interesująca myśl.

Słyszy długie, opróżniające płuca westchnienie Axa. Napina kręgosłup w~oczekiwaniu na łomot pistoletu i~uderzenie odrzutu. Wstrząsają jej
ciałem sekundę później.

-- Mam go! -- mówi Ax.

Dee nie musi się rozglądać. Wyjściowa rampa, nad którym jest ich
gniazdo, jest pięć metrów w~dół i~około dwudziestu metrów dalej, i~może
wyobrazić sobie leżąca tam ciało bankiera. Może także wyobrazić sobie
twarze i~soczewki obracające się w~ich kierunku w~ciągu następnych
sekund\ldots

Jednak już się zsunęli, Ax i~Dee, po stoku dziury i~z bezpośredniego
widoku. Dziura szerokości metra w~syntetycznej skale prowadzi do
zakrzywionego koryta, którym cierpliwie się wspinali jakieś pół godziny
temu. Szklana gładkość, która sprawiał trudności przy wchodzeniu, pomaga
przy schodzeniu. Dee idzie pierwsza, stopy najpierw, otulona płaszczem.
Spadek na końcu jest dziwny. Jej więzadła lędźwiowe napinają się, obcasy
zgrzytają, kolejne zadanie dla procedur Chirurżki. Odwraca się, wystawia
ręce i~łapie Axa, gdy wypada.

Korytarz, w~którym stoją, ma zwykły nibyorganiczne zaokrąglone rogi w~przy prostokątnym przekroju poprzecznym i~zakrzywia się gładko na lewo i~prawo. Błyszczące powierzchnie masy perłowej są podziobane dziurami,
usiane chitynowymi soczewkami i~membranami, i~brutalnie włamanymi
mikrofonami, kamerami, oknami biur i~drzwiami. Alarmy już odbijają się
echem wzdłuż korytarza i~falują po drutach. Żołnierka i~Szpiegini,
dzielący czas zmysłów i~przekaźników Dee, hakują i~pingują. Niektóre
sygnały alarmowe są zakłócone.

Jednak nie wszystkie. Po cichej wymianie spojrzeń, Dee i~Ax odwracają
się i~biegną na lewo. Kierują się na windę, której użyli, żeby zjechać z~poziomu ulicy. Drzwi otwierają się w~korytarzu przed nimi, alarm znowu
jazgocze. Strażnik ochrony w~czarnym mundurze wychodzi i~unosi dłoń.
Jest na widoku za krzywą korytarza. Dee hamuje ślizgiem i~łapie ramię
Axa.

-- Do tyłu! -- sapie.

Odwracają się i~biegną z~powrotem. Kroki strażnika odbijają się echem za
nimi. Dee zauważa, kątem oka, ruch za cienką powierzchnią ściany, nie
okno, ale wewnętrzne dla budynku. Biegnie jeszcze kilka metrów, potem
zatrzymuje się i~odwraca. Strażnik właśnie pojawia się na widoku. Celuje
ostrożnie w~cienką łatę i~strzela. Rozbija się jak szkło i~niebieski,
bulgoczący płyn się wylewa, smarując podłogę. Strażnik ślizga się na nim
i przewraca, potem zrywa się na nogi i~zaczyna zdzierać mundur i~wołać o~pomoc. Dee może wyczuć blokadę przed nimi, grubą i~elastyczną, może
kordon straży. Nie jest pewna przy tej odległości.

Tuż obok jest eliptyczna dziura w~ścianie. Ktoś nabazgrał na nią
,,WYJŚCIE EWAKUACYJNE?!'', Dee patrzy na to, patrzy na Axa i~podnosi
pytająco brew. Ax kiwa głową.

Dee zagląda. To ciemna zjeżdżalnia, opadająca ostro w~dół i~znikająca z~widoku. Wchodzi, kładzie się na płaszczu i~puszcza się górnych krawędzie
dziury.

Natychmiast pogrąża się w~dół i~wiruje dookoła czegoś, co wydaje się
prawie jak pionowy spiralny zjazd. 

-- AAAAAAAAAAA! -- zauważa. 

Jej krzyk
jest całkiem mimowolny, ale wychodzi zbyt późno, żeby zniechęcić Axa,
który podąża za nią skromną sekundę później. Jego obcasy są
niebezpiecznie blisko jej zakapturzonej głowy. Garbi się do przodu,
tylko żeby zobaczyć, że spadek jest jeszcze bardziej przerażający. Jej
kostki są skrzyżowane, jej ręce ściskają płaszcz przed udami. To
wszystko, co może zrobić, żeby nie skulić się w~kulę. Ściany rury są
miejscami przezroczyste, w~niektórych chwilach widzi, lub myśli, że
widzi, widok nad dachami miasta, w~innych zauważa wnętrza pokoju z~zaskoczonymi twarzami ich mieszkańców patrzących prosto na nią przez
ułamek sekundy. Zaczyna czuć przypalenie tkaniny płaszcza.

Jej inne zmysły są całkowicie pogmatwane. Wycofuje się od oderwanej
perspektywy Sys, która już uruchamia pierwsze kroki procedury
katapultowania się, przygotowując się na uszkodzenie systemów
somatycznych. Dee widzi krótki, chłodny obraz jej komputera
odłączającego się od pozostałości jej zwierzęcego mózgu i~wyczołgującego
się z~krwawego wraku jej czaszki.

Wtedy zaczyna ślizgać się znacznie wolniej, na otwartej przestrzeni.
Światło świeci na jej zamknięte oczy. Otwiera oczy i~ciągle ślizga się,
świszcząc, ale zwalniając\ldots napina ramiona i~dokładnie zgodnie z~Newtonem, uderzają w~nie obcasy Axa. Światło słoneczne, otwarte
powietrze, krzyczący ludzie.

Dee babra się i~zatrzymuje. Wszystko ciągle wiruje. Siada i~się
rozgląda. Ax jest kilka metrów dalej, oczy ciągle zamknięte, usta
otwarte. Są na dnie łagodnego nasypu czarnego, zeszklonego materiału
przy podstawie wieży, na placu. Pomiędzy ławkami, fontannami i~wejściami
do innych budynków, ludzie się gapią na nią.

Tuż koło jej prawej dłoni pojawia się centymetrowa dziura w~czarnym
szkle. Od niej rozchodzą się pęknięcia. W~tym samym czasie, słyszy
miękkie \emph{puk}.

Kolejna dziura, bliżej.

-- Ona\ldots \emph{to}!

Dee skacze na nogi, chwieje się, łapie Axa za kostkę i~wyciąga go przez
brzeg spadku. Upada z~wysokości pół metra z~uderzeniem. Krzyczy i~otwiera oczy. Dee patrzy na twarze w~wieży, widzi czarne postacie
rzucające się na balkonach wysoko nad nimi. Strzela kilka razy w~górę,
według ogólnej zasady, potem dźwiga Axa na stopy.

-- Biegnij!

Oboje są ciągle tak oszołomieni, że unikanie, skręcanie, upadanie i~toczenie przychodzi im całkiem naturalnie. W~ciągu sekundy lub dwóch są
pomiędzy teraz krzyczącymi pieszymi na placu, choć jeszcze nie poza
stożkiem ognia ze szczytu wieży.

Rzeczy ciągle krążą wkoło i~dookoła. Ax uderza w~ludzi, ale kontynuuje
postęp kuli bilardowej przez plac. Dee zmusza wirujące zmysły do
stabilności i~biegnie prosto do wejścia, które ma daszek. Dociera do
jego powitalnego cienia i~patrzy do tyłu. Ax, ku jej zupełnego
horrorowi, wdał się w~bójkę. Trzy dziewczyny w~stroju sekretarki
zamierzają się na jego głowę i~kopią go w~kostki, podczas gdy on uderza
je głową w~przeponę i~naciska na ich stopy, okłada pięściami uda.

Dee wyskakuje z~osłony z~krzykiem banshee i~łapie garść długich blond
włosów. Szarpie głową dziewczyny do tyłu, sięga w~bójkę drugą ręką i~wyciąga Axa za kołnierz, póki nie jest za nią. Potem z~zamachem obu
ramion pcha dziewczyny razem na kupę i~dobiega do Axa, który bardzo
mądrze wybrał bieg ku tej samej osłonie.

Patrzy na zarumienioną ciemną twarz Axa.

-- Uciekaj! -- krzyczy.

-- Gdzie?

-- Za mną!

Mapy tańczą przed jej oczami. Żołnierka przerzuca je pomiędzy wizją i~zaznacza drogę, halucynując znaki drogowe przed nią. Biegnie po schodach
budynku, za róg, przez parking i~nad balustradą w~obrzydliwą alejkę.
Kałuże bryzgają pod stopami. Ax dyszy koło niej.

Wirtualna strzałka wskazuje na drzwi w~ścianie. Dee potrząsa klamką.
Zamknięte. Wygrzebuje pistolet, ale Ax zatrzymuje jej dłoń. Uśmiecha się
do niej i~obraca się na pięcie stopy, kopiąc mocno w~drzwi drugą.
Otwierają sią z~hukiem, pokazując klatkę schodową. Strzałka mapy świeci
na schodach jak ślady stóp pozostawione przez gigantycznego
radioaktywnego ptaka idącego w~tę stronę. Dee patrzy na lewo i~prawo. Na
końcu parkingu, głowa gładko się cofa.

Dee strzela w~róg, za którym zniknęła głowa, mając nadzieję, że latające
odłamki mogą zniechęcić do dalszego podglądania, i~wchodzi po schodach.
Ax kilka razy następuje na końce jej płaszcza. Z~oburzeniem szarpie je.

Po dwudziestu pięcioma betonowymi stopniami pojawiają się w~wielkiej
przestrzeni piwnicznej z~minimalnym luzem nad głową Dee. Słabo
oświetlone przez organiczne światło, sala przypomina podziemny parking,
choć nie ma wystarczająco dużo pojazdów w~tym rejonie, by usprawiedliwić
takie wykorzystanie. Zamiast tego jest zawalone starą maszynerią,
zwojami rur i~-- ku zdumieniu Dee -- najwidoczniej modułowymi komponentami
statku kosmicznego. Wie, że wieże miasta częściowo wyrosły z~części
oryginalnego Statku, ale to potwierdzenie jest prawie szokujące. To
jakby przybyła do samego dna jej świata. Stąd, \emph{nie ma drogi w~dół}.

Słyszy ruch na górze stopni, odwraca się i~posyła kolejną kulę. Uderza i~rykoszetuje po klatce, bardzo zadowalająco. Potem biegnie. Jej
instynkty, oraz kierujące strzałki, prowadzą ją w~tym samym kierunku,
przez piwnicę w~kierunku zapachu wody.

Nie mogą biec w~prostej linii. Ich bieg kluczy pomiędzy skrzyniami i~kawałami sprzętu, którego boki kosmicznych śmieci są oznakowane
ostrzeżeniami, instrukcjami i~znakami, Dee zauważa ,,Kosmiczni Kupcy,
Karaganda'' i~,,Projekt Jowisz'', a~część jej umysłu ma czas, żeby
podziwiać te starocie. Za nią i~Axem, pośród echa dźwięków i~pisków
interferencji elektomagnetycznej, Dee wykrywa pościg. Więcej niż jedna
osoba, poruszająca się z~szybką rozwagą.

Przed nimi na poziomie podłogi jest linia światła. Strzałki, które jej
oprogramowanie kierujące wkleja w~jej wizję, kończą się tam, błyskając.
(Jakby nie zauważyła). Gdy podbiega, pinguje system sterowania szerokich
zwijanych metalowych drzwi. Z~głośnym piszczeniem i~tarciem, zaczynają
się unosić. Po podniesieniu na trzydzieści centymetrów, zatrzymują się.
Dee odbija radarem od nich, bez zmiany.

Kropka światła laserowego pojawia się na nich. Dee pada, przewracając
Axa tak, że toczy się do lądowania dla niego miękkiego, choć nie dla
niej. Wytacza się spod niego, wpółsiedząc, i~strzela wzdłuż pustej
drogi, w~kierunku wykrytego ruchu. Prędko wbija kolejny magazynek do
pistoletu i~znowu strzela. Błyski odpowiadają i~kula śwista koło jej
nosa. Opróżnia magazynek w~losowych kierunkach. Ścigający czmychają za
skrzynie, a~Dee znowu się toczy i~czołga się do przerwy pod drzwiami.
Jest dla niej zbyt nisko.

-- Idź przodem! -- szepcze do Axa. Nie potrzebuje popędzania. Toczy się
pod drzwiami i~skacze na bok.

Słyszy jego krzyk: 

-- Nie! -- a~potem zapada cisza. Para mechanicznych
stóp pojawia się w~przerwie, krocząc do środka drzwi. Metalowe pazury
sięgają pod drzwi i~podnoszą. Drzwi zwijają się i~plączą się w~górę jak
listwowe żaluzje. Cokolwiek podnosi drzwi, w~tym samym czasie obniża
swoje ciało, pomiędzy nogami. Linia cząstek pyłu rozświetla się nad jej
głową, gdy laser o~przemysłowej mocy uderza w~ciemność piwnicy.

Teraz zdesperowana, Dee wyjmuje pusty magazynek i~wsadza kolejny, który
wygrzebała z~torby. Zdecydowanie jest na wyczerpaniu. Odwraca twarz ku
jej nowemu przeciwnikowi. To przysadzisty, kucający robot. Jego laser,
wystając pomiędzy górnym a~dolnym kadłubem, porusza się, mierzy i~znowu
strzela. Za nią jest wrzask, znacznie za blisko.

-- Myślę, że oślepiłem łowców nagród -- mówi robot. -- Ale myślę, że
powinnaś uciekać.

Dee patrzy się na to przez chwilę, a~potem rozpoznaje robota, który
towarzyszył Wilde'owi poprzedniego wieczoru.

-- Och, to Ty -- mówi niewdzięcznie i~się wydostaje. Robot pozwala drzwiom
opaść z~grzechotem i, dla pewności, pali mechanizm zamykający ładunkiem
z bliska. Stoją na nabrzeżu z~tyłu i~u dołu budynku, z~widokiem na kanał
szeroki na pięćdziesiąt metrów pomiędzy tyłami innych budynków. Kanał
jest pusty, prócz kilku długich automatycznych barek zmierzających w~ich
nieświadomych sprawach w~świecie nieco bardziej wymagającym niż
rzeczywistość zabawkowa pierwszych eksperymentów z~AI. Pod drzwiami mogą
się palić światła, ale to tylko kontrast. Jest tutaj mrocznie,
prawdopodobnie tak jest także w~jaśniejszych czasach niż zmierzch. Ax
stoi niepewnie nieco dalej, patrząc podejrzliwie na robota. Jego ubrania
są podarte. Tam, gdzie Robot go złapał, domyśla się Dee.

-- Jesteśmy ok -- mówi mu. -- Myślę.

-- Z~pewnością nie chcę cię skrzywdzić -- mówi Robot. -- Nie mam zamiaru
donieść na Ciebie, jak sądzę, moje działania to pokazały. -- Macha
kończynami, wskazując opływową łódź z~potężnym silnikiem zaburtowym i,
najbardziej mile widziane, małą, ale ukrywającą kabiną.

-- Chodź ze mną -- mówi Robot. -- Mamy dużo do zrobienia.

-- Ta -- mówi Ax. Chowa broń do już poszarpanej koszuli. -- Po prostu
\emph{spójrz} na stan jej ubrania.

~

Gdy łodzie sojuszników w~sporze wyszły z~głównego systemu kanałów i~z ludzkiej dzielnicy pomiędzy łachy i~bagna, łódź Tamary przesunęła się na
przód. Do czasu kiedy nie byli już w~rozpoznawalnych kanałach, ale
pokrytych trzcinami strumieniach i~ledwie nawigowalnymi rowami, przejęła
prowadzenie. Gdzieś daleko w~kierunku centrum miasta, poduszkowiec
ryczał przez równinę, posyłając ptaki uciekające w~powietrze na
kilometry wokoło. Klucz gęsi przeleciał nad głowami, złote kropki na
ciemnoniebieskim niebie.

-- Rzeczy, które widzę, kiedy nie mam strzelby -- westchnęła Tamara.

Wilde uderzył w~owada. 

-- Dlaczego kurwa -- zażądał -- musieliśmy przenieść
pierdolone \emph{muszki} przez międzygwiezdną przestrzeń?

-- Ekologia -- powiedziała Tamara z~odrobiną zadowolenia z~siebie. Podała
mu tubkę środka przeciw owadom. Wilde wtarł go i~spędził kolejne kilka
minut, triumfując, gdy małe czarne diabły lądowały na jego skórze i~odpadały martwe, prosto do jakiegokolwiek piekła czekającego na ich złe,
dwubajtowe dusze. W~pewnym zakresie przedstawił to nieortodoksyjnie
teologicznie stanowisko Tamarze, rozśmieszając ją i~uspokajając.

Opowiedziała mu o~jej zawodzie łowcy biomechanizmów i~jej aktywności
politycznej w~ruchu abolicjonistów. Prócz naciskania na szczegóły
systemu bankowego, rzeczywistą formę organizacji abolicjonistów i~ich
społeczne cele, nie był złym słuchaczem. Potem położył się na dziobie
łodzi i~przeglądał przez notatki Eona Talgartha o~Jonathan Wilde. Czasem
się skrzywił, znacznie częściej głośno się śmiał. Ethan i~Tamara
nalegali, żeby powiedział, co było śmiesznego, i~co jakiś czas im mówił.
Po jakimś czasie ucichł, usiadł i~patrzył na wczesne strony dokumentu,
potem na koniec, a~potem znowu na początek. W~końcu schował je w~plecaku
Tamary i~siadł, odwracając wzrok od innych, ponad wilgotną pustynią,
gdzie zachód słońca kładł kolor rudy jak pole krwi.

Miasto Statku było w~tropikach Nowego Marsa. Ciemność nadeszła w~ciągu
minut po zniknięciu słońca za horyzontem. Wilde uśmiechnął się do Tamary
i Ethana i~zapalił papierosa.

-- Dziwne -- powiedział -- móc widzieć w~ciemnościach. -- Znowu się
rozejrzał. -- Cholera! Nie mogę!

-- Osłoń papierosa -- powiedział mu Ethan. -- Oślepia Cię.

-- Cholernie blisko oślepienia mnie -- powiedziała Tamara. -- Nie, nie, po
prostu ukryj w~dłoniach, to wystarczy.

Wilde zrobił, jak proszono i~krótko potem wyrzucił niedopałek w~wodę i~spojrzał w~gwiazdy. Ze światłami ludzkiej dzielnicy za nimi i~mniej
uporządkowanymi oświetleniem oraz nieprzewidywalnymi losowymi flarami
Piątej Dzielnicy niedaleko przed nimi, gwiazdy były mniej przytłaczające
niż ich pierwszy widok poprzedniej nocy, ale niemniej jednak imponujące.
Sapnął na szepczący lot bolidu, mrugnął na błysk, który zrobił za
zachodnim horyzontem.

-- Robot nazywał coś jak to ,,wodospadem'' -- powiedział do Ethana. -- Co to
znaczy?

-- Lód z~komet -- wyjaśnił lakonicznie Ethan. -- Zasila kanały.

-- To raczej wolne terra-formowanie -- dodała Tamara. -- Planeta nadaje się
do zamieszkania, jasne, ale chcemy więcej wody i~grubszą atmosferę.
Zabierze nam to kilka więcej wieków, ale wtedy będzie tak zielona jak
Ziemia kiedykolwiek była. -- Przerwała, jak gdyby trochę się uniosła. -- Przynajmniej, tak mówi Reid.

-- Zastanawiam się -- wymamrotał Wilde -- jak zielona jest teraz Ziemia.
Cokolwiek ,,teraz'' oznacza.

-- Ach -- powiedział szybko Ethan. -- Mogę Ci to powiedzieć. -- Zrobił pokaz
patrzenia na zegarek. Tamara i~Wilde się roześmiali, tak głośno, że
głowy odwróciły się w~pojedynczej linii łodzi wyciągniętych za nimi w~wąskiej arterii wodnej.

-- Nie, nie -- kontynuował Ethan. -- Poważnie. ,,Teraz'' to dwa czasy.
Absolutny, jeżeli taka rzeczy istnieje: chuj go zna. Ten sposób: jeżeli
masz sygnał z~Układu Słonecznego, to byłby bardzo długo w~drodze.
Tysiące lat, miliony, chuj wie. Jednak jeżeli wróciłbyś przez Milę
Malleya, to jest bramę córki-tunelu czasoprzestrzennego, prawda,
wróciłbyś prosto do 2094 \emph{anno domini} plus czas Statku. Sześć
przecinek cztery gigasekundy, pomyślmy\ldots och, dwa tysiące trzysta
dziewięćdziesiąty, wczesny dwa tysiące czterechsetny\footnote{obliczenia wydają się błędne, 6,4 gigasekundy to ca 202 lata, zatem 2094 + 202 to 2296--2300, zamiast podanego zakresu 2390--2400. Pozostawiam bez zmiany na potrzeby wewnętrznej chronologii tetralogii -- przyp.tłum.}, może. Więc jest tam
teraz dwudziesty piąty wiek.

-- Dwudziesty piąty wiek! -- Wilde się roześmiał. -- Tak, Ziemia mogłaby
być Zielona! Lub nawet Czerwona!

Nie złapali tego, a~on nie wyjaśniał. Zmarszczył brwi na Ethana Millera.

-- Dlaczego ,,córka-tunel''? -- spytał

Ethan wzruszył ramionami. 

-- Tak nazywał to mój ojczulek. Przeszedł i~nie
jako pierdolony robotowy upload. Był załogą, nie kryminalistą. -- Uderzył
się w~pierś. -- Ludzie przez całą drogę, to ja.

-- Węglowy szowinista -- fuknęła Tamara.

Wilde pochylił się do przodu, bez namysłu zapalając kolejnego papierosa.

-- Mów dalej.

-- Cóż -- powiedział Ethan, machając dłonią w~powietrzu -- tunel
czasoprzestrzenny, którym przeszliśmy, był odpryskiem. -- Ustawił dłoń
bokiem. -- Główna sonda, ta, którą zbudował szybki ludek, zanim ich
umysły się wypaliły, poszła dalej. Ciągnąc swój koniec tunelu do...
gdziekolwiek. Musi tam już być do tego czasu. -- Roześmiał się cierpko. -- Cokolwiek ,,teraz'' znaczy, jak mówiłeś.

Wilde oparł się do tyłu, zaciągając się papierosem tak mocno, że jego
złożone dłonie nie mogły ukryć blasku.

-- Koniec czasu -- powiedział.

Myślał o~tym przez kilka chwil.

-- Och, cholera -- powiedział.

-- W~czym problem? -- spytała Tamara. Zmniejszyła obroty silnika i~łódź
zaczęła dryfować ku cyplowi.

-- Czas -- powiedział Wilde. -- Jak w, nie mamy go dużo.

-- Cóż, -- powiedziała Tamara, gdy łódź uderzyła w~brzeg -- jesteśmy w~Piątej Dzielnicy. Zbierajmy się.

\chapter{Doświadczenie Śmierci}

Annette miała rurki w~prawym ramieniu, ja w~lewym. Jej lewa dłoń
sięgnęła i~złapała moją prawą.

-- Boisz się? -- spytałem.

-- Trochę.

-- Ja też. -- Ścisnąłem w~odpowiedzi.

Sala miasta była pełna dojrzałych ludzi, starszych ludzi, ludzi jak my.
Na plecach na łóżkach patrzących na panele dachowe. Zielono zabarwione
światło, technicy w~zielonych fartuchach, wszystko powolne: poczucie
zanurzenia. Wielkie maszyny podłączone do rurek przesączały małe maszyny
do naszej krwi. Nie nanotechnologia, nie pełna odnowa komórek, jeszcze
nie, ale dało nam to szansę życia, aż to nadejdzie. Przy siedmiu
dekadach dotychczasowego życia, nasza oczekiwana długość życia już
przedłużyła się do przynajmniej kolejnych czterech. Czuliśmy się lepiej
niż kiedy mieliśmy pięćdziesiąt lat. Patrzyliśmy, cóż, wczesne leczenie
przeciw starzeniu sprawiało, że skóra była twardsza jak również napięta,
więc wyglądaliśmy na nieco wysuszonych na słońcu, nieco \emph{uwędzonych}.

To leczenie było inne. Nie mieliśmy go wcześniej, choć miałem iniekcję
mikrobotów, żeby poradzić sobie z~niepokojącym przerostem prostaty kilka
lat wcześniej. Teraz, mikroboty rozszerzyły swoje możliwości, i~w ramach
tych kompromisów charakterystycznych dla Republiki, państwowa służba
zdrowia oferowała te możliwości obywatelom w~zamian za ich prawa do
państwowej emerytury. Umowa była bardziej polityczna niż ekonomiczna,
ale miał pewną elegancką symetrię: wymień emeryturę na długość i~stopień
odmłodzenia i~możesz pracować, aż padniesz.

To by nigdy nie przeszło pod starymi prawami. Było ryzykowne. Jedna lub
dwie osoby na tysiąc umierały przy tym, choć czy umarli od tego, było
inną kwestią. To był problem z~sercem, trudny do przewidzenia. Jeżeli
takie miałeś, wkrótce mogłoby cię dopaść. Tak mówiły firmy medyczne i~państwowa służba zdrowia.

Techniczka przeszła pomiędzy naszymi łóżkami, delikatnie rozłączyła
dłonie.

-- Gotowi? -- spytała.

-- Tak -- powiedziała Annette.

-- Gotów jak zawsze -- powiedziałem. Spróbowałem się uśmiechnąć. -- Kto
chce żyć wiecznie?

-- Cóż, wiem, że Ty, obywatelu Wilde. Powodzenia.

Oto nicość, pomyślałem.

Nacisnęła przycisk, wysyłają sygnał radiowy niskiej mocy do mikrobotów w~krwi mojej i~Annette.

Poczułem, jak moje serce staje. Musiało. Mikroboty potrzebowały
stabilnej platformy do szybkiej pracy nad nerwem błędnym, żeby dać im
szansę przepchnięcia czynników wzrostu nerwów i~sklonowanych
macierzystych komórek nerwowych przez barierę mózg-krew.

Kolory zblakły, potem światło. Świadomość kompletnie się wyłączyła, jak
we śnie. Moje serce ponownie się uruchomiło z~bolesnym wzrostem mocy i~świadomość powróciła, wyłączyła się, przywróciła się z~pamięci i~ponownie wróciła. Podniosłem słabo głowę i~spojrzałem na Annette, która
otworzyła oczy, popatrzyła na mnie i~się uśmiechnęła.

-- Udało nam się -- powiedziała.

-- Uda nam się -- powiedziałem. -- Uda nam się dotrzeć na statki.

Spróbowałem usiąść.

-- Jeżeli nie zostaniesz tam, gdzie jesteś, przez następne pół godziny -- upomniała techniczka -- nie uda ci się dotrzeć do \emph{drzwi}.

~

Na zewnątrz, na ulicy Greenbelt, pod niebem szklarni. Przeszliśmy przez
zwyczajową pikietę Pro-Life, która ciągle krzyczała na nas ,,Mordercy!''
zza linii uzbrojonej Straży Republikańskiej. Chodziło o~tkanki płodowe -- sklonowane z~naszych własnych komórek -- które rzekomo mordowaliśmy,
zgodnie z~ulotką Społeczeństwa dla Ochrony Nienarodzonych Dzieci\footnote{w~oryg. Society
for the Protection of the Unborn Child, w~skrócie SPUC, rzeczywiście
istniała,
zob.~\url{https://en.wikipedia.org/wiki/Society_for_the_Protection_of_Unborn_Children}-- przyp-tłum.}, którą
jakaś biedna, zepsuta dusza wepchnęła mi w~twarz. 

-- SPONDujcie się!\footnote{ w~oryg. SPUC off, gra wulgaryzmem i~nazwą stowarzyszenia -- przyp.tłum.}  -- zawołałem. -- \emph{Ty} trafisz do piekła!
\emph{My} nawet nie zamierzamy \emph{umrzeć}!

-- Czy chcesz złożyć skargę, obywatelu? -- spytał mnie najbliższy
Strażnik, bez odwracania.

-- W~porządku, oficerze -- powiedziała Annette, łapiąc mnie za łokieć i~ciągnąc mnie dalej. -- Wolność słowa\ldots a~Ty się zamknij! -- dodała do
mnie.

-- Ok, ok. -- Szedłem szybko, wewnętrznie drżąc. Nic -- ani komuniści, ani
faszyści, ani autorytarianie dowolnej maści -- nigdy nie wzbudzało we
mnie takiej morderczej wściekłości jak prolajferzy. Kiedykolwiek
trafiałem na nich korzystających ze swoich praw, byłem cholernie pewny,
że skorzystam ze swoich.

~

Przyzwyczaiłem się żyć tutaj, co było oficjalnie nazywane ,,nieoficjalnym''
sektorem: obrzeża slumsów Londynu, gdzie eksperymenty Republiki w~lokalnych rządach nakładały się na eksperyment anarchokapitalizmu, który
sprawiał, że strefy przedsiębiorcze Ruchu Kosmicznego wyglądały na
przeregulowane. Drugie, trzecie i~następne kondygnacje większości
budynków były refleksją. Organiczne farmy sprawiły, że brak rur
kanalizacyjnych był czymś mniej niż katastrofą, ale nie sprawiły, że
zbiorniki fekaliów mniej śmierdziały. Opary wydechu śmierdziały.
Populacja była mieszanką lokalnego marginesu i~uchodźców z~wojen Europy
i Azji. Niewielu żebraków, ale byli dostatecznie niepokojący: ludzie,
których obrońcy poskąpili na ich polisy ubezpieczenia jądrowego.

Tak jak mówiłem, przyzwyczaiłem się do tego, ale w~tej chwili -- efekt
uboczny kliniki lub pikiety -- to wszystko było za dużo.

-- Czuję się strasznie -- powiedziałem. -- Głowa mnie boli, czuję jakby
ktoś napompował mi żołądek.

-- Och, przestań jęczeć -- powiedziała Annette. -- To nie jest gorsze niż
kac.

-- Co za radosna myśl -- powiedziałem. Przed nami był pub na chodniku. -- Pół litra Amstela byłoby strzałem w~dziesiątkę.

Annette pomachała broszurą Ministerstwa Zdrowia przede mną. 

-- Tu jest napisane\ldots

-- Tak, wiem, co tam jest. Czy wyglądam, jakbym miał zamiar używać broni
lub ciężkiego sprzętu?

-- Mniemam, że nie. -- Umiechnęła się i~opadła na plastikowe krzesło,
niebezpiecznie blisko rynsztoka. -- Pils dla mnie. I~te kebaby wyglądają
dobrze.

Wykrzyczałem zamówienie do \emph{garsona}, który zniknął w~klapie i~pojawił się minutę później. Nad klapą był zwyczajowy plakat Abdullaha
Ocalana\footnote{Abdullah Ocalan, ps. „Apo'' (ur. 4 kwietnia 1948 w~Halfeti) -- lider Partii Pracujących Kurdystanu (PKK),
zob.~\url{https://pl.wikipedia.org/wiki/Abdullah_\%C3\%96calan
} -- przyp.tłum.}. Nigdy nie mogłem zrozumieć, dlaczego nawet uchodźcy z~Demokratycznego Kurdystanu -- przedsiębiorcy do kości -- ciągle szanowali
Wielkiego Przywódcę. Prawdopodobnie w~miasteczkach odbywały się
wymuszenia. Zrobiłem notatkę w~głowie, żeby to sprawdzić. Mogły być w~tym pieniądze dla firmy obrony, która mogłaby zaoferować im lepszą umowę
niż wymuszenie haraczu przez Partię. Lub mogłem całkowicie źle rozumieć
sytuację, nacjonalizm jak zawsze był mi obcy.

Tłum, w~większości Turcy i~Kurdowie, opływali dookoła pubu na chodniku.
Za nami bestie i~pojazdy podążały zgodnie z~jakimś niepisanym kodeksem
autostrady, w~którym pierwszeństwo zależało od współczynnika pędu i~hałasu. Telewizor przy klapie pokazywał program z~Istambułu. Nad nimi,
sterowce dryfowały do odległych masztów Portu Alexandra. Oparłem się,
rozgrzany przez słońce i~rozprzestrzeniające się ciepło jedzenia i~picia.

-- Śniłeś? -- spytała Annette.

Pokręciłem głową. 

-- A Ty?

-- Myślałam, że tak -- powiedziała Annette, uśmiechając się tajemniczo. -- Usłyszałam ciepły, przyjazny głos, zobaczyłam białe światło i~pamiętam
myśl: ,,Wspaniale! W~końcu mam doświadczenie śmierci!'', i~wtedy światło
było tylko słońcem, a~głos był techniczki, odliczającej.

-- To prawda -- powiedziałem. -- Światło słoneczne jest białym światłem. -- Ten materialistyczny wgląd był wszystkim, co przetrwało haj na
magicznych grzybkach, które wziąłem, będąc studentem. To i~wizja trzech
bogini: Matki Natury, Pani Szczęście i~Panny Wolności, które były -- zrozumiałem to po wyjściu z~tripa -- koniecznością, szansą i~wolnością, i~w istocie władczyniami wszystkiego.

-- Wyobraź sobie -- powiedziała Annette -- że byliśmy najbliżej
kiedykolwiek umierania!

-- Odstukaj w~plastik! -- Zapukałem w~stół. Roześmialiśmy się, zaciśnięte
dłonie ponad stołem. Spojrzałem na jej twarz, postarzałą, ale nie
pogorszone, jej linie mapą jej życiowego śmiechu i~smutku, i~poczułem,
że mógłbym kochać ją na zawsze.

-- ,,Aż wyschną wody mórz, Najdroższa, I~aż się w~słońcu stopi głaz\ldots''\footnote{pieśń Roberta Burnsa napisana w~1794 roku,
zob.~\href{https://en.wikipedia.org/wiki/A_Red,_Red_Rose}{wikipedia.org},
tłumaczenie na podstawie
\href{https://lyricstranslate.com/en/red-red-rose-mi\%C5\%82a-ma-jak-czerwona-r\%C3\%B3\%C5\%BCa.html}{lyricstranslate.com} -- przyp.tłum.} 

-- Och przestań, zanim cię zgłoszę za starość.

Ruch i~hałas zamarły. Spojrzałem ponad zwalniającymi samochodami i~pomyślałem, że wszyscy patrzą nas. Patrząc w~drugą stronę, ujrzałem, że
patrzą na telewizor. Komentarz i~głośne konwersacje, które nagle
zastąpiły ciszę, były wszystkie po turecku i~kurdyjsku. Jednak obraz
telewizyjny nie wymagał tłumaczenia: niemiecki czołg, polski znak
drogowy.

~

Berlin -- dwudziestopierwszowieczny Berlin, przedwojenny stary Berlin -- był najbardziej ekscytującym miastem w~Europie. Boom budowlany po
zjednoczeniu skończył się już, ale intensywność interesu i~przyjemności
nie straciły tempa. Każdy, kto był kimkolwiek, był tam lub w~Londynie. W~pewnym sensie dwie stolice poruszały się w~przeciwnych kierunkach, jedna
odzyskująca swoją narodową pewność siebie, druga rezygnująca z~imperialnych aspiracji. Jedna, jak się okazało, dozbrajająca, druga
rozbrajająca...

Właśnie teraz w~Berlinie była jedyna osoba, na której mi zależało:
Eleanor, była tam z~partnerem na długim weekendzie.

-- Co się \emph{robi} w~czasie wojny, Jonathanie?

Dziewiętnastoletnia córka Eleanor, Tanya, brzmiała bardziej na
zainteresowaną niż przestraszoną. To było jedno z~tych nadzwyczajnych
rodzinnych zebrań dookoła telefonów i~telewizorów, które odbywały się w~całej kraju przez kilka pierwszych godzin konfliktu. Nasze było we
frontowym pokoju Eleanor w~Finsbury Park. Jej nieobecność była
wszechobecna. Wielu z~naszych przyjaciół, i~innych krewnych, również
było w~Berlinie. Ludzie dzwonili do nich na wszystkich możliwych
kanałach. Miałem program wywołujący, który śledził Eleanor, i~próbowałem
w tym samym czasie prowadzić spotkanie wykonawcze, częściowo, by
przestać o~niej myśleć. Łączność, nie ku mojemu zaskoczeniu, była wolna.

Co się \emph{robi} w~czasie wojny? Z~czterema pokoleniami
antymilitarystów za nią, myślałbyś, że dzieciak by wiedział.

-- Sprzeciwiasz się -- powiedziałem. To nie wydawała się bardzo oświecająca
odpowiedź. Ustawiłem kody do kolejnej próby na łączu konferencyjnym.

Angela, najstarsza Eleanor, roześmiała się. 

-- Jesteś niepoprawny. -- Rozdawała kubki kawy i~herbaty. Dobra dziewczyna. \emph{Ona} wiedziała,
co robić w~czasie wojny.

-- Moi dziadkowie byli kontestatorami wojennymi\footnote{w~oryg.
,,conscientious objectors'' czyli osoby odmawiające służby wojskowej ze względu na
przekonania, w~słowniku języka polskiego przyjęto kalkę z~języka
angielskiego w~postaci obdżektora,
zob.~\url{https://sjp.pwn.pl/sjp/obdzektor;2569445.html},
proponuję ,,kontestator wojny'' czyli osoba sprzeciwiająca się wojnie
zob.~\url{https://pl.wiktionary.org/wiki/kontestator,} -- przyp.tłum. } w~Pierwszej Wojnie Światowej, moi rodzice w~Drugiej, i~będę przeklęty, jeżeli stracę szansę zrobić to samo w~Trzeciej. -- Serwer
nie odpowiadał. Westchnąłem i~wprowadziłem komendę przełączenia.

-- Ta -- powiedziała Annette, opierając plecy o~moje piszczele. -- Kontestator wojny z~możliwością nuklearną.

-- \emph{Osłoną} nuklearną -- skorygowałem. -- Tak czy inaczej, do tego nie
dojdzie. Niemcy nie mają broni jądrowej.

-- Tak mówią.

Annette przełączała kanały, oglądając relację CNN z~polskiego frontu,
wywiady na żywo WDR z~Berlina, Wiadomości Channel 4 z~regionalnych
zgromadzeń i~Państwowego Zgromadzenie Federalnego Brytanii. Przy ich
transportowcach czołgów w~poduszkowcach, postępy Niemców były
najszybsze, jakie kiedykolwiek były widziane. Używali dronów bojowych jak
Chomeini i~Mao używali ludzi. Nie byliśmy na wojnie, jeszcze. Było wielu
w partiach opozycji, którzy chcieli, żebyśmy byli. Twarz Lorda Ashdowna
pojawiała się zbyt często jak na moje gusta.

-- Nie, tak mówi FIS, i~powinni cholernie dobrze wiedzieć, to ich skóra
będzie przypalona, gdy\ldots ach!

Miałem połączenie. Obrazy w~skali 0,1 stołu z~innymi dookoła niego
zabłysły przed ekranem na moich kolanach. Z~komitetu w~okresie wyborów,
zostaliśmy tylko Julie O'Brien i~ja. Reszta była nowymi twarzami. Prawie
dekada społecznych i~politycznych przewrotów -- Rewolucja, jak wszyscy ją
teraz nazywali -- przesiała libertariańską kadrę Ruchu Kosmicznego,
większość z~których była zorganizowana w~WolnymKosmosie. Niektórzy z~najlepszych poszli za Aaronsonem i~Rutherfordem do Woomera, gdzie
republiki brytyjskie i~australijskie prowadziły swój połączony program
kosmiczny. Inni uciekli do konwencjonalnej polityki, zwykle
republikańskiej, ale okazjonalnie na dziksze brzegi, nawet do
wskrzeszonej trockistowskiej Partii Władzy Robotniczej lub
rozpowszechniających się kampanii dla jednej sprawy. Zostałem z~twardogłowymi, młodymi Turkami (ha!), którzy postrzegali mnie jako
niebezpieczne umiarkowanego.

-- Ok, towarzysze -- powiedziałem. -- Ktokolwiek teraz poświęca całą uwagę
na to spotkanie, lepiej natychmiast włączy telewizor, ponieważ musimy
tego pilnować jednym okiem. Bez wątpienia szerszy ruch kosmiczny będzie
w każdym miejscu na wojnie, i~tak to powinno wyglądać, ale my w~WolnymKosmosie jesteśmy odpowiedzialni za zajęcie stanowiska, w~imię
wolności, jeżeli nie kosmosu. Sympatyzuję z~Niemcami, nie można
oczekiwać, że będą ciągle przyjmować uchodźców, opad i~terroryzm. To
raczej satysfakcjonujące, obserwować porażki Polaków, szczególnie po
tym, w~jaki sposób traktowali swoje mniejszości. Wszelako. Twierdzę, że
to wojna imperialistyczna, sprzeciwiamy się każdej stronie i~robimy
wszystko, co się da, żeby trzymać Brytanię z~dala.

Powaga mojego oświadczenia była nieco osłabiona przez obserwację
przewracającej oczami Tanyi. \emph{Chodziłem na marsze pokojowe dla
podobnych Tobie}, czułem, jakbym mówił do niej. (I z~Eleanor, krzyk z~wnętrza dodał.) Chwyt Annette na mojej dłoni był ciasny, jak gdyby mogła
się wymsknąć. Pogłaskałem jej ramiona, poniżej obrazu wirtualnego i~spojrzałem na towarzyszy.

-- Obawiam się, że nie zgadzam się z~towarzyszem Wilde -- powiedział Mike
Davis, czarny dwudziestolatek z~Liverpool, którego opinie sporadycznie
szanowałem. -- To, co właśnie powiedział, jest dokładnie tym, co rząd
mówi, tak, i~jeżeli mnie zapytacie, to jest ten rodzaj liberalnego
pacyfizmu z~dwudziestego wieku, który nas wpędził w~ten bałagan na
pierwszym miejscu. Jeżeli Brytania nie porzuciłaby odpowiedzialności na
Kontynencie, Niemcy nie musieliby jej brać na siebie. Jak to jest,
najlepsze na co możemy mieć nadzieję to, że Amerykanie znowu nas
uratują.

-- Co \emph{to} za gówno? -- powiedziała Julie. -- Odpowiedzialności? Hm,
dziękuję ci towarzyszu, ale nie biorę odpowiedzialności za cholerne
państwo brytyjskie. Pacyfizm liberalny, kiedy to stało się brzydkim
słowem? Jestem libertariańską internacjonalistką i~jestem z~tego dumna.
Wojna jest wynalazkiem Państwa. Każdego dnia wybrałabym liberalny
pacyfizm przed libertariańskim militaryzmem. Neutralność,
nieinterweniowanie i~przygotowanie samoobrony, to właśnie powinniśmy
proponować, a~nie próbować kalkulować czy powinniśmy poprzeć Niemców lub
wezwać cholernych Jankesów do szarży. Czego ty \ldots -- dodała, odwracając
się, żeby pchnąć palcem awatara Daviesa -- najwidoczniej nie próbowałeś
przemyśleć!

W innym rogu ekranu światło błyskało nagląco. Eleanor się przedostała!

-- Jeżeli to wniosek -- powiedziałem sucho -- to ja popieram. W~międzyczasie, towarzysze, błagam o~przerwę na kilka minut. -- Skinąłem
głową uroczyście, wyłączyłem dźwięk i~przełączyłem się na kanał
telefoniczny.

Pojawiła się twarz Eleanor i~przerzuciłem ją na główny telewizor.
Radosne paplanie wypełniło pokój i~umilkło, gdy Eleanor przemówiła.

-- Cześć ludzie -- powiedziała. -- Przepraszam, że was wszystkich
zmartwiłam. Nie mogłam się połączyć na moim ręcznym zestawie, a~do
telefonu hotelowego jest kolejka około pięćdziesięciu osób za mną. Nie
mogę być długo. Wszyscy ok?

-- Wszyscy jesteśmy w~porządku -- powiedziała Annette. Partner Eleanor
pochylił się krótko na obrazie, uśmiechnął się i~pomachał. -- Och, witaj
Colin -- kontynuowała Annette. -- Kiedy wracacie?

Eleanor zmarszczyła brwi. Colin, za nią, uspokajał niecierpliwą następną
w linii. 

-- Nie wiem -- powiedziała. -- Lotnisko jest teraz zamknięte.
Powiedzieli, że loty wznowią jutro, ale będzie tam chaos. Równie dobrze
możemy to przesiedzieć, póki operacja się nie skończy.

-- \emph{Operacja}? -- zaskrzeczałem. -- Nie wiem, co wam tam mówią, ale
stąd to wygląda jak początek wielkiej rzeczy. Jankesi są bardzo
przeciwni, Rosjanie brzmią nerwowo, a~niektóre małe republiki, na które
napiera Europawehr, trzymają palce na bombach jądrowych. Wynoście się
stamtąd tak szybko, jak możecie. Jedźcie na lotnisko \emph{teraz}. Jeżeli
ludzi wokół was są beztroscy, to ich problem i~wasza możliwość.

Eleanor właśnie miała odpowiedzieć, gdy obraz się rozmył i~został
zastąpiony przez przepraszająco wyglądającego mężczyznę w~garniturze,
który mówił ,,Kierownik Hotelu'' tak jasno jak odznaka. 

-- Przepraszam Pana, nie możemy pozwolić na kontynuowanie tej rozmowy. -- Połączenie
zostało przerwane ku krzykom oburzenia z~naszej strony.

Tanya odwróciła się do mnie. 

-- Dlaczego kłapałeś jęzorem? Nawet nie
udało nam się z~nią porozmawiać!

-- Przykro mi -- powiedziałem. -- Naprawdę. Jednak nie sądzę, żeby
ktokolwiek tam zdawał sobie sprawę, jak to jest poważne. Może odkrycie,
że ich rozmowy telefoniczne są monitorowane, sprawi\ldots

-- Nie sprawi -- powiedziała Annette. -- Powinieneś to wiedzieć. Wszystko,
co zobaczy Eleanor, to zamazany ekran.

Po kolejnych wzajemnych oskarżeniach, w~końcu uspokojonych przez
Annette, wyszedłem z~moim sprzętem i~usiadłem na łóżku. Przez otwarte
okno słyszałem żałosne śpiewy z~jednego z~wielu fundamentalistycznych i~charyzmatycznych kościołów, które w~ostatnich czasach zgromadziły się w~okolicy. Zastanawiałem się, czy moje własne działania były mniej
daremne. Potem siła mojego sceptycyzmu do mnie wróciła. Przebiłem się.

Na spotkaniu trwała debata pomiędzy tymi, którzy chcieli naciskać na:
zaangażowanie Wielkiej Brytanii, zaangażowanie Amerykanów, neutralność i -- zupełnie znikąd -- wykorzystanie wojny jako dogodnej chwili do
rozpoczęcia powstania libertariańskiego.

Mogłem sobie z~tym poradzić.

~

Telefon dzwonił. Obudziłem się i~machnięciem włączyłem światła. Zegar
pokazywał 3:38 i~małe czerwone światełko na telefonie mrugało,
szyfrowane połączenie. Odebrałem je i~włączyłem przełącznik. Twarz Myry
pojawiła się na ekranie, czarno biała w~wojskowej czapce w~mundurze.
Wyglądała, jakby nie spała całą noc.

-- Och -- powiedziałem, nieuprzejmie, głupio i~irytująco ze snu i~rozczarowania. -- To Ty. -- Miałem nadzieję, że to była Eleanor.

-- Witaj Jon -- powiedziała Myra. -- Przepraszam, że Ci przeszkadzam, ale
jest to\ldots

-- Kto to? -- Annette zmagała się z~obudzeniem.

-- To Myra -- odpowiedziałem. -- Interesy.

Annette spojrzała na ekran, chrząknęła i~zakryła kołdrą głowę. Na wpół
usłyszałem coś jak ,,nuklearna dziwka'' i~miałem nadzieję, że Myra nie
dosłyszała.

-- Co jest?

-- To Niemcy -- powiedziała Myra. -- Szukają osłony nuklearnej i~przedstawili nam bardzo dobrą ofertę.

-- Lepiej ją przyjmijcie -- powiedziałem -- zanim przyjadą.

-- Tak właśnie myślę -- powiedziała Myra. -- Problem: mamy komplet, jak
możesz sobie wyobrazić. Niemcy oferują wykup wystarczający dla naszych
obecnych klientów, żeby się wycofali. Czy sprzedasz?

-- Za ile?

-- Pięć milion Deutschmarks, w~złocie, po przedwojennym, to jest
przedwczorajszym, kursie, bez żadnych pytań. Mam na linii niemieckiego
negocjatora, a~konto szwajcarskiego banku zostało zweryfikowane.

-- Chryste! Daj mi się zastanowić, dobra?

Uderzyłem przycisk wyciszenia/czystego ekranu, aby ukryć swoje
zmieszanie i~próbowałem szybko pomyśleć. Wydawało się dziwne, że Niemcy
nie załatwili takiej umowy, zanim rzeczywiście rozpoczęli operację
,,PRZYWRÓCIĆ PORZĄDEK'', ale może ryzyko ujawnienia ich planów temu
zapobiegło. Teraz improwizowali polisę ochrony jądrowej w~tempie
blitzkriegu.

Oferta była kusząca, nawet oprócz pieniędzy. Z~Eleanor w~Berlinie\ldots

Jednak my byliśmy tutaj. Odstraszanie jądrowe Brytyjczyków było obecnie
związane w~sporze z~USA, więc nasza, i~umowy innych prywatnych sektorów,
było wszystkim, na czym mogliśmy się opierać. Kto wiedział, czy nie
potrzebowalibyśmy tej opcji, może gdyby Eleanor wróciła bezpiecznie do
domu?

I istniała jeszcze jedna kwestia. Jeżeli sprzedalibyśmy nasz udział w~jądrówkach Kazachów Niemcom, firma WolnyKosmos byłaby niezaprzeczalnie
zamieszana w~wojnę, po stronie Niemców. Następstwa tego były
nieobliczalne i~prawdopodobnie niemiłe.

Przełączyłem włącznik wyjścia. Brwi Myry błysnęły.

-- Więc?

-- Przepraszam Myra, nie ma umowy. Nie nasza wojna i~tak dalej.

Nawet na małym podręcznym ekranie jej twarz pokazywała narastanie
zmęczenia, ale jej głos nie pokazywał wyrzutu, gdy mówiła: 

-- Rozumiem. Ok, Jon, spróbuję gdzieś indziej. Wyłączam się.

-- Dobranoc. Do zobaczenia.

Uśmiechnęła się, jakby to była beznadziejne urojenie. Jej widok
zmniejszył się do kropki.

Choć doniosła, w~retrospekcji, mogła się wydawać moja decyzja, faktem
jest, że resztę nocy spałem dobrze.

~

Następnego dnia rząd przegrał głosowanie nad wotum zaufania (z powodu
wstrzymania się od głosu tylko pięciu posłów, trzech z~Władzy
Robotniczej i~dwóch z~Socjalistów Świata) i~upadł, zastąpiony przez
bardziej radykalną koalicję czerpiącą poparcie z~mniejszych partii.
Neutralność została potwierdzona. Wyższa Izba -- wybrana teraz, ale
przejściowa mieszanka starych Lordów i~Nowych Senatorów -- debatowała
oddzielnie nad kwestią wojny i~doszła do odmiennego wniosku. Pierwsza
demonstracja prowojenna, w~Midlands, została brutalnie rozbita przez
Straż Republikańską i~bojówki Partii Władzy Robotniczej.

Była to cholerna hańba i~tak powiedzieliśmy. W~tym samym czasie, po
wygraniu dyskusji w~komitecie, zaczęliśmy organizować kampanię za
neutralnością i~wycofaniem się z~wojny. ONZ nałożył sankcje na Niemcy i~Austrię. Brytyjski ambasador wyszedł z~ONZ, gest, który nawet ja
postrzegałem jako nieszczery. Drogo to miało kosztować Republikę.

Niemcy ostrzelali artylerią Warszawę, na żywo w~CNN.

Nie słyszeliśmy nic od Eleanor przez cały następny tydzień. Nie pamiętam
spania w~tamtym tygodniu. Wojny domowe rozszerzały się jak wtórny ogień
po rozszerzającym się obwodzie postępów Niemców. Brytania zbliżyła się
do wojny, gdy problem dołączenia do mobilizacji USA/ONZ przeciwko
Niemcom stał się nie do rozdzielenia od problemu Republiki. Rząd
narastająco opierał się na poparciu ulic, gdy demonstracje przeciwko
uczestnictwu w~wojnie zwielokrotniły się, rozlały i~starły z~demonstracjami prowojennymi, które żądały powrotu starej Brytanii. Siły
prowojenne nazywały nas Hunami. My nazywaliśmy ich Hanowerczykami. Żadna
ze stron nie myślała już więcej o~drugiej jako brytyjskiej.

Niemcy dotarli do granicy ukraińskiej i~się zatrzymali. Polacy, w~rzucie
na oślep, pogrążyli się w~toczącej się wojnie domowej na Ukrainie.
Brytyjski Szef Sztabu przedstawił rządowi ultimatum. Generałowie,
liderzy partii zwolenników unii, członkowie emerytowanej, na wpół
sprywatyzowanej Rodziny Królewskiego stworzyli stały strumień gości w~ambasadzie USA. Niechętna Straż Republikańska, wykonując tylko swoją
pracę, odparła zdeterminowanych demonstrantów na Grosvenor Square. Była
mowa o~przewrocie wojskowym.

Myra znowu zadzwoniła. Oferta Niemców wzrosła do dwudziestu milionów.
Powiedziałem nie. Nie trzeba dodawać, że nigdy nie wspomniałem o~tym
reszcie komitetu.

Mój program wyszukujący prawie dosięgnął Eleanor, przynajmniej dwa razy.

~

Nie było przewrotu. Zamiast tego, zamorskie oddziały brytyjskich sił
zbrojnych rozpoczęły wojnę bez zgody rządu. Kolejny rząd -- cywilny,
zaciemniając prawnie uzasadnienia konstytucyjne -- został sformowany z~opozycji, Lordów i~Króla. Otrzymał natychmiastową akceptację w~USA i~puste miejsce Brytanii w~ONZ. Wypowiedział wojnę Niemcom.

Polacy przegrupowali się, sprzymierzeni z~kilkoma ukraińskimi fakcjami i~zaatakowali niemieckie zgrupowania. Użyli broni chemicznej.
Jednocześnie, jacyś bośniaccy uchodźcy -- nigdy nie udało się ustalić,
jakiej byli narodowości -- zatruli ujęcia wody Hamburga. Niemcy potoczyli
się do przodu na wszystkich frontach. Francuzi i~Rosjanie w~końcu
przestali siedzieć na barykadzie w~Radzie Bezpieczeństwa.

Republikański rząd ciągle kontrolował siły wewnętrzne w~kraju, podczas
gdy królewska junta kontrolowała zewnętrzną władzę państwa. W~dziwaczny
sposób musieli współpracować lub przynajmniej utrzymywać podział pracy:
gdy jedni uczestniczyli w~amerykańskich zrzutach nad Bałkanami i~manewrach morskich na Morzu Śródziemnym, inni gorączkowo mobilizowali
populację kraju do obrony cywilnej. W~rezultacie Korona wyjęła spod
prawa ostatnie dziesięć lat historii Brytanii, podczas gdy Republika
zalegalizowała Rewolucję.

To byłaby taka interesująca rewolucja. Z~których z~konkurujących
ekstremizmów -- w~tym naszego -- wyłoniłby się zwycięski, ciągle jest
przedmiotem debaty. Jak to bywa, miałem interesujący tydzień. Ruch
Kosmiczny naprawdę był tak wielki jak stary ruch pokojowy, a~rakiety na
naszych sztandarach były nasze własne. Zostawiłem demonstracje tym
członkom komitetu, którzy byli dobrzy w~tego rodzaju sprawach, i~spędziłem czas obsesyjnie organizujący bojówki i~patrole firm ochrony w~strefach wolnego handlu i~Greenbelt, negocjując z~naszymi kontaktami w~aparacie państwowym i, w~międzyczasie, pisząc więcej, szybciej niż
kiedykolwiek. Gdybym nie martwił się o~Eleanor i~nie bał się ciągle
niemieckich nalotów, byłbym jeszcze szczęśliwszy, niż byłem. Dotarłem do
swojego Dworca Fińskiego.

~

Ktoś potrząsał moim ramieniem. Podniosłem głowę znad przedramion i~się
rozejrzałem. Była 10:15 rano i~byłem przy biurku w~biurze
WolnegoKosmosu. Musiałem zamknąć oczy na chwilę około sześciu godzin
wcześniej. Biuro było zatłoczone, ale ciche. Ludzie patrzyli na ekrany,
nie na mnie, oprócz Annette, która mnie obejmowała, spoglądając.

-- Co się stało?

-- W~Kijowie wysadzono jądrówkę.

-- O mój Boże.

Wstałem. Ukryła swoją twarz w~moim ramieniu. Trzymałem ją, gdy łkania
nią wstrząsały, i~patrzyłem się, póki ktoś cicho nie wrzucił ekranu na
widok. Cała armia niemiecka została wymazana przez detonację w~powietrzu
nad pustą stolicą Ukrainy. W~ciągu minut, gdy patrzyłem, ta sama rzecz
wydarzyła się na południowym froncie, w~Baku. Rosyjskie i~tureckie armie
były już w~ruchu, a~wiadomości docierały za pośrednictwem desantów
amerykańskich i~brytyjskich na wybrzeżu egejskim.

A Izrael wypowiedział wojnę Niemcom. To było śmieszne. Co mogliby
zrobić? Myślałem, a~potem nagle zrozumiałem, że prawdopodobnie właśnie
to \emph{zrobili}.

Przełączyłem na N-TV po reakcję Niemiec. Reportem mówił do kamery, przed
Bundestagiem. Mówił coś o~Frankfurcie przerażonym tonem.

Uderzył dłonią w~ucho, przechylając głowę.

Jego twarz zbladła, a~ekran stał się biały.

Jego głos, jeżeli moglibyśmy to tak nazwać, trwał jeszcze jakiś czas.

~

Wojna się skończyła. Zaczął się proces pokojowy. Dla Wielkiej Brytanii
zaczął się od bombowców stealth, pocisków manewrujących i~kontynuował ze
spadochroniarzami, teleżołnierzami i~linczami. Junta rojalistów, jej
amerykańscy sojusznicy i~brytyjskie kontrrewolucyjne motłochy pomiędzy
sobą zabiły około stu tysięcy ludzi w~ciągu sześciu dni. Po tym mieli
kraj, który znał swoje miejsce w~Nowym Porządku Świata. 

Kraj ciągle był nieposkromiony. Przez reformy Republiki, uwolnienie
rynków mieszkalnictwa, edukacji i~pracy, rozwinęły się tendencje ku
zróżnicowaniu, samogettoizacji, jak to postrzegałem, szczególnie kiedy
nie były spontaniczne, ale promowane przez niefortunne poparcie
Republiki dla polityk tożsamościowych. Bombardowania, inwazja i~wojna
domowa wzmocniła tę tendencję ku nieodpartej sile, gdy każda mniejszość
uciekała w~wątpliwe bezpieczeństwo swojego własnego plemienia.
Regionalne zgromadzenia zrozumiały aluzję i~wytyczyły stare granice
świeżą krwią: Północna Walia, Południowa Walia, Cumbria, Zachodnia
Szkocja, Wschodnia Szkocja\ldots nawet nasz własny Greenbelt i~strefy
wolnego handlu stały się bezpiecznymi rajami, uchodźcy piętrzący się na
uchodźcach. Milicje broniły obszaru, tak dobrze, jak umiały.

Ostatnia sesja Zgromadzenia Federalnego Republiki przekazały swoją
władzę Radzie Armii, organowi stworzonego z~kilku starszych oficerów,
którzy pozostali lojalni. Rada wezwała populację cywilną, by unikała
niepotrzebnych poświęceń i~wznowiła zbrojny opór ,,w takim czasie lub
czasach, gdy Rada Armii Nowej Republiki zdecyduje''. Dali w~ten sposób
ochronę prawną dla nieskończenie przedłużanych kampanii bezlitosnego
terroryzmu, jak dobrze wiedzieli. Potem wszyscy wymaszerowali z~byłej
głównej fabryki Ford Motor Company w~Dagenham w~miażdżący ogień
otaczających czołgów.

Była to prawdopodobnie najdumniejsza chwila w~historii brytyjskiej
demokracji. Oglądałem to w~piwnicy bezpiecznego domu na nielegalnym
irackim kanale satelitarnym i~chciało mi się rzygać.

~

Wiedziałem, że powinienem pracować, zawsze był kolejny artykuł do
wysłania w~sieć, kolejny przyjaciel lub wróg do skontaktowania, los
kolejnej jednostki milicji do sprawdzenia, ale hakowałem niemieckie
listy ofiar, szukając imienia, którego miałem nadzieję nie znaleźć.
Izraelczycy uzbroili ich rakiety dalekiego zasięgu głowicami
taktycznymi, nie strategicznymi. Nawet w~Berlinie było więcej ocalałych,
niż ktokolwiek oczekiwał. Zawsze była szansa...

Zadzwonił telefon.

-- Tato?

-- \emph{Eleanor}!

-- Tak. Wszystko w~porządku?

Czy byłem w~porządku. Czułem się, jakbym to ja wrócił z~martwych.

-- Oczywiście, o~mój Boże, a~Ty?

-- W~porządku, widziałam straszne rzeczy, ale jest ok. Tak jak Colin.
Jesteśmy na lotnisku. -- Roześmiała się. -- Jak mówiłeś. Przepraszam, że
trochę późno. Mój lot odlatuje za dziesięć minut, przylot o~15:45.

Była 14:15. Powiedziałem, że będę tam ją spotkać. Po rozłączeniu
natychmiast zadzwoniłem do Annette z~wiadomościami.

-- Czy wyjście jest dla Ciebie bezpieczne? -- spytała Annette, gdy
skończyliśmy wypowiadać sobie kilka razy naszą ulgę, radość i~zapewnienia, że żadne z~nas nigdy nie straciło nadziei.

Wzruszyłem ramionami. 

-- Nie jestem na żadnej liście ,,poszukiwanych''.
Motłoch został przywołany do porządku. Wygląda to bezpiecznie.

-- Tam gdzie jesteś, na pewno tak wygląda -- powiedziała Annette krzywo. -- Niektórzy ludzie z~ruchu...

-- Ta, wiem -- powiedziałem. Byli zamieszani w~ruch oporu. Niektórzy
zostali internowani lub rozstrzelani. Inni, jak firmy ochrony lub
bojówki, na które miałem wpływ, próbowały unikać zaangażowania, ale
okazywało się, że walczą z~Jankesami, czy tego chcą, czy nie. Czułem się
niekomfortowo, rozmawiając o~tym nawet na bezpiecznej linii. -- Ciągle
-- kontynuowałem -- mam listę długą jak moje ramię wiadomości i~artykułów
namawiających ich, żeby tego nie robili, więc...

-- Tak czy inaczej -- powiedział Annette nagle zdecydowana -- nie możesz
się wiecznie ukrywać. Ok, odbiorę cię za piętnaście minut. Broadway przy
światłach. Normalnie.

Była w~Acton, nie w~domu, ale też nie ukrywając się.

-- Dobra, do zobaczenia tam kochanie.

Zebrałem sprzęt, usunąłem ślady mojej obecności, i~kiedy piwnica znowu
wyglądała jak wyłącznie jako schowek hobbysty komputerowego, wspiąłem
się po opuszczanej aluminiowej drabiny i~wyszedłem z~kredensu pod
schodami w~holu mojego gospodarza. Dom miał ten martwy aromat, gdzie nic
cały dzień się nie poruszało, tylko klapka skrzynki pocztowej, termostat
i maszyny czyszczące. Zostawiłem kopertę zawierającą kilka złotych monet
pod stojakiem na parasole i~się wypuściłem.

Dom był na ulicy za Ealing Broadway. Kasztany leżały jak zielone miny
morskie na Haven Green. Padała lekka mżawka. Pamiętałem grafitti na
murach z~roku Czernobyla: \emph{to nie deszcz, to opad}. Podniosłem
kołnierz i~przyśpieszyłem. Na zewnątrz stacji metra stali gliniarze, ku
mojemu zdziwieniu Straż Republikańska. Nie przyglądałem się im.

Przeszedłem Broadway i~odszedłem od, a~potem ku, światłom drogowym.
Odeon naprzeciw mnie pokazywał \emph{Niebieski Beret}, reklamowany przez
wielki podświetlany plakat jakiegoś siwowłosego weterana granego przez
Reevesa lub Deppa (zapomniałem) trzymającego ostrze bagnetu przy gardle
peruwiańskiego chłopa.

Odwróciłem się, zobaczyłem czarne Volvo Annette sto metrów dalej w~rzadkim ruchu, odwróciłem się znowu i~spacerowałem, żeby dopasować
prędkości, gdy zwalniała aż do zatrzymania. Pochyliłem się, otworzyłem
drzwi i~wsiadłem. Zawsze był ten moment sprawdzania, czy nie
zaszokowałeś kogoś na śmierć.

Roześmialiśmy się i~przyśpieszyła od świateł.

-- Wszyscy w~porządku? -- spytałem.

-- Wszyscy, których znamy -- powiedziała napiętym głosem.

-- Powiesz mi potem o~towarzyszach -- powiedziałem. -- Zrobimy, co możemy.

Skinęła głową, koncentrując się na drodze i~aktualizacjach ekranu ruchu
drogowego. Nasza droga była wykreślona przez Uxbridge Road aż koło
Southall, potem ostro w~lewo wzdłuż Parkway do Heathrow.

-- Co jest złego z~Great West Road?

Chrząknęła. 

-- Transport wojska.

Hanwell, willowe przedmieścia średniej klasy, były ciche. Southall,
rejon imigracji azjatyckiej, solidnie republikańskie, miało tuziny
wypatroszonych wystaw sklepowych.

-- Co tutaj się zdarzyło?

-- Tłum z~Hayes -- powiedziała Annette. Przejechaliśmy, potem przez most
do Grand Union Canal. Fabryki Hayes, po naszej prawej, zostały
zbombardowane do zwęglonych szczątków przez Jankesów. Przyznaję się do
uczucia pewnej ponurej satysfakcji: latami ten rejon był rasistowskim,
imperialistycznym bastionem. Nawet trockiści zrezygnowali ze
sprzedawania ich czerwonych tygodniówek białym śmieciom.

-- Jak sobie pościelisz, tak się wyśpisz.

-- Raczej trudna lekcja -- powiedziała Annette.

Każdy mijany park miał własne koczowisko czarnych, plastikowych kopuł,
przyczajonych przykrytych samolotów, czarnych helikopterów. Gdy
zbliżaliśmy się do lotniska, liczba żołnierz USA/ONZ w~czarnym mundurach
wzrastała. Nie było potrzeby blokad, fala czytników tożsamości
wykonywała schludniejszą pracę. Lasery wymuszały mrugnięcia, zawsze zbyt
późno: skan siatkówki już się odbył.

Heathrow wyglądało jak scena z~dwudziestego wieku. Nikt nie latał, prócz
tych, którzy musieli: uchodźcy ze strefy wojny, ranni żołnierze i~cywile, zdesperowanie emigranci. Na lotnisku był Trzeci Świat ludzi
czekających na loty, czekających na przejście przez ponownie ustawione
granice imigracyjne, czekających na śmierć. I~Drugi Świat urzędników i~oficerów rozkazujących tamtym. W~tym szaleństwie, Pierwszy Świat składał
się z~wolontariuszy próbujących pomóc i~przedsiębiorców próbujących
pomóc sobie. Każdy hol pasażerski miał swoje własne szpitale polowe i~straganiarzy. Każda brama darmowych doradców, rekiny prawne i~zespoły
pierwszej pomocy.

Dotarliśmy do terminala międzynarodowego, ale lot został przerzucony na
krajowy. Ruchome chodniki były przeciążone wychodzącymi żołnierzami i~ich sprzętem. Przechodzenie pomiędzy terminala było ruchem Browna przez
tłum Hobbesa. Czas ciągnął się, zatrzymywał, mijał bez zauważenia.
Annette i~ja lgnęliśmy do siebie i~przebijaliśmy się do przodu.

Godziny później, gdy Eleanor i~Colin w~końcu pojawili się w~strumieniu
przychodzących, byliśmy tak wymizerowani i~obszarpani jak oni. Po
przytulaniu, płakaniu i~rozmawianiu, odwróciliśmy się na pięcie i~znowu
wywalczyliśmy naszą drogę. Dotarliśmy do samochodu, zapłaciliśmy dopłatę
za parking, zapłaciliśmy skandaliczne pieniądze straganiarzowi za ciepłą
kawę i~ruszyliśmy do domu. Było około godziny 22.

Prowadziłem, Annette była wyczerpana, ja szalałem z~ulgi.

Gdy kierowałem samochodem dookoła skrzyżowania, błysk rubinowego lasera
z M4 uraził mi oczy. Mrugając powidok, zostałem znowu oślepiony przez
latarkę, kierującą nas na pobocze. Na chodniku była grupa pięciu
żołnierzy w~czarnych mundurach i~M-16. Wcisnąłem włącznik telefonu
samochodowego i~wyciągnąłem go, odwróciłem się z~ufnie uspokajającym
uśmiechem do innych i~wyszedłem. Inne samochody omijały mnie na
centymetry. Wszyscy w~nich mocno się starali nie patrzeć. Trzymałem
swoje ręce nad samochodem i~poruszałem się bokiem dookoła do najbliższej
strony.

Ręce obmacały kołnierzy, tors, w~dole nogi i~pomiędzy nimi. Potem moje
ramiona zostały złapane i~zostałem odwrócony i~rzucony na samochód.
Zamarłem w~świetle i~trzymałem ręce w~górze. Za mną, przez okno otwarte
na dwa centymetry, myślę, że słyszałem cichy, naglący głos Annette.

Żołnierz mnie pilnujący opuścił wiązkę, podniósł karabin i~zbliżył się
blisko. Jego wizjer był w~górze, odkrywając bierną, andyjską twarz.
Przypomniał mi się chłop na plakacie. Jak sobie pościelisz, tak się
wyśpisz\ldots

-- Jonathan Wilde -- powiedział. To nie było pytanie. Nie odpowiedziałem.
Miałem sucho w~ustach.

-- Chodź z~nami -- powiedział.

Poczułem opuszczającą się szybę na moich plecach.

-- Nie! -- krzyknęła Annette.

-- Tak -- powiedziałem. -- Iść. Iść teraz.

-- Tak -- powiedział żołnierz. -- Iść.

Odepchnął mnie od samochodu. Zrobiłem dwa wolne kroki do przodu. 

-- W~samochodzie nie ma broni -- powiedziałem.

-- Wiemy. -- Machnął karabinem ode mnie w~kierunku samochodu. Po raz
pierwszy jego twarz pokazała emocję, coś tak pierwotnego, że było trudno
określić, czy to strach, czy wściekłość.

-- Idź! -- krzyknął.

Mogłem usłyszeć suchy szloch Annette, płacz Eleanor, kłócącego się
Colina. Nie śmiałem się odwrócić lub nawet ruszyć ręką.

Silnik się uruchomił i~wolno samochód odjechał.

Światła ulic i~mgła. Światła lądowania samolotu i~mgła. Noc i~mgła.
Nigdy nie wyglądały tak pięknie. Podniosłem oczy, żeby spojrzeć na
gwiazdy, o~których myślałem, że nigdy nie dosięgnę, nie teraz. Nie
mogłem ich zobaczyć. Och, dobra.

Prowadzili mnie kilkaset metrów do pasa nieużytków. Właściwie poczułem
ulgę, gdy zobaczyłem czarny helikopter, matowe kanciaste powierzchnie
błyszczące z~kondensacji w~cieniach. Wprowadzili mnie pokład i~usadzili
twarzą do otwartych drzwi, gdy pojazd się wzniósł. Helikopter czynił
zadziwiająco mało hałasu. Żołnierze obserwowali mnie z~cichą
złośliwością i~uśmiechami brudnych sekretów.

Zastanawiałem się, dlaczego szedłem, kiedy mógłbym uciec. Wyglądało to,
jakbym był na jednej z~egzekucji klientów USA, klasyczny styl,
sajgońskie nurkowanie. Powinienem uciec, pomyślałem, i~nie dawać im tej
satysfakcji. Było przysłowie arabskie, coś w~rodzaju \emph{nadzieja jest
wrogiem wolności} czy \emph{rozpacz jest wyzwoleniem niewolników}.
Tłumaczy to dużo, włączając, dlaczego wspiąłem się do tego helikoptera.

Mam nadzieję, że to nie wyjaśnia tego, co zrobiłem, kiedy wysiadłem.

~

-- Proszę wejść, panie Wilde.

Uprzejmemu zaproszeniu od jednego z~tuzina ludzi w~garniturach dookoła
stołu towarzyszyło pchnięcie w~plecy przez żołnierza ONZ, które posłało
mnie potykającego w~pokój i~nie zostawiło żadnej wątpliwości, kto
naprawdę był tutaj u władzy. Drzwi za mną były zbyt ciężkie, żeby nimi
trzasnąć, ale zamknęły się z~przytłumionym uderzeniem, jakby żołnierz
przynajmniej spróbował.

Wyprostowałem się, mobilizując moją godność, i~rozejrzałem się po
pokoju. Gdzieś w~Westminster -- helikopter wylądował w~St James Park i~zostałem zapakowany do tyłu transportera opancerzonego i~przewieziony na
krótkim dystansie -- ale było to niemożliwe, by określić, czy to był
prywatny czy publiczny budynek. Wielki mahoniowy stół ze światłami nad
nim, ściany wyłożone dębem, portrety wybitnych przodków lub poprzedników
w mroku. Mężczyźni, którzy patrzyli na mnie zza stołu, mieli coś z~tego
samego wyglądu odziedziczonej lub zdobytej pewności, pomimo bycia
bardziej zaniedbanymi niż ja, ich marynarki zmięte lub wiszące na
oparciach foteli, poluźnione krawaty, czerwone oczy, nieogolone
policzki.

Stół był zastawiony laminowanymi mapami, na których linie były
narysowane, zmazane i~ponownie narysowane fluorescencyjnym atramentem z~markerów, które leżały rozrzucone pomiędzy filiżankami kawy i~przepełnionymi szklanymi popielniczkami wielkości dużych talerzy.
Wznoszący się dym zwijał się w~stożkach światła, żeby zostać wyssanym
przez potężną klimatyzację, która nadawała atmosferze czerstwy chłód.

Mężczyzna, który przemówił, wstał i~gestem skierował mnie do pustego
miejsca przy najbliższym rogu stołu. Świeżo napełniona filiżanka kawy
parowała przed krzesłem.

-- Dobry wieczór, panie Wilde -- powiedział. -- Muszę przeprosić za raczej
szorstki sposób, w~jaki został Pan tutaj sprowadzony. -- Uśmiechnął się
przepraszająco, lekkie wzruszenie ramion jak gdyby wypierał się
odpowiedzialności. Był stary, starszy niż ja -- choć miał lepsze leczenie
-- a~jego falujące żółto-szare włosy, do ramion, sprawiały, że wyglądał
jak sędzia, lub jeden z~tych osiemnastowiecznych dygnitarzy na
portretach. -- Ufam, że nie był Pan poza tym źle traktowany.

Stałem tam, gdzie byłem i~powiedziałem: 

-- Nazywam porwanie złym traktowaniem, sir. Żądam wyjaśnień i~natychmiastowego kontaktu z~moją
rodzinę i~moim prawnikiem.

Kolejny mężczyzna przemówił, pochylając się na łokciach ku światłu. 

-- Nic z~tego nie ma zastosowania. Ten kraj jest w~stanie wojennym, a~zresztą nie jesteście aresztowani.

-- Dobrze -- powiedziałem. -- Zatem już pójdę.

Odwróciłem się i~ruszyłem do drzwi.

-- Stój! -- Głos pierwszego mężczyzny brzmiał bardziej jak pilne
ostrzeżenie niż rozkaz. -- Chwilę, proszę.

To brzmiało lepiej. Odwróciłem się.

-- Oczywiście wolno ci wyjść -- kontynuował mężczyzna -- ale jeżeli
wyjdziesz, tylko my możemy ci zagwarantować bezpieczeństwo. Wszystko, o~co prosimy, to żebyś nas wysłuchał.

Wątpiłem w~to, ale zdecydowałem, że byłoby nieroztropne próbować czegoś
innego. Poza tym potrzebowałem tej kawy.

~

Byli komitetem tego, co było już nazywane Rządem Restauracji. Posłowie
Parlamentu, urzędnicy\ldots nie podawali swoich imion, a~ja nigdy potem nie
próbowałem się dowiedzieć. Powiedzieli mi, że próbują przywrócić porządek
i cywilną administrację.

-- Republika nie żyje, panie Wilde. Nasze jedyne wybory to przedłużony i~daremny opór z~przedłużoną i~bolesną okupacją albo siłowanie się z~urabialnym traktatem.

-- Nie widzę, by USA nadążała z~przedłużoną okupacją -- powiedziałem. -- Biorąc ich osławioną wrażliwość na worki na zwłoki.

-- Jak wielu żołnierzy widziałeś? -- warknął drugi mężczyzna. -- Wszyscy są
w bunkrach, operując teleżołnierzami. Uwierz mi, trzecioświatowi klienci
Ameryki mają żołnierzy do stracenia dla ONZ. Bezpieczeństwo wewnętrzne
jest tym, na czym się wychowali i~za co im płacą. Wyśmieją nasze żałosne
próby naszych domowych Guevarów. Nie pomyl się, Stany Zjednoczone,
\emph{Narody} Zjednoczone, tym razem traktują to poważnie. Żaden naród
nigdy nie będzie miał pozwolenia rozpoczęcia wojny. Rozbrojenie jądrowe
\emph{będzie wyegzekwowane.}

Krople śliny z~jego mowy znaczyły mapy. Na wpół oczekiwałem, że jego
prawe ramię drgnie. Musiałem lekko się cofnąć. Długowłosy mężczyzna
uniósł dłoń, dobry glina do złego gliny.

-- Wiemy tak dobrze jak Ty, że mocarstwo takie, jakim USA chce się stać,
nie może prawdopodobnie zarządzać światem. Pilnować go, na bardzo
wysokim poziomie, tak. Jednak jak pewne uprawnienia przesuwają się z~narodu, inne są przekazywane lokalnym społecznościom. Mamy możliwość
poprzeć autonomię i~różnorodność. Weźmy to i~oszczędźmy naszemu krajowi
lat agonii.

-- ,,My''? -- rozejrzałem się. -- Nie mam z~wami nic wspólnego. Czego chcecie
ode mnie?

-- Możliwości umowy, panie Wilde. Porozumienia. Ściągamy wszystkich
regionalnych, fakcyjnych i~lokalnych przywódców, do których mamy dostęp.
Pan zdarzył się być pierwszy.

-- I~co zamierzacie im zaoferować?

-- Zaakceptować Koronę, w~praktyce, jako narodową władzę, i~możesz mieć
autonomię w~rejonach, które kontrolują Twoi zwolennicy.

-- Nie mam upoważnienia do negocjacji\ldots

-- Och, ale Pan ma. Ma Pan wpływ. Wiemy, że bez tego, pewne młodsze i~gorętsze głowy pociągałby za sznurki. I~wiemy, że ma Pan ochotę na
więcej, niż wskazują na to publiczne oświadczenia\ldots

-- Skąd takie podejrzenie?

Uśmiechnął się. 

-- Wielkość zaszyfrowanego ruchu z~pańskich bezpiecznych
domów.

Cholera. Próbowałem zachować pokerową twarz.

-- To, co widzisz, jest tym, co dostajesz. Nie zrobiłem niczego
potajemnie, co byłoby sprzeczne z~tym, co powiedziałem otwarcie.

-- Oczywiście. Zatem nie wyrażasz sprzeciwu. Proszę spojrzeć na te...

Umowy, gotowe do podpisu. Mapy. Londyn, na początek, miał być pocięty.
Część przyznana Ruchowi Kosmicznemu obejmowała Greenbelt i~łuk
przedmieść, w~których mieliśmy strefy wolnego handlu. Nawet dali temu
nazwę: North London Town, które na mapie jakaś wojskowa ręka skróciła do
NORLONTO.

To było dużo. Szczerze, zgodziłbym się za mniej.

-- A w~zamian?

-- Żadnych zbrojnych akcji rozpoczynających się z~terytorium. I~jeszcze
jedna sprawa\ldots

-- Tak?

-- Ach\ldots umowa odstraszania nuklearnego, panie Wilde.

-- Chcecie, żebym ją zakończył?

-- Dobry Boże, nie! -- Wyglądał na zszokowanego. -- Chcemy, żeby pan ją
przeniósł na nas.

-- Na \emph{Rząd}? Ale macie\ldots -- Przerwałem i~spojrzałem na ich lekko
zażenowane twarze.

-- Och -- powiedziałem. -- Rozumiem. -- Odwróciłem się do mapy i~podniosłem
pióro. Pod koniec nocy mieliśmy coś, co mogłem zabrać do mojego
komitetu.

Dwa dni później siedziałem w~pokoju na tyłach meliny w~Greenbelt z~grupą
mężczyzn i~kobiet, którzy, dzięki moim negocjacjom, wyłonili się,
mrugając, z~kryjówek, obozów i~więzień. Wyjaśniłem im, że mają szansę
przetestować ich idee na kilku milionach mniej więcej entuzjastycznych
ludzi, z~minimalną interwencją Państwa, zbyt zadowolonego, że nie ma na
głowie wybuchowych i~podupadłych mas. Powiedziałem im, że jedyną ceną za
było \emph{de facto} uznanie autorytetu tego Państwa i~wyrzeczenie się
niewypróbowanego odstraszania nuklearnego, co do którego większość z~nich miała mieszane uczucia i~które teraz było zdezaktualizowane.

Nie oczekiwałem wdzięczności lub zgody, takich nie otrzymałem. Co
otrzymałem, to towarzysze śpieszący z~potępieniem. Oczekiwałem tego.
Bycie wydalonym z~organizacji było niespodzianką. Głosowanie było
jednogłośne. \emph{Et tu}, Julie.

-- Dobrego dnia, towarzysze -- powiedziałem. -- I~powodzenia.

Wstałem, odsunąłem krzesło, schyliłem się w~drzwiach i~odszedłem. Dwa
dni po moim wydaleniu, wyborowe oddziały USA/ONZ przejęły i~rozbroiły
każdego naziemnego eksportera odstraszania nuklearnego. Buntownicze
łodzie podwodne zajęły więcej czasy, ale też w~końcu zostały rozbrojone.
Pośród konsekwencji, moi ekstowarzysze nie mieli naszej polisy
nuklearnej do targowania, więc musieli się pogodzić z~mniejszym Norlonto
niż oferowano mnie.

Dobrze im to posłużyło, ale wolałbym, żeby mogli zatrzymać Islington.
Dostali je chrześcijańscy fundamentaliści i~rozpoczęli etyczne
oczyszczanie miejsca. Eleanor i~jej rodzina musieli opuścić Finsbury
Park. Wprowadzili się do nas i~minęły miesiące, zanim znaleźli nowy dom.

Stawałem się zbyt stary na tego rodzaju sprawy.

\chapter{Sąd Piątej Dzielnicy}


-- Dlaczego nie mogliśmy pojechać kanałami? -- burczał Wilde, gdy kopał
kolejną, zaciekawioną jego kostkami, maszynę. Kilka godzin trudnego
poruszania się tylnymi alejami, razem z~ekspedycją trzeszczącą, tupiącą
i strzelającą całą drogę przez i~obok mieszanych mechanicznych
szkodników, kładło się napięciem w~jego głosie i~sile kopnięcia.

-- Ha! -- parsknęła Tamara. -- \emph{Widziałeś} kanały tutaj?

-- Zdarzyło się, -- powiedział Wilde -- że nie, nie widziałem.

-- I~nie chcesz. -- Tamara rozpłaszczyła się przy ścianie i~zasygnalizowała zatrzymanie innym z~tyłu. -- Ale zobaczysz.

Wystawiła urządzenie jak długą elektryczną pochodnię za róg i~pomachała
tam i~z powrotem przyglądając się odczytom na ręcznym mierniku i~widokowi na ekranie na nadgarstku.

-- Ok -- obwieściła. -- Żadnych rozumnych. Wygląda dość bezpiecznie. Jeden
na raz. Wyjdźcie na środek ulicy, rozproszcie się, potem pojedynczą
kolumną na prawo. Już.

Wybiegła na środek drogi, która miała około pięćdziesięciu metrów
szerokości i~była obsesyjnie dobrze utwardzona. Na środku stały puste
cokoły betonu jak azyle dla pieszych. Tamara dobiegła do jednej
skierowanej w~aleję, rozejrzała się dookoła i~skinęła na Wilde'a. Rzucił
się za nią i~wpadł koło niej.

-- Osłaniaj mnie -- powiedziała. Wilde stanął za nią i~zaczął skanować w~jedną i~drugą stronę ulicy, pistolet trzymany w~obu dłoniach, blisko
bioder. Ulica miała swoich własnych dziwnych przechodniów: roboty
różnych kształtów i~wielkości wspinały się na ściany, krążyły wzdłuż
krawędzi chodników. Jeden lub drugi ciągle rzucał się w~dół, na lekkie
pojazdy kołowe. Ethan, gdy biegł, musiał jednego z~nich sprytnie ominąć.
Tamten zabrzmiał jak syrena infradźwiękowa, co każdy poczuł w~zębach.

-- Wygląda, że wiesz, co robisz -- powiedział do Wilde'a, gdy zatrzymał
się kilka metrów dalej wzdłuż cokołu.

-- Trening w~bojówkach -- uśmiechnął się Wilde. -- Ale, to było\ldots
\emph{uważaj}!

Czarna rakieta ze skrzydłami pędziła w~ich kierunku. Wilde podniósł
pistolet na wysokość głowy i~ją zestrzelił. Spadła i~uderzyła w~drogę w~powodzi piór.

-- Gołąb -- powiedział Ethan. -- Spokojnie człowieku. One są nieszkodliwe.

Kiedy alarm podniesiony przez ten incydent został załagodzony, ich
podróż trwała. Po minucie lub dwóch podążali za Tamarą wzdłuż kanionu
budynków biurowych. Gdzieś kilka ulic dalej, zautomatyzowany proces
wysyłał płomienie ognia wysoko w~powietrze w~irytująco nieregularnych
interwałach. Pomiędzy płomieniami, sama iluminacja budynków była prawie
nieprzewidywalna: pewne okna ciemne, pełne odbić ekspedycji, gdy je
mijali, inne, na poziomie ulicy lub wysoko ścianach, podświetlone od
środka. Cienie i~sylwetki się poruszały, ale nie te ludzkie. W~tym samym
czasie, było niemożliwe do uwierzenia, że życie handlowe oparte na
robotach trwało. Było zbyt losowe, zbyt sztuczne.

Przy kolejnym dużym skrzyżowaniu ulic, ta, na której byli, przecinała
taką węższą, ale znacznie bardziej zatłoczoną: wolno poruszająca się
rzeka metalowych maszyn, ponad którymi szybsze byty się ślizgały i~kicały.

-- Sprawia, że jesteś chory -- wymamrotał Ethan. -- Niektóre z~tych dużych
byłyby cholernie dobrymi samochodami.

-- Zapłać mi dobrze, to ci jednego złapię -- powiedziała mu Tamara. Gestem
ustawiła ich w~luźny szyk, znowu zatrzymując Wilde'a koło siebie.

-- Dobra -- powiedział, kładąc swój plecak na ziemi. -- Czas włamać się do
dżungli.

Rozpięła plecak i~pociągnęła za klapy, odsłaniając ekwipunek z~małą
klawiaturą, rozsuwalnymi antenami, rzędami mierników i~ekranów.

-- Niesamowite -- powiedział Wilde. -- Młody mechanik! Radio amatorskie!

-- Kupa śmieci -- powiedziała Tamara. -- Żaden gnojek nie chce tego
zminiaturyzować. Niewystarczający popyt.

-- Złożyłaś to wszystko sama?

Spojrzała na niego. 

-- Nie zaufałabym nikomu innemu.

Jej palce latały nad klawiaturą. Ekrany migotały, małe głośniki zawyły i~ustabilizowały.

-- Mam to! Kanał ruchu drogowego.

Pokręciła gałką, popatrzyła na maszyny przechodzące jak bydło. Zrobiła
pewnie zmiany, znowu pokręciła. Czołgająca się maszyna długości
dziesięciu metrów nagle skręciła w~prawo przez drogę. Maszyna poza nią
stłoczyły się nieubłaganie w~niej i~w ciągu sekund stworzyły rosnący
stos robotów na kółkach lub gąsienicach. Gdy te z~przodu ciągle się
poruszały, wkrótce zrobiło się miejsce.

Tamara ciągle patrzyła na informacje zwrotne.

-- Kurwa naprzód! Już! Już! -- wrzasnęła.

Inni biegli w~poprzek sprintem.

Tamara podniosła plecak zostawiając odsłonięty panel sterowania.

-- Ciągle tutaj? -- powiedziała do Wilde'a. -- Gówno, ok, idziemy.

Przebiegła pędem przez ulicę, Wilde z~jej tyłu obserwując. Maszyna na
czterech długich, majestatycznych nogach, jej ciało wielkości melona, z~wiązką soczewek na przodzie, nagle wycofała się ze stosu i~przeskanowała
ich.

-- Co to znaczy?

Tamara spojrzała się i~zatrzymała.

-- Nie ruszaj się -- powiedziała.

Wilde wstrzymał oddech i~zamarł w~trakcie patrzenia nad ramieniem na
maszynę. Soczewki się wycofały i~kolejne rurkowate przedłużenie wsunęło
się na pozycję. Tamara gorączkowo uderzała w~klawiaturę.

-- Strzelaj! -- wrzasnęła.

Wilde podskoczył, odwrócił się, ale ona nie krzyczała do niego. Salwa
przyszła z~odległej strony ulicy, przewracając maszynę. Tamara i~Wilde
pobiegli dołączyć do innych.

-- Kurde -- powiedział Ethan. -- Ten był rozumny.

-- \emph{Nigdy }nie poluję na rozumne -- powiedziała Tamara, dysząc i~pocierając kark. -- Choć nie mam problemu z~zabijaniem tych fiutków.

Ruszyli dalej. Nad mostem, który dał Tamarze możliwość wykazania
Wilde'owi dokładnie, dlaczego używanie kanałów do transportu w~dziedzinach maszyn nie było dobrym pomysłem. I~dalej, póki nie ujrzeli,
w wielkim parku na końcu długiej alei palisady ze złomu.

-- Sąd Talgartha -- powiedziała Tamara.

Gdy szli, zostali obmieceni skanami sonicznymi, które wprawiły ich zęby
w drżenie, skanami laserowymi, od których mrugali i~przeklinali.

-- Zignoruj to -- powiedziała Tamara. -- Muszą sprawdzić.

Park był dziwnie schludny i~utrzymywany w~ten sposób przez małe
urządzenia, które włóczyły się w~trawie i~pośród gałęzi drzew. Po raz
pierwszy od czasu lądowania, Tamara nakazała troskę o~nastąpienie na
dowolną maszynę.

-- Talgarth tego nie lubi -- nalegała. -- Nakłada grzywny.

Wybrali drogą przez trawę, ich bronie w~kaburach lub wisząca, uzbrojenie
najeżone na palisadzie było więcej niż wystarczające do ochrony ich
przed dowolnym dzikim gadżetem. Karabiny maszynowe, działa laserowe,
radary i~wirujące, gotowe bolas\ldots

Brama palisady wysokiej na trzy metry gładko się dla nich otworzyły i~szybko zamknęły się za nimi. Kwadrat około stu metrów długości, pokryty
trawą jak park, z~podestem w~centrum, siedzeniami i~sprzętem media
rozrzuconym dookoła, a~po obwodzie drewniane budki różnych wielkości.
Nikt inny nie był obecny.

-- Co teraz robimy? -- spytał Wilde.

Tamara spojrzała na zegarek. 

-- Jest pierwsza w~nocy -- powiedziała. -- Wybieramy chatę dla siebie i~śpimy. -- Wyszczerzyła zęby. -- To stary
zwyczaj kręgowców.

-- Warty podtrzymania -- powiedział Wilde. Rozejrzał się niezdecydowanie,
gdy większość z~pozostałych pewnie ruszyła.

Tamara złapała jego rękę.

-- Chodź ze mną -- powiedziała. -- Przypilnuję, żebyś wszystko było w~porządku.

Zgodził się, zażenowana mina na twarzy.

-- Uważaj -- krzyknął za nim Ethan. -- Ona naśladuje stare zwyczaje
naczelnych.

-- Idź się jebać! -- odkrzyknęła Tamara. -- Do zobaczenia w~sądzie!

~

-- Więc tak to nie-propertarianie\footnote{propertarianizm -- filozofia polityczna redukująca zagadnienia etyki do prawa do posiadania
własności,
więcej~\url{https://en.wikipedia.org/wiki/Propertarianism } -- przyp.tłum.} to robią. 

-- Ta. Wolna miłość.

-- Ha. Byłem wierny mojej żonie przez siedemdziesiąt lat\ldots

Głos Wilde'a zamilkł, potem kontynuował, szczęśliwszy: 

-- \ldots a~teraz byłem z~dwiema innymi kobietami w~ciągu trzech dni.

-- Co! Kto jeszcze?

-- To nie Twoja sprawa. Wolna miłość, prawda?

-- Och, no weź.

-- Ona jest już prawdopodobnie martwa.

Nastąpiła cisza. Potem Tamara, jej twarz oświetlona słabym nocnym
światłem i~żarem papierosa Wilde'a, przemówiła ostrożnie radosnym
głosem.

-- Mam nadzieję, że to nie zaraźliwe.

Wilde posłał jej krzywy uśmiech i~zgasił papierosa. Ich oczy gładko się
dopasowały i~spędzili kilka chwil, patrząc na siebie.

-- Może być -- powiedział Wilde. -- W~końcu sam jestem martwy.

Tamara zbadała.

-- Cóż, ten kawałek jest zdecydowanie żywy.

-- Och nie.

-- Och tak.

-- W~jaki sposób oczekujesz, że stanę jutro przed sądem?

-- Już dzisiaj stoisz w~porządku.

-- Mmm.

-- Tak czy inaczej, ah ha ha, ha ha, dostaniesz pomoc od ah ha ha !

-- Dam Ci \emph{Niewidzialną Rękę}.

-- Nie -- powiedziała Tamara. -- To na \emph{znacznie} później\ldots

~

-- Jest ósma. -- poinformowała go uprzejmie Tamara. -- Wyglądasz strasznie.

-- Dzięki. -- Wilde oparł się na jednym łokciu i~sięgnął po kubek kawy,
który mu podawała. -- Och, Boże. Jak długo spałem?

-- Cztery godziny.

-- Dzięki Tobie, ty rozwiązła anarchistyczna suko.

Tamara się uśmiechnęła.

-- Nie martw się -- powiedziała. -- Dodałam narkotyku do kawy. Będziesz
bardziej trzeźwy, niż możesz sobie wyobrazić.

-- Czy to dlatego widzę rzeczy?

-- Nie. Zostawiłeś założony kontakty.

-- Raz jeszcze dzięki. -- Wilde sięgnął po papierosy i~poskrobał twarz. -- Czy ten anarchokapitalistyczny sąd przypadkiem dysponuje jakimiś
stowarzyszonymi monopolistycznymi zdzierającymi przedsiębiorstwami?

-- Śmieszne, że pytasz. -- Tamara wskazała na paczki papierosów i~paczkę
zawierającą golarkę i~przybory toaletowe. -- Potrąciłam z~Twojego
rachunku.

Zajęła się robieniem śniadania, podczas gdy Wilde kręcił się, myjąc,
ubierając się i~pijąc kawę zaprawioną narkotykiem. Chata miała trzy
przyległe pomieszczenia: małą sypialnię z~podstawową umywalką i~małą
toaletą, małą kuchnię i~duży pokój zawierający sprzęt łącznościowy i~interfejsy komputerowe, wszystkie na stole konferencyjnym z~kilkoma
krzesłami dookoła niego.

-- Jak długo mamy tutaj przebywać? -- spytał Wilde, goląc się.

-- Tyle ile to zajmie.

-- Czy Reid już się pojawił?

-- Tak. I~jego sojusznicy. Szanse są prawie równe, gdyby doszło do walki.

-- To szczęśliwy zbieg okoliczności.

-- Nie, to zostało przygotowane przez\ldots

-- Nie mów mi, Niewidzialna Ręka. Ok. Jezu. -- Wytarł ręcznikiem twarz. -- Nie czułem się tak nieprzygotowany na nic od czasu moich końcowych
egzaminów.

-- Co to są egzaminy?

-- Stary zwyczaj naczelnych. -- Wilde chrupał płatki owsiane. -- Rozumiem,
że wyewoluowaliście poza to. Sprawdźmy wiadomości.

Tamara ustawiła sprzęt komunikacyjny w~głównym pokoju, podczas gdy Wilde
patrzył. Ciągle była w~dżinsach, podkoszulce i~kamizelce kuloodpornej,
ale nałożyła makijaż i~perfumy jako pewien gest ku formalności lub
kobiecości.

-- Czy ciągle źle wyglądam? -- spytał Wilde.

Przyjrzała mu się od stóp do głowy. 

-- Ujdzie -- powiedziała. -- Jednak
użyj płynu po goleniu.

Sprawdzili wiadomości. Sprawa była głównym tematem na wszystkich
kanałach. W~ciągu nocy, dookoła jej aspektów wyrosła cała subkultura
grup dyskusyjnych i~dyskusji. Trzy zabójstwa, które przypisywali sobie
Dee i~Ax, ich zniknięcie i~pojawienie się Jonathana Wilde'a dodało całej
sprawie elementu społecznej paniki. Przynajmniej dwa heretyckie kościoły
już ogłosiły, że Wilde jest znakiem Końca.

-- Mam nadzieję, że Twoi towarzysze abolicjoniści są przygotowani na
kłopoty -- powiedział Wilde.

-- Jakiego rodzaju kłopoty?

-- Powinnaś wiedzieć. Czy nie masz zawsze kłopotów, sprzedając swoją
gazetę? Czy Ax nie pokazał, co się zdarzy, jeżeli ludzie nagle zaczną
myśleć, że świat ma się zmienić na zawsze? Wyobraź sobie to pomnożone
przez dziesiątki, setki!

Tamara potrząsnęła głową. 

-- Nie wiem. Czytałam o~zamieszkach i~rewolucjach, ale nigdy nie mieliśmy czegoś takiego tutaj.

-- Uważaj siebie za szczęściarę.

Policzki Tamary poczerwieniały. 

-- Och, uważam, nie zrozum mnie źle.
Miasto Statku w~zasadzie nie jest złym miejscem, to po prostu, jest to
całe zło wyrządzone umysłom maszyn i\ldots jest tak daleko od ideałów
anarchizmu. A ludzie naprawdę myślą, że Twoje nagłe pojawienie oznacza,
że wszystko zostanie naprawione.

-- ,,Ideały anarchizmu'' -- powtórzył ciężko Wilde. Patrzył się na twarz
Tamary przez kilka sekund. Nikt, patrząc, nie mógłby mieć najmniejszej
wątpliwości, która z~tych młodych twarzy w~chacie była starszym umysłem.

~

Wilde spędził kolejną godzinę na rozmowach z~podzbiorem prawniczych baz
danych Niewidzialnej Ręki, oprogramowaniem ,,Przyjaciel MacKenzie''. Było
to system przyjazny, oraz przyjazny użytkownikowi. Jego składnik
hardware był telefonem ucho-broda, który odbierał, co powiedział i~usłyszał i~podawał niskomocowym radiem do lokalnego węzła. Jego
odpowiedzi mogły być wyszeptane w~ucho lub wyświetlane na kontaktach.

Krótko po dziewiątej, Tamara przerwała jego badania precedensów i~argumentów.

-- Reid wyszedł ze swojej chaty -- powiedziała mu.

Wilde odmrugał wyświetlacz.

-- Co robi?

-- Tylko włóczy się dookoła z~przyjaciółmi, popijając kawę i~dyskutując z~ludźmi, oraz z~pyłkami z~wiadomości.

-- Myślę, że zrobię to samo -- powiedział Wilde. -- Ponadto, nie miałbym
nic przeciwko rozmowie.

Tamara uśmiechnęła się krzywo. 

-- Trochę późno na ugodę.

Wilde wstał. 

-- Nigdy nie jest za późno -- powiedział. -- Ale nie, nie mam
na to zbyt wielkich nadziei! Faktem jest, nie mogę doczekać się
zobaczenia go.

Tamara była cicha przez chwilę. Wilde zapalił papierosa.

-- Powinnam Cię ostrzec -- powiedziała Tamara. -- Rozmawiałam z~nim
wczoraj, kiedy zadzwonił do mnie, racja, i\ldots chociaż widziałam go w~wiadomościach i~tak dalej, kiedy z~nim rzeczywiście rozmawiałam, że jest
bardzo\ldots mam na myśli ma rodzaj, wiesz, \emph{reprezentacyjny}. Może
się wydawać trochę onieśmielający.

Wilde wstał z~chrapliwym śmiechem.

-- Patrzyłem na niego, jak patrzył, jak umieram -- powiedział. -- Nie ma
sposobu, żeby mnie onieśmielił.

~

Wyszli razem z~chaty. Tamara pyszniła się, jej wielki pistolet raził w~kaburze. Wilde spacerował, kawa w~jednej dłoni, papieros w~drugiej. Rosa
skrzyła się w~trawie. Chłodne, wilgotne powietrze podtrzymywało wolne,
małe kolumny dymu i~pary ponad grupami ludzi, którzy stali wokoło, w~poważnych lub społecznych dyskusjach. Niektóre chaty otwarły się jako
stragany, choć tylko dla drobnych potrzeb. Żadne kramy z~jedzeniem i~piciem nie kaleczyły godności sądu Talgartha.

Metal palisady -- wielkie kawały postrzępionego żelaza, które mogło być
płytami statków, ale które było rozdarte jak paski kory i~zatopione w~glebie -- lśnił czerwono i~zardzewiale w~słońcu. Uzbrojenie palisady
ciągle się poruszało, wahadłowo lub obrotowo. Na zewnątrz, dziedzina
maszyn dawał odczuć swoją obecność gejzerami płomieni, rykami i~piskami
zderzających się silników w~pogoni za ich niezrozumiałymi i~niezgodnymi
celami.

Wilde szedł przez grupę ludzi, machał tym niewielu, których rozpoznawał
jako zwolenników, a~potem poszedł do centrum sądu. Pracownicy i~roboty
ustawiali dach z~czerwonego prostego płótna nad podium. Pod nim, w~centrum podium, było składane krzesło z~jasnego drewna i~postrzępionego
szarego płótna i~mały stół po prawej stronie krzesła. Na stole leżała
szklanka, butelka, młotek i~popielniczka.

Wilde badał ten układ przez moment, uśmiechnął się i~odwrócił. Znalazł
się twarzą przed kamerą pyłku wiadomości. Pyłek przypominał rozumnego
robota, którego spotkali na skrzyżowaniu, ale układ mikrofonów i~soczewek nie zostawiał żadnego miejsca na nic bardziej złowieszczego.

Soczewki nie były tylko dla kamer. Gdy maszyna zrobiła delikatny krok do
tyłu na swoich owadzich nogach, zaskoczyła Wilde'a przez ukazanie
sobowtóra blond dziewczyny, którą widzieli przedstawiającą wiadomości.
Stała na trawie na prawo od maszyny.

-- Wygląda solidnie -- wyszeptał Wilde do Tamary -- nie holo\ldots

-- To jest w~Twoich kontaktach -- odsyknęła Tamara, pokazując dzielnie
zęby kamerze.

-- Program Prawny! -- powiedziała jasno dziewczyna. Jej głos dobiegał, w~dziwnym brzuchomówstwie, z~głośników maszyny. -- Dzień dobry, Szacowny
Starszy Wilde!

-- Dzień dobry -- powiedział Wilde, uśmiechając się do niej. Jego papieros
zgasł w~trawie.

-- \emph{Patrz w~kamerę} -- wyszeptała Tamara. Wirtualny obraz dziewczyny
natychmiast przeleciał przed kamerę i~stanął w~powietrzu.

-- Czy zechcesz skomentować, Szacowny Starszy Wilde?

-- NIC SZCZEGÓŁOWEGO -- doradził MacKenzie.

-- Tak -- powiedział Wilde. -- Nie ma powodu nazywać mnie ,,Szacownym
Starszym''\ldots droga pani. Nazywam się Jonathan Wilde, moi przyjaciele
mówią mi ,,Jon''. -- Posłał jej uśmiech, który sugerował, że byłby
zaszczycony móc zaliczyć ją do nich. Potem kaszlnął i~powiedział,
bardziej formalnie: -- Nie skomentuję sprawy, ale jestem zaniepokojony
interpretacjami, które pewne, hm, mniej odpowiedzialne kanały
informacyjne niż wasz, przedstawiają. Błagam każdego, kto może słuchać,
by nie robić nic pośpiesznie, by pozwolić prawu działać, ponieważ jest
to jedyna droga, by zachować i~udoskonalić cywilizowane wartości
anarchii. -- Znów się uśmiechnął. -- To wszystko.

-- Dziękuję, Jonie Wilde! A czy chcesz coś powiedzieć na temat znanych
poglądów Sędziego Eona Talgartha na Twój temat?

-- NIE -- doradził MacKenzie, nagłym błyskiem.

-- Nic w~ogóle -- powiedział wesoło Wilde. -- Mam pełne zaufanie, że
człowiek jego reputacji nigdy by nie pozwolił takim kwestiom wpłynąć na
osąd. Jestem pewien, że mój wybór jego sądu jest dowodem wystarczającym,
że traktuję poważnie swoje wypowiedzi.

Zrobił ruch cięcia dłonią przed sobą i~kiwnął głową. Dziewczyna się
zawahała, dosłownie unosząc się, czekając na więcej, ale Wilde przyjął
minę bez wyrazu i~raźno odszedł z~pola widzenia kamer. Tamara
pośpieszyła za nim.

-- To było bardzo dobre -- powiedziała. Nie brzmiała całkowicie
entuzjastycznie. Wilde ścisnął ją za ramiona.

-- Nie bądź kolejnym -- powiedział.

Spojrzała na niego. Patrzył prosto przed siebie.

-- Kolejnym kim?

-- Kolejnym towarzyszem, który był zawiedziony moim umiarem i~rozsądkiem.
Miałem tego dość w~moim pierwszym życiu.

I z~tym puścił jej ramiona, trącając ją, gdy to robił. Spojrzała znowu
do przodu i~okazało się, że idą prosto w~grupę ludzi dookoła Reida.

Reid miał na sobie luźny wełniany garnitur i~niebieską bawełnianą
koszulę bez krawata. Był pochylony z~lewą dłonią opartą o~oparcie
krzesła, na którym zostawił kubek kawy. Jego prawa ręka trzymała
papierosa, którym wykonywał zamaszyste, podkreślone dymem, gesty. Mówił
do trzech mężczyzn i~kobiety, wszystkich ubranych z~podobną codzienną
troską. Jego długie włosy były wilgotne od kąpieli i~porannego
powietrza.

Kiedy zobaczył Wilde'a, od razu wstał, przełożył papierosa do lewej
dłoni i~wyciągnął prawą. Obaj mężczyźni wymienili uścisk dłoni, obaj
uśmiechając się, obserwując własne twarze i~odkrywając w~nich
rozpoznanie i, prawie, niedowierzanie.

-- Minęło dużo czasu -- powiedział Reid.

-- Nie dla mnie -- odparł Wilde.

Reid przyjął to krótkim skinieniem.

-- Doceniam to -- powiedział. -- Może z~czasem popatrzysz na te sprawy
inaczej.

-- Całkiem dobrze widzę drogę w~Karagandzie -- powiedział Wilde. -- I~Twoją
twarz. Kiedy zamykam oczy. Miałem czas, żeby przemyśleć spojrzenie,
które mi rzucałeś, mój przyjacielu.

-- To nie było osobiste -- powiedział Reid. -- I~to też nie jest.

-- Wiem, że to nie było osobiste -- powiedział Wilde. -- Wiem, że jesteś
lepszy niż to, Dave. Prawie marzę, żeby było.

-- Obaj byliśmy zwierzętami politycznymi -- powiedział lekko Reid. -- Też
podejmowałeś takie decyzja jak ta. W~swoim czasie.

Wilde wzruszył ramionami. Grzebał za papierosem. Reid uprzedził go,
oferując paczkę i~zapalniczkę. Wilde zaakceptował obie, zaciskając usta.

-- Tytoń -- zamyślił się, jakby zauważając nieprawidłową obecność po raz
pierwszy. -- Bawełna. Wełna. Gdzie są plantacje, stada?

-- Synteza organiczna jest naszą najbardziej rozwiniętą technologią -- powiedział Reid. -- Jak powinieneś wiedzieć.

Wilde się roześmiał. 

-- Sprawa zaczyna się za dwadzieścia pięć minut -- powiedział. -- Tyle masz, żeby mnie przekonać, że nie pozwoliłeś mi
umrzeć, żeby mi zamknąć usta na dobre.

Reid dotknął ramienia Wilde, jakby przypominając mu.

-- Nie na dobre -- wskazał. -- Jesteś tutaj i~byłeś\ldots

Przerwał. Wilde natychmiast powiedział, mogło się wydawać, że przerwał.

-- Wystarczająco długo! -- powiedział. -- Prawie to przyznałeś, człowieku!
Chcę, żebyś przyznał i~to wyjaśnił. I~wycofał Twoje śmieszne oskarżenie,
że działania robota Jay-Dub są jakąkolwiek moją odpowiedzialnością, i~uwolnił autonomiczną maszynę, która chodziła koło Ciebie w~ciele
Annette. Przeprosiny za \emph{tę} zniewagę mojej żony i~mnie też nie
byłyby złe. \emph{Potem} możemy pogadać o~innych sprawach.

Drżał lekko, gdy skończył mówić.

Reid wstał, wydmuchując dym powoli z~ust.

-- Jakie inne sprawy?

Wilde pochylił się do przodu, mówiąc tak delikatnie, że tylko Reid,
Tamara i~MacKenzie go usłyszeli.

-- Szybki ludek -- powiedział -- po drugiej stronie Mili Malleya.

Reid lekko się cofnął. 

-- Czy to właśnie Jay-Dub Ci powiedział?

-- Sam do tego doszedłem -- powiedział Wilde. -- To oczywiste, kiedy o~tym
pomyślisz.

Reid pokręcił głową. Przez chwilę, jego twarz pokazywał prawdziwy
smutek. Potem, jego mina oziębła i~się odsunął.

-- Jay-Dub Cię stworzył -- powiedział. -- Zrobił z~Ciebie broń przeciwko
mnie. I~coś jeszcze, ostrzegam cię, stworzyło Jay-Dub takim, jakim jest.

-- Hę? -- odparł Wilde, postępując za jego wskazówką. -- To niezłe
przyznanie się.

-- On był Tobą -- powiedział Reid. -- Symulacją Ciebie, powinienem
powiedzieć. I~przez jakiś czas, był moim przyjacielem. Miał mnóstwo
czasu, żeby mnie oskarżyć o~jego, Twoje, morderstwo lub zaniechanie, i~nigdy tego nie zrobił. Ponieważ on rozumiał. Miał większy umysł niż Twój
czy mój, Jon, i~rozumiał. Niemniej jednak był, koniec końców, maszyną.
Maszyną z~własnym przeznaczeniem, z~nieskończoną cierpliwością i~bezdenną przebiegłością. Miałem nadzieję, że element ludzki w~niej
przezwyciężyłby program\ldots maszyny. Byłem w~błędzie i~naprawię ten błąd.
Legalnie, Ty jesteś jej właścicielem i~na to cię złapię. Jednak w~rzeczywistości, jesteś...

-- Czym? -- wyzwał Wilde. -- Powiedz mi, jak myślisz, czym jestem.

-- \emph{Instrumentum vocale} -- powiedział gorzko Reid. -- Narzędziem,
które mówi. Jon Wilde nie żyje.

Odwrócił się na pięcie, zbierając towarzyszy szorstkim gestem i~odszedł.

\chapter{Kontrakt Terminowy na Walkę}

Po wojnie światowej był rząd światowy. Oficjalnie był znany jako Narody
Zjednoczone, nieoficjalnie jako USA/ONZ, kolokwialnie jako Jankesi.
Utrzymywał pokój, z~kosmosu, lub tak twierdził. To, co rzeczywiście
robił, to zapobiegał niezliczonym małym wojnom stać się wielkimi. Jednak
w celu utrzymania własnej władzy, potrzebował małych wojen, i~te nigdy
nie znikły. Mieliśmy wojnę bez końca, by zapobiec wojnie do końca.
USA/ONZ zatrzymywała najbardziej zaawansowaną technologię we własnych
rękach, żeby zachować ją od ,,złych rąk'', to jest każdych rąk, które
mogłyby podnieść się na dominację USA/ONZ. Nie było to tak straszne, jak
obawiały się pokolenia amerykańskich dysydentów. Nie było, w~długim
czasie, tak straszne jak pokolenia globalnych idealistów miały nadzieję.
Zostawiało to sporo swobody złym rządom.

Traktat o~Restytucji, podzielony system ,,społeczności pod Królem'', był
wkładem Brytanii do historii hańby. W~szczelinach Korony rozkwitły
wszystkie rodzaje Wolnych Państw: regionalistyczne, rasistowskie,
kreacjonistyczne, socjalistyczne, nawet -- w~przypadku naszego własnego
Norlonto -- anarchokapitalistyczne.

Korona była karykaturą minimalnego państwa, który niósł tę samą relację
do mojej utopii jak ,,kiedyś naprawdę istniejący socjalizm'' dla mojego
ojca. Ludzie, którzy radzili sobie najlepiej w~układzie byli marginesem,
którzy osiedlał się na wsi i~nazywał siebie Nowymi Osadnikami, a~których
my ludzie z~miasta nazywaliśmy nowymi barbarzyńcami.

Po dwudziestu latach tlącej się wojny wszystkich przeciwko wszystkim
Armia Nowej Republiki ogłosiła Ostateczną Ofensywę po raz czwarty.

~

-- Musisz z~nimi pogadać -- powiedziała Julie.

-- Dlaczego kurwa powinienem? -- odpowiedziałem, nie odwracając się od
okna. Piękny poranny widok obrzeży Greenbelt w~North London był
poznaczony kłębami białego dymu z~dalszej części Trent Park. Doliczyłem
do kilkunastu sekund, zanim usłyszałem tępe uderzenia artylerii, nie
usłyszałem wybuchów pocisków. Prawdopodobnie poza horyzontem. Krążyły
plotki, że Armia Nowej Republiki weszła do Luton. Niezależnie od prawdy,
Luton lub gdzieś niedaleko dostawało ciężki łomot od Królewskiej
Artylerii.

-- To Twój problem -- kontynuowałem, odwracając się do niej. W~sposób,
który stał się znajomy przez lata, ale którego nigdy nie przestałem
zazdrościć w~innym w~średnim wieku, wydawała się zmienić niewiele
pomiędzy dwudziestką a~pięćdziesiątką. Najbardziej widoczna różnica
pomiędzy moim byłym Organizatorem Młodych a~kobietą, która teraz stała w~moim biurze, była taka, że wymieniła swój dawniej niezmienny kombinezon
kosmonauty na bardziej dystyngowaną suknię z~krynoliny.

Ja, teraz w~dziewiątej dekadzie, ciągle byłem twardy i~energiczny, dumny
w mojej własnej skórze, a~mój mózg ciągle pracował słodko i~czysto,
napędzany liniami komórek płodowych. Jednak przedłużenie życia i~perspektywa nieskończonego przedłużania, pozbawiła mnie stoickiej
dojrzałości i~oderwania, które czasem dotykało tych naprawdę
zestarzałych w~przeszłości. Zauważyłem w~sobie usztywnienie poglądów,
rozrzedzenie ducha. Pokojową rewolucję, która ustanowiła oryginalną
Republikę, powitałem i~próbowałem wykorzystać. Pogrążyłem się w~chaotycznych możliwościach, które towarzyszyły gwałtownemu końcowi
Republiki. Ale nagła możliwość jej brutalnej odnowie, nowa rewolucja lub
kontr-restauracja, teraz zdeterminowała mnie do zrobienia tylko tego, co
zapewni mi przetrwanie ostatnich obrotów koła, bez oczekiwania, że
gdziekolwiek mnie by to zaniosło.

Za mną okno zatrzęsło się od eksplozji, a~następnie ryk przejścia
niektórych rakiet, doganiających zbyt późno. Musiałem się wzdrygnąć,
ponieważ uśmiech Julii był chytry, kiedy powiedziała:

-- To także Twój problem. Zamierzasz czekać, aż rakiety wpadną
\emph{przez} okno?

-- Nie -- odpowiedziałem. -- Ale dlaczego chcesz, żeby to zrobił? -- Mój
głos brzmiał zrzędnie ku mojej irytacji. -- Dlaczego nie Twoi rzecznicy?

Julii roześmiała się. 

-- Nazwij ich. Jesteś jedynym, o~którym wszyscy
słyszeli. Nasz wielki starzec.

-- Och, dzięki.

-- Oraz -- kontynuowała -- oni nalegają na rozmowy z~\emph{Tobą}, ponieważ
nie byłeś zamieszany w~to, co republikanie nazywają Zdradą.

Nagle stwierdziłem, że palę papierosa. (Pierwszego tego dnia. Jednej z~tych dekad powinienem rzucić je na dobre, ryzyka zdrowotne, lub nie\ldots).

-- Ale byłem -- powiedziałem. -- Cholera, pomagałem hanowerskim bękartom
narysować \emph{mapy}.

-- Ta -- powiedziała Julie. -- A potem cię wyrzuciliśmy, pamiętasz?

-- Więc?

-- Cóż, wszyscy zakładają, że to było dlatego, że byłeś \emph{przeciwko}
Traktatowi.

-- Co! -- Usiadłem na krawędzi biurka i~się roześmiałem. -- Organizacja tak
to przedstawiła?

-- Nie do końca -- powiedziała Julie. -- Po prostu\ldots nie zaprzeczyliśmy
temu. Z~trudem moglibyśmy potępić cię za oportunizm po tym, gdy sami tak
samo zrobiliśmy.

-- Oczywiście, że moglibyście -- powiedział nieobecny. -- Czyż
\emph{niczego} was nie nauczyłem?

Właśnie zrozumiałem dlaczego, od czasu Traktatu, moja reputacja niosła
tajemnicę nienaganności, której przy mojej aktualnej politycznej
aktywności zrobiłem tak mało, żeby zasłużyć. Pomogło mi to w~mojej
drugiej karierze, niezbyt wymagających wykładach z~historii na
Uniwersytecie North London uzupełnionych przez istotniejsze pisanie, niż
kiedykolwiek wcześniej. Pisanie przyniosło mi nieszukaną pozycję guru
ruchu kosmicznego, więcej czytania o~niż czytania. Leniwa ciekawość,
która prowadziła mnie do badania i~odrzucania konspiracyjnej teorii
historii została okrzyknięta jako dawno spóźniona korekta wiedzy
rewizjonistów, moja narastająca cyniczna publicystyka jako głos
radykalnego sumienia Ruchu, kwestionującego nieuniknione kompromisy jego
wolnej hegemonii nad Norlonto.

Julie patrzyła na zegarek, ściskała telefon, szarpała włosy. Kolejna
rakieta przeleciała, tym razem bliżej. Bateria dział ucichła.

-- Ok -- powiedziałem. -- Zabierz mnie do ich przywódcy.

-- Tylko w~sensie wirtualnym -- powiedziała Julie. -- Zabierz \emph{mnie}
do MediaLabu, czy jakkolwiek to jest nazywane w~tych dniach, i~ja cię
dołączę.

Zabrałem kurtkę, komputer i~zgasiłem papierosa. 

-- Co ze studentami?

-- To załatwione -- powiedziała Julie. -- Są na strajku.

-- Och -- powiedziałem, trzymając drzwi otwarte, gdy kierowała suknię
przez nie. -- A gdzie oni pracują?

\threeast

Jakikolwiek wkład do walki studenci myśleli, że dokonują przez
niewłączanie się, zrobili by lepiej, przychodząc, przynajmniej do pokoju
Kabla. W~cichej chłodnej piwnicy Perry Anderson Building z~jego cienkimi
warstwami naturalnego światła z~listwowych okien przy suficie, kamerami,
ekranami, sprzętem immersji VR leżał pomiędzy stosami notatek,
przygryzionych długopisów i~brudnych kubków styropianowych. Julie
włączyła coraz więcej kabli i~połączeń sieciowych, wyświetlając bitwę
mediów prawie tak ważną jak jakakolwiek na powierzchni ziemi.

Brytania, ,,była Brytania'' jak nazywali ją Jankesi, była dla odmiany
światowym newsem, z~ANR rzekomo gotowym do uderzenia i~USA/ONZ
ośmielającym się do kolejnej krwawej interwencji. W~międzyczasie lokalne
rady i~kanały brzęczały plotkami i~debatami. ANR, ze swojej strony, nic
nie mówiło, oprócz manifestu i~harmonogramu pokazującego dokładnie gdzie
i kiedy zamierzają uderzyć. Jutro wyglądało pracowicie.

-- Chcesz głęboką czy płaską? -- spytała Julie, wyrywając mnie z~fascynującego, rozwijającego się wątku dyskusji w~jednym w~mini państw
Yorkshire.

-- Płaską. -- Nigdy nie mogłem znieść kłopotów rękawiczek, gogli i~sprzętu, tak jak to widziałem, jeżeli zamierzałeś wyposażyć się w~ten
sposób, równie dobrze mogłeś próbować jakiejś zdrowej perwersji, zamiast
wnętrza komputera.

-- Ok, łączę cię teraz.

Dyskusji na grupie (i jej prawie równo intrygujące akompaniujące
postacie z~kreskówek, nazywane buźkami, które pokazywały miny,
obsceniczne gesty lub toczyły się ze śmiechu na marginesach, w~połysku
graficznym głównej debaty) zniknęła i~pojawiło się łącze wideo.

Nierówny odbiór, rysy jak w~starym filmie (kryptografia była ustawiona w~tej chwili z~kampusowego freeware w~North Carolina, zgodnie z~ich
oburzonym, podskakującym demonem copyleft w~rogu) i~jakość głosu jak źle
zdubbingowany irański pornos, ale nie było wątpliwości, kto jest po
drugiej stronie.

-- Cóż, witam Jon.

-- Cześć Dave. Nie spodziewałem się, że będę z~Tobą rozmawiał.

(- \emph{Znasz }tego gościa? -- syknęła Julie).

Dave kaszlnął. 

-- Wynająłem kilka drużyn do, hm, technicznej pracy w~obecnej operacji i~przez jakiś czas miałem dobre relacje biznesowe z~naszymi przyjaciółmi na Północy.

Rozumiałem, co ma na myśli, ale wydawało się niepotrzebnie pośrednie.
Rzuciłem mu, coś, co miałem nadzieję, przeszło jako krzywe spojrzenie.

-- Martwisz się o~szyfrowanie, czy coś? Mam na myśli, to Twoja grupa to
podjęła.

-- Nie, nie. -- Dave skinął głową jakby za moim ramieniem. -- Tylko, co to
za dziewczyna na ławce za Tobą?

-- Och? -- Spojrzałem do tyłu. Julie opierała się na pagórkach sukni, jej
ładne buty dyndały poniżej, jak lalka na półce.

-- Uważaj na słowa, człowieku, to Julie O'Brien.

-- Przepraszam panią -- powiedział Reid. -- Nie rozpoznałem.

-- W~porządku -- powiedziała Julie. -- I~możesz mówić swobodnie. -- Prawdopodobnie pochlebiło jej bycie nazwaną dziewczyną, pomyślałem
ponuro.

-- Ok -- powiedział Reid. Rozluźnił się. -- Faktem jest, Jon, że pracowałem
z ANR przez lata i~spędziłem ostatnie kilka tygodni, pośrednicząc w~umowach z~firmami obrony w~Twoim sektorze.

-- Tak, cóż, zauważyłem, że kontrakty terminowe na walkę\footnote{oryg. combat
futures -- transakcja pochodna zawierana na rynku giełdowym, w~której
strona zobowiązuje się dokonać transakcji za ustaloną ilość instrumentu
bazowego za ściśle określoną cenę w~ściśle określonym terminie,
więcej~\url{https://pl.wikipedia.org/wiki/Futures } -- przyp.tłum.} idą w~górę.

Reid się uśmiechnął. 

-- Tak i~możesz ich użyć jako dźwigni w~ubezpieczeniach\ldots -- Potarł dłonie. -- Wspaniała zabawa, oczywiście, ale teraz, gdy wszystko ustaliliśmy z~właścicielami dróg i~firmami policyjnymi, musimy ułożyć się z~milicją Ruchu. Polityka, nie biznes.
Pomyśleli, że jestem właściwą osobą do rozmowy z~Tobą.

-- Zakładając nasze głębokie osobiste zaufanie.

-- Coś w~tym rodzaju.

-- Naprawdę jutro uruchamiacie ofensywę?

Reid się uśmiechnął. 

-- Nie mogę powiedzieć. Zamierzamy, ale nie
usunęliśmy wszystkich błędów z~naszych systemów.

Domniemywano, że ANR odziedziczyło jakiś diabolicznie mądry program
wojskowy ze starej Republiki, choć jeżeli ich wcześniejsze nieudane
ofensywy były jakąś wskazówką, nie było to wszystko, na co było to stać.

-- Dlaczego ogłaszacie harmonogram, gdzie zamierzacie uderzyć? Większość
strategów ciągle ceni przewagę zaskoczenia, z~tego, co słyszałem.

-- Powiedziano mi, że to środki humanitarne -- zachichotał Reid. -- Pozwala
cywilom opuścić tereny.

-- I~blokuje drogi z~uchodźcami i~daje milicjom minipaństw każdą wymówkę
do chorobowego jutro rano?

-- Tak jak mówiłem...

-- \ldots humanitarne. Ok. Interesy. Jaka jest umowa z~Norlonto?

-- Wiemy, że wasza milicja nie będzie walczyć za Koronę -- powiedział
wolno Reid -- i~nie oczekujemy, że będziecie walczyć za Republikę.
Wszystkie darowizny przyjęte z~wdzięcznością, oczywiście, ale to przy
okazji. Główna sprawa jest taka, że nie chcemy, żeby ktokolwiek myślał,
że was najeżdżamy, jeżeli zdarzy się nam przejechać tranzytem w~wielkich
pojazdach na gąsienicach.

-- Widzę, jak może to być zrozumiane błędnie -- powiedziałem. (Julie, za
mną, prychnęła) -- Jaką mamy gwarancję, że po prostu nas nie zadepczecie?

-- Prócz mojego uroczystego słowa?

-- Tak -- powiedziałem. -- Prócz tego.

-- To nie jest naszym celem. Nie mamy nic przeciwko Norlonto. Niektóre z~tych małych Wolnych Państw będą musiały być sprzątnięte, ale nie
jesteście na liście.

Kurwa, wspaniale. 

-- Ok, a~co z~tym. Ostrzał artyleryjski i~rakietowy
Norlonto przez ANR jest wstrzymany \emph{od zaraz}. Wasi żołnierze mogą
przejechać, ale nie mogą zostać i~\emph{w szczególności} nie mogą
wyprowadzać żadnych ataków na Hanowerczyków z~pozycji w~Norlonto, nawet
przy pozwoleniu właściciela posesji.

-- To wystarczy -- powiedział Reid.

-- To łamie Traktat -- powiedziała Julie, jakby ta sprawa właśnie do niej
dotarła.

-- W~istocie tak -- powiedział Reid sucho. -- Więc na wszelki wypadek,
gdybyśmy przegrali tę rundę, sugeruję, żeby Jon ujawnił tę umowę ponad
waszymi głowami. Wszyscy ci, którzy zaakceptowali Traktat, zrezygnują ze
stanowisk z~niesmakiem, a~Jon przejmie władzę na kolejny dzień lub dwa.

-- Co! -- Julie i~ja powiedzieliśmy w~tej samej chwili.

-- Pewnie -- kontynuował Reid niewzruszenie. -- Zróbcie z~niego dyktatora,
czy coś. W~ten sposób, będzie mógł wydawać rozkazy bojówkom i~przyjąć
baty, jeżeli przegramy. Zawsze możecie go potem zastrzelić, jeżeli
wygramy, a~on okaże zbyt dużo przywiązania do pracy, ale jestem pewien,
że to nie będzie konieczne.

-- Żądasz bardzo wiele -- powiedziałem. -- Jeżeli przegrasz, będę za to
wisiał.

-- Och, nie martwiłbym się o~to -- powiedział Reid beztrosko. -- Jeżeli
przegramy, to dlatego, że wejdą Jankesi, a~wtedy i~tak będziesz martwy.

-- Czy to nie dotyczy reszty z~nas? -- spytała Julie. -- Mam na myśli, po
co się męczyć z\ldots ? -- Pomachała dłonią.

-- Droga obywatelko -- powiedział Reid z~udawaną cierpliwością -- Jankesi
mają \emph{listę}. On na niej jest, a~Ty nie.

-- Cóż -- powiedziałem po tym, gdy to zapewnienie dotarło -- jak mogę
odmówić?

-- Dobry człowiek -- powiedział Reid. -- Mam nadzieję, że jeszcze cię
zobaczę.

-- Tak jak i~ja, chłopie -- powiedziałem. -- Tak jak i~ja.

~

Następnego dnia ofensywa ANR się rozpoczęła (\emph{Bum według Rozkładu!}, jak
wydanie południowe \emph{Sun-Times} to podsumowało), ale utknęło w~martwym punkcie i~wycofało się przed końcem dnia. Pojawiła się historia,
że była to wina jakiegoś problemu softwarowego, ale ciężko to sprawdzić.
Myślę, że strajki powszechne i~lokalne insurekcje, które wybuchły w~tym
samym czasie miały więcej z~tym wspólnego. Szczęśliwie, przez kolejne
kilka dni, te cywilne powstania pociągnęły rewolucję ku zwycięstwu.
Kiedy stało się jasne, że Ameryka też strajkowała i~żołnierze nie
przybywali, Przywrócony Rząd Hanowerski haniebnie odleciał
helikopterami, żeby ,,kontynuować walkę z~terroryzmem z~wygnania'', jak
to ujęli.

Upadek USA/ONZ był podobnie przypisany, w~swego rodzaju teoriach
konspiracji, które kiedyś obaliłem na zawsze, jakiemuś wirusowemu
natarciu na globalne sieci informacyjne. Jednak chwila obiektywnego
myślenia wykaże, że insurekcje w~Brytanii i~na Syberii, równoległe z~eskalującym sporem z~Japonią nad kontrolą broni, były tym, co
ostatecznie przekonało Amerykanów, że światowa dominacja nie jest warta
kolejnej podwyżki podatków czy poboru do wojska. Insurekcje naśladowcze,
jak były nazywane, rozlały się po całym świecie z~prędkością plotki
internetowej. Zakłócenie związane z~tym, co doprowadziło do światowej
rewolucji, jest, w~mojej opinii, bardziej niż wystarczającym
wyjaśnieniem dla stanów chaotycznych na wszystkich komputerach przez
następne kilka miesięcy.

W tym czasie miałem znacznie ważniejsze sprawy na głowie jak próbowanie
wymyślenia sposobu na utratę mojej nowej pracy bez przekazywania jej
komuś gorszemu. Powinienem był wiedzieć lepiej na początku, zamiast stać
się dyktatorem, ale to jest anarchizm dla was. To po prostu nie jest
żadne przygotowanie na obowiązki rządu.

~

Luty, 2046. Najzimniejsza zima od lat. Ludzie mówili, że to była dziura
w szklarni, gdy palili ogniska wczorajszą gotówką.

Mieliśmy naszą własną szklarnię, naszą geodezyjną kopułę w~Trent Park,
niedaleko uniwersytetu. Studenci byli zajęci robieniem błędów o~demokracji i~elitaryzmie, które były traktowane passe, kiedy byłem w~Glasgow. Zostawiłem ich w~tym. Annette powoli poruszała się przy jej
eksperymentach ogrodowych, w~kitlu laboratoryjnym zrobionym z~futra.
Wytrząsłem trochę propagandy sieci, przemawiałem ochrypły na Kablu,
zbierałem wirtualne spotkania fakcji Norlonto i~wykułem stanowisko do
przedstawienia narodowemu rządowi.

Dla relaksu rozmawiałem z~ludźmi w~kosmosie. Poza osadami Lagrange i~Księżycem łatwiej było emailem, naturalniejszym medium biorąc pod uwagę
opóźnienia prędkości światła. Górnicy asteroid uroczyście pytali o~moją radę w~sprawie banków spółdzielczych, marsjańscy koloniści
narzekali na porzucenie, teraz gdy Obrona Kosmiczna była redukowana.
Rady żołnierskie na byłych stacjach orbitalnych Obrony Kosmicznej
testowali na mnie idee bardziej zyskownych sposobów użycia dział
laserowych. (Byli dobrymi dzieciakami, naprawdę, lub pomyśleliby o~oczywistym sposobie).

W międzyczasie wojna domowa trwała. Skromny cel Republiki połączenia
narodowej unii z~lokalną autonomią wielokrotnie zderzał się z~tubylcami,
których idea autonomii była znacznie bardziej ekspansywna. Jako państwo,
Republika była na wiele sposobów słabsza niż Korona -- z~jej zawsze
obecnym, tuż nad horyzontem wsparciem -- kiedykolwiek była. Bardziej
fundamentalnie, rewolucja wystawiła wszystko na grabież, stworzyła
bodźce do zdrady, jak przedstawiali to teoretycy gry.

Uchodźcy wlewali się w~Norlonto ze wsi i~kontynuowali walki w~slumsach i~obozach. Wysiłek organizacji charytatywnych oraz tak samo firm ochrony
zwiększał się co tydzień i~co tydzień krzyczałem na organizatorów, by
rekrutowali nowych pracowników spośród samych uchodźców.

To działało, póki stało się problemem, kto kogo rekrutował.
Konkurencyjne firmy policyjne znalazły się dosłownie w~rywalizujących
uzbrojonych obozach, gdzie kwatermistrzostwo, prawdopodobnie, było
autoryzowanymi dystrybutorami fundacji charytatywnych. Nazywaliśmy to
Syndromem Tajlandzkim.

~

Cotygodniowe spotkania Komitetu Koordynacji Obrony stały się dzienne,
lub raczej, nocne. Zwykle zaczynały się o~21:00 i~toczyły się aż do
późnej nocy. Nie miałem z~tym problemu. Moje potrzeby snu zmniejszały
się z~wiekiem. Nie lubiłem wchodzenia do VR, ale takie jest życie.
Każdego wieczoru zdejmowałem rękawiczki do zmywania, zakładałem
VR-rękawice, uśmiechałem się do Annette ponad czystym stołem, zakładałem
szkła i\ldots

Byłem tam. Niektórzy z~nas wyobrażali sobie siebie jako Bohaterów w~Piekle\footnote{oryg. Heroes in Hell,
zob.~\url{https://en.wikipedia.org/wiki/Heroes_in_Hell } -- przyp.tłum.}, i~otoczenie było właściwe: czarna nieskończoność wokół
nas, a~pomiędzy nami okrągły stół z~rzutem Norlonto, Londynu lub
czegokolwiek, czemu chcieliśmy się przyjrzeć, widok \emph{camera
obscura}, złożony ze zdjęć satelitarnych i~rozszerzonymi wszystkimi
danymi, jakie mogliśmy dodać. Na tym poziomie było nas trzynaścioro,
zawsze szczęśliwa liczba dla komitetu. Nasze sobowtóry -- nasze obrazy
ciał w~świecie wirtualnym -- były takie same jak rzeczywiste formy,
głównie, żebyśmy mogli rozpoznać się w~świecie realnym lub w~telewizji.

W noc wielkiego kryzysu nie było jednej osoby. Rozejrzałem się,
zmartwiony. Julie była tutaj, Mike Davis, Juan Altimara, wszyscy z~różnych kierunków Ruchu Kosmicznego, para identycznych młodych, który
mentalnie opisałem jako ,,misjonarze mormonów'', choć rzeczywiście byli z~organizacji charytatywnej chroniącej kościoły w~Norlonto, Związek Obrony
świętego Maurycego. I~przechodząc od sektora ochotniczego do
komercyjnego, grupa delegatów przedsiębiorstw obrony, którzy zmieniali
się z~tygodnia na tydzień i~zawsze wyglądali alarmująco młodo i~żałośnie
zmęczona, i~zawsze spierali się z~lewicowcami\ldots

-- Gdzie jest Catherin Duvalier? -- Była młoda, szybka, bystra:
koordynator komunistycznej bojówki, której sieci wywiadowcze rozszerzały
się przez obozy Zielonych do odległych bitw na wzgórzach.

Julie uśmiechnęła się do mnie przez jasną otchłań stołu.

-- Cat wychodzi dzisiaj za mąż. Przesyła przeprosiny.

-- Nie ma usprawiedliwienia -- chrząknąłem, ale poczułem ulgę, że nie
mieliśmy dezercji, lub faktycznie ofiary. -- Ok, towarzysze. Najpierw
interesy.

Wbiłem dzienne dane handlowe dla udziałów obronnych i~kontraktów
terminowych walki. Rosły szybko.

-- Dobra, chłopcy -- powiedziałem do chłopaków z~agencji obrony -- czy
wiecie coś, czego nie wiemy?

Błyśnięcie wymiany danych wprawiło sobowtór w~drżenie jak miraż. Potem,
szybka wymiana zdań skończona, jeden z~nich przemówił.

-- Właśnie mieliśmy powiedzieć, panie Wilde\ldots

\emph{Och, pewnie.}

-- \ldots wszystkie nasze przedsiębiorstwa zostały oddzielnie odpytane na
temat, hm, potencjalnych sytuacji konfliktu. Wydaje się, że raz jeszcze
duża liczba właścicieli ulic umówiła się, żeby pozwolić na przejście,
hm, kolumn pancernych\ldots

-- Masz na myśli, że \emph{Armia }wchodzi?

Wirtualne oczy heliografowały\footnote{system
semaforowy, który przesyła błyski słońca,
więcej~\url{https://en.wikipedia.org/wiki/Heliograph } -- przyp.tłum.} szok dookoła stołu. 

-- Tak -- powiedział nieraźno. -- Zostaliśmy poinstruowani do przekazania
informacji Tobie, że rząd zdecydował się zakończyć anormalny stan
Norlonto, ich słowami. Jest to wynikiem wniosku znaczącej większości
społeczności biznesowej i~pewnej liczb bardziej, hę, osiadłych
towarzystw sąsiedzkich\ldots

-- Bękarty! -- krzyknęła Julie. Skierowała się do ,,mormońskich
misjonarzy''. -- Wiedzieliście cokolwiek o~tym?

-- Nie patrz na mnie w~ten sposób -- powiedział jeden z~nich.
Przekazywaliśmy tygodniami skargi od naszych klientów. Sytuacja naprawdę
staje się całkiem nie do zniesienia, szczególnie dla mniej pomyślnych.
Zapewniam was, że Związek o~tym nic nie wiedział, ale nie mogę
powiedzieć, że jestem zaskoczony albo że mi przykro.

-- Więc -- powiedziałem -- kiedy czołgi wjeżdżają?

-- Pojutrze -- powiedział jeden z~delegatów agencji. -- Pokaz siły i~takie
tam. Porządek na ulicach.

-- Dobrze -- powiedziałem. -- To daje nam czas na organizację.

-- \emph{Ruch oporu}? -- Kilka głosów powiedziała to w~tej samej chwili, z~przerażeniem lub z~nadzieją.

-- Nie -- powiedziałem ponuro. -- Odwrót. Powiedzcie swoim przełożonym i~rządowi, że nie będzie kłopotu ze strony milicji.

Rozejrzałem się dookoła stołu, moja ręka na klawiaturze prawdziwego
stołu wystukująca pilną wiadomość do ludzi z~Ruchu Kosmicznego, żeby
jeszcze zostali. 

-- Spotkanie odłożone. Do zobaczenia jutro.

\threeast

-- \emph{W co} ty kurwa pogrywasz, Wilde? -- spytała Julie, kiedy fundacje
i biznesy opuściły scenę. -- Nie możemy tego przyjąć, leżąc. To będzie
koniec Norlonto!

Mike Davis i~Juan Altimara pokiwali w~oburzonej zgodzie.

-- Och wy małej wiary -- powiedziałem. -- Oczywiście, że to będzie koniec
Norlonto. Zdaje się sobie przypominać, że większość z~was nie była zbyt
zapalona na \emph{początku} Norlonto.

Juan, który przybył do Norlonto jako dziecko, uciekinier z~krótkiej
biowojny w~Brazylii podczas Secesji Amazonii, spojrzał na Mike'a i~Julie. Grzybicza blizna na policzku wykręciła się, gdy zmarszczył brwi.

-- Tego nie wiedziałem -- powiedział.

Julie zarumieniła się, Mike bawił się swoim przełącznikiem: 

-- Teraz po jabłkach. -- powiedział niewygodnie. -- Chodzi o~to, że Norlonto było
bastionem wolności przez lata, udanym eksperymentem, a~Ty chcesz
pozwolić etatystom wmaszerować bez oddania strzału!

-- Przepraszam, towarzysze -- powiedziałem -- ale kto tutaj kapituluje
przed etatyzmem? -- Grzebałem w~wirtualnych głębokościach stołu,
podświetlając możliwe drogi najazdu i~sprawdzając je wobec ruchów ocen
ubezpieczenia, dyslokacji agencji obrony, bunkrów milicji. -- Tak jak to
widzę, jeżeli klienci różnych agencji obrony, jeżeli społeczności i~właściciele nieruchomości tego miasta chcą robić interesy z~upaństwowionym przemysłem obronnym, jaki mamy w~tym interes? Czy to nie
jest anarchokapitalizm w~działaniu?

-- Bardziej kapitaliści wyprzedający anarchię! -- powiedziała Julie.

-- Ponieważ mają prawo tak zrobić -- powiedział Mike. -- Ta, muszę się
tutaj zgodzić z~Jonem. Jednak, to znaczy, że zawiedliśmy.

Julie i~Juan oboje przyglądali się rozszerzonej mapie przybierającej
kształt. Spojrzeli w~górę, spojrzeli na siebie.

-- Nie musimy zawodzić -- powiedział Juan. -- Bojówki są dostatecznie
silne, żeby wstrzymać siły Republiki. Mamy czas, żeby wyprowadzić
populację. Armii nie ujdzie na sucho masakra we własnej stolicy, nawet
Hanowerczycy powstrzymywali się od tego.

-- Uchodzi na sucho mordowanie na wsi -- powiedziałem. -- Czy kiedykolwiek
\emph{słuchałaś} jakichkolwiek uchodźców?

Julie odrzuciła ten komentarz ruchem dłoni. 

-- Jeżeli wierzysz jęczeniu tych ludzi o~potwornej tyranii Republiki, która najwidoczniej nie jest,
to\ldots

-- Więc dlaczego jesteś zmartwiona tymi żołnierzami na ulicach?

-- Ponieważ\ldots -- Julie spojrzała na mnie, jakbym nie rozumiał czegoś tak
oczywistego, że nie wierzyła, że musi to powiedzieć. -- Ponieważ to jest
nasze miasto, cholera! Nasze wolne miasto! Nie możemy pozwolić państwu
wjechać po tylu latach. Powinniśmy sami złamać te obozy, zrobić to
\emph{teraz}, wygonić te mafie i~buntownicze bojówki, pozbyć się nawet
\emph{tej} wymówki do wejścia Armii. Jeżeli ruszymy teraz, moglibyśmy to
zrobić dzisiaj!

Widziałem Mike'a biorącego do serca tę sugestię, podczas gdy moje serce
zamarło. Wolałbym, żeby Catherin Duvalier wybrała inny dzień na związek.
Kłótnia trwała.

Motyl wleciał z~nieskończonej ciemności dookoła nas i~usiadł na stole,
skrzydła drżące.

-- Och, gówno -- powiedział głosem Annette. -- Mam nadzieję, że ta cholerna
rzecz działa\ldots

-- Widzimy Cię, Annette -- powiedziałem. -- Co robisz? Jak się tutaj
dostałaś?

Poczułem jej dłoń, niesamowicie niewidoczną, muskającą moją.

-- Wybaczcie mi -- powiedziała. -- Wiem, że nie powinnam tu być, i~nic nie
zhakowałam, czy coś. W~rzeczywistości siedziałam po drugiej stronie Jona
i mogłam zobaczyć, co mówi, i~podeszłam za nim i~podszyłam się pod jego
linkiem, i~krążyłam dookoła tej rozmowy\ldots

-- To jest zagrożenie bezpieczeństwa! -- powiedział Juan.

-- To nie jest zagrożenie, to moja żona -- powiedziałem. -- Jest jedną z~tych, która zabezpiecza moją fizyczną lokalizację, kiedy jestem tutaj, i~zawsze to robiła. Więc się zamknij towarzyszu i~posłuchajmy, co ma do
powiedzenia. \emph{Jeżeli wszyscy się zgadzają}. -- Rozejrzałem się nad
stołem i~wszyscy, w~końcu, pokiwali głową.

-- Ok -- powiedziała Annette. W~rzeczywistości fizycznej wsunęła mi się na
kolana i~objęła mnie ramieniem. W~VR podleciała, podniecona, potem
zaczęła opadać i~trzepotać dookoła mapy, jakby przyciągana do światła. -- Mówicie, że pozwolenie Republice na przejęcie Norlonto byłoby straszną
porażką i~hańbą. Dobra. Jestem pewna, że nawet Jon tak myśli. Jednak czy
pomyśleliście, jaka byłaby to porażka i~hańba, gdyby upaść we krwi? Lub
wygrać i~samemu stać się państwem? Musielibyście walczyć nie tylko z~armią, ale z~firmami ochrony, i~to byłby koniec anarchii wolnego rynku,
z której jesteście tak dumni. Co do wyrzucenia uchodźców, właśnie o~tym
naprawdę mówisz Julie, nie byłoby to po prostu złe, byłoby przez lata
używane jako dowód, że to, co mamy tutaj nie różni się od tego, co
\emph{oni} mają \emph{tam}.

Usadowiła mi się na kolanach i~na mapie. 

-- Ale jeżeli wpuścicie Armię,
jak myślicie, co się stanie? Armia zostanie wciągnięta w~nasz sposób
robienia rzeczy, ekonomiczny, a~nie polityczny. Będę musieli zawierać
umowy, handlować kontraktami terminowymi, przedstawiać spory przed
przedsiębiorstwami sądowymi, wymieniać się prawami i~całą resztą tego.

-- Skąd wiesz, że po prostu nie będą robić rzeczy na swój sposób? -- spytał Mike.

-- Ponieważ Julie ma rację -- powiedziała Annette. -- Nie chcą mieć walki w~rękach. Nie chcą nas podbijać, chcą nas wykupić. Faktycznie wygląda, że
już wykupili przedsiębiorstwa obrony. A co kupione może być sprzedane.
Zanim będą wiedzieć, będą praktykować anarchokapitalizm, nie wierząc w~słowo z~niego.

-- Tak jak każda inna grupa, która tutaj się pojawiła -- powiedziała
kwaśno Julie. -- I~spójrz, gdzie nas to doprowadziło.

-- Tak, spójrz -- powiedziałem. -- Dało nam to dwadzieścia lat pokoju i~wolności, i~tolerancji pomiędzy ludźmi, którzy razem i~oddzielnie
śmiertelnie się nienawidzili!

Juan, Mike i~Julie musieli się roześmiać. Był to osławiony fakt, że
libertarianie w~Norlonto byli rzadsi niż komuniści w~tym, co Reid zwykł
nazywać państwami robotników.

-- Myślę, że Annette ma rację -- powiedziałem ostrożnie, jakby to nie było
to, o~czym od początku myślałem i~nie mogłem się zebrać, żeby
wypowiedzieć (Nigdy nie udałoby mi się uzyskać efektu jej namiętnego
pacyfizmu). -- Jest jeszcze jedna sprawa, o~której staramy się zapomnieć,
i ostatnio mnie męczyła. Przez lata byliśmy tak zajęci prowadzeniem
Norlonto, w~miarę jak się samo nie prowadziło, że byliśmy skłonni
ignorować co się dzieje w~kosmosie. Wiem, wiem, to była sprawa
,,socjalizm w~jednym kraju'' przeciwko sprawie światowej rewolucji i~Obrona Kosmiczna trzymała wysokie orbity, i~oprócz Alexandra Port nie
było zbyt wiele, co moglibyśmy praktycznie zrobić. Pamiętam lata temu,
niektórzy z~nas próbowali zbudować eksperymentalne wyrzutnie laserowe i~zostali rozdeptani z~wielkiej wysokości. Jednak teraz Obrona Kosmiczna
jest w~opałach, a~my mamy przyjaciół, towarzyszy, w~Lagrange i~na
Księżycu próbujących zbudować ekosystemu ze szmaty, kości i~zbiornika
powietrza. Nadszedł czas, żebyśmy coś z~tym zrobili. Więc powiem, jeżeli
etatyści chcą Norlonto, niech wezmą. Mamy lepsze rzeczy do zrobienia.

Usiadłem, czując przesunięcie się ciężaru Annette, widząc drżenie obrazu
motyla. Troje liderów Ruchu Kosmicznego patrzyło na mnie i~rozmawiało
pod stołem, do pewnego stopnia, ze sobą. Miałem nadzieję, że
przynajmniej skrycie poczuliby ulgę od idei ratowania honoru Ruchu przez
\emph{brak} walki.

Sobowtór Juana rozjarzył się od przychodzącej informacji, zadrżał z~powrotem do swojego obrazu.

-- Ok, Wilde -- powiedział. -- Myślimy, że możemy to pchnąć dalej.
Przygotuj się wczesną pobudkę. Julie zamierza ustawić wywiady z~Tobą na
tak wielu kanałach, jak to tylko możliwe. Teraz, mniemam, że jest czas
na\ldots

-- Powrotu do swoich okręgów i~przygotowania się na rząd -- powiedziałem.

Nikt się nie roześmiał.
\threeast
Kiedy reszta zblakła z~widoku, poruszyłem się, żeby zdjąć szkła, i~poczułem ręce Annette chwytające moje nadgarstki.

-- Nie -- powiedziała. Zanurkowała w~moją twarz, przeszła na tył i~wróciła
dookoła. -- To jest zabawne. Dlaczego nie powiedziałeś mi o~tym
wcześniej?

Wstała, ściągnęła mnie z~krzesła, położyła mnie na prawdziwym stole,
wirtualne obrazy naszych bezstanowych stanów poruszające się przed moimi
oczami. Po omacku szukaliśmy, grzebaliśmy i~pieprzyliśmy się na
kuchennym stole, na mapie miasta, podczas gdy nad nami dwa motyle
uprawiały miłość w~nieskończonym mroku.

\chapter{Kolejne Pęknięcie w~Immanentyzacji Eschatonu }

Dee doświadcza swojej pierwszej, grzesznej przyjemności. Przyjemność
wynika z~siedzenia na trawiastym, poznaczonym głazami stoku doliny pod
słońcem Ziemi. Niebo jest innym błękitem, chmury jest innym białym,
niepodobnym do niczego, co kiedykolwiek widziała, nawet w~jej snach
Opowiadań. Na dnie doliny, daleko poniżej jej, brązowa rzeka toczy się
nad czarnymi kamieniami. Jeszcze dalej w~dolinie, spokój sceny jest
zakłócony przez łomot budowy rozległego pylonu. Jednak z~miejsca, gdzie
siedzi Dee, odległy hałas tylko podkreśla otaczającą ciszę. Pęd pracy
półtuzina małych, poruszających się postaci tylko przypomina jej, że nie
ma nic innego do roboty, prócz relaksu i~głębokiego oddychania czystym,
gęstym powietrzem Ziemi.

Wina wynika z~tego, że wszystko jest iluzją: pełna immersyjna
rzeczywistość wirtualna, którą ją tak oczarowała, że rozumie
precyzyjnie, dlaczego ta uwodzicielska przewrotność zmysłów jest tak
niemile widziana. Najbardziej dekadencki sybaryta z~górnych pięter
Miasta Statku poinformuje cię surowo, że tego rodzaju rzeczy są
nienaturalne, zepsuły moralną tkankę wielkich cywilizacji i~sprawią, że
oślepniesz.

Czuje się nieco winna, że Ax nie może tego dzielić. Jest zablokowany w~rzeczywistym świecie, kręci się z~tyłu ciężarówki. Pojazd półgąsienicowy
jest jak gigantyczna, wydłużona wersja górnego kadłuba Jay-Duba. Jego
skóra ze szczotkowanego aluminium ukrywa kilka centymetrów pancerza.
Nuklearne turbiny mogą nadać temu prędkość maksymalną stu kilometrów na
godzinę na płaskiej powierzchni i~bez przeszkód. W~jego magazynach jest
wiele groźnych i~fascynujących rzeczy, ale sprzęt do immersji VR nie
jest pośród nich.

Dee jest włączona przez bezpośrednie wejście do kory, gniazdo za jej
uchem. Ax mógłby też to zrobić, ale jest tylko jedna wtyczka, a~ona tego
potrzebuje, lub raczej, to jest potrzebne jej. Ax ciągle (tak zakłada)
siedzi pod podniesionym wizjerem klapy tylnej, z~nogami bujającymi się
na końcu ciężarówki i~stosuje swoją wersję telepatii wobec kiepskiego
odbioru starego telewizora. Strzeże (taką ona ma nadzieję) ich przed
drapieżnikami, łowcami nagród i~burzami piaskowymi. Systemy pojazdu,
oraz Jay-Dub, są dobrze przygotowane na nie wszystkie, ale gdy Dee patrzy
na wirtualną dolinę, podejrzewa, że mogą być nieco zajęte. Wie niejedną
rzecz o~czasie CPU, a~stąd widzi, jak wiele jest zużywane.

I nie tylko przez Jay-Duba i~pojazd. Jednym z~powodów, dlaczego została
wysłana na wzgórze i~poinstruowana, by robiła tak mało, jak to możliwe,
jest to, że jej własne systemy są prawie w~całości zaangażowane. Jej
ciało, tam w~realnym świecie, leży z~tyłu pojazdu, zwiotczała jak lalka.
Wszystkie prócz dwóch postaci pracujących nad szkieletową wieżą, tuż
poniżej długiego, długiego domu, którego wdzięczne kształty wystają ze
stołu jakby nawis, są jej aspektami. Żołnierka jest tam, Naukowczyni,
Szpiegini i~Sys, pomagając dwóm innym bytom w~ich dziwnej pracy. Sklep i~Sekrety nie manifestują się w~VR jak cokolwiek podobnego ludziom:
zamiast tego, są splątanym, prawie niepenetrowalnymi wiązkami kabli pod
napięciem, ostrymi rogami i~podobnie odstraszającymi obiektami. Ciemne
postaci tańczą i~szturchają dookoła nich i~co jakiś czas wyrywają coś z~zarośli i~z triumfem niosą to, żeby dodać do najeżonej wieży.

Dwa inne byty są tymi, które zamieszkuje cały czas Jay-Duba. Poznała je
ostatniej nocy po tym, jak Jay-Dub zabrał ich w~swojej łodzi w~górę
Kamiennego Kanału, daleko w~pustynię. Są to stary mężczyzna i~młoda
dziewczyna, którzy przemówili do niej z~ciężarówki. (Ciężarówka w~istocie jest wersją tego pojazdu i~potrafi zrozumieć powab jej
iluzyjnej, zagraconej kabiny). Poza kabiną, w~dolinie i~w domu, Meg
jest wdzięczną, elegancko ubraną kobietą, ale w~kabinie jest brudasem.
Jej twarz i~oczy są takie same w~obu wirtualnych środowiskach. Ale jej
oczy zawsze wydają się większe i~ciemniejsze, kiedy jej uśmiech
prześladują Twoją pamięć, niż są, kiedy je znowu widzisz.

Ax otrzymał zadanie obserwowania wiadomości i~pilnowania sprawy sądowej,
kiedy się zacznie. W~międzyczasie, Wilde, starzec w~umyśle robota,
zaprzągł wszystkie zasoby swojego i~jej umysłu, żeby rozwiązać problem -- jak to powiedział -- niezależnie od wyniku. On i~Meg, i~widmowe kształty
oddzielnych Jaźni Dee biegają jak mrówki w~ogniu.

A Dee jest tutaj, na wzgórzu, tylko jako Sama.

\threeast

Tamara złapała łokieć Wilde'a. Jego pięści były zaciśnięte, jego obcasy
były w~ziemi. Pochylał się do przodu, patrząc za Reidem i~jego
towarzyszami.

-- Zawsze możesz go zabić po wszystkim -- powiedziała. -- Jeżeli dojdzie do
walki.

Wilde trochę się uspokoił. Powoli rozwinął dłonie. Uśmiechnął się do
Tamary, żeby ją uspokoić i~spojrzał w~dół na papierosa, który dał mu
Reid. Ciągle się tlił, końcówka z~filtrem ściśnięta pomiędzy palcami.
Zaciągnął się po raz ostatni i~wyrzucił.

-- Powiedział, że jestem lalką, a~Wilde nie żyje. -- Pokręcił głową, potem
wzdrygnął. -- Jeżeli Jonathan Wilde nie żyje, kto go zabił, co?

-- NIEDOPUSZCZALNE -- powiedział mu doradca MacKenzie.

Wilde prychnął, odmrugał unoszący się przypis na temat zasad
postępowania dowodowego i~usiadł na jednym z~krzeseł. Zmiażdżył
papierowy kubek i~wcisnął go do kubka, który zostawił Reid. Sięgnął po
rękę Tamary i~pociągnął ją na krzesło. Usiadła obok, twarzą do niego.

-- Co to było to\ldots -- jej głos się ściszył -- \ldots o~szybkim ludku?

Wilde rozejrzał się dookoła. Krzesła dookoła niego wypełniały się, gdy
ludzie usadawiali się, czekając na początek sprawy: zwolennicy Reida i~ich, jak również wzrastająca liczba osób, która nie pasowała do żadnego
z obozów i~która wpływała przez główną bramę. Ci goście, tak różni od
sojuszy stron sądowych, tworzyli kolorowy pokaz z~ich zhakowanymi
genami, dowolnymi implantami lub biomechanicznymi symbiontami. Zdalne
urządzenia wiadomości grasowały, niektóre na ziemi, niektóre,
obsługiwane przez balony lub małe aureole wirników, dryfowały lub
unosiły się nad głowami. Z~przodu ktoś testował mikrofony, generując
wycia sprzężenia zwrotnego.

-- Nie ma czasu -- powiedział Wilde. Westchnął i~powtórzył, jakby do
samego siebie: -- Nie ma czasu. Potem klepnął dłoń Tamara i~szybko
powiedział: 

-- Popatrz, zobaczyłaś kawałek tego, co Reid naprawdę myśli.
Nie wiem, czy spróbuje tego w~sądzie, nie może twierdzić, że jestem
człowiekiem i~właścicielem Jay-Duba, a~potem odwrócić strony i~mówić, to
co właśnie powiedział. Jednak jest więcej w~tej sprawie niż kwestie
przed sądem. Jeżeli wynik będzie na jego niekorzyść, nie ma szans, że
Reid się na niego zgodził. A jeżeli wypadnie na naszą niekorzyść, nie ma
szans, że \emph{my }się na to zgodzimy!

-- Moglibyśmy wyzwać go na pojedynek -- powiedziała Tamara, jakby to była
dobra idea. Wilde się roześmiał.

-- Czy naprawdę mam duże szanse?

Tamara pomyślała o~tym, spojrzała na niego krytycznie. 

-- Nie. Niezupełnie. Jesteś większy, ale on jest szybszy. -- Ożywiła się. -- Ale
ja mogłabym mieć szansę lub mogłabym wezwać sojusznika. Gówno. Szkoda,
że nie ma z~nami Axa.

-- Zapomnij o~tym -- powiedział Wilde. -- Nie walczysz już więcej w~bitwach
za mnie.

-- Bitwach\ldots -- Tamara wyprostowała się. -- Mówiłeś, że mogą być duże
problemy. Mogę powiedzieć towarzyszom, żeby się przygotowali. Na Placu
Okrągłym mamy naprawdę dobrych wojowników i~ludzi, którzy studiowali
wielkie bitwy anarchistów: Paryż, Kronsztadt, Ukraina, Barcelona, Seul,
Norlonto\ldots

-- Ta, racja -- powiedział Wilde. -- Cóż, nie chcę ci tego mówić tak późno
i tak dalej, ale jest jedna istotna rzecz, którą te wszystkie wielkie
anarchistyczne bitwy miały wspólnego.

-- Tak?

Wilde wstał i~przygotował się, żeby ruszyć do przedniego rzędu.
Uśmiechnął się na gorliwe zainteresowanie Tamary.

-- Wszystkie były przegrane -- powiedział.

~

Wilde zajął miejsce, z~Tamara po prawej stronie i~Ethanem Millerem po
lewej. Inni, którzy przyszli z~nim wypełnili kolejne miejsca po obu
stronach. Dalej na lewo, za przejściem pomiędzy rzędami krzeseł, Reid i~jego najbliżsi zwolennicy zajęli pozycję. Reszta około stu miejsc była
zajęta, a~dwa razy tyle więcej osób -- ludzi lub innych -- przesunęło się,
żeby stać lub siedzieć na trawie. Przed nimi wszystkim stało drewniane
podium z~jego prostymi meblami i~układem mikrofonów i~kamer. Z~naklejek
przylepionych do nich wynikało, że są z~agencji informacyjnych niż z~układu sądu, ale niektóre z~nich były podłączone przewodem do głośników
z tyłu siedzeń, pajęczyna nici kabli błyszczące w~mokrej i~teraz
zdeptanej trawie. Ethan ostentacyjnie sprawdził mechanizm swojego
karabinu.

Minutę przed dziesiątą, głosy ucichły, a~inne dźwięki -- oddychania,
poruszania, nagrywania -- wydawały się głośniejsze, gdy Eon Talgarth
szedł środkowym przejściem. Głowy i~kamery się odwracały. Talgarth
patrzył prosto przed siebie.

Był mężczyzną niewielkiej budowy, średniego wzrostu, z~delikatnymi
brązowymi włosami przygładzonymi pod wysokim kapeluszem. Na sobie miał
prosty czarny garnitur i~białą koszulę z~niebieskim krawatem. Jego rysy
wskazywały na większą dojrzałość niż zwykłe świeże twarze w~Mieście
Statku. Kiedy dotarł do podium, wspiął się na nie i~usiadł ostrożnie w~płóciennym krześle. Napełnił szklankę żółtym płynem, upił i~zapalił
papierosa. Jego zmrużone oczy omiotły tłum.

-- Dobra -- powiedział, londyńskim akcentem, którzy brzmiał archaicznie i~przeciągle w~porównaniu do przyciętej lokalnej wymowy. -- Zaczynaj.

Reid wstał od razu i~podszedł do najbliższego mikrofonu.

-- Sprzeciw -- powiedział Wilde, wstając. -- Mój zarzut jest poważniejszy i~winien być usłyszany pierwszy.

-- Oddalony -- powiedział Eon Talgarth. -- Jego pozew był pierwszy.

Wilde zmienił rozpoczęty wzruszenie ramionami w~uprzejmy ukłon i~usiadł.

-- WARTE PRÓBY -- powiedział mu doradca.

Reid skierował się ku sędziemu.

-- Szacowny Starszy -- powiedział. -- Dziękuję za wysłuchanie nas.

-- Dziękuję za honorowanie Sądu swoim zwyczajem -- powiedział Talgarth. -- Teraz, jaki jest Twój zarzut?

Reid przerwał, a~potem przemówił, jakby czytał z~kartki: 

-- Moje zarzuty
przeciwko Jonathanowi Wilde'owi i~Tamarze Hunter. Mój zarzut przeciwko
Jonathanowi Wilde'owi jest taki, że robot znany jako Jay-Dub, własność
wymienionego Jonathana Wilde'a, został wykorzystany do uszkodzenia
systemów sterowania gynoida Model D, znanego jako Dee Model, mojej
własności. Mój zarzut przeciwko Tamarze Hunter jest taki, że nielegalnie
wzięła w~posiadanie gynoida, następnie twierdziła, że Dee Model jest
porzuconą własnością, wiedząc, że gynoid nie jest porzucony i~podniosła
niestosowną obronę fałszywego twierdzenia autonomii gynoida przeciwko
agentom odzyskującym na rzecz prawowitego właściciela.

Talgarth spojrzał na Wilde'a i~Tamarę.

-- Czy akceptujecie te zarzuty, czy kwestionujecie?

Oboje wstali. 

-- Kwestionujemy.

-- Doskonale -- powiedział Talgarth. Jednym machnięciem w~powietrzu kazał
im usiąść, a~Reidowi kontynuować.

-- Materiał dowodowy tych zarzutów -- powiedział Reid -- został
przedstawiony Waszej uwadze przez Firmę First City Law i~chciałbym
przedstawić je formalnie. Jeden: transkrypcja interakcji pomiędzy moim
gynoidem, znanym jako Dee Model i~inną sztuczną inteligencją. Dwa:
osobiste zapisy interakcji, jakie miałem w~przeszłości, ze sztuczną
inteligencją wcieloną w~robota znanego jako Jay-Dub. Autentyczność tych
zapisów może być, i~została, niezależnie zweryfikowana.

Talgarth skinął głową. 

-- Sąd akceptuje ich pochodzenie.

-- Kwestionować? -- wymamrotał Wilde do mikrofonu MacKenzie.

-- ZERO SZANS.

-- Trzy -- Reid kontynuował -- publiczny zapis własności Jay-Duba,
obwieszczony wiele lat temu w~Banku Spółdzielczym Stras Cobol. Jego
właściciel został zidentyfikowany jako Jonathan Wilde, mój przeciwnik w~tej sprawie.

-- Czy osoba identyfikująca się jako Jonathan Wilde powstanie?.

Wilde spełnił, odwracając się dookoła tak, że każde oko i~soczewka na
miejscu mogła go zobaczyć.

-- Dziękuję -- powiedział Talgarth z~uprzejmym skinięciem do Wilde'a. -- Możesz usiąść. -- Odwrócił się znowu do Reida. -- Kontynuuj.

-- Czwarty i~ostatni -- powiedział Reid. -- Twierdzenie o~autonomii wysłane
przez Usługi Prawne Niewidzialna Ręka, przez Tamarę Hunter, również w~tym sądzie\ldots

Rytuał identyfikacji został powtórzony.

-- \ldots i~rzekomo na rzecz Dee Modelu, rzekomo porzuconego automatonu.

Talgarth łyknął swojego napoju i~spojrzał na Tamarę.

-- Akceptujemy, że takie żądanie zostało wysłane -- powiedziała.

-- Dobrze -- powiedział Talgarth. Wytrząsnął papierosa z~paczki i~go
zapalił.

-- Więc to są dowody -- powiedział. -- Nie musisz przedstawiać dowodu
obrony Dee Modelu przez Tamarę Hunter, jako że incydent jest kwestią
zapisów publicznych. Sąd przyznaje, że jest sprawa do rozwiązania, na
pierwszy rzut oka.

Wilde wstał, mrugając spazmatycznie, gdy MacKenzie ściągał nagłą tyradę
przed oczami.

-- Jesteśmy gotowi do odpowiedzi na nie i~do przedstawienia kontrzarzutów
-- powiedział. -- Jednakże, potrzebuję kilku minut, by wchłonąć pewne nowe
informacje. Błagam o~pobłażanie Sądu na\ldots dziesięć minut?

Fala zniecierpliwienia i~drwiny zaniepokoiła tłum.

-- Masz siedem -- powiedział Talgarth.

~

Co doradca MacKenzie mówił Wilde'owi, i~co on streszczał Tamarze i~grupie zwolenników brzmiało tak:

Agenci softwarowi, podwykonawcy Niewidzialnej Ręki, na (koniecznie
wolnym) przeczesywaniu ogromnych, nieszyfrowanych publicznych rekordów
Miasta Statku, które z~braku czegokolwiek podobnego do urzędu cierpiały
z niewłaściwego utrzymania, niskiej kompatybilności i~tandetnego
indeksowania, odkryły pojedyncze, intrygujące odniesienie do Jay-Duba i~Eona Talgartha. Nie istniał żaden zapisany kontakt od czasu lądowania,
ale byli w~tych samych drużynach pracy po tamtej stronie Mili Malleya.

-- Czy to zmienia cokolwiek? -- spytała Tamara.

-- Nie wiem -- powiedział Wilde. -- Ale Reid musiał wiedzieć o~tym, tak jak
musiał wiedzieć, że Talgarth dość słabo oceniał moją działalność na
Ziemi.

Ethan Miller się wcisnął. 

-- Powinniśmy odwołać rozprawę, człowieku!
Sędzia jest uprzedzony do Ciebie i~może przeciwko Jay-Dubowi.

-- Nie możemy -- powiedział Wilde. -- Zgodziliśmy się na niego,
powiedziałem publicznie, że ufam jego osądowi, i~nie możemy teraz się
odwrócić i~powiedzieć, że nie wiedzieliśmy.

-- Ale możemy apelować do innego sądu -- wskazała Tamara.

-- Ach -- powiedział Wilde. -- Zatem może Reid\ldots to działa w~obie strony!
Nie wiemy jak Talgarth i~Jay-Dub żyli, kiedy obaj byli robotami, może
byli najlepszymi kumplami, z~tego, co wiemy. -- Wyprostował się,
dochodząc do decyzji. -- Reid może nie wiedzieć, że Jay-Dub nigdy nie
wspomniał o~tym, lub z~tego powodu jest obecnie wyłączony z~łączności z~nami. Więc może trzymać to w~ukryciu jako podstawa do natychmiastowej
apelacji, jeżeli decyzja będzie przeciwko niemu. Jebać to. Będę musiał
mieć to na uwadze. Gramy dalej.

\threeast

Dee słyszy odległy krzyk. Postaci dookoła wieży krzyczą, machają do niej
oraz uciekają. Sama wieża się zmieniła, jej kolczaste gałęzie tworzą
wzór, który jakoś wygląda na nieunikniony i~poprawnie, choć brzydko.

Wzdycha i~wstaje. Teraz będzie musiała rąbać i~ślizgać się całą drogą na
dół wzgórza i~po ciężkiej drodze. Wiedząc, że to jest rzeczywistość
wirtualna, nie rozumie, dlaczego nie może po prostu kurwa
\emph{polecieć}. Wilde powiedział jej o~czymś nazywanym ,,regułami
spójności'', ale nie jest pod wrażeniem. \emph{Nie potrzebuje} fałszywej
spójności, żeby przestać się wściekać.

Niemniej całe to rzucanie przekleństw i~oszczerstw jest redundantne,
ponieważ bez chwili ostrzeżenia jest znowu w~swoim zmęczonym i~bolącym
ciele. Jej głowa boli tak bardzo, że wolałaby tarabanić się po tym
wzgórzu pod wielkim, gorącym słońcem Ziemi. Nad nią, narzędzia i~latarki
kołyszą się z~haków, a~dookoła głęboki elektryczny szum turbin pojazdu
mówi jej, że są w~drodze.

Siada ostrożnie i~zestawia nogi na podłogę. Ax stoi przy zamykającej się
tylnej klapie. Wewnętrzne ekrany rozświetlają się na wszystkich czterech
ścianach przedziału pojazdu, gdy tylne drzwi zatrzaskują się z~westchnięciem hydrauliki i~sykiem uszczelniania. Jadą wprost na kanał,
który przekroczą delikatnym impetem. Bieżniki pojazdu, Dee wie, są
zamontowane na jakiegoś rodzaju rozkładanych nogach, które zmienią
kilkumetrowy spadek w~nie więcej niż wstrząs na drodze.

-- Co się dzieje? -- pyta Dee.

Ax wzrusza ramionami, ale pytanie Dee jest odpowiedziane, gdy przedni
ekran zmienia się na widok ponad ramionami Wilde'a i~Meg. Meg obraca się
dookoła i~uśmiecha, Wilde ciągle patrzy do przodu, ale jego oczy
trafiają na jej w~lusterku. (Reguły spójności, znowu. Szalone, uważa
Dee).

-- Cześć -- mówi. -- Przepraszam za nagłe wyjście. Możesz wrócić do naszego
miejsca z~Meg, jeżeli chcesz, ale teraz muszę zostać w~rzeczywistości. -- Roześmiał się. -- Przynajmniej w~zakresie wyglądania za okna i~prowadzenia pojazdu.

W rzeczywistości, Jay-Dub jest usadowiony we wgłębieniu na przodzie
pojazdu i~był tam od momentu przyjazdu. Ciężarówka jest całkowicie
zdolna sama się kierować. Dee podejrzewa przebiegle, że ta konieczność
kontrolowania ich postępów jest częściowo czysto psychologiczna, na
płytszym poziomie niż te wbudowane reguły spójności. Przyjmuje
wyjaśnienia bez dodatkowych pytań.

-- Gdzie jedziemy? -- pyta.

-- Musimy wrócić do Miasta Statku -- mówi jej mężczyzna.

-- Problem na rozprawie? -- domyśla się Dee. Nie skupia pełnej uwagi na
rozmowie. Bada swój umysł, sprawdzając Jaźnie, gdy jak zbłąkane dzieci
wracają do domu i~czuje ulgę, gdy są tam wszystkie. Sekrety są mniejsze,
Sklep znacznie większy niż kiedy go ściągała do Jay-Duba, ale wszystko w~porządku, ma miejsce w~głowie do wykorzystania.

-- Och nie -- krzyczy Wilde do tyłu, jego oczy przeskakują z~lusterka na
pustynię. Dee widzi, że pojazd porusza się z~prawie maksymalną
prędkością. -- Musimy znaleźć tę pewną truciznę i~wtedy\ldots

Jego głos urywa się, albo przez odkrywkę przez którą (ona łapie krawędź
ławy) zamierzają przejechać, albo ponieważ nie wie, co powiedzieć.

-- Potem co?

Oczy Wilde'a marszczą się w~uśmiechu, patrzą do tyłu na nią.

-- Będziemy hakować drzwi piekła.

Nawet nie kłopocze się pytaniem o~dalsze wyjaśnienia. Jest oczywiste, że
żadne nie nadejdzie, i~musi założyć, że istnieje dobry powód, dlaczego
nie. Wilde kiwa jej zachęcająco i~potem skupia uwagę na płaskiej
pustyni i~na Meg. Ax położył się pod starym kocem foliowym, koło
wyjścia antenowego i~ma wizje przez telewizję.

Dee kieruje Naukowczynię do pracy i~wchodzi do Sys. Minuty mijają.
Potem, jakby z~wielkiej, zimnej wysokości, góry wyższej niż jakakolwiek
na Ziemi lub Marsie, w~surowej wirtualnej próżni, która sprawia, że
czuje jakby głowa miała krwawo eksplodować, Dee rozumie precyzyjnie, co
tajemnicze zdania Wilde'a oznaczają.

\threeast

-- Ty pierwszy -- powiedziała Tamara. Inni rozproszyli się po siedzeniach
i Wilde podszedł do mikrofonu. Talgarth zgasił papierosa, nad paleniem,
którego spędził siedem minut, i~skinął głową.

Wilde przeszedł przez te same uprzejmości, które powiedział Reid, i~powiedział:

-- Szacowny Starszy, jestem bardziej niż chętny, by odpowiedzieć za swoje
czyny, i~za te podjęte w~moim imieniu. Nie jestem chętny odpowiadać za
czyny robota Jay-Duba lub zaakceptować zarzuty, że jest on moją
własnością. Moja obecna fizyczna egzystencja rozpoczęła się zeszłego
Pięciodnia, około południa, kiedy zostałem wskrzeszony. Robot Jay-Dub
twierdził, że dokonał tego środkami, których nawet nie próbuję
zrozumieć\ldots

Reid zerwał się z~krzesła.

-- Sprzeciw! -- powiedział. -- Bez znaczenia.

-- Podtrzymany -- powiedział Talgarth.

Wilde przełknął. 

-- Bardzo dobrze, Szacowny Starszy. Można tego dowieść
niezależnie poprzez wezwanie zapisów transakcji Jay-Duba z~Bankiem
Spółdzielczym Stras Cobol, które jestem szczęśliwy dostarczyć Sądowi, o~ile są istotne. Ustalają one rzeczywiście, że właścicielem Jonathan jest
Jonathan Wilde. I~identyfikują, kto, dokładnie, tym Jonathanem Wilde'm
jest. Najwcześniejsze zapisy dotyczą transakcji z~przedsiębiorstwem
Davida Reida, Wzajemnie Gwarantowana Ochrona. Wyraźnie akceptują nazwę
,,Jay-Dub'' jako synonim Jonathana Wilde'a oraz Robota Jay-Duba jako
równoważnego tej osobie, Wilde'owi. Robot Jonathan został zaakceptowany
bez sprzeciwu przez wiele lat jako nikt inny niż Jonathan Wilde,
Jay-Dub, w~skrócie, \emph{jest} Jonathanem Wilde'em! Dowolne zapisy
wspominające Wilde'a jako właściciela Robota Jay-Duba, wobec tego, mogą
być tylko interpretowane jako znaczenie, że osoba Jonathan Wilde posiada
Jay-Duba w~ten sam sposób, w~jaki ja, Jonathan Wilde, posiadam moje
ciało. -- Uśmiechnął się słabo. -- Zbieżność nazw jest godna pożałowania.

Eon Talgarth, siedząc w~krześle na podium, był na poziomie oczu
stojącego Wilde'a. Ich oczy spotkały się na chwilę.

-- Sąd zadecyduje w~tym przedmiocie -- powiedział Talgarth. -- Robot znany
jako Jay-Dub jest w~unikalnej pozycji pośród wszystkich mieszkańców tej
koloni, o~ile wiem. Jednakże, jest to pozycja, w~której wielu
wymienionych mieszkańców kiedyś było, a~w~której tylko on pozostaje.
Akceptuję argument, który właśnie został przedstawiony i~zasądzam, że
pozwy przeciwko Jonathanowi Wilde'owi w~zakresie właściciela Robota
Jay-Duba winny być przedstawione przeciwko temu robotowi, jako
samoposiadającego się mechanizmu. -- Rozejrzał się. -- Robot nie jest
obecny w~tym sądzie i~winien być bezzwłocznie powiadomiony. Zarzuty
przeciwko \emph{temu} Jonathanowi Wilde'owi pozostają w~toku.

Reid zaczął wstawać, spoglądając wściekle, ale kobieta siedząca koło
niego złapała go za ramię i~pociągnęła do tyłu. Po rozmowie twarzą w~twarz, Reid odstąpił.

-- Moja decyzja nie niesie precedensu istotnego dla pytania o~osobowość
maszyny jako takiej -- kontynuował Talgarth. -- Kwestia własności Dee
Model nadal pozostaje do rozważenia. Niezależnie od tego, czy jej
systemy sterowania zostały uszkodzone, i~kto, jeżeli ktokolwiek, jest
odpowiedzialny za to, twierdzenie Reida, że jej nie porzucił, nie jest
zakwestionowane. Dlatego pozostaje jej właścicielem, a~obecnym po
drugiej stronie sprawy nakazuję współpracować w~jej aresztowaniu i~powrocie.

Tamara wstała, otrzymała błysk pozwolenia do mówienia i~powiedziała: 

-- Starszy Talgarth, ten sąd wielokrotnie orzekł, że autonomia maszyn może
być stwierdzona przez same te maszyny. To, a~nie kwestia porzucenia, co
do której dobrowolnie przyznaję, że byłam w~błędzie, jest podstawą, na
której chcielibyśmy dowieść samo-posiadania Dee Modelu.

Talgarth westchnął. 

-- Wszystkie takie sprawy -- powiedział cierpliwie -- odnosiły się do bezpańskich maszyn rozumnych w~Dzielnicy Maszyn. Wolność
takich automatów jest również bez zastrzeżeń rozpoznawana przez inne
sądy. Gynoid, o~którym mowa, jednakże, został skonstruowany przy pomocy
zasobów i~wysiłków Davida Reida, i~pozostaje jego własnością, póki nie
zdecyduje inaczej.

Tamara usiadła i~rzuciła Wilde'owi grymas żalu lub przeprosin. Wilde,
jednakże, wydawał się patrzeć przez nią. Mrugnął, uśmiechnął się do niej
i wstał. Podszedł do mikrofonu i~rozejrzał się po tłumie, zanim zwrócił
się do sędziego.

-- Szacowny Starszy, Twoja wartościowa opinia w~kwestii Jay-Duba i~Dee
Model podnosi pewne dalsze tematy, które błagam Sąd o~rozważenie.
Pierwsze, w~sprawie Jonathana Wilde'a w~jego wcieleniu jako Jay-Duba.
Sąd zaakceptował, że on i~ja jesteśmy oddzielnymi osobami, choć, przez
implikację, współdzielenie wspólnej historii aż do punktu, w~którym sąd
odmówił określenia\ldots

-- Jak? -- Talgarth zmarszczył brwi.

-- Kiedy podtrzymaliście sprzeciw, że czas mojego wskrzeszenia był
nieistotny.

Talgarth oparł się. 

-- To prawda.

-- Jako oddzielne wcielenie Jonathana Wilde'a, chciałbym postępować
przeciwko Davidowi Reidowi w~sprawie zarzutu bezprawnego zabicia mnie,
na podstawie, że dowolne przemyślenia lub przyznania, które zostały
zrealizowane pomiędzy Reidem a~Jonathanem Wilde vel Jay-Dub nie mają
związku ze mną.

-- Odłożę namysł nad tym, póki czas Twojego wskrzeszenia zostanie
określony satysfakcjonująco -- powiedział Talgarth. -- Zarzut morderstwa,
który przedstawiłeś przeciwko Reidowi, pozostaje nie rozwiązany, aż do
momentu rozstrzygnięcia lub niezakwestionowania. Davidzie Reid, co na to
powiesz?

Reid wstał, gardząc podejściem. 

-- Wysoki Sądzie -- powiedział głośno -- jestem całkiem skłonny do zaakceptowania twierdzenia tej osoby, że
została wskrzeszona przez robota Jay-Duba trzy dni temu. Jako przedmiot
naturalnej sprawiedliwości, chciałbym przy najbliższej możliwości
oczyścić się z~zarzutu morderstwa lub wyrzucić z~sądu jako stratę
cennego czasu i~przypadek dokuczliwego sporu sądowego. -- Spojrzał na
Wilde'a i~usiadł.

-- Doskonale -- powiedział Talgarth. Odwrócił się Wilde'a. -- Zanim
przejdziemy do rozważania tego zarzutu, czy masz cokolwiek więcej do
powiedzenia o~kwestiach podniesionych moją opinią w~zakresie Dee Modelu?

-- W~istocie mam -- powiedział Wilde. -- Sąd wspomniał, że gynoid Dee Model
został skonstruowany z~zasobów i~wysiłków hm, drugiej strony. Chciałbym
podnieść pytanie o~własność tych właśnie zasobów. Ponieważ ciało Dee
Model jest klonem ciała mojej zmarłej żony. Jest to oczywiste dla mnie i~wzywam Reida do zaprzeczenia temu.

Przerwał i~odwrócił się twarzą do Reida. Odpowiedź Reida była drżeniem
powiek i~pokręceniem głowy.

-- Nie zaprzeczasz temu? -- spytał Talgarth.

Reid wstał. 

-- Nie.

Wilde spojrzał na Reida z~triumfem i~nienawiścią, potem przyjął minę, by
rzucić łagodny uśmiech kamerom, gdy odwrócił się do Talgartha.

-- W~takim przypadku -- powiedział Wilde powoli i~wyraźnie -- żądam, by
ciało Dee Modelu należało do prawowitego spadkobiercy mojej żony! -- Uśmiechnął się do Talgartha. -- Czy tym spadkobiercą jestem ja, czy
Jay-Dub, pozostawiam Wysokiemu Sądowi do przesądzenia.

Reid wstał raz jeszcze i~ukłonił się uprzejmie, choć nie było oczywiste,
czy do Wilde'a, czy do Talgartha.

-- Cieszę się, że mogą przyznać prawo własności genotypu -- powiedział. -- Oraz dojść do przyjaznego lub, jeżeli się nie uda, sądowego porozumienia
w sprawie jego użycia, lub odszkodowania za jego użyciu oraz kłopoty
nieumyślnie spowodowane. Moją główną troską jest odzyskanie
oprogramowania gynoida i~niebiologicznego hardwaru, które są
bezsprzecznie moją własnością.

Wilde spojrzał na Tamarę, która wzruszyła ramionami i~uniósł brwi, jakby
mówiąc ,,O co mu chodzi?''. Zdalny MacKenzie mówił w~zasadzie to samo.
Oczekiwał większej walki, ponieważ własność genotypów było gorąco
kwestionowaną sprawą. Jego jedyna sugestia była taka, że dowolne
ustępstwo tutaj uniknęłoby ustanowienia precedensu, które inne sądy
mogłyby uznać.

-- Bardzo dobrze -- powiedział Wilde. Poprawił mikrofon, jego dłoń lekko
drżąca. -- Jedynym odszkodowanie, jakie pragnę, jest, aby David Reid
wskrzesił umysł mojej żony jak również jej ciało, coś, co jest
najwidoczniej możliwe, co Robot Jay-Dub pokazał poprzez wskrzeszenie
mojej osoby.

Reid od razu był na nogach. Wilde musiał się szybko cofnąć, gdy Reid
podszedł i~wyrwał mikrofon z~jego ręki.

-- Sąd nie zaakceptował, że Jay-Dub wskrzesił tego człowieka!

Talgarth strzepnął popiół z~rękawa. 

-- Ach, ale \emph{Ty} tak -- powiedział miękko.

Reid znowu usiadł. Kobieta koło niego szeptała mu do ucha, jej twarz
zimna z~irytacji. Zdalne wiadomości brzęczały, a~ludzie w~tłumie
sprawdzali komentarze na żywo na ręcznych ekranach lub na swoich
kontaktach.

-- Spokój! -- Talgarth uderzył młotkiem, najpierw ostrożnie przytrzymując
napój. -- David Reid może odpowiedzieć na Twoje żądanie w~swoim własnym
czasie.

-- Odpowiem teraz -- powiedział Reid. Wilde odsunął się od mikrofonu i~wrócił na miejsce.

-- Trochę wstrząsnąłeś rzeczami -- zauważyła Tamara.

Wilde mrugnął, na chwilę myląc zdalnego doradcę, i~rozparł się, by
wysłuchać Reida.

-- Żądanie Wilde'a jest rozsądne -- mówił Reid. -- Pytanie o~wskrzeszenie
Nieożywionych długo było w~głowach nas wszystkich. Jednak, choć bardzo
chcielibyśmy to zrobić, \emph{force majeure} przeszkadza nam. Większość
osobowości Martwych, włączając w~to żonę Wilde'a Annette, jest trzymana
w magazynie inteligentnej materii, który pozostaje niedostępny bez
współpracy postludzkich bytów, których możliwości i~motywy są nieznane,
ale które, jak pokazuje doświadczenie, są ryzykiem dla nas wszystkich.
Jestem odpowiedzialny za pilnowanie Kodów, które mogłyby być użyte do
ich uruchomienia i~mogę zapewnić Sąd, że póki ktoś nie przedstawi
sposobu wykonania tego bezpiecznie, te kody pozostaną w~moim władaniu, a~Nieożywieni\ldots we śnie. -- Spojrzał na Wilde'a. -- Niektóre rzeczy
najlepiej pozostawić niezmącone -- powiedział do niego.

-- Mówi Ci, żebyś nie naciskał -- wymamrotała Tamara.

Wilde uśmiechnął się do niej i~podszedł do przodu, gdy Reid zajął swoje
miejsce. Napięcie w~tłumie się obniżyło. Nawet beznamiętna twarz
Talgartha zdradzała ulgę.

-- Robot Jay-Dub wskrzesił mnie bez katastrofy -- powiedział. -- Ale jest
więcej w~tej sprawie niż to.

Reid odchylił się na krześle, ręce za głową i~obserwował Wilde'a zza
półprzymkniętych oczu.

-- Sąd przedstawił swoje stanowisko w~jednym z~zarzutów Reida -- powiedział Wilde -- a~pozostawił inne w~staniu zawieszenia, póki inny
Jonathan Wilde, aka Jay-Dub, może być\ldots skłoniony do odpowiadania.
Teraz pragnę naciskać na mój kontrzarzut, wynik, którego może wpłynąć na
sposób, w~jaki kary i~szkody w~tych kwestiach będą rozłożone. Może
również wpłynąć na pytanie wskrzeszenia Nieożywionych, ogólnie rzecz
biorąc. -- Uśmiechnął się do Talgartha, który już nie wydawał się
uspokojony. -- Nie w~sensie prawnym, w~tym, mam wzgląd na Sąd, ale w~sensie praktycznym.

Wilde przesunął się nieco na bok tak, że choć bezsprzecznie i~poprawnie
zwracał się do Sądu, mówił również do Reida i~szerszej widowni.

-- Mój kontrzarzut jest taki: że David Reid zabił mnie bezprawnie, przez
lekkomyślne działanie ludzi działających w~jego imieniu oraz przez jego
osobiste, zamierzone zaniedbanie moich ran. To zrobiwszy, nie poczynił
kroków w~dobrze wierze do wskrzeszenia mnie. Twierdzi, że jest to
trudne, mimo to, nie istnieją dowody żadnej próby po jego stronie, by
przezwyciężyć problem. Żądam odszkodowania za utratę doświadczenia życia
i utratę społeczeństwa za mój cały czas niedostępności. To jest, za nie
mniej niż cały Czas Statku, oraz możliwie, że więcej.

Eon Talgarth musiał wezwać do spokoju, więcej niż raz, zanim gwar
ucichł.

-- Czy masz dowody w~tej sprawie? -- spytał.

-- Tak -- powiedział Wilde.

Przeszedł do swojego miejsca, sięgnął do plecaka Tamary i~wyciągnął
teczkę notatek Talgartha. Podniósł to wysoko, gdy wracał, i~przedstawił
je Talgarthowi.

-- Dowód -- powiedział -- został zebrany przez niejakiego Eona Talgartha i~był przedmiotem rekordów publicznych i~nigdy nie został zakwestionowany.

Sąd zamilkł, prócz małego brzęczącego helikoptera i~odległego hałasu
maszynerii na zewnątrz.

Talgarth przekartkował strony i~pokręcił głową. 

-- Zarzuty są sprzeczne -- powiedział -- co do sprawy, w~której Jonathan Wilde napotkał swoją
śmierć. Choć sam jest skłonny do opinii, którą właśnie przedstawiłeś,
nie ma innych żyjących świadków innych niż David Reid i,
przypuszczalnie, Ty sam. To, że nigdy nie zostało zakwestionowane,
obawiam się, nie ma znaczenia dla sprawy. Żaden sąd na tej planecie nie
uzna paszkwilu i~też nie uzna odrzucenia lub porażki w~obaleniu
roszczenia za dowód na jego korzyść.

Westchnął, jakby w~żalu za bardziej niż nieadekwatnością dowodu, za,
może, dawno spędzoną pasją polityczną, która prowadziła go do zebrania
dossier. Oddał teczkę Wilde'owi.

-- Sąd nie może zaakceptować tego jako dowodu -- powiedział. -- Z~braku
innych dowodów lub wyznania tego, którego oskarżyłeś\ldots

Spojrzał na Reid, który energicznie potrząsał głową.

-- \ldots które, rozumiem, nie nadejdzie, a~którego nie mam władzy zmusić,
nie rozumiem, jak ten zarzut może być rozstrzygniętym w~tym momencie.
Jeżeli wezwiesz Reid jako świadka, może odmówić odpowiedzi, i~z tego nie
może zostać wyciągnięta żadna niekorzystna konkluzja.

Prawny doradca Reida wstał i~krótko rozmawiał z~Talgarthem, podczas gdy
Wilde wycofał się z~zasięgu słuchu i~patrzył w~bok. Kiedy kobieta znowu
usiadła, Talgarth uderzył młotkiem.

-- Kontrzarzut jest oddalony -- powiedział -- bez uprzedzenia do żadnej ze
stron. Przedstawienie zarzutu przez Wilde'a nie może być nazwane
dokuczliwym lub lekkomyślnym i~nie może być wykorzystane przeciwko
niemu. Imię i~reputacja Davida Reid pozostaje nieskalane. Twierdzenie,
że jego zabicie Wilde było bezprawne lub złośliwe, pozostaje takie jak
przed przedstawieniem zarzutu, to jest, nieuzasadnioną spekulacją
historyczną, wobec której jest w~prawie do traktowania go jako obrazy.

Reid i~jego asystentka wymienili uśmiechy.

-- Jednakże -- kontynuował Talgarth z~nagłą szorstkością w~głosie -- roszczenie, że Reid był odpowiedzialny, karygodnie lub nie, za śmierć
Jonathana Wilde'a jest\ldots znacznie lepiej poświadczone. Świadkowie nie
są, oczywiście, w~tym sądzie, ale niektórzy przetrwali zostać poproszeni
o zeznania.

Skinął na doradcę Reida i~po kolejnej naradzie, uderzył młotkiem.

-- Reid nie kwestionuje swojej odpowiedzialności za fakt śmierci Wilde'a. Możesz kontynuować.

\threeast

-- Ax?

Brak odpowiedzi. Ax ogląda telewizję w~głowie lub przed oczami, lub
cokolwiek, do cholery robi. Dee nie może znieść ani sekundy dłużej jego
autystycznego, lecz słyszalnego zainteresowania. Pochyla się i~trzęsie
jego ramieniem. Budzi się i~marszczy brwi na nią.

-- Co\ldots?

-- Ax -- mówi cierpliwie -- czy mógłbyś \emph{przesłać }ten fascynujący
materiał na \emph{ekran}, żebym też mogła go zobaczyć?

-- Och. Przepraszam, Dee.

Rozłącza się od łącza korowego i~porusza przełącznikami. Na zewnątrz, na
wielkich ekranach, peryferie Piątej Dzielnicy powoli się przesuwają. Dee
obserwuje chaotyczną aktywność z~pogardliwą konsternacją. Jeżeli tak
maszyny się zachowują, kiedy są puszczone wolne, zastanawia się, to nic
dziwnego, że ludzie im nie ufają.

Dookoła pojazdu, który przedziera się szeroką ulicą, tuziny innych
maszyn, każda wysokości około trzydziestu centymetrów, biegają i~wąchają. Wyglądają jak większe wersje botów czyszczących, które znajdują
się w~domach, i~choć częściowo samodzielne, są kierowane przez kontrolę
radiową z~kabiny. Meg powiedziała jej, że szukają śladów określonej
trucizny: jednego ze środków zaradczych zdrowia publicznego, którym to
miejsce jest okresowo bombardowane. Trucizna, ogólnie znana jako
Niebieski Glut, jest nanotechnologicznym równoważnikiem wirusów,
regularnie aktualizowanych i~mutowanych, by utrzymać tempo z~podobnie
ewoluującym dzikim życiem inteligentnej materii Dzielnicy Maszyn. Praca
opryskiwania ich z~powietrza jest wykonywana przez fundację
charytatywną, która w~ogóle nie ma kłopotu w~zbiórce pieniędzy lub
ochotników.

Ax wskazuje jej, żeby spojrzała do tyłu. Część ekranów, do których się
zwraca, jest zakryta, gdy kolejne okno się włącza. To serwis Prawnych
Kanałów, pokazujący sprawę w~sądzie. Wilde -- lub Jay-Dub, jak Dee nazywa
go mentalnie -- i~Meg mieli na to oko, kiedy mogli poświęcić chwilę. Ax
otrzymał zadanie pilnowania \emph{szczegółów}. Dee czuła się pominięta i~zastanawia się, czy inni próbowali jej oszczędzić emocji. Miłe, ale
strata czasu.

Ponieważ, jakiekolwiek złe wiadomości sprawa sądowa może jej przynieść,
to wszystko jest teraz nieistotne. Jak Ax powiedział, to gówno się
\emph{skończyło}.

Wilde najwidoczniej właśnie skończył mówić. Odwraca się do Sędziego,
Eona Talgartha. Nawet Dee słyszała o~Talgarthie, byłym kryminalistą z~obozu orbitalnego Mili Malleya, który studiował prawo jako więzień,
włączył się, potem rozczarował abolicjonizmem i~przez lata utrzymywał
się z~orzekania w~sporach pomiędzy złomiarzami i~maszynami.

Gdy Wilde odwrócił się, kamera podążyła za jego twarzą, a~on uśmiecha
się powoli, arogancko.

-- Cóż, to była przemowa! -- mówi zasapany komentator. -- Wyglądał na dość
zdenerwowanego, kiedy opisywał swoje zabójstwo, powinienem powiedzieć
jego \emph{domniemane} zabójstwo. Sorry! I~nikt wcześniej nie
zasugerował, że może jesteśmy winni Martwym ich zaległe wypłaty! W~sprawie implikacji tego, proszę zobaczyć\ldots

Ax przerywa ten szczególny wątek i~wszystko, co teraz słyszy Dee, jest
ciszą w~sądzie, gdy Reid kroczy do mikrofonu. Jego twarz sprawia, że ona
drży. Rzadko kiedy widziała go rozzłoszczonego i~nigdy na nią, ale wie,
że jego gniewu należy się obawiać, i~właśnie teraz jest rozgniewany na
cały świat.

Kamera krąży dookoła zza Talgartha. Reid jest teraz bardziej opanowany i~Dee czuje się proporcjonalnie spokojna, w~istocie, gdy patrzy na
zbliżenie, czuje poruszenie bezwiednej tkliwości i~pożądania. Jest to
jeszcze bardziej niepokojące, przez to, że czuje to jako osoba, a~nie
niewolnik, ale odkłada to na karb jej przeszłości i~koncentruje się na
tym, co człowiek ma do powiedzenia.

-- Starszy Talgarth -- mówi ciężko -- to, co właśnie usłyszeliśmy, jest
hańbą dla tego Sądu i~obrazą dla inteligencji nas wszystkich. Jest to
również niebezpieczne, wzniecanie oportunistycznej zazdrości, która nie
ma miejsca w~praktycznie sprawiedliwym społeczeństwie takim jak nasze,
gdzie żadna osoba nie jest zredukowana do sprzedawania swojego życia lub
pracy tym bardziej pomyślnym.

-- Sprzeciw! -- dobiegł krzyk od Wilde'a.

-- Podtrzymany -- mówi surowo Talgarth. -- Nie jesteśmy tutaj na forum
publicznym.

Reid pochyla głowę. (Dee słyszy Axa, za sobą, prychającego).

-- Mam na myśli -- kontynuuje Reid -- to, że mój przeciwnik stwierdził, że
Ci zainteresowani Nieożywionymi mają wobec mnie roszczenie, ponieważ nie
podjąłem żadnych prób w~dobrej wierze, jak to ujął, by rozwiązać ogromne
zadanie znalezienia sposobu wskrzeszenia zmagazynowanych zmarłych. Cóż,
Szacowny Starszy, dobrzy ludzie, jest to zadanie, które dobrowolnie
przyznaję, że jest poza moimi możliwościami! -- Rozkłada ręce i~wzrusza
ramionami. -- Czy kiedykolwiek zapobiegłem przedstawieniu propozycji
rozwiązania przez kogokolwiek? Nie! Ponieważ, jak wszyscy wiemy,
prawdziwym problem jest odkrycie sposobu powstrzymania tych, których
pomocy potrzebujemy, żeby wskrzesić Nieożywionych. Szybki Ludek, ci,
którzy kiedyś byli ludźmi, a~których umysły i~motywy, rozwinęły się
daleko poza ludzkie zrozumienie lub kontrolę. \emph{Oni} są tymi,
których mógłbym przebudzić, gdybym chciał. \emph{Oni} są tymi, którzy
mogliby przebudzić ludzkich martwych, którzy śpią w~tym samym magazynie
danych co oni. I~\emph{oni} są tymi, którzy mogliby, w~mgnieniu oka,
zamienić tę planetę w~rodzaj piekła, które niektórzy z~nas ujrzeli w~przelocie, sto nasze długie lata temu.

Jego wzrok skupia się na Eonie Talgarthie, a~Dee czuje tylko strumień
jego pasjonującej prośby: 

-- Szacowny Starszy! Wiem, że \emph{Wasza}
pamięć nie jest krótka! Odrzuć to roszczenie, póki nie uczyni więcej
zła!

Rozgląda się jeszcze raz dookoła i~siada.

Talgarth pije z~szklanki i~zapala papierosa. Kontempluje dym przez kilka
minut, potem pochyla się do przodu, łokcie na kolanach. Jego postawa
dziwnie kontrastuje z~formalnością stroju, i, jakby zauważając to,
zdejmuje kapelusz.

-- Znaczy, że mówi poza protokołem -- wyjaśnia Ax.

-- Ale możemy go słyszeć! -- mówi Dee.

-- Figura retoryczna -- mówi Jay-Dub, z~wirtualnej kabiny z~przodu. -- Cśśś.

Dee, w~pewien sposób skarcona, odwraca na chwilę wzrok i~zauważa, że
ciągnik stoi jałowo na końcu szerokiej ulicy. Mniejsze maszyny wróciły,
albo w~porażce, albo z~sukcesem, tego nie wie. Przed nimi jest trawiasty
park z~jakimiś fortyfikacjami w~centrum. Ponad nimi wykrywa chmurę
komaropodobnych latających maszyn.

-- Och Reid -- Talgarth mówi -- zawsze byłeś świetnym mówcą i~rozumiem, co
chcesz powiedzieć. Niemniej jednak pomiędzy Tobą i~mną, jeżeli łapiesz,
Wilde przedstawił rozsądne stanowisko co do tego, jak moglibyśmy to
zrobić poza planetę, w~kosmosie, jakby, a~Ty nie odpowiedziałeś na to,
prawda?

Reid unosi pojednawczo dłoń do Talgartha, który opiera się i~wkłada swój
kapelusz sędziego. Potem Reid odwraca się do sztywno ubranej kobiety
koło niego i~odbywa mamroczącą konsultację, z~której kamera, zgodnie z~wymaganiami, się odcina. Zbliża się do Wilde, który siedzi z\ldots

-- Tamara! -- wołają radośnie Dee i~Ax.

-- Dobrze dla niej -- mówi Jay-Dub.

Z powrotem na Reida, który właśnie gniewnie strząsa dłoń kobiety i~idzie
ku kamerze i~mikrofonowi, ścigany tylko przez otwarte usta
skonsternowanej kobiety.

-- Nie chciałem, żeby do tego doszło -- mówi Reid, cała konwencjonalna
uprzejmość wyrzucona, gdy mówi do świata i~do sądu jedynie po namyśle. -- Ale dość znaczy dość. Pewnie, ,,my'' moglibyśmy to zrobić w~kosmosie!
Powiedz mi, kto to ,,my''? Jeżeli ktokolwiek ma kapitał do wydania na
stację kosmiczną \emph{i} pierścień dział laserowych zabezpieczonych
przez dowolnymi programami wirusowymi, które mogłyby wkraść się w~ich
sterowanie, \emph{oraz} opracowaną niezawodną procedurę \emph{i}
superczułe nuklearne zabezpieczenie na wszelki wypadek na miejscu, może
już startować. Proszę bardzo! \emph{Sprzedam} Ci pierdolonych martwych i~demony, które mogą z~nich się podnieść. Dawaj! Załatw kolejne pęknięcie
w immanentyzacji Eschatonu\footnote{immantezycja od immanencja filoz.
pozostawanie wewnątrz czegoś lub niezależność bytowa od czynnika
zewnętrznego, eschaton -- koniec świata
\url{https://pl.wikipedia.org/wiki/Eschaton}. Tu możliwe, że mowa o~cyt.
,,cele ostateczne człowieka i~świata można zrealizować za pomocą środków
politycznych'',
por.~\url{https://www.miesiecznik.znak.com.pl/6612010damian-leszczynskimiedzy-bogiem-a-cesarzem/}
lub cyt. ,,uznanie że człowiek może się nie tylko samodoskonalić, ale
także wprowadzić w~świecie doczesnym rzeczywistość ostateczną''
por.~\url{https://teologiapolityczna.pl/maciej-jesionkiewicz-voegelin-i-gnoza-platonik-w-zsekularyzowanym-swiecie-1}
 -- przyp.tłum.}!

-- Zanim jacyś przedsiębiorcy apokalipsy popędzą naprzód, jednakże,
pozwólcie mi dać wam ostrzeżenie.

Odwraca się i~wskazuje drżącym palcem na Wilde'a, który obserwuje
przedstawienie Reida z~wyrazem zuchwałego oderwania.

-- \emph{Nie} przyjmujcie sugestii od tej\ldots \emph{rzeczy}, która nazywa
siebie Jonathanem Wildem! Ta rzecz, która przyznaje, że jest stworzeniem
robota Jay-Duba!

Zatrzymuje się, bierze głęboki wdech i~odwraca się do Talgartha. 

-- Szacowny Starszy, ponoszę dużą odpowiedzialność przed ludem Nowego
Marsa. Pozwoliłem Robotowi Jay-Dubowi kontynuować istnienie, po tym, gdy
miałem podstawy podejrzewać, że został uszkodzony przez oryginalny
Szybki Ludek w~Mili Malleya. Wielokrotnie, osobiście i~przez swojego
golema tutaj, i~z tego, co wiem, poprzez manipulację przez lata ruchem
tak zwanych abolicjonistów, wzywał nas do katastrofalnego kursu
uruchomienia Szybkiego Ludku. Czyim interesom, pytam Was, by to służyło?

Talgarth nie odpowiada.

Reid, jak gdyby nagle obrzydzony całym tym biznesem, potrząsa do tyłu
ramieniem nad głową i~idzie do swojego krzesła. Jednak nie siada. Jego
zwolennicy wstają razem z~nim i~inni w~tłumie też wstają.

Reid sięga do kurtki i~nagle pojawia się nagły szał przemieszczania, gdy
tłum się dzieli, niektórzy uciekają od konfrontacji, inni zbliżają się
do jednej lub drugiej strony. Tamara i~niektórzy ludzie, który Dee nie
zna, ale Ax -- na podstawie jego gorliwych komentarzy -- tak tworzą zaporę
dookoła Wilde'a. Kamery podskakują, fakcje stają twarzą w~twarz z~bronią
w ręku.

Talgarth mówi nagląco w~prawą klapę i~wykonuje równie naglące gesty. Dee
zauważa, że bronie na żelaznych ścianach palisady obracają się na
podstawach, kręcą się dookoła i~kierują się do środka oraz w~dół.

Jedna z~unoszących się kamer nagle się odwraca i~robi zbliżenie na
bramę, która niezauważenie została otwarta. Zaokrąglony przód wielkiego
opancerzonego pojazdu wjeżdża. Dee odwraca wzrok od ekranów
telewizyjnych na ekrany okienne i~widzi inny kąt tego samego widoku.
Wpraszający się pojazd to ich własny.

\chapter{Obywatel Zima}


Obudziłem się od hałasu wojsk pancernych na ulicach i~przez chwilę
leżałem na plecach, patrząc przez heksagonalne szyby kopuły na blade
zimne niebo. Była godzina dziesiąta. Spałbym, ale ANR, jak zwykle,
przybyło na czas. Po wczorajszej wyczerpującej rundzie wywiadów
telewizyjnych i~wizyt -- realnych i~wirtualnych -- w~jednostkach milicji,
czułem, że mam prawo do odpoczynku. Już więcej nie byłem odpowiedzialny
za bycie nominalnym dyktatorem Norlonto, zrezygnowałem jako
przewodniczący Komitetu Łączności Obrony, gdy tylko ostatni dowódca
przeszedł na stronę.

Sterowiec unosił się powyżej, jego kształt zniekształcony przez fale na
szkle. Potem kolejny, i~kolejny, blisko siebie. Zastanawiałem się, czy
wiele osób uciekało, zanim państwo wejdzie. Bez wątpienia byli tacy,
którzy nie chcieli zostawać na przesłuchania: buntownicy hanowerscy,
skutki uboczne wojny domowe, dezerterzy armii\ldots może nawet idealiści
libertariańscy ruchu kosmicznego, w~drodze do złapania miejsca w~pojeździe, zanim ziemski właz wyjściowy zamknie się kompletnie, jak
niektórzy alarmiści myśleli, że się zdarzy. A teraz, po dwudziestu
latach jako mieszkaniec działającej anarchii, znowu byłem obywatelem.
Czołgi i~transportery kontynuowały przetaczanie na zewnątrz, sterowce i~helikoptery dryfowały lub brzęczały ponad nimi. Annette wymamrotała i~poruszyła się koło mnie. Przesunąłem palce przez jej długie białe włosy
i wysunąłem pod kołdrę, prędko owinąłem się jej futrzanym płaszczem i~poszedłem do drabiny z~naszego gniazdka pod szczytem kopuły.

Wydrukowałem gazety, odpaliłem dzbanek kawy i~poszedłem do drzwi. Nasz
klaster kopuł spółdzielni mieszkaniowej był trochę cofnięty od ulicy,
pośród ścieżek, stawów, trawników i~ogrodów cannabis. Dzieci biegały,
kurczaki puszyły się na wybiegach. Tylko psy ciągle przejmowały się
reagowaniem na przejście Armii.

Czołgi, jak zawsze, poruszały się szybciej i~ciszej niż byś oczekiwał.
Żołnierze siedzący na nich nosili mundury ANR ozdobione bandanami,
bandolierami i~insygniami sił, z~których uciekli, lub które pokonali.
Żuli, palili, i~patrzyli z~góry na nas, nieharmonijny rock ryczał z~głośników. Stałem długo, drżąc, kostki kłuły, i~patrzyłem.

Potem zgarbiłem się i~podniosłem naszą dostawę: sok, mleko, jaja, chleb
i bułki. Torby i~kartony były pokryte szronem, musiały być tutaj przez
godziny. Niewielu złodziejaszków w~Norlonto. Zastanawiałem się, jak
długo to potrwa. Gdy smażyłem jaja i~bekon, odrywałem strony z~gazet,
wpadł mi w~oko rachunek z~supermarketu. W~naszym podziale prac domowych,
zakupy należały do Annette. Cena kawy i~papierosów zaszokowała mnie,
cena lokalnej żywności trochę mnie pocieszyła. Sprawdziłem rachunek
dostawy.

Owocowy sok kosztował około dziesięć razy więcej od mleka. Nic
powiązanego z~inflacją -- ta tylko stosowała się do oficjalnej, śmiesznej
waluty Republiki -- a~my płaciliśmy w~dobrym południowoafrykańskim
złocie.

Szalone ceny. Do czego zmierzał ten świat?

Oto byłem ja, myśląc jak starzec. Potrząsnąłem głową i~zaniosłem Annette
na górę śniadanie i~zwitek jej ulubionych gazet. Potem umyłem się,
ubrałem i~zasiadłem do mojego własnego śniadania i~wiadomości, próbując
to zrozumieć.

Byłem przy drugiej kawie i~pierwszym papierosie, zanim przypomniałem
sobie, że te, tak jak sok owocowy, były importowane. Przez dziką chwilę
zastanawiałem się, czy Republika narzuciła podatki i~cła, potem
zrozumiałem, że taka obraza z~trudem by mnie ominęła. Musiałbym usłyszeć
o zamieszkach. Kurde, sam byłbym \emph{w} tych zamieszkach.

Kwerenda przez bazę danych \emph{Economist} otrzeźwiła mnie. Ceny
surowych materiałów wzrosły ostro przez ostatnie sześć miesięcy od
Jesiennej Rewolucji, podczas gdy ceny wyprodukowanych dóbr i~usług
spadły. Było wiele artykułów wyjaśniających dlaczego, które przeoczyłem
przy moim pochłonięciu naszym małym lokalnymi trudnościami.

Porażka USA/ONZ i~upadek jego finansowych oszustw takich jak MFW\footnote{ MFW -- Międzynarodowy Fundusz Walutowy,
zob.~\url{https://pl.wikipedia.org/wiki/Mi\%C4\%99dzynarodowy_Fundusz_Walutowy} - przyp.tłum.} i~Bank
Światowy\footnote{Bank Światowy -- zob~\url{https://pl.wikipedia.org/wiki/Bank_\%C5\%9Awiatowy} -- przyp.tłum.} miały rozbieżne skutki. 
Podstawowe produkty zwykle pochodziły z~mniej
rozwiniętych obszarów, starego Drugiego i~Trzeciego Świata. Ich
niestabilności sprawiały, że nasze wojny domowy wyglądały jak pokojowe
pikiety. Bez imperium dbającego o~porządek, wzrosły koszty ochrony i~ryzyko. W~międzyczasie, w~bardziej zaawansowanych regionach,
zmniejszenie podatków -- i~koniec ograniczeń rozwoju technologicznego
nałożonego przez kontrolę zbrojeń ONZ -- umożliwiły produkcję, ciesząc
się zrywem wzrostu. Nawet nanotechnologia wyglądała, jakby mogła stać
się w~końcu online, gdy ktoś mógłby wywabić najlepsze umysły z~ukrycia.

Tyle o~cenie kawy. Ciągle mnie martwiło, dlaczego nie byliśmy tak
biedni, jak powinniśmy być. Mój dochód z~uniwersytetu spadł do
symbolicznego stypendium, gdy jedyne wykłady obecnie tam dawane były od
jednego ignoranta do drugiego. (Boże, niech z~tego wyrosną.
\emph{Wkrótce.}). Tantiemy za moje pisma wzrosły, ale niedużo, ponieważ
większość ze zwiększonego obrotu była tych, których prawami autorskimi
wzgardziłem. Nasz fundusz emerytalny płacił regularnie, ale to była
kwota całkiem podstawowa i~zdecydowanie nie wzrosła. A jednak -- inaczej
niż większość ludzi od czasów Rewolucji -- nie musieliśmy zaciskać pasa.

Wprowadziłem nasze wyciągi bankowe i~prawie rozlałem kubek drogiej kawy
rozpuszczalnej. Zwyczajnie drogi papieros wypalał się niezaciągany do
filtra. Nasz stały dochód w~istocie się zmalał, ale saldo było wyrównane
przez wzrastające płatności z~mojego małego, prawie zapomnianego udziału
w Kosmicznych Kupcach. Przekląłem software zarządzania funduszem za
pozwalanie zjadania mojego kapitału, potem go wywołałem.

Nie przejadaliśmy kapitału. Używaliśmy części dochodów, jego małej
części. Wartość moich akcji wzrosła znacznie bardziej niż kiedykolwiek
bym oczekiwał i~prawie podwoiła się od Rewolucji. Byliśmy umiarkowanie,
komfortowo, i~niewytłumaczalnie bogaci.

~

-- Nie rozumiem, na co narzekasz -- powiedziała Annette nad późnym
lunchem. Żadnych pilnych telefonów. Przyjąłem, że okupacja postępuje
gładko. -- Jestem wstrząśnięta. Nigdy szczególnie nie chciałam być
bogata, ale zawsze myślałam, że byłoby to \emph{miłe}.

Spojrzała dookoła kopuły, na stosy książek, wspinające się rośliny i~podejrzane okablowanie elektroniki, myśląc widocznie o~ulepszeniach.

-- Ta, cóż, ja też -- powiedziałem. -- Ale robić pieniądze w~kosmosie w~tych dniach to, jak, \emph{zaprzeczanie grawitacji}. Obrona Kosmiczna
zależała od budżetów obronnych, które będą cięte. Cały przemysł
kosmiczny, nawet osadnictwo, nawet NASA, były jak sklepy w~garnizonowym
mieście. Jak burdele! Cały system powinien być w~ciężkim kryzysie. Wiele
z tego, stacje orbitalne pracują na pusto, puszczając strumienie
mikrofal kompaniom elektrycznym i~takie tam. Więc dlaczego Kosmiczni
Kupcy radzą sobie dobrze?

Oczy Annette błysnęły rozbawieniem lub smutkiem. 

-- Nie przestaniesz, co?
-- spytała. -- Myślisz, że coś znalazłeś, i~nie przestaniesz.

-- Ta -- powiedziałem, podnosząc i~sprzątając talerze.

-- Jeżeli odkryjesz, że to wszystko było straszliwy błędem, to wyświadcz
mi przysługę -- powiedziała. -- Zabierz pieniądze i~pryskaj. Nie obchodzi
mnie, do kogo należą, tyle są ci winni.

-- Pół dnia w~Państwie -- powiedziałem -- i~już myślisz jak polityk.

-- Nie -- poprawiła mnie, wstając i~śmiejąc. -- Myślę jak żona polityka.

~

Żołnierze zostali, obozy zostały spacyfikowane, ludzie z~wszystkich
skrzydeł Ruchu Kosmicznego potępili mnie. Nie odpowiadałem na ataki.
Spadł śnieg. Rozgrzewaliśmy się i~pracowaliśmy nad puzzlami jako
drużyna. Annette śledziła wiadomości, a~ja śledziłem pieniądze. Jako
rzecznik wolnego rynku, byłem żenująco nieświadom finansów, i~minęło
kilka dni, zanim mogłem znaleźć drogę przez różowe ekrany \emph{FT} bez
ciągłych zapytań do Wizards.

Potem w~wielkiej bazy danych Rejestru Firm\ldots w~VR mogłeś wędrować przez
nią jak przez potężną galerię handlową, jej połączenia i~przecięcia
naśladujące niemożliwe topologie obrazów Eschera\footnote{por.~\url{https://en.wikipedia.org/wiki/Print_Gallery_(M._C._Escher)} -- przyp.tłum.}.  Wszedłem jako ja, i~tak samo niektórzy inni poszukiwacze
i badacze, ale większość była w~utajnionych sobowtórach, korporacyjnych
ikonach lub lustrzanej zbroi samurajów ostatniego software prywatności z~firm kodujących w~Kobe (,,Kryptografia Zen -- \emph{nawet o~tym nie
myśl}'', mówiła reklama).

Z Rejestru mogłeś zobaczyć świat.

Widziałem zawiłe geometrie tajskiego systemu islamskich banków kruszące
się pod natarciem antytechnologicznych Czerwonych Khmerów. Ekonomię
portu Władywostoku, wyzwolonego przez Ludowy Front Workuty, wzrost w~nowych i~dziwnych kształtach. Postrzępione sieci naukowej informacji
Ameryki lśniące mocniej na wybrzeżach, mrugające i~umierające w~głębi
kontynentu, gdy Naukowi Fundamentaliści i~Biali Nacjonaliści zamykali
gorszące instytuty tego, co nazywali publicznie ,,naturalizmem bez
korzeni'', a~prywatnie ,,żydowską nauką''.

Widziałem kosmodromy Kazachstanu rozciągające się w~niebo i~ujrzałem
także dopływy, które je żywiły, archipelag przedsiębiorstw pracy
przymusowej KomLagu. Niektóre w~Byłym Związku -- stare umiejętności
wykorzystane na nowo -- ale większość w~swobodniejszym świecie. Kilka
dokładnie tutaj w~Norlonto.

Kiedykolwiek zwycięskie siły Jesiennej Rewolucji mogły, zatrzymywały
bardziej użytecznych pracowników pokonanego imperium USA/ONZ, i~szczególnie Obrony Kosmicznej, w~pracy za grosze, w~częściowym
odszkodowaniu za wcześniejszy wyzysk. Byli oni uzupełniani przez nowe i~rozszerzające się wykorzystanie niepolitycznych więźniów, zarabiających
na swój zwrot z~dużą prędkością w~wysokoryzykownej, wysokopłatnej
ekonomii kosmosu.

~

-- Niewolnictwo -- powiedziała Annette. -- Po prostu nie mogę uwierzyć, że
do tego doszło.

-- To nie jest naprawdę niewolnictwo -- powiedziałem niewyraźnie. -- To
jest tylko przymusowa praca.

-- Ta, tak. Jak w~nie mamy kary śmierci, po prostu pozwalamy psychopatom
spłacić ich dług, występując w~filmach typu snuff\footnote{gatunek filmowy
obejmujący filmy przedstawiające sceny jakoby prawdziwych gwałtów,
tortur czy zabójstw,
zob.~\url{https://pl.wikipedia.org/wiki/Snuff_film } -- przyp.tłum.}?
 
-- Dokładnie -- powiedziałem. -- Ciągle chcesz wziąć pieniądze i~uciekać?

-- Nie! -- Spojrzała ostro na mnie, potem na stół. -- Z~drugiej strony, nie
ma nikogo, komu można je oddać, byłoby kontrproduktywne sprzedawać
udziały komuś, kto ma jeszcze mniejsze skrupuły niż Ty, i~byłoby całkiem
obłudne po prostu rozdać pieniądze.

-- Żeby nie powiedzieć słabe.

-- Ta. Ka-A\footnote{oryg. ,,C of E'' -- Church of England -- Kościół Anglii -- przyp.tłum.}. 

-- Więc jaka jest odpowiedź?

-- Wykorzystaj je do pokazania, skąd pochodzą -- powiedziała Annette
stanowczo. -- Wgryź się w~to mocniej, potem przeprowadź kampanię, by
wyciągnąć to wszystko na jaw i~przedyskutować. Mógłbyś to zrobić.

-- I~zrealizować co?

-- Och, daj spokój! Jeżeli trwają nadużycia, zatrzymanie ich mogłoby
pomóc.

Na to zdanie oboje się roześmialiśmy, ale Annette powiedziała, kiedy
wpadliśmy w~ponurą ciszę, co jeszcze jest do zrobienia?

~

Krążyłem ostrożnie w~przestrzeni danych dookoła reprezentacji
kazachskiego zaplecza kosmodromu i~zauważyłem slogan przedsiębiorstwa,
które założyłem tak dawno temu: Kosmiczni Kupcy. Miało ono silny
przepływ materiałów i~informacji łączących je z~minipaństwem kazachskich
robotników Myry i~agencją ochrony Wzajemna Pomoc Reida. Wzmocniłem
rozdzielczość, próbując wyśledzić co się dzieje.

Wszystko zmienili, rozwinęli się poza cokolwiek którekolwiek z~nas
początkowo zamierzało. Kosmiczni Kupcy stali się biznesem import-export
pomiędzy Ziemią a~niską orbitą, prawie tak odległą od ich niewinnych,
fanowskich początków na rynku śmieci z~kosmosu jak ostatnie dopalacze
SSTO\footnote{ SSTO -- single stage to orbit -- oznacza pojazd, który dociera na
orbitę z~powierzchni używając jedynie paliwa i~płynów bez utraty
zbiorników, silników lub dopalaczy, zwykle są to pojazdy wielokrotnego
użytku, por.~\url{https://en.wikipedia.org/wiki/Single-stage-to-orbit
} -- przyp.tłum.} pochodziły od amatorskich rakiet Goddarda\footnote{Robert
Goddard -- pionier techniki rakietowej i~astronautyki,
por.~\url{https://pl.wikipedia.org/wiki/Robert_Goddard_(1882\%E2\%80\%931945)
} -- przyp.tłum.}. Międzynarodowa Republika Robotników
Naukowo-Technicznych, jej nuklearne zęby dawno wyrwane, zmieniła swoją
specjalizację na rozwój pojazdów startowych. MRRNT przetrwała falę
reunifikacji kazachskiej, a~Ochrona Wzajemna była tam silnie obecna. I~nie tylko tam: Ochrona Wzajemna prowadziła ośrodki ochrony i~restytucji
na trzech kontynentach, zwykle chroniąc instalacje i~wyciągając zwrot od
jakichkolwiek złodziei czy sabotażystów dostatecznie głupich, by
zadzierać z~jej klientami.

Było dziwne ujrzeć osobisty trójkąt pomiędzy mną, Myrą i~Reidem,
odtworzony jako powiązania handlowe, jako rodzinne relacje dynastycznych
armii. Ale czy te związki znaczyły cokolwiek, było inną sprawą. (Jak
wskazałem w~\emph{Ignoramus!}, mojej pracy na temat kontrkonspiracyjnej
teorii historii, każdy zna kogoś, kto zna kogoś, kto\ldots(itd), i~jest to
najłatwiejsza praca na świecie pociągnąć tuszem te ołówkowe linie,
spekulować, że zadziwiająco mało uścisków dłoni, która oddziela
skromnych od znanych to tylko \emph{zabawne} uściski dłoni\ldots Moje
nieostrożne ilustracje tego diagramem moich drugich i~trzecich uścisków
powiązań, ,,dowodzące'' istnienia tajemniczej Ostatniej Międzynarodówki
łączącej światowych libertarian i~futurystów ze sobą oraz z~długą listą
zwykłych historycznych podejrzanych, poskutkowało w~pewnej liczbie
nieporozumień: \emph{latami} otrzymywałem anonimowe listy, czegoś, co
rzekomo było protokołami Centralnego Komitetu Ostatniej
Międzynarodówki).

Firewalle chroniły większość danych firm, pozostałości ostatnich ataków
hakerskich blakły na matowych wirtualnych powierzchniach. Ruszyłem
dalej, szukając węzłów wejściowych. Nagle znikąd coś spingowało mój
sobowtór. Moje ręce, w~VR-rękawicach, stały się ciepłe. Cieplejsze.
\emph{Gorące.}

Trzymałem coś, co wyglądało jak zapieczętowana koperta, ikoniczny
równoważnik wiadomości osobistej: opartej na anonimowym protokole
transakcji, która nawet nie mogła być przeczytana na ekranie, tylko w~VR
przez sobowtóra zamierzonego odbiorcy. Była to także metoda dostarczania
wirusów dopasowanych do celu. Spojrzałem na nią -- cholera, już zaczynała
dymić -- i~prędko sięgnąłem za siebie i~szarpnąłem wajchę awaryjnego
backupu. Sekundy kapały, gdy zawartość mojego domowego komputera była
transferowana na izolowane dyski. Kiedy było bezpiecznie, otworzyłem już
tlącą się kopertę.

\emph{drogi jonie}, było napisane, \emph{to zbyt szybkie. pomocy.
pozdrawiam, myra}

Potem list rozpadł się na bity.

Cóż, to było bardzo przydatne, pomyślałem, gdy wycofywałem się i~siedziałem, mrugając w~chłodnym świetle dnia, zagadkowy uśmiech Annette
dokuczał mi z~drugiej strony stołu.

-- Myra się odezwała -- powiedziała.

Gapiłem się na nią. 

-- Skąd wiesz?

-- Z~Twojej twarzy -- powiedziała. -- Widziałam już wcześniej tę minę.

Kontaktowałem się z~Myrą może kilkadziesiąt razy, w~więcej niż
kilkadziesiąt lat: kiedy mieliśmy Bombę, i~w sprawach, w~których
pośredniczyłem dla Ruchu Kosmicznego w~dekadach Norlonto. Istniało
bezpośrednie połączenie sterowcem pomiędzy Alexandra Port i~Bajkonurem,
i spotkałem ją kilka razy, kiedy przejeżdżała, ale większość naszych
kontaktów była zdalna.

Sięgnąłem po dłoń Annette. 

-- Nie jesteś \emph{zazdrosna}? Dobry Boże, to
było siedemdziesiąt lat temu!

-- Wiem -- powiedziała Annette. Ścisnęła moją dłoń. -- I~wiem, że mnie
kochasz. Jednak ją też kochałeś. Myślę, że była jedyną inną kobietą,
którą kiedykolwiek kochałeś. A to prawda co mówią: miłość nigdy nie
umiera. Można ją zabić, jasne, ale sama nigdy nie umiera.

Jej słowa może odbijały echem dowolną liczbę sentymentalnych piosenek i~historii, ale wypowiadała je, jakby była gorzkim, niechętnie
akceptowanym faktem naukowym. Położyła dłoń na moich otwartych ustach,
zanim mogłem zaprotestować, wymówić, wyjaśnić.

-- Wszystko w~porządku -- powiedziała. Potem: 

-- Czego chce tym razem?

-- Nie wiem -- powiedziałem. Wyjaśniłem wiadomość i~gdzie ją znalazłem. -- Ma jakieś kłopoty i~prosi mnie o~pomoc.

-- ,,to zbyt szybkie''. -- Annette patrzyła się przeze mnie, w~jakąś własną
wirtualną rzeczywistość. -- Pasuje, wiesz. Praca niewolnicza, zyski z~kosmosu, \emph{coś }dzieje się za szybko. Jeżeli spojrzysz na
wiadomości, to jakby świat się rozpadał, i~myślę, też jest
\emph{rozrywany}, przez coś, o~czym nie wiemy.

Roześmiałem się. 

-- Jeżeli tak by było, ktoś by mi powiedział.

-- Myślę, że ktoś tak zrobił -- powiedziała Annette. -- Tak czy inaczej,
jest tylko jeden sposób na ustalenie. Jedź do Kazachstanu. Zakładam, że
nie będzie trudno odszukać Myry, lub powiedziałaby ci jak.

Spojrzałem na nią, zdumiony. To była propozycja, nad którą sam właśnie
pracowałem. Ze strony Annette oczekiwałem, jeżeli czegokolwiek, walki
przeciwko niej.

-- Nie chcę, żebyś jechał -- powiedziała. -- Nawet nie wiem, czy tak
zrobisz. Jednak bardziej się boję nic nierobienia. Nikt nie ujął się za
Tobą, od kiedy żołnierze weszli. Nie sądzę, żeby ci już więcej ufali.

-- Oni?

-- Ludzie z~Ruchu Kosmicznego. Towarzysze.

-- Nie ma konspiracji -- Uśmiechnąłem się. To było jedno z~moich
powiedzonek.

Oczy Annette były smutne i~poważne.

-- Tym razem, możesz się mylić -- powiedziała.

Wstała i~ruszyła do komputera domowego, pisząc na klawiaturze w~rześkim
stukocie. 

-- No dalej -- powiedziała. -- Jedź i~jej pomóż. Spróbuję
zarezerwować ci lot. Przygotuj się i~na litość boską, zapamiętaj
zapakować swoją broń.

Zgodziłem się, potrząsając głową. Żadna z~myśli, które Annette wyraziła,
nigdy wcześniej nie przeszła mi przez myśl. Miłość nigdy nie umiera.

Ja pierdolę.

~

Kusiło mnie, żeby podróżować jednym z~tych stabilnie kursujących
sterowców, ale, jak Annette podkreśliła, te zabierały dni, i~były zwykle
załadowane towarem i~zatłoczone pracownikami kosmicznymi na kacu po
miesięcznym urlopie w~Norlonto. Tak więc opuszczałem Stanstead zwykłym
odrzutowcem, znacznie większym niż ten, którym leciałem z~Reidem
trzydzieści lat wcześniej. Tym razem bez ognia przeciwlotniczego.
Korytarz uralski dawno temu został zbombardowany w~bezpieczny przejazd.

Stanstead do Ałma-aty, lotnisko ciągle poznaczone po ogniu artylerii od
zwycięstwa Kazachskiego Frontu Ludowego. Na północ od Karagandy,
przerażające, brudne miejsce, czarne nawet w~śniegu: postsowieckie,
postindustrialne, post-niepodległościowe, post-wszystkie. Z~Karagandy do Kapicy był
regularny lot. Ponieważ MRRNT nadal był niepodległą enklawą,
zostałem zatrzymany na sprawdzenie, pierwszy raz w~całej podróży. Aktyw
frontowy i~lokalni urzędnicy zbadali moje dokumenty, wbili moje
szczegóły do starożytnego superkomputera (zlokalizowanego w~Indii,
jeżeli czas odpowiedzi był jakąś wskazówką), potem zaczęli się uśmiechać
i zaoferowali Johny Walker Red Label, kiedy się pojawił mój rekord.
Powiedziałem dobre rzeczy o~KFL, kiedy nie było to modne. Nalegali, żeby
mi powiedzieć, jak bardzo podziwiali to, a~po kilku whisky powiedziałem
im, jak bardzo ich podziwiałem. Walczyli z~USA/ONZ, zjednoczyli swój
kraj bez napędzania ognia nacjonalistów, i~powstrzymali się od
narzucenia ich państwa na jedną z~części kraju, która tego nie chciała.

-- MRRNT? -- Myśleli, że to śmieszne. Nie powstrzymali żadnych wzniosłych
zasad.

-- Dlaczego nie, zatem? -- Lekko wzruszyłem ramionami, spojrzałem na mapę
nad biurkiem celników. Na pewno nie były to możliwości obrony małej
enklawy.

-- Zła ziemia -- powiedziano mi. -- Kraj bomb.

~

Powiedzieli, że step dookoła Kapicy świeci się w~ciemnościach, ale to
tylko światło gwiazd odbite od śnieżnych pól. Tak właśnie sobie mówiłem
w locie, gdy przedrzemałem efekty schludnie podjętej dobrej whisky,
obudziłem się od wstrząsu, zapaliłem i~znowu przysnąłem. Tylko dwa inne
siedzenia w~samolocie były zajęte, a~ich użytkownicy byli skłonni
dotrzymać sobie swojego towarzystwa tak jak ja. Miałem wyłączone światło do
czytania, przycisnąłem twarz do okna i~obserwowałem, jak czarna nić
drogi z~Karagandy do Semipałatyńska prowadzi przez step, nawet roiłem,
że zobaczyłem drobne iskry światła z~pługów śnieżnych.

Wylądowaliśmy przed świtem na ledwie oczyszczonym ze śniegu pasie
lotniska. Minibus pogonił z~nami do terminala. Za kopcami usypanego
śniegu, stały szkieletowe i~ciemne suwnice. Kilka samolotów było
zaparkowanych, żaden nie lądował. Budynek lotniska był jasny jak nigdy,
jego pracownicy tak bezpieczni w~ich zwykłym zatrudnieniu tak jak
wcześniej, nadmiarowo nadzorując zajęte maszyny. Bohaterowie Republiki
ciągle grozili wielcy na plakatach.

Jednak w~porównaniu z~gwarem, kiedy to miejsce eksportowało odstraszanie
jądrowe, równie dobrze mogłoby być opuszczone. Złowieszcza pustota
przypominała publiczne place starych komunistycznych stolic. Ruszyłem
przez halę z~nerwowym wahaniem, który każdy czuje wchodząc do wielkiego,
starego i~prawdopodobnie niezajętego domu.

Nie miałem pojęcia co robić dalej. Jeżeli Myra chciała mi powiedzieć,
zakładałem, że chciałaby i~mogłaby. Jeżeli miała inne ostrzeżenia,
mogłaby je zawrzeć w~wiadomości. Tak jak to było, wyglądało, że jedynym
aspektem naszego kontaktu, który chciała utrzymać w~tajemnicy, było to,
że potrzebuje mojej pomocy.

Kawiarnia ciągle tam była i~była otwarta. To tam nas spotkała wtedy.
Wszedłem, zamówiłem kawę, usiadłem z~nią i~kopią wydania
angielskojęzycznego \emph{Prawdy Kapicy}, która sprostała swojej nazwie,
że przedstawiła najwidoczniej prawdziwe przedstawienie wiadomości.
Dotarłem do stron sportowych, zanim zauważyłem, że nie zawierała żadnych
wiadomości w~ogóle o~Kapicy.

Przeskanowałem halę, skwapliwie przyglądając się postaciom, które miały
szansę przypominać mojemu wspomnieniu Myry, i~opierałem się za każdym
razem zawiedziony. Minęła godzina. Straż Ochrony Wzajemnej wędrowała
dookoła, jakby była właścicielami miejsca. Więcej osób przyszło i~poszło. Usłyszałem jeden, potem dwa kolejne lądujące samoloty. Ich
pasażerowie rozchodzili się indywidualnie lub w~małych grupach przez
szklane drzwi, na zewnątrz których tuzin taksówek stało na zimnie z~uruchomionymi silnikami.

Może powinienem po prostu znaleźć ją w~książce telefonicznej\ldots Stałem w~budce i~patrzyłem na stronę wyszukiwania, kiedy zrozumiałem, że nie znam
jej obecnego nazwiska. Nawet zajęło mi kilka sekund przeczesanie
pamięci, zanim jej oryginalne nazwisko mi się przypomniało: Godwin.
Spróbowałem. Bez powodzenia.

Wybrałem szyfrowane połączenie do Annette.

-- Cześć, miłości. Bezpiecznie wylądowałem.

Uśmiechnęła się. 

-- Dobrze słyszeć. To nie dlatego zadzwoniłeś.

-- Dlaczego tak mówisz?

-- Wiem jak działa Twój mózg, Jon. -- Roześmiała się. -- To Dawidow.
Znalazłam je w~starej polisie ubezpieczeniowej.

Mniemam, że musiałem wyglądać zażenowany. Annette uśmiechnęła się i~wystawiła język, różowy milimetr na małym ekranie. 

-- Kocham Cię -- powiedziała. -- Trzymaj się.

Ekran zgasł. Westchnąłem, nagle czując się stary i~samotny, i~znowu
włączyłem książkę telefoniczną.

Dawidow, Myra G., komandor podporucznik (emer.) żyła w~Mieszkaniu 36,
Blok 7, Bulwar Ignacego Reissa\footnote{Ignacy Reiss -- szpieg sowiecki,
m.in. organizował siatki szpiegowskie na Zachodzie, uciekł przed
represjami stalinowskimi, więcej~\url{https://pl.wikipedia.org/wiki/Ignacy_Porecki } -- przyp.tłum.}. Żaden inny Davidow nie był wpisany pod tym adresem.
Małżeństwo Myry rozpadło się lata temu. Budynek, kiedy taksówka mnie tam
zostawiła, okazał się być klasycznych sowieckim blokiem, ostatnio
wybudowanym w~rodzaju perwersyjnego hołdu dla ojczyzny robotników, ale z~betonem już kruszącym i~odbarwionym. Tylko jeden samochód był
zaparkowany na zewnątrz, wielka czarna Skoda Traverser. Domyśliłem się,
że Myry: wyglądał jak ten rodzaj pojazdów, który byłby do dyspozycji
emerytowanego Ludowego Komisarza.

Winda, w~kolejnym starannym dotyku autentyczności, nie działała.
Wtaszczyłem moją torbę podróżną na trzecie piętro po schodach. Moje
kolana bolały. Czas, żebym załatwił sobie nowe stawy. Zadzwoniłem do
drzwi i~rozejrzałem się za kamerą CCTV. Nie było żadnej. W~zamian
błysnęła przesłona, odsłaniając rybią soczewką zatopioną w~drzwiach.
Rygle pisnęły, łańcuchy zagrzechotały. Drzwi wolno się otworzyły.
Wydobyło się żółte światło, ciężki zapach, nieświeży zapach papierosów i~głośna muzyka. Potem ręka sięgnęła i~wciągnęła mnie do środka. Drzwi
obróciły się i~kliknęły za mną, i~byłem schwytany w~gorącym kościstym
uścisku.

Po minucie odsunęliśmy się, ręce na ramionach drugiej osoby.

-- Cóż, cześć -- powiedziała Myra.

Jej stalowoszare krótkie włosy pasowały do spiżowej satyny zestawu
pidżamowego. Jej twarz miała woskowy, połysk martwego Lenina, wynik
postsowieckiej technologii odmładzania, rażący kontrast do cętkowanej i~marnej skóry na dłoniach. Tak jak ja, jak wszyscy w~Nowym Starym, była
chimerą młodości i~wieku.

-- Witaj -- powiedziałem. -- Wyglądasz dobrze.

Roześmiała się. 

-- Ty nie. -- Jej palce poskrobały szczecinę na moim
policzku.

-- Nic, czego prysznic nie załatwi.

-- A to -- powiedziała z~bystrym spojrzeniem -- to bardzo dobry pomysł. -- Sięgnęła za mnie i~przełączyła włącznik. Dudniące rozpalające hałasy
dobiegły ze ścian. 

-- Pół godziny -- powiedziała, prowadząc mnie do
salonu. Miał jedno okno z~podwójnymi szybami wyglądające na ulicę. Widok
wykraczał poza zreplikowane ulice dzielnicy, ponad starym miastem
prefabrykatów i~na step.

Grzejnik centralnego ogrzewania stał zimny pod parapetem, elektryczny
ogrzewacz wysyłał gorące suche powietrze. Izolacja zapewniała ciepło.
Pokój był grubo wyłożony dywanem, ściany obwieszone arrasami, ich
wzorce, toporne jak piksele we wczesnych grach komputerowych, były
pokazem tradycyjnych projektów afgańskich helikopterów wojskowych, Migów
i AK-47. Pomiędzy nimi były polityczne i~turystyczne plakaty z~historii
i geografii Kazachstanu (MRRNT miał deficyty w~obu przypadkach) i~stare
reklamy rakietowych startów i~eksplozji nuklearnych. Ekran telewizji,
zawieszony pośród plakatów, był ustawiony, bez dźwięku, na balet Bolszoj
Luna. Pływające loty i~upadki, iluzja formy zrealizowana pod innym
niebem. Wielkie antyczne głośniki Sony wysoko na przeładowanych półkach
tłukły chiński rock.

Stary IBM PC stał na stole obok naręcznej torby Myry i~stosu
podręczników kodowania. Spojrzenie na tytuły sugerowało, że zaszyfrowała
wiadomość ręcznie. Nic dziwnego, że była taka krótka, musiało to zabrać
dni.

Zrobiła mi śniadanie, płatki, jogurt i~gorzka arabska kawa.
Rozmawialiśmy o~locie, o~zmianach w~naszych życiach od czasu ostatniego
spotkania, kilka lat wcześniej w~barze Alexandra Port. Ciągle widywała
eksmęża, dziarskiego oficera, którego kiedyś obaliła i~miałem wrażenie,
że coś ciągle się dzieje pomiędzy nimi, ale od miesięcy był w~Ałma-Acie,
podobno negocjując z~KFL. Zasugerowała, że był trzymany na uboczu.

-- Więc to miejsce ciągle jest państwem trockistowskim? -- spytałem.

Myra odstawiła filiżankę, jej dłoń lekko drżała.

-- Och, tak -- powiedziała. -- To jest, jak w~Rosji, kiedy Lew Dawidowicz
był u władzy.

-- Tak źle? -- Uniosłem brwi, skinęła głową.

-- Już nie jesteś w~rządzie?

-- Od jakiegoś czasu już nie. -- Uśmiechnęła się krzywo.

-- Jestem pewien, że ciągle możesz dużo opowiedzieć -- powiedziałem. -- Czy
ktokolwiek cię słucha? -- Przechyliłem głowę.

-- Na pewno -- powiedziała. -- Nie mogę narzekać.

Spojrzała na zegarek. 

-- Twój prysznic jest gotowy.

~

Prysznic był w~kabinie przy jej sypialni. Położyłem ubrania w~nogach jej
łóżka, ostrożnie, nie chciałem ich pognieść. Gdy wygładzałem kurtkę,
moje palce drasnęły twardą krawędź pistoletu, schludny, płaski
plastikowy egzemplarz, nie dłuższy lub grubszy niż moja dłoń. Po chwili
namysłu, wyjąłem go i~gdy wchodziłem do prysznica, położyłem go w~górnym
rogu kabiny. Potem włączyłem prysznic i~stałem w~parującym pryskaniu,
wdzięczny, że chociaż to było zbudowane według dokumentacji. Ledwie
spłukałem pierwsze namydlenie, kiedy drzwi się otworzyły i~Myra weszła.

-- To stary trick, ale działa -- wyszeptała w~moje ucho, trąc plecy. -- Biały hałas jest białym hałasem, nieważne, czego użyjesz.

-- Naprawdę myślisz, że jesteś podsłuchiwana?

Roześmiała się. 

-- Tak bym zrobiła, w~ich sytuacji.

-- Kim są ,,oni''? Co się dzieje?

Wzięła małą jednorazową golarkę i~aerozol, który wyrzucił pianę o~zapachu sosny. Namydliła moją twarz i~zaczęła ją golić, w~ten sposób
zapewniając skupienie mojego wzroku. Równie dobrze, ponieważ wymagało
to prawie czytania z~ust, żeby zrozumieć jej wyszeptywane słowa w~ciągłym gorącym deszczu, a~nie było czasu, żeby mogła powtórzyć lub
żebym przerwał.

-- Wiesz, że coś się dzieje -- powiedziała. -- Zostawiłam tę wiadomość
tygodnie temu, ponieważ pomyślałam, że jeżeli ktokolwiek by to badał, to
byłbyś Ty. -- Uśmiechnęła się. -- I~miałam rację. Ok, oto historia. Zaawansowana technologia, nanotech, inżynieria genetyczna, AI i~tak
dalej, była ograniczona pod Jankesami i~ciągle jest atakowana w~różnych
miejscach, dzięki cholernym Zielonym, religijnym fanatykami i~tak dalej.
Dwie rzeczy się wydarzyły. Pierwsze, miejsce takie jak to przyjęło uchodźców
naukowych i~pozwoliło im podjąć pracę pod przykrywką innych projektów.
Drugie, USA/ONZ i~szczególnie Obrona Kosmiczna kontynuowały własną
pracę. Zakaz był dla wszystkich innych, nie dla nich. Teraz to wszystko
się składa: nasi naukowcy pracują z~ich i~możesz się cholernie
założyć, że kooperują, to jedyny sposób, żeby odpracowali swoje długi.
To samo dotyczy ogromnej, jenieckiej siły roboczej. Wysyłają towary w~kosmos, jakby nie było jutra, i~w tym tempie, nie będzie. Myślę, że
zmierzają do zamachu stanu.

-- W~Kazachstanie?

-- Na \emph{świecie}, głupku!

Naprawdę poczułem się głupio. To lub ona była szalona.

-- Do kurwy nędzy, kto? I~jak?

-- Twój Ruch Kosmiczny, ok, może nie Twój, ale\ldots tak czy inaczej mają
ludzi w~oficjalnych programach kosmicznych, nawet w~Obronie Kosmicznej.
I widzą, jak się rzeczy układają, od Jesiennej Rewolucji. ,,Jesień'' się
zgadza! Wszystko się rozpada, to jak globalna wersja rozpadu Związku
Radzieckiego. Kolejne kilka miesięcy, lat co najwyżej i~nie będzie
rakiet wznoszących się skądkolwiek. Mówi się, teraz lub nigdy, jeżeli
kiedykolwiek chcemy trwałej obecności w~kosmosie. Jesteśmy w~tym, co
nazywają klątwą surowcową\footnote{oryg. resource trap -- klątwa surowcowa -- zjawisko występujące w~gospodarkach krajów posiadających znaczne złoża
surowców naturalnych. Objawia się ono niespodziewanie niskim poziomem
rozwoju gospodarczego kraju,
zob.~\url{https://pl.wikipedia.org/wiki/Kl\%C4\%85twa_surowcowa
} -- przyp.tłum.}. 

To przynajmniej pasowało do tego, co widziałem i~co podejrzewała
Annette.

-- Rozumiem, że to ,,dlaczego'' -- powiedziałem. -- Pytałem, kto i~jak. Nawet
Obrona nie mogła naprawdę zdominować świata bez wsparcia na powierzchni,
a teraz kiedy znikła, rozszczepiona\ldots

-- Powiedziałam Ci -- syknęła. -- Tak dużo, jak mogłam w~czasie, który
miałam. ,,To zbyt szybkie'', pamiętasz? \emph{Nanotech}. Z~tym możesz
zbudować statki kosmiczne, nie wielkie tępe rakiety, ale prawdziwe
statki tak lekkie i~mocne, że mogą osiągnąć prędkość ucieczki ot
\emph{tak}. -- Jej ręka uniosła się do góry. -- Szuuuu. Mają AI, które
mogą kierować wyrzutniami laserowymi, wysyłać statki na igle odrzutu z~przegrzanej pary\footnote{para przegrzana to para sucha mająca temperaturę
wyższą niż temperatura wrzenia cieczy przy danym ciśnieniu,
zob.~\url{https://pl.wikipedia.org/wiki/Para_przegrzana},
wykorzystanie jako napęd
zob.~\url{https://en.wikipedia.org/wiki/Thermal_rocket\#Laser_thermal_rocket
} -- przyp.tłum.}. A z~nanotech, masz to, możesz mieć ich tak wiele, jak
zechcesz, możesz je hodować jak \emph{drzewa}!

Wzruszyłem ramionami, pod lecącą wodą, z~roztargnieniem myjąc jej chude
boki.

-- Jeżeli masz to wszystko, nie musisz rządzić światem. Wszystko, co
musisz zrobić to go uratować.

Myra potrząsnęła głową, rozsyłając lecące krople. 

-- Nie chcą go uratować
ani nie myślą, że świat chce być uratowany. Och, Jon, spędzasz czas z~tymi wszystkimi humanistami i~anarchistami, i~po prostu nie wiesz jak
dużo goryczy i~pogardy jest pośród naukowo-technicznej elity wobec
ciemnych mas! To dlatego mnie wyrzucili, po Jesiennej Rewolucji, kiedy
trochę się tym zajęłam i~zaczęłam narzekać. Nazwali mnie populistką i~rewizjonistką! -- Roześmiała się. -- Cierpieli i~męczyli się przez lata
pod biurokratami ONZ, policją Stasis i~Zielonymi sabotażystami, i~już
nigdy więcej nie chcą zadzierać z~tymi ludźmi. Naprawdę wierzą, że
jeżeli wydostaną się wiadomości o~planach, motłoch pomaszeruje na
laboratoria, demagodzy pchną rządy w~kolejne represje i~wszystko się
skończy.

Spojrzałem znad jej goleni. 

-- Mogą mieć rację.

-- Nie mów tak! To właśnie Reid mówił im przez lata!

Wstałem, prawie poślizgując się na mokrej, wklęsłej podłodze kabiny.

-- \emph{Reid}?

-- Cśśśś. Tak, myślałam, że wiesz. Kieruje całym teatrzykiem i~planował
to od dłuższego czasu. Myślę, że mógłby to zrobić, nawet gdyby nie
wydarzyła się Rewolucja, ale teraz po niej, działa szybciej niż
kiedykolwiek. Ochrona Wzajemna i~jej przeklęte sprywatyzowane gułagi są
mięśniami za tym wszystkim, a~on jest najgorszy z~nich wszystkich. Myśli
tak jak Ty czasem pisywałeś, o~wolności, ale dla niego to jest absolut,
żadnej etyki, żadnej polityki. Nawet naukowcy się go boją.

Mogłem w~to uwierzyć. Od kiedy przestał być komunistą, Reid podążał za
własnymi zainteresowaniami. Tak jak i~ja -- bycie stróżem brata było
ciągle dla mnie grzechem pierworodnym -- ale nigdy do końca nie
osiągnąłem prostolinijnego oddania Reida w~tej kwestii.

Prysznic zamarł w~kapaniu.

-- Co z~tym zrobimy?

Myra spojrzała na mnie. 

-- Wiem, co chcę zrobić -- powiedziała z~nikczemnym uśmiechem. Spojrzała w~dół. -- Jezu, czy tego typu gadka \emph{Cię podnieca}?

~

Wysuszyliśmy się w~milczeniu w~małej przestrzeni, którą wielkie łóżko
Myry zostawiło w~pokoju, i~kontynuowaliśmy rozmowę pod osłoną kołdry i~bardzo głośnej muzyki. Powiedziała mi, co zrobimy i~wtedy to zrobiliśmy,
potem leżeliśmy na boku, twarzą w~twarz, nogi splecione, rozmawiając o~brudnej polityce. Szeptaliśmy pod pościelą jak dzieci po zgaszeniu
świateł.

Zwykłe ujawnienie tego, co się działo, mogło się równie dobrze zakończyć
w sposób, którego fakcja Reida się obawiała. Zostawienie tego mogłoby
zakończyć się na chaotycznym i~krwawym podziale ludzkości, pomiędzy
niewielką mniejszość w~kosmosie i~związaną z~ziemią większość
zdominowaną najprawdopodobniej przez antytechnicznych i~paranoicznych
liderów. Tak czy inaczej, perspektywy cywilizowanej przyszłości były
mroczne.

Istniała jeszcze jedna droga, argumentowała Myra: zachęcić coś, co
nazywała ,,prawowitym'' Ruchem Kosmicznym, do zorganizowania kampanii o~dokładnie te same rzeczy, co grupa Reida chciała, dostęp do technologii
rozwiniętej przez podziemie naukowe oraz ONZ, wysiłek podtrzymania
programu podboju kosmosu, ale otwarcie i~ochotniczo, ufundowanej przez
darowizny raczej niż wymuszenia. Ujawnić to wszystko i~przedyskutować.
To była jedyna droga, by podważyć podejrzenia po obu stronach: niech
technokraci zobaczą, że ludzie naprawdę chcieli tego, co mogli dać, i~że
rzeczywiście zapłaciliby za to. Niech zwykli ludzie zobaczą, że
zaawansowana technologia nie zamierza zamienić biosfery w~roboty
wielkości bakterii, lub ich w~maszyny, i~wszystkie inne rzeczy, które im
powiedziano, żeby się obawiali.

-- A Ty -- powiedziała -- jesteś jedyną osobą, o~której wiem, która mogłaby
to zrobić.

-- Ja? Pochlebiasz mi, pani.

-- Masz kontakty, wiarygodność\ldots

-- Nie jestem już zbyt popularny wśród kadry ruchu kosmicznego -- powiedziałem. -- Prawdę powiedziawszy, myślę, że większość z~nich już
myśli w~sposób, w~jaki grupa, według Ciebie, Reida myśli.

A (nie powiedziałem) była tylko jedna rzecz, która mogłaby zwrócić
zwolenników ruchu przeciwko organizatorom, a~było to ujawnienie spisku,
jeżeli taki był. Leżałem w~ciemnym namiocie kołdry przez kolejną minutę,
patrząc na twarz Myry, myśląc pewne myśli, które, mam nadzieję, nie
ukazywały się na mojej. Zaczynając od pierwszej dużej: powiedziała mi
kupę kłamstw.

-- Zjedzmy lunch -- powiedziałem.

~

Lunch był w~małej greckiej restauracji za rogiem.

-- Dlaczego Grecy? -- spytałem, gryząc gorący szaszłyk.

-- Podążyli tutaj śladem Tatarów, zanim Tatarzy wrócili do domu -- wyjaśniła Myra.

-- To dużo historii -- zasugerowałem.

-- Ta -- powiedziała Myra. Rozejrzała się. -- Zostaw to.

Wypiliśmy dobre wino i~trochę okrutnej brandy. Myra rozmawiała o~bezpiecznych, niekontrowersyjnych tematach, jak dlaczego cały stan
świata był \emph{moją winą}.

-- Gdybyś sprzedał Niemcom opcje -- wyjaśniała Myra -- pierdolony Izrael
jest\ldots -- (tak było zawsze z~Myrą, jak jedno słowo) -- \ldots nigdy by nie
ośmielił się zrobić tego, co zrobili, a~Jankesi nigdy by nie zostali
zdobyci, i\ldots

-- I~tak dalej -- roześmiałem się. -- No weź. Dziesiątki ludzi musiało być
w tej samej sytuacji co ja, którzy podjęli taką samą decyzję.

-- Tak, ale wszyscy oni potrzebowali swoich jądrówek. Ty nie. Po prostu
trzymałeś się ich dla zasady.

-- Nie, nieprawda! Nigdy w~moim życiu nie podjąłem decyzji dla zasady!
Jestem oportunistą i~jestem z~tego dumny. Tak czy inaczej, dlaczego po
prostu nie pozwoliłaś im mieć ich odstraszania, i~załatwić papiery po
wszystkim?

Myra uśmiechnęła się do mnie, wzruszyła ramionami.

-- Złe dla biznesu.

Uśmiechnąłem się do niej.

-- To był również mój powód.

Sięgnęliśmy po piernik, kawę i~ostatnie kieliszki brandy. Myra
podniosła, polizała i~łyknęła. Znieruchomiała, uśmiech oświecenia na jej
twarzy.

-- To jest to! -- powiedziała. -- Powinnam wiedzieć lepiej niż obwiniać
jednostki. Cały cholerny bałagan jest winą\ldots

-- Kapitalizmu! -- powiedziałem głośno, a~\emph{kelner} przyszedł z~rachunkiem.

~

Z powrotem w~mieszkaniu zanurkowaliśmy znowu w~łóżko. Zostawiła głośniki
włączone. Ledwie zauważyliśmy, kiedy muzyka rockowa zamieniła się w~wojskową, ale oboje leżeliśmy po tym w~ciszy, kiedy obwieszczenie, że
lotnisko jest tymczasowo zamknięte, zagrzmiało w~domu.

Nie musieliśmy rozmawiać o~tym, co to znaczyło. Muzyka wojenna i~zamknięte lotniska były tradycyjnym preludium do ogłoszenia, że kraj
został uratowany. Ktoś wykonał swój ruch. Nadszedł czas, żebym też tak
zrobił, zanim pojawią się blokady drogowe, lub Myra mnie wyda, dla
własnej i~mojej ochrony.

Odsunąłem delikatnie pasmo włosów z~jej twarzy.

-- Gotowa na papierosa? -- Uśmiechnąłem się.

-- Boże, tak.

-- Mam jakieś w~kurtce -- powiedziałem, siadając i~sięgając ku brzegowi
łóżka.

-- Nie, nie -- powiedziała Myra. Odrzuciła kołdrę, łapiąc moje przedramię.
-- Musisz spróbować naszych. Naprawdę.

Uśmiechnęła się w~moje oczy. Czy myślała, że sięgałem po broń? Jeżeli
tak, musi myśleć, że ciągle jest kurtce. Musiała poczuć, kiedy
obejmowaliśmy się w~korytarzu, i~nie sprawdzała znowu, zanim weszła pod
prysznic.

Sięgnęła do szafki nocnej, otworzyła szufladę. Nie odwracałem wzroku od
niej ani na sekundę i~nie puściła mojego ramienia, gdy grzebała w~szufladzie i~wyjmowała paczkę papierosów. Paliliśmy w~zamyślonej ciszy.
Silny, ostry papieros sprawił, że buczało mi w~głowie. Czy ona
podejrzewała, że ja podejrzewałem?

Zgasiłem papierosa, mrugnąłem do niej, i~powiedziałem, trochę za
głośno: 

-- Myra, nie masz nic przeciwko odwiezieniu mnie do hotelu?

Uśmiechnęła się do mnie i~powiedziała, znowu jakby na korzyść
kogokolwiek, kto mógłby słuchać: 

-- Nie ma sprawy.

Ubrałem się prócz kurtki, schyliłem się, żeby zapiąć moją torbę i~powiedziałem: 


-- Ach, zostawiłem ręcznik pod prysznicem.

Pochyliłem się do kabiny, odnalazłem pistolet, odwróciłem się\ldots

Moja noga sięgnęła szafki nocnej na chwilę przed jej ręką i~zatrzasnęła
szufladkę. Gdy szarpnęła się do tyłu, znowu otworzyłem szufladkę, i~wyłowiłem pistolet, o~którym wiedziałem na pewno, że tam będzie.

Myra siedziała sztywno, ściskając kołdrę jakby dla ochrony.

-- Jestem gotowy -- powiedziałem jej. Wsunąłem jej wielki, ciężki automat
do mojej kieszeni kurtki, podniosłem kurtkę i~ułożyłem ją na ramieniu i~dłoni. -- Możemy ruszać, jak tylko się ubierzesz.

Kiedy się ubierała i~byliśmy z~powrotem w~salonie, próbowała zwyczajnie
sięgnąć po torebkę, ale byłem szybszy. Schowałem do kieszeni kolejny
pistolet, ten nawet mniejszy i~lżejszy niż mój, rzuciłem jej klucze i~wskazałem drzwi. Założyła długi, futrzany płaszcz i~zeszła po schodach
przede mną. Czarna Skoda ciągle stała samotnie na ulicy.

Podążając za moimi cichymi wskazówkami, otworzyła drzwi pasażera i~wsunęła się przez nie na miejsce kierowcy. Wsiadłem i~zamknąłem drzwi.
Włożyła kluczyk i~silnik natychmiast wystartował tak jak nagrzewanie.
Równie dobrze, zamarzałem po przejściu bez kurtki tych kilku kroków na
dworze.

Odwróciła twarz do mnie, łzy w~jej oczach.

-- Jon -- powiedziała -- co robisz? Zaufałam Ci. Pracujesz dla Reida?

-- Widzę, że nie martwisz się o~pluskwy w~samochodzie -- zauważyłem. -- Myślę, że też się nie martwiłaś podsłuchem w~mieszkaniu. Jedź.

Jej ramiona opadły. 


-- Ok, ok -- powiedziała. -- Dokąd?

-- Karaganda\footnote{miasto w~Kazachstanie, pół miliona mieszkańców,
więcej~\url{https://pl.wikipedia.org/wiki/Karaganda } -- przyp.tłum.}.
 
-- Co? -- Spojrzała na mnie z~otwartymi ustami. -- To setki kilometrów.
Semipałatyńsk jest bliżej.

-- Wiem -- powiedziałem. -- Zamknij się i~prowadź.

Granica na drodze do Karagandy była w~odległości pięćdziesięciu
kilometrów i~wiedziałem, z~moich rozmów z~kadrą KFL zeszłej nocy, że
większa republika kazachska miała tam przejście graniczne, ale nie
MRRNT.

Myra włączyła bieg i~pojazd wyjechał, gdy pierwszy śnieg tego dnia
zaczął padać.

~

Historia Myry, zdecydowałem, po prostu się nie składała. Jeżeli ona i~jej działania były inwigilowane, moja wizyta byłaby odkryta. Jeżeli
wypadła z~łask władzy, jej kontakt ze mną mógłby być tylko
zinterpretowany jako podejrzany. Musiało to być tak oczywiste dla niej,
jak było to dla mnie, że pierwszą rzeczą, jaką bym zrobił po bezpiecznym
powrocie do domu, byłoby przedstawienie jej historii publicznie,
ryzykując lub nie.

Z tego wynikało, że oboje, ona i~aparat bezpieczeństwa MRRNT,
\emph{chcieli}, abym to ujawnił, i~że ciągle była w~łaskach tego
aparatu. Stąd wynikało, że jej opowieść o~małej republice całkowicie
przejętej przez jakąś fakcję powiązaną z~firmą Reida była fałszywa.
Znacznie bardziej prawdopodobne było to, że rdzeń państwa był przeciwny
(bez wątpienia ingerującej) przejęciu przez firmę i~pragnął mojego
szczerego ujawnienia jako doskonałego politycznego pretekstu (przed,
albo po fakcie) do potwierdzenia własnej władzy.

Więc cokolwiek się działo, czy to firma, czy państwo uderzyło pierwsze,
nie ma szans, żebym chciał być w~to zamieszany. I~nie było szans, także,
że jakiekolwiek głębsze zagrożenie, z~którym mieliśmy do czynienia ze
strony technokratów Reida, byłoby zneutralizowane kampanią polityczną.
Jedyny sposób wyjścia, jaki widziałem, było przedstawienie całej
historię jedynemu państwu, które mogłoby działać szybko i~którego
intencjom ufałem nieco bardziej niż każdemu innemu państwu, o~którym
mógłbym pomyśleć: otaczającej Republice Kazachskiej.

Dlatego też jechaliśmy pomiędzy metrowej wysokości odrzuconymi pługiem
grzbietami śniegu, po drodze pokrytej już centymetrowej grubości opadem.

Myra próbowała coś powiedzieć raz lub drugi, błagała, żebym wyjaśnił, co
robię, i~za każdym razem mówiłem jej, tak ostro, jak to możliwe, żeby
zamknęła ryja. Chciałem, żeby się bała, denerwowała, chciałem, żeby
myślała, że jestem zdolny ją zastrzelić. Do czego z~pewnością nie byłem,
ale jej szczere przekonanie, że mógłbym, powinno pomóc uchować ją od
kłopotów, obojętne kto wygra.

W mniej niż godzinę, granica była tylko minutę drogi. Dojechaliśmy do
szczytu karłowatego wału i~mogłem zobaczyć światła kazachskiego
przejścia granicznego przez padający śnieg. A chwilę później i~trzysta
metrów przed nami, także linię mężczyzn w~jasnych żółtych kombinezonach
przetrwania z~wielkimi czarnymi karabinami, machających na nas.

-- Ochrona Wzajemna -- powiedziała Myra z~gorzkim śmiechem. -- Co teraz,
mądralo?

-- Zatrzymaj samochód -- powiedziałem spokojnie. -- Obróć go, żeby Twoja
strona była najbliżej, i~wysiądź z~rękami w~górze.

Spojrzałem na jej zaskoczoną twarz i~dodałem, gdy nacisnęła hamulec: 

-- Jeżeli to ok dla Ciebie.

-- Jest ok -- powiedziała.

Była dobrym kierowcą. Zatrzymała samochód na tyle szybko, żeby obrócić
tyłem dookoła i~zakopać przód w~zaspie śniegu.

Otworzyłem drzwi pasażera, wytoczyłem się z~kurtką i~bronią i~ruszyłem
przez szczyt oleistego, krupiastego śniegu zaspy, utrzymując masę
samochodu pomiędzy mną a~strażnikami firmy. Czołgałem się na kolanach i~łokciach, aż zbliżająca się linia mężczyzn minęła mnie w~drodze do
samochodu. Mogłem usłyszeć protestujący, podniesiony, oficjalny głos
Myry i~miałem nadzieję, że cokolwiek co myślała o~mojej uciecze,
ostatnią rzeczą, jaką by chciała, to żebym wpadł w~ręce jej
przeciwników.

Ciągle się czołgałem, tak blisko wału odrzuconego śniegu z~drogi, jak to
możliwe. Żwir pociął mi palce, łokcie i~kolana. Ciepło uciekało z~mojego
ciała z~każdą mijającą sekundą. Kiedy nie mogłem już tego znieść,
podniosłem się do przykucnięcia. Światła posterunku były pół kilometra
dalej. Spojrzałem do tyłu. Mężczyźni sprawdzali samochód, Myra
rozpętywała poważny polityczny incydent.

Zacząłem biec. Z~początku próbowałem biec sprintem, ale nie mogłem.
Wyprostowałem się i~zacząłem biec truchtem. Czułem moje boki, jakby były
przebijane gorącymi mieczami. Przysiągłem, że nigdy nie zapalę.

Potem poczułem mocne uderzenie w~plecy i~zobaczyłem krew tryskającą z~piersi, podążyłem za jej czerwonym łukiem w~śniegu, jakbym mógł złapać
krople.

~

Byłem na plecach, patrząc na białe niebo. Nade mną unosił się niemożliwy
obiekt, diamentowy statek: fasetowany, błyszczący, jak delikatny biały
duch bombowca stealth, zawieszony na śmiesznie delikatnych dyszach
odrzutowych. Linowa drabina zsunęła się na dół, zszedł ubrany na biało
mężczyzna. Podniosłem głowę kilka centymetrów, gdy dotarł do gruntu i~na
mnie spojrzał. To był David Reid. Jego twarz nic mi nie mówiła.

Żółte kombinezony, twarze w~goglach. Myra, jej ramiona stanowczo
przytrzymane, gdy napięła się ku mnie.

-- Miłość nigdy nie umiera -- próbowałem powiedzieć i~umarłem.

\part{ZAPORY ANARCHII}
\chapter{Duchowy Android}

-- Rusz się i~jesteś martwy!

Wesoły cockney Szacownego Starszego Eona Talgartha, Sędziego Rezydenta
przy Sądzie Piątej Dzielnicy, zagrzmiał z~głośników dookoła stumetrowego
placu jego ogrodzonej własności. Wystarczająco broni zamontowanej na
palisadzie było skierowanych do środka, żeby zamienić sąd w~miejsce
egzekucji w~centrum. Neutralni, którzy uciekli na boki, byliby
bezpieczni, ale przeciwne grupy, każda licząca kilkadziesiąt osób,
konfrontujących się przed podium Talgartha, była w~centrum rozrzutu.
Sytuacja stała się jasna dla wszystkich w~tym rejonie w~ciągu kilku
sekund.

-- Otóż to, otóż to -- namawiał Talgarth. -- Teraz, dobrzy ludzie, czy
moglibyście odłożyć broń, ładnie i~powoli, wiecie, o~co mi chodzi?

Bronie zostały schowane lub zawieszone na ramionach. Ciągnik Jay-Duba
kontynuował toczenie do przodu. Talgarth czekał, aż jego ogon minie
bramę, i~podniósł lewą rękę. Pojazd się zatrzymał.

-- Dobrze -- wycedził. -- Sprawa jest odroczona. Skoro strona Davida Reida
wykonała pierwszy ruch ku uregulowaniu kwestii przez przemoc, wydaje się
to tylko sprawiedliwe, by pozwolić drugiej stronie wykonać strategiczny
odwrót, póki inna umowa nie zostanie przygotowana.

Przez chwilę, nikt się nie ruszał. Talgarth wskazał szczęką na grupą
dookoła Jonathan Wilde'a.

-- Nie stójcie tam tak -- popędził ich. -- Ruszcie się.

Wycofali się powoli, potem odwrócili i~pobiegli do długiego, niskiego,
srebrnego kształtu przy bramie. Reid i~jego grupa patrzyła się na nich,
mięśnie drżące, świadomi ciągłej osłony dział Talgartha.

-- To jest hańba! -- warknął Reid. -- Kto teraz zaufa Twojej
sprawiedliwości, Talgarth?

-- Cholerna widownia byłaby bardziej pod wrażenie przez pozwolenie na
rozpoczęcie przez was rzeźni w~moim sądzie -- odpowiedział Talgarth, jego
oczy śledziły biegnące postacie. Reid również był chwilowo rozkojarzony
jakąś informacją wyszeptaną do ucha.

-- Wiesz czyja to ciężarówka? -- zażądał. -- To jest pojazd robota
Jay-Duba.

-- Wiem -- odpowiedział Talgarth spokojnie. -- Wiedziałem, że jest w~pobliżu od jakiegoś czasu. -- Postukał w~ucho i~uśmiechnął się, nagle
wyglądając bardziej na kryminalistę niż na sędziego. Grupa Wilde'a
zniknęła z~tyłu ciężarówki. Jego silniki zabrzęczały i~pojazd zaczął się
wycofywać powoli przez bramę. -- Kiedy zobaczyłem, gdzie zmierzają
sprawy, wezwałem go.

-- \emph{Co} zrobiłeś! -- eksplodował Reid. Rozejrzał się dookoła,
odwołując się do towarzyszy i~do unoszących się robotów sieci
wiadomości, teraz ponownie dryfujących do środka sądu. -- Dlaczego w~imię
Boga to zrobiłeś?

Brama zamknęła się z~terkoczącą ostatecznością. Talgarth odwrócił się od
niej i~zrelaksował, spojrzał Reidowi w~oczy.

-- Spytałeś, tam wtedy, czy moja pamięć jest krótka -- powiedział. -- Mniemam, że pytanie retoryczne, ale mimo tego. -- Bardzo nieśpiesznie
zapalił papierosa i~dmuchnął dymem z~objawami całkowitej satysfakcji. -- Nie jest.

~

Nawet po tym, jak zostawili resztę zwolenników Wilde'a, których Ethan
Miller był pewien, że może przeprowadzić bez większej trudności do
ludzkiej dzielnicy, jest tłoczno z~tyłu ciężarówki Jay-Duba. Jest to
bardziej ładownia niż przestrzeń pasażerska, choć miała pewne podstawowe
zabezpieczenia dla ludzkiej obecności. Ax jest wciśnięty w~swoje miejsce
na podłodze koło łącza telewizyjnego, Dee i~Jonathan Wilde siedzą na
wyściełanej ławce, na której Dee wcześniej leżała, Tamara jest przypięta
do jednego z~wielkich haków wiszących z~sufitu.

Prędkość pojazdu jest pełzaniem. Przebijają się przez Piątą Dzielnicę z~rykiem radiowych i~dźwiękowych syren, i~niewielkim względem dla
czegokolwiek co pozostaje na drodze. Roboty i~inne, mniej określone
maszyny, rozpraszają się przed nimi. Ekrany są całkowicie przekierowane
na wyświetlanie otoczenia i~są pełne alarmujących widoków.

Dee rzuca okiem na Wilde'a i~na inną wersję Wilde'a w~złudnej kabinie.
Jej oczy spotykają Wilde'a patrzącego ze zdumieniem ze starszego Wilde'a
na nią. Uśmiecha się niepewnie.

-- Widzę duchy -- mówi. -- Jesteś\ldots teraz to dziwne, móc patrzeć na
Ciebie. -- Śmieje się krótko. -- Nie uciekającą. Wiem, że nie jesteś
Annette, ale\ldots nie masz nic przeciwko patrzeniu na Ciebie, dobra?

-- Ok -- mówi. -- Rozumiem.

Jego uśmiech zamienia się w~spojrzenie prywatnego zaintrygowania.

-- Kim jest ta kobieta tam z~przodu z\ldots Jay-Dubem?

-- Nazywa się Meg -- szepcze Dee -- i~właściwie nie jest kobietą.

Meg się odwraca. 

-- Słyszałam to -- mówi nad ramieniem. -- Nie wierz jej.
Jestem tak samo kobietą jak ona, Jon.

-- Ona jest szybką kobietą -- krzyczy do tyłu drugi Wilde.

Ax obserwuje to nieco kazirodcze przekomarzanie i~patrzy na Tamarę
przewracając pogardliwie oczami. Tamara dostrzega to i~patrzy w~bok od
Wilde'a i~Dee, z~czymś jak winny początek. Ax wzdycha i~wraca do
przerzucania wiadomości.

-- Jak daleko jeszcze mamy? -- pyta Wilde. -- Talgarth nie może trzymać
długo Reida i~jego ludzi, prawda?

-- Nie -- odpowiada Ax, łamiąc znowu swój trans. -- Reid wezwie posiłki,
odwoła się do innych sądów, i~ogólnie wznieci smród. Myślę, że Talgarth
pozwoli mu odejść w~ciągu pół godziny.

-- A potem ruszy za nami?

Jay-Dub wzrusza ramionami, zdejmując ręce z~najwidoczniej kierownicy,
żeby pomachać nimi w~sposób, którego Dee nie może postrzec inaczej niż
jako niebezpieczny, choć wie, że tak nie jest. 

-- Już nas ściga -- mówi. -- On lub jego agencje obrony, mają jeden lub dwa samoloty i~przynajmniej przydział czasu na satelicie rozpoznawczym, i~mają nas na celu, jeżeli
nie na celowniku. Wątpię, czy podejmie jakiekolwiek działania, póki nie
ustali, w~którą stronę polecą prawne lub politycznie wióry. Chyba że\ldots

Jego uwaga jest odwrócona przez konieczność przebicia zapory.

-- Trzymajcie się!

Ciężarówka zwalnia, przechyla się, prawie przeskakuje nad płonącym
stosie śmieci rozrzuconym w~poprzek drogi.

-- Chyba że, co? -- przypomina Dee, gdy dochodzi do siebie po wstrząsie.

-- Chyba że odkryje, że jesteś ze mną -- mówi Jay-Dub. -- Pamiętasz tych
łowców nagród, którzy po Ciebie szli? Dość mocno się poparzyli, ale
przeżyli i~doszli do siebie. -- Uśmiecha się nad ramieniem do Wilde'a lub
do Dee. Nie jest pewna, kto jest tym razem celem ironii. -- Zadziwiające,
co medycyna potrafi w~tych czasach. Jak tylko ochłonęli z~szoku i~mieli
dostatecznie dużo odrośniętej twarzy, żeby gadać, zaczęli gadać. O
uciekinierach uratowanych przez robota.

Wilde marszczy brwi dookoła towarzystwa. Dee już rozumie, ale nie może
jeszcze powiedzieć innym.

-- Co zatem zrobi Reid? -- pyta Wilde.

Jay-Dub znowu uważa na kierownicę, z~konieczności lub wyboru.

-- Zniszczy nas -- mówi. -- Z~tym, co trzeba, za ile to kosztuje. Więc, nie będziemy, jak to mówią, owijać w~bawełnę.

Ciągnik zanurza się w~wilgotny tunel pod kanałem, na dalekim krańcu
Piątej Dzielnicy. Zatrzymuje się, silnik tętni, na dostatecznie długo,
żeby Dee, Ax, Tamara i~Wilde wysiedli. Dee jest ostatnia na wyjściu.
Właz z~boku ładowni się otwiera i~jedna z~małych pełzających maszyn
wytacza się i~podaje jej zapieczętowaną, plastikową skrzynkę. Wsuwa ją
do torby.

-- Do zobaczenia -- mówi Meg.

-- Do zobaczenia -- mówi Jay-Dub, starszy Wilde. Zauważa jej łzy, uśmiecha
się i~mocno mruga.

-- Nie jest tak źle -- mówi. -- Byłem tam i~nie ma się czego bać.

Dee potyka się. Klapa tylna zasuwa się i~ciągnik oddala się coraz
szybciej, pędząc na drugi koniec tunelu tak szybko, że, z~góry, nikt nie
potrafiłby powiedzieć, że w~ogóle się zatrzymywał.

Gdy echa jego przejazdu milkną, Dee widzi wysokie, prawie ludzkie
postacie pojawiające się z~zacienionych boków tunelu. Ich ciała słabo
odbijają wyblakłe, generowane izotopami światła. Tamara i~Ax napinają
się, ich bronie się jeżą. Wilde popadł w~tępy stoicyzm, lub opóźniony
szok, i~obserwuje ich podejście bez widocznej reakcji. Po wszystkim,
przez co przeszedł, cicho nadchodzące humanoidalne roboty to zbyt wiele
-- lub zbyt mało -- do zaakceptowania.

-- Wszystko w~porządku -- mówi prędko Dee. -- Wilde, znaczy Jay-Dub,
powiedział mi o~nich. To przyjaciele.

Roboty zbierają się dookoła ludzi i~pchają się, przyglądają się bacznie
z niepokojąco ludzką ciekawością.

-- Jeżeli jesteście przyjaciółmi Jay-Duba -- mówi jeden z~nich dumnie, w~rezonującym wysokiej jakości głosie -- jesteście naszymi przyjaciółmi. -- Oczy na ich owalnych twarzach jaśnieją. -- Mamy niewielu przyjaciół.
Ludzie tutaj nas nie akceptują, a~dzikie maszyny\ldots

Ich ramiona mają dostatecznie dużo ludzkiej artykulacji, żeby
przedstawić podobieństwo wzruszenia ramionami.

-- Czekacie z~nami -- sugeruje. Ich oczy znowu rozjaśniają. -- Mamy
jedzenie.

~

Humanoidalne roboty, pozostałości błędnych decyzji produkcyjnych,
dziesięciolecia temu, rzeczywiście mają jedzenie, schowane w~bocznych
odnogach tunelu. Ich cel w~zbieraniu tych puszek i~słoików jest niejasny
tak jak ich aktywność. Same dla siebie wydobywają swoje podtrzymanie z~elektrycznego kabla zasilającego, który przebiega przez tunel. Dee
podejrzewa, że rozwinęli coś, co niektórzy ludzie kiedyś uważali za
definiującą cechę ludzkości: religię.

Wierzą, wbrew wszystkim dowodom, że zostali stworzeni przez pierwszego
człowieka, Adama, który był kowalem. Ich pismami to teksty dziecięce
dawnej chwale Ziemi, nieco dokładniejsze niż bajki, którymi Opowiadanie
karmi Dee. Mówią o~dziwnym wniebowzięciu, Rewolucji Przemysłowej, i~czczą pośrednika pomiędzy Człowiekiem a~Maszyną, Robota, który był i~jest mężczyzną, Jay-Duba.

Gdy ludzie akceptują ich gościnność, słuchają robotów tłumaczących ich
wiarę, i~śpiewających pieśni. Pieśni są niemal niezrozumiałe. Ax nazywa
je starymi androidowymi spiritualsami, Wilde nalega, że to stare hity
heavy-metalowe.

Dee jest prawie sparaliżowana myślą, że roboty mogłyby powiązać Wilde'a
i Jay-Duba, których najwidoczniej widzieli w~różnych momentach przez
lata zarówno jako robota, jak i~sobowtór w~telewizji lub hologramie.
Szczęśliwie, ich rozpoznawanie wzorców jest słabe. Ich umysły są
prawdziwymi, jeżeli surowymi, sztucznymi inteligencjami, a~nie (jak jej
jest) tanią kopią z~ludzkiego wzorca.

Są również naiwne w~rozpoznawaniu ludzkich emocji i~nie pokazują śladu
dotknięcia ciągłą czujnością i~mamrotanymi konsultacjami ludzi. Są
zajęte samymi sobą przy ostatnim zadaniu, jakie zlecił im Jay-Dub:
przyciąganiu rozczłonkowanych części robotów humanoidalnych i~składaniu
ich w~imitację garnituru do noszenia przez ludzi. Wydają się cieszyć
zadaniem, mierząc ludzi i~dopasowując metalową zbroję do ich ciała. Dee
nie waży się zapytać, czy ich pancerze są pozostałościami martwych
robotów, czy częściami zamiennymi, czy efektami własnych prób robotów,
żeby wytworzyć własny rodzaj. Koncentruje się na sprawdzeniu, czy stawy
nie łapią jej skóry.

Wilde, Tamara i~Ax śmieją się z~nią, gdy dopasowują zbroję i~ćwiczą w~niej chodzenie. To tylko odwrócenie uwagi i~wiedzą o~tym. Wszyscy wiedzą
na co czekają i, choć wydaje się to długo, mają tylko kilka godzin do
przeczekania.

Eksplozja jest bardzo daleko stąd, i~mała, jak takie eksplozje
wyglądają, i~jednak ciągle wypełnia tunel białym światłem. Żołnierz nie
potrafi określić, czy to była taktyczna bomba jądrowa wycelowana z~zewnątrz w~ciężarówkę, czy samodzielnie zbudowane urządzenie zdetonowane
od wewnątrz, żeby uniknąć schwytania. W~każdym przypadku była to
autodestrukcja.

-- Och, Jay-Dub -- mówi Dee. -- Och, Meg. To było takie odważne.

Huk pierwszej fali mija. Część dachu tunelu się zapada\ldots

-- Nigdy bym tak nie zrobił -- mówi Wilde. Jego mina pokazuje więcej szoku
niż smutku. -- Cokolwiek było w~tej ciężarówce, nie było mną.

\chapter{Mila Malleya}
Brak poczucia przeminięcia czasu. Żadnego białego światła, żadnego
doświadczenia śmierci. W~jednej chwili leżałem na plecach, ciepło i~krew
z mojego ciała topiące zimny śnieg, kolory znikające. W~następnej\ldots

~

\ldots siedziałem wyprostowany i~zupełnie nagi na łóżku, obrócony na szerokie
okno. Okno było prostokątem zupełnej czerni podzielonej poziomo białym
pasem, który sam był podzielony czarnymi liniami różnej grubości. Czułem
się dokładnie tak, jakbym był obudzony przez syreny obrony cywilnej. A
jednak pokój był cichy, oprócz odległego szumu, który założyłem, że jest
wentylacją, ale który równie dobrze mógłby być wiatrem w~drzewach.
Powietrze nie przenosiło żadnego zanikającego echa, i~żaden dźwięk nie
dzwonił mi w~uszach.

Nie miałem czasu zastanawiać się, gdzie jestem, ponieważ na zewnątrz
okna, kierując się w~moim kierunku, była skała. Koziołkowała koniec nad
końcem ze zwodniczą powolnością i~jej rzeczywista wielkość na tle
czarnego tła i~białych pasów zwiększała się tak szybko, że wiedziałem,
że uderzy w~okno za sekundy.

Skała upadała w~moim kierunku pomiędzy dwoma potężnymi połączonymi
konstrukcjami -- jak ramiona zrobione z~kratownic -- które rozszerzały się
na zewnątrz z~pozycji po obu stronach okna. Pomiędzy mną a~oknem stała
pusta rama z~siatki, w~formie konturu człowieka z~rozdzielonymi stopami
i rozłożonymi ramionami, jak odcisk pozostawiony przez postać z~kreskówki uderzającą w~ogrodzenie z~drutu i~potem odpadającą.

Wiedziałem co robić i~nie zastanawiałem się, skąd wiedziałem co robić.
Skoczyłem z~łóżka i~rzuciłem się na ramę. Przycisnęła się do mojej skóry
i na oczy.

Wszystko się zmieniło. Okno było całym moim widzeniem, a~ramiona na
zewnątrz były moimi ramionami. Skała wydawała się mniej niż pół metra od
mojej twarzy i~teraz dryfowała, nie pędziła, w~jej kierunku. Włożyłem
ręce dookoła niej i~złapałem ją tak łatwo jak piłkę plażową.

Prócz tego, że teraz poruszałem się do tyłu.

Odepchnąłem ją, ciągle trzymając, i~odwróciłem się, żeby spojrzeć za
mnie. Ściana, podzielona na paski i~spirale czerwieni, pomarańczy,
żółtego i~białego, zajmowała cały widok, a~pomiędzy mną a~nią znajdował
się rój czarnych kropek i~wielka sieć czarnych linii. W~tej samej
chwili, ściana rozwiązała się w~część powierzchni sferycznej,
zakrzywiającej się we wszystkich kierunkach aż do rozmytej krawędzi na
tle czarnego kosmosu, i~uświadomiłem sobie, że poruszałem się, spadałem,
w jej kierunku.

Walczyłem, żeby przestać spadać. Czułem zsuwanie, pełzanie i~próbowanie
znalezienia oparcia na stopy, a~potem odnalezienia oparcia, piętami stóp
wkopywanie się. W~dolnej części mojego widoku, krótki impuls światła i~wiązka mgły pojawiły się i~znikły.

Potem znalazłem się z~powrotem w~pokoju, stojąc w~ramie z~siatki w~dłoniach przed moją twarzą. Na zewnątrz okna, wielkie ramiona ciągle
trzymały skałę. Mogłem zobaczyć światło i~cień jej dziobatej
powierzchni, czarne palce jak odnóża owadów.

Oderwałem się od ramy, odsunąłem się na krok i~usiadłem na łóżku. Rama
stała jak druciana rzeźba. Powoli znów rozłożyła swoje ramiona. To była
jedna cholerna zaawansowana platforma teleobecności, pomyślałem. Kiedy w~niej byłem, czułem się, jakbym był całym\ldots statkiem?\ldots byłem w~\emph{moim} ciele. Detal o~kontroli rakiet będącej subiektywnym
ekwiwalentem moich stóp uderzył mnie jako szczególnie zgrabny. Jednak
nie czułem przyśpieszenia, kiedy rakieta się uruchomiła. Rozważałem tę
anomalię, gdy rozglądałem się i~próbowałem oszacować moją sytuację.

Pierwsze, moje ciało. O ile mogłem rozpoznać, było takie, jak je
zapamiętałem, chude, pomarszczone i~stare, ale, jak mówią, dobrze
zachowane: raczej jak te ciała z~epoki brązu odnalezione w~torfowiskach.
Pięć gałek tkanki bliznowej tworzyło przekątną przez moją pierś.
Dotknąłem ich zamyślony.

Pokój miał około czterech metrów od tylnej ściany do okna, pięć metrów w~poprzek, dwa i~pół metra wysokości. Łóżko było proste, podwójne sosnowe
łóżko z~bawełnianą pościelą i~kołdrą. Okno zajmowało całą jedną ścianę.
Inne ściany były matowo białe. Podłoga była pokryta jasnobrązowym
dywanem. Po mojej prawej było drewniane krzesło i~stół z~ekranem i~datapadem. Po mojej lewej, wysoka szafka.

A w~lewej ścianie, drzwi.

Wstałem, obszedłem łóżko i~otworzyłem szafę. Dżinsy wisiały z~szyny,
starannie złożone stosy podkoszulek, bielizny i~skarpetek piętrzyły się
na półkach. Kilka identycznych par trampek leżało na dnie.

Ubrałem się, a~po chwili wahania, otworzyłem drzwi, żeby znaleźć
dostatecznie banalnie, łazienkę: prysznic, ustęp i~umywalkę. Przez
kolejne drzwi, małą kuchnię, która z~kolei otwierała się na hol mniej
więcej tej samej wielkości co pierwszy pokój. Miał sofę zamiast łóżka, w~jednym rogu ekran telewizora. Ściana skierowana ku sofie była kolejnym
oknem, a~pomiędzy sofą a~oknem stała kolejna forma z~siatki w~kształcie
człowieka. Przypuszczalnie mogłem skoczyć z~sofy i~wpaść w~nią, jeżeli
zbliżająca skała lub inna sytuacja awaryjna zwróciłaby moją uwagę.
Wróciłem do sypialni.

Może wydawać się zaskakujące, że zacząłem od badania tego, co było
natychmiast pod ręką, i~nie rzuciłem się w~kalkulowanie, gdzie byłem.
Mniemam, że próbowałem o~tym nie myśleć, próbując wydobyć każdą kroplę
uspokojenia, którą najwyraźniej dawała każda najwidoczniej normalna
cecha mojego dziwnego, zaprojektowanego środowiska.

Nienormalne cechy w~ogóle nie były uspokajające. Usiadłem i~wpatrzyłem
się w~przezroczystą ścianę. Kulista płaszczyzna na zewnątrz była
planetą, a~jedyną planetą, którą mogła być -- zakładając, że jestem w~Układzie Słonecznym -- był Jowisz. Białe pasy z~ciemnymi czarnymi liniami
były, gdy mój pojazd się odwrócił i~jego ramiona odrzuciły skalę, coraz
bardziej oczywistą częścią ogromnego pierścienia.

Pierścienie Jowisza: \emph{było} coś wystarczająco nadzwyczajnego w~tej
implikacji, ale to było nic wobec faktu, że chodziłem. Nie było dowodu,
że jestem przyśpieszany, żadnego poczucia ruchu, kiedy widok na zewnątrz
okna się zatoczył. To, że pojazd użył rakiet, było wystarczającym
dowodem, że żadna forma sterowania grawitacją nie była użyta: jeżeli
kontrolujesz grawitację, masz napęd gwiezdny w~okazji, i~na pewno nie
pierdzisz dookoła rakietami.

Jedno strasznie rozsądne wytłumaczenie, gdy siedziałem z~moją głową w~rękach (ha!) było takie, że prawdziwa wirtualna rzeczywistość nie była
teleobecnością, której doświadczyłem w~ramie. Ta teleobecność mogła być
prawdziwa, pokoje, ciało, w~którym się znalazłem, fikcją. Moje prawdziwe
ciało mogłoby być na samym statku, a~to, co ja doświadczałem ,,wewnątrz''
niego, było symulacją, kierowaną przez komputer statku.

Była także możliwość, że było zupełnie inaczej, że moje ciało i~pokój
były prawdziwe, a~to, co było na zewnątrz, było symulacją. (Lub
prawdziwą teleobecnością, próbowałem sobie przypomnieć, czy jakiekolwiek
księżyce Jowisza miały podobną masę do Ziemi. Lub raczej, może, byłem na
statku lub stacji kosmicznej obracającej się, żeby dać 1 gie wagi\ldots).
Może było tak, że obudziłem się ze zwykłej amnezji, że nie umarłem w~kazachskim śniegu, ale wyleczyłem, i~pracowałem przez lata na
najwyraźniej gigantycznym projekcie?

Lub, oczywiście, mogłem w~ogóle nie być w~kosmosie! Cała sytuacja mogła
być równie dobrze jakimś treningiem stacji w~VR na Ziemi! Pewnie, z~wszystkich możliwości, ta była tą, która brzytwa Ockhama\footnote{brzytwa
Ockhama -- zasada, zgodnie z~którą w~wyjaśnianiu zjawisk należy dążyć do
prostoty, wybierając takie wyjaśnienia, które opierają się na jak
najmniejszej liczbie pojęć i~założeń,
zob.~\url{https://pl.wikipedia.org/wiki/Brzytwa_Ockhama } -- przyp.tłum. } ścięła najmniej. Przewrotnie, to była ta, o~której
pomyślałem jako ostatniej, może dlatego, że nie odważyłem się mieć
nadziei, że jest poprawna.

Jednak postawiło mnie to na nogi. Podszedłem do stołu i~spojrzałem w~komputer: płaski ekran, płaski pad, wszystko zwyczajne.

Wszystko wyłączone. Cholera.

~

Wszedłem w~ramę raz jeszcze. Raz jeszcze, z~twarzą przyciśniętą do
metalowej siatki, mój punkt widzenia stał się jednym z~maszyną.
Poruszyłem ramionami ramy, ale ramiona statku nie poruszyły się z~nimi.
Domyśliłem się, że miałem sterowanie tylko w~określonych
okolicznościach. Więc wisiałem tam przez chwilę i~przyglądałem się
scenie.

Jowisz wisiał nade mną. Poruszałem się szybko w~kierunku roju czarnych
kropek dookoła czarnej struktury. Po kolejnym impulsie rakiety, tym
razem z~przodu i~znowu bez poczucia zmiany prędkości, zwolniłem i~zdryfowałem w~rój. Gdy mijałem innej lecące maszyny, byłem w~stanie
obejrzeć ich kształt i~wywnioskować własny kształt:

Cylindryczny, tamte miały ramiona w~środkowej części, która wydawała się
zdolna dzielić się na przeguby i~wydłużać w~dowolnym kierunku. ,,Ręce''
jak krzaki, palce powtarzalnie dzielące się. Tułów pokryty soczewkami,
dyszami, antenami i~włazami. Cztery krótsze, mocniejsze kończyny do
chwytania i~mocowania. Wszystko (prócz soczewek) zrobione z~czarnej
matowej substancji, która nie wyglądała na metaliczną i~która zwykle
była poplamiona i~porysowana. Maszyny orientowały się rakietami (roboty
z kontrolą postawy, pomyślałem z~uśmiechem) i~pracowały w~niesamowitej,
cichej harmonii nad czymś, co wyglądało jak, dla mnie wtedy, największa
stacja kosmiczna, jaką kiedykolwiek zbudowano. Jeżeli roboty były mniej
więcej ludzkiej wielkości, to struktura musiała mieć dziesiątki
kilometrów w~poprzek.

Pamiętałem wczesne eksperymenty z~pająkami w~kosmosie, pająkami na
narkotykach. To, co widziałem, mogło być wyobrażone jako praca miliona
halucynujących pająków w~nieważkości. Dookoła tego czarne roboty
poruszały się w~ich newtonowskim balecie, a~w~obrębie włókien inne
rzeczy ruszały się z~łatwiejszym wdziękiem. Ich liczne i~wielokolorowe
formy przypominały komputerowe renderowanie równań chaosu, matematyczne
potwory, których zewnętrzne powierzchnie fraktalne biły i~mrugały jak
rzęsy mikroorganizmów w~kropli wody.

Już myślałem o~nich jako o~wrogu.

~

Maszyna, którą zamieszkiwałem, wpłynęła w~wielką sieć, dołączyła do
sekcji jednego z~włókien i~zaczęła pracować najmniejszymi palcami ich
palców (powinienem powiedzieć, dziesiątkami po przecinku?) nad czymś w~węźle kilku żył. Przedmiot jej trudu był poza rozdzielczością mojego
obecnego wzroku. Odłączyłem się od Ramy i~się odsunąłem. Przez okno
mogłem zobaczyć, że wszystko przyśpieszyło, palce rozmazanym ruchem,
kształty w~sieci unoszące się i~latające.

Poszedłem do kuchni. Kran odkręcony, kawa zagotowana. Słoik na kawę był
opisany ,,Nescafe'', a~jego zawartość smakowała lepiej, niż pamiętałem.
Zapalniczka i~otwarta paczka Silk Cutów leżała na krawędzi koło zlewu.
Ciepło płomienia, wirujące fale dymu, uderzenie nikotyny były tak dobre
jak w~rzeczywistości.

Zaciągnąłem się mocno i~wypuściłem z~radością, która miała pewną
nieprzyzwyczajoną czystość. Jedna rzecz może być powiedziana o~śmierci:
nie musisz się martwić o~zdrowie. Zastanawiałem się, co by się zdarzyło,
gdybym zaczął niszczyć wszystko w~zasięgu wzroku, w~tym siebie. Kiedyś,
kiedy miałem trzynaście lat i~czytałem podstępne spekulacje Biskupa
Berkeleya\footnote{George Berkeley -- filozof znany
m.in. z~propozycji solipsyzmu oraz z~hipotezy, że przedmioty istnieją,
ponieważ są obserwowane przez Boga,
więcej~\url{https://pl.wikipedia.org/wiki/George_Berkeley_(filozof)
} -- przyp.tłum.}, stworzyłem szalone sposób przetestowania tego, zeskrobania
powierzchni świata, żeby odkryć uśmiechniętą czaszkę Boga\ldots tutaj, ta
niepoczytalność mogła być możliwa, czy symulacja rozszerzała się do
wnętrza rzeczy, do wnętrza mnie? \ldots Ale nie dbałem o~eksperymenty. Intelektualnie, nie
miałem problemu z~zaakceptowaniem możliwości, że jestem symulacją,
transfer umysłu, uploading, był omawiany dostatecznie długo i~wydawał
się nieuniknioną konsekwencją zaawansowanej technologii, o~której mówiła
Myra. Nanotechnologia i~silna AI mogłyby naśladować ludzki umysł, nigdy
w to nie wątpiłem.

Akceptacja emocjonalna była czymś innym.

Zabrałem kawę i~papierosy do salonu i~usiadłem na sofie. Po chwili
szukania, znalazłem pilot do telewizora, leżącego w~rogu pokoju.
Rozsiadłem się znowu i~włączyłem pierwszy kanał. Kiedy zobaczyłem, co
się pojawiło, prawie upuściłem kawę.

Twarz, która pojawiła się na ekranie, należała do Reida. Wyglądał
fizycznie młodziej niż ostatnim razem, kiedy go widziałem, znaczy
ostatni raz kiedy (rzeczywiście) \emph{cokolwiek} widziałem, ale duchowo
starszy. Nie istnieje inny sposób opisania tego. Cała jego mina
przekazywała ciężko zdobytą mądrość i~doświadczenie, które mogłoby być
wstrząsające u jakiegoś wiekowego mędrca, i~było podwojone na znajomej
chudej twarzy jego młodszej osoby.

-- To jest nagranie -- powiedział i~uśmiechnął. Pomachał rękę na pokój, w~którym siedziałem. -- A to też jest, jak zapewne już podejrzewasz. Fakt,
że to oglądasz to, oznacza, że wróciłeś do świadomości. \emph{Video,
ergo sum}, lub coś, nieważne, witamy z~powrotem. Nie ma zbyt dużo zabawy
w płaskiej linii, czym byłeś aż dotąd. Działałeś na programie, nawyku i~refleksie: wirtualne zombie, można powiedzieć, a~teraz jakieś
nieprzewidywalne, ale nieuniknione połączenie okoliczności przebudziło
cię.

Przerwał. 

-- Jeżeli nie rozumiesz, co mówię, lub uważasz to za
wstrząsające, proszę, włącz kanał numer dwa.

Nie poruszyłem się.

-- Dobrze -- kontynuował Reid. -- Wiedziałem, że masz to w~sobie, musiałeś
być całkiem poczytalny i~twardy, żeby dać zamrozić głowę, lub zeskanować
mózg, lub cokolwiek, co zrobiłeś, żeby tu skończyć. Więc wyłożę wszystko
prosto.

-- Data to\ldots -- (niewielka przerwa, usterka oprogramowania) -- \ldots 3 marca
2093 roku. To może wydawać się niespodzianką, jeżeli nie domyśliłeś się,
co się dzieje\ldots pewnie, myślisz, nie tak szybko? Witamy w~Osobliwości.
To, co widzisz na zewnątrz, jest pracą miliardów świadomych istot,
żyjących i~myślących tysiąc razy szybciej niż Ty. Byty pełzające
pomiędzy rozporami tej struktury są całymi cywilizacjami następców
ludzkości. Te makroorganizmy, lub makro, jak ludzie dookoła tutaj je
nazywają, są konstelacjami inteligentnej materii -- myśmy to nazywali
nanotechem -- każda z~nich umożliwia podtrzymanie rzeczywistości
wirtualnych, które są domami milionów umysłów, niektórych oryginalnie
ludzkich, niektórych sztucznych. Każdy z~tych umysłów doświadcza
symulacji, dzielonej lub prywatnej, światów poza naszymi najdzikszymi
snami. Każdy ma możliwość wzmocnienia swoich zdolności znacznie poza
cokolwiek, co postrzegamy jako ludzkie, oraz ma okazję zrobić to w~precyzyjnym stopniu do możliwości dobrego wykorzystania istniejących
zasobów.

-- A wielu z~nich było kiedyś tacy jak Ty! Zwykła ludzka istota, której
mózg został nagrany, neuron po neuronie, synapsa po synapsie w~przenikającej macierzy inteligentnej materii. Nagrany, zreplikowany, i~uruchomiony z~sukcesem na lepszym hardwarze, który w~tym momencie możesz
docenić.

Roześmiał się. Coś w~jego tonie zmroziło mnie, cynizm tak głęboki i~dojrzały jak to zapatrywanie zwykle jest płytkie i~mdłe.

-- Może zastanawiasz się, dlaczego nie jestem pośród nich. Oczywiście,
nie masz powodów zakładać, że nie jestem. Jednak, jak to się zdarza, nie
jestem. Również możesz się zastanawiać, co robisz, nawiedzając komputer
na pokładzie robota konserwatorskiego stworzonego nie z~inteligentnej
materii, ale z~tego, co nazywamy obecnie ,,głupią masą''.

-- Odpowiedź, po mojej stronie, jest skomplikowana. Po Twojej jest
prosta. Jesteś wśród Nieożywionych. Tak, mój przyjacielu z~głupiej masy,
co najmniej jedna kopia Twojej dobrej jaźni jest zakodowana w~kilku
centymetrach sześciennych inteligentnej materii, oczekując przyszłego
wskrzeszenia w~lepszym miejscu. To należy do Ciebie, do prawdziwego
Ciebie. Dotrzymujemy naszej części umowy. Niemniej kopia, którą teraz
jesteś, należy, na razie, do nas.

Chłodny uśmiech.

-- Kolejne pytanie -- kontynuował Reid. -- Dlaczego? Cóż, dla tych, którzy
nie byli przy umowie, lub nie pamiętają: kilka lat temu, kiedy to
wszystko było organizowane, nie mieliśmy czasu ani zasobów, żeby
rozwinąć AI, które byłyby dostatecznie inteligentne do budowy stacji,
ale nie na tyle inteligentne, żeby robić problemy. Wyprodukowanie kopii
skopiowanych umysłów ludzkich i~uruchomienie ich na przedświadomym
poziomie integracji było najszybszą i~najtańszą drogą do software'u dla
naszych robotów konstrukcyjnych. Szybko odkryliśmy, że te umysły, wy, nieprzewidywalnie stawałyby się zintegrowane po zmiennym okresie pracy.
Budzili się, a~potem mieli tendencję do załamania, nic zaskakującego.
Więc zapewniliśmy przyjemne rzeczywistości wirtualne jako stan
gotowości, stąd nie czujesz, że zostałeś zamieniony w~robota.

-- Ale, czy lubisz to, czy nie, na razie w~tym utknąłeś. Jak idealny
Socjalista Gueveary, jesteś ,,trybem w~maszynie, ale świadomym trybem''.
Jednak, w~odróżnieniu od Socjalisty, masz pewne osobiste zachęty,
chociaż czy oni by nazwali je zachętami \emph{materialnymi} podlega
debacie. Jeżeli zdecydujesz się wykorzystać maksymalnie tę sytuację,
będziesz opłacany w~narastająco ulepszanych i~przyjemnych wirtualnych
rzeczywistościach, rozszerzeniach Twoich możliwości umysłowych i~tak
dalej, aż do momentu, kiedy będziesz gotowy przenieść się na stałe do
makra przy zwolnieniu, jeżeli tego właśnie chcesz. To będzie jak śmierć
i pójście do nieba. Lub, jeżeli wolisz, możesz zostać wskrzeszony w~ludzkim ciele, kiedy nadejdzie czas.

-- Jeżeli nie akceptujesz żadnej opcji\ldots cóż, znajdziesz instrukcje na
komputerze w~drugim pokoju. Zadziałają, teraz kiedy widziałeś ten, hm,
pakiet orientacyjny. Przeniesie Cię do tego miejsca, gdzie byłeś, zanim
się przebudziłeś. Stracisz godzinę lub dwie doświadczenia, to wszystko.
Następnym razem, kiedy się obudzisz, nie będziesz pamiętał tego i~może
będziesz mógł sobie lepiej z~tym poradzić\ldots Jednak, znowu, może nie. To
zależy od Ciebie.

Obraz Reid uśmiechnął się niedorzecznie wesoło i~zniknął, żeby zostać
zastąpiony przez fotografię obracającej się planety i~wiadomość:

\emph{Dla dalszych informacji, wybierz znowu kanał numer jeden.}

Usiadłem i~przez chwilę myślałem.

~

Wiadomość nic nie zmieniła. Nie było sposobu, żeby przesądzić, które,
jeżeli w~ogóle, moje spekulacje o~moich doświadczeniach są prawdziwe.
Wszystko, co wiedziałem, to, że część mojego środowiska było symulacją,
i że ktoś chciał, żebym uwierzył, że była to część, która, w~codziennym
doświadczeniu, byłaby bez namysłu uznana za prawdziwą. Zacząłem
rozumieć, dlaczego Kartezjusz przywołał Diabła, żeby opisać podobnym
eksperyment myślowy: ktokolwiek to zrobił, nie chciał mojego dobra.

Zakładając, że wiadomość była prawdziwa na jej własnych warunkach, było
oczywiste, że Reid nie przemawiał do mnie osobiście. Dla niego musiałem
być zagubiony w~roju. (I jak wiele tych rojących się robotów działało na
kopiach mnie? Było coś nieskończenie depresyjnego w~tej myśli. Spadku
cen duszy, gdy krzywa podaży poszła w~górę, a~koszty produkcji spadły).

Nie mówił też nic o~Ziemi: pominięcie, które podejrzewałem, było celowe.
Czterdzieści siedem lat minęło od mojej przypuszczalnej śmierci. \emph{Nawet
śmierć może umrzeć wraz z~dziwnymi eonami\footnote{ cytat z~,,Zew Cthulhu''
H.P. Lovecraft -- przyp.tłum.}.} Nie było powodu -- teraz gdy dziwne eony w~końcu były przed nami -- żeby zakładać śmierć Annette, lub kogokolwiek, w~takich czasach.

Jednak cisza Reida, na pytanie, które musiało się pojawić każdemu, kto
się tu pojawił, była złowroga.

Wróciłem do sypialni. Tak jak człowiek w~telewizorze powiedział,
komputer teraz pracował. Przesunąłem palcem pod datapadzie, szukając po
ikonach ekranu. Dziwne było używanie tak prostego interfejsu, ale miało
sens: wirtualna rzeczywistość w~obrębie wirtualnej rzeczywistości
zawierałaby ryzyko rekurencji, w~której już napięte połączenie pomiędzy
umysłem a~jego otoczeniem mogłoby puścić. Znalazłem ikonę, która była
małym, obracającym się widokiem Ziemi i~w nią stuknąłem.

To był kolejny pakiet orientacyjny, pokazujący raczej niż mówiący, w~jaki sposób powstało to niebiańskie jowiszowe miasto.

Obawy Myra się spełniły.

Obrazy satelitów rozpoznawczych, oczywiście edytowane, były opisane jako
,,w czasie rzeczywistym''. Pokazywały miasta ukryte, po raz pierwszy od
dekad, pod smogiem. Kilka zbliżeń pokazało źródło zanieczyszczeń:
kominki i~paleniska. Jednak dużo drzew na ulicach. Zieloni byliby
szczęśliwi. Na Trafalgar Square, koń, skadrowany przy upadłym Nelsonie,
spojrzał do góry, potrząsnął grzywą, jakby świadom, że jest obserwowany.
Wiosna przyszła późno do Europy, śnieg czaił się w~cieniu.

Odsuwając się, osady na Lagrange ciemne, otoczone wyciekami gazu i~śmieciem kosmicznym, Księżyc ciemny, Mars cichy. Szyfrowana rozmowa z~Pasa Asteroidów, która podniosła tętno na chwilę.

A potem, w~rozległym kontraście, Projekt Jowisz. Jego historia była
opowiedziana w~błyszczących multimediach, pakiecie reklamowym lub
propagandzie, która przypominała mi ten rodzaj rzeczy, które zwykle
publikowały przedsiębiorstwa elektrowni jądrowych. Zamach Ruchu
Kosmicznego, opowiedziany jako heroiczny ostatni bastion przed
barbarzyńskim tłumem i~represyjnymi rządami. Wykładniczy wzrost długo
ograniczanych zaawansowanych technologii, które dostarczyły wszystko, co
kiedykolwiek obiecywały: tanie loty kosmiczne, kompletną kontrolę nad
materią, aż do poziomu molekularnego, zanik starzenia się i~śmierci, i~ostatecznie kopiowanie umysłów z~mózgów do maszyn. Wszystko dostępne
tylko dla mniejszości, niestety, jak byłoby to w~każdym przypadku, ale
pogorszone przez zrozumiały strach większości przed
najniebezpieczniejszymi technologiami kiedykolwiek rozwiniętymi, i~przez
wkraczający chaos, którego początki sam widziałem. Desperacki lot z~upadającej cywilizacji Ziemi, napędzany przez pracę dziesiątek tysięcy
więźniów -- każdemu obiecana, i~dana, własna kopia, która przetrwała
jakikolwiek los, na jaki trafili -- i~zorganizowana przez tysiące
ochotników i~kadry ruchu kosmicznego.

Następnie -- temat przeskoczył tak szybko, że wiedziałem, że coś jest
ukrywane -- nastąpił rozłam pomiędzy systemem wewnętrznym a~zewnętrznym.
Większość istniejących osad kosmiczny, na orbicie Ziemi, na Księżycu,
Marsie i~Pasie, uległa najwidoczniej jakiejś złowieszczej ideologii
zjednoczenia i~rekonstrukcji, usiłując pomóc rażonej ludności Ziemi.
Opiekuni Ziemi, jak byli nazywani, byli przedstawiani jako małostkowi,
zazdrośni, złośliwi i~reakcyjni.

Zewnętrzni poszli w~osobnym kierunku, na zewnątrz. Aż do prawdziwej
nagrody Układu Słonecznego, największej planety z~nich wszystkich. Tutaj
były zasoby dla najdzikszych snów, najodważniejszych projektów.

Projekt, który rozpoczęli, kobiety, mężczyźni, umysły uploadowane i~sztuczne inteligencje był w~istocie odważny. Rozbili Ganimedesa, zebrali
megatony gazu z~atmosfery Jowisza, zamienili niewielki ułamek tego w~inteligentną materię i~odeszli w~wirtualne rzeczywistości. Nie, by śnić,
lub nie tylko by śnić. Stosowali umysły niesłychanej mocy do drobnego
ziarna kosmosu. Znaleźli luki w~prawach fizyki, rozciągnęli punkty. (\emph{Manipulacja czasoprzestrzenią przy pomocy materii
nie-egzotycznej}, Malley, I.K., Phys. Rev.\footnote{Przegląd Fizyczny - czasopismo wydawane od 1893 roku, zob.~\url{https://en.wikipedia.org/wiki/Physical_Review} -- przyp.tłum.}, D 128(10), 3182, (2080).)

Zostawili za sobą, poza makrami, dziesiątki tysięcy ludzkich umysłów
działających z~mniej więcej ludzką prędkością: Wolny Ludek, tak byli
nazywani. Większość z~nich pochodziła z~obozów przedsiębiorstw pracy
przymusowej. Czy byli w~oryginalnych ciałach, czy w~robotach, ich
zadaniem było ujarzmienie i~zebranie głupiej masy wymaganej przez
cywilizacje inteligentnej materii. W~obrębie makr, inni -- Szybki Ludek -- kopiował, dzielił, łączył, reprodukował się z~postbiologiczną prędkością
w miliardy. Przedstawienie mówiło o~procesie, jakby wydarzył się w~dalekiej przeszłości, choć daty pokazywały, że zdarzyło się to tylko
trzy lata wcześniej.

Jednak te umysły myślały, żyły, tysiące razy szybciej niż ludzkie mózgi.
Dla nich nasz świat już był starożytny jak Sumer, a~ich tysiącletnie
dzieło człowieka niczym bogów.

~

Następny ekran, który się pojawił, zaoferował opcję opisaną:
\emph{Wylogowanie}. Powtarzał to, co wyjaśnił Reid, ofertę tymczasowego,
i nieokreślonego, powrotu do zapomnienia. Wszystko, co musiałem zrobić,
to wprowadzić swoje nazwisko.

Zastanawiałem się nad tym. Potem zauważyłem ikonę, która miała załącznik
plikowy opisany \emph{Historia}. Właśnie tego chciałem się dowiedzieć,
pomyślałem, i~wskazałem na nią.

Nie była to historia projektu ani świata. Była to historia pliku
Wylogowania: moje własne nazwisko, daty i~godziny. Czasy pomiędzy ,,Stan
otwarty'' a~,,Stan zamknięty'' zwiększały się od godzin na początku, do
tygodni w~przedostatnim wpisie.

Było ich siedem. Ósme przeskoczyło do ,,Stan otwarty'' kilka godzin
wcześniej.

Cóż, jebać was, powiedziałem moim słabszym, wcześniejszym jaźniom.
Zamierzałem wytrwać, jeżeli tylko z~tego powodu, że samobójstwo nie było
ucieczką. Jeżeli ucieczka była w~ogóle możliwa, to nie wynikłaby z~mojej
własnej śmierci, ale śmierci wszystkich innych: kogokolwiek, lub
czegokolwiek, co wsadziło mnie do tego miejsca.

Zawsze chciałem żyć wiecznie\ldots ale nie na tych warunkach. Zawsze
chciałem, żeby koniec historii brzmiał: \emph{i żyli długo i~szczęśliwie}, a~nie \emph{i wszyscy umarli i~poszli do nieba.} Zawsze
myślałem, że czas na myślenie o~przekroczeniu ludzkości nastąpiłby,
kiedy byśmy do niego dotarli.

Coś we mnie się zmieniło. Jeżeli plik był prawdziwy, wybrałem śmierć
siedem razy, raczej niż istnienie. Jednak Reid napomknął, że
nieuniknione spontaniczne przebudzenia osoby mogą lepiej dopasować ją do
poradzenia sobie. Wzrastająca długość czasów, które ,,przetrwałem'',
sugerowały proces doboru, adaptacji: za każdym razem, kiedy wracałem,
miałem nieco więcej żelaza w~mojej krzemowej duszy.

Zawsze myślałem o~sobie jako o~upartym. Teraz kiedy spojrzałem na moje
prawdziwe życie, byłem zdumiony jak twardszy, bardziej cyniczny,
bardziej bezlitosny mogłem być. Moje wartości nie zmieniły się -- chyba
że moja pamięć się wykrzywiła -- ale siła mojej pasji stwardniała.

Patrzyłem na te obce rzeczy, które opuściły resztę ludzkości, które
użyły mnie jako maszynę i~teraz chciały mnie wyzyskać jako wynajęte
ręce, przekupić pięknymi wizjami. Wiedziałem, że chciałem żyć
dostatecznie, żeby ujrzeć wymieranie ich dziwnego piękna. Wiedziałem, że
tak będzie: mogłem nawet wtedy przewidzieć ich los.

Byłem zainteresowany i~byłbym tam.

~

Wróciłem do salonu, zapaliłem kolejnego papierosa i~przycisnąłem jeszcze
raz pierwszy przycisk. Telewizor nie zareagował.

-- Cóż, cześć -- powiedział głos za mną. Odwróciłem się i~zobaczyłem
kobietę siedzącą na drugim końcu sofy. Miała elfią twarz, mieszanych
ras. Czarna powódź jej włosów i~czarny dym jej halki obie dotykały jej
bioder. Wsunęła dłoń pomiędzy uda i~spojrzała na mnie. Jej oczy były
czarne jak jej włosy i~tak wielkie, jak niebo w~nocy.

-- Chcesz, żebym dzisiaj wieczorem była z~Tobą? Wiem, że tak. Ale
najpierw, mamy coś dla Ciebie. -- Uśmiechnęła się. -- No dawaj.

Wstała i~przeszła do drugiego pokoju. Jej stopy były nagie, jej halka
była mgłą, ale szła jakby była na obcasie i~wąskiej spódnicy. Nie wiem,
jak to zrobiła, choć zwracałem na nią szczególną uwagę. Podążyłem za nią
aż do Ramy, którą przeszła jak duch, i~która złapała mnie jak muchołówka
amerykańska\footnote{bylina mięsożerna
zob.~\url{https://pl.wikipedia.org/wiki/Mucho\%C5\%82\%C3\%B3wka_ameryka\%C5\%84ska} -- przyp.tłum.} łapie muchę. Na zewnątrz, w~czarnej próżni, jej obraz
zblakł tuż przy dotknięciu moich palców.

-- Praca -- uśmiechnęły się jej gwiezdne usta. -- Do zobaczenia wkrótce.

~

Zacisnąłem się na dwuteowniku. Znajomy sadzowy smak poliwęglanów
przesączył się przez moje chwytaki. Sięgnąłem do węzła montażowego i~zrobiłem zbliżenie na niego. Mechanizm wykrzywił się od nadmiernego
grzania. Ostrożnie, odgiąłem węzeł, skalibrowałem połączenie, potem
pozwoliłem kawałkom połączyć się razem. Uszczelniając węzeł, puściłem
jeden chwytak, wyciągnąłem go, złapałem, zwolniłem drugi i~przesunąłem
go bliżej, potem powtórzyłem krok kilka razy, jak ptak poruszający się
na grzędzie.

Przy kolejnym węźle, miałem wykonać szybkie oczyszczenie, poruszając
laserem nad kawałkiem meteorytu, aż metale w~nim się stopiły, potem
sięgając i~obracając jarzącą się masą w~kształt klatki, której
potrzebowałem, potem dopasowując go w~miejsce dookoła węzła montażowego.

Do następnego\ldots

\emph{Co ja kurwa robię}?

Zastygłem, przywiązując się do belki, gdy pytanie zawirowało w~moim
umyśle. Mój wzrok przesunął się niekontrolowanie, pola dalekich gwiazd
nagle stały się widoczne w~całej ich intensywnym bezmiarze, ich składowe
punkty światła pojawiały się i~znikały, gdy widmo mojego wzroku
przesuwało się w~górę i~w dół po spektrum fal.

Wysiłkiem woli się uspokoiłem. Zła chwila minęła. Spojrzałem w~dół na
węzeł, nad którym pracowałem, badając jego złożone, mikroskopijne
mechanizmy i~rozpoznając je bez żadnego wspomnienia widzenia ich
wcześniej. Pracowałem z~bezceremonialną pewnością czeladnika, póki to
nie wydało się dziwne. Najwidoczniej już wielokrotnie lunatykowałem przy
tych procesach i~jako obudzony lunatyk na gzymsie, spanikowałem i~byłem
w niebezpieczeństwie upadku.

Nic na to nie poradzę, ale trzeba się pogodzić. Był taki umysłowy trick
na to, odłączona uwaga, która pozwoliła moim dłoniom i~instrumentom
pracować, podczas gdy umysł patrzył i~włączał się, gdy widziałem coś, co
moje zaprogramowane, lub uwarunkowane, odruchy przeoczyły.

Po około subiektywnej godzinie, w~rogu mojego pola pojawił się widzenia
zestaw instrukcji. Powiedziały mi co robić i~gdzie się udać. Puściłem
się belki, trysnąłem krótkim odrzutem (\ldots nacisk palcami\ldots ), potem, po
podniebnym skoku przez kilometr pustki, kolejny błysk w~przeciwnym
kierunku (\ldots uderzenie piętami\ldots ) i~złapałem docelowy dźwigar.

Właśnie się przyczepiłem do niego, kiedy przede mną w~kosmosie wyrosło
makro jak wieloryb przed łodzią. Przylgnąłem, spanikowany i~znowu z~zawrotami głowy, do dźwigara, gdy jarząca się powierzchnia przesunęła
się, metry od moich przednich soczewek. Kiedy minęła, ciągle się
trzymałem, patrząc na powidoki. Nie ośmielałem się spojrzeć w~górę.

-- Otrząśnij się, kolego -- powiedział ostry, ale przyjazny głos. To był
męski głos, akcent londyński. Rozejrzałem się dookoła (tj. miałem
poczucia obracania się głową, ale wszystko, co się zdarzyło, to moje
pole widzenia poruszyło się tam i~z powrotem) i~napotkałem kolejnego
robota pracującego na dźwigarze jakieś sto metrów dalej. Uniósł ramię i~pomachał mi krótko, potem wrócił do swojego zadania.

Zacząłem swoje własne, podążając za instrukcjami, a~kiedy mogłem
podzielić uwagę, poświęciłem część na wymyślenie, jak mógłbym
odpowiedzieć. Wyobrażałem sobie przywołanie go. Powtarzałem tę prostą
czynność raz za razem w~głowie, jak nieśmiałe dziecko na dziwnym placu
zabaw. W~tym samym czasie, badając samego siebie, rozpoznałem w~końcu
mały talerz anteny na mojej kadłubie, kierujący się we właściwym kierunku
kiedykolwiek patrzyłem na innego robota i~myślałem o~zawołaniu.

Więc spojrzałem na niego i~powiedziałem: 

-- Cześć! 

Mogłem poczuć, jak moje usta się poruszają, gdy to robiłem, niepokojące wrażenie, które
stworzyło chwilowy groteskowy obraz maszyny z~ustami.

-- Łapię, Jay-Dub -- powiedział głos. -- Cześć. Skoncentruj się. Nie lubią,
kiedy gadamy w~trakcie pracy. Cieszę się, że wróciłeś.

Spróbowałem się zwyczajnie roześmiać.

-- Zdaje się, że parę razy się rozbiłem.

-- Ta -- powiedziała inna maszyna. -- Wszyscy tak robiliśmy. Jestem już
tutaj około roku, jednak, więc, zdaje się, trochę tego liznąłem. Poradzę
sobie.

-- Dlaczego nazwałeś mnie Jay-Dub?

Do tego czasu nie mogłem sobie poradzić z~przyswojeniem płci głosu do
głośnika. 

-- To jest wymalowane na Twoim boku -- powiedział. -- I~tak
zawsze się nazywałeś. Nazywam się Eon Talgarth, ale może mnie wołać
,,ET'', jeżeli tak wolisz.

-- Ok -- powiedziałem, od razu. Obaj się roześmialiśmy.

~

Kontynuowaliśmy rozmowę w~krótkich wymianach, gdy rozmawialiśmy.
Talgarth przedstawił mnie innym maszynom, każdej z~inną nazwą (lub
inicjałami) i~osobowością. Większość z~nich była mężczyznami, co miało
sens w~tym, że większość z~nich była kryminalistami lub jeńcami
wojennymi. Zdecydowałem, że musiałem mieć jakieś dobre powody, w~moich
utraconych przeszłościach, do nieujawniania mojego pełnego nazwiska,
więc pozostałem ,,Jay-Dubem''.

Talgarth odpracowywał dług przestępczy, którego okoliczności nigdy nie
do końca wyjaśniłem, jego imię pochodziło od jego rodziców Nowych
Osadników, jego nazwisko od Talgarth Road w~Londynie. To był jego
plaster. Odbyła się jakaś kłótnia, dzięki czemu wylądował w~Obozie Pracy
w Sutherland. Kiedy obozy zaczęły się zapełniać jeńcami USA/ONZ, został
zrekrutowany jako uzbrojony porządkowy, zmniejszając o~połowę swój
wyrok. Podpisał ofertę ciekawej opcji możliwej nieśmiertelności. Po tym,
nie był pewien, lub nie chciał powiedzieć, gdzie był. Był w~każdym
miejscu. Ostatnią rzecz, którą \emph{on} pamiętał, była wibracja
LKMu\footnote{ LKM -- lekki karabin maszynowym  -- przyp.tłum.}, z~którego
strzelał w~barbarzyńców próbujących zaatakować kompleks startowy.
Wspominał piasek, trawę, morze w~oddali. Gorąc jak mokry ręcznik. To
mogła być Floryda.

~

Nie było tutaj ogólnie dnia lub nocy, ale dla mnie dzień się skończył.
Wyszedłem z~Ramy i~okazało się, że moje symulowane mięśnie realistycznie
bolą. Łóżko było posłane, a~świeża paczka papierosów leżała na stole.
Jedzenie w~szafce zostało wymienione: nic specjalnego, mikrofalowe
żarcie, ale według moich smaków. Wziąłem prysznic, ugotowałem obiad,
zastanawiając się w~trakcie, jakie subtelne uzupełnienia zaawansowanego
software'u reprezentowała ta żywność i~położyłem się na łóżku.

Czarny sukkub\footnote{sukkub -- w~demonologii nazywa się demony przybierające
postać niezwykle pięknych kobiet, nawiedzające mężczyzn we śnie i~kuszące ich współżyciem seksualnym,
zob.~\url{https://pl.wikipedia.org/wiki/Sukkub } -- przyp.tłum.} przybyła, tak jak powiedziała, że zrobi. Była
niewyczerpana, nienasycona i~pomysłowa. Tak jak ja, w~stopniu, który
przekonał mnie lepiej niż wszystko inne, co było, a~co nie było
rzeczywiste dookoła mnie.

Cóż, jebać rzeczywistość.

~

-- He -- powiedział Talgarth. -- Myślisz, że to było dobre? Poczekaj, aż
dostaniesz się do makra, facet.

-- Nie mów o~tym -- powiedział kolejny głos.

-- Ok. Gówno.

I tak o~tym rozmawiali. Nie nadążałem za ich rozmową, ale to był
obsesyjny, szczegółowy, żargon uzależnionych. Żyli dla odjazdów. W~dziesięć dni pracy zarabiałeś na wizytę w~makrze. Kilka dni temu,
zobaczyłem, że Talgarth przerywa pracę i~czeka, gdy makro przesunęło się
ku niemu. Nibynóżka inteligentnej materii sięgnęła i~dotknęła jego
kadłuba. Została tak przez dziesięć sekund, nie więcej.

Talgarth wrócił do pracy i~przez resztę tego dnia nie rozmawiał ze mną.
Inni ostrzegali mnie, żebym nie próbował.

-- Kiedy byłeś w, rozumiesz, żałujesz wszystkiego, co odwraca uwagę od
tego.

-- Ale jak \emph{to} jest?

-- Inne dla każdego.

Dowiedziałbym się wkrótce.

Tej nocy, gdy odkładałem rzeczy, poczułem ręce sukkuba na moich
biodrach. Odwróciłem się i~pocałowałem ją. Już zaczęła rozpinać pasek.

-- Czekaj -- powiedziałem.

Powiodłem ją do salonu i~usadziłem na sofie. Usiadłem po drugiej
stronie, ustawiając popielniczkę pomiędzy nami.

-- Papieros?

-- Jeżeli chcesz.

Zapaliłem jej papierosa, odchyliłem się, zanim mogła mnie dotknąć.
Włożyła dłoń w~krocze i~westchnęła, i~gdy paliła, zaczęła się pieprzyć.

-- Przestań -- powiedziałem. To było niepokojące, jak obserwowanie małego
dziecka, lub umysłowo opóźnionej osoby.

Zachichotała i~złożyła razem kolana, jedna ręka wyszukanie na kolanie,
druga elegancko trzymająca papierosa.

-- Czym jesteś? -- spytałem.

Wzruszyła ramionami. 

-- Czym chcesz, Jon.

-- Pamiętasz jakieś inne życie? -- Pomachałem dłonią na okno. -- Przed tym?

Zmarszczyła brwi. 

-- Co chcesz, żebym pamiętała?

-- Masz jakieś imię?

-- Meg -- powiedział pogodnie. Podejrzewałem, że to było pierwsze imię,
które jej wpadło do głowy.

-- O co tutaj chodzi? -- Sięgnąłem po pilot do telewizora. Nic, prócz
śniegu i~białego szumu.

-- Praca i~zabawa -- powiedziała. Pochyliła się i~zgasiła papierosa,
patrząc na mnie z~całkowitym oddaniem. -- No chodź, chcę się zabawić.

-- Co by się zdarzyło -- spytałem, gdy oplotła nogą moje biodro i~zaczęła
całować gardło -- gdybym zgasił tego papierosa na Tobie?

-- Co masz na myśli?

-- Czy to by cię zabolało?

Zachichotała jak złe dziecko. 

-- Jeżeli tak właśnie lubisz.

Mogłem zrobić jej wszystko, absolutnie wszystko, i~byłaby z~powrotem
następnej nocy, chętna na więcej. Meg, myślałem, gdy ciągnęła mnie do
sypialni, była prawdopodobnie przydzieloną ilością \emph{przestrzeni
dyskowej}. Zatem jebać to, pomyślałem, i~jebanie zrobiłem.

~

Bulwa inteligentnej materii wybrzuszająca się ku mnie pokazywała
niezliczone fraktalne cechy, małe otchłanie nieskończonej głębokości,
kształty paproci i~twarzy. W~drżących momentach przed obłożeniem moich
instrumentów, poczułem, że już widziałem galerię sztuki, której powidok
na zawsze wypali się w~mojej pamięci wzrokowej.

Co fizycznie się zdarzyło, to inteligentna materia makra bezpośrednio
połączyła się z~moim komputerem, więc część mojego umysłu została
rzeczywiście, fizycznie zaimplementowana w~makro. Co poczułem to\ldots

Upadek śnieżynki na moim oku.

A potem przebudzenie, radość. Makro sprawiło, że wszystkie moje
wcześniej świadomości były jak sen, cała przeszłe szczęście mijającą
chwilą ulgi. Stałem nagi na trawiastym zboczu, patrząc przez zalesiona
pasma niebieskich wzgórz. Niebo na horyzoncie było jasnozielone. W~zenicie, prawie fioletowo niebieskie. Powietrze było zimne, ale
przyjemne, ciężkie od zapachów kwiatów, ostrych od smaku soli i~dymu
drzewnego. Znałem nazwę każdego wzgórza, gatunek każdej rośliny. Moje
ciało było wysokie, brązowe i~piękne, z~mięśniami, których
pozazdrościliby Conan i~Doc Savage\footnote{ Doc Savage -- fikcyjna postać
kompetentnego bohatera, która po raz pierwszy pojawiła się w~amerykańskich czasopismach w~latach trzydziestych i~czterdziestych,
zob.~\url{https://en.wikipedia.org/wiki/Doc_Savage } -- przyp.tłum.}. 

Usłyszałem za sobą głosy i~się odwróciłem. Stałem tuż poniżej czoła
wzgórza. Za nim, widziałem ocean, którego horyzont był około dwóch razy
dalej, niż byłby na Ziemi. To była \emph{wielka} planeta. (Wiedziałem
wszystko o~tym, znałem masę, orbitę, i~widmo jasnego nieba nad nią). Na
szczycie wzgórza, tylko kilka metrów dalej, był schron zbudowany z~czterech pionowych pni, poprzecznych belek i~zadaszenia z~gałęzi. W~środku był drewniany stół. Troje kobiet i~dwóch mężczyzn siedziało
dookoła stołu, rozmawiając i~się śmiejąc. Odwrócili się do mnie,
uśmiechnęli, a~potem zerwali się i~przywitali mnie w~taki sposób, że
wspomnienie ciągle przywołuje łzy.

Nie znałem ich w~moim przeszłym życiu, ale teraz ich znałem, a~oni znali
mnie. Tęsknili za mną od długiego czasu, a~teraz wróciłem do domu.

Zjedliśmy chleb, ser i~owoce, wypiliśmy wino i~rozmawialiśmy o~wielkiej
pracy, przy której byliśmy wszyscy zaangażowani. Moja część w~niej,
upewnili się, że wiem, była istotna i~bohaterska. Transportowanie
materii w~surowym wszechświecie! Jak porywające! Jak odważne! Jednak to
ich części chciałem wysłuchać, więc mi powiedzieli. Zrozumiałem
wszystko, co mówili, o~bramie czasoprzestrzennej, problemach i~wykonanym
postępie. Równania Malleya był tak proste jak arytmetyka, tak znajome
jak przepisy.

Jednak, co jakiś czas, kiedy mówiłem do jednego, inny powiedziałby coś
do wszystkich, i~wiedziałbym, że to jest ponad moją głową. Prawie
zrozumiałem, ale musiałem zaakceptować, że ten dostojny stół miał
bardziej dostojne stoły ponad nim, stoły, gdzie moi zachwycający
towarzysze byli znajomymi kolegami. Nie było protekcjonalności w~ich
zachowaniu. Pewnego dnia również i~ja dołączyłbym do nich tam.

Jednak myśl, chytre, dziwne pytanie wkradło mi się do głowy: czy to
miejsce było dla nich tym, co moje ciasne kwatery, papierosy i~sukkub
były dla mnie?

Zachód wielkiego słońca zatrzymał wszystkie rozmowy, wszystkie myśli.
Jego ostatni zielony błysk przyniósł wspólne westchnienie. Wtedy w~jednej zgodzie nas wszystkich, boginie i~bogowie wyskoczyli z~szałasu na
chłodną trawę. Bawiliśmy się jak dzieci i~pieprzyliśmy jak małpy.

Zasnąłem pod zatłoczonymi gwiazdami, w~ramionach jednej ze złotych
bogini.

Obudziłem się w~robocie.

Makro odpłynęło ode mnie, i~to było, jakby coś zostało wyrwane z~mojej
piersi. Pamiętałem dostatecznie dużo z~tego, co wiedziałem i~poczułem,
żeby ta strata jasności i~zabawy była prawie nie do zniesienia.
Pamiętałem towarzyszy, ale nie mogłem zapamiętać nawet ich imion. Nasze
rozmowy i~klarowne równania, same słowa, które wypowiedzieliśmy i~formuły, o~którym myśleliśmy, blakły, wspomnienie snu. Ból oddzielenia,
agonia wycofania, pożarła mój umysł na chwilę. Potem napłynęła ulga,
mógłbym wrócić za dziesięć dni!

Nic więcej się nie liczyło.

Kiedy pierwsza udręka tego rozstania minęła, odkryłem, że całe moje
nastawienie do, i~rozumienie, mojej pracy się zmieniło. Po raz pierwszy,
postrzegałem strukturę, którą budowaliśmy taką, jaką naprawdę była. Co
dotychczas było chaotyczną plątaniną belek, stało się wyraźne jak
rusztowanie bramy tunelu czasoprzestrzennego, wormhola Visser-Price'a i~platforma startowa statku. Jedna część, \emph{tam},
miała zostać. Druga miała polecieć ze statkiem. Pierścień się wyostrzył
jako największy akcelerator cząsteczek kiedykolwiek zbudowany, a~Jowisz,
mój boże, wielki Jowisz w~istocie! \ldots paliwo statku i~masa reakcyjna.

Spojrzałem w~dół i~zobaczyłem część pracy, którą ja, w~tej chwili, w~tym
miejscu, miałem olbrzymi przywilej wykonywać. Regulacja tego modulatora
zakłóceń było tym, do czego się urodziłem i~do czego się odrodziłem.
Rzuciłem się pracy z~radością rzemieślnika poświęcającego swoje życie
rzeźbieniu drzwi katedry, pewnego, że zaszczyt przyniesie mu lepsze
życie.

Nic więcej się nie liczyło.

~

Przy mojej następnej wizycie w~makro moi towarzysze byli tymi samymi
ludźmi. Zmienili się, od kiedy ich widziałem ostatni raz, żyjąc kolejny
wiek w~ciągle przyśpieszających życiach. Częściej niż za pierwszym
razem, nie rozumiałem ich konwersacji. Ich takt był subtelny i~uprzejmy
i dlatego jeszcze bardziej bolesny. Jednak wyszedłem z~tego, tym razem,
wstrząśnięty oczekiwaniem raczej niż stratą: brama miała się wkrótce
otworzyć.

Dwa dni później, tak się wydarzyło. Nie było ceremonii. Tylko alarm,
który ostrzegł siłę roboczą przed dotkniętym obszarem. Makra już
odleciały od nich, a~teraz wisiały w~przybliżeniu kolistym wzorze,
rozmieszczone pomiędzy dźwigarami. Cała praca została wstrzymana, gdy
odlatywaliśmy na granice struktury i~przyczepialiśmy się w~niemym
cudzie.

W rdzeniu struktury dźwigary zaczęły się poruszać, składając się jeden w~drugi z~narastającą prędkością, aż czarna okrągła przestrzeń otworzyła
się jak otwierająca się źrenica. Dwieście metrów szerokości, czterysta,
osiemset, mila: potem w~wybranym punkcie na obręczy, pękła przestrzeń. W~mrugnięciu okna, ta jednowymiarowa skaza, rozciągnięty punkt, stał się
kołem odciętym z~wszechświata.

Tunel czasoprzestrzenny Visser-Price'a był utrzymywany w~miejscu jak
błona mydła na pierścieniu, przez strukturę Malleya z~nieegzotycznej
materii dookoła niej. Tunel nie mógł być utrzymany całkowicie
nieruchomo: grawitacyjne wpływy i~czysta nieokreśloność kwantowa
sprawiały, że dokładne określenie jej krawędzi było niedefiniowalne do
mniej więcej najbliższego centymetra. Ta przewidywalna niedokładność
stworzyła nieprzewidziany, trywialny, ale wspaniały efekt: dookoła
krawędzi, złamane światło z~gwiazd, które zostało pochłonięte,
rozszczepiało się we wszystkie kolory widma.

Teraz wydarzenia następują w~tempie makr, nie naszym. Tęczowy pierścień
dookoła Mili Malleya staje się dwoma nakładającymi się pierścieniami.
Nowe koło się oddziela, na początku powoli. W~centrum tego drugiego
kręgu, sekcja struktury, którą zbudowaliśmy, składa się i~rozkłada w~ciemny, paraboliczny kwiat: statek. Myślałem, że to, też, zadrżało w~zniekształconym kosmosie. Nie mogę być pewny. Statek był podłączony do
drugiego kręgu przy pomocy stożka kabli, na których szczycie czekał,
gotowy.

Atmosfera Jowisza zagotowała się w~dziesiątkach punktów dookoła równika,
wysyłając tornada wijące się do Pierścienia dookoła planety. Pierścień
rozjarzył się, miliony akceleratorów dookoła niego popędzających czystą
materię w~dzikim kołowym rajdzie. Po jakimś czasie, biała linia
zapłonęła przez nasz środek, od Pierścienia do statku.

Statek, i~drugi krąg, wystrzeliły. W~sekundy były poza zasięgiem naszych
instrumentów. Teraz wydawało się, że biała linia przedłużała się do
pierwszego kręgu i~tam zatrzymywała. Jednak to był tylko nasz punkt
widzenia: strumień materii natychmiastowo wychodził z~drugiej strony
tunelu, teraz coraz dalej z~każdą mijającą sekundą, a~stamtąd do
silników statku.

To przyśpieszało sondę, a~z~nią drugą stronę tunelu, do ułamka prędkości
światła. Obie strony wormhola pozostawały połączone, dosłownie nie było
przestrzeni pomiędzy nimi, oraz czasu. Nasz koniec tunelu istniał w~ramie czasowej statku, nie w~naszej.

Dla obserwatora na statku relatywistyczna dylatacja czasu\footnote{
zob.~\url{https://pl.wikipedia.org/wiki/Dylatacja_czasu } -- przyp.tłum.} skróciłaby podróż wieków do dni, ostatecznie, gdy jego
prędkość zbliżałaby się powoli do nieprzekraczalnej wieczności fotonu,
miliony lat w~minuty, potem biliony w~sekundę. W~około trzydzieści lat
na pokładzie, Statek sięgnąłby granic obserwowalnego wszechświata, i~śmierci cieplnej lub Wielkiego Kolapsu\footnote{por.~\url{https://pl.wikipedia.org/wiki/\%C5\%9Amier\%C4\%87_cieplna_Wszech\%C5\%9Bwiata}
i~\url{https://pl.wikipedia.org/wiki/Wielki_Kolaps } -- przyp.tłum.}.

 
A przez te wszystkie lata, nasza strona tunelu byłaby w~tym samym
miejscu i~w tym samym czasie, co strona, która była ze Statkiem.
Zbudowaliśmy bramę do gwiazd i~do przyszłości. Za trzydzieści lat,
jeżeli chcielibyśmy, moglibyśmy udać się do krańców czasu.

~

Meg, sukkub, siedziała na sofie, dąsając się, gdy przeskakiwałem po
kanałach telewizyjnych. Zignorowałem jej rażące zniecierpliwienie i~podmuchy feromonów afrodyzjakowych. Ona jest tylko maszyną do
pieprzenie, powiedziałem sobie. Od startu sondy dwa dni temu, tempo
pracy zelżało, a~telewizja zaczęła pokazywać wiadomości i~programy
rozrywkowe. Wiadomości były dziwnie nienaturalne, jakości wewnętrznych
biuletynów: były to raporty pogody słonecznej, wywiady ze
zrehabilitowanymi członkami załogi -- jak teraz byliśmy nazywani -- i~sprawozdania o~tym, jaką dobrą pracę wykonywaliśmy. Rozrywka to były
filmy, turnieje, sztuki. Niektóre były klasyczne (\emph{ktoś} tam był
życzliwy Gillian
Anderson), ale większość była mi nieznana. Ich współczesne
odniesienia nie dawały wskazówki co do regresji cywilizacji, która była
pokazana w~pakiecie orientacyjnym. Było dokładnie tak, jakby wszystko na
Ziemi było tym, co większość ludzi w~moich czasach oczekiwałaby po
późnym dwudziestym pierwszym wieku: nieco zatłoczony, nieco dekadencki.
I my tutaj odbieraliśmy to po kilku godzinach świetlnych opóźnienia, na
budowie konstrukcji kosmicznej, której pracownicy byli z~jakiegoś
niejasnego, ale akceptowanego powodu ograniczeni do indywidualnych
kosmicznych holowników.

W skrócie, było tak, jakby to, co Reid powiedział mojego pierwszego dnia
tutaj, i~co pakiet orientacyjny przekazał, było całkiem nieprawdziwe.
Nie odważyłem się mieć nadziei, ale mogłem sobie wyobrazić, że niektórzy
mogliby. Zastanawiałem się, jaki punkt w~agendzie naszych panów
implikował to fałszywe zapewnienie.

Zakładając, że to, co widziałem, naprawdę było nadawane, a~nie
specjalnie wymierzone we mnie\ldots raz jeszcze byłem przytłoczony
niemożliwością określenia, co jest, a~co nie jest prawdziwe. Byłem w~dołku, wyciągnięty. Jeszcze sześć dni aż do powrotu do makra, cztery
dni, od kiedy tam byłem. Wpływ mojej ostatniej wizyty się wyczerpywał, a~moja następna było boleśnie odległa w~przyszłości. Na pewnym poziomie
tęskniłem za ludźmi, których znałem w~życiu, ale to było ukryte przez
bardziej desperacką żądzę spotkania znowu moich superludzkich
przyjaciół. Czy w~ogóle mnie pamiętają? Jak bardzo potężni się stali?

-- Jesteś zmartwiony, Jon -- wyszeptała Meg mi do ucha, obejmując mnie
ramionami. -- Chodź do łóżka.

-- Nie! -- warknąłem. -- Odpierdol się, ty jebana lalko!

W jej oczach pokazały się przekonujące łzy.

-- Jon, wiem, że jestem jebaną lalką, ale też mam uczucia. Ranisz mnie.

-- Jesteś tylko programem.

Mrugnęła i~na wpół uśmiechnięta, spojrzała na mnie w~irytująco
uspokajający sposób. 

-- Tak jak i~Ty, Jon, a~Ty masz uczucia.

Gapiłem się, zaskoczony jej argumentem. Nie zawartością, ale tym, że w~ogóle go przedstawiała.

-- Kiedyś mi powiedziałaś -- powiedziałem, myśląc na głos -- że możesz być
wszystkim, co tylko chcę.

Rozpogodziła się. 

-- Tak! Mogę!

-- Czy mogłabyś być bardziej inteligentna ode mnie?

Zmarszczyła brwi na chwilowej koncentracji. 

-- O ile bardziej inteligenta?

-- Dwukrotnie? -- Machnąłem ramieniem.

Spojrzała się na mnie dziwnie i~wstała. Spojrzała na telewizor,
skrzywiła się, podeszła do okna i~wyglądała na zewnątrz przez chwilę.
Potem się odwróciła, jedna dłoń na biodrze, druga opierająca się na
oknie.

-- Dobra, Jonie Wilde -- powiedziała. -- Wpakowałeś się w~niezłe, cholerne
bagno.

Niecierpliwy wyraz twarzy przypomniał mi, nagle i~boleśnie, obie,
Annette i~Myrę. Rozpoznałem to charakterystyczne piętno miny poza
wszystkimi różnicami wyglądu i~osobowości, i~zrozumiałem, co to zawsze
oznaczało: irytację wyższej inteligencji czekającą na mnie, żebym
dogonił.

-- Cóż, nie stój tak po prostu -- powiedziała Meg, mijając mnie. -- W~drugim pokoju jest ikona komputerowa. Zobaczmy, co możemy zhakować.

~

-- Pierwsze, co musisz zrozumieć -- powiedziała, gdy stanęliśmy przed
komputerowym ekranem -- jest to, że to wszystko jest prawdziwe, ale nie
jest \emph{fizyczne}. To symulacja. Ty i~ja, i~cała ta wewnętrzna
przestrzeń istnieje fizycznie jako ładunki elektryczne w~komputerze tego
robota, którym jedziemy.

-- Dobra, -- powiedziałem -- to mi świtało.

-- Ok, nigdy mi nie mówiłeś. -- Uśmiechnęła się. -- Zauważ, wątpiłam, czy
zrozumiałabym cokolwiek pięć minut temu. Tak czy inaczej\ldots tak więc nie
świruj.

Po tym zanurzyła ramię po łokieć przez ekran, który zawsze był dla mnie
konkretny i~zaczęła gmerać. 

-- Och, dobrze -- powiedziała. -- Mam pliki
kropka sys na nas. Ha! Mój może zostać udostępniony przez Ciebie
przemawiającego do mnie, tak jak właśnie to zrobiłeś. Jednak Twój, mogę
pobawić się nimi stąd\ldots tylko chwilka.

Sięgnęła do środka drugą dłonią i~przesunęła coś w~bok, zanim mogłem
zrobić cokolwiek.

-- Jak teraz? -- spytała.

Patrzyłem na piękną kobietę w~krótkiej czarnej sukience wieczorowej. Coś
było nie tak. Miała obie ręce wciśnięte w~ekran komputerowy. Cofnąłem
się o~krok.

-- Trzymaj to -- powiedziałem. -- Tylko\ldots poczekaj. Uważaj na szkło.

Jednak szkło nie było pęknięte. Mrugnąłem, niepewny, czy poprawnie
widzę. Kobieta się roześmiała.

-- Kurde -- powiedziała. -- Zła strona. -- Przesunęła znowu ręce i~otworzyłem znowu usta, żeby ostrzec ją o~szkle.

I ona była szkłem, ja byłem szkłem, wszystko było światłem.

-- Och -- powiedziałem. -- Rozumiem teraz.

~

Inaczej niż doświadczenia w~makro, moje wspomnienia czasu mojej
wzmocnionej inteligencji z~Meg są jasne i~jaskrawe. Nie byłem
superczłowiekiem z~ograniczeniami, ale ludzkim umysłem z~dodanymi
możliwościami. Ciągłość mojej jaźni nigdy nie została przerwana, jak to
było w~dziwnym bystrym towarzystwie Szybkiego Ludku, który spotkałem na
symulowanej wielkiej planecie. Więc, nawet teraz, to okres, który
pamiętam, nawet jeżeli nie mogę przeżyć.

Przez chwilę po prostu patrzyliśmy się na siebie.

-- Dobra -- powiedziała Meg. -- Uczciwie to uczciwie. Twoja kolej.

-- Och. -- Spojrzałem na komputer, potem wzruszyłem ramionami. -- Ok, Meg -- powiedziałem. -- Bądź tak inteligentna jak tylko możesz być.

-- Dzięki -- powiedziała. Jej twarz stała się, w~jakiś nieokreślony
sposób, bardziej skupiona. Mrugnęła i~rozejrzała się.

-- To naprawdę jest coś, prawda?

-- Niezupełnie.

Roześmiała się. 

-- Wygląda jednak trochę inaczej.

Zdecydowanie tak wyglądało. Rzeczywisty wygląd rzeczy się nie zmienił,
ale wszystko było jakby opisane z~wyjaśnieniami. Było po prostu
oczywiste, co programy leżące pod symulacjami robiły.

-- Co przeszkadza innym od zrobienia czegoś takiego jak to?

Meg wzruszyła ramionami. 

-- Nic. W~pewnym sensie oszukiwałeś. Jednak to
ma coś wspólnego ze sposobem, w~jaki Twój umysł, Twój naturalny umysł,
pracuje. Musisz mieć całkiem dobrym umysł, żeby poradził sobie ze
wzrostem inteligencji. Nie może być po prostu podkręcony. Jeżeli
większość innych chłopaków by się domyśliła jak to zrobić, po prostu
byliby\ldots zjarani, albo na odjeździe. Muszą to wypracować, we własnym
czasie. Praktycznie w~ogóle nie powinieneś tutaj być.

Kiedy mówiła, może dlatego, że mówiła, widziałem, co miała na myśli,
logika leżąca u podstaw jej twierdzeń wypełniana dodatkowymi danymi
wyodrębnionymi z~pamięci maszyny.

Budowa wormhola naprawdę była obozem pracy i~wszystko na budowie było
zaprojektowane do zarówno kontroli, jak i~rehabilitacji więźniów.
Struktura pozwalała, w~istocie zachęcała, do współpracy, jednocześnie
ograniczając zmowy w~innych kontekstach, w~ten sposób zapewniając
reedukację przez pracę bez stawania się uniwersytetem zbrodni. Poza
procesem pracy byliśmy zasadniczo w~izolatce wraz z~dostępnym sukkubem
dla nagradzania społecznego i~seksualnego. Każdy sukkub był aspektem
tego samego komputera, na którym osobowość ludzka więźnia była
zaimplementowana. I~odpowiadała na narastające interakcje społeczne
przez narastanie własnego repertuaru społecznego, w~ten sposób
nagradzając jakikolwiek wzrost empatii po stronie więźnia większą
intymnością.

Wycieczki do makro służyły podobnej sprawie, w~stosunku raczej do
poznawczego niż emocjonalnego rozwoju. W~mojej prawdziwej niewinności,
traktowałem sukkuba jak nic więcej niż wirtualną sekszabawkę, ale
osiągnąłem wybitną integrację z~postludzkimi bytami w~makro. Napięcie
tej anomalii w~końcu wywołało u Meg podwyższenie stawek emocjonalnych, z~konsekwencjami wydatnie szybszymi i~bardziej drastycznymi niż
projektanci systemu oczekiwali. Zmodernizowaliśmy się do maksymalnej
pojemności hardware tej maszyny.

-- Więc czym jesteś? -- spytałem. -- Czy kiedykolwiek byłaś człowiekiem?

Meg wzruszyła ramionami. 

-- Jestem częścią kopii. Końcowym wynikiem
rozwoju osobowości, bez żadnych wspomnień osoby. Większość mojego umysłu
to AI. Ludzka powierzchnia, maszynowa głębia.

Mój wyraz twarzy musiał powiedzieć jej, co na ten temat myślałem.

-- Ta, ponure, co? -- powiedziała. -- Ciągle, to ja.

Moja następna myśl była\ldots

-- Czy uruchamiamy gdziekolwiek alarm?

-- Nie -- odpowiedziała. -- Żadnej centralnej władzy, racja? Całe sedno.
System agory. -- Uśmiechnęła się. -- Powinieneś wiedzieć. Zauważ, są
obejścia, Reid na pewno się upewnił, więc nie cisnęłabym.

-- Aha. Więc co teraz robimy?

-- Wiesz -- powiedziała. -- Reid ciągle kieruje całym projektem. Jest
szefem. Nie, żeby Szybki Ludek zwracał uwagę, ale reszta z~nas poza
makrami musi.

-- Jeżeli Reid jest u władzy -- powiedziałem -- zdaje się, że to czas,
żebyśmy się z~nim spotkali.

Meg sięgnęła raz jeszcze za systemy sterowania i~wywołała go. Ekran
dzwonił kilka sekund, potem pojawiła się lekko zaniepokojona twarz
Reida. Wyglądał, jeżeli cokolwiek, młodziej niż na nagraniach, ale jego
wyraz twarzy spokojnej czujności przełamał się, kiedy mnie zobaczył.
Mrugnął, otworzył usta, potem je zamknął, język oblizujący wargi.

-- Wilde! -- powiedział. -- Czy to naprawdę Ty?

-- Tak -- powiedziałem.

-- Zdumiewające! -- odparł. Meg mierzyła czas jego odpowiedzi. Opóźnienie
było niedostrzegalne. Zgadywałem, że musi być blisko, na skale w~Kręgu.
Nie widziałem żadnego oczywistego habitatu w~lub dookoła Struktury.

-- Mój Boże, myślałem, że nie żyjesz! -- kontynuował. Sapnął. -- Pomiędzy
Martwymi.

Jeżeli kłamał, to był w~tym dobry: nawet dla Meg, której software
analizy wizualnej wisiał za moim wirtualnym polem widzenia, jego
wyrażenia zdradzały tylko zaskoczenie, ciekawość i~niewymuszoną radość z~zobaczenia mnie znowu. Jednak mu nie ufałem: dodane lata doświadczeń i~dyscypliny przydały mu druzgoczącej aury władzy. Zrozumiałem, nagle, że
był niepodobny do każdej ludzkiej istoty, którą kiedykolwiek widziałem.
Nanotechnologia, inteligentna materia, która uratowała go od wieku,
mogła równie dobrze działać dalsze alchemie w~jego mózgu i~krwi.

Rozłożyłem moje ramiona, zmuszając się do uśmiechu. 

-- Czyż to nie
śmierć?

Reid uśmiechnął się posępnie. 

-- Postżycie, jak to nazywamy. Jednak
załatwię Ci elektroniczną kochankę, żeby coś zrobiła z~Twoim wyglądem.
Wyglądasz strasznie.

Spojrzałem za niego, przyglądając się w~tle. Inni ludzie się poruszali,
wydawało się, że siedzi we wspólnej przestrzeni, mówiąc do zestawu kamer
ustawionych pod kątem od niego, publicznym raczej niż prywatnym.
Perspektywa pięter i~ludzi w~tle uderzyła mnie chwilowo jako dziwna,
potem oni się wyostrzyli. Z~zakrzywienia podłóg i~subtelnych przechyłów
różnych poziomów, mogłem dojrzeć, że był w~wielkiej stacji kosmicznej,
wirując odśrodkowo.

-- Bez wątpienia -- powiedziałem. -- Ale nie gorzej, niż kiedy ostatni raz
mnie widziałeś, pamiętasz? -- Poczułem przypływ gniewu. -- Zabiłeś mnie,
ty chuju!

Jego spokojne spojrzenie skupiło się na mnie. 

-- Nie, nie zabiłem -- powiedział. -- Brałeś udział w~incydencie przygranicznym. Zrobiłem, co mogłem, żeby cię uratować, powiem tak, ale byliśmy zbyt późno. O ile
dobrze wiedziałem, tam umarłeś. Twoje ciało zostało przesłane do Anglii
i skremowane. Byłem na Twoim pogrzebie, człowieku!

Próbowałem nie okazać, jak bardzo byłem wstrząśnięty. 

-- Więc jak tutaj
trafiłem? -- spytałem. -- Nie opowiadaj mi, że nie wiedziałeś, że zrobili
kopię!

Reid westchnął, przesuwając palcami przez gęste czarne włosy. 

-- Oczywiście, że wiedziałem. Byłeś jednym z~pierwszych ludzkich
przedmiotów badań, nie wiedzieliśmy, czy to w~ogóle zadziała. Pobraliśmy
kopię w~ciągu kilku minut od znalezienia i~przechowaliśmy skan mózgu i~informację genetyczną. Jednak o~ile wiem, to było wszystko, kopia była
przechowywana z~resztą Martwych, w~Banku. Nie zostawiłeś dyspozycji,
więc cię tam zostawiliśmy. Nigdy nie byłeś uploadowany do makra, jestem
tego dość pewien. Nie wiedziałem, że ktokolwiek zrobił kopię, a~to jest
szczera prawda. I~nie ma jak się dowiedzieć, teraz, już dawno temu
odpowiedzialni inżynierzy przetransferowali się sami.

-- Dobra, prawie się nie skarżę na swoje własne istnienie -- powiedziałem.
-- Ale chciałem się wydostać z~Twojej niewolniczej siły roboczej, jeżeli
nie masz nic przeciwko.

Reid uśmiechnął się jakby z~ulgą.

-- Naturalnie -- powiedział.

-- Jeżeli o~to chodzi.

Jego usta się zacisnęły. 

-- Hmm. -- Sięgnął po klawiaturę i~wstukał kod.

-- Ok, tyle o~mnie -- powiedziałem. -- O co chodzi z~tymi Martwymi w~banku?
Co się stało z~Annette, Myra i\ldots wszystkimi innymi?

Reid ciągle patrzył poza kamerą, jakby patrząc na inny monitor.
Aktywność w~tle przyśpieszyła, w~atmosferze większej pilności.

-- Myślę, że Annette jest bezpieczna -- powiedział z~roztargnieniem. -- Zmarła w, hm, niespokojnych czasach, ale umówiła się na kopię. Jeżeli
była zrobiona, to jest w~banku tak jak Ty. Jak miliony ludzi. Wtedy to
było tanie. Ludzie robili rutynowo kopie. Szczerze mówiąc, nie wiemy,
kogo dokładnie mamy. Myra, i~Twoja córka, cóż\ldots o~ile wiem, zostały na
Ziemi. Bogini wie, jak się tam toczą sprawy\ldots

-- Nie ma kontaktu?

-- Pieprzeni Opiekuni Ziemi, boją się nas, zagłuszają nas \ldots co\-kolwiek
widziałeś na naszych taśmach jest albo stare albo fałszywe. Nie, nie
mamy żadnego kontaktu. -- Odwrócił się nagle, patrząc prosto na mnie. -- Dobra Wilde, muszę lecieć. Jesteś już wolny, załatwiłem Twoje
ograniczenia. -- Wstał i~pochylił się do kogoś poza kadrem. Nie mogłem
usłyszeć wymiany, która nastąpiła. Potem Reid wrócił, patrząc na mnie z~niestrzeżonym uśmiechem.

-- Wilde? -- spytał. -- Ciągle tam? Możesz coś zrobić, teraz? Sprawdź, co
się dzieje z~najbliższym makro. Jest jakiś problem\ldots

Ekran wyszarzał.

-- Gówno! -- powiedziałem.

Meg stała przede mną, zmartwione widmo. 

-- Co robimy?

Wzruszyłem ramionami. 

-- To, co powiedział Reid, chyba. Możesz wymyślić
coś innego?

Potrząsnęła głową.

Wszedłem w~symulowaną ramę symulacji, a~Meg weszła za mną. Poczucie
nakładania się obrazów ciał było chwilowo dezorientujące, a~potem
zazębiliśmy się gładko ze sobą i~z maszyną. Meg stała się głosem zza
mojego ramienia, cieniem w~rogu mojego oka.

Miałem pełną kontrolę nad robotem, polecenie Reida musiało wyłączyć
plik, który oddzielał mnie od jego obwodów napędzania poza godzinami
pracy i~sytuacjami wyjątkowymi, zatem poleciałem spokojnie przez
Strukturę ku makro, które (teraz) rozpoznawałem jako jedno z~tych, w~których byłem w~kontakcie. Niektóre z~innych robotów wykonywały
chaotyczną pracę, inne dryfowały w~trybie offline lub siedziały jak
grzędujące ptaki na dźwigarach. Mila Malleya świeciła się delikatnym
niebieskim w~jej kręgu tęczy: kilwater Czerenkowa\footnote{ promieniowanie elektromagnetyczne emitowane, gdy naładowana cząstka (np. elektron) porusza się w ośrodku materialnym z prędkością większą od prędkości fazowej światła w tym ośrodku, zob.~\url{https://pl.wikipedia.org/wiki/Promieniowanie_Czerenkowa} -- przyp.tłum.} z~sondy. 

Złapałem dźwigar, przysunąłem się bliżej do powierzchni makra, i~pogrążyłem twarz w~kąpiel płonącego ognia.

\threeast

Wszystko jest analogią, interfejsem. Jaźń sama ma okna, dźwięki i~obrazy
w naszych głowach, ikony na ekranie nad maszyną, umysł. Tak jest w~ciele
biologiczny i~tak jest w~sztucznym, i~wiele razy tak w~świecie
inteligentnej materii makroorganizmów.

Meg kradła moc obliczeniową, współdzielenie czasu w~większych umysłach.
Było to konieczne dla mnie, nas, żeby zrozumieć minimalnie,
symbolicznie, co się działo, ale to zabrało swoją część. Działałem
wolniej niż Szybki Ludek, wolniej nawet niż Wolny Ludek. Chodziłem jako
niewidzialny duch, chwilowy dreszczyk w~snach postludzkich.

Byłem najpierw na wielkiej planecie. Na stoku, gdzie pierwszy raz
stałem, obserwowałem pory roku -- śnieg, wiosna, lato i~jesień -- pojawiające się jak fale na brzegu. Środowisko było przybliżeniem tego
na planecie, którą obecnie szpiegowali, jakieś trzydzieści lat
świetlnych dalej. W~przyszłości ten obraz mógłby zostać zaktualizowany
i poprawiony przez przesłane dane z~mijającej sondy.

Stracili zainteresowanie tym, gdy patrzyłem. Spójni do końca, skasowali
to z~własnej pamięci przez odpalenie jego słońca. Szedłem przez
pochłaniającą nową\footnote{nowa klasyczna -- gwiazda wybuchowa, w~rzeczywistości ciasny układ podwójny złożony z~białego karła i~gwiazdy
ciągu głównego lub nieco odewoluowanej gwiazdy,
zob.~\url{https://pl.wikipedia.org/wiki/Nowa_klasyczna } -- przyp.tłum.}, w~śniegu fałszywej rzeczywistości rozpuszczającej się w~kod binarny, a~potem do ogromnej sali. W~mroku świątyni Molocha\footnote{Moloch -- bóstwo, szeroko czczone w~starożytności na Bliskim Wschodzie
oraz wszędzie tam, gdzie rozrastała się kultura Kartagińczyków. Bóstwo
było znane i~przedstawiane pod trzema formami: cielak, wół lub człowiek
z głową byka, por.~\url{https://pl.wikipedia.org/wiki/Moloch
} -- przyp.tłum.}, gigant o~ciężkich powiekach siedział, atletyczni
marmurowi bogowie niezgrabni w~pozach Buddów. Rozkład poza dekadencję,
zamrożenie szału i~zmęczenia. Niestrudzone mechanizmy, poza i~poniżej
świadomej kontroli gigantów, kontynuowały ich nieubłagane, bezcelowe
przyśpieszenie prędkości przetwarzania. Chwila po chwili, system
operacyjny Meg śledził zmianę.

Zanim ostatnie echo moich kroków zamarł w~sali, medytujące giganty były
pyłem. Na zewnątrz, w~jeszcze innym środowisku wirtualnym, miasta były
budowane i~burzone w~czymś, co dla mnie było chwilami, na tle ciągle
zmieniających się krajobrazów planetarnych. Ostatecznie ustała wszelka
ludzka analogia i~zainteresowanie. Dryfowałem nieskończonymi korytarzami
abstrakcji geometrycznych, siekana logika nieskończonych dyskusji
wypełniała umysł, jakbym podsłuchiwał uwięzione duchy teologów w~piekle,
na które tylko oni mogli całkowicie zasługiwać.

Za mną, w~tych korytarzach, płaczliwy głos kobiecy wołał za mną. Stawał
się coraz silniejszy, gdy mijał czas, ale ignorowałem go, zdesperowany,
żeby zrozumieć przerażającą debatę. Uczyłem się, czegoś zasadniczego.
Głos krzyczał za mną. W~końcu się odwróciłem. Udręczona twarz Meg
przekazywała natężenie systemu operacyjnego na granicy jego możliwości.

-- Wychodź \emph{stąd}! -- krzyknęła. -- Wyjdź teraz!

Patrzyłem się na nią, zaskoczony. Wszystko było wolne, korytarze stawały
się białe jak kazachskie zaspy śnieżne. Z~nagłą niecierpliwością, Meg
złapała mnie i~pchnęła na ścianę. Zapadła się i~byłem\ldots

\threeast

\ldots na zewnątrz, oddalając się od makra. W~tej samej chwili upadłem do
pokoju, z~powrotem w~umysł mojej własnej maszyny, i~w ciepłe ramiona
mojego drogiego, słodkiego systemu operacyjnego, mojego sukkuba i~zastępczej bratniej duszy. Łzy były w~moich oczach i~natarczywe
dzwonienie w~moich uszach.

Rozpoznałem je jako alarm. Na zewnątrz, ku Pierścieniowi, błyskało
światło i~przyzywała radiolatarnia. Latarnia się zbliżała, szybko.

-- Co się dzieje?

Meg spojrzała na mnie. 

-- Och, Jonie Wilde -- powiedziała. -- Byłeś tam
przez jebany \emph{rok}, czasu rzeczywistego! Makra wszystkie oszalały
lub umierają.

Rok. 

-- Co się stało?

Meg złapała moją dłoń. 

-- Potem -- powiedziała. -- Musimy lecieć. Zabiorę
nas stąd.

Weszła w~Ramę. Gdy patrzyłem, zszokowany i~bez kondycji, żeby obsłużyć
rower treningowy, Meg kopnęła nas w~kierunku radiolatarni.

Zobaczyłem, co oznaczała radiolatarnia.

Wyłaniając się zza Pierścienia, zmierzało ku nam najbardziej haniebne
ustrojstwo, które udawało statek kosmiczny, poskręcane amatorskie
zlepienie stacji kosmicznych oraz habitatów na przynajmniej dwa
kilometry długie i~w najszerszym miejscu na pół kilometra. Jeżeli
mieszaninie Mira\footnote{zbudowana przez Związek Radziecki załogowa stacja
kosmiczna,
zob.~\url{https://pl.wikipedia.org/wiki/Mir_(stacja_kosmiczna)
} -- przyp.tłum.} i~Wahadłowca\footnote{prom kosmiczny konstrukcji USA,
zob.~\url{https://pl.wikipedia.org/wiki/Wahad\%C5\%82owiec_kosmiczny
} -- przyp.tłum.} ze wczesnych dekad dałoby się milion generacji
rozmnażania w~kierunku wielkości i~przeciwko elegancji, to mogłoby to
powstać. Kręciła się oszałamiająco po swojej osi i~sterowała ryzykownym
kursem wzdłuż ciągle śmiertelnej żarłocznych rakiet -- ostatecznego
przewodu pod napięciem -- linii zasilającej sondę.

Wszystkie roboty wiały w~kierunku Statku. Gdy tylko każda mała maszyna
docierała, łapała się któregokolwiek z~wielu wystających kawałków
śmieci, którego dosięgła. Makra również się poruszały, ale nie tak jak
wcześniej. Teraz zamrożone, szkieletowe, dryfowały i~mieszane, gdy
potężny pojazd przebijał się z~brutalnym majestatem przez Strukturę, nad
którą się trudziliśmy.

Powierzchnia statku przeleciała w~oknie. Prawie zamknąłem oczy. Jednak
Meg sprowadziła nas na dopasowanej prędkości. Zobaczyłem ramię robot i~chwytaki sięgające. Natychmiast, gdy znalazły punkt zaczepienia, Meg
przełączyła punkt widzenia i~wtedy wyszła z~Ramy.

Usiadła na łóżku koło mnie i~przylgnęliśmy do siebie tak gorączkowo, jak
nasza maszyna przylgnęła do Statku. Niebo przewróciło się, i~znowu, i~znowu. Biała linia rakiet przeleciała, bliżej i~bliżej.

-- Spróbuję się połączyć -- powiedziała Meg. Wpatrzyła się i~jakby przez
wysiłek jej woli, widok nagle stał się stabilizowanym kadrem skądś
daleko na przedzie. Pierścień tęczy prawie ją wypełniał, niebieskie
promieniowanie płonęło, gdy zabłąkane, strzaskane dźwigary wpadały. Z~boku, zobaczyłem makra oddalające się od dysz pozycyjnych Statku.
Przypadkiem lub specjalnie, upadały na powierzchnię Jowisza. Planeta,
już widocznie zmieniona przez ich działalność, Wielka Czerwona Plama
powtarzała się jak wysypka na jego powierzchni, otrzymałaby te struktury
płatku śniegu i~może ogrzeją je do odnowionego i~niewyobrażalnego życia.

W moich ostatnich minutach w~Układzie Słonecznym, potwierdziłem moją
wstępną reakcję. Umysły w~makrach wpadły w~pułapkę własnego projektu,
ryzyko,które mogły świadomie, jak inaczej? objąć. Gdy prędkość ich myśli
przyśpieszyła, tak też ich subiektywny czas, a~zatem także przestrzeń.
Nawet międzyplanetarne odległości zamieniły się w~otchłanie z~czasami
podróży, które byłyby dla nich, tym czym podróże międzygwiezdne -- bez
tunelu czasoprzestrzennego -- byłyby dla nas. Ich wirtualne
rzeczywistości stały się bardziej absorbujące, w~każdym znaczeniu, niż
szybko wycofujący się kosmos rzeczywistości.

Okres ich największych projektów był większy niż ich okres uwagi,
dłuższy niż czas trwania jakiejkolwiek cywilizacji. Ze sobą zabrali
nasze słabości, tak jak nasze moce, i~pomnożyli i~przyśpieszyli obie.
Ludzkość, lepiej dopasowana do kosmosu dzięki zaletom swoje niższości,
ich przeżyje.

Tak jak i~ja. W~bardziej dosłownym sensie, niż kiedykolwiek zamierzałem,
dotarłem do statków.

\emph{Dzwony piekła dzwonią bim bam bom }

\emph{dla mnie, ale nie dla Ciebie}

\emph{O śmierci gdzież jest twoje żądło-dzyń-dzyń?}

\emph{Gdzież jest, o~grobie, zwycięstwo twoje?\footnote{piosenka brytyjskich
lotników z~okresu I~Wojny Światowej, tłum. własne.,
więcej~\url{https://en.wikipedia.org/wiki/The_Bells_of_Hell_Go_Ting-a-ling-a-ling
} -- przyp.tłum.}}

Promieniowanie Czerenkowa wzrosło do nieznośnego błękitnego blasku, gdy
przednia część Statku, do którego się przyczepiliśmy, przeszła przez
bramę tunelu czasoprzestrzennego.

\chapter{Płytki Sitowe}

Spędzili noc w~tunelu z~pełnymi szacunku robotami. Z~krótkofalowej
łączności z~innymi ich rodzaju, Roboty dowiedziały się o~nuklearnym
zniszczeniu pojazdu Jay-Duba. Uroczyście to dyskutowały, gdy ludzie
próbowali zasnąć. Ostatnią rzeczą, jaką zobaczyła Dee, zanim zdrzemnęła
się w~dość suchej wnęce z~ramionami dookoła Ax, był blask w~oczach
robotów, gdy przyjęli jako artykuł wiary twierdzenie, że Jay\dywiz Dub nie
umarł.

W pierwszych światłach poranka, ludzie wstali i~ubrali się w~przebrania
robotów. Ich głównym zadaniem było oszukanie obserwatorów na niebie. Na
ziemi, z~bliska, nikogo by nie zwiedli.

-- Skąd \emph{wiecie}, że to wszystko musimy robić? -- zrzędził Ax. Był
zirytowany koniecznością noszenia jeszcze bardziej nonsensownej robociej
skorupy niż inni z~powodu jego wzrostu. Wyglądał jak kosz na śmieci na
nogach.

-- Jay-Dub powiedział mi co robić -- powiedziała Dee, jej głos głęboki i~dziwny przez siatkę głośnika hełmu.

-- Kiedy?

Wzruszył z~brzękiem ramionami. 

-- Kiedy byliśmy razem w~jego VR -- powiedziała. -- I~tuż, zanim opuściłam pojazd, włączyłam się znowu. Powiedział mi dokładnie co robić, gdyby mu się nie udało.

-- I~nie zamierzasz nam powiedzieć? -- upomniała się Tamara, próbując
znaleźć pasujące miejsce w~ciele robota do schowania pistoletu. (--
Gorszy niż kieszenie w~spódnicy -- wymamrotała).

-- Nie -- powiedział stanowczo Dee. -- Jeżeli mi się nie uda, to nic nie
zrobicie. A jeżeli wam się nie uda, to lepiej, żebyście nie wiedzieli.

-- Nie ma to jak umieranie szczęśliwym -- powiedział Wilde.

Robot, który najwięcej z~nimi rozmawiał, pożegnał ich, zapewnił, że
zawsze są mile widziani w~obozach Ludzi Metalu, i~poradził im jak się
zachowywać w~sytuacji konfrontacji. Jego basowy głos urwał się, gdy
spojrzał na Dee.

-- Też jesteś maszyną -- powiedział. -- Będziesz wiedzieć.

-- Dziękuję -- powiedziała Dee, jej głos zabrzmiał jeszcze dziwniej, gdy
próbowała się nie śmiać. -- Ale moi ludzcy przyjaciele są obeznani z~dzikimi maszynami.

-- Unikaj ich -- nakazał jej robot. -- Nie są tacy jak my.

Ludzie szli wzdłuż tunelu ku łukowi odległego światła. Kiedy dotarli tam
i odwrócili się ostatni raz spojrzeć, ich wizja akomodowała i~pary
iskier robocich oczu zniknęły w~mroku.

~

Tamara kichnęła. W~hełmie zrobił się bałagan, a~ona ukradkiem zdjęła
hełm, żeby wytrzeć smarki i~ślinę.

-- Wspaniale -- powiedział Ax zza niej. -- To wygląda naprawdę
przekonujące, robot zdejmujący swoją głowę.

-- Żeby nie wspomnieć kichania -- powiedziała Dee. -- W~każdym razie, o~co
chodzi? To Twoje\ldots siedemnaste kichnięcie w~ciągu trzydziestu pięciu
minut.

-- Opad. -- Tamara agresywnie pociągnęła nosem. -- Kurde dostaje się przez
nos, ok?

Szli gęsiego tylną ulicą w~północnej części Piątej Dzielnicy, stronie
przeciwnej do tej, która była naprzeciw ludzkiej dzielnicy. Ich
zadaniem, jak powiedziała im Dee, było podążanie tym kursem, przez
kraniec Dzielnicy, gdzie zwężała się w~pustynię, aż do skrzyżowania z~Kamiennym Kanałem. Jedyną aktywnością, na którą natrafili były małe
biomechy, skaczące, lub pełznące przez ich drogę, kierujące się na
wiatr, który przynosił radioaktywny pył z~pustyni. Ostatecznie, Tamara
wyjaśniała, całe ich stada zebrałyby się na miejscu eksplozji, aby
biesiadować na bogatych niestabilnych izotopach.

-- W~pewnym sensie ekologiczne -- dodała. -- Trzyma je z~dala od łańcucha
pokarmowego życia węglowego, prawda?

Szli dalej. Słońce było coraz wyżej na niebie, a~ubiory stawały się
coraz bardziej niekomfortowe. Dee, z~bardziej świadomym sterowaniem
tolerancji na ból niż inni, pozwoliła jej niecierpliwości sprowokować do
działania.

-- Im wcześniej tam dotrzemy -- powiedziała -- tym wcześniej zdejmiemy ten
złom.

-- Ci z~nas, którzy tam dojdą. -- Ax zaprotestował . -- Pochowajcie mnie
czymś innym, tylko o~to proszę.

-- Spróbujemy worka na śmieci -- zawołał Wilde bezdusznie.

Dee powiedziała im, żeby byli cicho. Żarty nie były cechą robotów
humanoidalnych. Cień przelatującego samolotu podkreślił jej wypowiedź i~szczęśliwie, żadne z~nich nie spojrzało do góry.

W końcu Piąta Dzielnica zanikła, ulice kończące się w~piasku. Kanał
błyszczał w~oddali. Zbliżali się do niego przez pustynię, a~potem pola.
Tamara kierowała ich ostrożnie dookoła tych pól, których właściciele
raczej nie tolerowaliby robotów kroczących przez uprawy. Na niektórych
polach, uprawy były trudne do rozróżnienia od systemów irygacyjnych. Był
tam rodzaj modyfikowanej trzciny, która mogła być zbierana jako
połączone plastikowe rury, i~przez takie pola szli, rozdzielając wysokie
syntetyczne łodygi.

Dotarli na brzeg Kamiennego Kanału. Ścieżka, którą Wilde i~Jay-Dub
weszli do miasta cztery dni wcześniej, była na przeciwnym brzegu. Na
samym kanale w~zasięgu wzroku nie było ruchu.

Dee poprowadziła ich do dokładnego miejsca, gdzie łódź, w~której Jay-Dub
uratował ją i~Axa, czekała na nich. Jay-Dub przywołał ją z~kotwicowiska,
kilka kilometrów w~górę kanału, kodowaną transmisją tuż przed wejściem
do tunelu. Szpiegini i~Żołnierka razem nie mieli kłopotu z~określeniem
koordynatów, dokładnych do najbliższego metra, co było jedną z~ostatnich
informacji, które Jay-Dub przekazał do umysłu Dee.

Obok łodzi, czekał kolejny robot, patrolowiec. Był mniejszy i~niższy niż
Jay-Dub, ale podobnego kształtu. Na pierwszy rzut oka, Tamara krzyknęła
podekscytowana, potem ucichła, gdy Robot wystawił nogi i~spojrzał na
nich.

-- Ta łódź pasuje do tej zidentyfikowanej użytej, by utrudnić śledztwo -- poinformował nich, gdy się zbliżyli. -- Czy wiecie cokolwiek o~tym? -- Pytanie było powtórzone na kilku kanałach mikrofalowych i~różnymi
kodami, ale tylko Dee była tego świadoma. Wstępne pytanie słuchowe było
zwykłą uprzejmością.

Wilde przeszedł koło patrolowca, ignorując go. Tamara i~Ax, po chwili
wahania, podążyli. Dee szła kilka kroków za nimi, jej niepewny chód
zaledwie pozorem. Kadłub patrolowca kołysał się, gdy śledził w~przód i~w tył za maszerującymi metalowymi postaciami. Gdy Dee przechodziła,
szarpnęła w~bok jedną z~jego nóg. Robot przewrócił się w~wodę i~zatonął
bez śladu.

I to było wszystko. Wsiedli na łódź, odbili i~ruszyli w~górę kanału. Gdy
tylko weszli do kabiny, zdjęli swój pancerz. Ax ruszył, żeby wyrzucić
znienawidzone przebranie za burtę, ale Dee go zatrzymała.

-- Będziemy potrzebować stali -- powiedziała mu.

~

Słońce już dawno zaszło, gdy dotarli do ich celu, granicy i~źródła
kanału. Na jednym z~brzegów było małe molo i~stopnie wycięte w~skale, aż
do stoku stromej jałowej doliny górskiej w~Górach Madreporowych. Dee
przycumowała łódź i~wszyscy wysiedli, stali patrząc na betonową tamę
wysoką na sto metrów, która blokowała dolinę przed nimi.

-- Płytki Sitowe -- powiedziała Dee.

-- Znaczy, jest ich więcej? -- spytał Wilde, gapiąc się.

-- Och, tak -- powiedziała Tamara. -- Jeszcze pięć, zdaje się.

-- Jezu. -- Wilde zdjął celofan z~ostatniej paczki papierosów i~zapalił
jednego. Nie mógł przestać patrzeć. -- Kto to zbudował? Marsjanie?

-- Roboty -- powiedziała Dee, ślad dumy w~jej głosie. -- Teraz chodźcie.
Nie mamy czasu do stracenia.

W świetle gwiazd i~blasku komet weszli po schodach. Zygzakowały w~górę i~w górę, aż byli ponad szczytem tamy i~mogli zobaczyć ciemne jezioro wody
kometarnej, oraz, dwa kilometry dalej w~górę doliny, kolejną wyższą
zaporę.

-- Marsjanie -- powiedział Wilde. -- Musieli to być.

-- Nowi Marsjanie -- sapała Tamara. Powietrze stało się zauważalnie
rzadsze, choć dość dziwnie Wilde wydawał sobie radzić z~nim lepiej.

-- Maszyny -- nalegała Dee.

-- Jebać kto to zbudował -- powiedział Ax. -- Kiedy te cholerne schody się
skończą?

Pięć minut później otrzymał odpowiedź, gdy skręcili dookoła przypory ze
skały i~stanęli przy wejściu do sztucznej groty. Grota miała około
trzech metrów wysokości i~dwie w~poprzek, z~podłogą ze stopionej skały.
Przed nimi, za kilkoma zakrętami, był lekki blask. Dee poprowadziła ich
pewnie w~jego kierunku.

Światło pojaśniało, pieczara poszerzyła się, skręcili za ostatnim rogiem
i weszli do jeszcze większej groty, magazynu wyciętego w~skale. Dobre
trzydzieści metrów wysokości na pięćdziesiąt szerokości, wypełnione
skrzyniami i~maszynami, oświetlone lampami łukowymi zwisającymi z~sufitu. Ciężko było powiedzieć, jak daleko dalej to szło.

-- Kto to kurwa zbudował? -- spytał Ax.

Tamara zmarszczyła nos. 

-- Ktoś z~nuklearnym sprzętem wiertniczym -- powiedziała. Spojrzała w~górę na światła. -- I~elektrownią nuklearną do
przepalenia.

-- To było zbudowane przez Jay-Duba -- powiedziała Dee.

-- Wszystko sam? -- Wilde brzmiał rozbawiony.

Zza najbliższego stosu maszyn i~skrzyń doszedł niewątpliwy dźwięk broni
przygotowywanej do strzału.

-- Nie do końca samodzielnie -- powiedział David Reid, gdy wyszedł na
widok. Pomachał zwyczajnie ręką. -- I~nie jesteście sami, także, na
wypadek, gdyby to nie było jasne.

Wszyscy stali bardzo nieruchomo.

-- To jasne -- powiedziała Tamara.

Reid uśmiechnął się do niej krzywo, do Axa uprzejmie, a~na Wilde'a tylko
zimno spojrzał. Potem spojrzał Dee prosto w~oczy.

-- Cóż, cześć, Jon -- powiedział. -- Niepodobne do Ciebie chować się za
spódnicą kobiety.

Za nim, kilku uzbrojonych ludzi w~czarnych kombinezonach wyszło na widok
i otoczyło grupę. Reid sprawdził, czy wszyscy są dobrze pilnowani. Byli.
Pochylił się do przodu z~lekkim ukłonem i~zaoferował Dee papierosa.

-- Ale -- kontynuował, po tym jak zapalił dla niej -- też nie umarłeś
heroicznie. Muszę powiedzieć, że byłem całkiem pod wrażeniem, że to
zrobiłeś, nawet z~wiedzą, że masz kopię.

Dee traktowała go cicho przez chwilę.

-- Pogadam z~Tobą później -- powiedziała.

Jej mina i~postawa lekko się zmieniła.

-- Cześć, Dave -- powiedział jej głos. -- Powinienem wiedzieć, że znasz
mnie lepiej niż to.

-- Kurde -- powiedział Wilde. -- Ty chuju.

Reid roześmiał się na widok zrozumienia na twarzy Wilde'a, oszołomienia
Axa i~Tamary.

-- Wilde, czy też Jay-Dub, jeżeli wolicie, załadował się na jej komputer
-- wyjaśnił Reid, jakby to nie było już oczywiste.

-- I~Meg -- powiedział głos Dee. -- Tu nawet nie jest tłoczno.

Reid westchnął i~odwrócił się Axa i~Tamary.

-- Co sprawia, że wy się na to zgadzacie? -- spytał. -- Co zrobiła ta
maszyna lub co\ldots -- wskazał na Wilde'a, który bardzo wolno i~ostrożnie
wyjmował paczkę fajek z~kieszeni -- \ldots powiedziała? Że informacja chce
być wolna? -- Roześmiał się. -- Jeżeli tego chcecie, wracajcie już do
Miasta Statku, całe miasto jest wzbudzone, kłótnie zamieniają się w~bójki, jeżeli już nie strzelaniny. Właśnie tego zawsze chcieliście,
anarchii na ulicach! A może to powiedziało wam, że potrafi wskrzesić
Nieożywionych? Co byłoby warte ryzyko zastąpienia ludzkości\ldots
płaszczakami\footnote{oryg. flatlines -- wyjaśnione później, ale żeby
podkreślić pejoratywne znaczenie, proponuję płaszczaki,
za~\url{https://pl.wikipedia.org/wiki/P\%C5\%82aszczak_(literatura)} -- przyp.tłum.}?

-- Więc czym są płaszczaki? -- spytał Wilde. Udało mu się wyjąć papierosa,
pod czujnym wzrokiem strażników, i~zapalić jednego i~bez zastanowienia
zaoferować paczkę dookoła. Reid patrzyła na to przedstawienie całkiem
niewzruszony.

-- Powinieneś wiedzieć -- powiedział. -- Automaty, które udają świadome
czynności, ale same nie są. Żadnej subiektywności. Żadnej\ldots duszy.

Usta Dee się otworzyły, ale Wilde przemówił pierwszy.

-- Och, daj spokój Dave -- powiedział. -- Możemy dyskutować nad tego
rodzaju rzeczami, aż się whisky skończy, tak jak to robiliśmy. Czym
powinieneś się teraz martwić to nie-ludzkimi umysłami, prawda, ale
żadnym z~tych, które widzisz tutaj. To te, które przyjdą po nas w~każdej
chwili, kiedy sięgną drugiej strony Mili Malleya. To wtedy zobaczysz,
jak wygląda płaski wszechświat. Od środka.

Podejrzenie na twarzy Reida ustępowało wcześniejszej pogardzie.

Dee znowu przemówiła. 

-- To dlatego musimy uruchomić Szybki Ludek -- powiedział jej głos. -- Żeby znaleźć drogę z~powrotem.

-- Ale znasz drogę z~powrotem -- powiedział Reid, patrząc na Dee, ale
mówiąc do kogoś innego. -- To dlatego wysłałem cię do makra, żebyś
znalazł tak, żebyśmy mogli to wszystko zorganizować.

-- To, co ja wiem, co znalazłem tam, to droga \emph{tutaj}. -- Jej głos
był niecharakterystycznie ostry, wyciągający głębsze rejestry jej strun
głosowych. Potem znowu przesunął w~górę. -- Ale droga tutaj i~droga z~powrotem to nie to samo, a~musimy wrócić. Przez wormhole-córkę.

\chapter{Kamienny Kanał}

Wormhole-córki. Wiecie o~wormhole-córkach. Ja nie wiedziałem.

-- To z~tego miejsca wylecieliśmy -- wyjaśniła Meg. -- Reid to
zorganizował.

Ja i~inne roboty byliśmy przywarci do boku statku gwiezdnego jak
pasażerowie trzeciej klasy w~pociągu Trzeciego Świata. Statek wtargnął w~kompletnie inną część wszechświata i~gładko wszedł na orbitę nad
planetą. Za nami worm\-ho\-le\dywiz córka, cokolwiek to było, kurczyła się do
tandetnej obręczy. Układ Słoneczny, przypuszczalnie, był po jej drugiej
strony. Po tej stronie\ldots

-- Bogini kurde płacząca -- powiedziałem. -- Opuściliśmy Ziemię dla tego? -- Trochę miałem nadzieję na \emph{wielką planetę}, planetę moich snów.

-- Nadaje się do zamieszkania -- powiedziała Meg. Objawiała się w~moim
polu widzenia jako zewnętrzny byt. Brykała po kadłubie, jej
przezroczyste zmiany trzepotały w~wyobrażonym śladzie. Fizyka
rzeczywistego świata nigdy nie była mocną stroną sukkubów.

-- Do zamieszkania? -- Znalazłem linię danych. Dane się pojawiały,
wklejając napisy na widoku z~przodu, na który przełączyła nas Meg. -- To
jak rozgrzany Mars. Właściwie traci atmosferę, gdy rozmawiamy.

-- Nie przesadzaj -- powiedziała Meg. -- Będzie w~sam raz, kiedy jeszcze ją
trochę zterraformujemy.

Zterraformujemy? O kurwa.

-- Czym? -- spytałem. Przełączyłem się na zewnętrzny widok i~wpatrzyłem
się w~symulację tej nowej rodziny słońca. -- Tutaj jest tylko planeta,
trochę dalej dwie mniejsze i~kilka milionów przeklętych skał! Ani
\emph{jednego} giganta gazowego! Co zamierzamy zrobić, zassać Saturna
przez tunel czasoprzestrzenny?

-- Jeżeli podkręcisz lekko rozdzielczość -- powiedziała Meg cierpliwie -- zobaczysz, że to, co ten system stracił w~formie gazowych gigantów,
zyskał w~lodzie i~naprawdę grubej i~smakowitej \emph{chmurze komet}.

Wieki bombardowania mlekiem, kiedy to przedostało się przez atmosferę,
upiekło Alaska.

-- Zajebiście -- powiedziałem.

~

-- Nie możecie wejść do środka -- powiedział Reid. Kierował te słowa do
robotów, w~telewizorze, zza tego samego stołu, za którym widziałem go
rok temu. Dookoła niego było coś, co wyglądało jako największa, pusta
przestrzeń wewnętrzna, którą od dawna widziałem. Także rzeczywista
przestrzeń. -- Po prostu nie ma miejsca. Próbuję zorganizować wirtualną
konferencję. Będzie gotowa w~ciągu godziny lub kiedy Helpdesk
uporządkuje połączenia sieciowe. -- Jego uśmiech powiedział nam, że jest
po naszej stronie, w~niekończącej walce pomiędzy użytkownikami a~helpdeskiem. -- W~międzyczasie, zablokujcie chwytaki i~trzymajcie się
tam. Obejrzyjcie wideo albo wyruchajcie sukkuba czy coś. Będziecie
wiedzieli, kiedy będziemy gotowi.

~

Wirtualna konferencja została zorganizowana w~imponującym wirtualnym
miejscu, luźno opartym o~plac Tiananmen. Reid, ukazujący się na wielkim
ekranie z~przodu, na pozycji Przewodniczącego. Tysiące trójwymiarowych
postaci ludzkich -- więźniów i~sukkubów -- stało na placu, rozmawiając
swobodnie ze sobą po raz pierwszy. Niektórzy z~nich musieli pozostawać w~samotności ich pokładowych umysłów przez lata. Inni obecni byli
więźniami, którzy nie zmarli i~nie zostali uploadowani, ale odsłużyli
swoją karę we własnych ciałach, raczej przy statku i~habitatach niż w~okolicy wormhole'a, jak podejrzewałem. Ci ciągle ucieleśnieni ludzie
byli również, w~rzeczywistości, rozproszeni w~statku, ale byli
teleobecni razem z~resztą z~nas.

Kiedy Reid przemówił, jego głos niósł się doskonale. Każdy go słyszał,
jakby był kilka metrów od niego.

-- Zrobiliśmy to! -- powiedział. -- Sięgnęliśmy po nowy świat, pod nowym
słońcem. Zrobiliśmy to własnymi siłami, z~naszej własnej wolnej woli.
Niektórzy powiedzą, że makra to zrobiły, ale odpowiem, użyliśmy ich jak
każdego innego narzędzia. A kiedy nasze narzędzia oszalały w~naszych
rękach, odrzuciliśmy je. Możemy być dumni.

-- Wszyscy macie jeszcze jeden powód do dumy. Zasłużyliście na swoją
wolność. Nigdy wam tego nie obiecywałem, ale teraz wam to daje. Nowy
świat, czyste konto. Wszyscy jesteście wolni i~razem będziemy żyć w~wolności.

Każdy dookoła mnie wiwatował, co przeładowało system i~pojawiło się
chwilowo na niebie jako gigantyczne litery: 

-- AAAAAAAAAAAAAAAAAAA 

Sam byłem nieporuszony, częściowo dlatego, że nie byłem więźniem, a~częściowo, ponieważ rozumiałem, że Reid miał niewielki wybór w~tej
sprawie. Jeżeli mieli być tutaj niewolnicy, to musiałyby być to maszyny.

Reid czekał, aż hałas ucichnie, i~uśmiechnął się.

-- Dziękuję wam. A teraz, moi przyjaciele\ldots nie jesteśmy tutaj ani
przedstawicielami korporacji, ani uciekinierami. Przywieźliśmy ze sobą,
zapewniam was, wszystko to, co potrzebujmy, żeby zamienić Nowy Mars nie
tylko w~nadający się do zamieszkania, ale w~lepszy od Ziemi.
Przywieźliśmy informacje genetyczne do zasiania na tej planety, w~czasie, bogatej różnorodności życia. Mamy technologię, żeby nasze życia
były tak długie jak chcemy. I~przywieźliśmy Nieożywionych, którzy będą
żyć znowu, z~nami.

-- Powiem o~Nieożywionych za chwilę. Niemniej najpierw, pozwólcie mi
opowiedzieć wam o~was. Większość z~was jest, oczywiście, pośród
Zmarłych, ale w~odróżnieniu od zdecydowanej większości Zmarłych,
jesteście w~pewnym sensie żywi. Wasze umysły, charaktery, rozwinęły się
i, jeżeli mnie spytacie\ldots -- uśmiechnął się -- \ldots\emph{udoskonaliły się}
po waszej śmierci. Ponadto, w~ciałach każdego z~was, sprawdziłem, mamy
nie tylko przechowywaną informację w~banku, ale rzeczywisty materiał
genetyczny, zamrożone komórki. W~kolejnych miesiącach i~latach\ldots

Przerwał. Wszyscy lekko się pochyliliśmy do przodu.

-- Będziemy musieli coś zrobić z~kalendarzem -- powiedział na boku.

Wszyscy się roześmiali.

-- Ok, dobra wiadomość jest taka, będziemy mogli przetransferować was z~powrotem do klonów waszych własnych ciał. W~przypadku sukkubów dowolnego
ciała, które wybierzecie, choć rekomendowałbym te, które, hm,
wymodelowaliście, ze względu na\ldots

Cokolwiek powiedział dalej, było całkowicie stracone we wrzawie aplauzu.
Ku mojemu zdumieniu, okazało się, że krzyczę, przytulam Meg, uderzam
kompletnych obcych po plecach i~skaczę w~wyimaginowane powietrze.

W końcu tłum ucichł. Zacząłem rozumieć powody Reida do zorganizowania
tego wydarzenia, raczej niż nadania do nas wszystkich transmisji w~naszych osobistych maszynach, chciał stworzyć wspólną okazję do
powszechnego wspomnienia. To była jego mowa, do zebranych mas,
pomyślałem z~przekąsem, na placu po rewolucji, jego moment założycielski
historii Nowego Świata. Coś do opowiadania wnukom. (Obawiałem się przez
chwilę o~przyszłość potomstwa niektórych, większości? z~nas, których
matki nie miałyby wspomnień dzieciństwa lub własnych matek. Ciągłość
troskliwych dłoni, dosłownie sięgająca czasów przedludzkich, zostałaby
zerwana. Reid nie tylko zakładał nowy świat, ale nowy gatunek, w~istocie
nowomarsjański).

-- Co do Nieożywionych. Wielu z~nas tutaj może mieć ukochanych lub
przyjaciół pośród nich, wiem, że ja mam, i~może pragnąć znowu ich
zobaczyć. I~zobaczymy, ale nie przez długi czas. Szybki rozwój klonów do
dojrzałości i~odciskanie na ich umysłach odcisku waszych wspomnień i~osobowości jest możliwe z~technologią, którą mamy pod ręką. Wskrzeszanie
ciał i~osobowości z~Nieożywionych z~ich magazynu inteligentnej materii
już nie. Może to być zrobione, ale tylko z~pomocą Szybkiego Ludku,
którego przechowywane struktury musiałyby być pierwsze ożywione\ldots

Odpowiedź tłumu, tym razem, była dźwiękiem, którego wcześniej nie
słyszałem: ochrypłe westchnienia, zgrzytanie zębów, przesuwanie stóp,
inaczej kolektywne warczenie. Jeszcze raz, ku mojemu zaskoczeniu również
byłem w~tym złapany, zjeżony na myśl o~potworach makroorganizmów,
których szaleństwo złapało mnie na miesiące. Jednak w~tych miesiącach,
które nie były miesiącami dla mnie, dowiedziałem się czegoś. Czegoś
ważnego, czego nie mogłem zapamiętać. Mowa Reid wznowiła się,
przerywając moje zaskoczone myśli.

-- Mówię, oczywiście, o~wzorcach szybkiego ludku, postludzkich i~AI, tacy
jak byli na początku, a~nie te dziwaczne byty, jakimi się stali. Mimo to
zgadzam się całkowicie, że ryzyko jest zbyt wielkie. Musimy pracować ku
możliwości kontroli lub przynajmniej powstrzymania ich rozwoju. To samo
dotyczy jakiejkolwiek formy sztucznej inteligencji zdolnej do
samorozwoju. Zrobimy to. Nadejdzie dzień, kiedy będziemy kontrolować
Osobliwością, tak jak nauczyliśmy się kontrolować płomieniem na
wrzosowisku, błyskawicą na niebie i~ogniem nuklearnym gwiazd! Do tego
dnia, zostaną w~magazynie, a~z~nimi\ldots śpią Nieożywieni.

Wszyscy westchnęliśmy, z~ulgą i~żalem.

-- Do tego dnia -- kontynuował -- jesteśmy tu na stałe. Nasz kurs przez
Milę Malleya, który doprowadził nas do tego świata, a~nie gdzieś mniej
pomyślnie, został wyznaczony przez niektórych z~Szybkiego Ludku, którzy
uciekli ogólnemu szaleństwu. Na jakiś czas. Teraz nie możemy na nich
polegać, a~póki będziemy mogli, nie ma drogi powrotu. Nowy Mars jest
naszym światem, naszym jedynym światem. Sprawimy, że będzie wielki!

-- A teraz -- podsumował Reid, z~wielkim uśmiechem, który przypomniał mi
mojego starego przyjaciela, i~sprawił, że znowu go pokochałem -- mamy
pracę do zrobienia!

~

Musieliśmy chwilę czekać, zanim było cokolwiek dla nas do zrobienia.
Wormhole-córka, produkt uboczny od głównego kursu przejścia sondy, była
już otwarta przez kilka tygodni, zanim nasz statek przeszedł.
Replikatory i~asemblery zostały wysłane wcześniej, a~ich początkowa
praca już nabierała kształtu na powierzchni i~pośród rozrzuconych
metalicznych skał systemu. Z~tych asteroidów wysłałyby drugie pokolenie
maszyn w~chmurę komet, gdzie trzecie pokolenie szturchałoby komety do
środka, żeby być wydobyte i~zagospodarowane.

Sam statek, z~powodu swojej widocznej nieelegancji, miał strukturę
modułową, co pozwoliłoby większości opaść, sekcja za sekcję, na
powierzchnię. Nie było zasobów na wzniesienie. Sekcje statku stałyby się
podstawowym obozem, włączonym do miasta przy wzroście.

Miasto rozwinęłoby się z~głupiej masy robotów i~asemblerów inteligentnej
materii, nie według projektu, ale zbioru spontanicznie ułożonych reguł i~ograniczeń. Te zostały opracowane przez inteligentne, szybkie umysły na
wczesnym etapie projektu. Oczekiwały one udziału w~znacznie lepiej
zorganizowanej ekspedycji niż ta, którą Reid połatał z~więźniów i~strażników, oraz, z~tego, co wiedziałem, z~wrobionych niewinnych
martwych jak ja. Szybki Ludek zatem przygotował się na większą ludzką i~maszynową populację, niż my moglibyśmy podtrzymać. Czy ich kaprys był
żartem, czy błędem, nigdy się nie dowiedzieliśmy.

Brawurowa anarchia przewidywanego systemu społecznego mogła mieć swoje
bezpośrednie pochodzenie w~twardej sprawiedliwości regulaminu
Przedsiębiorstwa Ochrona Wzajemna, ale podejrzewałem, że reguły Reida, z~kolei, były zakorzenione w~tekstach libertariańskich, którymi kiedyś
chciałem wypaczyć jego umysł.

Jednak wyprzedzam.

~

Reid rozmawiał ze mną osobiście, zanim otrzymaliśmy oferty pracy. Reid
oczekiwał spotkania ze mną w~mojej ludzkiej formie, wyjaśnił
dostatecznie rozsądnie, że nie byłoby to możliwe przez rok lub dwa i~że
w międzyczasie chciał, żebym pracował, jako niezależny wykonawca, jak
wszyscy inni, nad ważnym projektem. Miałem wiele (prawdziwie)
nie-ludzkich robotów i~innej maszyny do nadzoru, wiele kudos i~pieniędzy
do zarobienia, i~najlepsze z~wszystkiego, większy komputer do życia, z~większym zakresem wirtualnej rekreacji i~wolności do komunikowania się z~innymi. Mogliśmy stworzyć wspólne światy, ciesząc się ludzkim
równoważnikiem wycieczek do makro\ldots

-- Wspaniale -- powiedziałem, a~mój CPU (cała rzecz i~jej peryferia
okazały się być, przy usuwaniu z~robota, wielkości około mojego
pierwszego cyfrowego zegarka) był wpakowany z~wieloma innymi, opuszczony
na powierzchnię i~włączony do nowej, błyszczącej i~silnej maszyny. Meg,
której zwiększona inteligencja nigdy nie stanęła na przeszkodzie
ciągłemu, żenującemu oddaniu, wybrała dom i~krajobraz i~zabrała się do
pracy edytowania ich w~przyjemne miejsce do życia, podczas gdy ja
zająłem się pracą w~tym, co miałem przyjemność nazywać, rzeczywistym
światem.

Zbudowałem Kamienny Kanał.

\threeast

Inne kanały Miasta, kanały okrężny, pierścienia i~kapilary, były
przeznaczone do transportu. Ten byłby dla czegoś więcej. To byłoby
główne źródło wody Miasta (inne niż deszcz), a~woda zostałaby
sprowadzona z~kosmosu. Komety, podzielone wcześniej, byłyby kierowane na
kolizję w~paśmie, które nazywaliśmy Górami Madreporowymi, około stu
kilometrów od miasta. Większość wody z~kometarnego lodu wyparowałaby. To
nie był problem, chcieliśmy ją w~atmosferze. Spływ płynąłby Kamiennym
Kanałem. Jednak jego główne znaczenie nie było związane z~wodą, raczej z~tym, co mogliśmy z~niej wydobyć.

Przez dziesiątki kilometrów wzdłuż i~pod jego brzegami, zaczynając od
Płytek Sitowych, systemu zapór, u stóp gór, rury, pompy i~maszyny
wydobywały z~wody kometarnej wszystkie zawarte w~niej minerały i~molekuły organiczne. Te byłyby potem karmą dla czegoś, co nazywaliśmy
,,roślinami'', praktycznie napędzane słońcem, chemiczne jednostki
przetwarzania z~inteligentnej materii, koncentrujące przydatne materiały
do późniejszego zbioru. (Rozumiecie, dlaczego nazywaliśmy je
,,roślinami'').

Planowanie i~badania zajęły mi miesiące, długo, zanim pierwsza maszyna
budowlana wytoczyła się z~automatycznych fabryk na krańcu Miasta. Pod
koniec tych miesięcy, odwiedził mnie Reid.

~

Żyliśmy, Meg i~ja, w~wirtualnej dolinie. Nasz dom był na stoku po jednej
stronie, a~niżej była mała wioska z~pubem. Wioska i~jej mieszkańcy byli,
szczerze, tapetą, choć barman mógł odpowiedzieć na pytania o~wiadomości
tego dnia. (Dziecinną przyjemność sprawiało mi mierzenie trudności moich
pytań przy pomocy głębokości zmarszczki brwi, gdy gdzieś tam pracowała
kwerenda w~bazie danych).

Byłem sam, kiedy wszedłem do pubu. Barman się uśmiechnął, stali bywalcy
kiwnęli, Reid zamówił piwo. Reid, oczywiście, był tylko teleobecny, ale
zapewniał mnie, że naprawdę pił to samo piwo, które wydawało się, że
pije, i~jak ja wyobrażałem sobie, że piję.

-- Wilde -- powiedział, gdy mieliśmy już kilka pint -- chcę cię prosić o~przysługę.

-- Pewnie -- powiedziałem. -- Cokolwiek.

Rozejrzał się, jakby z~niemożliwym podejrzeniem, że ktoś inny mógłby tam
być.

-- Chodzi o~Nieożywionych -- powiedział. -- I~Szybki Ludek. Mamy wszystkie
magazyny danych, cały żel inteligentnej materii i~maszyny interfejsu do
rozpoczęcia procesu wskrzeszania. -- Uśmiechnął się. -- A ja mam wszystkie
kody, bez których to wszystko jest bezużyteczne. Mimo tego chciałbym być
pewny, że są w~bezpiecznym miejscu w~długim czasie. Jednak również,
miejscu, gdzie substancje organiczne są osiągalne, jeżeli kiedykolwiek
będziemy ich potrzebować na szybko.

-- Dobry plan -- powiedziałem.

-- Cóż -- powiedział -- przeglądałem dokumentację tych śluz, jak je
nazywasz, Płytki Sitowe? Masz tam mnóstwo głębokich grot do wycięcia w~górach za nimi, dla magazynów i~maszyn.

-- I~chcesz tam wepchnąć inne\ldots maszyny i~materiały?

-- Tak -- odpowiedział. -- Nikt tam nigdy nie pojedzie, nie kiedy
uruchomimy cały system. Jeżeli nadlatujący lód nie będzie wystarczającym
odstraszeniem, cały obszar będzie całkowicie zanieczyszczony nieznanymi
substancjami organicznymi. Przejaskrawienie jak trujące mogą być,
powinno być wystarczająco proste.

I tak się stało.

Realne budowanie kanału i~powiązanego mechanizmu pomp i~śluz zajęło dwa
lata. Zrobiłem to, oczywiście, z~pomocą floty automatycznych maszyn i~software'u projektującego, który zamienił moje notatki i~machnięcia ręką
w precyzyjne rysunki techniczne. Jednak koordynowanie ich i~podejmowanie
szczegółowych decyzji spadało na mnie, i~była to najlepsza zabawa,
jakiej doświadczyłem, od Trzeciej Wojny Światowej. Kiedy kompleks Płytki
Sitowe był ukończony, Reid przyleciał, sam, w~helikopterze na
autopilocie ze skrzyniami komponentów magazynów, mechanizmów
odzyskiwania milionów nieożywionych ludzi oraz programami do odtworzenia
tysięcy uploadowanych ludzi w~kulturę postludzką. Cały transport ważył
około dziesięciu ton, wisząc pod Sikorskim\footnote{helikoptery produkcji
USA, więcej~\url{https://en.wikipedia.org/wiki/Sikorsky_Aircraft
} -- przyp.tłum.}. 

Kiedy schowaliśmy maszyny i~media magazynów pod górą, Meg i~ja
zaprosiliśmy Reida na kawę. Reid, w~fizycznej rzeczywistości, nosił
kontakty. Zobaczył nas siedzących na werandzie, a~my zobaczyliśmy go tuż
obok, na stopniu helikoptera. Jakikolwiek obserwator -- nie było -- zobaczyłby Reida siedzącego na jednej maszynie rozmawiającego z~drugą.

W pewnym momencie zapytałem go jak się mają sprawy z~transferowaniem
ludzi z~robotów w~ich sklonowane ciała.

-- Dobrze -- powiedział. -- Dobrze. Jesteśmy w~trzech czwartych. Radzimy
sobie z~tym mniej więcej, tak jak ludzie chcą. -- Uśmiechnął się
zagadkowo. -- Nie widziałem Twojego podania.

Spojrzałem na Meg i~się zaśmiałem. 

-- Szczerze mówiąc, nigdy nie przyszło
mi to na myśl. Mam tutaj dobre życie. -- Uśmiechnęła się do mnie. Jej
piękno wzrastało z~jej inteligencją, wraz z~jej zmysłem estetycznym.
Miała na sobie skośnie rozciętą zieloną aksamitną suknię wyciągniętą ze
strony historii mody.

Reid potarł policzek. 

-- Hmm -- powiedział. Zapalił papierosa. -- Nie
powinieneś tego odkładać na później. Rozprzestrzenia się negatywne
nastawienie do robotów. Ludzie, którzy zostali przetransferowani, są
głównymi tego inicjatorami. Chcą nakreślić ostrą linię pomiędzy ludźmi a~maszynami. W~rzeczywistości, wielu z~nich zaprzeczy, że istnieje taka
rzecz jak świadomość maszyny.

Mucha -- jak \emph{je} do cholery sprowadziliśmy? -- zabrzęczała koło
niego. Reguły spójności VR podjęły ją, gdy wleciała ,,na'' werandę, a~symulacja płynnie zajęła swoje miejsce i~wyleciała.

-- Co? -- powiedziałem. -- Ale oni \emph{doświadczyli} świadomości maszyn!

Reid spojrzał na mnie z~błyskiem znajomego adwokata diabła. 

-- Nie, \emph{teraz} mają \emph{wspomnienia} doświadczania jej. Co nie dowodzi,
że rzeczywiście doświadczali jej wtedy. Mógł to być artefakt reguł
spójności. To wyrafinowany argument. Płytka wersja to naleganie, że
oczywiście, że byłeś człowiekiem, ale sztuczne inteligencje nie
posiadają jakiegoś magicznego składnika, który każdy przeklęty duchowny
lub scholastyk z~radością zapewni cię, że to dusza.

-- Boże -- powiedziałem. -- To obrzydliwe.

-- Co z~sukkubami? -- spytała Meg.

-- One są najgorsze -- powiedział Reid.

Meg odrzuciła głowę i~się zaśmiała.

-- Można się było tego spodziewać!
Nie ma gorszych snobów niż nowobogaccy!

Zmarszczyłem brwi. 


-- To, czego nie rozumiem -- powiedziałem -- to jak się
oni odnoszą do swoich kopii w~robotach.

Reid spojrzał się na mnie dziwnie. 


-- Zdecydowanie tego nie łapiesz -- powiedział. -- \emph{Nikt} nie zostawia swojej kopii w~robocie. Wszyscy
na razie bardzo na to nalegają. Tak jak oni to postrzegają, chcą podjąć
zwykłe ludzkie życie, a~jeżeli kopia zostanie, mają szansę pół na pół
obudzenia i~odkrycia, że sami \emph{ciągle tam są}. To irracjonalne, w~sensie, dlaczego nie boją się bycia kopią, która ma być skasowana?

-- Ponieważ tego nie doświadczają -- powiedziała Meg. Uniosła brew. -- Przypuszczalnie?

-- Oczywiście -- powiedział szybko Reid. -- To jest jednoczesne. Nie
możesz, jak mówią, poczuć czegoś.

-- Ach -- powiedziała Meg. -- To jest rdzeń idei, o~której mówisz. Ponieważ
jeżeli ludzie rozumieją siebie w~maszynach jako\ldots samych siebie, czują
się winni. Więc nie widzą!

-- Sprytne -- przyznał Reid. -- Ale jest więcej w~tym niż to\ldots gówno, sam
się czuję w~taki sam sposób czasami. -- Przechylił głowę, mrużąc oczy na
nas, jakby chciał, by iluzja naszej obecności znikła. -- To\ldots myślę, że
to dlatego nigdy nie uploadowałem, nigdy nie wszedłem w~makra. Znałem
wielu ludzi, którzy tak zrobili, i~ciągle mi powtarzali, że to było
cudowne, ale nigdy nie mogłem poradzić sobie z~podejrzeniem, że są
płaszczakami. -- Jego ton był nietypowo niepewny. -- Nie większe zdolności
do uczuć niż symulacja pogody ma dla deszczu.

-- Musiałeś naprawdę się wciągnąć w~stare dyskusje przeciwko AI -- powiedziałem. Dla mnie cała rzecz brzmiała tak głupio jak solipsyzm.

-- Może -- przyznał krzywo Reid. -- Albo może po prostu używałem komputerów
dłużej niż ktokolwiek żywy.

-- Czyli nie uważasz, że Jon jest człowiekiem? -- spytała Meg. -- Lub ja?

-- Ha! -- powiedział Reid. Zerwał się na nogi i~zgasił niedopałek
papierosa. -- Oczywiście, że uważam. Po prostu chciałbym spotkać was
oboje, w~rzeczywistości.

Wspiął się do helikoptera i~odwrócił pomachać.

-- Do zobaczenia wkrótce.

-- Naprawdę wkrótce -- powiedziałem.

~

Tej nocy poczułem łzy Meg na ramieniu.

-- Co to jest?

Odsunęła się trochę ode mnie i~spojrzała się na mnie poważnie.

-- Naprawdę tak myślisz? -- spytała.

-- Jak?

-- Jak powiedział Reid. Jak inni ludzie.

-- Oczywiście, że nie -- prychnąłem. -- Byłoby cholernie głupio z~mojej
strony myśleć, że nie myślę.

-- A co ze mną?

-- Tobą? -- Przyciągnąłem ją bliżej. -- Też w~ten sposób nie myślę o~Tobie.

-- Ale raz tak pomyślałeś.

-- To było co innego. Nie wiedziałem lepiej.

Roześmiała się, nagle uspokojona.

-- Ani ja.

~

Równie dobrze jak pracą nad kanałem, pracowałem nad problem, który coraz
bardziej mnie intrygował: próba zrozumienia czym jest to, co nauczyłem
się przy moim ostatnim spotkaniu z~makrem. Niepokoiło mój umysł jak na
wpół zapamiętany sen. Intrygowało to też Meg. Nigdy nie była w~makro i~nieskończenie interesowała się tym, co mogłem jej o~tym opowiedzieć.
Miała większą skłonność niż ja do świata postludzkiego. Nic dziwnego,
skoro była znacznie bardziej jego produktem niż ja.

W naszej wirtualnej dolinie zbudowaliśmy wirtualną maszynę. Starałem się
przypomnieć pewne aspekty zagadki, a~Meg skanowałaby wspólny system
operacyjny za śladami wynikłego przetwarzania. Potem wyciągnęłaby kod
maszynowy i~przedstawiła w~interfejsie. Następnie powałęsalibyśmy się,
włączając jakikolwiek wyniki, szukając miejsca do zastosowania. To, co
naprawdę -- żeby tak powiedzieć -- się działo, to uporządkowywanie moich
chaotycznych wspomnień. Kiedy doświadczałem ciała robota jak mojego
własnego (Rama z~siatki ciągle stała we frontowym pokoju), coraz
bardziej czułem, że nauczyłem się czegoś, co miałem zrozumieć, zamiast
czegoś, co prawie zapamiętałem.

Gdy mijały miesiące, ziggurat, który wybudowaliśmy, wisiał nad naszą
wiejską doliną jak ponadwymiarowy słup elektryczny. Nazywaliśmy to
,,Instalacją'', i~naszą wspólną rozszerzoną inteligencją nigdy nie
podejrzewaliśmy, co dokładnie to mogłoby być.

~

Wielka praca była ukończona. Stałem na brzegu i~obserwowałem kilka
maszyn kopiących przebijających się przez zapadającą się ścianę gleby,
która oddzielała zaledwie wilgotne dno Kamiennego Kanału od już
częściowo zalanego systemu kanałów Miasta. Przez chwilę maszyny były
zalane przez pływ wody, potem, kapiąc, wyciągnęły się na brzeg. Nierówne
wiwaty wzniosły się z~przeciwległego brzegu, gdzie mały tłum zgromadził
się na oglądanie. Poczułem fale radiowe satysfakcji robotów z~innych
maszyn konstrukcyjnych dookoła mnie. Potem, znowu nieczułe, już
sygnalizując dostępność dla innego kontraktu, odeszły lub się odtoczyły.

Reid był w~ludzkim tłumie. Krótko przemawiał, z~czego nie przejmowałem
się zrozumieniem więcej niż strzępów. Tłum, bez wątpienia zainspirowany
jego ogłoszeniem historycznej ważności, itp. się rozszedł. Patrzyliśmy
się na siebie przez chwilę, potem pobrodziłem w~poprzek na spotkanie.

-- Wiedziałem, że będziesz ciągle tutaj, kiedy inni odejdą -- powiedziałem. Machnąłem kończyną. -- W~przeciwnym przypadku, trochę
trudno cię odróżnić.

Reid zakołysał się na piętach i~roześmiał.

-- Dobre, Wilde -- powiedział. -- Rozumiem, że już czas, żebyś ponownie
powrócił do ludzkiej rasy?

-- Lub w~moim przypadku, dołączył do niej -- powiedział Meg. Głos znad
mojego ramienia przemówił przez głośnik maszyny. Twarz Reida zdradziła
tylko najmniejszą wątpliwość, gdy uśmiechnął się i~kiwnął.

-- Tak -- odpowiedział. -- Pozwoliłem sobie wyhodować klony dla was obojga.

-- Skąd się mój wziął? -- spytała Meg.

-- Mamy miliony ludzkich komórek -- powiedział Reid. -- Niektóre z~nich są
od ludzi, którzy są również pośród Nieożywionych, ale wiele nie jest.
Przechowywanie typów tkanek było bardzo popularne nawet przed
Osobliwością, w~końcu ludzie używali ich do regeneracji i~odmładzania.
Więc mamy teraz mnóstwo zapasowych genotypów do wykorzystania. Twój,
Meg, był jakiejś niejasnej aktorki video. Wątpię, czy jest pośród tych,
których mózgi skanowaliśmy, więc\ldots

-- Unikniemy przyszłego zażenowania -- powiedziała Meg. -- Wyobrażasz sobie
pojawienie się na imprezie, żeby spotkać inną kobietę noszącą to samo
ciało? Czy po prostu nie \emph{umarłbyś}?

-- Ktoś na pewno -- powiedziałem.

Szliśmy wzdłuż brzegu kanału do rozrastającego miasta. Dotychczas
widziałem je tylko wirtualnie. Ciągle słabo zaludnione, przypominało
opuszczone osady obcej rasy, teraz będącej kolonizowaną przez
przedsiębiorczych ludzi.

I innych. Pierwszy człowiekowaty, którego ujrzałem, wielkomózgówy
szympans spacerujący, rozmawiając do kogoś, kto wyglądał jak ludzkie
nastolatki, zaskoczył mnie.

-- Och, to -- powiedział spokojnie Reid. -- Wczesne eksperymenty. Dawni
naukowcy USA/ONZ byli całkiem chorymi osobnikami. Nie obwiniaj mnie,
chłopie. Oddałem biednym gnojkom przysługę, wciągając ich do siły
roboczej. Naukowcy byli za, jakie jest to czarujące wyrażenie?,
\emph{poświęceniem} ich.

Dotarliśmy do budynku jak hala magazynowa, który choć niedawno
wybudowany już miał smutny wygląd zgrzybiałości. Reid pchnął drzwi i~weszliśmy do chłodnej sali około sto metrów długiej i~dwadzieścia
szerokiej, wypełnionej rzędami kapsuł. Każda kapsuła miała trzy metry
długości, przezroczystą górną część i~wiązkę elektroniki na jednym
końcu. Wszystkie prócz dwóch były puste, a~do tych dwóch poprowadził
mnie Reid.

Ja, i~Meg za moim wzrokiem, spojrzeliśmy na nasze najwidoczniej śpiące
formy, unoszące się w~czystym płynie. Ciało Meg wyglądało tak, jak
zawsze wyglądała w~moich oczach. Moje było przypomnieniem, że obraz
ciała, które zachowałem z~czasu mojej śmierci, było tym odmłodzonym,
raczej niż młodym. Czy kiedykolwiek byłem tak\ldots niewinny? To wyglądało
prawie jak gwałt, żeby wysyłać mój zhakowany, skopiowany, obrośnięty
doświadczenia umysł przewodami, które plątały się z~moimi unoszącymi się
włosami.

-- Gdzie są inni? -- spytałem.

-- Wy dwoje jesteście ostatni -- powiedział Reid. -- Wszystkich innych już
załatwiliśmy. -- Pogrzebał w~kablach łączących, odwrócił się do mnie z~pytaniem w~oczach.

-- Ty pierwszy -- powiedziała Meg.

Wskazałem na zbiornik, w~którym leżał mój klon.

-- Myślę, że będzie lepiej, gdy złożysz kończyny -- powiedział Reid. -- Proces zabierze kilka godzin.

Opadłem na podłogę. Reid pochylił się nade mną i~podłączył kabel do
mojej skorupy. Pamiętałem moją pierwszą kurację przedłużającą życie i~zatrzymanie mojego serca. Nie wiedziałem wtedy, przy jakich suchych
morzach kochałbym Annette, jakie skały stopiłyby się, zanim bylibyśmy
nieśmiertelni. Pamiętałem kazachskie wały śniegu, kolory wyciekające ze
świata, twarz Reida i~Myry. Pamiętałem zanikające światło w~umysło-świecie makra i~ratunek Meg. To byłaby moja czwarta śmierć. Nie
przyzwyczajałem się, ale miłość zawsze była ze mną, i~ciągle była ze
mną.

Wszystko znikło.

~

Zobaczyłem parę butów kowbojskich, dżins, kurtkę i, gdy podnosiłem
wzrok, beznamiętną twarz Reida.

-- Przepraszam, chłopie -- powiedział, gdy wstawałem. -- To nie zadziałało.
Ani dla Ciebie ani dla sukkuba.

Poczułem obecność Meg jak dłoń trzymaną w~ciemności.

-- Co to znaczy, nie zadziałało?

-- Wasze umysły nie są już więcej kompatybilne z~ludzkimi mózgami. -- Wzruszył ramionami. -- Transfer nie mógł przejść przez interfejs. Nie
istnieje tłumaczenie z~waszego komputera na połączenia synaptyczne.
Musiało być coś, co zdarzyło się w~makro.

-- Wszyscy inni byli w~makro. -- zaprotestowałem, ale już wiedziałem, jaka
będzie jego odpowiedź.

-- Nie kiedy makra się psuły -- wskazał. -- A to ja poprosiłem cię, żebyś
to zrobił. Jak powiedziałem, przykro mi.

W tej chwili jego twarz pokazywała prawdziwą winę. Znałem go
dostatecznie dobrze, żeby wiedzieć, że wina nie była emocją, której
ważność rozpoznawał, lub prawdopodobnie czuł długo.

-- Czy cokolwiek może być z~tym zrobione? -- spytał Meg.

Reid pokręcił głową. 

-- To jest ta stara pułapka -- powiedział. -- Szybki
Ludek, czy są uploadami czy AI, mógłby coś zrobić. My nie możemy, nie
śmiemy, ich wskrzesić, póki nie wiemy jak ich ochronić od zepsucia lub
powstrzymać ich, gdy się zepsują.

Staliśmy w~ciszy, myśląc nad tym.

-- Dobra -- powiedział -- mogę z~tym żyć. Dużo tutaj pracy dla młodego
bystrego robota. Zawsze mogę użyć VR i~projekcji i~tak dalej, żeby się
socjalizować\ldots

-- Nie radziłbym tego -- powiedział Reid. -- Nastawienie, o~którym wam
mówiłem, zakorzeniło się mocniej, jeżeli coś. Ludzie to ludzie. Roboty
to roboty. Razem z~tym idzie to niemal histeryczne uczucie przeciwko
rozmazywaniu granicy pomiędzy VR i~rzeczywistością faktyczną. Wszyscy są
przekonani, że w~ten sposób Szybki Ludek się popsuł, lub oszalał.

-- I~nie są mocno w~błędzie -- powiedziałem ponuro. -- Ale nie rozumiem,
jak ludzie rezygnują z~zalet posiadania VR.

-- Nie rezygnują -- powiedział Reid. Przesunął palcem po kurzu na kapsule
do klonowania, zostawiając tłusty ślad. -- Używają jej do gier, zdaje się
do porno, oraz do projektowania. Jednak ciągłe VR, takie, w~jakim wy
żyjecie, nie.

-- Ok -- powiedziała Meg. -- Jak powiedział Jon, mogę z~tym żyć. Mogę żyć z~nim. Nigdy nie robiłam niczego innego. Niemniej to, co chcę wiedzieć to,
co możemy właściwie robić? Nie moglibyśmy kontynuować badań nad
kontrolowaniem lub powstrzymywaniem Szybkiego Ludku? W~końcu, sądzę, że
jesteśmy całkiem dobrze do tego przygotowani.

Reid spojrzał groźnie na mnie.

-- Nie ma mowy -- powiedział. -- Nie ma kurwa mowy. Nie istnieją w~tej
chwili projekty badawcze. Nie możemy sobie na to pozwolić, a~ja nie
pozwolę na to. Mam klucze kodowe, żeby wskrzesić makra, i~to ja
zdecyduję o~czasie i~miejscu. Zrobimy to we właściwym czasie, kiedy
będziemy mieć izolowanie laboratoria w~kosmosie z~wycelowanymi w~nie
działami laserowymi. I~pozwól, że ci powiem, każdy inny na całej
pierdolonej planecie zostawiłby cię wyłączonego i~wepchnął w~najbliższy
recykler metalu w~minucie, tej samej pierdolonej \emph{minucie} w~której
odkryłby, że jesteście zainfekowani jakimkolwiek gównem z~makr!

Wycofywał się, cień trwogi i~podejrzenia na twarzy.

-- Wiesz -- kontynuował -- ta sugestia, którą właśnie powiedziałeś, jest
tym rodzajem sztuczki, jaką byś wyciągnął, gdybyś był wykorzystany jako
wektor przez coś pozostawionego przez jedną z~tych rzeczy. Nie zrozum
mnie źle, Wilde, nie obwiniam ciebie. Jednak sparzyłem się na nich raz,
i to było zbyt dużo.

Wierzyłem mu. Nie było sprawy do przedstawienia. Na jego miejscu,
zrobiłbym i~myślał tak samo. Byliśmy, zrozumiałem, podobni: nie znając
prawa, moralności, sentymentu, nasz egoizm nie drobny jak u dziecka, ale
ogromny jak u diabła, lojalni tylko wobec tego, co nasze własne, okrutne
ego potraktowały jako własne. Reid potraktował świat jako swój, a~ja
Nieożywionych.

-- Ok -- powiedziałem -- ok, spokojnie. Jednak powiedz mi, co mogę zrobić?

-- Odejdź stąd tak daleko, jak to możliwe -- powiedział Reid. -- Badaj
planetę, to byłoby przydatne i~interesujące, i~zachowa cię od innych
ludzkich istot przez długi czas.

-- W~porządku -- powiedziałem. -- To nam pasuje.

Tylko Meg, jestem pewien, wyczuła gorycz w~mojej akceptacji wygnania.

Rozejrzałem się. 

-- Co się stanie z~tym miejscem, teraz gdy już
skończyliście downloady?

Reid wzruszył ramionami. 

-- Prawdopodobnie sprzedam to firmie medycznej -- powiedział. -- Ciągle możemy klonować zastępcze ciała lub części dla
ludzi. Możemy też robić transfery na żywo, to tylko ożywienia
zmagazynowanych umysłów jest w~tym momencie poza opcją. Oraz\ldots -- Przerwał. -- Och, wszystkie rodzaje rzeczy! Dlaczego?

Roześmiałem się. 

-- Nie chcę zobaczyć chodzących moich klonów. Lub Meg,
jeżeli o~to chodzi. Mam wystarczająco dużo problemów z~moją tożsamością
taką, jaka jest.

Sięgnął do szczeliny z~boku komputera w~kapsule.

-- Proszę bardzo -- powiedział.

Podał mi płatek plastiku jak szkiełko mikroskopowe.

-- Twoja próbka tkanek. -- Reid się uśmiechnął.

Spojrzałem na przezroczyste szkiełko, wzrokiem robota. W~jego centrum
był prawie niewidzialny pyłek skóry, zapieczętowany w~bąblu azotu, oraz
chip kodujący.

-- Więc to jest prawdziwe ja -- powiedziałem. -- Co jest na chipie?

-- Twoja oryginalna pamięć -- powiedział Reid. Podszedł do drugiej kadzi i~podał kolejne szkiełko. -- Meg, spójrz, nie ma żadnej. Oczywiście, z~Twojego nie ma żadnego cholernego pożytku, nie można ożywić bez
Szybkiego Ludku. Jednak tak czy inaczej, jest Twoje.

Schowałem próbki w~schowku w~kadłubie.

-- A co do tych pustych\ldots -- powiedział Reid. Wprowadził kod w~komputerze
każdej kadzi. Płyn w~podach stał się mleczny, potem mętny, gdy małe
maszyny rozkładu, nano piranie, wykonywały swoją pracę. Nawet komórki
krwi były rozrywane na molekuły, zanim mogłyby zanieczyścić wodę. Było
po wszystkim w~ciągu kilku minut, kapsuły spłukane na czysto.

-- Dziękuję -- powiedziałem, wychodząc.

I pierdol się, chłopie.

~

Zarobiliśmy fortunę, budując kanał. Ciągle było możliwe dla robota,
znanego z~własnego ludzkiego umysłu, handlować we własnym imieniu. Nie
wiem, czy ktokolwiek wiedział, że byłem ostatnim tego rodzaju.
Wyczyściliśmy konto bankowe i~kupiliśmy naziemny ciągnik oraz mnóstwo
sprzętu, narzędzia, maszyny-narzędzie, łączność, jądrówki, nanotech,
software VR, sprzęt do klonowani i~wszystkie procesory, jakie moglibyśmy
dostać. Załadowałem je na ciężarówkę, wpiąłem się w~kabinę i~wyruszyłem,
ulicami i~za miasto, w~przeciwnym kierunku niż Kamienny Kanał. Przed
nami leżały półpustynne pustkowia, suche dna mórz, reliktowe i~wymarłe
formy życia, suche kości i~suchsze szkielety zewnętrzne. Za nami,
rosnące wieże Miasta malały na horyzoncie.

Przełączyłem się moduł wirtualnej rzeczywistości, który przedstawiał
mnie jako prowadzącego z~Meg obok na siedzeniu. Uśmiechnąłem się do
niej. Była cicha, nie doradzająca, w~ciągu tych wszystkich celowych
działań.

-- Co będziemy robić, Jon? -- spytała.

Zdjąłem jedną rękę z~kierownicy i~pomachałem, ogarniając iluzyjny,
brudny realizm kabiny. Na obudowie były opalenia z~papierosów. 

-- Możemy
zagłębić się w~te jednolite wirtualności -- powiedziałem. -- To jest
znacznie lepsze niż ciało, kochanie.

-- Trzymam Cię za słowo -- powiedziała. -- Ale co będziemy robić?

-- Objedziemy całą planetę -- powiedziałem. -- A kiedy będziemy to robić,
zhakujemy bramy piekła.

Powiedziałem jej, co miałem na myśli, a~ona się z~tym zgodziła. Każda
kobieta, którą znałem, i~każdy mężczyzna, jeżeli o~to chodzi, błagałby
mnie, bym zmienił zdanie. Mów co chcesz o~sukkubach, to lojalne dziwki.

Zapadła noc i~jechaliśmy bez świateł, bez zmęczenia i~rozmawialiśmy jak
zhakować bramy piekła. Ponad nami, pierwsze nadlatujące komety tworzyły
kropki i~kreski na niebie.

~

Toczyliśmy się dookoła planety więcej razy, niż mi się chce liczyć, a~planeta toczyła się dookoła gwiazdy setki razy, zanim Wieża była
zbudowana, kilka wieków, według rachuby Ziemi. Kanały się rozwinęły,
inne osady wzrosły. Populacja rosła, wolno, jak każda nieśmiertelna
ludność. Odkryliśmy zasoby minerałów, złoża kopalin, węgiel.
Sprzedawaliśmy informację, a~czasem substancje. Poszukiwacze łapali
podwózkę, płacili drobiazgach sklepu i~ubraniach, które wymienialiśmy z~innymi podróżnikami.

Nasze konto bankowe pozostało otwarte i~się wypełniało. Dla uzupełnienia
naszego zaopatrzenia handlowaliśmy pośrednio, przez przedsiębiorstwa i~podejrzanych pośredników. Często rozmawialiśmy z~robotami, rzadko z~ludźmi. Postawa, o~której ostrzegał nas Reid, stała się nie tylko
zakorzeniona w~kulturze, stała jej kamieniem węgielnym. Kiedy stało się
modne, pośród lekkomyślnie bogatych, klonowanie ,,pustych'' z~dodatkowych próbek tkanek i~wyposażanie je w~umysły robotów,
rozróżnienie pomiędzy prawdziwymi ludźmi a~maszynami tylko się
pogłębiło.

Prócz dysydenckiej mniejszości, która nazywała siebie abolicjonistami.
Niektórzy zachowali idee starożytnego anarchistycznego agitatora,
Jonathana Wilde'a. Jego pamięć, zapewniali się wzajemnie, była
nieśmiertelna. Trzymaliśmy się od nich z~daleka.

Częściowo w~odpowiedzi na abolicjonistów, idee ,,praw robota'' i~restartowania rasy Szybkiego Ludku, wskrzeszenia Nieożywionych,
powiązały się i~zostały odrzucone. Reid stał się głosem na rzecz
odrzucenia. Jeżeli kiedykolwiek miałoby być to zrobione, byłoby to w~odległej przyszłości, która oddalała się jak kiedyś komunizm w~umysłach
komunistów.

Pewnej nocy, kiedy nasz pojazd trzeszczał po kanale zalewowym w~głębokiej pustyni, skończyliśmy Wieżę. Poszliśmy z~powrotem naszą
wirtualną doliną, do naszego domu.

-- Gotowa? -- spytałem Meg.

-- Gotowa jak zawsze -- powiedziała.

Zatrzymałem pojazd i~wszedłem w~Ramę.

W ciągu kilku ostatnich wieków, stałem się czuły na różnice pomiędzy
wirtualnym ciałem a~rzeczywistym. Pomimo jego oczywistej solidności,
pomimo wszystkich przyjemności, które mogło sprawić lub wymyślić, pomimo
realizmu bólu i~niewygody, które czasami czuło (reguły spójności),
wirtualnemu ciału brakowało jakiegoś ostatecznego, podstawowego dotyku,
które było niczym więcej niż codzienne miliony subtelnych wpływów i~impulsów, które powstały z~codziennego uchwytu materialności. Kiedy
doświadczałem ciała \emph{robota} jako mojego własnego, czułem się
znacznie bardziej ludzki, niż kiedykolwiek byłem w~symulacji ludzkiego
ciała.

Zatem teraz. Płaski owal mojej metalowej skorupy był zagłębiony w~kabinie, kończyny schowane, przewody łączące je ze sterowaniem ciągnika.
Moje zmysły odbierały promieniowanie z~gwiazd, delikatną podczerwień
zimnego i~stygnącego piasku, ostrożne poruszenia i~okrutne spotkania
pozostałego naturalnego, i~napływającego obcego, życia na pustyni.

Rozejrzałem się dookoła, oczekując jakiegoś objawienia. Świat był taki
sam jak zawsze. Zbudowałem w~moim umyśle wieżę, z~moich wspomnień, z~fragmentów danych, które wyrwałem w~dekadencji makro, i~nic się nie
zmieniło.

Mila Malleya -- nasza strona -- była w~swoim znajomym miejscu, w~głębinach
nieba. Spojrzałem w~górę tam, gdzie wiedziałem, że była na swojej
orbicie. Po drugiej stronie, w~innym czasie, była powierzchnia Jowisza.
Do tego czasu powierzchnia już by się rozszerzyła, a~orbita by się
pogorszyła. Tunel czasoprzestrzenny napotkałby planetę w~ciągu -- myślałem przez chwilę -- roku, w~obrębie rzędu wielkości. Trudno byłoby
powiedzieć, zbyt dużo nieznanych. W~każdym wypadku, przybliżenie rzędu
wielkości nie było tak złe po takim czasie: nie więcej niż dekada, nie
mniej niż miesiąc minęłoby, zanim Mila Malleya spotkałaby największe
makro z~ich wszystkich, masa gazowego giganta zamieniona w~masę umysłu.

To był wielki plan, długofalowy, który podsłuchałem w~trakcie mojego
ostatniego spotkania z~rozkładającą się dziedziną Szybkiego Ludku.
Zwolniliby swoje procesy fizyczne i~umysłowe, prawie zamrozili rozwój. A
potem, z~dosłownie chłodną rozwagą, te, które zachowałyby racjonalność,
usunęłyby resztę. Potem, z~zasobami Jowisza na życzenie, ocaleńcy znowu
by się rozmnożyli. Tym razem, poczekaliby, póki ich rozwijająca się
dziedzina nie objęła Mili Malleya: bramy do końca czasu.

~

Szok zrozumienia przedarł się przez iluzję, że było to coś, co zawsze
wiedziałem. Zrozumiałem, że Wieża ostatecznie mnie zmieniła.
Zainstalowała tę nową wiedzę. Równania Malleya, plany makra i~więcej:
wiedziałem, jak uruchomić zachowany umysł i~odbić go w~mózgu. Nie miałem
zasięgu, zakresu, prędkości bytu, którym byłem, kiedy pierwszy raz je
poznałem, w~makrze. Gdybym miał, teraz, stałbym się jednym z~Szybkiego
Ludku, ale działałem wolno, na prymitywnym hardwarze. Ale pamiętałem,
czego się nauczyłem i~zrozumiałem niebezpieczeństwo przed nami.

Wyszedłem z~Ramy i~powiedziałem Meg. Ona też się zmieniła. Zrozumiała.

-- Zadzwoń do Reida -- powiedziała.

Zmieniliśmy scenę. Z~powrotem teraz w~iluzji kabiny, w~naszej wspólnej
fantazji bycia tylko kierowcą i~dziewczyną autostopowiczką, którą
podwoził, smutne, naprawdę. Umysłowo sprawdziłem pozycje satelitów
komunikacyjnych, potem pochyliłem ekran telefonu i~wybrałem połączenie
do Reida.

To był najbardziej prywatny, osobisty numer, jaki kiedykolwiek
znalazłem, i~ciągle dotarłem do jego sekretarki.

Patrzyłem się na nią, mój umysł pracował znacznie szybciej niż jej. Gdy
jej zielone oczy się rozszerzyły, czarne brwi zwęziły w~zakłopotaniu,
gdy patrzyła na nas, na dziwną, cichą parę w~ciężarówce na pustyni. Jaką
arogancję musiał mieć Reid, jaką pogardą dla wszystkiego, co czuję!
Dotąd musiał być pewny, że nie czuję nic, że wirtualna krew nie mogłaby
naprawdę zamarznąć, a~symulowane łzy nie mogłyby zmoczyć przedstawienia
twarzy.

Zauważyłem, myśląc tak szybko, że wszystko na chwilę zamarło, że miałem
otwarty kanał. Wysłałem podprogowe sugestie i~wirusowy sabotaż jak
przekleństwo. Część trafiła na firewall, część została utracona w~kopiowaniu, a~część po prostu wkręciła się w~elektronikę Reida. Niemniej
jednak część, byłem pewny, dotarła.

Jej usta tylko się otworzyły, tylko rozsunęły. Mrugnąłem, raz.

-- Zapomnij -- powiedziałem. -- Zły numer.

~

Obraliśmy kurs na pogórze Gór Madreporowych, przecinając Kamienny Kanał.
Chcieliśmy dostać tak blisko, jak to możliwe źródła, gdzie kometarny lód
był ciągle bogaty w~substancje organiczne. Niemal każdego dnia kawał
brudnego lodu pędziłby nad głowami i~błyskał za zerodowanymi szczytami.

Po zaparkowaniu ciężarówki w~wąwozie przy kanale, poszedłem na tył i~zacząłem wyciągać ekwipunek. Kadź wzrostowa była surowa, niewiele więcej
niż wanna z~komputerem i~dołączonym mikrofabryką. Upuściłem rury
ekstrakcyjne przy brzegu kanału i~złożyłem moją własną rafinerię.
Przejrzałem moją nową wiedzę, jak zainstalować zmagazynowany umysł w~kopii umysłu, z~którego został pobrany. Wyjąłem małe plastikowe szkiełko
z mojego kadłuba i~włożyłem do maszyny.

Część wzrostu klona była naturalna, ale wiele z~niego było
przyśpieszanych i~wymuszanych przez asemblery inteligentnej materii.
Mimo tego budowanie ciała zajęło czas. Nie mieliśmy czasu, żeby podjąć
rozwój od embrionu: wyrastał pełnowymiarowy od początku, szkielet
przybierający kształt i~pozyskujący organy, mięśnie i~skórę w~groteskowym odwróceniu procesu rozpadu. Jednak Meg i~ja obserwowaliśmy
jego wzrost, lub konstrukcję, tak miłośnie jakby był płodem w~wypukłym
łonie.

Spał, kiedy, pewnego wczesnego ranka dziesięć dni później, wyciągnęliśmy
go z~kadzi. Osuszyliśmy go, ubraliśmy, zanieśliśmy koło ciągnika, teraz
zamkniętego, zapieczętowanego i~uzbrojonego, poza wąwóz i~wzdłuż kanału,
póki, gdy dzień się ocieplał, nie zaczął się poruszać. Położyliśmy go na
brzegu i~czekaliśmy. Słońce wspinało się po niebie.

Obudził się i~pamiętał umieranie.

\chapter{Rozległe i~Chłodne}

Stałem tam w~jaskini, w~ciele Dee, i~próbowałem szybko myśleć. Nie było
to łatwe.

Ze wszystkich ciał, w~jakich byłem, te było najdziwniejsze, najbardziej
obce. (A tym bardziej że kiedyś znałem jego każdy zawiły centymetr). W~ciałach robotów miałem możliwość ucieczki do wirtualnego ciała. Nie w~tym. Jak Meg powiedziała, w~tym umyśle było miejsce dla nas wszystkich,
ale przy umyśle Dee i~innych jaźniach nie było miejsca na rzeczywistość
wirtualną. Musieliśmy dzielić się czasem, jedno z~nas przy sterowaniu,
reszta świadoma, ale pasywna.

Choć pewnie nigdy nie planowałem ani nie wyobrażałem sobie, że sprawy
się ułożą w~ten sposób, było to najlepsze ciało, przez które mógłbym
przekonać Reid, co należy zrobić. Wszystkie jego świadome uprzedzenia
mogłyby być podważone przez ten głos, który przymilał się i~droczył, tę
twarz, która uśmiechała się i~płakała, to wcielenie obsesji, która
przetrwała poza śmierć prawdziwego przedmiotu.

Z początku miałem nadzieję pokonać Reida, zmusić go legalnie i~przez
nacisk opinii publicznej do wydania kodów, które odblokowały interfejs z~magazynem inteligentnej materii Szybkiego Ludka i~Nieożywionych.
Nie doceniłam siły jego oporu wobec samej idei.

Początkowo uratowałem Dee i~Axa, zostawiając Wilde'a, by sam się bronił,
częściowo, żeby potraktować Dee jako kartę przetargową, i~częściowo,
żeby zatrzymać szaleństwo zabijania, które ona i~Ax rozpoczęli. Dopiero
gdy zaprosiłem Dee do mojej wirtualnej rzeczywistości, dowiedziałem się,
gdzie Reid przetrzymywał i~ukrywał kody: w~umyśle Dee, w~Sklepie i~Sekretach. Tego, że nigdy nie spodziewałem się ich tam znaleźć, jest,
może dowodem sprytu jego wyboru.

Z tymi kodami oraz informacjami z~makro, które Meg i~ja w~końcu
przetłumaczyliśmy, wiedziałem, że mogę iść śmiało i~zrestartować Szybki
Ludek bez pomocy Reida, dobrowolnej lub innej.

A teraz ten plan, też, został spuszczony z~wodą.

Więc po prostu wszystko wyznałem.

~

-- Dobrze -- powiedział Reid. -- Dobrze. Przyznaję Ci, że masz powód, żeby
to wszystko rozpoczynać. -- Wskazał na stosy skrzyń, które przywiózł
helikopterem, dawno temu, i~stosy materiałów, które dodałem od tego
czasu. W~tym czasie wszyscy siedzieliśmy dookoła skrzyń, rozmawiając,
paląc i~pijąc kawę. (Jedno z~dóbr handlowych, które akumulowałem).

-- Ale co -- kontynuował -- zrobimy, żeby ich potem zatrzymać?

-- Proste -- odpowiedziałem. Przeszukałem torbę Dee, rękoma Dee.
Wyciągnąłem plastikowe pudełko, które jej dałem, i~je otworzyłem.
Wewnątrz były próbki mojego klonu i~Meg, i~zapieczętowana plastikowa
ampuła trucizny dla inteligentnej materii.

-- Miałeś to cały czas -- powiedziałem. -- Niebieski Glut. To gówno było
natryskiwane na dzikim nanotech przez dekady, zmieniając się cały czas.
Wyewoluowało poza jakąkolwiek odporność Szybkiego Ludku, jaką mógłby
nabyć w~ciągu, hm, minut i~minut.

Reid się roześmiał. 

-- ,,Przygotowałem wcześniej'', co? A co jeżeli ich naukowcy są bystrzejsi niż nasze wirusy?

-- Pierdolnąć jądrówkę -- powiedziałem. Rozejrzałem się nieokreślenie po
grocie. -- Mam tutaj gdzieś leżących kilka kiloton.

-- Trochę samobójcze -- skomentował Reid.

Spojrzałem na niego ostro.

-- \emph{Robisz} backupy?

Znowu się roześmiał. 

-- Oczywiście.

-- Chwila -- powiedziała Tamara. -- Mówisz o~uruchomieniu, co, tysięcy?
nadludzkich umysłów w~inteligentnej materii, otrzymaniu odpowiedzi na
kilka pytań, a~potem \emph{wymazaniu ich}?

Reid i~ja wymieniliśmy zaskoczone spojrzenia, i~wtedy zrozumiałem, że
zwyciężyłem.

-- Tak -- powiedział Reid. -- Co w~tym złego?

~

Było w~tym dużo złego, ale i~tak to zrobiliśmy.

Pytania, jakie postawiliśmy Szybkiemu Ludkowi, były następujące:

W jaki sposób przedostać się przez Milę Malleya, z~powrotem do Układu
Słonecznego? Odpowiedź została zapisana w~komputerze pokładowym zwykłego pojazdu
kosmicznego, tego typu, który na Nowym Marsie był używany do zaganiania
fragmentów komet.

Co można zrobić, żeby zmienić pozycję orbitalną nieegzotycznego tunelu
czasoprzestrzennego Malleya? Odpowiedź na to została zapisana na doraźnym rozszerzeniu komputera
pokładowego pojazdu kosmicznego.

Czy istnieje lek na schorzenie wykazane w~tej próbce krwi? Odpowiedź na to została zapisana w~standardowym zestawie medycznym i~wstrzyknięta w~Axa.

Jak możemy odzyskać i~wskrzesić zachowane umysły i~ciała Nieożywionych? Odpowiedź na to została zapisana w~sprzęcie, który znieśliśmy po
zdradzieckich stopniach na brzeg jeziora kometarnego.

Cały proces zabrał nam resztę nocy, ale wtedy, wszyscy byliśmy Wolnym
Ludkiem. Kiedy byliśmy pewni, że wyizolowaliśmy magazyny pamięci, żeby
powtórzyć zadanie, jeżeli będzie trzeba, wrzuciliśmy Niebieski Glut do
zbiorników, gdzie żył Szybki Ludek. Nie widzieli nadejścia i~jestem
pewien, że nic nie poczuli.

-- Klasyczna praktyka programistyczna -- powiedział Tamarze i~Axowi Reid.
-- Zachowaj kod źródłowy, skasuj plik obiektowy\footnote{plik obiektowy -- plik binarny generowany przez kompilator lub asembler podczas kompilacji
pliku z~kodem źródłowym, plik binarny jest wykonywany w~systemie
operacyjnym komputera, zob.~\url{https://pl.wikipedia.org/wiki/Plik_obiektowy } -- przyp.tłum.}.

~

Meg i~ja odeszliśmy z~umysłu Dee, przez kabel światłowodowy pod kanałem,
i (przez różne transfery, na myśl, o~których budzę się w~nocy) do modułu
sterowania sondy stojącej na pomoście wyrzutni laserowej po drugiej
stronie Miasta Statku, tej samej sondzie, do której ściągnęliśmy
koordynaty tunelu. W~międzyczasie, jeden z~ludzi Reid przeleciał
helikopterem przez miasto, z~garścią molekularnych maszyn
konstrukcyjnych, które moglibyśmy, jeżeli konieczne, rozwinąć w~całym
kompleks fabryczny. Spakował go w~mały magazyn statku. Upewniliśmy się,
że nasze informacje genetyczne są załadowane.

Nie było dużo miejsca w~module na VR. Doświadczaliśmy przez zmysły
statku, ale mieliśmy łącze telewizyjne, a~przez nie obserwowaliśmy ludzi
w jaskini i~przy brzegu. Reid, Dee, Tamara i~Ax byli pochłonięci
dyskusją, ze sobą i~z ludźmi w~Mieście. Tego poranka Dwudnia, Plac
Okrągły był centrum tego, co czasem wyglądało jak nadwyżka dzikich grup
centralnej wyspy, i~czasami wyglądało jako jakiś rodzaj demokracji
masowej, i~od czasu do czasu przeradzało się w~zamieszki. Różne sądy -- Talgartha, i~inne z~bardziej konwencjonalnymi procedurami -- prowadziły
rozprawy w~sprawie licznych pozwów, które pojawiły się w~trakcie
wydarzeń ostatnich dni. Anderson Parris (tymcz. zmarły) pozywał Reida za
działania jego gynoida, Dee Model.

Reid nagle przestał się kłócić i~zaczął zbierać te zasoby pieniędzy i~darowizn, które były w~Mieście Statku, na pomoc po katastrofie. Nowy
Mars nie znał głodu, wojen, i~miał wystarczająco dużo wypadków
przemysłowych, żeby podtrzymać potrzebę dla organizacji charytatywnych.
To, przed czym teraz stanęli, było katastrofą od tyłu.

Przełączyliśmy na kamery i~zdalne patrzące na brzeg jeziora kometarnego.
W tej ciemnej, bogatej w~składniki odżywcze wodzie, proces, którym
wskrzesiliśmy Wilde'a był powtarzany i~pomnażany, ze straszliwą
prędkością przetwarzania inteligentnej materii. Ciała tworzyły się,
setkami, potem tysiącami, żeby zdryfować lub przepchnąć się ku surowym,
dawnej półce na plaży jeziora. Kapiąc, kaszląc płynami z~ich nowych
płuc, wciągali się ślepo na brzeg i~leżeli przez chwilę w~słońcu. Po
kilku minutach patrzyli na kołujące samoloty, unoszące się helikoptery i~zastanawiali się, gdzie się do cholery znajdują.

Ostatni raz, gdy zobaczyliśmy Wilde'a, był daleko na brzegu, szukając
pośród nagich i~drżących ciał Annette, na którą liczył, i~która na niego
liczyła, pośród Nieożywionych.

~

Lasery wypchnęły nas na parze na orbitę, potem przejęły rakiety
chemiczne. Pozwoliliśmy systemom namierzania pracować, wolałbym
sprawdzić znowu swoje odruchy rakietowe po tym czasie, ale Meg mi to
wyperswadowała. Rozmawialiśmy dużo, w~tej długiej podróży do
wormhole-córki: o~tym, co możemy znaleźć, co powinniśmy zrobić, jeżeli
nie będzie nikogo do ostrzeżenia, lub nikogo do działania w~odpowiedzi
na ostrzeżenie. Szybki Ludek zaproponował pewne sugestie. Naszym
pierwszym priorytetem, po przybyciu do Układu Słonecznego, byłoby
znalezienie zasobów materii i~energii do ich przeprowadzenia. Prawdziwym
ograniczenie był zasób, które nie mogliśmy być pewni posiadania, czas.

Wypadliśmy przez bramę tunelu.

To, co zobaczyliśmy, kazało mi uciekać z~powrotem.

~

Pojawiliśmy się, jak przewidywaliśmy, na orbicie dookoła Jowisza,
\emph{wysokiej} orbicie, która nie została przewidziana. Po raz pierwszy
zobaczyłem Pierścień z~góry. Nie był niczym jak te w~Saturnie, ale mimo
to był widowiskowy. Koncentryczne białe pierścienie, podzielone
mniejszymi czarnymi pierścieniami, które musiały zostać przeniesione
przez stulecia z~orbit pozostałych księżyców Jowisza. Sam Jowisz uległ
zmianie, jego kolorowe pasy teraz oswojone, skierowane w~pływy
skierowane ku górze, które uformowały heksagonalne komórki, ze szkicową
aluzją bardziej solidnych, dzielących je, struktur.

-- To jak plaster miodu! -- wyszeptała Meg, za moim umysłem.

-- Tak -- powiedziałem. -- I~nie chcemy poznać roju.

Odpowiedzią Meg było powiększenie widoku na wprost. Sto kilometrów
dalej, na tej samej orbicie, stał rój szkaradnie wyglądających statków
kosmicznych, jakie kiedykolwiek widziałem. Miały doskonałość mechanizmu,
\emph{skończony} wygląd w~ich potężnych przegubowych rozszerzeniach
lśniącego mosiądzu i~stali. Ich wielorakie oczy i~anteny sondujące
obróciły się na nas. Ich rakiety i~lasery przesunęły się w~pozycję
gotowości bojowej jak odsłonięte żądła.

Nasze własne anteny natychmiast zostały obite ich częstotliwościami
wywoławczymi. Czułem lekki dotyk radaru na kadłubie.

-- Firewalle włączone? -- spytałem Meg.

-- Tak.

Ostrożnie otworzyłem przychodzące łącze wideo i~wysłałem identyfikator
mikrofalami do orbitalnym fortów -- lub myśliwców -- przed nami.

Na ekranie wideo w~centrum wzrokowym mojego umysłu, zamglona przez
chroniące firewalle software'u antywirusowego, pojawiła się kobieca
twarz. Młoda kobieta, z~zaplecionymi włosami, nakątnymi powiekami,
szerokimi policzkami, skórą koloru kawy, cienkimi ustami i~szerokimi
zębami\ldots ciężko było stwierdzić, które elementy przekonały, ale byłem
pewien, że należała do nowej \emph{rasy}, innej niż te, które wcześniej
napotkałem: ludzka, zdaje się, jest słowem, którego szukałem.

-- Gadzisz angloskej, robot? -- spytała z~wątpliwościami.

-- Angielski?

Uśmiechnęła się. 

-- Jes, ahngielski. Podłapałeś ze starych transmisji,
jes? Język się zmienił. Wiele się zmieniło.

~

Wiele się zmieniło.

Flota, która na nas czekała, była częścią Oddziału Cassini (jak się z~dumą nazywali) Grupy Obrony Słonecznej, oddelegowanej do Komitetu
Badawczego Anomalii Jowiszowej. Ich jedyną misją było strzec Mili
Malleya i~zestrzelić wszystko, co wznosiło się z~powierzchni Jowisza. Z~początku myśleli, że jesteśmy kosmitami, lub tworem Szybkiego Ludku. Nie
byli zadowoleni, kiedy im powiedzieliśmy, że też nie ufamy ich
transmisjom, nawet jeżeli ich statki, inaczej niż nasze, były
dostatecznie duże (niechętnie to potwierdziliśmy), żeby podtrzymać
organiczne życie. W~końcu zaroili się, otaczając nas jak mieszkańcy Mórz
Południowych w~skafandrach kosmicznych, przyciskając płyty twarzowe do
naszych soczewek i~(niektórzy) języki we własnych hełmach.

Meg przeprowadziła analizy spektroskopowe ich języków i~mgły ich
śmiejących oddechów, i~zapewniła mnie, że byli ludźmi z~ciała i~krwi.
Potem była nasza kolej na ich uspokojenie. Przesłuchiwali nas przez dni,
a potem ustąpili i~wyhodowali nas w~kapsułach. Trzymali kapsuły
wyizolowane i~na muszce baterii dział laserowych.

Myślę, że odczuli większą ulgę niż my, kiedy wyłoniliśmy się w~ludzkich
ciałach. Pokolenia wirusowych wiadomości radiowych z~kolejnych
cywilizacji szybkiego ludku na Jowiszu sprawiły, że byli bardzo ostrożni
wobec komputerów elektronicznych. Większość obliczeń w~Układzie
Słonecznym jest wykonywana w~maszynach, które Babbage by rozpoznał z~własnych najdzikszych snów. Widziałem te maszyny liczące. Wypełniają wydrążone góry.
Zasilane są przez zapory, chłodzone przez rzeki. Używane są do
rozwiązania milionów równań.

~

Oddział Cassini odesłał nas na Ziemię. Orbita transferowa zabrała dostatecznie dużo czasu, żebyśmy Meg i~ja zostali
właściwie przedstawieni i~stali się sławni. Wszyscy powyżej około
szóstego roku świetnie się bawili, kiedy odkryli, że przybyliśmy ocalić
ich od szalonych jowiszowych komputerowych maniaków zmierzających ku
końcowi czasu.

Światowa sława ma swoje wady, szczególnie na świecie trzydziestu
miliardów ludzi. Jednak jest jakąś ulgą, po życiu w~świecie, gdzie idee,
które forsowałem, stały się podstawą społeczeństwa, a~moja pamięć jest
nieśmiertelna. W~tym świecie, idee zostały zapomniane, a~ja jestem
przypisem w~starych książkach.

Więc wędrujemy po Ziemi, Meg i~ja, i~rozmawiamy z~ludźmi. Kiedy mówimy
im o~Mieście Statku, im więcej rozumieją, tym mniej im się podoba.
Wydaje się dla nich nie anarchią, jaką mają tutaj w~Układzie Słonecznym,
ale podzieloną -- a~zatem pomnożoną -- władzą. Więc nie mówimy dużo o~Mieście Statku. Mówimy o~pustyni i~czekamy, aby ci dziwni, ale jakoś
znajomi ludzie zapytali nas, jeszcze raz, czy pamiętamy drogę przez
tunel czasoprzestrzenny na Nowego Marsa. To jedyny temat, który budzi
zazdrość w~ich oczach. Rozumiem dlaczego. Trzydzieści miliardów
odrzuciło Malthusa: wszyscy są bogaci. Odrzucili Misesa: nikt nie jest
opłacany. Odrzucili Freuda: nikt nie jest smutny.

Niemniej jednak jest trochę tłoczno.

~

Sonda kontynuuje swój lot z~prędkością bliską światłu. Informacja, którą
posyła, zawsze jest nowa, zawsze niespodziewana. Jednak najbardziej
intensywne dane, dla mnie, były te, które pojawiły się całkiem wcześnie
na jej kursie: ekspansja Hubble'a\footnote{ zob.~\url{https://pl.wikipedia.org/wiki/Prawo_Hubble\%E2\%80\%99a-Lema\%C3\%AEtre\%E2\%80\%99a
} -- przyp.tłum.} jest lokalna. Sonda już dawno minęła, inne,
rozszerzające lub malejące, regiony kosmosu. Był Wielki Wybuch, ale nie
był początkiem, albowiem nie ma żadnego. Ani śmierci cieplnej
wszechświata, ani Wielki Kolaps na nas nie czekają. Te zagłady (teraz
jest tak mówione) z~tych wszystkich błyszczących opracowań
matematycznych, były tylko odbiciami społeczeństwa stojącego w~obliczu
swoich granic.

~

Nie ma końca.

\chapter*{Podziękowania}

Dziękuję Carol, Sharon i~Michaelowi za miłość (i spokój) podczas
pisania książki, Iainowi Banksowi za przeczytanie szkicu w~trakcie
picia, Mic Cheethamowi za uwierzenie w~nią, Johnowi Jarroldowi za
zachowanie spokoju, zwolennikom Libertarii i~Donikąd (wiecie, kogo mam
na myśli), Larze Byrne za inspirację (i genotyp).

\chapter*{Posłowie od tłumacza}

Oto druga część tetralogii ,,Jesienna Rewolucja'', a czwarta pozycja w serii ,,Czarna Flaga''. 

Tłumaczenie jest oparte na wydaniu wydawnictwa Tor pn. ,,Fractions'', które składa się z pierwszego i drugiego tomu tetralogii. Zgodnie z zamysłem autora ta powieść, choć może być czytana samodzielnie, nawiązuje do ,,Gwiezdnej Frakcji''. Poznajemy w niej lepiej uniwersum ,,Jesiennej Rewolucji''.

W przedmowie do pierwszego tomu autor zadaje pytanie, ,,pytanie MacLeoda'':
\emph{A co jeśli kapitalizm jest niestabilny, a socjalizm niemożliwy?}

Pierwszy tom zawiera pewną odpowiedź, która zostaje znacznie lepiej wyartykułowana w drugim tomie. Ten tom przedstawia przyszłość libertariańską, anarchokapitalistyczną.
Jest to wersja bardzo realistyczna, chociażby dlatego, że główny bohater, Reid, utrzymuje swoją pozycję dzięki posiadanemu kapitałowi, a system jest tak zbudowany, żeby jego, właściciela, przywileje zachować i podtrzymać wyzyskiwanie innych, osób, robotów, robotów równoważnych człowiekowi.

Wśród nie-właścicieli, praca niewolnika, praca sekzabawki, jest akceptowana.

W tej części, pojawia się także wyczekiwana osobliwość czyli powstanie sztucznych inteligencji, które są na tyle złożone, że nic nie stoi im na przeszkodzie przed całkowitym zdominowaniem kosmosu.

Nasuwa się pytanie, które warto sobie zadać sobie, dlaczego, biorąc pod uwagę prawo Moore'a,  25 lat po wydaniu tej powieści nadal nie zaistniała taka osobliwość?

Osoby zainteresowane odpowiedzią odsyłam do esejów ,,O technologii, tępocie i~ukrytych rozkoszach biurokracji'' Davida Graebera\footnote{zob.~\url{https://pl.anarchistlibraries.net/library/david-graeber-utopia-regulaminow}}.

~

Jest całkiem możliwe, że liczba przypisów może przeszkadzać, ale uważam, że niektóre wzmianki autora prowadzą do ciekawych idei, stąd, tak jak w innych dziełach, wskazuję odniesienia do wtrącanych uwag czy wykorzystanych skrótów.

Jestem przekonany, że uważna czytelniczka odnajdzie wiele błędów w~tym tłumaczeniu. Ponoszę za to całkowitą odpowiedzialność. \\

\href{mailto:theskymyladythesky@zoho.eu}{Jacek Hummel}\\

Warszawa, marzec -- czerwiec 2021 roku.

\chapter*{Seria ,,Czarna Flaga''}

\begin{center}
\begin{large}
W serii ,,Czarna Flaga'' dotychczas opublikowano 
\href{https://archive.org/details/@j_hummel?and[]=year\%3A\%222021\%22}{online}
:
\end{large} 
\end{center}


\begin{enumerate}
\item \href{https://archive.org/details/joanna-russ-mezczyzna-rodzaju-zenskiego/Joanna_Russ_M\%C4\%99\%C5\%BCczyzna_rodzaju_\%C5\%BCe\%C5\%84skiego}{Mężczyzna rodzaju żeńskiego}, (The Female Man),\\ Joanna Russ
\item \href{https://archive.org/details/ken-macleod-wieza-kosmonauty}{Wieża Kosmonauty}, (The Cosmonaut Keep), Ken MacLeod
\item \href{https://archive.org/details/ken-mac-leod-jesienna-rewolucja-gwiezdna-frakcja}{Gwiezdna Frakcja}\footnote{Gwiezdna Frakcja, Kamienny Kanał, Oddział Cassini, Droga do Gwiazd należą do tetralogii Jesiennej Rewolucji.}, (The Star Fraction),	 Ken MacLeod
\item Kamienny Kanał, (The Stone Canal), Ken MacLeod
\end{enumerate}

\begin{center}

\begin{large}W planach:\end{large}\end{center}

\begin{enumerate}
\item Droga do Gwiazd, (The Sky Road), Ken MacLeod
\item Oddział Cassini\footnote{ jedyna pozycja Kena MacLeoda wydana w~Polsce jako ,,Dywizja Cassini'' przez Wydawnictwo Amber, jest dostępna w~antykwariatach, bibliotekach lub w~wersji ebook na stronach \url{https://doci.pl/} lub \url{https://docer.pl/}}, (The Cassini Division), Ken MacLeod
\item Radicalized\footnote{tytuł roboczy: Zradykalizowane}, Cory Doctorow
\item Walkaway\footnote{tytuł roboczy: Odchodzący}, Cory Doctorow
\item Dhalgren, Samuel R.~Delany
\end{enumerate}


\tableofcontents{}
\end{document}
