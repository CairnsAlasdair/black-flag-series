\documentclass[oneside,polish,11pt,rmheadings]{mwbk}
%polonizacja 
\usepackage[T1]{fontenc}
\usepackage[polish]{babel}
\usepackage[utf8]{inputenc} 
\usepackage{polski}  
\frenchspacing
\usepackage{indentfirst}  
%koniec polonizacja 
%font
%\usepackage{garamondlibre} %for epub
%\usepackage{ebgaramond}
%\usepackage{palatino}
%\usepackage{times}
%\usepackage{newcent}

%grafika 
\usepackage{graphicx,longfbox,caption} 

\usepackage[a5paper]{geometry} %wielkość papieru (148x210-book w~PL) 
\newcommand{\threeast}{\par\centerline{*\,*\,*}\medskip\par}


%EPUB 
%\usepackage[hyperfootnotes=true,unicode, pdftex]{hyperref}  
\usepackage[hyperfootnotes=true]{hyperref} %original 
%move footnotes to endnotes \usepackage{enotez \let\footnote=\endnote \setenotez{   list-name = Przypisy,   backref = true 

%pdf anonimize %dla EPUB wykomentować 
%\pdfsuppressptexinfo=-1 %Suppress PTEX.Fullbanner and info of imported PDFs 
%pakiet odnośników i~pdf metadata 

%\usepackage[unicode, pdftex]{hyperref}
%\hypersetup{pdfauthor={Cory Doctorow},
%                        pdftitle={Dla Wygranej},
%                     pdfsubject={For The Win},
%                     pdfkeywords={red. Jacek Hummel, Creative Commons, tłumaczenie CC BY 4.0, powieść, novel, science-fiction},             pdfcreator={pdfLaTeX}} %dla EPUB koniec wykomentowania 

\begin{document}
%For the win
\begin{titlepage} % Suppresses displaying the page number on the title page and the subsequent page counts as page 1
	\newcommand{\HRule}{\rule{\linewidth}{0.5mm}} % Defines a new command for horizontal lines, change thickness here
	
	\center % Centre everything on the page
	\normalsize 
	\includegraphics[width=0.4\textwidth]{CC.png}\\
	%\\[1cm] 
\href{https://creativecommons.org/licenses/by/4.0/deed.pl}{ Uznanie autorstwa 4.0 Międzynarodowe}\\[1.5cm]
	
	\textsc{\Huge Cory Doctorow}\\[0.5cm] % Author
	
	\rule{\textwidth}{1.6pt}\vspace*{-\baselineskip}\vspace*{2pt} % Thick horizontal rule
	\rule{\textwidth}{0.4pt}\\[0.4cm] % Thin horizontal rule
		
	{\huge\bfseries For The Win}%\\[0.4cm] % Title of your document

	\rule{\textwidth}{0.4pt}\vspace*{-\baselineskip}\vspace{3.2pt} % Thin horizontal rule
	\rule{\textwidth}{1.6pt}\\[0.5cm] % Thick horizontal rule	

	
    \textsc{\Large Dla Wygranej}\\[0.2cm] % Major heading such as course name
%	\textsc{\large Etnografia}\\[2.5cm] % Minor heading such as course title
	
		\vfill\vfill
	{\large\textsc{redakcja:}}\\
	 \href{mailto:theskymyladythesky@zoho.eu}{Jacek \textit{Hummel}}
	\vfill
		

	{\large Warszawa, 2023} % Date, change the \today to a set date if you want to be precise
\end{titlepage}

\pagestyle{plain} 

\begin{center}
\includegraphics[width=\textwidth]{ftw-title-page.jpg}
\end{center}


\newpage

\begin{center}
\textsc{PRZECZYTAJ TO}


Ta powieść jest rozpowszechniana w~ramach licencji 

\noindent Uznanie autorstwa-Użycie niekomercyjne-Na tych samych warunkach 3.0 Międzynarodowe (CC BY-NC-SA 3.0). 

To oznacza, że wolno wam: 

\end{center}

\begin{enumerate}
\item korzystać -- kopiować, 
\item rozpowszechniać i~przekazywać dzieło 
\item  remiksować -- adaptować utwór 
\end{enumerate}


Pod następującymi warunkami: 
\begin{itemize}
\item Uznanie autorstwa. Musicie wskazać autora dzieła w~sposób określony przez autora lub licencjodawcę (ale nie w~sposób, który sugeruje, że popierają Ciebie lub korzystanie z~utworu). 
\item  Niekomercyjne. Nie możecie używać tej pracy do celów komercyjnych. 
\item  Na tych samych warunkach. Jeśli modyfikujecie, przekształcacie lub tworzycie na podstawie tego dzieła, możecie rozpowszechniać powstałe dzieło tylko na tej samej lub podobnej licencji jak ta.
\item  W przypadku ponownego wykorzystania lub rozpowszechniania należy wyjaśnić innym warunki licencji tego dzieła. Najlepszym sposobem na to jest link \url{http://craphound.com/ftw}
\end{itemize}

\begin{center}


\noindent Każdy z~powyższych warunków może zostać uchylony, jeśli uzyskasz moją zgodę. 

\noindent \href{http://creativecommons.org/licenses/by-nc-sa/3.0/}{Więcej informacji tutaj} 

\noindent Pełna wersja prawnicza znajduje się na końcu tego pliku. 
\end{center}

\newpage

\begin{flushright}
Dedykacja:
\medskip 

\textit{Dla Poesy:}

\textit{Żyj tak, jakby to były pierwsze dni lepszego narodu. }
\end{flushright}

\bigskip


\chapter*{Wprowadzenie}


\textit{Dla Wygranej} (For the Win) to moja druga powieść dla młodzieży i~podobnie jak moja książka \textit{Mały Brat }z 2008 roku, jej celem jest coś więcej niż tylko opowiedzenie historii. \textit{Za Zwycięstwo }to książka o ekonomii (temat, który nagle stał się o wiele bardziej istotny w~połowie pisania tej książki, kiedy światowa gospodarka bezceremonialnie wpadła do kibla i~tam utknęła), sprawiedliwości, polityce, grach i~pracy. Powieść łączy kropki między sposobem, w~jaki robimy zakupy, sposobem organizowania się i~sposobem grania, a także tym, dlaczego niektórzy ludzie są bogaci, inni biedni i~w jaki sposób tam utknęliśmy. 


Mam nadzieję, że czytelnicy tej książki zostaną zainspirowani do głębszego zagłębienia się w~tematykę ,,ekonomii behawioralnej'' (i pokrewnych tematów, takich jak ,,neuroekonomia'') i~zaczną zadawać trudne pytania o to, w~jaki sposób dostajemy posiadane rzeczy i~ile kosztuje produkcja naszych ludzkich braci i~siostry, i~dlaczego uważamy, że potrzebujemy tych rzeczy. 


Jednak to kiepska polityka, którą możemy wyrazić się tylko poprzez wybór: kupowania lub nie-kupowania czegoś. Czasami (często!) trzeba się zorganizować, żeby coś zmienić. 


To złoty wiek dla organizowania. Jeśli istnieje jedna rzecz, którą Internet zmienił na zawsze, to względna trudność i~koszt zebrania grupy ludzi w~tym samym miejscu, działających dla tego samego celu. Nie zawsze jest to dobre (bandyci, łobuzy, rasiści i~szaleńcy nigdy nie mieli tak dobrze), ale zasadniczo zmienia to \textit{zasady gry}. 


Trudno sobie przypomnieć, jak trudne kiedyś było zorganizowanie się: jak trudno było zrobić coś tak trywialnego, jak zebranie dziesięcioro przyjaciół, aby zgodzili się na kolację i~film, nie mówiąc już o zebraniu milionów ludzi, aby zebrać pieniądze na kandydata politycznego, zagłosowania, zaprotestowania przeciwko korupcji lub uratowania zagrożonej i~ukochanej instytucji. 


Sieć nie rozwiązuje problemu niesprawiedliwości, ale rozwiązuje pierwszy trudny problem naprawienia krzywd: zebranie wszystkich razem i~utrzymanie ich razem. Nadal musisz wykonać tę \textit{cięższą }pracę, ryzykując życiem, zdrowiem, osobistym majątkiem, reputacją. 


Każda cudowna rzecz w~naszym świecie została wywalczona. Za nasze prawa, nasze powodzenie, nasze szczęście i~wszystko, co słodkie, zapłacili kiedyś zaangażowani ludzie, którzy zaryzykowali wszystko, aby zmienić świat na lepszy. Zagrożenia te nie są ani o jotę zmniejszane przez sieć. Ale nagrody są równie słodkie. 

\bigskip
\noindent AUDIOBOOK  


Dobrzy ludzie z~Random House Audio stworzyli \textit{fantastyczną }edycję audio tej książki. Możesz ją kupić na płycie CD lub możesz kupić wersję MP3 w~różnych księgarniach internetowych. \href{https://craphound.com/category/ftw/}{\textit{Sprzedaję ją również na mojej stronie.}}  


Niestety, nie możesz kupić tej książki od najpopularniejszych światowych dostawców audiobooków: iTunes firmy Apple i~Audible firmy Amazon. To dlatego, że żaden z~tych sklepów nie pozwoliłby mi sprzedawać audiobooka na warunkach, które uważam za uczciwe i~sprawiedliwe. 


W szczególności Apple odmówił dostarczania książki, chyba że było na niej włączone ,,zarządzanie prawami cyfrowymi''. Jest to technologia, która blokuje muzykę na urządzeniach Apple. Przenoszenie plików z~uszkodzonymi mechanizmami DRM na urządzenia, których Apple nie pobłogosławił, jest nielegalne, co oznacza, że jeśli zachęcę Cię do kupowania moich prac za pośrednictwem Apple, stracę możliwość kontynuowania sprzedaży u konkurencji Apple w~późniejszym terminie. Wydaje się, że to zły interes dla nas wszystkich. 


Trzeba przyznać, że firma Audible (która dostarcza wszystkie audiobooki w~iTunes) \textit{była }gotowa sprzedać tę książkę bez DRM, ale nalegali na włączenie swojej niezwykle uciążliwej ,,umowy licencyjnej użytkownika końcowego'', która zabrania również przenoszenia mojej książki na urządzenie, których Audible nie zatwierdził. Aby im to ułatwić, zaproponowałem po prostu nagranie krótkiego wstępu, który mówił: ,,Cory Doctorow i~Random House Audio udzielają ci pozwolenia na wykorzystanie tej książki w~sposób, który nie narusza praw autorskich''. W ten sposób nie musieli wprowadzać żadnych zmian w~swojej witrynie ani w~umowach, które musisz kliknąć, aby z~nich skorzystać. Jednak Audible odmówił. 


Nie sprzedałbym tej książki przez Wal-Mart, gdyby się upierali, że można ją wystawić tylko na regałach marki Wal-Mart oraz nie sprzedam jej za pośrednictwem żadnego internetowego sprzedawcy, który nakłada ten sam wymóg na twoje wirtualne półki na książki. Dlatego też nie znajdziesz moich książek na sprzedaż w~sklepach Kindle lub iPad -- oba sklepy nalegają na prawo do zablokowania Cię warunkami, które moim zdaniem są niesprawiedliwe i~złe dla nas obojga. 


Jestem z~tego powodu dość przygnębiony. Chętnie sprzedawałbym zarówno przez Apple, jak i~Audible, gdyby pozwolili mi sprzedawać je bez DRM i~na podstawie najkrótszej na świecie umowy EULA ( ,,Nie naruszaj prawa autorskiego''). Tymczasem z~góry dziękuję za opiekę sprzedawców audiobooków online, którzy szanują prawa zarówno autorów, jak i~odbiorców. I jestem szczególnie wdzięczny Random House Audio za wsparcie mnie w~tej walce o uczciwą umowę dla nas wszystkich. 

\bigskip
\noindent SPRAWA Z PRAWAMI AUTORSKIMI 


Licencja Creative Commons znajdująca się na górze tego pliku prawdopodobnie powiadomiła cię o tym, że mam dość niekonwencjonalne poglądy na temat praw autorskich. Oto co, o tym myślę, w~skrócie: trochę wystarczy, a dużo to za dużo. 


Podoba mi się fakt, że prawa autorskie pozwalają mi sprzedawać prawa moim wydawcom, studiom filmowym i~tak dalej. Fajnie, że nie mogą po prostu zabrać moich rzeczy bez pozwolenia i~wzbogacić się na tym bez mojego udziału. Jestem w~całkiem dobrej pozycji, jeśli chodzi o negocjacje z~tymi firmami: mam świetnego agenta i~dziesięcioletnie doświadczenie w~prawie autorskim i~licencjach (w tym pracę jako delegat w~WIPO, agencji ONZ, która zajmuje się traktatami dotyczącymi praw autorskich). Co więcej, tych negocjacji jest niewiele, nawet jeśli sprzedam pięćdziesiąt czy sto różnych wydań \textit{For the Win }(co umieściłoby to w~górnej milionowej części percentyla dla beletrystyki), to i~tak tylko sto negocjacji, z~czym bym sobie poradził. 


\textit{Nienawidzę }faktu, że fani, którzy chcą robić to, co zawsze robili czytelnicy, muszą grać w~tym samym systemie, co wszyscy ci agenci i~prawnicy. \textit{Głupotą }jest mówić, że klasa w~szkole podstawowej musi porozmawiać z~prawnikiem wielkiego światowego wydawcy, zanim wystawią sztukę opartą na jednej z~moich książek. Śmieszne jest twierdzenie, że ludzie, którzy chcą ,,pożyczyć'' swoją elektroniczną kopię mojej książki przyjacielowi, muszą uzyskać na to \textit{licencję}. Wypożyczanie książek istnieje dłużej niż jakikolwiek wydawca na Ziemi i~to dobra rzecz. 


Prawa autorskie są coraz częściej uchwalane bez demokratycznej debaty i~kontroli. W Wielkiej Brytanii, gdzie mieszkam, Parlament właśnie uchwalił ustawę o Gospodarce Cyfrowej (Digital Economy Act), złożone prawo autorskie, które pozwala korporacyjnym gigantom odłączyć całe rodziny od Internetu, jeśli ktokolwiek w~domu zostanie oskarżony (bez dowodu) o naruszenie praw autorskich; tworzy również ,,Wielką Zaporę Sieciową Wielkiej Brytanii'' (Great Firewall of Britain), która służy do cenzurowania każdej witryny, której nie lubią wytwórnie płytowe i~studia filmowe. Ta ustawa została uchwalona bez żadnej poważnej debaty publicznej w~Parlamencie, pospiesznie wykorzystując brudny proces, w~którym nasi wybrani przedstawiciele zdradzili opinię publiczną, aby dać ogromny, zapakowany prezent swoim kolegom z~korporacji. 


Jest coraz gorzej: na całym świecie bogate kraje, takie jak USA, UE i~Kanada, negocjują tajny traktat dotyczący praw autorskich o nazwie ,,Umowa handlowa dotycząca zwalczania obrotu towarami podrobionymi'' (ACTA -- The Anti-Counterfeiting Trade Agreement), który zawiera wszystkie problemy, jakie miała ustawa o gospodarce cyfrowej, plus jeszcze trochę. Plan jest taki, aby zgodzić się na to w~tajemnicy, bez publicznej debaty, a następnie zmusić najbiedniejsze kraje świata do zarejestrowania się, blokując im sprzedaż towarów do bogatych krajów, o ile tego nie zrobią. W Ameryce plan jest taki, aby przyjąć ten traktat bez debaty w~Kongresie, wykorzystując władzę wykonawczą prezydenta. Chociaż zaczęło się to za Busha, administracja Obamy też realizowała ten plan z~wielkim entuzjazmem. 


Więc jeśli nie łamiesz teraz praw autorskich, wkrótce to zrobisz. A kary będą znacznie gorsze. Jako ktoś, kto polega na prawach autorskich, aby zarabiać na życie, robi mi się niedobrze. Jeśli wielkie firmy rozrywkowe postanowiły zniszczyć misję praw autorskich, nie mogłyby zrobić nic lepszego niż to, co robią teraz. 


Więc w~zasadzie \textit{chrzanić to}. Albo, jak to ujął wokalista, Wobbly i~organizator związków, Woody Guthrie: 

\textit{,,Ta piosenka jest chroniona prawami autorskimi w~USA, pod pieczęcią praw autorskich sygnatura 154085, przez okres 28 lat, a wszyscy przyłapani na śpiewaniu jej bez naszej zgody będą naszymi wielkimi, dobrymi przyjaciółmi, ponieważ nam to kompletnie nie przeszkadza. Opublikuj ją. Wydrukuj ją. Zaśpiewaj ją. Zatańcz. Zajodłuj. Napisaliśmy ją, i~to wszystko, co chcieliśmy zrobić''.}

\bigskip
\noindent DAROWIZNY I SŁOWO DO NAUCZYCIELI I BIBLIOTEKARZY 


Za każdym razem, gdy udostępniam online za darmo, otrzymuję e-maile od czytelników, którzy chcą przesłać mi darowizny na książkę. Doceniam ich hojność, ale nie interesują mnie darowizny pieniężne, ponieważ moi wydawcy są dla mnie bardzo ważni. Wnoszą niezmierny wkład do książki, ulepszając ją, przedstawiając ją odbiorcom, do których nigdy nie mogłem dotrzeć, pomagając mi robić więcej z~moją pracą. Nie mam ochoty odcinać się od nich. 


Musi być jakiś dobry sposób, aby dobrze wykorzystać tę hojność i~myślę, że go znalazłem. 


Oto propozycja: jest wielu nauczycieli i~bibliotekarzy, którzy chcieliby dostać egzemplarze tej książki dla swoich dzieci, ale nie mają na to budżetu (nauczyciele w~USA wydają około 1200 dolarów z~kieszeni każdy na materiały szkolne, czego ich budżety nie pokryją, dlatego sponsoruję klasę w~Ivanhoe Elementary w~mojej starej dzielnicy w~Los Angeles; możesz samodzielnie adoptować zajęcia \href{https://www.adoptaclassroom.org/}{\textit{tutaj}}). 


Istnieją hojni ludzie, którzy chcą wysłać mi trochę gotówki, aby podziękować mi za darmowe e-booki. 


Proponuję, że złożymy się razem. 


Jeśli jesteś nauczycielem lub bibliotekarzem i~chcesz otrzymać bezpłatną kopię \textit{For the Win}, wyślij e-mail na adres freeftwbook@gmail.com, podając swoje imię i~nazwisko oraz nazwę i~adres swojej szkoły. Zostanie on wysłany do \url{http://craphound.com/ftw/donate/}przez moją fantastyczną pomocnicę, Olgę Nunes, aby potencjalni darczyńcy mogli to zobaczyć. 


Jeśli podobało Ci się elektroniczne wydanie For the Win i~chcesz przekazać coś w~podziękowaniu, przejdź do \url{http://craphound.com/ftw/donate/}i znajdź nauczyciela lub bibliotekarza, którego chcesz wesprzeć. Następnie przejdź do Amazon, BN.com lub ulubionej księgarni elektronicznej i~zamów kopię do klasy, a następnie wyślij e-mailem kopię paragonu (możesz najpierw usunąć swój adres i~inne dane osobowe!) na adres freeftwbook@gmail.com, żeby Olga może oznaczyć tę kopię jako wysłaną. Jeśli nie chcesz być publicznie doceniony za swoją hojność, daj nam znać, a zachowamy anonimowość, w~przeciwnym razie podziękujemy Ci na stronie darowizny. 


Zrobiłem to dotychczas z~trzema moimi tytułami i~dzięki waszej hojności przekazałem w~ręce czytelników ponad tysiąc książek. Jestem za to bardziej wdzięczny, niż mogą wyrazić to słowa -- jeden z~moich czytelników nazwał to -- spłatą długów z~natychmiastową gratyfikacją. To cholerna racja, prawda? 

\bigskip
\noindent  DEDYKACJE DLA KSIĘGARNI 


Wiele scen w~tym pliku zostało poświęconych księgarniom: sklepom, które kocham, sklepom, które pomogły mi odkryć książki, które otworzyły mi umysł, sklepom, które pomogły mi w~karierze. Sklepy nic mi za to nie zapłaciły -- nawet im o tym nie powiedziałem -- ale wydaje mi się, że to słuszne. W końcu mam nadzieję, że przeczytasz tego ebooka i~zdecydujesz się kupić papierową książkę, więc warto zasugerować kilka miejsc, w~których możesz ją kupić! 



\chapter{Gracze i gry, pracownicy i praca }

Ta scena jest poświęcona BakkaPhoenix Books w~Toronto w~Kanadzie. Bakka to najstarsza księgarnia science fiction na świecie, dzięki której stałem się mutantem, jakim jestem dzisiaj. Po raz pierwszy wszedłem do środka w~wieku około dziesięciu lat i~poprosiłam o kilka rekomendacji. Tanya Huff (tak, Tanya Huff, ale wtedy nie była sławną pisarką!) zabrała mnie do sekcji używanej książki i~wcisnęła w~moje ręce kopię ,,Kudłaczka'' H. Beama Pipera i~na zawsze zmieniła moje życie. W wieku 18 lat pracowałem w~Bakka, przejąłem ją od Tanyi, kiedy przeszła na emeryturę, aby pisać na pełny etat, i~nauczyłem się na całe życie lekcji o tym, jak i~dlaczego ludzie kupują książki. Myślę, że każdy pisarz powinien popracować w~księgarni (a wielu pisarzy pracowało w~Bakka przez lata!). Na 30-lecie księgarni, zebrali antologię pisarzy z~Bakki, która zawierała pracę Michelle Sagara (aka Michelle West), Tanyi Huff, Nalo Hopkinsona, Tary Tallan \ldots  i~moją!). 


\href{https://www.bakkaphoenixbooks.com/}{\textit{BakkaPhoenix Books}}: 697 Queen Street West, Toronto ON Kanada M6J1E6, +1 416 963 9993 

\bigskip
\threeast

W Grze, postacie Matthew zabijały potwory, tak jak robiły to każdej nocy. Jednak dziś wieczorem, gdy Matthew w~zamyśleniu schwytał pałeczkami kluskę ze styropianowego opakowania, zanurzył ją w~ostrym sosie i~włożył do ust, jego mały szwadron zrobił coś niezwykłego: zaczął \textit{wygrywać}. 


Na jego biurku stało osiem monitorów, ustawionych w~dwa rzędy po cztery, z~górnym rzędem opartym na półce, którą kupił od starszej pani handlującej złomem przy targu Dongmen. Sprzedała mu również monitory, kręcąc głową na jego idiotyzm: w~czasach, gdy wszyscy chcieli gigantycznych, 30-calowych ekranów, dlaczego chciał tę kolekcję dziwacznych, małych 9-calowych wyświetlaczy? 


Żeby wszystkie zmieściły się na biurku.


Niewiele osób mogło grać w~osiem jednoczesnych partii Svartalfaheim Warriors. Po pierwsze, Coca Cola (która była właścicielem gry) poświęciła dużo czasu programistów, aby uniemożliwić granie w~więcej niż jedną grę na jednym komputerze, więc musiałeś jakoś umieścić osiem komputerów na jednym biurku, z~ośmioma klawiaturami oraz z~ośmioma myszami, do tego mieć wystarczająco miejsca na pierogi, popielniczkę, stos hinduskich komiksów i~ten głupi topór bojowy, który dał mu Ping, zeszyty, szkicownik, laptop i~\ldots  


To było zatłoczone biurko. 


I było głośno. Zainstalował osiem tanich głośników, każdy przyklejony do odpowiedniego monitora, przyciszonych do normalnego szumu Svartalfaheim -- brzęk toporów, ryk lodowych gigantów, niesamowita muzyka czarnych elfów (która brzmiała bardzo podobnie do programów demonstracyjnych na klawiaturach elektrycznych, przy których produkcji jego matka spędziła pół życia). Teraz wszystkie odtwarzały hałas kasyna, hałas \textit{wypłat}, gdy jego oddział zaczął sprzątać. Złoto popłynęło na ich konta. Grał trollami -- w~Svartalfaheim toczyły się wojny trolli kontra elfy, chociaż był tam moduł rozszerzający z~lekkimi elfami i~jakimś rodzajem chodzącego drzewa -- i~przeszedł przez instancję lochu, który był podziemnym legowiskiem pomniejszego książątka z~rasy mrocznych elfów. Legowisko było tylko średnio trudne, z~mnóstwem gównianych małych potworów na początku, potem trochę elfiego mięsa armatniego do wykoszenia, pułapki i~level boss, czarownik, który musiał być załatwiony przez czarodziejów w~grupie Matthew, podczas gdy uzdrowiciele leczyli ich, a tanki zabijały wszystko, co próbowało ich zaatakować. 


Na razie w~porządku. Matthew zmapował ten loch podczas swojej drugiej nocy w~grze, szybkie reko, które pokazało, że może się spodziewać, że zarobi tam około czterystu złota w~ciągu dwudziestu minut, co oznaczało, że był to dość kiepski sposób na zarabianie na życie. Ale Matthew prowadził bardzo dobre \textit{notatki}, a wśród jego notatek była info, że ostatnia grupa strażników upuściła trochę mareridtbane, które było częścią potężnego zaklęcia Żywy Koszmar w~nowym module rozszerzeń. Gracze z~całych Niemiec, Szwajcarii i~Danii kupowali mareridtbane po osiemset złota za roślinę. Jego wstępne rozpoznanie pozwoliło mu zebrać \textit{pięć} roślin. To dało łączny oczekiwany dochód z~tego lochu w~wysokości czterech tysięcy czterystu złota w~ciągu dwudziestu minut, czyli 13 200 sztuk złota w~godzinę, co, według dziennego kursu wymiany, było warte około 30 dolarów, czyli 285 renminbi.


To było, pomyślał przez chwilę, ponad siedemdziesiąt jeden misek klusek. 


Bingo.


Jego ręce przeleciały nad myszami, przejmując bezpośrednią kontrolę nad oddziałem. Wymyśli teraz optymalną ścieżkę przez loch, a potem pójdzie do kafejki internetowej Huoda i~zobaczy, kto mógłby z~nim biegać. Przy odrobinie szczęścia mogliby zabrać, spojrzał na sufit, gdy znów pomyślał, \textit{milion }złota z~lochu, gdyby udało im się nakłonić do pracy całą kawiarnię. Wyrzucaliby złoto po drodze, a zanim administratorzy systemów Coca Coli zorientowaliby się, że coś jest nie tak, wyciągnęliby z~gry prawie trzy tysiące dolarów. To roczny czynsz za jedną noc pracy. Ręce mu drżały, gdy otworzył notatnik na nowej stronie i~zaczął robić notatki lewą ręką, podczas gdy prawa ręka pracowała w~grze. 


Już miał zamknąć notes i~ruszyć do kawiarni -- po drodze potrzebował więcej pierogów, czy mógłby się na nie wpaść? Czy mógł sobie na to pozwolić? Musiał jeść. I kawa. Dużo kawy \ldots  kiedy drzwi roztrzaskały się i~uderzyły o ścianę, odbijając się, zanim zostały ponownie uchylone, wpuszczając zimne fluorescencyjne światło z~zewnątrz do jego maleńkiej jaskini. Trzech mężczyzn weszło do jego pokoju i~zamknęło za sobą drzwi, przywracając ciemność. Jeden z~nich znalazł włącznik światła i~kliknął go kilka razy bez efektu, po czym zaklął w~mandaryńskim chińskim i~walnął Matthew w~ucho tak mocno, że głowa obróciła się na jego szyi, gdy próbowała nie odbić się od biurka. Ból był oślepiający, piekący, nagły. 


-- Światło -- rozkazał jeden z~mężczyzn, a jego głos dotarł do Matthew przez wysoki pisk jego dzwoniącego w~uchu. Niezdarnie sięgnął do lampy stojącej za hinduskimi komiksami, przewrócił ją, a potem jeden z~mężczyzn chwycił ją brutalnie i~włączył, oświetlając całą twarz Matthew, sprawiając, że zmrużył łzawiące oczy. 


-- Zostałeś ostrzeżony -- powiedział mężczyzna, który go uderzył. Matthew go nie widział, ale nie musiał. Znał głos, charakterystyczny akcent Wenzhou, prawie niemożliwy do zrozumienia. 

-- Teraz kolejne ostrzeżenie. -- Rozległ się \textit{trzask }teleskopowej pałki, a Matthew wzdrygnął się i~spróbował podnieść ręce, by osłonić głowę, zanim broń spadnie. Pozostali dwaj trzymali go teraz za ramiona, a pałka gwizdnęła mu koło ucha. 


Jednak nie zmiażdżyła mu kości policzkowej ani obojczyka. Raczej to ekran przed nim został rozbity, rozsyłając maleńkie, ostre odłamki szkła w~chmurę, która wydawała się rozszerzać w~zwolnionym tempie, pokrywając jego twarz i~dłonie. Potem poleciał kolejny ekran. I kolejny. I kolejny. Jeden po drugim mężczyzna beznamiętnie rozbił wszystkie osiem ekranów, wydając drobne pomruki palacza podczas pracy. Potem, z~dużo większym, gardłowym chrząknięciem, chwycił jeden koniec półki i~przechylił ją, posyłając rozbite monitory na podłogę, razem z~komiksami, miską, popielniczkę, wszystko to, ześlizgnęło się na wąskie łóżko, które było przyciśnięte do biurka, a potem na podłogę z~hukiem tak głośnym jak mecz koszykówki w~hucie szkła. 


Matthew poczuł, jak dłonie na jego ramionach się zaciskają, a sam zostaje podniesiony z~krzesła i~odwrócony twarzą do mężczyzny z~akcentem, który pracował jako nadzorca w~fabryce pana Winga, prawie zawsze milcząc. Kiedy mówił, wszyscy podrygiwali na swoich miejscach, nigdy do końca niepewni, czy jego ledwie powstrzymywana wściekłość pęknie, czy ktoś zostanie zabrany z~hali fabrycznej, a potem wróci do akademika nocą, posiniaczony, pocięty, czasem płaczący w~nocy za rodzicami pozostawionymi na prowincji. 


Twarz mężczyzny była teraz spokojna, jakby przemoc wymierzona w~maszyny podrapała go w~swędzącym miejscu nie do podrapania, które sprawiało, że przez cały czas zaciskał i~otwierał pięści. 

-- Matthew, pan Wing chce, abyś wiedział, że myśli o tobie jako o krnąbrnym synu i~nie ma do ciebie pretensji. Zawsze jesteś mile widziany w~jego domu. Wszystko, co musisz zrobić, to poprosić o wybaczenie, i~wszystko będzie dobrze. -- Była to najdłuższa przemowa, jaką Matthew kiedykolwiek usłyszał od tego człowieka, i~została wygłoszona z~zaskakującą czułością, więc było nie lada niespodzianką, gdy mężczyzna uderzył kolanem w~jądra Matthew, tak mocno, że Matthew zobaczył gwiazdy. 


Dłonie go puściły i~osunął się na podłogę, z~dziwnym dźwiękiem w~uszach, o którym po chwili zdał sobie sprawę, że musiał to być jego własny głos. Ledwie zdawał sobie sprawę, gdy mężczyźni poruszali się po jego maleńkim pokoju, kiedy sapnął jak ryba, próbując wciągnąć powietrze do płuc, wystarczająco dużo, by wykrzyczeć niesamowity, promienny ból w~pachwinie. 


Usłyszał jednak okropny szum elektryczny, gdy potraktowali taserem pudło, w~którym znajdowały się jego komputery, osiem komputerów na ośmiu oddzielnych płytach, wetknięte w~pogiętą blaszaną skrzynkę, którą kupił od tej samej starszej pani. Zapach ozonu przypomniał mu małe mieszkanko dziadka, zapach kurzu na grzejniku elektrycznym, który staruszek włączał tylko wtedy, gdy przychodził z~wizytą. Usłyszał, jak zbierają zeszyty, ciężko depczą obudowę komputera i~zamykają za sobą roztrzaskane drzwi. Światło z~lampki na biurku malowało na suficie szalony owal, w~który wpatrywał się przez długi czas, zanim wstał, skomląc z~bólu w~jądrach. 


Nocny strażnik stał na końcu korytarza, gdy wykuśtykał w~noc. Był tylko chłopcem, nawet młodszym od Matthew, szesnastolatkiem, w~mundurze o dwa rozmiary za dużym na jego chudą klatkę piersiową, w~kapeluszu, który zawsze zsuwał mu się na oczy, więc musiał spojrzeć w~górę spod ronda jak chłopiec w~kapeluszu ojca. 


-- W porządku? -- spytał chłopiec. Jego oczy były szeroko otwarte, twarz blada. 

Matthew poklepał się po~dole, krzywiąc się na ból w~uchu i~przeszywający ból szyi. 


-- Tak myślę -- powiedział. 


-- Będziesz musiał zapłacić za drzwi -- powiedział strażnik. 


-- Dzięki -- powiedział Matthew. -- Wielkie dzięki 


-- W porządku -- powiedział chłopiec. -- To moja praca. 


Matthew rytmicznie zaciskał pięści, ruszył w~noc Shenzhen, kuśtykając po schodach w~neonową poświatę. Dochodziła północ, ale ulica Jiabin wciąż tętniła muzyką, jedzeniem, straganiarzami i~naganiaczami, starsze panie goniły cudzoziemców po ulicy, szarpiąc ich rękawy i~proponując im ,,piękne młode dziewczyny'' po angielsku. Nie wiedział, dokąd pójść, więc po prostu szedł, szybko, szybko, jak tylko mógł, próbując rozchodzić ból i~ogrom swojej straty. Zbudowanie komputerów w~jego pokoju nie kosztowało wiele, ale od początku nie miał zbyt wiele. Były prawie wszystkim, co posiadał, poza komiksami, kilkoma ubraniami i\ldots  toporem bojowym. Och, topór wojenny. To była zabawna wizja, gdy podniósł go i~wymachiwał nad głową jak mroczny elf, gwizd ostrza przecinał powietrze, mięsne \textit{łup}, gdy uderzało w~ludzi. 


Wiedział, że to było śmieszne. Nie brał udziału w~bójce, odkąd skończył dziesięć lat. Był \textit{wegetarianinem} do zeszłego roku! Nie zamierzał nikogo uderzyć toporem wojennym. Topór był równie bezużyteczny jak jego rozbite komputery. 


Stopniowo zwolnił tempo. Był teraz poza centralnym obszarem wokół stacji kolejowej, w~zewnętrznym pierścieniu centrum miasta, gdzie było ciemno i~tak cicho, jak nigdy dotąd. Oparł się o stalowe okiennice nad sklepem spożywczym, położył ręce na udach i~pozwolił opaść obolałej głowie. 


Ojciec Matthew był niezwykły wśród ich znajomych -- Kantończyk, który odniósł sukces w~nowym Shenzhen. Kiedy premier Deng zmienił zasady, aby Delta Rzeki Perłowej stała się światową fabryką, prowincja przodków jego rodziny wypełniła się przez noc ludźmi z~innych prowincji. ,,Wskoczyli do morza'' -- porzucili bezpieczną pracę w~rządowych fabrykach, by szukać szczęścia tutaj, na południowym wybrzeżu Chin, i~wszystko się zmieniło dla rodziny Matthew. Jego dziadek, chrześcijański pastor, który został wysłany do obozu pracy podczas rewolucji kulturalnej, nigdy nie dokonał korekty, problem, który dotykał wielu rdzennych Kantończyków, którzy zdawali się stać w~miejscu, gdy obcy pędzili obok nich, aby zostać bogatymi i~potężnymi. 


Ale nie ojciec Matthew. Stary człowiek zaczął jako kierowca dla szefa fabryki obuwia, uczył się jeździć w~pracy, prawie rozbijając samochód więcej niż raz, chociaż właściciel nie wydawał się mieć nic przeciwko. W końcu szef też nigdy nie jeździł samochodem, zanim nie osiągnął sukcesu w~Shenzhen. Jednak pewnego dnia dziadek wyłamał się, gdy główny wykrawacz był zbyt chory, by pracować i~cała produkcja została wstrzymana, podczas gdy dziewczyny, które pracowały przy linii, kłóciły się o najlepszy sposób cięcia skóry dla nowego zamówienia, które nadeszło. 


Ojciec Matthew uwielbiał opowiadać tę historię. Przez cały dzień słyszał kłótnię w~tę i~z powrotem, gdy linia powoli szarpała, potem usiadł na krześle i~myślał i~myślał, a potem wstał, zamknął oczy i~wyobraził sobie spokojny ocean, aż grzmot jego bicia serca zwolnił do normalnego rytmu. Potem wszedł do biura właściciela i~powiedział: 

-- Boss, mogę ci pokazać, jak ciąć te skóry. 


Nie było to łatwe zadanie. Skóry były w~różnym stanie -- w~końcu krowy nie były identyczne -- a niektóre części były lepszej jakości niż inne. Sam but, włoskie mokasyny męskie, potrzebował sześciu różnych części na każdy egzemplarz, a tylko niektóre z~nich były widoczne. Części, które znajdowały się wewnątrz buta, nie musiały być wykonane z~najlepszej skóry, ale części zewnętrzne już tak. Wszystko to wchłaniał ojciec Matthew, siedząc na krześle i~słuchając argumentów. Zawsze lubił rysować, zawsze miał dobrą głowę do przestrzeni i~projektowania. 


I zanim jego szef zdążył wyrzucić go z~biura, zebrał się na odwagę, chwycił długopis z~biurka i~wygrzebał z~kosza pogniecioną paczkę papierosów -- drogie zagraniczne papierosy, które wszyscy właściciele fabryk kupowali jako pokaz bogactwa -- rozdarł ją i~narysował schludną skórę bydlęcą i~szybko pokazał, jak buty można dopasować do skóry przy minimalnych stratach, projekt, który dawał dziesięć par butów na skórę. 


-- Dziesięć? -- spytał szef. 


-- Dziesięć -- potwierdził dumnie ojciec Matthew. Wiedział, że Mistrz Yu, zatrudniony krojczy, uzyskuje co najwyżej dziewięć par. -- Jedenaście, jeśli użyjemy dużej skóry lub zrobimy małe buty. 


-- Możesz to wykroić? 


Wtedy, przed tym dniem, ojciec Matthew nigdy w~życiu nie ciął skóry, nie miał pojęcia, jak przecinać miękką skórę, która przyjechała od garbarza. Tego ranka wstał dwie godziny wcześniej, zanim ktokolwiek inny się obudził, i~wziął skórzaną kurtkę, prezent z~okazji ukończenia studiów od własnego ojca, która była jego własnością i~była skarbem od dziesięciu lat, i~wziął najostrzejszy nóż w~kuchni i~pociął kurtkę na wstążki, ćwicząc, aż mógł sprawić, by nóż przeciął skórę tymi samymi niezawodnymi, skutecznymi łukami, które jego wzrok i~umysł mógł prześledzić. 


-- Mogę spróbować -- powiedział skromnie. Był zdenerwowany swoją śmiałością. Jego szef nie był miłym człowiekiem i~zwolnił wielu pracowników za niesubordynację. Gdyby zwolnił ojca Matthew, straciłby pracę i~kurtkę. A czynsz się należał, a rodzina nie miała oszczędności. 


Szef spojrzał na niego, spojrzał na szkic. 

-- OK, spróbuj. 


I to był dzień, w~którym ojciec Matthew przestał być Kierowcą Fongiem i~stał się Mistrzem Fongiem, młodszym krojczym w~Fabryce Obuwia Nieskończonej Jakości. Niecały rok później został szefem, a rodzina rozkwitła. 


Dorastając, Matthew słyszał tę historię tyle razy, że mógł ją wyrecytować słowo w~słowo ze swoim ojcem. To było coś więcej niż historia: to była rodzinna legenda, ważniejsza niż jakakolwiek historia, której nauczył się w~szkole. Jako opowieść, była dobra, ale Matthew był zdeterminowany, aby jego własne życie miało jeszcze lepszą historię. Matthew nie byłby drugim mistrzem Fong. Będzie Bossem Fongiem, Pierwszym -- człowiekiem z~własną fabryką, własną fortuną. 


I podobnie jak jego ojciec, Matthew miał dar. 


Podobnie jak jego ojciec, Matthew mógł przyjrzeć się pewnemu problemowi i~\textit{zobaczyć }rozwiązanie. A problemy, które Matthew potrafił rozwiązać, polegały na zabijaniu potworów i~zbieraniu ich złota i~prestiżowych przedmiotów, lepiej i~wydajniej niż ktokolwiek inny, kogo kiedykolwiek spotkał lub o którym słyszał. 


Matthew był farmerem złota, ale nie jednym z~tych facetów, do których podchodził właściciel kafejki internetowej i~oferował siedem lub osiem RMB za prawo do kontynuowania gry, przekazujących całe zdobyte złoto szefowi, który je sprzedawał w~jakimś tajemniczym procesie. Matthew był Mistrzem Fong, farmerem złota, który potrafił raz przejść przez loch i~powiedzieć ci dokładnie, jak przejść go ponownie, aby uzyskać maksymalną ilość złota w~jak najkrótszym czasie. Podczas gdy normalny farmer może zarobić 50 sztuk złota w~godzinę, Matthew mógł zarobić 500. A jeśli obserwowałeś, jak Matthew gra, też mogłeś tak zrobić. 


Pan Wing szybko zauważył talent Matthew. Pan Wing nie lubił gier, nie przejmował się legendami Islandii, Anglii, Indii czy Japonii. Pan Wing rozumiał, jak zmusić chłopców do pracy. Pokazywał ich dzienne wpływy na wielkich tablicach na obu końcach swojej fabryki, zabierał najlepszych pracowników na obfite posiłki i~imprezy baijiu w~prywatnych pokojach w~swoim klubie karaoke, gdzie były piękne dziewczyny. Matthew pamiętał te wieczory niewyraźnie, jak przez mgłę: dziewczyna siedząca po jednej stronie na sofie, przyciśnięta do niego, jej perfumy w~jego nosie, napełniająca mu kieliszek, podczas gdy pan Wing wznosił toast za bohatera, wychwalając jego osiągnięcia. Dziewczyny wzdychały, ach, och i~przyciskały się mocniej do niego. Pan Wing zawsze śmiał się z~niego następnego dnia, ponieważ mdlał, zanim poszedł z~jedną z~dziewczyn do jeszcze \textit{bardziej }prywatnego pokoju. 


Pan Wing upewniał się, że wszyscy inni chłopcy wiedzieli o porażce, upewniał się, że kpią z~niezdolności ,,Mistrza Fong'' do trzymania alkoholu, jego nieśmiałości w~stosunku do dziewczyn. A Matthew widział dokładnie to, co robił Boss Wing: ustawiał Matthew jako bohatera, ponad jego przyjaciół, a następnie upewniał się, że jego przyjaciele wiedzieli, że nie jest aż \textit{tak }wielkim bohaterem, żeby nie można go było obalić. I tak wszyscy farmili złoto przez długie godziny, jedząc pierożki przy komputerach i~krzycząc na siebie nad ekranami do późnej nocy i~w papierosowej mgiełce. 


Godziny przeciągnęły się w~dni, dni w~miesiące i~pewnego dnia Matthew obudził się w~pokoju w~akademiku wypełnionym pierdzeniem, chrapaniem i~zapachem dwudziestu młodych mężczyzn w~zbyt małym pokoju i~zdał sobie sprawę, że miał dość pracy dla Boss Winga. Wtedy zdecydował, że zostanie swoim własnym człowiekiem. Wtedy postanowił zostać Boss Fong. 


\bigskip
\threeast

Ta scena jest poświęcona Amazon.com, największej księgarni internetowej na świecie. Amazon jest \textit{niesamowity }-- ,,sklep'', w~którym można dostać praktycznie każdą opublikowaną książkę (wraz z~praktycznie wszystkim innym, od laptopów po tarki do sera), gdzie podnieśli rekomendacje do wysokiej sztuki, gdzie pozwalają klientom bezpośrednio komunikować się ze sobą, gdzie ciągle wymyślają nowe i~lepsze sposoby łączenia książek z~czytelnikami. Amazon zawsze traktował mnie jak złoto -- założyciel Jeff Bezos opublikował nawet recenzję czytelniczą do mojej pierwszej powieści! -- i~robię tam zakupy jak szalony (patrząc na moje arkusze kalkulacyjne, wygląda na to, że kupuję coś od Amazona mniej więcej co \textit{sześć dni}). Amazon jest w~trakcie odkrywania na nowo tego, co to znaczy być księgarnią w~XXI wieku, a ja nie potrafię wyobrazić sobie lepszych ludzi, którzy zmierzą się z~tym trudnym zbiorem problemów. 


\href{https://www.amazon.com/exec/obidos/ASIN/0765322161/downandoutint-20}{Amazon} 

\bigskip
\threeast

Wei-Dong Goldberg obudził się minutę przed uruchomieniem alarmu, świecące cyfry wskazywały 12:59. Pierwsza w~nocy w~Los Angeles, 18 w~Chinach, nadszedł czas na rajd. 


Przetarł oczy i~wstał z~wąskiego łóżka, jego mama wciąż kładła na nim te cholerne prześcieradła Spongebob, więc namalował permanentnym markerem brody, rogi i~papierosy na wszystkich twarzach, i~przeszedł w~ciszy do tornistra i~wydobył laptopa, po czym pomacał na biurku w~poszukiwaniu małej słuchawki Bluetooth, wkręcił ją w~ucho. 


Ułożył stos poduszek przy wezgłowiu i~usiadł przy nich ze skrzyżowanymi nogami, podnosząc wieko i~odpalając przeglądarkę, szukając swoich kumpli, aż do Shenzhen. Gdy ekran wypełnił się imionami i~grami, w~których można ich było znaleźć, uśmiechnął się do siebie. Nadszedł czas grania. 


Trzy kliknięcia później był w~Savage Wonderland, odradzając się na swoim mechanicznym koniu z~mieczem w~dłoni, pośród ogrodu rozmawiających, syczących kwiatów, gotowy do walki. A obok niego jechali jego chłopcy, ich mechaniczne wierzchowce parskały i~gnały do bitwy. 


-- Ni hao! -- powiedział do słuchawek, najgłośniejszym szeptem, na jaki się odważył. Jego ojciec miał problemy z~pęcherzem, wstawał całą noc i~nigdy nie spał zbyt głęboko. Wei-Dong nie mógł sobie na to pozwolić. Gdyby rodzice przyłapali go raz jeszcze, zabraliby mu komputer. Uziemiliby go. Wysłali do akademii wojskowej, gdzie ogoliliby mu głowę i~zostałby pobity pod prysznicem, bo to budowało charakter. Był przyzwyczajony do wszystkich tych gróźb i~nie tylko, i~robiły na nim wrażenie. 


Oczywiście nie wystarczająco, żeby przestał grać w~gry w~środku nocy. 


-- Ni hao! -- powiedział ponownie. Rozległ się śmiech, odległy i~nakładający się w~kipiącej sieci. 


-- Witaj, Leonardzie -- powiedział Ping. -- Dobrze, że uczysz się chińskiego. -- Ping nadal nazywał go \textit{Leonardem}, ale przynajmniej teraz rozmawiał z~nim po mandaryńsku, co było dużą poprawą. Chłopaki zwykle lubili ćwiczyć na nim swój angielski, co oznaczało, że nie mógł na \textit{nich }ćwiczyć chińskiego. 


-- Ćwiczę -- powiedział. 


Znowu się roześmiali i~wiedział, że coś mu poszło nie tak. Intonacja. Zawsze się mylił. Chciał powiedzieć: ,,Pójdę rozwalić te demony, a ty wzmocnij kleryka'', a wychodziło: ,,Jestem miską makaronu, mam piękne rzęsy''. Ale był coraz lepszy. Zanim dotrze do Chin, będzie miał to obcykane. 


-- Robimy rajd? -- spytał. 


-- Tak! -- powiedział Ping, a pozostali się zgodzili. -- Musimy tylko poczekać na gweilo. 

Wei-Dong uwielbiał, że nie był już gweilo. Gweilo znaczyło ,,zagraniczny diabeł'' i~technicznie się kwalifikował. Jednak teraz był jednym z~najeźdźców, a gweilo byli płacącymi klientami, którzy wyrzucali dobre dolary, euro, rupie lub funty, aby grać razem z~nimi. 


Oto gweilo. Można było to dojrzeć, ponieważ często kierował konia ze ścieżki w~wijący się uścisk żywych roślin, zatrzymując się raz za razem, aby wyrąbać ich chwytające pnącza. Po obejrzeniu tego widowiska przez minutę lub dwie, Leonard ruszył do przodu i~rzucił wokół nich zaklęcie ochronne, a pnącza zaskwierczały na świecącej czerwonej bańce, która ich otaczała. 


-- Dzięki -- powiedział gweilo. 


-- Nie ma problemu -- powiedział. 


-- Wow, mówisz po angielsku? -- Gweilo miał mocny akcent z~New Jersey. 


-- Trochę -- powiedział Wei-Dong z~uśmiechem. \textit{Lepiej niż ty, głupku}, pomyślał. 


-- OK, zróbmy to -- powiedział gweilo, a reszta grupy ich dogoniła. 


Gweilo zapłacił im za rajd na instancję Ogrodu Morsa, dość trudnego podwodnego lochu, w~którym znajdowało się kilka naprawdę dobrych łupów, składniki do eliksirów, całkiem niezła broń i~oczywiście mnóstwo złota. Było tam kilka prestiżowych przedmiotów, które tam wpadały, choć rzadko, można było zdobyć wibracyjne ostrze vorpal i~hełm, jeśli miało się szczęście. Umowa polegała na tym, że gweilo zapłacił im, aby przeprowadzili go przez instancję, a on mógł po prostu trzymać się z~tyłu i~pozwolić łupieżcom ponosić ciężary, ale ruszał do przodu, by zadać coup de grace każdemu wielkiemu bossowi, którego pokonali, żeby zdobyć punkty doświadczenia. Miał zatrzymać złoto, broń, prestiżowe przedmioty, wszystko, za niewielką cenę 75 dolarów. Łupieżcy dostawali gotówkę, gweilo szybko awansował i~zbierał mnóstwo skarbów. 


Wei-Dong często zastanawiał się, jaki typ osoby zapłaciłby nieznajomym, aby pomóc im awansować w~grze? Zwykłym powodem, dla którego gweilo zatrudniali najeźdźców, było to, że chcieli bawić się ze swoimi przyjaciółmi, a ich przyjaciele byli bardziej zaawansowani od nich. Wei-Dong dołączył do gier po swoich przyjaciołach i~będąc noobem w~swojej małej grupie, po prostu poprosił swoich kumpli, aby zabierali go ze sobą na najazdy, poprawiając go, aż jego postać będzie na ich poziomie. Więc jeśli ten gweilo miał tak wielu kumpli w~tej grze, że chciał awansować, aby się z~nimi spotkać, dlaczego nie mógł załatwić podniesienia poziomu swojej postaci razem z~nimi? Dlaczego płacił najeźdźcom? 


Wei-Dong podejrzewał, że to dlatego, że facet nie miał przyjaciół. 


-- Cholera, czy mógłbyś na to spojrzeć? 

To był co najmniej dziesiąty raz, kiedy facet powiedział to w~ciągu dziesięciu minut, kiedy jechali nad morze. Tym razem była to impreza herbaciana, nieustanna walka w~zwarciu, w~której sztućce gwizdały w~powietrzu, dzikie krzesła wędrujące w~paczkach, ścigając nieszczęsnych graczy, którym zdarzyło się ich rozwalić, i~szalenie trudna łamigłówka, w~której trzeba było zebrać i~tak ułożyć naczynia, ogłuszając każdy kawałek, aby nie wyczołgał się, zanim z~nim skończysz. Wei-Dong musiał przyznać, że to było całkiem fajnie (rozwiązał zagadkę w~ciągu dwóch dni ciężkiej gry i~za swoje kłopoty zdobył czajniczek, którego mógł użyć do przywołania dżinów w~chwilach wielkiej potrzeby). Jednak gweilo zachowywał się tak, jakby nigdy nie widział grafiki komputerowej. 


Jechali dalej, rozmawiając po chińsku na prywatnym kanale. Przeważnie działo się to zbyt szybko, by Wei-Dong mógł nadążyć, ale rozumiał sedno. Rozmawiali o pracy, rajdach, które organizowali do końca wieczora, szefie i~jego głupich zasadach, pieniądzach i~tym, co z~nimi zrobią. Dziewczyny. Zawsze mówili o dziewczynach. 


W końcu znaleźli się nad morzem i~Wei-Dong rzucił Kieszonkę Powietrzną Czerwonej Królowej, używając do tego ostatnich ostryg. Wszyscy zsiedli, komicznie trzepocząc skrzela, gdy wpadali do wody ( \textit{Cholera}, szepnął gweilo). 


Ogród Morsa był trudnym rajdem, ponieważ za każdym razem był inny, teren regenerował się dla każdej drużyny. Jako czarodziej, Wei-Dong miał za zadanie utrzymywać włączone światła i~przepływ powietrza, aby bez względu na to, co się nadchodziło, zobaczyli to na czas, by się przygotować i~pokonać. Najpierw przyszły ośmiornice, unosząc się z~dna w~obłoku piasku, płynąc w~ich kierunku przez wodę. Lu, tank, ustawił się między drużyną a ośmiornicami, a po chwili miotania się i~wystrzeleniu w~nie kilku pocisków, żeby je sprowokować, znieruchomiał, gdy jedna po drugiej owijały się wokół niego, miażdżąc go swoimi długimi mackami, ich twarze wyglądały jak szalone maski czystej wrogości. 


Kiedy już wszystkie były pochłonięte tankiem, reszta grupy otoczyła je, cała czwórka wyciągnęła broń z~wodnistym \textit{szczękiem} i~zabrała się do pracy w~wijącym się supełku. Wei-Dong bacznie przyglądał się zdrowiu tanka i~rzucał zaklęcia leczące w~razie potrzeby. Gdy każda ośmiornica była bliska śmierci, najeźdźcy odsunęli się, a Wei-Dong syknął do swojego mikrofonu: 

-- Wykończ je! -- Gweilo babrał się przy pierwszych dwóch bestiach, ale pod koniec poruszał się sprawnie. 


-- To było \textit{chore }-- powiedział gweilo. -- Całkowity twardziel! Ale jak ten facet zaabsorbował te obrażenia? 


-- To tank -- powiedział Wei-Dong. -- Klasa wojownika, ciężka zbroja. Mnóstwo wzmocnień. A ja cały czas utrzymywałem zaklęcia leczące. 


-- Też jestem w~klasie wojownika, prawda? 


\textit{Nie wiesz? }Ten gość ma \textit{więcej} forsy niż mózgu, to na pewno. 


-- Właśnie zacząłem grać. Nie jestem jakimś tam graczem. Ale wiesz, wszyscy moi przyjaciele \ldots  


\textit{Wiem}, pomyślał Wei-Dong. Wszyscy fajni, których znałeś, grali, więc zdecydowałeś, że musisz do nich dołączyć. Nie masz przyjaciół, jeszcze. Jednak myślisz, że będziesz miał, jeżeli będzie grał. 

-- Pewnie -- odpowiedział. -- Trzymaj się blisko, poradzisz sobie. Podniesiesz poziom przed śniadaniem. 

To była kolejna oznaka przeciwko gweilo, miał pieniądze, żeby zapłacić za sesję z~gildią najeźdźców, ale nie zamierzał płacić premium, żeby odbyć ją w~amerykańskiej strefie czasowej. To były dobre wiadomości dla reszty gildii, oszczędzało im to czasu szukania gdzieś miejsca do grania w~ciągu dnia w~Chinach, kiedy kafejki internetowe były wypełnione klientami, ale to oznaczało, że Wei-Dong musiał wstać w~środku nocy, a potem przetrwać szkołę następnego dnia. 


Nie, żeby nie było warto. 


Teraz znaleźli się między skałami i~jaskiniami ogrodu, unikając węgorzy i~gigantycznych homarów, które wyskakiwały z~jam, gdy przechodzili. Wei-Dong znalazł kilka skorup homarów i~ukradkiem je podniósł. Technicznie rzecz biorąc, były gweilo, który miał pierwszeństwo, ale były potrzebne, jeśli zamierzał dalej rzucać Kieszonkę Powietrzną, co może być konieczne, jeśli utrzymają to wolne tempo. A i~tak gweilo tego nie zauważył.  


-- Nie jesteś w~Chinach, prawda? -- spytał gweilo.  


-- Niezupełnie -- powiedział, patrząc przez okno na niebo nad hrabstwem Orange, najnudniejszy kod pocztowy w~Kalifornii. 

-- Gdzie jesteście?  


-- Oni w~Chinach. Tam, gdzie mieszkam, co wieczór można oglądać fajerwerki w~Disneylandzie.  


-- \textit{Cholera }-- powiedział. -- Nie masz lepszych rzeczy do roboty niż pomaganie jakiemuś idiocie w~awansowaniu w~środku nocy?  


-- Chyba nie -- odpowiedział. Z tyłu wmieszali się faceci śmiejący się i~wołający do siebie po chińsku na swoim kanale. Uśmiechnął się, słysząc ich. 


-- To znaczy, do diabła, rozumiem, dlaczego ktoś w~Chinach robi gównianą robotę za marne 75 dolców, ale jeśli jesteś w~Ameryce, stary, powinieneś mieć trochę \textit{dumy}, dostać prawdziwą pracę! 


-- A dlaczego ktoś w~Chinach miałby wykonywać gównianą robotę? -- Chłopaki teraz słuchali. Nie mówili zbyt dobrze po angielsku, ale znali na tyle, żeby sobie poradzić. 


-- Wiesz, to \textit{Chiny}. Są ich \textit{miliardy}. Biedne jak brud i~ciemne. Nie obwiniam ich. Nie możesz ich winić. To nie ich wina. Ale do diabła, kiedy wyjedziesz z~Chin i~dostaniesz się do Ameryki, powinieneś \textit{zachowywać }się jak Amerykanin. Nie wykonujemy takiej pracy. 


-- Co sprawia, że myślisz, że wyjechałem z~Chin? 


-- A nie? 


-- Urodziłem się tutaj. Moi rodzice urodzili się tutaj. Ich rodzice urodzili się tutaj. Ich rodzice przyjechali tu z~Rosji. 


-- Nie wiedziałem, że w~Rosji mają Chińczyków. 


Wei-Dong się roześmiał. 

-- Nie jestem Chińczykiem, koleś. 


-- Nie jesteś? Cóż, \textit{do cholery}, przepraszam. Myślałem, że tak. Kim więc jesteś, ich szefem czy kimś takim? 


Wei-Dong zamknął oczy i~policzył do dziesięciu. Kiedy otworzył je ponownie, cieśle wypłynęły przed nimi z~rozbitego galeonu, mając w~pogotowiu swoje T-kwadraty i~piły. Poruszali się, budując wokół siebie drewniane skrzynie i~bramy, które pełniły rolę barykad, i~pracowali \textit{szybko}. Na lądzie można było spalić belki, ale to nie zadziała pod morzem. Kiedy już cię opakowali, wbijali długie gwoździe w~deski wokół ciebie. To był makabryczny, powolny sposób na śmierć. 


Oczywiście w~mgnieniu oka otoczyli gweilo i~wszyscy musieli się ułożyć, aby walczyć z~nimi swobodnie. Xiang wezwał swojego chowańca, dzika, a Wei-Dong wyczarował mu własny pęcherzyk powietrza, który zabrał się do pracy, rozrywając deski kłami. Kiedy w~końcu stolarzom udało się go zabić, zamienił się w~dziecko i~wypłynął bez życia na powierzchnię oceanu w~akompaniamencie upiornego płaczu. Savage Wonderland \textit{wyglądało }jakby to był tylko śmiech, ale kiedy się do tego przystąpiło, było naprawdę ponuro, a łamigłówki były trudne, a wielcy bossowie \textit{naprawdę }twardzi. 


A propos bossów: załatwili ostatnich cieśli, a gdy to zrobili, wirujący prąd poruszył dno morza, wzbijając piasek, który powoli osiadał, odsłaniając wibracyjne ostrze i~zbroję pokrytą pąklami. A gweilo wydał z~siebie okrzyki, krzyki i~niezgrabnie zanurkował, gdy wszyscy natychmiast krzyknęli, żeby się zatrzymał, poczekał, a potem\ldots  


A potem uruchomił pułapkę, o której wszyscy wiedzieli, że tam jest. 


A potem nastąpiły \textit{kłopoty}.  


Jabberwock rzeczywiście miał ogniste oczy i~wydawał ,,bulgoczący'' dźwięk, tak jak to było napisane w~wierszu. Ale Jabberwock robił o wiele więcej niż tylko brzydkie spojrzenia i~beknięcie. Jabberwock był \textit{wredny}, pochłaniał wiele obrażeń i~był tak dobry, jak tylko mógł. Był też szybki, szybszy niż stolarze, więc w~jednej minucie można było być za nim, a potem robił beczkę -- ogon jak bicz, trzaskając i~odrzucając wszystko, co stanie mu na drodze -- i~był twarzą do ciebie, wznosząc się z~rozpostartymi wrzecionowatymi pazurami i~falującą wąską klatką piersiową. Szczęki, które gryzą, pazury, które łapią, a kiedy już cię złapią, Jabberwock tłukł tobą o najtwardszą powierzchnię w~zasięgu, zadając szalone obrażenia, podczas gdy ty wiłeś się, by się uwolnić. A bulgotanie? Nie tak bardzo jak odbijanie, naprawdę: bardziej jak dźwięk mięsa przechodzącego przez maszynkę, paskudny dźwięk. \textit{Krwawy }dźwięk. 


Gdy po raz pierwszy Wei-Dongowi udało się zabić Jabberwocka -- po weekendowej nieustannej zabawie -- wstrząsnęło to nim i~miał koszmary związane z~tym dźwiękiem. 


-- Niezła zabawa, dupku -- powiedział Wei-Dong, uderzając w~klawiaturę, próbując uruchomić wszystkie swoje zaklęcia i~nie dać się wypatroszyć przez koszmarną bestię przed nimi. Miało Lu i~wybijało z~niego wszystkie płyny, ale to było w~porządku, to był tylko Lu, jego zadaniem było dać się pobić. Wei-Dong rzucił zaklęcia lecznicze na Lu, gdy ten odpłynął tak szybko, jak tylko mógł. 


-- To nie jest miłe -- powiedział gweilo. -- Skąd, do diabła, miałem wiedzieć \ldots  


-- Nie byłeś. Nie wiedziałeś. Nie wiesz. O to właśnie \textit{chodzi.} Dlatego \textit{nas }zatrudniłeś. Teraz zużyjemy wszystkie nasze zaklęcia i~eliksiry, walcząc z~tym czymś\ldots  -- przerwał na sekundę i~nacisnął kilka klawiszy -- a zabierze to kilka \textit{dni}, żeby wszystko odebrać, tylko dlatego, że nie mogłeś czekać z~tyłu, tak jak \textit{powinieneś}. 


-- Nie muszę tego słuchać -- powiedział gweilo. -- Jestem klientem, do cholery.  


-- Chcesz być \textit{martwym }klientem, kolego? -- powiedział Wei-Dong. 

Podczas całego nalotu ledwo miał czas porozmawiać ze swoimi gildiami, utknął w~rozmowie z~tym głupim anglojęzycznym. Teraz facet mu pyskował. To sprawiło, że chciał rzucić komputerem o ścianę. Widzisz, co daje ci bycie miłym? 


Jeśli gweilo odpowiedział, Wei-Dong tego nie usłyszał, ponieważ Jabberwock naprawdę grzał. Skończyły mu się eliksiry i~zaklęcia lecznicze, a Lu nie wytrzyma długo. O \textit{gówno}. Miał teraz Pinga w~drugim szponie i~męczył jego zbroję długim kłami, próbując go obrać jak winogrono. Połączył się z~kontrolerem czatu głosowego i~włączył chiński kanał na pełny, wyłączając gweilo. 


To był chaos szybkiego, bluźnierczego dialektu, slangowego chińskiego, który mieszał się z~przekleństwami z~japońskich komiksów i~indyjskich filmów. Wszyscy chłopcy krzyczeli, zbyt szybko, by mógł zrozumieć coś więcej niż sens. 


Był jednak Ping, który go wołał. 

-- Leonard! Uzdrowienie! 


-- Nic nie mam! -- powiedział, nienawidząc tego, jak to wszystko szło. -- Jestem całkowicie pusty. Zużyłem to wszystko na Lu!  


-- W takim razie to wszystko -- powiedział Ping. -- Nie żyjemy. -- Wszyscy wyli z~rozczarowania. Wbrew sobie Wei-Dong uśmiechnął się. -- Myślisz, że zmieni termin, czy będziemy musieli oddać mu pieniądze? 


Wei-Dong nie wiedział, ale miał przeczucie, że ten goober nie będzie zbyt chętny do współpracy, jeśli powiedzą mu, żeby wstał w~środku nocy bez powodu. Nawet jeśli to jego wina. 


Wciągnął kilka świszczących oddechów przez nos i~próbował się uspokoić. Była prawie druga w~nocy. W otaczającym go domu panowała cisza. Gdzieś daleko, gdzieś daleko, włączył się silnik, ale noc była tak cicha, że dźwięk dotarł do jego sypialni. 


-- Dobrze -- powiedział. -- OK, pozwól mi coś z~tym zrobić.  


Każda gra miała kilka BFG, Big Friendly Guns (lub przynajmniej \textit{jakiś }rodzaj Big Gun), które były prawie niemożliwe do zdobycia i~prawie niemożliwe do odparcia. W Savage Wonderland były również prawie niemożliwe do załadowania: rzadki potworny garłacz, do którego trzeba było spędzać miesiące na zbieraniu części do wystrzelonych ogromnych partii zaostrzonych sztućców z~Tea Party, a zebranie wystarczającej ilości na jeden ładunek zajmowało osiem lub dziewięć godzin rozgrywki. Niemożliwe do zdobycia -- niemożliwe do załadowania. Praktycznie nikt go nie miał. 


Ale Wei-Dong miał. Ignorując krzyki w~słuchawkach, wycofał się na skraj zasięgu garłacza i~zaczął go uzbrajać, co było pracochłonnym procesem wrzucania wszystkich sztućców do pyska.

-- Stańce frontem do tego -- powiedział. -- Teraz, przed niego!  


Jego gildie widziały, co teraz robi, i~ryczały triumfalnie, ustawiając swoje plutony na jego przodzie, zajmując uwagę potwora, oczyszczając linię ognia. Potrzebował tylko jeszcze \ldots  jednej \ldots  sekundy. 


Pociągnął za spust. Rozległ się trzask i~syk, gdy proszek w~panewce zaczął się palić. Dźwięk sprawił, że Jabberwock obrócił głowę na długiej, wężowej szyi. Spojrzał na niego płonącymi oczami i~rzucił Ping i~Lu na dno oceanu. Proszek na panewce rozbłysnął i~--- umarł. 


Niewypał.  


\textit{gównogównogówno}, mamrotał, uderzając \textit{uderzając} sekwencję uzbrajania, jego palce rozmyte na myszce. 

-- Gównogównogówno. 


Jabberwock uśmiechnął się i~ponownie wydał ten mokry, mięsisty dźwięk. \textit{Burble burble, chłopczyku, idę po ciebie}. To był dźwięk jego koszmaru, dźwięk jego umierającego marzenia o bohaterstwie. Dźwięk marnowania amunicji na cały dzień i~gry na noc. Był trupem. 


Jabberwock wykonał jedną z~tych bijących, falujących beczek, które były jego znakiem rozpoznawczym. Uderzyły go prądy, które kołysały się z~boku na bok. Poprawiał, przeprawiał, poprawiał ponownie, wciskał przycisk ponownego uzbrajania, przycisk strzału, przycisk ponownego uzbrajania, przycisk strzału\ldots  


Jabberwock stał teraz naprzeciw niego. Cofnął się, napinając pazury, zaciskając szczęki. Za chwilę znajdzie się na nim, otworzy go od krocza do gardła i~pożre jego wnętrzności, lada chwila\ldots  


\textit{Bum!} Dźwięk garłacza był jak eksplozja w~szufladzie z~widelcami, milion metalicznych brzęków i~szczęków, gdy morze zostało pocięte przez szybko rozwijający się stożek śmiertelnej, ryczącej metalowej zastawy stołowej. 


Jabberwock \textit{rozpuścił się}, rozdarty na powoli rosnący grzyb z~mięsa, pazurów i~skórzastych łusek. Lewa strona jego głowy oddarła się w~jego kierunku i~odbiła od niego, osiadając na piasku. Woda zrobiła się różowa, potem czerwona, a śmiertelny zgrzyt Jabberwocka zdawał się spływać z~wody i~zalewać go raz za razem. To był \textit{fantastyczny }dźwięk. 


Jego gildie szalały jedenaście tysięcy kilometrów dalej, wykrzykując jego imię, a nie \textit{Leonard}, ale \textit{Wei-Donga}, intonując je w~swojej kafejce internetowej przy Jiabin Road w~Shenzhen. Wei-Dong szczerzył się dziko w~swojej sypialni, wygrzewając się w~śpiewie. 


A kiedy woda się oczyściła, w~skorupie pąkli znów pojawiły się wirowe ostrze i~hełm, niewinnie osadzone na dnie oceanu. Gweilo -- gweilo, zapomniał o gweilo! -- ruszył niezdarnie w~tamtym kierunku. 


-- Nie sądzę -- powiedział Ping całkiem nieźle po angielsku. 

Jego pluton poruszał się tak szybko, że gweilo prawdopodobnie nawet go nie zauważył. Miecz Pinga zawył, a głowa gweila opadła na piasek z~głupim, zdradzonym wyrazem twarzy. 


-- Co \ldots  


Wei-Dong wyrzucił go z~czatu. 


-- To twój skarb, bracie -- powiedział Ping. -- Zarobiłeś to.  


-- Ale pieniądze\ldots   


-- Możemy zarobić pieniądze jutro wieczorem. To było \textit{zabójcze, stary}! -- Było to jedno z~ulubionych angielskich zwrotów Pinga i~najwyższe uznanie w~ich gildii. A teraz miał wibracyjne ostrze i~hełm. To była dobra noc. 


Wynurzyli się, wiosłowali do brzegu i~ponownie wyczarowali swoje wierzchowce, po czym pojechali z~powrotem do siedziby gildii, rozmawiając przez całą drogę i~bez większego zamieszania pokonując od czasu do czasu pomniejszą bestię. Chłopaki nie przerażali się tym, że są o 75 dolców biedniejsi, niż się spodziewali. Na pierwszym miejscu to byli gracze, na drugim ludzie biznesu. I to było \textit{zabawa}. 


A teraz była 2:30 i~za cztery godziny musiał być na nogach do szkoły, a w~tym tempie miał leżeć bezsennie przez \textit{długi }czas. 

-- OK, idę chłopaki -- powiedział w~swoim najlepszym chińskim. Pożegnali się z~nim i~kanał czatu przestał działać. W nagłej ciszy swojego pokoju słyszał bicie swojego pulsu w~uszach. I jeszcze jeden dźwięk -- kroki na podłodze za jego drzwiami. Ręka na klamce\ldots  


gównogównogówno


Udało mu się opuścić pokrywę laptopa i~podciągnąć kołdrę, zanim drzwi się otworzyły, ale nadal trzymał maszynę pod prześcieradłem, a spojrzenie ojca zza drzwi powiedziało mu, że nikogo nie oszukuje. Bez słowa, wciąż wściekle, jego ojciec przeszedł przez pokój i~delikatnie wyjął słuchawkę z~ucha Wei-Donga. Świeciła charakterystycznie niebieskim światłem, mrugając, szukając laptopa, który spał teraz pod artystycznie odświeżoną pościelą Spongeboba Wei-Donga. 


-- Tato \ldots  -- zaczął. 


-- Leonard, jest 2:30 rano. Nie zamierzam teraz z~tobą o tym rozmawiać. Ale porozmawiamy o tym rano. A potem będziesz miał dużo, bardzo dużo czasu, żeby o tym myśleć -- Odsunął kołdrę i~wyjął laptop z~bezwładnej już dłoni Wei-Dong.  


-- Tato! -- powiedział, gdy jego ojciec odwrócił się i~wyszedł z~pokoju, ale ojciec nie dał żadnego znaku, że słyszał, zanim mocno i~autorytatywnie zamknął drzwi sypialni. 


\bigskip
\threeast

Ta scena jest poświęcona Borderlands Books, wspaniałej niezależnej księgarni science fiction w~San Francisco. Borderlands słynie nie tylko ze wspaniałych wydarzeń, podpisywania kontraktów, klubów książki i~tym podobnych, ale także z~niesamowitego bezwłosego egipskiego kota Ripleya, który lubi przysiadać jak brzęczący gargulec na komputerze przed sklepem. Borderlands to najbardziej przyjazna księgarnia, o jaką możesz poprosić, wypełniona wygodnymi miejscami do siedzenia i~czytania i~obsługiwana przez niezwykle kompetentnych urzędników, którzy wiedzą wszystko o science fiction. Co więcej, zawsze chętnie przyjmowali zamówienia na moją książkę (przez internet lub telefon) i~będą trzymać ją do podpisania, kiedy wpadnę do sklepu, a następnie wyślą ją na terenie Stanów Zjednoczonych za darmo! 


\href{https://www.borderlands-books.com/v2/index.html}{\textit{Borderland Books 866 Valencia Ave, San Francisco CA USA 94110 +1 888 893 4008}} 

\bigskip
\threeast

Mala tęskniła za śpiewem ptaków. Kiedy mieszkali w~wiosce, każdego ranka rozlegał się śpiew ptaków, przerywając doskonały spokój nocy, aby dać im znać, że słońce wschodzi i~zaczyna się dzień. Wtedy była małą dziewczynką. Tu, w~Bombaju, o świcie rozległy się chorowite głosy koguta, ale niemal zagłuszyła je niekończąca się pieśń uliczna: klaksony, szum silników, nawoływania o świcie. 


W wiosce rozlegały się śpiewy ptaków, cisza i~spokój, czasy, kiedy wszyscy nie zawsze obserwowali. W Bombaju nie było niczego oprócz ludzi, ludzi wszędzie, więc każdy oddech, który brałaś, smakował ustami, które go wydychały, zanim go dostałaś. 


Ona, jej matka i~jej brat spali razem w~maleńkim pokoju nad fabryką recyklingu plastiku pana Kunala w~Dharavi, wielkim slumsie lokatorów na północnym krańcu miasta. W ciągu dnia pomieszczenie służyło do sortowania plastiku do kilkunastu pojemników -- plastiku pochodzącego z~niekończącej się procesji ogromnych worków ryżowych, które były napełniane w~stoczniach. Statki płynęły do Ameryki, Europy i~Azji wypełnione towarami wyprodukowanymi w~Indiach i~wracały wypełnione śmieciami, plastikiem, który zbieracze z~Dharavi sortowali, czyścili, topili i~przekształcali w~granulki, a następnie wysyłali do fabryk, aby mogły zostać przetworzone na towary i~wysłane z~powrotem do Ameryki, Europy i~Azji. 


Kiedy przybyli do Dharavi, Mala uznała to za przerażające: wąskie chaty zasłaniające niebo, drogi między nimi z~rynnami biegnącymi opalizując na niebiesko i~czerwono z~farbiarni, duszący się zawsze zapach płonący plastik, ryk motocykli pędzących między budynkami. I oczy, oczy z~każdego okna i~dachu, wszystkie obserwujące, jak mamaji prowadzi ją i~jej młodszego brata do fabryki pana Kunala, gdzie mieli mieszkać teraz i~na wieki. 


Ale minął zaledwie rok, a zapach zniknął. Oczy stały się przyjazne. Mogła przeskakiwać z~jednego pasa na drugi z~całkowitą pewnością siebie, nigdy nie gubiąc się w~drodze na rynek lub na popołudniowe zajęcia w~małej sali szkolnej nad restauracją. Praca przy sortowaniu była nudna, ale nigdy ciężka i~zawsze było jedzenie, były też inne dziewczyny do zabawy, a mamaji zaprzyjaźniła się z~przyjaciółmi, którzy im pomagali. Kawałek po kawałku stała się dziewczyną Dharavi, a teraz patrzyła na nowo przybyłych z~mieszaniną hojności i~litości. 


A praca\ldots  no cóż, ostatnio znacznie się poprawiła. 


Zaczęło się, gdy była w~kawiarni gier z~Yasmin, kradnąc godzinę po lekcjach, aby wydać kilka rupii z~pieniędzy, które zaoszczędziła z~pakietu wypłat (prawie wszystko, oczywiście, trafiło do rodziny, ale mamaji czasami pozwalała jej zatrzymać trochę i~radziła jej, aby wydała je na smakołyk w~sklepie na rogu). Yasmin nigdy nie grała w~Zombie Mecha, ale oczywiście oboje widziały filmy w~małym domu przy drodze, która oddzielała muzułmańską i~hinduską część Dharavi. Mala \textit{kochała }Zombie Mecha i~była w~tym dobra. Wolała serwery PvP, na których gracze mogli polować na innych graczy, próbując przewrócić swoje gigantyczne mecha-skafandry, aby zombie wokół nich mogły się nad nim zaroić, otworzyć osłonę kokpitu i~ucztować na avatarach w~środku. 


Większość dziewczyn z~kawiarenki gier przychodziła i~grała w~małe gry z~uroczymi zwierzętami i~wymieniała się sercami i~klejnotami. Ale dla Mali akcja była w~niesamowitej rzezi w~wieloosobowych grach wojennych. Tylko kilka minut zajęło przyuczenie Yasmin, aby złapała podstawy pilotowania swojej małej eskadry, a potem mogła zabrać się do \textit{taktyki}. 


To było to, czego żaden z~pozostałych graczy nie zdawał się rozumieć: \textit{taktyka }była wszystkim. Traktowali grę tak, jakby to był przypadkowy chaos piszczących rakiet i~eksplozji, zamieszanie, w~które należy wkroczyć i~przeżyć, najlepiej jak się da. 


Ale dla Mali zamieszanie było czymś, co przydarzało się innym ludziom. Dla Mali eksplozje, drżenie kamery i~pisk zombie były tylko drobnymi szczegółami, które można było zauważyć wśród Wielkiego Obrazu, armii ustawionych na polu bitwy w~jej umyśle. Na tym polu bitwy zmasowane siły nabrały gęstości i~koloru, które pokazały, gdzie są ich mocne i~słabe strony, jak są ze sobą połączone i~jak pchnięcie tutaj, przewróciłoby tamtego. Możesz stawić czoła swoim wrogom, rakiety przeciwko rakietom, broń przeciwko pistoletom, a wtedy zwycięzcą będzie ten, który ma więcej szczęścia, lub ten z~największą ilością amunicji, lub ten z~najlepszymi tarczami. 


Ale jeśli byłaś \textit{mądra}, nie musiałaś mieć szczęścia ani być twardsza. Mala lubiła miotać rakietami i~granatami \textit{ponad }przeciwnymi armiami, po ich lewej i~prawej stronie, tworząc kaniony z~gruzu i~ruin, które blokowały im ucieczkę. W międzyczasie kilka jej błotniaków krążyło w~chwastach, atakując ogromne stada zombie, doprowadzając je do \textit{prawdziwego} szaleństwa, zbierając je, aż stały się jak szarańcza, zacierając ziemię we wszystkich kierunkach, prowadząc je coraz bliżej do kanionu. 


Tuż przed ich pojawieniem się, jej czołowa siła oderwała się, uciekając w~pozornym akcie tchórzostwa. Jej wrogowie zostaną przyciągnięci fałszywą pewnością siebie i~ruszą w~pościg, dopóki nie zobaczą, że błotniaki zbliżają się wprost na nich, z~niepowstrzymaną, ulewną zarazą zombie depczących im po piętach. W większości przypadków byli zbyt zszokowani, by \textit{cokolwiek }zrobić, nawet strzelać do błotniaków, gdy biegli prosto na ich linie i~\textit{przez nie}, do jedynej ucieczki pozostawionej w~kanionie, wysadzając szczelinę, gdy wychodzili. Potem była już tylko kwestia czekania, aż zombie przytłoczą i~pożrą twoich przeciwników, podczas gdy ty chichotałaś, jadłaś słodycze i~napiłaś się herbaty z~czajnika przy kasie. \textit{Szczególnie }satysfakcjonujące były odgłosy zombie rozdzierających armie jej wrogów i~obgryzających ich kości. 


Yasmin była rozproszona przez zombie, obrzydliwe wnętrzności, lśniące rakiety. Ale widziała, o tak, \textit{widziała}, jak strategie Mali były w~stanie zniszczyć znacznie większe armie przeciwnika i~przebolała swoje przeczulenie. 


I tak dalej grały, przyciągając publiczność: najpierw pohukujący, szyderczy chłopcy (którzy zamilkli, gdy patrzyli, jak armie padają przed nią, i~którzy bez cienia drwiny zaczęli nazywać ją \textit{Generał Robotwalla}), a potem dziewczęta, z~początku nieśmiałe, zaglądające chłopcom przez ramiona, potem przepychające się do przodu, wiwatujące i~walące pięściami w~ściany i~tupiące nogami przy każdym dramatycznym zwycięstwie. 


Nie było jednak tanie. Starannie zgromadzony zapas rupii Mali skurczył się, nieco buforowany kilkoma monetami od innych graczy, którzy zapłacili jej trochę tu i~tam, aby nauczyła ich, jak naprawdę grać. Wiedziała, że mogła pożyczyć pieniądze lub pozwolić, aby jakiś chłopak wydał je na nią -- już była zacięta rywalizacja o prawo do przejścia przez drogę do drinkwalla i~kupienia jej masala coli, musującej, pieniącej się pikantnej eksplozji coli oraz przyprawy masala z~kruszonym lodem, które łagodziły surowość z~tyłu gardła, która była jej stałym towarzyszem od czasu ich przybycia do Dharavi. 


Ale miłe dziewczyny ze wsi nie pozwalały chłopcom kupować im rzeczy. Chłopcy chcieli czegoś w~zamian. Wiedziała o tym, wiedziała o tym z~filmów i~z otaczającego ją życia. Wiedziała, co się dzieje z~dziewczynami, które pozwalają chłopcom zadbać o swoje potrzeby. Zawsze był rachunek. 


Kiedy ten dziwny mężczyzna po raz pierwszy podszedł do niej, pomyślała o miłych dziewczynach i~chłopakach i~o tym, czego się spodziewali, i~nie chciała z~nim rozmawiać ani patrzeć mu w~oczy. Nie wiedziała, czego chciał, ale nie zamierzał tego od niej dostać. Kiedy więc wstał z~krzesła przy kasie, gdy weszła do kawiarni, wstał i~podszedł, by ją zatrzymać w~swoim eleganckim lnianym garniturze, dobrych butach, krótkich, starannie naoliwionych włosach i~małym wąsiku, ona go wyminęła. Wyminęła go, udając, że nie słyszała, jak mówi: 

-- Przepraszam panienko 

I

-- Panna? Panno? Proszę, tylko chwilkę swojego czasu. 


Ale pani Dibyendu, właścicielka kawiarni, krzyknęła na nią: 

-- Mala, posłuchaj tego człowieka, posłuchaj, co ma ci do powiedzenia. Nie bądź niegrzeczna w~moim sklepie, o, nie! -- A ponieważ pani Dibyendu również pochodziła ze wsi, a jej matka powiedziała, że Mala może grać w~gry, ale tylko w~kawiarni pani Dibyendu, ponieważ pani Dibyendu jest osobą, której można ufać, że nie pozwoli na niewłaściwe postępowanie, narkotyki czy przemoc, czy przestępczość, Mala zatrzymała się i~odwróciła do mężczyzny, milcząc, oczekując. 


-- Ach -- powiedział. -- Dziękuję. -- Skinął na panią Dibyendu. -- Dziękuję. 

 Odwrócił się do niej i~do armii chłopców i~dziewcząt, którzy zgromadzili się wokół niej, do \textit{jej }armii, tych, którzy nazywali ją generałem Robotwallah i~naprawdę tak myśleli. 


-- Słyszałem, że jesteś bardzo dobrym graczem -- powiedział. Mala kiwała brodą w~przód i~w tył, przymykając oczy, pozwalając jej minie powiedzieć: \textit{Tak, jestem dobrym graczem i~jestem na tyle dobra, że nie muszę się tym przechwalać}.  


-- Czy ona jest dobrym graczem?  


Mala zwróciła się do swojej armii, 


która była zdyscyplinowana, by milczeć, dopóki nie skinęła im głową. Pomachała na nich brodą: \textit{dalej}. 


I wybuchli entuzjastycznym bełkotem, wychwalając zalety ich generała Robotwallah, epickich bitew, które stoczyli i~zwycięstw z~niemożliwymi do pokonania przeciwnościami. 


-- Mam trochę pracy dla dobrych graczy. 


Mala słyszała o tym plotki. 

-- Reprezentujesz ligę? 


Mężczyzna uśmiechnął się lekko i~pokręcił głową. Pachniał cytrusową wodą kolońską i~betelem, słodką kombinacją zapachów, których nigdy wcześniej nie czuła. 

-- Nie, nie liga. Wiesz, że w~grze są gracze, którzy nie grają dla zabawy? Gracze, którzy grają, aby zarabiać pieniądze? 


-- Jaki rodzaj pieniędzy nam oferujesz? 


Poruszył brodą i~zachichotał. 

-- Nie, nie do końca. Są gracze, którzy grają, aby gromadzić gierczane pieniądze, które sprzedają innym graczom, którzy są zbyt leniwi, by grać sami. 


Mala zastanawiała się nad tym przez chwilę. Kontenery wyjechały z~Indii wypełnione towarami i~wróciły wypełnione śmieciami dla Dharavi. Gdzieś tam, w~Ameryce z~filmów, był świat ludzi z~niewyobrażalnym bogactwem. 

-- Zrobimy to -- powiedziała. -- Mam już więcej kredytów, niż mogę wydać. Ile za nie płacą? 


Znowu chichot. 

-- Właściwie -- powiedział i~przerwał. Jej armia milczała teraz całkowicie, trzymając się każdego jego słowa. Z maszyn dochodził cichy trzask wojen, toczących się w~świecie wewnątrz sieci, przez cały dzień i~całą noc. -- Właściwie to nie do końca to. Chcemy, abyś ty i~twoi przyjaciele zniszczyli ich, zabili ich awatary, zabrali im fortunę. 


Mala pomyślała przez kolejną chwilę, zdziwiona. Kto chciałby zabić tych innych graczy? 

-- Jesteś rywalem? 


Mężczyzna potrząsnął brodą. \textit{Może tak, może nie.}  


Pomyślała trochę więcej. 

-- Pracujesz dla gry! -- powiedziała. -- Pracujesz dla gry i~nie chcesz \ldots  


-- Nieważne, dla kogo pracuję -- powiedział mężczyzna, unosząc palce. Na jednej dłoni nosił obrączkę, a na drugiej dwie złote obrączki. Zauważyła, że \hspace{0pt}brakuje mu górnych stawów na trzech palcach. To było powszechne we wsi, gdzie rolnicy zawsze wpadali w~maszyny. Oto mężczyzna z~wioski, mężczyzna, który przybył do Bombaju i~stał się mężczyzną w~schludnym garniturze ze schludnymi wąsami i~złotymi pierścionkami połyskującymi na tym, co zostało z~jego palców. To był powód, dla którego jej matka przywiozła je do Dharavi, powód bólu gardła, piekących oczu i~niekończącej się pracy nad wannami do sortowania plastiku. 


-- Ważne jest to, że zapłacimy Tobie i~Twoim przyjaciołom\ldots  


-- Mojej armii -- powiedziała, przerywając mu bez zastanowienia. Przez chwilę jego oczy błysnęły niebezpiecznie i~wyczuła, że zaraz ją spoliczkuje, ale nie ustępowała. Była już wielokrotnie spoliczkowana. Parsknął raz przez nos, po czym kontynuował. 


-- Tak, Mala, twojej armia. Zapłacilibyśmy ci za zniszczenie tych graczy. Powiedziano by ci, jakiego rodzaju mechami pilotowali, jakie były ich imiona graczy, i~musiałabyś ich wykorzenić i~zniszczyć. Zatrzymasz całe ich bogactwo i~dostaniesz też rupie. 


-- Ile?  


Zrobił zbolały wyraz twarzy, jakby miał trochę gazu. 

-- Może powinniśmy porozmawiać o tym później na osobności? W obecności twojej matki? 


Mala zauważyła, że nie powiedział ,,Twoich rodziców'', ale raczej ,,Twojej matki''. Pani Dibyendu i~on już rozmawiali. Wiedział o Mali, a ona nic nie wiedziała o nim. W końcu była tylko dziewczyną ze wsi, a to był świat, w~którym wciąż próbowała to wszystko zrozumieć. Była generałem, ale była też dziewczyną ze wsi. Dziewczyna Generał ze Wsi. 


Przyszedł więc tej nocy do fabryki pana Kunala, a matka Mali nakarmiła go thali i~papadamami z~kobiecego kolektywu papadamów, i~gotowali czaj w~czajniku elektrycznym, a mężczyzna udawał, że jego piękne ubrania i~złoto należą do tego miejsca i~przykucnął na piętach jak człowiek w~wiosce, owłosione kostki wystające spod skarpet. Nikt, kogo Mala znała, nie nosił skarpetek. 


-- Panie Banerjee -- powiedziała mamaji -- nie rozumiem tego, ale znam panią Dibyendu. Jeśli mówi, że można panu ufać \ldots  -- Urwała, bo tak naprawdę nie znała pani Dibyendu. W Dharavi istniało wiele niebezpieczeństw dla młodej dziewczyny. Mamaji martwiła się o nich bez końca, podczas gdy nocą czesała włosy Mali, na wszystkie sposoby, w~jakie dziewczyna może się tutaj zrujnować lub zranić. Ale pieniądze. 


-- Lakh rupii co miesiąc -- powiedział. -- Plus premia. Oczywiście, będzie musiała zapłacić swojej ,,armii''  \ldots  -- potrząsnął trochę brodą, \textit{widzisz, pamiętam }-- z~tego. Ale ile zależałoby do niej.  


-- Te dzieci nie miałyby pieniędzy, gdyby nie moja Mala! -- powiedział mamaji, urażona ich wyimaginowanymi proszącymi rękami. -- Oni tylko grają w~grę! Powinni się cieszyć, że mogą się z~nią bawić! 

Mamaji była wściekła, kiedy odkryła, że Mala bawiła się w~kawiarni przez te wszystkie popołudnia. Myślała, że Mala gra tylko raz na jakiś czas, nie każdą rupią i~każdą wolną chwilę. Ale kiedy mężczyzna -- pan Banerjee -- wspomniał o jej talencie i~pieniądzach, jakie może zarobić dla rodziny, nagle mamaji została dyrektorem biznesowym jej córki. 


Mala zobaczyła, że pan Banerjee wiedział, że tak się stanie i~zastanawiała się, co jeszcze pani Dibyendu powiedziała mu o ich rodzinie. 


-- Mamaji -- powiedziała cicho, patrząc w~dół, tak jak robili to w~wiosce. -- To moja armia i~muszę zapłacić, jeśli dobrze grają. W przeciwnym razie nie będą moją armią na długo. 


Mamaji spojrzał na nią twardo. Obok nich, młodszy brat Mali, Gopal, wykorzystał ich nieuwagę, aby wykraść ostatni kawałek bakłażana z~talerza Mali. Mala zauważyła, ale udała, że nie, i~skoncentrowała się na spuszczaniu oczu. 


-- Dobrze, Mala -- powiedziała  mamaji -- wiem, że chcesz być dobra dla swoich przyjaciół, ale najpierw musisz pomyśleć o swojej rodzinie. Znajdziemy uczciwy sposób, aby im to zrekompensować, może moglibyśmy przygotować dla nich cotygodniową ucztę tutaj, wykorzystując część pieniędzy. Jestem pewna, że wszystkim przydałby się dobry posiłek. 


Mala nie lubiła sprzeciwiać się matce, a nigdy nie robiła tego w~obecności obcych, ale\ldots  


Ale to była jej armia, a ona była ich generałem. Wiedziała, co ich motywuje, i~słyszeli, jak pan Banerjee oznajmił, że za ich usługi dostaną gotówkę. Wierzyli w~uczciwość. Nie pracowaliby na jedzenie, podczas gdy ona pracowała za lakh (\textit{lakh }-- \textit{100 000 }rupii! Cała rodzina żyła za 200 rupii dziennie!) gotówki. 


-- Mamaji -- powiedziała -- to nie byłoby słuszne ani sprawiedliwe. -- Mali przyszło do głowy, że pan Banerjee wspomniał o pieniądzach przed armią. Mógł być bardziej dyskretny. Może to było celowe. -- I wiedzieliby o tym. Sama nie zarobię tych pieniędzy dla rodziny, Mamaji. 


Jej matka zamknęła oczy i~oddychała przez nos, znak, że starała się opanować. Gdyby nie było pana Banerjee, Mala była pewna, że dostałaby porządne lanie, takie, jakie dostała od ojca, zanim ich opuścił, kiedy była niegrzeczną małą dziewczynką w~wiosce. Ale gdyby nie było tu pana Banerjee, nie musiałaby też odpowiadać w~ten sposób matce. 


-- Przepraszam za to, panie Banerjee -- powiedziała Mamaji, nie patrząc na Malę. -- Dziewczyny w~tym wieku stają się buntownicze\ldots  niemożliwe. 


Mala myślała o przyszłości, w~której zamiast być generałem Robotwallah, będzie musiała poświęcić swoje życie na błaganie i~zastraszanie swojej armii, aby bawiła się z~nią, aby mogła zatrzymać wszystkie pieniądze, które zarobili dla jej rodziny, podczas gdy ich rodziny głodowały, a ich matki domagały się, aby wracali do domu prosto ze szkoły. Kiedy pan Banerjee wspomniał o swojej gigantycznej kwocie, wyczarował wizję niewypowiedzianego bogactwa, prawdziwego domu, pięknych ubrań dla nich wszystkich, Mamaji mogłaby swobodnie spędzać popołudnia, gotując dla rodziny i~odpoczywając od upału, życie z~dala od Dharavi, dymu, piekących oczu i~bólu gardła. 


-- Myślę, że twoja mała dziewczynka ma rację -- powiedział pan Banerjee z~cichym autorytetem, a cała rodzina Mali wpatrywała się w~niego oniemiała. Dorosły, która staje po stronie Mali przy matce? -- Z tego, co widzę, jest bardzo dobrym przywódcą. Jeśli mówi, że jej ludzie potrzebują zapłaty, wierzę, że ma rację. -- Wytarł usta chusteczką. -- Oczywiście z~całym szacunkiem. Oczywiście nie marzyłbym o tym, żeby mówić ci, jak wychowywać dzieci. 


-- Oczywiście\ldots  -- powiedział Mamaji jak we śnie. Jej oczy były spuszczone, ramiona opadłe. Żeby słyszeć coś w~ten sposób, we własnym domu, od nieznajomego, przed swoimi dziećmi! Mala poczuła się okropnie. Jej biedna matka. I to wszystko była wina pana Banerjee: wspomniał o pieniądzach przed jej armią, a potem doprowadził jej matkę do tego punktu\ldots  


-- Znajdę sposób, żeby zmusić ich do walki bez zapłaty, Mamaji \ldots  -- Ale przerwała jej ręka matki, pojawiającej się, wyciągającej do niej dłoń. 


-- Cicho, córko -- powiedziała. -- Jeśli ten mężczyzna, ten \textit{dżentelmen}, mówi, że wiesz, co robisz, no cóż, to nie mogę temu zaprzeczyć, prawda? Jestem zwykłą kobietą ze wsi. Nie rozumiem tych rzeczy. Ty oczywiście musisz robić to, co mówi ten dżentelmen. 


Pan Banerjee wstał i~wygładził garnitur dłońmi. Mala zobaczyła, że dostało się trochę chany na koszulę i~klapy, i~to sprawiło, że poczuła się lepiej, jakby był śmiertelnikiem, a nie jakąś straszną siłą natury, która przybyła, by zniszczyć ich małe życia. 


Zrobił małe namaste ku Mamaji, z~rękami zaciśniętymi na piersi, z~małą nutą ukłonu. 

-- Dobranoc, pani Vajpayee. To była cudowna kolacja. Dziękuję. -- powiedział. -- Dobranoc, generale Robotwallah. Przyjdę do kawiarni jutro o trzeciej, żeby porozmawiać o twoich misjach. Dobranoc, Gopal -- powiedział, a jej brat spojrzał na niego z~poczuciem winy, bakłażan wystający z~kącika ust. 


Mala pomyślała, że Mamaji może ją spoliczkować, gdy mężczyzna wyjdzie, ale wszyscy poszli spać razem bez słowa, a Mala przytuliła się do matki tak samo jak co noc, gładząc jej długie włosy. Kiedy opuścili wioskę, lśniły i~były czarne, ale rok później były przeszyte szarością i~wydawało się, że są druciane. Dłoń Mamaji chwyciła ją i~uspokoiła, odciski na jej palcach szorstkie. 


-- Śpij, córko -- mruknęła. -- Masz teraz ważną pracę. Potrzebujesz snu. 


Następnego ranka unikały wzajemnie wzroku i~przez tydzień było ciężko, dopóki nie przyniosła do domu pierwszej paczki z~wypłatą, starannie złożonej w~podeszwie buta. Jej armia przebiła się przez siły wroga jak tasaki, którymi rzeźnik oddzielał głowy od kurczaków. W ich paczce była duża premia i~nawet po tym, jak zapłaciła pani Dibyendu i~kupiła wszystkim masalę colę w~sąsiednim hotelu Hajj, i~wypłaciła pensje wojsku, zostało prawie 2000 rupii, a ona zabrała Mamaji do najmniejszej sortowni na poddaszu fabryki, po drabinie. Oczy Mamaji rozbłysły, gdy zobaczyła pieniądze, i~pocałowała Mali w~czoło i~wzięła ją w~najdłuższy, najbardziej zaciekły uścisk w~ich wspólnym życiu. 


A teraz wszystko było między nimi cudownie. Mamaji zaczęła szukać dla nich miejsca bliżej środka Dharavi, starej części, w~której budynki z~blachy i~złomu były stopniowo zastępowane ceglanymi, gdzie piece garncarskie dymiły czystym dymem z~drewna zamiast brudnego, szorstkiego plastikowego dymu w~pobliżu fabryki pana Kunala. Mala miała nowe szkolne ubrania, nowe buty, tak samo Gopal, a Mamaji nowe szczotki do włosów i~nowe sari, które nosiła po pracy, wyglądając ładnie i~młodo, tak jak Mala pamiętała z~czasów wsi. 


A bitwy były \textit{wspaniałe}. 


Weszła do kawiarni z~topniejącego, zakurzonego słońca późnego dnia i~stanęła w~drzwiach. Jej armia była już zebrana, ćwiczyła na swoich maszynach, przepuszczała gupszup w~cieniu ciemnego, hałaśliwego pokoju lub łzawie patrzyła po sobie w~półmroku. Ledwo zdążyła się uśmiechnąć, a potem ukryć uśmiech, zanim ją zauważyli i~wstali na nogi, stojąc prosto i~dumnie, pozdrawiając ją. 


Nie wiedziała, które z~nich rozpoczęło salutowanie. Zaczęło się od żartu, ale teraz było to coś poważnego. Wibrowali na baczność, wszystkie oczy zwrócone na nią. Mieli na sobie lepsze ubrania, wyglądali na dobrze odżywionych. Generał Robotwallah prowadził swoją armię do zwycięstwa i~dobrobytu. 


-- Zagrajmy -- powiedziała.

W kieszeni miała telefon komórkowy z~najnowszą wiadomością od pana Banerjee z~lokalizacją celu dnia. Yasmin była na swoim zwykłym miejscu, po prawej ręce Mali, a po jej lewej Fulmala, która ciężko utykała z~powodu złamanej nogi, która się nie zagoiła prawidłowo. Ale Fulmala była bystra i~szybka, a taktykę znała lepiej niż ktokolwiek w~kawiarni, z~wyjątkiem samej Mali. A Yasmin, cóż, Yasmin potrafiła sprawić, by chłopcy się zachowywali, co było wielkim osiągnięciem, ponieważ pozostawieni samym sobie lubili sprzeczać się i~przegrywać, w~lekkomyślnej spirali, która zawsze kończyła się źle. Ale Yasmin potrafiła z~nimi rozmawiać w~sposób surowy, jak starsza siostra, i~dopasowywali się do porządku. 


Mala miała swoją armię, poruczników i~swoją misję. Miała swoją maszynę, najszybszą w~kawiarni, z~większym monitorem niż którykolwiek z~pozostałych, i~była gotowa iść na wojnę. 


Poprawiła swoje ekrany, przekręciła głowę z~boku na bok i~ponownie poprowadziła swoją armię do bitwy. 


\bigskip
\threeast


Ta scena jest poświęcona Barnes and Noble, amerykańskiej krajowej sieci księgarni. Gdy amerykańskie drobne księgarnie znikały, Barnes i~Noble zaczęli budować te gigantyczne świątynie do czytania w~całym kraju. Zaopatrując dziesiątki tysięcy tytułów (w księgarniach handlowych i~w sklepach spożywczych tylko niewielką część z~nich) i~utrzymując długie godziny, które były wygodne dla rodzin, ludzi pracy i~innych potencjalnych czytelników, sklepy B\&N utrzymywały kariery wielu pisarzy, unosząc ich na powierzchni, zaopatrując się w~tytuły, których mniejsze sklepy nie mogły sobie pozwolić na przechowywanie na swoich ograniczonych półkach. B\&N zawsze prowadziło silne programy dla społeczności, a ja wykonałem niektóre z~moich najlepiej uczęszczanych, najlepiej zorganizowanych spotkań z~podpisywanie w~sklepach B\&N, w~tym wspaniałe wydarzenia w~(niestety nieistniejącym) B\&N na Union Square w~Nowym Jorku, gdzie miało miejsce megapodpisywanie po rozdaniu nagród Nebula, oraz B\&N w~Chicago, który był gospodarzem tego wydarzenia po Nebulach kilka lat później. Najlepsze jest to, że nabywcy-geeki B\&N naprawdę to rozumieją, jeśli chodzi o science fiction, komiksy i~mangę, gry i~podobne tytuły. Są pasjonatami i~wiedzą w~tej dziedzinie, co widać w~doskonałej selekcji na wystawie w~sklepach. 


\href{https://www.barnesandnoble.com/b/contributor/cory-doctorow/_/N-2qlt}{Barnes and Noble, cały kraj} 

\bigskip
\threeast

Złoto. Chodzi o złoto. 


Ale nie zwykłe złoto, takie, jakie wykopujesz z~ziemi. To były rzeczy z~zeszłego stulecia. Po pierwsze, jest go za mało: całe złoto, jakie kiedykolwiek wykopano z~ziemi w~historii świata, równałoby się jedynie sześcianowi o bokach długości kortu tenisowego. I co ciekawe, jest go też za dużo: wszystkie świadectwa własności złota wydane na świecie sumują się do sześcianu dwa razy większego. Niektóre z~tych certyfikatów nic nie znaczą i~nikt nie wie, które. Nikt nie przeprowadził niezależnego audytu Fort Knox od 1956. Z tego, co wiemy, jest pusty, złoto przeszmuglowane i~sprzedane, umieszczone w~skarbcu, sprzedane jako certyfikaty, a następnie ponownie skradzione i~umieszczone w~innym skarbcu, używane jako podstawa dla kolejnych certyfikatów. 


Nie zwykłe złoto. 


\textit{Wirtualne} złoto. 


Nazwij je, jak chcesz: w~jednej grze nazywa się ,,Kredyty'', w~innej ,,Volcano Bucks''. Są jeszcze grosze, Disney Dollary, kauri, moola, Fool's Gold i~milion innych rodzajów złota. W przeciwieństwie do prawdziwego złota nie istnieje skarbiec rezerw podtrzymujący certyfikaty. W przeciwieństwie do pieniędzy, w~ich emisję nie jest zaangażowany żaden rząd. 


Wirtualne złoto emitowane jest przez firmy. Firmy zajmujące się grami. Kompanie gier, które deklarują: ,,Tyle sztuk złota kosztuje ten kawałek zbroi'' lub ,,Tyle kredytów kosztuje statek kosmiczny'' lub ,,Za tyle Jools można kupić ten sterowiec''. A ponieważ tak mówią, to prawda. Kraje i~ich banki muszą grzebać w~brzydkim biznesie przekonywania obywateli, by uwierzyli w~to, co mówią: rząd może powiedzieć: ,,Ten czek ubezpieczenia społecznego zaspokoi wszystkie twoje potrzeby na miesiąc'', ale to nie znaczy, że kupcy, którzy zaspokajają te potrzeby, też się zgodzą. 


Firmy nie mają tego problemu. Kiedy Coca Cola mówi, że za 76 groszy kupisz jeden krasnoludzki topór w~Svartalfaheim Warriors, to tyle: cena siekiery to 76 groszy. Nie lubisz? Idź pograć gdzie indziej. 


Wirtualne pieniądze nie są wspierane przez złoto ani rządy: są wspierane przez \textit{zabawę}. Tak długo, jak gra jest fajna, gracze gdzieś będą chcieli się w~nią kupić, ponieważ o ile gra jest zabawą, zawsze jest fajniej, jeśli jesteś jednym z~posiadaczy, ze wszystkimi niesamowitymi zbrojami i~zabójczą bronią, niż gdybyś był jakiś skromnym noobem ze sztyletem, walczącym o swój pierwszy miecz. 


Ale tam, gdzie są pieniądze do wydania, są pieniądze do zarobienia. Dla niektórych graczy najfajniejszą grą ze wszystkich jest gra, w~której wykrawają kawałek tortu. Nie cała akcja należy do gigantycznych firm w~swoich wieżowcach i~gier, które tworzą. Wielu z~nas może wziąć udział w~akcji od dołu, gdzie są brudne ludziki. 


Oczywiście to sprawia, że firmy \textit{szaleją}. Są wielkimi tatusiami, wiedzą, co jest najlepsze dla ich światów. Oni mają \textit{władzę}. Projektują poziomy i~trudność, aby wszystko było idealnie zbalansowane. Projektują puzzle. Dekretują, że lekkie elfy nie mogą rozmawiać z~mrocznymi elfami, że gracze na rosyjskich serwerach nie mogą wskakiwać na chińskie serwery, że przeciętnemu graczowi zajęłoby 32 godziny, aby zdobyć napęd Von Klausewitza i~48 godzin, aby zdobyć Order Pancernego Pingwina. Jeśli ci się to nie podoba, powinieneś \textit{odejść}: nie powinieneś po prostu \textit{kupować }sobie wejścia. A jeśli już kupujesz, to powinieneś mieć na tyle przyzwoitości, żeby kupić \textit{je} od nich. 


A mała rzecz, o której ci nie powiedzą, brzmi, ci Bogowie Wirtualizmu: nie \textit{mogą }tego kontrolować. Dzieciaki, przestępcy i~dziwacy na całym świecie podziurawili ich bezpieczne, małe światy terrariów tunelami prowadzącymi na wspaniałe tereny. Istnieje wiele konkurencyjnych wymian międzyświatowych: chcesz wymienić swoje bogactwo Zombie Mecha na w~pełni załadowany statek kosmiczny i~załogę wesołych kosmicznych piratów, którzy będą załogą? Dziesięć różnych gangów chce twojego biznesu, załatwią cudzy statek kosmiczny i~zabiorą twojego mecha, broń i~amunicję do ekwipunku dla następnej osoby, która chce wyemigrować do Zombie Mecha z~jakiegoś innego magicznego świata. 


A Bogowie są bezsilni, by to powstrzymać. Na każdą barierę, którą postawią, istnieje setka inteligentnych, zmotywowanych graczy Wielkiej Gry, którzy ją obalą. 


Można by pomyśleć, że to niemożliwe, prawda? W końcu nie są to zwykłe zabawy gliniarzy i~złodziei, rozgrywane w~prawdziwych miastach wypełnionych prawdziwymi ludźmi. Nie potrzebują szczegółowego biuletynu, aby znaleźć zbiega na wolności: każda osoba na świecie jest w~bazie danych i~oni są właścicielami tej bazy. Nie potrzebują nakazu przeszukania, aby znaleźć kontrabandę ukrytą pod twoimi deskami podłogowymi: deski podłogowe, kontrabanda, dom i~wy wszyscy jesteście w~bazie danych -- a oni są właścicielami tej bazy danych. 


Powinno to być niemożliwe, ale tak nie jest, a oto dlaczego: największymi sprzedawcami złota i~skarbów, poziomów i~doświadczenia na światach są same \textit{firmy zajmujące się grami}. Och, nie \textit{nazywają }tego wyrównywaniem poziomu władzy i~uprawianiem złota -- nadają temu ładniejsze, bardziej smakowite nazwy, takie jak ,,pakiet premium przyspieszonego postępu'' i~,,Wszystko Razem Teraz(TM)'' i~wiele innych dziwacznych nazw, które nie oszukują nikogo. 


Ale Bogowie nie są zadowoleni z~tego, że zarobią pieniądze na graczach, którzy są zbyt leniwi, by awansować w~grze. Mają dużo, dużo dziwniejszą grę. Sprzedają złoto ludziom, \textit{którzy nawet nie grają w~tę grę}. Zgadza się: jeśli jesteś wielkim finansistą i~szukasz miejsca, w~którym można schować milion dolców, gdzie może zrobić coś dobrego, możesz kupić wirtualne złoto warte milion dolarów, trzymać się go w~miarę rozwoju gry i~to staje się coraz bardziej zabawne, gdy wartość złota rośnie i~rośnie, a następnie możesz je odsprzedać za prawdziwe pieniądze za pośrednictwem oficjalnych banków w~grze, zgarniając gruby zysk za kłopot. 


Więc kiedy pilotujesz swojego mecha, wymachujesz toporem lub dowodzisz swoją flotą kosmiczną, grupa dziwnych, starych dorosłych w~garniturach w~eleganckich biurach na całym świecie z~niecierpliwością obserwuje twoją grę, próbując domyślić się, czy wartość w~grze złoto będzie rosło lub spadało. Kiedy gra zaczyna być do niczego, wszyscy spieszą się, aby sprzedać swoje zasoby, pozbywając się złota tak szybko, jak to możliwe, zanim jego wartość zostanie zniszczona przez znudzonych graczy przechodzących na konkurencyjną usługę. A kiedy gra staje się \textit{fajniejsza}, cóż, to jeszcze większe szaleństwo, ponieważ wojny licytacyjne nabierają tempa, każdy bankier na świecie próbuje kupić to samo złoto w~tym samym świecie. 


Czy można się dziwić, że osiem z~20 największych gospodarek świata znajduje się w~krajach wirtualnych? I czy można się dziwić, że granie stało się tak poważnym biznesem? 


\bigskip
\threeast

Ta scena jest poświęcona Secret Headquarters w~Los Angeles, moim ulubionym sklepie z~komiksami na świecie. Jest mały i~selektywny w~stosunku do tego, co przechowuje, i~za każdym razem, gdy wchodzę, wychodzę z~trzema lub czterema kolekcjami, o których nigdy nie słyszałem pod pachą. To tak, jakby właściciele, Dave i~David, mieli niesamowitą zdolność przewidywania dokładnie tego, czego szukam, i~wykładają to na kilka sekund, zanim wejdę do sklepu. Odkryłem około trzech czwartych moich ulubionych komiksów, wędrując do SHQ, chwytając coś interesującego, zatapiając się w~jednym z~wygodnych krzeseł i~przenosząc się do innego świata. Kiedy ukazała się moja druga kolekcja opowiadań, OVERCLOCKED, współpracowali z~lokalnym ilustratorem Martinem Cenredą nad stworzeniem darmowego minikomiksu opartego na Printcrime, pierwszym opowiadaniu w~książce. Wyjechałem z~LA około rok temu i~z wszystkich rzeczy, za którymi tęsknię, Secret Headquarters jest na szczycie listy. 


\href{https://www.thesecretheadquarters.com/}{Secret Headquarters: 3817 W. Sunset Boulevard, Los Angeles, CA 90026 +1 323 666 2228} 

\bigskip

\threeast

Matthew stał przed drzwiami kafejki internetowej, oddychając głęboko. Podczas spaceru udało mu się trochę uspokoić, ale gdy podszedł bliżej, coraz bardziej był przekonany, że chłopcy Bossa Winga będą tam na niego czekać, a wszyscy jego przyjaciele będą zwinięci w~kłębki na ziemi, pobici do nieprzytomności. Wyprowadził ze sobą czterech najlepszych graczy z~fabryki Boss Wing i~wiedział, że Boss Wing \textit{wcale }nie był z~tego zadowolony. 


Hiperwentylował, w~głowie mu się kręciło. Nadal bolało. Czuł się tak, jakby miał w~bieliźnie czerwone słońce bólu wielkości piłki do piłki nożnej, a jedną z~rzeczy, których najbardziej i~najmniej chciał zrobić, to znalezienie prywatnego miejsca, aby tam zajrzeć. W kawiarni była łazienka, więc już czas wejść do środka. 


Wspinał się boleśnie po czterech kondygnacjach schodów, przechodząc pod gigantycznymi malowidłami ściennymi z~uniwersów gier, unikając plastikowych roślin na każdym podeście, które śmierdziały sikami graczy, którzy nie chcieli czekać na łazienkę. Od trzeciego piętra w~górę otulała go znajoma chmura zapachu ciała, dymu papierosowego i~przekleństw, które mówiły mu, że jest w~drodze do swojego prawdziwego domu. 


W drzwiach zatrzymał się i~rozejrzał dookoła, szukając jakichkolwiek śladów zbirów Boss Wing, ale było jak zwykle: rzędy stołów z~komputerami, kilka par dzielących się maszynami, ale głównie bawili się chłopcy, chudzi, z~koszulami podwiniętymi na brzuchy, żeby złapać wiatr, który mógłby się przedostać przez pokój. Nie było wiatru, tylko wiry w~dymie spowodowane rykiem tych wszystkich wentylatorów PC, które jęczały, gdy wysysały zadymione powietrze nad przegrzanymi płytami głównymi i~potwornymi kartami graficznymi. 


Przekradł się obok stanowiska rejestracji, obsadzonego dziś wieczorem przez nowego dzieciaka, kogoś innego właśnie przybyłego z~prowincji, by znaleźć swoją fortunę tutaj, w~starym, złym Shenzhen. Matthew chciał złapać dzieciaka i~zanieść go do granic miasta, wyjaśniając, że nie ma już tu fortuny, wszystko należało do ludzi takich jak Boss Wing. \textit{Idź do domu}, pomyślał do chłopca, \textit{Idź do domu, to miejsce jest skończone}.  


Jego chłopcy grali przy swoim zwykłym stole. Zbudowali piramidę z~naprzemiennych warstw paczek papierosów Double Happiness i~pustych kubków po kawie. Gdy się do nich zbliżył, spojrzeli w~górę, uśmiechając się i~śmiejąc z~jakiegoś żartu. Wtedy zobaczyli wyraz jego twarzy i~zamilkli. 


Usiadł na wolnym krześle i~wpatrywał się w~ich ekrany. Oczywiście grali. Zawsze się bawili. Kiedy pracowali w~fabryce Boss Wing, mieli 18-godzinną zmianę, a potem odpoczywali, grając jeszcze trochę, prowadząc własne postacie przez lochy, które farmili przez cały dzień. To dlatego Boss Wing miał tak łatwy czas na rekrutację do swojej fabryki: podejście było uwodzicielskie. ,,Zarabiaj na graniu! '' 


Ale to nie to samo, kiedy pracowałeś dla kogoś innego. 


Próbował znaleźć słowa na początek i~nie mógł. 


-- Matthew? -- To był Yo, najstarszy z~nich. Yo właściwie miał rodzinę, żonę i~córkę. Opuścił fabrykę Bossa Winga i~poszedł za Matthew. 


Matthew wpatrywał się w~swoje ręce, wziął głęboki oddech i~podjął decyzję.

 -- Przepraszam, właśnie pokłóciłem się w~drodze tutaj. Mam jednak dobre wieści: znalazłem sposób, aby nas bardzo wzbogacić w~bardzo krótkim czasie. -- I, z~pamięci, Mistrz Fong opisał sposób, w~jaki znalazł się w~bogatym lochu Wojowników Svartalfaheim. Zarekwirował komputer i~pokazał im, jak skrócić sekundy przejścia, gdzie zatrzymać się, złapać i~podnieść. A potem każdy wziął maszynę i~poszedł do pracy. 


Z czasem ból w~jego spodniach zniknął. Ktoś dał mu papierosa, potem drugiego. Ktoś przyniósł mu pierogi. Mistrz Fong zjadł je bez czucia smaku. On i~jego zespół byli w~pracy i~zarabiali pieniądze, a wkrótce będą mieli fortunę, dzięki której Boss Wing będzie wyglądał jak drobiazg. 


Podczas zmiany zadzwonił jego telefon. To była jego matka. Chciała mu życzyć wszystkiego najlepszego. Właśnie skończył 17 lat. 


\bigskip
\threeast


Ta scena jest poświęcona Powell's Books, legendarnemu ,,City of Books'' w~Portland w~stanie Oregon. Powell's to największa księgarnia na świecie, niekończący się, wielopiętrowy wszechświat papierowych zapachów i~wysokich półek. Wystawiają nowe i~używane książki na tych samych półkach -- coś, co zawsze kochałem -- i~za każdym razem, gdy wpadałem, mieli istną górę moich książek i~byli niesamowicie uprzejmi, prosząc mnie do podpisania zapasów magazynowych. Sprzedawcy są przyjaźni, asortyment jest fantastyczny, a na lotnisku w~Portland jest nawet Powell, co czyni ją najlepszą księgarnią lotniskową na świecie za moje pieniądze! 


\href{https://www.powells.com/SearchResults?keyword=Cory+Doctorow}{\textit{Powell's Books: 1005 W Burnside, Portland, OR 97209 USA +1 800 878 7323}} 

\bigskip
\threeast

Zawieszenie Wei-Donga od gry trwało całe 20 minut. Tyle czasu zajęło mu zasymulowanie migreny, zdobycie przepustki, wkradnięcie się do centrum zasobów, pokonanie filtra sieciowego i~zalogowanie się. W Chinach robiło się bardzo późno, ale to było w~porządku, chłopcy pracowali do późna i~cieszyli się, że jest z~nimi. 


Prawdziwe nazwisko Wei-Dong nie brzmiało oczywiście Wei-Dong. Jego prawdziwe nazwisko brzmiało Leonard Goldberg. Wybrał Wei-Dong po zbadaniu znaczenia chińskich imion i~wymyśleniu Siły Wschodu, czego brzmienie mu się spodobało. Ten system wybierania imion działał dobrze dla chińskich dzieci, które znał -- kiedy ich rodzice migrowali do Stanów, wybierali po prostu jakieś angielskie imię i~to wszystko. Dlaczego nie? Dlaczego lepiej było wybrać imię, ponieważ miał je twój dziadek, niż dlatego, że podobał ci się jego dźwięk? 


Próbował to wyjaśnić rodzicom, ale nie zrobiło to na nich większego wrażenia. Fajnie było, że interesował się innymi kulturami, ale to nie znaczyło, że mógł zrezygnować z~bar-micwy lub żeby nazywali go Wei-Dong. I nie oznaczało to, że aprobowali, by przez całą noc grał ze swoimi kumplami w~Chinach, zarabiając pieniądze. 


Wei-Dong wiedział, że to wszystko może być postrzegane jako bardzo słabe, dzieciak wyrzutek tak zdesperowany, by się zaprzyjaźnić, że całkowicie porzucił szkołę średnią i~zamiast tego podpinał się do kogoś na innej półkuli za darmową pracę. Ale tak nie było. Wei-Dong miał mnóstwo przyjaciół w~liceum Ronalda Reagana. Wiele dzieci myślało, że Chiny są najciekawszym miejscem na świecie, uwielbiały filmy, jedzenie, komiksy i~gry. W szkole było też dużo chińskich dzieciaków i~choć kilkoro najwyraźniej myślało, że jest dziwny, o wiele więcej go rozumiało. W końcu większość z~nich była zafascynowana w~Indiach, tak jak on w~Chinach, więc mieli to coś wspólnego. 


A co z~tego, że opuszczał zajęcia? To były nauki społeczne, naranykrzysztofa! Mieli studiować Chiny, ale Wei-Dong wiedział na ten temat dziesięć razy więcej niż nauczyciel. Kiedy szeptał po mandaryńsku do swojej słuchawki w~uchu, pomyślał, że to jest jak niezależny projekt badawczy. Jego nauczyciele powinni dać mu dodatkowe oceny. 


-- Co teraz? -- powiedział. -- Jaka jest misja?  


-- Myśleliśmy o uruchomieniu Ogrodu Morsa jeszcze kilka razy, teraz gdy mamy to świeżo w~głowach. Może moglibyśmy podnieść kolejne ostrze vorpal. 

Tak właśnie robili faceci, gdy nie było żadnych płacących gweilo, szli na najazdy po prestiżowe przedmioty. Nie była to najbardziej ekscytująca rzecz ze wszystkich, ale nigdy nie wiedziałeś, co może się wydarzyć. 


-- Wchodzę -- powiedział. Po tym miał wolną lekcję, potem lunch, więc technicznie mógł grać przez trzy godziny. I tak do tego czasu wszyscy byliby gotowi do wylogowania się i~pójścia spać. 


-- Jesteś dobrym gweilo, wiesz? -- Wei-Dong wiedział, że Ping żartuje. Nie obchodziło go, że chłopaki nazywają go gweilo. To nie był rasistowski termin, nie w~stylu ,,chinol'' czy ,,skośnooki''. Tylko określenie uczucia. I jako przezwisko, ,,zagraniczny duch'', było całkiem fajne. 


Więc udali się do Ogrodu, przeszli i~poradzili sobie całkiem nieźle, wyszli, włożyli pieniądze do banku gildii i~wrócili po więcej. Potem zrobili to ponownie. Gdzieś tam zadzwonił dzwonek. Gdzieś tam przyszli jego przyjaciele i~porozmawiali z~nim, a on wyciszył słuchawkę i~odpowiedział im kilka rzeczy, ale tak naprawdę nie wiedział, co powiedział. Coś. 


Potem, przy trzecim biegu, stało się coś złego. Dotarli już prawie do brzegu i~wypędzili swoje wierzchowce. Wei-Dong przygotowywał Kieszonkę Powietrzną Królowej, sięgając do ogromnego zapasu muszli ostryg, które zgromadził podczas poprzednich rajdów. 


I nagle wyszli, tuzin rycerzy na ogromnych, przerażających czarnych rumakach, jednogłośnie wynurzyli się z~wody, rozdzierając powietrze gniewnym chórem wierzchowców i~okrzykami bojowymi. Woda wytrysnęła wokół nich i~spadła na Wei-Dong i~jego gildie. 


Krzyknął coś do swojej słuchawki, ostrzeżenie, a wokół niego w~centrum informacyjnym dzieci podniosły wzrok znad rozmów, by na niego spojrzeć. Stał się derwiszem, stukając w~klawiaturę i~wściekle poruszając myszką, z~oczami utkwionymi w~ekranie. 


Czarni jeźdźcy poruszali się niesamowicie synchronicznie. Albo to potwory -- takie potwory, jakich Wei-Dong nigdy nie spotkał -- albo najlepiej wyćwiczona, współpracująca grupa rabusiów, jaką kiedykolwiek widział. Wyciągnął teraz swoje vorpalne ostrze, jego gildia również walczyła. W jego słuchawce przeklinali w~chińskich dialektach sześciu różnych prowincji. W innych okolicznościach Wei-Dong robiłby notatki, ale teraz walczył o życie. 


Lu odważnie stanął między jeźdźcami a drużyną, ogromny czołg, stał mocno z~maczugą i~pałaszem, atakując wszystkich dwunastu rycerzy bez względu na własne bezpieczeństwo. Wei-Dong rzucił na niego uzdrawiające zaklęcia, gdy próbował wycisnąć na jeźdźcach swój własny ślad za pomocą wibracyjnego ostrza, trzy razy dłuższego od niego. 


Migające ostrze mogło wyrządzić niewiarygodne obrażenia, ale nie było łatwe w~użyciu. Dwukrotnie Wei-Dong przypadkowo pokroił członków swojej własnej partii, choć nie tak źle -- dzięki Bogu, inaczej nigdy nie zapomną -- ale nie mógł dostać się do czarnych rycerzy, którzy byli dla niego za szybcy. 


Potem Lu upadł, klękając na jedno kolano, przebity przez gardło piką dzierżoną przez jeźdźca, którego oczy rumaka miały barwę lodowatego błękitu mgły Gąsienicy. Jeździec uniósł Lu w~powietrze, jego stopy bezwładnie wierzgały, a inny rycerz ściął mu głowę pogardliwym zamachem miecza. Lu spadł w~dwóch kawałkach na piaszczysty plażowy piasek, a w~słuchawce przeklął ich, używając wyrażenia, które Wei-Dong skrupulatnie przetłumaczył na ,,Pieprzyć osiem pokoleń swoich przodków''. 


Gdy Lu odpadł, reszta była praktycznie bezradna. Walczyli dzielnie, koordynując swoje ataki, rozpalając ogień ze swoich magicznych przedmiotów i~najlepszych zaklęć, ale czarni rycerze byli nie do pokonania. Zanim zginął, Wei-Dong zdołał trafić jednego z~wibracyjnym ostrzem i~miał chwilową satysfakcję obserwowania rycerza zataczającego się i~ściskającego pierś, ale wtedy wojownik zbliżył się do niego, wyciągając parę krótkich mieczy, którymi obrócił, jak mag wykonujący sztuczki z~nożem. Nie było mowy o sparowaniu tego, a kilka sekund później Wei-Dong leżał na piasku, patrząc, jak kolczasty but rycerza spada mu na twarz, słysząc chrzęst jego kości policzkowych i~nos pękający pod ciężarem. Potem odrodził się w~odległym Jeziorze Łez, nagi i~nieuzbrojony, i~musiał pobiec do swojego ciała, zanim dranie zdobędą jego wibracyjne ostrze. 


Usłyszał, jak jego ludzie umierają w~słuchawce, jeden po drugim, gdy biegł upiornie i~eterycznie przez wzgórza i~doliny Krainy Czarów. Dotarł do swojego trupa w~samą porę, by zobaczyć, jak rycerze plądrują ciało i~ciała jego kolegów z~drużyny. Znowu wstał, bezradny i~nieuzbrojony, ciało swojej postaci, bezbronny. 


Jeden z~rycerzy wysłał mu prośbę o czat. Kliknął, uciszając odgłosy w~tle z~Shenzhen. 


-- Wy, farmerzy, nie jesteście tu już mile widziani, towarzyszu -- powiedział głos. Miał akcent, którego nie rozpoznawał. Może rosyjski? A mówca był tylko dzieckiem! -- Teraz patrolujemy. Wrócicie znowu, będziemy na was polować i~zabijać znowu i~znowu i~znowu. Rozumiesz mnie, Chino? -- Nie tylko dziecko: \textit{dziewczynka }, mała dziewczynka, która grozi mu z~jakiegoś miejsca na świecie. 


-- Kto ci dał dowództwo, \textit{panienko}? -- powiedział. -- A w~ogóle, dlaczego myślisz, że jestem Chińczykiem? 


Rozległ się paskudny śmiech. 

-- Panienko, co? Ja dowodzę, bo właśnie skopałam ci tyłek i~mogę go znowu skopać tyle razy, ile potrzebuję. I nie obchodzi mnie, czy jesteś w~Chinach, Wietnamie, Indonezji, to nie ma znaczenia. Zabijemy ciebie i~wszystkich farmerów w~Krainie Czarów. Ta gra nie nadaje się już do farmienia. Skończyłam z~tobą rozmawiać. -- A czarny rycerz ściął mu głowę z~pogardliwą łatwością. 


Wrócił do kanału gildii, gotowy opowiedzieć im o tym, co się właśnie wydarzyło, jego umysł wirował, i~wtedy podniósł wzrok i~spojrzał w~twarz swojego ojca, stojącego nad nim, z~wyrazem twarzy, który mógłby zepsuć mleko. 


-- Wstawaj, Leonardzie -- powiedział. -- I chodź ze mną. 


Nie był sam. Byli tam pan Adams, wicedyrektor i~szkolny policjant, funkcjonariusz Turner, i~doradczyni zawodowy, pani Ramirez. Mieli kamienne twarze Mount Rushmore, twarze bez cienia litości. Jego ojciec wyciągnął rękę i~delikatnie, ostrożnie wyjął słuchawkę z~ucha. Następnie, z~dokładnie taką samą ostrożnością, upuścił słuchawkę na wypolerowaną betonową podłogę centrum zasobów i~przycisnął do niej piętę, \textit{trzask }głośny w~idealnie cichym pokoju. 


Leonard wstał. Pokój był pełen dzieciaków udających, że na niego nie patrzą. Wszyscy na niego patrzyli. Wyszedł za ojcem na korytarz i~gdy drzwi się zamknęły, bez wątpienia usłyszał setki chichotów. 


Szli koło niego w~drodze do gabinetu wicedyrektora, nie wypuszczając. Nie, żeby uciekał, nie miał \textit{dokąd }uciec, ale nadal czuł się klaustrofobicznie. To nie było dobre. To było bardzo, bardzo złe. 


Oto jak było źle: 

-- Wyślesz mnie do \textit{szkoły wojskowej}? 


-- Nie szkoła wojskowa -- powiedziała pani Ramirez. Powiedziała to tym denerwującym, protekcjonalnym tonem doradcy zawodowego. -- Akademia Martindale nie ma komponentu wojskowego ani wojennego. To tylko bardzo zorganizowane, nadzorowane środowisko. Mają fantastyczne osiągnięcia w~pomaganiu uczniom takim jak ty skoncentrować się na ocenach i~wydostać się z~kłopotów akademickich. Mają piękny kampus w~pięknym miejscu, a chłopcy Martindale wypełniają wiele ważnych\ldots  


I tak dalej. Połknęła broszurę sprzedażową jak burrito i~teraz zaczęło się jej odbijać. Wyłączył się i~spojrzał na swojego ojca. Benny Rosenbaum nie był osobą, którą łatwo czytać. Ludzie, którzy pracowali dla niego w~Rosenbaum Shipping and Logistics, nazywali go Murem, bo nic nie mogło przez niego przejść, pod nim, przez niego ani nad nim. Nie, żeby był twardym przypadkiem, ale nie dał się zwieść emocjonalnym argumentom: jeśli próbowałeś podejść do niego z~czymś mniej niż w~pełni skomputeryzowaną logiką, równie dobrze możesz o tym zapomnieć. 


Ale istniały drobne wskazówki, pewne sposoby, by dowiedzieć się, jaka była pogoda w~starym Benny. To, co robił ze swoim paskiem zegarka, pracując przy zaczepie, to była jedna z~nich. Podobnie jak mały skok w~zawiasie szczęki, jakby żuł niewidzialny zwitek gumy. Połącz je z~faktem, że nie było go w~pracy w~środku dnia, kiedy powinien się upewniać, że gigantyczne stalowe pojemniki brzęczą po całym świecie, cóż, dla Leonarda oznaczało to, że lawa była całkiem blisko powierzchni Mount Benny dziś po południu. 


Zwrócił się do swojego taty. 

-- Czy nie powinniśmy rozmawiać o tym jako rodzina, tato? Dlaczego robimy to tutaj? 


Benny spojrzał na niego, bawiąc się paskiem od zegarka, skinął głową doradcy i~wykonał mały gest ,,dalej'', który niczego nie zdradzał. 


-- Leonard -- powiedziała. -- Leonard, musisz zrozumieć, jak poważne to jest. Jesteś o jeden semestr od oblania dwóch przedmiotów: historii i~biologii. Przeszedłeś od ocen piątkowych w~matematyce, angielskim i~naukach społecznych do trójek z~minusem. W tym tempie stracisz semestr przed Świętem Dziękczynienia. Ujmijmy to w~ten sposób: przeszedłeś z~dziewięćdziesiątego percentyla uczniów drugiego roku liceum Ronalda Reagana do \textit{dwunastego}. To sygnał, Leonardzie, od ciebie do nas i~to sygnał S-O-S, S-O-S. 


-- Myśleliśmy, że bierzesz narkotyki -- powiedział jego ojciec, absolutnie spokojny. -- Właściwie przetestowaliśmy mieszek włosowy z~twojej poduszki. Miałem faceta, który za tobą chodził. Prawie, o ile wiem, palisz trochę zioła ze swoimi przyjaciółmi, ale właściwie już ich nie widujesz, prawda? 


-- Testowałeś moje włosy? 


Jego ojciec wykonał ten gest ,,dalej'', jego stary ulubiony. 

-- I śledziłem Ciebie. Oczywiście, że tak. Odpowiadamy za ciebie. Jesteśmy odpowiedzialni za ciebie. Nie jesteś naszą własnością, ale jeśli spieprzysz tak źle, że spędzisz resztę życia jako włóczęga, to będzie nasza wina i~my będziemy musieli cię ratować. Rozumiesz, Leonardzie? Jesteśmy za ciebie odpowiedzialni i~zrobimy wszystko, co w~naszej mocy, aby zapewnić, że nie schrzanisz swojego życia.  


Leonard stłumił ripostę. Uczucie zapadania się, które zaczęło się od zmiażdżenia słuchawki, opadło tak nisko, jak się da. Teraz pociły mu się dłonie, serce waliło mu w~piersi i~nie miał pojęcia, co wyjdzie z~jego ust następnym razem, gdy będą rozmawiać. 


-- Kiedy byłem w~twoim wieku, nazywaliśmy to interwencją -- powiedział wicedyrektor. Wciąż wyglądał jak agent nieruchomości, którym był przed przejściem do nauczania, kiedy ostatni raz załamał się rynek. Był uprzejmy, nieszkodliwy, miał szerokie oczy i~był godny zaufania. Na korytarzach nazywali go Dziecinka Adams. Ale Leonard znał się na sprzedawcach, wiedział, że bez względu na to, jak bardzo wyglądali przyjaźnie, zawsze szukali słabości do wykorzystania. -- I zrobilibyśmy to dla narkomanów. Ale nie sądzę, że jesteś uzależniony od narkotyków. Myślę, że jesteś uzależniony od gier. 


-- Och, \textit{daj }spokój -- powiedział Leonard. -- Nie ma czegoś takiego. Mogę ci pokazać artykuły naukowe. Uzależnienie od gier? Po prostu wymyślili to, żeby sprzedawać gazety. Tato, daj spokój, naprawdę w~to nie wierzysz, prawda? 


Jego tata wyraźnie odmówił spojrzenia mu w~oczy, kierując swoją uwagę na wicedyrektora. 


-- Leonard, wiemy, że jesteś bardzo mądrym młodym mężczyzną, ale nikt nie jest tak mądry, by nigdy nie potrzebować pomocy. Nie chcę spierać się z~tobą o definicje uzależnień \ldots  


-- Bo przegrasz. -- Leonard wypluł to, zaskakując się gwałtownością. Stary Dziecinka uśmiechnął się swoim uprzejmym uśmiechem sprzedawcy: O tak, proszę panie, z~pewnością jest pan bardzo sprytny . A teraz, czy mogę pokazać Panu coś w~pozornym dwupoziomowym stylu Tudorów z~garażem na trzy samochody i~basenem naziemnym?


-- Jesteś bardzo mądrym młodym mężczyzną, Leonardzie. Nie ma znaczenia, czy jesteś medycznie uzależniony, psychicznie zależny, czy po prostu -- machnął ręką, szukając właściwych słów -- czy po prostu spędzasz za dużo czasu na granie w~gry, a za mało czasu w~prawdziwym świecie. Nic z~tego nie ma znaczenia. Liczy się to, że masz kłopoty. A my Ci w~tym pomożemy. Ponieważ zależy nam na Tobie i~chcemy zobaczyć, jak ci się udaje. 


To nagle zapadło. Leonard wiedział, jak to się dzieje. Gdzieś w~tej chwili funkcjonariusz Turner czyścił jego szafkę i~ładował jej zawartość do dwóch papierowych worków po zakupach Trader Joe. Gdzieś jakiś sekretarz zdejmował jego nazwisko z~listy na każdej z~jego lekcji. W tej chwili jego matka pakowała walizkę w~domu, napełniając ją trzema lub czterema ubraniami na zmianę, świeżą szczoteczką do zębów -- i~niczym więcej. Kiedy opuści ten pokój, zniknie z~Orange County tak dokładnie, jakby porwali go z~ulicy seryjni mordercy. 


Tylko że to nie jego okaleczone ciało pojawi się na powierzchni za kilka miesięcy, rozłożone i~makabryczne, lekcja poglądowa dla wszystkich dzieciaków z~John Wayne High, by mieć się na baczności przed niebezpiecznymi nieznajomymi. Wypłynie na powierzchnię jego okaleczona \textit{osobowość}, ospały człowiek, który został wepchnięty w~formę szczęśliwego, dobrze przystosowanego, obywatela, co poprowadzi go przez dorosłość jako dobrego, bezproblemowego pracownika-pszczoły w~ulu. 


-- Tato, daj \textit{spokój}. Nie możesz mi tego po prostu zrobić! Jestem twoim synem! Zasługuję na szansę, żeby podnieść swoje oceny, prawda? Zanim wyślesz mnie do jakiegoś ośrodka prania mózgu? 


-- Miałeś okazję podnieść swoje stopnie, Leonardzie -- powiedziała pani Ramirez, a wicedyrektor energicznie skinął głową. -- Miałeś cały semestr. Jeśli planujesz ukończyć szkołę i~iść na studia, to jest ten moment, aby zrobić coś drastycznego, aby tak się stało. 


-- Czas już iść -- powiedział ojciec, ostentacyjnie spoglądając na zegarek. 

Szczerze, kto jeszcze nosił zegarek? Leonard wiedział, że ma telefon, tak jak wszyscy normalni ludzie. Staromodny nakręcany zegarek był w~dzisiejszych czasach równie przydatny jak trąbka na uszy czy kolczuga. Miał ich całą skrzynkę -- dziesiątki. Jego ojciec mógł mieć wszystkie śmieszne upodobania i~hobby, jakie chciał, wydać na nie małą fortunę i~nikt nie chciał wysłać \textit{go }do wariatkowa. 


To było tak cholernie \textit{niesprawiedliwe}. Chciał to wykrzyczeć, gdy prowadzili go do nieskazitelnego, małego Huawei Dartera jego ojca. Kupował nowy co roku, otrzymując solidną zniżkę prosto z~fabryki, która ładowała jego osobisty samochód do własnego kontenera i~dźwigała go na jeden z~dużych statków taty w~porcie w~Kantonie. Samochód pachniał czarnymi lukrecjowymi słodyczami, które ssał tata, i~gigantycznym, stalowym termosem z~kawą, który tata wkładał każdego ranka do uchwytu na kubek, napełniając go przez cały dzień w~kilku barach, w~których mówili mu po imieniu i~pozwalali na otwarcie rachunku. 


A za oknami, przez subtelny szary odcień, mijały ulice Anaheim, rzędy identycznych domów rozgałęziały się na ogromnej, podzielonej ośmiopasmowej arterii. Znał te ulice przez całe życie, chodził po nich, spotykał żebraków, którzy pracowali w~turystyce, pracowników Disneya o obolałych stopach, którzy spóźnili się na prom, pokonując kilometr do parkingu dla członków obsady, spacerujących emerytowanych dziwaków, ich psy, inne larwy-ludzi hrabstwa Orange, które były jeszcze zbyt młode, biedne lub miały pecha, by mieć samochód. 


Niebo było tak czyste, jak w~OC, bez chmur, pocztówkowe uśmiechnięte słońce prawie w~samo południe, idealne do zdjęć turystycznych. Leonard zobaczył to wszystko po raz pierwszy, naprawdę to \textit{zobaczył}, ponieważ wiedział, że widział to po raz ostatni. 


-- Nie jest tak źle -- powiedział jego tata. -- Przestań zachowywać się, jakbyś szedł do więzienia. To wytworna szkoła z~internatem, na miłość boską. I nie jest to jedna z~tych szkół, w~których biją cię w~łazience czy coś takiego. Tam są praktycznie hippisami. Matka i~ja nie wysyłamy cię do gułagu, dzieciaku. 


-- Nieważne, co powiesz, tato. Po prostu zapomnij. Oto fakty: porwałeś mnie z~mojej szkoły i~wysyłasz do jakiegoś miejsca, gdzie mają mnie ,,naprawić''. Nie pozwoliłeś mi się wypowiedzieć. Nie skonsultowałeś tego ze mną. Możesz powiedzieć, jak bardzo mnie kochasz, jak bardzo to dla mojego dobra, mówić i~mówić i~mówić, ale to nie zmieni tych faktów. Mam szesnaście lat, tato. Mam tyle lat, ile Zaidy Shmuel, kiedy poślubił Bubbie i~przyjechał do Ameryki, wiesz o tym? 


-- To było podczas wojny \ldots  


-- Kogo to obchodzi? Był twoim dziadkiem i~był na tyle dorosły, że mógł założyć rodzinę. Możesz się założyć, że nie stałby w~miejscu, gdyby został porwany \ldots  -- Ojciec parsknął. -- \textit{Porwany}, ponieważ jego hobby nie było pomysłem rodziców na dobrą zabawę. Boże! Co się z~tobą dzieje? Wiedziałem, że jesteś trochę gnojkiem, ale \ldots  


Jego ojciec spokojnie skierował samochód do krawężnika i~zjechał na pobocze, zmieniając płynnie trzy pasy, ze sprawdzeniem w~lusterkach przed każdym pasem, przeciskając się przez ruch turystyczny i~pickupy ogrodników bez jednego klaksonu. Jedną ręką nacisnął hamulec bezpieczeństwa, a drugą odpiął pas bezpieczeństwa, obracając się na siedzeniu, by zbliżyć twarz do twarzy Leonarda. 


-- Jesteś na cienkim, cholernym lodzie, dzieciaku. Możesz zrobić ze mnie złoczyńcę, jeśli chcesz, jeśli musisz, ale wiesz, gdzieś w~tym twoim przyćmionym hormonami nastoletnim mózgu, że to była \textit{twoja }sprawka. Ile razy, Leonard? Ile razy rozmawialiśmy z~tobą o równowadze, o utrzymywaniu lepszych ocen, poświęcaniu trochę czasu poza graniem? Ile miałeś przed tym szans? 


Leonard roześmiał się gorąco. W jego oczach pojawiły się łzy wściekłości, próbujące się wydostać. Ciężko przełknął ślinę. 

-- Porwany -- powiedział. -- Porwany i~wywieziony, bo uważasz, że nie mam wystarczająco dobrych ocen z~matematyki i~angielskiego. Jakby to miało znaczenie, kiedy ostatnio rozwiązałeś równanie kwadratowe, tato? Kogo \textit{obchodzi}, czy dostanę się na dobry uniwersytet? Jaki dyplom pomoże mi przetrwać następne dwadzieścia lat? Chwila, tato, z~czego zdobyłeś dyplom? Och, zgadza się, \textit{języki starożytne}. Założę się, że \textit{to} się często przydaje, gdy wysyłasz gigantyczne kontenery z~plastikowymi śmieciami z~Chin, co? 


Jego ojciec potrząsnął głową. Za nimi samochody hamowały i~trąbiły na siebie, manewrując wokół zatrzymanego Huawei.

 -- Tu nie chodzi o mnie, synu. Tu chodzi o ciebie \ldots  o tracenie swojego życia na jakąś głupią grę. Przynajmniej mówienie po łacinie pomaga mi zrozumieć hiszpański. Co będziesz miał z~tych godzin i~lat zabijania smoków? 


Leonard się uniósł. Znał gdzieś odpowiedź na to pytanie. Gry zawładnęły światem. Tam można było zarobić pieniądze. Uczył się pracy w~zespołach. To wszystko i~wiele więcej, to były powody do grania i~żaden z~nich nie był tak ważny jak najważniejszy powód: po prostu \textit{czułem się dobrze}, podróżując w~świecie\ldots  


Za nimi rozległ się szczególnie głośny pisk hamulców, który się zbliżał, coraz głośniejszy i~głośniejszy, a także ryk klaksonu, który nie ustawał, stał się głośniejszy, niż można było sobie wyobrazić. Odwrócił głowę, by spojrzeć przez ramię i\ldots  


Bum


Wydawało się, że samochód wyskoczył w~powietrze, unosząc się najpierw na przednich oponach, a potem przednie koła obróciły się i~samochód pomknął dziesięć metrów do przodu w~ciągu sekundy. Rozległ się dźwięk kruszącego się metalu, przekleństwo ojca, a potem brzęk dzwonów świątynnych, gdy głowa odbiła mu się od deski rozdzielczej. Świat pociemniał. 


\bigskip
\threeast


Ta scena jest poświęcona Books of Wonder w~Nowym Jorku, najstarszej i~największej dziecięcej księgarni na Manhattanie. Znajdują się zaledwie kilka przecznic od biur Tor Books w~Flatiron Building i~za każdym razem, gdy wpadam na spotkanie z~ludźmi Tora, zawsze wymykam się do Books of Wonder, aby przejrzeć ich zapasy nowych, używanych i~rzadkich książek dla dzieci. Jestem nałogowym kolekcjonerem rzadkich wydań Alicji w~Krainie Czarów, a Books of Wonder zawsze podnieca mnie piękną, limitowaną edycją Alicji. Mają mnóstwo wydarzeń dla dzieci i~jedną z~najbardziej zachęcających atmosfer, jakich kiedykolwiek doświadczyłem w~księgarni. 


\href{https://booksofwonder.com/}{\textit{Books of Wonder}}: 18 West 18th St, New York, NY 10011 USA +1 212 989 3270 

\bigskip
\threeast

Mala była na świecie z~małym oddziałem szturmowym, składającym się tylko z~kilku członków jej armii. Było późno -- po północy -- i~pani Dibyendu przekazała kawiarnię swojemu idiotycznemu siostrzeńcowi, żeby zarządzał sprawami. Ostatnio kawiarnia była otwarta, kiedy Mala i~jej armia chcieli z~niej skorzystać, w~dzień i~w nocy, i~zawsze byli żołnierze, którzy walczyliby o zaszczyt eskortowania generała Robotwallaha później do domu. Mamaji \ldots Mamaji miała nowe, ładne mieszkanie, z~dwoma kompletnymi pokojami, a jeden z~nich był tylko dla Mamaji, aby spała bez sapania i~gderania dwójki jej dzieci. W Dharavi były miejsca, w~których dziesięcioro lub piętnaścioro mogło dzielić ten pokój, śpiąc w~płaszczach, albo na sobie nawzajem. Mamaji miała materac, który przyniósł jej silny młody człowiek z~Chor Bazaar, niesiony ze sobą na dachu pociągu Marine Line wśród upału godzin szczytu i~nacisku ciał. 


Mamaji nie narzekała, gdy Mala grała po północy. 


-- Więcej, właśnie tam -- powiedział Sushant. Był od niej o dwa lata starszy, najwyższy ze wszystkich, z~krótkimi włosami i~szalonym uśmiechem, który przypominał jej pysk psa, którego żołądek był pełen ecstasy. 


I oto oni, trzy mechy w~trójkącie, metodycznie bijące zombie po głowach, rozpryskując ich zgniłe mózgi i~zrzucając je na coraz większe stosy. W końcu gra wysłała ghule, aby odciągnęły ciała, ale na razie ułożyły się one w~pasie wokół mechów pierwszego poziomu. 


-- Mam je -- powiedziała Yasmin, jej celownik łapiący cel. 

To był dla nich nowy rodzaj misji, wyniszczenie tych małych trójek mechów, które bez końca walczyły z~zombie. Pan Banerjee zlecił im to zadanie po tym, jak ich armia wytropiła i~wybiła bardziej agresywnych wojowników. Według pana Banerjee każdy z~tych był rozgrywany przez jedną osobę, kogoś, kto otrzymywał wynagrodzenie za podnoszenie poziomu podstawowego mecha do poziomu czwartego lub piątego, który potem miał być sprzedany na aukcji bogatym graczom. Zawsze trójkami, zawsze miażdżą zombie, zawsze w~tej części świata, jak robactwo. 


-- Ognia -- powiedziała, a broń pulsacyjna wystrzeliła koncentryczne pierścienie siły w~trójkę. Zamarły, systemy zagotowały się, a gdy Mala patrzyła, zombie roiły się nad mechami, przewracając je, pracując bez wytchnienia, dopóki nie znalazły drogi do środka. Czerwona mgła trysnęła w~niebo, gdy rozczłonkowywali pilotów. 


-- Niezły strzał -- powiedziała, wyginając plecy na krześle, siorbiąc resztki kubka czaju, który wystygł przy jej boku. 

Bratanek idiota pani Dibyendu stał boso w~drzwiach kawiarni, plując betelem na ulicę, a słodki zapach wracał do niej falami. Sen zbierał się w~jej umyśle, czekając, by rzucić się na nią, więc nadszedł czas, aby odejść. Odwróciła się, żeby powiedzieć o tym swojej armii, gdy jej słuchawki wypełniły grzmoty nadlatujących mechów, i~to w~\textit{wielu}. 


Uderzyła mocno, siadając w~fotelu, obróciła się, palce powędrowały do klawiatury, oczy wpatrywały się w~ekran. Wrogie mechy zbliżały się zamknięte w~konfiguracjach megamecha, piętnaście -- nie \textit{dwadzieścia }-- z~nich połączyło się, tworząc bota tak wielkiego, że wyglądała obok niego jak komar.  


-- Do mnie! -- zawołała, i~-- Formacja -- a jej żołnierze ruszyli do klawiatur, jej armia zainicjowała własną sekwencję megamecha, ale trwało to zbyt długo i~było ich za mało, i~chociaż walczyli dzielnie, gigantyczny statek wroga rozerwał ich na kawałki, podnosząc każdego robota bojowego i~zaglądając pod jego maskę, gdy rozrywał zbroję i~puszczał wijącego się pilota na falę zombie u jego stóp. Za późno, Mala przypomniała sobie swoją strategię, przypomniała sobie, jak to było, kiedy zawsze dowodziła słabszymi siłami, na jakiej postawie obronnej powinna była postawić swoją armię, gdy tylko zobaczyła, jak przegrywa. 


Za późno. Chwilę później jej własny mech znalazł się w~szponach wroga, został podniesiony do jego twarzy, a gdy się do niego zbliżyła, światła na jej konsoli zmieniły się i~zabrzmiał cichy klakson: robot próbował zinfiltrować systemy jej własnego statku, aby połączyć się z~ich, aby je zhakować. To była kolejna gra w~tej grze, gra w~hack-and-be-hacked, i~była w~tym bardzo dobra. Wymagało to rozwiązania serii zagadek logicznych, rozwiązywania ich szybciej niż wróg, a ona klikała i~pisała, gdy odkryła, jak zbudować most z~bloków o nieregularnych rozmiarach, gdy odkryła, jak otworzyć zamek, którego zapadki musiały być kliknięte tylko po to, aby mechanizm zadziałał, jak się domyśliła\ldots  


Nie była wystarczająco szybka. Jej armia zgromadziła się wokół niej, gdy jej konsola się zablokowała, wróg wewnątrz jej mecha, teraz biegnie od bootloadera do miotacza ognia. 


-- Cześć -- odezwał się głos w~jej słuchawkach. To było coś, co można było zrobić, gdy kontrolowało się zbroję innego gracza, można było przejąć jego komunikację. Pomyślała o wyjęciu słuchawek i~przełączeniu na głośnik, żeby jej armia też mogła słuchać, ale jakieś przeczucie powstrzymało jej rękę. Ten wróg zadał sobie trud, by osobiście z~nią porozmawiać, żeby usłyszała, co ma do powiedzenia. 


-- Nazywam się Big Sister Nor -- powiedziała i~to \textit{była }ona, kobiecy głos, nie, \textit{dziewczęcy } \ldots może coś pomiędzy. Jej hindi było dziwnie akcentowane, jak u chińskich aktorów w~filmach, które widziała. 

-- Walka z~tobą była przyjemnością. Twojej gildii poszło dobrze. Oczywiście, nam poszło lepiej.

 Mala usłyszała nierówne wiwaty i~zdała sobie sprawę, że na kanale czatu były dziesiątki wrogów, wszyscy nasłuchiwali. To, co pomyliła z~szumem na kanale, to w~rzeczywistości dziesiątki wrogów, gdzieś na świecie, wszyscy oddychający do swoich mikrofonów jak mówiła ta kobieta. 


-- Jesteście bardzo dobrymi graczami -- powiedziała Mala, szepcząc to tak, że słyszał tylko jej mikrofon. 


-- Nie jestem tylko graczem, ty też nie, moja droga. 

 W tym głosie było coś siostrzanego, bez radosnej rywalizacji, jaką Mala czuła wobec graczy, których pokonała wcześniej w~grze. Mala wbrew sobie stwierdziła, że trochę się uśmiecha. Kołysała brodą z~boku na bok -- \textit{Och, jesteś mądra, mów dalej }-- a jej żołnierze wokół niej wykonali ten sam gest. 


-- Wiem, dlaczego walczysz. Myślisz, że wykonujesz uczciwą robotę, ale czy kiedykolwiek zastanawiałeś się, dlaczego ktoś miałby ci zapłacić za atakowanie innych robotników w~grze? 


Mala wygoniła swoją armię, wskazując gestem w~kierunku drzwi. Kiedy była sama, powiedziała:

 -- Ponieważ psują grę prawdziwym graczom. Wtrącają się. 


Olbrzymi mech powoli pokręcił głową. 

-- Czy naprawdę jesteś tak ślepa? Czy myślisz, że syndykat, który ci płaci, robi to, ponieważ zależy mu na tym, by gra była \textit{fajna}? Och, kochanie. 


Umysł Mali zawirował. To było jak rozwiązywanie jednej z~tych zagadek. Oczywiście pan Banerjee nie dbał o innych graczy. Oczywiście nie pracował w~grze. Gdyby pracował dla gry, mógłby po prostu zawiesić konta graczy, z~którymi walczyła Mala. Czyściej i~gustowniej. Rozwiązanie pojawiło się w~jej umyśle.

-- A więc są rywalami biznesowymi? 


-- O tak, jesteś tak sprytna, jak myślałam. Tak, rzeczywiście. Są rywalami biznesowymi. Gdzieś tam jest grupa graczy jako one, opłacani, żeby podnieść level mecha, uprawiać złoto, zdobyć ziemię lub zrobić cokolwiek z~tych rzeczy, które mogą zamienić pracę w~pieniądze. A jak myślisz, do kogo trafią te pieniądze? 


-- Do mojego szefa -- powiedziała. -- I jego szefów. Tak to jest. -- Wszyscy dla kogoś pracowali. 


-- Czy to brzmi uczciwie? 


-- Dlaczego nie? -- powiedziała Mala. -- Pracujesz, robisz coś lub coś robisz, a osoba, dla której to robisz, płaci ci coś za twoją pracę. Taki jest świat, tak to działa. 


Co robi osoba, która ci płaci, aby zarobić na swoją część?, pomyślała Mala.

 -- Wymyśla, jak zamienić pracę w~pieniądze. Płaci mi za to, co robię. Wiesz, to są głupie pytania. 


-- Wiem -- powiedziała Big Sister Nor. -- To głupie pytania, które mają jedne z~najbardziej zaskakujących i~interesujących odpowiedzi. Większość ludzi nigdy nie pomyśli o zadaniu głupich pytań. Czy wiesz, czym jest związek? 


Mala pomyślała. W całym Bombaju istniały związki, ale żaden w~Dharavi. Słyszała jednak, że wielu ludzi o nich mówiło. 

-- Grupa pracowników -- powiedziała. -- Którzy sprawiają, że ich szefowie płacą im więcej. -- Pomyślała o wszystkim, co usłyszała. -- Powstrzymują innych pracowników przed zabieraniem im pracy. Zaczynają strajkować. 


-- Tak właśnie \textit{działają} związki zawodowe, dobrze. Ale nie ma większego sensu, czym one są. Powiedz mi tak: jeśli poszłabyś do swojego szefa i~poprosiła o więcej pieniędzy, krótsze godziny pracy i~lepsze warunki pracy, jak myślisz, co on by odpowiedział? 


-- Roześmiałby się i~odesłał -- powiedziała Mala. To było niewiarygodnie głupie pytanie. 


-- Prawie na pewno masz rację. Ale co, jeśli wszyscy pracownicy, do których poszedł, powiedzieliby to samo? A gdyby wszędzie, gdzie się udał, byli pracownicy mówiący: ,,Jesteśmy tyle warci'' i~,,Nie będziemy tak traktowani sposób'' oraz ,,Nie możesz odebrać nam pracy, jeśli nie ma ku temu sprawiedliwego powodu''? A gdyby wszyscy pracownicy na całym świecie domagali się takiego traktowania?. 


Mala odkryła, że kręci głową. 

-- To niedorzeczny pomysł. Zawsze znajdzie się ktoś biedny, który podejmie tę pracę. To nie ma znaczenia. To nie zadziała. -- Stwierdziła, że jest wściekła. -- Głupia!  


-- Przyznaję, że to wszystko jest raczej nieprawdopodobne -- powiedziała kobieta z~wyraźnym rozbawieniem w~głosie. -- Ale pomyśl przez chwilę o swoim pracodawcy. Czy wiesz, gdzie są jego pracodawcy? Czy wiesz, gdzie są gracze, z~którymi walczysz? Gdzie są ich klienci? Czy wiesz, gdzie ja jestem? 


-- Nie rozumiem, dlaczego to ma znaczenie\ldots   


-- Och, to ma znaczenie. Ma znaczenie, bo chociaż wszyscy ci ludzie są na całym świecie, nie ma między nimi prawdziwego dystansu. Rozmawiamy tu jak sąsiedzi, ale ja jestem w~Singapurze, a ty w~Indiach. Gdzie? Delhi? Kalkuta? Bombaj? 


-- Bombaj -- przyznała. 


-- Nie brzmisz jak Bombaj -- powiedziała. -- Masz piękny akcent. Uttar Pradesh? 


Mala była zaskoczona, słysząc, że jej i~jej wioski rodzinny stan tak łatwo został odgadnięty. 

-- Tak -- powiedziała. Była dziewczyną ze wsi, była generałem Robotwallah i~ta kobieta bardzo szybko się na nie poznała. 


-- Ta gra ma siedzibę w~Ameryce, w~mieście o nazwie Atlanta. Korporacja jest zarejestrowana na Cyprze, w~Europie. Gracze są na całym świecie. Ci, z~którymi walczyłaś, są w~Wietnamie. Mieliśmy cudowną rozmowę, zanim przyszłaś i~rozwaliłaś ich na kawałki. Jesteśmy wszędzie, ale wszyscy jesteśmy tutaj. Każdy, kto kiedykolwiek zatrudnił twojego szefa do wykonywania twojej pracy, skończyłby tutaj, a my moglibyśmy znaleźć tego pracownika i~porozmawiać z~nim. Gdziekolwiek Twój szef pójdzie, wszyscy jego pracownicy przyjdą tu i~będą pracować. Porozmawiamy z~nimi w~ten sposób i~porozmawiamy z~nimi o tym, jaki świat moglibyśmy mieć, gdyby wszyscy pracownicy współpracowali, by chronić nawzajem swoje interesy. 


Mala wciąż kręciła głową. 

-- Po prostu by cię zdmuchnęli. Wynajęli armię taką jak moja. To głupi pomysł. 


Olbrzymia metamecha uniosła ją na twarz, gdzie szczękały i~brzęczały jej gigantyczne zęby. 

-- Myślisz, że istnieje armia, która mogłaby nas pokonać? 


Mala pomyślała, że może jej armia mogłaby, gdyby była w~sile, gdyby była przygotowana. Potem pomyślała o tym, jak bardzo udaną wojnę trzeba by przeprowadzić, aby wygrać z~jedną z~tych gigantycznych bestii. 

-- Może nie. Może potrafisz zrobić to, co mówisz, że potrafisz. -- Pomyślała trochę więcej. -- Ale w~międzyczasie nie mielibyśmy żadnej pracy. 


Olbrzymia metalowa twarz skinęła głową. 

-- Tak, to prawda. Na początku możesz nie mieć swoich zarobków. I może twoi współpracownicy wsparliby trochę, aby ci pomóc. To kolejna rzecz, jaką robią związki, nazywa się to płacą strajkową. Ale w~końcu ty i~ja i~wszyscy chcielibyśmy cieszyć się światem, w~którym otrzymujemy pensję na życie, gdzie pracujemy w~warunkach umożliwiających życie, a nasze miejsca pracy są uczciwe i~przyzwoite. Czy nie jest to warte małego poświęcenia? 


Oto było to: 

-- Prosisz mnie o poświęcenie. Dlaczego miałabym się poświęcać? Jesteśmy biedni. Walczymy o bardzo mało, bo mamy jeszcze mniej. Dlaczego uważasz, że powinniśmy się poświęcać? Dlaczego \textit{Ty }się nie poświęcisz ? 


-- Och, siostro, wszyscy się poświęciliśmy. Rozumiem, że to wszystko jest dla ciebie bardzo nowe i~że trzeba trochę się przyzwyczaić. Jestem pewna, że kiedyś znowu się zobaczymy. W końcu wszyscy grają tutaj w~tym samym świecie, prawda? 


Mala zdała sobie sprawę, że oddechy, które słyszała, inne głosy na czacie, ucichły. Przez krótki czas to była tyla tylko Mala i~ta kobieta, która nazywała ją ,,siostrą''. 


-- Jak masz na imię? 


-- Jestem Nor-Ayu -- powiedziała. -- Ale nazywają mnie ,,Big Sister Nor''. Na całym świecie tak mnie nazywają. Jak mam cię nazywać? 


Imię Mali było na końcu jej języka, ale nie wypowiedziała. Zamiast tego powiedziała: 

-- Generał Robotwallah. 


-- Bardzo dobre imię -- powiedziała Big Sister Nor. -- To była przyjemność cię poznać. -- Po tych słowach gigantyczny mech upuścił ją, odwrócił się i~odsunął, miażdżąc zombie pod swoimi stopami. 


Mala wstała i~poczuła liczne trzaski i~trzaski jej kręgosłupa i~mięśni. Siedziała przez wiele godzin. 


Przekręciła głowę z~boku na bok na szyi, ćwicząc sztywność i~zobaczyła, że przygląda jej się głupi siostrzeniec pani Dibyendu. Jego wargi były wypełnione śmierdzącą śliną betelu i~patrzył na nią ze szczerością, która sprawiła, że skręciła się aż do dołka żołądka. 


-- Zostałaś dla mnie -- powiedział z~szerokim uśmiechem na twarzy.

 Jego zęby były brązowe. Tak naprawdę nie był idiotą, w~każdym razie nie w~głowie. Był jednak bardzo gruby i~bardzo powolny, z~brutalną siłą, którą pani Dibyendu zawsze określała jako jego ,,wyjątkowy hart ducha''. Mala myślała, że był tylko zbirem. Widziała go spacerującego wąskimi uliczkami Dharavi. Nigdy nie przesuwał się dla kobiet lub starszych ludzi, co sprawiało, że okrążali go, nawet jeśli oznaczało to wchodzenie w~błoto lub coś gorszego. I cały czas żuł betel. Wiele osób żuło betel, to było jak palenie, ale jej matka nie znosiła tego nałogu i~tyle razy mówiła jej, że to ,,niski'' nawyk i~brudny, że ona nie mogła nie myśleć gorzej o żuciu betelu. 


Spojrzał na nią przekrwionymi oczami. Nagle poczuła się bardzo wrażliwa, tak jak czuła się przez cały czas, kiedy po raz pierwszy przybyli do Dharavi. Ona zrobiła krok w~prawo, on też zrobił krok w~prawo. To była przekroczona granica: kiedy zablokował jej wyjście, ogłosił intencje, że zamierza ją skrzywdzić. To była podstawowa strategia wojskowa. Wykonał pierwszy ruch, więc miał inicjatywę, ale też szybko pokazał zamiary, więc\ldots  


Ona sfingowała w~lewo, a on dał się na to nabrać. Spuściła głowę jak byk i~wbiła ją w~środek jego klatki piersiowej. Już wytrącony z~równowagi, poleciał na plecy. Nie przestawała się ruszać, nie oglądała się za siebie, po prostu szła dalej, myśląc o jakimś szarżującym byku, który by przejechał go, gdy ruszyła do drzwi bez zatrzymywania się. Jedna pięta spadła na jego klatkę piersiową, druga na twarz, rozgniatając usta i~nos. Marzyła, aby coś \textit{chrupnęło}, ale nic się nie stało. 


W jednej chwili była przy drzwiach i~wyszła na chłodne powietrze ciemnej, ciemnej nocy Dharavi. Wokół niej odgłosy biegających po dachach szczurów, odległe odgłosy dróg, chrapanie. I wiele innych, mniej rozpoznawalnych dźwięków, dźwięków, które mogły czatować w~cieniach wokół niej. Stłumiona mowa. Odległy pociąg. 


Nagle odesłanie jej armii nie wydawało się dobrym pomysłem. 


Za sobą usłyszała znacznie wyraźniejszy dźwięk groźby. Siostrzeniec-idiota wpadł przez drzwi, jego buty na ubitej ziemi. Wślizgnęła się z~powrotem do alejki między dwoma budynkami, niewiele szerszej od niej, jej stopy chlapały przez jakiś rodzaj ciepłego płynu, który unosił się paskudnym smrodem. Siostrzeniec-idiota wtoczył się w~noc. Stanęła w~miejscu. Cofnął się, rozglądając się za nią we wszystkich kierunkach. 


Stała tam, czekając, aż się podda, ale on nie chciał. Szarżował tam i~z powrotem. Stał się bykiem, rozwścieczonym, niestrudzonym, głupim. Usłyszała jego głos zgrzytliwy w~piersi. Trzymała w~dłoni telefon komórkowy, drugą ręką przykrywając go, osłaniając zdradzieckie światło, które emanowało z~jego malutkiego ekranu. Była teraz 12:47, a przez całe 14 lat nigdy nie była sama o tej porze. 


Mogłaby napisać do kogoś ze swojej armii -- przyjechaliby po nią, prawda? Czy nie spali, albo czy obudziły ich ćwierkające telefony. Nikt jednak nie czuwał o tej godzinie. A jak to wytłumaczyć? Co powiedzieć? 


Czuła się jak idiotka. Wstydziła się. Powinna była to przewidzieć, powinna była być generałem, powinna była zastosować strategię. Zamiast tego została otoczona. 


Mogła poczekać. W razie potrzeby całą noc. Nie ma potrzeby, aby jej armia dowiedziała się o jej słabości. Siostrzeniec idiota zmęczy się albo wzejdzie słońce, dla niej to to samo. 


Przez cienkie ściany domów po obu stronach dźwięki chrapania. Z cieczy pod nią w~rowie unosił się paskudny zapach, a palcami stóp ściskała coś śliskiego. To paliło jej skórę. Szczury przemykały nad głowami, brzmiąc jak deszcz na blaszanych dachach. Głupia, głupia, głupia, to była jej mantra, ciągle w~jej umyśle. 


Byk się męczył. Następnym razem, gdy ją mijał, jego oddech był straszliwym świszczącym oddechem, od którego smród betelu rozchodził się przed nim jak słodka zgnilizna. Mogła poczekać na jego następne przejście, a potem uciec. 


To był dobry plan. Nienawidziła tego. On\ldots  On jej zagroził. Przestraszył ją. Powinien \textit{zapłacić}. Była generałem Robotwallah, a nie tylko jakąś dziewczyną z~wioski. Pochodziła z~Dharavi, twarda. Mądra. 


Przemknął obok, a ona wymknęła się z~zaułka, jej stopy wyswobodziły się z~błota ze słyszalnym \textit{plusk}. Wciąż patrzył w~innym kierunku, jeszcze jej nie słyszał i~był do niej odwrócony plecami. Głupi chłopcy z~jej armii walczyli tylko twarzą w~twarz, mówili o ,,honorze'' uderzenia od tyłu. Honor to tylko głupie chłopięce rzeczy. Zwycięstwo przebija honor. 


Zebrała się w~sobie i~podbiegła do niego, obie ręce sztywne, ręce na wysokości ramion. Uderzyła go wysoko i~ruszyła dalej, tak jak przedtem, a on znowu upadł, zupełnie nieprzygotowany na atak od tyłu. Dźwięk, który wydał na ziemi, przypominał odgłos kozy spadającej na blok rzeźnika. Próbował się przetoczyć, a ona odwróciła się i~pobiegła na niego, podskakując w~powietrze i~lądując z~obiema zabłoconymi stopami na jego głowie, wbijając mu twarz w~błoto. Krzyknął z~bólu, dźwięk stłumiony przez brud, a potem leżał oszołomiony. 


Wróciła do niego i~uklękła przy jego głowie, jego włochaty płatek ucha kilka cali od jej ust. 


-- Nie czekałam na ciebie w~kawiarni. Zajmowałam się własnymi sprawami -- powiedziała. -- Nie lubię cię. Nie powinieneś gonić dziewczyn, bo dziewczyny mogą się odwrócić i~cię złapać. Rozumiesz mnie? Powiedz, że mnie rozumiesz, zanim wyrwę ci język i~podetrę nim tyłek. 

Chłopcy cały czas rozmawiali w~ten sposób na kanałach z~grami, a ona nigdy tego nie aprobowała. Ale słowa miały moc, czuła ją w~ustach, gorącą jak krew z~odgryzionego języka. 


-- Powiedz mi, że mnie rozumiesz, idioto! -- syknęła. 


-- Rozumiem -- powiedział, a słowa wyszły ze zgniecionych warg i~zmiażdżonego nosa. 


Odwróciła się na pięcie i~zaczęła odchodzić. Jęknął za nią, po czym zawołał: 

-- Dziwka! Głupia dziwka! 


Nie myślała, po prostu działała. Odwróciła się, pobiegła do jego wciąż leżącego ciała, niewyraźnego w~półmroku, jeden krok, dwa kroki, jak mistrzowski piłkarz wchodzący po rzut karny, a potem go \textit{kopnęła}, cuchnąca woda pryskała z~przesiąkniętego palca jej buta, gdy uderzyła w~jego wielką, głupią klatkę piersiową. Coś tam pękło -- może kilka rzeczy i~och, czy to nie było \textit{cudowne} uczucie? 


Był każdym mężczyzną, który ją przestraszył, który wykrzykiwał za nią paskudne rzeczy, który terroryzował jej matkę. Był kierowcą autobusu, który groził, że wyrzuci ich na pobocze, jeżeli nie zapłacą mu łapówki. Wszystko i~wszyscy, którzy kiedykolwiek sprawiali, że czuła się drobną i~przestraszoną, dziewczyną z~wioski. Wszyscy. 


Odwróciła się. Trzymał się za bok i~ryczał teraz, płakał głupimi łzami na głupich policzkach, rozświetlony w~przytłumionym świetle księżyca, które przesączało się przez mgiełkę plastikowego dymu wiszącego nad Dharavi. Wstała i~wykonała kolejny krok, jeden krok, dwa kroki, \textit{kopnięcie }i \textit{chrupnięcie}, znowu ten satysfakcjonujący dźwięk z~jego żeber. Jego szloch ugrzązł w~piersi, a potem wziął ogromny, drżący oddech i~\textit{zawył }jak ranny kot w~nocy, krzyczał tak głośno, że tutaj, w~Dharavi, zapaliły się światła, a przez okna dobiegły głosy. 


To było tak, jakby zaklęcie zostało złamane. Trzęsła się i~była zlana potem, a ludzie przyglądali się jej w~ciemności. Nagle zapragnęła jak najszybciej wrócić do domu, jeśli nie szybciej. Czas iść. 


Pobiegła. Mala uwielbiała biegać po polach jako mała dziewczynka, z~włosami rozwianymi za nią, kolanami i~ramionami pompującymi, po drogach gruntowych. Teraz biegła w~nocy, smród wody z~rowu uderzał ją w~nos przy każdym mlaszczącym kroku. Głosy goniły ją przez całą noc, choć docierały do niej przez puls jej pulsu w~uszach, a później nie mogła powiedzieć, czy były prawdziwe, czy wyimaginowane. 


Ale w~końcu była w~domu i~wbiegała po schodach do mieszkania na trzecim piętrze, które wynajęła dla swojej rodziny. Jej grzmiące kroki wywołały krzyki sąsiadów z~dołu, ale zignorowała je, pogrzebała przy kluczu i~weszła do środka. 


Jej brat Gopal spojrzał na nią znad maty, mrugając w~ciemności, z~nagą chudą klatką piersiową. 

-- Mala? 


-- W porządku -- powiedziała. -- Nic. Śpij, Gopal.  


Osunął się z~powrotem. Buty Mali śmierdziały. Zdjęła je, używając tylko czubków palców, i~zostawiła za drzwiami. Być może zostałyby skradzione, choć trzeba by naprawdę być desperatem, by ukraść te buty. Teraz jej stopy śmierdziały. W kącie stał duży dzban z~wodą i~czerpak. Ostrożnie zaniosła chochlę do okna, otworzyła skrzypiącą okiennicę i~powoli wylała wodę na stopy, opierając najpierw jedną, a potem drugą o parapet. Gopal znów się poruszył. 

-- Bądź cicho -- powiedział -- to czas snu. 


Zignorowała go. Nadal brakowało jej tchu, a rzeczywistość tego, co zrobiła, zapadała w~niej. Kopnęła siostrzeńca-idiotę \ldots  ile razy? Dwa? Trzy? I za każdym razem coś w~jego ciele \textit{pękało}. Dlaczego ją zablokował? Dlaczego poszedł za nią w~noc? Co sprawiło, że wielcy i~silni uprawiali ten sport terroryzowania słabszych? Całe grupy chłopców robiły to dziewczynkom, a czasem nawet dorosłym kobietom -- podążały za nimi, wzywały ich, dotykały ich, czasami nawet prowadziło to do gwałtu. Nazywali to ,,dokuczaniem'' i~traktowali jak grę. To nie była gra, jeśli byłaś ofiarą. 


Dlaczego kazali jej to zrobić? Dlaczego wszyscy zmusili ją do tego? Dźwięk trzasku był wtedy tak satysfakcjonujący, a teraz był tak obrzydliwy. Trzęsła się, chociaż noc była taka gorąca, jedna z~tych parnych nocy, kiedy wszystko było śliskie od nisko wiszącej, wilgotnej wilgoci. 


I ona też płakała, płacz wydobywający się z~niej, którego nie była w~stanie opanować, i~tego też się wstydziła, bo tak zrobiłaby dziewczyna z~wioski, a nie dzielny generał Robotwallah. 


Zrogowaciałe ręce dotknęły jej ramion, ścisnęły je. Zapach matki w~nosie: czysty pot, przyprawa do gotowania, mydło. Silne, cienkie ramiona otoczyły ją od tyłu. 


-- Córko, córko, co się Ci stało? 


I chciała wszystko opowiedzieć Mamaji, ale wszystko, co wyszło, to płacz. Odwróciła głowę na piersi matki i~zaczęła szlochać, łkając falami, czując, że wywrócą ją na lewą stronę. Gopal wstał i~przeszedł do sąsiedniego pokoju, cichy i~przerażony. Zauważyła to, zauważyła to wszystko z~dużej odległości, jej ciało szlochające, jej umysł gdzieś daleko, chłodny i~odległy. 


-- Mamaji -- powiedziała w~końcu. -- Był chłopiec. 


Matka ścisnęła ją mocniej. 

-- Och, Mala, słodka dziewczyno \ldots  


-- Nie, Mamaji, on mnie nie dotknął. Próbował. Przewróciłam go. Dwa razy. I kopnęłam go, kopałam go, aż usłyszał pękanie, a potem uciekłam do domu. 


-- Mala! -- matka trzymała ją na odległość ramienia. -- Kim on był? -- W znaczeniu: Czy był kimś, kto może nas ścigać, kto może sprawić nam kłopoty, kto może nas zrujnować tutaj w~Dharavi?


-- Był siostrzeńcem pani Dibyendu, ten duży, ten, który cały czas sprawia kłopoty. 


Palce matki zacisnęły się na jej ramionach, a jej oczy się rozszerzyły. 


-- Och, Mala, Mala \ldots  och, nie. 


A Mala wiedziała dokładnie, co jej matka miała na myśli, dlaczego ogarniało ją przerażenie. Jej związek z~panem Banerjee pochodził od pani Dibyendu. A mieszkanie, ich życie, telefon i~ubrania, które nosili -- wszystko pochodziło od pana Banerjee. Balansowali na chwiejnym filarze relacji, a pani Dibyendu znajdowała się na dole, relacja opierająca się na jej ramionach. A siostrzeniec-idiota mógłby ją przekonać, żeby wzruszyła ramionami i~wszystko by się zawaliło -- pieniądze, bezpieczeństwo, to wszystko. 


To była największa niesprawiedliwość ze wszystkich, niesprawiedliwość, która popychała ją do kopania, kopania i~kopania \ldots  ten głupi chłopak wiedział, że może ujść na sucho z~jego chwytaniem i~zastraszaniem, ponieważ nie mogła go powstrzymać. Ale powstrzymała go i~nie mogła\ldots  nie chciała \ldots  żałować. 


-- Mogę porozmawiać z~panem Banerjee -- powiedziała. -- Mam jego numer telefonu. Wie, że jestem dobrym pracownikiem, sprawi, że wszystko będzie dobrze. Zobaczysz, Mamaji, nie martw się. 


-- Dlaczego, Mala, dlaczego? Nie mogłaś po prostu uciec? Dlaczego musiałaś skrzywdzić tego chłopca? 


Mala poczuła, jak napływa do niej część gniewu. Jej matka, jej własna matka\ldots  


Ale zrozumiała. Jej matka chciała ją chronić, ale jej matka nie była generałem. Była tylko dorosłą dziewczyną ze wsi. Została pobita przez zbyt wielu chłopców i~mężczyzn, zbyt wiele krzywdy, biedy i~strachu. To było to, co było Mali przeznaczone, być kimś, kto uciekł od swoich napastników, ponieważ nie mogła ich rozgniewać. 


Nie zrobiłaby tego. 


Bez względu na to, co stało się z~panem Banerjee, panią Dibyendu i~jej głupim bratankiem-idiotą, nie zamierzała zostać taką osobą. 


\bigskip
\threeast


Ta scena jest poświęcona Borders, światowemu gigantowi księgarskiemu, którego można znaleźć w~miastach na całym świecie -- nigdy nie zapomnę wejścia do gigantycznych Borders na Orchard Road w~Singapurze i~odkrycia półki wypełnionej moimi powieściami! Przez wiele lat Borders na Oxford Street w~Londynie gościło comiesięczne wieczory science fiction Pata Cadigana, podczas których miejscowi i~przyjezdni autorzy czytali swoje prace, opowiadali o science fiction i~spotykali swoich fanów. Kiedy jestem w~obcym mieście (co często się zdarza) i~potrzebuję świetnej książki na następny lot, zawsze wydaje mi się, że Borders jest pełen wspaniałych wyborów -- szczególnie lubię Borders na Union Square w~San Francisco. 


\href{https://en.wikipedia.org/wiki/HTTP_404}{Borders na całym świecie} 

\bigskip
\threeast

Jeśli chcesz się wzbogacić bez robienia czegokolwiek, czego ktoś potrzebuje lub chce, musisz być \textit{szybki}. 


Terminem technicznym na to jest \textit{arbitraż}. Wyobraź sobie, że mieszkasz w~bloku mieszkalnym i~pada tak mocno, że nikt nie chce skoczyć do sklepu spożywczego. Twoja sąsiadka po prawej, pani Głodna, chce banana i~jest gotowa zapłacić za niego 0,50 dolara. Twój sąsiad po lewej, pan Pełny, ma całą szafę pełną bananów, ale w~tym miesiącu ma problemy z~opłaceniem rachunku telefonicznego, więc sprzeda tyle bananów, ile chcesz kupić, po 0,30 dolara za sztukę. 


Można by pomyśleć, że po sąsiedzku byłoby zadzwonić do pani Głodnej i~powiedzieć jej o Panu Pełnym, pozwalając im sfinalizować transakcję. Jeśli tak myślisz, zapomnij o bogaceniu się bez pożytecznej pracy. 


Jeśli jesteś arbitrażystą, to pomyślisz o godnej pożałowania ignorancji sąsiadów jako o szansie. Łapiesz wszystkie banany pana Pełnego, a potem biegniesz do mieszkania pani Głodnej z~wyciągniętą ręką. Za każdego banana, który kupuje, wkładasz do kieszeni 0,20 dolara. Nazywa się to arbitrażem. 


Arbitraż to bardzo ryzykowny sposób zarabiania na życie. Co się stanie, jeśli pani Głodna zmieni zdanie? Utknąłeś z~bananami, ot co. 


Albo co się stanie, jeśli jakiś inny arbitrażysta będzie pierwszy u drzwi pani Głodnej, wypełniając jej mieszkanie wszystkimi bananami, których może potrzebować? Po raz kolejny utkniesz z~garścią bananów i~nie masz gdzie ich włożyć (chociaż tutaj sugeruje się kilka wybranych otworów). 


W prawdziwym świecie arbitrażyści nie noszą bananów, kupują i~sprzedają za pomocą komputerów podłączonych do sieci, sprawdzają wszystkie zaległe zamówienia ( ,,oferty'') i~pytają, a kiedy znajdą kogoś, kto jest gotów zapłacić za coś więcej niż ktoś inny, płacąc za to, kupują ten przedmiot po niższej cenie, dodają narzut i~sprzedają. 


A dzieje się to bardzo, bardzo szybko. Jeśli masz zamiar pokonać innych arbitrażystów towarem, jeśli chcesz dotrzeć tam, zanim kupujący zmieni zdanie, musisz działać szybciej niż prędkość myśli. Dosłownie. Arbitraż nie polega na tym, że człowiek czujnie obserwuje na ekranach różnice cenowe. 


Nie, arbitraż odbywa się za pomocą zautomatyzowanych systemów. Ci mali handlowcy krążą po światowych rynkach sieciowych, szukając okazji do arbitrażu, kupując coś i~sprzedając w~mniej niż mikrosekundę. Dobry dom arbitrażowy przeprowadza codziennie \textit{miliard }lub więcej transakcji, wyciskając z~każdej po kilka centów. Miliard razy kilka centów to dużo pieniędzy, jeśli masz szybki klaster komputerowy, dobrego programistę i~szybkie połączenie sieciowe, możesz zarobić \textit{dziesięć lub dwadzieścia }milionów dolarów dziennie. 


Nieźle, biorąc pod uwagę, że wszystko, co robisz, to wykorzystywanie faktu, że jest tu osoba, która chce coś kupić, i~osoba, która chce to sprzedać. Nieźle, biorąc pod uwagę, że jeśli ty i~wszyscy twoi kumple z~arbitrażu mielibyście jutro zniknąć, gospodarka i~świat nawet by tego nie zauważyli. Nikt nie potrzebuje ani nie chce twojej ,,usługi'', ale wciąż jest to słodki sposób na wzbogacenie się. 


Najlepszą rzeczą w~arbitrażu jest to, że nie musisz wiedzieć ani jednej, pojedynczej rzeczy na temat rzeczy, które kupujesz i~sprzedajesz, aby się na tym wzbogacić. Niezależnie od tego, czy są to banany, czy miecze vorpal, wszystko, co musisz wiedzieć o kupowanych rzeczach, to tyle, że ktoś \textit{tutaj }chce je kupić za więcej, niż ktoś \textit{tam }chce je sprzedać. To też dobrze, -- jeśli finalizujesz transakcję w~mniej niż mikrosekundę, nie ma czasu, aby usiąść i~poszukać kilku faktów na temat towaru. 


A towar jest dość dziwny. Zacznijmy od tego, że wiele z~tych rzeczy nawet nie istnieje -- miecze vorpal, młoty grabthara, złoto tysiąca wyimaginowanych krain. 


Teraz pomyśl, że ludzie handlują więcej niż złotem: Bogowie Gry sprzedają wszelkiego rodzaju śmieszne pieniądze. Co powiesz na to: 


Oferowane: obligacje wojowników Svartalfaheim o wartości 100 000 złota, płatne za sześć miesięcy od teraz. To nawet nie jest \textit{prawdziwe }fałszywe złoto -- to obietnica prawdziwego fałszywego złota w~przyszłości. Wstaw to na rynek przez kilka miesięcy, kochany, i~obserwuj, jak idzie. Oto handlowiec, który zapłaci o pięć procent więcej, niż było to warte wczoraj, obstawia, że za jakiś czas za sześć miesięcy gra stanie się popularniejsza, a więc wartość towarów w~grze wzrośnie w~tym samym czasie. 


A może obstawia, że Bogowie Gry po prostu podniosą cenę za wszystko i~utrudnią miażdżenie wystarczającej liczby potworów, by zebrać złoto, aby to dostać, wypędzając wszystkich poza najtwardszymi graczami, którzy zapłacą wszystko, aby położyć swoje ręce na wygranej. 


A może jest idiotą. 


A może myśli, że \textit{jesteś }idiotą i~jutro dasz mu dziesięć procent, dochodząc do wniosku, że wie coś, czego ty nie wiesz. 


A jeśli uważasz, że to dziwne, oto jeszcze lepsze! 


Coca-Cola sprzedaje ci sześciomiesięczną obligację Svartalfaheim Warriors o wartości 100 000 złota, ale obawiasz się, że jej wartość spadnie między teraz a D-Day, kiedy obligacja dojrzeje. Więc znajdujesz innego tradera i~prosisz go o jakieś ubezpieczenie: oferujesz mu 1,50 dolara, aby ubezpieczył swoją obligację. Jeśli wartość obligacji wzrośnie, on zatrzyma 1,50 dolara, a ty zatrzymasz zyski z~obligacji. Jeśli wartość obligacji spadnie, będzie musiał zapłacić różnicę. Jeśli to więcej niż 1,50 dolara, traci pieniądze. 


Jest to w~zasadzie polisa ubezpieczeniowa. Jeśli pójdziesz do towarzystwa ubezpieczeniowego na życie i~poprosisz go o polisę na twoje życie, postawią zakład o to, jak prawdopodobne jest, że walniesz w~kalendarz, i~naliczą ci tyle, żeby średnio zarobić (pod warunkiem, że dokładnie odgadują twoje szanse na śmierć). Więc jeśli handlowiec, z~którym rozmawiasz, myśli, że Wojownicy Svartalfaheim zamierzają wzrastać, może zażądać od Ciebie 10 lub 100 USD. 


Jak dotąd, jasne, racja? 


Teraz robi się jeszcze dziwniej. Pójdź krok dalej. 


Wyobraź sobie, że w~tej transakcji uczestniczy osoba trzecia, jakiś facet siedzący na uboczu, trzymający pulę pieniędzy i~próbujący wymyślić, co z~nią zrobić. Obserwuje, jak idziesz do tradera i~kupujesz polisę ubezpieczeniową za 1,50 dolara -- jeśli Svartalfaheim Warriors się polepszy, tracisz 1,50 dolara, jeśli sytuacja się pogorszy, trader musi nadrobić różnicę. 


Po zawarciu umowy ta trzecia strona, będąca czymś w~rodzaju ghula, podchodzi do tego samego tradera i~mówi: -- Hej, a co powiesz na to? Chcę postawić na ten sam zakład, który właśnie załatwiłeś z~tym facetem. Dam ci 1,50 dolara, a jeśli jego akcje wzrosną, zatrzymasz pieniądze. Jeśli jego kaucja spadnie, zapłacisz mi \textit{i }jemu różnicę. -- Zasadniczo ten facet zakłada, że twoje akcje to śmieci, więc może znajdzie chętnego. 


Teraz ma swój zakład, który nie jest nic wart, jeśli twoja obligacja wzrośnie, i~wart nieznaną kwotę, jeśli twoje obligacje się spadną. A wiesz, co on z~tym robi? 


Sprzedaje je.  


Pakuje je i~znajduje jakiegoś frajera, który chce kupić jego zakład za 1,50 dolara na twoją obligację za więcej niż 1,50 dolara, które będzie musiał wykrztusić, jeśli twoja obligacja wzrośnie. I frajer kupuje to, a potem \textit{sprzedaje}. A potem inny frajer kupuje to i~\textit{sprzedaje}. I zanim się zorientujesz, obligacja o wartości 100 000 sztuk złota, którą kupiłeś za 15 USD, ma wartość 1000 USD. 


I to jest \textit{ten }rodzaj rzeczy, które arbitrażysta kupuje i~sprzedaje. Nie nosi bananów od Pana Pełnego do Pani Głodnej -- kupuje i~sprzedaje zakłady na polisy ubezpieczeniowe na obietnice wyimaginowanego złota. 


I to właśnie on nazywa uczciwą pracą. 


Dobra robota, jeśli możesz ją dostać. 


\bigskip
\threeast


Ta scena jest poświęcona Compass Books/Books Inc, najstarszej niezależnej księgarni w~zachodnich Stanach Zjednoczonych. Mają sklepy w~Kalifornii, w~San Francisco, Burlingame, Mountain View i~Palo Alto, ale najfajniejsze ze wszystkiego jest to, że prowadzą zabójczą księgarnię w~środku Disneylandu Downtown Disney w~Anaheim. Jestem maniakiem parku Disneya (jeśli nie wierzysz, zobacz moją pierwszą powieść Down and Out in the Magic Kingdom) i~za każdym razem, gdy mieszkałem w~Kalifornii, kupowałem sobie roczną przepustkę do Disneylandu, i~praktycznie przy każdej wizycie wpadam do Compass Books w~Downtown Disney. Posiadają świetny wybór nieautoryzowanych (a nawet krytycznych) książek o Disneyu, a także dużą różnorodność książek dla dzieci i~science fiction, a kawiarnia obok robi niesamowite cappuccino. 


\href{https://www.booksinc.net/search/author/%22Doctorow%2C%20Cory%22}{Compass Books/Books Inc} 

\bigskip
\threeast

Matthew Fong i~jego pracownicy robili rajdy przez całą noc aż do następnego dnia, zbierając tyle złota, ile zdołali wydostać poza swój poziom, póki pozyskiwanie było dobre. Spali na zmiany i~dokooptowali każdego, kto popełnił błąd pytaniem, co robią, zmuszając ich do kopania w~lochach razem z~nimi. 


Przez cały czas mistrz Fong usuwał złoto z~ich kont tak szybko, jak w~nich lądowało. Wiedział, że kiedy Bogowie Gry dowiedzą się o jego operacji, wpadną do akcji, zawieszą konta wszystkich i~przejmą wszelkie złoto, które mieli w~swoim ekwipunku. Cała sztuka polegała na upewnieniu się, że nie mają czego uchwycić. 


Wskoczył więc do sieci i~trafił na wielkie fora maklerskie. To nie były tylko szary rynek, to były najczarniejsze z~czarnych i~trzeba było znać kogoś ważnego, żeby się do nich dostać. Ważnym Matthew był facet z~Syczuanu, chudy i~roztrzęsiony, z~kilkoma brakującymi zębami. Nazywał siebie ,,Cobra'' i~to on przedstawił Matthew Bossowi Wingowi te kilka miesięcy wcześniej. Cobra pracował dla kogoś, kto pracował dla kogoś, kto pracował dla jednego z~wielkich karteli, twardych organizacji przestępczych, które trzymały wszystkie rynki do zamiany złota z~gier w~gotówkę. 


Cobra dał mu login i~instruktaż na temat zawierania transakcji w~sieci brokerskiej. Teraz gdy mijała noc, Matthew przeszedł przez interfejs, wymieniając swoje złoto i~ustalając cenę wywoławczą, która była o połowę niższa od ceny sprzedaży podanej w~białym, naziemnym sklepie ze złotem, w~którym gweilos kupowali złoto z~gry od brokerów. 


Czekał i~czekał i~czekał, ale nikt nie kupił jego złota. Każdy świat gry został podzielony na lokalne serwery i~shardy, a podczas rejestracji trzeba było określić, na którym serwerze chcesz grać. Kiedy już wybrałeś serwer, utykałeś tam -- twój pluton nie mógł tak po prostu wędrować między równoległymi wszechświatami. To sprawiało, że kupowanie i~sprzedawanie złota było jeszcze trudniejsze: jeśli gweilo chciał kupić złoto do swojego pionka na serwerze A, musiał znaleźć farmera, który wydobywał złoto na serwerze A. Jeśli wydobyłeś całe złoto na serwerze B, nie miałeś szczęścia. 


W tym momencie pojawili się brokerzy. Kupowali złoto od wszystkich i~trzymali je w~ciągle zmieniającej się sieci kont, miliony łajdaków, którzy rozeszli się po całym świecie i~wymieniali niewielkie ilości złota w~nieregularnych odstępach czasu, by oszukać szpiegów szukających prania pieniędzy w~logice gry, którzy bezlitośnie polowali na farmerów i~brokerów. 


Unikanie tych filtrów było nauką, którą przez dziesięciolecia kumulowano w~prawdziwym świecie, zanim przeniesiono je do gier. Jeśli duży fundusz emerytalny w~prawdziwym świecie chciałby kupić akcje w~Google o wartości pół miliarda dolarów, ostatnią rzeczą, jaką chciałby zrobić, to poinformować wszystkich, że zamierzają wtopić tyle gotówki w~Google. Gdyby to zrobili, wszyscy inni kupiliby akcje Google wcześniej, dodali marżę i~ich wypatroszyli. 


Więc każdy, kto chce kupić dużo czegokolwiek -- kto chce przenosić dużo pieniędzy -- musi wiedzieć, jak to zrobić w~sposób niewidoczny dla szpiegów. Transakcje muszą być statystycznie nieistotne, co oznacza, że pojedyncza duża transakcja musi zostać podzielona na miliony małych transakcji, które wyglądają jak zwykli frajerzy, którzy kupują i~sprzedają trochę akcji. 


Bez względu na to, jakie sekrety próbujesz zachować i~przed kim próbujesz je zachować, techniki są takie same. W każdym świecie gry były tysiące pozornie normalnych postaci, które robiły pozornie normalne rzeczy, dając sobie nawzajem pozornie normalne sumy pieniędzy, ale pod koniec dnia wszystko to dało miliony złota w~handlu, odbywającym się tuż pod nosem bogów gry. 


Matthew obniżył cenę swojego złota, szukając ceny, za którą makler raczyłby go zauważyć i~kupić od niego. Cały handel odbywał się w~slangowym, szybkim języku chińskim -- to był jeden ze sposobów, w~jaki brokerzy utrzymywali swoją pozycję na rynku, ponieważ nie było zbyt wielu Rosjan, Indonezyjczyków i~Hindusów, którzy mogliby go śledzić i~grać -- przepełnionym obelgami i~nagabywaniem. W końcu Matthew znalazł magiczną cenę. Była niższa, niż się spodziewał, ale niewiele, a teraz, gdy ją znalazł, był w~stanie przenieść złoto drużyny tak szybko, jak tylko mógł je zgromadzić, wożąc postacie-atrapy do i~z lochów, które starały się zabrać gotówkę do botów prowadzonych przez brokerów. 


W końcu źródło wyschło. Najpierw gwałtownie spadła ilość złota w~lochach, a złoto spadło z~12 000 na godzinę do 8 000, potem 2 000, potem marne 100. Następnie zniknęły mareridtbane, a szkoda, bo był w~stanie sprzedać je bezpośrednio, sprzedawał ją w~dużych miastach, wklejał, wklejał i~wklejał swoją ofertę na czat, gdzie mogli ją zobaczyć prawdziwi gracze. A potem weszli gliniarze, moderatorzy otoczeni specjalnymi aureolami, którzy wrzucali na czat całe wykłady, surowe ostrzeżenia o naruszeniu warunków korzystania z~gry. 


A potem zawieszenia kont, gry znikające z~jednego ekranu po drugim, strzelające jak bańki mydlane. Wszyscy zostali wyrzuceni z~powrotem na ekrany logowania i~osunęli się, uśmiechając się szaleńczo i~wyczerpani, na swoich miejscach, patrząc na siebie z~wyczerpaną ulgą. Nareszcie było po wszystkim.  


-- Ile? -- spytał Lu, rzucając się do tyłu przez krzesło, nie otwierając oczu ani nie podnosząc głowy. -- Ile, mistrzu Fong? 


Matthew nie miał już swoich zeszytów, więc śledził wnętrze paczek papierosów Double Happiness, długie, zgrabne zestawienia liczb. Jego pióro przeskakiwało z~arkusza na arkusz, sprawdzając ostatni raz matematykę, a potem cicho:

 -- 3400 dolarów. 


Zapadła oszołomiona cisza. 

-- Ile? -- Lu miał teraz otwarte oczy. 


Matthew zrobił pokaz ponownego sprawdzania liczb, ale to wszystko, pokaz. Wiedział, że liczby się zgadzają. 

-- Trzy tysiące czterysta dwa dolary i~czternaście centów. 

To był dwa razy najwyższy wynik, jaki kiedykolwiek zdobyli dla Boss Winga. To była największa kwota, jaką ktokolwiek z~nich kiedykolwiek zarobił. Jego udział w~tym był większy, niż jego ojciec zarobił w~ciągu miesiąca. I zarobił to w~jedną noc. 


-- Przepraszam, \textit{ile}?  


-- 8080 misek pierogów, Lu. Tyle.  


Cisza była jeszcze gęstsza. To było dużo pierogów. To wystarczyło, by wynająć własne mieszkanie na fabrykę, miejsce z~komputerami i~szybkim łączem internetowym oraz sypialnie do spania, miejsce, w~którym mogli zarabiać i~zarabiać, gdzie mogli się bogacić jak każdy szef. 


Lu zerwał się z~krzesła i~krzyknął, dźwięk tak głośny, że cała kawiarnia odwróciła się w~ich stronę, ale ich to nie obchodziło, wszyscy wstali już ze swoich miejsc, krzycząc, tańcząc i~przytulając się nawzajem. 


A teraz nadszedł ten dzień, nowy dzień, kiedy słońce wzeszło, zaszło i~wschodziło podczas ich długiej pracy w~kawiarni, i~wygrali. To był nowy dzień dla nich i~dla wszystkich wokół nich. 


Wyszli na słońce, a na ulicach byli ludzie, tłumy kupujących i~sprzedających, krzątające się naganiacze, ładne dziewczyny w~dobrych ubraniach chodzące ramię w~ramię pod jednym parasolem. Ciepło dnia było jak wielki piec po klimatyzowanym chłodzie kawiarni, ale to też było dobre, wypalało smród papierosów, kawy, głodu. Nagle żaden z~nich nie był śpiący. Wszyscy chcieli jeść. 


Więc Matthew zabrał ich na śniadanie. W końcu byli jego zespołem. Zajęli tylny stolik w~indyjskiej restauracji niedaleko dworca kolejowego, w~miejscu, o którym słyszał, jak jego wujek Yiu-Yu opowiadał rodzicom i~chwalił się jakimś wspólnikiem, który go tam zabrał. Bardzo wyrafinowany. I tak dużo czytał o indyjskim jedzeniu w~swoich komiksach, że nie mógł się doczekać, żeby go spróbować. 


Wszyscy pozostali klienci byli albo obcokrajowcami, albo mieszkańcami Hongkongu, ale nie dopuścili, żeby ich to dotknęło. Chłopcy siedzieli przy stoliku z~tyłu, bawili się widelcami i~jedli talerz za talerzem curry i~świeżych, gorących podpłomyków zwanych naan. Było to pyszne i~dziwne i~stanowiło idealne zakończenie tego, co okazało się idealnej nocy. 


W połowie deseru -- pyszne lody z~mango -- bezsenność w~końcu dopadła ich wszystkich. Siedzieli na siedzeniach w~odrętwieniu, z~rękami na brzuchu, z~na wpół otwartymi oczami, a Matthew zażądał rachunku. 


Wyszli ponownie na światło. Matthew postanowił udać się do domu rodziców, przespać się przez chwilę na sofie, zanim wymyślił, co zrobić ze swoim rozbitym pokojem z~rozbitymi drzwiami. 


Gdy mrugali w~świetle, znajomy głos z~akcentem Wenzhou powiedział: 

-- Nie jesteś zbyt mądrym chłopcem, prawda? 


Matthew odwrócił się. Był tam człowiek Bossa Winga i~trzech jego przyjaciół. Podbiegli do przodu i~złapali chłopców, zanim zdążyli zareagować, jeden z~nich był tak duży, że chwycił chłopca w~obie ręce i~prawie podniósł go w~powietrze. 


Jego przyjaciele próbowali się uwolnić, ale człowiek Bossa Winga metodycznie bił ich tak długo, aż przestali. 


Matthew nie mógł uwierzyć, że to się dzieje, w~biały dzień, tuż obok stacji kolejowej! Ludzie przechodzili przez ulicę, aby ich ominąć. Matthew przypuszczał, że on też by to zrobił. 


Człowiek Bossa Winga pochylił się tak blisko, że Matthew wyczuł w~oddechu zapach ryby, którą zjadł na lunch. 

-- Dlaczego jesteś głupim chłopcem, Matthew? Nie wydawałeś się głupi, kiedy pracowałeś dla Boss Winga. Zawsze wydawałeś się mądrzejszy od tych dzieci. -- Machnął lekceważąco ręką na chłopców. -- Ale Boss Wing, szkolił cię, zapewniał schronienie, karmił, płacił\ldots  czy uważasz, że to honorowe czy sprawiedliwe, żebyś wziął całą tę inwestycję i~uciekł z~nią? 


-- Nie jesteśmy nic winni Boss Wing! -- krzyknął Lu. -- Myślisz, że możesz zmusić nas do pracy dla niego? 


Człowiek Bossa Winga potrząsnął głową. 

-- Co za pasjonat. Nikt nie chce cię do czegokolwiek zmuszać, dziecko. Po prostu uważamy, że nie jest w~porządku, żebyś wziął całe szkolenie i~inwestycje, które w~ciebie włożyliśmy, przebiegł przez ulicę i~założył konkurencyjny biznes. To nie w~porządku, a Boss Wing tego nie toleruje. 


Curry obróciło się w~żołądku Matthew. 

-- Mamy prawo rozpocząć własną działalność gospodarczą. -- Słowa były odważniejsze, niż się czuł, ale to byli \textit{jego }chłopcy i~dawali mu odwagę. -- Jeśli Boss Wing nie lubi konkurencji, niech znajdzie inną pracę. 


Człowiek Bossa Winga nie dał mu żadnego ostrzeżenia, zanim uderzył Matthew tak mocno, że jego głowa zadzwoniła jak gong. Cofnął się o dwa kroki, potknął się o pięty i~upadł na tyłek, lądując na brudnym chodniku. Człowiek Bossa Winga postawił stopę na jego klatce piersiowej i~spojrzał na niego. 


-- Chłopczyku, to tak nie działa. Oto umowa\ldots  Boss Wing rozumie, że jeśli nie chcesz pracować w~jego fabryce, w~porządku. Jest gotów sprzedać ci franczyzę, żebyś otworzył własną filię jego firmy. Wystarczy, że zapłacisz mu opłatę franczyzową w~wysokości 60 procent swoich zarobków brutto. Obserwowaliśmy sprzedaż złota w~Svartalfaheim. Możesz wykonywać tyle pracy, ile chcesz, a Boss Wing nawet zajmie się sprzedażą rzeczy za Ciebie, dzięki czemu będziesz mógł skoncentrować się na swojej pracy. A ponieważ to Twoja firma, możesz decydować o tym, jak podzielić pieniądze \ldots  możesz z~tego zapłacić sobie za wszystko, co chcesz. 


Matthew płonął ze wstydu. Wszyscy jego przyjaciele patrzyli na niego wytrzeszczonymi oczami, przestraszeni. Ciężar stopy na jego klatce piersiowej wzrastał, aż nie mógł zaczerpnąć tchu. 


W końcu wydyszał: 

-- \textit{Dobrze }-- i~ciśnienie zniknęło. Człowiek Bossa Winga wyciągnął rękę i~pomógł mu wstać. 


-- Rozsądnie -- powiedział. -- Wiedziałem, że jesteś rozsądnym chłopcem. -- Zwrócił się do przyjaciół Matthew. -- Wasz mały szef to rozsądny człowiek. Zabierze was w~różne miejsca. Teraz go słuchajcie. 


Potem bez słowa odwrócił się na pięcie i~odszedł, a jego ludzie poszli za nim. 


\bigskip
\threeast


Ta scena jest poświęcona Anderson's Bookshops, legendarnej dziecięcej księgarni w~Chicago. Anderson's to stara, stara rodzinna firma, która zaczynała jako dawna drogeria sprzedająca niektóre książki na boku. Dziś jest to dynamicznie rozwijające się imperium książek dla dzieci w~wielu lokalizacjach, z~kilkoma niewiarygodnie innowacyjnymi praktykami sprzedaży książek, które łączą książki i~dzieci w~naprawdę ekscytujący sposób. Najlepsze z~nich to mobilne targi książek w~sklepie, na których wysyłają ogromne regały na kółkach, już zaopatrzone w~doskonałe książki dla dzieci, bezpośrednio do szkół na ciężarówkach -- voila, natychmiastowe targi książek! 


\href{https://en.wikipedia.org/wiki/HTTP_404}{\textit{Anderson's Bookshops,}: 123 West Jefferson, Naperville, IL 60540 USA +1 630 355 2665 }

\bigskip
\threeast

Samochód, który wjechał w~samochód ojca Wei-Dong, był prowadzony przez bardzo zirytowanego, bardzo zmęczonego Brytyjczyka, grubego i~łysego, z~dwójką wściekłych dzieci na tylnym siedzeniu i~zagniewaną żoną na przednim siedzeniu. 


Stale, cicho przeklinał po angielsku, co przypominało przeklinanie po amerykańsku, ale było w~nim dużo więcej ,,choler''. Przechadzał się chodnikiem obok rozbitego Huaweia, jego żona wołała go z~wnętrza samochodu, by wrócił do cholernego samochodu, Ronald, ale Ronald nic z~tego sobie nie robił. 


Wei-Dong siedział na wąskim pasie trawy między drogą a chodnikiem, oszołomiony południowym słońcem, czekając, aż jego wizja przestanie pływać. Benny usiadł obok niego, trzymając zwitek Kleenexu, żeby zatamować krwawienie ze złamanego nosa, który odbił od deski rozdzielczej. Wei-Dong uniósł ręce do czoła, by ponownie dotknąć guza tam. Jego ręce pachniały nowym plastikiem, zapachem poduszki powietrznej, z~której musiał się wydostać. 


Gruby mężczyzna przykucnął obok niego. 

-- Chryste, synu, wyglądasz, jakbyś był na wojnie. Ale wszystko będzie dobrze, prawda? Mogło być znacznie gorzej. 


-- Proszę pana -- powiedział Benny Rosenbaum cichym głosem stłumionym przez Kleenex. -- Proszę zostawić nas teraz w~spokoju. Kiedy przyjedzie policja, wszyscy będziemy mogli porozmawiać, dobrze?  


-- Oczywiście, oczywiście. -- Jego dzieci krzyczały teraz, krzyczały z~tylnego siedzenia o dostaniu się do Disneylandu, kiedy dojeżdżają do Disneylandu? 

-- Zamknijcie się, potwory -- ryknął. 

Ten dźwięk sprawił, że Wei-Dong się wzdrygnął. Zerwał się na nogi. 


-- Usiądź, Leonardzie -- powiedział jego ojciec. -- Nie powinieneś był wysiadać z~samochodu, a już na pewno nie powinieneś chodzić. Możesz mieć wstrząs mózgu lub uraz kręgosłupa. Usiądź -- powtórzył, ale Wei-Dong musiał zejść z~trawy., musiał rozejść mdłości w~żołądku. 


Oho. Ledwo dotarł do krawężnika, z~rękami zaciśniętymi na pogniecionej, łuszczącej się tylnej części Huawei, zanim zaczął rzygać, gejzer zużytego jedzenia, który wystrzelił prosto z~jego wnętrzności i~poleciał po całym wraku samochodu. Chwilę później ręce ojca były na jego ramionach, podtrzymując go. Ze złością otrząsnął się z~nich. 


Zbliżały się teraz syreny, a grubas rozmawiał intensywnie ze starym Bennym, choć było na tyle cicho, że Wei-Dong zdołał rozróżnić tylko kilka słów -- \textit{ubezpieczenie, wina, wakacje }-- a wszystko to pochlebnym tonem. Jego ojciec starał się wtrącić słowo, ale facet przegadywał go. Wei-Dong mógł mu powiedzieć, że to nie była dobra strategia. Nic nie było pewniejszego, żeby Wulkan Benny wybuchł. I oto przyszło. 


-- Zamknij na chwilę usta, dobrze? Po prostu ZAMKNIJ SIĘ.  


Krzyk był tak głośny, że nawet dzieci na tylnym siedzeniu zamilkły. 


-- UDERZYŁEŚ NAS, cholerny idioto! Nie zamierzamy płacić połowy szkód. Nie rozliczymy tego za gotówkę. Nie \textit{interesuje} mnie, że jesteś w~jet-lagu, nie \textit{dbam, }że nie kupiłeś dodatkowego ubezpieczenia twojego wypożyczonego samochodu, nie \textit{obchodzi }mnie, czy to zrujnuje ci wakacje. Mogłeś nas zabić, rozumiesz, kretynie? 


Mężczyzna uniósł ręce i~skulił się za nimi. 

-- Byłeś zaparkowany na środku drogi, kolego -- powiedział błagalną nutą w~głosie. 


Wszyscy ich obserwowali, dzieci i~żona faceta, ciekawscy, którzy zwolnili, żeby zobaczyć wypadek. Obaj mężczyźni byli całkowicie skupieni na sobie. 


Innymi słowy, nikt nie pilnował Wei-Donga. 


Pomyślał o dźwięku, który wydała słuchawka, zgrzytając pod metalowym butem ojca, usłyszał zbliżające się syreny i\ldots  


\ldots poszedł. 


Prześlizgnął się w~stronę krzaków otaczających mini-centrum handlowe i~stację benzynową, nonszalancko, ściskając tornister, jakby dopiero się orientował, ale kierował się w~tamtą szczelinę, wąską, przez którą ledwo zdołał się przecisnąć. Wpadł na parking wokół mini-centrum handlowego, wypełnionego sklepami sprzedającymi koszulki za 3 dolary, śnieżne kule i~duże butelki filtrowanej wody. Po tej stronie krzewów świat był normalny i~zajęty, wypełniony turystami w~drodze do lub z~Disneylandu. 


Przyspieszył kroku, odwracając twarz od sklepów i~kamer CCTV na zewnątrz. Pomacał w~kieszeni, wyczuł te kilka dolarów. Musiał uciekać, daleko, szybko, jeśli w~ogóle miał uciec. 


I tam było jego zbawienie, autobus turystyczny, który przejeżdżał ulicami Anaheim Resort District, przewożąc ludzi od hoteli do restauracji do parków, pełen przesłodzonych dzieciaków i~uczestników konwencji z~odznakami na szyjach, i~jechał na przystanek zaledwie kilka metrów dalej. Zerwał się do biegu, potknął się z~bólu, który przeszył mu głowę jak błyskawica, a potem zdecydował się iść tak szybko, jak tylko mógł. Syreny były teraz bardzo, bardzo głośne, po drugiej stronie krzaków, a on był już prawie w~autobusie, a tam był głos jego ojca, wołającego jego imię, a tutaj był autobus i\ldots  


jego stopa opadła na najniższy stopień, tylna stopa podniosła się, by do niej dołączyć, a niecierpliwy kierowca zamknął za nim drzwi i~z wielkim westchnieniem zwolnił hamulec i~autobus ruszył do przodu. 


-- Wei-Dong Rosenbaum -- szepnął do siebie, -- właśnie uciekłeś z~porwania przez rodziców do szkoły wojskowej, co teraz zrobisz? -- Wyszczerzył zęby. -- Jadę do Disneylandu! 


Autobus toczył się w~dół Katelli, kierując się do wejścia autobusowego, a potem wypluł ładunek rozgorączkowanych turystów. Wei-Dong mieszał się z~nimi, niewidoczny w~masie ludzkości przeskakującej obok ogromnych słupów drogowych w~podstawowym kolorze. Był na autopilocie, pozostał na autopilocie, gdy odpiął tornister, żeby znudzony bandyta z~ochrony przejrzał go. 


Miał roczną przepustkę do Disneylandu, odkąd był na tyle duży, że mógł jeździć autobusem. Wszystkie dzieci, które znał, też je miały -- chodzenie do centrum handlowego po szkole nie wystarczyło, i~chociaż po jakimś czasie robiło się to nudne, nie mógł wymyślić lepszego miejsca, w~którym mógłby się zniknąć, zastanawiając się nad kolejnymi krokami. 


Ruszył Main Street, kierując się do małego, różowego zamku na końcu drogi. Wiedział, że na chodnikach wokół zamku znajdują się odosobnione ławki, miejsca, w~których mógł usiąść i~chwilę pomyśleć. Czuł, jakby jego głowa była pełna waty cukrowej. 


Pierwszą rzeczą, którą zrobił po znalezieniu ławki, było sprawdzenie telefonu. Dzwonek był wyłączony -- regulamin szkolny -- ale czuł, jak ciągle wibruje w~jego kieszeni. Piętnaście nieodebranych telefonów od ojca. Wybrał pocztę głosową i~słuchał przemowy taty o powrocie \textit{teraz już} i~wszystkich okropnych rzeczach, które by mu się przydarzyły, gdyby tego nie zrobił. 


-- Dzieciaku, cokolwiek myślisz, że robisz, mylisz się. W końcu wrócisz do domu. Im szybciej do mnie oddzwonisz, tym mniej będziemy mieć kłopotów. A im dłużej czekasz\ldots  \textit{posłuchaj tego Leonardzie }\ldots  im dłużej czekasz, \textit{tym będzie gorzej}. Są gorsze rzeczy niż szkoła z~internatem, dzieciaku. Dużo, dużo gorsze. 


Wpatrywał się bezmyślnie w~niebo, słuchając tego, a potem upuścił telefon, jakby został oparzony. 


\textit{Miał GPS}. Używali telefonów do znalezienia uciekinierów, złych gości i~zgubionych turystów. Podniósł telefon z~chodnika, zdjął obudowę, wyjął baterię, a potem włożył baterię do kieszeni kurtki, a telefon do dżinsów. Nie był zbyt dobrym uciekinierem. 


Kiedy odszedł, policja była w~drodze na wypadek. Przybyli kilka minut później. Stary zdecydował, że uciekł, więc powie to glinom. Był nieletni, wagarował, miał wypadek samochodowy i~do diabła, nie oszukujmy się, jego rodzina była bogata. Oznaczało to, że policja zwróci uwagę na jego ojca, co oznacza, że zrobią wszystko, co w~ich mocy, aby go zlokalizować. Gdyby jeszcze nie dowiedzieli się, gdzie jest jego telefon, szybko wiedzieliby, przejrzeliby logi i~znaleźli połączenie z~Disneylandu na jego pocztę głosową. 


Ruszył, przepychając się przez tłum, kierując się z~powrotem w~górę Main Street. Schylił się za budką fryzjerską i~zdał sobie sprawę, że stoi przed bankomatem. W każdej chwili zablokowaliby jego kartę -- albo, jeśli byli sprytni, pozostawiliby kartę aktywną i~wykorzystali ją do wyśledzenia go. Potrzebował gotówki. Poczekał, aż dwóch niemieckich turystów pogrzebało w~maszynie, a następnie włożył do niej swoją kartę i~wypłacił 500 dolarów, czyli najwięcej, jakie wydałaby maszyna. Spróbował ponownie za kolejne 500 dolarów, teraz skrępowany grubym na dwa centymetry zwitkiem dwudziestek w~dłoni. Próbował trzeciej wypłaty, ale automat powiedział mu, że osiągnął swój dzienny limit. W każdym razie nie sądził, że ma w~banku dużo więcej niż 1000 dolarów -- to były pieniądze z~kilku lat urodzin,plus nieco za letnią pracę w~chińskim warsztacie PC przy mini-galerii handlowej w~Irvine. 


Złożył zwitek, wsadził go do kieszeni i~wyszedł z~parku, nie zawracając sobie głowy stemplem dłoni. Zaczął iść na ulicę, ale potem odwrócił się na pięcie i~skierował w~stronę kompleksu handlowego Downtown Disney i~przylegających do niego hoteli. Stały tam tanie autobusy wycieczkowe, które jeździły stamtąd do LA, do San Diego, na wszystkie lotniska. Nie było prostszego, tańszego sposobu na dostanie się stąd daleko. 


Hol Grand Californian Hotel wzniósł się na niewyobrażalne wysokości, gigantyczne belki przecinały przepastną przestrzeń. Wei-Dong zawsze lubił to miejsce. Zawsze wydawała się tak \textit{wyrenderowana}, jak wyimaginowane miejsce, z~misternymi marmurowymi intarsjami na podłodze, wysokimi na trzy metry witrażami osadzonymi w~przesuwanych drzwiach i~haftowaną tapicerką sof. Teraz jednak chciał tylko przez to przejść i~wsiąść do autobusu, żeby\ldots  


Gdzie? 


Gdziekolwiek. 


Nie wiedział, co będzie dalej robić, ale wiedział jedno, że nie zostanie odesłany do jakiejś szkoły za wpadki, wykopany z~Internetu, wyrzucony z~gier. Jego ojciec nie pozwoliłby nikomu \textit{mu }tego zrobić, bez względu na to, jakie miał problemy. Stary nigdy nie dałby się tak popychać i~potrząsać. 


Jego matka by się martwiła \ldots  ale zawsze się martwiła, prawda? Wysyłałby jej e-maile, gdy już by gdzieś dotarł, e-maile codziennie, dając jej znać, że wszystko w~porządku. Była dla niego dobra. Cholera, stary był dla niego dobry, jeżeli o to chodzi. Przeważnie. Ale teraz miał siedemnaście lat, nie był dzieckiem, nie był zepsutą zabawką, którą trzeba odesłać do producenta. 


Mężczyzna na recepcji nawet nie mrugnął okiem, gdy Wei-Dong zapytał o rozkład jazdy na lotnisko, po prostu go podał. Wei-Dong usiadł w~najciemniejszym kącie przy kamiennym kominku, najbardziej niepozornym miejscu w~całym hotelu. Zaczynał wpadać w~paranoję, mógł rozpoznać to uczucie, ale nie pomogło mu to uspokoić, gdy podskakiwał i~patrzył na wszystkich gliniarzy Disneya, którzy przechadzali się po holu, bez wątpienia wyglądał na winnego jak masowy morderca. 


Następny autobus jechał do LAX, a następny na lotnisko w~Santa Monica. Wei-Dong zdecydował, że LAX to właściwe miejsce. Nie, żeby mógł wsiąść do samolotu, gdyby jego tata wezwał policję, był pewien, że mieliby jakiś ślad w~okienkach sprzedaży biletów. Nie wiedział dokładnie, jak to działa, ale rozumiał, jak działają wąskie gardła dzięki graniu. W tej chwili mógł być w~dowolnym miejscu w~LA, co oznaczało, że musieliby poświęcić gigantyczną ilość wysiłku, aby go znaleźć. Ale gdyby próbował wylecieć samolotem, byłoby znacznie mniej miejsc, które musieliby sprawdzić, żeby go złapać -- stanowiska linii lotniczych na czterech lub pięciu lotniskach w~mieście -- i~to było o wiele bardziej praktyczne. 


Ale LAX miał też tanie autobusy do \textit{każdego miejsca }w LA, autobusy, które jeździły do każdego hotelu i~okolicy. Na pewno zajęłoby to dużo czasu -- półtorej godziny z~Disneylandu do LAX, kolejna godzina lub dwie, żeby wrócić do LA, ale to było w~porządku. Potrzebował czasu \ldots  czasu, by wymyślić, co będzie dalej robić. 


Bo kiedy był ze sobą całkowicie szczery, musiał przyznać, że nie miał żadnego cholernego pomysłu. 


\bigskip
\threeast


Ta scena jest poświęcona University Bookstore na Uniwersytecie Waszyngtońskim, której sekcja science fiction rywalizuje z~wieloma specjalistycznymi sklepami, dzięki bystremu, oddanemu kupcowi science fiction, Duane Wilkinsowi. Duane jest prawdziwym fanem science fiction -- po raz pierwszy spotkałem go na konwencji World Science Fiction w~Toronto w~2003 roku -- i~widać to w~eklektycznych i~świadomych wyborach wystawianych w~sklepie. Jednym z~wielkich czynników prognostycznych dobrej księgarni jest jakość ,,recenzji półek'' -- te małe kawałki tektury przyklejone do półek z~(zazwyczaj ręcznie pisanymi) recenzjami personelu wychwalającymi zalety książek, których w~innym przypadku można by nie zauważyć. Pracownicy księgarni uniwersyteckiej wyraźnie skorzystali z~opieki Duane'a, ponieważ recenzje półek w~księgarni uniwersyteckiej nie mają sobie równych. 


\href{https://en.wikipedia.org/wiki/HTTP_404}{\textit{University Bookstore } 4326 University Way NE, Seattle, WA 98105 USA +1 800 335 READ }

\bigskip
\threeast

Mala obudziła się wcześnie, po niespokojnym śnie. W wiosce często wstawała rano i~słuchała ptaków. Ale nie było śpiewu ptaków, gdy mrugała rozespana, tylko podejrzliwość Dharavi -- samochody, szczury, ludzie, odległe odgłosy fabryk, kozy. Kogut. Cóż, to był rodzaj ptaka. Na jej ustach pojawił się lekki uśmiech i~poczuła się trochę lepiej. 


Niewiele. Usiadła i~przetarła oczy, wyciągnęła ramiona. Gopal nadal spał, chrapiąc cicho, leżąc na brzuchu, tak jak wtedy, gdy był niemowlęciem. Potrzebowała toalety, a ponieważ było jasno, zdecydowała, że wyjdzie do wspólnej toalety trochę dalej, zamiast korzystać z~przykrytego wiadra w~pokoju. W wiosce mieli porządną latrynę, głęboko wykopaną, a przy niej garnek z~czystą wodą, który kobiety cały czas napełniały. Tutaj, w~Dharavi, wspólna toaleta była o wiele bardziej zamkniętym, cuchnącym miejscem, nigdy bardzo czystym. Szanowane rodziny w~Dharavi miały własne prywatne toalety, więc z~publicznych korzystali tylko nowo przybyli. 


Dziś rano nie było tak źle. Były panie, które wstawały nawet wcześniej niż ona, żeby spłukać ją wodą przyniesioną z~pobliskiego kranu. O zmroku smród będzie łzawił w~oczach. 


Włóczyła się po ulicy przed domem. Nie było jeszcze zbyt gorąco, ani zbyt tłoczno, ani zbyt głośno. Chciałaby, żeby tak było. Może hałas i~tłumy zagłuszą zmartwienia przemykające jej przez głowę. Może upał to wypali. 


Przyniosła ze sobą komórkę. Grała powiadomieniami o nowych rzeczach, za oglądanie których mogłaby zapłacić -- serialach, rysunkach i~politycznych wiadomościach, wysyłanych w~nocy. Odrzuciła je niecierpliwie i~przewinęła książkę adresową, zatrzymując się przy nazwisku pana Banerjee i~wpatrując się w~nie. Jej palec uniósł się nad przyciskiem wysyłania. 


Za wcześnie, pomyślała. Spałby. Ale nigdy nie spał, prawda? Pan Banerjee wydawał się nie spać o każdej porze, wysyłając jej wiadomości z~nowymi celami, do których ma zabrać swoją armię. Nie śpi. Nie spałby całą noc, rozmawiając z~panią Dibyendu. 


Jej palec unosił się nad przyciskiem \textit{Wyślij}. 


Telefon zadzwonił. 


Prawie upuściła go ze zdziwienia, ale zdołała ułożyć go w~dłoni, wyłączyć dzwonek i~spojrzeć na twarz. Pan Banerjee, oczywiście, jakby został wyczarowany w~telefonie przez jej myśli i~niepokój. 


-- Halo? -- powiedziała. 


-- Mala -- powiedział. Brzmiał poważnie. 


-- Pan Banerjee. -- Wyszło piskliwie. 


Nie powiedział nic więcej. Znała tę sztuczkę. Używała go w~swojej armii, zwłaszcza na chłopcach. Cisza tworzyła balon w~głowie przeciwnika, taki, który puchł, żeby wypełnić głowę, aż zaczynał odbijać echo lękami i~wątpliwościami. Sztuczka działała bardzo dobrze. Działała, nawet gdy wiedziała, jak ma działać. Działała dobrze na nią. 


Przygryzła wargę. W przeciwnym razie mogłoby się jej coś wyrwać, może \textit{on chciał mnie skrzywdzić} lub \textit{zbierało mu się} albo \textit{nie zrobiłam niczego złego}. 


\textit{Lub} jestem wojowniczką i~się nie wstydzę \ldots


\textit{Tam}  pojawiła się myśl, która chciała uciec i~się schować za \textit{on zamierzał mnie skrzywdzić}, to była myśl, której potrzebowała, pluton, który musiała wystawić na front. Uporządkowała myśl, zmieniła ją w~uporządkowaną linię bojową i~poprowadziła. 


-- Wczoraj wieczorem ten idiota pani Dibyendu próbował mnie zaatakować, na wypadek, gdybyś nie słyszał. -- Odczekała chwilę. -- Nie pozwoliłam mu. Nie sądzę, żeby znów tego próbował. 


Na linii telefonicznej rozległo się bardzo słabe parsknięcie. Tłumiony śmiech? Ledwo powstrzymywany gniew? 

-- Słyszałem o tym, Mala. Chłopiec jest w~szpitalu. 


-- Dobrze -- powiedziała, zanim zdążyła się powstrzymać. 


-- Jedno z~jego żeber złamało się i~przebiło płuco. Ale mówią, że przeżyje. Mimo to było dość blisko. 


Poczuła się chora. Czemu? Dlaczego tak musiało być? Dlaczego nie mógł jej zostawić w~spokoju? 

-- Cieszę się, że przeżyje. 

-- Pani Dibyendu zadzwoniła do mnie w~nocy, żeby powiedzieć, że zaatakowano jedynego syna jej siostry. Że został zaatakowany przez okrutny gang Twoich przyjaciół. Waszą ,,armię''. 


Teraz \textit{ona }prychnęła.

-- Mówi to, bo wstydzi się, że został przeze mnie tak mocno pobity, tylko mnie, tylko dziewczynę. 


\textit{Ponownie cisza wypełniła rozmowę. }Czekał, żebym powiedziała, że mi przykro, że jakoś mu to wynagrodzę, że może zapłacić za to z~moich zarobków. \textit{Przełknęła. }Nie zrobię tego. Ten idiota sprawił, że go zaatakowałam, a on zasłużył na to, co dostał.


-- Pani Dibyendu -- zaczął, po czym przerwał. -- Są wydatki, które wynikają z~czegoś takiego, Mala. Wszystko ma swoją cenę. Wiesz o tym. Gra w~kawiarni pani Dibyendu kosztuje. Kosztuje mnie, gdy robisz coś takiego. Cóż, to też ma swój koszt. 


\textit{Teraz nadeszła jej kolej, by być cicho i~myśleć o nim tak mocno, jak tylko potrafi,} O tak, cóż, chyba już wyegzekwowałem zapłatę od idiotycznego siostrzeńca. Myślę, że pokrył koszty.


-- Słuchasz mnie?  


Mruknęła z~aprobatą, nie ufając sobie, gdy otworzy usta. 


-- Dobrze. Posłuchaj uważnie. W przyszłym miesiącu pracujesz dla \textit{mnie}. Każda rupia jest moja i~sprawiam, że ta zła rzecz, którą sobie sprowadziłaś, znika.  


Odsunęła telefon od głowy, jakby rozgrzał się do czerwoności i~ją spalił. Wpatrywała się w~ekran. Z bardzo daleka pan Banerjee powiedział: 

-- \textit{Mala}?\textit{Mala}? 

Przyłożyła telefon z~powrotem do ucha. Teraz ciężko oddychała. 

-- To niemożliwe -- powiedziała, starając się zachować spokój. -- Armia nie będzie walczyć bez wynagrodzenia. Moja matka nie może żyć bez mojego wynagrodzenia. Stracimy dom. Nie -- powtórzyła -- to niemożliwe. 


-- Niemożliwe? Mala, lepiej, żeby było możliwe. Niezależnie od tego, czy pracujesz dla mnie, czy nie, będę musiał to załatwić z~panią Dibyendu. To mój obowiązek, jako twojego pracodawcy. A to będzie kosztować pieniądze. Zaciągnęłaś dług, który muszę za ciebie uregulować, a to oznacza, że musisz być przygotowany na rozliczenie ze \textit{mną}. 


-- W takim razie nie załatwiaj tego -- powiedziała. -- Nie dawaj jej ani jednej rupii. Są inne miejsca, w~których możemy zagrać. Jej siostrzeniec sam to sprowadził na siebie. Możemy zagrać gdzie indziej. 


-- Mala, czy ktoś \textit{widział}, jak ten chłopak kładł na tobie ręce? 


-- Nie -- odpowiedziała. -- Poczekał, aż będziemy sami. 


-- A dlaczego byłaś z~nim sama? Gdzie była twoja armia? 


-- Już poszli do domu. Zostałam do późna. -- Pomyślała o Big Sister Nor i~jej metamechu, o Związku. Pan Banerjee byłby jeszcze bardziej zagniewany, gdyby powiedziała mu o Big Sister Nor. 

-- Studiowałam taktykę -- powiedziała. -- Praktykuję na własną rękę. 


-- Zostałaś sama z~tym chłopcem w~środku nocy. Co się naprawdę stało, Mala? Chciałaś zobaczyć, jak to jest całować go jak gwiazdę fillum, a potem wymknęło się to spod kontroli? Czy to się stało? 


-- \textit{Nie!} -- krzyknęła tak głośno, że usłyszała ludzi jęczących w~łóżkach, wołających sennie zza otwartych okien. -- Zostałam do późna, żeby ćwiczyć, próbował mnie powstrzymać. Powaliłam go, a on mnie gonił. Powaliłam go, a potem nauczyłam go, dlaczego nie powinien mnie próbować złapać. 


-- Mala -- powiedział, a ona pomyślała, że stara się teraz brzmieć po ojcowsku, surowo, staro i~męsko. -- Powinnaś była wiedzieć lepiej i~nie stawiać się w~takiej sytuacji. Generał wie, że wygrywasz niektóre walki, w~ogóle się w~nie nie angażując. Cóż, nie jestem nierozsądnym człowiekiem. Oczywiście ty, twoja matka i~twoja armia wszyscy potrzebują moich pieniędzy, jeśli zamierzasz dalej walczyć. Możesz pożyczyć ode mnie wypłatę w~tym miesiącu, coś, czym zapłacę wszystkim, a potem możesz spłacać je stopniowo, w~ciągu mniej więcej przyszłego roku. Wezmę pięć na dwadzieścia rupii przez 12 miesięcy, a będziemy kwita. 


To była nadzieja, straszna, okropna nadzieja. Szansa na utrzymanie armii, mieszkania, szacunku. Kosztowało ją to tylko jedną czwartą jej zarobków. Zostałyby jej trzy czwarte. Trzy czwarte było lepsze niż nic. To było lepsze niż powiedzenie Mamaji, że to już koniec. 


-- Tak -- powiedziała. -- W porządku, dobrze. Ale nie gramy już w~kawiarni pani Dibyendu. 


-- O nie -- powiedział. -- Nie chcę tego słyszeć. Pani Dibyendu ucieszy się, gdy wrócisz. Oczywiście będziesz musiała ją przeprosić. Możesz przynieść jej pieniądze dla jej siostrzeńca. Dzięki temu poczuje się lepiej, jestem pewien, i~to uleczy wszelkie rany w~waszej przyjaźni. 


-- Czemu? -- Na jej policzkach pojawiły się łzy. -- Dlaczego nie pozwolisz nam pójść gdzie indziej? Dlaczego to ma znaczenie? 


-- Ponieważ, Mala, ja jestem szefem, a ty jesteś robotnikiem i~to jest fabryka, w~której pracujesz. Dlatego. -- Jego głos był teraz twardy, cały powiew fałszywej troski zniknął, pozostawiając po sobie zgrzytanie jak kamień o kamień. 


Chciała rozłączyć się z~nim, tak jak robili to w~filmach, kiedy mieli swoje gigantyczne, krzyczące awantury, i~wrzucali telefony do studni lub rozbijali je o ścianę. Ale nie mogła sobie pozwolić na zniszczenie telefonu i~nie mogła pozwolić sobie na rozgniewanie pana Banerjee. Powiedziała więc: 

-- W porządku -- cichym głosikiem, który brzmiał jak mysz starająca się nie zostać zauważona. 


-- Dobra dziewczyna, Mala. Mądra dziewczyna. Teraz mam dla ciebie następną misję. Jesteś gotowa? 


Odrętwiała zapamiętała szczegóły misji, kogo miała zabić i~gdzie. Pomyślała, że jeśli wykona tę pracę szybko, może poprosić go o kolejną, a potem jeszcze jedną \ldots  pracować dłużej, szybciej spłacać dług. 


-- Mądra dziewczyna, dobra dziewczynka -- powtórzył, kiedy powtórzyła mu szczegóły, a potem odłożył słuchawkę. 


Schowała telefon do kieszeni. Wokół niej budziło się Dharavi, mijając nią, jakby była skałą w~rzece, przeciskając się obok niej z~obu stron. Mężczyźni z~łopatami i~taczkami, chłopcy z~ogromnymi workami ryżu na każdym ramieniu, wypełnionymi brudnymi plastikowymi butelkami w~drodze do jakiejś sortowni, mężczyzna z~długą brodą, w~jarmułce kufi i~zwisającej do kolan koszuli kurty prowadzącej kozę z~kawałek liny. Trzy kobiety w~sari, z~rozciągniętymi brzuchami i~prążkowanymi śladami po urodzeniu dzieci, niosące ciężkie wiadra wody z~gminnego kranu. W powietrzu unosił się zapach gotowania, skwierczenie dhalu na grillu i~pachnący zapach herbaty. Przeszedł obok niej chłopak, młodszy od Gopala, w~trzepoczących sandałach i~krótkich spodniach, i~splunął pod jej stopy snopem mdlącego, słodkiego betelu. 


Zapach przypomniał jej, gdzie jest, co się wydarzyło i~co musi teraz zrobić. 


Minęła rodzinę Dasów na parterze i~wdrapała się po schodach do ich mieszkania. Mamaji i~Gopal nie spali i~się krzątali. Mamaji przyniosła wodę i~przygotowywała śniadanie nad palnikiem propanowym, a Gopal miał na sobie koszulę szkolnego mundurka i~spodnie do kolan. Szkoła Dharavi, do której uczęszczał, trwała pół dnia, co dało mu trochę czasu na zabawę i~odrabianie lekcji, a potem jeszcze kilka godzin na pracę u boku Mamaji w~fabryce. 


-- Gdzie byłaś? -- spytała Mamaji. 


-- Telefon -- powiedziała, poklepując małą kieszonkę wszytą w~jej tunikę. -- Z panem Banerjee. -- Poruszyła brodą z~boku na bok, mówiąc, że \textit{Biznes}. 


-- Co powiedział? -- Głos Mamaji był cichy i~pełen fałszywej nonszalancji. 


Mamaji nie musiała wiedzieć, co zaszło między panem Banerjee a nią. Mala była generałem i~potrafiła zarządzać własnymi sprawami. 


-- Powiedział, że wszystko zostało wybaczone. Chłopiec na to zasłużył. Poradzi sobie z~panią Dibyendu i~będzie dobrze. -- Znowu poruszyła brodą z~boku na bok \textit{wszystko w~porządku. Zajęłam się tym.} 


Mamaji wpatrywała się w~patelnię i~skwierczące w~niej jedzenie, i~kiwnęła głową do siebie. Chociaż nic nie widziała, Mala skinęła głową. Była generałem Robotwallah i~potrafiła sprawić, by wszystko było dobrze. 

\bigskip
\threeast

Ta scena jest poświęcona Forbidden Planet, brytyjskiej sieci sklepów z~książkami, komiksami, zabawkami i~filmami science fiction i~fantasy. Forbidden Planet ma sklepy w~całej Wielkiej Brytanii, a także placówki sportowe na Manhattanie i~Dublinie w~Irlandii. Postawienie stopy na Forbidden Planet jest niebezpieczne -- rzadko udaje mi się uciec z~nienaruszonym portfelem. Forbidden Planet naprawdę przewodzi grupie, która zbliża gigantyczną publiczność telewizyjną i~filmową science fiction do książek science fiction -- coś, co jest absolutnie kluczowe dla przyszłości tej dziedziny. 


\href{https://forbiddenplanet.co.uk/}{Forbidden Planet, Wielka Brytania, Dublin i~Nowy Jork} 

\bigskip
\threeast

Wei-Dong był kiedyś w~centrum Los Angeles, na wycieczce klasowej do Disney Concert Hall, ale wtedy wjechali do środka, zaparkowali i~maszerowali jak kaczątko do sali, a potem z~powrotem, nie spędzając czasu na włóczeniu się po okolicy. Pamiętał, jak z~okna autobusu obserwował mijające ulice, wyblakłe witryny sklepowe i~wolno poruszających się ludzi, kasy czekowe i~sklepy monopolowe. I kafejki internetowe. Mnóstwo kafejek internetowych, zwłaszcza w~Koreatown, gdzie każdy pasaż handlowy miał jaskrawy szyld reklamujący ,,PC Baang'' -- po koreańsku na kafejkę internetową. 


Ale nie wiedział dokładnie, gdzie jest Koreatown, i~potrzebował kafejki internetowej, żeby to wygooglować, więc złapał autobus LAX do sali koncertowej Disneya, myśląc, że mógłby prześledzić trasę autobusu i~znaleźć drogę do tych sklepów, połączyć się z~internetem, porozmawiać z~jego ziomkami w~Kantonie, wymyślić następny krok. 


Ale Koreatown okazało się trudniejsze do znalezienia i~dalej, niż sądził. Zapytał kierowcę autobusu o drogę, który spojrzał na niego jak na wariata i~wskazał w~dół wzgórza. I tak zaczął schodzić i~chodzić, i~mijać blok za zakurzonym blokiem. Z okna szkolnego autobusu centrum LA wyglądało na wolne i~wyblakłe jak zbyt długie zdjęcie pozostawione w~oknie. 


Na pieszo było szalone, ruch autobusów, bezdomni przechodzący, jeżdżący na kółkach lub kuśtykający obok niego, proszący o pieniądze. W przedniej kieszeni dżinsów miał 1000 dolarów i~wydawało mu się, że wybrzuszenie musi być tak oczywiste, jak erekcja przy tablicy w~klasie. Pocił się nie tylko z~powodu upału, który wydawał się o dziesięć stopni cieplejszy niż w~Disneylandzie. 


A teraz nie był w~pobliżu Koreatown, ale raczej trafił na Santee Alley, ogromny piracki targ na świeżym powietrzu w~centrum LA. Słyszał o tym miejscu już wcześniej, widywał je cały czas w~wiadomościach o popiersiach podrabianych towarów, zdjęciach wyprowadzanych Meksykan, podczas gdy posępnie usatysfakcjonowani gliniarze w~garniturach lub mundurach zwijali góry podrabianych koszul, podrabianych płyt DVD, podrabianych dżinsy, fałszywe gry. 


Aleja Santee była mile widzianą ulgą od otaczających ją ulic. Wszedł w~głąb rynku, wszystkie witryny sklepowe wykrzykiwały w~jego stronę technobregą i~reggatonem, a domokrążcy wykrzykiwali nazwy swoich towarów. To było jak prawdziwy rynek, na którym opierały się wszystkie setki rynków w~grach, które odwiedził, i~stwierdził, że zwalnia i~patrzy na ubrania gangsterów, kiepskie pamiątki i~fałszywą elektronikę. Kupił ze straganu duży kubek napoju arbuzowego i~kilka empanad, ostrożnie wyciągając z~kieszeni jedną dwudziestkę, nie wyciągając całej rzeczy. 


Potem znalazł kafejkę internetową pełną Gwatemalczyków rozmawiających z~rodzinami w~domu, noszących śliskie i~maleńkie słuchawki. Dziewczyna za ladą -- niewiele starsza od niego -- sprzedała mu jedną, która podawała się za Samsunga za 18 dolarów, a następnie wypożyczyła mu komputer, z~którym mógł go używać. Fałszywa słuchawka pasowała tak samo jak jego prawdziwa, chociaż miała szorstki szew plastiku biegnący wzdłuż jej długości, podczas gdy jego była gładka jak szkło plażowe. 


Ale to nie miało znaczenia. Miał połączenie sieciowe, słuchawki i~swoją grę. Czego więcej mógł potrzebować? 


Cóż, na początek jego ekipa. Nigdzie ich nie było. Spojrzał na zegarek i~nacisnął przycisk, który przestawił go na chińską strefę czasową. 5 rano. Cóż, to było wyjaśnienie. 


Sprawdził swój inwentarz, sprawdził bank gildii. Nie był w~stanie biegać po trupach po tym, jak został wyrwany z~gry przez swojego ojca i~policję myśli Liceum Johna Wayne'a, więc nie spodziewał się, że nadal będzie miał swoje vorpalne ostrze, ale tak zrobił, co oznaczało, że jeden z~gangu uratował go dla niego, co było strasznie rozważne. Ale w~końcu to właśnie robiły dla siebie towarzysze z~gildii. 


Zbliżała się pora obiadowa na wschodnim wybrzeżu, co oznaczało, że Savage Wonderland zaczynał wypełniać się ludźmi wracającymi do domu z~pracy. Pomyślał o czarnych jeźdźcach, którzy wymordowali ich tego ranka, i~zastanawiał się, kim byli. Było wielu ludzi, którzy polowali na farmerów złota, albo dlatego, że pracowali dla gry lub dla konkurencyjnego klanu farmy złota, albo dlatego, że byli znudzonymi bogatymi graczami, którzy nienawidzili idei, że biedni ludzie najeżdżają ,,ich'' przestrzeń i~pracują tam, gdzie oni grali. 


Wiedział, że powinien przejść do swojego e-maila i~sprawdzić wiadomości od rodziców. Nie lubił używać poczty elektronicznej, ale jego rodzice byli od niej uzależnieni. Bez wątpienia teraz wariowali, wzywając armię, marynarkę wojenną i~gwardię narodową, by znalazły ich krnąbrnego syna. Cóż, mogli wariować, ile tylko chcieli. Nie zamierzał wracać i~nie musiał wracać. 


Miał w~kieszeni 1000 dolarów, miał prawie 17 lat i~było wiele sposobów na przetrwanie w~wielkim mieście, które nie wymagały sprzedaży narkotyków ani własnego ciała. Jego gildie mu to pokazały. Wszystko, czego potrzebowałeś, aby zarobić na życie, to połączenie z~siecią i~mózg w~głowie. Rozejrzał się po kawiarni na dziesiątki Gwatemalczyków rozmawiających z~domem na swoich słuchawkach, wielu niewiele starszych od niego. Jeśli mogliby zarabiać na życie -- nie mówiąc językiem, nielegalni w~pracy, bez formalnej edukacji, prawie nie mieli pojęcia, jak korzystać z~technologii poza odrobiną wiedzy niezbędnej do taniego dzwonienia do domu -- to z~pewnością i~on mógłby. Jego dziadek przyjechał do Ameryki i~znalazł pracę, gdy był w~wieku Wei-Donga. Właściwie to była rodzinna tradycja. 


Nie chodziło o to, że nie kochał swoich rodziców. Kochał. Byli dobrymi ludźmi. Kochali go na swój sposób. Ale żyli w~bańce nierzeczywistości, bańce zwanej Orange County, gdzie wciąż mieli rzędy schludnych identycznych domów i~schludne identyczne życie, podczas gdy wokół nich wszystko się waliło. Jego ojciec nie mógł tego zobaczyć, chociaż prawie nie było dnia, żeby nie wracał do domu i~gorzko narzekał na kontenery, które spadły z~jego statku podczas kolejnej potwornej burzy, na cenę oleju napędowego płynącego w~stratosferę, o spadającym dolarze, rosnącym w~górę renminbi i~coraz bardziej zaciskających pasach Amerykanów, których zamówienia na towary z~południowych Chin uderzały w~jego interesy. 


Wei-Dong zrozumiał to wszystko, ponieważ zwracał uwagę i~widział rzeczy takimi, jakie były. Ponieważ rozmawiał z~Chinami, a Chiny mu odpowiadały. Gruby i~wygodny świat, w~którym dorastał, nie był trwały; wydrapany w~piasku, a nie wyrzeźbiony w~kamieniu. Jego przyjaciele w~Chinach widzieli to lepiej niż ktokolwiek inny. Lu pracował jako ochroniarz w~fabryce w~Shilong New Town, mieście, które produkowało urządzenia na sprzedaż w~Wielkiej Brytanii. Trochę czasu zajęło Wei-Dongowi zrozumienie tego: całe miasto, cztery miliony ludzi, robiło tylko urządzenia na sprzedaż w~Wielkiej Brytanii, kraju z~osiemdziesięcioma milionami ludzi. 


Aż pewnego dnia fabryki po obu stronach Lu's zostały zamknięte. Wszyscy produkowali towary dla kilku różnych firm, zatrudniając armie młodych kobiet do obsługi maszyn i~składania części, które z~nich wyszły. Młode kobiety zawsze dostawały najlepszą pracę. Szefowie lubili je, ponieważ ciężko pracowały i~nie kłóciły się za bardzo, przynajmniej tak wszyscy mówili. Kiedy Lu opuścił swoją wioskę w~prowincji Syczuan, aby udać się do południowych Chin, rozmawiał z~jedną z~dziewcząt, które wróciły z~fabryk na Święto Środka Jesieni, dziewczyną, która wyjechała kilka lat wcześniej i~znalazła bogactwo w~Dongguan, która za swoje pieniądze kupiła rodzicom piękny, nowy dwupiętrowy dom, która co roku na Festiwal wracała do domu w~eleganckich ubraniach z~nowym telefonem komórkowym w~designerskiej torbie, wyglądając jak kosmita lub modelka, która wyszła świeżo z~reklamy w~prasie. 


-- Jeśli idziesz do fabryki i~nie ma tam pełno młodych dziewcząt, nie bierz tam pracy -- radziła. -- Każde miejsce, które nie może przyciągnąć wielu młodych dziewcząt, jest coś nie tak.

 Ale fabryka, w~której pracował Lu -- wszystkie fabryki w~Shilong New Town -- były wypełnione młodymi dziewczynami. Jedyne zawody dla mężczyzn to kierowcy, ochroniarze, sprzątaczki i~kucharze. Fabryki kwitły, każda była małym miastem, z~własnymi kuchniami, własnymi akademikami, własną przychodnią i~własnym punktem kontroli celnej, gdzie każdy pojazd i~gość wjeżdżający lub wyjeżdżający z~muru był sprawdzany. 


A te niezłomne miasta się rozpadły. Fabryka Zmywarek Najwyższej Jakości została zamknięta w~poniedziałek. W środę padła ciepłownia Boundless Energy Enterprises. Każdego dnia Lu widywał szefów wchodzących i~wysiadających samochodami, machając do nich po tym, jak błysnęli mu swoimi dowodami osobistymi. Pewnego dnia uzbroił się w~nerwy i~oparł o okno, a jego twarz dzieliła zaledwie kilka centymetrów od twarzy człowieka, który co miesiąc wypłacał mu pensję. 


-- Radzimy sobie lepiej niż sąsiedzi, co, szefie? -- Starał się o jowialny uśmiech, najlepszy, na jaki mógł się zdobyć, ale wiedział, że nie był zbyt dobry. 


-- Mamy się dobrze -- warknął szef. Miał bardzo gładką skórę i~elegancki sportowy płaszcz, ale jego ramiona były pokryte łupieżem. -- I niech nikt nie mówi inaczej!  


-- Tak jak mówisz, szefie -- powiedział Lu i~wychylił się przez okno, starając się utrzymać uśmiech na swoim miejscu. Ale widział to po twarzy szefa, fabryka zostanie zamknięta. 


Następnego dnia na przystanek nie przyjechał żaden autobus. Normalnie na autobus czekałoby pięćdziesiąt lub sześćdziesiąt osób, głównie młodzi mężczyźni, kobiety mieszkały głównie w~akademikach. Ochroniarze i~dozorcy nie cenili pokoi w~akademikach. Tego ranka na przystanek autobusowy czekało osiem osób. Minęło dziesięć minut, kilka kolejnych spływało do przystanku, a autobus nadal nie nadjeżdżał. Minęło trzydzieści minut -- Lu oficjalnie spóźnił się do pracy -- a autobus nadal nie przyjeżdżał. Poszukał kolegów kelnerów, żeby sprawdzić, czy ktoś nie zbliża się do jego fabryki i~może zechcieć dzielić taksówkę -- luksus skądinąd nie do pomyślenia, ale utrata pracy była nawet bardziej nie do pomyślenia. 


Inny facet, z~akcentem Shaanxi, był chętny i~wtedy zauważyli, że po drodze też nie jeżdżą żadne taksówki. Tak więc Lu, będąc Lu, poszedł do pracy, piętnaście kilometrów w~palącym, topniejącym, ociekającym upale, z~koszulą i~płaszczem ochroniarza na ramieniu, z~podkoszulkiem podwiniętym, by odsłonić brzuch, z~kurzem osadzającym się na butach. A kiedy przybył do fabryki suszarek kondensacyjnych Miracle Spirit, znalazł się w~tłumie tysięcy skrzeczących młodych kobiet w~fabrycznych kitlach, stłoczonych wokół ogrodzenia i~grzechoczących podwójnymi kłódkami i~krzyczących do zaciemnionych drzwi fabryki. Wiele dziewcząt miało na ziemi małe plecaki lub worki marynarskie, przeładowane i~przeciekające bieliznę i~kosmetykami do makijażu. 


-- Co się dzieje? -- krzyknął do jednej, wyciągając ją z~tłumu. 


-- Dranie zamknęli fabrykę i~nas wyrzucili. Zrobili to podczas przerwy. Nacisnęli alarm przeciwpożarowy i~krzyknęli ,,Ogień'' i~,,Dym'', a kiedy wszystkie tu wyszłyśmy, wybiegli i~zamknęli bramę na kłódkę! 


-- Kto? -- Zawsze myślał, że jeśli fabryka ma zostać zamknięta, to zrobią to za pomocą ochroniarzy. Zawsze myślał, że przynajmniej dostanie ostatnią wypłatę z~firmy. 


-- Szefowie, sześciu z~nich. Pan Dai i~pięciu jego przełożonych. Zamknęli frontową bramę, a potem przejechali przez tylną, zamykając ją za sobą. Wszystkie jesteśmy wywalone. Wszystkie moje rzeczy są tam! Mój telefon, moje pieniądze, moje ubrania\ldots  


Jej ostatnia wypłata. Zostały tylko trzy dni do wypłaty i~oczywiście firma zachowała pensję z~pierwszych ośmiu tygodni, kiedy wszyscy zaczęli pracować. Musiałeś poprosić szefa o pozwolenie, jeśli chciałeś zmienić pracę i~zabrać pieniądze, w~przeciwnym razie musiałeś zrezygnować z~dwumiesięcznej pensji. 


Wokół Lu krzyki narastały, a małe kobiece pięści wymachiwały w~powietrzu. Na kogo one krzyczały? Fabryka była pusta. Fabryka była pusta. Gdyby wspięły się na ogrodzenie, przecięły u góry drut kolczasty, a potem wyłamały zamki w~drzwiach fabryk, kierowałyby tym miejscem. Nie mogły wynieść suszarki kondensacyjnej -- w~każdym razie niełatwo -- ale było mnóstwo drobiazgów: narzędzia, krzesła, rzeczy z~kuchni, rzeczy osobiste dziewcząt, które nie pomyślały, aby je zabrać, kiedy zabrzmiał alarm przeciwpożarowy. Lu wiedział o wszystkich rzeczach, które można przemycić z~fabryki. Był ochroniarzem. Albo już nie był. Częścią jego pracy było przeszukiwanie innych pracowników, kiedy wychodzili, aby upewnić się, że nie kradną. Jego przełożony, pan Chu, przeszukiwał \textit{go }po kolei pod koniec każdej zmiany. Nie był pewien, czy ktokolwiek, przeszukiwał pana Chu. 


Miał małe narzędzie wielofunkcyjne, które każdego ranka przypinał do paska. Mając przy sobie cały czas zestaw szczypiec, nóż i~śrubokręt, zmieniało sposób, w~jaki postrzegasz świat -- stało się to miejscem do cięcia, krojenia, podważania i~odkręcania. 


-- Czy to Twoja jedyna kurtka? -- krzyknął do ucha dziewczyny, z~którą rozmawiał. Była trochę niższa od niego, z~dużym pieprzykiem na policzku, który raczej lubił. 


-- Oczywiście, że nie! -- odpowiedziała. -- Mam w~środku trzy inne.  


-- Jeśli dostaniesz te trzy, czy mogę użyć tej? 

 Rozłożył szczypce na swoim multitoolu. Były połączone przez zestaw kół zębatych, które potęgowały siłę ściskającej dłoni, a w~szczęki szczypiec było wstawione parę paskudnie ostrych przecinaków do drutu. Dziewczyna z~jego wioski pracowała przez jakiś czas w~fabryce SOG w~Dongguan, dała mu parę i~życzyła powodzenia w~południowych Chinach. 


Dziewczyna z~trzema kurtkami spojrzała na drut kolczasty. 

-- Zostaniesz pocięty na wstążki -- powiedziała. 


Uśmiechnął się. 

-- Może -- powiedział. -- Myślę jednak, że mogę to zrobić. 


-- Chłopcy -- wrzasnęła mu do ucha. 

Czuł w~jej oddechu zapach jej śniadania, zmieszanego z~pastą do zębów. To sprawiło, że zatęsknił za domem. 

-- W porządku. Ale bądź ostrożny! -- Zrzuciła kurtkę, odsłaniając mocno umięśnione ramiona, wypracowane do zręczności na linii. Owinął ją wokół swojej lewej dłoni, a potem owinął wokół niej własny płaszcz, tak że jego dłoń wyglądała jak rysunkowa rękawica bokserska, pod którą spływały rękawy. 


Nie było łatwo wspiąć się na płot z~jedną ręką owiniętą w~kilkanaście grubości materiału, ale zawsze był świetnym wspinaczem, nawet w~wiosce, odważnym chłopcem, który zyskał reputację wspinania się na wszystko, co stoi w~miejscu : drzewa, domy, a nawet fabryki. Miał jedną zdrową rękę, dwie stopy i~jedną zabandażowaną rękę, i~to wystarczyło, by wspiąć się na pięć metrów w~górę. Tam ostrożnie owinął lewą rękę wokół drutu kolczastego, uważając, aby pociągnąć go prosto w~dół i~nie przesuwać z~boku na bok. Pomyślał o sobie samym poślizgującym się i~upadającym, drut kolczasty odcinający mu palce z~dłoni tak, że upadły na drugą stronę ogrodzenia, wijące się jak robaki w~kurzu, gdy ściskał swoją zmasakrowaną dłoń i~krzyczał, zalewając dziewczyny gejzerem krwi. 


\textit{Więc lepiej się nie poślizgnij}, pomyślał ponuro, ostrożnie rozkładając multitool drugą ręką, obracając nim jak nożem motylkowym (ruch, który często ćwiczył, bawiąc się w~rewolwerowca w~swoim pokoju lub kiedy nikogo innego nie było w~pobliżu bramy). Ostrożnie wsunął go wokół pierwszego zwoju drutu i~ścisnął w~dół, obserwując, jak zęby kół zębatych zazębiają się i~napinają jeden po drugim, zamieniając dźwignię prawej dłoni w~setki kilogramów nacisku skierowanego w~dół dokładnie na krawędź tnącą szczypiec. Wbiły się w~drut, złapały, a potem rozdzieliły. 


Zwój drutu wyskoczył z~głośnym dźwiękiem \textit{dzyń}, a on uchylił się w~samą porę, żeby uniknąć odcięcia nosa -- a może ucha i~oka -- przez drut. 


Ale teraz mógł przenieść lewą rękę na szczyt ogrodzenia, przyłoży do niej większy ciężar i~sięgnąć po drugi zwój drutu za pomocą przecinaków, zwisający daleko od ogrodzenia, tak daleko, jak tylko mógł, aby uniknąć zwoju, gdy wyskoczył zwolniony. Co zrobił, rozcinając się równie łatwo jak pierwszy zwój i~lecąc bezpośrednio na niego, i~tylko uwolnienie nóg i~zawiśnięcie jedną ręką na płocie, uderzając swoim ciałem, udało mu się uniknąć podcięcia gardła. W tym momencie drut zrobił długą rysę z~tyłu głowy, która zaczęła swobodnie krwawić wzdłuż jego pleców. Zignorował to. Albo była płytka i~zatrzymałaby się sama, albo była głęboka i~potrzebował pomocy medycznej, ale tak czy inaczej, zamierzał oczyścić płot. 


Teraz pozostały tylko trzy pasma drutu kolczastego, które były trudniejsze do przecięcia niż drut żyletkowy, ale kolce były szeroko rozstawione, a sam drut był mniej podatny na zwariowane latanie pił niż zwinięty drut żyletkowy. Gdy każdy z~nich się dzielił, dziewczyny pod nim wyły z~aprobatą i~chociaż jego skóra głowy piekła mocno, pomyślał, że to może być po prostu jego najlepsze chwile, pierwszy raz w~życiu, kiedy był kimś więcej niż ochroniarz, który opuścił swoje zacofane miasto, by znaleźć nic ważnego w~prowincji Guangdong. 


A teraz był w~stanie odwinąć kurtki z~ręki i~po prostu przeskoczyć przez płot i~zejść po drugiej stronie jak małpa, szczerząc się przez całą drogę do hordy młodych dziewcząt, które wspinały się z~drugiej strony w~wielkiej fali. Nie minęło dużo czasu, zanim dogoniła go dziewczyna z~trzema kolejnymi kurtkami. Wytrząsnął jej kurtkę -- przeciętą w~czterech lub pięciu miejscach -- jak kelner oferujący damie płaszcz, a ona delikatnie wsunęła w~nią te muskularne ramiona, a potem odwróciła go i~szturchnęła go w~skórę głowy. 


-- Płytka -- powiedziała. -- Będzie mocno krwawić, ale wszystko będzie w~porządku. 

Złożyła mu siostrzany pocałunek na policzku. 

-- Jesteś grzecznym chłopcem -- powiedziała, a potem pobiegła, by dołączyć do strumienia dziewcząt, które wchodziły do fabryki przez rozbite drzwi. 


Wkrótce znalazł się sam na dziedzińcu fabrycznym, pośród schludnych żwirowych ścieżek i~przystrzyżonych trawników. Wpuścił się do fabryki, ale właściwie nie mógł się zmusić, żeby cokolwiek wziąć, chociaż byli mu winni prawie trzymiesięczne pensje. W jakiś sposób wydawało mu się, że dziewczyny, które używały narzędzi, powinny mieć swój wybór narzędzi, że mężczyźni, którzy przygotowywali posiłki, powinni wybierać rzeczy z~kuchni. 


W końcu zdecydował się na jeden z~ogólnodostępnych rowerów, które były schludnie zaparkowane w~pobliżu bram fabryki. Były one używane przez wszystkich pracowników w~równym stopniu, a poza tym musiał wrócić do domu, a powrót z~raną głowy w~południowym upale nie wydawał się zbyt dobrym planem. 


W drodze do domu świat wydawał się bardzo zmieniony. Przede wszystkim stał się przestępcą, co wydawało mu się bardzo odległe od ochroniarza. Ale to było coś więcej: powietrze wydawało się czystsze (później przeczytał, że powietrze \textit{było }czystsze dzięki wszystkim zamkniętym fabrykom i~autobusom, które pozostały zaparkowane). Większość sklepów wydawała się zamknięta, a reszta była obsługiwana przez apatycznych sklepikarzy, którzy siedzieli na werandzie lub grali na nich w~Mah-Jonga, chociaż był środek dnia. Wszystkie restauracje i~kawiarnie były zamknięte. Na przejeździe kolejowym patrzył, jak przejeżdża pociąg międzymiastowy, każdy wagon jest zapchany młodymi kobietami i~ich torbami, zostawiając Shilong New Town, by znaleźć drogę gdzie indziej, gdzie wciąż była praca. 


Tak po prostu, w~ciągu tygodnia lub dwóch umarło to gigantyczne miasto. To wszystko wydawało się tak niewiarygodnie potężne, kiedy przybył, nowe brukowane drogi, nowe sklepy i~nowe budynki oraz fabryki szybujące na tle nieba, gdziekolwiek spojrzysz. 


Zanim dotarł do domu -- zawroty głowy od bolącego skaleczenia na głowie, spocony, głodny -- wiedział, że magiczne miasto było tylko stosem betonu i~górą potu robotników i~że miało całą trwałość snu. Gdzieś, w~odległej krainie, której nazwy ledwo znał, ludzie przestali kupować pralki, a więc jego miasto umarło. 


Myślał, że położy się tylko na najkrótszą drzemkę, ale kiedy wstał, zebrał kilka rzeczy do worka marynarskiego i~wrócił na rower, nie zawracając sobie głowy zamykaniem za sobą drzwi swojego mieszkania, to stacja kolejowa była już zabarykadowana, a drogą do oddalonego o co najmniej dwa dni drogi do Shenzhen ciągnęła się długa kolejka uchodźców. Cieszył się, że zabrał rower. Później znalazł działający bankomat i~wyciągnął trochę gotówki, co było bardziej uspokajające, niż się spodziewał. Przez chwilę wydawało się, że świat się skończył. Z ulgą dowiedziałem się, że to tylko jego mały kącik. 


W Shenzhen zaczął przesiadywać w~kafejkach internetowych, ponieważ były to najtańsze miejsca do siedzenia w~domu, z~dala od upału, i~ponieważ były wypełnione młodymi mężczyznami, takimi jak on, przemykającymi obok. A ponieważ mógł stamtąd rozmawiać z~rodzicami, opowiadając im zmyślone historie o swoim nieistniejącym poszukiwaniu pracy, obiecując, że wkrótce zacznie wysyłać pieniądze do domu. 


I właśnie tam znalazła go gildia, Ping i~jego przyjaciele, i~mieli tego kumpla po drugiej stronie planety, tego gościa Wei-Donga, który wisiał zafascynowany na każdym słowie jego opowieści, który powiedział mu, że opisał to w~raporcie z~badań społecznych w~szkole, co ich wszystkich rozśmieszyło. Znalazł szczęście i~pracę, a także prawdę: świat nie został zbudowany na skale, ale raczej na piasku, i~zmieni się na zawsze. 


Wei-Dong nie wiedział, jak długo potrwa biznes jego ojca. Może trzydzieści lat, ale myślał, że to znacznie mniej. Codziennie budził się w~swojej sypialni pod pościelą Spongebob i~zastanawiał się, bez których z~tych rzeczy mógłby żyć, jak \textit{podstawowe }może stać się jego życie. 


I oto była szansa, żeby się przekonać. Kiedy jego pradziadkowie byli w~jego wieku, byli uchodźcami wojennymi, przeprawiali się przez ocean na zatłoczonej łodzi, podróżowali na skradzionych papierach, z~niemowlęciem w~ramionach prababki i~drugim w~jej brzuchu. Jeżeli mogli to zrobić, Wei-Dong mógł to zrobić. 


Potrzebował mieszkania, co oznaczało pieniądze, co oznaczało pracę. Gildia dałaby mu część pieniędzy z~rajdów, ale to nie wystarczyło, by przeżyć w~Ameryce. A może wystarczyło? Zastanawiał się, ile Gwatemalczycy wokół niego zarabiali na nielegalnych pracach związanych ze zmywaniem naczyń, sprzątaniem i~pracami ogrodniczymi. 


W każdym razie nie musiałby się tego dowiadywać, bo miał coś, czego oni nie mieli: numer ubezpieczenia społecznego. I tak, to oznaczało, że w~końcu jego rodzice mogliby go znaleźć, ale za miesiąc miałby 18 lat i~byłoby już za późno, by cokolwiek z~tym zrobili, gdyby nie chciał współpracować. 


W tych godzinach, w~których planował upadek rodzinnej fortuny, szybko wybrał najłatwiejszą pracę, jaką mógł wykonać: Mechaniczny Turek. 


Turcy byli armią robotników w~przestrzeni gier. Wszystko, co musiałeś zrobić, to udowodnić, że jesteś przyzwoitym graczem -- gra miała statystyki, aby to wiedzieć -- i~zarejestrować się, a następnie zalogować się, gdy tylko chcesz zmianę. Gra będzie cię pingować za każdym razem, gdy gracz zrobi coś, czego gra nie umiała zinterpretować -- zbyt intensywnie rozmawia z~postacią niebędącą graczem, wbije miecz tam, gdzie nie należy, wspina się na drzewo, którym nikt nawet nie zawracał sobie głowy, aby dodać jakieś szczegóły -- i~musiałbyś odegrać sędziego punktowego. Grałbyś postacią niegrywalną, wybierał zachowanie dźgniętego obiektu lub podejmował decyzję z~menu możliwych rzeczy, które możesz znaleźć na drzewie. 


Nie płaciło to dużo, ale też nie zajmowało dużo czasu. Wei-Dong obliczył, że gdyby grał na dwóch komputerach -- co był pewien, że nadążałby -- i~wykonywał nową pracę co dwadzieścia sekund na każdym, mógłby zarobić tyle, co starsi menedżerowie w~firmie jego ojca. Musiał to robić przez dziesięć godzin dziennie, ale spędzał mnóstwo weekendów grając po 12, a nawet 14 godzin dziennie, więc do diabła, to były praktycznie pieniądze w~banku. 


Więc użył wynajętego komputera, aby zalogować się na swoje konto, i~zaczął wypełniać dokumenty, aby ubiegać się o pracę. Przez cały czas był świadomy swojego rzadko używanego konta e-mail i~wiadomości od rodziców, które z~pewnością na niego czekały. Formularze były długie i~nudne, ale dość łatwe, nawet te małe pytania opisowe, w~których trzeba było odpowiedzieć na kilka hipotetycznych pytań o to, co byś zrobił, gdyby gracz zrobił to lub powiedział tamto. I ten e-mail od rodziców czaił się, żądając, aby go ściągnął i~przeczytał\ldots  


Wszedł do przeglądarki i~wyświetlił swój e-mail. Minęły tygodnie, odkąd ostatnio to sprawdzał i~był zatłoczony setkami spamów, ale tam, na górze: 


RACHEL ROSENBAUM --- GDZIE JESTEŚ??? 


Oczywiście jego matka była tą, która wysłała e-maile. To ona zawsze wysyłała mu e-maile, które przez cały dzień szkolny wysyłały mu zachęcające notatki, przypominające mu o urodzinach dziadków, kuzynów i~ojca. Ojciec używał poczty, kiedy musiał, zwykle o drugiej w~nocy, kiedy nie mógł spać z~powodu martwienia się o pracę i~musiał krzyczeć na swoich menedżerów, nie budząc ich przez telefon. Ale gdyby telefon był opcją, tata by ją wybrał. 


GDZIE JESTEŚ??? 


Temat mówił wszystko, prawda? 


\textit{Leonard, to szaleństwo. Jeśli chcesz być traktowany jak dorosły, zacznij się tak zachowywać. Nie chowaj się za naszymi plecami, grając w~gry w~środku nocy. Nie uciekaj do Bóg-wie-gdzie, żeby się podąsać.}


\textit{Możemy to negocjować jak rodzina, jak dorośli, ale najpierw musisz WRÓCIĆ DO DOMU i~przestać zachowywać się jak ZEPSUTY BACHOR. Kochamy cię, Leonardzie, martwimy się o ciebie i~chcemy ci pomóc. Wiem, że kiedy masz 17 lat, łatwo jest poczuć, że znasz wszystkie odpowiedzi\ldots} 


Przestał czytać i~wydmuchał gorące powietrze przez nozdrza. Nienawidził, gdy dorośli mówili mu, że czuje się tak, jak on, tylko dlatego, że był \textit{młody}. Jakby bycie młodym było jak bycie szalonym lub pijanym, jak przekonania, które miał, były halucynacjami spowodowanymi chorobą psychiczną, którą można wyleczyć czekaniem tylko przez pięć lat. Dlaczego po prostu nie wsadzić go do pudełka i~zamknąć, dopóki nie skończy 22 lat? 


Zaczął klikać odpowiedź, po czym zdał sobie sprawę, że był zalogowany bez przechodzenia przez anonimizator. Jego gildie zajmowały się tymi sprawami -- były to serwery, które przekazywały Twój ruch, ukrywając Twoją tożsamość i~adresy, których próbowałeś uniknąć. Najlepsze pochodziły z~Falun Gong, dziwnego kultu religijnego, który chiński rząd zamierzał wykorzenić. Falun Gong umieszczał nowe przekaźniki online mniej więcej co godzinę, wyprzedzając Wielką Chińską Zaporę Sieciową, wszechwidzącą, wszechwiedzącą, kontrolującą wszystko farmę serwerów, która miała powstrzymać 1,6 miliarda Chińczyków od patrzenia na niewłaściwy rodzaj informacji. 


Nikt w~gildii nie miał zbyt wiele czasu na Falun Gong ani na jego dziwaczne wierzenia, ale wszyscy zgadzali się, że świetnie pracowali, jeśli chodzi o wybijanie dziur w~Wielkim Firewallu. Szybki trolling przez ciągle obracające się strony indeksowe dla przekaźników Falun Gong pozwolił znaleźć Wei-Dongowi maszynę, która przejmowała jego ruch. \textit{Potem }odpowiedział swojej mamie. Niech spróbuje pobiec jego śladem -- to byłby ślepy zaułek do osławionego chińskiego kultu religijnego. To dałoby jej coś, o co mogła się martwić, w~porządku! 


\textit{Mamo, wszystko w~porządku. Zachowuję się jak dorosły (dbam o siebie, podejmuję własne decyzje). To mogło wyglądać źle okłamywanie was na temat tego, co robiłem ze swoim czasem, ale porywanie własnego syna do szkoły wojskowej jest tak niedorosłe, jak to tylko możliwe. Odezwę się, kiedy będę miał okazję. Kocham was. Nie martw się, jestem bezpieczny.}


Czy naprawdę był? Tak samo bezpieczny, jak jego pradziadkowie schodzący ze statku w~Nowym Jorku. Tak samo bezpieczny, jak Lu jadący rowerem popękaną drogą do Shenzhen. 


Znalazłby miejsce na nocleg -- mógłby wygooglować -- tani hotel w~centrum Los Angeles tak samo, jak następny dzieciak. Miał pieniądze. Miał NUS. Miał pracę -- dwie prace, wliczając pracę w~gildii -- i~miał mnóstwo misji treningowych, które musiał wykonać, zanim zacznie zarabiać. I nadszedł czas, aby się do tego zabrać. 


\chapter{Ciężka praca w zabawie }



Ta scena jest poświęcona niezrównanej Mysterious Galaxy w~San Diego w~Kalifornii. Ludzie z~Mysterious Galaxy namawiali mnie do podpisywania książek za każdym razem, gdy byłem w~San Diego na konferencji lub nauczaniu (Warsztat Pisarzy Clarion ma siedzibę na UC San Diego w~pobliskim La Jolla w~Kalifornii) i~za każdym razem pojawiają się, pakują cały budynek. To sklep z~lojalnymi zwolennikami zagorzałych fanów, którzy wiedzą, że zawsze będą mogli uzyskać świetne rekomendacje i~świetne pomysły w~sklepie. Latem 2007 roku zabrałem lekcję pisania z~Clarion do sklepu na premierę ostatniej książki o Harrym Potterze o północy i~nigdy nie widziałem tak szalonej, niesamowicie zabawnej imprezy w~sklepie. 


\href{https://www.mystgalaxy.com/book/9780765322166}{\textit{Mysterious Galaxy}}: 7051 Clairemont Mesa Blvd., apartament nr 302 San Diego, CA USA 92111 +1 858 268 4747 

\threeast

Zaatakowali robotników w~grze i~w prawdziwym świecie, skoordynowany atak, który rozbił organizację Big Sister Nor. 

Tej pamiętnej nocy zajęła zaplecze Headshot, PC Baang w~dzielnicy Geylang w~Singapurze, dzielnicy tętniącej przez całą noc od ryczącego handlu seksem z~legalnych burdeli i~nielegalnych prostytutek. O każdej porze po zmroku ulice Geylang były zatłoczone ludźmi, od żądnych przygód gości jedzących w~doskonałych całonocnych restauracjach (prawie wszystkie z~nich halal, co zawsze wywoływało u niej uśmiech) po pracowników migrantów i~Singapurczyków grasujących w~poszukiwaniu niedozwolonych emocji po dziewczyny biegające na przerwach do całonocnych supermarketów na zakupy. 


Geylang był tak rozpięty, jak Singapur, jednym z~niewielu miejsc, w~których można było być ,,poza granicami'' -- robić coś nielegalnego, niemoralnego, nie do wzmianki lub szkodliwego dla harmonii społecznej -- bez przyciągania zbyt dużej uwagi. Headshot przez całą noc migotał od sieciowych gier pokerowych, wielkich turniejów shoot-em-up, tanich telefonów do domu, przekrzykujących hałaśliwą mieszankę wszystkich tych gier, a tej nocy Big Sister Nor i~jej klan. 


Nazywali siebie Webblies, co było niejasnym żarcikiem, który okropnie cieszył Big Sister Nor. 

Prawie sto lat temu grupa robotników utworzyła związek o nazwie Robotnicy Przemysłowi Świata, Industrial Workers of the World, pierwszy związek, który powiedział, że wszyscy pracownicy muszą bronić się nawzajem, że każdy pracownik jest mile widziany bez względu na kolor jego skóry, bez względu na to, czy pracownik był kobietą, bez względu na to, czy wykonywał pracę ,,wykwalifikowaną'' czy ,,niewykwalifikowaną''. Nazywali siebie Wobblies. 


Informacje o Wobblies były tylko jednym z~wielu tematów ,,poza granicami'', które były blokowane w~singapurskim Internecie, więc oczywiście Big Sister Nor postanowiła dowiedzieć się o nich więcej. Im więcej czytała, tym więcej sensu nadawała ta nieistniejąca już grupa dla \textit{dzisiejszego} świata -- wszystko, co zrobiło IWW, musiało być zrobione dzisiaj, a co więcej, \textit{dziś }byłoby łatwiej niż wtedy. 


Weźmy organizowanie pracowników. Wtedy trzeba było wejść do fabryki lub przynajmniej stanąć u jej bram, żeby porozmawiać z~robotnikami o podpisaniu legitymacji związkowej i~żądaniach lepszych warunków, wyższych płac i~krótszych godzin pracy. Teraz możesz dotrzeć do tych samych osób online, z~dowolnego miejsca na świecie. Kiedy zostali członkami, mogli rozmawiać ze wszystkimi innymi członkami, używając tych samych narzędzi. 


Postanowiła nazwać swoją małą grupę Industrial Workers of the World Wide Web, IWWWW, i~to był kolejny z~tych żartów, który bardzo ją ucieszył. A IWWWW rosła i~rosła i~rosła. Farmerzy złota byli łatwym łupem: pracowali w~strasznych warunkach na całym świecie, za straszne pensje, znienawidzeni zarówno przez producentów gier, jak i~bogatych graczy. Zrozumieli już, jak pracować w~zespołach, założyli już własne małe gildie, i~byli lepsi w~korzystaniu z~Internetu niż kiedykolwiek byliby ich szefowie. 


Teraz, rok później, IWWWW miała ponad 20 000 członków zapisanych w~sześciu krajach, płacących składki i~wypełniających gruby fundusz strajkowy, który w~końcu został użyty w~Shenzhen, ostatnim miejscu, w~którym Big Sister Nor nigdy się nie spodziewała strajku. 


Ale zastrajkowali! Szef, postać o imieniu Wing, zadeklarował lock-in, zamknięcie pracowników w~trzech swoich ,,fabrykach'' -- kafejkach internetowych, które przejął, by wesprzeć rozrastającą się armię robotników -- w~celu wykorzystania z~dziury w~Mushroom Kingdom, grze MMO opartej na Mario, która miała wielu fanów w~Brazylii. Jeden z~jego pracowników znalazł sposób na potrojenie złota, które wydobywali z~jednego z~lochów, a on chciał wydobyć każdy grosz, jaki mógł, zanim Nintendo-Sun się załapie. 


Następną rzeczą, o której się dowiedziała, było to, że jej telefon brzęczał od pilnych wiadomości przekazywanych z~jej różnych tożsamości w~grze, aby powiedzieć jej, że pracownicy odepchnęli zarząd fabryki i~strażników i~wybiegli, wspinając się po bokach budynków lub słupach, przecinając kable sieciowe kawiarni. Uformowali front i~zaczęli skandować zaimprowizowane hasła, w~większości zaadaptowane z~ich okrzyków bitewnych w~grze. A teraz chcieli wiedzieć, co robić. 


-- To wildcat, dziki, strajk -- powiedziała Big Sister Nor do swoich poruczników, The Mighty Krang i~Justbob, ten pierwszy to mały Chińczyk z~oszronionymi fioletowymi końcówkami we włosach, drugi to tamilska dziewczyna w~pięknym, nieskazitelnym sari i~jedwabnych kapciach, dziewczyna, która wcześniej latała z~jednym z~najbardziej znanych gangów dziewczęcych w~Azji i~spędziła trzy lata w~więzieniu za swoje kłopoty. 

-- Oni wyszli z~fabryki w~Shenzhen. -- Przesłała tweety, blipy i~alerty ze swojego telefonu, a następnie pokazała im swój ekran, gdy czekali, aż komunikaty wylądują na ich urządzeniach. 


-- To szaleństwo -- powiedział Mighty Krang, tańcząc z~podniecenia z~nogi na nogę. -- To szaleństwo, to szaleństwo, to \ldots  


-- Cudownie -- powiedziała Justbob, kładąc dłonie na jego ramionach i~sprowadzając go z~powrotem na ziemię. -- I spóźnione. Przewidziałem to. Przewidziałem to od samego początku. Jak tylko zaczniesz zbierać składki na ,,fundusz strajkowy'', ktoś zacznie strajkować. Raz dwa trzy, wszyscy strajkujemy.


Następnym krokiem było udanie się do siedziby głównej, zaplecza w~Headshot, aby usiąść na krzesłach i~uderzyć w~światy, rozgłaszając wiadomość o pierwszym strajku do wszystkich 20 000 członków. Big Sister Nor poszła pracować nad planem: 


--- 1. Rozpowszechnij masowo info


--- 2. Zrekrutuj pikiety w~świecie, aby zablokować miejsce pracy, aby Boss Wing nie mógł sprowadzać łamistrajków -- pracowników zastępczych -- aby wykonać pracę. 


--- 3. Zadzwoń do liderów strajku i~porozmawiaj o prawnikach zajmujących się prawami człowieka, płacach za strajk, kwaterach sypialnych dla wszystkich pracowników, którzy polegali na fabryce łóżek w~akademikach 


--- 4. Zdobądź materiał filmowy i~raporty w~czasie rzeczywistym od strajkujących do mediów praw człowieka, załatw wywiady z~prasą przywódcom strajku  


Robiła to już wcześniej, w~prawdziwym życiu, z~drugiej strony, jako przywódca dzikiego strajku, kiedy zeszła z~linii, gdy szefowie w~jej fabryce tkackiej w~Taman Makmur ogłosili cięcia płac, ponieważ ich duży europejski dystrybutor zmniejszył zamówienia. Zdarzało się to rokrocznie, ale bardzo ją to rozzłościło, pracownicy nie otrzymywali premii, dzieląc się szczęściem, gdy dystrybutorzy zwiększali swoje zamówienia, ale zmuszano ich do dzielenia się ciężarem, gdy zamówienia spadały. Cóż, zapomnij o tym, dość tego. Wstała na środku hali fabrycznej i~potępiła szefów jako chciwych, niemoralnych drani, którymi byli, a kiedy ochrona wkroczyła, by ją zabrać, stała dumna i~silna, gotowa na pobicie za jej bezczelność. 


Zamiast tego jej współpracownice stanęły w~jej obronie, młode kobiety wokół niej wstały i~otoczyły ją, dopingując ją, wznosząc lamentujące krzyki, które odbijały się od sufitu i~wypełniały pokój i~jej serce, czyniąc je wszystkie odważnymi, tak że ochroniarze cofnęli się i~przejęły fabrykę, zablokowały bramy, zamknęły ją, a potem był tam ktoś z~Malezyjskiego Związku Pracowników Włókiennictwa, żeby skłonić ich do podpisania kart, i~ktoś zrobił ją kapitanem pikiety, a potem\ldots  


A potem to wszystko zwaliło się na nie, nadjeżdżały policyjne furgonetki, policja ustawiała się w~linii i~kazała im rozejść się, wrócić do pracy, powstrzymać tę głupotę, zanim ktoś zostanie ranny, wykrzykiwała rozkazy przez megafon, wściekała się na nie spod hełmów bojowych, uderzając pałkami w~tarcze, spryskując ich gazami łzawiącymi. 


Ich linia zafalowała, rozpadła się, cofała. Ale sformowali się w~alejce niedaleko fabryki, pośród bandy gapiących się dzieci, a kobiety z~MUTE założyły im kołnierzyki i~kazały im biec po mleko -- krowie, kozie, co tylko udało im się znaleźć, i~organizatorzy MUTE przepłukiwali oczy mlekiem, trzymając twarze nieruchomo, gdy kaszlały i~się krztusiły. Rozpuszczalny w~tłuszczach gaz CS wypłukał się, pozostawiając je ze łzami w~oczach, ale mogły widzieć, a kaszel się rozproszył, a ktoś wyciągnął torbę masek rowerowych z~filtrem węglowym, a ktoś inny miał torbę okularów pływackich i~kobiety je założyły i~naciągnęli hidżaby na nosy, na maski, tak że wyglądali jak jakiś gatunek zwierząt z~pyskami, po czym utworzyły swoją linię i~pomaszerowały z~powrotem, skandując hasła. 


Policja ponownie ich zagazowała, ale tym razem kapitanki pikiet były w~stanie utrzymać linię, wysłać odważne kobiety naprzód, aby chwyciły dymiące kanistry i~przerzuciły je z~powrotem przez linie policyjne. Przez chwilę wydawało się, że policja zaatakuje, ale strajkujący i~organizatorzy przesyłali do Internetu strumień zdjęć za pomocą telefonów komórkowych, które przechodziły przez ogólnokrajową zaporę ogniową, podłączając je do stron dotyczących praw człowieka, a więc Ministerstwo Pracy otrzymywało telefony z~prasy zagranicznej, oni rozmawiali z~Ministerstwem Sprawiedliwości, a policja się wycofała. 


Pierwsza potyczka się skończyła, a napastnicy przygotowali się do długiego oblężenia. Nikt nie wchodził ani nie wychodził z~fabryki bez bycia nękanym przez setki młodych kobiet, wypychających literaturę opisującą warunki pracy, skargi i~żądania przez okna autobusów i~samochodów. Niektórzy pracownicy zastępczy weszli, niektórzy wszczynali bójki, niektórzy odwrócili się i~odeszli. Związkowy kierowca ciężarówki odmówił przekroczenia ich linii i~nie chciał zabrać ładunku, który został mu zlecony, więc ładunek po prostu leżał w~dokach. 


Dni zamieniły się w~tygodnie, a oni karmili swoje rodziny najlepiej, jak potrafili, płacą strajkową, która stanowiła jedną trzecią tego, co zarobili w~fabryce, ale właściciele fabryk -- filia holenderskiej firmy -- też cierpieli. Organizatorzy MUTE wyjaśnili, że spółka-matka musi publikować kwartalne oświadczenie swoim akcjonariuszom, którzy domagają się wyjaśnienia, dlaczego ta duża fabryka stoi bezczynnie, zamiast zarabiać. Organizatorzy zapewnili z~przekonaniem, że kiedy to nastąpi, żądania robotników zostaną spełnione, strajk się zakończy i~będą mogli wrócić do pracy. 


Więc trzymali się tam, podtrzymując się na duchu, a potem\ldots  


Fabryka została zamknięta. 


Big Sister Nor dowiedziała się o tym pewnej nocy, grając w~Theatre of War VII, grę, w~którą grała, odkąd była małą dziewczynką. Jedną z~jej towarzyszek z~gildii była dziewczyna, której brat minął fabrykę w~drodze do domu ze szkoły i~widział, jak wywozili maszyny z~fabryki, odjeżdżali wielkimi ciężarówkami. 


Wysłała SMS-a do wszystkich, których znała: \textit{Idźcie teraz do fabryki}, ale zanim tam dotarli, fabryka była martwa, pusta, bramy zamknięte łańcuchami. Nikt ze związku ich nie spotkał. Żaden z~nich nie odpowiedział na jej telefony. 


A kobiety, które nazwała siostrami, kobiety, które uratowały ją, kiedy powiedziała \textit{dość}, wszystkie spojrzały na nią i~spytały: \textit{Co teraz zrobimy}?  


A ona nie wiedziała. Udało jej się powstrzymać łzy, dopóki nie wróciła do domu, ale potem popłynęły, a jej rodzice -- którzy wątpili w~nią i~przemawiali do niej na każdym kroku -- skarcili ją za jej głupotę, powiedzieli jej, że to jej wina, że wszyscy jej przyjaciele byli bez pracy. 


Tej nocy leżała w~łóżku, nieszczęśliwa, i~obudził ją cichy dźwięk telefonu. 


\textit{Jestem na zewnątrz}. To była Affendi, organizatorka MUTE, z~którą była najbliżej. \textit{Wyjdź pod drzwi.} 


Wyszła na zewnątrz po cichu i~ledwo zdążyła dostrzec zarys Affendi, zanim upadła w~ramiona Nory. Affendi została pobita do krwi, miała czarne oczy, złamane dwa palce, zmiażdżone usta i~brakowało jej jednego zęba. Zdołała się uśmiechnąć i~wyszeptała: 

-- To wszystko jest częścią pracy. 


Tani hotel, w~którym czterej organizatorzy dzielili pokój, został napadnięty tuż po kolacji, a policja ich zabrała. Byli na to przygotowani, mieli prawników gotowych im pomóc, kiedy to się stało, ale nie mogli wezwać prawników. Nie poszli do więzienia. Zamiast tego przewieziono ich do dzielnicy nędzy za główną stacją kolejową, gdzie trzej policjanci stali na straży, podczas gdy grupa prywatnych ochroniarzy z~fabryki na zmianę biła ich pałkami, pięściami i~butami, krzycząc na nich obelgi, nazywając dziwkami, szarpiąc ich ubrania, bijąc w~piersi i~uda. 


Zatrzymało się dopiero wtedy, gdy jedna z~kobiet straciła przytomność, krwawiła z~rany głowy, powieki trzepotały. Mężczyźni wtedy uciekli, po zabraniu ich pieniędzy i~dokumentów tożsamości, pozostawiając je płaczące i~ranne. Affendi zdołała ukryć zapasowy telefon komórkowy -- maleńki przedmiot wielkości zapałek -- w~gumce majtek, co pozwoliło jej zadzwonić po pomoc do centrali MUTE. Kiedy karetka była w~drodze, przyjechała po Nor. 


-- Pewnie po ciebie też przyjdą -- powiedziała. -- Zazwyczaj starają się dać przykład robotnikom, którzy zaczynają kłopoty. 


-- Ale powiedziałaś mi, że będą musieli ustąpić ze względu na swoich udziałowców \ldots  


Affendi podniosła złamaną rękę. 

-- Myślałam, że tak. Ale zdecydowali się wyjechać. Sądzimy, że prawdopodobnie jadą do Indonezji. Nowe przepisy znacznie utrudniają organizowanie pracowników. Czasami tak to jest. -- Wzruszyła ramionami, po czym skrzywiła się i~wciągnęła powietrze przez zęby. -- Pomyśleliśmy, że będą chcieli tu zostać. Rząd prowincji dał im zbyt wiele, by tu przyjechać \ldots  ulgi podatkowe, nowe drogi, darmowa woda i~prąd przez pięć lat. Ale są nowe Specjalne Strefy Ekonomiczne w~Indonezji, które mają jeszcze lepsze oferty. -- Znowu wzruszyła ramionami, znów się skrzywiła. -- Oczywiście możesz być tutaj bezpieczna. Może po prostu pójdą dalej. Ale pomyślałem, że powinieneś mieć szansę dotarcia z~nami w~bezpieczne miejsce, jeśli chcesz. 


Nor potrząsnęła głową. 

-- Nie rozumiem. Bezpieczne miejsce?  


-- Związek związkowy ma kryjówkę po drugiej stronie granicy prowincji. Możemy cię tam zabrać dziś wieczorem. Pomożemy ci znaleźć pracę, rozpocząć działalność. Możesz nam pomóc w~założeniu związku w~innej fabryce. 


Padał drobny deszcz, bębniąc po palmach otaczających jej ulicę i~rozpryskując się w~mokrych, tłustych kroplach, wydobywając z~ziemi ziemisty zapach. Gruba kropla zsunęła się z~niewidocznego liścia nad głową i~rozprysnęła się na szyi Nor, przypominając jej, że wyszła z~domu bez hidżabu, czego prawie nigdy nie robiła. Wydawało jej się to omenem, jakby jej życie zmieniało się pod każdym względem. 


-- Gdzie idziemy?  


-- Dowiesz się, kiedy tam dotrzemy. Ja też nie wiem. Dlatego to jest kryjówka, nikt nie wie, gdzie to jest, chyba że musi. Organizatorzy MUTE zostali zamordowani, rozumiesz. 


\textit{Dlaczego nie powiedziałaś mi tego, kiedy to się zaczęło?}, chciała spytać. Ale jej rodzicie przecież \textit{mówili} o tym. Kierownictwo ostrzegało ich, przez megafony, że ryzykowały wszystkim. Śmiała się z~nich, wypełniona uczuciem siostrzeństwa, bezpieczeństwa, \textit{mocy}. To uczucie już zniknęło. 


I poszła z~Affendi, pracowała w~fabryce, która bardzo przypominała fabrykę, którą opuściła, i~była walka związkowa, podobna do tej, z~którą walczyła, ale tym razem były lepiej przygotowane, a robotnice nazwały Nor ,,starszą siostrą'', co trochę ją przestraszyło, wypowiadane z~ust kobiet znacznie starszych od niej, słyszane od młodych dziewcząt, które nigdy nie potrafiły docenić niebezpieczeństwa. 


I tym razem właściciele nie uciekli, robotnicy zdobyli lepsze warunki, a Big Sister Nor stwierdziła, że nie chce już robić tekstyliów. Odkryła, że lubi walkę. 


Teraz w~Shenzhen był młody człowiek, niejaki Matthew Fong, który polegał na niej, że pomoże mu zdobyć godność, uczciwe zarobki oraz bezpieczne miejsce pracy. I robił to w~Chinach, gdzie nieoficjalne związki zawodowe były nielegalne i~gdzie organizatorzy pracy czasami znikali na lata w~więzieniu. 


Mighty Krang potrafił mówić pięknie mandaryńskim, jak również swoim ojczystym kantońskim, więc był odpowiedzialny za wysyłanie info zagranicznej chińskiej prasie, tej sieci źródeł wiadomości, która służyła setkom milionów ludzi chińskiego pochodzenia mieszkających za granicą. Byli kluczowi, ponieważ byli ściśle związani z~całym rozrastającym się przedsiębiorstwem importu i~eksportu, a kiedy przemawiali, biurokraci w~Pekinie słuchali. A Mighty Krang potrafił przybrać głos tak gładko przekonujący, że można by przysiąc, że to prezenter wiadomości. 


Justbob była odpowiedzialna za moralne wsparcie dla napastników, rozmawiając z~nimi w~łamanym kantońskim i~singlijskim oraz w~mowie graczy podczas telekonferencji, podtrzymując ich morale. Mogła obsługiwać trzy telefony i~dwa komputery jak ludzka ośmiornica, jej uwaga była podzielona na tuzin rozmów, nie gubiąc wątku w~żadnej z~nich. 


A Big Sister Nor? Była w~świecie, w~kilku światach, zbierając Webblies na stronę Mushroom Kingdom, znajdując graczy zbiegających się z~całej Azji -- gdzie była noc -- i~z Europy -- gdzie był dzień -- i~Ameryki\ldots  gdzie był ranek. Kierownictwo nie traciło czasu przy załatwianiu pracowników na zastępstwo. W chińskich prowincjach zawsze byli zdesperowani podwykonawcy, dziesięcioro dzieciaków w~wymarłym przemysłowym mieście w~Dongbei, których zwabiono do komputerów, gadając o tym, ile zarobią za granie. Przez tuzin różnych odłamków tego samego świata Mushroom Kingdom, tuzin alternatywnych rzeczywistości, przybyli, a Big Sister Nor grało generała w~potyczce przeciwko nim, gdy strajkujący zablokowali wejście do lochu i~wysłali strumień prozwiązkowych czatów i~adresy URL w~nich, nawet gdy walczyli z~nimi, aby utrzymać ich z~dala od lochów. 


Bitwa nie była wielką walką, nie na początku. Pracownicy zastępczy mieli zabijać głupie postacie nie-graczy w~nudny, przewidywalny sposób, który nie wyzwoliłby Mechanicznych Turków i~nie zwróciłby uwagi Nintendo-Sun na ich operację. Wszyscy byli doświadczonymi graczami i~byli przyzwyczajeni do gry zespołowej, a wielu Webblies nigdy wcześniej nie walczyło ramię w~ramię. Ale Webblie walczyli o ruch, a robotnicy zastępczy -- nazywali ich ,,łamistrajkami'', kolejne stare słowo z~historii -- walczyli, ponieważ nie wiedzieli, co jeszcze mogą zrobić. 


To był pogrom. Łamistrajki zostali odesłani z~powrotem do swoich punktów odrodzenia tysiącami, nie mogąc wrócić do pracy, dopóki nie zakończą biegu po trupach, a Webblies unieśli miecze i~wystrzeliwali kule ognia w~niebo, wiwatując w~kilkunastu językach. 


Wieści od Shenzhen też były dobre, sądząc po tym, co Justbob mówiła do jej słuchawek i~pisała na ekranach. Linia strajku utrzymywała się, a kiedy policja tam była, nie wkroczyli, w~rzeczywistości brzmiało to tak, jakby przenieśli się, by powstrzymać ochronę prywatnej fabryki! 


Po cichu Big Sister Nor podziękował Matthew Fongowi za wybranie walki, którą, pozornie, byli w~stanie wygrać. Krzyknęła do Ezhila przed Headshot, wzywając dookoła na bubble-tea z~żeń-szeniem, korzeń żeń-szenia dałby im wszystkim trochę energii. Nie można żyć samą kofeiną i~tauryną! 


-- Ezhil! -- krzyknęła minutę później, podnosząc wzrok znad myszy. -- Bubble-tea! -- Gdyby uważała, zauważyłaby pisk w~jego głosie, tak jak obiecał od razu, od razu. 


Ale jej uwaga była skupiona na ekranach, ponieważ tam nagle wszystko potoczyło się naprawdę źle. To, co wzięła za zwycięskie kule ognia wystrzeliwane w~niebo, lądowało teraz wśród graczy, zadając poważne obrażenia. W chwili, gdy to zauważyła, zza ekranu wyślizgnęła się seria ślizgających się, kolczastych skorup żółwi w~dwunastu światach naraz. 


Zasadzka!


Wyszczekała to słowo do zestawu słuchawkowego po mandaryńsku, potem po kantońsku, potem w~hindi i~po angielsku. Okrzyk podjęli gracze i~zebrali się, tworząc formacje bitewne, uzdrowiciele w~środku, czołgi na zewnątrz, zwinni złodzieje i~zwiadowcy rozbiegający się po grzybowych lasach w~poszukiwaniu zasadzki. 


To działałoby znacznie lepiej, gdyby byli zwykłą gildią, wszyscy grającą po stronie złego Bowsera lub dzielnej księżniczki Peach, ponieważ gdybyś był po tej samej stronie, gra koordynowałaby twoje ruchy za ciebie, dawała ci radar gdzie i~jak poruszali się wszyscy pozostali gracze. Ale napastnicy pochodzili z~obu stron moralnej monety Mushroom Kingdom i~jeśli chodzi o grę, byli zaprzysięgłymi wrogami. Ich wiadomości błyskawiczne były dla nich niezrozumiałe, a domyślną opcją dla każdego ,,przeciwnego programu'', na który kliknąłeś, był ATAK, co prowadziło do wielu przypadkowych potyczek. 


Ale farmerzy złota wiedzieli wszystko o graniu we własną grę, która żyła poza grą, w~którą chciałyby grać firmy. Narzędzia komunikacyjne w~grze były potężne i~łatwe, ale nic (poza śmieszną ,,umową'', którą trzeba było klikać za każdym razem, gdy uruchamiałeś grę) nie powstrzymywało cię przed użyciem czegokolwiek, co chciałeś. Preferowali darmowe systemy czatu, opracowane, aby pomóc korporacyjnym grupom roboczym współpracować; ponieważ te usługi zawsze miały dostępne bezpłatne wersje demo, w~nadziei, że przekonasz kogoś z~biura, aby kupił 30 000 licencji dla ich megakorporacji. Systemy te pozwalały nawet przesyłać strumieniowo zrzuty ekranu z~własnych komputerów, a Big Sister Nor dopilnowała, aby były one ułożone sekwencyjnie, tworząc ogromny, panoramiczny widok całego pola bitwy. 


Przeglądała sceny bitewne i~węzeł komunikacyjny, przesuwając palcami po klawiaturze. Mieli Koopa Turbo Hammer na siedmiu światach, ogromny, wirujący boski młot, który jednym rzutem mógł zwalić dziesiątki napastników, a ona przeniosła go na front, używając zrzutów ekranów zwiadowców, żeby wskazać pozycje wroga, wręczając je miotaczom młotów, pasmo potężnych Kongów z~wystającymi kłami i~ogromnymi, owłosionymi piersiami. 


To było siedem bitew załatwionych; w~pozostałych pięciu kazała Peaches uformować się z~gotowymi parasolami, a następnie kazała każdemu z~nich ,,odbić'' dwóch Bowserów, przyklejając się do nich, wyrządzając minimalne obrażenia. Peaches rozłożyły parasole i~poszybowały w~powietrze, zabierając ze sobą Bowsery, by schować się za liniami wroga, gotowi ziać ogniem i~deptać wrogie siły. To był niszczycielski atak, który był możliwy tylko wtedy, gdy grałeś w~grę farmerów, współpracując przez boczny kanał -- zwykle Bowsers i~Princess Peaches byli po przeciwnych stronach Wielkiej Wojny, która była w~centrum historii Mushroom Kingdom. 


To powinno zadziałać -- młoty, Bowserzy, wytrawni gracze z~kilkunastu gildii, najeżeni uzbrojeniem i~zbroją, zaklęciami, strzelaniem i~potyczkami. 


Powinno zadziałać, ale tak się nie stało. 


Tajemniczy napastnicy -- nazwała ich w~myślach ,,Pinkertonami'' po łamiących strajki bandytach z~Agencji Detektywistycznej Pinkertona, którzy byli najgorszymi wrogami starych Woblies -- mieli pozornie nieskończoną liczbę i~każdy atak, który przeprowadzili, wydawało się, że zadaje maksymalne obrażenia. W międzyczasie byli w~stanie wykonać niesamowite uniki i~obrony przed atakami napastników. A ich cel! Każda kula ognia, każdy żółw, każda bomba dźwiękowa, każdy rzucony topór trafiał w~cel z~idealną dokładnością. 


To było prawie tak, jakby \ldots  oszukiwali! 


To musiało być to. Używali aimhacków, dodgehacków, całego niedozwolonego oprogramowania dodatkowego, które gra miała być w~stanie wykryć i~wyłączyć. Jakoś udało im się przebić przez obronę gry. To nie miało znaczenia. Gra zawsze była ułożona przeciwko farmerom złota. 


-- Wycofać się! -- krzyknęła. -- Odwrót! 

To miała być wojna partyzancka, wojna w~dżungli, ukrywanie się w~krzakach i~strzelanie do nich, tak jak oni strzelali do niej. Zwabi ich na polanę, która wyznaczała wejście do lochów, a oni przemykaliby wokół nich do grzybowego lasu, wykorzystując swoją doskonałą koordynację, by przebić się z~hackami i~liczbami, które Pinkertonowie mieli po swojej stronie. W słuchawkach słyszała nierówny oddech, przekleństwa w~sześciu językach, śmiech i~krzyki graczy z~całego świata, słuchając jej poleceń wycofania we wszystkich wersjach Mushroom Kingdom, w~których walczyli. 


Stwierdziła, że się uśmiecha. To było \textit{zabawne}. To było o \textit{wiele }przyjemniejsze niż bycie zagazowanym. 


To był pomysł Big Sister Nor, by wykorzystać te gry do organizacji. Po co ryzykować kark w~fabryce lub stojąc u jej bram, skoro można wślizgnąć się między pracowników, bez względu na to, gdzie się znajdują, i~porozmawiać z~nimi o przyłączeniu się? Wielu ze starej gwardii MUTE myślało, że jest szalona, ale było też dużo wsparcia, zwłaszcza gdy Nor pokazała im, że mogą skontaktować się z~indonezyjskimi robotnikami tekstylnymi, którzy odziedziczyli jej pracę po zamknięciu jej fabryki i~przeprowadzce, po prostu logując się do Spirals of the Golden Snail, gry, która szturmem podbiła cały Półwysep Malajski. 


Nie miało znaczenia, gdzie walczyłeś, ważne było, czy wygrałeś. A im więcej o tym myślała, tym bardziej zdawała sobie sprawę, że mogą wygrać w~grze. Bossowie byli lepsi w~strzelaniu do nich gazami łzawiącymi, ale oni byli lepsi w~miotaniu kulami ognia, pulsacyjnej broni energetycznej, torpedami fotonowymi i~dzikimi latającymi rybami \ldots  i~zawsze będą. Co więcej, napastnik, który przegrał potyczkę w~grze, musiał jedynie odrodzić się i~wykonać bieg po trupach, prawdopodobnie tracąc przy tym trochę ekwipunku. Napastnik, który przegrał potyczkę AFK -- z~dala od klawiatury -- mógł skończyć martwy. 


Big Sister Nor żyła w~ciągłym strachu przed czyjąś śmiercią na jej rękach. 


Bitwa znów się toczyła. Wszyscy Pinkertonowie wpadli w~jej gambit, pozwalając biegać obok i~z powrotem do grzybowego lasu, skutecznie zamieniając się miejscami. Teraz kopali w~lesie, zastawiali małe zasadzki, umacniali pozycje i~rzucali miażdżący ogień ze wszystkich kierunków. Oddychające, zdyszane, triumfalne mamroczące głosy w~jej głowie i~pospiesznie brzęczący czat w~grze dawały jej wrażenie, jakby bitwa spoczywała delikatnie w~równowadze na jej palcach, a każda zmiana i~taniec były odczuwane jako drżenie delikatnych opuszków palców. 


Big Sister Nor ponownie poprosiła o bubble tea, zdając sobie sprawę, że rzeczywiście minęło bardzo dużo czasu, odkąd zamówiła ją po raz pierwszy. Tym razem nikt nie odpowiedział. Skóra na karku swędziała i~zdjęła słuchawki z~głowy. Justbob i~Mighty Krang złapali sekundę później, usuwając swoje słuchawki douszne. Z przodu Headshota nie było żadnego hałasu, żadnego normalnego, nadpobudliwego nawoływania dzieciaków-graczy, ani okrzyków gości dzwoniących do domu na tanich słuchawkach. 


Big Sister Nor wstała cicho i~szybko i~cofnęła się pod ścianę, dając znak innym, by zrobili to samo. Na ekranie zobaczyła kolejny rajd Pinkertona, który wykorzystał nagły brak strategicznego przywództwa, by zdobyć kilka małych twierdz napastników. Posuwała się powoli w~kierunku drzwi i~bardzo, bardzo, \textit{bardzo }powoli przechyliła głowę, żeby spojrzeć na framugę, po czym odepchnęła ją tak szybko, jak tylko mogła. 


\textit{UCIEKAJCIE}, powiedziała bezgłośnie do poruczników, oni ruszyli do tylnego wejścia, włazu ucieczkowego, o którym Big Sister Nor zawsze się upewniała, zanim zajęła się pracą związkową. 


Deptali im po piętach Pinkertonowie, Pinkertonowie z~prawdziwego świata, Malajowie w~robotniczych ubraniach, biedni mężczyźni, mężczyźni uzbrojeni w~grube kije i~kilka łańcuchów, mężczyźni, którzy szli do drzwi, gdy Big Sister Nor przypadkowo zajrzała za nie. 


Krzyczeli teraz za nimi podekscytowanymi i~napiętymi głosami, jak krzyki pijanych chłopców na rogach ulic, kiedy czuli odwagę liczb, hormonów i~alkoholu. To był niebezpieczny dźwięk. To był głos głupców, którzy się podjudzali. 


Big Sister Nor uderzyła obiema dłońmi w~klamkę na tylnych drzwiach, uderzając w~nie całym ciężarem ciała. Blokada w~drzwiach była zepsuta, więc odskoczyły jak pułapka na myszy i~dobrze się stało, bo poruszały się tak szybko, że dwaj Pinkertonowie czekający na zablokowanie tego wyjścia nie zdążyli zejść z~drogi. Jeden przewrócił się na tyłek, a drugi uderzył w~ścianę z~pustaków z~trzaskiem, który Big Sister Nor poczuła w~dłoniach. 


Drzwi odbiły i~uderzyły w~nią, wpychając ją z~powrotem w~The Mighty Krang, który ją złapał, popchnął z~rękami na łopatkach, urywający się oddech w~uszach. 


Znajdowali się w~ciemnej, wąskiej, śmierdzącej alejce za tą, która łączyła dwa Lorangi, małe uliczki wychodzące z~Geylang Road, i~nadszedł czas, aby uciekać i~strzelać, czyli to, co robiłaś, gdy wszystkie twoje inne plany upadły. Big Sister Nor przemyślała to wystarczająco długo, by upewnić się, że mają tylne drzwi, ale nie dalej. 


Pinkertonowie byli tuż za nimi, ale wszyscy stłoczyli się w~niewiarygodnie wąskich ciasnych uliczkach i~nikt nie mógł biegać ani poruszać się szybciej niż desperackie szuranie nogami. 


Ale potem wyrwali się do następnego Lorangu, a Big Sister Nor skręciła w~lewo, mając nadzieję, że dotrą wystarczająco daleko w~górę drogi, by zobaczyć jedzących w~całonocnych restauracjach. 


Nie udało się jej. 


Jeden z~mężczyzn rzucił w~nią swoją pałką, która trafiła ją prosto między ramiona, zapierając dech w~piersiach i~powodując, że upadła na jedno kolano. Justbob wsunęła jedną rękę w~jej bluzkę, postawił ją na nogi z~odgłosem rozdzieranego materiału i~pociągnęła dalej, ale stracili krok przez jej upadek, a teraz mężczyźni byli na nich. 


Justbob odwróciła się, warcząc, wykrzykując nieziemski krzyk, wykorzystując ten ruch jako bezwładność do dzikiego kopniaka z~półobrotu, który połączył się z~jednym z~Pinkertona, mężczyzną o zaspanych oczach i~gęstym wąsach. Stopa Justboba złapała go w~bok i~wszyscy usłyszeli dźwięk jego żeber pękających pod czubkiem jej skromnego sandała z~fałszywymi klejnotami. Sandał poleciał dalej i~z brzękiem odleciał na drogę z~tanim dźwiękiem klejonych klejnotów. 


Mężczyźni tego się nie spodziewali i~był taki moment, kiedy zatrzymali się w~miejscu, wpatrując się w~upadłego towarzysza, i~w tej chwili Big Sister Nor pomyślała, że może, tylko może, uda im się uciec. Ale pierś Justbob falowała, jej twarz wykrzywiła się we wściekłości, a ona \textit{skoczyła }na następnego mężczyznę, grubego mężczyznę w~spoconej sportowej kurtce, z~kciukami wycelowanymi w~jego oczy, a gdy do niego dotarła, mężczyzna obok niego podniósł pałkę i~uderzył w~dół, zerkając na jej wysoką, delikatną kość policzkową, a następnie uderzając w~obojczyk. 


Justbob zawyła jak ranny pies i~cofnęła się, uderzając mocno w~krocze napastnika, gdy upadała. 


Ale teraz Pinkertonowie byli na nich, ich ręce były uniesione, ich pałki uniesione wysoko, a kiedy pierwsza z~nich wbiła się w~lewą pierś Big Sister Nor, krzyknęła, a jej umysł wypełnił Affendi i~jej połamane palce, jej nierozpoznawalna, posiniaczona twarz. Gdzieś, zaledwie kilka kuszących metrów w~górę Lorang, nocni ludzie jedli wielką ucztę z~ryb i~kóz w~curry, zapachy w~powietrzu. Ale to było tam. Tutaj Big Sister Nor był nieskończenie daleko od nich, a pałki unosiły się i~opadały, a ona skuliła się, by chronić głowę, piersi, brzuch, odsłaniając delikatne nerki, delikatne krótkie żebra i~tam leżała, znosząc okres w~piekle, który trwał przez półtorej wieczności. 


\bigskip
\threeast


Ta scena jest poświęcona Chapters/Indigo, narodowemu kanadyjskiemu megachainowi. Pracowałem w~Bakka, niezależnej księgarni science fiction, kiedy Chapters otworzył swój pierwszy sklep w~Toronto i~wiedziałem, że od razu dzieje się coś wielkiego, ponieważ dwóch naszych najmądrzejszych, najlepiej poinformowanych klientów wpadło, żeby mi powiedzieć, że zostali zatrudnieni do prowadzenia sekcji science fiction. Od samego początku Chapters podniósł poprzeczkę w~kwestii tego, czym może być duża księgarnia firmowa, wydłużając godziny otwarcia, dodając przyjazną kawiarnię i~mnóstwo miejsc siedzących, instalując w~sklepie terminale samoobsługowe i~zaopatrując się w~niesamowitą różnorodność tytułów. 


\href{https://www.indigo.ca/en-ca/for-the-win-a-novel/9780765322166.html}{Chapters/Indigo} 

\bigskip
\threeast

Connor Prikkel czasami myślał o matematyce jako o pięknej dziewczynie, takiej dziewczynie, o której marzył, że się zaleca, randkuje, a nawet żeni, kiedy siedział z~tyłu każdej klasy, która nie była związana z~matematyką, marząc na jawie. Piękna dziewczyna, taka jak Jenny Rosen, która miała z~nim lekcje przez całe liceum, która zawsze znała odpowiedź, bez względu na temat, z~lekkimi piegami wokół nosa i~dziwacznym półuśmiechem. Która ubierała się w~dżinsy, które sama uszyła, w~t-shirty, które zmieniła, zszywając ze sobą wiele koszulek, by zrobić obcisłe pół-koszule, wyszukane szale, imitacje golfów. 


Jenny Rosen wydawała się mieć wszystko: piękno i~inteligencję, a przede wszystkim racjonalność: nie podobał jej się sposób, w~jaki pasują te kupione w~sklepie dżinsy, więc zhakowała własne. Nie podobały jej się t-shirty, które wszyscy nosili, więc zmieniła koszulki według własnego gustu. Była zabawna, mądra, a on był w~niej zakochany po uszy, od drugiego roku angielskiego aż do ostatniej klasy historii Ameryki. 


Przez cały ten czas byli przyjaźni, choć nie byli przyjaciółmi. Przyjaciele Connora interesowali się grami i~komputerami, przyjaciele Jenny byli osiłkami i~dzieciakami ze szkolnej gazetki. Ale przyjaźni, jasne, wystarczająco, by przywitać się na korytarzu, wystarczająco, by zostać partnerami laboratoryjnymi w~drugiej klasie fizyki (była uważnym notującym, a jej włosy pachniały \textit{niesamowicie}, a ich dłonie ocierały się o siebie sto razy w~tym semestrze). 


A potem, na ostatnim roku, zaprosił ją do kina. Potem zaprosiła go na wiec. Potem poprosił ją, żeby współpracowała z~nim przy projekcie historii Ameryki na temat chińskich pracowników kolei, który obejmował chodzenie do Chinatown po szkole, i~tam zjedli gigantyczne danie dim sum, a potem siedzieli w~parku i~rozmawiali godzinami, potem przestali rozmawiać i~zaczęli się całować. 


I jedna rzecz doprowadziła do drugiej, a pocałunki doprowadziły do więcej pocałunków, a potem wszyscy ich przyjaciele zaczęli szeptać: 

-- Słyszałeś o Connorze i~Jenny? -- a ona poznała jego rodziców, a on poznał jej. I wszystko wydawało się idealne. 


Ale nie było idealne. W ogóle. 


W ciągu czterech miesięcy, dwóch tygodni i~trzech dni, kiedy oficjalnie byli parą, mieli około 2 453 212 kłótni, z~których każda była bardziej płomienna niż poprzednia. Teoretycznie wiedział wszystko o niej, czego potrzebował. Kochała sport. Uwielbiała używać umysłu. Kochała humor. Kochała głupie komedie i~powolną muzykę bez słów. 


Odchodził więc i~planował dokładnie, jak przekazać jej te wszystkie rzeczy, podłączając jej ulubione rzeczy jak zmienne do równania, wymyślając skomplikowane schematy, by to jej dostarczyć. 


Ale to nigdy nie zadziałało. Rozpracował to, żeby mogli pójść na mecz w~AT\&T Park, a ona chciała zamiast tego pójść na koncert do Cow Palace. Zabierałby ją na nową zwariowaną komedię, a ona chciałaby wrócić do domu i~popracować nad zaległym zadaniem. Bez względu na to, jak bardzo starał się dopasować jej rzeczywistość do swojej teorii, zawsze mu się to nie udawało. 


W głębi serca wiedział, że to nie jej wina. Wiedział, że ma pewien brak, który sprawił, że żył w~wyimaginowanym świecie, który czasami uważał za ,,krainę teorii'', kraj, w~którym wszystko zachowywało się tak, jak powinno. 


Po ukończeniu studiów licencjackich z~czystej matematyki w~Berkeley, magisterskich z~przetwarzania sygnałów w~Caltech i~pierwszego roku doktoratu z~ekonomii w~Stanford, miał okazję umawiać się z~wieloma pięknymi kobietami i~za każdym razem odnajdywał siebie. zmielony na miazgę między trybami świata rzeczywistego i~świata teorii. Zrezygnował z~kobiet i~doktoratu pewnego pięknego dnia w~październiku, mówiąc profesorowi, który miał być jego doradcą, że może znaleźć kogoś innego, kto poprowadzi jego kursy matematyki na pierwszym roku, oceni jego prace i~odpowie na jego e-maile. 


Wyszedł z~kampusu Stanford na zamożne ulice Palo Alto, spakował samochód i~pojechał do nowej pracy, jako główny ekonomista działu gier Coca Coli, i~wreszcie znalazł prawdziwy świat, który pasował do pięknej elegancji. teorii-krainy. 


Coca Cola prowadziła lub udzielała franczyzy na od kilkunastu do trzydziestu światów gier w~dowolnym momencie. Liczba gier rosła lub spadała zgodnie z~brutalną, elegancką logiką ekonomii zabawy: 

\begin{center}

pewna ilość trudności 


plus  


pewna liczba twoich znajomych 


plus  


pewna liczba ciekawych nieznajomych 


plus  


pewna kwota nagrody 


plus  


pewna liczba okazji 


równa się 


\textit{zabawa }

\end{center}




To było równanie, które przyszło mu na myśl pewnego dnia na początku drugiego semestru doktoratu, grom inspiracji niczym boży palec sięgający do jego mózgu. Magia była w~znaku równości, tuż przed zabawą, ponieważ kiedy już można było wyrazić zabawę jako funkcję innych zmiennych, można było ustalić jej związek z~tymi zmiennymi  \ldots  jeśli zmniejszymy trudność i~liczbę grających znajomych, czy możemy zwiększyć nagrodę i~sprawić, by zabawa pozostała taka sama?

Ten sposób myślenia skłonił go do zadzwonienia do swojego doradcy i~skierowania się prosto do domu, gdzie pisał, rysował, gryzmolił i~myślał, myślał i~myślał, i~zgłosił chorobowe następnego dnia i~następnego \ldots  i~potem był weekend, a on wyłączył telefon, wyłączył pocztę i~komunikator i~pracował, jedząc, kiedy musiał.

Zanim zorientował się, że wpycha do ust masło z~palcami, po opróżnieniu lodówki, wiedział, że ma coś.

Nazwał je równaniami Prikkela i~w eleganckiej, czystej, abstrakcyjnej matematyce opisał związek między wszystkimi zmiennymi, które szły w~zabawę, a tym, jak zabawa równa się pieniądzowi, ponieważ ludzie płaciliby za zabawne gry i~płacili więcej za przedmioty, które miały wartość w~tych grach.

Technicznie rzecz biorąc, powinien był wysłać artykuł do swojego doradcy. Podpisał kontrakt, kiedy został przyjęty na uniwersytet, oddając na zawsze wszystkie swoje pomysły uczelni, w~zamian za obietnicę dodania ,,dr'' do jego nazwiska. Wtedy nie wydawało się to dobrym pomysłem, ale alternatywą był niesamowicie gówniany rynek pracy, więc podpisał go.

Ale nie zamierzał dać tego Stanfordowi. Nie zamierzał \textit{dać }tego nikomu. Zamierzał to \textit{sprzedać}.

Po tym nie wrócił do kampusu, ale raczej zanurzył się w~szereg wirtualnych światów, kreśląc czas w~godzinach, jaki zajęło mu wykonanie różnych zadań, i~porównując to z~ceną złota w~kolorze czarnym, szarym i~wymiany na białym rynku za bogactwo w~grze.

Każda liczba pasowała idealnie, dokładnie tam, gdzie się spodziewał. Jego równania \textit{pasują}, a świat pasuje do jego równań. W końcu znalazł miejsce, w~którym irracjonalne stało się zrozumiałe. Co więcej, mógł \textit{manipulować }światem za pomocą swoich równań.

Zdecydował się na mały handel fantazjami: korzystając ze swoich równań, przewidział, że złoto w~Klątwie Shlabotnika magazynu MAD jest szalenie niedowartościowane. To była niesamowicie zabawna gra -- a przynajmniej spełniała zabawne równanie -- ale z~jakiegoś powodu pieniądze z~gry i~elitarne przedmioty szły za grosze. Rzeczywiście, w~ciągu 36 godzin jego wyimaginowane pieniądze z~MAD były warte 130 dolarów w~wyimaginowanych prawdziwych pieniądzach.

Następnie wziął swoją stawkę 130 \$ i~zatopił ją w~czterech innych walutach gry, rozkładając swoje zakłady. Trzech z~czterech trafiło w~dziesiątkę, zwiększając jego sumę do 200 dolarów w~wyimaginowanych dolarach. Teraz postanowił wydać trochę prawdziwych pieniędzy, wiedział już, że nie wróci do kampusu, co oznaczało, że jego stypendium magisterskie wkrótce zniknie. Będzie musiał zapłacić czynsz, kiedy będzie szukał kupca do swoich równań.

Już udowodnił ku własnej satysfakcji, że potrafi przewidzieć ruch walut w~grze, ale teraz chciał zająć się dziwniejszymi obszarami ekonomii gry: elitarnymi przedmiotami, rzadkimi prestiżowymi przedmiotami, które były szalenie trudne do zdobycia w~grze. Niektóre z~nich miały pewną wrodzoną wartość -- potężna broń i~zbroje, składniki przydatnych zaklęć -- ale inne wydawały się mieć wartość przez czystą rzadkość lub nowość. Dlaczego fioletowa zbroja miałaby kosztować dziesięć razy więcej niż czerwona, skoro obie zbroje miały dokładnie taką samą wartość w~grze?

Oczywiście fioletowy był znacznie trudniejszy do zdobycia. Trzeba było albo kupić go z~niewyobrażalnymi górami złota -- więc gracze, którzy zobaczyli twojego avatara chodzącego w~niej, założyliby, że spędziłeś mnóstwo czasu, żeby na to zasłużyć -- albo wykonać fantastyczne wyczyny, aby go zdobyć, na przykład zrobić rajd z~sześćdziesięcioma graczami na prawie niemożliwego do zabicia bossa. Podobnie jak markowa metka na niezbyt imponującym elemencie garderoby, te przedmioty były cenne, ponieważ ludzie, którzy je widzieli, zakładali, że muszą one dużo kosztować lub być trudne do zdobycia, i~lepiej myśleli o kimś, kto je miał. Innymi słowy, dużo kosztują, bo \ldots  dużo kosztują!

Jak dotąd, tak dobrze, ale czy mógłbyś użyć równań Prikkela, aby przewidzieć, \textit{ile będą} kosztować? Connor tak myślał. Pomyślał, że możesz użyć formuły, która łączyłaby iloraz zabawnego czasu gry i~liczbę godzin potrzebnych do zdobycia przedmiotu, i~wyznaczyć ,,wartość'' dowolnego elitarnego przedmiotu, od fioletowej zbroi przez złote prążki na twoim statku kosmicznym do ciasta o smaku kremu bananowego wielkości bloku mieszkalnego.

Tak, to zadziała. Connor był tego pewien. Zaczął obliczać prawdziwą wartość różnych elitarnych przedmiotów, rozglądając się za niedowartościowanymi przedmiotami. To, co odkrył, zaskoczyło go: podczas gdy wirtualna waluta zwykle pozostawała blisko swojej rzeczywistej wartości, plus minus pięć procent, różnica wartości w~elitarnych przedmiotach była \textit{gigantyczna}. Niektóre przedmioty rutynowo wymieniano za dwieście lub trzysta procent ich rzeczywistej wartości -- jak przewidział w~każdym razie jego równania -- a niektóre sprzedawano za grosze.

Ani przez chwilę nie wątpił w~swoje równania, chociaż osoba bardziej pokorna lub bardziej ostrożna mogła zacząć. Nie, Connor spojrzał na ten paradoksalny obraz i~pierwszą rzeczą, jaka przyszła mu do głowy, nie było ,,Ups''. To było \textit{KUPUJ!}

I kupował. Wszystko, co było niedowartościowane w~wielkich magazynach, kupował, tak dużo, że musiał tworzyć alternatywne i~drugorzędne postacie w~wielu światach, ponieważ jego główne postacie nie mogły \textit{udźwignąć }wszystkich niedowartościowanych śmieci, które kupował. Wydał sto, dwieście, trzysta dolarów, kupując aktywa i~sporządzając arkusz kalkulacyjny ich wartości nominalnej. Na papierze był niesamowicie, niewypowiedzianie bogaty. Na papierze mógł sobie pozwolić na wyprowadzenie się ze swojego jednopokojowego mieszkania, które było trochę zbyt blisko biednego i~przerażającego East Palo Alto jak na jego podmiejski gust, kupienie McMansion gdzieś na półwyspie i~pełnoetatowe prowadzenie biznesu, spędzając dni na kupowaniu magicznej zbroi, sterowców i~płonących hamburgerów oraz wieczornych czeków.

W rzeczywistości był spłukany. Teoria mówiła, że te aktywa były szalenie niedowartościowane. Rynek powiedział inaczej. Opanował rynek kilkoma rodzajami cudownych gadżetów, ale wydawało się, że nikt tak naprawdę nie chce ich od niego kupować. Pamiętał Jenny Rosen i~wszystkie miażdżące sposoby, w~jakie teoria i~rzeczywistość mogły czasami przestać komunikować się ze sobą.

Kiedy pojawiły się pierwsze czerwone rachunki, wsadził je pod klawiaturę i~kupował dalej. Nie musiał płacić rachunku za telefon komórkowy. Nie potrzebował telefonu komórkowego, żeby kupić magiczne jaszczurki. Jego pożyczki studenckie? Nie był już studentem, więc nie wiedział, dlaczego miałby się o nie martwić, nie mogli wyrzucić go ze szkoły. Płatności samochodowe? Pozwolił im go odebrać (i zrobili to, pewnej nocy, o drugiej w~nocy, i~pomachał na pożegnanie małemu kawałkowi śmieci, gdy sprzedawca go odjechał, a potem wrócił do klawiatury). Rachunki z~karty kredytowej? Dopóki była jedna karta, która była nadal dobra, jedna karta, której mógł użyć do uiszczenia opłat za subskrypcję swoich gier, tylko to się liczyło.

Mieszkanie w~pobliżu wschodniego Palo Alto miało swoje zalety: po pierwsze, znajdowały się tam banki żywności, miejsca, w~których mógł ustawiać się w~szeregu z~innymi biednymi ludźmi po gigantyczne cegły rządowego sera, worki jednodniowego chleba, pudełka nieregularnych i~nieładnych warzywa korzeniowych. Usmażył wszystkie te warzywa podczas całodniowego festiwalu skrobiowego i~zamroził je, a potem zaczął żyć z~kanapek z~serem i~ziemniakami, i~pewnego ranka zdał sobie sprawę, że całe jego ciało i~wszystko, co z~niego wyszło -- oddech, bekanie, pierdzenie, nawet jego mocz -- pachniało kanapkami z~serem. Nie obchodziło go to. Można było kupić strusie pióropusze.

Katastrofa uderzyła: stracił rachubę, którą kartę kredytową ignorował, a połowa jego kont została zawieszona, gdy jego miesięczna opłata za subskrypcję spadła. Połowa jego bogactwa zniszczona. A druga karta nie była daleko w~tyle.

Pomyślał, że prawdopodobnie mógłby zadzwonić do rodziców, poczołgać się i~kupić bilet autobusowy do Petalumy, zaszyć się w~piwnicy swoich rodziców, lizać rany i~być kolejną małomiasteczkową porażką, która wróciła do domu z~podkulonym ogonem. Potrzebował oczywiście rolki ćwierćdolarówek i~automatu telefonicznego, ponieważ jego telefon był teraz bezwładną, niespłaconą, nawiedzoną długami cegłą. Na szczęście dla niego East Palo Alto było miejscem, w~którym spotykało się wielu ludzi, którzy byli zbyt biedni, by nawet zadłużyć się z~telefonem komórkowym, ludzi, którzy również musieli korzystać z~automatów telefonicznych.

Położył się do swojego brudnego łóżka w~środę rano i~pomyślał: \textit{Jutro, jutro do nich zadzwonię}. 

Ale następnego dnia tego nie zrobił. A w~piątek nie zrobił tego, chociaż skończył mu się rządowy ser i~nie kwalifikował się na więcej aż do poniedziałku. Mógł jeść kanapki ziemniaczane. Nie mógł już kupować aktywów, ale wciąż je śledził, obserwował, jak handlują i~identyfikował okazje, które \textit{kupiłby}, gdyby tylko miał trochę więcej płynności, trochę więcej gotówki.

W sobotę umył zęby, bo czasem o tym pamiętał, i~krwawiły mu dziąsła, a na wewnętrznej stronie jamy ustnej były rany i~\textit{teraz }był gotowy zadzwonić do rodziców, ale jakoś była 23, jak szybko przeleciał dzień i~kładli się spać o 21 każdego wieczoru. Zadzwoni do nich w~niedzielę.

A w~niedzielę --- w~niedzielę --- w~tę magiczną, cudowną niedzielę, w~niedzielę --

RYNEK RUSZYŁ!

Wyceniał aktywa, zapisywał ich wartości w~swoim arkuszu kalkulacyjnym i~zdał sobie sprawę, że to zasób, który rezerwował -- steampunkowa skórzana maska przeciwgazowa ozdobiona pękiem ogromnych skórzanych trąbek do uszu oraz mosiężnych trybów i~nitów (nie lepszych niż standardowe maska przeciwgazowa w~zniszczonym świecie ekotastrof, jakim były Rising Seas, ale nieskończenie fajniejsza) -- została już wpisana do jego arkusza kilka tygodni wcześniej. Rzeczywiście, zarezerwował maskę, gdy jej wartość pieniężna w~świecie rzeczywistym wynosiła około 0,18 USD, w~porównaniu z~4,54 USD przewidywanymi przez Equations. A teraz księgował ją po 1,24 dolara, co oznaczało, że 750 z~nich, które miał w~ekwipunku, właśnie podskoczyło ze 135 dolarów do 930 dolarów, co oznaczało zysk w~wysokości 795 dolarów.

Rozległ się dziwny dźwięk. Po chwili zdał sobie sprawę, że to jego żołądek warczący o jedzenie. Mógłby teraz sprzedać maski przeciwgazowe, wziąć 795 dolarów na jedną ze swoich kart debetowych PayPal i~zjeść jak król. Może nawet będzie w~stanie odkupić część utraconych kont i~odzyskać swoje aktywa.

Ale Connor nawet przez sekundę tego nie rozważał. Pobiegł do zlewu, napełnił wodą trzy garnki i~zaniósł je z~powrotem na biurko wraz z~filiżanką. Napełnił kubek i~wypił go, napełnił go i~wypił, napełniając żołądek wodą, aż przestał żądać napełnienia. W końcu to była Kalifornia, gdzie ludzie płacili dobre pieniądze, aby chodzić na ,,odosobnienia'' z~,,płynnym postem'' i~,,detoks''. Więc mógł przeczekać jedzenie przez dzień lub dwa \ldots  W końcu jego równania przewidywały, że te rzeczy powinny sięgnąć 3405 dolarów. Dopiero zaczynał.

A teraz maski przeciwgazowe się drożały. Wstawał, szedł do łazienki -- jego nerki z~pewnością ćwiczyły! -- i~wracał, aby sprawdzić aukcje na oficjalnych stronach wymiany i~na czarnorynkowych, na których spotykali się hodowcy złota. Miał mały wzór na obliczenie prawdziwej ceny, używając tych dwóch cen jako swego rodzaju latarni. Bez względu na to, jak to obliczył, jego maski przeciwgazowe szły w~górę.

I tak, niektóre z~jego innych aktywów również rosły. Pies-robot, z~1,32 dolara do 1,54 dolara, wciąż dość daleko od przewidywanych 8,17 dolara, ale był właścicielem tysiąca rzeczy, co oznaczało, że właśnie zarobił tu 1318,46 dolara i~dopiero zaczynał.

Ceny szły w~górę i~w górę, gdy aktywa po aktywach osiągały wzrost, i~zaczął podejrzewać, że jego szał kupowania aktywów zbiegł się z~kryzysem międzyświatowym we wszystkich wirtualnych gospodarkach, co odpowiadało za ogromne ilości niedowartościowanych aktywów, które znalazł leżące luzem. Prawdopodobnie istniał interesujący powód, dla którego te wszystkie wirtualne gospodarki załamały się naraz, ale było to coś do przestudiowania innego dnia. W rzeczywistości bardziej interesował go fakt, że gospodarki odbijały się, gdy siedział na górach tanich, wyimaginowanych gadżetów, bibelotów, czubków i~białych słoni, a ich wartości nabierały szalonego tempa.

A teraz nadszedł czas, aby zamienić część tych aktywów na pieniądze, a część na żywność, czynsz i~opłacone rachunki. Jego kolekcja przegubowych macek z~,,Nemo's Adventures on the Ocean Floor'' na dnie oceanu ładnie dojrzewała -- kupił je po 0,22 dolara, wycenił je na 3,21 dolara, a teraz kosztowały 3,27 dolara -- więc je sprzedał i~żałował, że to kupił tylko 400 sztuk. Mimo to zdołał je sprzedać z~poręcznym zyskiem w~wysokości 1150 dolarów (do czasu, gdy sprzedał ich 300, cena znowu zaczęła spadać, gdy podaż macek wzrosła, a popyt zmalał).

Pieniądze spłynęły na jego konto PayPal i~wykorzystał je, aby zamówić trzy pizze, galon soku pomarańczowego i~dziesięć pudełek sałatek, spłacił zawieszone konta i~wysłał właścicielowi 400 dolarów zamiast 3500 dolarów należnych za dwa miesiące czynszu wraz z~listem błagalnym obiecującym spłatę reszty w~ciągu dnia lub dwóch.

Czekając na przybycie pizzy, zdecydował, że lepiej weźmie prysznic, ogoli się i~spróbuje zrobić coś z~włosami, które od miesiąca, bez szczotki do włosów, zaczęły zamieniać się w~dredy. W końcu po prostu wyciął kołtuny i~po raz pierwszy od tygodnia ubrał się w~coś innego niż swój brudny szlafrok, podziwiając, jak jego dżinsy zwisają z~jego wydatnych bioder, jak koszulka przylega do jego zmarnowanej klatki piersiowej, jego żebra niczym ksylofon przez bladą skórę. Otworzył wszystkie okna, świadom smrodu ciała i~stęchłego powietrza przefiltrowanego przez komputer w~swoim mieszkaniu, i~zdał sobie sprawę, że jest ranek, i~podziękował swoim szczęśliwym gwiazdom, że mieszka w~miasteczku uniwersyteckim, gdzie można dostać pizzę dostarczoną o 8:30.

Po zjedzeniu pierwszej pizzy zaczął rzygać, trafiając większością do wielkiego garnka, w~którym trzymał wodę pitną, duże kawałki skórki i~pepperoni, cuchnące kwaśnym żołądkiem. Nie pozwolił, żeby go to zniechęciło. Jego konto PayPal napuchło do 50 000 \$, a on dopiero zaczynał. Przerzucił się na sałatki i~sok, sądząc, że przyzwyczajenie się do jedzenia zajmie trochę czasu, a nie miał teraz czasu na długą przerwę biologiczną. Jego ciało będzie musiało poczekać. Zamówił dzban z~kawą z~miejsca, które zajmowało się spotkaniami firmowymi, tego rodzaju rzeczy, która mieściła 80 filiżanek, i~dorzucił talerz pokrojonych warzyw i~trochę ciastek.

Sprzedawanie stało się teraz łatwiejsze. Gospodarki się odbijały, a z~tonu wiadomości z~podziękowaniami, jakie otrzymał od swoich kupujących, zrozumiał, że w~powietrzu wisi rodzaj paniki odwróconej, poczucie, że gracze na całym świecie zaczynają się tym martwić, gdyby teraz nie kupili tego złomu, nigdy nie byliby w~stanie go kupić, ponieważ ceny będą rosły i~rosły w~nieskończoność.

I właśnie wtedy miał swój drugi wielki błysk, po raz drugi, gdy palec Boga sięgnął i~dotknął jego umysłu z~siłą, która zrzuciła go z~krzesła i~kazała mu chodzić po salonie jak tygrys, mamrocząc do siebie.

Kiedyś, kiedy pracował nad swoimi magisterium, brał udział w~studiach dla kumpla z~wydziału ekonomii. Zamknęli dwudziestu pięciu studentów w~pokoju i~dali każdemu z~nich żeton do pokera. 

-- Z tymi chipami możecie zrobić, co chcecie -- powiedział eksperymentator. -- Ale możecie zechcieć je zatrzymać. Co godzinę, o każdej pełnej godzinie, będę otwierał te drzwi i~dawał wam dwadzieścia dolarów za każdy żeton do pokera, który trzymasz. Zrobię to osiem razy, przez następny osiem godzin. Potem otworzę drzwi po raz ostatni i~będziesz mógł wrócić do domu, a twoje żetony do pokera będą bezwartościowe, chociaż będziecie mogli zatrzymać wszystkie pieniądze, które zdobyliście w~trakcie eksperymentu.

Parsknął i~przewrócił oczami na innych studentów, którzy w~większości robili to samo. Zapowiadało się dłuuuugie osiem godzin. W końcu wszyscy wiedzieli, jaka jest wartość żetonów: 160\$ w~pierwszej godzinie, 140\$ w~następnej, 120\$ w~następnej i~tak dalej. Jaki byłby sens wymieniania żetonów pokerowych z~kimkolwiek innym za mniej niż to, co było warte?

Przez pierwszą godzinę wszyscy siedzieli i~narzekali, że to wszystko jest nudne. Następnie eksperymentator wrócił do pokoju z~tacą kanapek i~25 dwudziestodolarówkami. 

-- Poproszę żetony do pokera -- powiedział, a oni posłusznie wyciągnęli swoje żetony i~jeden po drugim, każdy otrzymał nowy, świeży banknot 20-dolarowy.

-- Jedna poszła, pozostało siedem -- powiedział ktoś, gdy eksperymentator wyszedł. Kanapki były w~dużej mierze nietknięte. Czekali. Flirtowali w~znudzony sposób lub rozmawiali. Minęła godzina.

Potem, po 55 minutach, jeden facet, prawdziwy żartowniś z~rudymi włosami i~psotnymi piegami, wstał ze zniszczonej starej pomarańczowej sofy i~zwrócił się do najładniejszej dziewczyny w~pokoju, uroczej Chinki z~krótkimi włosami i~domowej roboty ubrania, które przypominały Connorowi modę Jenny, i~powiedział: 

-- Wypożycz mi żeton do pokera na pięć minut? Zapłacę ci 20 dolarów.

To rozwaliło cały pokój. To była doskonała demonstracja absurdalności siedzenia i~czekania na 20-dolarową godzinę. Chińska dziewczyna też się roześmiała i~uroczyście się wymienili. Pięć minut później wszedł absolwent, z~kolejnym zwitek dwudziestek i~chłodziarką wypełnioną koktajlami w~tetrapakach. 

-- Poproszę żetony do pokera -- powiedział, a joker podniósł swoje dwa żetony. 

Wszyscy uśmiechnęli się do siebie, jakby mieli jednego ucznia, a on też się uśmiechnął i~wręczył rudzielcowi dwie dwudziestki. Chińska dziewczyna uniosła swoją dodatkową dwudziestkę, pokazując, że ma to samo, co wszyscy inni. Kiedy odszedł, Rudy oddał jej chip. Schowała go do kieszeni i~wróciła do siedzenia w~jednym z~zakurzonych starych foteli.

Pili koktajle. Były szemrane rozmowy i~wydawało się, że wiele osób wymienia swoje żetony tam i~z powrotem. Connor roześmiał się, widząc to, i~nie był jedynym, ale to wszystko było zabawne. W końcu dwadzieścia dolarów to aktualna stawka za godzinę wynajmu, dokładnie i~doskonale racjonalna suma.

-- Daj mi swój żeton na 20 minut za 5 dolarów? -- Pytająca znajdowała się w~młodszym końcu pokoju, miała około 22 lat, z~delikatnym, kulturalnym południowym akcentem. Była też bardzo ładna. Sprawdził zegar na ścianie: 

-- Jest dopiero wpół do drugiej -- powiedział. -- Jaki jest sens? 

Uśmiechnęła się do niego. 

-- Zobaczysz. 

Wypisano banknot pięciodolarowy i~żeton do pokera opuścił jego opiekę. Ładna dziewczyna z~południa rozmawiała z~inną dziewczyną i~po chwili 10\$ wymieniło ręce, dość rzucając się w~oczy. 

-- Hej -- zaczął, ale dziewczyna z~południa mrugnęła do niego i~zamilkł.

Z niepokojem patrzył na zegar, czekając, aż minie 20 minut. 

-- Potrzebuję z~powrotem chipa -- powiedział do dziewczyny z~południa.

Wzruszyła ramionami. 

-- Musisz z~nią porozmawiać -- powiedziała, wskazując kciukiem przez ramię, po czym ostentacyjnie wyciągnęła z~plecaka powieść w~miękkiej okładce -- \textit{Źródło} Ayn Rand -- i~schowała w~niej nos. Poczuł skomplikowane uczucie: chciał się śmiać i~chciał krzyczeć na dziewczynę. Wybrał śmiech, świadom tego, że wszyscy go obserwują, i~podszedł do drugiej dziewczyny, która była wysoka i~solidnie zbudowana, o rzeczowym wyglądzie, który idealnie pasował do jej rzeczowych ubrań i~fryzury.

-- Tak? -- spytała, kiedy podszedł do niej.

-- Masz mój chip -- powiedział.

-- Nie -- odpowiedziała. -- Nie mam. 

-- Ale chip, który ci sprzedała, tylko jej wypożyczyłem.

-- Musisz to załatwić z~nią -- powiedziała dziewczyna, która miała jego chip. 

-- Ale to mój chip -- powiedział. -- To nie było jej do sprzedania. 

Nie chciał powiedzieć, że \textit{jestem też dość onieśmielony przez każdego, kto ma czelność wykonać taki numer}. Czy to była jego wyobraźnia, czy dziewczyna z~południa uśmiechała się do siebie, zadowolonym z~siebie uśmiechem?

-- Obawiam się, że to nie mój problem -- powiedziała. -- Szkoda.

Teraz \textit{wszyscy }obserwowali bardzo uważnie i~poczuł, że się rumieni, traci zimną krew. Przełknął ślinę i~spróbował przybrać przekonujący uśmiech. 

-- Tak, chyba naprawdę powinienem być bardziej ostrożny, komu ufam. Sprzedasz mi mój chip? 

-- Mój chip -- powiedziała, podrzucając go w~powietrze. Kusiło go, by spróbować go złapać z~powietrza, ale to mogło doprowadzić do walki zapaśniczej właśnie tutaj, na oczach wszystkich. Ale wstyd!

-- Tak -- powiedział. -- Twój chip. 

-- Dobrze -- powiedziała. -- 15 dolarów.

\textit{-- Umowa -- powiedział, myśląc:}Zarobiłem już tu 45 dolarów, stać mnie na odpuszczenie 15 dolarów.\textit{ }

-- Za siedem minut -- powiedziała. Spojrzał na zegar: była 11:54. Za siedem minut, dostanie jego 20 dolarów. Poprawka: \textit{jej }20 dolarów.

-- To niesprawiedliwe -- powiedział.

Uniosła na niego jedną brew, unosząc ją tak wysoko, że wydawało się, że dotknie jej linii włosów. 

-- Naprawdę? Myślę, że ten żeton jest wart 120 dolarów. 15 dolarów wydaje się okazją.

-- Dam ci 20 dolarów -- powiedział rudy.

-- 25 dolarów -- powiedział ktoś inny, śmiejąc się.

-- Dobrze, dobrze -- powiedział pospiesznie Connor, teraz rumieniąc się tak mocno, że poczuł się oszołomiony. -- 15 dolarów.

-- Za późno -- powiedziała. -- Cena wynosi teraz 25 dolarów.

Usłyszał chichot pokoju, poczuł, że przygotowuje się do wykrzyczenia nowej ceny  \ldots  40 dolarów? 60 dolarów?  \ldots  i~szybko rzucił ,,25 dolarów'' i~wyciągnął portfel.

Dziewczyna zabrała mu pieniądze  \ldots  skąd wiedział, że da mu chip? Poczuł się jak idiota, gdy tylko dał pieniądze, a potem wszedł eksperymentator. 

-- Lunch! -- zawołał, jadąc wózkiem wyładowanym sałatkami w~pudełkach, wegetariańskim sushi i~kilkoma wiaderkami smażonego kurczaka. 

-- Żetony do pokera! -- Rozdano dwudziestki.

Dziewczyna z~jego pieniędzmi spędziła zbyt dużo czasu na wybieraniu lunchu, po czym w~końcu odwróciła się do niego z~udawanym zaskoczeniem i~powiedziała: 

-- O tak, tutaj -- i~wręczyła mu chip. Facet z~rudymi włosami zachichotał.

Cóż, to był początek gry, rzecz, która zmieniła następne pięć godzin w~jedno z~najbardziej intensywnych, emocjonalnych przeżyć, w~jakich kiedykolwiek brał udział. Gracze tworzyli frakcje kupujące, wykupywali innych graczy, gromadzili swoje bogactwa. Ktoś podstępnie zmienił zegar ścienny, a potem wszyscy spędzili 30 minut, kłócąc się o to, kogo zegarek lub telefon jest dokładniejszy, dopóki badacz nie wrócił z~kilkoma dwudziestkami.

W szóstej godzinie eksperymentu Connor nagle zdał sobie sprawę, że jest w~mniejszości, odstającym od dwóch wielkich frakcji: z~których jedna kontrolowała prawie wszystkie żetony do pokera, a druga prawie całą gotówkę. Zostały tylko dwie godziny, co oznaczało, że jego pojedynczy żeton był wart 40 \$.

I coś zaczęło go gryźć w~brzuch. Strach. Zazdrość. Panika. Pewność, że po zakończeniu eksperymentu będzie jedynym biednym, jedynym, który nie ma ogromnego pliku gotówki. Bystrzy handlarze wokół nich w~jakiś sposób dopracowali się do pozycji władzy i~bogactwa, podczas gdy on był niepewny przez swoje złe wczesne doświadczenia i~trzymał się, podczas gdy wszyscy inni tworzyli rynek.

Więc postanowił kupić więcej żetonów. Albo sprzedać swój chip. Nie obchodziło go które, po prostu chciał być bogaty.

Nie tylko on: po siódmej godzinie na całym rynku wybuchła furia kupna i~sprzedaży, co nie miało \textit{cholernego sensu}, ponieważ \textit{teraz }wszystkie żetony były warte dokładnie 20 dolarów za sztukę, a w~ciągu zaledwie kilku minut, byłyby absolutnie bezwartościowe. Powtarzał sobie to w~kółko, ale też licytował coraz mocniej o żetony. Na szczęście nie był najbardziej przestraszoną osobą w~pokoju. Okazało się, że to rudowłosy, który ścigał żetony jak palant w~pogoni za kamieniem, tracąc cały niezobowiązujący luz, z~którym zaczął, i~goniąc za żetonami za pieniądze, za weksle.

Oto rzecz, gotówka powinna być \textit{królem}. Za godzinę gotówka wciąż będzie coś warta. Żetony do pokera były jak bańki mydlane, które zaraz pękną. Ale ci, którzy trzymali żetony, byli królami i~królowymi Gry, Rynku. W ciągu siedmiu krótkich godzin zostali przyzwyczajeni do myślenia o czipach jak o bankomatach, które wypluwały dwudziestki, i~chociaż ich racjonalne umysły wiedziały lepiej, ich serca podpowiadały im, żeby skonfrontować się z~czipem.

O 4:53, siedem minut przed ostatecznym wypłatą żetonu, sprzedał go kobiecie z~Fontanny za 35 dolarów, uśmiechając się do niej, aż się odwróciła i~sprzedała rudzielcowi za 50 dolarów. Badacz wszedł do pokoju, rozdał swoje dwudziestki, podziękował im za poświęcony czas i~odesłał ich.

Nikt nie spojrzał nikomu w~oczy, kiedy odchodzili. Nikt nie oferował nikomu numeru telefonu, adresu e-mail ani wiadomości błyskawicznych. To było tak, jakby wszyscy po prostu zrobili coś, czego się wstydzili, jakby brali udział w~pobiciu tłumu lub paleniu czarownic, a teraz po prostu chcieli uciec. Daleko stąd.

Przez lata Connor zastanawiał się nad manią, która zawładnęła tym pokojem pełnym skądinąd zdrowych ludzi, która znalazła dom we własnym sercu, napędzała ich jak nałóg. Co go sprowadziło do tego haniebnego miejsca?

Teraz gdy obserwował, jak wartość jego wirtualnych aktywów rośnie, rośnie i~rośnie, rośnie wyżej, niż przewidywały jego równania, wyżej niż jakakolwiek rozsądna osoba powinna być skłonna na nie wydać, \textit{zrozumiał}.

Emocja, która kierowała nimi w~laboratorium tego eksperymentatora, która kierowała niewidzialnymi oferentami na całym świecie: to nie była chciwość.

To była \textit{zazdrość}.

Chciwość była przewidywalna: jeśli jeden kawałek pizzy jest dobry, to ma sens, gdy intuicja podpowiada, że pięć lub dziesięć kawałków byłoby jeszcze lepsze.

Ale zazdrość nie dotyczyła tego, co było dobre: chodziło o to, co ktoś inny uważał za dobre. To diabeł szeptał ci do ucha o samochodzie twojego sąsiada, jego pensji, ubraniach, jego dziewczynie, lepszych niż twoje, droższych niż twoje, piękniejszych niż twoje. To był sztylet wbity w~twoje serce, który mógł doprowadzić cię od szczęścia do nieszczęścia w~sekundę, nie zmieniając ani jednej rzeczy w~twojej sytuacji. Może zmienić twoje idealne życie w~idealny bałagan, po prostu porównując je z~kimś, kto miał więcej/lepiej/ładniej.

To zazdrość napędzała tę burzę kupowania i~sprzedawania w~laboratorium. Rudowłosy, wypisujący weksel IOU i~opróżniający portfel: kierował nim strach, że stracił to, co otrzymywała reszta. Connor sprzedał swój żeton w~ciągu ostatniej godziny, ponieważ wydawało się, że wszyscy inni się wzbogacili, sprzedając swoje. Mógł zachować swój czip dla siebie przez osiem godzin, wyjść o 160 dolarów bogatszy i~wykorzystać ten czas na naukę, drzemkę lub jogę z~tyłu pokoju. Ale poczuł ten syreni śpiew: \textit{Ktoś inny się wzbogaca, dlaczego nie ty}? 

A teraz rynki działały i~\textit{wszystko }drożało: jego kolekcja czerwonych ogonów wołowych (przydatnych w~przygotowaniu zaklęcia Revelations w~Endtimes) powinna sprzedawać się po 4,21 dolara za sztukę. Kupił je po 2,10 dolara za sztukę. Zostały one obecnie wycenione na \textit{14,51 }dolarów za sztukę.

To było szalone.

To było cudownie.

Connor wiedział, że to nie może trwać długo. W końcu rynek zda sobie sprawę, że były one zawyżone, tak jak rynek niedawno zdał sobie sprawę, że były one zaniżone. Licytacja ustałaby. Ostatnia, najbardziej przerażona osoba, która kupiła zawyżony zasób gry, nie byłaby w~stanie go odwrócić, musiałaby za niego zapłacić.

Racjonalnie przypuszczał, że powinien sprzedawać według przewidywanej przez równanie liczby. Cokolwiek wyższego było tylko zakładem na czyjąś irracjonalność. Ale nadal  \ldots  czy naprawdę byłoby lepiej, gdyby spuścił swoje 50 wołowych ogonów za 200 dolarów, kiedy mógłby poczekać kilka minut i~sprzedać je za 700 dolarów? Nie musiało to być wszystko albo nic. Podzielił swój majątek na dwie grupy; te, które kupił najtaniej, odłożył na bok, by pozwolić im wzrosnąć tak wysoko, jak tylko mogły. Reprezentowały jego inwentarz najniższego ryzyka, najtańsze straty do wchłonięcia. Pozostałe aktywa sprzedał w~chwili, gdy osiągnęły wartość przewidywaną przez jego równania.

Szybko wyprzedał się z~drugiej grupy, pozostawiając sobie możliwość obserwowania, jak aktywa spekulacyjne rosną coraz wyżej. Na swoim komputerze miał otwarty tuzin gier, przerzucając się od jednej do drugiej, monitorując rozmowy i~powiązane z~nimi strony internetowe i~targowiska, orientując się, dokąd zmierzają. Filtrując tweety i~wiadomości o statusie w~sieciach społecznościowych, poczuł dziwne poczucie znajomości: wariowali w~sposób, który był niemal identyczny z~szaleństwem, które ogarnęło grupę w~eksperymencie z~żetonami pokerowymi. W głębi serca wszyscy wiedzieli, że pawie pióropusze i~fioletowa zbroja są zdecydowanie przewartościowane, ale wiedzieli też, że niektórzy ludzie się na nich bogacą i~że jeśli ceny będą rosły, sami nigdy nie będą w~stanie ich posiadać.

Nieważne, że \textit{wcześniej }nie chcieli ich mieć, oczywiście! Najważniejszą rzeczą nie było to, czego potrzebowali lub kochali, ale pomysł, że ktoś inny miałby coś, czego oni nie mogli mieć.

Connor dokonał drugiego wielkiego odkrycia: zazdrość, a nie chciwość, była najpotężniejszą siłą w~każdej gospodarce.

(Później, gdy Connor pisał o tym artykuły do błyszczących magazynów i~podróżował po całym świecie, aby o tym rozmawiać, wiele osób z~działów marketingu wskazywało, że wiedzieli o tym od pokoleń, spędzili wieki na tworzeniu reklam, które celowały w~splot słoneczny zazdrości. To była prawda, musiał przyznać, ale prawdą było również, że praktycznie każdy ekonomista, jakiego kiedykolwiek spotkał, uważał ludzi z~marketingu za bandę płytkich, głupich nadwornych błaznów o słabych umiejętnościach matematycznych i~dlatego w~dużej mierze ich ignorował)

Przyglądał się rosnącej zazdrości i~starał się to wszystko wyczuć, śledzić nastroje, gdy bulgotały. Było to trudne -- praktycznie niemożliwe, szczerze -- ponieważ wszystko było rozproszone i~nikt nie napisał programów do czatów, gier, sieci społecznościowych i~twitterów, aby śledzić tego rodzaju rzeczy. Skończył z~tuzinem otwartych przeglądarek, każda z~dziesiątkami kart, przeglądając je z~dużą szybkością, nie czytając dokładnie, ale przeglądając, pochłaniając \textit{poczucie}, jak się sprawy mają. Czuł pieniądze, myśli i~towary balansujące na jego palcach, czuł, jak ich ciężar przesuwa się tam i~z powrotem.

I tak poczuł to, kiedy sprawy zaczęło dziać się źle. To była garść subtelnych wskaźników, skok cen na tym rynku, radosny tweet gracza, który właśnie odkrył łatwego do zabicia minibossa z~ogromnym magazynem wypchanym pawimi piórami. Bańka zawiści runęła. Ktoś wybił dziurę i~powietrze świsnęło.

\noindent SPRZEDAWAĆ!

W tym momencie jego aktywa spekulacyjne były teoretycznie warte ponad \textit{czterysta tysięcy dolarów}, ale dziesięć minut później było to 250 000 dolarów i~spadało jak skała. To też znał -- strach -- strach, że wszyscy inni wyszli, gdy sprawy były w~porządku, gdy wszystkie muzyczne krzesła były zajęte, że byłeś najbardziej przerażoną osobą w~łańcuchu sterroryzowanych ludzi, którzy kupowali drogie śmieci, ponieważ ktoś jeszcze bardziej przestraszony kupiłby je od ciebie.

Ale Connor mógł wznieść się ponad strach, przelecieć nad nim, przerzucić swoje aktywa w~metodyczny, szybki sposób. Wyszedł z~ponad 120 000 \$ w~gotówce, plus 80 000 \$, które otrzymał ze swoich ,,racjonalnie wycenionych'' aktywów, a teraz jego konta PayPal były pełne zysków i~było po wszystkim.

Tyle że tak nie było.

Jeden po drugim jego konta w~grach zaczęły się zamykać, jego postacie wyrzucane, a hasła zmieniane. Był bezwładny z~wyczerpania, ręce mu drżały, gdy pisał i~ponownie wpisywał swoje hasła. A potem zauważył nowy e-mail od czterech firm, które kontrolowały dwanaście gier, w~które grał: wszystkie odcięły go za naruszenie Terms of Service. Konkretnie: ,,Wtrącał się w~ekonomię gry, angażując się w~rozgrywkę, która mogła wywołać panikę finansową''.

-- Co to do cholery znaczy? -- krzyknął do komputera, opierając się chęci rzucenia myszą o ścianę. Nie spał już od ponad 48 godzin, zarobił setki tysięcy dolarów w~zwykły weekend i~został uświęcony piorunem uświadomienia sobie sposobu, w~jaki działała światowa gospodarka. Och, i~sprawdził swoje równania.

Mógł rozwiązać ten problem później.

Nie poszedł nawet do łóżka. Zwinął się na podłodze, w~gnieździe pudełek po pizzy i~kocach, i~spał przez 18 godzin, aż obudził go komornik, który przyszedł go eksmitować za trzymiesięczne opóźnienie w~czynszu.

\bigskip
\threeast

Ta scena jest poświęcona San Francisco Booksmith, która znajduje się w~historycznej dzielnicy Haight-Ashbury, zaledwie kilka drzwi dalej od Ben and Jerry's, dokładnie na rogu Haight i~Ashbury. Ludzie z~księgarni naprawdę wiedzą, jak prowadzić wydarzenie autorskie, kiedy mieszkałem w~San Francisco, schodziłem na dół, aby posłuchać przemawiających niesamowitych pisarzy (William Gibson był niezapomniany). Produkują również małe karty kolekcjonerskie w~stylu bejsbola dla każdego autora, mam tam dwie z~moich własnych występów.


\href{https://en.wikipedia.org/wiki/HTTP_404}{\textit{Booksmith}: 1644 Haight St. San Francisco CA 94117 USA +1 415 863 8688}

\bigskip
\threeast


Yasmin już nie widywała Mali. Jeśli nie byłaś w~gangu, ,,Generał Robotwallah'' nie chciała z~tobą rozmawiać.

A Yasmin nie chciała być w~gangu.

Ona też została odwiedzona przez Big Sister Nor. Kobieta mówiła z~sensem. Wykonali całą pracę, prawie nic nie zarobili. Nie tylko w~grach, jej rodzice przez całe życie harowali dla innych, a ci inni stawali się coraz bogatsi i~bogatsi, a oni zostali w~Dharavi.

Pan Banerjee zapłacił armii Mali więcej, niż mogło zarobić jakiekolwiek inne dziecko ze slumsów, to prawda, i~otrzymywali pieniądze za grę, co wydawało się cudem -- na początku. Ale im więcej Yasmin o tym myślała, tym mniej cudowne się to stawało. Big Sister Nor pokazała jej zdjęcia w~grze, przedstawiające pracowników, którym przeszkadzali w~pracy. Niektórzy byli w~Indonezji, niektórzy w~Tajlandii, niektórzy w~Malezji, niektórzy w~Chinach. Wielu z~nich było w~Indiach, na Sri Lance, w~Pakistanie i~Bangladeszu, skąd pochodzili jej rodzice. Wyglądali jak ona. Wyglądali jak jej przyjaciele.

A \textit{oni }też po prostu próbowali zarabiać pieniądze. Po prostu próbowali pomóc swoim rodzinom, tak jak zrobiła to armia Mali. 

-- Nie musisz krzywdzić innych pracowników, aby przeżyć -- powiedziała jej Big Sister Nor. -- Wszyscy możemy razem prosperować.

Dzień po dniu Yasmin wkradała się do kafejki internetowej pani Dibyendu, zanim Armia się spotkała -- nie u pani Dibyendu, ale w~nowym sklepie internetowym nieco dalej, w~pobliżu kobiecego kolektywu papadamów -- i~rozmawiała z~Big Sister Nor i~słuchała jej opowieści o tym, jak to może być.

Nigdy o tym nie rozmawiała z~nikim innym z~armii. O ile wiedzieli, była lojalną porucznik Mali, silną i~niezawodną. Musiała wymusić dyscyplinę w~szeregach, co oznaczało powstrzymywanie chłopców od zbytniego kłócenia się, a dziewcząt przed łączeniem się ze sobą syczącymi, szeptanymi plotkami. Dla nich była surowym, potężnym wojownikiem, kimś, komu się jest bezwarunkowo posłusznym w~bitwie. Nie mogła podejść do nich i~powiedzieć: 

-- Czy kiedykolwiek myślałeś o walce o robotników, zamiast walki przeciwko nim?

Nieważne, jak bardzo Big Sister Nor jej tego chciała.

-- Yasmin, oni cię słuchają, la, kochają cię i~patrzą na ciebie. Sama to mówisz. -- Jej hindi było dziwnie akcentowane i~pełne angielskich i~chińskich słów. Ale było wiele zabawnych akcentów w~Dharavi, dialektach i~językach całej Matki Indii.

W końcu zgodziła się to zrobić. Nie po to, żeby rozmawiać z~żołnierzami, ale żeby porozmawiać z~Malą, która była jej przyjaciółką, odkąd Yasmin znalazła ją niosącą do domu wielki worek ryżu ze sklepu pana Bhatta ze swoim młodszym bratem, wyglądającą na zagubioną i~przestraszoną w~zaułkach Dharavi. Od tamtego czasu ona i~Mala były nierozłączne, a Yasmin zawsze była w~stanie powiedzieć jej wszystko.

-- Dzień dobry, Generale -- powiedziała, podchodząc do Mali, idąc do lokalnego kranu z~konewką w~każdej ręce. Wzięła jedną puszkę od Mali, wzięła jej teraz wolną ręką i~ścisnęła siostrzańsko.

Mala uśmiechnęła się do niej i~odwzajemniła, a uśmiech był jak stara Mala, Mala sprzed powstania Generała Robotwallah. 

-- Dzień dobry, poruczniku. 

 Mala była ładna, kiedy się uśmiechała, jej poważne oczy były pełne figlarności, a jej kwadratowe małe zęby były wyeksponowane. Kiedy tak się uśmiechała, Yasmin czuła się, jakby miała siostrę.

Rozmawiały cicho, czekając na kran, przekazując gupszup o swoich rodzinach. Matka Mali spotkała w~fabryce pana Bhatta mężczyznę, którego rodzice przybyli do Bombaju pokolenie wcześniej, ale z~tej samej wioski. Dorastał na opowieściach o życiu w~wiosce i~przez cały dzień mógł słuchać mamaji Mali opowiadających o tej ziemi obiecanej. Był łagodny i~chętnie się śmiał, co Mala aprobowała. Nani Yasmin, jej babcia, kontaktowała się ze swatką w~Londynie i~groziła, że znajdzie tam męża Yasmin, chociaż jej rodzice nic z~tego nie robili.

Kiedy już miały wodę, Yasmin pomogła Mali zanieść ją z~powrotem do jej budynku, ale powstrzymała ją, zanim tam dotarli, pod osłoną zwisającego zsypu, którym robotnicy używali do zrzucania tektury z~fabryki na drugim piętrze do przewoźników na ziemi. Fabryka jeszcze nie ruszyła, więc teraz było cicho.

-- Big Sister Nor poprosiła mnie, żebym z~tobą porozmawiała, Mala.

Mala zesztywniała, a jej uśmiech zniknął. Nie rozmawiały już jak siostry. Twarde spojrzenie, spojrzenie generała Robotwallah, było w~jej oczach. 

-- Co ona Ci powiedziała? 

-- To samo, co powiedziała Tobie, zdaje się. Ludzie, z~którymi walczymy, są również pracownikami, takimi jak my. Dzieciakami, takimi jak my. Że możemy żyć bez krzywdzenia innych. Że możemy z~nimi pracować, z~robotnikami na całym świecie \ldots 

Mala uniosła rękę, rozkaz milczenia w~sali wojennej. 

-- Słyszałam to, słyszałam to. I co, myślisz, że ma rację? Chcesz to wszystko porzucić i~wrócić do tego, jak miałyśmy wcześniej? Z powrotem do szkoły, z~powrotem do pracy, z~powrotem do braku pieniędzy i~braku jedzenia i~lęku przez cały czas?

Yasmin nie pamiętała, żeby cały czas się bała, a szkoła nie była taka zła, prawda? 

-- Mala -- powiedziała uspokajająco. -- Chciałam tylko z~tobą o tym porozmawiać. Uratowałaś nas, nas wszystkich w~armii, wyrwałaś nas z~nędzy do bogactwa i~pracy. Ale my pracujemy i~pracujemy dla pana Banerjee, dla jego szefów i~nasi Rodzice pracują dla szefów, a dzieci, z~którymi walczymy w~grze, pracują dla szefów, a ja po prostu myślę \ldots  -- Wzięła oddech. -- Myślę, że mamy więcej wspólnego z~pracownikami niż z~szefami. Być może, jeśli wszyscy się spotkamy, będziemy mogli domagać się od nich wszystkich lepszych warunków \ldots 

Oczy Mali płonęły. 

-- Chcesz dowodzić armią, prawda? Chcesz zabrać nas w~tę swoją misję, aby \textit{zaprzyjaźnić }się ze wszystkimi, dołączyć do nich w~walce z~panem Banerjee i~szefami, panem Bhattem, który jest właścicielem fabryki i~ludźmi, którzy są właścicielami gry? A jak będziesz walczyć, mała Yasmin? Zamierzasz wywrócić cały świat, aby wreszcie stał się \textit{sprawiedliwy } i~\textit{miły }dla wszystkich?

Yasmin cofnęła się, ale wzięła głęboki oddech i~spojrzała w~przerażające oczy generała. 

-- Co jest złego w~dobroci, Mala? Co jest takiego strasznego w~przetrwaniu bez krzywdzenia innych ludzi?

Warga Mali wykrzywiła się w~grymasie czystego obrzydzenia. 

-- Nie wiesz już, Yasmin? Jeszcze tego nie rozgryzłaś? Rozejrzyj się wokół nas! 

Machała dziko swoją puszką z~wodą, niemal uderzając w~staruszkę, która przechodziła obok, niosąc własne puszki z~wodą. 

-- Rozejrzyj się! Wiesz, że na całym świecie są ludzie, którzy mają dobre samochody i~wyśmienite posiłki, służących i~pokojówki? Na całym świecie są ludzie, którzy mają \textit{toalety}, Yasmin i~\textit{bieżącą wodę}, i~którzy mają swoje własne sypialnie z~pięknym łóżkiem do spania! Czy myślisz, że ci ludzie zrezygnują ze swoich pięknych łóżek, wspaniałych domów i~samochodów dla Ciebie? A jeśli nie zrezygnują, skąd to wezmą? Jak wiele łóżek i~samochodów tam jest? Czy wystarczy dla nas wszystkich? Na tym świecie, Yasmin, po prostu nie wystarczy. To znaczy, że zawsze będą zwycięzcy i~przegrani, jak w~każdej grze, a ty decydujesz, czy chcesz być po stronie zwycięzców, czy przegranych.

Yasmin wymamrotała coś pod nosem.

-- Co? -- Mala krzyknęła na nią. -- Co mówisz, dziewczyno? Mów głośniej, żebym mogła cię usłyszeć!

-- Nie sądzę, żeby tak było. Myślę, że możemy być dobrzy dla innych ludzi, a oni będą dla nas mili. Myślę, że możemy trzymać się razem, jak drużyna, jak armia i~wszyscy możemy pracować razem, by uczynić świat lepszym miejscem.

Mala roześmiała się, ale zabrzmiało to wymuszenie, a Yasmin wydało się, że widzi łzy w~oczach przyjaciółki. 

-- Wiesz, co się dzieje, gdy się tak zachowujesz, Yasmin? Znajdują sposób, by cię zniszczyć. Zmusić cię do zostania zwierzęciem. Ponieważ \textit{oni} są zwierzętami. Chcą wygrać, a jeśli zaoferujesz im rękę, to odetnę ci palce. Musisz być zwierzęciem, żeby przeżyć.

Yasmin potrząsnęła głową, zaprzeczając wszystkiemu. 

-- To nieprawda, Mala! Nasi sąsiedzi tutaj nie są zwierzętami. To ludzie. To dobrzy ludzie. Nie mamy nic, a jednak wszyscy współpracujemy. Pomagamy sobie nawzajem \ldots 

 -- No dobrze, może uda ci się zdobyć tu małą grupę przyjaciół, ludzi, którzy musieliby spojrzeć ci w~oczy, gdyby zrobili ci brudną sztuczkę. Ale to wielki świat. Czy myślisz, że przyjaciele Big Sister Nor w~Singapurze, w~Chinach, Ameryce, Rosji \ldots  czy myślisz, że \textit{pomyślą }dwa razy, zanim cię zniszczą? W Afryce, w~\ldots  -- Machnęła ręką, oglądając wszystkie kraje, których nazw nie znała, wypełnione licznymi masami drapieżnych robotników, gotowych odebrać im pracę. -- Słuchaj: czy naprawdę tak bardzo zależy ci na Chińczykach, Rosjanach i~tych wszystkich innych ludziach? Czy odejmiesz swój chleb z~ust, żeby im go dać? Dla grupy \textit{obcokrajowców}, którzy nawet nie splunęliby, gdybyś się paliła?

Yasmin myślała, że zna swoją przyjaciółkę, ale nie przypominało to niczego, co kiedykolwiek słyszała od Mali. Skąd wziął się cały ten indyjski patriotyzm? 

-- Mala, to obcokrajowcy są właścicielami wszystkich gier, w~które gramy. Kogo obchodzi, czy są obcokrajowcami? Czy nie wystarczy fakt, że są ludźmi? Czy nie denerwował Cię ten głupi system kastowy i~nie mówiłaś, że wszyscy zasłużyli na równość?

-- Zasłużyli! -- Mala wypluła to słowo jak przekleństwo. -- Kogo obchodzi, na co zasługujesz, jeśli tego nie dostaniesz. Napełnij swój brzuch zasługą. Śpij na łóżku zasług. Zobacz, co otrzymujesz z~zasług!

-- Więc twoja armia ma zabierać wszystko, co może dostać, nawet jeśli to zrani kogoś innego? 

Mala wyprostowała się. 

-- Zgadza się, to \textit{moja }armia, Yasmin. Moja armia! A ty już nie jesteś jej częścią. Nie kłopocz się, ponieważ, ponieważ \ldots 

-- Bo nie jestem już twoją przyjaciółką ani porucznikiem -- powiedziała Yasmin. -- Rozumiem, generale Mala Robotwallah. Ale twoja armia nie będzie trwała wiecznie, a nasza siostrzana społeczność mogłaby przetrwać, gdybyś tylko bardziej ją ceniła. Przykro mi, że podejmujesz tę decyzję, generale Robotwallah, ale to należy do ciebie. Twoja karma. 

Odstawiła kanister, odwróciła się na pięcie i~ruszyła, zesztywniałe plecy, czekając, aż Mala wskoczy na nią i~wepchnie ją w~błoto, czekając, aż podbiegnie, przytuli ją i~zacznie błagać o przebaczenie. Dotarła do następnego rogu, wąskiej alejki między kolejnymi fabrykami recyklingu plastiku, i~udało jej się spojrzeć przez ramię, gdy się odwróciła, udając, że robi unik, by uniknąć pary kóz prowadzonych przez starego Tamila.

Mala stała wyprostowana jak żołnierz z~płonącymi oczami, a one przeszyły ją na chwilę, zamroziły ją w~kroku, tak że naprawdę \textit{musiała }omijać kozy. Kiedy znów się obejrzała, generał odeszła, jej chude ramiona napinały się pod bańkami z~wodą.

Big Sister Nor powiedziała jej, żeby była wyrozumiała.

-- Ona nadal jest twoją przyjaciółką -- powiedziała kobieta, a jej głos emanował z~gigantycznego robota, który pilnował grupę farmerów złota Webblies, którzy metodycznie napadali na starą zbrojownię, usuwali zombie i~zbierali gotówkę i~zrzuty broni, które pojawiały się za każdym razem, gdy biegli do lochu. -- Może o tym nie wiedzieć, ale jest po stronie pracowników. Druga strona, strona szefa, skorzysta z~jej usług, ale nigdy nie wpuszczą jej do swojego obozu. Co najwyżej może mieć nadzieję na bycie ukochanym zwierzakiem, cennym kawałkiem wynajętego mięśnia. Nie sądzę, żeby się na to zgodziła, co?

Ale to nie było zbytnie pocieszenie. Pewnego ranka Yasmin straciła najlepszą przyjaciółkę i~zawód. Zaczęła znowu chodzić do szkoły, ale w~ciągu sześciu miesięcy nieobecności miała zaległości w~pracy, a teraz nauczyciel chciał, żeby została na rok i~siedziała z~uczniami czwartej klasy, co było krępujące. Zawsze była dobrą uczennicą i~drażniło ją siedzenie z~młodszymi dziećmi, a co gorsza, była wysoka jak na swój wiek i~górowała nad nimi. Stopniowo przestała chodzić do szkoły.

Oczywiście jej rodzice byli oburzeni. Ale byliby oburzeni, kiedy Yasmin też wstąpiła do Armii, a jej ojciec bił ją przez dziesięć dni z~rzędu, podczas gdy ona nie chciała płakać, nie chciała łamać swojej woli. W końcu przekonał ich jej upór. I oczywiście pieniądze, które przyniosła do domu.

Yasmin poradzi sobie z~rodzicami.

Kafejka internetowa pani Dibyendu była teraz smutnym miejscem, odkąd Armia ruszyła dalej. Mala narzuciła to panu Banerjee i~uznała to za wielki pokaz swojej siły, gdy zwyciężyła. Ale Yasmin myślała, że nigdy nie wygrałaby kłótni, gdyby pani Dibyendu nie była tak chętna do pozbycia się Armii.

Yasmin wątpiła jednak, czy pani Dibyendu przewidziała wpływ, jaki odejście Armii wywrze na jej sklepik. Kiedy armia odeszła, każdy dzieciak w~Dharavi przeprowadził się z~nimi, nikt poniżej 30 roku życia nie postawił stopy w~kawiarni. Nikt prócz Yasmin, która teraz siedziała tam całymi dniami, walcząc o robotników.

-- Jesteś w~tym bardzo dobra -- powiedziała jej Justbob. Była porucznikiem Big Sister Nor, a jej hindi było okropne, więc radzili sobie łamaną angielszczyzną, którą ledwo mogły zrozumieć. Niemniej jednak gra Justboba była agresywna i~po prostu ta lekkomyślna, całkowicie nieustraszona, a kiedy grała, wykrzykiwała przerażające okrzyki bojowe w~języku tamilskim i~chińskim, co rozśmieszyło Yasmin, nawet gdy włosy na jej ramionach stanęły dęba. Justbob lubiła powierzać Yasmin kierownictwo strategii, podczas gdy ona prowadziła armie obrońców z~całego świata, którzy grali po ich stronie, broniąc robotników przed ludźmi takimi jak Mala.

-- Dziękuję -- powiedziała Yasmin i~wysłała eskadrę do zwodu na lewą flankę dwudziestu krążowników zardzewiałych wozów bojowych, najeżonych przykręcanymi karabinami maszynowymi i~wyrzutniami granatów. Ostatnio grała głównie w~Mad Max: Autoduel and Civilization, unikając Zombie Mecha i~innych gier, w~których rządziła Mala i~jej armia. Autoduel był teraz ogromny, powiązany z~programem telewizyjnym, w~którym szaleni biali ludzie walczyli ze sobą na pustyniach w~Australii zabójczymi samochodami, takimi jak te w~grze.

Wroga armia kupiła fintę, obracając się szerokim łukiem, by zaprezentować swoje wysunięte działa swoim zwinnym, małym motocyklowym zwiadowcom, którzy musieli wyglądać na łatwych łup, szybkie motocykle nie były w~stanie utrzymać żadnej prawdziwej broni ani zbroi, więc każdy kierowca ograniczał się do broni ręcznej, głównie Uzi na pełnej automatyce, strzelających pociskami w~stalowym płaszczu w~kierunku ciężko opancerzonych pysków wroga, który odpowiadał miażdżącym ogniem z~zamontowanymi na trójnogach karabinami maszynowymi i~granatami.

Ale kiedy się odwrócili, wtoczyli się w~podwójny rząd min, które Yasmin ukryła na początku bitwy, a potem, gdy samochody zakołysały się i~zderzyły ze sobą i~wyrwały się spod kontroli, dragoni Justbob wdarli się z~po lewej, a ich wspaniały wóz bojowy nadjeżdżał z~prawej -- niezdarny dwupiętrowy kamper, opancerzony potrójnie grubym pancerzem, z~otworami na broń dla baterii miotaczy ognia i~automatycznej broni balistycznej, strzelającej głównie pociskami ze zubożonego uranu, które przecinały samochody wroga jak masło. Nie było trudno prześcignąć taki wóz bojowy, ale wróg nie miał dokąd się uciekać, a kilka minut później po wrogu pozostały tylko zapalone płomienie benzyny i~potwornie okaleczone ciała.

Yasmin wyzoomowała i~pojechała swoim trójkołowcem dowodzenia wokół wydmy, gdzie grupa robocza nadal pracowała, wykonując swoją pracę, wykopując zakopane miasto pełne dzikich mutantów i~po raz dziesiąty zbierając jego bogate składy amunicji i~skarby sztuki tego dnia. Yasmin tak naprawdę nie mogła z~nimi rozmawiać, pochodzili z~Chin, zwanych Fujian, a poza tym byli zajęci. Opuścili swojego szefa i~utworzyli spółdzielnię pracowniczą, która dzieliła zarobki po równo, ale musieli mocno się zadłużyć, żeby kupić komputery, a z~tego, co Yasmin zrozumiała, ich rodziny mogły być zranione, a nawet zabite, jeśli przegapili płatność, ponieważ musieli pożyczyć pieniądze od gangsterów.

Byłoby miło, gdyby mieli dostęp do lepszego źródła pieniędzy, ale z~pewnością nie byłaby to Yasmin. Pieniądze z~jej armii skończyły się kilka tygodni po jej opuszczeniu Mali i~chociaż IWWWW zapłaciła jej trochę pieniędzy za pilnowanie punktów związkowych, nie wychodziło to zbyt wiele, zwłaszcza w~porównaniu z~pieniędzmi, które pan Banerjee musiał rozrzucać.

Przynajmniej nie krzywdziła innych biednych ludzi, żeby przeżyć. Zbiry, których właśnie zlikwidowała, dostaną zapłatę, nawet jeśli przegrali. I musiała to przyznać: to było \textit{zabawne}. W grze był prawdziwy dreszczyk emocji, granie w~nią dobrze, skłonienie tej armii ludzi do podążania za jej przykładem, do współpracy i~stania się niepowstrzymaną bronią.

Wtedy Justbob zniknęła. Nawet nie wpisała pospiesznie ,,gtg'', po prostu nie było jej na końcu mikrofonu. I rozlegały się odgłosy łoskotu, krzyki w~języku, którego Yasmin nie znała. Odległy krzyk.

Yasmin przeszła do Minerwy, serwisu społecznościowego, który lubili Webblies, tak jak robiła to tysiące razy dziennie. Minerva została stworzona dla graczy i~zawierała wszelkiego rodzaju ładne pulpity nawigacyjne, które pokazywały, w~jakich światach byli wszyscy twoi przyjaciele, w~jakich bitwach walczyli i~tak dalej. Łatwo było zgubić się w~Minervie, wpadać w~trans zrzutów ekranu ze słynnych bitew, gadania między gildiami, wściekłe kłótnie o najlepszy sposób na przejście poziomu, i~niekończące się rundy bicia farmerów złota. Jedną z~rzeczy, które uwielbiała w~Minervie, była funkcja automatycznego tłumaczenia, której baza danych zawierała wszelkiego rodzaju skróty i~slangi międzynarodowych graczy, wiedzącej, że Kekekekeke jest koreańskim dla LOL i~z milionem innych ważnych dialektów. To sprawiło, że Minerva była szczególnie przydatna dla globalnej sieci gildii Webblies, pracowniczych spółdzielni, lokalnych i~klanów.

Jej deska rozdzielcza \textit{zwariowała}. Webblies z~całego świata tweetowali o czymś, co dzieje się w~Chinach, o wielkim strajku grupy farmerów złota, którzy opuścili swojego szefa, a teraz pikietują przed swoimi fabrykami. Gracze z~całego świata pędzili do miejsca w~Mushroom Kingdom, aby zablokować jakiś exploit, który wydobywali, zanim wyszli z~pracy. Yasmin nigdy nie grała w~Mushroom Kingdom i~na nic by się tam nie przydała -- trzeba było dużo wiedzieć o broni, fizyce i~typach graczy na świecie, zanim zdołasz wyrządzić jakiekolwiek szkody. Ale sądząc po pasku stanu, który minął, na każdym shardzie było mnóstwo Webblies, aby wypełnić lukę.

Śledziła wiadomości, gdy przechodzili obok, obserwowała wiece i~odwroty, zwycięstwa i~porażki, i~czekała na szpicy, aż bitwa się skończy, kiedy GM odkryją, co zamierzają i~zablokowali konta wszystkich. To była tajna broń we wszystkich tych bitwach: każdy, kto donosił pracownikom firm, które zarządzały światami, mógł zniszczyć obie drużyny, wymazując ich konta i~łupy w~mgnieniu oka. Nikogo nie było na to stać, i~nikogo nie było stać na walkę w~tak ogromnych bitwach, że przyciągały uwagę GMów.

A jednak byli tu Webblies, setki, wszyscy ryzykowali swoje konta i~środki do życia, aby odeprzeć zbirów, którzy próbowali złamać strajk. Krew Yasmin śpiewała, to było to, o tym zawsze mówiła Big Sister Nor: Solidarność! Szkoda jednego jest szkodą dla wszystkich! Wszyscy jesteśmy w~tej samej drużynie, i~trzymamy się razem.

Były też filmy i~zdjęcia z~strajku, chudzi chińscy chłopcy mrugali jak sowy w~świetle dnia, na ruchliwych ulicach odległego kraju, stojąc z~rękoma splecionymi przed szklanymi drzwiami, skandując hasła po chińsku. Przechodnie gapili się na nich, wskazywali palcem lub śmiali się. Przeważnie były to dziewczyny, starsze od Yasmin, starsze nastolatki i~wczesne dwudziestki, bardzo dobrze ubrane, z~modnymi fryzurami i~krótkimi spódniczkami, wyprasowanymi bluzkami i~lśniącymi włosami. Patrzyły, a niektóre z~nich rozmawiały z~chłopcami, którzy pławili się w~uwadze. Yasmin wiedziała o chłopcach i~dziewczynach oraz o tym, jak się nawzajem zachowywali, czy nie widziała i~nie wykorzystywała tej wiedzy, kiedy była porucznikiem Mali?

A teraz coraz więcej dziewcząt przyłączało się do chłopców, niezupełnie dołączających, ale tłoczących się wokół nich, stojąc w~grupkach, rozmawiając między sobą. I przyjechała też policja, mnóstwo zdjęć pojawiającej się policji i~serce Yasmin zamarło. Okiem swojego stratega widziała, jak policyjne pozycje będą działać w~planowaniu natarcia na strajkujących, odcinając im drogi ucieczki, zamykając ich w~boksach i~blokując, gdy wkroczy policja.

Teraz zdjęcia zwolniły, teraz filmy się zatrzymały. Ręce w~rękawiczkach sięgnęły i~wyrwały aparaty fotograficzne, zasłaniając obiektywy. Ostatnim przekazem dźwiękowym były krzyki, złość, przerażenie, zranienie \ldots 

A teraz pasek na dole jej ekranu zwariował jeszcze bardziej, wiadomości z~pikiet w~Chinach o policyjnej akcji, a Yasmin poczuła się przez chwilę nierzeczywiście, jakby znowu czytała o bitwie w~grze, osadzonej w~jakimś świecie gry wzorowanym na przemysłowych Chinach, miejscu, które wydawało jej się równie obce jak Zombie Mecha czy Mad Max. Ale to byli prawdziwi ludzie, walczący z~prawdziwą policją, tłuczeni prawdziwymi pałkami. Wyobraźnia Yasmin dostarczała obrazy ludzi krzyczących, wijących się, tratujących się nawzajem z~całą ostrością jednej z~jej gier. To była znajoma scena, ale zamiast zombie byli to młodzi, bladzi chińscy chłopcy i~piękne, modne chińskie dziewczyny złapane w~ścisku, upadające pod pałkami.

A potem wiadomości ucichły, gdy wszyscy na scenie zamilkli. Pasek nadal pełzał wśród innych Webblies na całym świecie, ktoś powiedział, że chińska policja może wyłączyć wszystkie urządzenia mobilne w~mieście lub okolicy, jeśli tylko zechcą. Więc może ludzie wciąż tam byli, wciąż nagrywali i~zapisywali. Może nie wszyscy zostali aresztowani i~zabrani.

Yasmin ukryła twarz w~dłoniach i~ciężko oddychała. Pani Dibyendu krzyknęła coś na nią, może zaniepokojona. Nie sposób było odróżnić pieśni krwi w~jej uszach i~młota krwi w~piersi.

Tam, Webblies na całym świecie walczyli o lepszą ofertę dla biednych ludzi i~jakie to miało znaczenie? Jak jej solidarność może pomóc tym ludziom w~Chinach? Jak mogli \textit{jej }pomóc, kiedy tego potrzebowała? Gdzie była Big Sister Nor, Justbob i~Mighty Krang teraz, kiedy ich potrzebowała?

Wyszła na światło, mrugając, myśląc o tych chudych chińskich chłopcach i~policji na strategicznych pozycjach wokół nich. Nagle znajome zaułki i~zaułki Dharavi wydały się złowrogie i~klaustrofobiczne, jakby ludzie obserwowali ją z~każdej strony, przygotowując się do ataku. W końcu była tylko dziewczynką, małą dziewczynką, a nie potężnym wojownikiem czy generałem.

Zdradzieckie stopy poprowadziły ją w~dół drogi, za róg, za podwórko, na którym kobiece spółdzielnie piekarnicze wystawiały na słońcu swoje papapadamy, i~obok nowej kawiarni, w~której walczyła Mala i~jej armia. Byli tam teraz, a dźwięk ich hałaśliwej zabawy unosił się w~powietrzu jak dym, jak apetyczny zapach gotowania potraw.

O czym oni krzyczeli? Jakaś bitwa, którą toczyli  \ldots  bitwa w~Mushroom Kingdom. Bitwa przeciwko Webbliesom. Oczywiście. Byli najlepsi. Kogo jeszcze zatrudniłbyś do walki z~armiami Webblies? Poczuła mdłości w~brzuchu, uczucie, że ziemia ucieka jej spod stóp. Była teraz sama, naprawdę samotna, wróg jej dawnych przyjaciół. Nie było nikogo po jej stronie, z~wyjątkiem kilku odległych ludzi w~odległej krainie, których nigdy nie spotkała, których prawdopodobnie nigdy nie spotka.

Zniechęcona odwróciła się i~skierowała do domu. Jej ojciec wyjechał na kilka dni, podróżując do Pune, aby zainstalować podłogę do pracy. Pracował w~fabryce płytek samoprzylepnych, gdzie drukowali fałszywe wzory z~kamienia na pokrytych klejem kwadratach trwałego winylu, które można było łatwo układać w~biurowcach parków przemysłowych w~Pune. W ich domu zawsze były kafelki, a Yasmin nigdy nie zwracała na nie większej uwagi, dopóki nie zaczęła bawić się z~Malą, a potem pewnego dnia zauważyła z~szokiem, że dziwne, kanciaste rozmycie wokół krawędzi pięknego ,,marmuru'' żyły na kafelkach były tymi samymi smugami kompresji, które pojawiły się, gdy grafika gry zaczęła się dusić, ,,artefakty JPEG'', nazywano je na forach dyskusyjnych. To było jakby drobne niedoskonałości, które sprawiały, że gry były lekko nierealistyczne, wkradały się w~rzeczywisty świat.

To uczucie towarzyszyło jej, teraz gdy oddalała się jak duch z~kawiarni, ale została przywrócona do rzeczywistości przez dotknięcie jej ramienia. Odwróciła się, zaskoczona, czując, z~jakiegoś powodu, jakby miała zostać uderzona.

To był Sushant, najwyższy chłopiec w~armii Mali, który nigdy nie rzucał się w~oczy i~nie walczył jak inni chłopcy, ale wpatrywał się uważnie w~ekran, jakby chciał móc do niego uciec. Yasmin zorientowała się, że patrzy mu prosto w~oczy, a on potrząsnął przepraszająco brodą i~uśmiechnął się do niej nieśmiało.

-- Myślałem, że widziałem, jak przechodzisz -- powiedział. -- A ja pomyślałem \ldots  -- Spuścił oczy.

-- Co pomyślałeś? -- spytała. Wyszło to szorstko, złość, o której nie wiedziała, że czuje.

-- Pomyślałem, że wyjdę i~\ldots  -- urwał.

-- Co? Co sobie pomyślałeś, Sushant? 

Jej podbródek kołysał się teraz z~boku na bok i~pochyliła twarz w~jego stronę, nosy ledwo oddzielone. W jego oddechu czuła zapach jego lunchu ze szpinakiem.

Skurczył się, skrzywił. Yasmin zdała sobie sprawę, że jest przerażony. Zdał sobie sprawę, że prawdopodobnie dużo zaryzykował, wychodząc z~nią porozmawiać. Dyscyplina była wszystkim w~armii Mali. Czyż Yasmin nie była odpowiedzialna za egzekwowanie dyscypliny?

-- Przepraszam -- powiedziała, cofając się. -- Miło cię znowu widzieć, Sushant. Jadłeś? -- To była formalność, bo wiedziała, że tak, ale to było to, co jeden przyjaciel powiedział drugiemu w~Dharavi, w~Bombaju, może w~całych Indiach, z~tego, co Yasmin wiedziała.

Znów się uśmiechnął, chwiejnym, nieśmiałym uśmiechem. To było bolesne. Yasmin zdała sobie sprawę, że nigdy nie mówiła mu wiele, kiedy była poruczniczką Mali. Nigdy nie potrzebował namawiania ani ostrych słów, żeby zabrać się do pracy, więc praktycznie go zignorowała. 

-- Myślałem, że wyjdę i~przywitam się, bo wszyscy za tobą tęskniliśmy. Miałem nadzieję, że może ty i~Mala mogłybyście \ldots  -- Znowu się zawahał, a Yasmin poczuła, jak jej broda wysuwa się mimowolnie w~uparty, wściekły sposób.

-- Mala i~ja wybrałyśmy różne drogi -- powiedziała, świadomie starając się brzmieć spokojnie. -- To ostateczne. Czy to dobrze dla niej i~dla ciebie?

Pokiwał głową. 

-- Wygrywamy każdą bitwę.

-- Gratulacje. 

-- Ale teraz \ldots  ostatnio \ldots  myślałem \ldots 

Czekała, aż powie więcej. Chwila się przeciągnęła. Dorośli przeszli obok nich i~zdała sobie sprawę, że prawdopodobnie myśleli, że to zaloty, będąc razem chłopcem i~dziewczynką. Jeśli wiadomość o tym dotarła do jej ojca \ldots 

Ale to już nie miało dla niej znaczenia. Jej ojciec montował artefakty JPEG w~parku IT w~Pune. Odeszła z~Armii, nie miała przyjaciół ani szkoły. Co może mieć jakiekolwiek znaczenie.

-- Rozmawiam z~twoimi przyjaciółmi -- powiedział w~końcu.

-- Moi przyjaciele? -- Nie wiedziała, że jakiś ma.

-- Webblies. Twoja nowa armia. Przychodzą do mnie, kiedy walczę, wysyłają mi prywatne wiadomości. Na początku ich ignorowałem, ale ostatnio jestem wolny i~mam dużo czasu na myślenie. I wysłali mnie zdjęcia \ldots  ludzi, których skrzywdziłem. Dzieciaki takie jak ty i~ja na całym świecie. I to dało mi do myślenia. -- Przerwał, oblizał usta. -- O karmie. O krzywdzeniu ludzi, by żyć. O wszystkich rzeczach, które mówią. Nie sądzę, że chcę to ciągle robić. Lub, że mogę to robić na zawsze.

Yasmin zabrakło słów. Czy naprawdę byli inni ludzie, tutaj w~Dharavi, tutaj w~armii Mali, którzy czuli się tak jak ona? Jakoś nigdy nie wyobrażała sobie czegoś takiego. Ale oto on był.

-- Wiesz, że armia Mali płaci dziesięć razy więcej niż możesz dostać od Webblies, prawda?

-- Na razie -- powiedział. -- O to właśnie chodzi, prawda? Chee! Jeśli teraz będziemy walczyć, możemy podnieść płace każdemu, kto pracuje na utrzymanie, zamiast posiadać rzeczy na całe życie, prawda?

-- Nigdy nie myślałem o podziale w~ten sposób. Mam na myśli posiadanie rzeczy do życia.

Jego nieśmiałość ustąpiła. Wyraźnie cieszył się, że ma z~kim o tym porozmawiać. 

-- Wszystko sprowadza się do posiadania versus pracowania. Ktoś musi zorganizować, jak sądzę  \ldots  nie byłoby Zombie Mecha, gdyby ktoś nie zebrał wielu ludzi razem, pracujących nad stworzeniem tego całego kodu. Ktoś musi zapłacić mistrzom gry i~całe to wszystko. Rozumiem tę część. To ma dla mnie sens. Moja mama pracuje w~farbiarni pani Dotty. Ktoś musi kupić barwniki, zdobyć materiał, kupić kadzie i~narzędzia, umówić się na sprzedaż, gdy praca się skończy, w~przeciwnym razie moja mama nie miałaby pracy. Zawsze tam się zatrzymywałem, myśląc, w~porządku, jeśli pani Dotta wykonuje całą tę pracę i~robi pracę dla mojej matki, dlaczego nie miałaby dostawać za to zapłaty?

-- Ale teraz myślę, że nie ma powodu, aby praca pani Dotta była ważniejsza niż praca mojej matki. Mamaji nie miałaby pracy bez fabryki pani Dotta, ale pani Dotta nie miałaby fabryki bez pracy mamaji, prawda? -- Poruszył wyzywająco brodą.

-- Zgadza się -- powiedziała Yasmin. Była zdenerwowana przebywaniem publicznie z~tym chłopcem, ale musiała przyznać, że słyszeć to wszystko od niego było ekscytujące.

-- Więc dlaczego pani Dotta miałaby mieć prawo zwolnić moją matkę, ale moja matka nie ma prawa zwolnić panią Dotta? Jeśli są od siebie zależne, dlaczego jedno z~nich miałoby zawsze mieć prawo żądać, a drugie zawsze musi prosić o przysługę?

Yasmin poczuła jego podniecenie, ale wiedziała, że musi być w~tym coś więcej. 

-- Czy pani Dotta nie podejmuje całego ryzyka? Czy nie musi znaleźć pieniędzy na uruchomienie fabryki i~czy nie straci ich, jeśli fabryka zostanie zamknięta?

-- Czy Mamaji nie ryzykuje utraty pracy? Czy Mamaji nie ryzykuje zachorowania z~powodu oparów i~chemikaliów w~barwnikach? Nie ma w~tym nic wiecznego, doskonałego ani naturalnego! To po prostu coś, na co wszyscy się zgodziliśmy, szefowie są na czele, zamiast być po prostu innym rodzajem pracownika, który wnosi inny rodzaj pracy!

-- A myślisz, że to właśnie dostaniesz od Webbliesów? Koniec z~szefami?

Spuścił wzrok, rumieniąc się. 

-- Nie -- powiedział. -- Nie, nie sądzę. Myślę, że to zbyt wiele, aby o to prosić. Ale może pracownicy znajdą lepszą ofertę. O tym mówi Big Sister Nor, prawda? Dobra płaca, dobre miejsca do pracy, uczciwość? Nie zostaniesz zwolniony tylko dlatego, że nie zgadzasz się z~szefem?

\textit{Lub Generałem}, pomyślała Yasmin. Głośno, powiedziała: 

-- Więc opuścisz Armię? Chcesz być Webblies?

Teraz on spojrzał w~dół. 

-- Tak -- odpowiedział, w~końcu. -- W końcu. Wszystko kręci się w~mojej głowie. Nie wiem, czy jestem jeszcze gotowy. -- Zaryzykował spojrzenie na nią. -- Nie wiem, czy jestem tak odważny jak ty.

Ogarnął ją gniew, gorący i~irracjonalny. Jak on \textit{śmie }mówić o jej ,,odwadze''? Używał tego tylko jako pretekstu do dalszego wzbogacania się w~armii Mali. \textit{Tak dobrze }rozumiał, co jest nie tak i~co należy zrobić. Rozumiał to lepiej niż Yasmin! Ale nie chciał rezygnować z~komfortu i~przyjaźni. To nie było tchórzostwo, to była \textit{chciwość}. Był zbyt chciwy, żeby z~tego zrezygnować.

Musiał zobaczyć to w~jej twarzy, ponieważ cofnął się o krok i~uniósł ręce. 

-- Nie chodzi o to, że kiedyś tego nie zrobię \ldots  ale nie wiem, co by mi dało, gdybym zrobił to dzisiaj, sam. Co by się zmieniło, gdybym przestał walczyć dla armii Mali? Ona jest tylko jednym generałem z~jedną armią spośród setek na całym świecie, a ja jestem tylko jednym wojownikiem w~armii. Ja \ldots  -- Zawahał się. -- Jaki sens jest oddawać tak dużo, jeśli nie zrobi to żadnej różnicy?

Gniew Yasmin kipiał w~niej, zjadł ją jak kwas, ale ugryzła się w~język, ponieważ ten cichy głos w~jej wnętrzu mówił: ,,Jesteś zła, bo myślałaś, że masz towarzysza, kogoś, kto dotrzyma ci towarzystwa i~okazało się, że chciał tylko przyznać się tobie i~abyś mu wybaczyła''. I to była prawda. Była o wiele bardziej zdenerwowana swoją samotnością niż jego tchórzostwem, chciwością, czy czymkolwiek to było.

-- Ja. Muszę. Iść. Teraz -- powiedziała, gryząc słowa, starając się powstrzymać gniew w~głosie czystą siłą woli.

Nie czekała, aż podniesie oczy, po prostu odwróciła się na pięcie i~szła, szła i~szła znajomymi alejkami Dharavi, nigdzie nie idąc, ale i~tak próbując uciec, jak przykute zwierzę wychodzące ze swojego łachy. Została przykuta łańcuchem, przykuta przez urodzenie i~okoliczności. Jej rodzina mogła być bogata. Mogli należeć do wysokiej kasty. Może przebywać w~innym kraju, w~Ameryce, Chinach, Singapurze, we wszystkich odległych krajach. Ale była tutaj i~nie miała nad tym kontroli. Tam był cały świat i~tutaj umieścił ją los.

Nie zmieniłaby świata. Nie pojedzie do żadnego z~tych miejsc. Nie opuściła nawet Dharavi, z~wyjątkiem jednego razu z~matką, kiedy zabrała Yasmin i~jej braci pociągiem na plażę, gdzie było gorąco i~piaszczysto, a woda była zbyt niebezpieczna, by w~niej pływać, więc stali na brzegu, a potem poszli ulicą z~eleganckimi sklepami, gdzie nie było ich stać na zakupy, a potem znów czekali na autobus i~wrócili do domu. Yasmin widziała wieloświaty gier, ale nawet nie widziała Bombaju.

Gdzie teraz? Była zmęczona i~głodna, zła i~wyczerpana. Dom? Było jeszcze popołudnie, więc jej matka i~bracia pracowali albo byli w~szkole. Ta pustka \ldots  To ją przerażało. Nie była przyzwyczajona do samotności. To nie był stan naturalny w~Dharavi. Była bardzo spragniona, wiatr dmuchał jej plastikowym dymem w~oczy i~twarz, sprawiając, że bolały ją nozdrza, zatoki i~gardło. Kawiarnia pani Dibyendu będzie miała czaj, a pani Dibyendu dałaby jej filiżankę i~trochę czasu na korzystanie z~komputera na kredyt, ponieważ pani Dibyendu desperacko starała się uratować kawiarnię przed bankructwem, teraz gdy Armia ją porzuciła.

Idiotyczny bratanek pani Dibyendu niechętnie podał jej filiżankę herbaty. Nie nauczył się niczego od okrutnego bicia, które dała mu Mala. Wciąż stał zbyt blisko, wciąż wychodził na ,,dokuczanie'' ze swoją bandą badamaash. Yasmin wiedziała, że z~radością zemściłby się na Mali i~że Mala nigdy nie wychodziła po zmroku bez trzech lub czterech największych chłopców z~Armii. To ją wkurzało. Bez względu na to, jak bardzo Mala ją skrzywdziła, miała prawo obejść swój dom, nie obawiając się tego idioty. Jego górna warga była wykrzywiona w~nieustannym szyderstwie, dzięki bliźnie, którą pozostawiły stopy Mali.

Usiadła do komputera, zalogowała się. Była pewna, że ten idiota bratanek używał wszelkiego rodzaju szkodliwego oprogramowania do szpiegowania tego, co robili na komputerach, ale kupiła breloczek do logowania w~jednym ze sklepów na obrzeżach Dharavi i~to robiło magię, logując ją przy użyciu innego hasła za każdym razem, gdy siadała, dzięki czemu jej konta PayPal i~gry były bezpieczne.

Bezmyślnie wróciła do swojej zwykłej rutyny. Zaloguj się do Minerwy, sprawdź misje ochrony Webblies w~światach, w~których grała. Ale nie czekały żadne misje. Wszystkie kanały Webblies płonęły plotkami o strajku w~Shenzhen, plotkami o liczbie aresztowanych, plotkami o strzelaninach. Patrzyła, jak mijają bezradnie, zastanawiając się, skąd wzięły się te wszystkie plotki. Każdy zdawał się wiedzieć coś, czego ona nie wiedziała. Skąd wiedzieli?

Na jej ekranie pojawiła się bezpośrednia wiadomość. Pochodziła od nieznajomego, ale był to ktoś z~wewnętrznej grupy afinicji Webblies, co oznaczało, że Big Sister Nor, Mighty Krang lub Justbob ręcznie ją zatwierdzili. Każdy mógł dołączyć do zewnętrznych Webblies, ale było bardzo niewielu wewnętrznych Webblies.

{\textgreater} Witam, czy możesz to przeczytać?

Było to pełne zdanie, z~interpunkcją, a pytanie było tak głupie, jak można sobie wyobrazić. To był rodzaj wiadomości, którą mógłby wysłać jej ojciec. Od razu wiedziała, że komunikuje się z~osobą dorosłą, która nie gra.

{\textgreater} tak

{\textgreater} Nasz wspólny przyjaciel B.S.N. poprosił mnie o skontaktowanie się z~Tobą. Jesteś w~Bombaju, prawda?

Zawahała się przez chwilę. To był bardzo dorosły sposób pisania, typu nie-jestem-graczem. Może to był ktoś pracujący dla drugiej strony? Ale Bombaj był ogromny jak świat. ,,W Bombaju'' było tylko trochę bardziej konkretne niż ,,w Indiach'' czy ,,Na Ziemi''.

{\textgreater} tak

{\textgreater} Gdzie jesteś? Czy mogę przyjść i~cię zabrać? Muszę z~Tobą porozmawiać.

{\textgreater} mówisz teraz lol

{\textgreater} Co? Rozumiem. Nie, muszę z~tobą POROZMAWIAĆ. To jest oficjalna sprawa. B.S.N. wyraźnie powiedziała, że muszę się z~tobą skontaktować.

Kilka razy przełknęła ślinę, wysączyła resztki herbaty.

{\textgreater} ok 

{\textgreater} Wspaniale. Skąd mam przyjść i~cię zabrać?

Znowu przełknęła ślinę. Kiedy poszli na plażę, jej matka jasno stwierdziła: \textit{Nie mów nikomu, że jesteś z~Dharavi. Dla Mumbaikarów Dharavi jest jak piekło, miejsce wiecznych męki, a ci, którzy tu mieszkają, są potworami. }Ten dorosły brzmiał bardzo uczciwie. Może pomyśli, że Dharavi jest piekłem i~zostawi ją w~spokoju.

{\textgreater} dziewczyna dharavi 

{\textgreater} Chwila.

Nastąpiła długa pauza. Zastanawiała się, czy próbował skontaktować się z~Big Sister Nor, powiedzieć jej, że jej wojownik jest dzieckiem ze slumsów, znaleźć kogoś lepszego do pomocy.

{\textgreater} Znasz to miejsce?

Był to obraz meczetu Dharavi, wysokiego i~imponującego, górującego nad całą dzielnicą muzułmańską.

{\textgreater} oczywiście!!

{\textgreater} Będę tam za około godzinę. To ja.

Inny obrazek. Nie był to mężczyzna w~średnim wieku w~garniturze, którego oczekiwała, ale młody mężczyzna, niewiele starszy niż nastolatek, z~krótkimi, nażelowanymi włosami, skórzaną kurtką, stylowymi niebieskimi dżinsami i~czarnymi motocyklowymi butami.

{\textgreater} Czy możesz mi podać swój numer telefonu? Zadzwonię, kiedy będę blisko.

{\textgreater} lol 

{\textgreater} przepraszam?

{\textgreater} dziewczyna z~dharavi - -  telefon nie dla nas

Miała telefon, kiedy była w~armii Mali. Wszyscy mieli telefony. Ale to była pierwsza rzecz, którą straciła, kiedy odeszła z~wojska. Wciąż miała go w~szufladzie, nie mogła go sprzedać, ale nie działał już jako telefon, chociaż czasem używała go jako kalkulatora (wszystkie gry wyłączały się zaraz po odłączeniu usługi, ku jej rozczarowaniu).

{\textgreater} Przepraszam, przepraszam. Oczywiście. Spotkamy się tam za około godzinę.

Serce waliło jej w~piersi. Spotkanie z~obcym mężczyzną, załatwianie sekretnej sprawy, w~opowieściach takie rzeczy zawsze kończyły się straszliwą tragedią, skalaniem i~morderstwem. A za godzinę od teraz będzie \ldots 

{\textgreater} nie mogę się spotkać w~meczecie

Będzie w~samym środku ,,Asr'', popołudniowych modlitw, a meczet będzie zatłoczony przez przyjaciół jej ojca. Wystarczyłoby, żeby któryś z~nich zobaczyła ją z~obcym mężczyzną o nażelowanych włosach, Hindusem, sądzącym po rakhi na jego nadgarstku, wyswobadzającym się ze skórzanej kurtki. Jej ojciec \textit{oszalałby}.

{\textgreater} spotkajmy się na stacji węzłowej Mahim zamiast przy barierkach

Dojście tam zajęłoby jej godzinę, ale byłoby bezpieczne.

Nastąpiła pauza. Potem kolejne zdjęcie: dwóch chłopców siedzących okrakiem na jednej z~ogromnych betonowych barier przed stacją. To tam ona i~jej bracia czekali, podczas gdy ich matka ustawiała się w~kolejce po bilety.

{\textgreater} Tutaj?

{\textgreater} tak

{\textgreater} No to OK. Będę na skuterze Tata 620.

Kolejne zdjęcie pięknie wypolerowanego małego motoru, dumnego fioletowego zbiornika paliwa na chromowanej ramie. W Dharavi było ich tysiące, kierowanych przez niedoszłych badamaash, którzy zaoszczędzili trochę pieniędzy na parę kół.

{\textgreater} będę tam

Podała swój kubek idiotycznemu siostrzeńcowi, nie widząc nawet grymasu na jego twarzy, gdy minęła go, wybiegła na ulicę, wróciła do domu, żeby się przebrać i~schować kilka rzeczy do torby, zanim jej matka lub bracia wrócą do domu. Nie wiedziała, dokąd jedzie ani jak długo będzie nieobecna, a ostatnią rzeczą, jakiej pragnęła, było wyjaśnienie tego matce. Zostawi notatkę, jeden z~jej braci przeczyta ją matce. Po prostu mówiła: ,,Wyjeżdżam, sprawy związku. Niedługo wracam. Kocham cię''. A to musiało wystarczyć, bo przecież to było wszystko, co wiedziała.

Podczas długiego spaceru do stacji Mahim Junction czuła na przemian nerwowe podniecenie i~nerwowe przerażenie. Z pewnością było to głupie, ale to było też wszystko, co jej zostało. Jeśli Big Sister Nor poręczyła za tego mężczyznę  \ldots  chee! nie znała nawet jego imienia!  \ldots  więc kim była Yasmin, żeby w~niego wątpić?

Kiedy zbliżyła się do krawędzi Dharavi, alejki rozszerzały się w~ulice, wystarczająco szerokie, by chudzi chłopcy bez butów mogli grać w~krykieta. Krzyczeli na nią ,,obrażając przyzwoitość'', jak zawsze nauczyciel, pan Hossain, mówił, kiedy badamaash zbierali się na zewnątrz szkoły, żeby wołać do dziewczyn różne rzeczy, gdy opuszczały klasę. Ale wiedziała, jak ich zignorować, a poza tym przyniosła lathi swojego brata Abdura, używając jej jako laski, przywiązując do góry zapasową chusteczkę hidżabu, aby wydawała się bardziej nieszkodliwa. Grali na gimnastyce na dziedzińcu szkolnym kijami jak lathis, ale bez żelaznej oprawy na czubku. Mimo to była przekonana, że może wymachiwać nią wystarczająco przerażająco, by odstraszyć każdego badamaash, który stanąłby jej na drodze tego pamiętnego dnia. Dopiero na stacji zrozumiała, że nie ma pojęcia, jak niosłaby ją na małym skuterze.

Przyniosła ze sobą telefon, żeby sprawdzić godzinę, a teraz minęła godzina i~nie było śladu mężczyzny z~krótkimi nażelowanymi włosami. Minęło kolejne dwadzieścia minut. Była do tego przyzwyczajona: nic w~Dharavi nie miało dokładnego czasu, z~wyjątkiem wezwań do modlitwy z~meczetu, rannego piania kogutów i~wezwań do mobilizacji w~armii Mali, które zawsze były precyzyjnie zaplanowane, z~surową dyscypliną dla maruderów, którzy spóźnili się na bitwę.


Przyjeżdżały pociągi i~wyjeżdżały pociągi. Zobaczyła kilku mężczyzn, których rozpoznała: przyjaciół jej ojca, którzy pracowali w~Bombaju, którzy rozpoznaliby ją, gdyby nie nosiła hidżabu podciągniętego do nosa i~tam przypiętego. Była dotkliwie świadoma spojrzenia hinduskich chłopców. Oficjalnie hindusi i~muzułmanie się nie dogadywali. Oczywiście nieoficjalnie znała tylu Hindusów, co muzułmanów w~Dharavi, w~Armii, w~szkole. Ale na bezosobową, wielką skalę zawsze była \textit{inna}. Oni byli to ,,Mumbaikars'' -- ,,prawdziwi'' ludzie z~Mumbai. Jej rodzice uparli się, by nazywać miasto ,,Bombajem'', co było starą nazwą miasta sprzed zmiany jej przez zaciekłych nacjonalistów hinduskich, głoszących, że Indie są wyłącznie dla hinduistów i~tylko hinduistów. Ona i~jej ludzie mogli wrócić do Bangladeszu, do Pakistanu, do jednej z~muzułmańskich fortec, gdzie byli w~większości i~zostawić Indie prawdziwym Hindusom.

Przeważnie jej to nie dotykało, ponieważ w~większości spotykała tylko ludzi, którzy ją znali i~których znała, lub ludzi, którzy byli całkowicie wirtualni i~którym bardziej zależało na tym, czy była orkiem, czy elfką ognia, niż czy była elfką, niż czy była muzułmanką. Ale tutaj, na skraju znanego świata, była dziewczyną w~hidżabie, z~rozcięciem na oczy, w~długiej, skromnej sukience i~z grubym kijem, i~wszyscy się na nią \textit{gapili}.

Bawiła się myślami o tym, jak zaatakować lub obronić stację, używając różnych systemów uzbrojenia w~grach. Gdyby wszystkie były zombie, ustawiałaby mechy tu, tu i~tu, wykorzystując kolejkę jako kanał, by zwabić walczących w~zasięg miotaczy ognia. Gdyby walczyli na motocyklach, krążyłaby tam ze swoimi samochodami, w~tę stronę ze swoimi motocyklami i~wciągała tam ciężarówkę śmierci. To wywołało uśmiech na jej twarzy, bezpiecznie schowanej za hidżabem.

A oto mężczyzna, który wjeżdżał na parking swoim zielonym motocyklem, ocierając kurz z~okularów końcówką koszuli, zanim wsunął ją z~powrotem do kurtki. Nerwowo rozejrzał się po ludziach na zewnątrz stacji -- ludzi pracy krążących tam i~z powrotem, badamaash i~żebraków kręcących się i~przechadzających się, wchodzących wszystkim w~drogę. Kilku żebraków szło teraz w~jego stronę, dzieci z~wyciągniętymi rękami, niektórzy nosili mniejsze dzieci na biodrach. Nawet przez hałas tłumu Yasmin słyszała ich smutne, wyćwiczone krzyki.

Sięgnęła pod brodę i~sprawdziła szpilkę utrzymującą jej hidżab na miejscu, po czym podeszła do jeźdźca, przechodząc przez żebraków, jakby ich tam nie było. Uciekali od jej lathi jak muchy przed uniesioną ręką. Był tak zakłopotany przez żebraków, że dopiero po minucie zauważył stojącą przed nim młodą dziewczynę z~welonem, ściskającą półtorametrowy kij oprawiony w~żelazo.

-- Yasmin? -- Jego hindi było jak u gwiazdy fillum. Z bliska był bardzo przystojny, miał proste zęby, starannie przystrzyżony wąsik oraz mocny nos i~podbródek.

Skinęła głową.

Spojrzał na jej lathi. 

-- Mam kilka lin bungee -- powiedział. -- Myślę, że możemy to przymocować z~boku motoru. I przywiozłem ci kask.

Znowu skinęła głową. Nie wiedziała, co powiedzieć. Podszedł do zamkniętego pudła transportowego z~tyłu roweru, odpychając małego żebraka, który dotykał zamka, i~przysunął kciuk do czytnika odcisków mechanizmu blokującego. Otworzył się, a on sięgnął do środka, wyciągając hełm, który wyglądał jak coś z~kreskówki manga, opływowy, z~zawiłymi wzorami wyrytymi na jego powierzchni w~jaskrawożółtym i~różowym kolorze. Z przodu hełmu znajdowała się naklejka przedstawiająca Sai Babę, świętego, na którego godzili się zarówno muzułmanie, jak i~hindusi. Yasmin uznała, że to dobry znak, nawet jeśli był hinduskim chłopcem, przyniósł jej hełm, który mogła nosić, nie kalając islamu.

Wzięła od niego manga-hełm Sai Baby, zauważając, że naklejka była holograficzna i~że Sai Baba odwrócił się, by spojrzeć jej prosto w~oczy, gdy ją unosiła. Był cięższy, niż się wydawało, z~grubą wyściółką w~środku. Nikt w~Dharavi nie nosił kasków ochronnych na motocyklach, a chłopiec też go nie miał. Ale kiedy zastanawiała się nad wąskim siodełkiem, pomyślała o spadnięciu z~prędkością 70 kilometrów na godzinę na jakiejś drodze do Bombaju i~uznała, że cieszy się, że przywiózł. Więc skinęła głową po raz trzeci i~uniosła go nad głowę. Trwało to powoli, jej głowa wsuwała się jak ręka schwytana w~splątany rękaw, pchając, by przesunąć materiał, który powoli ustąpił. Potem znalazła się w~środku, a dźwięki wokół niej były martwe i~odległe, wszystkie widoki zabarwiły się na żółto przez jednokierunkowy lustrzany wizjer. Niepewnie pomacała swoją głowę -- która miała wrażenie, że gdyby odwróciła twarz zbyt szybko -- miała wrażenie, jakby wychyliła się do przodu pod ciężarem hełmu, i~znalazła zaczep przyłbicy, podniosła ją. Dźwięk stał się nieco jaśniejszy i~ostrzejszy.

W międzyczasie chłopiec mocował lathi wzdłuż roweru, ku uciesze dziecięcych żebraków, którzy udzielali rad i~szyderstw. Miał garść linek bungee, które wyciągnął z~pudła roweru, i~owijał je raz za razem wokół kija, znajdując miejsca na chromowanym szkielecie roweru, aby zamocować haki, testując kierownicę, aby upewnić się, że nadal może sterować. W końcu stęknął, wstał, otrzepał ręce o dżinsy i~odwrócił się do niej.

-- Gotowa? 

Wzięła głęboki oddech, wreszcie się odezwała. 

-- Gdzie jedziemy? 

-- Andheri -- powiedział. -- W pobliżu studiów filmowych.

Skinęła głową, jakby wiedziała, gdzie to jest. W pewnym sensie oczywiście, że tak: było mnóstwo filmów o, no cóż, złotej erze robienia filmów, kiedy Andheri było \textit{tym} miejscem do zaistnienia, wspaniałym i~tętniącym życiem. Ale większość z~tych filmów opowiadała o tym, jak słońce Andheri zaszło, kiedy wszystkie wielkie wytwórnie filmowe się wyprowadziły. Jak by to było dzisiaj?

-- A kiedy wrócimy?

Poruszył brodą, myśląc. 

-- Oczywiście dziś wieczorem. Upewnię się. A niektórzy ludzie związkowi mogą wrócić z~nami i~upewnić się, że bezpiecznie dotrzesz do swoich drzwi. Przemyślałem wszystko.

-- A jak masz na imię?

Patrzył na nią przez chwilę, jego szczęka opadła ze zdumienia. 

-- OK, nie przemyślałem wszystkiego! Jestem Ashok. Umiesz jeździć na skuterze?

Potrząsnęła głową. Widziała mnóstwo ludzi jeżdżących na motocyklach i~skuterach, dwójkami, a nawet trójkami i~czwórkami -- czasami całą rodziną, z~dziećmi na kolanach matek na plecach -- ale nigdy nie wsiadła na jednego. Stojąc teraz obok, wydawał się niematerialny i~no, \textit{śliski}, tego rodzaju rzeczy, z~których łatwiej było spaść niż pozostać.

-- OK -- powiedział, machając brodą, biorąc pod uwagę jej ubranie.

-- Z sukienką jest trudniej -- powiedział. -- Będziesz musiała siedzieć bokiem na siodełku. -- Wspiął się na siodełko roweru i~zademonstrował, trzymając kolana razem i~przyciśnięty do boku roweru, skręcając ciało. 

-- Będziesz musiała mocno się mnie trzymać. -- Uśmiechnął się swoim uśmiechem gwiazdy filmowej.

Yasmin zdała sobie sprawę, jaki to wszystko był błąd. Ten dziwny człowiek. Jego motocykl. Wyjazd do Bombaju, z~dala od Dharavi, do dziwnego miejsca, z~dziwnego powodu. I miał jej lathi, które nawet nie należało do niej, i~gdyby odwróciła się na pięcie i~wróciła do Dharavi, nadal musiałaby wyjaśnić brakujące lathi bratu i~list matce. A teraz miała zginąć w~ruchu ulicznym w~Bombaju z~zupełnie obcą osobą w~drodze do ulubionego miasta duchów Bollywood.

Choć było to beznadziejne, nie było tak beznadziejne, jak samotność, ani w~Armii, ani w~szkole, ani w~Webblies. Nie tak beznadziejna jak bycie biedną Yasmin, dziewczyną z~Dharavi, urodzoną w~Dharavi, wychowaną w~Dharavi.

Wsunęła się bokiem na skuter, a Ashok wspiął się na siodło i~usiadł ze skórzaną kurtką dociśniętą do jej boku. Próbowała wyprostować biodra do przodu i~znalazła się w~tak niepewnej pozycji, że prawie przewróciła się do tyłu.

-- Musisz się trzymać -- powiedział Ashok, a żebrzące dzieci szydziły i~wykonywały niegrzeczne gesty. Zamknęła oczy, objęła go ramionami w~pasie, czując, jak chudy jest pod tą fantazyjną kurtką, i~splotła palce wokół jego brzucha. Teraz było mniej niebezpiecznie, ale wciąż czuła, że lada chwila upadnie, a nawet jeszcze się nie ruszali!

Ashok kopnął podnóżek motocykla i~włączył silnik. Z rury wydechowej wydostała się chmura spalin z~biodiesla, pachnąca starym olejem spożywczym -- prawdopodobnie początkowo był to stary olej spożywczy, oczywiście -- ostro i~nieświeżo. Żołądek Yasmin zabulgotał, a ona zarumieniła się pod hidżabem, pewna, że poczuł skręcanie się jej pustego żołądka. Ale on tylko odwrócił głowę i~powiedział: 

-- Gotowa? 

-- Tak -- powiedziała, ale jej głos zabrzmiał jak pisk.

Ledwo przejechali pięćdziesiąt metrów, zanim krzyknęła ,,Stop! Stop!'' w~jego ucho. Nigdy w~życiu nie bała się bardziej. Zmusiła palce do rozsznurowania i~wsunęła drżące ręce z~powrotem na kolana.

-- Co jest nie tak? 

-- Nie chcę umrzeć! -- krzyknęła. -- Nie chcę umrzeć na twoim maniakalnym rowerze w~tym maniakalnym ruchu ulicznym!

Poruszył brodą. 

-- Chodzi o sukienkę -- powiedział. -- Gdybyś tylko mogła usiąść na siedzeniu. 

Yasmin żałośnie poklepała się po udach, po czym podciągnęła sukienkę, odsłaniając salwar -- luźne spodnie -- które nosiła pod nią. Ashok skinął głową. 

-- To wystarczy -- powiedział. -- Ale musisz związać nogawki, żeby nie zostały złapane przez kierownicę. Ponownie otworzył skrzynię ładunkową i~podał jej dwa plastikowe paski, którymi zawiązała każdą kostkę.

-- Dobra, ruszamy -- powiedział, a ona usiadła okrakiem na skuterze, ponownie obejmując go w~pasie. Pachniał żelem do włosów, skórą i~potem z~drogi. Czuła się, jakby udała się teraz na inną planetę, chociaż wciąż widziała za sobą Mahim Junction. Ścisnęła jego talię, by ocalić życie, gdy włączył silnik i~skierował motocykl z~powrotem w~ruch.

Zdała sobie sprawę, że już wcześniej dla niej był spokojny, jadąc stosunkowo wolno i~równo, ze względu na jej niepewną pozycję. Teraz kiedy była bardziej zabezpieczona, prowadził jak najgorszy badamaash, jakiego kiedykolwiek widziała w~jakimkolwiek filmie akcji. Wjechał małym skuterem na skraj rowu, obok szarpanego, wolnego ruchu, zawsze o krok od przewrócenia się do śmierdzącego rowu, zabicia przez skręcającego kierowcę lub nagle otwierające się drzwi, aby kierowca mógł wypluć strumień betel; albo przejechanie jednego z~żebraków, którzy stali na skraju drogi, stukając w~szyby i~robiąc smutne miny do uwięzionych kierowców.

W swojej karierze jako gracz pilotowała milion wirtualnych pojazdów, z~dużą prędkością, przez niebezpieczny teren. W najmniejszym stopniu nie było tak samo, nawet z~filtrującą rzeczywistość wyściółką i~wizjerem hełmu. Słyszała w~głowie własne skomlenie. Każdy nerw w~jej ciele krzyczał. \textit{Odejdź od tego, póki możesz!} Ale jej racjonalny umysł wciąż upierał się, że ten chłopak najwyraźniej codziennie jeździł na rowerze przez Bombaj i~zdołał przeżyć.

Poza tym, gdy pędzili drogą, było tyle Bombaju do zobaczenia, a to było o wiele bardziej interesujące niż martwienie się o nieuchronną śmierć. Gdy pędzili groblą, zbliżyli się do ogromnego mostu wiszącego, szerokiego na osiem pasów, z~białego betonu i~stalowych lin, dumnie ogłoszonego łączem morskim Bandra-Worli w~zawiłych znakach w~języku hindi i~po angielsku. Przyspieszyli po rampie, jadąc blisko stalowych dźwigarów wzdłuż krawędzi mostu, a pod nimi morze lśniło błękitem i~wydawało się, że jest tak blisko, że mogła sięgnąć w~dół i~muskać palcami fale. Powietrze pachniało solą i~morzem, duszące spaliny uliczne unoszone przez wiatr, który potargał jej sukienkę i~spodnie, przyklejając je do ciała. Jej strach zniknął, gdy przeszli przez most i~nie wrócił, gdy staczali się z~niego, z~powrotem do Bombaju, z~powrotem na ulice zatłoczone ruchem ulicznym i~ludźmi. Ominęli saddhu, nagich świętych mężczyzn pokrytych farbą. Kręcili się wokół dabbahwallah, mężczyzn, którzy dostarczali domowe obiady od żon mężom w~całym mieście, w~wiaderkach z~tiffin ułożonych w~ogromne drewniane ramy, balansując na głowach.

Wiedziała, że byli już prawie w~Andheri, kiedy minęli gigantyczne centrum handlowe Infinity, a potem skręcili wzdłuż wysokiego, starożytnego muru z~cegły, który biegł przez setki metrów, ogradzając ogromną posiadłość, która musiała być jednym ze studiów filmowych. Za murem, wzdłuż rowu melioracyjnego, rozciągał się gwarny targ, pełen sprzedawców, restauracji pod gołym niebem, żebraków, rzemieślników, a wśród nich filmowców w~eleganckich garniturach i~ciemnych okularach, trzymających w~dłoniach telefony komórkowe. Motor skręcił przez to wszystko, omijając długi rząd drogich, nieskazitelnie czystych ciemnych samochodów, które ciągnęły się wzdłuż muru w~niekończącej się kolejce do przejścia przez punkt kontroli bezpieczeństwa przy stróżówce.

Przyjęła to wszystko, gdy pędzili wzdłuż muru, skręcając ostro na końcu i~podążając za nim do znacznie węższej bramy. Stało przed nią dwóch strażników z~karabinami przymocowanymi do pasów na łańcuchach i~podnieśli broń, gdy Ashok się zbliżał. Potem zbliżył się jeszcze bliżej, a strażnicy rozpoznali go i~odsunęli się, odsłaniając wąską szczelinę w~ścianie, która była ledwie na tyle szeroka, by mógł przez nią przejechać motocykl, chociaż Ashok przyspieszył, a Yasmin sapnęła, gdy jej falujące rękawy zgrzytnęły o starożytną ceglaną ścianę.

Przejście przez bramę było jak przejście do innego świata. Przed nimi studia rozprzestrzeniły się na zawsze, najdalsza krawędź zagubiona we mgle zanieczyszczeń. Drogi i~ścieżki tworzyły labirynt, omijając największe budynki, jakie Yasmin kiedykolwiek widziała, ogromne budynki, które wyglądały jak stacje kolejowe lub hangary lotnicze z~filmów wojennych. Cały teren składał się z~wypielęgnowanej trawy, uporządkowanych drzew owocowych i~robotników, którzy robili tajemnicze sprawy, z~pasami narzędzi, które brzęczały wokół bioder, niosąc ogromne wiązki rur, drewna i~materiału.

Ashok zawiózł ich obok hangarów -- to muszą być sceny dźwiękowe, w~których kręcili filmy, w~grze Zombie Mecha była dobra mapa studyjna, w~której można było walczyć z~zombie w~szeregu scenerii filmowych z~drewna -- i~w kierunku serii nisko zawieszonych przyczep, które przylegały do ściany po ich lewej stronie. Każda z~nich miał przed sobą miniaturowe ogrodzenie i~mały ogródek kwiatowy, tak schludny i~schludny, że w~pierwszej chwili pomyślała, że kwiaty muszą być sztuczne.

W końcu Ashok zwolnił, a następnie wyhamował, wyłączając silnik. Hałas silnika wciąż jednak buczał jej w~uszach, a ona nadal czuła dudnienie motocykla w~nogach i~pośladkach. Uwolniła dłonie z~pasa Ashoka, rozsuwając palce i~zsiadła z~roweru, potykając się palcem u nogi o lathi i~upadając na trawę. Rumieniąc się, wstała niepewnie, ale wyprostowana.

Ashok uśmiechnął się do niej. 

-- Wszystko w~porządku, siostro?

Chciała w~odpowiedzi powiedzieć coś ostrego i~ciętego, ale nic nie nadeszło. Słowa zostały wyrwane z~niej podczas jazdy. Nagle poczuła się tak, jakby ledwo mogła oddychać, a materiał jej hidżabu wydawał się wypełniony kurzem drogowym, który z~każdym wdechem uwalniał się do jej nosa i~ust. Ostrożnie odpięła szpilkę i~przesunęła hidżab tak, aby nie zakrywał już jej twarzy.

Ashok wpatrywał się w~nią z~przerażeniem. 

-- Ty \ldots  jesteś tylko małą dziewczynką!

Zjeżyła się i~słowa znów do niej dotarły. 

-- Mam \textit{14 }lat  \ldots  w~Dharavi były dziewczyny w~moim wieku z~mężami i~dziećmi! Jestem utalentowaną wojowniczką i~dowódczynią. Nie jestem dziewczynką!

Zarumienił się i~przepraszająco splótł ręce na piersi. 

-- Wybacz mi -- powiedział. -- Ale \ldots  No cóż, założyłem, że masz 18 lub 19 lat. Jesteś wysoka. Przywiozłem cię na tę drogę i~jesteś \ldots  cóż, jesteś dzieckiem! Twoi rodzice będą szaleć ze zmartwień!

Posłała mu swoje najlepsze stalowe spojrzenie, którym zwykła sprawiać, że chłopcy z~Armii zachowywali się, kiedy robili się zbyt, no cóż, \textit{chłopięcy}. 

-- Zostawiłam im notatkę. I wrócę dziś wieczorem. I jestem wystarczająco dorosła, by martwić się takimi rzeczami na własny rachunek, bardzo ci dziękuję. Teraz, skoro przeciągnąłeś mnie przez połowę Indii z~tajemniczego powodu i~jestem pewna, że nie chodziło tylko o to, żebym tu stała i~opowiadała o moim życiu rodzinnym.

Doszedł do siebie i~znów się uśmiechnął. 

-- Przepraszam, przepraszam. Racja, jesteśmy tutaj na spotkaniu. To ważne. Webblies nigdy nie mieli zbyt dużego kontaktu z~prawdziwymi związkami, ale teraz, gdy Nor miała kłopoty, poprosiła mnie, abym zajął się jej sprawą w~tutejszych związkach. Spotkania takie jak to odbywają się dziś na całym świecie, w~Chinach i~Indonezji, w~Pakistanie, Meksyku i~Gwatemali. Ludzie czekają na nas w~środku, to przywódcy związkowi, przedstawiciele związku pracowników przemysłu odzieżowego, hutnicy, nawet Związek Pracowników Transportu i~Doków \ldots  największe związki zawodowe w~Bombaju. Dzięki ich wsparciu Webblies mogą mieć dostęp do pieniędzy, ludzi na linie pikiet, wpływów i~władzy. Ale oni nie wiedzą nic o tym, co wy robicie, nigdy nie grali w~grę. Uważają, że Internet służy do poczty elektronicznej i~pornografii. Więc jesteś tutaj  \ldots  \textit{jesteśmy tutaj } \ldots  aby im to wyjaśnić. 

Kilka razy przełknęła ślinę. W tym wszystkim było tyle rzeczy, których nie rozumiała, a to, co \textit{zrozumiała}, nie była o tym zbyt szczęśliwa. Na przykład ten \textit{prawdziwy }związek  \ldots  Webblie byli prawdziwym związkiem! Ale była bardziej pilna sprawa niż jej irytacja, na przykład: 

-- Co masz na myśli, że \textit{jesteśmy tutaj, aby wyjaśnić}? Czy jesteś graczem?

Potrząsnął głową ze smutkiem. 

-- Nie mam do tego cierpliwości. Jestem ekonomistą. Ekonomistą pracy. Spędziłem dużo czasu z~BSN, opracowując z~nią strategię.

Nie była do końca pewna, kim jest ekonomista, ale czuła też, że przyznanie się do tego może jeszcze bardziej podważyć jej wiarygodność wobec tego mężczyzny, który nazwał ją dzieckiem. 

-- Potrzebuję mojego lathi -- powiedziała.

-- Na tym spotkaniu nie potrzebujesz lathi -- powiedział. -- Nikt nas nie zaatakuje. 

-- Ktoś to ukradnie -- powiedziała.

-- To nie jest Dharavi -- powiedział. -- Nikt tego nie ukradnie.

To wystarczyło. Mogła porozmawiać o problemach w~Dharavi. \textit{Ona }była dziewczyną z~Dharavi. Ale ta nieznajoma nie miała prawa mówić złych rzeczy o swoim domu. 

-- Potrzebuję mojej lathi, na wypadek, gdybym musiała rozwalić ci nią łeb za śmiecenie w~moim domu -- powiedziała przez zaciśnięte zęby.

-- Przepraszam, przepraszam. 

 Przykucnął obok roweru i~zaczął odwijać linki bungee wokół lathi. Uklękła też na jedno kolano i~zaczęła się martwić o opaski, które spinały jej nogawki w~kostkach, ale szły tylko w~jednym kierunku, a gdy już mocno się zapięły, nie poluzowały się. Ashok podniósł wzrok znad linek bungee.

-- Musisz je odciąć -- powiedział. -- Tutaj, chwileczkę. 

Sięgnął do kieszeni spodni i~wyciągnął nikczemny rozkładany nóż, który otworzył z~zatrzaskiem. Delikatnie chwycił pasek na jej prawej kostce i~wsunął ostrze między nim a jej nogę. Wstrzymała oddech, gdy przeciął pasek, po czym obrócił nóż, odwrócił się do jej drugiej nogi i, chwytając ją za kostkę, odciął drugi pasek. Spojrzał na nią. Ich oczy spotkały się, a potem odwróciła wzrok.

-- Uważaj -- powiedziała, chociaż skończył. Wręczył jej lathi. Chwyciła je zdrętwiałymi palcami, prawie upuściła, chwyciła.

-- Dobrze -- powiedział. -- OK. -- Potrząsnął głową. -- Ludzie tam nie wiedzą nic o tobie ani o tym, co robisz. Są trochę, no wiesz, staromodni. -- Uśmiechnął się i~wydawało się, że coś sobie przypomina. -- W niektórych przypadkach bardzo staromodni. I nie radzą sobie zbyt dobrze z~dziećmi. Mam na myśli młodych ludzi. -- Uniósł ręce, gdy podniosła lathi. -- Chcę tylko ostrzec. -- Zastanawiał się nad nią. -- Może mogłabyś znów zakryć twarz? 

Yasmin zastanawiała się nad tym przez chwilę. Oczywiście nie chciała zakrywać twarzy. Chciała po prostu wejść jako ona. Dlaczego nie miałaby być w~stanie? Ale noszenie hidżabu miało pewne zalety, a jedną z~nich było to, że nikt nie zapytałby cię, dlaczego zakrywasz twarz. Ashok najwyraźniej wierzył, że jest znacznie starsza, dopóki go nie rozsunęła.

Bez słowa odpięła materiał, przyłożyła go do twarzy i~ponownie zapięła. Uniósł jej kciuki i~powiedział: 

-- W porządku! Wiesz, to dobrzy ludzie. Bardzo dobrzy ludzie. Chcą być po naszej stronie. -- Przełknął ślinę, pomyślał trochę, kołysał brodą z~boku na bok. -- Ale może jeszcze tego nie wiedzą.

Pomaszerował do drzwi, które były wykonane z~ciężkiego metalowego ekranu na szkle, otworzył je, po czym machnął ręką do środka. Starając się wyglądać tak dostojnie, jak to tylko możliwe, weszła w~mrok przyczepy, gdzie było chłodno, pachniało betelem, chai i~wybielaczem, a leniwy wentylator na suficie tłukł powietrze, ciągnąc za sobą długie smugi kurzu.

To właśnie zauważyła jako pierwsza, a nie ludzi siedzących w~pokoju na kanapach i~fotelach. Ci ludzie zagłębili się w~swoich krzesłach i~siedzieli w~milczeniu, ich oczy gubiły się w~cieniu. Ale po chwili oczy zaczęły się delikatnie zmieniać, patrząc na nią. Ashok wszedł za nią i~powiedział: 

-- Witajcie! Witajcie! Cieszę się, że wszystkim się udało!

A potem wstali i~wszyscy byli znacznie starsi od niej, znacznie starsi niż Ashok. Najmłodszy był w~wieku jej matki, był gruby i~smukły, miał wielkie policzki i~krótkie włosy z~grzywką wokół uszu. Było jeszcze trzech, inny mężczyzna w~kurcie z~muzułmańską czapką i~dwie bardzo stare kobiety w~sari, które pokazywały pomarszczoną skórę na brzuchu.

Ashok przedstawił ich, pana Phadkara ze związku hutników, pana Honnenahalli ze związku pracowników transportu i~pracowników doków oraz panią Rukmini i~panią Muthappa, obie ze związku pracowników przemysłu odzieżowego.

-- Ci dobrzy ludzie są zainteresowani pracą Big Sister Nor, więc poprosiła mnie, abym przyprowadził cię, abyś porozmawiała z~nimi. Panie i~panowie, to jest Yasmin, zaufana działaczka organizacji IWWWW. Jest tutaj, aby odpowiedzieć na wasze pytania.

Wszyscy przywitali ją uprzejmie, ale ich uśmiechy nigdy nie dotarły do ich oczu. Ashok usiadł w~kącie, gdzie stał dzbanek na herbatę i~filiżanki, rozlewając masala chai dla wszystkich i~przynosząc go na tacy. 

-- Będę waszym ćajwalą -- powiedział. -- Po prostu rozmawiajcie.

W gardle Yasmin strasznie wyschło, ale była zasłonięta, więc przekazała czaj, ale szybko tego pożałowała, gdy zaczęła się rozmowa.

-- Rozumiem, że twoja ,,praca'' to tylko granie w~gry, zgadza się? -- powiedział pan Honnenahalli, grubas, który pracował ze związkiem pracowników transportu i~doków.

-- Pracujemy w~grach, tak -- powiedziała Yasmin.

-- A więc organizujesz ludzi, którzy grają w~gry. Jak oni pracują? Dla mnie brzmią jak gracze. W branży transportowej my pracujemy.

Yasmin kołysała brodą z~boku na bok i~cieszyła się ze swojego welonu. Przypomniała sobie rozmowę z~Sushant. 

-- Przypuszczam, że pracujemy tak, jak każdy. Mamy szefa, który prosi nas o wykonanie pracy i~wzbogaca się dzięki naszej pracy.

To sprawiło, że dwie stare ciocie się uśmiechnęły i~chociaż w~pokoju było ciemno, pomyślała, że to autentyczne.

-- Siostro -- powiedział pan Phadkar, ten w~czapce -- opowiedz nam o tych grach. Jak się w~nie gra?

Powiedziała im więc, zaczynając od Zombie Mecha, poparta faktem, że pan Phadkar rzeczywiście widział jeden z~wielu filmów opartych na grze. Ale kiedy zagłębiała się w~klasy postaci, zdobywanie poziomów, odblokowywanie osiągnięć i~tak dalej, zobaczyła, że ich traci.

-- To wszystko brzmi bardzo skomplikowanie -- powiedział pan Honnenahalli po trzydziestu minutach mówienia, a jej gardło było tak suche, że czuła się, jakby zjadła kęs piasku i~soli. -- Kto gra w~te gry? Kto ma czas?

Często słyszała to od swojego ojca, więc powiedziała panu Honnenahalli to, co zawsze mu mówiła. 

-- Miliony ludzi, bogatych i~biednych, mężczyzn i~kobiet, chłopców i~dziewcząt, na całym świecie. Wydają krore i~krore rupii i~tysiące godzin. To gra, tak, ale jest to tak skomplikowane jak życie, w~pewnych sprawach.

Pan Honnenahalli wykrzywił twarz w~kwaśno cytrynowym wyrazie. 

-- Ludzie w~życiu \textit{robią }rzeczy, które mają znaczenie. Oni nie tylko \ldots  -- Machnął ręką, naśladując jakąś bezsensowną pracę. -- Nie tylko naciskają przyciski i~bawią się w~udawanie.

Poczuła, jak rumienią się jej policzki i~znów ucieszyła się z~welonu. Ashok uniósł rękę. 

-- Jeśli skromny ćajwala może tu interweniować. -- Pan Honnenahalli spojrzał na niego wrogo, ale skinął głową. -- ,,Naciskanie guzików i~granie w~udawanie'' opisuje kilka ważnych sektorów gospodarki, nie tylko całą branżę finansową. Czym jest bankowość, jeśli nie naciskaniem guzików i~proszeniem wszystkich, aby uwierzyli, że wyniki mają wartość?

Stare ciocie uśmiechnęły się, a pan Honnenahalli chrząknął. 

-- Jesteś sprytnym gnojkiem, Ashok. Zawsze możesz być sprytny, ale spryt nie karmi ludzi ani nie daje im uczciwego interesu od pracodawców.

Ashok skinął głową, jakby nigdy mu to nie przyszło do głowy, chociaż Yasmin była całkiem pewna po jego uśmiechu, że on też się tego spodziewał. 

-- Panie Honnenahalli, w~tej branży pracuje ponad 9 000 000 osób, a co roku przynoszą one ponad 500 crore rupii. To średnio 6\% wzrostu kwartalnego. A osiem z~20 największych gospodarek na świecie to nie kraje, to gry, emitujące własne waluty, prowadzące własne polityki fiskalne i~ustalające własne prawa pracy.

Pan Honnenahalli skrzywił się, sprawiając, że jego policzki zadrżały, i~uniósł brwi. 

-- Oni mają kodeks pracy w~tych grach? 

-- O tak -- powiedział Ashok. -- Ich polityka polega na tym, że nikt nie może pracować w~ich światach bez ich zgody, że mają absolutną władzę do ustalania płac, zatrudniania i~zwalniania, że mogą cię wygnać, jeśli cię nie lubią lub z~dowolnego innego powodu, i~że każdy przyłapany na łamaniu zasad może zostać pozbawiony wszelkiej wirtualnej własności i~wydalony bez dostępu do procesu, sędziego lub wybranych urzędników.

To zwróciło ich uwagę. Yasmin znała ten opis. Słyszała, jak Big Sister Nor mówiła podobne rzeczy, ale to było lepsze niż jakiekolwiek wcześniejsze wykonanie. I nie można było zaprzeczyć, że ma to wpływ na pokój, podskoczyli, jakby byli zszokowani i~wszyscy otworzyli usta, żeby coś powiedzieć, a potem je zamknęli.

W końcu jedna z~ciotek powiedziała: 

-- Powiedz mi, mówisz, że w~tych miejscach pracuje dziewięć milionów ludzi: gdzie? Bangalore? Pune? Kalkuta?. -- Były to stare miasta IT, gdzie znajdowały się centrale telefoniczne i~firmy technologiczne. 

Ashok skinął. 

-- Niektórzy z~nich tam. Niektórzy tutaj, w~Bombaju. 

Spojrzał na Yasmin, wyraźnie czekając, aż coś powie.

-- Pracuję w~Dharavi -- powiedziała. I czy ona to sobie wyobraziła, czy też ich nosy trochę się pomarszczyły, czy wszyscy subtelnie oddalili się od niej, jakby uciekali przed gównianym smrodem dziewczyny z~Dharavi?

-- Pracuje w~Dharavi -- powiedział Ashok. -- Ale tylko milion lub dwa pracują tutaj w~Indiach. Większość z~nich znajduje się w~Chinach, Indonezji lub Wietnamie. Niektórzy w~Ameryce Południowej, niektórzy w~Stanach Zjednoczonych. Gdziekolwiek jest IT, są ludzie, którzy pracują w~grach .

Teraz ciocia usiadła. 

-- Rozumiem -- powiedziała. -- Cóż, to bardzo interesujące, Ashok, ale co mamy wspólnego z~Chinami? Nie jesteśmy w~Chinach.

Yasmin pokręciła głową. 

-- Gra nie jest w~Chinach -- powiedziała, jakby wyjaśniała coś dziecku. -- Gra jest wszędzie. Wszyscy gracze są w~tym samym miejscu.


-- Nie rozumiesz, siostro -- powiedział pan Phadkar.  -- Pracownicy w~tych miejscach konkurują z~naszymi pracownikami. Duże firmy chodzą tam, gdzie praca jest najtańsza i~najbardziej niezorganizowana. Nasi członkowie tracą pracę dla tych ludzi, ponieważ nie mają szacunek dla samego siebie, aby walczyć o godziwą płacę. Nie możemy konkurować z~Chińczykami, Indonezyjczykami czy Wietnamczykami, nawet tutejsi żebracy oczekują lepszej płacy, niż dostają! 

Pan Honnenahalli poklepał się po brzuchu i~skinął głową. 

-- Jesteśmy indyjskimi robotnikami. Reprezentujemy ich. Ci robotnicy, co się z~nimi dzieje  \ldots  to nie nasza sprawa.

Ashok skinął głową.

 -- Cóż, to w~porządku wobec twojego związku i~członków. Ale związek, dla którego pracuje Yasmin \ldots 

Pan Honnenahalli prychnął, a jego policzki zadrżały. 

-- To nie jest związek -- powiedział. -- To banda dzieciaków grających w~gry!

-- To dziesiątki tysięcy zorganizowanych robotników solidaryzujących się ze sobą -- powiedział łagodnie Ashok, jakby był nauczycielem korygującym ucznia. -- W 14 krajach. Spójrz, ci gracze, są już zorganizowani w~gildie. To już praktycznie związki zawodowe. Martwisz się, że związkowe prace w~Indiach mogą stać się pozazwiązkowymi pracami w~Wietnamie  \ldots  cóż, oto jak możesz też zorganizować pracowników w~Wietnamie! Firmy są międzynarodowe  \ldots  dlaczego pracownicy mają nadal trzymać się granic? Co właściwie oznacza granica?

-- Dużo, jeśli granica jest z~Pakistanem. Ludzie \textit{umierają }za granice, synu. Możesz siedzieć tam, z~wykształceniem wyższym i~rozmawiać o tym, że granice nie mają znaczenia, ale wszystko to oznacza, że jesteś całkowicie poza kontaktem z~przeciętnym indyjskim robotnikiem. Indyjscy robotnicy chcą indyjskich miejsc pracy, a nie pracy dla Chińczyków czy co tam. Niech Chińczycy zorganizują Chińczyków.

-- \textit{Organizują} -- wtrąciła się Yasmin. -- W tej chwili strajkują w~Chinach! Cała fabryka wyszła z~sali, a policja ich pobiła. A ja pomagałam w~ich pikiecie.

Pan Honnenahalli przygotowywał się do dalszego wygrażania, ale jedna ze starych ciotek położyła słabą dłoń na jego przedramieniu.

-- Jak pomogłaś w~pikietowaniu w~Chinach z~Dharavi, córko?

I tak Yasmin opowiedziała im historię bitwy o Mushroom Kingdom, historię bitwy pod Shenzhen i~to, co widziała i~słyszała.

-- Dziki strajk -- powiedział Honnenahalli. -- Szaleństwo. Bez strategii, bez organizacji. Skazani. Ci pracownicy mogą już nigdy nie ujrzeć światła dziennego.

-- Nie, chyba że ich towarzysze zjednoczą się dla nich -- powiedział Ashok. -- Towarzysze tacy jak Yasmin i~jej grupa. Chcesz zobaczyć coś, o co pracownicy są gotowi walczyć? Musisz udać się do kafejki internetowej i~zobaczyć. Zobacz, kto nie ma kontaktu z~pracownikami. Możesz mówić, co chcesz o ,,hinduskich pracownikach'', ale dopóki nie będziesz solidarny ze \textit{wszystkimi }robotnikami, nigdy nie będziesz w~stanie ochronić swoich cennych \textit{indyjskich pracowników}. -- Teraz tracił panowanie nad sobą, tracił chłód nauczyciela. -- Ci pracownicy zostali źle potraktowani przez swojego pracodawcę, więc zastrajkowali. Ich miejsca pracy można po prostu przenieść  \ldots  do Wietnamu, Kambodży, do Dharavi  \ldots  i~ich strajk został złamany. Nie \textit{widzisz tego}? \textit{W końcu mamy te same narzędzia co szefowie!} Dla fabrykanta wszystkie miejsca są takie same i~nie ma znaczenia czy koszule są szyte tu czy tam, o ile po zakończeniu można je załadować do kontenera transportowego. Ale teraz dla nas wszystkie miejsca też są takie same! Możemy pojechać wszędzie, po prostu siadając przy komputerze. Przez czterdzieści lat sytuacja pracowników stawała się coraz trudniejsza  \ldots  teraz nadszedł czas, aby to zmienić.

Yasmin poczuła, że uśmiecha się pod zasłoną. O to chodzi, Ashok, powiedz to! Ale wtedy zobaczyła twarze starych ludzi w~pokoju: kamienne i~pozbawione serca.

-- To miłe słowa -- powiedziała jedna z~ciotek. -- Szczerze. To piękna wizja. Ale moi pracownicy nie mają komputerów. Nie chodzą do kafejek internetowych. Całymi dniami farbują ubrania. Kiedy ich praca wyjeżdża za granicę, nie mogą ich ścigać Twoimi komputerami.

-- Mogą też być częścią Webblies! -- powiedziała Yasmin. -- W tym jest piękno. Ci, którzy pracują w~grach, możemy iść wszędzie, organizować się wszędzie, a gdziekolwiek są twoi pracownicy, my też! Możemy iść wszędzie, nikt nie może nas powstrzymać. Możemy organizować farbiarzy wszędzie, przez graczy.

Pan Honnenahalli skinął głową. 

-- Tak myślałem. A kiedy to wszystko zostanie zrobione, Webblies zorganizują wszystkich robotników na świecie i~nasze związki, co się z~nimi stanie? Znikną? Czy zostaną wchłonięci przez was? O tak, rozumiem bardzo dobrze. Bardzo fajny układ. Z pewnością gracie w~gry, tam w~Webblies.

Ashok i~Yasmin zaczęli mówić jednocześnie, po czym oboje przerwali i~wymienili spojrzenia. 

-- To nie tak -- powiedziała Yasmin. -- Oferujemy pomoc. Nie chcemy przejęcia.

-- Być może nie, ale może ktoś inny -- powiedział pan Honnenahalli.  --  Czy możesz mówić w~imieniu wszystkich? Mówisz, że nigdy nie spotkałaś tej Big Sister Nor, ani jej poruczników, Mighty cokolwiek i~Justboba.

-- Spotkałam ich dziesiątki razy -- powiedziała cicho Yasmin.

-- Och, na pewno. W grze. Jaki jest ten stary żart z~Ameryki? W Internecie nikt nie wie, że jesteś psem. Być może ci twoi przyjaciele to starcy albo małe dzieci. Może są w~następnej kafejce internetowej w~Dharavi. Internet jest pełen kłamstw, sztuczek i~brudu, siostrzyczko \ldots  -- Jej plecy zesztywniały. Jedną rzeczą było nazywanie się ,,siostrą'' ale ,,siostrzyczka'' nie było przyjazne. To było odprawienie. -- A kto powiedział, że nie dałaś się nabrać na jedną z~tych sztuczek?

Ashok uniósł rękę. 

-- Może to wszystko jest snem. Być może wszyscy jesteście wytworami mojej wyobraźni. Dlaczego mielibyśmy wierzyć w~cokolwiek, jeśli to jest standard, do którego wszyscy muszą się wznieść? Rozmawiałem wiele razy z~Big Sister Nor oraz z~innymi członkami IWWWW na całym świecie. Reprezentujesz dwa miliony pracowników budowlanych, ilu z~nich \textit{ty} spotkałeś? Skąd mamy wiedzieć, że \textit{oni }są prawdziwi?

-- To prowadzi nas donikąd -- powiedziała jedna z~ciotek. -- Byłeś bardzo miły, że przyszedłeś odwiedzić nas, Ashok, i~ty też, Yasmin. Było to bardzo uprzejme, gdy powiedziałaś nam, co zamierzasz. Dziękuję.

-- Zaczekajcie -- powiedział Ashok. -- To nie może być wszystko! Przybyliśmy tu, by poprosić cię o pomoc  \ldots  o \textit{solidarność}. Właśnie mieliśmy pierwsze uderzenie, a nasza komórka wykonawcza jest niedostępna i~zaginęła \ldots  -- Yasmin odwróciła głowę. Co to oznaczało? -- A potrzebujemy pomocy: funduszu strajkowego, wsparcia administracyjnego, pomocy prawnej \ldots 

-- Wykluczone -- powiedział pan Honnenahalli.

-- Obawiam się, że tak -- powiedział pan Phadkar. -- Przepraszam, bracie. Nasz statut nie pozwala nam interweniować w~innych związkach, zwłaszcza w~organizacji, którą reprezentujesz.

-- To niemożliwe -- powiedziała jedna z~ciotek z~zaciśniętymi ustami i~przepraszająco. -- To po prostu nie jest coś, co robimy.

Ashok podszedł do czajnika i~zabrał się do robienia więcej chai. 

-- Cóż, przepraszam, że zmarnowałem wasz czas -- powiedział. -- Jestem pewien, że coś wymyślimy.

Wszyscy wpatrywali się w~siebie, po czym pan Honnenahalli wstał ze świszczącym oddechem, podniósł leżącą u jego stóp wypchaną teczkę i~wyszedł z~małego budynku. Pan Phadkar podążył za nim, uśmiechając się miękko do ciotek i~niepewnie machając do Yasmin. Nie spojrzała mu w~oczy. Jedna z~ciotek wstała i~próbowała coś powiedzieć Ashokowi, ale on wzruszył ramionami. Wróciła do swojej partnerki i~pomogła jej stanąć na stare, niepewne stopy. Obie ścisnęły ramiona Yasmin przed odejściem.

Gdy drzwi zatrzasnęły się za nimi, Ashok odwrócił się i~zasyczał \textit{banchode} na pokój. Yasmin codziennie słyszała gorsze słowa niż te w~zaułkach Dharavi i~w pokoju gier, kiedy Armia walczyła, i~usłyszenie ich od tego miękkiego chłopca niemal wywołało u niej chichot. Ale usłyszała zdławienie w~jego głosie, jakby powstrzymywał łzy, i~nie miała już ochoty się uśmiechać. Sięgnęła w~górę i~odpięła hidżab, zakładając go na szyję, uwalniając twarz, by ostygła w~parnym powietrzu, którym otaczał ich wentylator. Podeszła do Ashoka, wzięła od niego filiżankę herbaty i~wypiła ją tak szybko, jak mogła, rozkoszując się ciepłą wilgocią na jej suchym, drapiącym gardle. Teraz gdy jej twarz była wolna od hidżabu, poczuła silny zapach starej śliny betelu i~zobaczyła, że listwy przypodłogowe porysowanych ścian były poplamione na różowo starą śliną.

-- Ashok -- powiedziała, używając głosu, którego używała do wymuszania dyscypliny w~Armii. -- Ashok, spójrz na mnie. O co chodziło \ldots  w~tym \textit{spotkaniu}? Dlaczego tu byłam?

Usiadł na krześle, które właśnie zwolnił pan Phadkar, i~sączył herbatę.

-- Och, zrobiłem z~tego cholerny bałagan, tak -- powiedział.

-- Ashok -- powiedziała z~surową nutą w~głosie. -- Poskarż się później. Mów teraz. Po co właśnie przeciągnąłeś mnie przez połowę Bombaju?

-- Pracuję nad tym spotkaniem od miesięcy, odkąd poprosiła mnie o to Big Sister Nor. Powiedziałem jej, że myślę, że związki zawodowe mogłyby przyjąć Webblies, zobaczyłyby moc globalnego ruchu pracowniczego, który mógłby zorganizować się wszędzie na raz. Od razu spodobał jej się ten pomysł i~od tego czasu słodko rozmawiam z~dyrektorami związku, próbując przekonać ich, aby zobaczyli potencjał. Z ich członkami pomagającymi nam  \ldots  a nasi członkowie im pomagający  \ldots  moglibyśmy zmienić świat. Zmienić go o tak! -- Pstryknął palcami. -- Ale potem wybuchł strajk i~Big Sister Nor powiedziała mi, że potrzebuje pomocy \textit{na już}, w~przeciwnym razie ci towarzysze wylądowaliby w~więzieniu na zawsze lub gorzej. Powiedziała, że pomyślała, że mogłabyś pomóc, i~planowaliśmy porozmawiać o tym, zanim zeszliśmy na dół, ale potem, kiedy jechałem po ciebie \ldots  -- Przerwał, napił się czaju, zapatrzył się na brudne, przesłonięte oknami wypielęgnowane tereny studia filmowego. -- Dostałem telefon od Mighty Krang. Zostali pobici. Mocno. Cała trójka, chociaż Krangowi udało się uciec. Big Sister Nor jest nieprzytomna w~szpitalu. Mighty Krang powiedział, że myślał, że to jeden z~chińskich właścicieli fabryk  \ldots  stają się coraz bardziej złośliwi, wysyłają groźby. I mają mnóstwo kontaktów w~Singapurze.

Yasmin dopiła herbatę. Włosy swędziały od kurzu i~potu, wsunęła pod nie palec i~podrapała kropelkę potu, która spływała jej po głowie. 

-- W porządku -- powiedziała. -- Czego oczekiwałeś od tych starych ludzi?

-- Pieniędzy -- powiedział. -- Wsparcia. Mają wejście do prasy. Jeśli ich członkowie domagali się sprawiedliwości dla robotników w~Shenzhen, zbierali się pod konsulatami chińskimi w~całych Indiach \ldots  -- Machnął rękami. -- Nie jestem pewien, szczerze mówiąc. To miało się stać za kilka tygodni, po tym, jakbym dużo więcej naszeptał im do uszu, dowiadując się, czego chcą, co mogą dać, co my możemy im dać. To nie miało się wydarzyć w~środku strajku. -- Wpatrywał się żałośnie w~podłogę.

Yasmin pomyślała o Sushant, o jego strachu przed opuszczeniem armii Mali. Dopóki żołnierze tacy jak on walczyli po drugiej stronie, Webblies nie będą w~stanie zablokować ataków w~grze. Więc. Więc będzie musiała powstrzymać armię Mali. Powstrzymać wszystkie armie. Żołnierze, którzy walczyli dla szefów, byli po złej stronie. Zobaczą to.

-- A jeśli byśmy sami sobie pomogli? -- spytała. -- A co, jeśli staniemy się tak wielcy, że związki zawodowe będą musiały do nas dołączyć?

-- Tak, co jeśli, co jeśli. Tak łatwo jest zagrać w~co-jeśli. Ale nie widzę, jak to się stanie.

-- Myślę, że w~grach mogę zdobyć więcej wojowników. Możemy obronić każdy strajk.

-- Cóż, to jest dobre dla gier, ale to nie pomaga graczom. Big Sister Nor nadal jest w~szpitalu. Webblies w~Shenzhen nadal są w~więzieniu.

-- Mogę zrobić tylko to, co mogę -- powiedziała Yasmin. -- Co możesz zrobić? Co robią ekonomiści?

Wyglądał smutno. 

-- Idziemy na uniwersytet i~uczymy się dużo matematyki. Używamy matematyki, aby przewidzieć, co duża liczba ludzi zrobi ze swoimi pieniędzmi i~pracą. Następnie staramy się wymyślić zalecenia, jak na to wpłynąć.

-- I to właśnie robisz ze swoim życiem?

-- Tak, wydaje mi się, że to wszystko brzmi cholernie bezsensownie, co nie? Może właśnie dlatego jestem gotów brać te gry tak poważnie  \ldots  nie są mniej wyimaginowane niż wszystko, co robię. Ale zostałem ekonomistą, ponieważ nic bez tego nie miało sensu. Dlaczego moi rodzice byli biedni? Dlaczego nasi kuzyni w~Ameryce byli tacy bogaci? Dlaczego Ameryka miałaby wysyłać swoje śmieci do Indii? Dlaczego Indie miałyby wysyłać swoje drewno do Ameryki? Dlaczego kogoś obchodzi złoto?

-- To jest naprawdę dziwne. Złoto jest tak bezużyteczną rzeczą, wiesz? Jest ciężkie, nie nadaje się zbytnio do robienia rzeczy  \ldots  zbyt miękkie do naprawdę długiej biżuterii. Stal nierdzewna jest o wiele lepsza do pierścionków. -- Stuknął w~prawą rękę zawiły pierścień w~oparcie krzesła. -- Oczywiście nie ma go dużo. Całe złoto, jakie kiedykolwiek wydobyliśmy z~ziemi, utworzyłoby sześcian o bokach długości kortu tenisowego. -- Yasmin widziała zdjęcia kortów tenisowych, ale nie wiedziała, jak duże rzeczywiście były. Przypuszczała, że nie bardzo duże. -- Wykopujemy to z~jednej dziury w~ziemi, a następnie wkładamy do innej dziury w~ziemi, gdzieś w~jakimś skarbcu i~nazywamy to pieniędzmi. Wydawało się to śmieszne.

-- Ale wszyscy \textit{wiedzą}, że złoto jest cenne. Jak wszyscy się z~tym zgodzili? Właśnie wtedy zacząłem się naprawdę fascynować. Ponieważ złoto i~pieniądze są naprawdę blisko spokrewnione. Kiedyś pieniądze były po prostu łatwym sposobem na noszenie złota. Rząd wypełniał złotem dziurę w~ziemi, a następnie drukował noty mówiące: ,,Ten papier jest wart tyle gramów złota''. Czyli zamiast nosić ze sobą ciężkie złoto, żeby kupować rzeczy, moglibyśmy nosić ze sobą łatwe papierowe pieniądze.

-- To zabawne, prawda? Wykopujemy złoto z~dziur w~ziemi, ważymy je, a następnie wkładamy do innej dziury w~ziemi! Co dobrego jest w~złocie? Cóż, stawia granicę, ile naprawdę pieniędzy może stworzyć państwo. Jeśli chcą zrobić więcej pieniędzy, muszą skądś zdobyć więcej złota. 

-- Dlaczego to ma znaczenie, ile pieniędzy drukuje dany kraj?

-- Cóż, wyobraź sobie, że rząd zdecydował się wydrukować krorę rupii dla każdego mieszkańca Indii. Wszyscy bylibyśmy bogaci, prawda?

Yasmin zastanawiała się przez chwilę. 

-- Nie, oczywiście, że nie. Wszystko by zdrożało, prawda?

Poruszył brodą. Znowu brzmiał jak nauczyciel. 

-- Bardzo dobrze -- powiedział. -- To jest inflacja: więcej pieniędzy sprawia, że wszystko jest droższe. Gdyby inflacja była równomierna, nie byłoby tak źle. Powiedzmy, że twoja pensja podwoiła się z~dnia na dzień, podobnie jak wszystkie ceny  \ldots  wszystko będzie w~porządku, ponieważ możesz po prostu kupić tyle, ile mogłeś poprzedniego dnia, chociaż ,,kosztuje'' dwa razy więcej. Ale jest z~tym problem. Czy wiesz, co to jest?

Yasmin myślała. 

-- Nie wiem. 

Pomyślała trochę więcej. Ashok kiwał jej głową, a ona czuła, że było to coś oczywistego, prawie widocznego. 

-- Po prostu nie wiem.

-- Podpowiedź -- powiedział. -- Oszczędności. 

Pomyślała o tym trochę więcej. 

-- Oszczędności. Gdybyś miał zaoszczędzone pieniądze, nie podwoiłyby się one wraz z~zarobkami, prawda? -- Potrząsnęła głową. -- Nie rozumiem jednak, dlaczego jest to taki problem. Zaoszczędziliśmy trochę pieniędzy, ale to tylko kilka tysięcy rupii. Jeśli zarobki podwoją się, szybko odzyskamy je z~napływających nowych pieniędzy.

Wyglądał na zaskoczonego, potem się roześmiał. 

-- Przepraszam -- powiedział. -- Oczywiście. Ale są ludzie, firmy i~rządy, które mają \textit{dużo }oszczędności. Bogaci ludzie mogą zaoszczędzić miliony rupii  \ldots  te oszczędności zostałyby zmniejszone o połowę z~dnia na dzień. Albo szpital może zaoszczędzić wiele milionów rupii na nowy budynek. Albo rząd lub związek może mieć miliony oszczędności na emerytury. Co, jeśli pracujesz całe życie na emeryturę w~wysokości dwóch tysięcy rupii miesięcznie, a potem, na rok przed planowanym rozpoczęciem jej zbierania, zostanie ona obcięta w~połowie? 

Yasmin nie znała nikogo, kto miałby emeryturę, choć o nich słyszała. 

-- Nie wiem -- powiedziała. -- Przypuszczam, że byś pracował.

-- Nie ułatwiasz tego -- powiedział Ashok. -- Pozwólcie, że ujmę to w~ten sposób: jest wielu potężnych, bogatych ludzi, którzy byliby bardzo zdenerwowani, gdyby inflacja zniszczyła ich oszczędności. Ale rządy są kuszone przez inflację. Powiedzmy, że toczysz kosztowną wojnę i~musisz kupować czołgi i~płacić żołnierzom, wypuszczać samoloty w~powietrze i~produkować pociski w~fabrykach. To drogie rzeczy. Musisz za to jakoś zapłacić. Możesz pożyczyć pieniądze \ldots 

-- Rządy pożyczają pieniądze?

-- O tak, są szokującymi żebrakami! Pożyczają je od innych rządów, od firm  \ldots  nawet od własnych ludzi. Ale jeśli nie masz szans na wygranie wojny, lub jeśli zwycięstwo cię zniszczy, wtedy jest mało prawdopodobne, aby ktokolwiek dobrowolnie pożyczył ci pieniądze na walkę. Ale rządy nie muszą polegać na dobrowolnych płatnościach, prawda?

Yasmin widziała, dokąd to zmierza. 

-- Mogą po prostu opodatkować ludzi.

-- Dobrze -- powiedział. -- Gdybyś nie była tak wyraźnie rozsądną dziewczyną, sugerowałbym, żebyś spróbowała kariery jako ekonomista, Yasmin! OK, więc rządy mogą po prostu podnieść podatki. Ale ludzie, którzy muszą płacić zbyt wysokie podatki, raczej nie zagłosują na ciebie następnym razem. A jeśli jesteś dyktatorem, nic nie sprowadza rewolucjonistów na ulicę szybciej niż niekontrolowane opodatkowanie. Tak więc podatki mają ograniczone zastosowanie w~opłacaniu wojny.

-- I dlatego rządy lubią inflację, prawda?

-- Znowu poprawnie! Po pierwsze, rządy mogą drukować dużo pieniędzy, które mogą wykorzystać na zakup rakiet, czołgów i~tak dalej, jednocześnie pożyczając jeszcze więcej, tak szybko, jak tylko mogą. Potem kiedy ceny i~płace rosną i~rosną  \ldots  powiedzmy, sto razy  \ldots  potem nagle bardzo łatwo jest spłacić wszystkie pożyczone pieniądze. Może trzeba było tysiąca pracowników do opodatkowania, aby dodać do krora rupii przed inflacją, a teraz wystarczy jeden. Oczywiście, osoba, która pożyczyła ci pieniądze, jest w~tarapatach, ale do tego czasu wygrałeś wojnę, zostałeś ponownie wybrany, a wszystko to bez sparaliżowania twojego kraju długiem. Brawo.

Yasmin obróciła to w~głowie. Uznała, że jest to zaskakująco łatwe do naśladowania, wszystko, co musiała zrobić, to pomyśleć o tym, co dzieje się z~cenami towarów w~różnych grach, w~które grała, i~mogła łatwo zobaczyć, jak inflacja będzie działać na korzyść niektórych graczy, a nie innych. 

-- Ale rządy nie muszą wykorzystywać inflacji tylko do wygrywania wojen, prawda? -- Pomyślała o politykach, którzy przeszli przez Dharavi, zbierając głosy, które mogli oddać tam ludzie. Pomyślała o ich obietnicach. -- Możesz użyć inflacji do budowy szkół, szpitali, tego typu rzeczy. Potem kiedy dług cię dopadnie, możesz po prostu użyć inflacji, aby go zlikwidować. Otrzymasz w~ten sposób wiele głosów, jestem całkiem pewna.

-- O tak, to druga strona równania. Rządy zawsze starają się o reelekcję za pomocą broni lub masła, lub obu. Z pewnością można uzyskać wiele głosów, kupując wiele szpitali i~szkół pod wpływem inflacji, ale inflacja jest jak tłuste jedzenie, w~końcu zawsze płacisz za to cenę. Gdy nadejdzie hiperinflacja, nikt nie może zapłacić nauczycielom, pielęgniarkom ani lekarzom, więc następne wybory prawdopodobnie zakończą twoją karierę.

-- Ale pokusa jest potężna, bardzo potężna. I tu właśnie pojawia się złoto. Czy możesz pomyśleć, jak?

Yasmin pomyślała jeszcze trochę. Złoto, inflacja; inflacja, złoto. Tańczyły w~jej głowie. Wtedy to miała. 

-- Nie możesz zrobić więcej pieniędzy, jeśli nie masz więcej złota, prawda?

Uśmiechnął się do niej. 

-- Złota Gwiazda! -- powiedział -- Dokładnie to. To jest dokładnie to, co bogaci ludzie lubią w~złocie. To dyscyplina, policjant w~skarbcu i~powstrzymuje rząd przed pokusą finansowania ich szaleństwa fałszywymi pieniędzmi. Jeśli masz dużo oszczędności, chcesz zdyscyplinować rządowe nawyki drukowania pieniędzy, ponieważ każda rupia, którą drukują, dewaluuje twoje własne bogactwo. Ale żaden rząd nie ma wystarczająco dużo złota, aby pokryć pieniądze, które wydrukował. Niektóre rządy wypełniają swoje skarbce innymi cennymi rzeczami, takimi jak inne dolary lub euro.

-- A więc dolary i~euro są oparte na złocie?

-- Zupełnie nie! Nie, wspierają je inne waluty, małe kawałki metalu, marzenia i~przechwałki. Tak więc w~ostatecznym rozrachunku wszystko opiera się na niczym!

-- Zupełnie jak złoto z~gry! -- powiedziała.

-- Kolejna złota gwiazda! Nawet \textit{złoto }nie jest oparte na złocie! W większości przypadków, jeśli kupujesz złoto w~prawdziwym świecie, po prostu kupujesz zaświadczenie, że posiadasz sztabkę złota w~jakimś skarbcu gdzieś na świecie. Listonosz nie dostarcza złotej cegiełki przez skrzynkę pocztową. A oto brudny sekret złota: jest więcej złota według dostępnych certyfikatów depozytowych niż kiedykolwiek wykopano z~ziemi.

-- Jak to możliwe? 

-- Jak myślisz, że to możliwe? 

-- Ktoś drukuje certyfikaty bez złota na ich poparcie?

-- To dobra teoria. Myślę, że tak się dzieje. Powiedzmy, że masz skarbiec pełen złota w~Hongkongu. Nazwij to tysiącem sztabek. Sprzedajesz złoto tysiąca sztabek przez rynek certyfikatów i~zamykasz drzwi. Jakiś czas później ktoś, ochroniarz, dyrektor banku, wchodzi do skarbca i~wychodzi z~dziesięcioma sztabkami złota ze środka stosu. Te dziesięć sztabek złota jest sprzedawanych na rynku metali i~ląduje w~skarbcu w~Szwajcarii, który drukuje certyfikaty \textit{swoich }zasobów złota i~sprzedaje je dalej. Później pewnego dnia dyrektor szwajcarskiego banku bierze sobie dziesięć sztabek z~\textit{tego }skarbca i~są one sprzedawane na rynku metali. Zanim się spostrzeżesz, Twoje dziesięć sztabek złota zostało sprzedanych setce różnych osób. 

-- To inflacja! 

Klasnął. 

-- Wspaniała uczennica! Prawidłowo. Jest takie powiedzenie z~fizyki: ,,Żółwie są aż do samego dołu''. Znasz je? Pochodzi z~historii o brytyjskim fizyku Bertrandzie Russellu, który wygłosił wykład o wszechświecie, o tym, jak Ziemia krąży wokół Słońca itd. A mała stara babcia na widowni mówi: ,,To wszystko bzdury! Świat jest płaski i~spoczywa na grzbiecie żółwia!''. A Russell mówi: ,,Jeśli tak, to na czym stoi żółw?''. A babcia mówi: ,,Nie możesz mnie oszukać, synku, to żółwie aż do dołu!'' Innymi słowy, to, co żyje pod iluzją, to kolejna iluzja, a pod nią znowu jest inna iluzja. Podobno dobra waluta jest wspierana przez złoto, ale samo złoto nie istnieje. Zła waluta nie jest poparta złotem, ale jest oparta na innych waluty, a \textit{one }nie istnieją. Pod koniec dnia wszystko, na czym to wszystko się opiera, to, co, możesz mi powiedzieć?

-- Wiara -- powiedziała Yasmin. -- Albo strach, tak? Obawiasz się, że jeśli przestaniesz wierzyć w~pieniądze, nie będziesz mogła niczego kupić. To \textit{jak }złoto do gry! Pamiętam, jak kiedyś Zombie Mecha zaczął pobierać premie, które kiedyś były darmowe i~z dnia na dzień wszyscy gracze odeszli. Ludzie, którzy zostali, byli tak zdesperowani, krążyli wokół, próbując sprzedać swoje złoto i~broń, oferując ceny, które były niskie w~porównaniu z~zaledwie kilkoma dniami wcześniej. To było tak, jakby wszyscy przestali w~to wierzyć Zombie Mecha, a potem gra przestała istnieć! A potem gra zniżyła ceny, a ludzie wrócili, a ceny znów wzrosły.

-- Nazywamy to ,,zaufaniem'' -- powiedział Ashok. -- Jeśli ,,ufasz'' gospodarce, możesz użyć jej pieniędzy. Jeśli nie masz zaufania do gospodarki, chcesz od niej uciec i~żeby od ciebie też uciekła. A to żółwie aż do samego końca. Nie ma prawie nic, co jest warte \textit{czegokolwiek}, poza zaufaniem. Idź do odlewni stali tutaj w~Bombaju, a znajdziesz ludzi ryzykujących życie, pracujących w~ogniu piekielnym na bosych stopach bez hełmów i~rękawic, odlewających stal, by zrobić wielkie metalowe płyty zakrywające włazy do kanałów ściekowych w~Ameryce. Dlaczego to robią? Ponieważ dostają rupie, które są nic nie warte, jeśli nie masz do nich zaufania. I dlaczego dostają rupie? Ponieważ ktoś, szef, myśli, że dostanie dolary za swoje stalowe dyski. Ile są warte dolary?

-- Nic? 

-- \textit{Nic}! Chyba, że w~nie wierzysz. A co z~dyskami, na co one? Mają złą średnicę do otworów ściekowych w~Bombaju. Mogłabyś je stopić i~zrobić coś innego z~nimi, ale poza tym, to po prostu cholernie ciężkie herbatniki, które nie służą żadnemu celowi. Więc dlaczego tak się dzieje?

-- Och, to proste -- odpowiedziała  Yasmin. -- Naprawdę nie wiesz?

-- To łatwe? Proszę, powiedz mi. To nie jest dla mnie łatwe, a studiowałem to przez całe życie. 

-- To wszystko dzieje się, ponieważ to \textit{gra}!

Wyglądał na urażonego. 

-- Może jest to gra dla bogatych i~potężnych  \ldots  ale nie jest to zabawne dla biednych, robotników i~oszczędzających, którzy źle ją kończą.

-- Gry nie muszą być \textit{zabawne}, tylko muszą być, nie wiem, \textit{interesujące}? Nie, \textit{wciągające}! Jest tak wiele razy, kiedy gram, gram i~gram, i~nie mogę przestać nawet, chociaż wszystko stało się bardzo nudne i~powtarzalne. ,,Jeszcze jedno zadanie'', mówię sobie. ,,Jeszcze jedno zabójstwo''. A potem znowu: ,,Jeszcze raz, jeszcze raz, jeszcze raz''. Ważną rzeczą w~grze nie jest to, jaka jest fajna, ale jak łatwo jest zacząć grać i~jak trudno jest przestać.

-- Aha. OK, to ma sens. Co konkretnie sprawia, że trudno jest przestać? -- 

-- Och, wiele drobiazgów. Na przykład w~Zombie Mecha, jeśli przestaniesz grać bez pójścia do mecha-bazy, stajesz się ,,zmęczony''. Więc kiedy wracasz do gry, grasz gorzej i~zdobywasz mniej punktów za te samo zabijanie i~bieganie w~tych samych lochach. Więc myślisz: ,,OK, skończyłem na dzisiaj, czas wracać do bazy''. I biegniesz do bazy, która nigdy nie jest zbyt blisko zadań, a po drodze dostajesz nowe zadanie, krótkie, które ma wiele dobrych nagród. Wykonujesz zadanie. Teraz znów kierujesz się do bazy, ale znowu znajdujesz się na misji, ale ta jest trochę dłuższa, niż się wydawało, a teraz minęło jeszcze więcej czasu. W końcu docierasz do bazy, ale grałeś tak dużo, że to prawie wyższy poziom, i~szkoda byłoby przestać grać, teraz gdy tylko kilka przypadkowych killów dałoby ci wyższy poziom, a potem możesz kupić kilka bardzo dobrych nowych broni i~szkolenia w~bazie, więc polujesz na niektórych gryzoni wokół wejście bazy, a teraz awansujesz, dostajesz kilka dobrych nowych broni, a także właśnie odblokowałeś wiele nowych zadań. Te misje są przyznawane ci, gdy dotrzesz do bazy, a niektóre z~nich wyglądają bardzo interesująco, a teraz dołączyli do ciebie niektórzy twoi znajomi, więc możesz się z~nimi grupować i~wspólnie wykonywać zadania, co będzie znacznie szybsze i~będzie dużo więcej zabawy. A kiedy przestajesz grać, minęło trzy, czasem cztery godziny więcej, niż myślałeś. 

-- To się często zdarza? 

-- O tak. U mnie wiele razy w~tygodniu . I nawet nie gram o punkty  \ldots  gram, aby pomóc związkowi! Im więcej grasz, tym więcej sensu ma granie dalej. Cały ten biznes ze złotem, rupiami, dolarami i~stalowymi płytami, gramy w~tę grę cały czas, prawda? Oczywiście, że to działa. Wszyscy grają, ponieważ wszyscy grają w~nią przez całe życie.

-- Rozumiem, dlaczego Big Sister Nor powiedziała mi, że muszę z~tobą porozmawiać -- powiedział. -- Jesteś bardzo mądrą dziewczyną.

Spojrzała w~dół.

-- Co zrobimy z~Big Sister Nor?

-- Uważa, że musimy znaleźć pieniądze i~wsparcie dla strajkujących. Myślę, że potrzebuje pieniędzy i~wsparcia dla \textit{siebie}. Mówi, że nic jej nie jest, ale jest w~szpitalu i~wygląda na to, że została ciężko pobita.

-- Jak możemy dać jej wsparcie stąd? . Są tak daleko -- Myśląc: \textit{przeciwległy róg Bombaju jest dla mnie daleko, Chiny mogą równie dobrze być księżycem lub Mushroom Kingdom.} -- A skąd mamy wiedzieć, że Big Sister Nor będzie bezpieczna tam, gdzie jest?

-- Dwa dobre pytania -- powiedział. -- To frustrujące. Są tak blisko, gdy wszyscy jesteśmy online, ale tak daleko, gdy musimy zrobić coś, co dotyczy świata fizycznego. -- Zaczął chodzić. -- To jest wydział Big Sister Nor. Widzi sposób na powiązanie świata wirtualnego ze światem rzeczywistym, aby przenieść pracę, pomysły i~pieniądze z~jednego do drugiego.

-- Może powinniśmy więc po prostu skoncentrować się na grach? To część, którą wiemy, jak używać.

-- Ale ci ludzie mają kłopoty w~prawdziwym świecie -- powiedział Ashok, zaciskając dłonie w~pięści.

A Yasmin stwierdziła, że chichocze, a potem się śmieje, naprawdę się śmieje. To było takie oczywiste!

-- Och, Ashok -- powiedziała -- och tak, na pewno mają.

I wiedziała, co z~tym zrobić.

\bigskip
\threeast

Ta scena jest poświęcona Waterstone, krajowej sieci księgarni w~Wielkiej Brytanii. Waterstone's to sieć sklepów, ale każdy z~nich sprawia wrażenie wspaniałego niezależnego sklepu, z~mnóstwem osobowości, świetnymi zasobami (zwłaszcza audiobookami!) i~kompetentnym personelem. Na szczególną uwagę zasługuje sklep Manchester Deansgate, który ma \textit{znakomitą }sekcję s-f.

\href{https://www.waterstones.com/}{Waterstone}

Lu nie wiedział, gdzie iść. Akademiki Boss Winga oczywiście nie wchodziły w~rachubę. I chociaż znał tuzin kafejek internetowych w~Shenzhen, gdzie mógł usiąść i~zalogować się do gry, tak naprawdę nie chciał teraz grać. Nie, gdy wszyscy są w~więzieniu.

Jednak musiał usiąść. Został mocno uderzony w~głowę i~w ramię i~miał duże zawroty głowy. Już raz zwymiotował, trzymając się słupka przystankowego i~pochylając się nad rynsztokiem, zyskując cmoknięcie dezaprobaty od starej kobiety, która przechodziła obok, ciągnąc ogromny wózek pełen elektronicznych odpadów.

Myślał o napisaniu SMS-a do Matthew i~pozostałych, aby dowiedzieć się, czy policja ich zatrzymała, ale bał się, że policja namierzy go, jeśli to zrobi, korzystając z~sieci telefonicznej, aby go zlokalizować i~odebrać.

To wszystko było takie \textit{cudowne}. Wstali od swoich komputerów, skandując ze złością pieśni wojenne z~gier, w~które Boss Wing i~jego zbiry nigdy nie grali, więc wszystko to było dla nich całkowicie kłopotliwe. Ich twarze zmieniły się od zdziwienia do złości do strachu, gdy wszyscy chłopcy w~pokoju stanęli razem i~wyszli z~kawiarni, blokując drzwi, aby nikt nie mógł wejść.

Były też dziewczęta, stare babcie i~młodzi mężczyźni, którzy zatrzymali się, by je podziwiać, gdy stali ramię w~ramię, dzielnie intonując do tchórzliwych zbirów z~fabryki Bossa Winga, zbirów, którzy byli tak twardzi zaledwie kilka minut wcześniej, chętni, by uderzyć cię w~głowę, jeśli za dużo mówiłeś, gotowi do odebrania swojej pensji. Odkąd próbowali wyjść na własną rękę, życie stawało się coraz gorsze. Boss Wing prowadził ogromną operację, z~mnóstwem postaci w~grze, by strzec się bogatych graczy, którzy polowali na farmerów złota dla sportu, ale był okrutny i~tani, a ty miałeś szczęście, jeśli w~końcu zobaczyłeś połowę zarobków, po odjęciu kar za ,,łamanie zasad'' zostały naliczone na pensję.

Ich telefony dzwoniły i~brzęczały zdjęciami z~innych fabryk Boss Wing, w~których robotnicy też wychodzili, a w~Mushroom Kingdom wybuchły wojny, ponieważ Webblies nie pozwalali innym pracować w~ich strefie. Przyjechała policja i~pozostali odważni, Matthew, Ping i~wszyscy jego przyjaciele. Byli robotnikami, byli wojownikami, byli armią i~ich sprawa była słuszna. Nie dadzą się zastraszyć.

A potem przyszedł gaz. A potem zaczęły spadać pałki. I wtedy zaczęły się krzyki. A potem Lu pobiegł, przebiegł przez kłujące chmury gazu i~chaos bitwy -- tak podobny i~tak niepodobny do milionów bitew, które stoczył w~grach -- i~zwymiotował, a teraz \ldots 

Teraz nie miał pojęcia, dokąd się udać.

A potem zadzwonił jego telefon. Numer był pusty, co sprawiło, że pulsowało mu w~gardle. Czy tajna policja wymazywała numer, kiedy do ciebie dzwonili? Ale gdyby tajna policja wiedziała, że istnieje i~ma jego numer telefonu, mogłaby po prostu odebrać go tam, gdzie stał, używając przeklętej funkcji śledzenia telefonu.

To nie była policja. Z niepokojem przesunął palcem po przycisku rozmowy na ekranie.

-- Wei? -- powiedział ostrożnie.

-- Lu? Czy to ty? -- Rozmowa miała dziwny, pogłos taniej usługi połączeń internetowych, cyfrowy szum pakietów, które podróżowały trzecią klasą w~globalnej sieci. Akcent też był trudny, gruby i~dziwny. Znał dźwięk i~znał głos.

-- Wei-Dong?

-- Tak!

-- Wei-Dong w~\textit{Ameryce}? 

Nie miał wiadomości od dziwnego gweilo, odkąd poszli do Boss Wing i~Ping musiał go wyrzucić z~gildii. Boss Wing nie pozwalał im na najazdy z~ludźmi z~zewnątrz, a nawet rozmawianie z~nimi w~grze. Na wszystkich swoich komputerach miał oprogramowanie szpiegujące, które informowało go, kiedy złamałeś te zasady, i~traciłeś dzień za pierwsze wykroczenie, tydzień za drugie.

-- Lu, to ja! Słuchaj, czy właśnie oglądałem ciebie i~Pinga bitych przez gliny?

-- Nie wiem, oglądałeś? -- Dezorientacja spowodowana raną głowy była gwałtowna i~zastanawiał się, czy rzeczywiście prowadzi tę rozmowę. To było bardzo dziwne.

-- Ja \ldots  właśnie widziałem, jak pobito cię na filmie z~Shenzhen. Myślę, że tak. Czy to ty?

-- Właśnie zostaliśmy pobici -- powiedział. -- Jestem ranny.

-- Jesteś ciężko ranny? Nie mogłem dodzwonić się do Pinga, więc spróbowałem do ciebie. -- Był podekscytowany, jego głos był napięty. -- Co się stało? 

Lu wciąż zmagał się z~myślą, że gweilo właśnie zadzwonił do niego z~odległości tysięcy kilometrów. 

-- Widziałeś mnie w~Internecie w~Ameryce? 

-- Każdy gracz na świecie cię widział, Lu! Nie mogłeś lepiej zaplanować tego czasu! Po obiedzie jest najbardziej zajęty czas na serwerach, a słowo krążyło jak nic, co kiedykolwiek widziałem. Wszyscy w~każdej grze rozmawiali o tym, podając linki do strumieni wideo i~zdjęć. Było to nawet w~prawdziwych wiadomościach! Mój sąsiad walił w~moją ścianę i~zapytał, czy coś o tym wiem. To było niesamowite!

-- Widziałeś, jak bito mnie w~Internecie?

-- Stary, \textit{wszyscy }widzieli, jak zostałeś pobity w~Internecie.

Lu nie wiedział, co powiedzieć. 

-- Czy dobrze wyglądałem? 

Wei-Dong śmiał się jak hiena. 

-- Wyglądałeś \textit{świetnie}! 

Zerwała się tama, Lu śmiał się, śmiał i~śmiał, gdy całe napięcie wypływało z~niego. W końcu przestał, wiedząc, że jeśli tego nie zrobi, znowu zwymiotuje. Znajdował się teraz przy stacji kolejowej, w~dużym ruchu pieszym, wokół niego celowo poruszali się wszyscy ludzie, gdy stał nieruchomo, jak zamroczona wyspa w~rwącym strumieniu. Cofnął się na klatkę schodową przed salonem kosmetycznym i~osunął się na pośladki, przykucnięty i~trzymając telefon przy głowie.

-- Wei-Dong?

-- Tak. 

-- Po co do mnie dzwonisz? 

Na linii zapadła nieprzyjemna cisza, przerywana miękkim cyfrowym pogłosem.

 -- Chciałem ci pomóc -- powiedział w~końcu. -- Pomóc Webbliesom.

-- Wiesz o Webblies? -- Lu wierzył, że Matthew ich wymyślił, wymyśloną armię tysięcy wyimaginowanych przyjaciół, którzy będą o nich walczyć.

-- Wiesz o nich? Lu, to gildia, która kopie najwięcej dup na świecie! Nikt nie może ich pokonać! Coca-Cola Games wysyła nam dziennie trzy notatki na ich temat!

-- Dlaczego Coca-Cola wysyła ci notatki? 

-- Och. -- Więcej ciszy. -- Nie mówiłem? Teraz dla nich pracuję. Jestem Turkiem.

-- Och -- odpowiedział Lu. 

Wiedział o Turkach, ale nigdy tak naprawdę nie myślał o tym, jacy ludzie będą pracować w~dziesięciosekundowych odstępach, tworząc dialog dla postaci niezależnych lub zastanawiając się, co się stanie, gdy strzelisz garłaczem w~krzesło biurowe. 

-- To musi być interesujące.

Wei-Dong wydał z~siebie mokry dźwięk. 

-- To nędza -- powiedział. -- Prowadzę jednocześnie cztery różne sesje i~ledwo zarabiam na opłacenie czynszu. A oni zarabiają na nas tyle pieniędzy! W zeszłym miesiącu ogłosili kwartalne zyski, a gry z~Turkami zarabiają 30 procent więcej niż te bez. Zatrudniają więcej Turków tak szybko, jak tylko mogą \ldots  tutaj jest bałagan. Ale nasze zarobki nie rosną. Więc pomyślałem o Webbliesach, wiesz \ldots  -- Urwał. -- Może wy możecie nam pomóc, jeśli pomożemy wam? Wszyscy gramy na nasze pieniądze, prawda? Więc dlaczego nie mielibyśmy być po tej samej stronie.

-- Brzmi dobrze -- powiedział Lu. Wciąż próbował pojąć fakt, że Webblies byli najwyraźniej sławni wśród amerykańskich nastolatków. -- Zaczekaj -- powiedział, odtwarzając akcentowaną, niegramatyczną mowę Wei-Donga. -- Płacisz czynsz?

-- Tak -- powiedział Wei-Dong. -- Tak! Teraz mieszkam sam. To jest super! Mam kiepski pokój w~hotelu, nie wiem, jak go nazywasz. Ale dla ludzi, którzy nie mają pieniędzy. Ale mogę tu mieć połączenie bezprzewodowe i~mam cztery maszyny i~jest mnóstwo rzeczy, do których mogę chodzić, przynajmniej w~porównaniu z~domem \ldots  -- Zaczął bełkotać o swoich ulubionych restauracjach i~klubach, które miały wieczory dla wszystkich grup wiekowych i~milion drobnych nieistotnych szczegółów na temat Los Angeles, które równie dobrze mogło być Mushroom Kingdom, na tyle, o ile to znaczyło dla Lu. Pozwolił mu się wylać i~próbował wymyślić miejsca, do których mógłby się udać, aby się zregenerować. Przelotnie zapragnął swojej matki, która zawsze znała jakiś tradycyjny chiński lek na jego dolegliwości. Często nie działały, ale czasami działały, a delikatne nakładanie ich przez matkę też było magicznie skuteczne.

Nagle poczuł mdłości, przytłaczająco zatęsknił za domem. 

-- Wei-Dong -- powiedział, przerywając wirtualną wycieczkę po Los Angeles. -- Muszę teraz pomyśleć. Nie wiem, co robić. Jestem ranny, jestem na ulicy i~nie mogę do nikogo zadzwonić, na wypadek, gdyby policja namierzyła telefon. Co mam zrobić? 

-- Och. Cóż. Nie wiem dokładnie. Miałem nadzieję, że będziesz wiedział, co \textit{ja }powinienem zrobić, prawdę mówiąc. Chcę się zaangażować!

-- Myślę, że chcę się \textit{odangażować}. -- Tęsknota za domem Lu zamieniała się w~gniew. Kim był ten \textit{chłopiec}, który zadzwonił do niego z~drugiego końca świata, żądając ,,zaangażowania''? Czy nie miał dość własnych problemów? -- Co możesz dla mnie zrobić stamtąd? Co to jest \ldots  ile te \textit{śmieci }są warte? Jak wszyscy w~więzieniu poprawią moje życie? Jak bicie mnie po głowie sprawi, żeby było lepiej? Jak? 

-- Nie wiem. -- Głos Wei-Dong był cichy i~zraniony. Lu starał się kontrolować swój gniew. Ten gweilo chciał pomóc. To nie jego wina, że nie umiał pomóc. Lu też nie wiedział, jak pomóc.

-- Ja też nie wiem -- powiedział Lu. -- Dlaczego nie pomyślisz, jak mi pomóc i~oddzwonisz. Muszę znaleźć miejsce na odpoczynek, może pielęgniarkę lub lekarza. OK?

-- Jasne -- powiedział gweilo. -- Jasne. Oczywiście. Niedługo oddzwonię, nie martw się.

Za każdym razem, gdy pociąg z~Hongkongu wjeżdżał na dworzec kolejowy w~Shenzhen, wyrzucał ogromny tłum ludzi: ludzi z~Hongkongu w~ostrym stylu biznesowym, bogate dzieciaki, obcokrajowcy i~robotnicy z~Shenzhen powracający z~zagranicznych kontraktów, ściskający plecaki. Gęsta grupa została rozbita na postój taksówek i~centrum handlowe i~pojawiła się jako rozproszona chmura na ulicy, na której rozmawiał Lu. Teraz przedzierał się przez ten tłum, słuchając fragmentów setek rozmów o biznesie, produkcji i~farmowaniu złota.

To było na ustach wszystkich, mówili o strajku, o akcji policji, o rolnikach. Oczywiście większość ludzi w~Chinach słyszała o uprawie złota i~o wszystkich historiach o pieniądzach, które można zarobić, grając w~gry wideo, ale nigdy nie słyszałeś, żeby taki biznesmen mówił o tym. Nie bystrzy, wytworni ludzie z~oczywistym bogactwem i~władzą, którzy przeskakiwali tam i~z powrotem między Hongkongiem a Shenzhen, szybko przemawiając do swoich słuchawek, mówiąc innym, co mają robić.

Co powiedział gweilo? \textit{Wszyscy widzieli, jak bito Cię w~Internecie! }Czy ci ludzie przyglądali mu się uważnie? Teraz wydawało się, że tak. Oczywiście był zakrwawiony, gapił się, miał czerwone oczy. Dlaczego nie mieliby się na niego gapić? Ale może  \ldots 

-- Jesteś jednym z~nich, prawda? 

Miała 22 lub 23 lata, z~doskonałymi paznokciami na dłoni, którą opierała na jego ramieniu, podchodząc do niego od tyłu. Mimowolnie pisnął i~podskoczył, a ona trochę zachichotała. 

-- Musisz być -- powiedziała. Podniosła telefon. -- Obejrzałem film pięć razy w~pociągu. Powinieneś zobaczyć komentarze. Takie brzydkie!

Wiedział o tym. Za każdym razem, gdy coś, co sprawiało, że rząd wyglądał źle, trafiało do Internetu, armia komentatorów tweetowała, publikowała i~komentowała, jak rząd miał rację, jak cała historia była niewłaściwa, jak ludzie w~historii byli winny wszelkiego rodzaju strasznych rzeczy. Lu wiedział, że nie powinien w~to wierzyć, ale nie dało się przeczytać tego wszystkiego bez odrobiny wątpliwości, potem trochę więcej, a potem, jak kostka lodu na siniaku, oburzenie, które poczuł na początku, by zdrętwiało.

Myśl, że on sam jest w~centrum jednej z~tych burz, sprawiła, że poczuł się, jakby miał znowu zwymiotować. Dziewczyna musiała to zobaczyć, bo lekko ścisnęła jego ramię. 

-- Och, nie wyglądaj tak poważnie. Świetnie wyglądałeś na filmie. Jestem pewien, że nikt nie wierzy w~te wszystkie bzdury! -- Zacisnęła usta. -- Cóż, oczywiście, to nieprawda. Jestem pewna, że wielu ludzi w~to wierzy. Ale są głupcami. I jestem pewna, że wielu innych zostało zainspirowanych. Jestem Jie.

-- Lu -- powiedział Lu, po próbie znalezienia pseudonimu. Nie został stworzony na zbiega. -- Miło było cię poznać -- powiedział, strząsnął jej rękę i~ruszył w~głąb tłumu.

Znowu złapała go za ramię. 

-- Och, proszę, przestań. Musimy porozmawiać. Proszę? 

Zatrzymał się. Nie miał dużego doświadczenia z~dziewczynami, ale coś w~jej głosie sprawiło, że chciał zostać. 

-- Dlaczego musimy rozmawiać? 

-- Chcę poznać twoją historię -- powiedziała. -Do mojego programu. 

-- Twojego \textit{program}?

Pochyliła się blisko, tak blisko, że mógł poczuć jej perfumy, i~wyszeptała: 

-- Jestem Jiandi.

Spojrzał na nią tępo.

Potrząsnęła głową.

-- Jiandi -- syknęła. -- Jiandi! Z Factory Girl Show!

Wzruszył ramionami. 

-- Jakiego rodzaju show? 

-- Każdej nocy! -- powiedziała. -- O 21:00! Słucha mnie dwanaście milionów pracowniczek fabryk! Dzwonią do mnie ze swoimi problemami. Wychodzimy przez sieć, audio, przez \ldots  hm -- obniżyła głos -- proxy Falun Gong.

-- Och -- powiedział i~zaczął się oddalać.

-- To nie jest religijne -- powiedziała. -- Po prostu pomagam im w~ich problemach. -- \textit{Proxy} -- dodała szeptem -- to po prostu sposób, w~jaki wprowadzamy program do fabryk. Próbują mnie zablokować, ponieważ mówimy prawdę o warunkach pracy  \ldots  dziewczyny, które są pod presją seksualną swoich szefów, oszustwa marketingowe, zdzierstwo płacowe, uzależnienie \ldots 

-- Dobrze -- powiedział. -- Rozumiem. Dziękuję, ale nie.

-- \textit{Chodź }-- powiedziała i~spojrzała mu głęboko w~oczy. Jej były ciemne i~pokryte cienkim, precyzyjnym zielonym ołówkiem do oczu, a brwi ukształtowane w~zdziwione, wyrafinowane łuki. -- Wyglądasz, jakbyś potrzebował miejsca do umycia i~może posiłku. Mogę ci to załatwić.

-- Możesz? 

-- Lu, jestem \textit{sławna}! Mam reklamodawców, którzy \textit{dużo }płacą za sponsorowanie mojego programu. Mam miliony zwolenników w~całym Shenzhen, nawet w~Kantonie i~Dongguan. Nawet w~Szanghaju i~Pekinie! Jestem dla nich bohaterką, Lu. Mogę w~ten sposób przedstawić twoją historię każdemu pracownikowi w~delcie Rzeki Perłowej, ot \textit{tak}! -- Pstryknęła palcami przed jego nosem, sprawiając, że mrugnął i~się cofnął. Roześmiała się. -- Jesteś słodki -- powiedziała. -- Chodź, będzie cudownie. 

-- Gdzie idziemy? -- powiedział ostrożnie.

-- Och, mam miejsce -- powiedziała.

Chwyciła go za rękę, jej palce były suche i~chłodne, dotykały zimnymi miejscami, w~których pierścienie, które nosiła, stykały się z~jego skórą. Poprowadziła go przez tłum, który wydawał się magicznie przed nią rozstąpić. Wszystko stało się teraz jak sen, z~bólem spychającym wizję Lu do zamglonego tunelu. Zastanawiał się, czy miałaby coś na ból. Zastanawiał się, czy zna jakąś tradycyjną medycynę, czy nie zmieszałaby mu gorzkiej herbaty ze skomplikowanymi zapachami i~unoszącymi się w~niej małymi kawałkami twardych substancji. Zastanawiał się nad tym wszystkim, a ulice i~chodniki prześlizgiwały się pod ich stopami jak magia. Możesz automatycznie śledzić swoich towarzyszy w~gildiach w~grze, po prostu kliknij na nie i~wybierz ,,podążaj'', i~cała gildia mogła to zrobić, gdy odległość do pokonania była duża, tak że tylko jeden gracz musiał zwracać uwagę na długi marsz przez świat, podczas gdy inni się relaksowali, palili, jedli, korzystali z~toalety, gdy ich avatary szły jak stado zwierząt za przewodnikiem.

Tak to się czuło, jakby był postacią, której gracz wyszedł po papierosa i~siku, a postać bezmyślnie biegła za liderem.

-- Czy tutaj mieszkasz? -- powiedział, kiedy dotarli do holu wysokiego budynku mieszkalnego. 

Był to ,,budynek uścisku dłoni'', tak blisko sąsiadującego z~nim budynku, że najemcy mogli wychylać się przez okna i~uścisnąć dłoń sąsiadom po drugiej stronie ulicy. W holu śmierdziało gotowaniem i~potem, ale było czysto, a przy drzwiach działał domofon i~zamek.

-- Nie -- odpowiedziała. -- Niektóre z~moich show robię stąd. Są dwa lub trzy z~nich, żeby zmylić jingcha. 

Pomyślał, że to zabawne słyszeć, jak używa określenia klanu graczy dla policji. 

Zobaczyła to i~powiedziała: 

-- O tak, zengfu myślą, że jestem bardzo bantai i~zrobiliby mi PK, gdyby mogli. 

Roześmiał się na to, ponieważ był to prawie nieprzenikniony slang, rząd myśli, że jestem zboczeńcem, więc chcą ,,zabić gracza'' -- zniszczyć -- mnie, jeśli mogą. Co innego usłyszeć chłopaka z~podwiniętą koszulą i~papierosem zwisającym mu z~twarzy mówiącego to, a co innego usłyszeć tę delikatną, pięknie umalowaną dziewczynę.

Winda była zepsuta, więc poprowadziła go po pięciu kondygnacjach schodów, ściany ozdobiono bogatym graffiti: malowidłami z~przekleństwami, scenami z~życia fabryki, numerami telefonów, pod które można było zadzwonić, aby kupić fałszywe dokumenty tożsamości, stopnie, certyfikaty. Własny pokój Lu w~akademiku znajdował się w~budynku, który wynajął Boss Wing, i~każdego dnia pokonywał dwa razy tyle schodów, ale wydawało się, że ta wspinaczka go zabije. Na piętrze Jie przy drzwiach klatki schodowej w~korytarzu kucała starsza pani. Skinęła im głową.

-- Pani Yun -- powiedziała Jie, -- Chciałabym, aby poznała Pani Hui. Jest mechanikiem, który przyszedł naprawić mój klimatyzator. -- Staruszka skinęła głową i~odwróciła wzrok.

Jie zaatakowała jedne z~drzwi mieszkania kółkiem na klucze, otwierając cztery różne zamki dużymi, skomplikowanymi, grubymi kluczami, a następnie wciskając ramieniem drzwi, które odskoczyły ciężko do tyłu, uderzając o odbojnik z~metalicznym dźwiękiem. Wskazała mu wejście do środka i~zamknęła drzwi, wystrzeliwując cztery rygle od wewnątrz i~uderzając w~kilka włączników światła.

Mieszkanie składało się z~dwóch dużych pokoi, salonu, w~którym stali, i~połączonej sypialni, którą widział z~drzwi. Obok nich pod ścianą znajdował się mały aneks kuchenny, a resztę pokoju zajmowała sofa i~duże biurko z~krzesłami po obu jego stronach, pokryte śmietnikiem sprzętu do nagrywania: mikser, kilka dużych zestawów słuchawek i~kilka chudych mikrofonów na stojakach. Każdy centymetr ściany był \textit{pokryty }papierem: wycinki z~gazet, listy, rysunki -- wszystko obficie posypane naklejkami, serduszkami, uroczymi rysunkami zwierząt.

Jie machnęła na to ręką. 

-- Moje studio! -- powiedziała i~okręciła się dookoła. -- Cała moja poczta od fanów i~prasa. 

 Przejechała lekko palcami po ścianie. Przyglądając mu się uważniej, Lu zobaczył, że każdy list zaczynał się od ,,Droga Jiandi'' i~że wszystkie zostały napisane schludnymi, dziewczęcymi rękami. 
 
 -- Mam skrzynkę pocztową w~Makau. Moi przyjaciele wysyłają tam listy, skanują je i~wysyłają do mnie pocztą elektroniczną. Tuż pod nosem zengfu! 

-- A starsza pani w~holu?

Opadła na kanapę, spódnica podwinęła się wokół jej ud i~skopała buty fachowym łuku na matę przy drzwiach. 

-- Odpowiedź naszego budynku na jednostkę detektywów babci z~obwiązanymi stopami -- powiedziała, a on znów się roześmiał, słysząc slang. 

W Nanjing używali tego określenia, by mówić o małych staruszkach, które zawsze węszyły, plotkując o tym, kto robi coś złego lub niegodziwego. Tak naprawdę nie miały skrępowanych stóp, praktyka obwiązywania stóp małych dziewczynek do tego stopnia, że dorastały niezdolne do prawidłowego chodzenia, była martwa, a on nigdy nie widział naprawdę związanej stopy poza muzeum, chociaż babcie zawsze wykrzykiwał nad stopami dziewczynek, rzucając złe uwagi, jeśli dziewczyna miała duże stopy, gruchając, jeśli miała małe, ale i~tak wszystkie zachowywały się jak uszczypnięte.

-- A ona uwierzy, że jestem mechanikiem? Nie mam żadnych narzędzi!

-- O nie -- Jie znów się zaśmiała. To był ładny dźwięk. Lu widział, że byłaby bardzo popularną prezenterką netshow. Ten śmiech był zaraźliwy. -- Nie, pomyśli, że uprawiamy seks!

Poczuł, że robi się czerwony i~się jąka. 

-- Och \ldots  Uch \ldots 

Teraz wyła ze śmiechu z~odrzuconą do tyłu głową i~włosami rozrzuconymi na poduszce sofy. 

-- Powinieneś zobaczyć swoją twarz! Słuchaj, dopóki babcia Mao myśli, że jestem tylko nieznaczącą dziwką, to nie będzie podejrzewała, że naprawdę jestem Jiandi, Plagą Biura Politycznego i~Głosem Delty Rzeki Perłowej, prawda? A teraz zdejmij buty i~przyjrzyjmy się tej ranie głowy.

Zrobił, jak mu kazano, starannie układając buty przy drzwiach i~ostrożnie wchodząc na zakurzoną drewnianą podłogę. Jie wstała i~poprowadziła go za ramiona do jednego z~krzeseł na kółkach przy biurku i~popchnęła go na nie, po czym pochyliła się nad nim i~uważnie wpatrywała się w~jego skórę głowy. 

-- Dobrze -- powiedziała. -- Po pierwsze, musisz zmienić szampon, masz bardzo przetłuszczające się włosy, to wstyd. Po drugie, wydaje mi się, że z~głowy wyrasta ci gołębie jajo, które musi trochę szczypać. Powiem ci tak, przyniosę ci coś zimnego do trzymania przez kilka chwil, a potem chcę, żebyś wziął prysznic i~dobrze to wyczyścił. Wygląda na to, że trochę krwawiło, ale nie bardzo, co jest dla ciebie szczęściem, ponieważ rany głowy zwykle krwawią jak szalone. A kiedy już wprowadzimy cię w~bardziej cywilizowany stan, ja wstawię Cię do Internetu i~sprawię, że będziesz jeszcze bardziej sławny. Brzmi ok? 

Otworzył usta, żeby zaprotestować, ale ona już odwracała się i~grzebała w~małej lodówce, kucając, włosy opadały jej na ramiona w~taki sposób, że Lu nie mógł przestać się gapić. Teraz trzymała torbę mrożonych pierożków z~kurczakiem Hahaomai -- rozpoznał opakowanie, to właśnie jedli na obiad przez większość wieczorów w~dormitorium Bossa Winga -- i~owijała ją ściereczką i~przyciskała do głowy. Miał wrażenie, że ważyły 500 kilogramów i~zostały schłodzone do zera absolutnego, ale również sprawiło, że jego głowa przestała pulsować niemal natychmiast. Osunął się na krzesło, zamknął oczy i~przytrzymał pierogi w~miejscu, w~którym zengfu -- slang był zaraźliwy -- klepnął go z~miłością. Śledził ruchy Jie wokół siebie na podstawie dźwięków, które wydawała, a także obłoków perfum i~włosów, ilekroć przechodziła blisko. To nie było złe, pomyślał, znacznie lepsze niż rzeczy godzinę temu, kiedy kucał przed stacją i~rozmawiał z~gweilo.

-- Dobrze -- powiedziała -- weź te. 

 Otworzył oczy i~zobaczył, że podaje mu dwie kredowobiałe pigułki i~szklankę wody.

-- Co to jest? -- powiedział, mrużąc oczy na blask zachodzącego słońca, wpadającego przez okno. Prawie spał.

-- Trucizna -- powiedziała. -- Postanowiłam wybawić cię z~twojego nieszczęścia. Weź je.

Wziął je.

-- Prysznic jest tam -- powiedziała, wskazując na sypialnię. -- Na desce klozetowej jest ręcznik i~znalazłam piżamę, która powinna na ciebie pasować. Opłuczemy twoje ubrania i~położymy je na grzejniku, żeby wyschły podczas rozmowy. Bez obrazy, Panie Bohaterze Pracy, ale pachniesz jak coś dawno martwego. 

Znowu się zarumienił, wiedział, że nie pozostało mu nic innego, jak tylko uchylić się i~przemknąć przez sypialnię, odniósł pogmatwane wrażenie wąskiego łóżka z~cienkim kocem zmiętym na dole, miotu wypchanych zwierząt i~sterty fałszywych torebek przepełnione ubraniami i~przyborami toaletowymi. Potem znalazł się w~łazience, półka zlewu pokryta tajemniczymi garnkami i~eliksirami, wszystkie drobiazgi dziewczyny, o których sugerowały miliony billboardów, ale których nigdy nie widział na swoim miejscu, z~przekrzywionymi pokrywkami i~wysypującym się proszkiem. Wszystko było o wiele mniej efektowne niż na billboardach, gdzie wszystko wyglądało, jakby było lekko mokre i~lśniące, ale było o wiele bardziej ekscytujące.

Każda pozioma przestrzeń pod prysznicem wydawała się podtrzymywać jakąś butelkę. Lu kupował duże, dwulitrowe butle żelu pod prysznic, których mógł też używać jako szamponu, ale po przyjrzeniu się etykietom znalazł jeden, który wydawał się przeznaczony do ciała, a drugi do włosów, i~wykorzystał oba. Woda wydawała się uderzać w~głowę jak małe ostre kamienie, a jego ramię zaczęło pulsować, gdy wcierał szampon. Po prysznicu starł parę z~lustra i~wykręcił się, żeby na nie spojrzeć, i~mógł po prostu dostrzec tam ogromny, wypukły siniak, fioletową, posiniaczoną linię w~kształcie maczugi otoczoną aureolą zielono-żółtej opuchlizny.

-- Coś, co możesz nosić, leży na łóżku -- krzyknęła Jie z~drugiej strony drzwi. Ostrożnie przekręcił gałkę i~stwierdził, że zaciągnęła zasłonę na drzwiach sypialni, zostawiając go samego w~nagim półmroku. Na łóżku, schludnie złożone, były spodnie od dresu i~koszulka z~biura pośrednictwa pracy, rodzaj rzeczy, które rozdawali ludziom, którzy stali przed nimi przez cały dzień, płacili za każdą osobę, którą przyprowadzili, by ubiegała się o pracę. Był ciasno dopasowany, ale włożył go, zwinął w~kulkę ubranie, które naprawdę śmierdziało, i~wyjrzał przez zasłonę.

-- Halo? 

-- Chodź tutaj, piękny! -- powiedziała, gdy wyszedł, bosymi stopami na zakurzonej płytce. Pochyliła się i~powąchała go delikatnym nosem. 

-- Mmm, wybrałeś szampon Dang-gui. Bardzo dobry. Bardzo dobry na problemy reprodukcyjne kobiet. -- Poklepała go po brzuchu. -- Będziesz miał tam małe dziecko w~mgnieniu oka! 

Teraz czuł, jakby zemdlał ze wstydu, dosłownie, pokój wirował wokół niego.

Musiała zobaczyć to w~jego twarzy, bo przestała się śmiać i~ścisnęła jego rękę. 

-- Nie martw się -- powiedziała. -- To tylko dokuczanie. Dang-gui jest dobry do wszystkiego. Twoja matka musiała ci go dawać. 

I tak, zdał sobie teraz sprawę, stąd znał ten zapach, przypomniał sobie, że chciałby, aby jego matka była tam i~dała mu trochę ziół, i~to pragnienie musiało prowadzić jego rękę wśród wielu butelek pod prysznicem.

-- Czy tutaj mieszkasz? -- powiedział.

-- W tej jamie? -- Zrobiła minę. -- Nie, nie! To tylko jedno z~moich studiów. Posiadanie wielu miejsc, w~których mogę pracować, pomaga. Utrudnia życie zengfu.

-- Ale ubrania, łóżko?

-- Tylko kilka rzeczy, które zostawiam na wieczory, kiedy pracuję do późna. Mój program może czasem trwać całą noc, w~zależności od tego, ilu mam dzwoniących. -- Znów się uśmiechnęła. Miała dołeczki. Nigdy wcześniej nie zauważył dołków u dziewczyny. Uraz głowy przyprawiał go o zawroty głowy. A może to była miłość.

-- A teraz? 

-- A teraz porozmawiamy z~tobą o tym, co widziałeś -- powiedziała. -- Mój program zaczyna się za \ldots  -- spojrzała na ekran swojego telefonu -- 12 minut. Wystarczająco dużo czasu na wodę i~ułożenie się wygodnie. 

Wyłowiła dzbanek z~filtrem ze swojej lodówki i~napełniła szklankę z~małego stojaka obok maleńkiego zlewu. Wziął ją i~wypił łapczywie, a ona przyniosła mu filtr, odłożył go po jednej stronie biurka, po czym usiadła na krześle po drugiej stronie.

Zaczęła klikać, pisać i~marszczyć brwi w~uroczy sposób, wkładając ogromne słuchawki i~ustawiając mikrofon. Pomachała do niego, a on usiadł na krześle naprzeciwko, napełniając szklankę.

-- Jeszcze raz, co to za show? 

-- Jesteś takim \textit{chłopcem}! -- powiedziała, podnosząc wzrok znad ekranu, palcami wciąż uderzając w~klawiaturę owadzimi kliknięciami wypielęgnowanych paznokci.

Spojrzał na siebie. 

-- Przypuszczam, że tak -- powiedział.

-- Chodzi mi o to, że gdybyś był dziewczyną, wiedziałbyś o tym wszystko. Każda dziewczyna z~fabryki mnie słucha, uwierz w~to. Zaczynam audycje po kolacji, one logują się, dzwonią do mnie, czatują, i~mówią mi o wszystkich kłopotach, a ja mówię im, co muszą usłyszeć. Przeważnie sprowadza się to do tego: jeśli twój szef chce cię pieprzyć, znajdź inną pracę lub bądź przygotowana na pieprzenie na więcej niż jeden sposób. Jeśli twój chłopak jest śmieciem, który nie chce pracować i~pożycza od ciebie pieniądze, znajdź nowego chłopaka, nawet jeśli jest ,,miłością twojego życia''. Jeśli twoje dziewczyny gadają o tobie bzdury, skonfrontuj się z~nimi, wypłacz się i~zacznij od nowa. Jeśli twoja kumpela pieprzy się z~twoim chłopakiem, pozbądź się ich obu. Jeśli pieprzysz chłopaka swojej przyjaciółki, przestań, rzuć go, wyznaj jej i~nie rób tego ponownie. -- Odznaczała te rzeczy na palcach jak listę zakupów.

-- Brzmi to trochę powtarzalnie -- powiedział. 

Zastanawiał się, czy ona to zmyśliła, czy może ma urojenia. Czy naprawdę może istnieć przedstawienie, którego słuchała każda dziewczyna z~fabryki, o którym on nigdy nie słyszał? Pomyślał o tym, jak mało rozmawiały z~nim dziewczyny z~fabryki w~Shilong New Town, kiedy pracował jako ochroniarz, i~zdecydował, że tak, to jest całkowicie możliwe.

-- To bardzo powtarzalne, ale wszystkie lubimy to w~ten sposób, moje dziewczyny i~ja. Niektóre problemy są uniwersalne. Niektórych rzeczy po prostu nie można mówić zbyt często. Zresztą to nie wszystko. Mamy różnorodność! Mamy Ciebie!

-- Mnie -- powiedział. -- Zamierzasz pokazać mnie z~tymi wszystkimi dziewczynami? Czemu? Czy to nie sprawi, że policja będzie chciała mnie jeszcze więcej?

-- Kochanie, policja już cię chce. Pamiętaj o wideo. Twoja twarz jest wszędzie. Im bardziej jesteś sławny, tym trudniej będzie im cię aresztować. Zaufaj mi.

-- Skąd możesz być pewna? Czy kiedykolwiek to robiłaś?

-- Codziennie -- powiedziała z~szeroko otwartymi oczami. -- Jestem moim własnym studium przypadku. Policja ściga mnie od dwóch lat, a ja trzymam się z~dala od ich szponów. Robię to, będąc zbyt popularna, by mnie złapać!

-- Myślę, że nie rozumiem, jak to działa -- powiedział.

Spojrzała na ekran swojego telefonu. 

-- Mamy tylko minutę. Tutaj, szybko, wyjaśnię: jeśli jesteś zbiegiem, bycie biednym jest trudne. Nawet trudniejsze niż dla nie-uciekinierów. Ucieczka jest droga. Potrzebujesz wielu miejsc do życia. Mnóstwo różnych telefonów, które możesz porzucić. Musisz umieć płacić li \ldots  ,,łapówki'' \ldots  i~mieć możliwość szybkiego przemieszczania się. Bycie sławnym oznacza, że masz dostęp do pieniędzy i~przysług od wielu różnych ludzi. Moi słuchacze wspierają mnie poprzez bezpośrednie darowizny lub przez moich reklamodawców.

-- Masz reklamy? Kto kupiłby reklamę w~audycji radiowej uciekiniera?

Wzruszyła ramionami. 

-- Tajwańczycy -- powiedziała. 

Wyspa Tajwan uważała się za oddzieloną od Chin od 1949 roku, ale Chiny nigdy nie przestały do niej mieć roszczeń, bez większych sukcesów. 

-- Czasami Falun Gong. -- Zobaczyła wyraz szoku na jego twarzy. -- Nie martw się, nie \textit{jestem }religijna. Ale wezmę ich pieniądze. Nie obchodzi ich, czy będę się z~nich wyśmiewać w~programie, pod warunkiem, że wyświetlam też ich reklamy.

Potrząsnął głową. 

-- To wszystko jest zbyt dziwne -- powiedział.

Uniosła rękę, by uciszyć, i~opuściła mały mikrofon z~jednego z~nauszników słuchawek. 

-- Cześć dziewczyny! -- zawołała do mikrofonu, klikając myszką. -- To twoja najlepsza przyjaciółka, siostra Jiandi, przyjaciółka, na której zawsze możesz polegać, przyjaciółka, która nigdy cię nie zawiedzie, przyjaciółka, której możesz zwierzyć się ze wszystkich swoich sekretów  \ldots  pod warunkiem, że nie masz nic przeciwko temu, aby dowiedziało się o tym osiem milionów dziewczyn z~fabryki! -- Zachichotała na własny żart. -- Och, siostry, to będzie dobra noc, mogę powiedzieć! Mam dla was specjalną niespodziankę trochę później, ale najpierw porozmawiajmy! Dziś wieczorem korzystam z~czatu Amazon France, chat.amazon.fr, więc idźcie i~się teraz zarejestrujcie. Jestem pod jiandi88888. Pamiętajcie, aby użyć kilku najnowszych serwerów proxy FLG przed nawiązaniem połączenia  \ldots  i~wygląda na to, że usługi tłumaczeniowe w~Yahoo.ru i~123india.in są w~tej chwili odblokowane, co powinno ułatwić rejestrację. Cóż, na co czekasz? Zarejestruj się! 

Kliknęła coś i~usłyszał rozbrzmiewającą reklamę Falun Gong w~jego słuchawkach i~zsunął jedną słuchawkę z~boku głowy. Jie odsunęła mikrofon i~wskazała na niego palcem.

 -- Czujesz już magię?

-- To jest to? To jest twoje wielkie przedstawienie? 

-- O tak -- powiedziała. -- Prawdopodobnie będziemy musieli dziś wieczorem przełączać czaty trzy lub cztery razy, ponieważ zaktualizują zaporę ogniową. Fajnie! Zaczekaj, zobaczysz. -- W jego uchu reklama się kończyła i~wsunął drugą słuchawkę z~powrotem na miejsce.

-- Mów do mnie -- powiedziała Jie, jej głos był pełen ciepła. Chwilę zajęło mu uświadomienie sobie, że mówi do swojego mikrofonu, do publiczności, nie do niego. Jej palce pracowały na klawiaturze i~myszy.

-- Halo? 

-- Tak, kochanie, cześć. Jesteś na linii. Mów, mów! Mamy tylko całą noc!

-- Och, um \ldots  -- Głos był kobiecy, z~silnym akcentem Henan i~był przestraszony.

-- W porządku, kochanie, moje serce, w~porządku. Powiedz mi. 

Głos Jie był gruchaniem, mruczeniem, uwodzeniem. Jej oczy były wilgotne, usta zaciśnięte w~geście czystej troski. Lu chciał jej zdradzić \textit{swoje }sekrety.

-- Po prostu \ldots  -- Głos się urwał. Odgłosy płaczu. W tle odgłosy ruchliwej, fabrycznej sypialni, dziewczęce nawoływania oraz śmiech i~rozmowa. Jie wydawała kojące dźwięki szzzz szzzz. 

-- To mój szef -- powiedziała dziewczyna. -- Na początku był dla mnie taki \textit{miły}. Powiedział, że interesuje się mną, ponieważ oboje jesteśmy z~Henan. Powiedział, że będzie mnie chronił. Oprowadzi mnie po mieście. Poszliśmy do ładnych miejsc. Restauracja przy giełdzie. Zabrał mnie do parku Windows on the World i~przebraliśmy się za starożytnych wojowników.

-- I chciał czegoś w~zamian, prawda?

-- Wiedziałam, że tak. Słucham twojego programu. Ale myślałam, że dla mnie będzie inaczej. Myślałem, że on jest inny. Ale on \ldots  -- Urwała. -- Kiedy mnie pocałował, powiedział, że chce więcej. Wszystko. Powiedział, że jestem mu to winna. Że wiedziałam to, kiedy akceptowałam zaproszenie, i~że oszukiwałabym go, gdybym nie wiedziała \ldots  -- Zaczęła płakać.

Jie skrzywiła się, zakręciła palcem w~niecierpliwym geście. Lu był przerażony jej bezdusznością. Ale kiedy płacz ustał, jej głos znów był pełen współczucia i~zrozumienia.

-- Och, słodkie dziecko, źle ci się stało, prawda? Cóż, oczywiście wiedziałaś, że tak się stanie, ale serce i~głowa nie zawsze się ze sobą zgadzają, prawda? Pytanie nie brzmi, czy zachowywałaś się jak głupia  \ldots  ponieważ tak zrobiłaś, zachowywałaś się jak doskonały głupek  \ldots  pytanie brzmi, co możesz teraz z~tym zrobić. Mam rację?

-- Tak. -- Głos był tak cichy i~miękki, że ledwo go słyszał. Wyobraził sobie dziewczynę skurczoną do rozmiarów myszy, drżącą ze strachu.

-- Cóż, to proste. Niełatwe, ale proste. Zrezygnuj z~zarobków z~ostatnich ośmiu tygodni i~jutro rano wyjdź z~fabryki. Idź do pośrednika pracy na ulicy Xi Li i~znajdź coś  \ldots  cokolwiek  \ldots  co możesz zacząć od nowa. Potem dzwonisz do żony szefa  \ldots  czy jest żonaty?

-- Tak. -- Głos był teraz trochę mocniejszy.

-- Zadzwoń do jego żony i~powiedz jej wszystko. Powiedz jej, co zrobił, co powiedział, co odpowiedziałaś. Powiedz jej, że jest ci przykro i~powiedz jej, że jest ci przykro, że jej mąż jest takim workiem zgniłych, śmierdzących śmieci. Powiedz jej, że zrezygnowałaś z~wypłaty, którą wstrzymywał, i~że porzuciłaś pracę. A potem znowu zacznij pracować. I bez względu na to, co powie lub zrobi twój nowy szef, nie wychodź z~nim. Rozumiesz? 

-- Zadzwonić do jego żony \ldots  

-- Zadzwoń do jego żony, odejdź od swojej pensji i~zacznij od nowa. Nic innego nie zadziała. Nie możesz rozmawiać z~tym mężczyzną. Zgwałcił cię \ldots  to właśnie jest to, wiesz, kiedy ktoś z~władzą zmusza Cię do seksu, to gwałt, po prostu gwałt  \ldots  a on będzie to robił raz za razem. Zrobi to z~innymi dziewczynami w~fabryce. Powiedz dziewczynom jak najwięcej, dlaczego odchodzisz. Właściwie, powiedz mi, w~jakiej fabryce pracujesz i~jak się nazywa szef, a wtedy dowiedzą się o tym miliony dziewczyn. Będą trzymać się z~daleka od tego psa i~może uratujesz kilka dusz twoją odwagą. Co powiesz?

-- Chcesz, żebym wymieniła mojego szefa? Teraz? Ale myślałam, że to poufne \ldots 

-- Nie musisz. Ale czy chcesz, aby inna dziewczyna przeszła przez to, przez co ty właśnie przeszłaś? Jak myślisz, co by się stało, gdybyś usłyszała, jak inna dziewczyna wymawia jego imię w~tym programie, w~zeszłym miesiącu, zanim z~nim wyszłaś. Co zrobiłabyś? Czy uratujesz swoje siostry od bólu, który przeżywasz? A może ochronisz swoje posiniaczone ego i~pozwolisz cierpieć następnej dziewczynie i~następnej? 

Odczekała chwilę. Dziewczyna przez telefon nic nie powiedziała, choć odgłosy ludzi poruszających się po akademiku wciąż były słabo słyszalne. Lu wyobraził sobie ją pod kocem na pryczy, trzymającą słuchawkę telefonu, szepczącą swoje sekrety milionom dziewczyn. Co za dziwny świat. 

-- No i~jak? 

-- Zrobię to -- powiedziała dziewczyna. 

-- Co? Powiedz to głośno! 

-- Zrobię to! -- powiedziała dziewczyna i~zaśmiała się lekko, a śmiech powtórzyły dziewczęce głosy obok niej, gdy dziewczyny w~jej akademiku zdały sobie sprawę, że wyznanie, którego słuchały na swoich komputerach, telefonach i~radiach, pochodziło z~pryczy pośrodku. Rozległ się pisk sprzężenia zwrotnego, gdy jeden z~radiotelefonów zbliżyło się zbyt blisko telefonu, a palce Jie stukały w~klawiaturę, zgniatając sprzężenie zwrotne, ale jakoś zostawiając inne piski, dziewczęce piski. Dopingowały ją, dziewczyny w~akademiku, wiwatowały i~skandowały jej imię, jej prawdziwe imię, teraz w~radiu, ale to nie miało znaczenia, ponieważ dziewczyna śmiała się mocniej niż kiedykolwiek.

-- To Bau Peixiong -- powiedziała ze śmiechem. -- Bau Peixiong w~fabryce sportowej Huaxia. -- Roześmiała się, wyzwolony dźwięk.

-- OK, OK, dziewczęta -- powiedziała Jie do mikrofonu rozkazującym tonem. Głosy ucichły. -- Teraz wasza siostra właśnie poświęciła się za was wszystkie, więc musicie jej pomóc. Ona potrzebuje pieniędzy  \ldots  ten szef świnia nie da jej ośmiotygodniowego wynagrodzenia, które trzyma, zwłaszcza po tym, jak zadzwoni do żony. Potrzebuje pomocy w~pakowaniu, w~znalezieniu pracy. Ktoś tam myśli o zmianie pracy, ktoś tam wie, gdzie jest praca dla tej dziewczyny. Powiedzcie jej. Pomóżcie jej się wyprowadzić. Pomóżcie jej znaleźć nową pracę. To jest wasz obowiązek wobec waszej siostry. Obiecajcie mi!

Z telefonu głosy dziewczyn mówiących: -- Obiecuję! Obiecuję! 

-- Bardzo dobrze -- powiedziała Jie. -- Teraz bądźcie czujni przyjaciele, bo wkrótce zaprezentuję cudowną niespodziankę! 

Kliknięcie myszką, a potem kolejna reklama, tym razem firmy, która dostarczała fałszywe dane uwierzytelniające osobom poszukującym pracy, co gwarantowało przejście przez test w~bazach danych. Oboje zdjęli słuchawki i~Jie opróżniła szklankę wody, strużka spływała jej po brodzie i~gardle. Lu stłumił jęk. Była \textit{tak }piękna, a cała ta siła i~pewność siebie \ldots 

-- To był całkiem niezły początek, prawda? -- powiedziała, unosząc na niego brwi.

-- Czy tak jest przez cały czas?

-- Och, to było szczególnie dobre. Ale tak, przez większość nocy tak to wygląda. Trwa to sześć czy siedem godzin. Nadal myślisz, że stanie się powtarzalne?

-- Widzę, że to pozostanie interesujące.

-- W końcu przez całą noc zabijasz te same potwory, prawda? To musi być dość nudne.

Rozważał to. 

-- Niezupełnie -- powiedział. -- To chyba praca zespołowa. Wszyscy pracujemy razem i~nie za każdym razem jest tak samo  \ldots  gry bardzo różnią się od pojawiania się potworów. Czasami dostajesz też naprawdę dobre dropy  \ldots  to może być bardzo ekscytujące! Idziesz korytarzem, który oczyściłeś kilkanaście razy, i~odkrywasz, że tym razem jest wypełniony 200 wampirami, a potem jeden z~nich upuszcza epicki miecz i~wcale nie jest to nudne. -- Wzruszył ramionami. -- Mój kolega z~gildii, Matthew, mówi, że to sporadyczne wzmocnienie.

Uniosła palec i~powiedziała: 

-- Trzymaj się tego.

A potem kliknęła i~znów zaczęła mówić do swojego mikrofonu, odbierając telefon od innej dziewczyny z~fabryki, tym razem bardziej wściekłej niż smutnej. 

-- Miałam przyjaciółkę, która sprzedawała franszyzy dla linii ziołowych leków -- powiedziała, a Jie przewróciła oczami.

-- Mów dalej -- powiedziała. -- Brzmi jak świetna okazja. -- Sarkazm w~jej głosie był niewątpliwy.

-- Tak właśnie pomyślałam -- powiedziała dziewczyna. Brzmiała, jakby chciała coś uderzyć. -- Na początku myślałam, że chodzi o sprzedaż ziołowych remediów i~podobało mi się to, ponieważ moja mama zawsze dawała mi zioła, gdy byłam chora jako dziewczynka, i~pomyślałam, że wiele dziewczyn tutaj też będzie chciało kupić te remedia, bo tęskniły za domem.

-- Tak -- powiedziała Jie. -- Kto nie chciałby pamiętać swojej mamusi? 

-- Dokładnie! Dokładnie tak myślałam. A koleżanka powiedziała mi, ile pieniędzy mogę zarobić, ale nie na sprzedaży ziół! Powiedziała, że sprzedaż ziół będzie moją pracą dla ,,downlinerów'' i~że będę nimi zarządzać. Byłabym szefem! 

-- Kto nie chciałby być szefem?

-- Prawda! Powiedziała, że rekrutuje mnie do najwyższego szczebla organizacji, a następnie zwerbuję dwóch moich przyjaciół, aby byli moimi sprzedawcami. Oni płacili mi za prawo podpisania kolejnych downliners i~że wszyscy downlinerowie kupią ode mnie zioła i~wtedy dostanę udział w~ich wszystkich zyskach. Pokazała mi, jak gdyby moi dwaj downlinerzy zapisali się po jeszcze dwoje, a każda z~\textit{nich }zapisze się na jeszcze dwa i~tak dalej, że będę miał setki downlinerów pracujących dla mnie w~ciągu zaledwie kilku dni! A gdybym dostawała tylko kilka RMB z~każdego z~nich, zarabiałbym tysiące każdego miesiąca za samo zapisanie dwóch osób.

-- Bardzo hojna przyjaciółka -- powiedziała Jie i~chociaż brzmiała, jakby żartowała, nie uśmiechała się.

-- Tak, tak! Tak właśnie myślałam. I wszystko, co musiałem zrobić, to zapłacić jej niewielką opłatę za prawo do sprzedaży downline, a ona dostarczyła mi zioła, zestawy sprzedażowe i~wszystko, czego potrzebuję. Powiedziała, że zapisuje mnie, ponieważ byłam z~Fujian, tak jak ona, a ona chciała się mną zaopiekować. Powiedziała, że powinnam znaleźć dziewczyny, które wciąż były w~wiosce, dziewczyny, z~którymi chodziłam do szkoły, zadzwonić do nich i~zapisać je, ponieważ musiały zarabiać pieniądze.

-- Dlaczego dziewczęta w~wiosce miałyby potrzebować ziół? Czy nie miałyby swoich matek?

To powstrzymało gniewną, szybko gadającą dziewczynę. 

-- Nie pomyślałam o tym -- powiedziała w~końcu. -- Wyglądało na to, że będę bohaterką dla wszystkich i~że ucieknę z~fabryki i~się wzbogacę. Moja przyjaciółka powiedziała, że odejdzie za kilka tygodni i~dostanie własne mieszkanie. Myślałem o wyprowadzce z~akademika, mając pieniądze do wysłania do domu \ldots 

-- Marzyłaś o pieniądzach i~wszystkim, co można za nie kupić, ale nie poświęciłaś tyle uwagi zastanawianiu się, czy ta rzecz mogłaby zadziałać, prawda?

Kolejna cisza. 

-- Tak -- odpowiedziała. -- Muszę powiedzieć, że to prawda.

-- I wtedy? 

-- Zaczęło się dobrze. Sprzedałam kilka downline'ów, ale miały problemy z~wywiązywaniem się ze swoich zobowiązań. A potem moja przyjaciółka zaczęła prosić mnie o jej procent z~mojego dochodu. Kiedy powiedziałem jej, że nie otrzymuję dochodu, że moi downliners byli mi winni, ona się zmieniła.

-- Mów dalej. -- Oczy Jie były utkwione w~ścianie za głową Lu. Wyglądało na to, że była w~innym świecie, wyobrażając sobie dziewczynę i~jej problem.

-- Zdenerwowała się. Powiedziała, że zobowiązałem się wobec niej i~że na tej podstawie zobowiązała się wobec swoich upline'ów i~że będę musiała jej zapłacić, aby mogła zapłacić ludziom, którym była winna. Sprawiła, że poczułam, jakbym ją zdradziła, zdradziła niesamowitą szansę. Powiedziała, że jestem zwykłą dziewczyną ze wsi, nie nadaje się na bizneswoman. Dzwoniła do mnie cały dzień, w~kółko, krzycząc: ,,Gdzie są moje pieniądze ?''.

-- Więc co zrobiłaś? 

-- W końcu do niej poszłam. Płakałam. Powiedziałam jej, że nie wiem, co robić. A ona powiedziała mi, że wiem, ale że nie mam odwagi, by to zrobić. Powiedziała, żebym poszła do moich downlinerów, była dla nich twarda, wyciągnęła od nich pieniądze. A gdyby nie płacili, musiałabym zdobyć pieniądze w~inny sposób: od moich rodziców, znajomych, oszczędności. Mogłabym zdobyć nowe downlinerki w~przyszłym miesiącu.

-- A więc zadzwoniłaś do swoich downlinerów?

-- Tak. -- Wzięła głęboki oddech. -- Na początku byłam dla nich delikatna i~życzliwa, ale moja przyjaciółka dzwoniła do mnie raz za razem i~się wkurzyłam. Złościłam się na nie, nie na nią. To była ich wina, że musiałam stracić cały ten czas i~energię, że nie mogłam spać ani jeść. Więc stałam się bardziej wredna. Groziłam im, błagałam, krzyczałam na nie. Te dwie dziewczyny, to były moje stare przyjaciółki. Znałam je, odkąd byliśmy małymi dziećmi. Znałam ich tajemnice. Zagroziłam, że zadzwonię do ojca mojej przyjaciółki i~powiem mu, że pozwoliła chłopcu robić jej nagie zdjęcia, gdy miała 15 lat. Zagroziłam, że powiem siostrze mojej drugiej przyjaciółki, że pocałowała jej chłopaka.

-- Czy zapłaciły to, co były ci winne?

-- Na początku. W pierwszym miesiącu zapłaciły. Jednak w~następnym miesiącu musiałam do nich zadzwonić i~jeszcze raz na nie krzyczeć. To było tak, jakbym siedziała nad sobą, patrząc, jak szalona nieznajoma mówi te okropne rzeczy moim starym, najstarszym przyjaciółkom. Ale znowu zapłaciły. A potem, w~trzecim miesiącu \ldots  

Urwała nagle. Cisza narastała. Lu poczuł, że robi się coraz grubszy, nieruchomy.

-- Co się stało? 

-- Wtedy jedna przyjaciółka zjadła trutkę na szczury. -- Jej głos był cichym, odległym szeptem. Więcej ciszy. -- Powiedziałam jej, że pójdę do jej ojca i~\ldots  i~\ldots  -- Cisza. -- W ten sposób jej matka popełniła samobójstwo, kiedy oboje byłyśmy małe. Ten sam rodzaj trucizny. Jej ojciec był twardym mężczyzną, Starym Sto Imion, który przeżył Rewolucję Kulturalną. Nie ma w~nim litości. Kiedy nie mogła dostać pieniędzy, ukradła je. Została złapana. Miał się dowiedzieć. A jeśli nie, powiedziałabym mu o zdjęciach, które zrobiła. A ona nie mogła się z~tym pogodzić. Doprowadziłam ją do samobójstwa. To byłam ja. Zabiłam ją.

-- Zabiła się -- powiedziała Jie głosem pełnym współczucia. -- To choroba kobiet w~Chinach. Jesteśmy jedynym krajem na świecie, w~którym więcej kobiet niż mężczyzn popełnia samobójstwa. Nie możesz się za to obwiniać. -- Zatrzymała się. -- Nie za wszystko.

-- To nie wszystko -- powiedziała dziewczyna, cały gniew zniknął z~jej głosu, nie pozostało nic poza destylowaną rozpaczą.

-- Oczywiście, że nie -- powiedziała Jie. -- Nadal jesteś winna za ten miesiąc. I za miesiąc następny i~kolejny miesiąc.

-- Moja przyjaciółka, ta, która mnie w~to wciągnęła, ona wie \ldots  \textit{rzeczy} \ldots  o mnie. Takie rzeczy, które wiedziałam o moich przyjaciołach. Rzeczy, które mogą mnie kosztować pracę, dom, chłopaka \ldots  

-- Oczywiście. Tak działa cuanxiao. 

Lu słyszał już to określenie. Znaczyło ,,sprzedaż sieciowa''. Zawsze ktoś próbował ci coś sprzedać w~ramach cuanxiao. Śmiał się z~tego. Teraz wydawało się to znacznie poważniejsze. 

-- A gdzieś, upline stąd, jest ktoś inny w~cuanxiao, kto ma coś na nią. I są kaznodzieje, którzy mogą cię przekonać, że z~cuanxiao zarobisz fortunę i~że musisz tylko zainspirować swoją rodzinę i~przyjaciół.

-- Znasz go? Pan Lee. Mój przyjaciel zabrał mnie na spotkanie. Pan Lee wyglądał, jakby był w~ogniu, i~upewnił mnie, że stanę się bogata, jeśli tylko \ldots 

-- Nie znam pana Lee. Ale w~prowincji Guangdong są setki panów Lee. Wiesz, jak ich nazywamy? Faraonowie, jak egipscy królowie, których pochowali w~piramidach. To dlatego, że siedzą na piramidzie głupców jak ty. Pod faraonem jest para downlinerów, a pod nimi dwie pary, a pod nimi jeszcze dwie pary i~tak dalej, wszyscy przekazują pieniądze do jakiegoś feudalnego idioty ze wsi, który wie, jak mówić dobre teksty i~nigdy w~życiu nie przepracował ani dnia. Czy kiedykolwiek uczyłaś się matematyki?

-- Dostałam złoty medal na olimpiadzie matematycznej w~naszym kantonie! 

-- Bardzo dobrze! Matematyka jest przydatna na tym świecie. Policzmy coś. Jeśli każdy poziom piramidy ma podwójną liczbę członków z~poprzedniego poziomu, ilu członków jest na dziesiątym poziomie piramidy?

-- Co? Och. Um. dwa do dziesiątej. To  \ldots  

\textit{1024}, pomyślał Lu. 

-- 1024, prawda? 

-- Dokładnie. Ilu na 30 poziomie? 

-- Hmm \ldots 

Lu wyjął telefon, skorzystał z~kalkulatora, trochę policzył.

-- Um \ldots  

-- Och, tylko zgadnij.

-- To dużo. Sto tysięcy? Nie! Około pięciuset tysięcy.

-- Powinnaś oddać swój medal, siostro. To ponad miliard. -- Jie wstukała kilka cyfr na klawiaturze. -- Dokładnie 1 073 741 824. W Chinach jest 1,6 miliarda ludzi. Twoi sprzedawcy ziół mieli rekrutować nowych downlinerów co dwa tygodnie. W tym tempie \ldots  -- Wpisała trochę więcej. -- To byłby nieco ponad rok, zanim każda osoba w~Chinach pracowałaby w~twojej piramidzie, nawet malutkie dzieci i~najstarsze babcie.

-- Och.

-- Wiesz o sprzedaży sieciowej, musisz wiedzieć. W którym jesteś roku? -- To znaczy, ile lat minęło, odkąd opuściłaś wioskę?

-- Czwartym -- przyznała dziewczyna. -- Wiedziałam o tym. Oczywiście. Ale myślałem, że to coś innego. Myślałam, że to był prawdziwy produkt, ponieważ to tylko dwie osoby naraz \ldots 

-- Nie sądzę, że myślałaś o tym, siostro. Myślę, że myślałaś o posiadaniu dużego mieszkania i~mnóstwa pieniędzy. Czy to nie w~porządku?

-- Ale były pieniądze! Działało tygodniami! Moja przyjaciółka tak wiele zarobiła \ldots 

-- Na jakim poziomie piramidy była? 10? 20? Kiedy okradasz nowych ludzi, aby zapłacić starym, to jest to dobry interes dla starych ludzi. Nie tak dobry dla nowych ludzi. Ludzi jak ty lub twoje downlinerki.

-- Jestem głupia -- powiedziała dziewczyna. -- Jestem potworem! Zniszczyłam życie moich przyjaciółek! -- Teraz szlochała, wykrzykując spowiedź dla milionów ludzi.

-- To prawda -- powiedziała łagodnie Jie. -- Jesteś głupia i~potworem tak jak tysiące innych ludzi. A teraz co zamierzasz z~tym zrobić? 

-- Co mogę zrobić? 

-- Możesz przestać szlochać i~wziąć się w~garść. Twoja przyjaciółka, ta, która cię zwerbowała? Ktoś ma coś na nią, tak jak ona miała coś na ciebie. Usiądź z~nią i~zrób wszystko, aby ją wydostać. Najgorszą rzeczą w~tych piramidach jest to, że obracają przyjaciela przeciwko przyjacielowi, sprawiają, że zdradzamy ludzi, których kochamy, aby sami nie zostali zdradzeni. Nawet jeśli jesteś jedną z~nielicznych szczęśliwych na szczycie, którzy zarabiają na tym trochę pieniędzy, to płacisz cenę swoją uczciwością, przyjaźniami i~duszą. Jedynym sposobem na wygraną jest nie-granie.

-- Ale  \ldots 

-- Ale, ale, ale! Słuchaj, głupia dziewczyno! Zadzwoniłaś do mnie dziś wieczorem, ponieważ twoja dusza jest splamiona złem, które zrobiłaś. Myślałaś, że powiem ci tylko, że wszystko w~porządku, zrobiłaś to, co musiałaś, żadnej winy? Nie! Znasz mnie, jestem Jiandi. Nie udzielam rozgrzeszenia. Mówię ci, co musisz zrobić, aby zapłacić za swoje zbrodnie. Nie możesz się przyznać, poczuć się lepiej i~odejść. Musisz teraz wykonać ciężką pracę  \ldots  musisz naprawić wszystko, pomóc swoim przyjaciołom, przywrócić uczciwość i~sumienie. Słyszysz mnie? 

-- Słyszę cię. -- Cicho, potulnie. 

-- Powiedz to głośniej. -- Pstryknęła jak generał wydający rozkaz.

-- Słyszę cię! 

-- GŁOŚNIEJ! 

-- SŁYSZĘ CIĘ! 

-- Dobrze! -- Zaśmiała się i~potarła jedno ucho. -- Myślę, że słyszą cię w~Makau! Dobra dziewczyna. Idź i~zrób to teraz! 

I kliknęła coś i~w słuchawkach Lu pojawiła się kolejna reklama. Zdjął je i~stwierdził, że jego oczy są wilgotne od łez. 

-- Ta biedna dziewczyna -- powiedział.

-- Są tysiące takich jak ona, -- powiedziała Jie. -- To choroba, jak hazard. Pochodzi z~niezrozumienia liczb. Wszyscy wygrywają swoje małe medale matematyczne, ale nie wierzą w~liczby. A teraz miałeś mi powiedzieć o jakimś wzmocnieniu.

-- Okresowe wzmocnienie -- powiedział. -- Mój przyjaciel Matthew, przewodzi naszej gildii, opowiedział mi o tym. Pochodzi to z~eksperymentów na szczurach. Wyobraź sobie, że masz szczura, który dostaje jedzenie za każdym razem, gdy naciska dźwignię. Jak myślisz, jak często naciska dźwignię? 

-- Przypuszczam, że tak często, jak jest głodny. Kiedyś trzymałam myszy  \ldots  wiedziały, kiedy nadszedł czas na jedzenie i~biegły do rogu klatki, do którego wrzucałam nasiona i~ser.

-- Zgadza się. Dobrze, co z~poziomem, który daje jedzenie co piąty raz, gdy naciśnie się dźwignię.

-- Nie wiem \ldots  mniej?

-- Właściwie to tak samo. Po pewnym czasie szczury dochodzą do wniosku, że potrzebują pięciu naciśnięć na granulkę jedzenia i~za każdym razem, gdy chcą się nakarmić, podchodzą i~uderzają w~nią pięć razy. A co z~dźwignią, która podaje jedzenie losowo? Czasem jedna naciśnięcie, czasem sto uderzeń? 

-- Poddałyby się, prawda?

-- Źle! Naciskają jak szalone, dniami i~nocami. To jak ktoś, kto przez tydzień wygrywa trochę pieniędzy na loterii, a potem gra co tydzień, na zawsze. Niepewność doprowadza ich do szaleństwa, to najbardziej uzależniający system ze wszystkich. Matthew mówi, że to najważniejsza część projektowania gry  \ldots  pewnego dnia udaje ci się zabić naprawdę twardego NPCa szczęśliwym zamachem, a z~niego wypadnie jakiś niesamowicie epicki przedmiot i~zarobisz więcej pieniędzy w~dziesięć sekund niż zarobiłeś przez cały tydzień, i~musisz ciągle wracać do tego miejsca, szukać takiego potwora, myśląc, że to się powtórzy.

-- Ale to losowe, prawda?

-- Nie jestem pewien -- powiedział. -- Matthew mówi, że tak. Czasami myślę, że firma zajmująca się grami celowo manipuluje szansami, tak, że gdy masz zamiar rzucić palenie, dostajesz kolejną wygraną. -- Wzruszył ramionami. -- W każdym razie tak bym zrobił.

-- Jeśli jest losowo, nie powinno mieć znaczenia, co robisz i~gdzie grasz. Jeśli rzucisz monetą dziesięć razy i~wypadnie orzeł dziesięć razy z~rzędu, masz dokładnie taką samą szansę, że się pojawi orzeł jedenasty raz, niż gdybyś dostał wszystkie reszki, lub pół na pół. -- 

-- Matthew cały czas mówi takie rzeczy. Mówi, że chociaż mało prawdopodobne, że trafisz dziesięć orłów z~rzędu, każdy rzut ma dokładnie taką samą szansę.

-- Matthew brzmi, jakby znał się na matematyce.

-- Tak. Powinnaś go kiedyś spotkać. -- Przełknął. -- To znaczy, jeśli kiedykolwiek wyjdzie z~więzienia.

-- Och, musimy coś z~tym zrobić.

Odebrała sześć kolejnych telefonów, prowadząc program przez kolejne dwie godziny, przerywając na reklamy i~obiecując wszystkim swoim słuchaczom najbardziej ekscytujące wydarzenie ich życia, jeśli po prostu poczekają. Na początku Lu słuchał uważnie, ale bolała go głowa i~był taki zmęczony i~w końcu osunął się na swoje miejsce i~drzemał, dryfując w~snach i~wychodząc ze snów, słuchając, jak Jie beształa głupie dziewczyny z~fabryk z~południowych Chin.

Obudził się z~kropelką lodowatej wody na twarzy, sapnął i~usiadł, otwierając oczy w~samą porę, by zobaczyć Jie tańczącą w~oddaleniu od niego, śmiejącą się, jej twarz promieniała z~podniecenia. 

-- \textit{Uwielbiam }robić ten program! -- powiedziała. -- Jesteś następny, przystojniaku! 

Spojrzał na telefon i~zdał sobie sprawę, że drzemał ponad godzinę i~że minęła już pora kolacji. Zaburczało mu w~żołądku. Jie zdjęła buty i~skarpetki i~rozpięła dwa górne guziki swojej czerwonej bluzki. Miała rozpuszczone włosy i~rozmazany makijaż. Wyglądała, jakby bawiła się najlepiej w~swoim życiu. 

-- Co? -- głowa pulsowała mu i~smakowało, jakby coś użyło jego ust jako toalety.

-- \textit{Chodź }-- powiedziała i~znów się zbliżyła, zakładając mu słuchawki. -- Zbliża się 20:00. Wtedy moja widownia osiąga szczyty. Wróciły z~obiadu, skończyły plotkować i~wszystkie siedzą na swoich łóżkach, nastawiając na komputery, telefony i~radia. Ekscytuję Tobą od wielu \textit{godzin}. Każda ładna dziewczyna w~delcie Rzeki Perłowej czeka na ciebie, jesteś gotowy?

-- Ja \ldots  ja \ldots  -- Nagle nie mógł znaleźć języka. -- Tak! -- zdołał powiedzieć.

-- Załóż słuchawki -- zawołała, podbiegając do swojej strony biurka i~rzucając się na swoje miejsce. -- Jesteśmy na żywo za 10, 9, 8 \ldots  

Pogrzebał w~słuchawkach, opuścił mikrofon, sięgnął po szklankę z~wodą i~przełknął za dużo, krztusił się, próbował ją zatrzymać, krztusił się więcej, rozlał wodę na cały przód. Jie roześmiała się głośno, przełykając wodę, gdy mówiła do swojego mikrofonu.

-- Wróciliśmy, wróciliśmy, wróciliśmy, a teraz siostry, mam specjalną niespodziankę, którą obiecywałem wam przez całą noc! Rycerz ludu, bohater fabryki, zabójca, który polował na piratów w~kosmosie i~smoki na wzgórzach, zawodowy farmer złota o imieniu \ldots  -- Urwała. -- Jak mam cię nazwać, bohaterze?

-- Och! -- Zastanowił się przez chwilę. -- Tank -- powiedział. -- To taki rodzaj gracza, jakim jestem, czołg.

-- Tank! -- Zachichotała. -- To jest po prostu idealne. Och, siostry, gdybyście tylko mogły zobaczyć ten wielki, umięśniony czołg, który mam tutaj w~moim studio. Pozwólcie, że opowiem wam o Tanku. Oglądałam krótkie video tego popołudnia i~przyłapałam się na tym, że oglądam coś niesamowitego: dziesiątki chłopców, ustawionych w~kolejce przed kafejką internetową, mrugających i~bladych jak nowonarodzone myszy w~świetle dnia. Wydawało się, że byli innym rodzajem chłopaka z~fabryki, legendarnymi farmerami złota z~Shenzhen, i~żądali lepszej pracy, lepszej płacy, lepszych warunków i~koniec ich okrutnych, chciwych szefów. Czy to brzmi znajomo, siostry?

-- Przybyła policja, brudna jingcha, z~hełmami, pałkami i~gazem, tchórze z~ukrytymi twarzami i~brutalną bronią w~ręku, by walczyć z~tymi chłopcami, którzy chcieli tylko sprawiedliwości. Ale czy chłopcy uciekli? Nie! Czy wrócili do pracy i~przeprosili szefów? Nie! Mysia armia twardo broniła się, twierdząc, że ich miejsce pracy jest prawowitym domem, miejscem, które opłacała ich praca. I co zrobili jingcha? Powiedz mi, Tank, co oni zrobili?

Lu spojrzał na nią, jakby była szalona. Gdy cisza się przedłużała, wykonała natarczywe gesty. 

-- Ja, to znaczy, pobili nas! 

-- Z pewnością to zrobili! Siostry, proszę, pobierzcie teraz ten film! Zobaczcie, jak jingcha szarżuje na chłopców z~Shenzhen, rozbijając im głowy, gazując ich, uderzając ich pałkami. A teraz zatrzymajcie na odważnym chłopcu, z~lewej dokładnie na 14:22. Mocny podbródek, szerokie oczy, małe piegi na nosie, włosy w~nieładzie. Widzicie, jak się nie cofa podczas szarży z~towarzyszami u boku? Widzicie jak jingcha z~pałką, który zamierza się na niego od tyłu i~uderza go w~ramię, przewracając? Widzicie pałkę, która unosi się i~ląduje na głowie chłopca, krew, która leci z~rany?

-- To, siostry, Tank, chłopak siedzący naprzeciwko mnie, zakrwawiony, ale nieugięty, odważny i~silny, broniący praw robotników \ldots  -- Zakończyła chichotem. Lu też zachichotał, nic na to nie mógł poradzić. 

-- Och, przepraszam, przepraszam. Słuchajcie, to bardzo miły chłopak i~nieźle się na niego patrzy, a jingcha biła go po głowie i~ramieniu, jakby przygotowywali stek, a wszystko, co robił, to upieranie się, że ma prawo pracować jak człowiek, a nie jak zwierzę. I nie jest sam. Nazywają to ,,Chińską Republiką Ludową'', ale lud nie ma nic do powiedzenia na temat sposobu jej prowadzenia. To wszystko jest korupcją i~wyzyskiem.

-- Pomyślałam, że wideo było niesamowite, prawdziwa inspiracja. A potem zobaczyłam jego, naszego Tanka, wędrującego oszołomionego i~zakrwawionego \ldots  -- przerwała. -- W lokalizacji, której nie ujawnię, aby jingcha nie wiedział, który materiał wideo musi przejrzeć. Widziałam go i~powiedziałam, że chcę go wam przedstawić, moi przyjaciele, a potem powiedział mi niesamowitą historię, którą słyszałam, a \textit{wiecie}, że słyszę tu wiele niesamowitych historii każdego wieczoru. Opowieść o globalnym ruchu mającym na celu poprawę losu pracowników na całym świecie i~mam nadzieję, że to jest historia, którą opowie nam dziś wieczorem. Tank, kochanie, zacznij od ran. Czy możesz opisać je naszym przyjaciołom?

I Lu to zrobił, a potem zorientował się, że potem zaczął opowiadać o tym, jak został farmerem złota, jak wyglądało jego życie, historie, które opowiedział mu Matthew o tym, jak Boss Wing zmusił go i~jego przyjaciół do powrotu do pracy w~jego fabryce, mówiąc i~rozmawiając, aż woda znikła, a usta mu wyschły, i~na szczęście wezwała kolejną reklamę.

Opadł na krzesło, a ona dała mu trochę wody. 

-- Powinieneś zobaczyć czaty -- powiedziała. -- Wszyscy są w~tobie zakochani, ,,Tanku''. Sposób, w~jaki uratowałeś rzeczy tych dziewcząt w~Shilong New Town! Jesteś ich bohaterem. Są dziesiątki z~nich, które twierdzą, że były tam tego dnia, że widziały, jak wspinasz się po płocie. Posłuchaj tego: ,,Jego mięśnie falowały jak żelazne opaski, gdy wspinał się po płocie jak potężne stworzenie z~dżungli \ldots '' -- Wciągnął wodę do nosa, a Jie ścisnęła jego biceps. -- Musisz jeszcze trochę poćwiczyć, stworzonko z~dżungli, twoje mięśnie zmiękły! 

-- Jak możesz mieć fora dyskusyjne? Czy oni ich nie blokują? 

-- Och, to proste -- powiedziała. -- Wybieramy losowego bloga w~sieci, zwykle takiego, na którym nikt nie publikował w~ciągu roku lub dwóch, i~przejmujemy tablicę komentarzy na jednym z~jego postów. Kiedy zablokują, albo serwer przestanie odpowiadać, przełączamy się na kolejny. To proste  \ldots  i~zabawne! 

Roześmiał się i~potrząsnął głową, co ponownie wywołało u niego ból głowy. Skrzywił się i~ścisnął głowę w~dłoniach. 

-- Czysty geniusz!

Teraz reklama się kończyła i~oboje szybko usiedli na krzesłach i~ustawili mikrofony na miejscu. Lu był teraz w~tym dobry, rozmowa przychodziła do niego tak, jak wtedy, gdy rozmawiał ze swoimi towarzyszami z~gildii. Zawsze był gawędziarzem całej bandy.

Historia toczyła się dalej, opowiedział o tym, jak Webblies przyszły do niego i~jego gildii w~grze, rozmawiał z~nimi o potrzebie solidarności i~wzajemnej pomocy, aby chronić się przed bossami, graczami, którzy polowali na farmerów złota, przed firmą zajmującą się grami.

-- Chcą zjednoczyć chińskich robotników -- powiedziała Jie, mądrze kiwając głową.

-- Nie! -- Zaskoczył się swoją gwałtownością. -- Zjednoczenie chińskich pracowników byłoby bezużyteczne. Przy farmowaniu złota praca może po prostu przenieść się do Indonezji, Wietnamu, Kambodży, Indii  \ldots  wszędzie tam, gdzie pracownicy nie są zorganizowani. Tak samo jest z~całą pracą teraz  \ldots  wasza praca może się przenieść w~mgnieniu oka w~ogóle do dowolnego miejsca, gdzie można zbudować fabrykę i~zadokować kontenerowiec. Nie ma już czegoś takiego jak ,,chińscy'' robotnicy. Tylko robotnicy! I tak Webblies organizują nas wszystkich, wszędzie!

-- To dużo pracowników -- powiedziała. -- Ilu macie? 

Zwiesił głowę. 

-- Jiandi -- powiedział. -- Wszyscy widzimy licznik i~wszyscy wiwatujemy, gdy podnosi się o kilkaset, ale jesteśmy daleko.

-- Och, Tank -- powiedziała. -- Nie zniechęcaj się. Dziesiątki tysięcy ludzi! To fantastyczne  \ldots  i~jestem pewna, że możemy pozyskać dla ciebie kilku członków. Jak moi słuchacze mogą się przyłączyć?

-- Ech? Och! -- Z trudem pamiętał, jak to zrobić. -- Musicie uzyskać co najmniej 50 procent swoich współpracowników, aby zgodzili się zarejestrować, a następnie certyfikujemy związek dla całej waszej fabryki.

-- Ajaah! 50 procent! Wielkie fabryki mają po 50 000 pracowników! Jak to zrobić? 

Wzruszył ramionami. 

-- Nie jestem pewien -- powiedział. -- Przeważnie podpisywaliśmy małe fabryki gier, nie są większe niż 200 pracowników. Ale to musi być możliwe. Związki zawodowe na całym świecie zorganizowały się w~fabrykach każdej wielkości. -- Przełknął, rozumiejąc, jak kiepsko brzmi. -- Słuchaj, to zazwyczaj strona Matthew. On to wszystko rozumie. Jestem tylko czołgiem, rozumiesz? Stoję z~przodu i~pochłaniam wszystkie obrażenia. I nie możesz rozmawiać z~Matthew, ponieważ jest w~więzieniu. 

-- Ach tak, więzienie. Opowiedz nam o tym, co się dzisiaj wydarzyło.

Więc opowiedział im historię bitwy, tym wszystkim milionom dziewcząt w~miastach Guangdong, i~znalazł się \ldots  przeniesiony. Zabrany z~powrotem do kawiarni, krzyki, policja i~wrzaski, jego głos docierający do jego uszu z~daleka przez zapamiętane krzyki w~uszach. Kiedy przestał, wrócił do rzeczywistości i~znalazł Jie wpatrującą się w~niego mokrymi oczami i~rozchylonymi ustami. Spojrzał na swój telefon. Była prawie północ.

Wzruszył ramionami z~suchymi ustami. 

-- Ja \ldots  Cóż, to chyba to.

-- Wow -- odetchnęła Jie i~uruchomiła kolejną reklamę. -- Nic ci nie jest? 

-- Czuję się, jakby moja głowa została zmiażdżona między dwiema ciężkimi skałami -- powiedział. Przesunął tyłek na krześle i~się skrzywił. -- A moje ramię płonie.

-- Naprawdę trzymałam cię na nogach -- powiedziała. -- Ale prawie tu skończyliśmy. Jesteś naprawdę twardym draniem, wiesz o tym?

Nie czuł się twardy. Prawdę mówiąc, czuł się okropnie z~powodu faktu, że uciekł, podczas gdy wszyscy jego kumple z~gildii byli aresztowani. Logicznie wiedział, że nie skorzystaliby na uwięzieniu go razem z~nimi, ale to była logika, a nie uczucia.

-- Dobrze -- powiedziała. -- Wróciliśmy. Co za \textit{historia}! Siostry, czy nie mówiłem wam, że mam dzisiaj coś wyjątkowego? Niestety, już prawie czas kończyć \ldots  wszyscy potrzebujemy trochę snu, zanim wrócimy rano do pracy, nieprawdaż? Jeszcze tylko jedno: \textit{co z~tym zrobimy}? 

Nagle przestała być śpiąca i~uspokajająca. Jej oczy były szeroko otwarte i~mocno ściskała krawędź biurka. 

-- Przybywamy tu z~naszych wiosek, chcąc wykonać uczciwą pracę za przyzwoitą płacę, abyśmy mogli pomóc naszym rodzinom, abyśmy mogli żyć i~przetrwać. Co otrzymujemy? Lepkich zboczeńców, którzy ruchają nas na pracy i~poza nią! Kryminalnych bękartów, którzy niszczą każdego, kto rzuca wyzwanie ich dochodom! Gliny, którzy biją nas, wsadzają do więzienia, jeżeli ośmielimy się podważyć status quo!

-- Siostry, to \textit{nie może trwać }! Tank tutaj powiedział, że nie ma już czegoś takiego jak chiński robotnik, jest tylko robotnik. Nie słyszałem o tych jego Webblies przed dzisiejszym wieczorem i~nie wiem, czy są nielepsi niż twój szef lub złodziej, który zdziera ze sprzedaży sieciowej, a mnie to nie obchodzi. Jeśli na całym świecie są pracownicy organizujący się w~celu uzyskania lepszej umowy, chcę być tego częścią i~tak też zrób i~ty!

-- Powiem wam, co będzie dalej. Tank i~ja znajdziemy Webblies i~zaplanujemy coś wielkiego. Coś \textit{wielkiego}! Nie wiem, co to będzie, ale to wszystko zmieni. Są nas \textit{miliony!} Wszystko, co robimy, jest \textit{wielkie}. 

-- Muszę się do czegoś przyznać. -- Jej głos stał się cichszy. -- Grzech do wyznania. Robię ten program, ponieważ zarabiam na nim pieniądze. Dużo pieniędzy. Muszę dużo wydawać, aby wyprzedzić zengfu, ale wiele zostało. Więcej niż zarabiacie, muszę przyznać. Minęło dużo czasu, odkąd byłam biedna jak dziewczyna z~fabryki. Jestem praktycznie bogata. Nie bogata jak szef, ale bogata, rozumiecie? 

-- Ale jestem z~wami. Nie zaczęłam tego programu, żeby się wzbogacić. Zaczęłam, ponieważ byłam dziewczyną z~fabryki i~troszczyłam się o moje siostry. Przyjeżdżamy do prowincji Guangdong, odkąd Deng Xiaoping zmienił zasady i~sprawił, że fabryki tutaj wyrosły. Minęły pokolenia, siostry, a my przyjeżdżamy, biedne myszy ze wsi, i~jesteśmy zmielone przez fabryki, w~których jesteśmy niewolnicami. Za każdego juana, którego wysyłamy do domu, nasi szefowie wkładają setkę do kieszeni. A kiedy skończymy, to co? Zostajemy jedną ze starych babć żebrzących przy drodze. 

-- Więc słuchajcie jutro. Dowiemy się więcej o tych webblies, ułożymy plan i~przedstawimy go wam. W międzyczasie nie przyjmujcie gówna od szefów. Nie pozwólcie gliniarzom popychać was, twoich sióstr i~braci. I bądźcie dla siebie dobre  \ldots  jesteśmy po tej samej stronie.

Kliknęła myszką i~opuściła pokrywę laptopa.

-- Uff! -- powiedziała. -- Co za \textit{noc}! 

-- Czy twój program jest taki co wieczór? 

-- Nie tak dobry, Tank. Z pewnością poprawiłeś rzeczy. Cieszę się, że porwałam cię ze stacji kolejowej.

-- Ja też -- powiedział. Był taki zmęczony. -- Chyba zadzwonię jutro w~sprawie następnego występu? Może moglibyśmy spotkać się rano i~spróbować dotrzeć do Webblies albo znaleźć sposób, żeby spróbować zadzwonić do moich gildii i~sprawdzić, czy nadal są w~więzieniu?

-- Zadzwonić do mnie? Nie bądź głupi, Tank. Nie spuszczę cię z~oczu. 

-- W porządku -- powiedział. -- Znajdę miejsce do spania. 

 Kiedy po raz pierwszy przybył do Shenzhen, spędził kilka nocy w~parkach. Mógł tak zrobić znowu. Nie było tak źle, jeśli w~nocy nie padało. Czy tego dnia były chmury? Nie pamiętał.

-- Z pewnością możesz \ldots  przez te drzwi, właśnie tam. -- Wskazała na sypialnię.

Nagle był całkowicie rozbudzony.

-- Och, nie mógłbym \ldots 

-- Zamknij się i~idź do łóżka. Masz kontuzję głowy, głupcze. I właśnie dałeś mi godziny świetnej audycji radiowej. Więc potrzebujesz tego i~zasłużyłeś na to. Łóżko. Teraz.

Był zbyt zmęczony, by się spierać. W drodze do łóżka potknął się trochę, a ona zrzuciła ubrania, zabawki i~torebki z~łóżka na podłogę tuż przed nim. Okryła go prześcieradłem i~pocałowała go w~czoło, gdy się ułożył. 

-- Śpij, Tanku -- szepnęła mu do ucha.

Zastanawiał się mgliście, gdzie będzie spać, kiedy wyszła z~pokoju i~ponownie usłyszał, jak pisze na swoim komputerze. Zasnął z~dźwiękiem klawiszy w~uszach.

Ledwo się obudził, kiedy wsunęła się razem z~nim pod kołdrę, przytuliła się do niego i~zaczęła cicho chrapać w~jego ucho.

Ale całkowicie się obudził już godzinę później, gdy dziesięć radiowozów zatrzymało się przed budynkami Houhai, wyły syreny, a reflektor helikoptera skąpał cały budynek w~świetle tak białym jak światło dzienne. Zesztywniała obok niego pod kołdrą, a potem praktycznie wylewitowała z~łóżka.

-- Dwadzieścia sekund -- warknęła. -- Buty, telefon, wszystko, czego potrzebujesz. Nie wrócimy tutaj. 

Lu czuł się niejasno dumny z~tego, jak spokojny był, kiedy wstał i~bez pośpiechu, spokojny sposób zabrał swoje buty -robotnicze tenisówki, tanie i~wszechobecne -- i~zawiązał je, a następnie wciągnął kurtkę, potem przeszedł sprawnie do salonu, gdzie Jie spryskiwała rozpuszczalnikiem wszystkie płaskie powierzchnie w~pokoju. Zapach był tak ostry jak jego ból głowy i~jeszcze go spotęgował.

Kiwnęła mu raz głową, a potem wskazała kolejną buteleczkę z~rozpuszczalnikiem i~powiedziała: 

-- Zrób łazienkę i~sypialnię. 

Zrobił, szybko pracując. Domyślał się, że to usunie wszystko, jak odcisk palca lub charakterystyczny rodzaje brudu. Skończył w~minutę, a może krócej, a ona stała przy jego łokciu z~torbą strunową pełną kurzu. 

-- Odkurzone z~siedzeń pociągu Hongkong-Shenzhen -- powiedziała. -- Komórki skóry od dobrego miliona ludzi. Rozłóż je równomiernie, proszę. Teraz szybko.

Kurz unosił mu się do nosa i~kichał, zapadał się w~załamania dłoni i~wszystko było trochę obrzydliwe, ale jego głowa była czysta i~pełna syren i~grzmotu helikoptera. Rozrzucając materiał genetyczny, patrzył, jak Jie wyjmuje dysk ze swojego komputera i~upuszcza smukły stick w~dekolt, i~\textit{to }w końcu przebiło się przez jego spokój. Nagle zdał sobie sprawę, że spędził noc, śpiąc obok tej pięknej dziewczyny i~nawet jej nie pocałował, a tym bardziej nie dotknął tych tajemniczych i~intrygujących piersi, które teraz ciepło obejmowały niezwykle kompromitujący kawałek nośnika pamięci, płat nośnika magnetycznego, które mogłyby wsadzić ich oboje do więzienia na zawsze.

Rozejrzała się i~odhaczyła mentalną listę kontrolną na palcu. Potem rzuciła zdecydowanym skinieniem głowy i~powiedziała: 

-- W porządku, chodźmy. 

Wyprowadziła go na korytarz, który był jasno oświetlony i~pusty, przez co czuł się bardzo odsłonięty. Wyciągnęła z~torebki krótki łom i~umiejętnie otworzyła stalowe drzwi przy panelu bezpieczników przy windach, odsłaniając schludne rzędy czarnych plastikowych wyłączników. Znowu sięgnęła do torebki i~wyszła z~jednorazową zapalniczką butanową, którą zapaliła, przykładając płomień do małego, białego winylu lub błyszczącego papieru wystającego jak języczek z~dyskretnego szwu w~panelu. Zasyczało i~błysnęło, a potem uniósł się z~niego smużka czarnego dymu, a potem papier spłonął, iskra zniknęła w~panelu.

Sekundę później cała powierzchnia panelu wybuchła deszczem iskier, dymu i~płomieni. Jie przyglądała się temu z~satysfakcją, gdy czarny dym wylewał się z~płyty. Potem wszystkie światła zgasły i~zaczęły bić alarmy przeciwpożarowe, głębokie do szpiku kości di-da-di-da-di-da, które zagłuszyły helikopter, syreny.

Włączyła małą czerwoną diodę LED, która skąpała jej twarz w~demonicznym świetle. Wyglądała na bardzo zadowoloną z~siebie. To sprawia, że Lu poczuł się spokojny.

-- Co teraz? -- powiedział.

-- Teraz wyjdziemy pieszo ze wszystkimi, którzy uciekają przed alarmami przeciwpożarowymi.

W całym budynku otwierały się drzwi, wyłaniały się zasmucone rodziny, kłębił się czarny i~gryzący dym. Skierowali się w~stronę schodów, tuż za Babcią z~Obwiązanymi Stopami, którą poznali dzień wcześniej. Na klatce schodowej spotkali setki, a potem tysiące kolejnych uchodźców z~budynku, wszyscy niosący naręcza cennych rzeczy, dzieci, starszych członków rodziny.

Na dole policja próbowała zebrać ich w~uporządkowaną grupę przed budynkiem, ale było za dużo ludzi, za dużo zamieszania. W końcu łatwo było prześlizgnąć się przez linie policyjne i~wmieszać się w~tłum gapiów z~pobliskich budynków, którzy przybyli popatrzeć.

\bigskip
\threeast

Ta scena jest poświęcona wielojęzycznym Sophia Books w~Vancouver, różnorodnym i~ekscytującym sklepie wypełnionym najlepszymi z~dziwnych i~ekscytujących światów popkultury wielu krajów. Sophia była tuż za rogiem mojego hotelu, kiedy pojechałem do Van, aby wygłosić wykład na Uniwersytecie Simona Frasera, a ludzie z~Sophii wysłali mi wcześniej e-mail z~prośbą o wpadnięcie i~podpisanie ich akcji, gdy będę w~okolicy. Kiedy tam dotarłem, odkryłem skarbnicę nigdy wcześniej niewidzianych prac w~oszałamiającej liczbie języków, od powieści graficznych po grube traktaty akademickie, pod przewodnictwem dobrodusznego (nawet slapstickowego) personelu, który tak wyraźnie sympatię do swojej pracy rozprzestrzeniał na każdego klienta, który przekroczył próg.

\href{https://en.wikipedia.org/wiki/HTTP_404}{\textit{Sophia Books}: 450 West Hastings St., Vancouver, BC Kanada V6B1L1 +1 604 684 0484} 


\bigskip
\threeast


Niezależnie od tego, czy jesteś rewolucjonistą, właścicielem fabryki, czy organizatorem hokeja z~małej ligi, jest jeden czynnik, którego nie możesz zignorować: koszt transakcji Coase'a.

Ronald Coase był amerykańskim ekonomistą, który zmienił wszystko dzięki pracy opublikowanej w~1937 roku zatytułowanej ,,Teoria firmy''. Artykuł Coase'a argumentował, że prawdziwym biznesem każdej organizacji jest organizowanie ludzi. Religia to system organizowania ludzi do modlitwy i~dawania pieniędzy na budowę kościołów i~płacenie księżom, pastorom lub rabinom; fabryka obuwia to system organizowania ludzi do produkcji butów. Rewolucyjny spisek to system organizowania ludzi w~celu obalenia rządu.

Organizowanie jest rodzajem podatku od działalności człowieka. Z każdą minutą spędzoną na \textit{robieniu rzeczy }musisz poświęcić kilka sekund na upewnienie się, że nie wyprzedzasz, nie jesteś z~tyłu ani nie stajesz u boku innych osób, z~którymi robisz różne rzeczy. Sekundy, w~których płacisz dziesięcinę organizacji, to koszt Coase'a, podatek od twojej pracy, który płacisz za to, że jesteśmy ludźmi, a nie mrówkami, pszczołami czy jakimś innym gatunkiem, któremu udaje się maszerować zgodnie z~czystym instynktem.

Och, możesz pokonać ten koszt: po prostu trzymaj się projektów, przy których nie potrzebujesz niczyjej pomocy. Na przykład \ldots  wiązanie butów? (Nie, chyba że zaplatasz własne sznurowadła). Opiekanie własnej kanapki (chyba że sam zebrałeś drewno na ogień i~pszenicę na chleb i~mleko na ser).

Faktem jest, że wszystko, co robisz, polega na współpracy, gdzieś tam ktoś inny miał w~tym swój udział. A część kosztów tego, co robisz, jest przeznaczana na upewnienie się, że koordynujesz prawidłowo, że ser dostaje się do lodówki i~że prąd brzęczy w~przewodach.

Nie możesz wyeliminować kosztów Coase'a, ale możesz je obniżyć. Można to zrobić na dwa sposoby: opracować lepsze techniki organizacyjne (powiedzmy ,,księgowość z~podwójnym zapisem'', wstrząsający Ziemią wynalazek z~XIII wieku, który jest sercem każdej organizacji zarabiającej pieniądze na świecie, od kościołów po korporacje do rządów) lub zdobyć lepszą technologię.

Rozważmy wyjście do kina. Jest piątek wieczorem i~myślisz o obejrzeniu filmu, ale nie chcesz iść sama. Wyobraź sobie, że był rok 1950, jak rozwiązałabyś ten problem?

Cóż, musiałabyś znaleźć gazetę i~zobaczyć, co jest grane. Wtedy trzeba by zadzwonić do wszystkich domów znajomych (bez telefonów komórkowych, pamiętaj!) i~zostawić im wiadomości. Potem trzeba było czekać, aż niektórzy lub wszyscy oddzwonią i~zgłoszą swoje preferencje filmowe. Potem trzeba by ich wydzwonić pojedynczo i~dwójkami i~zobaczyć, czy uda ci się przekonać ich masę krytyczną do obejrzenia tego samego filmu. Potem trzeba było iść do teatru, zlokalizować się nawzajem i~mieć nadzieję, że przedstawienie nie zostało wyprzedane.

Ile to kosztuje? Cóż, najpierw zobaczmy, ile wart jest ten film: jednym ze sposobów, aby to zrobić, jest przyjrzenie się, ile ktoś musiałby ci zapłacić, aby przekonać cię do rezygnacji z~chodzenia do kina. Innym jest stopniowe podnoszenie cen biletów, aż w~końcu zdecydujesz się nie oglądać filmu.

Gdy już uzyskasz tę liczbę, możesz obliczyć swój koszt Coase'a: możesz zapytać, ile kosztowałoby Cię zapłacenie komuś innemu za przygotowanie się za Ciebie lub ile możesz zarobić na pracy po szkole, jeśli nie grałeś telefonem ze znajomymi.

Otrzymasz równanie, które wygląda tak:

\noindent [Wartość filmu] --~[Koszt zorganizowania spotkania znajomych] =~[Wartość netto wieczoru]

Dlatego zrobisz coś mniej zabawnego (zostaniesz w~domu i~obejrzysz telewizję), ale prostego, zamiast wychodzić i~robić coś zabawniejszego, ale bardziej skomplikowanego. Nie chodzi o to, że filmy nie są zabawne, ale jeśli sprowadzanie znajomych na wieczór jest zbyt dużym problemem, to liczba filmów, które oglądasz, spada.

A teraz pomyśl o wieczorze w~kinie dzisiaj, obecnie. Jest 18:45 w~piątek wieczorem i~wszystkie filmy zaczną się za 20-50 minut. Wyciągasz telefon i~wyszukujesz w~Google listy posortowane według odległości od Ciebie. Następnie wysyłasz wiadomość tekstową do swoich znajomych -- jeśli Twój telefon jest bardzo inteligentny, możesz ją wysłać tylko do znajomych z~sąsiedztwa -- wymieniając dostępne filmy. Odpowiadają sobie nawzajem, a po kilku seriach odpowiedzi znalazłaś grupę ludzi, z~którymi możesz obejrzeć film. Kupujesz bilety przez telefon.

Ale wtedy docierasz tam i~odkrywasz, że tłumy są tak ogromne, że nie możecie się znaleźć. Więc dzwonicie do siebie i~umawiacie się na spotkanie przy barze z~przekąskami, a chwilę później siedzicie na swoich miejscach i~jecie popcorn.

Więc? Dlaczego ktoś miałby dbać o to, ile kosztuje wykonanie rzeczy? Ponieważ koszt Coase'a to cena bycia \textit{nadludzkim}.

W dawnych czasach -- bardzo, bardzo dawnych czasach -- twoi przodkowie byli samotnymi małpami. Pracowały w~pojedynkę lub w~parach, aby robić wszystko, czego potrzebowała małpa, od zbierania jedzenia przez opiekę nad dziećmi, wypatrywanie drapieżników po budowanie gniazd. Miało to swoje ograniczenia: jeśli opiekujesz się dziećmi, nie możesz zbierać jedzenia. Jeśli zbierasz jedzenie, możesz przepuścić tygrysa i~stracić dzieci.

Pojawia się plemię: grupy małp, które pracują razem, dzieląc pracę. Teraz nie są tylko samotnymi małpami, są grupami małp i~mogą zrobić więcej niż jedna małpa. Wyszły ponad małpiowatość. Są \textit{super małpami}.

Bycie super małpą nie jest łatwe. Jeśli jesteś indywidualną super małpą, istnieją dwa sposoby na radzenie sobie: możesz bawić się ze wszystkimi małpimi kumplami, aby nakarmić dzieci i~mieć oko na tygrysy, lub możesz ukryć się w~krzakach i~drzemać, udając, że pracujesz, pojawiając się tylko w~porze posiłków.

Z indywidualnego punktu widzenia bycie leniwą małpą ma sens. W wielkim plemieniu małp jeden lub dwóch głupców nie doprowadzi do bankructwa grupy. Jeśli zamiast pracować, uchodzi ci na sucho drzemka, a mimo to dostajesz się do jedzenia, dlaczego by tego nie zrobić?

Ale jeśli \textit{wszyscy }to robią, to tyle, jeśli chodzi o super małpy. Teraz nikt nie dostaje owoców, nikt nie opiekuje się dziećmi i~cholera, myślałem, że \textit{ty} uważasz na tygrysy! Zbyt wiele leniwych małp plus tygrysy to obiad.

Tak więc małpy -- i~ich bezwłosi potomkowie, tacy jak ty -- potrzebują specjalistycznego sprzętu do wykrywania oszustów i~karania ich, zanim ktoś skorzysta z~idei i~pojawią się tygrysy. Ten wyspecjalizowany sprzęt to warstwa tkanki owinięta wokół górnej części mózgu, zwana korą nową czy ,,neo-cortex''. Kora nowa jest odpowiedzialna za śledzenie małp. To ta część mózgu, która organizuje ludzi, kontroluje ich, zakochuje się w~nich, wytwarza wrogość. To część twojego mózgu, która zostaje rozświetlona, gdy bawisz się z~Facebookiem lub innymi portalami społecznościowymi, i~jest to część twojego mózgu, w~której znajdują się lokalne kopie ludzi z~twojego życia. To właśnie stamtąd dobiega głos twojej matki, która każe ci umyć zęby.

Kora nowa to koszt Coase'a zastosowany do mózgu. Każdy łyk powietrza, którym oddychasz, każda spożyta kaloria, każdy bumbum twojego serca idzie na nakarmienie tej nowej kory, która śledzi innych ludzi w~twojej grupie i~co robią, niezależnie od tego, czy są w~kolejce, czy poza rezerwacją.

Koszt Coase'a to granica twojej zdolności do bycia nadludzkim. Jeśli koszt Coase'a jakiejś działalności jest niższy niż wartość, którą byś z~tego uzyskała, możesz zebrać kilku przyjaciół i~\textit{zrobić to}, przekroczyć ograniczenia, które natura nałożyła na samotne, bezwłose małpy i~stać się \textit{nadczłowiekiem}.

Wynika z~tego, że wysokie koszty Coase powodują, że jesteś mniej potężny. Co więcej, duże instytucje dysponujące dużą ilością pieniędzy i~władzy mogą pokonać wysokie koszty Coase: rząd może umieścić na polu bitwy 10 000 żołnierzy z~czołgami, żywnością i~medykami; ty i~twoje kumpele nie możecie. Tak wysokie koszty Coase mogą ograniczyć \textit{twoją }zdolność do bycia nadludzkim, pozostawiając bogatych i~potężnych w~posiadaniu supermocy, których nigdy nie możesz osiągnąć.

I to jest prawdziwy powód, dla którego potężni boją się otwartych systemów i~sieci. Jeśli ktokolwiek może nawiązać bezpłatne połączenie głosowe z~kimkolwiek na świecie za pomocą sieci, to wszyscy możemy komunikować się z~taką samą łatwością, jaka jest standardem dla wysokich i~potężnych. Jeśli ktokolwiek może tworzyć i~sprzedawać wirtualne bogactwo w~grze, to wszyscy jesteśmy w~tych samych ekonomicznych butach, co międzynarodowe megakorporacje, które rozpoczynają gry.

A jeśli jakikolwiek pracownik, gdziekolwiek, może komunikować się z~każdym innym pracownikiem, gdziekolwiek, za darmo, natychmiast, bez zgody szefa, to bracie, uważaj, bo to podstawa kosztu żądania lepszej płacy, lepszych warunków pracy i~kawałka tortu właśnie stała się \textit{dużo }tańsza. A ludzie, którzy mają władzę, nie będą siedzieć spokojnie i~pozwalali, by banda roboli im ją odebrała.

\bigskip
\threeast

Ta scena jest poświęcona MIT Press Bookshop, sklepowi, który odwiedzałem podczas każdej podróży do Bostonu w~ciągu ostatnich dziesięciu lat. MIT jest oczywiście jednym z~legendarnych węzłów źródłowych globalnej kultury nerdów, a księgarnia kampusowa spełnia niesamowite oczekiwania, jakie miałem, gdy po raz pierwszy postawiłem w~niej stopę. Oprócz wspaniałych tytułów publikowanych przez prasę MIT, księgarnia to wycieczka po najbardziej ekscytujących publikacjach high-tech na świecie, od hakerskich zinów, takich jak 2600, po grube akademickie antologie dotyczące projektowania gier wideo. To jeden z~tych sklepów, w~których muszę prosić o wysłanie moich zakupów do domu, ponieważ nie mieszczą się w~mojej walizce.

\href{https://mitpressbookstore.mit.edu/}{\textit{MIT Press Bookshop}: Budynek E38, 77 Massachusetts Ave., Cambridge, MA USA 02139-4307 +1 617 253 5249} 

\bigskip
\threeast

Centrala dowodzenia Coca Cola Games została zaprojektowana przez jednego z~czołowych światowych scenografów. Dokument wymagał pomieszczenia, które wyglądało na to, że można go wykorzystać do zarządzania imperium zła, wystrzeliwania międzygalaktycznego statku odkrywczego lub dowodzenia zaawansowaną technologicznie armią najemników. Wszystko było zakrzywione, ze szczotkowanej stali, oświetlone punktowo, a to, co nie było chromowane, było czarne, z~wyjątkiem akcentów popękanej, zniszczonej czarnej skóry zebranej ze starych motocyklowych kurtek. Wszędzie były ekrany wbudowane w~stoły, zwinięte w~suficie lub podłodze, nawet jeden z~tyłu drzwi. Dowolną ścianę można było zapisać specjalnymi pisakami, które wykorzystywały RFID i~akcelerometry do śledzenia ich ruchów i~przesyłania ich do komputera, który rejestrował to wszystko i~rozrzucał na bezprzewodowych ekranach wielodotykowych, które były zapinane na rzepy w~całym pomieszczeniu.

Zgrabne zdjęcia Centrali zdobiły stronę rekrutacyjną Coca Cola Games i~znalazły się w~serii filmów dokumentalnych o próżności, które CCG zleciła o sobie, wyglądając świeżo po projektantach, wypełnionych wysportowanymi, intensywnymi, śmiejącymi się młodymi ludźmi w~eleganckich ubraniach, robiącymi inteligentne rzeczy.

Centrum dowodzenia Coca Cola Games było kłamstwem.

Dziesięć sekund po przeniesieniu gamerunnerów do Command Central każdy multitouch został złamany lub skradziony. Zagłębione terminale umieszczone w~stołach były przestarzałe, zanim zostały zainstalowane, a teraz spotkał ich haniebny los: służąc jako podstawki dla najnowocześniejszych laptopów wyposażonych w~karty graficzne, które tak się nagrzewały, że ich wentylatory brzmiały jak silniki odrzutowe.

Piętnaście sekund później każdą płaską powierzchnię pokrywały opakowania po śmieciowym jedzeniu, pudełka po pizzy, puszki po napojach energetycznych, stare powieści science-fiction, zużyte chusteczki, hełmy orków w~origami złożone z~karteczek samoprzylepnych, zgrabne kapelusze i~nieskończenie różnorodne, licencjonowane bzdury, które CCG zrobiło z~gry, od dozowników Pez, przez zakrętki na wentyle rowerowe, po karty kolekcjonerskie i~scyzoryki.

Dwadzieścia sekund później pokój nabrał zapachu gamerunnerów, uderzającej mieszanki smaru do pizzy przetartego przez pory pod pachami, taniej wody kolońskiej, niemytych włosów, klasycznego japońskiego dżinsu i~oleju silnikowego.

A teraz lśniące legowisko supergeniusza stało się ekskluzywną jaskinią spotkań plemienia dzikich, hiperkonkurencyjnych, niezwykle dobrze opłacanych producentów, którzy zaszyli się tam, zgrzytając zębami i~krzycząc na siebie za każdą godzinę, którą Bóg zesłał. Żaden sprzątacz nie wchodził do pokoju, a nawet osobiści asystenci szli tylko do drzwi, gdzie żałośnie wykrzykiwali imiona swoich szefów i~unikali obrzydliwych opakowań po jedzeniu, rzucanych im w~głowy przez gamerunnerów, którzy nie czuli się uprzejmie, że przerywa im się pracę.

Connor Prikkel znalazł swoje plemię. Technicznie był wiceprezesem, ale nikt się do niego nie zgłaszał, z~wyjątkiem asystenta, którego zadaniem było wyciąganie go z~Centrum Dowodzenia kilka razy w~miesiącu, czyszczenie parowe w~firmowej siłowni, wsadzanie go do firmowego jetu i~wystrzelenie go w~tłumy graczy i~prasy na całym świecie, aby wyjaśnić, z~wyniosłym uśmieszkiem, w~jaki sposób Coca Cola Games zdołało nadzorować trzy z~dwudziestu największych gospodarek świata.

Przez resztę czasu praca Connora polegała na pracy na fingerspitzengefuhl. To było przydatne słowo. To było oczywiście niemieckie słowo. Niemcy mieli słowa na \textit{wszystko}, stworzone przez prosty zabieg polegający na zderzaniu tylu mniejszych słów, ile trzeba było razem, aż do uzyskania jednego potwornego mordercy ust, jak fingerspitzengefuhl, które dokładnie i~precyzyjnie oddawało coś, do czego żaden inny język nawet się nie zbliżył.

Fingerspitzengefuhl oznacza ,,czucie opuszkiem palca'' -- to uczucie, które pojawia się, gdy świat spoczywa na grubej poduszce zakończeń nerwowych na czubkach palców. To uczucie, gdy trzymasz lekko piłkę do koszykówki w~dłoni i~wiesz dokładnie, dokąd zabierze ją następne odbicie, gdy ją puścisz. To uczucie, które masz, kiedy trzymasz dziecko i~możesz wyczuć, czy ona teraz zasypia, czy się budzi. To uczucie, które odczuwasz, gdy dłonie lekko spoczywają na kierownicy roweru, skacząc po stromym zboczu, delikatnie naciskając na hamulce, jeżdżąc po ostrej jak brzytwa linii między wykonaniem manewru i~bezpiecznym dotarciem do dna.

Propriocepcja to twoja zdolność do wyczuwania, gdzie w~przestrzeni znajduje się twoje ciało w~stosunku do wszystkiego innego. To szósty zmysł, a ty nawet nie wiesz, że go masz, dopóki go nie stracisz, jak wtedy, gdy splatasz palce i~przeplatasz ręce przez ramiona i~stwierdzasz, że poruszasz lewym palcem, gdy chcesz poruszyć prawym; lub gdy nadepniesz na duchowy krok na szczycie schodów i~twoja stopa nie spadnie na nic.

Fingerspitzengefuhl to propriocepcja świata, rozszerzenie twojego szóstego zmysłu na wszystko wokół ciebie. Masz fingerpitzengefuhl, kiedy możesz rozpoznać, tylko po tym, jak czujesz powietrze, że twoja klasa jest w~złym humorze lub że twój kolega z~drużyny jest na dobrej pozycji i~czeka na podanie piłki.

Palce Connora oznaczały, że czuł \textit{wszystko}, co działo się w~grach, które prowadził. Mógł powiedzieć, kiedy w~Wojownikach Svartalfaheim doszło do gonitwy po złoto, albo kiedy kredyty Zombie Mecha spadają. Mógł powiedzieć, kiedy Fortecę Odyna atakowała ogromna gildia, sześciuset ludzi wcielonych w~sześćset statków powietrznych, koordynowanych przez generałów, kapitanów i~poruczników. Mógł powiedzieć, kiedy był korek na Moście Brooklińskim w~Zombie Mecha, ponieważ zbyt wielu roninów próbowało wjechać na Manhattan, aby oczyścić Flatiron Building i~ukończyć Publishing Quest.

Cała ta wiedza dotarła do niego poprzez jego stale zmieniające się, ciągle zmieniające się kanały -- wykresy, transkrypcje czatów, logi serwera, paski przedstawiające obciążenie i~pamięć czy wskaźnik odejścia subskrybentów oraz każdy inny fragment zmieniających się informacji z~gry. Migały w~kolorowej rolce, na ekranie jego potwornego, szerokoekranowego laptopa, przezroczystość zmniejszona do 10 procent w~oknach znajdujących się nad jego ekranami, na których uruchomił cztery avatary w~obu grach.

Każdy gamerunner miał inny sposób na osiągnięcie fingerpitzengefuhl, tak osobistą jak myśl, którą podążasz, aby iść spać lub powód, dla którego się zakochujesz. Niektórzy lubią \textit{dużo }ekranów -- cztery lub pięć. Niektórzy słuchali dużo czytanego na głos tekstu i~podsłuchiwali gamechat. Niektórzy tylko oglądali wykresy, niektórzy tylko dzienniki, niektórzy tylko ekrany gier. Coca Cola Games zatrudniło kilku psychologów przemysłowych, aby spróbowali odkryć metody producentów gier, spróbować stworzyć system do ich odtwarzania i~udoskonalania. Przetrwali dzień, zanim zostali wyrzuceni z~centrum dowodzenia pośród nawałnicy obelg i~wulgaryzmów.

Prowadzący grę nie chcieli być usystematyzowani. Nie chcieli być badani. Bycie producentem, bycie gamerunnerem oznaczało zdobycie fingerpitzengefuhl i~vice versa. Gamerunnerzy nie potrzebowali psychiatrów, żeby im powiedzieć, kiedy mają fingerpitzengefuhl. Kiedy miałeś fingerpitzengefuhl, wpadałeś w~ciepłą kąpiel, rodzaj hiperczujnej śpiączki, w~której wiedza przepływała do i~z każdego otworu z~maksymalną prędkością. Fingerspitzengefuhl potrzebowało kawy i~napojów energetycznych, fast foodów i~głośnej, cholernej muzyki, pomruków twoich współpracowników. Fingerspitzengefuhl nie potrzebowało psychologii przemysłowej.

Fingerspitzengefuhl Connora było najlepsze. Kierował nieprzytomnym tańcem palców na laptopie, podsłuchiwał właściwe rozmowy, monitorował właściwe działania, wypatrywał rozpoczynającej się walki Webblies z~Pinkertonami. Chrząknął tym specjalnym chrząknięciem, które ostrzegało resztę jego plemienia o niebezpieczeństwie, i~dźgnął ekran grubym palcem posmarowanym olejem do pizzy. Wiedza przepłynęła przez pokój jak fala, brzuchy i~podbródki chwiały się, gdy całe plemię dostroiło się do walki.

-- Powinniśmy wyciągnąć z~tego wtyczkę -- powiedziała Fairfax, projektantka, która dotarła do Centrum Dowodzenia.

-- Zapomnij o tym -- powiedział Kaden. -- Dwadzieścia tysięcy złota na Webblies.

-- Dwa do jednego? -- powiedział Palmer, ekonomista numer dwa, który zdobył doktorat, ale nie wymyślił równań Prikkela.

-- Żadnych zakładów -- powiedział Connor. -- Po prostu obejrzyjcie grę. 

-- Jesteś takim maniakiem walki -- powiedział Kaden. -- Wybrałeś złą specjalizację. Powinieneś być strategiem wojskowym.

-- Złe wynagrodzenie, głupie ubrania i~musisz pracować dla rządu -- warknął Connor, zauważając zesztywniałe kręgosłupy Kadena i~Billa, zwerbowanych z~dowództwa antyterrorystycznego Delta Force Pentagonu, aby pomóc przeanalizować dowództwo wielkich gildii, ich struktury i~dowiedzieć się, jak uzyskać z~nich więcej pieniędzy.

-- Spójrz, jak idą! -- powiedział Fairfax. 

Connor poświęcił jej dużo czasu, chociaż często się ze sobą nie zgadzali. Prowadziła duże zespoły projektantów poziomów, grafików, specjalistów AI, programistów, wszystko to i~miała dobry widok z~góry na dół i~od dołu do góry.

-- Są dobrzy -- powiedział Connor. Kliknął trochę i~pokolorował każdego avatara flagą narodową reprezentującą kraj, w~którym zarejestrowany był adres IP gracza. -- I to jest cholerne ONZ graczy, spójrz na to. W jakim języku mówią? 

Pstryknął jeszcze trochę i~przejął głośniki w~pokoju, sprytnie wpuszczone w~ściany i~podłogi, teraz zakopane pod górami kartonu do pizzy. Pokój wypełnił bełkot mocno akcentowanego angielskiego zmieszanego z~mandaryńskim. Ucho wyczuwało akcenty indyjskie, chińskie, coś jeszcze -- malajski? indonezyjski? W tym tłumie byli gracze z~całego Półwyspu Malajskiego.

-- I spójrz na Pinkertonów -- powiedziała Fairfax. 

Miała doświadczenie w~programowaniu sztucznej inteligencji, handlu, który bardzo się zmienił, odkąd Mechaniczni Turcy wkroczyli, by powstrzymać AI w~grze. Ale wpadła na pomysł, aby nadać ścieżce dźwiękowej gry własną sztuczną inteligencję, zdolną do podniesienia współczynnika dramatyzmu w~muzyce, gdy nadchodziły ważne wydarzenia, a za ten holistyczny pogląd na rozgrywkę wylądowała w~centrum dowodzenia. To ona zamawiała zdrową żywność i~gigantyczne sałatki zamiast worków hamburgerów i~kwarty lodów. 

-- Są prawie tej samej dystrybucji co Webblies! Spójrz na to \ldots  -- powiększyła przewijaną listę adresów IP, po czym wyjęła kolejną tabelę, majstrując przy ich kolejności sortowania. -- Spójrz! Ci Pinkertonowie walczą z~netblocka, który jest w~promieniu 200 metrów od tych Webblies! Są sąsiadami! Och, to jest \textit{cholernie dziwne}.

To była prawda. Connor stworzył szybki skrypt, aby znaleźć i~sparować graczy, którzy byli blisko siebie fizycznie i~spróbować pokazać na mapie, jeżeli były dostępne. Przeważnie nie były -- próbował wcześniej tropić te szczury, próbował zobaczyć, gdzie mieszkają, ale skończyło się ślepym zaułkiem. Nie mieszkali przy ulicach, mieszkali w~nielegalnych squatach, dzielnicach nędzy w~strefach slumsów świata. Najlepsze, co mógł zrobić, to zdjęcia tych labiryntów sprzed miesiąca, ukazujące góry tlących się śmieci, otwarte toksyczne kanały ściekowe, zagrody dla zwierząt gospodarskich \ldots  Connor czuł, że powinien odwiedzić jedno z~tych miejsc, wywieźć grupę szczurów do centrum dowodzenia w~firmowym odrzutowcu, umieścić ich w~laboratorium, przestudiować i~dowiedzieć się, jak je eksterminować.

Ponieważ był jeden wykres, którego Connor nie musiał wczytywać, wykres pokazujący ogólną stabilność ekonomii gry: jego fingerpitzengefuhl wypełniało go. Ekonomia gry była \textit{zepsuta}.

-- OK ludzie, jest tu wiele do zrobienia. Nikt inny nie odradza się na tym serwerze. Stwórz nową instancję dla Caverns, aby prawdziwi gracze, którzy uderzą, nie musieli przedzierać się przez ten bałagan. Zdobądźcie wszystkie te konta i~zamroźcie ich aktywa. 

Esteban, który kierował obsługą klienta, jęknął.

-- \textit{Wiesz}, że w~większości są zhackowane -- powiedział. -- Są ich setki! Będziemy rozplątywać aktywa przez \textit{miesiące}.

Connor o tym wiedział. Prawdziwi gracze, których konta zostały skradzione przez walczące klany oszustów z~trzeciego świata, nie zasługiwali na zamrożenie swoich aktywów. Co więcej, byłoby ich wielu, których aktywa były częścią większego banku gildii, który mógł mieć bogactwo dziesiątek lub setek graczy. Oczywiście Źli Faceci wiedzieli o tym i~polegali na tym, wiedzieli, że to sprawi, że gracze będą ostrożni i~wolni, gdy przyjdzie czas na zamknięcie kont, których używali do przemycania ich nielegalnego bogactwa.

Nawiązał kontakt wzrokowy z~Billem, szefem ochrony. Zastanawiali się, czy warto wciągnąć część budżetu Connora do działu bezpieczeństwa, aby opracować oprogramowanie kryminalistyczne, które wykryje historie transakcji skradzionych kont i~ustali, jakie aktywa posiadał oryginalny gracz i~gdzie brudne pieniądze trafiły po tym, jak opuściły jego konto. Connor nienawidził rozstawać się z~budżetem, zwłaszcza gdy dotyczyło to Billa, który był nadętym dupkiem, który lubił zachowywać się, jakby był jakimś supercyberpolicjantem, a nie wychwalanym administratorem systemów.

Ale czasami trzeba było przełknąć gorzką pigułkę. 

-- Zajmiemy się tym -- powiedział. -- Prawda, Bill? 

Szef ochrony kiwnął głową i~zaczął walić w~klawiaturę, bez wątpienia zatrudniając grupę swoich starych kumpli hakerów, aby weszli na pokład za najwyższego pieniądze i~napisali kod.

-- Tak -- dodał Bill. -- Nie martw się o to, mamy to obstawione.

Jeden po drugim, walczący znikali, gdy zamykano i~zamrażano ich konta. Niektórzy żołnierze pojawili się ponownie w~nowej instancji -- równoległym wszechświecie zawierającym identyczny loch, ale nie z~tymi samymi graczami -- używając nowych awatarów, ale mogli powiedzieć, kim byli, ponieważ pochodzili z~tych samych adresów IP, co wyrzucone konta. 

-- To świetnie -- powiedział Connor. -- Jeśli tak dalej utrzymają, do końca dnia zniszczymy wszystkie ich konta.

Ale Pinkertonowie i~Webblies musieli pomyśleć o tym samym, ponieważ logowanie spadło prawie do zera, a potem do zera. Ekrany przesunęły się, odgłosy jedzenia zaczęły się od nowa i~Connor wrócił do swoich wykresów ekonomicznych. Jak czuł, cena aktywów, waluty i~instrumentów pochodnych zwariowała. Rynek jakoś wiedział, kiedy pojawiły się kłopoty w~Gold Farmer Land, i~zaczął huśtać się w~oczekiwaniu, że ceny towarów wkrótce się zmienią.

Własne aktywa Connora spadły o 18 procent w~ciągu 25 minut, co kosztowało go 321 498,18 dolarów.

Otworzył czat z~Billem.

\noindent {\textgreater} te rzeczy, które zlecasz moim budżetem

\noindent {\textgreater} tak?

\noindent {\textgreater} chcę tego użyć, aby wytropić każdego farmera złota i~wyrzucić go z~gry

\noindent {\textgreater} co?

\noindent {\textgreater} będzie tam, w~historii transakcji. jakiś odcisk palca w~stylu gry i~wydatkach, który pozwoli nam automatycznie wykryć farmerów i~ich wyrzucić. będziemy mieć idealną, kontrolowaną, wolną od farmerów gospodarkę. pierwsza w~swoim rodzaju

\noindent {\textgreater} Connor każdy złożony ekosystem ma pasożyty.

\noindent {\textgreater} nie ten

\noindent {\textgreater} to nie zadziała

\noindent {\textgreater} chcesz się założyć? niech będzie 10k\$. dam ci 2-1

\bigskip
\threeast

Ta scena jest poświęcona The Tattered Cover, legendarnej niezależnej księgarni w~Denver. Na The Tattered Cover trafiłem zupełnie przypadkowo: Alice i~ja właśnie wylądowaliśmy w~Denver, przyjeżdżając z~Londynu, było wcześnie i~zimno i~potrzebowaliśmy kawy. Jeździliśmy bezcelowo w~kółko wypożyczonym samochodem i~właśnie wtedy zauważyłem to, znak Tattered Cover. Coś w~tym zamrowiło mi w~tyłomózgowiu -- wiedziałem, że słyszałem o tym miejscu. Zatrzymaliśmy się (zdobyliśmy kawę) i~weszliśmy do sklepu -- cudownej krainy z~ciemnego drewna, przytulnych kącików do czytania i~wielu mil półek z~książkami.


\href{https://www.tatteredcover.com/search/author/%22Doctorow%2C%20Cory%22}{\textit{The Tattered Cover} 1628 16th St., Denver, CO USA 80202 +1 303 436 1070} 

\bigskip
\threeast


Ashok prowadził swój śliczny rower przez wąskie uliczki Dharavi, jego latarka czołowa przecinała noc. Matka Yasmin byłaby sztywna ze zmartwień i~gniewu i~prawdopodobnie by ją zbiła, ale to było OK. Ona i~Ashok siedzieli godzinami w~tej szopie w~studio, rozmawiając o tym, próbując wcielić się w~jej pomysł, a on zostawił długie, szczegółowe wiadomości dla Big Sister Nor, zanim zabrał ją z~powrotem na swój rower.

Yasmin klepała go w~ramię na każdym skrzyżowaniu, pokazując mu, w~którą stronę skręcić. Wkrótce byli już prawie w~domu jej rodziny i~krzyknęła, by się zatrzymał, krzycząc przez hełm. Zgasił silnik i~reflektor, a jej tyłek w~końcu przestał wibrować, jej nogi narzekały na godziny, które spędziła, trzymając motocykl wewnętrzną stroną ud. Zeskoczyła chwiejnie z~roweru i~podniosła ręce do kasku.

Jej ręce były na hełmie, kiedy usłyszała głosy.

-- Czy to ona?

-- Nie umiem powiedzieć.

Szeptali głośno, a kratki nad nausznikami hełmu pozwoliły jej usłyszeć dźwięk, jakby dobiegał tuż obok niej. Mocno położyła dłoń na ramieniu Ashoka i~ścisnęła.

-- To ona. -- Głos należał do Mali, twardy.

Yasmin puściła ramię Ashoka i~położyła dłoń na linach mocujących lathi do motocykla, podczas gdy jej wolna ręka przeniosła się na wizjer hełmu, machając nim w~górę. Założyła hidżab na szyję i~teraz była zadowolona, że to zrobiła, ponieważ miała całkiem dobrą widoczność. Minęło sporo czasu, odkąd walczyła fizycznie, ale dobrze rozumiała zasady, znała swoją taktykę.

Lathi była naprawdę dobrze zakotwiczona -- Ashok nie chciał, żeby odleciała, kiedy jechali autostradą -- a teraz opuściła drugą rękę, by pracować na ślepo, nie spuszczając oczu z~otaczających ją cieni, nasłuchując kroków.

-- A co z~mężczyzną?

-- On też -- powiedziała Mala.

A potem zaatakowali, cała ich armia, wychodząc z~cieni wokół nich. 

-- UCIEKAJ! -- krzyknęła do Ashoka, próbując powstrzymać go przed zsiadaniem z~roweru, ale on wstał, wyprostował ramiona i~odwrócił się od niej, w~stronę szarżujących go żołnierzy. 

Kamień lub grudka cementu odbiły się od jej hełmu, wydając dźwięk przypominający upadający na podłogę garnek, a teraz szarpnęła z~całej siły, jak tylko mogła, lathi, aż w~końcu odskoczyła, stalowe haki na czubkach Kable bungee skręciły się i~uderzyły boleśnie w~jej ręce. Ledwo zauważyła, wirując z~dwumetrowym kijem trzymanym nad głową jak kij do krykieta.

I szybko cofnęła.

Najbliższym jej chłopcem był Sushant. Sushant, która tego popołudnia opowiadał o tym, jak bardzo pragnął przyłączyć się do jej sprawy. Jego twarz była maską przerażenia w~słabym świetle sączącym się z~otaczających ich domów. Stalowy czubek drżał na jej ramieniu, gdy drgały jej nadgarstki. Wszystko, co musiałaby zrobić, to zamachnąć się, pozwolić długiemu drążkowi i~jego stalowej końcówce gwizdnąć w~powietrzu z~całą siłą trzaskającego bicza na końcu lathi, a ona rozwaliłaby głowę biednej Sushant.

Czemu nie? W końcu po to była tu armia Mali.

Cała ta myśl w~mgnieniu oka, tak szybko, że nawet nie zauważyła, że tak pomyślała, ale nie machnęła lathi w~powietrzu w~kierunku głowy Sushant. Zamiast tego zamiotła nim u jego stóp, machając tak, że po prostu odrzuciła go do tyłu, wlatującego na dwóch kolejnych żołnierzy za nim, chłopców, którzy kiedyś przyjmowali od niej rozkazy.

-- Wycofać się! -- warknęła głosem rozkazującym i~machnęła lathi do tyłu jak miotłą w~kierunku stóp armii. 

Jednogłośnie cofnęli się o gigantyczny krok, z~oczami oszalałymi i~przewracającymi się w~słabym świetle. Sushant płakał. Słyszała, jak łamie się kość, kiedy czubek lathi zetknął się z~jego kostką. Trzymał się za ramiona dwóch żołnierzy, których przewrócił, a oni starali się utrzymać go w~pozycji pionowej.

Nikt nic nie powiedział i~był tylko zbiorowy oddech Dharavi, tysiące i~tysiące klatek piersiowych unoszących się i~opadających zgodnie, wdychając wzajemnie powietrze, wdychając smród garbarzy i~palący smród z~fabryk barwników i~żądło plastikowego dymu.

Wtedy Mala wystąpiła naprzód. W ręku trzymała \ldots  co? Butelka?

Butelka. Z końca zwisała tłusta szmata. Bomba benzynowa. 

-- Mała! -- powiedziała i~usłyszała szok we własnym głosie. -- Spalisz całe Dharavi! -- Był to ton głosu, którego używasz, krzycząc do słuchawek do kogoś z~gildii, który miał skończyć imprezę, przypadkowo drażniąc jakiegoś gigantycznego szefa. Ton, który mówił: \textit{Jesteś idiotą, przestań}. 

To był niewłaściwy ton do użycia z~Malą. Zesztywniała, a jej druga ręka pracowała za kierownicą jednorazowej zapalniczki -- \textit{snzz snzz}.

Znowu poruszyła się, zanim pomyślała, dwa biegnące kroki, podczas gdy uniosła lathi przez ramię, czując, jak uderza o coś za jej plecami, gdy kroi, a potem ponownie wcina w~dół, tym dzikim, tnącym łukiem, w~dół na Mali chude nogi, machając nimi z~całej siły swojego ciała, a Mala odskoczyła do tyłu, odskoczyła od lathi, potknęła się, przewróciła się do tyłu \ldots 

 \ldots  i~lathi \textit{dotknęło}, solidny cios, który wydał dźwięk, jakby nóż rzeźnika oddzierał głowę kozy od jej szyi, a krzyk Mali był tak straszny, że faktycznie sprowadził ludzi do ich okien (normalnie wrzask w~nocy sprawiłby, że trzymali się z~daleka od tego). Z jej nogi wystawała kość, błyszcząca wśród krwi, która tryskała z~rany.

I wciąż miała bombę benzynową, wciąż miała zapalniczkę, a teraz zapalniczka była zapalona. Yasmin cofnęła nogę do kopnięcia piłkarza, wiedząc, że dobrym kopnięciem może okaleczyć rękę Mali, kończąc karierę generała Robotwallah.

Potem przypomniała sobie głos, który gonił się wokół jej głowy, gdy cofała się do kopniaka:

\textit{Zrób to, zrób to i~zakończ swoje kłopoty. Zrób to, bo zrobiłaby to tobie. Zrób to, ponieważ odstraszy to jej armię od walki z~tobą i~Webblies. Zrób to, bo cię zdradziła. Zrób to, ponieważ zapewni Ci bezpieczeństwo.}

I opuściła stopę i~zamiast tego \textit{skoczyła }na Mali, przyciskając jej ręce do ciała. Płomień zapalniczki lizał jej ramię, paląc ją, a ona go zgasiła. Czuła na gardle oddech Mali, prychający i~bolesny. Chwyciła lewy nadgarstek Mali, potrząsnęła ręką, w~której trzymała bombę, uderzyła nią o ziemię, aż pękła i~wylała śmierdzącą benzynę do rowu biegnącego obok chat. Wstała.

Twarz Mali była poszarzała, nawet w~złym świetle. Zapach krwi i~benzyny były wszędzie.

Yasmin spojrzała na Ashoka. 

-- Musisz zabrać ją do szpitala -- powiedziała.

-- Tak -- powiedział. Trzymał się za bok głowy z~zaciśniętymi oczami. -- Oczywiście, że tak. 

-- Co Ci się stało? 

Wzruszył ramionami. 

-- Zbliżyłem się za bardzo do lathi -- powiedział i~spróbował odważnie się uśmiechnąć. Przypomniała sobie \textit{brzęk}, gdy cofnęła się, by wykonać zamach.

-- Przepraszam -- powiedziała.

Armia Mali stała z~daleka i~wpatrywała się.

-- Idźcie! -- -- powiedziała Yasmin. -- Już. To była katastrofa. To było głupie, złe i~nieszczęśliwe. Nie jestem waszym wrogiem, idioci. Idźcie!

Poszli.

-- Musimy włożyć w~łubki -- powiedział Ashok. -- Zrób też nosze. Nie możesz jej tak ruszać.

Yasmin spojrzała na niego i~uniosła brew.

-- Mój ojciec jest lekarzem -- powiedział.

Yasmin weszła do mieszkania, weszła po schodach. Jej matka usiadła, gdy weszła do pokoju i~otworzyła usta, żeby coś powiedzieć, ale Yasmin podniosła do niej rękę i~cudem się zamknęła. Yasmin rozejrzała się po pokoju, wzięła krzesło stojące w~rogu, naręcze szmat z~tobołka, którego używali do utrzymywania pokoju w~czystości, i~wyszła bez słowa.

Ashok rozbił krzesło na kawałki, rozbijając je o pobliską ścianę. To była tania rzecz i~szybko się rozpadła. Yasmin uklękła przy Mali i~wzięła ją za rękę. Jej oddech był płytki, ciężki.

Mala lekko ścisnęła jej rękę. Potem otworzyła oczy i~rozejrzała się, zdezorientowana. Jej oczy spoczęły na Yasmin. Spojrzały na siebie. Mala próbowała cofnąć rękę. Yasmin nie puściła. Ręka była silna, zwinna. Rozwaliła niezliczoną ilość zombie i~potworów.

Mala przestała walczyć, zamknęła oczy. Ashok przyniósł łubki i~szmaty i~przykucnął obok nich.

Tuż przed tym, jak zaczął nad nią pracować, Mala coś powiedziała. Yasmin nie mogła tego do końca zrozumieć, ale pomyślała, że może to być: \textit{Wybacz mi}. 

\bigskip
\threeast

Ta scena jest dedykowana Hudson Booksellers, księgarzom, którzy są praktycznie na każdym lotnisku w~USA. Większość stoisk Hudson ma tylko kilka tytułów (choć często są one zaskakująco zróżnicowane), ale te duże, takie jak ten w~terminalu AA w~O'Hare w~Chicago, są równie dobre, jak każdy sklep w~sąsiedztwie. Potrzeba czegoś wyjątkowego, aby nadać lotnisku osobisty charakter, a Hudson's uratował mi umysł na więcej niż jednym długim postoju w~Chicago.

\href{https://www.hudsongroup.com/}{Hudson Booksellers}

Wei-Dong nie mógł wyrzucić Lu z~głowy. Barbarzyńca dźgnął dynię, a on zdecydował, że miecz utkwi na trzy sekundy, a następnie odtworzy standardowy dźwięk zgniatania z~karty dźwiękowej. Nie mógł wyrzucić Lu z~głowy. Kieszonkowiec próbował ukraść pióro feniksa, a on kazał feniksowi odwrócić się i~przekląć gracza, plując płomieniami, krzycząc na niego po mandaryńsku, a jego głos przefiltrowany przez fazer brzmiał jak ptak. Nie mógł wyrzucić Lu z~głowy. Przywódca hordy zombie próbował wedrzeć się do zabarykadowanego mini-centrum handlowego, próbując przejść przez szyld ,,Koniec biznesu'', który był tylko teksturą odwzorowaną na zewnętrznej powierzchni, która nie miała wnętrza. Wei-Dongowi spodobała się pomysłowość tego faceta, więc zdecydował, że rozbicie go zajmie 3000 minut zombie, a kiedy rozbije, skierowałby się do wnętrza sklepu sportowego, gdzie były fajne pałki, kusze i~maczety.

I nie mógł wyrzucić Lu ze swojego umysłu.

Zawsze lubił Lu. Ze wszystkich facetów Lu był tym, który naprawdę \textit{wciągnął} się w~gry. Kochał nie tylko pieniądze czy przyjaźń: uwielbiał \textit{grać}. Uwielbiał rozwiązywać łamigłówki, eliminować wielkich bossów podczas wielkiego rajdu, odblokowywać nowe lądy i~osiągnięcia dla swoich avs. Czasami, gdy Wei-Dong pracował na długie zmiany, podejmując drobne decyzje dotyczące gry, myślał o tym, o ile lepiej byłoby grać dzięki pracy, którą wykonywał, i~wyobrażał sobie, że Lu zaaprobuje artyzm. Fajnie było być po drugiej stronie gry, robiąc zabawę, zamiast po prostu ją konsumować. Praca była długa, ciężka, nie opłacała się dobrze, ale on był \textit{częścią show}.

Ale to już nie było przedstawienie.

Jego telefon zaczął wibrować w~kieszeni. Wyjął go, spojrzał na twarz, położył na biurku. To była jego mama. Ustąpił i~dał jej swój nowy numer, kiedy skończył 18 lat, usprawiedliwiając się tym, że jest teraz dorosły i~nie mogła go wytropić i~ściągnąć z~powrotem. Ale tak naprawdę to dlatego, że nie mógł stawić czoła samotnemu spędzeniu swoich osiemnastych urodzin. Ale nie chciał teraz z~nią rozmawiać. Przełączył ją do poczty głosowej.

Znowu zadzwoniła. Telefon zabrzęczał. Przerzucił na pocztę głosową. Sekundę później telefon znów zabrzęczał. Sięgnął, żeby go wyłączyć, po czym zatrzymał się i~odebrał.

-- Cześć mamo? 

-- Leonard -- powiedziała. -- To twój ojciec.

-- Co? 

Wzięła głęboki oddech, wypuściła powietrze. 

-- Atak serca. Duży. Zabrali go do \ldots  -- Urwała, wzięła głęboki oddech. -- Zabrali go do Hoag Center. Jest na Intensywnej Terapii. Mówią, że to najlepsze \ldots  -- Kolejny oddech. -- Podobno jest najlepszy.

Żołądek Wei-Donga opadł i~zapadł się gdzieś pod jego krzesłem. Miał wrażenie, że jego głowa może odlecieć. 

-- Kiedy? 

-- Wczoraj -- powiedziała.

Nic nie powiedział. \textit{Wczoraj}? Chciał to wykrzyczeć. Jego ojciec był w~szpitalu od \textit{wczoraj }i nikt mu o tym nie powiedział?

-- Och, Leonardzie -- powiedziała. -- Nie wiedziałam, co robić. Nie rozmawiałeś z~nim, odkąd odszedłeś. I \ldots 

I \ldots ? 

-- Przyjadę i~go odwiedzę -- powiedział. -- Mogę wziąć taksówkę. To zajmie chyba około godziny.

-- Godziny odwiedzin się skończyły -- powiedziała. -- Byłam z~nim cały dzień. Nie jest zbyt przytomny. Ja \ldots  Nie pozwalają ci tam korzystać z~telefonu. Nie na Oddziale.

Przez wiele miesięcy Wei-Dong żył jako dorosły, żył życiem, które określiłby jako idealne, zanim zadzwonił telefon. Znał ciekawych ludzi, jeździł w~ekscytujące miejsca. \textit{Grał w~gry }przez cały dzień, żeby zarobić na życie. Znał tajniki przestrzeni gier.

Teraz zrozumiał, że pod jego satysfakcją kryło się uczucie intensywnej samotności, bulgocząca otchłań rozpaczy, która cuchnęła porażką i~nieszczęściem. Wei-Dong kochał swoich rodziców. Chciał ich aprobaty. Ufał ich osądowi. To dlatego był tak przerażony, kiedy odkrył, że planowali go odesłać. Gdyby nie dbał o nich, nic z~tego nie miałoby znaczenia. Gdzieś w~swojej głowie miał scenkę z~okazji spotkania z~rodzicami, zapraszającą ich do eleganckiej, miejskiej restauracji, może jednego z~tych surowych miejsc w~Echo Park, o których czytał cały czas w~Metroblogach. Odbyliby kulturalną, wyrafinowaną rozmowę o wielu niesamowitych rzeczach, których nauczył się na własną rękę, a jego ojciec musiałby zeskrobać szczękę z~talerza, żeby dokończyć rozmowę. Potem wsiadłby na swój zgrabny skuter Tata, pokryty około tysiącem warstw lakieru na cienkich bambusowych paskach, i~odjechałby, podczas gdy jego rodzice patrzyli na siebie, podziwiając niesamowitego syna, którego spłodzili.

To było głupie, wiedział o tym. Chodziło jednak o to, że zawsze traktował ten czas jako wakacje, małą przerwę w~życiu rodzinnym. Jego poszukiwanie wizji, kiedy odszedł, by stać się mężczyzną. Prawdziwa bar-micwa, taka, która coś znaczy.

Myśl, że może już nigdy więcej nie zobaczyć ojca, nigdy się z~nim nie pogodzić, uderzyła go jak cios, jakby uderzył młotkiem w~gwóźdź i~zamiast tego roztrzaskał sobie rękę.

-- Mamo \ldots  -- Jego głos zabrzmiał zachrypniętym. Odchrząknął. -- Mamo, przyjdę jutro i~do zobaczenia. Wezmę taksówkę.

-- OK, Leonardzie. Myślę, że twój ojciec chciałby cię zobaczyć.

Chciał, żeby powiedziała coś o tym, jak samolubny był, gdy ich zostawił, jakim był złym synem. Chciał, żeby powiedziała coś \textit{niesprawiedliwego}, żeby mógł być zły, zamiast czuć tę okropną, okropną winę.

Ale powiedziała: 

-- Kocham cię, Leonardzie. Nie mogę się doczekać, kiedy cię zobaczę. Tęskniłam za tobą.

Poszedł więc do łóżka z~milionem nienawistnych myśli śpiewających zgodnie w~jego umyśle, i~godzinami leżał w~swoim łóżku w~hotelu, słuchając myśli, krzyczących włóczęgów, bywalców klubów i~ludzi uprawiających seks w~innych pokojach z~muzyką unoszącą się z~okien samochodu przez wiele godzin, a on ledwo zasnął, gdy obudził go budzik. Wziął prysznic, zeskrobał swój mały, puszysty wąsik jednorazową brzytwą, zjadł kanapkę z~masłem orzechowym i~zrobił sobie poczwórne espresso za pomocą prasy ręcznej na nitro, którą kupił za pierwszą wypłatę, wezwał taksówkę i~wyszorował zęby, gdy na to czekał.

Taksówkarz był Chińczykiem i~Wei-Dong poprosił go, w~jego najlepszym mandaryńsku, by zabrał go do Orange County, do domu rodziców. Mężczyzna był wyraźnie rozbawiony młodym białym chłopcem, który mówił po chińsku i~porozmawiali trochę o pogodzie i~ruchu ulicznym, a potem Wei-Dong spał, drzemiąc ze zwiniętą kurtką zamiast poduszką, przesypiając kofeinowe drżenie poczwórnego strzału, gdy wczesny poranek w~Los Angeles wpełzł na 5 Aleję.

Zapłacił taksówkarzowi prawie jednodniową pensję, wyjął klucze z~kieszeni marynarki, poszedł do swojego domu i~wszedł do środka, a jego matka siedziała przy kuchennym stole w~podomce, z~czerwonymi i~spuchniętymi oczami, tylko się gapiąc w~przestrzeń.

Stanął w~drzwiach i~spojrzał na nią, a ona spojrzała na niego, a potem wstała niepewnie, podeszła do niego i~uściskała go mocno i~drżąco, a jego szyja była wilgotna, gdzie spływały po niej łzy.

-- Odszedł -- szepnęła mu do ucha. -- Dzisiaj rano, około 3 nad ranem. Kolejny atak serca. Bardzo szybko. Powiedzieli, że to było praktycznie natychmiastowe. -- Płakała jeszcze trochę.

A Wei-Dong wiedział, że znowu przeprowadzi się do domu.

\bigskip
\threeast

Szpital wypisał Big Sister Nor, The Mighty Krang i~Justbob dwa dni wcześniej, tylko po to, by się ich pozbyć. Po pierwsze, nie zostawali w~swoich pokojach, zamiast tego ciągle zakradli się do szpitalnej stołówki, gdzie zdobyli trzy lub cztery stoły, mozolnie zsuwając je razem, poruszając się o kulach i~wózkach inwalidzkich, a następnie rozkładając komputery, telefony, notatniki, projekty z~frędzlami, maleńkie ołowiane miniatury, które The Mighty Krang zawsze malował za pomocą pięknych pędzli z~wielbłądziej sierści, kartki, kwiaty, czekoladki i~ciastek wysyłanych przez zwolenników Webblies.

Na domiar złego, Big Sister Nor odkryła, że trzy kobiety na jej oddziale to filipińskie pokojówki, które były bite przez swoich pracodawców i~prowadziła spotkania podnoszące świadomość, podczas których uczyła je, jak pisać oficjalne listy ze skargami do Ministerstwa Pracy. Pielęgniarki ich pokochały -- rok wcześniej głosowały w~związku zawodowym -- a administracja szpitala \textit{nienawidziła }rozpalonym do białości upałem tysiąca słońc.

Tak więc niecałe dwa tygodnie po tym, jak zostali pobici prawie na śmierć, Big Sister Nor, The Mighty Krang i~Justbob wyszli, mrugając, w~duszący upał południa w~Singapurze, zawinięci w~bandaże, łubki i~gipsy. Ich ciała były połamane, ale ich duchy były lekkie. Bicie było, no cóż, \textit{wyzwalające}. Po latach życia w~strachu przed skopaniem prawie na śmierć przez zbirów pracujących dla szefów przeszli przez to i~przeżyli. Prosperowali. Ich strach się wypalił.

Kiedy spojrzeli na siebie, z~lepkimi włosami i~twarzami zarumienionymi od parującego gorąca, zaczęli się uśmiechać. Potem chichotać. Potem śmiać się tak głośno i~tak głęboko, jak pozwalały na to ich rany.

Justbob odgarnął jej włosy z~przepaski na oko, która zakrywała zniszczenie jej lewego oka, podrapał się pod gipsem na jej ramieniu i~powiedział: 

-- Powinni byli nas zabić.


\chapter{Ponzi}

Ta scena jest poświęcona Harvard Bookstore, wspaniałej i~eklektycznej księgarni w~sercu jednej z~najwspanialszych, światowej klasy dzielnic księgarskich, odcinka Mass Ave, który biegnie między Harvardem a MIT. Ostatnim razem, gdy odwiedziłem sklep, właśnie dostali maszynę Espresso do drukowania książek na żądanie, która była podłączona do zadziwiającej biblioteki Google zeskanowanych książek z~domeny publicznej i~mogli wydrukować i~oprawić praktycznie każdą wyczerpaną książkę z~całą historię ludzkości za kilka dolarów w~kilka minut. Aby zgłębić niewyobrażalną głębię ludzkiej kreatywności, którą to reprezentowało, sklep miał kogoś, kogo zadaniem było po prostu myszkować i~znajdować dzikie tytuły spoza historii do wydrukowania i~wystawienia na półkach wokół maszyny. Rzadko odczuwałem obecność przyszłości tak silnie jak tamtej nocy.

\href{https://www.harvard.com/}{\textit{Harvard Bookstore}: 1256 Massachusetts Avenue, Cambridge MA 02138 USA, +1 (617) 661-1515} 

\bigskip
\threeast

Wnętrze kontenera było dużo gorsze, niż przewidywał Wei-Dong. Kiedy zdecydował się przemycić do Chin, dużo czytał na ten temat, zaczynając od wyszukiwań dotyczących handlu ludźmi -- które były opowieściami grozy o 55 stopniach Celsjusza w~południe w~pudełku do pieczenia, wypełnionym trzydziestoma innymi osobami -- a potem do ruchu zrównoważonego budownictwa mieszkaniowego, w~którym architekci prześcigali się w~prostych i~eleganckich modernizacjach kontenerów w~urocze małe mieszkania.

Dlaczego nikt nie pomyślał o połączeniu tych dwóch dyscyplin, było poza nim. Jeśli zamierzasz przemycać ludzi przez ocean, dlaczego nie skorzystać z~uroczego małego zestawu, aby przekształcić stalowe pudełko w~przytulny mały kamper? Czy czegoś mu brakowało?

Nie. 

Poza tym, że wszyscy przemytnicy byli kryminalistami, nie mógł znaleźć żadnego powodu, dla którego przemytnik nie mógłby cieszyć się dziesięcioma dniami na morzu w~wielkim stylu. Zwłaszcza jeśli przemytnik był teraz współwłaścicielem ogromnej firmy spedycyjnej i~logistycznej z~siedzibą w~Los Angeles, która zarządzała magazynem i~posiadała przepustkę Homeland Security do portu.

Prace nad pojemnikiem zajęły Wei-Dongowi trzy tygodnie. Zestaw do konwersji wysyłkowej twierdził, że może być złożony w~terenie przez dwóch niewykwalifikowanych robotników w~obszarze katastrofy za pomocą narzędzi ręcznych w~ciągu dwóch dni. Zajęło mu to dwa tygodnie, co było trochę zawstydzające, ponieważ zawsze określał się jako ,,wykwalifikowany'' (ale proszę bardzo).

W końcu miał specjalne potrzeby. Czytał na temat ochrony portu i~wiedział, że będą czujniki wyszukujące charakterystyczny koktajl gazów wydzielanych przez ludzi: aceton, izopren, alfa-pinen i~wiele innych egzotyczne spaliny wydzielane z~każdym oddechem w~określonej proporcji. Zbudował więc mały pojemnik wewnątrz kontenera, hermetyczne pudełko, które miało przechowywać jego gazy, dopóki nie znajdą się na morzu, pomyślał, że przeżyje w~nim dobre dziesięć godzin, zanim zużyje całe powietrze, pod warunkiem, żeby za dużo nie ćwiczyć. Policjanci portowi mogliby zbadać jego kontener, ile tylko chcieli, i~dostaliby normalną mieszankę lotnych substancji gotujących się z~farby po wewnętrznej stronie kontenera transportowego, nieskażonych ludzkim wydechem. Pod warunkiem, że faktycznie nie otworzyli jego pojemnika, a potem nie zainteresowali się hermetycznie zamkniętym pudełkiem w~środku,

W każdym razie, zanim skończył, miał naprawdę zajebiste małe gniazdo. Załadował Huawei swojego taty meblami z~IKEA na całe mieszkanie, a potem zhakował je, przybił gwoździami, przykręcił i~przykleił do wnętrza kontenera, tworząc przytulną kabinę na statku z~łóżkiem typu king-size, toaletą chemiczną, mikrofalówką, biurkiem i~kącikiem gry. Gdy znaleźli się na morzu, mógł otworzyć swój mały właz i~wyciągnąć odbiornik Wi-Fi -- podłączenie się do pokładowego Wi-Fi używanego przez załogę byłoby proste, ponieważ nie poświęcali dużo energii na trzymanie się z~dala od freeloaderów, kiedy byli na środku oceanu -- i~panelu słonecznego. Miał kilka bardzo długich drutów dla obu, ponieważ ustawił listy przewozowe tak, aby jego kontener znajdował się głęboko w~środku stosu, obok jednej ze szczelin, które biegły między nimi, a nie na zewnętrznej krawędzi: jeden procent kosztów wysyłki to kontenery, które wylądowały na dnie morza, wyrzucone za burtę na wzburzonych wodach, a on chciał zminimalizować ryzyko śmierci, kiedy jego kontener imploduje pod ciśnieniem setek atmosfer głębokiego oceanu.

Spadek był wygodniejszy niż przypuszczał. Udało mu się wejść na stronę Huawei i~zamówić dziesięć baterii do ich całkowicie elektrycznych samochodów typu runabout, z~których każdy mógł przejechać 120 kilometrów. Dostarczono je bezpośrednio na molo, na którym czekał jego kontener transportowy (rozważał możliwość, że power-packi zostały wysłane do Ameryki w~tym samym kontenerze, w~którym je zainstalował, ale wiedział, że szanse na to były astronomiczne, co sekundę do wybrzeży Ameryki przybywało \textit{wiele }kontenerów). Ułożył je równo na jednym końcu pojemnika, z~przyklejonym listem przewozowym z~kodem kreskowym, który stwierdzał, że zostały zwrócone jako wadliwe. Przybyły naładowane i~był prawie pewien, że będzie w~stanie utrzymać je naładowane między portem Los Angeles a Shenzhen, używając paneli słonecznych, które zamierzał rozłożyć na szczycie stosu kontenerów. Przetestował arkusze fotowoltaiczne na Huaweiu swojego ojca i~stwierdził, że może je w~pełni naładować w~sześć godzin, i~obliczył, że powinien być w~stanie uruchomić swój laptop, klimatyzator i~pompy wody przez cztery dni na każdej baterii. 16 dni mocy wystarczyłoby na ukończenie przeprawy, nawet w~przypadku złej pogody, ale dobrze było wiedzieć, że doładowanie jest opcją.


Woda zmusiła go do pomyślenia. Ludzie zużywają \textit{dużo }wody i~chociaż w~jego kapsule kosmicznej było dużo miejsca -- jak zaczął myśleć o pojemniku -- pomyślał, że musi istnieć lepszy sposób na zaspokojenie swoich potrzeb na płyny podczas podróży niż po prostu wrzucenie trzech lub czterech tony wody do pudełka. Pogrążył się w~myślach, kiedy zdał sobie sprawę, że wszystkie panele słoneczne są wodoodporne i~można je łatwo przekształcić w~lejek, który prowadzi w odcinek rury PCV, którą mógłby wciągnąć ze szczytu stosu kontenerów do kapsuły kosmicznej, gdzie kilka sterylnych pustych bębnów przetrzymałoby wodę, dopóki nie był gotowy do jej wypicia lub prysznica. Potem jego ścieki można było po prostu wypompować na pokład statku, gdzie zostałyby spłukane za burtę z~całą pozostałą wodą, która spadła na statek. Jeśli spakował wystarczająco dużo wody, by mógł brać prysznic i~gotować przez tydzień, istniało duże prawdopodobieństwo, że nadejdzie ulewa i~zostanie ona uzupełniona, a jeśli nie, będzie mógł racjonować swoją pozostałą wodę i~przyjechać do Chin trochę bardziej śmierdzący niż na początku.

Kochał te rzeczy. Planowanie było świetną zabawą, prawdziwym googlefestem interesujących HOWTO i~porad. Wiele części problemu samowystarczalności na morzu rozważano już wcześniej, chociaż nikt nie poświęcał zbyt wiele uwagi problemowi podróżowania w~stylu i~tajemnicy w~kontenerze. Był pionierem. Robił notatki i~planował je opublikować po zakończeniu przygody.

Oczywiście nie wspomniałby o \textit{przyczynie}, dla której musiał przemycić się do Chin, zamiast po prostu ubiegać się o wizę turystyczną.

Matka Wei-Dong nie wiedziała, co sądzić o swoim synu. Śmierć jego ojca wstrząsnęła nią i~przez połowę czasu zdawała się mówić do niego zza zasłony z~gazy. Znalazł antydepresanty przepisane przez jej lekarza, sprawdził skutki uboczne i~zdecydował, że jego matka prawdopodobnie nie będzie w~stanie zauważyć, że kombinuje coś dziwnego. Przeważnie wydawała się po prostu odczuwać ulgę, że ma go w~domu i~że jest pilnie zaangażowany w~rodzinny biznes. Nawet nie mrugnęła, kiedy powiedział jej, że wybiera się w~podróż wzdłuż wybrzeża, na miłą długą przejażdżkę na Alaskę z~minimalnym dostępem do sieci, aktywnością telefoniczną i~tak dalej.

Ostatnim ładunkiem, jaki trafił do kosmicznej kapsuły, były trzy kartonowe pudła, wystarczająco małe, by można je było załadować do bagażnika Huawei, którego pozostawił na długoterminowym parkingu i~zamknął podwójnie po załadowaniu. Każda z~nich była potrójnie owinięta w~wodoodporny plastik, a w~środku znajdowało się dwadzieścia pięć tysięcy przedpłaconych kart do różnych gier MMO. Wartość nominalna tych kart przekraczała 200 000 dolarów, chociaż żadne pieniądze nie zmieniły właściciela, gdy zebrał je, w~paczkach po kilkaset, w~chińskich sklepach w~całym Los Angeles i~hrabstwie Orange. Zebranie całego ładunku zajęło trzy dni i~jak dotąd była to najtrudniejsza część triku. Karty były częścią regularnej umowy, w~ramach której wielcy farmerzy złota korzystali z~sieci zagranicznych sprzedawców detalicznych, aby przejąć czas gry w~USA i~wysłać go z~powrotem do Chin, aby ich pracownicy używali amerykańskich serwerów.

Technicznie oznaczało to, że wszyscy sprzedawcy, których odwiedzał, należeli do ogromnego podziemia przestępczego, ale żaden z~nich nie wydawał się aż tak niebezpieczny. Mimo to, jeśli któryś z~nich był podejrzliwy wobec białego dzieciaka z~kiepskim mandaryńskim akcentem, który regularnie odbierał towar, kto wiedział, co może się stać?

Ale tak się nie stało. Teraz miał cenny ładunek, pudła z~niemożliwymi do wyśledzenia, niesekwencyjnymi kredytami w~grze, które pozwoliłyby mu zarobić złoto z~gry. To wszystko było takie dziwne, teraz kiedy siedział tam na swojej czerwonej skórzanej sofie z~Ikei, popijając mrożoną herbatę, przeżuwając batonik i~kontemplując swój łup.

Pod zdrapkami karty te zawierały unikalne numery wygenerowane przez duży generator liczb losowych na serwerze w~Ameryce, następnie wydrukowane w~Chinach, a następnie wysłane z~powrotem do Ameryki, teraz ponownie przeznaczone dla Chin. Pomyślał, o ile prościej byłoby najpierw wymyślić losowe liczby w~Chinach, zachichotał i~położył nogi na pudełkach.

Oczywiście, gdyby to zrobili, nie miałby żadnego usprawiedliwienia, żeby zbudować kosmiczną kapsułę i~przemycić się do Chin.

\bigskip
\threeast

Ta scena jest poświęcona londyńskim Clerkenwell Tales, znajdującym się tuż za rogiem mojego biura w~Clerkenwell, cudownej i~eklektycznej dzielnicy w~centrum Londynu. Peter Ho, właściciel, jest weteranem Waterstone's i~otworzył dokładnie ten rodzaj małego, fachowo wyselekcjonowanego sklepu osiedlowego, który każda osoba zajmująca się książkami pragnie mieć w~pobliżu. Peter stara się zaopatrywać w~małe, ręcznie robione wydania z~lokalnych drukarni, w~wyniku czego zawsze wpadam, żeby się przywitać podczas przerwy na lunch i~wychodzę z~naręczem znakomitych i~wspaniałych książek. To śmiertelne. W dobry sposób.

\href{https://www.clerkenwell-tales.co.uk/}{\textit{Clerkenwell Tales}: 30 Exmouth Market EC1R 4QE Londyn +44 (0)20 7713 8135} 

\bigskip
\threeast

Ashok najlepiej myślał na papierze, na dużych arkuszach. Wiedział, że to śmieszne. Mądrą rzeczą byłoby zachowanie wszystkich plików w~postaci cyfrowej, zaszyfrowanych na współdzielonym dysku w~sieci, gdzie wszystkie Webblies mogłyby się do nich dostać. Ale liczby nabierały o wiele więcej sensu, gdy zostały napisane porządnie na papierze flipchartowym i~przypięte na ścianach jego ,,pokoju wojennego'', pokoju na zapleczu kawiarni pani Dibyendu, wynajętego przez Malę z~zarobków armii od pana Bannerjee.

O tak, Mala wciąż pobierała pensję od pana Bannerjee, a jej żołnierze nadal walczyli w~misjach, na które ich wysłał. Ale potem, w~swoim czasie, walczyli na własnych misjach, w~sklepie pani Dibyendu. Pani Dibyendu serdecznie ich witała, wdzięczna za interesy w~jej sklepie, któremu groził upadek i~bankructwo. Siostrzeniec-idiota został odesłany do Uttar Pradesh, aby zamieszkał z~rodzicami, kuśtykając do domu z~ogonem między nogami i~zostawiając panią Dibyendu, aby sama zajmowała się pustym sklepem.

Pani Dibyendu nie przeszkadzały duże arkusze papieru. \textit{Kochała }Ashoka, elegancko ubranego i~dobrze wystrojonego, i~wyraźnie myślała, że pomiędzy nim a Yasmin coś się dzieje. Ashok próbował delikatnie wyprowadzić ją z~błędu, ale nic z~tego sobie nie robiła. Przynosiła mu słodki czaj przez cały dzień i~całą noc, gdy pracował nad swoimi kartami.

-- Ashok -- zawołała Mala, kuśtykając w~jego stronę przez pustą kawiarnię, opierając się o kozłowe stoliki podtrzymujące długie rzędy dyszących komputerów.

Wstał od stołu, ocierając dłonią czaj z~brody, wycierając dłoń o spodnie. Mala go denerwowała. Odwiedził ją w~szpitalu z~Yasmin i~usiadł przy jej łóżku, podczas gdy ona nie chciała na żadne z~nich spojrzeć. Odebrał ją, kiedy została zwolniona, a ona spojrzała na niego płonącym spojrzeniem, jak święta kobieta, skinęła mu raz głową i~zapytała, jak jej armia może pomóc.

-- Mala -- powiedział. -- Jesteś wcześnie. 

-- Niewiele dzisiaj bitew -- powiedziała, wzruszając ramionami. -- Walka z~Webblies jest jak walka z~dziećmi. Źle zorganizowanymi dziećmi. Przed obiadem pokonaliśmy ponad dwadzieścia miejsc pracy i~musiałam zarządzić przerwę. Wojsko się nudziło. Zadałam im treningi, prowadzenie bitew przeciwko sobie.

-- Jesteś dowódcą, generale Robotwallah, jestem pewien, że wiesz najlepiej.

Miała bardzo ładny uśmiech, choć rzadko można go było zobaczyć. Przeważnie widziane były jej brzydkie uśmiechy, uśmiechy, które wydawały się mieć zbyt wiele ostrych zębów. Ale jej piękny uśmiech był jak słońce. To zmieniało cały pokój, rozświetlało serce. Rozumiał, jak taka dziewczyna może dowodzić armią. Przez chwilę wpatrywał się w~ten piękny uśmiech, a jego język w~ustach wysechł i~zgęstniał.

-- Chcę z~tobą porozmawiać, Ashok. Siedzisz tutaj ze swoim papierem i~swoimi liczbami i~ciągle mówisz nam, żebyśmy poczekali, poczekali trochę, a wszystko wyjaśnisz. Minęły miesiące, Ashok, a ty wciąż mówisz: czekaj, wyjaśnię. Jestem zmęczona czekaniem. Armia jest zmęczona czekaniem. Bycie podwójnymi agentami było zabawne przez chwilę i~fajnie jest walczyć z~prawdziwymi Pinkertonami w~nocy, ale nie będą czekać wiecznie.

Ashok wyciągnął ręce w~uspokajającym geście, który często działał na Malę. Musiała wiedzieć, że była szefem. 

-- Słuchaj, to nie jest prosta sprawa. Jeśli mamy zmierzyć się z~czterema wirtualnymi światami naraz, wszystko musi działać jak w~zegarku, każdy element odpalający po drugim. W międzyczasie \ldots 

Pomachała mu lekceważąco. 

-- W międzyczasie Bannerjee staje się coraz bardziej podejrzliwy. Ten człowiek jest idiotą, ale nie kretynem. W końcu zorientuje się, że coś jest nie tak. Albo jego panowie. I wtedy \ldots 

-- I wtedy będziemy musieli go udobruchać lub zmylić. Generale, to jest gra zaufania, oszustwo, działająca na czterech wirtualnych światach i~dwudziestu prawdziwych narodach, z~setkami wspólników. Gry w~zaufanie wymagają planowania i~sprytu. Nie wystarczy wejść, strzelając z~pistoletów \ldots 

-- Myślisz, że nie rozumiemy planowania? Myślisz, że nie rozumiemy \textit{przebiegłości}? Ashok, nigdy nie walczyłeś. Powinieneś walczyć. Pomogłoby ci to zrozumieć ten biznes, w~który się wpakowałaś. Myślisz, że jesteśmy zbirami, idiotami z~mięśniami. Prowadzenie bitwy wymaga tyle samo umiejętności, co wszystko, co robisz  \ldots  nie mam dobrego wykształcenia, jestem tylko dziewczyną z~wioski, jestem tylko szczurem Dharavi, ale jestem \textit{bystra} Ashok i~nie nigdy tego nie zapominaj.

Najgorsze było to, że miała rację. Często \textit{myślał }o niej jako o bandycie. 

-- Mala, chcę grać, ale granie odciąga mnie od planowania.

-- Nie możesz planować, jeśli nie grasz. Jestem generałem i~rozkazuję. Jutro o~10 rano dołączysz do plutonu młodszych na manewrach, będzie walka, trochę teorii, a potem bitwy nadzorowane przez starszy pluton, kiedy przybędzie. To będzie dla ciebie dobre. Będą cię trochę szarpać, bo jesteś nowy, ale to też będzie dobre dla ciebie.

To spojrzenie jej oczu, to ogniste, powiedziało mu, że nie śmiał się nie zgodzić. 

-- Tak, generale -- powiedział.

-- A teraz wyjaśnisz mi tę sprawę. Poznasz mój świat, ja poznam twój.

-- Mala \ldots  

-- Wiem, wiem. Weszłam i~krzyknęłam na ciebie, ponieważ zajęło ci to zbyt dużo czasu, a teraz nalegam, abyś to zrobił szerzej. -- Posłała mu ten uśmiech. Nie była ładna, jej rysy były zbyt ostre jak na ładną, ale była piękna, kiedy się uśmiechała. Będzie łamać serca, kiedy dorośnie. \textit{Jeżeli} dorośnie. 

-- Tak, generale.

-- Czaj! -- zawołała do pani Dibyendu, która szybko go przyniosła, odwracając wzrok od Mali.

-- Dobrze, zacznijmy od podstawowej teorii oszustwa. Kogo najłatwiej oszukać? 

-- Głupca -- powiedziała natychmiast.

-- Źle -- powiedział. -- Głupcy są często podejrzliwi, ponieważ zostali wykorzystani. Najłatwiej oszukać osobę odnoszącą sukcesy, im większy sukces, tym lepiej. Dlaczego tak jest?

-- Mają więcej pieniędzy, więc warto ich oszukać?

Ashok poruszył brodą. 

-- Nie, przepraszam  \ldots  według tego rozumowania powinni być \textit{bardziej }podejrzliwi, a nie mniej.

Mala przesunęła krzesło po podłodze, usiadła i~skrzywiła się do niego. 

-- Poddaję się, powiedz mi. 

-- To dlatego, że jeśli człowiek odnosi sukces w~jednej rzeczy, jest skłonny zakładać, że odniesie sukces we wszystkim. Wierzy, że jest braminem, obdarzonym boską mądrością i~siłą charakteru, aby odnieść sukces. Nie może znieść myśli, że po prostu miał szczęście albo że jego rodzice po prostu mieli szczęście i~zostawili mu kupę rupii. Nie może znieść myśli, że zrozumienie fizyki, komputerów czy aparatów fotograficznych nie czyni go ekspertem od ekonomii, pszczelarstwa czy gotowania.

-- A jego inteligencja i~duma współpracują ze sobą, aby \textit{łatwiej }go oszukać. Jego duma, oczywiście, ale także inteligencja: jest wystarczająco inteligentny, aby zrozumieć, że istnieje wiele sposobów na wzbogacenie. Jeśli opowiesz mu skomplikowaną historię o tym, jak jakiś rynek działa i~może go oszukać, może podążać po nierównym terenie, który mógłby zagubić głupszego człowieka.

-- I jest trzeci powód, dla którego ludzi sukcesu łatwiej jest oszukać niż głupców: boją się, że zostaną pokazani jako głupcy. Kiedy ich oszukasz, możesz ich oszukać ponownie, sprawić, by uwierzyli, że tylko plan się nie powiódł. Nie chcą iść na policję lub powiedzieć swoim przyjaciołom, bo jeśli się rozejdzie, że jakiś potężny i~władczy człowiek został oszukany, może stracić swoją reputację, bez której może nie odzyskać swojej fortuny.

Mala poruszyła brodą. 

-- To wszystko ma sens, jak sądzę.

-- Tak -- powiedział Ashok.

-- Jestem osobą odnoszącą sukcesy i~potężną -- powiedziała. Jej oczy były jak kocie szparki.

-- Jesteś -- powiedział Ashok ostrożniej.

-- Więc łatwiej byłoby mnie oszukać niż kogokolwiek z~głupców z~mojej armii?

Ashok roześmiał się. 

-- Jesteś tak ostra, generale, to cud, że się nie skaleczyłeś. Tak, możliwe, że to wszystko jest gigantycznym potrójnym blefem, którego celem jest oszukanie ciebie. Ale po co bym chciał cię oszukać? Tak bogata, jak twoja armia cię uczyniła, musisz wiedzieć, że mógłbym być tak samo bogaty, pracując jako młodszy wykładowca ekonomii w~IIT. Ale generale, pod koniec dnia albo mi ufasz, albo nie. Nie mogę ci udowodnić, że jesteś wewnątrz planu, a nie jego celem. Jeśli chcesz się wydostać, w~porządku. To zaszkodzi planowi, ale nie będzie jego końcem. Jest w~to zaangażowanych wiele osób.

Mala uśmiechnęła się swoim słonecznym uśmiechem. 

-- Jesteś mądrym człowiekiem -- powiedziała. -- A na razie będę ci ufała. Kontynuuj.

-- Cofnijmy się trochę. Chcesz poznać trochę historii?

-- Czy to pomoże mi zrozumieć, dlaczego tak długo trwa?

-- Tak myślę -- powiedział. -- W każdym razie myślę, że to cholernie dobra historia.

Zrobiła gest i~popijała herbatę z~bardzo wyprostowanymi plecami, z~królewską miną.

-- W latach 30. największe oszustwa nosiły nazwę ,,Wielki Sklep''. Były to małe sztuki sceniczne, w~których była tylko jedna publiczność -- ,,klient'', albo frajer, czyli ofiara. \textit{Wszyscy inni }brali udział w~przedstawieniu. Klient spotykał w~pociągu ,,dublera'', który wyczuwał go, by sprawdzić, czy miał jakieś pieniądze. Czasami dawał mu przedsmak pieniędzy, które miał zarobić, może podzieliliby się tajemniczymi ,,znalezionymi'' pieniędzmi, które zostawił. Takie rzeczy sprawiają, że klient ufa ci bardziej i~również stawia go w~twojej mocy, bo teraz wiesz, że jest skłonny trochę oszukiwać.

-- Kiedy pociąg wjechał do dziwnego miasta i~klient wysiadł, każda osoba, którą spotkał lub z~którą rozmawiał, byłaby częścią kombinacji. Jeżeli klient był dobry w~finansach, potykacz przekazałby go satelicie, ,,człowiekowi w~środku'', który powiedziałby mu o oszustwie, które miał ustawione na wyścigach konnych; jeśli klient był dobry w~wyścigach konnych, oszustwo polegałoby na ustawieniu giełdy, innymi słowy, o czym klient wiedział najmniej, to było centrum gry.

-- Klient zostałby pokazany w~salonie zakładów lub biurze maklerskim pełnym tętniących życiem, aktywnych ludzi \ldots  tak wielu ludzi, że nie można było uwierzyć, że \textit{wszyscy }mogą być częścią oszustwa. Potem mechanizm byłby mu przedstawiony: dom maklerski lub zakład bukmacherski otrzymywał swoje dane z~biura telegraficznego, to było przed komputerami, które telefonowało z~wynikami. Następnie klientowi było pokazywane ,,biuro telegraficzne'', kolejny całkowicie fałszywy biznes, i~spotykał ,,przyjaciela'', wewnętrznego człowieka, który był skłonny opóźnić wyniki o kilka minut, podając je dublerowi i~rynkowi na tyle szybko, aby pozwolić im obstawić lub kupić. Wcześniej znaliby zwycięzców, zanim biuro to zrobiło, więc stawiali na pewną rzecz.

-- I próbowali tego, i~to działało! Klient mógł włożyć kilka dolarów i~odejść z~kilkoma setkami. To było niesamowite przeżycie, prawdziwy dreszczyk emocji. Wyobraźnia klienta zaczęłaby pracować. Gdyby mógł zamienić kilka dolarów w~setki, wyobraź sobie, co mógłby zrobić, gdyby mógł włożyć \textit{wszystkie }swoje pieniądze, razem z~tymi, które mógłby ukraść ze swojej firmy, rodziny, przyjaciół, wszystkich. Nawet by ich nie kradł, bo byłby w~stanie spłacić wszystkim, gdy wygra dużą sumę. I poszedłby po wszystkie pieniądze, jakie mógł zdobyć, postawił zakład i~przegrał!

-- I to byłaby jego wina. Satelita nie byłby w~stanie w~to uwierzyć, powiedziałby: ,,Miałeś postawić na tego konia w~pierwszym wyścigu'', a nie ,,Postawić na tego konia o pierwsze miejsce'' lub podobne nieporozumienie. Zły słuch klienta kosztował ich wszystko, wszystkich. Następowała wielka scena i~zanim się zorientowałaś, policja jest tam gotowa aresztować wszystkich. Ktoś strzela do policjanta, jest krew i~krzyki, miejsce pustoszeje, a klient uważa, że miał szczęście, że uszedł z~życiem. Oczywiście, cała krew i~strzelanina też są fałszerstwami, tak samo jak policjant. Ma trochę krwi w~worku w~ustach; nazywali to ,,żabipęcherz''': dobre słowo, nie?

-- Teraz, na tym etapie, może się zdarzyć, że klient jest całkowicie, całkowicie zepsuty, nie ma ani jednego paisa. Jeśli tak jest, ucieka i~nigdy więcej nie słyszy od dublera ani człowieka w~środku. Spędza resztę życia jako bankrut, złamany, nienawidząc siebie za to, że źle usłyszał instrukcje w~krytycznym momencie. I nigdy, przenigdy nikomu nie mówi, bo gdyby to zrobił, uczyniłoby to tego wielkiego człowieka głupcem.

-- Ale jeśli jest jakaś szansa, że zdobędzie więcej pieniędzy, przyjaciela, którego nie wyczyścił, konto bankowe firmy, do którego ma dostęp, mogą skontaktować się z~nim \textit{ponownie }i zaoferować mu szansę na wyrównanie rachunków. Możesz się założyć, że spróbuje, w~końcu jest królem wśród ludzi, przeznaczonym do rządzenia, który dorobił się fortuny, ponieważ jest lepszy od wszystkich innych. Dlaczego nie miałby grać ponownie, skoro jedynym powodem, dla którego przegrał ostatnim razem, było to, że źle usłyszał instrukcję. Na pewno to się więcej nie powtórzy!

-- Ale się powtórzy -- powiedziała. Jej oczy błyszczały.

-- O tak, rzeczywiście. I znowu i~znowu \ldots 

-- I znowu. dopóki nie zostanie wyczyszczony.

-- Nauczyłaś się pierwszej lekcji -- powiedział Ashok. -- Teraz przejdźmy do tematów zaawansowanych. Wiesz, jak działa piramida, tak?

Pomachała lekceważąco.

-- Oczywiście. 

-- Teraz piramida jest tylko rodzajem szkieletu i~tak jak szkielet, można na nim powiesić wiele różnych ciał. Może to wyglądać jak plan sprzedaży mydła, plan sprzedaży witamin lub coś innego. Ale najważniejsze jest to, że cokolwiek sprzedaje, musi wydawać się dobrą ofertą. Pomyśl o wielkim sklepie, jak sprawić, by coś wyglądało na dobrą ofertę?

Mala pomyślała uważnie. Ashok widział, jak koła zębate kręcą się w~jej głowie. La! Ta dziewczyna z~Dharavi była \textit{mądra}!

-- Dobrze -- powiedziała. -- OK  \ldots  powinno to być coś, o czym klient nie wie zbyt wiele.

-- Uchwycone w~locie! -- powiedział Ashok. -- Jeśli klient jest sprytny i~utalentowany, założy, że wie wszystko o wszystkim. Pomachaj przynętą, której tak naprawdę nie rozumie, a on się pojawi. Ale jest sposób, aby nawet znajome tematy stały się nieznane. Proszę, spójrz na to. -- Pisał na nieużywanym komputerze na rogu biurka, wygooglował obraz stołu do gry w~kości w~kasynie.

-- To jest gra hazardowa, kości. Grają w~nią kostkami.

-- Widziałam mężczyzn grających w~to na ulicy -- powiedziała Mala.

-- To jest wersja kasynowa. Widzisz wszystkie linie i~oznaczenia?

Skinęła głową.

-- Te znaki reprezentują różne zakłady, podwójne, jeśli wypadnie w~ten sposób, potrójne, jeśli wypadnie w~ten sposób. Zakłady mogą być bardzo, bardzo skomplikowane.

-- Teraz, kości nie są aż tak skomplikowane. Istnieje tylko 36 sposobów, na które rzut może wypaść: jeden-jeden, jeden-dwa, jeden-trzy i~tak dalej, aż do sześciu-sześciu. Powinno być łatwo stwierdzić, czy zakład jest dobry: skorzystaj z~szansy na wyrzucenie dwóch szóstek dwa razy z~rzędu: kurs wynosi 36 razy 36 do 1. Jeśli zakład wypłaci mniej niż ten kurs, ostatecznie stracisz pieniądze. Jeśli zakład wypłaci więcej niż ten kurs, ostatecznie zarobisz pieniądze.

Mala pokręciła głową. 

-- Naprawdę nie rozumiem.

-- Wyobraź sobie rzucanie monetą. -- Wyjął portfel, otworzył klapkę i~wyciągnął starą mosiężną chińską monetę z~dziurką pośrodku. -- Jedna strona to orzełki, a druga to reszki. Zakładając, że moneta jest ,,uczciwa'', to znaczy, zakładając, że obie strony monety ważą tak samo i~mają taki sam opór wiatru, wtedy szanse na wylądowanie monetą z~obu stron pokazywane są 50-50 lub 1 do 1 lub po prostu ,,równe''.

-- Teraz gramy w~uczciwą grę. Rzucam monetą, a ty mówisz, po której stronie myślisz, że wyląduje. Jeśli zgadniesz, podwajasz zakład; jeśli nie, zabieram twoje pieniądze. Jeśli gramy w~tę grę długo, oboje będziemy mieli tyle samo pieniędzy, z~którymi zaczęliśmy, to nudna gra.

-- Ale co, jeśli zamiast tego zapłacę ci trzykrotnie, jeśli wyląduje na orłach, pod warunkiem, że wziąłeś zakład typu orzełki? Wszystko, co musisz zrobić, to ciągle stawiać pieniądze na orłach, a w~końcu skończysz z~wszystkimi moimi pieniędzmi: kiedy nadejdą reszki, wygrywam trochę; kiedy wypadnie orzełek, wygrywasz dużo. Z biegiem czasu przejmiesz wszystko. Więc jeśli zaproponowałem ci to, powinieneś ją przyjąć.

-- W porządku -- powiedziała Mala.

-- Ale co, jeśli byłby to bardzo skomplikowany zakład? Co by było, gdyby były dwie monety, a wypłata zależała od długiej listy czynników; zapłacę ci trzykrotnie za każdą dwójkę lub reszkę, pod warunkiem, że nie jest to taki sam wynik jak poprzednio, chyba że jest to trzeci zduplikowany wynik. Czy to dobry zakład, czy zły? 

Mala wzruszyła ramionami.

-- Ja też nie wiem, musiałbym obliczyć szanse na papierze. Ale co z~tym: co jeśli zapłacę ci \textit{300 do jednego}, jeśli wygrasz zgodnie z~zasadami, które właśnie ustaliłem. Połóż dziesięć rupii i~wygraj, oddam ci \textit{3000}?

Mala przekrzywiła głowę. 

-- Prawdopodobnie wzięłabym zakład. 

-- Większość ludzi by to zrobiła. To fantastyczny koktajl: wymieszaj jedną część mylących zasad z~jedną częścią wysokich szans, a ludzie będą odkładać swoje pieniądze przez cały dzień. Powiedz, czy postawiłabyś dziesięć rupii na wyrzucenie podwójnych szóstek, trzydzieści razy z~rzędu? 

-- Nie! -- powiedziała Mala. -- To praktycznie niemożliwe.

Ashok rozłożył ręce. 

-- A teraz masz drugą lekcję: każdy ma jakąś intuicję co do szans, nawet jeśli jest, przepraszam, dziewczyną, która nigdy nie studiowała statystyki. -- Mala zaczerwieniła się, ale trzymała język za zębami. W końcu to była prawda. -- Większość ludzi nie postawi na prawie niemożliwe rzeczy, nawet jeśli dajesz wspaniałe szanse. Ale możesz ukryć prawie niemożliwe, wykonując wiele akrobacji, czyniąc zasady gry bardzo skomplikowanymi, a potem mnóstwo ludzi, nawet mądrych, postawi zakłady na propozycje, które są tak mało prawdopodobne, jak trzydzieści podwójnych szóstek z~rzędu. W rzeczywistości mądrzy ludzie są \textit{szczególnie }skłonni obstawiać takie zakłady \ldots 

Mala podniosła rękę. 

-- Ponieważ są tak mądrzy, że myślą, że wiedzą wszystko.

Ashok klasnął. 

-- Gwiazdorska uczennica! Powinnaś być oszustem albo ekonomistą, gdybyś tylko nie była tak dobrym generałem, generale. 

 Uśmiechnęła się. Ashok wiedziała, że uwielbia słyszeć, jak dobrym jest generałem. Nie winił jej: gdyby był dziewczyną z~Dharavi, która przechytrzyła slumsy i~stworzyła sobie życie, też byłby trochę niepewny. To była tylko jedna rzecz, którą można lubić w~Mali i~jej ponurym, twardym blasku. 
 
 -- Teraz, moja gwiazdorska uczennico, zbierz to wszystko dla mnie.

Zaczęła recytować, odliczając na palcach, jak uczennica opowiadająca lekcję. 

-- Aby stworzyć schemat Ponzi, który działa, który naprawdę działa, musisz mieć

\noindent bystrych ludzi

\noindent którzy są otoczeni przez wspólników przestępców

\noindent którzy mają szansę postawić na coś skomplikowanego

\noindent w sposób, którego nie potrafią zrozumieć.

Ashok zaklaskał, a Mala ukłoniła się ze swojego miejsca małym, ironicznym ukłonem.

-- Więc to właśnie tutaj robię. Opracowując plan, który weźmie jako zakładników gospodarki czterech światów, sprawimy, że będziemy je niszczyć, jak uznamy za stosowne. Aby to zrobić, muszę wykonać bardzo dobrą pracę. 

Mala wskazała na wykres pełen nabazgranych równań i~notacji. 

-- Wyjaśnij -- rozkazała.

-- To zupełnie inny rodzaj lekcji -- powiedział Ashok. -- Na inny dzień. A może rok.

Oczy Mali  się zwęziły.

-- Mój drogi generale -- powiedział Ashok, lejąc tak grubo, że oboje wiedzieli, że to robi, i~zobaczył, jak kąciki ust Mali drżą, gdy próbują powstrzymać jej uśmiech -- Gdybym poprosił cię o wyjaśnienie rozkazu bitwy, mogłabyś zrobić dwie rzeczy: albo przekazać kilka przydatnych, filozoficznych zasad dowodzenia siłami, albo mogłabyś zwymiotować statystyki i~szczegóły całego życia dotyczące każdej broni, każdej klasy postaci, każdej techniki i~triku. Szanse są takie, że nigdy nie zapamiętałbym ani jednej dziesiątej tego, co miałaś mi do powiedzenia. Nie mam do tego podstaw. A po zapamiętaniu nigdy nie będę w~stanie tego użyć, ponieważ nie miałbym doświadczenia, które zdobyłaś, jai ho! , więc nie będę miał w~swoim umyśle szkieletu, na którym mógłbym położyć ciało twojej nauki, mój guru. 

Sprawdził, czy nie przesadził, zdecydował, że nie, uśmiechnął się do niej i~skłonił się tylko po to, żeby dołożyć. Mala pokiwała królewsko głową, zachowując kamienną twarz tak długo, jak mogła, ale kiedy wyszła z~pokoju, kuśtykając na lasce, był pewien, że usłyszał  dziewczęcy chichot.

\bigskip
\threeast

Pierwszy talerz pierogów Matthew smakował tak dobrze, że prawie zakrztusił się śliną, która zalała mu usta. Po dwóch miesiącach w~obozie pracy, jedzeniu kurzych łapek i~ryżu i~nigdy ich dość, zamarzaniu w~nocy i~pieczeniu w~ciągu dnia, pomyślał, że doskonale zrekonstruował w~głowie smak pierogów. W dni, kiedy kopał, każde ugryzienie czubka łopaty w~ziemię przypominało moment, w~którym jego zęby przebiły skórę pieroga, pozwalając uciec parze i~olejowi, a mięso w~środku wydzielało zapach, który unosił się do jego nozdrzy. W dni, kiedy walił młotkiem, okrągłe kamienie były delikatnymi pierogami w~górach, wytarta ziemia była skrzypiącą styropianową tacą. Pierogi tańczyły w~jego myślach, gdy leżał na podłodze między dwoma innymi więźniami; były w~jego umyśle, kiedy wstawał rano. Jedynym okresem, gdy nie myślał o pierogach, było, gdy jadł kurze łapki i~ryż, ponieważ były tak obrzydliwe, że same miały moc wyprowadzanie ducha pierogów z~jego wyobraźni.

To były czasy, kiedy myślał o tym, co zrobi, kiedy wyjdzie z~więzienia. Co zamierzał zrobić w~grze. Co planowali Webblies i~jak miałby odegrać swoją rolę w~tym planie.

Urzędnik więzienny, który go wypuścił, przypuszczał, że jest jednym z~milionów nielegalnych pracowników ze sfałszowanymi dokumentami, którzy udali się do Kantonu, do delty Rzeki Perłowej, by szukać swojej fortuny. Był w~połowie surowego, wyszczekiwanego wykładu na temat unikania kłopotów i~powrotu do swojej wioski w~Gui-Zhou, Syczuanie lub na jakimkolwiek zubożałym zakątku, z~którego pochodził, zanim mężczyzna faktycznie spojrzał na swoje akta i~zobaczył, że Matthew jest rzeczywiście, Kantończykiem, i~że wkrótce zostanie przetransportowany, na koszt rządu, z~powrotem do Shenzhen. Mężczyzna zamilkł, a Matthew, przytłoczony komedią chwili, nie mógł nie podziękować mu wylewnie, po kantońsku.

W pociągu były pierogi, sprzedawane przez ponurych mężczyzn i~kobiety z~głębokimi zmarszczkami na twarzy wyciętymi przez lata, zmartwienia, głód i~nędzę. Były to prowincje, terytoria zewnętrzne, tajemnicze Chiny, które wysłały miliony dziewcząt i~chłopców do Kantonu, aby zarobiły fortunę w~Delcie Rzeki Perłowej. Matthew znał wszystkie ich dziwne akcenty, mówił ich dziwnym językiem mandaryńskim, ale był kantończykiem, a to nie był jego lud.

To nie były jego pierogi.

Dopiero gdy wysiadł na obrzeżach Shenzhen i~przesiadł się do metra, poczuł się jak w~domu. Dopiero wtedy zaczął myśleć o pierogach. Dziewczyny w~metrze były takie, jak je zapamiętał, piękne, obyte, śmiejące się i~dobrze odżywione. Przyczajony w~drzwiach pociągu, obserwując swoje odbicie w~ciemnym szkle, zobaczył, jakim okropnym szkieletem się stał. Był młodym mężczyzną, kiedy wszedł, właściwie chłopcem. Teraz wyglądał na pięć lat starszego, był zmieszany i~zapadnięty, a na jego policzkach widniał kosmyk rzadkiej brody, podkreślający ich pustkę. Wyglądał jak jeden z~masy przestępców, oszustów i~łajdaków, którzy kręcili się wokół stacji kolejowej i~na rogach ulic, twardy i~zdesperowany jak szczur ściekowy. Nieobliczalny.

Dlaczego nie? Szczury ściekowe dostawały mnóstwo pierogów. Miały ostre zęby i~ostry rozum. Były \textit{szybkie}. Matthew uśmiechnął się do swojego odbicia, a dziewczyny w~pociągu ominęły go szerokim łukiem, kiedy wjeżdżali na następną stację.

Lu spotkał go na stacji Guo Mao, na poziomie ulicy, gdzie mężczyźni i~kobiety w~dziarskich garniturach energicznie spacerowali wokół giełdy, idealny tłum ludzi, w~którym można się zgubić. Lu ujął jego obie dłonie w~długim, uduchowionym, cichym potrząsaniu i~poprowadził ich w~stronę giełdy, gdzie znajdowali się fałszerze tożsamości.

Ci ludzie utrzymywali Shenzhen i~całą prowincję Guangdong. Mogliby zrobić ci wszelkie potrzebne dokumenty: pozwolenia na pracę pozwalające dziewczynie z~farmy na przeprowadzkę z~Xi'an do Shenzhen i~produkcję iPodów; dokumenty mówiące, że byłeś prawnikiem, lekarzem, inżynierem; prawa jazdy, licencje sprzedawcy, a nawet licencje pilota, według ulotki, którą dał mu jeden z~nich. Były to starsze panie, przyjazna twarz kryminalnych imperiów rządzonych przez twardych mężczyzn z~wiecznymi papierosami i~łupieżem na ramionach ciemnych garniturów.

Szli w~milczeniu przez krzyczące tłumy, miotające się kartkami reklamującymi fałszywe dokumenty wpychane im w~ręce przez babcie ze wszystkich stron. Lu zatrzymał się przed babcią, pochylił się i~szepnął jej do ucha. Skinęła raz głową i~wróciła do machania kartami, ale musiała jakoś zasygnalizować wspólnika, bo chwilę później młody mężczyzna wstał z~ławki i~wszedł do gigantycznego centrum handlowego z~elektroniką, a oni podążyli za nim, przeciskając się przez stragany części do telefonów komórkowych -- klawiatur, ekranów, klawiatur, diod -- w~górę ruchomymi schodami na kolejne piętro z~częściami, na kolejne schody ruchome i~kolejne piętro, i~jeszcze jedno na piętro, które było całkowicie opuszczone. Nawet gniazdka elektryczne były puste, z~gniazd zwisały gołe przewody, czekając na podłączenie do wtyczek.

Chłopak był sto metrów przed nimi, a oni podążali za nim, wślizgując się do korytarza, który prowadził w~stronę schodów ewakuacyjnych. Małe boczne drzwi były lekko uchylone i~Lu pchnął je. Chłopca tam nie było -- musiał zejść po schodach -- ale był tam jeszcze inny chłopak, młodszy od Lu czy Matthew, siedzący przed komputerem i~intensywnie grający w~Mushroom Kingdom. Matthew uśmiechnął się, zawsze dziwnie było widzieć Chińczyka grającego w~grę tylko dla zabawy, a nie dla pracy. Podniósł wzrok i~skinął im głową. Lu bez słowa podał mu pakunek, który chłopiec starannie przeliczył, mieszając dolary z~Hongkongu i~chińskie renminbi. Sprawił, że pieniądze zniknęły zwinnym gestem palca, po czym wskazał na stołek w~rogu pokoju, za którym znajdował się biały ekran. Matthew usiadł, wciąż bez słowa, i~zobaczył, że na biurku chłopca znajduje się mała kamera internetowa, która wskazuje na niego. Skomponował swoje rysy w~wyrazie zawstydzonej powagi, w~rodzaju okropnego wyrazu twarzy, który nosiły wszystkie dokumenty tożsamości, a chłopiec kliknął myszką i~wskazał drzwi. 

-- Jedna godzina -- powiedział.

Lu przytrzymał drzwi dla Matthew i~poprowadziła go schodami przeciwpożarowymi, z~powrotem do centrum handlowego, z~powrotem na ulicę, z~powrotem wśród fałszerzy i~niedaleko do straganu z~makaronem, który był zatłoczony ludźmi, i~wtedy właśnie usta Matthew zaczęły pracować, żeby wytworzyć tyle śliny, że musiał ukradkiem osuszyć kąciki ust rękawem taniej bawełnianej marynarki.

Chwilę później jadł. I jadł. I jadł. Pierwszą miską była wieprzowina. Następnie wołowina. Następnie krewetka. Potem pierożki szanghajskie z~wieprzowiną. I nadal jadł. Jego żołądek wyciągnął się, a pasek dżinsów uwierał go, odpiął górny guzik i~zjadł trochę więcej. Lu cały czas gapił się na niego, przynosząc w~razie potrzeby więcej misek z~kluskami, przynosząc sos chili i~serwetki. Wysłał i~odebrał kilka wiadomości, a Matthew podniósł wzrok znad swojej pracy przy jedzeniu w~tych chwilach, by obserwować zaciekłą koncentrację Lu, gdy stukał w~klawiaturę telefonu.

-- Kim ona jest? -- spytał Matthew, odchylając się do tyłu i~pozwalając ostatniej warstwie pierogów osiąść w~żołądku.

Lu pochylił głowę i~zarumienił się. 

-- Przyjaciółka. Jest świetna. Zorganizowała, wiesz \ldots  -- Machnął pałeczkami w~kierunku targu fałszerzy. -- Ona \ldots  nie wiem, co bym bez niej zrobił. To dlatego nie jestem w~więzieniu.

Matthew uśmiechnął się krzywo. 

-- Do tej pory już byś się wydostał. -- Szarpał luźną koszulę. -- Chociaż może o~kilka rozmiarów mniejszy.

Lu pokazał Matthew na swoim telefonie zdjęcie dziewczyny z~południowych Chin. Wyglądała jak idealny model południowochińskiej kobiecości, modne ubrania i~włosy, starannie umalowana powieka, wyraz figlarności i, co, władzy? To poczucie bycia na szczycie swojego świata i~świata w~ogóle. Matthew skinął głową z~uznaniem. 

-- Szczęściarz Lu -- powiedział.

Lu zniżył głos.

-- Ona jest niesamowita -- wyszeptał. -- Załatwiła mi papiery, anulowała mój telefon, pozwoliła, by numer zniknął, a potem zgarnęła go z~powrotem, podając inną tożsamość, a potem przesłała dalej przez \ldots  -- rozejrzał się dramatycznie wokół i~zniżył głos jeszcze niżej -- centralę Falun Gong w~Makau, a potem z~powrotem do tego telefonu. Dlatego mogłeś do mnie zadzwonić. To niewiarygodne  \ldots  wciąż jestem w~kontakcie ze wszystkimi, ale to wszystko przez tak wiele zasłon, że zengfu nie mają pojęcia, gdzie jestem ani jak mnie wyśledzić. 

-- Skąd ona to wszystko wie? -- spytał łagodnie Matthew, a pierogi osiadły mu w~żołądku jak kamienie. Był trupem. -- Skąd wiesz, że ona sama nie jest policjantką?

-- Nie może być -- odpowiedział Lu. -- Zobaczysz dlaczego, kiedy się z~nią spotkamy. Tyle jestem pewien.

Ale Matthew nie mógł pozbyć się świadomości, że ta dziewczyna zabierze go z~powrotem do więzienia. W więzieniu wszyscy byli informatorami. Jeśli informowałeś na współwięźniów, dostałeś więcej jedzenia, więcej snu, lżejsze obowiązki. Najlepsi informatorzy byli jak mali szefowie, a inni więźniowie zabiegali o~ich przychylność, jakby byli na zewnątrz, dając im odpowiednik wolności golfa, dziewczyny i~hazard, tym, co mogli zeskrobać z~więzienia. Matthew nigdy nie informował i~nigdy nie został poinformowany. Zawsze wybierał gry, w~które grał, i~nigdy nie grał w~grę, której nie mógł wygrać.

I był tak odrętwiały, kiedy spotkał Jie, która cudownie pachniała, miała fantastyczne maniery i~promienny uśmiech. Miała jego nowe dokumenty tożsamości, z~właściwym zdjęciem, ale z~innym nazwiskiem i~numerem identyfikacyjnym oraz odciskiem palca na odwrocie, który na pewno nie należał do niego. Gdy szli, gawędziła przyjaźnie o~drobnych sprawach, pogodzie i~jedzeniu, wynikach piłkarskich i~plotkach o~celebrytach, zbyt doskonała gadka, która czyniła go jeszcze bardziej podejrzliwym wobec tej dziewczyny i~jej nienagannego aktorstwa.

Zaprowadziła ich do małego, zaniedbanego budynku w~starej kantońskiej części miasta, gdzie budynki rosły tak blisko siebie, że można było wystawić rękę przez okno sypialni i~uścisnąć dłoń osobie w~sypialni po drugiej stronie ulica. To tam dorastał Matthew, ,,miasto w~mieście'', w~którym wtłoczono Kantończyków, gdy południowe Chiny przestały być tylko miejscem i~stały się symbolem Nowych Chin, światowej fabryki. Powrót na te znajome ulice sprawił, że stał się jeszcze bardziej drażliwy, dając mu pełzającą pewność, że w~każdej chwili zostanie rozpoznany, że jakiś jego biedny przyjaciel z~dzieciństwa zostanie naznaczony przez tajną policjantkę i~wysłany razem z~nim do więzienia. Zmuszał się, by iść dalej, choć z~każdym krokiem chciał się odwrócić i~rzucić.

Mieszkanie, do którego ich zaprowadziła, było kiedyś połową maleńkiego mieszkania; teraz zostało zredukowane do jednego, maleńkiego pokoju ze stertami dziewczęcych ciuchów i~butów, kilkoma komputerami ustawionymi na tanich biurkach, umywalką z~brzegiem pokrytym kosmetykami i~odgrodzoną częścią, która prawdopodobnie skrywała toaletę. Prysznic znajdował się obok pieca i~zlewu, w~rogu wykafelkowany kwadrat z~odpływem wpuszczonym w~podłogę, głowicą prysznicową przytwierdzoną do ściany i~karniszem przykręconym do sufitu.

Kiedy drzwi zostały zamknięte, dziewczyna Lu zmieniła zachowanie tak gwałtownie, jakby zdjęła maskę. Jej twarz była teraz ożywiona przenikliwą inteligencją, jej zachowanie -- agresywne i~przenikliwe. 

-- Musimy kupić ci nowe ubrania -- powiedziała. -- Golenie, strzyżenie, trochę pieniędzy \ldots 

Jedną z~rzeczy, których Matthew nauczył się w~więzieniu, było to, by nie dać się ponieść scenariuszom innych ludzi. Silna osoba mogła to zrobić: napisać scenariusz, wymyślić go dla ciebie, umieścić cię w~roli i~zanim się zorientowałeś, przemycałeś zapieczętowane paczki z~jednej części więzienia do drugiej. Kiedy ktoś inny pisał scenariusz, byłeś prawie bezradny.

-- Poczekaj -- powiedział. -- Przestań.

Spojrzała na niego łagodnie. Lu był mniej spokojny, Matthew od razu mógł stwierdzić, że był całkowicie w~mocy tej kobiety. 

-- Madame, nie chcę być niegrzeczny, ale kim do diabła jesteś i~dlaczego miałbym ci ufać?

Roześmiała się. 

-- Chcesz wiedzieć, czy jestem zengfu -- powiedziała. Lu wyglądał na zgorszonego, ale dobrze to znosił. -- Oczywiście, że tak. Mam pieniądze, mieszkania, wiem, gdzie zdobyć dobre dokumenty tożsamości \ldots 

-- A ty jesteś bardzo apodyktyczna -- powiedział Matthew. 

-- Na pewno jestem! -- odpowiedziała. -- A czy słyszałeś kiedykolwiek o~Jiandi? 

\textit{Słyszał }to imię. Pomyślał o~tym przez chwilę, cofając się myślami do odległego, podobnego do snu czasu przed więzieniem. 

-- Pani z~radia? -- powiedział powoli. -- Ta, która rozmawia z~dziewczynami z~fabryki?

-- Tak -- powiedziała. -- Ta właśnie.

-- Dobrze -- powiedział. -- Słyszałem o~niej.

Lu uśmiechnął się. 

-- A teraz ją poznałeś!

Matthew zastanawiał się nad tym przez chwilę, wpatrując się w~starannie umalowane oczy dziewczyny, otoczone długimi, ciemnymi rzęsami. Wreszcie powiedział: 

-- Bez obrazy, ale każdy może twierdzić, że jest kimś, kogo nikt nigdy nie widział.

Lu zaczął mówić, ale uniosła rękę i~uciszyła go. 

-- On ma rację -- powiedziała. -- Tank, jedynym powodem, dla którego chodzę za darmo i~nadal nadaję, jest to, że jestem bardzo paranoiczną damą. Paranoja twojego przyjaciela to po prostu zdrowy rozsądek. Dużo bywałam tu na antenie, ale nigdy mnie nie \textit{słuchałeś}, co Tank?. Byłeś tyle razy na audycjach, ale nigdy się nie dostroiłeś. O ile wiesz, \textit{jestem }zengfu, infiltruję twoje szeregi gigantyczną, wymyślną podróbką, do której dzwonią inni gliniarze, udając słuchaczy programu, który nigdy nie wychodzi dalej niż do pokoju, w~którym siedzę. -- Usta Lu otwierały się i~zamykały, otwierały i~zamykały. Śmiała się z~niego. -- Nie martw się, nie jestem gliną. Po prostu podkreślam, że jesteś bardzo ufnym chłopcem. Może zbyt ufnym. Twój przyjaciel jest trochę bardziej ostrożny, to wszystko. Całkowicie aprobuję. 

Matthew miał nadzieję, że ta dziewczyna nie jest policjantką z~tego prostego powodu, że zaczął ją lubić. Nie wspominając o~tym, że gdyby była policjantką, poszedłby prosto z~powrotem do więzienia, ale teraz, gdy jego panika ustępowała, był w~stanie zastanowić się, jaka byłaby jako towarzyszka. Podobał mu się pomysł.

-- Dobrze -- powiedział. -- Więc jeśli jesteś Jiandi, to powinno ci być łatwo to udowodnić. Po prostu zrób program, a ja się dostroję i~go posłucham. 

-- Skąd wiesz, że Jiandi nie jest gliną? -- Miała błysk w~oku.

-- Nawet gliniarze nie są tak przebiegli -- powiedział. -- Nie mogliby znieść tych wszystkich reklam Falun Gong i~wszystkich tych wywrotowych rozmów o~partii  \ldots  nie potrwałoby to tygodnia, nie mówiąc już o~latach.

Skinęła głową. 

-- Też tak myślę. Lu, zgadzasz się?

Lu, wciąż nieszczęśliwy, skinął ponuro głową.

-- Rozchmurz się -- powiedziała. -- Musisz spędzić trochę czasu w~pojedynkę ze swoim przyjacielem!

Wylądowali w~nowej kafejce gier, daleko na linii metra, przy parku rozrywki Windows on the World. Ojciec Matthew zabrał go tam kiedyś, a on przebrał się w~starą zbroję bojową i~strzelał strzałami do celów, podczas gdy mężczyzna z~kantońskim akcentem ubrany jak amerykański Indianin dawał mu wskazówki. Było fajnie, ale nic tak przyjemnego jak gry, w~które Matthew już grał.

Metro wypuściło ich tuż za rogiem, przed ogromnym, zaniedbanym hotelem, który był zamknięty, kiedy Matthew przejeżdżał tędy ostatnim razem. Kawiarnia z~grami znajdowała się w~dawnej restauracji, w~stylu pirackim, z~ogromnym fałszywym statkiem pirackim na dachu. Wewnątrz było gęsto od dymu, a stoły uformowano w~zwyczajne, długie odcinki z~komputerem co metr. Mniej więcej połowa z~nich była zajęta, a w~jednym kącie restauracji znajdowało się pięćdziesięciu lub sześćdziesięciu graczy, którzy byli wyraźnie farmerami złota, pracującymi pod czujnym okiem starszego łobuza z~twardą twarzą i~papierosem w~kąciku ust. W kawiarni było niesamowicie gorąco, o~pięć stopni cieplej niż na zewnątrz, a było ciemno i~wilgotno jak w~jaskini. Matthew od razu poczuł się jak w~domu.

Lu podsunął starcowi za ladą kilka złożonych banknotów, złowrogiemu, bezzębnemu dziadkowi z~wyraźnym garbem i~dwoma brakującymi palcami u jednej ręki. Lu spojrzał na Matthew, po czym zamówił również talerz z~kluskami. Mężczyzna wyjął styropianową tacę z~zamrażarki skrzyniowej, przebił folię na wierzchu i~włożył do mikrofalówki obok niego w~recepcji. 

-- Idź -- zaskrzeczał -- przyniosę wam.

Matthew i~Lu usiedli przy sąsiednich komputerach, z~dala od reszty tłumu, obok okna zasłoniętego gazetami. Matthew przyłożył oko do rozdarcia w~gazecie i~wyjrzał na ruiny wymyślnego basenu o~tematyce morskiej na zewnątrz, wraz z~krętymi zjeżdżalniami i~fontannami, teraz zielonymi i~brudnymi. 

-- Ładny hotel -- powiedział.

Lu przesuwał się myszką na stronę Jiandi, przeplatając połączenie przez serię serwerów proxy, sprawdzając najnowsze adresy swoich serwerów strumieniowych i~znajdując taki, który działał. 

-- Myślę, że będziemy mieli co najmniej 45 minut, zanim ktokolwiek zauważy, że ten komputer robi coś poza granicami. Ufam, że będzie to mnóstwo czasu, aby zaspokoić swój podejrzliwy umysł.

Matthew zauważył, że Lu jest naprawdę zły, i~przełknął własną złość, coś jeszcze, w~czym miał mnóstwo praktyki w~więzieniu. 

-- Chcę być po prostu bezpieczny, Lu. To nie jest gra. -- Potem usłyszał własne słowa i~się uśmiechnął. -- OK, to \textit{jest }gra. Ale to także prawdziwe życie. Ma konsekwencje. -- Skubał koszulę, która wisiała luźno na jego chudym ciele. -- Nie zaszkodzi być bardziej ostrożnym.

Lu nic nie powiedział, ale jego usta były zaciśnięte i~białe. Starzec przyniósł im pierogi i~zjedli je w~milczeniu. Były to nędzne pierożki, wypełnione czymś, co smakowało jak posiekany papier, ale i~tak lepsze niż więzienne łapki kurze.

Matthew spojrzał na chłopca. Zawsze był troskliwy -- dziwna rzecz dla czołga -- i~taktowny i~odważny. Nie należał do pierwotnej gildii Matthew, ale kiedy Boss Wing umieścił go na czele całej elitarnej drużyny, przybyli z~własnej woli, widząc w~Matthew stratega, który mógłby ich poprowadzić do zwycięstwa. A kiedy Matthew zaczął im szeptać o~Webblies, Lu był tak samo podekscytowany jak wszyscy. Wszystko to wydawało się tak dawno temu, innym życiem i~innym czasem, zanim pałka policjanta powaliła go na ziemię, zanim poszedł do więzienia, zanim zmienił się w~człowieka, którym był teraz. Ale Matthew wrócił teraz na świat, a Lu od miesięcy żył ze sprytu i~\ldots 

-- Jestem ci winien przeprosiny -- powiedział, odkładając pałeczki. -- Wciąż nie wiem, czy mogę zaufać twojej przyjaciółce, ale mogłem być trochę mądrzejszy, jak to powiedziałem. To był dziwny dzień, 36 godzin temu miałem na sobie więzienny mundurek.

Lu zaczął na niego, a potem mały uśmiech wkradł się w~kąciki jego ust. 

-- W porządku -- powiedział. -- Tutaj, ona zaczyna. 

Wyjął słuchawkę, już sparowaną z~systemem dźwiękowym komputera, wytarł ją rękawem i~wręczył Matthew. Matthew wkręcił go sobie w~ucho.

-- Witam, siostry -- rozległ się znajomy głos. -- Jest trochę za wcześnie, wiem, ale to jest krótka i~specjalna transmisja dla was, szczęśliwe panie, które mają dzień wolny, są chore w~ambulatorium lub zdarzyło się, że wniosły słuchawki do fabryki. Witam, witam, witam. Odbierzemy telefon lub dwa?

Lu uśmiechnął się do Matthew, wstał i~wyszedł z~kawiarni. Matthew dotknął skorki, pomyślał o~pójściem za nim, postanowił tego nie robić. Chwilę później Jiandi powiedział: 

-- Idziemy, witaj, witaj.

-- Witaj Jiandi -- powiedział Lu. 

Matthew ponownie przyłożył oko do szczeliny w~pokrytym gazetą szkle i~stwierdził, że wpatruje się w~uśmiechniętego Lu, stojącego za budynkiem z~telefonem przy głowie.

-- Tank! -- pisnęła. -- Jak fantastycznie znów cię słyszeć. Minęły wieki, odkąd byłeś w~moim programie! Powiedz mi, Tank, o~czym dzisiaj myślisz?

-- O sprawiedliwości -- powiedział Lu/Tank. Matthew zaczął się cicho śmiać i~pochylił głowę, żeby nie zwracać na siebie uwagi. -- Sprawiedliwość dla ludzi pracy. Przyjeżdżamy do prowincji Guangdong, ponieważ mówią, że będziemy bogaci. Ale kiedy tu dotrzemy, dostajemy złe warunki pracy, złe wynagrodzenie i~wszystko jest ułożone przeciwko nam. Nikt nie może zdobyć prawdziwych dokumentów, aby żyć tutaj, więc wszyscy kupujemy podróbki, a policja wie, że w~każdej chwili może nas zatrzymać i~wsadzić do więzienia lub odesłać, bo nie mamy prawdziwych dokumentów. Nasi szefowie o~tym wiedzą, więc zamykają nas lub biją nas, albo kradną nam pensję. Jestem tu już od pięciu lat i~widzę, jak to działa: bogaci się bogacą, biedni się wyczerpują i~wracają do wioski, zrujnowani. Skorumpowany rząd bierze łapówki, a nie sprawiedliwość, a każda próba zorganizowania się przez ludzi pracy dla lepszego porozumienia spotyka się z~przemocą i~wojną. Skorumpowani biznesmeni kupują skorumpowanych policjantów, którzy pracują dla skorumpowanego rządu.

-- Mam już dosyć! Czas, aby ludzie pracy się zorganizowali, jeden z~nas jest niczym. Razem nie można nas powstrzymać. Rewolucje w~Chinach przychodzą i~odchodzą, a mimo to niewielu jest bogatych, a wielu jest biednych. Nadszedł czas na światową rewolucję: pracownicy w~Chinach, Indiach, Ameryce  \ldots  na całym świecie  \ldots  muszą walczyć razem. Będziemy korzystać z~Internetu, ponieważ jesteśmy lepsi w~Internecie niż nasi szefowie. Internet ma kształt organizacji pracowniczej: chaotyczny, rozproszony, bez kilku liderów podejmujących wszystkie decyzje. Wiemy, jak się nim kontaktować. Nasi szefowie rozumieją Internet tylko wtedy, gdy potrafią go ukształtować tak, jak oni, wymuszając, by wszystkie nasze kliknięcia szły przez kilka wąskich gardeł, które mogą posiadane i~kontrolowane. Nie możemy być kontrolowani. Nie możemy być powstrzymani. Wygramy! 

Jiandi roześmiał się do mikrofonu gardłowym, seksownym dźwiękiem. 

-- Och, Tank! Tak poważnie! Swoją gadką sprawiasz, że wszyscy czujemy się jak głupie dzieci! Ale on ma rację, siostry, wiecie, że tak. Martwimy się naszymi małymi problemami, nasi szefowie próbują nas wkręcić lub oszukać; policja nas ściga, nasze sieci są zainfekowane i~szpiegowane, ale nigdy nie pytamy \textit{dlaczego}, po co ten system? -- Wzięła głęboki oddech. -- Nigdy nie pytamy, co możemy zrobić.

Długa cisza. Matthew włączył komputer i~sprawdził, czy rzeczywiście jest dostrojony do Factory Girl Show. Poczuł w~piersi, w~brzuchu niedającą się nazwać emocję. Była tym, za kogo się podawała. Nie gliniarz. Nie szpieg.

Cóż, albo to, albo cała ta sprawa była ogromnym układem, a policja prowadziła operację tej kobiety od lat, oszukując miliony, tylko po to, by mieć tego wtajemniczonego w~środku. To był niesamowicie dziwny pomysł. Ale czasami politbiuro było niesamowicie dziwne.

-- Będziemy wiedzieć, co robić. Niedługo, siostry, nie bójcie się. Słuchajcie dalej  \ldots  nastawcie się dziś wieczorem na nasz regularny program .. a pewnego dnia \textit{już wkrótce }powiemy wam, co możecie zrobić. Czekajcie.

-- A wy, policjanci, biurokraci rządowi i~szefowie słuchacie teraz? Bójcie się.

Jej głos ucichł, a wesoły wariat zaczął mówić szalone rzeczy o~tym, jak wspaniałe jest Falun Gong, tradycyjne reklamy śmieci, które słyszał wcześniej w~programie Jiandi.

W zamyśleniu przeżuł kolejnego gazetowego pierożka i~czekał, aż Lu wróci do kawiarni. Wyszedł z~więzienia niecałe dwa dni, a jego życie było milion razy ciekawsze niż zaledwie kilka godzin wcześniej. I miał pierogi. Działy się rzeczy, wielkie rzeczy.

Lu ponownie uścisnął jego dłoń i~obaj szybko wyszli, kierując się w~stronę wejścia do metra. Kiedy zbiegali po schodach, Lu pochylił się i~powiedział cicho: 

-- Poczekaj, aż usłyszysz, co zaplanowaliśmy. -- Jego głos był napięty, podekscytowany. Prawie radosny.

-- Nie mogę się doczekać -- powiedział Matthew.

Wybuchło w~nim uczucie nadziei. Kiedy ostatni raz czuł nadzieję? O tak. To było wtedy, gdy opuścił farmę złota Bossa Winga, zabierając ze sobą swoich towarzyszy i~zakładając własny biznes. Oczywiście nie skończyło się to dobrze. Ale nadzieja była \textit{pyszna}. Teraz była pyszna.

\bigskip
\threeast

Justbob miała całą swoją sieć online. Byli to najlepsi zawodnicy IWWWW, pełni pasji i~zaangażowania. Od roku walczyli z~Pinkertonami i~unikali zabezpieczeń gry, co sprawiło, że byli twardzi. Niektórzy z~nich zostali pobici w~prawdziwym życiu, tak jak Justbob, Krang i~BSN, i~zastąpienie ikony użytkownika zdjęciem swoich obrażeń było nie lada honorem, zdjęcie rentgenowskie pełne połamanych kości, zbliżenie przerażającego rzędu szwów.

Kochała swoich wojowników. I kochali ją.

-- Cześć, ślicznotki -- zagruchała w~słuchawkę, poprawiając worek z~lodem, który wcisnęła między kość ogonową a krzesło. Działali teraz w~nowej kawiarni, wciąż w~Geylang, która była najlepszym miejscem do przebywania w~Singapurze, jeśli chciałeś być trochę poza zasięgiem, nie przyciągając zbyt dużej uwagi policji. -- Gotowi na najnowsze info? 

Z całego świata rozległ się chór wiwatów. Justbob mówiła po malajsku, indonezyjsku, angielsku, tamil i~trochę mandaryńskim i~hindi, ale zwykle planowali rzeczy po angielsku, którym wszyscy trochę mówili. Był oczywiście kanał zwrotny, czat tekstowy, na którym ludzie pomagali w~tłumaczeniach. Musieli mówić powoli, ale to działało.

-- Zamierzamy zabrać się za cztery światy jednocześnie: Mushroom Kingdom, Zombie Mecha, Svartalfaheim Warriors i~Magic of Hogwarts. -- Oglądała kanał zwrotny i~czekała, aż wszystkie tłumaczenia zostaną uporządkowane. -- Co mam na myśli przez ,,zabrać się''? Mam na myśli \textit{przejęcie}. Zamierzamy przejąć kontrolę nad gospodarkami wszystkich czterech światów: większością złota, prestiżowymi przedmiotami i~władzą. Zrobimy to szybko. Będziemy nie do powstrzymania: kiedykolwiek operacja zostanie zakłócona, będziemy mieć w~pogotowiu jeszcze trzy. Będziemy kontrolować los każdego szefa, którego pracownicy trudzą się w~tych światach. Zakołyszemy korporacyjnymi panami. Będziemy walczyć z~każdym Pinkertonem, albo nawracając ich do naszej sprawy, albo bijąc ich tak mocno, że zmienią karierę.

-- Aby to zrobić, będziemy potrzebować wielu tysięcy graczy pracujących w~koordynacji. Przede wszystkim oznacza to robienie tego, co robią najlepiej: zarabianie złota. Ale spodziewamy się również dużego oporu, gdy dowiedzą się, co robimy. Oczywiście będziemy potrzebować wojowników, którzy będą bronić naszych linii przed Pinkertonami, ale potrzebujemy również dużo odwrócenia uwagi i~ingerencji, w~tym, nie, \textit{szczególnie}, w~światach, do których się \textit{nie }wybieramy. Chcemy, aby kierownictwo gry było tak zdezorientowane, aż będzie za późno. Będziecie potrzebować serwerów proxy, \textit{dużo ich }i tyle awatarów, ile zdołacie przejść na kolejny poziom. To teraz twoje zadanie numer jeden  \ldots  wypoziomować jak najwięcej awatarów, aby móc przełączać konta i~wskoczyć do nowego bojowca, gdy drugi stary zostanie odłączony. -- Przez chwilę przyglądała się gadaninie, po czym dodała: -- Tak, oczywiście, my pracujemy nad tym teraz. Za mniej więcej dzień będziemy mieć przedpłacone karty do konta dla was wszystkich. Do działania będą potrzebować amerykańskich serwerów proxy, więc upewnijcie się, że macie ich dobrą listę.

Przez chwilę przyglądała się paplaninie. 

-- Oczywiście tak, będą próbowali wyłączyć proxy, ale jeśli to zrobią, ich amerykańscy gracze będą \textit{krzyczeć}. Czy wiesz, ilu Amerykanów wymyka się ze swoich sieci roboczych, aby grać w~ciągu dnia, korzystając z~tych serwerów proxy? Jeśli zaczną blokować serwery proxy, będą blokować niektórych ze swoich najlepszych klientów. I oczywiście wielu Mechanicznych Turków korzysta z~sieci szkolnych, używając serwerów proxy do logowania się w~swoich miejscach pracy. Nie stać ich na blokowanie wszystkich tych serwerów proxy \ldots  nie na długo! 

Wybuchł kanał zwrotny. Podobało im się to. To była dobra strategia, jak wtedy, gdy zaatakujesz bossa, a następnie znalazłeś schronienie, które postawiło między tobą a złymi ludźmi niskiego poziomu i~sprowokowałeś walkę, w~której wszyscy walczyli ze sobą zamiast z~tobą. Justbob żałowała, że nie może powiedzieć więcej na ten temat, ponieważ przebiegłość tego wszystkiego sprawiła, że uśmiechała się przez cały dzień, cały tydzień, cały miesiąc, kiedy rozpracowywali to podczas jednego z~zebrań na wysokim szczeblu. Ale rozumiała potrzebę zachowania tajemnicy. Można było się założyć, że niektórzy z~bojowników na tej konferencji pracowali dla drugiej strony; w~końcu niektórzy z~\textit{ich }szpiegów byli w~firmach, prawda?

-- W porządku -- powiedziała -- w~porządku. Dość gadania. Zabijmy coś. 

 Jej słuchawki wybuchły urwanymi wiwatami, a ona walczyła ze swoimi dowódcami przez szczęśliwą godzinę, dopóki The Mighty Krang nie przyszedł i~odciągnął ją, aby mogła zjeść obiad.

Big Sister Nor poczekała, aż usiądzie z~jedzeniem na talerzu -- skwierczącym char kway teow i~smażonym makaronem Hokkien, pachnącym jak niebo -- zanim zaczęła mówić.

-- W porządku -- powiedziała. -- Nasz człowiek jutro wyląduje w~Shenzhen. Mamy ludzi, którzy pomogą mu bezpiecznie wydostać się z~portu, a on mówi, że ma nasz ładunek, żadnych problemów. Logował się w~trakcie podróży, mówi, że może nam załatwić setki Turków.

Mighty Krang pomachał pałeczkami. 

-- Wierzysz mu?

Big Sister Nor żuła i~połykała w~zamyśleniu. 

-- Myślę, że tak -- powiedziała. -- On jest entuzjastyczny, ten. Jest jednym z~tych dzieciaków, które absolutnie \textit{kochają }gry i~chciały być częścią,,magii'', ale odkrył, że pracuje każdą godzinę zesłaną przez Boga i~zawsze istnieją ukryte zasady, które kończyły się na cięciach jego płacy. 

 Pozostała dwójka energicznie pokiwała głową, rozpoznali wzór, był to szablon sweatshopów na całym świecie.
 
 -- Jego pracodawcy powiedzieli mu, żeby był wdzięczny za taką cudowną okazję i~czy nie wiedział, że jest wielu innych, którzy będą mieli jego pracę, jeśli jej nie chce?

-- OK, więc jest zdenerwowany  \ldots  dlaczego sądzisz, że może dostarczyć wielu innych zdenerwowanych ludzi? 

Wzruszyła ramionami i~nadziała krewetkę. 

-- On jest urodzonym networkerem, prawdziwym wykonawcą. Powinniście usłyszeć, jak opowiada o~swoim kontenerze transportowym! To prawdziwy hotel na pełnym morzu. Bardzo pomysłowy. A jego kumple z~gildii mówią, że jest cholernie towarzyski. Miły facet. Facet, którego słuchasz.

-- Rodzaj faceta, za którym idziesz? -- spytała Justbob, drapiąc się po jej oczodole w~bliznach. Mogła zapomnieć o~swędzeniu i~bólu z~boku twarzy, kiedy rozmawiała ze swoimi wojownikami, ale przez resztę czasu straciła to cenne rozproszenie. A jej sny były pełne upiornych bólów ze zrujnowanego oczodołu i~czasami budziła się ze łzami na twarzy.

Big Sister Nor powiedziała: 

-- Tak myślę.

Mighty Krang wypił trochę soku z~arbuza i~za pomocą kondensacji narysował glify na stole. Kelnerka, ładna Tamilka, skrzywiła się na niego z~udawaną teatralnością i~wytarła to. Wszystkie kelnerki kochały się w~The Mighty Krang. Nawet Justbob musiała przyznać, że był ładny. 

-- Nie podoba mi się ten pomysł -- powiedział. -- Chodzi o, no wiesz, \textit{robotników}.

Big Sister Nor utkwiła w~nim spokojne spojrzenie. 

-- Masz na myśli ,,jest biały, nie ufam mu''. On też jest pracownikiem, nawet jeśli pracuje dla gry. \textit{Wszyscy }jesteśmy robotnikami. To jest sens Webblies. Wszyscy pracownicy w~jednym wielkim związku, solidarni. Zacznij różnicować pracowników, którzy zasługują na związek, i~tych, którzy nie rób tego i~następną rzeczą, o~której wiesz, twoja praca zostanie przekazana pracownikom, których pominąłeś w~swoim małym prywatnym klubie. Krang, jeśli nie masz jasności co do tego, jesteś w~złym miejscu. Absolutnie złym miejsce. Czy wyrażam się jasno? 

To była inna Big Sister Nor niż ta, którą zwykle znali, matczyna, cierpliwa, wyrozumiała. Jej głos był kruchy i~surowy, jej spojrzenie było przeszywające. Krang wyraźnie zwiądł pod jego blaskiem. 

-- Dobrze -- powiedział bez większego przekonania. -- Przepraszam. 

Justbob czuła się za niego zakłopotana, ale nie współczuła. On wiedział lepiej.

Skończyli posiłek w~milczeniu. Zadzwonił telefon Big Sister Nor. Spojrzała na twarz, zobaczyła numer, odłożyła go z~powrotem. Obowiązywała zasada: nie odbierać telefonów podczas ,,rodzinnych obiadów'' we trójkę. Ale BSN wyraźnie zależało na odebraniu tego. Zaczęła jeść szybciej, tak szybko, jak tylko mogła skręconą ręką.

-- Kto to był? -- zapytał Justbob.

-- Chiny -- powiedziała. -- Pilne. Nasz chłopak z~Ameryki. 

\bigskip
\threeast

Ping nie lubił portu. Za dużo gliniarzy. Miał dobre papiery, ale nawet najlepsze nie wytrzymałyby długo przed gliniarzem, który rzeczywiście odpytał ID przez radio. Fałszerze twierdzili, że do podróbek używali dobrej tożsamości, prawdziwych ludzi, którzy nie mieli żadnych kłopotów, ale kto wiedział, czy można im wierzyć?

W każdym razie to było po prostu szalone. Gweilo miał poczekać, aż statek wpłynie do doku, przebrać się w~czyste ubranie, przypiąć identyfikator z~firmy ojca i~po prostu \textit{wyjść z~}portu, błyskając swoim identyfikatorem każdemu, kto zechce zapytać chudego, białego dzieciaka, co robi, wynosząc dwa ciężkie kartonowe pudła z~bezpiecznego obszaru. Gdy wydostanie się z~portu, Ping może go zabrać, sprawić, by zniknął w~mieszance obcokrajowców, kupców i~biznesmenów zatłoczonych w~regionie.

Ping rozpytywał się dookoła, znalazł Webbliesa, którego brat pracował rok wcześniej jako przewoźnik, uzyskał informacje o~tym, gdzie najprawdopodobniej pojawi się Leonard, i~wysłał wszystkie te informacje do Leonarda, gdy przemierzał ocean.

Ale nie powinno być \textit{tylu }gliniarzy, prawda? Wyglądało na to, że były ich setki, i~to nie tylko mundury. Było mnóstwo szczególnie wysokich mężczyzn z~podciętymi włosami i~nausznikami, ubranych jak cywile, ale poruszających się ze zdecydowanie zbyt dużą koordynacją i~celowością. Ping dwukrotnie przeszedł obok wejścia, po raz pierwszy prowadząc wyimaginowaną kłótnię z~kimś przez telefon, próbując emanować aurą rozproszenia, która sprawi, że będzie wydawał się nieszkodliwy. Za drugim razem przeszedł obok, wpatrując się uważnie w~mapę turystyczną, starając się zachować pozory bezradności. W międzyczasie sprawdził zegarek, zobaczył, że Leonard spóźnił się o~godzinę, wysłał wiadomość do Lu i~poprosił go, żeby sprawdził, czy mógłby wysłać e-maila do Big Sister Nor i~dowiedzieć się, co się dzieje. To był najtrudniejszy moment, odkąd łącze satelitarne statku nie działało, gdy znajdowało się w~doku, więc  kradzione połączenie sieciowe Leonarda zostało przerwane. Kiedy wyjdzie z~portu, dadzą mu telefon na kartę, sprowadzą go z~powrotem do sieci, ale do tego czasu \ldots 

Prawie upuścił mapę turystyczną, kiedy wyłączył się jego telefon. Pobliski gliniarz, najwyższy mężczyzna, jakiego kiedykolwiek widział, spojrzał na niego twardo, a on uśmiechnął się nieśmiało i~wyjął telefon, próbując opanować drżenie rąk, gdy przytknął go do życia, mając nadzieję, że hałas go nie przeszkodził.

-- Czy on jest z~tobą? -- Mandaryński Big Sister Nor był mocno akcentowany, ale dobry. Natychmiast rozpoznał głos z~wielu nocnych czatów i~nalotów.

-- Cześć! -- powiedział jasnym, szorstkim głosem, próbując brzmieć, jakby rozmawiał z~dziewczyną lub siostrą. -- Wspaniale Ciebie słyszeć! 

-- Jeszcze go nie widziałeś?

-- Zgadza się! -- powiedział, przyklejając na twarz fałszywy uśmiech na korzyść ochroniarza.

-- Cholera. Miał wyjść kilka godzin temu. -- Big Sister Nor zamilkła. -- OK, jest sprawa. Cokolwiek mu się przydarzyło, potrzebujemy tych pudełek. -- Przeklinała w~jakimś innym języku. -- Powinnam była po prostu kazać mu włożyć pudła do kontenera. Tak bardzo chciał się z~wami zobaczyć \ldots  -- Urwała.

-- OK! -- powiedział, odchodząc najswobodniej, jak tylko mógł, od gliniarza. Było miejsce, drzwi przed zamkniętym sklepem spożywczym dalej na drodze. Mógł tam iść, usiąść, porozmawiać o~tym.

-- Dużo gliniarzy tam, gdzie jesteś, co? Nie odpowiadaj. Słuchaj, Ping, muszę wiedzieć, czy możesz dostać się do portu? Jeżeli mu się nie uda?

Przełknął. 

-- Nie sądzę -- wyszeptał. Był już prawie przy drzwiach.

-- A jeśli będziesz musiał?

Był liderem rajdu, mistrzem strategii. Nie był Matthew, ale mimo to rozumiał, jak wchodzić i~wychodzić z~ciasnych miejsc. I był całkiem niezłym wspinaczem kilka lat temu, zanim znalazł farmy złota. Może mógłby przejść przez płot? Miał ochotę zwymiotować na tę myśl. Było tyle kamer, tylu gliniarzy, ogrodzenie było \textit{tak wysokie}.

-- Spróbowałbym -- powiedział. -- Ale prawie na pewno pójdę do więzienia. -- Był przetrzymywany przez trzy dni w~miejscowym areszcie wraz z~większością strajkujących, a następnie zwolniony. Wystarczająco źle, nie tak źle, jak historie Matthew, a on nigdy nie chciał tam wracać. -- Musisz zobaczyć to miejsce, Nor, to jest jak forteca. 

Westchnęła. 

-- Wiem, jak wyglądają porty -- powiedziała. -- OK, powiem ci coś \ldots  poczekaj jeszcze godzinę, zobacz, czy możesz go znaleźć. Popracuję tutaj nad czymś innym i~zadzwonię do ciebie. 

-- Dobrze -- powiedział.

Od niechcenia dryfował wzdłuż wysokiego ogrodzenia, które strzegło portu, doskonale świadom kamer wwiercających się w~jego kark. Ile razy mógł przejść obok, zanim ktoś zdecydował się dowiedzieć, co tam robił? Powinni byli sprowadzić całą drużynę, pół tuzina gangu, który mógłby zamienić się na szukanie głupiego gweilo. Ping potrząsnął głową z~obrzydzeniem. Fajnie było poznać Leonarda, gdy był dzieckiem w~Kalifornii i~oni byli pięciorgiem dzieciaków w~Chinach, nawet egzotyczne. Nikt inny nie imprezował z~egzotycznymi obcokrajowcami ze złym akcentem.

To było nawet ekscytujące, gdy gweilo zamienił się w~przemytnika dla sprawy, przemierzając ocean ze swoim łupem ciężko zarobionych przedpłaconych kart do gry, które pozwoliłyby im wszystkim latać pod radarem firm produkujących gry.

Ale teraz, kiedy miał iść do więzienia, nie było to już ekscytujące, ponieważ jakiś głupi dzieciak zza oceanu nie mógł wymyślić, jak wyciągnąć swój tyłek z~portu w~Shenzhen.

\bigskip
\threeast

Poszło lepiej, niż Wei-Dong miał prawo oczekiwać. Kiedy wypłynęli na morze, jak masło wciął się w~WiFi frachtowca i~wskoczył na ich łącze satelitarne. Było wolne -- zbyt wolne do grania -- ale nadawało się do wysyłania wiadomości i~utrzymywania kontaktu zarówno z~Webbliesami, jak i~komórką Turków, które złożył z~najlepszych ludzi, jakich znał. Pierwszej nocy wyszedł z~kontenera i~wspiął się na szczyt stosu, ciągnąc za sobą platformę słoneczną i~kolektor wody, i~przymocował oba w~niepozornym miejscu na zewnętrznej powierzchni najwyższych kontenerów, gdzie żaden członek załogi nie mógł ich dostrzec. Znowu operacja przebiegła bez zakłóceń.

Ale trzeciego dnia zaczął szukać kłopotów. Nie mógł poświęcić zbyt wiele czasu na oglądanie planów pojawiających się na tablicach Webblies, zwłaszcza że tak wiele fragmentów planu było ściśle strzeżonymi tajemnicami, widocznymi tylko jako puste plamy w~jego zrozumieniu, dokąd zmierza i~dlaczego tam idzie. Tysiąc razy dziennie uderzało go absolutne szaleństwo swojej pozycji, przemytnik na pełnym morzu, który w~młodym wieku 18 lat robi rewolucję w~Azji! To było bajeczne i~przerażające, w~zależności od nastroju.

Główny nastrój był \textit{nudą}.

Nie było nic do roboty, a piątego dnia wgryzał się w~cały ruch na łodzi, obserwując zakochaną w~miłości załogę sześciu filipińskich marynarzy wysyłających romantyczne notatki do swoich tęskniących dziewczyn. Pobieranie słownika tagalskiego było wystarczająco zabawne, żeby mógł sprawdzić niektóre frazy, które wpisywali do listów, ale po chwili to też zbladło.

I były jeszcze \textit{dni }drogi, a deszcze nadeszły i~wypełniły jego zbiorniki, więc miał wodę do picia i~gotowania, więc nawet nie swędziała go skóra ani nie był niedożywiony, co by go rozpraszały, więc zaczął robić głupie rzeczy.

Zaczął się skradać.

Och, oczywiście tylko w~nocy i~na początku tylko wśród kontenerów, gdzie załoga rzadko się zapuszczała. Ale w~przestrzeniach kontenerów nie było wiele do zobaczenia, tylko nieprzerwane, użebrowane przestrzenie kontenerów, oznakowanych radiowo i~pomalowanych ogromnymi numerami, zaklejonych i~szczelnie zamkniętych.

Więc wtedy zaczął przekradać się do kwater załogi.

Wiedział, jak będą wyglądać. Możesz zarezerwować przeprawę na frachtowcu, wybrać się długie, dziwne wakacje dryfujące od portu do portu na całym świecie. Biura podróży, które sprzedają te samotne, proste rejsy, miały mnóstwo zdjęć i~filmów online oraz panoram zakwaterowania i~świetlic. Wszędzie wyglądały jak sale instytucjonalne, z~dużymi, porysowanymi, płaskimi wyświetlaczami, zniszczonym i~poplamionym dywanem, obwisłymi sofami, podrapanymi stołami i~krzesłami. Różnica polegała na tym, że na statku wszystkie te rzeczy zostały przykręcone.

Ale po dniach spędzonych w~jego małej sekretnej fortecy samotności każda zmiana scenerii brzmiała jak wycieczka do Disneylandu i~pół. I tak oto znalazł się w~kuchni statku o~drugiej nad ranem, żyli w~czasie pacyficznym, a on przerzucił się na czas chiński po tym, jak wypłynęli na morze, więc nie było to zbyt trudne. W lodówce, składniki do kanapek, filipińskie rożki z~lodami na jedną porcję, gotowana herbata boba z~ogromnymi perłami tapioki w~środku i~puszki frappucino Starbucks. Poczęstował się, wkładając to wszystko do torby na ramię, którą przyniósł ze sobą, i~pospiesznie wrócił do swojego legowiska.

To była pierwsza noc. Drugiej nocy zjadł przekąskę w~pokoju telewizyjnym, oglądając bootlegowe DVD z~aktualnie wydanego filmu komediowego, który miał premierę w~dniu, w~którym opuścił LA. Ściszył dźwięk, a nawet skorzystał z~łazienki na zewnątrz pokoju wspólnego na korytarzu, który prowadził do kwater załogi. Skradał się na palcach i~wyciszał telewizor za każdym razem, gdy statek skrzypiał, serce waliło mu jak młotem, gdy jego oczy rzucały się w~każdy kąt pokoju, szukając nieistniejącej kryjówki wśród przykręconych mebli.

To była jak dotąd najlepsza noc w~podróży.

Więc następnej nocy musiał iść dalej. Po tym, jak wypuścił się trzeci raz i~obejrzał bollywoodzki film komediowy science fiction o~robocie w~turbanie, który zaatakował Bangalore, tylko po to, by zostać pokonanym przez informatyków, wkradł się do maszynowni.

\textit{To }była zmiana scenerii. Drzwi do maszynowni były zaryglowane, ale nie zamknięte, tak jak wszystkie inne drzwi na statku, których próbował. W końcu znajdowali się na środku przeklętego oceanu, to nie tak, że musieli się martwić o~włamywaczy, prawda? (Oczywiście z~wyjątkiem obecnych!).

Wielkie silniki wysokoprężne były głośne jak odrzutowce. Znalazł parę zatłuszczonych, dźwiękoszczelnych nauszników i~nałożył je na uszy, zmniejszając nieco hałas, ale hałas nadal wibrował przez podeszwy jego tenisówek, powodując, że jego kości drżały. Wszystko tutaj było świeże i~lśniące, wypolerowane, naoliwione i~pomalowane. Przesunął palcami po panelach kontrolnych, wskaźnikach, zaworach odcinających, uniósł ręce, żeby połaskotać giętkie węże zwijające się nad głową. Grał w~kilka map osadzonych w~takich pomieszczeniach, ale doświadczenie w~prawdziwym życiu było czymś innym. Właściwie znajdował się \textit{wewnątrz }maszyny, w~silniku tak potężnym, że mógł przewieźć tysiące ton stali i~ładunku przez pół świata.

Nieźle.

Gdy zdjął mufy i~ostrożnie je odłożył, zauważył coś, co naprawdę powinien był zauważyć po drodze: mały czujnik optyczny przy drzwiach maszynowni na szczycie stalowych, pofałdowanych, antypoślizgowych schodów i~obok niego kamera wielkości szpilki otoczona diodami podczerwieni. Co znaczyło \ldots 

Co oznaczało, że gdy wszedł do pokoju, uruchomił niewidzialny alarm i~złamał wiązkę, a odkąd przybył, był nagrywany. Co znaczyło \ldots 

Co oznaczało, że był \textit{stracony}.

Palce mu drżały, kiedy otwierał zamek w~drzwiach i~wymknął się do stalowej szopy, która strzegła wejścia do maszynowni na pokładzie dla załogi. Rozejrzał się w~lewo i~w prawo, czekając, aż reflektor przetnie smolistą noc, czekając, aż syrena przetnie ryk oceanu, gdy przecięli ją na pół potężnym dziobem łodzi.

Było cicho. Było ciemno. Na razie. Statek miał tylko jednego nocnego oficera wachtowego i~jednego nocnego pilota, a z~jego sieci szpiegowskiej wiedział, że wachta była pretekstem do wysyłania e-maili i~pobierania pornografii, więc mogło być tak, że żaden z~nich nie zauważył ostrzeżenia \ldots  jeszcze.

Zakradł się z~powrotem między kontenery, poruszając się tak szybko, jak tylko się odważył, boleśnie świadom tego, jak wyraźnie będzie się wyróżniał dla każdego, kto, choć od niechcenia, spojrzy w~dół z~mostka statku na szczycie nadbudówki. Kiedy dotarł do pojemników, wśliznął się na wąski chodnik, który otaczał statek na zewnątrz, i~zaczął biec, ścigając się do swojego gniazda. Po drodze sporządził w~myślach listę rzeczy, które będzie musiał zrobić, gdy już tam dotrze, zwijając panele słoneczne i~anteny, kolektory wody. Zapinał swój pojemnik tak ciasno jak żabi tyłek i~mogliby szukać miesiącami, zanim dotrą do niego, a tymczasem będzie w~Shenzhen za kilka dni. Wtedy byłaby to tylko kwestia ominięcia ochrony portu, która byłaby w~stanie najwyższej gotowości, gdy załoga zaalarmuje ich o~pasażerze na gapę. Aaa. Był \textit{takim }idiotą. To wszystko się rozwali w~drobny mak, tylko dlatego, że się \textit{znudził}.

Przeklinając siebie, hiperwentylując, biegnąc, poślizgnął się na pokładzie i~wbił twarz w~pomalowaną, poplamioną ptakami stal. Ból był szalony. Krew wylewała się z~jego nosa, który był pewien, że był złamany. A teraz statek kołysał się i~kołysał mocno i~cholera, spójrz na te chmury przecinające niebo!

Nie szło dobrze. Zakręcił się chwiejnie w~róg wokół stosu kontenerów, miał moment strachu, gdy ogromny statek przetoczył się pod nim, a jego ręka machała dziko w~kierunku poręczy, a potem złapał się i~dokończył zakręt, pędząc do swojego pojemnika. Gdy tam dotarł, ruszył biegami, które wyznaczały bieg macek podtrzymujących życie, wybiegających z~jego pudełka, i~odłączył każdą, pracując drżącymi rękami. Przytulając do piersi elastyczny wąż, kable, ogniwa słoneczne i~antenę, prześlizgnął się po ściankach pojemnika i~wślizgnął się do środka w~chwili, gdy kolejna ruch rzucił go na tyłek.

Odepchnął włazy w~swoim hermetycznym wewnętrznym sanktuarium i~wszedł do środka. Statek kołysał się teraz mocno, a jego sprzęty kuchenne, niedbale pozostawione leżące, grzechotały w~tę i~z powrotem. Z początku zignorował to, sięgając po laptopa i~sprawdzając dzienniki ruchu z~sieci statku, ale gdy puszka tuńczyka uderzyła go w~policzek, tworząc pręgę, odłożył komputer i~zapiął go na rzepy, a potem zebrał wszystko, co było luźne, i~wrzucił to do przyśrubowanych skrzyni. Potem wrócił do swoich zrzutów ruchu sieciowego, szukając czegokolwiek, co brzmiało jak oficjalne zawiadomienie o~jego odkryciu.

Nocny ruch był zawsze niewielki, trochę telemetrii, zalotne e-maile od szkieletowej załogi. Dzisiejsza noc nie była wyjątkiem. Plik zatrzymał się w~momencie, w~którym zwinął antenę, ale prawdopodobnie i~tak nie wytrzymałby dużo dłużej. Deszcz bębnił teraz, jak prawdziwa pieska pogoda, brzmiąc jak grad żwiru na stalowe pojemniki dookoła niego. Po kilku minutach stwierdził, że żałuje, że nie wziął nauszników. Kilka minut później zapomniał o~nausznikach i~chwycił za torbę, do której miał włożyć skradzione jedzenie. Rzucanie i~toczenie nie ustało, po prostu trwało i~trwało, jego żołądek był pusty, próbując wywrócić się na lewą stronę, śliskie wymiociny wszędzie w~maleńkiej kabinie. Próbował sobie przypomnieć, co powinieneś zrobić z~chorobą morską. Obserwuj horyzont, prawda? W kontenerze nie było horyzontu, tylko pochylające się ściany i~podłoga oraz niestabilne światło z~zasilanych bateryjnie opraw LED, które przykleił do sufitu. Cienie podskakiwały i~majaczyły, zwiększając dezorientację.

To były najbardziej nieszczęśliwe chwile, w~jakich kiedykolwiek był. Wydawało się, że to się nigdy nie skończy. W pewnym momencie zaczął się zastanawiać, jak by to było, gdyby był stłoczony z~10 czy 20 innymi osobami, w~ciemności boiska, bez chemicznej toalety, tylko z~wiadrem, które może się przewrócić przy pierwszym przechyle. Zatłoczone i~zamknięte, drzwi nie będą jeszcze otwarte przez kilka dni i~nie ma sposobu, aby wiedzieć, co może cię powitać po drugiej stronie.

Nagle nie czuł się tak nieszczęśliwy. Pobudził się na tyle, by jeszcze trochę popatrzeć na komputer, ale wpatrywanie się w~ekran natychmiast przywołało chorobę morską. Przypomniał sobie, jak pakował kilka tabletek imbiru, które miały działać uspokajająco na żołądek -- czytał o~nich na stronie z~najczęściej zadawanymi pytaniami dla osób wybierających się na swój pierwszy rejs po oceanie -- i~szukanie ich w~bujanym pudełku rozproszyło go na chwilę. Połknął dwie z~nich wodą, zauważając, że zbiornik był tylko w~połowie pełny i~postanowił uratować każdą kroplę, teraz gdy jego kolektor był wyłączony.

Nie był pewien, ale wydawało się, że burza słabnie. Wypił trochę więcej wody, sprawdził mdłości -- trochę lepiej -- i~wrócił do ekranu. To był drobny cud, ale nie było żadnego doniesienia o~jego zauważeniu, żadnego pilnego komunikatu do centrali korporacji na temat pasażera na gapę. Może nie zauważyli? Może skupili się na burzy?

I potem znów pojawiła się burza, z~powrotem i~jeszcze silniejsza niż wcześniej. Kołysanie budowało, budowało, budowało. To już nie było obrzydliwe, to było \textit{gwałtowne}. W pewnym momencie Wei-Dong zorientował się, że wisi na łóżku obiema rękami i~nogami, z~laptopem wciśniętym między klatkę piersiową a materacem, gdy cały statek przetoczył się na lewą burtę i~zawisł tam, chwiejąc się pod kątem prawie poziomym przed zakołysaniem się w~\textit{przeciwnym} kierunku. Raz, dwa razy statek przewrócił się, a Wei-Dong zacisnął zęby, pięści i~oczy i~modlił się do bezimiennego boga, aby się nie przewrócił i~nie opadł na dno oceanu. Kontenerowce nie tonęły zbyt często, ale \textit{tonęły}. I nie tylko to, około pół procenta kontenerów ginęło na morzu, wylatywało za burtę na wzburzonej wodzie. Jego ojciec zawsze brał to do siebie. Jeden procent to niewiele, ale, jak lubił mu przypominać ojciec Wei-Dong, to było 20 000 kontenerów, z~których można było zbudować wieżowiec. Liczba ta rosła z~roku na rok, gdy morze stawało się coraz bardziej wzburzone, a pogoda coraz trudniejsza do przewidzenia.
 
 Wszystko to przeszło przez głowę Wei-Donga, gdy kurczowo trzymał się swojego przyśrubowanego łóżka, poobijany od stóp do głów przez luźne przedmioty, które przeoczył, kiedy pakował wszystko do skrzynek. Statek jęczał i~napinał się, a potem rozległ się głęboki metaliczny zgrzyt, który poczuł aż do swoich jąder, a potem \ldots 

 \ldots  kontener się \textit{poruszył}.

To była długa chwila i~wydawało się, że wszystko ucichło, gdy wrażenie przesuwania się po masywnym pokładzie przeniknęło przez jego ucho wewnętrzne i~prosto do ośrodka strachu w~mózgu. W tym momencie wiedział, że zaraz umrze. Miał tonąć, tonąć i~tonąć w~nieważkiej wieczności, gdy ciśnienie oceanu wokół niego rosło, aż pojemnik imploduje i~rozsmaruje go po zmiętych ścianach, rozpraszając się czerwonymi serpentynami, gdy pojemnik opadnie na dno morza.

A potem statek się wyprostował. W jego oczach pojawiły się łzy, a w~kroczu wilgoć. Zsikał się. Kołysanie zwolniło, zwolniło. Zatrzymało się. Teraz statek kołysał się normalnie i~Wei-Dong wiedział, że przeżyje.

Jego kryjówka była wrakiem. Jego ubrania, zabawki, sprzęt do przetrwania -- wszystko rozrzucone na cztery wiatry. Na szczęście chemiczna toaleta pozostała na swoim miejscu, a jej pokrywa była mocno zaciśnięta. To byłby \textit{bałagan}. Wymioty, woda i~inne wycieki szorowały każdą dostępną powierzchnię. Według jego zegarka była czwarta rano na jego osobistym zegarze. To sprawiło, że jest godzina 11 rano czasu statku, który był ustawiony na Los Angeles. Jeśli dobrze wykonał obliczenia, na ich szerokości geograficznej była około szóstej rano, co powinno odpowiadać mniej więcej w~Nowej Zelandii. Co oznaczało, że słońce wzejdzie, a załoga bez wątpienia będzie się roić na pokładzie, badając uszkodzenia i~zabezpieczając pozostałe kontenery najlepiej jak potrafią za pomocą małego dźwigu statku i~traktorów. A \textit{to }oznaczało, że będzie musiał pozostać na miejscu, pośród chorego, złego powietrza i~bałaganu, poczekać, aż na noc statku, a może nawet do następnej nocy. Nie miał też WiFi.

Gówno. 

Przywiózł ze sobą tabletki nasenne, na wszelki wypadek, jako część swojej apteczki pierwszej pomocy. Znalazł zapieczętowaną plastikową skrzynię nadal przytwierdzoną do jednego z~drucianych regałów, obok dwóch cennych pudełek z~kartami przedpłaconymi, nadal bezpiecznie przymocowanymi do ramy. Kiedy przełamał opakowanie i~wlał skąpy łyk wody do blaszanego kubka, na chwilę przerwał: co by było, gdyby odkryli jego pojemnik, gdy był odurzony narkotykami?

A co, jeśli odkryliby to, kiedy był zupełnie przytomny? Przecież nie mógł \textit{uciec.}

Co z~niego za idiota.

Zjadł tabletki, a potem zabrał się do sprzątania swojego mieszkania najlepiej, jak mógł, używając starych koszulek jako szmat. Odwrócił materac, aby odsłonić stronę, która nie była zasikana, i~zastanawiał się, kiedy pigułki zaczną działać. A potem stwierdził, że jest zbyt zmęczony, by zrobić coś innego poza położeniem się z~policzkiem na nagim materacu i~zapadnięciem w~głęboki sen bez marzeń sennych.

Tabletki miały być preparatem ,,nie-ospałe'', ale obudził się z~uczuciem, że jego głowa jest owinięta gumą piankową. Może to było doświadczenie bliskie śmierci. Był teraz środek nocy na statku, prawdziwa noc. Teoretycznie na zewnątrz będzie ciemno, i~mógłby się wymknąć, zbadać uszkodzenia, może zamontować antenę WiFi i~dowiedzieć się, czy nie zostanie aresztowany, kiedy dotrą do portu. Ale kiedy ostrożnie wyszedł ze swojego wewnętrznego pudełka i~próbował otworzyć drzwi swojego kontenera, odkrył, że został on zaklinowany. Nie tylko lepki, czy zgięty na zawiasie, ale właściwie zablokował się na~następnym kontenerze, z~kilkoma tonami ładunku po drugiej stronie drzwi, żeby mógł usunąć go z~drogi. Albo nie.

Usiadł. Miał zapaloną czołówkę, bo wnętrze pojemnika było ciemne jak wnętrze puszki coli. Rzucał szalone cienie na ściany, stos baterii (pochwalił się za przezorność w~użyciu potrójnych warstw stalowych taśm, aby utrzymać je na miejscu), właz prowadzący do jego wewnętrznego sanktuarium.

Według jego obliczeń byli tylko trzy dni od Shenzhen, plus lub minus wszelkie korekty kursu, jakie musieliby wprowadzić, teraz gdy burza minęła. Teoretycznie mógłby to zrobić. Miał wodę, jedzenie, elektryczność, pod warunkiem, że racjonuje wszystkie trzy. Ale Webblies spodziewali się, że zamelduje się wcześniej, a nuda doprowadziłaby go do szaleństwa.

Pomyślał o~próbie przecięcia stalowego pojemnika. To było możliwe, na forach dyskusyjnych zajmujących się konwerterami kontenerów nie brakowało rozmów o~tym, co trzeba zrobić, aby pociąć kontener i~wykorzystać go do innych celów. Ale nic w~jego zestawie narzędzi nie mogło sobie z~tym poradzić. Najbliżej, jak mógłby się zbliżyć, byłoby wywiercenie dziury w~pokrywie bezprzewodową wiertarką. Użył jej do złożenia swojego gniazda, miał w~skrzynce z~narzędziami kilka zapasowych pudełek z~szybkimi wiertłami. Jego największy kawałek, mała piła tarczowa, przebije dziurę wielkości kciuka, ale dopiero po wywierceniu otworu prowadzącego w~stali. Stal o~grubości 2 mm, kilka razy grubsza niż rozpórki, które rozwiercał podczas prac wewnętrznych.

Zrobiłoby to okropny hałas, ale był na pokładzie ładunkowym, daleko od nadbudówki. Zakładając, że nikt nie patrolował pokładu, nie było mowy, żeby usłyszano go ponad szumem morza i~dudnieniem diesli. Powiedział sobie, że warto było ryzykować odkrycie, ponieważ wybicie dziury oznaczałoby wyjęcie anteny, a tym samym dołączenie do sieci i~sprawdzenie, czy będzie bezpieczny po dotarciu do Chin.

Teraz albo nigdy. Znalazł skrzynkę z~narzędziami w~większej, przykręconej skrzynce i~wydobył wiertło. Miał do tego zapasową ładowarkę z~falownikiem, który rozładowałby cały stos akumulatorów. Potrzebowałby wielu baterii, żeby przedostać się przez sufit.

Kilka godzin później zdał sobie sprawę, że sufit mógł być błędem. Jego barki, ramiona i~klatka piersiowa paliły i~bolały. Coraz częściej robił sobie przerwy, wymachując rękami, ale ból nie ustępował. Bolały go też uszy, od echa świszczącego odgłosu wiertarki, setek koszmarów na fotelu dentystycznym. Nie spuszczał oka z~zegarka, powtarzając sobie, że będzie pracował do porannej zmiany, aby zmniejszyć ryzyko, że dźwięk będzie słyszalny. Ale do końca zmiany pozostała jeszcze godzina, gdy zgasł akumulator w~jego wiertarce, i~odkrył, że ostatnim razem, gdy zmieniał akumulatory, zaniedbał wepchnięcia całkowicie martwej do ładowarki, a teraz obie jego baterie były wyczerpane.

To była dobra wymówka, żeby przestać. Dotknął wgniecenia, które zrobił w~stalowej blasze podczas godzin spędzonych na wierceniu. Jego palec wbił się w~nią, ale ledwie się w~nim zagłębił. Zdjął krzesło z~zaczepów, przeciągnął je, stanął na nim, przyjrzał mu się i~zobaczył punkcik brudnego szarego światła, pierwsze światło świtu, migoczące na dnie jego wywierconego otworu.

Sen nie pomógł jego ramionom. Jeśli już, to tylko je pogorszyło. Zajęło mu pięć minut, zanim doszedł do punktu, w~którym mógł unieść ręce na twarz, przesuwając je tam i~z powrotem. Miał w~apteczce mały słoiczek z~Tygrysim Balsamem, czerwonym, pachnącym chińskim środkiem na mięśnie, i~wmasowywał go w~ramiona, ramiona, klatkę piersiową i~szyję, myśląc tak samo: \textit{te rzeczy nic nie zrobią}. Kilka minut później na jego skórze pojawiło się nowe pieczenie, ogniste, miętowe uczucie, jednocześnie gorące i~zimne. Na początku było to niepokojące, ale kilka sekund później było to \textit{niewiarygodne}, jakby jego mięśnie jednocześnie puściły napięcie. Wziął wiertarkę, sprawdził zegarek -- w~połowie pierwszej zmiany, ale chrzanić, silniki jęczały, nikt tego nie słyszał -- i~zabrał się do pracy.

Przebił się pięć minut później. Pięć minut! Był tak blisko! Znowu przyłożył oko do dziury, zobaczył niebo, chmury, cienie innych pojemników w~pobliżu. Jego bezprzewodowa antena czekała. Miała dużą, ciężką podstawę magnetyczną, potężne magnesy ziem rzadkich, których użył do przymocowania jej do wcześniejszego miejsca. Pracowały tak dobrze, że musiał postawić obie stopy po obu jego stronach i~dźwigać upartą marchewkę. Teraz nie potrzebował podstawy, tylko smukłą różdżkę samej anteny. Zdemontował antenę, przymocował ją z~powrotem do gołych końcówek drutu, a następnie delikatnie, ostrożnie wsunął ją przez otwór wielkości dziesięciocentówki.

Zatrzymał się na chwilę, gdy miał już dość, wyobrażając sobie, jak wystaje między równymi, gładkimi powierzchniami blatów kontenerów, tak oczywista jak erekcja przy tablicy, ale wiercił tak długo, że wydawało się szaleństwem przestać. Głos w~jego głowie powiedział mu, że bycie złapanym jest jeszcze bardziej szalone, ale uciszył ten głos, mówiąc mu, żeby się zamknął, ponieważ uzyskanie informacji o~stanie statku byłoby niezbędne do ukończenia misji. A potem antena została podniesiona.

Chwycił laptopa, zalogował się do sieci i~zaczął podsłuchiwać ruch. Mógł oglądać w~czasie rzeczywistym -- jego sniffer pomagał grupować przechwycone e-maile, kliknięcia, strony, wyszukiwane hasła i~komunikatory we własnych panelach raportowych -- ale to było po prostu frustrujące, jak obserwowanie przesuwającego się po ekranie paska postępu.

Zamiast tego wszedł do swojego sanktuarium i~zrobił sobie filiżankę błyskawicznego makaronu ramen, używając nieco więcej swojej cennej elektryczności i~wody, a następnie otworzył puszkę zielonej herbaty z~mlekiem sojowym, aby ją popić. Jadł tak wolno, jak tylko mógł, próbując rozkoszować się każdym kęsem i~przekazać swojemu żołądkowi, że jedzenie jest w~porządku, pomimo rock and rolla minionego dnia. Podczas posiłku usłyszał kroki w~pobliżu kontenera, pomruk ciężkiego sprzętu pracującego przy kontenerach, a usta wyschły mu na myśl o~wystającej tam antenie.

Dlaczego to tam położył? Ponieważ nie mógł znieść myśli o~siedzeniu znudzony i~niespokojny w~swoim pudełku przez kolejne dni. Dlaczego to robił? Dlaczego był w~drodze do Chin? Dlaczego opuścił dom, aby zostać graczem? Dlaczego w~ogóle nauczył się chińskiego? Uwięziony we własnych myślach, znalazł się w~obliczu dość brzydkich odpowiedzi. Nie chciał być taki jak wszystkie inne dzieciaki. Chciał się wyróżniać, być wyjątkowym. Innym. Wiedzieć, rozumieć i~być biegłym w~rzeczach, o~których jego ojciec nic nie wiedział. Zatriumfować. Być częścią czegoś większego niż on sam, ale być \textit{ważną }częścią. Być romantycznym i~wyjątkowym. Dbać o~sprawiedliwość, o~której istnieniu jego przyjaciele nawet nie wiedzieli.

To sprawiało, że czuł się smutny, żałosny i~potrzebujący. To sprawiło, że chciał podłączyć się do laptopa i~uciec od swoich myśli.

Zadziałało. To, co znalazł na swoim laptopie, było zdumiewające. Najpierw od kapitana do firmy żeglugowej przesłano e-mailem szereg zdjęć, na których widać pokład ładunkowy statku, który wygląda jak przewrócona wieża Jenga, z~kontenerami porozrzucanymi wszędzie, na bokach, na plecach, pod dziwnymi kątami. Wyglądało to tak, jakby cała górna warstwa pudeł zsunęła się do oceanu, a potem jeszcze kilka warstw po lewej stronie. Przyjrzał się dokładniej. Jego kontener znajdował się na prawej burcie, a kontener z~odpowiedniej pozycji po drugiej stronie wydawał się zniknąć. Zajrzał do wykazu statku, znalazł numer seryjny kontenera, dopasował go do listy skrzynek za burtą i~przełknął. To była czysta przypadkowa szansa, że jego pudło znalazło się na prawej burcie. Gdyby poszedł w~drugą stronę, byłby dżemem malinowym w~zmiażdżonej puszce na dnie oceanu.

Przeszukał ruch e-mailowy w~poszukiwaniu informacji o~tajemniczym pasażerze na gapę, ale wyglądało na to, że burza dosłownie zdmuchnęła wszelkie obawy o~niego za burtę. W manifeście wymienił wartość celną wszystkich kontenerów na statku. Większość z~nich była pusta, a przynajmniej częściowo pusta, ponieważ Ameryka nie miała zbyt wiele, czego potrzebowały Chiny, z~wyjątkiem pustych kontenerów, aby napełnić je towarami do Ameryki. Mimo to łączna wartość brakujących pojemników sięgała setek tysięcy dolarów. Skrzywił się. To będzie ogromny rachunek za ubezpieczenie.

Teraz nadszedł czas na otrzymanie \textit{jego }e-maila, coś, co odkładał, bo to było jeszcze bardziej ryzykowne; gdyby administratorzy statku podsłuchiwali własną sieć, zobaczyliby jego ruch. Och, nie wyglądałoby to na e-mail od niego do Big Sister Nor, jego gildii i~Turków w~Ameryce. Wyglądałoby to na gigantyczne ilości losowych śmieci, pochodzące z~wewnętrznego adresu, który nie odpowiadał żadnej znanej maszynie na statku. Jego cel był niejasny, natychmiast wskoczył do TOR, The Onion Routera, który podrzucił go jak groszek w~marakasy po otwartych przekaźnikach globu. Liczył na luźne zabezpieczenia informatyczne statku i~na fakt, że załoga zawsze podłączała nowe urządzenia, takie jak telefony i~przenośne gry, które odbierali w~porcie, aby pomóc mu prześlizgnąć się przez oczy sieci. Nadal, jeżeli szukali pasażera na gapę, mogliby szukać po ruchu sieciowym.

Siedział przy klawiaturze z~uniesionymi palcami i~dyskutował ze sobą. W głębi duszy wiedział, jak zakończy się ta debata. Nie mógł trzymać się z~dala od sieci i~przyjaciół tak samo, jak nie mógł pozostać zamknięty w~puszce bez wytykania anteny ze statku.

Więc to zrobił. Wysyłał e-maile, obserwował ruch w~sieci, wstrzymał oddech. Na razie w~porządku. Potem: dudnienie i~stukot i~para grzmiących \textit{brzęków }z góry. Serce waliło mu w~uszach, a przez ciasną przestrzeń rozlegały się metaliczne dźwięki. Co to było? Umieścił dźwięki, połączył je ze zdjęciami, które widział wcześniej. Załoga wyjęła wózek widłowy i~traktor, a dźwig się kołysał, a następnie przestawiali kontenery w~celu zapewnienia stabilności i~wyważenia. Wciągnął antenę do środka i~zanurkował do wewnętrznego sanktuarium, zahaczając o~właz i~wrzucając wszystkie luźne przedmioty do szafek, po czym rzucił się na łóżko, chwycił słupek i~przylgnął do niego palcami rąk i~nóg, gdy pojemnik kołysał się i~toczył drugi raz w~ciągu 24 godzin.

\bigskip
\threeast

-- Więc gdzie skończyłeś? -- zapytał Ping, podając Wei-Dongowi kolejną paczkę ryżu longzai i~kurczaka w~liściu lotosu. 

Ping chciał iść do Pizza Hut, ale Wei-Dong wyglądał na tak dotkniętego i~urażonego tą sugestią, i~tak nalegał na zjedzenie czegoś ,,prawdziwego'', że zabrał gweilo do kawiarni w~dzielnicy kantońskiej w~pobliżu budynków uścisku dłoni. Wei-Dong pokochał to miejsce od chwili, gdy usiedli, i~zamówił pewnie, imponując zarówno Pingowi, jak i~kelnerowi swoją znajomością południowochińskich potraw.

Wei-Dong przeżuł, skrzywił się. 

-- Na cholernym szczycie stosu, trzy piętra! -- powiedział. -- Z większą ilością pojemników wciśniętych po każdej stronie, z~wyjątkiem strony drzwi, na szczęście! Ale nie mogłem po nich zejść ze stosu. -- Uderzył w~brudne, poobijane kartonowe pudła obok stołu. -- Więc musiałem przenieść karty do plecaka, a następnie wspinać się po tym stosie, raz za razem, aż miałem wszystko na ziemi. Potem zrzuciłem zwinięte kartony, wspiąłem się na dno i~zapakowałem wszystko znowu w~porządku. 

Szczęka Pinga opadła. 

-- Zrobiłeś to wszystko w~\textit{porcie} ? -- Pomyślał o~wszystkich strażnikach, których widział, o~wszystkich kamerach.

Wei-Dong potrząsnął głową. 

-- Nie -- powiedział. -- Nie mogłem zaryzykować. Zrobiłem to w~nocy, sztafetą, w~noc przed wejściem do środka. I przykryłem to wszystko jakąś plastikową płachtą, którą miałem, co jest dobrą rzeczą, ponieważ wczoraj padało. Było dużo wody na pokładzie, a część przeciekła przez plastik, ale pudełka wydają się w~porządku. Miejmy nadzieję, że karty są nadal czytelne. Myślę, że muszą być  \ldots  są w~plastikowych pudełkach w~środku.

-- Ale co z~załogą, która cię zobaczyła?

Wei-Dong się roześmiał. 

-- Och, przez cały czas srałem ze strachu, serio! Przez większość czasu byłem w~zasięgu wzroku sterówki, chociaż na szczęście nie było księżyca. Ale tak, to było dość dziwaczne.

Ping spojrzał na gweilo, jego chude ramiona, puszyste wąsy dojrzewania, potargane włosy, nieprzyjemny zapach. Kiedy chłopiec w~końcu wyszedł z~bramy, pewnie pokazując strażnikowi jakąś odznakę, Ping chciał go udusić za to, że się tak spóźnił i~wyglądał na tak \textit{zrelaksowanego}. Teraz jednak nie mógł nie podziwiać swojego starego towarzysza z~gildii. Powiedział to.

Wei-Dong zarumienił się, a jego klatka piersiowa napełniła się i~wyglądał na tak dumnego, że Ping musiał to powtórzyć. 

-- Jestem zachwycony -- powiedział. -- Co za historia! 

-- Zrobiłem po prostu to, co musiałem zrobić -- powiedział Wei-Dong z~nieprzekonującym, nonszalanckim wzruszeniem ramion. 

Jego mandaryński był lepszy, niż pamiętał go Ping. Może chodziło po prostu o~bycie twarzą w~twarz, a nie przez rozmytą, zawodną sieć, możliwość zobaczenia całego ciała, całej twarzy.

Całe wcześniejsze zmartwienie i~irytacja Pinga zniknęły. Ogarnęła go fala uczucia do tego dzieciaka, który przebył tysiące kilometrów, aby być częścią tej samej wielkiej gildii. 

-- Nie zrozum tego źle -- powiedział -- ale muszę ci to powiedzieć. Kilka godzin temu byłem na ciebie bardzo zły. Myślałem, że to tylko ego, lub głupota, twój przyjazd całą drogę z~pudłami. Chciałem cię udusić. Myślałem, że jesteś głupim, rozpieszczonym \ldots  -- Zobaczył wyraz twarzy Wei-Dong, czysty ból serca, i~urwał, uniósł ręce. -- Czekaj! Próbuję powiedzieć, że myślałem o~tym wszystkim, ale potem spotkałem cię i~usłyszałem twoją historię, i~zdałem sobie sprawę, że chcesz tego tak samo jak ja i~grasz teraz o~taką samą stawkę. Że jesteś prawdziwym, prawdziwym \textit{towarzyszem}. 

 To słowo było zabawne, stare komunistyczne słowo, które zostało wypłukane z~koloru i~znaczenia przez dziesięć milionów godzin rewolucyjnego śpiewania piosenek w~szkole. Ale pasowało.

I zadziałało. Klatka piersiowa Wei-Donga spuchła jeszcze bardziej, jak balon, który ma odpłynąć, a jego policzki błyszczały jak czerwone węgle. Szukał słów, ale jego chiński najwyraźniej mu uciekł, więc Ping roześmiał się i~wręczył mu kolejny liść lotosu, tym razem wypełniony owocami morza.

-- Jedz! -- powiedział. -- Jedz! 

 Sprawdził godzinę na swoim telefonie, przeczytał zakodowane wiadomości od Big Sister Nor. 
 
 -- Masz dziesięć minut na dokończenie, a potem musimy dostać się do domu gildii na wielkie zebranie! 

\bigskip
\threeast

Jesteś w~obcym mieście lub w~obcej części miasta. Już trochę zdezorientowany, to klucz. Może to po prostu dziwny czas, żeby wyjść z~domu z~samego rana w~dzielnicy biznesowej, albo późną nocą w~klubie, albo w~środku dnia na przedmieściach i~nikogo innego nie ma w~pobliżu.

Podchodzi do ciebie nieznajomy. Jest dobrze ubrany, uśmiechnięty. Jego język ciała mówi, że \textit{jestem przyjacielem i~trochę tu nie pasuję}. On coś trzyma. To tafla szkła, duża, krucha, wielkości atlasu drogowego lub tablicy Monopoly. On się z~tym zmaga. To jest ciężkie? Śliskie? Gdy podchodzi bliżej, mówi z~nutą samoświadomości absurdalności tego wszystkiego: 

-- Czy możesz to potrzymać na chwilę? -- Brzmi też trochę zdesperowany, jakby miał to rzucić.

Łapiesz się tego. Kruche. Wielkie. Ciężkie. Bardzo dziwne.

I wciąż uśmiechnięty nieznajomy metodycznie i~szybko zanurza ręce w~twoich kieszeniach i~zaczyna przenosić klucze, portfel i~gotówkę do własnych kieszeni. Nigdy nie zrywa kontaktu wzrokowego w~ciągu 10 czy 15 sekund, jakie zajmuje mu wykonanie zadania, po czym odwraca się na pięcie i~odchodzi (nie biegnie, to ważne) bardzo szybko, przez kilkanaście kroków, a potem wpada w~sprint z~wiatrem, nabierając mocy jak Kaczor Duffy zrywający się na Elmera Fudda.

Nadal trzymasz szybę.

Dlaczego trzymasz się tej szyby?

Co jeszcze zamierzasz z~tym zrobić? Upuścić go i~pozwolić jej pęknąć na dziwnym chodniku? Odłożyć ostrożnie?

Powiem ci jedną rzecz, której nie zrobisz. Nie będziesz z~nią biec. Bieganie z~dziesięciokilogramowym kawałkiem szkła o~ostrych krawędziach jest jeszcze głupsze niż trzymanie go w~dłoniach.

\bigskip
\threeast

-- Co w~pracy? 

 Big Sister Nor była w~oknie wideokonferencji, z~Mighty Krang i~Justbob po obu jej stronach, z~głowami opuszczonymi na ekrany, utrzymując aktywny czat tekstowy na drugim kanale, podczas gdy Big Sister Nor prowadziła wykład. Mówiła po mandaryńsku, potem po hindi. Czat tekstowy był żywy w~trzech alfabetach i~pięciu językach, a pod słowami pojawiały się tłumaczenia maszynowe. Angielski dla Wei-Dong, chiński dla jego gildii. Kilka tysięcy osób było zalogowanych bezpośrednio, a dziesiątki tysięcy miały się zameldować później, kiedy skończą swoją zmianę.

-- Dingleberry w~KL mówi ,,dezorientacja'' -- powiedział Mighty Krang, nie podnosząc wzroku.

Big Sister Nor skinęła głową. 

-- I? 

-- ,,Umowa społeczna'' -- powiedział Justbob. -- To MrGreen w~Singapurze.

BSN pokazała zęby w~twardym uśmiechu. 

-- Singapur, gdzie wiedzą wszystko o~umowie społecznej! Tak, tak! To po prostu to. Ktoś przychodzi do ciebie i~prosi o~pomoc, ty pomagasz; to jest w~naszych instynktach, to jest w~naszym wychowaniu. To nas wszystkich trzyma cywilizowanych.

A potem opowiedziała im historię o~grupie robotników z~Phenom Penh, farmerów złota, którzy pracowali dla kogoś, kto miał być dla nich bardzo uprzejmy i~dobry, zabierał ich na lunch raz w~tygodniu, przynosił dobre obiady i~filmy, kiedy pracowali do późna, ale którzy zawsze robił małe \ldots  \textit{błędy} \ldots  w~kopertach wypłat. Niewiele i~zawsze był zakłopotany, kiedy to się stało i~płacił, a jeszcze bardziej zawstydził się, kiedy ,,zapomniał'', że nadszedł dzień wypłaty i~było dzień, dwa dni, trzy dni opóźnienia w~zapłacie. Ale był ich przyjacielem, ich dobrym przyjacielem i~mieli z~nim niepisany kontrakt, który mówił, że wszyscy są dobrymi przyjaciółmi i~nie nazywasz swojego dobrego przyjaciela złodziejem.

A potem zniknął.

Pewnego dnia przyszli do pracy -- trzy dni po wypłacie i~oczywiście jeszcze nie otrzymali zapłaty -- a człowiek, który prowadził kafejkę internetową, po prostu wzruszył ramionami i~powiedział, że nie ma pojęcia, dokąd poszedł ich szef. Kilku pracowników przepracowało nawet cały dzień, a nawet następny, ponieważ ich dobry przyjaciel musiał wkrótce pojawić! A potem ich konta przestały działać; wszystkie konta, wszystkie postacie, które upgradowali, osobiste postacie, których używali do wielkich rajdów z~rzadkimi łupami, wszystko.

Część z~nich wróciła do domu, część znalazła inną pracę. I w~końcu niektórzy z~nich ponownie wpadli na swojego starego szefa. Prowadził nową farmę złota, gdzie pracowali dla niego nowi młodzi mężczyźni. Szef był tak przepraszający, że nawet płakał i~błagał o~wybaczenie; jego wierzyciele zażądali pożyczek, a on musiał uciekać, żeby im uciec, ale chciał to wynagrodzić robotnikom, swoim przyjaciołom, których kochał jak synów. Zatrudniłby ich jako starszych członków swojej nowej farmy, za dwukrotność ich starych zarobków, po prostu dajcie mu jeszcze jedną szansę.

Pierwsza wypłata była spóźniona. Jeden dzień. Dwa dni. Trzy dni. Potem szef w~ogóle nie przyszedł do pracy. Część młodszych, nowszych pracowników chciała jeszcze trochę popracować, bo przecież szef był ich drogim przyjacielem. A starzy, ci, których właśnie nabrano po raz drugi, w~końcu przyznali się swoim współpracownikom do tego, co wiedzieli od początku: szef był oszustem i~właśnie ich wszystkich obrabował.

-- Tak to działa. Łamiesz umowę społeczną, druga osoba nie wie, co z~tym zrobić. Nie ma na to scenariusza. Jest chwila, w~której czas się zatrzymuje i~w tym momencie możesz opróżnić mu kieszenie .

Było więcej takich historii, które rozśmieszały wszystkich, serie ,,kekekekeke'', ale kiedy to się skończyło, Wei-Dong poczuł pierwsze drżenie wątpliwości.

-- Co jest? -- zapytała go Jie. 

Była bardzo piękna i~z tego, co mógł zrozumieć, była bardzo znaną osobą z~radia, jakąś lokalną bohaterką dla dziewczyn z~fabryki. Było jasne, że Lu jest w~niej zakochany po uszy, a wszyscy inni też się jej podporządkowali. Kiedy zwróciła na niego uwagę, cały pokój odwrócił się wraz z~nią. Pokój -- mieszkanie w~dziwnej, starej części miasta -- był pełen ludzi, gorący i~głośny od wentylatorów z~komputerów.

-- Po prostu -- powiedział Wei-Dong, machając rękami. 

Nagle był bardzo zmęczony. Nie zdrzemnął się ani nawet nie wziął prysznica, odkąd wymykał się z~portu, a spotkanie z~tymi wszystkimi ludźmi, wideokonferencja z~Big Sister Nor, to było tyle. Jego chiński uciekł przed nim i~zaczął szukać słów. Przełknął, przemyślał to. 

-- Spójrz -- powiedział. -- Chcę pomóc wszystkim robotnikom uzyskać lepszą ofertę, Turkom, rolnikom, dziewczynom z~fabryki. -- Wszyscy ostrożnie skinęli głowami. -- Ale czy to jest to, co tutaj robimy? Czy zdobędziemy jakieś prawa, no wiesz, będąc oszustami? Grabiąc ludzi?

Grupa wybuchnęła głosami. Najwyraźniej rozpoczął starą debatę i~pokój wyłamywał się na swoje tradycyjne strony. Chiński był szybki i~slangowy, i~bardzo szybko zgubił się, a wtedy ogrom tego, co w~końcu zrobił, naprawdę go \textit{uderzył}. Oto on, tysiące kilometrów od domu, nielegalny imigrant w~kraju, w~którym wyróżniał się jak bolesny kciuk. Miał się zaangażować w~przestępcze przedsięwzięcie -- do diabła, był \textit{już }w to zamieszany -- które miało wstrząsnąć światem do fundamentów. A miał tylko 17 lat. Czuł się wysoki na dziesięć centymetrów i~płaski jak naleśnik.

-- Wei-Dong -- powiedział mu do ucha jeden z~chłopców. To był Matthew, który miał zabawny, chudy, znoszony wygląd, ale jego oczy błyszczały inteligencją. -- Chodź, zabierzemy cię stąd. Będą siedzieć nad tym godzinami.

Spojrzał na Matthew od góry do dołu. Technicznie rzecz biorąc, byli członkami gildii, ale kto wie, co to oznaczało? Jaki rodzaj umowy społecznej mieli \textit{naprawdę}, ci nieznajomi i~on?

-- Chodź -- powiedział Matthew z~miłą i~troskliwą twarzą. -- Zabierzemy cię gdzieś spać, znajdziemy ci ubrania.

Ta oferta była zbyt dobra, by z~niej zrezygnować. Matthew wyprowadził go z~mieszkania, z~budynku i~na ulice. Słońce zaszło, kiedy odbywali konferencję, a ciepło zniknęło z~powietrza. Matthew poprowadził go w~górę i~w dół kilkoma podobnymi do labiryntu alejkami, przez gigantyczne bloki mieszkalne, a potem do innego budynku, jeszcze bardziej zaniedbanego niż poprzedni. Weszli po dziewięciu kondygnacjach schodów i~zanim dotarli na właściwe piętro, Wei-Dong miał wrażenie, że się załamie. Uda go paliły, klatka piersiowa unosiła się i~bolała, a pot spływał mu po twarzy, szyi, plecach, pośladkach i~udach.

-- Miałem to samo pytanie co Ty -- powiedział Matthew. -- Kiedy wyszedłem z~więzienia.

Wei-Dong zmusił się, by nie odsuwać się od Matthew. Mieszkanie było wypełnione cienkimi materacami, pokrywającymi prawie całe piętro jak jakiś szalony, gruby dywan. Siedzieli na sąsiednich łóżkach bez butów. Wei-Dong musiał pokazać swoje zaskoczenie, bo Matthew uśmiechnął się smutno. 

-- Poszedłem do więzienia za strajk z~innymi Webbliesami. Nie jestem mordercą, Wei-Dong.

Wei-Dong poczuł, że się rumieni. Wymamrotał przeprosiny.

-- Miałem długą rozmowę z~Big Sister Nor. Oto co mi powiedziała: powiedziała, że tradycyjny strajk, w~którym odbierasz swoją pracę szefom i~żądasz lepszej umowy, tu nie zadziała. Musieliśmy to zrobić, ale musieliśmy też pokazać wszystkim, którzy mieli nas na łasce, że przecenili swoją władzę. Kiedy szefowie mówią: ,,Pobijemy cię'' albo kiedy policja powie: ,,Wsadzimy cię do więzienia'' lub gdy firmy produkujące gry powiedzą: ,,Wyrzucimy cię'', musimy być w~stanie powiedzieć: ,,O nie, nie zrobisz tego!''.

Czysta radość, jaką włożył w~to ostatnie zdanie, sprawiła, że Wei-Dong się uśmiechnął, mimo że był tak zmęczony, że ledwo mógł poruszać twarzą.

Potarł oczy wierzchem dłoni i~powiedział: 

-- Słuchaj, myślę, że moje emocje są dzisiaj na trampolinie. To był bardzo długi dzień. -- Matthew zachichotał. -- Rozumiesz. 

-- Rozumiem. Chciałem tylko powiedzieć, że nie chodzi tylko o~bycie oszustem. Chodzi o~zmianę dynamiki siły w~bitwie. Jesteś wojownikiem, rozumiesz to, prawda? Słyszałem, że grasz uzdrowicielami. Wiesz, jak wygląda nalot z~uzdrowicielem i~bez niego?

Wei-Dong skinął głową. 

-- To zupełnie inna walka -- powiedział. -- Inna taktyka, inne wyczucie.

-- Inna dynamika. Jest matematyka, żeby to opisać, wiesz? Znalazłem artykuł naukowy na ten temat. To fascynujące. Prześlę ci kopię. To, co my tutaj robimy, to zmieniamy dynamikę, układ siły dla robotników na całym świecie. Zobaczysz.

Wei-Dong ziewnął i~machnął słabo pięścią nad ustami.

-- Musisz spać -- powiedział Matthew. -- Dobranoc, towarzyszu. 

Wei-Dong obudził się raz w~nocy, każdy materac był wypełniony, wszyscy chrapali i~oddychali, wciągali nosem i~drapali. Musiało być z~nim w~pokoju z~dwudziestu facetów, ludzki dywan niespokojnej energii, oddechu papierosów i~czosnku, zapachu stóp, zapachu ciała i~stłumionych pomruków. To było zupełnie niepodobne do statku, do jego pokoju w~hotelu Cecil w~Los Angeles, do domu jego rodziców w~hrabstwie Orange \ldots  Ziemia rzeczywiście wydawała się opadać na chwilę, jak miotany sztormem pokład kontenerowca i~przez szaloną, zdezorientowaną chwilę myślał, że nastąpiło trzęsienie ziemi, i~wyobraził sobie wieżowce, które widział zbite w~gromadę po drodze, które zderzają się ze sobą jak domino. Potem ziemia znów się wyprostowała i~panika zniknęła.

Pomyślał o~swojej matce i~wiedział, że będzie musiał znaleźć komputer i~zadzwonić do niej następnego dnia. Wymieniali dużo e-maili, gdy był na statku, wiele wspomnień o~jego ojcu, a on czuł się z~nią bliższy niż przez lata.

Myśl o~matce dawała mu dziwne uczucie spokoju, a nie tęsknotę za domem, której na wpół się spodziewał, i~znów odpłynął wśród pierdów, pomruków i~ludzkich odgłosów ludzi, wśród których się znalazł.

\bigskip
\threeast

Fingerspitzengefuhl Connora wariował. Jak wszyscy producenci gier, miał pokaźny portfel aktywów gier i~instrumentów pochodnych. To nie było do końca sprawiedliwe, obstawianie przyszłości gry w~złoto, gdy masz coś do powiedzenia w~tej przyszłości, daje ci znaczną przewagę nad ludźmi po drugiej stronie zakładów. Ale pieprzyć ich, jeśli nie mogą znieść żartu.

Poza tym jego portfolio było tak duże i~złożone, że sam nie był w~stanie nim zarządzać. Jak wszyscy inni, miał maklera, faceta, który pracował dla jednego z~dużych domów, firmy, która kiedyś była producentem samochodów, zanim zbankrutowała, została wykupiona, wyciśnięta, wykręcona i~sfinansowana, aż to, co pozostało, co miało w~niej jakąkolwiek wartość, to ta część firmy, która pakowała i~sprzedawała frajerom kredyty samochodowe.

A jego broker kochał go, ponieważ za każdym razem, gdy Connor dzwonił z~zamówieniem na pewien złożony instrument pochodny -- powiedzmy, zlecenie kupna polis ubezpieczeniowych o~wartości 300 000 USD na sześciomiesięczne kontrakty terminowe na broń gatling od Zombie Mecha -- warto było założyć się, że w~ciągu sześciu miesięcy w~Zombie Mecha będzie o~wiele mniej dział gatlinga (lub że działo gatlinga otrzyma wzmocnienie, być może amunicję ze zubożonym uranem, która może przebić się przez dziesięć zombie, zanim się zatrzyma), zwiększając cenę broni dużo, dużo w~górę. Pośrednik z~kolei mógłby zarabiać na tej prognozie, pozwalając swoim najlepszym klientom na zawarcie transakcji, kupując polisy ubezpieczeniowe na broń gatlinga, a nawet kontrakty terminowe na broń gatlinga lub kontrakty terminowe na ubezpieczenie na broń gatlinga, zgarniając grube prowizje i~wzbogacając wszystkich innych w~tym samym czasie.

Więc Connor miał przewagę. Czy kto narzekał? Kogo to bolało? 

Z kolei makler Connora lubił dzwonić do niego z~gorącymi wskazówkami na temat innych instrumentów finansowych, które mógłby chcieć rozważyć, instrumentów finansowych, które przyszły mu od innych klientów, zróżnicowanej grupy wysoko postawionych osób, które były wtajemniczone w~różnego rodzaju tajemnice i~wiedzę wewnętrzną. Każdego dnia w~tym tygodniu makler Ira dzwonił do Connora i~prowadził rozmowę, która wyglądała tak:

\noindent Ira: -- Hej, stary, czy to dobry czas? 

\noindent Connor (z roztargnieniem, zamknięty w~walce z~wieloma ekranami i~wieloma kanałami): -- Zawsze mam dla ciebie czas, kolego. Masz moje pieniądze.

\noindent Ira: -- Cóż, doceniam to. Postaram się być szybki. W tym tygodniu mamy nowy produkt, który trochę nas zaskoczył. Pochodzi z~Mushroom Kingdom, co jest dla nas dziwne, ponieważ Nintendo ma tendencję do rozgrywania tych wszystkich rzeczy bardzo blisko i~ciasno, nie pozostawiając nic na stole dla reszty z~nas. Ale mamy linię dotyczącą w~pełni zabezpieczonego pakietu bez ryzyka, którą chciałem najpierw dać wam do kupienia, ponieważ mamy ograniczoną podaż \ldots 

I stamtąd przeszło to w~nieczytelny bełkot bankiera, jak wiązka zautomatyzowanego tekstu generowanego przez wyszukiwanie w~Internecie hasła ,,w pełni zabezpieczonego'' (co oznacza, że mamy zakład, który płaci się, jeśli wygrasz, i~inny, który się opłaca, jeśli przegrasz, więc bez względu na wszystko, wychodzisz do przodu, coś, co wszyscy obiecali i~nikt nigdy nie dostarczył) i~dmuchanie wokół tekstu, który pojawił się we fragmentach wyników wyszukiwania, jak werbalna trąba powietrzna z~,,pełnym zabezpieczeniem'' w~środku.

Chodziło o~to, że Connor był \textit{naprawdę dobry }w mówieniu po bankiersku, a to po prostu się nie składało. Wypłata była gigantyczna, 15 procent w~jednym kwartale, do 45 procent w~idealnym scenariuszu, i~to na ścisłym rynku, na którym większość ludzi była zadowolona z~jednego lub dwóch procent. To był rodzaj obietnicy, którą kojarzył z~szalonymi, ryzykownymi przedsięwzięciami, a nie z~niczym ,,w pełni zabezpieczonym''.

Przerwał entuzjastyczne bełkotliwe wyjaśnienia Iry, mówiąc: 

-- Mówiłeś, że nie ma ryzyka, kolego?

Ira wciągnął powietrze. 

-- Czy ja to powiedziałem? 

-- Tak. 

-- No wiesz, \textit{wszystko }jest ryzykowne. Ale tak, wkładam w~to własne pieniądze. -- Przełknął. -- Nie chcę cię naciskać \ldots  

Connor nie mógł się powstrzymać, parsknął. Ira miał wiele rzeczy ok, ale był nachalnym sukinsynem.

-- Naprawdę! -- Ale brzmiał na skruszonego. -- OK, pozwól, że będę z~tobą szczery. Sam też w~to nie wierzyłem. Nikt z~nas nie wierzył. Wiesz, jacy są sprzedawcy obligacji, widzieliśmy już wszystko. Ale w~biurze są dzieciaki, od razu po szkole. Te dzieciaki mają o~wiele więcej czasu na zabawę niż my \ldots  -- Connor stłumił parsknięcie, ale ledwo. Ostatnim razem, kiedy Ira grał w~grę, był to World of Warcraft, u zarania dziejów. Był kompetentnym, choć pozbawionym wyobraźni brokerem, ale nie był graczem. W porządku, on też nie był hodowcą wieprzowiny, ale wciąż mógł kupować opcje na wieprzowinę. --  \ldots  i~słyszeli o~tym od innych graczy. Zaczęli kupować dla siebie, korzystając ze swoich miesięcznych premii, wiesz, to rodzaj tradycji jest traktowanie tych bonusowych pieniędzy jak grosze z~nieba i~wydawanie ich na długoterminowe zakłady. W każdym razie zaczęli sprzątać, sprzątać i~sprzątać.

-- Więc skąd wiesz, że nie jest to już skończone?

-- O to chodzi. Kilka starych wyjadaczy kupiło to i~wiesz, zaczęli też zbierać. A potem wsiadłem do tego \ldots 

-- Jak dawno temu? 

-- Dwa miesiące temu -- powiedział nieśmiało. -- To zwraca średnio 16 procent miesięcznie. Zacząłem też przenosić do niego swoje długoterminowe oszczędności.

-- Dwa miesiące? Ilu z~twoich innych klientów sprowadziłeś na tę umowę? -- Poczuł dziwną mieszankę gniewu i~uniesienia, jak Ira śmie zachować to dla siebie i~jak dobrze, że miał zamiar się tym podzielić!

-- Nikogo! -- Ira mówił teraz szybko. -- Słuchaj, Connor, wszystkie moje karty są teraz na stole. Jesteś moim najlepszym klientem, jakiego mam. Bez ciebie, do diabła, moja pensja do domu prawdopodobnie spadłaby o~połowę. Jedyny powód, dla którego ci tego nie przyniosłem przedtem, jest, wiesz, nie było już więcej do zrobienia! Za każdym razem, gdy była oferta na te rzeczy, była rozchwytywana w~sekundę.

-- Więc co się stało? Czy wszyscy twoi chciwi kumple najedli się do syta?

Ira roześmiał się. 

-- Nie bardzo! Ale wiesz, jak to jest, jak tylko coś wystartuje, jak te kupony, wielu ludzi próbuje wymyślić, jak zarobić z~nich więcej. Okazuje się, że jest bank, jeden z~tych na morzu, który jest trochę jak prywatna fortuna księcia z~Dubaju, a książę ma wątpliwości. Bank sprzedaje bardzo długie zakłady przeciwko tym obligacjom na świetnych warunkach. To zakłady na rok i~płacą \textit{dużo}, jeśli obligacje się nie rozbiją. Więc teraz jest pewna niepewność w~puli, a niektórzy rzucają się, obstawiając, że książę wie coś, czego oni nie wiedzą, kupując papiery i~sprzedając swoje obligacje. My poszliśmy lepiej: mamy pływającą pulę zabezpieczonych pakietów, które się równoważą zakłady księcia i~te obligacje, więc bez względu na to, co się stanie, jesteś na zielono. Kupujemy lub sprzedajemy codziennie w~oparciu o~stawki na każdej z~nich. To jest  \ldots 

-- Bez ryzyka? 

-- Praktycznie bez ryzyka. Absolutnie.

Usta Connora były suche. Tutaj coś się działo, coś wielkiego. Jego umysł był w~stanie wojny z~samym sobą. Finanse były grą, największą grą, a zasady ustalali gracze, a nie projektant. Czasami zasady oszalały i~pojawiało się małe szaleństwo, w~którym mały zakład mógł dać niewyobrażalne wygrane. Wiedział, jak to działa. Oczywiście, że tak. Czy nie ścigał farmerów złota po dziewięciu światach, próbując odnaleźć ich własne małe kieszenie o~wysokim zwrocie i~wywrócić ich na lewą stronę? W tym samym czasie, po prostu nie było czegoś takiego jak darmowy lunch. Coś, co wyglądało zbyt pięknie, aby mogło być prawdziwe, prawdopodobnie było zbyt piękne, aby mogło być prawdziwe. Wszystkie te i~wszystkie inne powiedzonka, z~którymi dorastał, cały ten zdrowy rozsądek, którym obdarzyli go jego prości rodzice, oni ze swoim domem w~małym miasteczku, bez kredytu hipotecznego i~rozsądnymi funduszami emerytalnymi, które kazałyby im odcinać kupony i~chodzić na wyprzedaże przez resztę ich życia.

-- Dwadzieścia kawałków -- wypalił. 

To było dużo, ale mógł sobie z~tym poradzić. Zarobił więcej na swoich inwestycjach w~ciągu ostatnich 90 dni. Mógłby to nadrobić w~ciągu następnych 90 dni, gdyby \ldots 

-- \textit{Dwadzieścia}? Żartujesz? Connor, spójrz, takie rzeczy zdarzają się raz w~życiu! Przyszedłem do ciebie \textit{pierwszy}, kolego, żebyś mógł wejść do środka. Cholera, kolego, sprzedam ci dwadzieścia kawałków warte tych rzeczy, ale powiem ci, co \ldots 

To sprawiało, że czuł się mały, chociaż wiedział, że miało to \textit{sprawić}, że poczuł się mały. To było tak, jakby było dwóch Connorów, chłodny, racjonalny i~emocjonalny, gorzko walczący o~kontrolę nad swoim ciałem. Racjonalny wygrał, choć było to trudne zadanie.

-- Dwadzieścia to wszystko, co mam teraz w~gotówce -- skłamał, emocjonalny Connor wygrywając to małe ustępstwo. -- Gdybym mógł sobie pozwolić na więcej \ldots  

-- Och! -- powiedział Ira, a Connor usłyszał w~jego głosie uśmiech. -- Connor, kolego, nie robię tego zbyt często i~byłbym wdzięczny, gdybyś zachował to dla siebie, ale co powiesz na to, że obiecam ci, że twoje normalne transakcje na dzisiaj dostaną dodatkowe, uch, zróbmy jeszcze 20, w~sumie 40 tysięcy. Czy chciałbyś przeznaczyć ten zysk na te szczenięta?

Connorowi wyschło w~ustach. Wiedział, jak to działa, ale już dawno zrezygnował z~bycia tego częścią. Było to najstarsze oszustwo brokerskie na świecie: każdego dnia brokerzy dokonywali wielu transakcji pozaksięgowych, kupując akcje, obligacje i~instrumenty pochodne, mając przeczucie, że pójdą w~górę. Bycie ,,pozaksięgowym'' oznaczało, że te transakcje nie były przypisane do żadnego konkretnego konta klienta; pieniądze na ich zakup wychodziły z~konta głównego domu maklerskiego.

Pod koniec dnia niektóre -- może wszystkie -- z~tych transakcji wyszłyby na jaw. Niektórzy -- może wszyscy -- wyszliby na minusie. I wtedy właśnie zaczynała się magia. Dzięki wstecznemu datowaniu ksiąg broker mógł przypisać gówniane transakcje do gównianych klientów, tandetnych klientów lub dużych, zamkniętych, wolno poruszających się klientów, takich jak luźno zarządzane posiadłości dla dawno zmarłych ludzi, których majątek był przechowywany w~zaufaniu. Zyski można by przypisać najlepszym klientom brokera, takim jak jakiś miliarder, z~którym broker miał nadzieję zrobić więcej interesów. W ten sposób każdy broker ma codziennie pewną dozę dyskrecji w~wyborze, kto będzie zarabiał pieniądze, a kto je stracił. To była po prostu lepsza wersja baristy w~kawiarni, która od czasu do czasu podawał swoim stałym bywalcom dużą zamiast średniej, bez pobierania opłat za upgrade. Partnerzy, którzy prowadzili biura maklerskie, wiedzieli, że tak się dzieje, podobnie jak wielu klientów. Niemożliwe było udowodnienie, że w~ten sposób straciłeś lub zyskałeś pieniądze, chyba że Twój broker powiedział Ci o~9:15 we wtorek rano, że Twoje konto będzie miało dodatkowe 20 000 USD do godziny 17:00.

Ira właśnie podjął duże ryzyko, mówiąc Connorowi, co zamierza dla niego zrobić. Teraz kiedy miał to przyznanie, teoretycznie mógł aresztować Irę za oszustwa związane z~papierami wartościowymi. To znaczy, dopóki nie da Irze zielonego światła, w~którym to momencie \textit{obaj }byliby winni, tym razem.

A tam racjonalny i~emocjonalny Connor walczył na krawędzi między bogactwem i~spiskiem a bezsensowną, beznadziejną uczciwością. Wpadli na stronę konspiracji. W końcu Connor i~broker naginali zasady za każdym razem, gdy Connor zlecał transakcję na jeden z~kontraktów futures Coca Cola Games. To było dokładnie to samo, tylko więcej.

-- Zrób to -- powiedział. -- Dzięki, Ira.

Oddech Iry świsnął przez telefon i~Connor zdał sobie sprawę, że pośrednik wstrzymywał oddech i~czekał na jego odpowiedź, czekając, aby dowiedzieć się, czy nie posunął się za daleko. Sprzedawca naprawdę chciał mu sprzedać ten pakiet.

Później, w~Centrum Dowodzenia, Connor obserwował swoje kanały i~myślał o~tym, i~coś wydawało się \ldots  \textit{dziwaczne}. Dlaczego Ira był taki chętny? Ponieważ Connor był takim świetnym klientem, a Ira myślał, że jeśli zarobi Connorowi mnóstwo pieniędzy, to Connor odda mu je, aby dalej inwestował, zarabiał dla niego coraz więcej i~coraz więcej prowizji dla brokera?

A teraz, gdy jego anteny były pobudzone, zaczął dostrzegać wszelkiego rodzaju duchy w~swoich kanałach, małe wzmianki o~złocie i~elitarnych przedmiotach zmieniających właściciela w~zabawny sposób, wycenionych zbyt wysoko lub niewystarczająco wysoko, wszystko nie w~porządku z~rzeczywistą wartością w-grze. Oczywiście, kto wiedział, jaka naprawdę może być wartość czegokolwiek w~grze? Powiedzmy, że producenci gry postanowili sprawić, by działo Zombie Mecha gatlinga strzelało amunicją ze zubożonego uranu, za sześć miesięcy. Proste obliczenia wykazały, że broń gatlinga zyskała na wartości w~ciągu sześciu miesięcy, ponieważ umożliwiłoby to Mechom przedzieranie się przez gigantyczne hordy zombie, nie dając się obezwładnić. Ale co, jeśli to sprawi, że gra będzie \textit{zbyt }łatwa i~wielu graczy odejdzie? Gdy twoi kumple przeszli do Anthills and Hives i~zaczęli grać zespołowo, walcząc z~inteligencję rojową, czy chciałbyś kręcić się w~pobliżu Zombie Mecha, samotny i~opuszczony, strzelający z~pistoletu gatlinga do zombie? Czy zombie przestaną być zabawnymi celami i~zaczną być jedynie zbiorami warczących pikseli?

Potrzeba było subtelnego palca wróżki, aby naprawdę przewidzieć, co stanie się z~grą, gdy osłabisz lub wzmocnisz jedną klasę postaci, broń lub potwora. Każda taka zmiana była uważnie obserwowana przez producentów gry przez tygodnie, przez całą dobę, i~zmieniali charakterystykę zmiany z~minuty na minutę, starając się doprowadzić grę do równowagi.

Kanały opowiadały historię. Tam, w~krainie gier, było piekielnie dużo aktywności, wymiany tam i~z powrotem, i~to go martwiło. Zaczął pytać innych uczestników gry, czy zauważyli coś niezwykłego, ale wtedy coś innego wyskoczyło z~jego kanałów: tam! Farmerzy złota!

Szukał ich wszędzie i~znajdował. Farma złota miała wiele sygnatur, które można było zauważyć przy odpowiednim feedzie. Za każdym razem, gdy ktoś logował się z~tajemniczego azjatyckiego adresu IP, podchodził do najbliższego punktu handlowego, zdejmował każdy kawałek zbroi i~ozdoby i~sprzedawał je, a następnie zabierał całą otrzymaną gotówkę i~całą zawartość jej banku gildii i~przekazywał ją jakiemuś noob pierwszego poziomu na darmowym koncie próbnym, który zaczął się zaledwie godzinę wcześniej, który z~kolei przekazywał pieniądze kilku setkom kolejnych noobów, którzy szybko rozpraszali się i~deponowali we własnych bankach gildii, cóż, to było pewny, że znalazłeś jakiegoś farmera złota, który włamywał się na konta. Do diabła, w~połowie przypadków można było stwierdzić, kim byli farmerzy, po prostu patrząc na nazwy, które nadali ich gildiom: prawdziwi gracze mieli albo heroiczne (,,Dziki Grom'') albo ironiczne (,,Pasterze Nerfów''), albo tytularne (,,Rozbójnicy Jima''), ale rzadko wybierali ,,asdfasdfasdfasdfasdfasdfasdfasdf2329'' lub, na litość boską, 707A\-55DF\-0D7E\-15BB\-B9FB\-3BE1\-6562\-F22C\-02\-6A\-88\-2E\-40\-16\-4C\-7B\-14\-9B\-15\-DE\-71\-37\-ED\-1A.

Ale gdy tylko poprawił swoje feedy, aby ich złapać, rolnicy wymyślili, jak ich uniknąć. Gildie zyskały dobre imiona, zhakowani gracze zaczęli zachowywać się bardziej wiarygodnie -- prowadząc na wpół dialogi z~postaciami, które wzmacniali wszystkimi swoimi dobrami -- a gangi, które skupiały się na każdym przypadkowym macierzystym złożu w~grze, wykonywały dużo realistycznego gadania i~rozmawiania w~łamanym angielskim. Coraz częściej gracze logowali się za pomocą kart przedpłaconych przekierowywanych z~USA przez amerykańskie proxy, co czyniło ich nie do odróżnienia od lukratywnej sprzedaży amerykańskim dzieciakom, które były skłonne do rozpoczęcia gry, kupując kilka kart przedpłaconych wraz z~colą i~gumą do żucia w~sklepie spożywczym. Te dzieciaki miały okres uwagi komara, a jeśli wyłączyłeś je po pomyleniu ich z~farmerem złota, opuszczali i~szli do konkurencji, nigdy nie wracali do twojej gry czy do księgowości.

To było niesamowite, jak szybko informacje rozchodziły się wśród tych dziwaków. Cóż, nie niesamowite. W końcu informacje rozchodziły się wśród zwykłych graczy szybciej, niż można by sądzić, to było świetne, prawie nie trzeba było kiwnąć palcem ani wydawać ani grosza na marketing, kiedy wydałeś kilka nowych elitarnych przedmiotów lub odsłoniłeś nowy świat. Gracze rozmawialiby o~tym za ciebie, rozpowszechniając informacje z~prędkością plotek. I ten sam telegraf z~dżungli biegł przez podziemie farmerów, widział, jak działa.

I było ich więcej, mała dwudziestoosobowa gildia, wszyscy mielący i~szarpiący w~tej samej kampanii. Były to świeże postacie, stworzone dwa dni wcześniej i~zostały stworzone przez graczy, którzy wiedzieli, co robią, była to po prostu idealna równowaga między rezzerami, czołgami i~casterami, dobra mieszanka AOE i~broni do walki w~zwarciu. Wyrównali się cholernie szybko, wyciągnął kilka informacji na temat niektórych pionków, poczuł mrowienie w~palcach, gdy gra gasła jak płomień na wietrze. Zainstalował pakiety kryminalistyki, ignorując wycie protestów zespołu administratorów, którzy pokazywali mu wykres po wykresie na temat tego, jak prowadzenie historii, którą chciał zobaczyć, wpłynie na wydajność serwera. Dostał swoje statystyki, ale dopiero po tym, jak obiecał, że będzie ich używał oszczędnie.

I oto było: gracze zrównali się nawzajem, przechodząc do obszaru turniejowego PvP -- Player versus Player -- i~wielokrotnie zabijając się nawzajem. Gdy tylko jeden z~nich wspiął się na poziom, stałby niebroniony i~pozwolił drugiemu graczowi szybko go zabić. Gra dawała megapunkty za zabicie gracza na wyższym poziomie. Kiedy dwaj gracze zamienili się, zamienili się miejscami i~wspięli się, jeden po drugim, na wyżyny, których osiągnięcie normalnym graczom zajęłoby wieczność.

Kampania, którą prowadzili, była prosta: zgarnięcie mieszanki skrzydełek wróżek ziemi i~niektórych czapek grzybowych, oddanie ich mistrzowi eliksirów, który zapłacił im złotem. Nie było to nic specjalnego i~było trochę poniżej ich poziomu, ale kiedy wyliczył zwroty w~złocie i~doświadczeniu na godzinę, zobaczył, że ktoś nieostrożnie stworzył misję, która wypłaciłaby prawie trzykrotnie więcej, niż zakładała zwykła kampania. Potrząsnął głową. \textit{Jak do diabła oni to rozgryźli}? Musiałbyś sporządzić wykres każdej drobnej misji w~grze, a były \textit{dziesiątki tysięcy misji}, stworzonych przez projektantów, którzy używali algorytmów oprogramowania do rozłożenia podstawowego scenariusza na setki wariantów.

I tam byli, radośnie zbierając kapelusze grzybów, zabijając brązowe wróżki i~wyrywając im skrzydła. Od czasu do czasu natrafiali na większego potwora, który zabłąkał się do ich strefy, a oni z~łatwością go wykańczali.

Jego palec drżał nad makro, które miało zawiesić ich konta i~wystartować z~serwera. Nie poruszył się.

Podziwiał ich, to był problem. Robili coś sprawnie, cicho i~dobrze, bez zbędnego zamieszania. Rozumieli grę prawie tak dobrze jak on, bez korzyści z~Centrum Dowodzenia i~jego licznych kanałów. On  \ldots 

Zalogował się.

Wybrał avatara, które wzmocnił do poziomu 43, w~połowie drabiny do maksimum, czyli 90. Regulus był elfem uzdrowicielem, wysoki i~chudy jak bicz, z~ogromnym plecakiem wypełnionym ziołami i~eliksirami. Był nominalnym członkiem jednej z~gildii średniej wielkości, jednej z~tych, które za niewielką opłatą zaakceptowałyby nawet każdego gracza, która oferowała kursy szkoleniowe, bankowość gildii, zaplanowane wydarzenia, a wszystko to przy wdzięcznej sankcji Coca Coli. Właściwi ludzie.

\noindent {\textgreater} Witaj

Dwa miesiące wcześniej gracze kontynuowaliby swoją misję, beztrosko go ignorując. Ale to była jedna z~cech ostrzegawczych, których szukały jego feedy, aby wybrać rolników. Zamiast tego wszystkie te bajki machały do niego i~robiły małe emotikony, z~których niektóre były całkiem dobrymi niestandardowymi zadaniami, w~tym ruchami tanecznymi, wyszukaną miną i~innymi gestami. Gdyby jego kanały nie wybrały tych żartownisiów jako farmerów, uznałby ich za hardkorowych graczy. Ale tak naprawdę nic z~nim nie rozmawiali. Prawie na pewno byli Chińczykami, a angielski byłby dla nich trudny.

\noindent  {\textgreater} Chcesz grupować?

Zaproponował im naprawdę śliwkowe zadanie, które miało szalenie wysokie złoto i~nagrodę w~postaci doświadczenia za stosunkowo bliski cel: odzyskanie run Dvalinna z~głębokiej jaskini, do której musieliby się przedrzeć, zabijając bandę krępych krasnoludów i~po drodze kilku przyzwoitych szefów. Zadanie było powiązane z~innym, które doprowadziło do walki z~Fenrisulfrem, jednym z~największych bossów w~Wojownikach Svartalfaheim, megabossem, którego trzeba było pokonać ogromną drużyną, ale który nagrodził cię ogromnym skarbem. Całość była przynętą na farmerów, którą przygotował specjalnie na tego rodzaju misję.

Po przyzwoitej przerwie -- krótkiej, ale wystarczająco długiej, by gracze mogli zastanowić się nad maszynowym tłumaczeniem tekstu zadania -- chętnie dołączyli, wysyłając proste podziękowania przez tekst.

Udawał, że nie widzi nic dziwnego w~ich milczeniu, gdy zbliżali się do celu, ale w~międzyczasie skoncentrował się na obserwowaniu ich z~bliska, próbując wyobrazić sobie ich wokół stołu w~zadymionej kawiarni w~Chinach, Wietnamie, Kambodży czy Malezji, dwudziestu chudych chłopcy z~tłustymi włosami i~pryszczami, z~papierosami w~kącikach ust, mrużącymi oczy wokół kłębka dymu. Może byli w~więcej niż jednym miejscu, dwóch lub nawet trzech grupach. Prawie na pewno mieli jakiś kanał zwrotny, czy to głos, tekst, czy po prostu krzyczeli na siebie nad stołem, ponieważ poruszali się z~dobrą koordynacją, ale z~wystarczającym indywidualizmem, że wydawało się nieprawdopodobne, aby to wszystko był jeden facet z~dwudziestką botów.

\noindent {\textgreater} Skąd jesteś?

Musiał zdawać sobie sprawę, że prawdopodobnie próbowali dowiedzieć się, czy jest z~gry, a jeśli ułatwi im sprawy, może ich ostrzec.

Jeden z~graczy, rzucający ogr z~ogromną maczugą i~bandolier z~mistycznych czaszek wyrytych runami, odpowiedział

\noindent {\textgreater} jesteśmy Chińczykami, mam nadzieję, że to w~porządku

To było bardziej szczere, niż się spodziewał. Inne grupy, do których zbliżył się z~tą samą sztuczką, były o~wiele bardziej zamknięte, twierdząc, że pochodzą z~nieprawdopodobnych miejsc na środkowym zachodzie, takich jak Sioux Falls, miejsc, które wydawały się wybrane przez losowe kliknięcie mapy USA.

\noindent {\textgreater} Chiny!

pisał,

\noindent {\textgreater} Radzisz sobie całkiem nieźle z~angielskim!

Ogr -- książę Simon, według jego statystyk -- zrobił mały ukłon.

\noindent {\textgreater} Uczyłem się w~szkole. Moi koledzy nie są tak dobrzy.

Connor pomyślał o~tym, kim udaje: młodym zawodnikiem w~wielkim amerykańskim mieście, takim jak LA. Co by powiedział tym ludziom?

\noindent {\textgreater} Czy jest tam późno?

\noindent {\textgreater} Tak, po obiedzie. Zawsze gramy po obiedzie.

\noindent {\textgreater} Brzmi jak świetna zabawa! Chciałbym mieć dużą grupę przyjaciół, którzy byli wolni po obiedzie. To zawsze praca domowa praca domowa praca domowa

Fikcyjna postać Connora wyostrzała się teraz dla niego, samotny dzieciak z~liceum w~La Jolla lub San Diego, gdzieś nad oceanem, gdzieś biały, z~klasy średniej i~odizolowany. Gdzieś bez chodników. Dzieciak, który może natknąć się na śliwkową misję, żyje runami Dvalinna i~musi zebrać grupę nieznajomych, aby z~nim poszli.

\noindent {\textgreater} to dobry czas -- odpowiedział ogr. Pauza.

\noindent {\textgreater} Mój przyjaciel chce wiedzieć, czego się uczysz?

Jego osobowość umieściła odpowiedź w~jego głowie.

\noindent {\textgreater} Niedługo skończę szkołę. Ubiegałem się o~inżynierię lądową w~kilku szkołach. Mam nadzieję, że się dostanę!

Ogr powiedział:

\noindent {\textgreater} Przed wyjazdem z~domu byłem inżynierem budownictwa. Zaprojektowałem mosty, pięć mostów. Dla systemu pociągów dużych prędkości.

Connor zmienił mentalnie swój wizerunek chłopców na młodych mężczyzn, dorosłych.

\noindent {\textgreater} Kiedy opuściłeś dom?

\noindent {\textgreater} 2 lata. Nigdy więcej pracy. Myślę, że niedługo wrócę do domu. Mam tam rodzinę. Mały synek, tylko 3

Ogr przesłał mu obraz. Uśmiechnięty chiński chłopiec w~marynarskim garniturze, zębaty, trzymający ociekający lodami rożek jak pałeczkę, machający nim jak dyrygent.

Fikcyjny 17-latek Connora nie zareagował na zdjęcie, ale jego 36-letnie ja tak. Ojciec zostawiający syna, odjeżdżający w~poszukiwaniu pracy. Connor nigdy nie musiał nikogo wspierać, ale dużo o~tym myślał. W świecie Connora, w~którym ludzkimi motywami kierowała zazdrość i~strach, zdjęcie tego dziecka było sejsmiczne, trzęsienie ziemi wstrząsało rzeczami i~sprawiało, że meble spadły na podłogę i~się roztrzaskały. Walczył o~znalezienie swojego charakteru.

\noindent {\textgreater} Słodkie! Musisz za nim tęsknić

\noindent {\textgreater} Bardzo. To jak bycie w~wojsku. Będę to robił przez kilka lat, a potem wrócę do domu.

Co za świat! Oto ten inżynier budowlany, znakomity, zakochany ojciec, mieszkający daleko, cały dzień pracujący nad gromadzeniem wirtualnych skarbów, bawiący się w~kotka i~myszkę z~Connorem i~jego ludźmi.

\noindent {\textgreater} Jaką masz radę dla kogoś, kto zajmuje się inżynierią lądową?

Ogr wybuchnął wielkim śmiechem.

\noindent {\textgreater} Nie próbuj znaleźć pracy w~Chinach

Connor też się roześmiał, i~poprowadził grupę do runów Dvalinna, zatracając się w~sztuce, nawet gdy starał się zachować ostrożność i~spostrzegawczość. Niektórzy z~jego kolegów gamerunnerów od czasu do czasu spoglądali mu przez ramię, obserwowali, jak prowadzą misję, i~robili krótkie uwagi. Wśród twórców gier sama gra była nieco zlekceważona, co było czymś, w~czym mogą grać znaki. Prawdziwą grą, wielką grą, była gra polegająca na projektowaniu gry, gra polegająca na podkręcaniu wszystkich zmiennych w~gigantycznej klatce chomika, za którą płacili wszyscy frajerzy.

Ale Connor nigdy nie zapomniał, jak doszedł do gry, skąd pochodzą jego równania: z~\textit{zabawy}, tysięcy godzin spędzonych na światach, wchłaniania ich fizyki i~rzeczywistości palcami, uszami i~oczami. Według niego nie mogłeś wykonywać swojej pracy w~grze, chyba że też w~nią grałeś. Zaznaczył zasmarkane słowa, zauważył, kto je dostarczył, i~spisał swoje mentalne oszacowanie każdego z~nich o~kilka punktów.

Teraz znajdowali się w~lochu, który właśnie złożył, ale który mimo to sprawił, że naprawdę mu się podobało. Jako gildia najazdowa Chińczycy byli znakomici: skoordynowani, sprytni, szybcy. Miał tendencję do myślenia o~farmerach złota jako bezmyślnych droidach, powtarzających zadanie postawione im przez jakiegoś szefa, który pokazał im, jak używać myszy i~odszedł. Ale oczywiście farmerzy złota grali cały dzień, każdego dnia, nawet bardziej niż najbardziej zagorzali gracze. \textit{Byli }zagorzałymi graczami. Hardkorowych graczy, których przysiągł wyeliminować, ale nie mógł zapomnieć, że \textit{byli }hardkorowymi.

Wywalczyli sobie drogę do wielkiego bossa, a drużyna była tak dobra, że Connor nie mógł się powstrzymać, sięgnął do wnętrzności gry i~wywalił bossa z~piekła rodem, znacznie podnosząc swój poziom i~wyposażając go w~całą masę specjalnych ataków z~biblioteki Nasties, którą trzymał w~swoim prywatnym miejscu pracy. Teraz szef był niesamowicie onieśmielający, wyzwanie, które wymagało bezbłędnej gry całego zespołu.

\noindent {\textgreater} O nie

napisał.

\noindent {\textgreater} Co zamierzamy zrobić?

I ogr rzucił się do akcji, a gracze utworzyli dwa szeregi, ci z~atakami wręcz w~awangardzie, czarodzieje, uzdrowiciele, atakujący dystansowi i~atakujący obszarowo z~tyłu, szukający półek i~innych wysokich miejsc poza zasięgiem bossa. ogromnego, strasznego wilka z~wieloma zaklęciami dystansowymi, a także okrutnym ugryzieniem i~potężnymi łapami, które potrafiły uderzyć i~przygwoździć gracza, dopóki wilk nie zdołał przyłożyć do niego paszczy.

Boss miał grupę mniejszych wojowników, krasnoludów, którzy w~wielkiej obfitości wylewali się z~jaskiń prowadzących do centralnej jaskini, nękając tylny szereg i~przechwytując główne ataki zgromadzonej straży przedniej. Jako uzdrowiciel i~rezzer Connor biegał tam i~z powrotem, szukając bezpiecznych miejsc, w~których mógłby usiąść, medytować i~rzucać leczniczą energię na walczących na czele, którzy przyjmowali niesamowite obrażenia od wielkiego bossa i~jego sługusów. Stracił koncentrację na sekundę i~dwaj krasnoludy uderzyły go toporami, wysoko i~nisko, i~znalazł się bez czapki, rozciągnięty na dnie jaskini, z~kolejnymi złoczyńcami w~drodze.

Serce waliło mu jak młotem, to stare uczucie, które przypominało mu, że jego ciało nie potrafi odróżnić podekscytowania na ekranie od niebezpieczeństwa w~prawdziwym świecie, a kiedy inny gracz, jeden z~Chińczyków, z~którymi w~ogóle nie rozmawiał, uratował go, poczuł przypływ wdzięczności, która była całkowicie szczera, pochodząca z~kręgosłupa i~żołądka, a nie z~głowy.

W końcu 12 z~20 graczy zostało nieodwracalnie zabitych w~bitwie, odradzając się w~jakimś odległym miejscu, zbyt daleko, by dotrzeć do nich przed zakończeniem bitwy. Szef w~końcu zawył, potężny dźwięk, który sprawił, że stalaktyty grzmiały z~sufitu i~rozpryskiwały się w~rozpryski ostrych skał, które zadawały niewielkie obrażenia ocalałym z~ich drużyny, szkody, od których i~tak się wzdrygali, ponieważ wszyscy biegli na czerwono. Punkty doświadczenia były niesamowite -- wszedł na pełny poziom -- i~było kilka bardzo dobrych dropów. Prawie sięgnął do swojego miejsca pracy, aby dodać kilka nowych, aby nagrodzić swoich towarzyszy za ich umiejętności i~odwagę, przymusowo przypominając sobie, że \textit{nie jest po ich stronie}, że to badania i~infiltracja.

\noindent {\textgreater} Jesteście wspaniali!

Ogr  ukłonił i~odtańczył mały taniec zwycięstwa, kolejny niestandardowy numer, który był jednocześnie pełen wdzięku i~zabawny.

\noindent {\textgreater} Dobrze grasz. Powodzenia z~Twoimi studiami.

Palce Connora zawisły nad klawiszami.

\noindent {\textgreater} mam nadzieję, że wkrótce zobaczysz swoją rodzinę

Ogr zareagował krótkim uściskiem, co sprawiło, że Connor przez chwilę poczuł się zawstydzony tym, co zrobił dalej. Ale zrobił to. Dodał całą gildię do swojej listy obserwowanych, aby każda wiadomość i~ruch były rejestrowane, tłumaczone maszynowo na angielski. Każda transakcja, którą dokonali -- całe złoto, które sprzedali lub rozdawali -- byłaby śledzona w~ramach wysiłków Connora mających na celu rozwikłanie złożonych, wielotysięcznych sieci, które były wykorzystywane do przechowywania, konwersji i~dystrybucji dóbr z~gier. Miał już setki kont w~bazie danych i~przy kursie, w~jakim zmierzał, do końca tygodnia miałby tysiące -- a była dopiero środa.

\bigskip
\threeast

Policja zrobiła nalot na studio Jie, kiedy ona i~Lu jedli pierogi i~patrzyli sobie w~oczy. To było jedno z~jej studiów zapasowych, ale przepracowali tu dwa dni z~rzędu, a mieli pracować przez trzeci. Było to pogwałcenie podstawowych zasad bezpieczeństwa, ale wiele mieszkań Jie szybko zapełniało się Webbliesami, którzy zrezygnowali z~pracy na farmie z~frustracji i~przyłączali się do pełnoetatowych wysiłków, aby zgromadzić złoto i~skarby na potrzeby Planu.

Sklep z~pierogami był prowadzony przez młodą kobietę, która opiekowała się swoim dwuletnim synem i~czteroletnią córką swojej siostry, ale mimo to zawsze była pogodna, gdy wchodzili, choć miała skłonność do sugestywnych uwag o~młodej miłości i~niebezpieczeństwach wczesnego rodzicielstwo.

Właśnie wręczała im rachunek -- Lu po raz kolejny udawał, że sięga po niego, choć nie tak szybko, żeby Jie nie mogła mu go wyrwać i~zapłacić sama, ponieważ to ona miała wszystkie pieniądze w~związku -- kiedy jego telefon zwariował.

Wyciągnął go, spojrzał na jego twarz i~zobaczył, że to Big Sister Nor, dzwoni z~numeru, którego zgodnie z~protokołem nie powinna używać przez następne 24 godziny. Oznacza to, że martwiła się, że jej stary numer został narażony, co oznaczało, że było źle. Odwrócił się do ściany i~zakrył ręką słuchawkę, odpowiedział.

-- Wei?

-- Zostałeś spalony. -- Był to The Mighty Krang, którego tajwański akcent był natychmiast rozpoznawalny. -- Oglądamy teraz kamery internetowe w~studiu. Dziesięciu gliniarzy rozrywa mieszkanie na strzępy.

-- Gówno! -- powiedział to tak głośno, że czterolatek zachichotał śmiechem, a dama z~pierogami spojrzała na niego spode łba. Jie przysunęła się do niego i~przyłożyła swój policzek do jego, od razu poczuł się trochę lepiej w~jej towarzystwie, i~szepnęła: 

-- Co jest?

-- Wszyscy jesteście bezpieczni, prawda? 

Pomyślał o~tym przez chwilę. Wszystkie ich dyski były zaszyfrowane i~po dziesięciu minutach bezczynności blokowały się samoczynnie. Policja nie byłaby w~stanie odczytać niczego z~żadnej z~maszyn. Miał przy sobie dwa komplety legitymacji, ten aktualny, który miał zostać spłukany tego samego dnia zgodnie z~normalną procedurą, a drugi, ukryty w~kieszeni wszytej w~wewnętrzną stronę nogawki spodni. Tak samo dla jego obecnych i~następnych kart SIM, jedna załadowana do jego obecnego telefonu i~sakiewka nowych w~kolejności planowanego użycia włożona do szczeliny w~jego pasku. Zakrył ustnik i~szepnął do Jie: 

-- Studio poszło. -- Wciągnęła powietrze przez zęby. -- Czy jesteśmy wszyscy bezpieczni? 

Pstryknęła językiem. 

-- Nie martw się o~mnie, robię to o~wiele dłużej niż ty. 

Zaczęła metodycznie przeklinać pod nosem, przekopując torebkę, wymieniając identyfikatory i~otwierając telefon, by zmienić kartę SIM. 

-- Miałam naprawdę fajne rzeczy w~tym miejscu -- powiedziała. -- Dobre ciuchy. Mój ulubiony mikrofon. Jesteśmy takimi idiotami. Nigdy nie powinienem nagrywać tam dwa razy z~rzędu. 

Mighty Krang musiał to usłyszeć, bo zachichotał. 

-- Wygląda na to, że oboje jesteście w~porządku?

-- Cóż, Jiandi nie będzie dziś w~stanie wyjść na antenę -- powiedział.

-- Pieprzyć to -- powiedziała Jie. Wzięła od niego telefon. -- Powiedz Big Sister Nor, że dziś wieczorem będziemy nadawać o~zwykłej porze. Normalna usługa, bez przerw.

Lu nie usłyszał odpowiedzi, ale po ponuro zadowolonym wyrazie twarzy Jie widział, że Mighty Krang ją pochwalił. To był pomysł Big Sister Nor, aby wszystkie studia były wyposażone w~kamery internetowe, do których mieli dostęp Webblies, tylko we frontowych pokojach. To było trochę dziwne, próbować zignorować te wszystkowidzące oko kamery przykręconej nad drzwiami. Ale kiedy sypiasz po 20 osób w~jednym pokoju, łatwo jest porzucić swoje wyobrażenia na temat prywatności, ale mimo to Lu i~Jie siedzieli daleko od siebie podczas nadawania i~wkradali się do łazienki, aby potem się pocałować.

A teraz kamery internetowe się opłaciły. Odebrał telefon i~słuchał, jak The Mighty Krang relacjonuje odtwarzanie wideo, gliniarze wyważają drzwi, zabezpieczając przestrzeń. Następnie zespół dowodowy, który włączył baterie do kabli zasilających komputerów, aby można je było odłączyć bez wyłączania (Lu był wdzięczny, że Big Sister Nor wydała dekret, że cały ich sprzęt musi być skonfigurowany do odmontowywania i~ponownego szyfrowania dysków, gdy jest bezczynny), pobrał odciski palców i~DNA. Oczywiście mieli już DNA Lu, ponieważ wywęszyli jedno z~innych mieszkań Jie. Ale Jie wyprzedziła to znacznie: miała mały kieszonkowy odkurzacz, przeznaczony do usuwania okruchów i~mazi z~klawiatur, i~ukradkiem odkurzała siedzenia za każdym razem, gdy jechała pociągiem lub autobusem, wysysając losowe DNA tysięcy osób ludzi, które ostrożnie rozrzucała po lokalu, kiedy tam wchodziła. Śmiał się z~tej pomysłowości, a ona powiedziała mu, że czytała o~tym w~powieści.

Zespół dowodowy przyniósł kamerę panoramiczną i~umieścił ją na środku pokoju, a policja natychmiast zniknęła, gdy okrążyła ciasny, precyzyjny mechaniczny krąg, tworząc otaczający obraz pomieszczenia o~wysokiej rozdzielczości. Potem policjanci wkroczyli z~powrotem, bez papierowych kaloszy, i~włożyli każdy skrawek papieru, każdy kawałek nośnika optycznego, magnetycznego i~optycznego do większej liczby toreb, a potem zniszczyli to miejsce.

Pracując łomami i~małymi nożami, a zaczynając od kąta pod frontowymi drzwiami, metodycznie rozbijali każdy mebel, każdą płytkę podłogową, każdą ścianę z~gipsu, zamieniając to wszystko w~kawałki nie większe niż karty do gry, układając je w~stosy za nimi, gdy przechodzili dalej. Pracowali niemal w~ciszy, bez pośpiechu, i~wydawali się nie lubić tego zadania. To nie był wandalizm, to była absolutna zagłada. Policjanci mieli regulaminowe krótkie włosy, identyczne niebieskie mundury, papierowe maski i~kevlarowe rękawiczki. Jeden z~nich zbliżał się coraz bardziej do kamery, zauważył ją -- małą główkę od szpilki z~odklejanym podkładem samoprzylepnym w~zakurzonym kącie -- i~odkleił. Jego twarz przez chwilę wydawała się wielka, pory, zabłąkane włosy wystające z~nozdrzy, oczy martwe i~drapieżne. Potem chaos i~nic.

-- Sądzimy, że nadepnął na nią -- powiedział Mighty Krang. -- To tyle, jeśli chodzi o~kamery internetowe. Następnym razem to będzie pierwsza rzecz, jakiej będą szukać. Mimo to uratowała nam tyłek, prawda?

Opis na chwilę zabrał oddech Lu. Wszystkie jego rzeczy, zapasowe ubrania, komiksy, które czytał, na wpół przeżuta paczka gum energetycznych, które kupił poprzedniego dnia, zniknęły w~trzewiach nieubłaganego autorytarnego państwa. To mógł być on.

-- Przeniesiemy się do następnej kryjówki -- powiedział. -- Znajdziemy miejsce do nadawania od dzisiejszego wieczoru.

-- Masz cholerną rację, że tak zrobimy -- powiedział Jie ze swojej strony.

Omijali stary budynek szerokim łukiem, kiedy schodzili do metra, i~świadomie zmuszali się, by nie wzdrygać się za każdym razem, gdy wyła policyjna syrena. Kiedy wrócili na poziom ulicy, Jie wzięła Lu za rękę i~kącikiem ust powiedziała: 

-- W porządku, Tank, co teraz zrobimy?

Wzruszył ramionami. 

-- Nie wiem. To było, uhm, \textit{blisko}. -- Przełknął. -- Nie złość się, jeśli coś powiem?

Ścisnęła jego palce. 

-- Powiedz to. 

-- Nie musisz tego robić -- powiedział. 

Zatrzymała się i~spojrzała na niego z~białą twarzą. Zanim się pocałowali, zawsze czuł między sobą pustkę, niewidzialne pole siłowe, przez które musiał się przebić, żeby powiedzieć jej, jak się czuje. Kiedy zostali parą, pole siłowe osłabło, ale nie zniknęło, i~za każdym razem, gdy mówił lub robił coś głupiego, czuł, że to go odpycha. Teraz znów było pole. Mówił szybko, mając nadzieję, że jego słowa przebiją się przez to: 

-- To znaczy, to \textit{szaleństwo}. Prawdopodobnie wszyscy pójdziemy do więzienia lub zginiemy. -- Wciąż się na niego gapiła. -- Jesteś tylko \ldots  -- przełknął ślinę. -- Jesteś w~tym \textit{dobra}, staram się powiedzieć. Prawdopodobnie mogłabyś nadawać swój program jeszcze przez dziesięć lat, nie dając się złapać i~nie przechodząc na emeryturę bogatej kobiety. Nie musisz tego wyrzucać dla nas.

Jej oczy się zwęziły. 

-- Czy obiecałam, że się nie będę złościć?

Spróbował trochę nerwowego uśmiechu. 

-- Raczej?

Rozglądała się tam i~z powrotem. 

-- Chodźmy -- powiedziała. -- Wyróżniamy się tutaj. 

Poszli. Jej palce były bezwładne w~jego dłoni, a potem się wysunęły. Pole siłowe się wzmocniło. Czuł się bardziej przestraszony niż wtedy, gdy The Mighty Krang opisywał akcję z~kamery w~studio. 

-- Myślisz, że robię to wszystko dla pieniędzy? Mogłabym mieć więcej pieniędzy, gdybym chciała. Mogłabym brać brudnych reklamodawców. Mogłabym zacząć oszustwo marketingowe dla moich dziewcząt i~prosić, aby przysłały mi pieniądze, jest ich miliony  \ldots  gdyby każda przysłała mi tylko kilka RMB, byłabym tak bogata, że mogłabym przejść na emeryturę.

Budynki ,,uścisku dłoni'' wyłaniały się wokół nich, a ona urwała, gdy szli pojedynczo wąską uliczką między dwoma budynkami. Dogoniła go i~pochyliła się bliżej, mówiąc tak cicho, że prawie szeptem. 

-- Mogłabym być po prostu kolejną przestępczynią, która przyjeżdża do południowych Chin, kradnie wszystko, co może, i~wraca do domu na wieś. \textit{Nie }robię tego. Wiesz dlaczego?

Szukał słów, a ona wbiła mu paznokcie w~dłoń. Umilkł.

-- To pytanie retoryczne -- powiedziała. -- Robię to, ponieważ \textit{wierzę w~to}. Mówiłam moim dziewczynom, aby walczyły przeciwko swoim szefom, zanim jeszcze zagrałeś w~swoją pierwszą grę. Z tobą lub bez ciebie, powiem im, żeby walczyły. Podoba mi się twoja grupa, podoba mi się sposób, w~jaki tak łatwo przekraczają granice, nawet łatwiej niż ja, gdy jadę tam i~z powrotem z~Hongkongu. Więc wspieram twoich przyjaciół i~mówię moim dziewczynom, żeby ich wspierały. Twój problem to problem \textit{pracowników}, nie problem Chin, nie problem graczy. Dziewczyny z~fabryki są pracownikami i~chcą dobrych warunków, tak samo jak ty i~twoi przyjaciele.

Lu zauważyła, że oddycha ciężko, gniewnie parska przez nos.

Próbował coś powiedzieć, ale wszystko, co wyszło, to mamrotanie.

-- Co? -- powiedziała, ponownie wbijając paznokcie.

-- Przepraszam -- powiedział. -- Po prostu nie chciałem, żebyś została zraniona.

-- Och, Tank -- powiedziała. -- Nie musisz być moim wielkim, silnym obrońcą. Zajmuję się sobą, odkąd opuściłam dom i~przyjechałam do południowych Chin. To może być dla ciebie ogromną niespodzianką, ale dziewczyny nie potrzebują dużych, silnych chłopców, aby się nimi opiekowali.

Milczał przez chwilę. Byli już prawie przy wejściu do bezpiecznego domu. 

-- Czy mogę po prostu przyznać, że jestem idiotą i~tak to zostawimy?

Udawała, że przez chwilę się nad tym zastanawia. 

-- Dla mnie to brzmi dobrze -- powiedziała.

 I pocałowała go ciepłym, miękkim pocałunkiem, od którego spociły mu się stopy, a włosy na karku stanęły mu dęba. Przygryzła przez chwilę jego dolną wargę, zanim puściła, po czym wykonała niegrzeczny gest w~kierunku chłopców, którzy wołali na nich z~balkonu wysoko nad głową.

-- OK -- powiedziała, -- Chodźmy zrobić audycję.

\bigskip
\threeast

Wszystko zostało tak starannie zaplanowane. Poczekają, aż nadejdzie pora monsunowa z~ulewnymi deszczami; po Diwali z~jego obrzędami religijnymi i~petardami; po Święcie Środka Jesieni, kiedy tak wielu robotników wracało do swoich wiosek, gdzie inwigilacja była znacznie mniej intensywna. Poczekają, aż nadejdą duże zamówienia na sezon Święta Dziękczynienia w~USA, kiedy sprzedawcy o~spoconych dłoniach mieli nadzieję, że ich lata przyniosą zyski dzięki ogromnej sprzedaży towarów wytwarzanych i~wysyłanych z~całego Pacyfiku.

To był dobry plan. Wszyscy go lubili. Wei-Dong, chłopak, który przepłynął ocean ze swoimi przedpłaconymi kartami do gry, prawie zmoczył spodnie od błyskotliwości. 

-- Będziesz ich miał na celowniku -- powtarzał. -- Będą \textit{musieli }się poddać i~to \textit{szybko}.

Projekt w~grze działał bardzo dobrze. Ten człowiek Ashok w~Bombaju opracował bardzo sprytny plan sygnalizowania wigoru ich różnych ,,wehikułów inwestycyjnych'', a analitycy, którzy to obserwowali, łapali się na to. Sprzedawali więcej złego papieru, niż mogli wydrukować. Zaskoczyło to wszystkich, nawet Ashoka, i~faktycznie musieli ściągnąć kilku Webblies ze sprzedaży: okazało się, że zaskakująca liczba ludzi uwierzy w~każdą plotkę, którą usłyszą na tablicy inwestycyjnej lub w~stołówce w~grze.

Mighty Krang i~Big Sister Nor również byli bardzo zadowoleni z~tej daty i~wbili w~nią metaforyczną szpilkę i~zaczęli planować. Justbob nie miała nic przeciwko temu, ale była wojownikiem, więc rozumiała, że pierwszą ofiarą każdej bitwy jest plan ataku. Tak więc podczas gdy Big Sister Nor i~Krang oraz inni porucznicy w~Chinach i~Indonezji, Singapurze, Wietnamie i~Kambodży odkładali plany na przyszłość, Justbob prowadziła harcowników w~ćwiczeniach, ogromnych, obejmujących cały świat bitwach, w~których jej wojownicy prowadzili swoje armie przeciwko sobie tysiącami.

Big Sister Nor nienawidziła tego, mówiła, że jest zbyt głośna, że informuje graczy, że w~przestrzeni gry gromadzą się armie, a potem naturalnie zastanawialiby się, \textit{po co }gromadzą się gracze, i~wszystko się rozpadnie. Justbob uważała, że o~wiele bardziej prawdopodobne jest, że farmerzy złota i~wymyślni oszuści dadzą im znać, widząc, że armie były mniej więcej tak powszechne w~przestrzeni gry, jak cebula smażona. Nie próbowała powiedzieć tego Big Sister Nor, która już prawie nie grała w~gry. Zamiast tego zgodziła się posłusznie uspokoić, zachować ostrożność i~tak dalej.

A potem ponownie wysłała swoje armie przeciwko sobie.

Nie przypominała żadnej innej gry, w~którą ktokolwiek kiedykolwiek grał. Armie były ogromne, liczyły tysiące i~każdego dnia rosły. Ćwiczyła je godzinami, a generałowie, przywódcy, komendanci i~wszyscy, jak się tam nazywali, wymyślali najlepsze strategie i~taktyki, obmyślali koszmarne zasadzki i~podstępne wojny partyzanckie, i~ostrzyli noże na sobie.

Gdy skargi Big Sister Nor stawały się coraz poważniejsze, Justbob przedstawiła jej statystyki dotyczące liczby postaci wysokiego poziomu, które Webblies mieli teraz do swojej dyspozycji, ponieważ potyczki były szybkim sposobem na awans. Miała graczy, którzy kontrolowali pięć lub sześć absolutnie najwyższego poziomu, każdy powiązany z~własnym kontem przed-płaconym, każdy dostępny przez inny serwer proxy i~niemożliwy do wyśledzenia dla innych. Big Sister Nor ponownie ostrzegła ją, aby była ostrożna, a Mighty Krang wziął ją na bok i~powiedział, jak nieodpowiedzialna była, że narażała cały wysiłek jej wojowaniem. Zdjęła opaskę na oko i~podrapała sączące się blizny nad zrujnowanym oczodołem, niepokojąca sztuczka, która nigdy nie zawiodła i~sprawiła, że The Mighty Krang pozieleniał na twarzy.

Justbob starała się ukryć uśmiech na jej twarzy, gdy Big Sister Nor obudziła ją w~środku nocy, by powiedzieć jej, że plan jest martwy, a akcja rozpoczęła się właśnie wtedy, w~środku pory monsunowej, w~środku Diwali, zostało tylko kilka tygodni przed Świętem Środka Jesieni.

-- Co się stało? -- powiedziała, wciągając długą sukienkę i~owijając głowę hidżabem. Spędziła większość swojego życia w~zachodnich strojach, ubierając się na szok i~na łatwe ucieczki, ale ponieważ wyszła na prostą, zdecydowała się na bardziej tradycyjny strój. Brak mobilności nadrabiała chłodem, anonimowością i~dezorientującym wpływem na mężczyzn, którzy kiedyś jej grozili (choć nie powstrzymało to zbirów, którzy kosztowali ją oko).

-- Kolejny strajk w~Dongguan. Tym razem w~Kantonie. Jest duży.

\bigskip
\threeast

W pokoju było duszno. Te pokoje zawsze takie były. Jednak wrześniowe upały podniosły temperaturę do stratosferycznych poziomów, tak że kawiarnia tliła się jak kaldera niestrawnego wulkanu. Właściciel kawiarni, pokryty bliznami staruszek, o~którym wszyscy wiedzieli, że jest przykrywką dla jakichś ciężkich gangsterów, wysłał technika ze śrubokrętem, aby zdjął wszystkie obudowy z~komputerów, aby ciepło mogło łatwiej się rozproszyć od spoconych płyt głównych i~tych ogromnych kart graficznych, które były najeżone dodatkowymi wentylatorami i~lśniły miedzianymi radiatorami. Mogło to być lepsze dla komputerów, ale sprawiło, że pokój stał się jeszcze gorętszy i~wypełnił go ryk silnika odrzutowego, który był tak głośny, że gracze nie mogli nawet używać słuchawek z~redukcją szumów do rozmowy: musieli ograniczyć całą komunikację do tekstu.

Kawiarnia była kiedyś przeznaczona dla graczy z~ulicy, wraz z~zakochanymi dziewczynami z~fabryki, które spędzały długie noce na rozmowach ze swoimi wirtualnymi chłopakami, pracownikami tęskniącymi za domem, którzy zalogowali się, by snuć kłamstwa o~ich wspaniałym życiu w~południowych Chinach dla ludzi w~domu, a także od czasu do czasu zagubionych turystów, którzy mieli nadzieję na trochę czasu online, aby pogadać z~przyjaciółmi i~znaleźć tanie pokoje hotelowe. Ale przez ostatnie dwa lata mieściła się tu wyłącznie stale rosnąca kadra farmerów złota, wysyłanych tam przez swoich szefów, którzy nadzorowali tuzin zmieniających się, powiązanych ze sobą przedsiębiorstw, które tworzyły się i~rozpadały w~ciągu nocy, za każdym razem, gdy pojawiały się małe kłopoty, i~wygodnie było zwinąć sklep i~zniknąć jak dżin.

Chłopcy w~kawiarni tego wieczoru byli wszyscy młodzi, żaden powyżej 17 roku życia. Wszyscy starsi chłopcy zostali usunięci miesiąc wcześniej, kiedy zażądali przerwy po 22-godzinnym zamknięciu, aby zrealizować ogromne zamówienie dla dostawcy. Pozbycie się tych awanturników miało dwa miłe efekty dla ich szefów: pozwoliło im wprowadzić tańszą siłę roboczą i~uniknąć płacenia za te wszystkie godziny zamknięcia. Zawsze było więcej chłopców, którzy zarabiali na życie, grając w~gry.

A ci chłopcy umieli się \textit{bawić}. Po 12-godzinnej zmianie kręcili się i~robili jeszcze cztery lub pięć godzin najazdów \textit{dla zabawy}. Pokój był kociołkiem, w~którym chłopcy, ciepło, hałas, pierogi i~połączenia sieciowe tworzyli niekończący się zapas bogactwa dla niektórych przeważnie niewidocznych starszych mężczyzn.

Ruiling wiedział, że wcześniej pracowali tam inni chłopcy, starsi chłopcy, którzy pokłócili się z~szefami. Nie myślał o~nich zbyt wiele, ale kiedy to robił, wyobrażał sobie powolnych, chciwych głupców, którzy nie chcieli naprawdę pracować na życie. Lamerów, których tyłki mógł skopać z~powrotem do prowincji Syczuan lub jakiegokolwiek odległego miejsca, z~którego przekradli się do delty Rzeki Perłowej.

Ruiling był piekielnym graczem. Jego specjalnością było PvP -- gracz kontra gracz -- ponieważ miał talent do obserwowania ruchów innego gracza przez kilka sekund, a następnie budowania niemal pełnego obrazu jego dziwactw i~słabych punktów. Nie potrafił tego wyjaśnić, wiedza po prostu przebijała się do niego jak strzała w~oczodole. W rezultacie nikt nie mógł awansować postaci szybciej niż Ruiling. Po prostu wędrował po grze z~chińskim nazwiskiem, rozmawiając po chińsku z~napotkanymi graczami. W końcu jeden z~nich -- jakiś bogaty, gruby, głupi człowiek z~Zachodu, który chciał bawić się w~strażnika -- zaczął go wyzywać i~żądać pojedynku. On by się zgodził. Skopałby tyłek. Zdobyłby punkty.

To było niesamowite, jakie to było satysfakcjonujące.

Ruiling właśnie skończył dwanaście godzin tego i~zamówił na tacy pierogi z~wieprzowiną, oblał je ostrym wietnamskim czerwonym sosem i~wrzucił do ust tak szybko, jak tylko mógł przeżuć, a teraz był gotów odpocząć, grając po pracy. W tym celu zawsze używał własnego toon, postać, którą zaczął się bawić, gdy był chłopcem w~Gansu. Pod pewnymi względami ten toon był \textit{nim}, żył z~nim tak długo, z~miłością go polerował, szkolił, ubierał w~najrzadsze skarby. Wytrenował niezliczoną ilość zabawek i~widział, jak zostały wyprzedane, ale Ruiling należał do \textit{niego}.

Dziś wieczorem Ruiling imprezował z~kilkoma innymi farmerami, których znał z~innych części Chin, niektórych znał już w~swojej wiosce, a niektórych nigdy nie spotkał. Byli zaciekłą gildią nocnych rozbójników, która wykonywała najtrudniejsze misje na światach, śmietankę graczy. Wieści się rozeszły i~teraz każdej nocy miał publiczność graczy, którzy właśnie zostali zatrudnieni, i~patrzyli z~podziwem, jak skopuje fantastyczne ilości tyłków. Uwielbiał to, uwielbiał odpowiadać na ich pytania po zakończeniu gry, pomagając całej drużynie stać się lepszym. I wiecie, oni też go kochali i~to było równie wspaniałe.

Kierowali fortecą Buri, pałacem dawno zmarłego boga, ojca bogów, potężnej, żywiołowej siły, która zrodziła Svartalfaheim i~wszechświat, w~którym leżał. Miała przerażających strażników, wymagała potężnych zaklęć, by dosięgnąć, i~nigdy nie była w~pełni zarządzana w~historii Svartalfaheim. Taka właśnie misja, którą Ruiling uwielbiał. To będzie jego szósta próba i~był gotów na rajdy przez sześć godzin z~rzędu, jeśli to będzie potrzebne, podobnie jak reszta jego drużyny.

A potem zdobył Ząb Fenrira. Był to najrzadszy i~najbardziej legendarny zrzut ze wszystkich Wojowników Svartalfaheim, potężny talizman, który zamieni każde wilcze stado lub zniewoli je dla posiadacza Zęba. Fora ogłoszeniowe były pełne rozmów na ten temat i~kilka razy odbywały się na nim oszukańcze aukcje, ale nikt nigdy wcześniej go nie widział.

Po tym, jak Ruiling go podniósł -- Ząb pochodził z~epickiej bitwy z~armią Niebiańskich Gigantów, w~której zginęła cała grupa najeźdźców -- był tak oszołomiony, że przez chwilę nie mógł mówić. Po prostu wskazał na ekran, podczas gdy jego usta otwierały się i~zamykały na chwilę.

Obserwujący go gracze również zamilkli, podążając za jego wzrokiem i~palcem, powoli zdając sobie sprawę z~tego, co się właśnie wydarzyło. Szmer narastał w~tłumie, nabierając pary, zwiększając głośność, zmieniając się w~\textit{ryk}, triumfalny krzyk, który sprawił, że cała kawiarnia to zobaczyła. Ponad hałasem fanów brzęczały z~podekscytowania, przesiąknięte hormonami triumfalne, triumfalne ćwiczenie bicia klatki piersiowej, które pochłonęło ich wszystkich. Każdy chłopiec wyobrażał sobie, jak by to było iść na misję z~Zębem Fenrira, zdolnym do pokonania każdej siły jednym ruchem myszy, który wysłałby wilki przeciwko twoim wrogom. Serce każdego chłopca waliło mu w~piersi.

Ale był inny dźwięk, coraz głośniejszy i~bardziej natarczywy. Starszy głos, chrapliwy od miliona papierosów, twardy głos. 

-- Siadać! Siadać! Wracajcie do pracy! Wszyscy wracają do pracy!

Był to brygadzista Huang, który krzyczał z~przerażającym akcentem z~Fujianu. Mówiono, że był byłym Wężogłowym, wyrzuconym z~gangu przemytników ludzi za zabicie zbyt wielu migrantów brutalnym traktowaniem. Zwykle siedział jak jaszczurka i~nieruchomo w~kącie, paląc serię tanich chińskich podróbek Marlboro klasy D, szorstkich i~niefiltrowanych, leniwy kłąb dymu, który sprawiał, że stale zezował po jednej stronie twarzy. Czasami gracze zapominali, że tam jest, a ich krzyki i~konie wymykały się trochę spod kontroli, a wtedy podkradał się za nich na cichych, kocich stopach i~zadawał mocny cios w~ucho, który powodował, że się zataczali. Wystarczyła lekcja poglądowa -- ,,Nie denerwuj Wężogłowego, bo pobije cię tak, że nie zapomnisz'' -- że prawie nigdy nie musiał tego powtarzać.

Teraz jednak bił chłopców z~lewej i~prawej strony, wykrzykując rozkazy donośnym, ochrypłym głosem. Chłopcy w~pośpiechu wycofywali się do komputerów, zostawiając Ruilinga samego na swoim miejscu z~niepewnym uśmiechem na twarzy.

-- Szefie -- powiedział -- widzisz, co zrobiłem? -- Wskazał na swój ekran.

Twarz Huanga była jak zawsze beznamiętna. Położył twardą, ciężką dłoń na ramieniu Ruilinga i~pochylił się, by przeczytać ekran, z~głową otuloną dymem. Wreszcie wyprostował się. 

-- Ząb Fenrira -- powiedział. Pokiwał głową. -- Premia dla ciebie, Ruiling. Bardzo dobrze.

Ruiling skurczył się.

-- Szefie -- powiedział z~szacunkiem, przemawiając głośno, by usłyszeć go ponad wentylatorami komputera. -- Szefie, to moja postać. Teraz nie pracuję. To moja osobista postać.

Huang odwrócił się, by na niego spojrzeć, jego oczy były twarde, a wyraz twarzy bezbarwny. 

-- Premia -- powtórzył. -- Bardzo dobrze. 

-- To \textit{moja }postać -- powiedział Ruiling, mówiąc głośniej. -- Żadnych premii. Jest \textit{moje}! \textit{Zasłużyłem }na to osobiście, w~swoim własnym czasie.

Nawet nie widział ciosu, był tak szybki. W jednej chwili gorąco oświadczył, że Ząb Fenrira należy do niego, w~następnej leżał rozciągnięty na plecach na podłodze, a jego głowa dzwoniła jak gong. Brygadzista postawił stopę na jego gardle.


-- Bez premii -- powiedział wyraźnie i~dobitnie, tak aby wszyscy wokół mogli słyszeć. 

Potem wypluł ogromny kęs trującej zielonej śliny z~przesiąkniętych smołą głębin swoich poczerniałych płuc i~starannie splunął Ruilingowi w~twarz.

Od czwartego roku życia Ruiling ćwiczył wushu, treningi, którym podlegali wszyscy dorośli, z~mężczyzną w~wiosce. Mężczyzna został wysłany na północ podczas rewolucji kulturalnej, potępiony, pobity i~głodzony, ale nigdy się nie złamał. Był delikatny i~cierpliwy jak babcia, stary jak wzgórza i~jednym ruchem nadgarstka potrafił posłać napastnika w~powietrze; złamać deskę swoimi starymi rękami, kopnąć cię do następnego życia jedną starą, sękata nogą. Przez 12 lat Ruiling chodził trzy razy w~tygodniu na treningi ze starym mężczyzną. Wszyscy chłopcy musieli. To była tylko część życia w~wiosce. Nie ćwiczył, odkąd przyjechał do południowych Chin, prawie zapomniał o~tym relikcie innych Chin.

Ale teraz pamiętał każdą lekcję, pamiętał ją głęboko w~swoich mięśniach. Chwycił kostkę stopy, która znajdowała się na jego gardle, lekko \textit{skręcił}, aby uzyskać maksymalną dźwignię, i~zastosował niewielki, kontrolowany nacisk i~\textit{wyrzucił }brygadzistę w~powietrze, posyłając go po perfekcyjnym, pełnym gracji łuku, który zakończył się, gdy jego głowa \textit{uderzyła }o bok jednego z~długich kozłów, przewracając go i~posyłając tuzin płaskich ekranów na ziemię, a trzask był słyszalny nad wentylatorami komputera.

Ruiling wstał ostrożnie i~spojrzał na brygadzistę. Mężczyzna jęczał na ziemi, a Ruiling nie mógł powstrzymać uśmiechu na jego twarzy. To było \textit{dobre }uczucie. Stwierdził, że stoi w~gotowej postawie, ciężar równomiernie rozłożony na każdej stopie, stopy rozłożone dla stabilności, ciało zwrócone bokiem do człowieka na ziemi, prezentując mniejszy cel. Jego ręce były luźno uniesione, jedna przed drugą, gotowe do złapania ciosu, zablokowania ręki i~rzucenia napastnika, gotowego do kontrataku wysoko lub nisko. Chłopcy wokół niego wiwatowali, skandowali jego imię, a Ruiling uśmiechał się szerzej.

Brygadzista podniósł się z~podłogi, bez wyrazu twarzy, straszna pustka, a Ruiling poczuł pierwsze przeczucie strachu. Coś w~tym, jak mężczyzna się zachowywał, jak stał, nie było podobne do postawy w~grach sztuk walki, w~które grał w~wiosce. Coś zupełnie poważniejszego. Ruiling usłyszał wysoki jęk i~zdał sobie sprawę, że pochodzi on z~jego własnego gardła.

Opuścił lekko ręce, wyciągnął jedną w~przyjazny sposób, dłonią do góry. 

-- No weź -- powiedział. -- Bądźmy dorośli w~tej sprawie.

I wtedy właśnie brygadzista sięgnął pod ramię źle dopasowanej, wymiętej, upstrzonej łupieżem marynarki i~wyciągnął tani mały pistolet, wycelował go w~Ruilinga i~strzelił mu prosto w~czoło.

Jeszcze zanim Ruiling upadł na ziemię, z~jednym okiem otwartym, drugim zamkniętym, chłopcy wokół niego zaczęli ryczeć. Brygadzista miał sekundę, aby zarejestrować dźwięk setki głosów narastających w~gniewie, zanim chłopcy wykipieli, wspinali się na siebie, by do niego dotrzeć. Za późno, spróbował zacisnąć palec na spuście pistoletu, który nosił od czasu opuszczenia prowincji Fujian przed laty. Do tego czasu trzech chłopców przypięło się do jego ramienia i~zepchnęło je w~dół tak, że pistolet celował w~mięsień jego starego uda, a wyciśnięty przez niego pocisk kalibru 22 wbił się w~dużą kość udową, zanim rozpłaszczył się na złamanej kości, rozpościerając się jak ołowiana moneta.

Kiedy otworzył usta, by krzyczeć, palce znalazły się w~jego policzkach, wściekle je szarpiąc, podczas gdy inne dłonie wplatały się w~jego włosy, mocowały się do jego stóp i~ramion, a nawet szarpały za uszy. Ktoś uderzył go mocno w~jądra, dwukrotnie, i~nie mógł oddychać wokół dłoni w~ustach, nie mógł krzyczeć, gdy upadał. Pistolet wyrwano mu z~ręki w~tej samej chwili, gdy dwie pięści wbiły mu się w~oczy, a potem było mroczne, bolesne i~nieskończone, chwila, która przeciągnęła się w~jego nieświadomość, a potem w~\ldots  unicestwienie.

\bigskip
\threeast

-- I co teraz? 

 Justbob siorbała swoje congee, po którą posłali, razem z~mocną kawą i~talerzem świeżych bułek. O 3 nad ranem w~Geylang wybór jedzenia był nieco ograniczony, ale nigdy nie zniknął całkowicie.

Mighty Krang wyciągnął wideo, poczekał, aż się zbuforuje, a potem przewinął szybko obok. 

-- Trzech chłopców złapało strzelaninę ,,egzekucję'' na swoich telefonach. Ten zbir, który upadł, cóż, nie wygląda zbyt dobrze. 

Strzał z~wnętrza ciemnego pokoju, teraz opuszczonego, brygadzista na plecach pośród wraku zepsutych komputerów i~monitorów, nieruchomy, z~rękami złamanymi w~łokciach, twarzą w~ruinie z~galaretki i~krwi.

-- Zakładamy, że nie żyje, ale strajkujący nikogo nie wpuszczają.

-- Strajkujący -- powiedziała Justbob, a Mighty Krang kliknął kolejny film. 

Ten trwał dłużej, aby się załadować, jakiś serwer gdzieś jęczał pod obciążeniem wszystkich ludzi próbujących uzyskać do niego dostęp jednocześnie. To się jeszcze nigdy nie zdarzyło, minęły lata, odkąd to się stało, i~to uświadomiło Justbob, jak szybko to się musi rozprzestrzeniać. Uświadomienie sobie przebiło się przez jej oszołomienie, sprawiło, że jej oko się otworzyło, a drugie zrujnowało efekt pod jej łatą.

Film został załadowany. Setki chłopców zebranych przed anonimowym wielopiętrowym budynkiem, takim miejscem, których mija się tysiące. Zawiązali sobie koszule wokół twarzy i~wymachiwali pięściami w~powietrzu, a coraz więcej ludzi wychodziło, by do nich dołączyć. Chłopcy, starzy ludzie, dziewczęta \ldots 

-- Dziewczyny? 

-- Dziewczyny z~fabryki. Jiandi. Nagrała specjalną transmisję. Głupia. Prawie została złapana, wygoniona z~innej kryjówki. Kończą się jej dziury. Ale przekazała wiadomość. 

-- Wiedzieliśmy?

Twarz Big Sister Nor była chmurą burzową, złowrogą i~ciemną. 

-- Oczywiście, że nie. Gdybyśmy wiedzieli, powiedzielibyśmy jej, żeby tego nie robiła. Wyluzuj. Poczekaj. Mamy harmonogram, mnóstwo ruchomych części.

-- Martwy chłopiec?

-- Tam \ldots  -- powiedział Krang i~wskazał myszką krawędź nagrania. 

Stół na kozłach, ustawiony obok chłopców, z~udrapowanym na nim martwym chłopcem. Przyglądając się uważnie, widziała dziurę po kuli w~jego czole, smugę krwi spływającą po jego twarzy.

-- Aha -- powiedziała Justbob. -- Cóż, teraz niczego nie wyluzujemy.

-- Nie wiemy tego. Wciąż jest szansa \ldots -- powiedziała Big Sister Nor.

-- Nie ma szans -- powiedziała Justbob, a jej palec wbił się w~ekran. -- Są ich tam \textit{tysiące}. Co się dzieje na świecie?

-- To katastrofa -- powiedział Krang. -- Każda operacja związana z~uprawianiem złota jest pogrążona w~chaosie. Webblies atakują je tysiącami. Z każdym dniem jest coraz gorzej. Po prostu budzą się w~Chinach, więc nowe siły powinny nadciągać \ldots 

Justbob przełknęła. 

-- To nie katastrofa -- powiedziała. -- To jest bitwa. I wygrają. I będą dalej wygrywać. Od tej chwili zdziwiłabym się, gdyby na rynkach pojawiło się \textit{nowe }złoto w~dowolnej grze. Możemy zmieniać loginy tak szybko, jak gamerunnerzy zamykają konta, a co więcej, jest wielu zwykłych graczy, którzy toczą z~nami potyczki dla zabawy, którzy będą krzyczeć cholerne morderstwo, jeśli stracą konto. Mamy połączone gry. -- Zachowała obojętną twarz, sięgnęła po filiżankę herbaty, upiła ją i~odstawiła.

Big Sister Nor wpatrywała się w~nią przez długi czas. Były przyjaciółkami od dawna, ale w~przeciwieństwie do Kranga, Justbob nie kochała Nor z~czcią. Wiedziała, jak bardzo ludzka może być Big Sister Nor, widziała, jak schrzaniła w~małych i~dużych sprawach. Big Sister Nor też o~tym wiedziała i~miała siłę charakteru, by słuchać Justbob, nawet gdy mówiła rzeczy, których Nor nie chciała słyszeć.

Krang patrzył w~tę i~z powrotem między dwiema młodymi kobietami, czując się jak zawsze odcięty, starając się tego nie pokazać, bezskutecznie. Wstał od stołu, mrucząc coś o~wyjściu na kawę, na co żadna z~kobiet nie zwróciła na to uwagi.

-- Myślisz, że jesteśmy gotowi? -- powiedziała Big Sister Nor, gdy drzwi kryjówki się zamknęły.

-- Myślę, że musimy być -- powiedziała Justbob. -- Pierwsza ofiara jakiejkolwiek bitwy \ldots  

-- Wiem, wiem -- powiedziała Big Sister Nor. -- Możesz teraz przestać tak mówić.

Kiedy Mighty Krang wrócił, od razu zobaczył, jak się sprawy potoczyły. Rozdał kawę i~zabrał się do pracy.

\bigskip
\threeast

Kawiarnia pani Dibyendu była szczelnie zamknięta, a okna i~drzwi zasłonięte okiennicami.

-- Hej! -- zawołał Ashok, pukając do drzwi. -- Hej, pani Dibyendu! Tu Ashok! Hej!

 Była prawie 7 rano, a pani Dibyendu zawsze otwierała kawiarnię o~6:30, łapiąc część porannego handlu, gdy robotnicy, którzy mieli pracę poza Dharavi, szli na swoje przystanki autobusowe lub na dworzec kolejowy. To było nie do pomyślenia, żeby się spóźniła. 

-- Hej! -- zawołał ponownie i~zastukał breloczkiem w~metalową okiennicę, a dźwięk odbijał się echem w~blaszanej ramie budynku.

-- Idź stąd! -- zawołał męski głos. 

Z początku Ashok zakładał, że pochodzi z~jednego z~dwóch pokoi nad kawiarnią, gdzie pani Dibyendu wynajmowała tuzinowi pensjonariuszy -- dwie duże rodziny stłoczone w~małych pomieszczeniach. Wyciągnął szyję do góry, ale tam też okna były zamknięte.

-- Hej! -- ponownie zastukał w~drzwi, głośno na ulicy wczesnym rankiem.

Ktoś zrzucił rygle po drugiej stronie drzwi i~pchnął je tak mocno, że odbiły się od jego palca u nogi i~czubka nosa, sprawiając, że oboje szczypały. Odskoczył i~drzwi ponownie się otworzyły. Stał tam chłopiec, 17 lub 18 lat, z~ogromną, podziurawioną maczetą długości przedramienia. Chłopiec był chudy do granic głodu, z~nagim torsem i~żebrami wystającymi jak ksylofon. Patrzył na Ashoka zaczerwienionymi, kamiennymi oczami, odgarniał z~czoła chude, tłuste włosy wierzchem dłoni, która nie trzymała maczety. Wymachiwał nią przed twarzą Ashoka.

-- Nie słyszałeś mnie? -- powiedział. -- Czy jesteś głuchy? Odejdź! 

Maczeta zakołysała się w~jego dłoni, tańcząc w~powietrzu przed jego twarzą, tak blisko, że zezował oczami.

Cofnął się, a chłopak wyciągnął rękę dalej, trzymając maczetę blisko twarzy. 

-- Gdzie jest pani Dibyendu? -- spytał Ashok, zachowując swój głos tak spokojnym, jak tylko mógł, co nie było łatwe.

-- Odeszła. Z powrotem do wioski. -- Chłopak uśmiechnął się szalonym, złym uśmiechem. -- Kawiarnia jest zamknięta.

-- Ale \ldots  -- zaczął. Chłopak zrobił kolejny krok do przodu, a wraz z~nim pojawiła się fala alkoholu i~zapach potu, silny zapach nawet pośród smrodu zapachów Dharavi. 

-- Mam tam dokumenty -- powiedział Ashok. -- Są moje. Na zapleczu.

Z kawiarni dobiegły teraz inne, poruszające dźwięki, w~drzwiach pojawiło się więcej chudych chłopców. Więcej maczet. 

-- Idź już -- powiedział prowadzący i~splunął strumieniem różowej śliny poplamionej betelem pod stopy Ashoka, barwiąc mankiety jego dżinsów. -- Idź, póki możesz iść. 

Ashok cofnął się o~kolejny krok. 

-- Chcę porozmawiać z~panią Dibyendu. Chcę porozmawiać z~właścicielem! -- powiedział, zbierając całą odwagę, by nie odwrócić się na pięcie i~uciec. Chłopcy schodzili teraz do małego, osłoniętego obszaru przed wejściem. Uśmiechali się.

-- Właściciel? -- powiedział chłopiec. -- Jestem jego przedstawicielem. Możesz mi powiedzieć.

-- Chcę moje dokumenty.

-- Moje papiery -- powiedział chłopiec. -- Chcesz je kupić?

Pozostali chłopcy chichotali, odgłosy hieny. Dźwięki drapieżnika. Wszystkie te maczety. Każdy nerw w~ciele Ashoka krzyczał \textit{ucieczka}. 

-- Chcę porozmawiać z~właścicielem. Powiedz mu. Wrócę dziś po południu. Porozmawiam z~nim.

Brawura była nieprzekonująca nawet dla niego, a dla tych uliczników musiał brzmieć jak pierdnięcie podczas wichury. Śmiali się głośniej i~jeszcze głośniej, gdy chłopak zrobił kolejny pospieszny krok w~jego stronę, wymachując maczetą, tylko chybił, ostrze przeleciało obok niego z~przerażającym świstem, gdy cofnął się o~kolejny krok, wpadł na mężczyznę niosącego młot kowalski domowej roboty w~drodze do pracy, pisnął, rzeczywiście \textit{pisnął }i uciekł.

Matka Mali po dłuższej chwili odpowiedziała na jego pukanie, przyglądając mu się podejrzliwie. Spotkała go przy dwóch innych okazjach, kiedy odprowadził ,,Generał'' do domu po późnej bitwie, i~ani razu go nie lubiła. Teraz spojrzała otwarcie i~zablokowała drzwi. 

-- Nie jest ubrana -- powiedziała. -- Daj jej chwilę.

Mala przepchnęła się obok niej, z~włosami związanymi w~luźny kucyk, jej chód był stanowczy, wściekłe utykanie. Wysłała pobieżny pocałunek w~policzek matki, chybiony o~kilka centymetrów, i~szorstko wskazała na schody. Ashok pospieszył w~dół, przez dolny pokój z~własną rodziną, krzątał się i~szykował do pracy, a potem zszedł kolejnym zbiegiem na halę produkcyjną, a potem wyszedł na piekące powietrze Dharavi. W pobliżu ktoś palił plastik, smród silniejszy niż zwykle, natychmiastowy ból głowy.

-- Co? -- spytała, już w~trybie służbowym.

Opowiedział jej o~kawiarni.

-- Bannerjee -- powiedziała. -- Zastanawiałem się, czy by tego spróbował. 

 Wyjęła telefon i~zaczęła wysyłać SMS-y. Ashok stał obok niej, o~głowę wyższy od niej, ale czuł się jakoś mniejszy niż ta dziewczyna, ta kula talentu i~gniewu w~dziewczęcej formie. Dharavi już się budziło, a nad chatami i~fabrykami niosło się wezwanie muezzin do modlitwy z~wielkiego meczetu. Odgłosy zwierząt, koguty, kozy, krowi dzwonek i~wielkie bydlęce kichnięcie. Płaczące dzieci. Kobiety zmagały się z~dzbanami na wodę.

Pomyślał o~tym, jak nierealne było to wszystko dla większości ludzi, których znał, przywódców związkowych, z~którymi dorastał, jego własnej rodziny. Kiedy rozmawiał z~nimi o~sprawach Webblies, kpili z~nierealności życia w~grach, ale co z~nierealnością życia w~Dharavi? Oto milion ludzi żyjących życiem, którego wielu innych nie mogło sobie nawet wyobrazić.

-- Chodź -- powiedziała. -- Spotykamy się w~hotelu UP.

Kiedy przybył do Dharavi, ,,hotele'' przy głównej drodze w~dzielnicy Kumbharwada intrygowały go, dopóki nie dowiedział się, że ,,hotel'' to po prostu inne określenie na restaurację. Webblies lubili Hotel UP, spółdzielnię robotniczą, w~której pracowały wyłącznie kobiety pochodzące z~wiosek w~biednym stanie Uttar Pradesh. To było odwzajemnione, kobiety cieszyły się możliwością matkowania tym poważnym dzieciom, gdy rozmawiały swoim nieprzeniknionym żargonem, mieszanką indyjskiej angielszczyzny, języka graczy, chińskich przekleństw i~hindi, dziwacznym dialektem, o~którym myślał jako \textit{Webbli}, jak w~\textit{hindi}.

Webblies, wstając z~łóżek wczesnym rankiem, tłoczyli się sennie, domagając się chai, masala coli, dhosa i~aloo poori. Właścicielki restauracji podrzucały im naleśniki i~popovery ze smażonych ziemniaków w~wielkich stosach, Mala płaciła za nie zwitkiem tłustych rupii, które trzymała w~małej torebce, którą trzymała przed sobą. Ashok siedział obok niej po jej lewej ręce, a Yasmin po prawej z~na wpół przymkniętymi oczami. Armia była na zewnątrz późną nocą, na grupowej wycieczce do małego pałacu filmowego w~sercu Dharavi, aby obejrzeć trzy filmy z~rzędu w~nagrodę za serię naprawdę doskonałej gry. Ashok wykręcił się, chociaż trenował z~armią na rozkaz Mali. Lubił Webblies, ale nie był do nich całkiem podobny. Nie był graczem i~zawsze tak będzie, bez względu na to, jak dużo walczył.

-- Dobrze -- powiedziała Mala. -- Opcje. Możemy znaleźć inną kawiarnię. Jest 1000 Palm, gdzie kiedyś walczyliśmy \ldots  -- skinęła głową na Yasmin, pozostawiając resztę niewypowiedzianą, \textit{kiedy jeszcze byliśmy Pinkertonami, wciąż przeciwko Webblies}. -- Ale Bannerjee ma coś na właściciela, widziałem to na własne oczy.

-- Bannerjee ma coś na każdą kawiarnię w~Dharavi -- powiedział Sushant. Był bardzo żądny przygód, szukając innych miejsc do zabawy, na rozkaz Yasmin. Wszyscy w~armii wiedzieli, że podkochiwał się w~Yasmin, z~wyjątkiem Yasmin, która najwyraźniej nie była tego świadoma.

-- A co z~panią Dibyendu? -- powiedziała Yasmin. -- A co z~jej biznesem, całą pracą, którą w~to włożyła?

Mala skinęła głową. 

-- Zadzwoniłam do niej trzy razy. Nie odbiera. Może ją przestraszyli albo zabrali jej telefon. Albo \ldots  -- Znowu nie musiała tego mówić, albo \textit{nie żyje}. Ashok wiedział, że stawka jest wysoka. Bardzo wysoka. -- I jest coś jeszcze. Strajk się rozpoczął. 

Ashok podskoczył lekko. \textit{Co?} To było za wcześnie -- tygodnie za wcześnie! Wciąż było tyle planowania! Wyciągnął telefon, zdał sobie sprawę, że zostawił go wyłączonego, włączył go, wpatrywał się niecierpliwie w~ekran startowy, wsłuchując się w~gwar otaczających go żołnierzy. Czekały na niego \textit{dziesiątki }wiadomości od Big Sister Nor i~jej poruczników, od agentów specjalnych, którzy pracowali z~nim nad przekrętem, od amerykańskiego chłopca, który koordynował działania z~Mechanicznymi Turkami. Toczyły się walki online i~poza nimi przez całą noc, a Chińczycy tłoczyli się na ulicach, uciekali przed gliniarzami, przegrupowywali się. Gamespace pogrążył się w~chaosie. A on kłócił się w~kawiarni z~pijanymi bandytami, jedząc aloo poori i~łykając czaj, jakby to był tylko kolejny dzień. Jego serce zaczęło bić szybciej.

-- Musimy połączyć się z~Internetem -- powiedział. -- Pilnie.

Mala przerwała intensywną dyskusję na temat możliwości umieszczenia komputerów PC gdzieś w~mieszkaniu i~podłączenia łącza sieciowego, aby na niego spojrzeć. 

-- Tak źle?

Podniósł telefon. 

-- Widziałaś, wiesz. 

-- Nie sprawdzałam, odkąd przyjechałeś do mnie. Wiedziałam, że nic nie możemy zrobić, dopóki nie znajdziemy miejsca do pracy. W takim razie jest źle. -- To nie było pytanie.

Wszyscy na niego patrzyli.

-- Potrzebują naszej pomocy. -- powiedział.

-- W porządku -- powiedziała Mala. -- W porządku. Więc. Idziemy i~znowu zajmiemy miejsce pani Dibyendu. Bannerjee nie jest jego właścicielem. Wszyscy na jej ulicy o~tym wiedzą. Staną po naszej stronie. Muszą.

Ashok przełknął ślinę. 

-- Siłą? -- Przypomniał sobie chłopca: pijanego, nieustraszonego, puste oczy, drżącą ostrą maczetę.

Spojrzenie Mala, które zwróciła na niego, było równie płaskie. Mogła się tak zmienić w~sekundę, w~\textit{mgnieniu oka}. Mogła przejść od ładnej młodej dziewczyny, charyzmatycznej, otwartej, mądrej i~roześmianej, do generała Robotwallaha o~kamiennej twarzy, dzikiego i~bezkompromisowego. Jej płaskie oczy błyszczały.

-- Siłą, jeśli to konieczne, zawsze -- powiedziała. -- Wystarczająca siłą, aby odeszli i~nie wrócili. Uderzyć mocno, wystraszyć ich do ich dziur. 

 Wokół stołu wpatrywało się w~nią trzydziestu Webblies, ich twarze były odzwierciedleniem jej twarzy. Była ich generałem i~zanim pojawiła się w~ich życiu, byli szczurami Dharavi, pracującymi w~fabrykach sortujących plastik, chodzącymi codziennie na kilka godzin do szkoły, aby dzielić się książkami z~czterema innymi uczniami. Teraz byli członkami rodziny królewskiej, mieli więcej pieniędzy niż ich rodzice, pracę i~szacunek. Podążą za nią z~klifu. Podążą za nią \textit{w Słońce}.

Ale Yasmin odchrząknęła. 

-- Siła, jeśli musimy -- powiedziała. -- Ale na pewno nie więcej, niż jest to konieczne, a nawet nie, jeśli możemy temu pomóc.

Mala odwróciła się do niej ze sztywnymi plecami, sznurem na szyi, zaciśniętą szczęką. Yasmin spojrzała jej w~oczy spokojnymi oczami, a potem \ldots  \textit{uśmiechnęła }się drobnym, słodkim i~szczerym uśmiechem. 

-- Oczywiście, jeśli generał się zgodzi.

A Mala roztopiła się, napięcie z~niej ustąpiło, i~odwzajemniła uśmiech Yasmin. Coś się między nimi zmieniło od nocy, kiedy Mala ich zaatakowała, coś zmieniło się na lepsze. Teraz Yasmin mogła rozbroić Malę spojrzeniem, uśmiechem, dotykiem, a armia szanowała to, traktując Yasmin z~szacunkiem, czasami zwracając się do niej ze swoimi skargami.

-- Oczywiście -- powiedziała Mala. -- Nie więcej siły, niż jest to absolutnie konieczne. 

 Podniosła laskę, zwieńczoną srebrną czaszką, prezentem od jej żołnierzy,  i~wykonała kilka okrutnych machnięć w~powietrzu, wykonywanych z~gracją szermierza. Wiedział, że u podstawy laski jest ołowiany ciężarek i~widział, jak uderzenie wybija dziury w~cegle. Jej mocno umięśnione przedramiona prawie nie drżały, gdy dzierżyła laskę. Za nią jedna z~pań prowadzących restaurację patrzyła z~rozdzierającym serce smutkiem, a Ashok zastanawiał się, ilu młodych ludzi widziała zrujnowanych w~swojej wiosce i~tu, w~mieście.

-- Idziemy -- powiedziała Mala i~odsunęła krzesło.

 Ashok wpadł obok niej i~armia pomaszerowała główną drogą troje w~rzędzie, powodując, że wokół nich rozjechały się skutery, motocykle, kozy i~trójkołowe autoriksze. Wiele razy Ashok widywał na ulicach dumne gangi badamaash, schodził im z~drogi. Teraz był w~jednym, zbiorze dzieciaków, po prostu dzieciaków, najmłodszy miał zaledwie 13 lat, najstarszy nie miał jeszcze 20 lat, prowadzony przez kulejącą dziewczynę z~długą szyją i~włosami związanymi w~luźny kucyk, a wokół nich ludzie reagowali tym samym strachem. Serce Ashoka nabrzmiało, moc i~strach, a on poczuł się zawstydzony i~podekscytowany.

Przed drzwiami pani Dibyendu Mala pochyliła się i~podważyła palcami kamień z~kruszącego się chodnika, nie zważając na brud, który go zamulał. Rzuciła go z~niesamowitą dokładnością, rzucając nim jak piłką do krykieta, \textit{bum}, w~blaszane drzwi kawiarni. Natychmiast pochyliła się, by podnieść kolejny kamień, wyrywając go, zanim ucichły echa pierwszego. Wokół nich, na wąskiej uliczce, z~okien i~drzwi wyłaniały się głowy, a ciekawscy przechodnie przystawali, żeby się przyjrzeć.

Drzwi otworzyły się z~hukiem i~pojawił się chłopak, który wcześniej groził Ashokowi, z~przekrwionymi i~różowymi oczami nawet z~bezpiecznej odległości. Uniósł maczetę jak miecz, z~warknięciem na ustach. Uśmiech zniknął, gdy kontemplował 30 żołnierzy ustawionych przed nim. Wielu miało kawałki drewna lub żelaza albo podnosiło własne kamienie. Wpatrywali się niewzruszenie w~chłopca.

-- Co to jest? -- Próbował brawury, ale na końcu głos się załamał z~piskiem. Maczeta zadrżała.

-- Ostrożnie -- szepnął Ashok do siebie, do Mali, do każdego, kto chciał słuchać. Przestraszony tyran był jeszcze mniej przewidywalny niż pewny siebie.

-- Pani Dibyendu poprosiła nas, żebyśmy ponownie otworzyli jej kawiarnię -- powiedziała Mala, gestykulując telefonem trzymanym w~wolnej ręce. -- Możesz już odejść. 

-- Nowy właściciel poprosił nas, żebyśmy pilnowali \textit{jego }kawiarni -- powiedział chłopiec i~wszyscy na ulicy usłyszeli kłamstwa zarówno Mali, jak i~chłopca. Ashok próbował dowiedzieć się, ile lat ma chłopiec. 14? 15? Młody, głupi, pijany, wściekły i~uzbrojony.

-- Ostrożnie -- wyszeptał ponownie.

Mala schowała telefon do kieszeni i~podniosła kamień, nie spuszczając oczu z~chłopca.

-- Pięć -- powiedziała.

Uśmiechnął się do niej i~wypluł strumień różowej śliny betelowej w~kierunku jej stóp. Nie poruszyła się. Nikt się nie poruszył.

-- Cztery.

Uniósł maczetę, celując ją prosto w~nią. Wydawała się tego nie zauważać.

-- Trzy. 

W alejce zaległa cisza. Ktoś na motocyklu próbował przecisnąć się przez tłum, po czym zatrzymał się, wyłączając silnik.

-- Dwa. 

Oczy chłopca skakały w~lewo, w~prawo, znowu w~lewo. Zagwizdał wtedy, mocno i~głośno, a za jego plecami z~kawiarni dobiegł skrobanie bosych stóp.

-- Jeden -- powiedziała Mala. i~podniosła kamień, zwijając się znowu jak kręgarz do krykieta, całe ciało zwinięte, a Ashok pomyślał, \textit{muszę coś zrobić. Muszę ich powstrzymać. To szaleństwo}. Ale jego usta, ręce i~stopy miały inne pomysły. Pozostał zamrożony w~miejscu.

Chłopak uniósł maczetę na piersi, a trzymająca ją dłoń zadrżała jeszcze bardziej. Nagle Mala rzuciła. Kamień leciał tak szybko, że w~gorącym, wilgotnym porannym powietrzu wydał z~siebie skwierczący dźwięk, ale nie roztrzaskał głowy chłopca, tylko roztrzaskał się na kawałki o~framugę drzwi za nim, wyraźnie ją wgniatając. Chłopiec wzdrygnął się, gdy strzaskany kamień odbił się od jego nagiej twarzy, klatki piersiowej, ramienia i~pleców, a kilka zabłąkanych kawałków odbijało się od maczety.

-- Odejdź -- powiedziała Mala. Za chłopcem, pięciu kolejnych chłopców, tłoczących się w~drzwiach, każdy ze swoją maczetą. Podnieśli ręce. 

-- Walka! -- syknął jeden z~chłopców, najmniejszy. 

Coś było nie tak z~jego głową, pajęczyna blizn i~niejednolitych włosów spływających po lewej stronie, jakby miał rozwaloną głowę lub był wleczony. Ashok nie mógł odwrócić wzroku od tego małego chłopca. Miał kuzyna tej wielkości, małego chłopca, który lubił grać w~gry w~salonie i~biegać z~przyjaciółmi. Mały chłopiec z~butami i~jasnymi oczami, trzy posiłki dziennie i~matka, która co wieczór tuliła go z~pocałunkiem w~czoło.

Mala utkwiła w~chłopcu wzrok. 

-- Nie walcz -- powiedziała. -- Jeśli walczysz, przegrywasz. Zranisz się. Uciekaj. 

Armia uniosła broń i~wydała niski, dudniący dźwięk, który przerodził się w~warczenie. Jeden z~chłopców rozmawiał przez telefon i~szeptał do niego natarczywie. Ashok dostrzegł ich strach i~poczuł lekką ulgę, ci odejdą, a nie będą walczyć. 

-- Uciekaj! -- powiedziała Mala i~ruszyła naprzód. Wszyscy chłopcy się wzdrygnęli .

A niektórzy z~armii zachichotali, nienawistny dźwięk, który słyszał tysiące razy w~grze, drwiący dźwięk, który rozszedł się po szeregach jak wąż pełzający wokół ich stóp, a strach na twarzach chłopców się zmienił. Stał się gniewem.

Chwila balansowała na nici cienkiej jak pajęczy jedwab, chichoczący żołnierze, gotujący się chłopcy, maczety, maczugi i~kije, kamienie \ldots 

Chwila się skończyła. Najmniejszy chłopiec trzymał maczetę nad głową i~zaszarżował na nich, wykrzykując coś bez słów, wyjąc, naprawdę, dźwięk, jakiego Ashok nigdy nie słyszał od chłopca. Zrobił trzy kroki, zanim złapały go dwa kamienie, jeden w~ramię, a drugi w~twarz, rozprysk krwi, chrzęst kości i~ząb, który uniósł się wysoko w~powietrze, gdy chłopiec upadł do tyłu.

I chwila minęła. Maczety podniosły się, pozostałych pięciu chłopców pobiegło na Armię z~szalonym wyrazem twarzy. Ashok zdążył zastanowić się, czy mały chłopiec leżący nieruchomo na ziemi jest młodszym bratem jednego z~pozostałych badamaash i~wtedy dołączył do walki. Najwyższy chłopak, ten, który tego ranka otworzył drzwi i~splunął na niego, przedarł się przez dwóch żołnierzy, zadając głębokie rany na ich klatce piersiowej i~ramionach, twarz Ashoka pokryta delikatną mgiełką krwi tętniącej życiem gejzeru \ldots  twarz wykrzywiona wściekłością. Szedł po Malę, stojącą centymetry od Ashoka, a krew spływała z~jego maczety i~po ramieniu.

Mala wydawała się zamrożona w~miejscu, a Ashok myślał, że zaraz umrze, żeby zobaczyć, jak umiera jako pierwsza, i~napiął się, krew ryczała mu w~uszach tak głośno, że zagłuszyła straszliwe krzyki otaczających go wojowników, zdesperowanych i~gotowych do chwytania chłopca. Ale kiedy przeniósł wagę, Mala warknęła ,,\textit{NIE!}'' do niego, nie odrywając wzroku od przywódcy, a on się powstrzymał, potykając się o~pół kroku do przodu. Chłopak z~maczetą patrzył na niego przez krótką chwilę, a Mala \textit{zawirowała}, rozwinęła się, odepchnęła od siebie obciążoną laskę z~czubkiem czaszki, a potem szarpnęła ręką, gest, który widział niezliczoną ilość razy na lekcjach walki, a obciążony czubek wbił się w~przedramię chłopca z~trzaskiem, który słyszał ponad odgłosami bitwy, trzaskiem, który ostatnio słyszał tej nocy tyle miesięcy wcześniej, kiedy Mala i~jej armia przyszli po niego i~Yasmin w~nocy. Syn lekarza Ashoka wiedział dokładnie, co oznacza to chrupnięcie.

Rozmazany materiał, gdy Yasmin przetańczyła przed nim, schylając się z~gracją, by podnieść maczetę, a chłopak tylko patrzył z~zaszklonymi oczami i~już wkradał się szok. Yasmin delikatnie i~celowo kopnęła go w~rzepkę, celnym kopnięciem czubkiem sandała, nadchodząc z~boku, a chłopiec upadł, płacząc małym chłopięcym głosikiem, wołając matkę dźwiękiem jak żałosny jak pisklę, które spadło z~gniazda.

Minęły zaledwie sekundy, ale to już koniec. Dwóch chłopców uciekało, jeden szlochał przez zakrwawione usta, dwóch było nieprzytomnych. Ashok szukał rannych żołnierzy. Trzech zostało pociętych maczetami, w~tym dwoje, których widział zranionych przez przywódcę, gdy biegł do Mali. Pamiętając o~krwi tętniczej, czerwonej i~bogatej, Ashok pierwszy odnalazł jej właściciela, leżącego na ziemi z~półotwartymi oczami, z~ciężkim oddechem. Przesunął dłonie po kontuzji, głębokie rozcięcie na lewym ramieniu, które wystrzelało z~każdym uderzeniem w~klatkę piersiową chłopca, i~krzyknął: ,,Koszula, cokolwiek, bandaż'', a ktoś wcisnął koszulę w~jego zakrwawione ręce i~mocno naciskał, tamując krew. 

-- Niech ktoś wezwie lekarza -- powiedział, nawiązując kontakt wzrokowy z~Anam, żołnierką, z~którą wcześniej prawie nie rozmawiał. 

-- Masz telefon? -- Dziewczyna lekko drżała, ale skinęła głową i~poklepała torebkę u boku, z~roztargnieniem wymachując żelazem w~dłoni. Upuściła je. 

-- Dzwonisz do lekarza, rozumiesz? -- Skinęła głową. -- Co zrobisz? 

-- Zadzwonię do lekarza -- powiedziała sennie, ale zaczęła wybierać numer.

 Odwrócił się i~złapał za rękę, która podała mu koszulę, i~zobaczył, że należała do Mali, która zdjęła ją z~innego chłopca w~swojej armii. Jej pierś falowała, ale jej spojrzenie było spokojne.

-- Trzymaj to -- rozkazał, bez chwili skrupułów, by dyktować generałowi.

 To była pierwsza pomoc, do tego został przeszkolony przez ojca, na długo przed studiowaniem ekonomii, i~nie znosiła sporów. Przycisnął jej rękę do zakrwawionej szmaty i~wstał, nie słysząc trzasku stawów. Odwrócił się i~znalazł następną ranną osobę i~następną.

A potem podszedł do chłopca, małego chłopca, którego zniekształcona głowa przykuła jego uwagę. Chłopiec, który został uderzony z~góry i~z dołu dwoma twardymi kamieniami. Cały przód jego szczęki był zmiażdżony, koszmar z~białawych fragmentów kości i~zębów pływających w~galaretce na wpół zakrzepłej krwi. Kiedy Ashok odsunął każdą powiekę, zobaczył, że lewa źrenica była szeroka jak wejście do kanału i~nie skurczyła się, gdy odsunął się i~pozwolił, by słońce świeciło na nią w~pełni. 

-- Wstrząśnienie mózgu -- mruknął w~powietrze, a Yasmin spytała: 

-- Czy to źle?

-- Jego mózg krwawi -- powiedział Ashok. -- Jeśli za bardzo krwawi, umrze. 

 Powiedział to po prostu, jakby czytał z~podręcznika. Chłopiec pachniał okropnie, miał rany na ramionach, klatce piersiowej i~kostkach, opuchnięte, porysowane i~zainfekowane ukąszenia owadów i~czyraki. 
 
 -- Musi iść do lekarza. -- Spojrzał z~powrotem na krwawiącego żołnierza. -- On też. 

Znalazł dziewczynę, która obiecała wezwać lekarza.

 -- Gdzie jest lekarz? 
 
 Nie miał pojęcia, ile czasu minęło, odkąd kazał jej zadzwonić. Mogło to zająć dziesięć minut lub dwie godziny.

Wyglądała na zdezorientowaną. 

-- Karetka -- zaczęła. Rozejrzała się bezradnie. -- Przyjedzie, powiedzieli.

A teraz, kiedy tego nasłuchiwał, usłyszał to, odległe di-da, di-da. Wąska uliczka, przy której mieściła się kawiarnia pani Dibyendu, nigdy nie przyjęłaby szerokiej karetki pogotowia. Yasmin bez słowa pobiegła do głównej drogi, żeby ją zawołać. A teraz, kiedy Ashok słuchał, słyszał: sąsiadów z~głowami wystającymi z~okien i~drzwi, przekazujących wściekłe opinie i~gupszup. Dopingowali armię Mali, spuszczali klątwy na badamaash ze swoimi maczetami, opłakiwali odejście pani Dibyendu, paplali jak tropikalne ptaki o~tym, jak została zmuszona do ucieczki, płacząc i~ścigana drogą w~ciemności nocy.

Ashok był pokryty krwią. Pokrywała jego ręce, ramiona, klatkę piersiową, twarz. Jego usta były pokryte zaschniętą krwią, a w~ustach miał miedziany posmak. Jego koszula i~spodnie -- przemoczone. Wyprostował się i~rozejrzał po zatłoczonej alejce, w~górę na gadających, mrugających jak sowy. Wokół niego żołnierze i~ranni.

Mala szeptała natarczywie do ucha Sushant, chłopiec słuchał uważnie. Potem zaczął poruszać się wśród żołnierzy, wpychając ich do środka. Webblies mieli pracę do wykonania. Wkrótce przybędzie policja, a ludzie w~budynku będą mieli moralną kompetencję, by twierdzić, że to ich. Chłopcy ze swoimi maczetami, ranni lub zabici, nie będą mieli żadnych roszczeń. Ashok zastanawiał się, czy zostanie aresztowany, a jeśli tak, to czy zdoła się wydostać. Może jego ojciec mógłby się tym zająć. Ważny człowiek, lekarz, mógłby się opiekować \ldots 

Przybyło dwóch sanitariuszy pogotowia ratunkowego, niosąc ciężkie torby i~zwinięte nosze. Byli to miejscowi, z~akcentem Dharavi, wysłani ze szpitala Lokmanya Tilak, ogromnego budynku o~dobrej reputacji. Szybko opisał obrażenia mężczyzn i~rozdzielili się, aby przyjrzeć się najpoważniejszym przypadkom, głębokiemu przecięciu tętnicy i~wstrząsowi mózgu. Ashok trzymał się blisko małego chłopca, czując się w~jakiś sposób odpowiedzialny za niego, bardziej odpowiedzialny niż za swojego kolegę z~drużyny, patrzył, jak technik zakłada mu opaskę na szyję, a następnie uruchamia napełniający go pojemnik z~powietrzem, unieruchamiając mu głowę. Technik ostrożnie umieścił plastikowy pierścień w~środku ortezy w~kształcie pączka, na zniszczonej szczęce i~nosie chłopca, aby plastik nie przeszkadzał mu w~oddychaniu. Rozłożył nosze, zacisnął szelki i~spojrzał na Ashoka.

-- Znasz procedurę?

Zamiast odpowiedzieć, Ashok ustawił się na chudych biodrach chłopca, kładąc dłoń na każdym z~nich, gotowy do podciągnięcia go w~tym samym czasie, co medyk, utrzymując całe ciało w~jednej linii, aby uniknąć pogorszenia jakichkolwiek obrażeń kręgosłupa. Sanitariusz wsunął nosze na miejsce, a Ashok odwrócił chłopca. Przez krótką chwilę podtrzymywał w~dłoniach prawie cały ciężar chłopca, a dziecko wydawało się nie ważyć nic, zupełnie nic, jakby był pusty. Ashok stwierdził, że płacze, ciche łzy spływają mu po twarzy, zbierając krew, wślizgując się do ust, podwójnie słona mieszanka krwi i~łez.

Mala w~milczeniu wsunęła rękę w~jego. Było jej bardzo ciepło w~przytłaczającym upale poranka. Niedługo spadnie deszcz, wilgotność nie utrzyma się tak wysoka przez cały dzień, woda wkrótce się połączy, a potem krew spłynie do szorstkich rynsztoków biegnących wzdłuż alejki.

-- Był odważnym dzieckiem -- powiedziała Mala.

Ashok nie mógł znaleźć odpowiedzi.

-- Myślę, że myślał, że jeśli zaatakuje nas tym nożem, pokroi jednego z~nas, tak się przestraszymy, że odejdziemy na zawsze.

-- Więc naprawdę go rozumiesz? -- Ashok zobaczył, jak Yasmin podkrada się do nich, wsuwa palce w~palce Mali.

Mala nie odpowiedziała.

-- Wszyscy myślą -- powiedziała Yasmin -- że można wygrać walkę, uderzając pierwszy. -- Ramię Mali zacisnęło się na ramieniu Ashoka. -- Ale czasami wygrywasz walkę, nie walcząc.

Mala odpowiedziała: 

-- Powinniśmy nazywać cię generałem Gandhiji.

-- To byłby zaszczyt, ale nie mogłem dorównać Gandhiemu. Był wspaniałym człowiekiem. -- powiedział Ashok. -- Gandhi przyznał się do bicia swojej żony. Był wspaniałym człowiekiem, ale nie świętym. -- Przełknął. -- Nikt nie wspomina, że Gandhi miał w~sobie całą tę przemoc. Myślę, że to czyni go lepszym, ponieważ oznacza to, że jego droga nie była tylko naturalnym instynktem, z~którym się urodził. To było coś, o~co walczył, w~jego własnym umyśle, codziennie. 

Spojrzał w~dół na czubek głowy Mali, przez chwilę zaskoczony, gdy zdał sobie sprawę, że jest niższa od niego. Miał tendencję do myślenia o~niej jako o~wielkiej, większej niż życie.

Mala spojrzała na niego i~wydawało się, że jej ciemne oczy jarzyły się w~gorącym, parnym powietrzu, patrząc spod jej długich rzęs. 

-- Kontrolowanie siebie jest przereklamowane -- powiedziała. -- Jest wiele do powiedzenia o~odpuszczaniu.

Było na nich tak wiele oczu, tak wielu ludzi obserwowało ich z~każdego zakątka drogi, a Ashok nagle poczuł się bardzo skrępowany.

Wewnątrz kawiarnia była ledwo rozpoznawalna. Śmierdziało jak legowisko jakiegoś chorego zwierzęcia, które zapadło się na ziemię, a jeden róg był używany jako toaleta. Wiele komputerów zostało niedbale przesuniętych, odłączając przewody, a jeden ekran leżał w~kawałkach na podłodze. Na podłodze widniały smugi śliny betelu i~puste butelki taniego, ognistego alkoholu, tak okropnego, że nie chcieli go pić nawet starzy pijacy na ulicach.

Ale było też zdjęcie, mocno pogniecione i~złożone, przedstawiające znoszoną, ale wciąż ładną kobietę, formalnie upozowaną, trzymającą niemowlę i~nieco większego chłopca, którego Ashok pamiętał ze starcia. Pomyślał, że to dziecko musiało być tym młodszym chłopcem i~zastanawiał się, co się stało z~kobietą i~jak została oddzielona od synów, których trzymała z~taką miłością. A im więcej się zastanawiał, tym bardziej czuł się zdrętwiały i~smutek, aż smutek zalewał go czarnymi falami, jak zbliżający się przypływ, aż ugiął się w~kolanach i~upadł na podłogę, a jeśli któryś z~żołnierzy zobaczył, że obejmuje się i~płacze, nikt nie powiedział ani słowa.

Jego dokumenty były nienaruszone, głównie w~pokoju na zapleczu, w~którym pracował, a połączenie sieciowe wciąż działało, wszystkie śmieci zostały wymiecione przez drzwi, okna zostały otwarte, i~wkrótce odgłosy radosnej walki i~żołnierskiego dobrego samopoczucia wypełniały kawiarnię pani Dibyendu tak jak przez wiele dni wcześniej. Ashok wpadł w~liczby i~arkusze, widząc, jak poradzi sobie z~nowymi datami, i~był tak pochłonięty, że nawet nie zauważył nagłej ciszy w~kawiarni, która oznaczała przybycie policjanta.

Policjant -- gruby, skorumpowany, sam stary szczur z~Dharavi, większa kreatura ze slumsów niż dzieci -- zebrał już relację od sąsiadów, dowiedział się, że uzbrojone w~maczety badamaash były tu najeźdźcami, i~on nie zamierzał poddać się egzorcyzmowi w~imieniu sześciu małych nic nie znaczących, takich jak oni. Ale kiedy była śmierć, musiała być papierkowa robota \ldots 

-- Śmierć? -- powiedział Ashok.

-- Ten mały. Martwy, zanim dotarł do szpitala.

Ashok miał wrażenie, że podłoga od niego odchodzi, a jedyną rzeczą, która go rozpraszała i~powstrzymywała przed upadkiem, było westchnienie przerażenia Yasmin za nim, dźwięk, który początkowo był wydechem, ale zamienił się w~skomlenie. Odwrócił się i~zobaczył, że zbladła tak bardzo, że rzeczywiście była zielona, a syn lekarza w~nim zauważył, że jej źrenice skurczyły się do ukłuć.

Gruby policjant spojrzał na nią, a jego usta wykrzywiły się w~mokrym, sarkastycznym uśmiechu. 

-- Wszystko w~porządku, panienko?

-- Nic jej nie jest -- powiedziała stanowczo Mala. 

Stała bliżej policjanta, niż było to absolutnie konieczne, zbyt niska, by spojrzeć mu w~oczy, ale wciąż wydawała się spoglądać w~dół. Nieświadomie policjant przeniósł ciężar ciała do tyłu, cofnął się o~krok i~odwrócił.

-- Do widzenia -- powiedział, wymachując notesem, w~którym był numer dowodu osobistego Ashoka; wszyscy żołnierze twierdzili, że nigdy nie zostali zarejestrowani po kartę, w~co Ashok naprawdę wątpił, ale policjant nie kwestionował tego, gdy powietrze świszczało mu z~nozdrzy i~pocił się w~mundurze. 

Deszcze w~końcu nadeszły, niebo otworzyło się jak śluzy, deszcz padający w~postaci arkuszy koloru zanieczyszczenia, które pochłonęli, spadając z~nieba. Stukot na blaszanych ścianach i~dachu przypominał strzelaninę w~jakiejś taniej grze, w~której wszystkie pistolety wydawały metaliczne dźwięki \textit{pong }i \textit{ping}.

Ashok patrzył, jak Yasmin odpływa do małego ,,biura'' pani Dibyendu, pokoju, w~którym przygotowywała herbatę nad małym palnikiem gazowym; patrzyła, jak Mala podąża za nią. Próbował pracować nad swoimi obliczeniami, ale nie mógł się skoncentrować, dopóki nie zobaczył Mali wynurzającej się z~twarzą przyciśniętą do wyrazu twarzy General Robotwallah, ale wciąż były ślady po łzach na jej policzkach. Spojrzała prosto przez niego i~zaczęła szczekać rozkazy swoim żołnierzom, którzy naprawiali kawiarnię i~ponownie uruchamiali wszystkie systemy. Chwilę później wszyscy klikali, krzyczeli, z~założonymi słuchawkami, z~napiętymi ramionami, w~innym świecie i~dołączyli do bitwy.

Ashok znalazł drogę do gabinetu pani Dibyendu, zastał Yasmin kucającą pod ścianą, z~obcasami płasko na ziemi i~rękami przed nią. W milczeniu wpatrywała się w~te dłonie, owijając je wokół siebie jak węże.

-- Yasmin -- wyszeptał. -- Yasmin? 

Spojrzała na niego. W jej oczach nie było łez, tylko wyraz bezdennego smutku. 

-- Rzuciłam kamieniem -- powiedziała. -- Kamieniem, który uderzył tego małego chłopca. Rzuciłam nim. Tym, który trafił go w~usta. On był \ldots  -- Przełknęła ślinę.

-- Biegł na nas z~maczetą -- powiedział Ashok. -- Zabiłby nas \ldots  

Rąbnęła ręką w~powietrzu, gestem pełnym nietypowej przemocy. 

-- \textit{Postawiliśmy się w~takiej sytuacji}, w~sytuacji, w~której musielibyśmy go zabić! To była Mala. Mala, ona zawsze chce wygrać przed stoczeniem bitwy, wygrać przez \textit{unicestwienie wroga}. A potem mówi o~\textit{Gandhim}? 

 Wyglądała, jakby miała zamiar coś uderzyć, małe dłonie zacisnęły się w~pięści, a potem nagle rzuciła się do przodu i~zwymiotowała, obficie, kompletnie wyrzucając całą zawartość żołądka, więcej wymiocin, niż kiedykolwiek widział Ashok z~ludzkiego gardła. Pomiędzy konwulsjami na wpół prowadził, na wpół wyprowadzał ją z~kawiarni, we wszechogarniający deszcz, i~pozwalał rzucać na alejkę, która stała się rwącą rzeką, deszcz przelewał się przez wąskie rowy po obu stronach z~tego. Woda płynęła aż do popękanej płyty cementu, która służyła za próg drzwi pani Dibyendu, a hidżab Yasmin natychmiast zmoczył się, gdy wychyliła się, by opryskać wzburzoną powierzchnię wody poories, herbatą i~żółcią. Długa suknia przylegała do jej wąskich pleców i~ramion i~falowała wraz z~nimi, gdy walczyła o~oddech. Ashok też był przemoczony, w~ustach znów pojawił się posmak krwi, gdy woda spłukała zaschniętą krew po jego twarzy. Deszcz uniemożliwiał rozmowę, więc nie musiał się martwić o~kojące słowa.

W końcu Yasmin wyprostowała się, a potem opadła na niego. Objął ją ramieniem, wdzięczny za uczucie drugiego człowieka, ten kontakt, który przeniknął jego odrętwienie. Coś przeszło między nimi, niosło się łomotem ich serc, przenoszonym przez skórę i~przez chwilę czuł, jakby tu, tu nareszcie był ktoś, kto wszystko o~nim rozumiał i~tu był ktoś, kogo rozumiał. Chwila dobiegła końca, znikała, aż stali w~zakłopotanym, niezręcznym pół-uścisku, po czym bez słowa rozplątali się i~wrócili do środka. Ktoś wytarł wymiociny szmatami, które pozostawili po sobie baadmashe, a potem kopnął je w~śmierdzący stos w~rogu. Yasmin usiadła przy komputerze i~zalogowała się, słuchając uważnie paplaniny wokół niej, łapiąc porządek bitwy, podczas gdy Ashok poszedł do swojego komputera i~przygotował się na rozmowę z~Big Sister Nor.

\bigskip
\threeast

W dniu rozpoczęcia strajku Wei-Dong był w~trakcie swojego drugiego zadania specjalnego -- pierwszym było przywiezienie pudełka z~kartami przedpłaconymi, które zostały przekazane sieci Webblies w~celu zdrapania, a następnie wpisania klucza i~wysłania do Big Sister Nor, aby mogła podzielić je między bojowników.

Drugie zadanie było pod pewnymi względami trudniejsze: dostał rozkaz znalezienia innych Mechanicznych Turków, którzy mogliby sympatyzować ze strajkującymi, i~zwerbowania ich. Wei-Dong nigdy nie uważał się za przywódcę -- w~szkole zawsze był samotnikiem -- ale Big Sister Nor długo z~nim rozmawiała o~wszystkich sposobach, w~jakie mógłby przekonać innych Turków do rozważenia, dołączając do tego dziwnego przedsięwzięcia.

Technicznie rzecz biorąc, było to dość proste. Jako Turek miał dostęp do rankingów aktywności Turków, z~których Coca-Cola Online zrobiła wielką sprawę, aktualizując je co dziesięć minut. Tabele wyników zawierały nazwiska każdego Turka i~pokazywały, w~jakich częściach gry się bawił, ile zapytań obsługiwał w~ciągu godziny, jak wysoko oceniane były decyzje Turków i~odgrywanie ról przez graczy, którzy zostali przebadani w~sposób losowy przez bota satysfakcji, który rozdawał rzadkie odznaki każdemu graczowi, który wypełnił kwestionariusz w~grze. Pomysł polegał na zainspirowaniu Turków, pokazując im, o~ile lepiej radzą sobie ich rówieśnicy. To też zadziałało, Wei-Dong spędził wiele nocy, próbując podkręcić swoje statystyki, aby móc wyprzedzić innych Turków, osiągając najwyższe wyżyny, zanim został powalony przez czyjś całonocny bieg. I oczywiście, kiedy wyprzedziłeś innego Turka, musiałeś zostawić dla niego publiczne ,,przesłanie zachęty'', nie dłuższe niż 140 znaków, aby można było je wysłać na Twitterze i~wysłać SMS-em bezpośrednio do nich, a te wiadomości stworzyły nowe granice skrajnie lakonicznych wulgaryzmów i~przechwałek. 

Wei-Dong miał nowe zastosowanie dla plansz: używał ich, aby dowiedzieć się, którzy gracze prawdopodobnie zmienią strony. Prowadzący gry stworzyli obiekt do masowego pobierania z~nich danych historycznych, a Turków zachęcano do robienia szalonych miksów i~wizualizacji pokazujących, czyja gra była najlepsza. Wei-Dong miał inny pomysł.

Od tygodni pobierał gigantyczne ilości danych z~tablic, przesyłając je do bazy danych, którą Matthew pomagał mu zbudować, a teraz mógł uruchomić na niej bardzo wyspecjalizowane zapytania, takie jak: ,,Pokaż mi Turków, którzy dowodzili rankingiem, ale odpadli pomimo długich godzin pracy'' lub ,,Pokaż mi Turków, którzy używają dużo wulgaryzmów podczas wypełniania okna dialogowego dla postaci niezależnych''. A zwłaszcza: ,,Pokażcie mi Turków, którzy mają poniżej przeciętnego poziom donoszenia na farmerów złota do bossów''. To ostatnie było dużym przedsięwzięciem wśród Turków, którzy otrzymywali dużą premię za każdym razem, gdy przyłapali farmera. Większość Turków dość często ,,odwszawiała'', chcąc zgarnąć dodatkową gotówkę. Ale znaczna mniejszość nigdy, przenigdy nie polowała na rolników, a to był naturalny punkt startowy dla Wei-Donga. 

Miał długą listę kanałów, a dla każdego z~nich miał harmonogram zwyczajowych godzin logowania Turka i~części świata, w~których Turek pracował najczęściej. Potem wystarczyło zalogować się za pomocą jednego z~Webbliesowych, wielu, wielu toonów, kierując się do tej części świata, wzywając Turka i~mając nadzieję, że pojawi się właściwa osoba. Łatwiej byłoby po prostu korzystać z~forów dyskusyjnych Turków, ale gdyby to zrobił, zostałby złapany i~zwolniony w~ciągu kilku sekund. Ta droga była mniej wydajna, ale o~wiele bezpieczniejsza.

Teraz był w~Gwiezdnych Polach Goomby, chmurach w~Mushroom Kingdom, gdzie gwiazdy doładowania były uprawiane w~niekończących się rzędach. Gracze mogli tu wykonywać zadania, podejmując pracę u komicznych farmerów, którzy zmuszali ich do pracy przy odchwaszczaniu gwiezdnych łat i~wyrywaniu dojrzałych gwiazd. To było dobre dla treningu twoich umiejętności; wysoko postawiony Star Farmer mógłby wydobyć więcej mocy ze swoich gwiazd.

A oto rolnik, żujący łodygę kukurydzy i~miotający się po swojej stodole, która też była zrobiona z~chmur. Zaoferował Wei-Dongowi misję, niskopoziomową, po prostu wyrywanie chwastów z~niektórych łatwiej dostępnych chmur, tych, które nie były patrolowane przez wrogich Lakitusów. Wei-Dong przyjął zadanie, a następnie otworzył czat z~rolnikiem: 

-- Od jak dawna jesteś właścicielem tej farmy?

-- Och, młodzieńcze, pracuję na tej farmie, odkąd byłem chłopcem, a mój tato pracował to przede mną, a jego tato przed nim. Ta, mógłbyś powiedzieć, że jesteśmy farmerską grodziną, heee!

To był oczywiście domyślny dialog. Żaden Turek nigdy nie zdołałby się zmusić do pisania czegokolwiek tak wiejskiego. Rolnik NPC miał całą gamę zgryźliwych odpowiedzi na głupie pytania. Sztuką przywołania Turka było wyjście poza pole.

-- Lubisz rolnictwo? 

-- Ay-ya, możesz powiedzieć, że tak. To dobre życie  \ldots kiedy świeci słońce! Heej!

Wei-Dong przewrócił oczami. Kto to \textit{napisał}? 

-- Jakie problemy masz jako rolnik?

-- Och, to dobre życie  \ldots  kiedy świeci słońce! Heej!

Wei-Dong uśmiechnął się lekko. Gdy NPC zacznie się powtarzać, zostanie przywołany Turk. Rolnik wydawał się lekko drgnąć.

-- Czy masz jakieś problemy poza brakiem słońca? 

-- Och, młodzieńcze, nie chcesz słuchać narzekań starego rolnika. Wiele razy harowałem na tych polach i~ręce mam zmęczone. Porozmawiajmy o~przyjemniejszych rzeczach, jeśli łaska. -- To było bardziej podobne. Dialog był czymś, co wymyśliłby entuzjastyczny Turek w~odgrywaniu ról, a to pasowało do profilu Turka, którego szukał.

-- Czy masz na imię Jake Snider? -- wpisał

Postać nie poruszyła się przez sekundę. 

-- Nie rozumiem tego Jake Snider, młodzieńcze. Lepiej byś teraz wykonywał swoje obowiązki.

-- Myślę, że \textit{jesteś }Jake Sniderem i~myślę, że wiesz, że nie dostajesz uczciwej umowy z~Colą. Pracujesz więcej godzin niż kiedykolwiek, ale twoja pensja jest o~wiele niższa. Jak myślisz, dlaczego tak jest? Czy wiesz, że Coca-Cola Games właśnie miało swój najlepszy kwartał w~historii? I że cała grupa kierownicza dostała 20-procentową podwyżkę? Czy wiesz, że Coca-Cola systematycznie rotuje Turków, którzy zarabiają zbyt dużo pieniędzy na ich służbie, zastępując ich nowicjuszami, którzy nie wiedzą, jak zmaksymalizować swoje przychody?

Rolnik zaczął odchodzić, grabiąc przez ramię. Wei-Dong podążył za nim.

-- Czekaj! Jest sprawa. \textit{Nie musi tak być}! Pracownicy mogą się organizować i~żądać od swoich szefów lepszych warunków. Pracownicy się \textit{organizują }. Jeszcze dwa miesiące i~wylądujesz na ulicy. Czy nie warto walczyć o~twoją pensję i~godność?

Rolnik zmierzał do swojego domu. Wei-Dong pomyślał przez chwilę, że znowu rozmawia z~NPC, że Turek się wylogował. Ale nie, w~ruchach rolnika było trochę niezdarności, trochę wahania. W domu wciąż był ktoś. 

-- Wiem, że nie możesz ze mną rozmawiać w~grze. Oto adres e-mail D9FA754516116E89833A5B92CE055E19BCD2FA7@gmail.com. Wyślij mi wiadomość, a porozmawiamy na osobności.

Wstrzymał oddech. Turek mógł wydać go zarządcy gry, w~którym to przypadku jego pionek zostałby zniszczony w~ciągu kilku minut, a Webblies straciliby jeszcze jedną postać i~jeszcze jedną kartę przedpłaconą. Ale NPC wszedł do jego domu i~nic się nie stało. Wei-Dong poczuł trzepotanie w~klatce piersiowej, a potem kolejne, kilka minut później, gdy jego e-mail zadzwonił.

\noindent {\textgreater} Powiedz mi więcej

Nie był podpisany, ale wiedział, od kogo pochodzi.

\bigskip
\threeast

-- Powinnaś pojechać do Hongkongu -- powiedział Lu do Jie, trzymając ją mocno za rękę i~patrząc jej w~oczy. -- Stamtąd możesz zrobić audycje. To bezpieczniejsze.

Jie odwróciła głowę i~dmuchnęła strumieniem powietrza. Ścisnęła jego dłoń. 

-- Wiem, że chcesz najlepiej, Tank, ale nie zrobię tego i~chcę, żebyś przestał o~tym mówić. Jestem Webblies tak jak ty, tak jak wszyscy tutaj. Jasne, mogę nadawać z~Hongkongu, \textit{technicznie }rzecz biorąc, ale \textit{o czym }miałbym nadawać? Jestem dziennikarką, Tank. Muszę tu być, żeby zobaczyć, co się dzieje, żeby o~tym zrelacjonować. Nie mogę tego zrobić z~HK.

-- Ale to nie jest bezpieczne \ldots 

Przerwała mu gestem cięcia. 

-- Oczywiście, że to nie jest bezpieczne! Nie interesowałam się bezpieczeństwem od dnia, w~którym weszłam na antenę. Nie jesteś bezpieczny. Moje dziewczyny z~fabryki nie są bezpieczne. Webblies na pikiecie nie są bezpieczni. Dlaczego ja powinnam być bezpieczna?

Lu przygryzła słowa: \textit{bo cię kocham}. Potajemnie poczuł ulgę. Nie wiedział, co by zrobił, gdyby Jie był w~Hongkongu, a on w~Shenzhen. Ostatnia z~jej kryjówek, inne mieszkanie w~budynku ,,uścisków dłoni'', była zatłoczona Webblies, czterdziestu chłopców, którzy uważnie ich ignorowali, ale wiedział, że nasłuchują. Spali tutaj na zmiany, po czterdziestu na raz, a osiemdziesięciu kolejnych poszło do pracy w~przyjaznych kafejkach internetowych, uważając, aby nigdy nie wysłać więcej niż dwóch lub trzech do jednej kawiarni, aby nie zwrócić na siebie uwagi. Zaledwie dzień wcześniej dwóch chłopców wyszło z~kawiarni śledzeni przez dwóch anonimowych twardych mężczyzn, którzy metodycznie wykopali z~nich te wiecznie kochane bzdury, prosto na publicznej ulicy, wysyłając jednego do szpitala.

-- Wiesz, to tylko kwestia czasu, zanim to miejsce zostanie wysadzone w~powietrze -- powiedział Lu. -- Ktoś stanie się nieostrożny i~będzie śledzony do domu albo jeden z~sąsiadów zacznie opowiadać o~chłopcach, którzy o~każdej porze wchodzą i~wychodzą z~mieszkania, a potem \ldots 

-- A potem przejdziemy do innego -- powiedziała. -- Wynajmuję i~tracę mieszkania dłużej, niż ty zabijasz trolle. Dopóki reklamy będą się opłacać, będę zarabiać, a jeśli zarabiam, będę mogła wynajmować.

-- Jak długo reklamodawcy będą płacić za spędzanie trzech godzin każdej nocy na mówieniu dziewczynom z~fabryki, by walczyły przeciwko swoim szefom?

Uśmiech igrał na jej ustach, tajemniczy, pewny siebie uśmiech, który zawsze topił jego serce. 

-- Och, Tank -- powiedziała. -- Reklamodawców nie obchodzi, o~czym mówię, dopóki dziewczyny z~fabryki słuchają i~słuchają.

Poklepała go po dłoniach. 

-- A teraz chcę, żebyś poszedł i~znalazł mi Webbly'ego do przeprowadzenia wywiadu dziś wieczorem, kogoś, kto może mi powiedzieć, jak to wszystko idzie. Jeszcze jakieś protesty?

Potrząsnął głową. 

-- Żadnych głośnych. Zbyt wiele aresztowań. -- W więzieniu w~całych południowych Chinach przebywało ponad stu Webblies. -- Ale słyszałaś o~Dongguanie?

Potrząsnęła głową.

-- Webblies mają nowy rodzaj demonstracji. Zamiast robić dużo hałasu i~wykrzykiwać slogany, wszyscy chodzą bardzo powoli po dworcu autobusowym, w~samym środku miasta, jedząc lody.

-- Lody? 

Uśmiechnął. 

-- Lody. Po tym, jak jingcha zaczął aresztować każdego, kto tylko \textit{wyglądał}, jakby miał zamiar zaprotestować, zaczęto umieszczać te bardzo publiczne ogłoszenia: ,,pojaw się w~takim a takim miejscu i~kup lody''. Dziesiątki, a potem setki osób jadły lody, uśmiechały się jak wariaci, a policja była tam, gapiąc się na siebie jak manekiny, jakby: ,,\textit{Czy aresztujemy tych chłopców za zjedzenie lodów}?'', a potem ktoś wpadł na pomysł kupienia \textit{dwóch }lodów i~oddania jednego przypadkowemu przechodzącemu. To najłatwiejsze narzędzie rekrutacyjne, jakie możesz sobie wyobrazić! 

Śmiała się tak długo i~mocno, że łzy spływały jej po twarzy. 

-- Kocham was, chłopaki -- powiedziała. -- Nie mogę się \textit{doczekać}, aby porozmawiać o~tym w~dzisiejszym programie.

-- Jeśli zostaną aresztowani za jedzenie lodów, przestawią się na wspólne spędzanie czasu i~\textit{uśmiechanie }się do siebie. Wyobrażasz sobie? \textit{Czy aresztujemy tych chłopców za uśmiechanie się?}

Jej śmiech przedarł się przez niewidzialną ścianę, która oddzielała ich od wypoczywających, niepracujących Webbliesów, którzy chcieli się dowiedzieć, co jest tak zabawnego. Nie wszyscy z~nich wiedzieli o~lodach, byli zbyt zajęci patrolowaniem światów, nie dopuszczając do prowadzenia farm złota z~zastępczymi robotnikami, ale wszyscy zgadzali się, że to czysty geniusz.

Wkrótce ściągali filmy o~jedzeniu lodów, a potem pojawiła się kolejna zmiana chłopców, którzy chcieli posłuchać dowcipów, i~zanim się zorientowali, planowali własny festiwal jedzenia lodów, a ogólna wesołość trwała, dopóki Jie i~Lu nie wymknęli się, by obsadzić swój program na wieczór, chwytając kilka rozhisteryzowanych Webblisów na wywiad pomiędzy telefonami od dziewczyn z~fabryki.

Kiedy Lu położył głowę na poduszce i~objął ramieniem wąskie ramiona Jie i~wtulił twarz w~jej gęste, pachnące włosy, miał chwilę spokoju i~radości, prawdziwej radości, wiedząc, że nie mogą przegrać.

\bigskip
\threeast

Strajk wkraczał w~swój drugi tydzień, gdy imperium odpowiedziało. Connor wiedział o~strajku od kilku dni, ale nie od razu podjął działania. Na początku nie był pewien, czy \textit{chce }działać. W końcu pasożyty zajmowali się nawzajem, a strajkujący wykonywali lepszą robotę zamykania rynków złota niż kiedykolwiek (choć przyznanie się do tego bolało). Poza tym w~organizacji tych postaci było coś \textit{fascynującego }, wszyscy łączyli się przez proxy, ale obserwując ich harmonogramy snu i~wąchając ich paplaninę, wiedział, że są rozproszeni po całym Pacyfiku i~na subkontynencie. Siedząc tam w~oku cyklona, w~Centrali Dowodzenia, czuł się, jakby miał miejsce w~pierwszym rzędzie do niesamowitego i~dzikiego cyrku pcheł, w~którym egzotyczne, opancerzone owady walczyły ze sobą bez końca, poruszając się w~precyzyjnych żołnierskich liniach, która mówiła o~dyscyplinie wojskowej.

Ale nie mógł ich zostawić, by robili to w~nieskończoność. Nie był jedynym w~Command Central, który zauważył, że to się dzieje, a rynki instrumentów pochodnych zaczęły wyłapywać wiadomości, skacząc tak szaleńczo, że nawet prasa głównego nurtu zaczęła węszyć. Kilka lat temu rynki złota z~gier były egzotyczną, niemądrą porą roku, ale obecnie jedynymi osobami, które zwracały na nie uwagę, byli gracze: masowi handlowcy kontrolujący ogromne fortuny, które kupowały i~sprzedawały złoto z~gier i~jego liczne subskrypcje oraz gatunki tak szybko jak możliwe. Dopóki, oczywiście, nie zaczęły się rozchodzić informacje o~tych Webblies i~ich toczonych bitwach, ich lodzianych spotkaniach towarzyskich, ich globalnym zasięgu, a teraz korporacyjny PR dzwonił do Centrum Dowództwa pięć razy dziennie, próbując umówić się na spotkanie, aby mogli uzgodnić, co powiedzieć prasie.

Tak więc w~poniedziałek rano zebrał całe Centrum Dowodzenia, wraz z~kilkoma fajniejszymi -- to znaczy mniej neurotycznie paranoicznymi -- prawnikami i~kilkoma starszymi PR-owcami w~jednej z~bezpiecznych sal posiedzeń zarządu Coca-Coli na sesję z~tablicą.

-- Powinniśmy po prostu wytępić te pasożyty -- powiedział Bill. -- Możesz mieć dziesięć kawałków. 

 Zakład Connora i~Billa stał się żartem w~Centrum Dowodzenia, ale Connor i~Bill wiedzieli, że był śmiertelnie poważny. Obaj byli częścią rynków finansowych i~wiedzieli, że zakład jest po prostu innym rodzajem transakcji finansowej i~musi być honorowany.

Uśmiech Connora był ponury. Nie wiedział, czy szef ochrony przejdzie na jego stronę; był takim pragmatykiem w~tych sprawach. Może jednak coś zrobią. 

-- Wiesz, że jestem z~tobą, ale pytanie brzmi, jak wysoką cenę jesteśmy gotowi zapłacić za pozbycie się tych ludzi?

-- Żadna cena nie jest zbyt wysoka -- powiedział Kaden, który szczycił się tym, że jest najbardziej macho w~Centrum Dowodzenia, typem faceta, który nie zamyka się na temat swojej kolekcji broni i~umiejętności karate. Kaden mógł mieć czarny pas 20 lat temu, ale pięć lat w~Centrali Dowodzenia uczyniło go hojnie grubym i~niezdolnym do wchodzenia po schodach bez utraty tchu.

Bill -- sam nie był lekki -- wyciągnął głowę i~spojrzał łamiącym się wzrokiem na Kadena. Mruknął lekceważąco i~spytał: 

-- Naprawdę?

Kaden, zawołał przed salą pełną ludzi, czerwony, pociągnął dalej. 

-- Cholerna racja. Ci oszuści są w~\textit{naszych }światach. Możemy ich przewyższyć i~wymanewrować. Musimy mieć jaja, żeby zrobić to, co trzeba, zamiast tchórzyć jak cipa tak, jak zawsze.

Bill znów chrząknął, odgłos przypominający mikser z~niestrawnością. 

-- Żadna cena nie jest zbyt wysoka? 

-- Nie. 

-- Może zamknąć grę? Czy ta cena jest zbyt wysoka? 

-- Nie bądź głupi. 

-- Nie sądzę, żebym to ja był głupi. Istnieje górna granica tego, ile ta firma może wydać na tych palantów. Jeśli usunięcie ich z~gry kosztuje nas więcej niż pozostawienie ich tam, po prostu strzelamy sobie w~głowę. Więc przestańmy mówić o~,,tchórzeniu'' i~,,żadne koszty nie są zbyt wysokie'' i~ustawmy pewne parametry, które możemy zamienić w~działanie, dobrze?

-- Chcę tylko powiedzieć \ldots 

Bill wstał ze swojego miejsca i~odwrócił się do Kadena, mierząc go miażdżącym spojrzeniem. 

-- Wyjdź -- powiedział. -- Po prostu wyjdź. Jesteś całkiem niezłym projektantem poziomów, ale widziałem lepszych. A jako osoba, jesteś porażką. Nie masz nic użytecznego do dodania do tej dyskusji, prócz głupich sloganów. Słyszeliśmy głupie slogany. Idź, wzmocnij swojego paladyna czy coś i~pozwól dorosłym się tym zająć.

W sali konferencyjnej zapadła cisza. Connor, stojąc z~przodu sali, myślał o~tym, żeby powiedzieć Billowi, żeby się wycofał, ale chodziło o~to, że miał rację, Kaden był totalnym dupkiem, a pozwolenie mu na mówienie tylko odwróciłoby ich uwagę od wykonania zadania.

Kaden siedział przez chwilę z~otwartymi ustami jak ryba, po czym rozejrzał się w~poszukiwaniu wsparcia. Nie znalazł żadnego. Bill wykonał protekcjonalny drobny gest sio. Twarz Kadena zmieniła się z~czerwonej na fioletową.

-- Po prostu idź -- powiedział Connor i~to przerwało chwilę. Kaden wykradł się z~pokoju jak zbity pies i~wszyscy odwrócili się do Connora.

-- Dobrze -- powiedział Connor. -- Oto chodzi o~to, że chodzi o~rozwiązanie problemu, a nie o~pozowanie lub uderzanie w~piersi. Więc trzymajmy się problemu. -- Skinął na Billa.

Bill wstał i~odwrócił się twarzą do publiczności. 

-- Oto, co nie działa: adresy IP. Przychodzą z~serwerów proxy w~całych Stanach Zjednoczonych i~mogą znaleźć serwery proxy szybciej niż my możemy je umieścić na czarnej liście. Dodatkowo mamy mnóstwo legalnych klientów, głównie emigrantów, którzy mieszkają w~Chinach i~Azji i~używają tych serwerów proxy, aby uciec od lokalnych blokad sieci. Ale nawet gdybyśmy byli gotowi wrzucić tych klientów pod autobus, aby zatrzymać farmerów złota, nie moglibyśmy.

-- Również nie działa: śledzenie płatności. Te konta są kupowane na legalnych kartach przedpłaconych. Innymi słowy, wszyscy farmerzy są płacącymi klientami. Moglibyśmy wyłączyć karty przedpłacone i~nalegać na karty kredytowe, ale oni po prostu załatwiliby sobie przedpłacone karty kredytowe. A każde dziecko w~Ameryce, Kanadzie i~Europie, które płaci za swoje konto kartami przedpłaconymi ze sklepu na rogu, nie będzie miało szczęścia. To wielu klientów do wrzucenia pod autobus, a oni po prostu przejdą do jeden z~naszych konkurentów. Poza tym te karty przedpłacone są \textit{złote}. Dzieci kupują je i~przez połowę czasu nie używają, to dla nas darmowe pieniądze.

-- Wreszcie nie działa: profilowanie behawioralne. Tak, te postacie mają pewne stereotypowe zachowania, takie jak wielogodzinne wykonywanie tych samych zadań grindowania lub angażowanie się w~te gigantyczne, epickie bitwy. Ale jest to również charakterystyczne dla ogromnej liczby normalnych graczy, znowu to są ludzie, których nie chcemy wrzucać pod autobus.

-- Więc co będzie działać? 

Connor skinął głową. 

-- Jedno, co wiem, że możemy zrobić, to zwiększyć przebieg z~nalotów, które robimy. Gdy już pozytywnie zidentyfikujemy farmera, powinniśmy być w~stanie wyeliminować całą jego sieć, cofając się do osób, z~którymi rozmawiał, tych, z~którymi się bawił, jego gildii.

Bill kręcił głową i~wydał z~siebie dudniący dźwięk. 

-- To dźwięk twojego autobusu przejeżdżającego przez jeszcze więcej legalnych graczy. Te koty mogą łatwo zepsuć tę strategię, po prostu rekrutując normalnych graczy do ich najazdów i~walk. Do diabła, \textit{zaprojektowaliśmy }to w~ten sposób.

-- Łatwiej będzie wyśledzić pieniądze -- powiedział Fairfax, przerywając im. Patrzyła od jednego do drugiego. -- Mam na myśli to, że ci farmerzy muszą pozbyć się swojego złota, a jeśli odbierzemy je każdemu graczowi, który je kupił \ldots 

-- Zwariowaliby -- powiedział Connor.

-- To niezgodne z~warunkami korzystania z~usługi -- powiedziała. -- Wiedzą, że oszukują. To byłaby sprawiedliwość. Na jakiej podstawie mogliby złożyć skargę? Zgadzają się na warunki za każdym razem, gdy się logują.

Connor westchnął. Warunki usługi miały długość 18 ekranów i~wymagały wykształcenia prawniczego, aby je zrozumieć. Zakazali wszelkiej możliwej aktywności w~grze, włącznie z~zabawą. Technicznie rzecz biorąc, każdy gracz codziennie naruszał warunki, co oznaczało, że gdyby chciał, mógł wykopać każdego w~dowolnym momencie (oczywiście było to również dozwolone w~warunkach: ,,Coca-Cola Games, Ltd zastrzega sobie prawo do wypowiedzenia swoje konto w~dowolnym momencie i~z dowolnego powodu''). 

-- Problem polega na tym, że zbyt wielu graczy uważa, że kupowanie złota jest w~porządku. W końcu sprzedajemy złoto na naszych własnych giełdach przez cały czas. Jeżeli rozwalilibyśmy każde konto zaangażowane w~nabycie złota z~farmy, zmniejszylibyśmy populację świata o~jakieś 80 procent. Nie stać nas na to.

-- 80 procent? Nie ma mowy \ldots  

-- Spójrz -- powiedział. -- Od miesięcy poluję na farmerów. Po raz pierwszy staramy się traktować ich systematycznie, zamiast po prostu poklepywać ich, gdy aktywność staje się zbyt intensywna. Mogę pokazać ci liczby, jeśli chcesz, pokaż ci, jak to wypracowałem, ale na razie powiedzmy, że jestem ekspertem w~tym temacie i~nie zmyślam tego.

Fairfax wyglądała na skarconego. 

-- Dobrze -- powiedziała. -- Więc chcesz iść za znanymi współpracownikami farmerów, których złapaliśmy, chociaż wszyscy widzimy, jak łatwo będzie to pokonać.

Connor wzruszył ramionami. 

-- OK, jasne. W końcu to obejdą. Ale będziemy mieli trochę czasu, żeby się do nich dostać.

Bill odchrząknął i~ponownie pokręcił głową. 

-- Masz pojęcie, ile danych transakcyjnych będziemy musieli przechowywać, aby rejestrować każdą osobę, z~którą każdy gracz kiedykolwiek rozmawiał lub walczył? A wtedy ktoś będzie musiał przejrzeć wszystkie te transakcje, jedna po drugiej, za każdym razem, gdy wysadzamy gracza, żeby mieć pewność, że trafimy na prawdziwych oszustów, a nie niewinnych przechodniów. Skąd ci wszyscy ludzie przyjdą?

Ktoś z~publiczności, to był Baird, prawnik, którego Connor nienawidził najmniej, powiedział: 

-- A co z~Mechanicznymi Turkami?

Connor i~Bill wpatrywali się w~siebie z~otwartymi ustami. Prawnik wyglądał na lekko zdenerwowanego.

 -- To znaczy \ldots 

-- \textit{Oczywiście }-- powiedział Connor. -- I moglibyśmy to zrobić za darmo. Po prostu pozwól Turkom zatrzymać jakiekolwiek złoto z~kont odpadłych graczy.

Jednym z~pozostałych ekonomistów był młody Palmer, który kilka lat temu przypomniał sobie Connora. Connor go nienawidził. Jego chętna ręka wystrzeliła w~górę. 

-- Myślałem, że chodzi o~to, aby całe to złoto nie wchodziło na rynek -- powiedział. -- Jak możemy kontrolować podaż pieniądza, jeśli pozwoli się tym goombom zalać rynek tanimi pieniędzmi?

Connor machnął rękami.

 -- Tak, teoretycznie te koty są poza naszym planem monetarnym, ale nawet jeśli są na wyczerpaniu, po prostu nie poruszają zbytnio rynkiem. A jeśli tak, możemy ograniczyć podaż po naszej stronie lub dostosować podstawowe elementy w~grze kosztami w~górę lub w~dół \ldots  I to nie jest tak, że Turcy od razu odwrócą się i~wydadzą złoto, albo wyrzucą je na jedną z~oficjalnych giełd, zwłaszcza jeśli utrzymamy niski kurs wymiany przez ten okres.

Młody Palmer ponownie otworzył usta i~Connor go powstrzymał. 

-- Słuchaj, to wszystko można modelować. Załóżmy, że możemy zająć się podażą pieniądza i~iść dalej. 

W głębi duszy wiedział, że odrzuca potencjalnie wybuchowy problem, niż było to naprawdę uzasadnione, ale faktem było to, że była to jego szansa, by raz na zawsze zająć się farmerami złota, z~całym ciężarem firmy za nim, a jeśli to trochę schrzani gospodarkę, cóż, naprawią to później. W końcu kontrolowali gospodarkę.

Później, przy swoim biurku w~Centrali Dowodzenia, podniósł wzrok znad swoich kanałów i~zobaczył pokój pełen najmądrzejszych, najtwardszych ludzi w~firmie -- na świecie -- zajmujących się tym samym zadaniem, wyszukujących pasożyty, które gonił od miesięcy. A jeśli on sam był kiedyś farmerem złota, spekulantem aktywów w~grze, to co z~tego? Przeszedł do czegoś lepszego.

Faktem było, że na świecie nie było miejsca, aby kilka milionów hodowców złota przekształciło się w~wysoko opłacanych dyrektorów gier wideo. Faktem było, że gdybyś musiał pokroić ciasto na tyle kawałków, by dać je wszystkim, w~końcu pokroiłbyś je tak cienko, że można by przez nie przejrzeć. ,,Kiedy 30 000 osób dzieli się jabłkiem, nikt nie odnosi korzyści, zwłaszcza jabłko''. Był to cytat, który jeden z~jego profesorów ekonomii trzymał zapisany w~rogu swojej tablicy, a za każdym razem, gdy student zaczynał trąbić o~współczuciu dla biednych, stary profesor po prostu stukał w~tablicę i~mówił: 

-- Czy chcesz podzielić się swoim lunchem z~30 000 osób? 

I do diabła, na świecie było co najmniej trzy miliony farmerów złota. Niech dostaną własne cholerne jabłka.

\bigskip
\threeast

,,Poziom morza'' to termin odnoszący się do średniego poziomu wszystkich oceanów na świecie. Pomyśl o~świecie jak o~gigantycznej niecce wypełnionej do połowy wodą. Możesz dmuchać na jedną część powierzchni i~wywoływać małe fale, których grzbiety są wyższe niż reszta wody. Możesz przechylać miskę z~boku na bok i~sprawiać, że woda chlupocze wokół, dzięki czemu z~jednej strony będzie wyższa niż z~drugiej. Ale ogólnie rzecz biorąc, ta woda ma jeden poziom, wysokość powierzchni, którą można łatwo rozpoznać.

To samo z~oceanami. Chociaż przypływy mogą przeciągnąć wodę z~jednego brzegu morza na drugi -- i~tak naprawdę jest tylko jedno morze, pojedynczy, ciągły zbiornik wodny w~kształcie puzzli, który otacza wszystkie kontynenty -- chociaż burze mogą tu i~ówdzie stworzyć fale, w~końcu w~oceanie jest tylko tyle wody, że mniej więcej dochodzi do łatwo uzgodnionej wysokości. Poziom morza.

To samo z~pieniędzmi. Na świecie jest tylko tyle wartości: tylko tyle rzeczy do kupienia. Gdybyś miał wszystkie pieniądze na świecie, mógłbyś je wymienić na wszystkie rzeczy na ziemi (przynajmniej wszystkie rzeczy, które są na sprzedaż). Tak naprawdę nie ma znaczenia, czy pieniądze są w~dolarach, sztukach złota, grzybach, ringgitach, euro czy jenach. Dodaj to wszystko razem, a otrzymasz ocean. Masz tylko poziom morza.

Co się stanie, jeśli ktoś po prostu wydrukuje dużo więcej pieniędzy? Co się stanie, jeśli podwoisz ilość pieniędzy w~obiegu? Czy morza monetarne podniosą się, zatapiając ziemię?

Nie.

Drukowanie większej ilości pieniędzy nie daje większych pieniędzy. Drukowanie większej ilości pieniędzy jest jak mierzenie oceanu w~litrach zamiast w~galonach. Zamiana 343 trylionów galonów oceanu na 1,6 trylionów litrów (mniej więcej) nie daje więcej wody. Galony i~litry to miary wody, a nie samej wody.

A dolary są miarą wartości, a nie samą wartością. Jeśli podwoisz ilość waluty w~obiegu, podwoisz cenę wszystkiego na Ziemi. Ilość rzeczy jest stała, ilość waluty nie. To się nazywa inflacja i~może być brutalna.

Powiedz, że jesteś dyktatorem marnej republiki. Przez dziesięciolecia nabijałeś sobie kieszenie na koszt ludzi, opodatkowując gówno z~wszystkich i~defraudując je na swoim tajnym zagranicznym koncie bankowym w~Hondurasie. W końcu wyprowadziłeś z~kraju tyle bogactwa, że ludzie są gotowi zjeść buty. Zaczynają się denerwować. Na Ciebie.

Normalnie kazałbyś po prostu swoim żołnierzom iść i~zrobić przykład z~kilkuset dysydentów i~zostawić ich makabryczne, pocięte szczątki przy drodze w~płytkich grobach, aby poinformować lojalnych poddanych o~tym, czego mogą się spodziewać, jeśli zachowają w~ten sposób.

Ale żołnierze -- nawet prawdziwi niedorozwinięci sadyści -- nie pracują za darmo. Chcą płacić. A jeśli wywiozłeś wszystkie pieniądze z~kraju i~włożyłeś je na swoje konto bankowe, potrzebujesz czegoś, czym możesz je zapłacić.

Nie ma problemu. Jesteś dyktatorem. Po prostu zadzwoń do departamentu skarbu i~każ im wydrukować kilka bilionów dukatów, certyfikatów złota, wahoonii, czy jak tam nazywasz swoje pieniądze, i~zaczynasz płacić żołnierzom. Działa -- przez chwilę. Żołnierze zabierają swoje pieniądze do miasta i~kupują za nie napoje, odlotowe ubrania i~obfite posiłki. Wysyłają je do domu do swoich rodzin, które używają go do kupowania drewna, dachówek, stali i~cementu do ulepszania swoich domów lub do kupowania narzędzi rolniczych i~opłacania pomocników, aby pomóc im przy następnych plonach.

Ale w~miarę jak ilość pieniędzy w~obiegu rośnie, stopniowo stają się one mniej warte. Bar podnosi ceny napojów, ponieważ właściciel podniósł czynsz. Właściciel podniósł czynsz, bo wzrosły koszty wyżywienia jego rodziny, bo rolnik nie chce sprzedawać swoich plonów po starych cenach, bo płaci dwa razy za olej napędowy do traktora i~trzy razy za wodę.

A potem żołnierze zjawiają się w~pałacu dyktatora i~dosadnie tłumaczą bagnetami (jeśli to konieczne), dlaczego ich stare zarobki już nie wystarczają.

Nie ma problemu. Po prostu zadzwoń do skarbca i~zamów kolejny bilion wahoonie. I patrz, jak to wszystko się powtórzy.

Nazywa się to inflacją i~jest to tani cukrowy haj rządów. Tak jak wkuwający student wsysający napoje energetyzujące, rząd może drukować pieniądze tylko przez tak długi czas, zanim będzie musiał zapłacić cenę. To też nie jest ładne. Rodziny, które starannie oszczędzały całe życie na emeryturę, nagle odkrywają, że ich schludne zaskórniaki nie wystarczają, aby pokryć cenę obiadu na mieście. Każdy grosz oszczędności zostaje zniszczony w~mgnieniu oka i~nagle potrzebujesz o~wiele więcej żołnierzy w~pracy, aby powstrzymać lojalnych poddanych przed wypatroszeniem cię jak rybą i~powieszeniem do góry nogami z~najwyższego komina własnego pałacu.

Jeśli jesteś \textit{bardzo }bezczelnym dyktatorem, pójdziesz jeszcze dalej: weź wszystkie oszczędności w~bankach, które są denominowane w~prawdziwych pieniądzach -- euro, dolarach lub jenach -- i~zamień je na wahoonies po dzisiejszym kursie wymiany. Wykorzystaj wszystkie te prawdziwe pieniądze, aby zapłacić armii za dzień lub dwa więcej, ale lepiej zaoszczędź wystarczająco dużo, aby zapłacić za przelot do jakiegoś bardzo, bardzo odległego miejsca.

Jeśli uważasz, że inflacja jest przerażająca, spróbuj \textit{deflacji}. W miarę jak ludzie stają się biedniejsi -- w~miarę, jak w~obiegu jest coraz mniej pieniędzy -- wartość pieniądza rośnie. To dobra wiadomość dla oszczędzających: wahoonie, które zapłaciłeś w~zeszłym roku, w~tym roku jest warte dwa razy więcej. Ale to zła wiadomość dla wszystkich innych: tylko idiota pożycza pieniądze w~czasach deflacji, ponieważ wahoonie, które pożyczysz dzisiaj, będzie warte dwa razy więcej w~przyszłym roku, kiedy je spłacisz. Deflacja też jest nierówna: koszt żywności może spaść z~powodu jakiegoś niesamowitego nowego nawozu, co oznacza, że możesz kupić dwa razy więcej manioku za wahoonie. Ale to oznacza, że rolnicy zarabiają tylko połowę mniej i~nie będą płacić tyle za telewizję kablową. Telewizja nie obniżyła jednak \textit{kosztów}, więc obniżona opłata oznacza mniejsze zyski. Firmy zaczynają upadać, co oznacza, że więcej ludzi ma mniej pieniędzy, co powoduje ciągły spadek cen. Niedługo nikt nie będzie mógł sobie pozwolić na zrobienie lub kupienie \textit{czegokolwiek}.

Innymi słowy, ilość pieniędzy w~obiegu to wielka sprawa. Teoretycznie ta kwota jest bacznie obserwowana przez sprytnych, poważnych ekonomistów. W praktyce wszystkie pieniądze świata są w~jednym wielkim wirującym basenie. Dolary, dukaty, wahoonie i~euro, zmieszane ze sobą, chcąc nie chcąc, a kiedy jeden rząd idzie do prasy i~zaczyna masowo wypuszczać bele banknotów, wszyscy mają wysoki poziom cukru. A kiedy wszystko się załamuje, a oszczędności ludzi idą z~dymem, deflacyjna spirala śmierci wkracza, ceny spadają, a coraz więcej firm upada, a rządy wracają do prasy drukarskiej.

Tak więc w~praktyce ten wielki silnik, który decyduje o~tym, ile uprawia się żywności, czy będziesz musiał sprzedać nerki, by wyżywić rodzinę, czy pobliska fabryka wyprodukuje Zeppeliny, czy restaurację na rogu stać na ziarna kawy, wszystkie te ważne rzeczy \textit{nie mają nikogo za nie odpowiedzialnego}. To pędzący pociąg, maszynista nie żyje na rozjeździe, pasażerowie kurczowo trzymają się kurczowo swojego dobytku, gdy ich dobytek wylatuje z~wagonów towarowych i~wylatuje przez okna, a każdy zakręt torów grozi zerwaniem go z~torów.

Z tyłu pociągu jest niewielka liczba osób, które zażarcie kłócą się o~to, kiedy zjedzie z~torów, czy maszynista naprawdę nie żyje i~czy pociąg może spowolnić każdy, kto tylko się uspokoi i~zachowywać jak gdyby wszystko było w~porządku. Ci ludzie są ekonomistami, a niektórzy pasażerowie pierwszej klasy bardzo dobrze płacą im za przewidywania, czy pociąg ma się dobrze i~na którą stronę wagonu powinni się pochylić, aby czapki nie spadły na następnym zakręcie.

Wszyscy inni ich ignorują.

\bigskip
\threeast

-- Hej, Connorze! -- powiedział jego makler napiętym i~nerwowym głosem, a jego radość była wyraźnie fałszywa.

-- Co jest nie tak? 

-- Przejść do sedna, co, człowieku? -- Głos Iry był tak napięty, że aż zabrzęczał. -- Jesteś takim celnym strzelcem. Dlatego jesteś moim ulubionym klientem. 

-- Co się \textit{dzieje}, Ira? -- Centrala Dowodzenia ryczała wokół niego, gwar krzyków, rozmów i~wulgaryzmów.

-- Więc pamiętasz te akcje, w~które cię zabraliśmy?

Klatka Connora zacisnęła się. Zmusił się do zachowania spokoju. 

-- Pamiętam je.

-- Cóż, opłacali się naprawdę dobrze, widziałeś oświadczenia. Osiem procent w~zeszłym miesiącu \ldots 

-- Widziałem oświadczenia. 

-- Dobrze. 

-- Ira -- powiedział Connor. -- Przestań być takim cholernym sprzedawcą i~powiedz mi, co się u diabła dzieje, albo odłożę słuchawkę i~zadzwonię do twojego szefa.

-- Connor -- powiedział Ira bolesnym głosem. -- Słuchaj, jesteśmy kumplami \ldots  

-- Nie jesteśmy kumplami. Jesteś sprzedawcą. Jestem twoim klientem. Rozłączam się. 

-- Czekaj! Chodź, czekaj! OK, oto co jest. Jest mały \ldots  kryzys płynności w~aktywach bazowych. 

Connor przetłumaczył mowę brokera na angielski.

-- Oni nie mają pieniędzy.

-- W \textit{tym miesiącu }nie mają żadnych pieniędzy -- powiedział. -- Słuchaj, kupon na ten kontrakt szedł w~górę ponad rok. Ostatecznie też nie może stracić, ponieważ zapakowaliśmy go w~swap ryzyka kredytowego. Ale teraz, w~tej chwili, mają trudny, jednorazowy ścisk.

Po spłaceniu odsetek za pierwszy miesiąc Connor zlikwidował kilka innych udziałów i~kupił więcej obligacji, kupując duże. Tak dużo, że dom maklerski przysłał mu kurierem butelkę szampana. Stracił rachubę, jak bardzo związał się z~,,w pełni zabezpieczonym'' planem Iry, ale wiedział, że było to co najmniej 150 000 dolarów. Wydawało się, że to dobry zakład \ldots 

-- Jaki rodzaj jednorazowego ściśnięcia?

-- Nintendo -- powiedział makler. -- Ostatnio poluzowali swoją politykę monetarną. Gwiezdni farmerzy w~Mushroom Kingdom zbierają ogromne plony, więc ceny monet Mario spadają. Ale mówi się, że to tylko tymczasowy gambit, ponieważ mieli tak ogromny napływ nowych graczy, których nie stać na dotrzymanie kroku starym graczom, więc próbują obniżyć ceny towarów, aby utrzymać tych graczy na pokładzie. Ale kiedy ci gracze nadrobią zaległości i~zaczną domagać się więcej ulepszeń, ceny wrócą.

Dla Connora brzmiało to wiarygodnie. W końcu robili podobne rzeczy we własnych grach. Doświadczeni gracze wyli, gdy inflacja obniżyła wartość ich oszczędności, ale gracz, który szlifował swoje umiejętności przez dwa lata, nie zamierzał zrezygnować z~czegoś takiego. Nowa krew była niezbędna do utrzymania gry na właściwym torze, zastępowania graczy, którzy się zestarzali, znudzili lub byli biedni, to jeden z~powodów, dla których niektórzy gracze rezygnowali co miesiąc.

Przerób był jednym z~jego największych problemów ekonomicznych. Możesz to zminimalizować na wiele podstępnych sposobów: napisz do byłego gracza, aby poinformować go, że zamierzasz usunąć toon, którego nie tknął od roku, a szansa na to, że zagra, jest jedna na trzy, zamiast skazywać ten zapomniany awatar do wiadra bitów. Ale ostatecznie niektórzy gracze odeszli, a jedyną rzeczą, jaką było to, było wprowadzenie nowych graczy.

Pośrednik nadal ględził.

--  \ldots  tak naprawdę, spodziewamy się ogromnego wzrostu w~ciągu czterech do ośmiu tygodni, więcej niż wystarczająco, aby zrekompensować spadek. A jeśli sprawy pójdą wystarczająco źle, zawsze jest książę i~jego zakłady \ldots 

-- Jaki jest ostateczny wynik? -- powiedział Connor.

-- Dolna granica -- powiedział Ira. -- Najważniejsze jest to, że w~tym miesiącu nie ma kuponu. Bazowe obligacje sprzedają się z~20-procentowym dyskontem od wartości nominalnej. -- Głośno przełknął ślinę. -- To sześćdziesiąt procent mniej niż zapłaciłeś za nie w~tym pakiecie. Ale jeśli sprawy przybiorą wystarczająco zły obrót, odzyskasz z~ubezpieczenia \ldots 

Connor próbował dalej słuchać, ale jego oddech nabierał ciasnych, małych sapnięć. Sześćdziesiąt procent! Właśnie sprawił, że ponad połowa jego majątku rozpłynęła się w~powietrzu. Najgorsze było to, że miał inne zobowiązania, kredyt hipoteczny, płatności należne od niektórych małych start-upów, w~które wkupił się, pieniądze na opłacenie wykonawców, którzy remontowali domek letniskowy, który kupił jako nieruchomość na wynajem na Bermudach. Bez gotówki, której oczekiwał po tych inwestycjach, mógłby wszystko stracić.

Nieświadomy tego makler mówił dalej. 

--  \ldots  dlatego naszą rekomendacją jest dziś kupowanie. Podwójnie. 

-- Przepraszam? -- powiedział Connor na tyle głośno, że najbliżsi mu ludzie w~Centrali Dowodzenia podnieśli wzrok znad swoich kanałów, by na niego spojrzeć. Krzywił się na nich, dopóki nie odwrócili wzroku. -- Powiedziałeś \textit{kupować}?

-- Nigdy nie było lepszego czasu -- powiedział sprzedawca. 

Connor wyobraził go sobie w~swoim boksie, krótkowłosego faceta w~średnim wieku w~starym garniturze, który kiedyś był szyty na miarę, zbiór złych nawyków przyklejonych do telefonu, z~obgryzionymi paznokciami i~drgającymi kolanami, obok kosza na śmieci wypełnionego na puste kubki po kawie, ekrany wokół niego migoczą jak stare, nieme filmy. 

-- Słuchaj, każdy idiota może kupować, gdy rynek jest w~górę, ale o~ile wyżej idzie rynek, gdy jest już na szczycie? Jedynym sposobem na zarobienie prawdziwych pieniędzy, dużych pieniędzy, jest obstawianie przeciwko stadu. Kiedy wszyscy porzucają swoje zasoby, to jest czas na zakup, kiedy wszystko jest w~piwnicy.

Connor wiedział, że to ma sens. To była podstawa jego równań Prikkela, była podstawą wszystkich fortun, które do tej pory zgromadził. Kupowanie rzeczy, których wszyscy chcieli, było bezpiecznym, nieciekawym zakładem, który praktycznie nic nie płacił. Kupowanie rzeczy, których wszyscy inni byli zbyt głupi, by chcieć -- w~ten sposób się \textit{wzbogacałeś}.

-- Ira -- powiedział Connor -- słyszę, co mówisz, ale widziałeś moje rachunki. Nie mogę sobie pozwolić na podwojenie. Jestem na maksie.

-- Connor, kolego -- powiedział, a Connor usłyszał uśmiech w~jego głosie i~sam się uśmiechnął, odruchem, którego nie mógł stłumić, nawet gdyby chciał. -- Nie jesteś skończony. Masz problem z~płynnością. Masz relację z~tym domem maklerskim. To jest coś warte. Do diabła, to jest warte \textit{wszystko}. Wciągnęliśmy cię w~ten problem i~wyciągniemy cię z~niego. Jeśli potrzebujesz kredytu, jest to absolutnie możliwe. Pozwól, że porozmawiam z~naszym działem kredytowym i~wrócę do Ciebie. Jestem pewien, że wszystko nam się uda.

Connora ogarnęło niesamowite, schizofreniczne wrażenie. Wyglądało to tak, jakby jego mózg rozpadł się na dwie części. Jeden kawałek energicznie potrząsał głową, mówiąc: \textit{O nie, zwariowałeś, nie ma mowy, żebym włożył w~to więcej pieniędzy. Nie, nie, nie, Chryste, nie! }

\textit{Ale była inna część jego umysłu, która mówiła \ldots }Ma rację, najlepszy czas na zakup jest przy minimum rynku. Te rzeczy bardzo się opłacały. Wyjaśnienie ma sens. Pomyśl tylko, jak się poczujesz, gdy nie kupisz wpisowego, a zabezpieczenia odbiją się, wszystkie te pieniądze, które stracisz. Pomyśl, jak się poczujesz, jeśli załatwisz to i~kupisz większy dom, inną nieruchomość dochodową, nowy samochód. Pomyśl o~tym, jak te wszystkie palanty ślinią się z~zazdrości, kiedy zabijasz. 

A jego usta się otworzyły i~wyszły z~nich słowa: 

-- Dobra, to brzmi dobrze. Wezmę tyle, ile możesz sprzedać na marginesie. 

Na marginesie: wtedy kupowałeś papiery wartościowe za pożyczone pieniądze, ponieważ byłeś pewien, że zakład się opłaci, zanim będziesz musiał spłacić pieniądze. To była niebezpieczna gra: jeśli wezwanie do uzupełnienia depozytu nadeszło, zanim zakłady się opłaciły -- lub jeśli nigdy się nie opłaciły -- może cię zlikwidować.

Ale tak naprawdę to nie były zakłady. Sposób, w~jaki dom maklerski je zapakował, były w~pełni zabezpieczone. Im gorsze były obligacje bazowe, tym bardziej opłacały się zakłady księcia przeciwko nim. Mogły pojawić się jakieś drobne comiesięczne zmiany, ale kiedy to wszystko zostało powiedziane i~zrobione, po prostu nie mógł przegrać.

-- Kupuj -- powiedział. -- Kupuj, kupuj, kupuj.

Przez resztę dnia był tak pochłonięty troską o~swoją niepewną pozycję, że nawet nie zauważył, kiedy wszyscy inni dyrektorzy w~Centrali Dowodzenia prowadzili niemal identyczną rozmowę ze \textit{swoimi} maklerami.

\bigskip
\threeast

Matka Wei-Donga była idealnym sprawdzianem rzeczywistości, jeśli chodzi o~gry i~Webblies. Nigdy nie doceniał tego przed wyjazdem z~domu, ale kiedy poszedł do pracy jako Turek, jego mama próbowała nawiązać kontakt, wycinając historie o~grach i~graczach i~wysyłając je do niego e-mailem. To zawsze było coś, co kilka miesięcy wcześniej wchłaniał przez pory, o~czym informowano obcokrajowców z~wielkimi krzyczącymi nagłówkami OMG WTF, które sprawiły, że parsknął śmiechem.

Ale zaczął doceniać wycinki swojej mamy jako spojrzenie na równoległy świat nie-graczy, ludzi, którzy po prostu nie zrozumieli, jak ważne stało się to wszystko. Najlepsze pochodziły z~prasy finansowej, próbując wyjaśnić dziwakom, którzy zainwestowali w~złoto do gier, dokładnie, co kupili.

A te wycinki były jeszcze ważniejsze, teraz kiedy przyjechał do Chin. Mama wciąż myślała, że jest na Alasce, więc od czasu do czasu wysyłał do niej e-maile z~odniesieniami do długich nocy i~krótkich dni, dzikiej przyrody, ludzi, wiele z~nich wycinał i~wklejał dosłownie z~tweetów rzeczywistych turystów z~Alaski.

Dziś, po trzech tygodniach strajku, wysłała mu to:

\noindent ZWIĄZEK ZAWODOWY DLA GRACZY WIDEO?

\textit{Nazywają siebie Industrial Workers of World Wide Web i~twierdzą, że jest ich dziś ponad 100 000, w~porównaniu z~20 000 zaledwie kilka tygodni temu. Spędzają dnie i~noce w~wieloosobowych grach wideo, trudząc się, by wydobyć bogactwo z~silników gier, naruszając wyłączny monopol producentów gier na wartość gier. Plony zbierane przez tych ,,farmerów złota'' są sprzedawane bogatym graczom w~Ameryce, Europie i~reszcie rozwiniętego świata, a firmy kontrolujące te gry twierdzą, że może to zakłócić starannie zbilansowaną gospodarkę wewnętrzną \ldots }

Wei-Dong przerwał artykuł, przesuwając się w~dół. Ciekawie było zobaczyć, jak jeden z~kanałów jego matki mówi o~Webbliesach, ale oni byli tak \ldots  \textit{starzy }na ten temat. Wyjaśniający wszystko.

Potem zatrzymał się i~przewinął w~górę.

\textit{ \ldots tajemnicza, wpływowa piracka prezenterka radiowa, która nazywa siebie Jiandi, której ma podobno dziesiątki milionów słuchaczy, tworząc rzadki i~nieprawdopodobny sojusz między tradycyjnymi pracownikami fabryki a graczami. Zjawisko to podobno powtarza się wokół Pacyfiku, w~Indonezji, Malezji, Kambodży i~Wietnamie, choć nie jest jasne, czy oddziały ,,IWWWW'' w~tych krajach są tylko naśladowcami, czy też są formalnie powiązane pod jednym dowództwem.}

Wei-Dong spojrzał znad ekranu na materac, na którym Lu i~Jie upadli po zatoczeniu się po ostatniej transmisji, twarz Jie była znacznie młodsza w~spoczynku. Czy naprawdę mogła być tą sławną DJ-ką, o~której mama -- \textit{mama}, tam daleko po drugiej stronie świata w~Los Angeles -- czytała?

Było ich więcej, ekranów i~ekranów więcej, ale to, co naprawdę przykuło jego uwagę, to wspomnienie o~,,zamieszaniu na rynku'', które powodowało, że ceny obligacji i~akcji skakały w~górę i~w dół. Nie rozumiał tego zbyt dobrze -- za każdym razem, gdy ktoś próbował mu to wyjaśnić, jego oczy szkliły się -- ale było jasne, że rzeczy, które tutaj robili, miały wpływ, \textit{ogromny }efekt, na całym świecie.

Prawie roześmiał się na głos, ale się powstrzymał. Matthew spał całe dwadzieścia centymetrów od miejsca, w~którym siedział, i~prowadził potyczki na pikiecie przez 22 godziny bez przerwy, zanim się przewrócił. Wei-Dong też walczył, ale jego głównym zadaniem było zwerbowanie większej liczby Turków do swojej małej listy przyjaznych agentów, co było znacznie mniej intensywną grą. Mimo to powinien spać, a nie dziobać w~laptopa. Za sześć godzin wróci na zmianę, mając tylko miskę kongee i~talerz klusek na rozpoczęcie dnia.

Opuścił pokrywę laptopa i~wyciągnął ręce nad głowę, zauważając przy tym obrzydliwy zapach pach. Pojedynczy prysznic -- otoczony przerażająco wyglądającym grzejnikiem elektrycznym, który podgrzewał wodę przechodzącą przez głowicę prysznicową -- nie wystarczał wszystkim Webbliesom, którzy spali w~mieszkaniu, a on nie kąpał się przez dwa dni z~rzędu. Nie był jedyny. W mieszkaniu pachniało jak w~szkolnej szatni albo schronisku dla bezdomnych w~pobliżu Santee Alley, które mijał, kiedy wychodził po zakupy.

Usłyszał ciche ćwierkanie gdzieś w~pobliżu, cichy świerszcz brzęczenia telefonu komórkowego. Patrzył, jak Jie sennie grzebała w~małej torebce przy poduszce, której pasek był już owinięty wokół jej ramienia, i~wyjęła telefon, mętnie odpowiadając na nią: 

-- Wei?

Jej zaspane oczy otworzyły się z~taką siłą, że rzeczywiście usłyszał, jak zmarszczyły się jej powieki. Jej przekrwione oczy ukazywały całą tęczówkę, a ona zerwała się, krzycząc po chińsku, tak szybko, że z~początku nie mógł jej zrozumieć. Ale potem złapał: 

-- Policja! Na zewnątrz! GO GO GO!

W kryjówce spało 58 Webblies i~w jednej chwili wszyscy wystrzelili z~koców, większość z~nich była już ubrana, wepchnęli stopy w~buty i~chwycili małe torby na ramię zawierające ich dane i~rzeczy osobiste i~stłoczyli się przy drzwiach. Pracowali niemal w~ciszy, jedynymi dźwiękami naglących szeptów i~przekleństw, gdy nadepnęli sobie nawzajem na buty. Niektórzy rzucili się do okna, wyskoczyli, by złapać balkon przeciwległego budynku z~uściskiem dłoni, a teraz z~ulicy dobiegły ich krzyki, gdy zauważyła ich nadjeżdżająca policja.

Dołączył do tłumu ciał, wpychając się ponuro w~wąski korytarz, a następnie biegnąc w~przeciwnym kierunku niż większość Webblies, ponieważ widział Jie biegnącą w~tym kierunku, trzymającą się mocno za rękę Lu, a Jie wydawała się mieć instynkt przetrwania miejskiego szczura. Jeśli ona prowadziła w~tę stronę, on też biegłby w~ten sposób.

Ale wyprzedziła go, a kiedy poślizgnął się za róg i~zobaczył, że patrzy na krótki odcinek korytarza kończącego się nieoznakowanymi drzwiami, ale ani ona, ani Lu nigdzie nie było. Zatrzymał się na chwilę, po czym nieomylny dźwięk wystrzału i~narastająca fala panicznych krzyków popchnęły go do przodu, pędząc do nieoznakowanych drzwi, z~ręką wyciągniętą, by przekręcić gałkę \ldots 

 \ldots  która była zamknięta!

Odbił się od drzwi, oszołomiony, poszedł na tyłek i~krzyknął pojedyncze, spanikowane ,,Cholera!'' gdy uderzył głową o~brudną podłogę wyłożoną kafelkami. Kiedy wrócił do pozycji siedzącej, zobaczył, że drzwi pękają. Przekrwione oko Jie spojrzało na niego i~zaklęła w~pomysłowym, slangowym chińskim.

 -- Gweilo -- syknęła -- szybko!

Szybko wstał i~w dwóch szybkich krokach dotarł do drzwi. Jej długie paznokcie wbiły się w~jego ramię, gdy zaciągnęła go do słabo oświetlonej przestrzeni, która teraz była rodzajem szafy z~zapasami, którą ktoś przerobił na sypialnię, ze zwiniętym łóżkiem w~jednym kącie i~wyczyszczonym rogiem półki środków czyszczących i~środków dezynfekujących oraz skromnym stosem ubrań, zbiorem przyborów toaletowych i~małym lusterkiem.

-- Opiekunka -- powiedziała Jie, szepcząc tak cicho, że Wei-Dong ledwo ją słyszał. -- Może tu mieszkać za darmo. Ona i~ja mamy układ. 

Lu klęczał za nią na rękach i~kolanach, cicho przesuwając się w zatłoczonej przestrzeni, pracując z~małą latarką LED zaciśniętą między zębami. Oddychał ciężko, jego chude ramiona drżały, gdy podnosił gigantyczne butelki z~wybielaczem i~usiłował je postawić bez wydawania dźwięku.

-- Czy mogę pomóc? -- wyszeptał Wei-Dong.

Jie przewróciła oczami. 

-- Czy wygląda na to, że jest miejsce na pomoc? -- powiedziała. 

Była tak blisko niego, że widział jej pojedyncze rzęsy, puszyste włosy na płatkach uszu. Gdyby wziął głęboki oddech, prawdopodobnie by ją zmiażdżył.

Delikatnie potrząsnął głową. 

-- Przepraszam. 

Lu chrząknął z~satysfakcją i~odczepił całą dolną półkę od wspornika. Wei-Dong zauważył, że odkrył właz osadzony w~ścianie, który obsypywał podłogę kurzem i~odpryskami farby w~kształcie skrzydeł karalucha, gdy je luzował. Podał go do tyłu i~Jie próbowała go złapać, ale nie było miejsca na manewrowanie nim na małej przestrzeni.

Za drzwiami słyszał tupot, tupot, tupot ciężkich butów, łomotanie i~łomotanie do drzwi, stłumione i~przestraszone rozmowy ludzi zrywanych z~łóżek w~środku nocy.

Z~niskim, sfrustrowanym, przestraszonym dźwiękiem Jie chwyciła pokrywę włazu i~odsunęła ją, uderzając go tak mocno w~nos, że musiał wepchnąć pięść do ust, żeby przestać krzyczeć. Rzuciła mu pogardliwe spojrzenie i~wepchnęła mu właz w~ręce. Miał około czterystu centymetrów kwadratowych, brudny, niezręczny, wykonany ze zmiękczonej przez wieki sklejki.

Lu przeszedł już przez właz, a teraz Jie szła za nim, jej gołe nogi migały w~półmroku pokoju, a potem Wei-Dong został sam i~tupot butów był głośniejszy. W korytarzu ktoś szamotał się, mężczyzna krzyczał z~oburzenia; kobieta krzycząca z~przerażenia; wyjące dziecko.

Wei-Dong uklęknął i~zajrzał w~maleńki otwór. Było tam ciemno. Ostrożnie oparł kołdrę o~ścianę obok otworu, a następnie wszedł do środka. Podłoga po drugiej stronie była niedokończonym betonem, zapiaszczonym i~zakurzonym. Nic nie widział, gdy podciągnął się na łokciach w~stylu komandosa, z~oddechem chrapliwym w~uszach. Posunął się do przodu, ostrożnie wyczuwając przeszkody, po czym odkrył, że trzyma coś miękkiego, giętkiego i~ciepłego. 

Pierś Jie.

Syknęła jak wąż i~gwałtownie odepchnęła jego rękę. Zaczął jąkać się z~przeprosinami, ale ona znowu syknęła: 

-- Cśś!

Zdusił słowa.

-- Zamknij kratę -- powiedziała. 

Ostrożnie zaczął się odwracać. Niewielka przestrzeń miała zaledwie metr wysokości i~wielokrotnie uderzał głową w~sufit, wzdłuż którego biegło kilka bezlitosnych metalowych rur, najeżonych złośliwymi złączami i~trójnikami. I kilka razy kopnął Jie i~Lu.

Ale w~końcu znalazł się z~głową i~rękami przy włazie i~desperacko wsunął palce do wnętrza grilla i~wsunął go na miejsce. Prawie niemożliwe było manewrowanie nim w~ciasnej przestrzeni, ale udało mu się, jego palce były białe, a przez cały czas dźwięki z~korytarza stawały się coraz głośniejsze i~głośniejsze.

-- Zrobione -- sapnął i~odsunął się. 

Zza drzwi dobiegały teraz głosy, głębokie, niecierpliwe męskie głosy i~wściekły, piskliwy kobiecy głos, mówiący im, że to głupia szafa na miotły i~żeby przestali być tak głupi. Ktoś potrząsnął klamką, a potem wbił ramię w~drzwi, które zadrżały.

Wei-Dong ugryzł się w~język, żeby powstrzymać pisk i~odepchnął się jeszcze bardziej, strach w nim znana, żywa istota w~jego klatce piersiowej. Jie i~Lu odepchnęli go, gdy się z~nimi zderzył, ale ledwo to poczuł. Jedyne, co czuł, to strach, strach przed uzbrojonymi mężczyznami po drugiej stronie drzwi, którzy mieli wejść i~zobaczyć szafę i~wyraźną lukę na dolnej półce, gdzie rzeczy zostały odsunięte na bok. Wei-Dong nagle i~boleśnie uświadomił sobie, jak daleko jest od domu, nielegalny imigrant bez praw w~kraju, w~którym nikt inny też nie miał praw. Płakałby, gdyby nie bał się wydać dźwięku.

-- Chodź -- szepnęła Jie, ledwo słyszalnym dźwiękiem, gdy drzwiami wstrząsnął kolejny trzask.

 Ktoś włożył teraz klucz w~zamek, potrząsając nim. Włączyła maleńką czerwoną diodę LED, która pokazała mu kształt przestrzeni: długi, niski obszar konserwacji kanalizacji. Rury nad nimi bulgotały i~świszczały cicho, gdy przepływała przez nie woda.

Lu był obok niego, Jie przed nimi, a ona czołgała się po przeciwnej stronie obszaru. Szedł za nim tak szybko, jak mógł, wypatrując w~uszach jakiegokolwiek dźwięku dobiegającego zza jego pleców.

Jie zaklęła pod nosem.

-- Co? -- spytał Lu.

-- Nie mogę znaleźć drugiej kraty -- powiedziała. -- Myślałam, że to właśnie tutaj, ale \ldots 

Wei-Dong teraz zrozumiał. Strefa konserwacji zajmowała martwą przestrzeń między ich budynkiem a budynkiem za nim, a gdzieś w~tym miejscu była krata podobna do tej, przez którą przeszli, mały tunel czasoprzestrzenny do innego poziomu gry. Instynkt przetrwania Jie był niewiarygodnie ostry, to było oczywiste, więc nie był całkowicie zaskoczony odkryciem, że przygotowała tylne drzwi.

Wpatrywał się w~ciemność, całe jego ciało śliskie od potu i~brudne od pradawnego kurzu pokrywającego podłogę.

-- Ostatnim razem było światło po drugiej stronie. Łatwo było ją znaleźć -- powiedziała głosem bliskim paniki. Usłyszał nieomylny dźwięk policji wchodzącej do schowka za nimi, a potem głosy.

-- Musimy przeszukać całą ścianę -- powiedział Lu. -- Rozdzielić się. 

Więc Wei-Dong stwierdził, że wije się nad nagimi łydkami Jie, rozrywając przy tym dżinsy na jednej z~niskich rurek. Na ślepo poklepał ścianę, obmacując. Z dala od małego czerwonego światła było ciemno, dezorientująco, przerażająco. W pobliżu usłyszał odgłosy poszukiwań Jie i~Lu.

A potem go znalazł, jego mały palec wsunął się w~kratkę, po czym poklepał ją, poczuł jej pełny zasięg. 

-- Tutaj tutaj! -- wyszeptał głośno, a pozostali zaczęli przedzierać się do niego. 

Potrząsnął kratą, próbując znaleźć sztuczkę, która sprawiłaby, że zniknie, ale wydawało się, że została wkręcona. Coraz bardziej zdesperowany, potrząsnął kratą, powodując deszcz kurzu i~zaschniętej farby opadający mu na ręce. Trzymał metal tak mocno, że czuł, jak przecina jeden palec, strużka krwi zamienia się w~błoto, gdy miesza się z~brudem.

-- Światło -- powiedział. -- Nic nie widzę. 

Dłoń poklepała go po nodze, przesuwając się w~górę ciała, do ramienia, a potem wcisnęła mu w~dłoń małą lampkę. Dłoń Jie, szczupła i~dziewczęca. Włączył czerwone światło i~wpatrywał się uważnie w~kratę. Nie była wkręcona, ale musiała zostać lekko wypchnięta do przodu, zanim się wyjmie. Wsunął rączkę latarki między zęby, \textit{pchnął }i \textit{uniósł}, a krata odskoczyła.

Gdy tak się stało, długi stożek światła przeciął przestrzeń, a potem wojskowy głos zażądał ,,Stop!''. Światło oblało go, sprawiając, że zmrużył oczy, a Jie uderzyła go w~udo i~powiedziała: 

-- Idź!

Poszedł, znów czołgając się jak komandos, szczupłe dłonie Jie popychały go, by się pośpieszył. Wyszedł na wyłożoną kafelkami przestrzeń, brudną i~ciemną, z~mokrą i~śliską podłogą. Wstał ostrożnie, martwiąc się, że znowu uderzy się w~głowę, po czym pochylił się, by pomóc Jie przejść. Z drugiej strony kraty dobiegały teraz kolejne krzyki, a światło wylało się z~niej i~malowało zielonkawe szumowiny na starej, popękanej posadzce z~szarych płytek. ,,Stój!'' ponownie i~,,Stop'' jeszcze raz, gdy Jie skończyła się przeciskać i~pochyliła się, by chwycić Lu, zaglądając do jasno oświetlonej przestrzeni. Lu szukał kraty na drugim końcu pełzającej przestrzeni i~pełzł tak szybko, jak tylko mógł, z~twarzą maskującą determinację i~strach, wargami odsłaniającymi zęby, krwią spływającą swobodnie z~rany głowy.

,,Stop!'' ponownie, a Lu przyśpieszył i~rozległ się charakterystyczny dźwięk odbezpieczanego pistoletu. Oczy Lu rozszerzyły się, wyciągnął przed siebie ręce, wbił ręce w~ziemię i~podciągnął się, szarpiąc palcami u nóg.

-- Chodź -- błagał Wei-Dong, praktycznie ze łzami w~oczach. -- Chodź, Lu! 

Strzał z~broni palnej, ten płaski dźwięk, który słyszał w~oddali, kiedy mieszkał w~centrum Los Angeles, ale z~niepokojącym zestawem jęków, gdy pocisk odbijał się od jednej rury do drugiej. Woda zaczęła wylewać się na podłogę, a Lu wciąż był za daleko. Wei-Dong upadł na brzuch i~wczołgał się do połowy przestrzeni, wyciągając ramiona: ,,Chodź, chodź'' nucąc to teraz, nie wiedząc, czy mówi po angielsku, czy po chińsku.

I Lu był już blisko i: ,,STÓJ!'' i~kolejny wystrzał, potem jeszcze dwa, a woda była wszędzie, a rykoszety jęczące były wszędzie, a potem \ldots 

Lu \textit{wrzasnął}, dźwięk, jakiego Wei-Dong nigdy nie słyszał. Najbliższy, jaki kiedykolwiek usłyszał, to było zawodzenie kota, którego kiedyś widział potrąconego przez samochód przed jego domem, kota, który leżał na ulicy ze złamanym kręgosłupem przez wieczność, krzycząc prawie jak człowiek, wycie, od którego swędziała mu skóra od kostek do płatków uszu. Wtedy Lu się \textit{zatrzymał}. Położył się nieruchomo. Wei-Dong ugryzł się w~język tak mocno, że krew wypełniła mu usta. Oczy Lu zwęziły się, źrenice się zawęziły. Otworzył usta, jakby właśnie miał powiedzieć najważniejszą rzecz w~swoim życiu, a potem krew pociekła z~jego ust, po wargach i~po brodzie. 

-- Lu! -- zawołał Wei-Dong i~był rozdarty między chęcią pójścia naprzód i~dorwania go a chęcią wycofania się i~ucieczką tak szybko, jak tylko mógł, aż do Kalifornii, gdyby mógł \ldots 

A potem ,,ZOSTAŃ TAM, GDZIE JESTEŚ'' tym szczekającym, brutalnym chińskim i~broń została ponownie odbezpieczona. Poczuł krew z~własnych ust i~z Lu, a Lu osunął się do przodu. Potem zapach prochu. Następnie  \ldots 

 \ldots  kolejny strzał, który jęczał i~odbijał się ze śmiertelnym dźwiękiem, od którego dzwoniło mu w~uszach.

-- ZOSTAŃ TAM, GDZIE JESTEŚ -- powiedział głos, a Wei-Dong cofnął się tak szybko, jak tylko mógł.

Jie szarpnęła go na nogi, jej twarz była ubrudzona kurzem i~pokryta łzami.

-- Lu? -- spytała.

Potrząsnął głową, cały jego chiński zniknął na chwilę, żadne słowa nie były dla niego dostępne.

Wtedy Jie zrobiła niezwykłą rzecz. Zamknęła oczy, wzięła głęboki oddech, wciągnęła go i~wciągała, ścisnęła pięści, ramiona i~mięśnie szyi, tak że wszystkie się wyróżniały, napięte i~spięte.

A potem wypuściła wszystko, rozluźniła pięści, rozluźniła szyję i~otworzyła oczy.

-- Chodźmy -- powiedziała i~jednym płynnym ruchem odwróciła się do drzwi za sobą i~zatrzasnęła rygiel, przekręciła gałkę i~otworzyła ją na kolejny korytarz w~kamienicy, pachnący przyprawami do gotowania i~starodawnymi, zmielonymi zapachami ciała i~pleśń. Przyćmione światło z~korytarza wydawało się jasne w~porównaniu do zmierzchu, w~którym przebywał od czasu, gdy zanurkował przez otwór, i~zobaczył, że znajduje się pod nieużywanym wspólnym prysznicem, a ściany zielone od starej pleśni i~szlamu.

Jie wyjęła z~torebki parę sandałów z~paskami i~spokojnie i~sprawnie je włożyła. Wyjęła dwie zapieczętowane paczki wilgotnych chusteczek, jedną wręczyła Wei-Dongowi i~zawartością drugiej wytarła twarz, ręce, gołe nogi, pracując energicznymi pociągnięciami. Chociaż serce Wei-Dong'a waliło jak młotem, a adrenalina wzbierała w~jego ciele, zmusił się do zrobienia tego samego, wpychając brudne chusteczki do kieszeni, aż nie było ich więcej. Z krat za nimi dobiegały kolejne krzyki i~odległe dźwięki z~ulicy poniżej, a Wei-Dong wiedział, że to beznadziejne, wiedział, że są otoczeni.

Ale jeśli Jie miał zamiar maszerować, on też to zrobi. Lu był za nim, z~zapachem miedzianej krwi, zapachem ogniska i~prochu strzelniczego. Przed nim były Chiny, całe Chiny, kraj, o~którym marzył od lat, już nie sen, ale brutalna rzeczywistość.

Jie zaczęła żwawo iść, jej ręka machała tam i~z powrotem jak metronom, gdy przeszła przez budynek i~otworzyła drzwi na schody, nie zwalniając kroku. Wei-Dong starał się nadążyć. Zeszli po trzech kondygnacjach schodów, przez brudne, zakratowane okna przepuszczające tylko szare światło. Na zewnątrz był świt.

Pozostał tylko jeden zbieg schodów i~Jie gwałtownie się zatrzymała, obróciła się na pięcie i~spojrzała mu w~oczy. Jej oczy były pokryte czerwienią, ale jej twarz była opanowana. 

-- Dlaczego musisz być biały? -- powiedziała. -- Tak bardzo się wyróżniasz. Idź pięć kroków za mną, trzy kroki w~bok, a jeśli cię złapią, nie zatrzymam się.

Przełknął. Próbował przełknąć. Jego usta były zbyt suche. Lu nie żył na górze. Policja była za drzwiami -- słyszał wołania, rozmowy radiowe, silniki, syreny, krzyki -- i~byli morderczy.

Chciał powiedzieć: \textit{Czekaj, nie otwieraj drzwi, schowajmy się tutaj}. Ale tego nie powiedział. Byli tu skazani. Policja wiedziała, do którego budynku weszli. Im dłużej czekali, tym szybciej będzie, zanim zamkną wyjścia i~przeszukają każdy zakątek.

-- Zrozumiałem -- zdołał i~zmienił twarz w~gładką maskę.

Jeszcze jedne schody.

Jie uchyliła drzwi i~na jej twarzy pojawiło się różowe światło świtu. Na chwilę przyłożyła oko do szczeliny, po czym otworzyła je trochę szerzej i~się wymknęła. Wei-Dong powoli policzył do trzech, oddychając tak wolno, gdy liczył, po czym sam wyszedł.

Chaos. 

Ulica była nieco szersza niż większość pasów w~pobliżu budynków uścisku dłoni, główna droga była akurat na tyle duża, że mógł wjechać samochód. Na jednym końcu stał samochód, na zewnątrz dwóch policjantów. Trzech kolejnych policjantów właśnie wchodziło do budynku, z~którego wyszedł, używając szklanych drzwi kilka metrów dalej. Niebieskie światła policyjnego radiowozu malowały otaczające ich ściany powtarzającymi się wzorami błękitu i~czerni. Gdzieś w~pobliżu, krzyki. Dużo krzyku. Chłopięce wrzaski przerażenia i~agonii, łomot pałek, krzyki z~balkonów, żadnych słów, tylko bezsłowna ścieżka dźwiękowa z~rzeźni dziesiątek pobitych Webblies. Pobici, podczas gdy Lu leżał martwy lub umierający w~tunelu.

Skręcił w~lewo, w~kierunku, w~którym poszła Jie, w~samą porę, by zobaczyć, jak znika w~wąskiej uliczce, skręcając bokiem, by w~nią wejść. Nie był pewien, jak mógł zastosować się do jej nakazu, by trzymać się z~boku w~tak wąskiej przestrzeni, ale uznał, że to go nie obchodzi. Nie zamierzał próbować wydostać się na własną rękę z~labiryntu miasta kantońskiego.

Jednak gdy tylko wszedł do zaułka, pożałował tego. Policjant, który przypadkiem spojrzał w~alejkę, natychmiast go zobaczył i~stałby się siedzącym celem, nie do przeoczenia. Spojrzał przez ramię tak bardzo, że potknął się i~prawie się przewrócił, tylko powstrzymując się przed upadkiem na mokry, śmierdzący beton między budynkami, wbijając ręce w~ściany po obu stronach. Przed nim Jie przeszła na drugi koniec alejki i~skręciła w~prawo. Pospieszył ją złapać.

W chwili, gdy sam wyszedł z~alejki, usłyszał jeszcze trzy strzały, a potem serię strzałów, tak wielu, że nie mógł ich zliczyć. Zamarł, ale odgłosy były dalej, tam, gdzie Webblies wyszli z~bezpiecznego domu. To mogło znaczyć tylko jedno. Ugryzł się w~policzek i~przełknął mdłości rosnące w~gardle i~starał się nadążyć za Jie.

Jie szła szybko -- zbyt szybko; prawie ją stracił więcej niż raz. Ale w~końcu skręciła w~stację metra, a on poszedł za nią. Już wcześniej korzystał z~automatów do kupowania biletów -- były oznakowane po chińsku i~angielsku -- i~kupił bilet, który zabrał go na koniec kolejki, wkładając kilka banknotów RMB ze swojego portfela. Maszyna wyrzuciła plastikową monetę jak żeton do pokera do swojego zasobnika, a on wziął ją, potarł o~punkt styku kołowrotu i~z brzękiem zbiegł po schodach z~nielicznym tłumem pracowników zmierzających na wczesne zmiany.

Ustawił się przy jednych z~drzwi i~sięgnął do kieszeni po zniszczony przewodnik turystyczny po Shenzhen, wyjęty z~wolnego stosu w~budce informacyjnej na stacji kolejowej. To był doskonały kamuflaż, rodzaj niewidzialności. W metrze zawsze jeden czy dwa gweilo zastanawiały się nad mapą turystyczną, starannie ignorowani przez stada doskonale wystrojonych dziewczyn z~fabryki, które unikały ich jako prawdopodobnych zboczeńców i~wyraźnych źródeł zakłopotania.

Jie wysiadła cztery przystanki później i~zeskoczył w~ostatniej chwili. Kiedy to robił, dostrzegł swoje odbicie w~szybie drzwi samochodu i~zobaczył, że jedna strona jego włosów była pokryta zaschniętą krwią, która również spływała mu po szyi i~tam zaschła. Przeklął siebie za swoje samozadowolenie. Niewidzialny! Był prawdopodobnie najbardziej pamiętną rzeczą, jaką pasażerowie metra widzieli przez cały dzień, brudny, zakrwawiony gweilo w~pociągu.

Poszedł za Jie ruchomymi schodami i~zobaczył, jak wymownie kiwa głową w~stronę drzwi toalety. Poszedł i~poruszył klamką, ale była zablokowana. Odwrócił się do wyjścia i~drzwi się otworzyły. Za nim stała stara babcia ze strasznym garbem, który zginał ją prawie podwójnie.

Posłała mu mleczne spojrzenie, zacisnęła usta i~zaczęła zamykać drzwi.

-- Czekaj! -- powiedział natarczywym, niskim chińskim.

-- Mówisz po chińsku? 

Pokiwał głową. 

-- Trochę -- powiedział. -- Muszę skorzystać z~łazienki. 

-- Dziesięć RMB -- powiedziała. 

Był prawie pewien, że nie była oficjalną opiekunką do łazienki, ale nie zamierzał się z~nią kłócić. Sięgnął do kieszeni, znalazł dwie zmięte piątki i~podał jej. To było 1,25 dolara i~był prawie pewien, że to szalona suma pieniędzy za korzystanie z~łazienki, ale w~tym momencie nie obchodziło go to.

Łazienka była malutka i~zatłoczona rzeczami starej kobiety w~ogromnych winylowych torbach na zakupy. Ustawił się przy zlewie i~wpatrywał się w~swoje odbicie w~porysowanym lustrze. Wyglądał, jakby przeszedł przez blender, głową do przodu. Spuścił wodę i~za pomocą złożonych dłoni spryskał ją bezskutecznie na włosy i~szyję, mocząc przy tym jego koszulkę.

-- To nie sposób na to -- krzyknęła stara kobieta zza jego pleców. 

Zakręciła kran artretyczną ręką. Spojrzał na nią w~milczeniu. Nie chciał wdawać się w~kłótnię z~tą dziwną starą staruchą.

-- Zdjąć koszulę -- powiedziała surowym głosem. 

Kiedy się zawahał, niecierpliwie klepnęła go w~nadgarstek. 

-- Już! -- powiedziała. -- Zdejmuj koszulę, pochyl się do przodu, włosy pod kranem. Uczciwie!

Zrobił tak, jak mu kazano, pochylając się głęboko w~pasie, by schować włosy pod kranem małej, brudnej umywalki. Odkręciła kran do oporu i~drżącymi rękami umyła mu włosy i~wyszorowała zakrwawioną szyję. Kiedy wstał, klepnęła go po plecach i~powiedziała: 

-- Czekaj!

On czekał. W końcu pozwoliła mu wstać i~przekopywała się przez swoje torby, aż znalazła postrzępioną starą męską koszulę, którą mu podała. 

-- Suche -- powiedziała.

Koszula pachniała pleśnią i~miastem, ale była czystsza niż wszystko, co miał na sobie. Wytarł włosy ręcznikiem, uważając na delikatne skaleczenie na głowie.

-- Nie jest głębokie -- powiedziała. -- Byłam pielęgniarką, nic ci nie będzie. Jeden szew lub dwa, jeśli nie chcesz blizny.

-- Dziękuję -- udało się powiedzieć Wei-Dong. -- Dziękuje bardzo. 

-- Dziesięć RMB -- powiedziała i~uśmiechnęła się do niego, praktycznie bezzębna. Dał jej jeszcze dwie piątki i~włożył koszulkę. Śmierdziała okropnie, gęstym smrodem ciała i~krwi, ale była to czarna koszulka ze zdjęciem szarżującego orka i~nie było na niej krwi.

-- Idź -- powiedziała. -- Nigdy więcej walki. 

Wyszedł oszołomiony i~trafił na stację, szukając Jie. Czekała przy ruchomych schodach na powierzchnię, poprawiając makijaż w~małym lusterku, które akurat pokazywało jej widok na drzwi łazienki. Zatrzasnęła puderniczkę i~wynurzyła się na powierzchnię. Poszedł za nią.

\bigskip
\threeast

-- Czterdzieści dwóch martwych -- powiedziała do Justboba i~Mighty Krang Big Sister Nor. -- Czterdziestu dwóch zabitych w~Shenzhen. Krwawa łaźnia.

-- Wojna -- powiedział Justbob.

-- Wojna -- powiedział Mighty Krang z~zaciekłością, której żaden z~nich nigdy wcześniej od niego nie słyszał. Zobaczył ich spojrzenia, zacisnął pięści i~spojrzał na nich gniewnie. -- Wojna -- powtórzył.

-- Nie wojna -- powiedziała Big Sister Nor. -- Strajk.

\bigskip
\threeast

-- Strajk -- ogłosił generał Robotwallah swoim żołnierzom. -- Żadne złoto nie wchodzi ani nie wychodzi z~żadnej z~naszych gier.

-- Czterdziestu dwóch zabitych -- powiedziała Yasmin głosem ołowianym ze smutku.

\textit{Czterdziestu trzech}, pomyślał Ashok, przypominając sobie chłopca i~z pewnością Yasmin myślała o~\textit{czterdziestu trzech}, gdy siadała.

-- Potrzebujemy tu obrony -- powiedział generał Robotwallah. -- Bannerjee znajdzie więcej badamaash, aby spróbować nas wyrzucić z~tego miejsca.

Sushant wstał i~podniósł maczetę, którą chłopcy zostawili. 

-- Zajęliśmy to miejsce. Utrzymamy je -- powiedział z~brawurą nastolatków. Ashok czuł, że będzie chory.

Yasmin i~generał patrzyli na siebie intensywnie, odbywała się cicha rozmowa.

-- Nigdy więcej przemocy -- powiedziała Generał rozkazującym głosem.

Sushant spuścił powietrze, wyglądał na upokorzonego.

-- A jeśli przyjdą po nas z~nożami, kijami i~pistoletami? -- powiedział wyzywająco. 

Yasmin wstała i~podeszła do swojego generała. 

-- Upewnimy się, że nie -- powiedziała.

Ashok wstał, poszedł do swojego małego pokoju na zapleczu i~zaczął dzwonić.

\bigskip
\threeast

-- Siostry! -- powiedziała Jie, odchylając głowę do tyłu i~zaciskając pięści.

Była wystarczająco spokojna, gdy usiadła w~piwnicy kafejki internetowej, prywatnego pokoju, który właściciel dyskretnie wynajmował freelancerom porno, którzy potrzebowali połączenia sieciowego z~dala od opinii publicznej. Ale teraz wyglądało na to, że cały smutek i~ból, które wepchnęła w~siebie, kiedy Lu został postrzelony, wylewał się z~siebie.

-- \textit{SIOSTRY}! -- powtórzyła i~było to wycie, równie straszne jak hałas, który wydał Lu, tak straszny jak hałas, który na wpół martwy kot wydał na ulicy przed domem Wei-Dong.

Kawiarnia mieściła się w~zamkniętym na klucz hotelu Intercontinental, w~tematycznej restauracji, z~której z~dachu wystawał pełnowymiarowy statek piracki z~żaglami w~strzępach. Mężczyzna za biurkiem żwawo negocjował z~Jie, starannie ignorując Wei-Dong czającego się kilka kroków za nią. Skinęła na niego ruchem głowy i~zaprowadziła go do prywatnego pokoju, który kiedyś był magazynem restauracji.

Gdy drzwi zamknęły się za nimi, wyjęła bootowalną pamięć USB i~ponownie uruchomiła z~niej komputer, włożyła do ucha elegancką, smukłą słuchawkę i~podała jedną Wei-Dongowi, którą wkręcił sobie we własne ucho. Po kilku zabawach z~komputerem dała mu znak, że żyją i~zaczęła wyć jak ranna istota.

-- \textit{Siostry! Moje siostry! }-- powiedziała, a łzy popłynęły jej po twarzy. -- Zabili go dziś wieczorem. Biedny Tank, mój Tank. Jego imię, jego prawdziwe imię to Zha Yue Lu, i~kochałam go, a on nigdy nie skrzywdził drugiego człowieka, a jedyne, czego był winien, to żądał przyzwoitej płacy, przyzwoitych warunków pracy, wakacji, bezpieczeństwa pracy  \ldots  to, czego wszyscy chcemy od naszej pracy. Rzeczy, które nasi \textit{szefowie }biorą za pewnik.

-- Wczoraj w~nocy napadli na nas, okrutna jingcha, pracująca dla szefów, jak zawsze było i~zawsze będzie. Wywalili drzwi, a chłopcy biegli jak wiatr, ale złapali ich, złapali i~złapali. Lu a ja próbowałem uciec tylnym wyjściem, a oni \ldots  -- Urwała, a łzy spływały jej po twarzy, szloch większy niż sam pokój uciekł z~jej piersi. Odczyty miksera na ekranie komputera poczerwieniały od wybuchu dźwięku. -- Zastrzelili go jak psa, zastrzelili go.

Znowu szlochała, a szloch nie ustawał. Uderzyła pięściami w~stół, szarpała się za włosy, krzyczała, jakby była cięta nożami, krzyczała, aż Wei-Dong była pewna, że ktoś wyważy drzwi, spodziewając się, że znajdzie morderstwo w~toku.

Niepewnie rozprostował nogi, wstał, podszedł do niej i~złapał uderzające pięści w~swoje dłonie. Spojrzała na niego niewidzącym wzrokiem i~wsunęła twarz w~jego klatkę piersiową, gorące łzy przesiąkły przez jego koszulkę, krzyki napływały i~napływały. Odsunęła się na chwilę, sapnęła: 

-- Przepraszam, wrócę za kilka minut -- i~coś kliknęła, a poziomy miksera na ekranie się wyrównały.

Płakała bez przerwy, a wkrótce Wei-Dong też płakał, płakał za swoim ojcem, płakał za Lu, płakał za wszystkimi strzałami, które słyszał, gdy wychodził z~budynków, w~których uścisk dłoni. Kołysali się i~płakali razem w~ten sposób przez, jak się wydawało, wieczność, a potem Jie delikatnie oderwała się i~wróciła do swojego komputera i~kliknęła jeszcze trochę.

-- Siostry -- powiedziała, -- od lat siedzę przy tym mikrofonie, rozmawiając z~wami o~miłości, rodzinie, marzeniach i~pracy. Tak wiele z~nas przyjechało tutaj, żeby uciec od biedy, szukając przyzwoitych zarobków za przyzwoity dzień pracy, a zamiast tego byłyśmy bite przez zboczonych szefów, obrabowywano nas przez programy marketingowe, traciłyśmy zarobki i~byłyśmy wyrzucane na ulicę, gdy rynek się zmienia.

-- Nigdy więcej, -- powiedziała, oddychając tak cicho, że Wei-Dong musiał się wysilić, żeby to usłyszeć. -- Nie! -- powiedziała głośniej. -- JUŻ NIE! -- krzyknęła, wstała i~zaczęła chodzić, gestykulując jak ona.

-- Nigdy więcej proszenia o~pozwolenie na pójście do łazienki! Nigdy więcej utraty wynagrodzenia, ponieważ zachorujemy! Nigdy więcej zamykania w~fabryce, gdy nadchodzą duże zamówienia. Nigdy więcej nadgodzin bez wynagrodzenia. Nigdy więcej oparzeń rąk i~rąk od pracy maszyny do formowania gumy  \ldots  ile z~was ma na sobie idiotyczne logo jakiejś głupiej firmy po wypadku, któremu można było zapobiec dzięki porządnej odzieży ochronnej?

-- Nigdy więcej brakujących oczu. Nigdy więcej straconych palców. Nigdy więcej skalpów odrywanych od głowy krzyczącej dziewczyny, gdy jej włosy są wciągane przez jakąś gigantyczną maszynę o~sile wołu i~mózgu mrówki. NIGDY WIĘCEJ!

-- Jutro nikt nie pracuje. Nikt. Siostry, już czas. Jeśli któraś z~was odmówi pracy, po prostu cię zwalniają, a maszyny mielą dalej. Jeśli wszyscy odmówicie pracy, \textit{maszyny się zatrzymają}.

-- Jeśli jedna fabryka zostanie zamknięta, wysyłają policję, aby otworzyła ją ponownie, żołnierzy z~bronią, pałkami i~gazem. Jeśli \textit{wszystkie fabryki }zostaną zamknięte, nie będzie wystarczającej liczby policji na świecie, aby je ponownie otworzyć.

Spojrzała na swój ekran. To było szalone. Kliknęła w~połączenie. Wei-Dong usłyszał to w~słuchawce.

-- Jiandi -- powiedział zdyszany, dziewczęcy głos. -- Czy to Jiandi? 

-- Tak, siostro, to prawda -- powiedziała. -- Kto inny? -- Uśmiechnęła się cienkim uśmiechem.

-- Słyszałeś o~innych zgonach w~dzielnicy kantońskiej w~Shenzhen? Chłopcach, których zastrzelili?

Wei-Dong miał wrażenie, że spada. Dziewczyna wciąż mówiła.

--  \ldots  42 z~nich, tak słyszeliśmy. Były zdjęcia, wysłane z~telefonu na telefon. Wygoogluj ,,42 poległych'' i~znajdziesz ich. Policja powiedziała, że to kłamstwa, a przed chwilą powiedzieli, że byli gangiem przestępczym, ale rozpoznałem niektórych z~tych chłopców z~poprzedniego strajku, tego, o~którym nam opowiadałaś \ldots 

Wei-Dong wyciągnął telefon i~zaczął googlować, pisząc tak szybko, że zgniótł przyciski i~musiał trzykrotnie wpisywać ponownie zapytanie, proces ten był jeszcze bardziej uciążliwy ze względu na konieczność korzystania z~serwerów proxy, aby ominąć blokady w~sieci swojego telefonu. Ale potem to zrozumiał, a zdjęcia spływały do przeglądarki telefonu tak wolno jak lodowce, i~wkrótce patrzył na kolejne ujęcie poległych chłopców, leżących na wąskich uliczkach, z~rękami wyrzuconymi lub zaciśniętymi wokół twarzy, bezwładnymi nogami. Zdjęcia z~kamery były trochę nieostre, a mały ekran telefonu sprawiał, że były jeszcze mniej wyraźne, ale widok wciąż uderzał go jak uderzenie młotkiem.

Dziewczyna wciąż mówiła. 

-- Wszyscy je widzieliśmy, dziewczyny w~moim akademiku są przestraszone, a teraz mówisz nam, żebyśmy wyszły z~naszej pracy. Skąd wiesz, że my też nie zostaniemy zastrzeleni?

Usta Jie otwierały się i~zamykały jak ryba. Wyciągnęła rękę i~pstryknęła palcami na Wei-Donga, który podał jej swój telefon. Jej twarz była okropna, usta oderwane od zębów, które rytmicznie klikały, gdy patrzyła na zdjęcia.

-- Och -- powiedziała, jakby nie usłyszała pytania dziewczyny. -- Och -- powiedziała, jakby właśnie uświadomiła sobie jakąś głęboką prawdę, która umykała jej przez całe życie.

-- Jiandi? -- powiedziała dziewczyna.

-- Możesz zostać zastrzelona -- powiedziała Jie powoli, jakby wyjaśniała coś dziecku. -- Mogę zostać zastrzelona. Ale nie mogą zastrzelić nas wszystkich.

Zatrzymała się, zastanawiając się. Łzy spłynęły jej po brodzie i~poplamiły kołnierzyk koszuli.

-- Czy mogą? 

Kliknęła coś i~zaczęła się reklama.

-- Nie mogę tego dokończyć -- powiedziała martwym głosem. -- W ogóle nie mogę tego dokończyć. Powinnam iść do domu.

Wei-Dong spojrzał na swoje dłonie. 

-- Nie sądzę, żeby to było bezpieczne.

Potrząsnęła głową. 

-- \textit{Dom }-- powiedziała. -- Wioska. Wrócić. Zostało trochę pieniędzy. Mogłabym wrócić do domu, a moi rodzice mogliby znaleźć dla mnie chłopaka do małżeństwa, a ja mogłabym być tylko kolejną starzejącą się dziewczyną w~wiosce. Mieć moje jedno dziecko i~błagać, żeby to był chłopiec. Połknąć pestycydy, kiedy zrobi się za ciężko. -- Spojrzała mu w~oczy, a on musiał się zebrać, żeby się nie wzdrygnąć. -- Czy wiesz, że Chiny są jedynym krajem, w~którym więcej kobiet popełnia samobójstwa niż mężczyzn?

Wei-Dong przemówił drżącym głosem. 

-- Nie mogę udawać, że wiem, jak wygląda twoje życie, Jie, ale nie mogę uwierzyć, że chcesz to zrobić. Jest 42 zabitych. Nie sądzę, żebyśmy mogli się tutaj zatrzymać. -- Myśląc, że \textit{jestem tak daleko od domu i~nie wiem, jak wrócę}. Myśląc, \textit{jeśli odejdzie, będę sam}. A potem myśląc, \textit{Tchórz }i chęć uderzenia głową o~coś, dopóki myśli się nie zatrzymają.

Sięgnęła po klawiaturę, a on wiedział wystarczająco dużo o~jej środowisku pracy, by zobaczyć, że szykuje się do zamknięcia.

-- Czekaj! -- powiedział. -- Chodź, przestań. -- Wyłowił słowa. 

W ciągu kilku tygodni, odkąd przyjechał do Chin, zaczął myśleć po chińsku, czasem nawet w~tym marzyć, ale teraz to go zawiodło. 

-- Ja \ldots  -- Z frustracji uderzył pięściami w~uda. -- To się teraz nie skończy -- powiedział. -- Jeśli wrócisz do domu, w~wiosce, będzie szło dalej, ale nie będzie ciebie. Nie będzie Jiandi, starszej siostry wszystkich dziewczyn z~fabryki. Kiedy Lu powiedział mi o~Tobie, pomyślałem, że jest szalony, myślał, że nie ma mowy, abyś mógł mieć tylu słuchaczy. Myślał, że jesteś jakimś bogiem lub królową, przywódcą armii milionów. Powiedział mi, że uważa, że nie rozumiesz, jak ważna jesteś. Jak \ldots  -- Przerwał, zebrał słowa. -- Jesteś lśniąca. Tak powiedział. Błyszczysz, jesteś jak ta jasna, błyszcząca rzecz, za którą ludzie chcą po prostu gonić, za nią podążać. Każdy, kto cię spotyka, każdy, kto cię słyszy, ufają ci, chcą, abyś był ich przyjacielem.

-- Jeśli odejdziesz, Webblies nadal będą walczyć, ale bez ciebie myślę, że przegrają.

Spojrzała na niego. 

-- Prawdopodobnie też przegrają ze mną. Czy masz pojęcie, jak straszny ciężar na mnie nałożyłeś? \textit{Wszyscy }na mnie nałożyliście? To absolutnie niesprawiedliwe. Nie jestem twoim bogiem, nie jestem twoją królową. Jestem radiowcem!

Gorąc wzrósł w~Wei-Dong. 

-- Zgadza się! Jesteś dziennikarzem radiowym. Ale nie pracujesz dla jakiegoś kanału rządowego, takiego jak CTV, prawda? Jesteś podziemiem, przestępcą. Spędziłaś lata, mówiąc dziewczynom z~fabryki, by walczyły o~swoje prawa, lata życia w~kryjówkach i~noszenia fałszywych dowodów tożsamości. Ustawiłaś się tu, gdzie jesteś teraz. Nie mogę uwierzyć, że o~tym nie śniłaś. Spójrz mi w~oczy i~powiedz, że nie \textit{marzyłaś} o~byciu przywódca milionów, aby wszyscy podążali za tobą i~patrzyli na ciebie! Powiedz mi!

Zrobiła coś zupełnie nieoczekiwanego. Roześmiała się. Mały śmiech, urwany śmiech, śmiech z~odłamkami szkła, ale i~tak był to śmiech. 

-- Tak -- powiedziała. -- Tak, oczywiście. Ze szczotką do włosów zamiast mikrofonu, przed lustrem moich rodziców udając didżeja, którego wszyscy słuchali. Oczywiście. Co jeszcze?

Jej uśmiech był tak smutny i~promienny, że Wei-Dong osłabły w~kolanach. 

-- Nigdy nie sądziłem, że tu skończę. Myślałem, że będę ładną dziewczyną w~telewizji, rozpoznawaną na ulicy. Nie zbiegiem.

Wei-Dong wzruszył ramionami, wracając na znajome terytorium. 

-- Przyszłość jest dziwniejsza, niż myśleliśmy, że będzie, gdy byliśmy małymi dziećmi. Spójrz na farmy złota, jakie to dziwne?

Uśmiechnęła się. 

-- Nie dziwniejsze niż robienie gumowych bananów na wystawy w~szwedzkich domach towarowych. To była moja pierwsza praca, kiedy tu przyjechałem, wiesz? 

Podwinęła rękawy i~pokazała mu ramiona. Były poprzecinane starymi bliznami po oparzeniach. 

-- Potem robienie tanich koralików do czegoś, co nazywa się ,,Mardi Gras''. Szef Chan lubił mnie, podobał mi się sposób, w~jaki pracowałem z~gorącym plastikiem. Bez narzekania, mimo że nie mieliśmy masek, mimo że byłam cały czas poparzona. 

Wykręciła przedramię i~zobaczył, że ma logo Nike wypisane z~tyłu, w~bąbelkowej, pomarszczonej bliźnie. 

-- Później pracowałam na tej samej maszynie, w~fabryce obuwia. Widzisz logo? Wiele z~nas je ma. To tak, jakbyśmy byli bydłem, a fabryka znakowała nas pojedynczo.

-- Czy zamierzasz znowu porozmawiać z~ludźmi?

Opadła. Wsunęła słuchawkę. Zaczęła szturchać komputer. 

-- Tak -- powiedziała. -- Tak, muszę. Dopóki będą słuchać, muszę.

\bigskip
\threeast

Matthew płakał, idąc ulicami, nie widząc. Był jednym z~pierwszych, którzy wyszli z~budynku, kiedy policja najechała, i~prześlizgnął się przez kordon, zanim go zacisnęli, wślizgując się do innego budynku ,,uścisku dłoni'', w~którym grał jako chłopiec, i~wbiegł po schodach na dach, gdzie leżał na brzuchu pośród potłuczonego szkła i~kamyków, wpatrując się w~ulicę poniżej, gdy policja goniła i~łapała jego przyjaciół, jeden po drugim, linię twarzy Webblies \ldots  leżeli na ziemi, jęcząc od sporadycznych kopnięć lub uderzeń, gdy naruszali ciszę i~próbowali ze sobą rozmawiać.

Policja zaczęła ich metodycznie skuwać i~nakładać kaptury, zaczynając od jednego końca, pracując trójkami -- jeden do kajdanek, jeden do kaptura i~jeden do pilnowania z~karabinem. Wydawało się, że to trwa wiecznie i~Matthew zauważył, że nie jest jedyną osobą, która obserwuje to chore widowisko: zwisające z~prania balkony budynków, w~których stosuje się uścisk dłoni, drżały, gdy ludzie wpadali na nie z~telefonami komórkowymi wycelowanymi w~uliczkę w~dole. Matthew wyjął swój własny telefon, metodycznie przybliżając każdą twarz, próbując zrobić zdjęcie każdego Webbly'ego, zanim został zakapturzony, myśląc niejasno o~umieszczeniu zdjęć na dużych tablicach Webblies i~przesłaniu ich do prasy zagranicznej, blogerów-dysydentów, którzy korzystali ze swoich serwerów offshore.

Potem nagły ruch. Ping miotał się po ziemi, wymachując kończynami, uderzając głową o~chodnik na tyle mocno, że słychać go było z~pozycji Matthew sześć pięter wyżej. Matthew wiedział z~beznadziejną pewnością, że był to jeden z~napadów padaczkowych jego przyjaciela, które nie pojawiały się zbyt często, ale były gwałtowne i~przerażające dla otaczających go osób. Policjanci próbowali złapać go za ręce i~nogi, a jeden z~nich dostał mocnego kopniaka w~kolano, a wtedy ręka Pinga roztrzaskała zakapturzonego więźnia obok niego, który odtoczył się, potknął się na nogi, a gliniarze wdarli się do środka, kolby karabinów podniesione i~gotowe.

To, co wydarzyło się później, wydawało się trwać wieczność, wieczność, podczas której Matthew walczył, by nie krzyczeć, walczył na granicy niezdecydowania, impotencji, bycia zmuszonym do ucieczki na ulicę poniżej swoich towarzyszy i~bycia zbyt przestraszonym, by ruszyć z~miejsca.

Policjant uderzył zakapturzonego Webbly'ego, który był na nogach, po nerkach, a chłopiec pisnął, zachwiał się i~przypadkiem chwycił kolbę karabinu. Obaj chwycili za broń, podczas gdy chłopcy na chodniku krzyczeli, inni policjanci zbliżali się, a potem jeden z~nich wyjął rewolwer i~spokojnie strzelił zakapturzonemu chłopcu w~głowę, kaptur zbryzgany i~czerwony, gdy chłopiec padał.

To było to. Chłopcy zerwali się na równe nogi i~\textit{zaatakowali}, wojownicy wykrzykujący okrzyki bojowe, nieuzbrojone dzieci przestraszone, odważne i~głupie, a policja strzelała, strzelała i~strzelała.

Jego zmysły obezwładnił zapach kordytu, podobny do fajerwerków, które on i~jego przyjaciele używali w~Nowy Rok. Zmieszał się z~tym zapach krwi, zapach gówna chłopców, którym puściły kiszki. Matthew płakał cicho, kierując telefon w~stronę rzezi, robiąc zdjęcie za zdjęciem, a potem policjant spojrzał na tłum obserwujący masakrę i~krzyknął coś niewyraźnie, obiektyw aparatu na jego kasku zalśnił w~świetle świtu, a Matthew uchylił się, gdy pozostali policjanci podnieśli wzrok i~wtedy usłyszał krzyki, wrzaski zewsząd, ze wszystkich balkonów.

Pędził po dachu, kierując się do następnego budynku, z~łatwością przeskakując wąską szczelinę między nimi. Jeszcze dwa razy przeskakiwał od budynku do budynku, kierując się czystym instynktem przetrwania, a jego umysł był pusty. Potem znalazł się na ulicy, nie pamiętając, że schodził po schodach, szybkim krokiem, kierując się w~stronę centrum miasta, ulic pełnych ekskluzywnych sklepów i~alfonsów, biznesmenów i~kafejek internetowych wypełnionych wrzeszczącymi chłopcami zabijającymi orki i~zdmuchującymi kosmicznych piratów z~nieba i~pokonując złych superzłoczyńców.

Łzy spływały mu po policzkach, a poranny pośpiech ludzi w~drodze do pracy omijał go szerokim łukiem. Nie był pierwszym chłopcem, który we łzach chodził po ulicach Shenzhen i~nie byłby ostatnim. Przypadkowo wsiadł do autobusu, zapłacił za przejazd i~usiadł, chowając twarz w~dłoniach, tłumiąc szloch. Jechał autobusem przez całą godzinę, zanim zadał sobie trud, żeby spojrzeć w~górę i~zobaczyć, dokąd zmierza.

Potem musiał się uśmiechnąć. W jakiś sposób wsiadł do autobusu jadącego do Dafen, ,,wioski malarstwa olejnego'', gdzie tysiące malarzy pracujących w~małych fabrykach wyprodukowało miliony obrazów. Poszedł tam kiedyś z~Pingiem i~chłopcami, w~rzadki wolny dzień, by wędrować wąskimi uliczkami i~podziwiać płótna rozwieszone wszędzie, na straganach pod gołym niebem, w~otwartych sklepach i~ogromnych galeriach. Malowidła były w~większości w~stylu europejskim, staromodne, przedstawiające życie w~starożytnych europejskich miastach, albo umęczonego Jezusa (sprawiały, że Matthew wiercił się i~przypominał historie prześladowań ojca) albo doskonałe owoce leżące na stołach. W niektórych sklepach i~straganach pracowali malarze, kopiując obrazy z~książek, wykonując zręczne małe pociągnięcia pędzlem i~zamykając resztę świata. Same książki zostały wydrukowane w~Dongguan -- Matthew znał dziewczynę z~fabryki, która pracowała w~drukarni – i~coś w~całej tej scenie napełniło Matthew nieopisanym uczuciem na myśl o~tych wszystkich malarzach, którzy tworzą dzieła oczami i~rękami swoich artystów do użytku przez obcokrajowców, którzy nigdy nie przybyli do Chin, nigdy nie wyobrażają sobie twarzy i~rąk malarzy, którzy wykonali to dzieło.

I oto oni podjeżdżali pod pięciometrową rzeźbę ręki trzymającej pędzel, wyrzucając dziesiątki pasażerów na poboczu drogi. Wokół niego wznosiły się wysokie bloki mieszkalne i~długie budynki fabryczne, powietrze pachniało śniadaniem, farbą olejną i~terpentyną.

Matthew otrząsnął się na tyle, by zauważyć, że wielu jego współpasażerów nosiło poplamione farbą ubrania robocze i~nosiło drewniane pudełka z~farbami, i~dołączył do ogólnego tłumu, który wdarł się do Dafen, wśród szmerów rozmów, gdy robotnicy witali się z~przyjaciółmi i~przekazywali plotkę.

Kiedy odwiedził Dafen, zawędrował do galerii, która sprzedawała współczesne obrazy chińskich malarzy, przedstawiające chińskie dekoracje. Nigdy nie miał zbytniego pożytku ze sztuki, ale oni byli zaskoczeni tymi. Jeden przedstawiał cztery dziewczyny z~fabryki, piękne i~młode, trzymające telefony komórkowe i~markowe torby, idące wiejską uliczką podczas Święta Środka Jesieni, z~frontami domów i~witrynami sklepowymi obwieszonymi latarniami. Wioska była stara i~biedna, ulica zepsuta, ludzie obserwowali je z~progów z~pomarszczonymi i~wyschniętymi twarzami chłopów. Cztery dziewczyny były czarującymi kosmitami z~innego świata, dziećmi, które zostały odesłane, by znaleźć swoją fortunę, które wróciły przemienione w~zupełnie inny gatunek.

A był tam też obraz starej babci śpiącej w~schronie dla autobusów w~Dongguan z~otwartymi bezzębnymi ustami, owiniętej pod fałszywym markowym płaszczem, poplamionym brudem i~podartym. I obraz Kantończyka na drabinie między dwoma budynkami ,,uścisku dłoni'', zawieszającego nielegalny kabel. Obrazy były przejmujące, bolesne i~piękne, a on stał tam, patrząc na nie, dopóki właściciel galerii go nie wygonił. Były dla ludzi z~pieniędzmi, a nie dla takich jak on.

Teraz, przechodząc obok tego samego sklepu, poczuł dreszcz rozpoznania, gdy zobaczył zdjęcie czterech dziewczyn z~fabryki, obejmujących się nawzajem w~oknach sklepu. Nie sprzedał się, a może malarz wyrzucił je przy ciężarówce. Może była tam fabryka pełna malarzy oddanych kopiowaniu tego obrazu.

Uświadomił sobie odległy gwar, niewyraźny ryk gniewnych głosów. Myślał, że słyszał to już od jakiegoś czasu, ale zostało to zatopione w~dźwiękach otaczających go ludzi. Teraz stawało się coraz głośniejsze i~nie tylko on to zauważył. To była pieśń, grzmiąca i~nieubłagana, z~dudniącymi, rytmicznymi stopami. Tłum wykręcił szyje, aby zlokalizować zamieszanie, a on do nich dołączył.

Potem skręcili za róg i~zobaczył, co to było: grupa młodych mężczyzn i~kobiet, poplamionych farbą, trzymających kartki papieru z~pięknie wykaligrafowanymi hasłami: ,,FABRYKA BEZ FORMUŁ MALARSTWA NIESPRAWIEDLIWA! '' ,,WYMAGAMY PŁAC!'' ,,SZEF SIU JEST SKORUMPOWANY!''. Szyldy były ozdobione artystycznymi zawijasami, a on zauważył, że na drugim końcu pikiety znajdowało się trzech malarzy przykucniętych nad stosem papieru, wściekle pracującymi pędzlami. Pojawił się nowy znak: ,,PAMIĘTAJ O 42!'' a potem taki, który po prostu mówił ,,IWWWW'' w~zabawnym zachodnim piśmie, a Matthew poczuł przypływ uniesienia.

-- Kim są 42? -- zapytał jednego z~malarzy, ładną młodą kobietę z~kilkoma wydatnymi pieprzykami na twarzy. Założyła włosy za uszy. -- To było trzy godziny temu -- powiedziała, po czym spojrzała na godzinę w~telefonie. -- Cztery godziny temu. -- Potrząsnęła głową, przywołała kilka zdjęć na swoim telefonie. -- Policja dokonała egzekucji 42 chłopców w~kantońskim mieście. Mówią, że chłopcy byli przestępcami, ale sąsiedzi twierdzą, że byli tylko farmerami złota. 

Pokazała mu zdjęcia. Jego przyjaciele, na ziemi, z~głowami w~kapturach, ostrzeliwani przez policjantów, zataczający się pod ogniem. Policjanci anonimowi za maskami. Dziewczyna zobaczyła wyraz jego twarzy i~skinęła głową. 

-- Okropne, prawda? Po prostu okropne. A to, co mówi o~nich armia pięćdziesiąt centów \ldots  

Armia pięćdziesięciu centów była wielkim legionem blogerów, którzy dostawali pięćdziesiąt centów, 4 RMB, za pisanie patriotycznych komentarzy i~postów o~rządzie.

Stwierdził, że siedzi na brudnym chodniku, trzymając telefon dziewczyny. Uklękła przy nim i~spytała: 

-- Hej, proszę pana, wszystko w~porządku?

Odruchowo skinął głową, po czym nią potrząsnął. Ponieważ nie był w~porządku. Nic nie było w~porządku. 

-- Nie -- powiedział.

Dziewczyna spojrzała na znak, który malowała, a potem na niego. Odwróciła się plecami do obrazu i~ujęła go pod brodę, unosząc twarz w~górę. 

-- Jesteś ranny? 

-- Nie boli -- powiedział. -- Ale. -- Potrząsnął głową. Wskazał na swój telefon. Wyciągnął własne. Przywołał zdjęcia, które zrobił, gdy drżał na dachu.

-- Te same zdjęcia? -- powiedziała. Potem przyjrzała się bliżej. -- Różne zdjęcia. Skąd je masz? 

-- Zrobiłem je. -- powiedział charkoczącym szeptem. -- Oni byli moimi przyjaciółmi. 

Podskoczyła, jakby była zszokowana, po czym przygryzła wargę i~przejrzała zdjęcia. Pachniała terpentyną, a jej palce były bardzo długie i~eleganckie. Przypominała Matthew elfa. 

-- Byłeś tam? 

To było tylko pół pytania, ale i~tak skinął głową. 


-- Och, och, och -- powiedziała, oddając mu telefon i~mocno, siostrzańsko przytulając. -- Ty biedny chłopcze -- powiedziała.

-- Słyszeliśmy o~tym godzinę temu, kiedy zabieraliśmy się do pracy. Zebraliśmy się, aby o~tym porozmawiać, zostawiając nasze płótna, a nasz szef, Boss Siu, przyszedł i~zażądał, abyśmy wszyscy wrócili do pracy. Nie pozwoliłby nam powiedzieć mu, dlaczego się zebraliśmy. Nigdy tego nie robi. To tak, jak mówi Jiandi w~swojej audycji radiowej, kontroluje nasze przerwy w~łazience, pobiera nasze pensje za rozmowę lub czasami po prostu za zbyt długie patrzenie w~górę. A kiedy powiedział nam, że wszyscy stracą dniówkę, jedna z~dziewcząt wstała i~wykrzyczała hasło, coś w~stylu ,,Boss Siu jest niesprawiedliwy!'' I chociaż to było zabawne, było też tak \textit{prawdziwe}, prosto z~jej serca, i~wszyscy też wstaliśmy, a potem \ldots  -- Wskazała na linię.

Matthew pamiętał dzień, w~którym odeszli, milion lat temu, pamiętał przybycie policji i~zabranie ich do więzienia, pamiętał, jak przyrzekł, że nigdy więcej nie pójdzie do więzienia. A potem podniósł znak, który robiła, chwycił go za rogi i~dołączył do szeregu. Nie był jedyny. Wykrzykiwał hasła, a jego głos nie był już ochrypły, był mocny i~donośny.

A kiedy policja w~końcu przyjechała, wydarzyło się coś cudownego: ogromny tłum malarzy i~innych robotników, którzy zebrali się w~fabryce, połączył się z~pikietami i~podchwycił ich hasła. Trzymali w~górze telefony i~fotografowali zbliżającą się policję w~maskach, hełmach, tarczach i~pałkach.

Utrzymali swoje miejsce.

Policja strzelała z~kanistrów z~gazem.

Malarze z~wielkimi maskami filtrującymi z~fabryk chwycili kanistry i~spokojnie wyrzucili je przez okna fabryki, wypalając szefów i~ochroniarzy, którzy tam się kulili, a oni wyszli, kaszląc, płacząc i~sapiąc.

Tłum rozszerzył się, ruszył w~\textit{stronę }policji, zamiast się od niej \textit{oddalić}, a policjant wyskoczył do przodu ze swojej linii z~uniesioną pałką, szeroko otwartymi ustami i~oczami za maską na twarzy, a trzy dziewczyny z~fabryki ominęły go, potknął się, a tłum zamknął się nad nim. Linia policji zadrżała, gdy mężczyzna zniknął z~pola widzenia i~kiedy wydawało się, że zaatakują, tłum cofnął się, a mężczyzna był tam, poruszając się trochę, ale boleśnie, leżąc na ziemi. Jego hełm, pałka i~tarcza zniknęły, podobnie jak pas z~bronią, gazem i~wiązką plastikowych kajdanek.

\textit{Teraz mamy broń}, pomyślał Matthew i~z daleka zauważył, że znowu myśli jak taktyk, a nie jak sterroryzowany chłopiec, i~wie, z~której strony powinna nadejść policja, że tamta alejka, gdyby ją zajęli, kontrolowaliby wszystkie wejścia na plac, zatrzymując pikietujących.

-- Potrzebujemy tam ludzi -- krzyknął do malarki, która miała na imię Mei i~stała u jego boku z~uniesionym smukłym ramieniem, gdy wykrzykiwała z~nim hasła. -- Tam i~tam. Dużo ich. Jeśli policja odgrodzi te obszary \ldots 

Skinęła głową i~przecisnęła się przez tłum, klepiąc ludzi po ramionach i~krzycząc im do uszu przez ryk tłumu, syreny policyjne i~nadlatujący helikopter. Ten helikopter sprawił, że ręce Matthew się spociły. Gdyby coś na nich zrzuciło -- na pewno \textit{gaz, nie bomby, na pewno nie bomby}, które myślał jak modlitwa -- nie byłoby się gdzie ukryć. Protestujący ruszyli bronić zaułków, które wskazał, uzbrojeni w~cegły, kamienie i~aparaty fotograficzne. Te same lejkowate wyloty zaułków, które czyniłyby te zaułki tak śmiercionośnymi w~rękach wrogów, ułatwiłyby ich obronę.

Śmigłowiec nadlatywał teraz, a kamery wycelowały w~niebo, a potem helikopter zboczył i~skierował się w~zupełnie innym kierunku. Kiedy Matthew podniósł swój telefon, żeby go sfotografować, zobaczył, że kilka nieodebranych połączeń nie dotarło do niego. Numer, którego nie rozpoznawał za granicą. Nastawił go z~powrotem, przykucając nisko w~lesie tuptających stóp, aby uciec przed hałasem.

-- Witam? -- odezwał się kobiecy głos po angielsku.

-- Czy mówisz po chińsku? -- spytał po kantońsku.

Nastąpiła pauza, po czym telefon został przekazany komuś innemu.

-- Kto to jest? -- odezwał się męski głos po mandaryńsku.

-- Nazywam się Matthew -- powiedział. -- Dzwoniłeś do mnie? 

-- Jesteś jednym z~grupy Shenzhen? -- Mężczyzna powiedział.

-- Tak -- powiedział.

-- Mamy innego ocalałego! -- zawołał i~brzmiał na autentycznie uszczęśliwionego.

-- Kto to jest? 

-- To jest Mighty Krang -- powiedział mężczyzna. -- Pracuję dla Big Sister Nor. Tak się cieszymy z~wiadomości od ciebie, chłopcze! Wszystko w~porządku? Czy jesteś bezpieczny?

-- Jestem w~trakcie strajku -- powiedział. -- Tysiące malarzy w~Dafen. To wioska w~Shenzhen, gdzie malują \ldots 

-- Jesteś w~Dafen? Widzieliśmy stamtąd zdjęcia, wygląda to szaleńczo. Powiedz mi, co się dzieje. 

Bez zastanowienia, po prostu działając, Matthew wspiął się na ławkę w~parku, stanął bardzo wysoki i~podyktował zwarty, kompetentny raport sytuacyjny The Mighty Krang, którego widział na wielu wideokonferencjach z~Big Sister Nor i~Justbob, chichoczącego i~błaznującego w~tle. Teraz brzmiał na absolutnie poważnego i~zdecydowanego, prosząc Matthew o~powtórzenie kilku szczegółów, aby upewnić się, że zrozumiał.

-- A widziałeś inne strajki?

-- Inne strajki?

-- Wszędzie wokół ciebie -- powiedział. -- Lianchuang, Nanling i~Jianying Gongyequ. W Jianying Gongyequ płonie fabryka. To zły interes. Dzikie strajki \ldots  gdyby z~nami porozmawiali, powiedzielibyśmy, żeby tego nie robili. Jednak. -- Przerwał. -- Te zdjęcia były czymś. 42 poległych.

-- Mam więcej. 

-- Skąd je masz?

-- Byłem tam. 

-- Och.

Długa pauza.

-- Matthew, czy jesteś bezpieczny tam, gdzie jesteś?

Matthew wstał ponownie. Linia policji cofnęła się, demonstracja nabrała nieco karnawałowej atmosfery, artyści śmiali się i~rozmawiali intensywnie. Niektórzy mieli instrumenty i~improwizowali muzykę.

-- Bezpieczny -- powiedział.

-- OK, wyślij mi te zdjęcia. I bądź bezpieczny. 

Teraz jeszcze dwa helikoptery, nie leciały w~ich stronę. Domyślał się, że zmierza do płonącej fabryki w~Jianying Gongyequ. Miał nadzieję, że nikogo w~niej nie było.

\bigskip
\threeast

Pan Bannerjee przyszedł po nich tej nocy wraz z~inną grupą bandytów, ale nie byli to chudzi chuligani, ale dorośli, brudni mężczyźni z~nożami i~pałkami, mężczyźni, którzy pachnieli betelem, potem, dymem i~ognistym alkoholem, zapach, który poprzedzał jak posłaniec krzyczący ,,uważaj, uważaj''. Przybyli, wołając i~żartując przez Dharavi, tłum, który Webblies słyszeli z~daleka. Sąsiedzi pani Dibyendu podeszli do ich okien, gdakali z~niepokojem i~posłali swoje dzieci, by położyły się na podłodze.

Pan Bannerjee prowadził procesję w~swoim ładnym garniturze, błoto zasysało jego piękne buty. Stał na alejce przed drzwiami do kawiarni pani Dibyendu, położył ręce na biodrach i~zapalił papierosa, robiąc z~tego pokaz, z~całą nonszalancją, gdy ożywiał go i~puszczał strumień w~gorące, wilgotne powietrze.

Czekał.

Mala pokuśtykała do drzwi i~otworzyła je. Za nią kawiarnia była ciemna i~nic się nie poruszało.

Żadne z~nich nie powiedziało ani słowa. Sąsiedzi przyglądali się zmartwionemu milczeniu.

-- Mala -- powiedział pan Bannerjee, rozkładając ręce. -- Bądź rozsądna.

Mala wyszła na ganek kawiarni i~usiadła, niezgrabnie podkładając pod siebie nogi. Czystym, donośnym głosem powiedziała: 

-- Pracuję tutaj. To jest moja praca. Domagam się prawa do bezpiecznych warunków pracy, godziwych zarobków oraz sprawiedliwego i~uczciwego miejsca pracy.

Pan Bannerjee prychnął. Mężczyźni za nim się roześmieli. Zrobił krok do przodu, po czym się zatrzymał.

Jedna po drugiej, armia Mali wyszła z~kawiarni w~zdyscyplinowanym, wojskowym rządzie. Każdy usiadł, aż mały ganek zapełnił się siedzącymi dziećmi.

Pan Bannerjee znów parsknął, po czym roześmiał się. 

-- Nie możesz być poważna -- powiedział. -- Chcesz, chcesz, chcesz. Kiedy cię znalazłem, byłaś szczurem Dharavi, bez pieniędzy, bez pracy, bez nadziei. Dałem ci dobrą pracę, dobre zarobki, a teraz chcesz, chcesz i~chcesz? -- Wydał lekceważący dźwięk i~machnął na nią ręką. -- Usuniesz się z~mojej kawiarni i~zabierzesz ze sobą swoich uczniów albo zostaniesz zraniona. Bardzo mocno.

Sąsiedzi wydali na to zgorszone cmokanie, a pan Bannerjee ich zignorował.

-- Nie skrzywdzisz nas -- powiedziała Mala. -- Wrócisz do swojego pięknego domu i~swoich wspaniałych przyjaciół i~zostawisz nas samych, abyśmy kierowali naszym przeznaczeniem.

Pan Bannerjee nic nie powiedział, tylko dopalił papierosa i~wpatrywał się w~nich, patrząc na nich jak naukowiec, który odkrył nowy gatunek owadów.

-- Robisz psoty, Mala. Wiem, co kombinujesz. Zaburzasz sprawy, które są większe od ciebie. Mówię ci jeszcze raz. Usuń się z~mojej kawiarni.

Mala wydała z~siebie bardzo cichy, pełen pogardy odgłos plucia.

Pan Bannerjee podniósł rękę, a jego tłum zamilkł, przygotowując się.

A potem był dźwięk. Odgłos kroków, setki kroków. Tysiące. Armia maszerowała alejką z~obu stron, a potem byli przy nich. Ashok prowadzący kolumnę z~lewej, stara pani Rukmini i~pan Phadkar prowadzący kolumnę z~prawej.

Same kolumny składały się z~robotników związkowych -- robotników tekstylnych, hutników, kolejarzy. Rozmowy telefoniczne, zdjęcia i~historie Ashoka się opłaciły. Wysłano setki SMS-ów, robotnicy zostali zerwani z~łóżek, po czym pospiesznie się ubrali i~zebrali, by zostać przewiezionymi przez autobusy związkowe przez cały Bombaj do Dharavi, zaprowadzeni do sklepu pani Dibyendu przez Ashoka, który wyszeptał podziękowania dla przywódcy, którzy udzielili mu wsparcia.

Robotnicy zatrzymali się zaledwie kilka kroków od gangsterów i~ich paskudnych zapachów. Ashok spojrzał na dwie grupy, siedzącą armię i~stojący gang, po czym rozmyślnie i~powoli usiadł.

To samo zrobiły przepięknie starsze panie stojące na czele drugiej kolumny. Siadanie rozszerzyło się, przesuwając się z~powrotem przez grupę, a jeśli jakikolwiek pracownik pomyślał o~jego spodniach lub jej sari przed siedzeniem w~brudnej alei Dharavi, nikt nie powiedział ani słowa i~nikt się nie wahał.

Bannerjee głośno przełknął ślinę. Jeden z~sąsiadów wychylających się z~okna zachichotał. Bannerjee spojrzał na okna. 

-- Domy w~takich slumsach cały czas płoną -- powiedział, ale głos mu drżał. 

Sąsiad, który zachichotał -- młody mężczyzna bez koszuli z~oparzeniami w~górę i~w dół po jakimś starym wypadku -- zamknął okiennice. Chwilę później był na ulicy. Podszedł do Bannerjee, spojrzał mu w~oczy, a potem celowo skrzyżował nogi i~usiadł przed nim. Bannerjee uniósł nogę, jakby chciał kopnąć, a tłum \textit{zawarczał}, niski, dziki dźwięk, który sprawił, że włosy na karku Mali stanęły dęba, mimo że sama to zrobiła. Brzmiało to tak, jakby cały Dharavi był wściekłym psem, napinającym smycz, grożącym rzuceniem się.

Na ulicę wyszło więcej sąsiadów -- starych i~młodych, mężczyzn i~kobiet. Znali panią Dibyendu od lat. Widzieli ją wyprowadzoną z~domu i~firmy. Wydawali ten sam dźwięk. Oni też siadali.

Pan Bannerjee spojrzał na Malę i~otworzył usta, jakby chciał coś powiedzieć, po czym się zatrzymał. Spojrzała na niego z~całkowitym spokojem, po czym uśmiechnęła się szeroko. 

-- Buu -- powiedziała cicho, a on cofnął się o~krok.

Jego ludzie śmiali się z~tego, a on poczerwieniał w~przyćmionym świetle ulicy. Mala ugryzła się w~język, żeby się nie śmiać. Wyglądał tak komicznie!

Odwrócił się z~wielką godnością, aby spojrzeć na swoich ludzi, którzy byli tak spięci, że praktycznie wibrowali. Mala patrzyła z~osłupieniem w~podziwie, jak chwycił jednego na chybił trafił i~uderzył go mocno w~twarz, a dźwięk rozbrzmiał na wąskiej uliczce. To był najgłupszy akt przywódczy, jaki kiedykolwiek widziała, tak całkowicie głupi, że można by go włożyć do słoika i~pokazać, by ludzie mogli go podziwiać.

Mężczyzna przyglądał się przez chwilę Bannerjee z~wściekłymi oczami i~zaciśniętymi pięściami. Był niższy od Bannerjee, ale dźwigał kawałek drewna, a mięśnie jego nagich przedramion drgały i~napinały się jak kosz pełen węży. Mężczyzna rozmyślnie wypluł w~twarz Bannerjee kroplę złej, różowej, poplamionej betelem śliny, odwrócił się na pięcie i~odszedł, delikatnie przedzierając się przez siedzących Webblies, robotników i~sąsiadów. Po chwili reszta gangu Bannerjee podążyła za nim.

Bannerjee stał sam. Ślina spłynęła mu po twarzy. Mala pomyślała, że \textit{jeśli wyjmie pistolet i~zacznie strzelać, nie zdziwi mnie to w~najmniejszym stopniu}. Został całkowicie pobity, upokorzony przed dziećmi i~biednymi z~Dharavi, a w~nocy tańczyło tyle błysków aparatu, że przypominało to dyskotekę w~filmie.

Ale może Bannerjee nie miał broni, a może miał większą samokontrolę, niż sądziła Mala. W każdym razie on też odwrócił się na pięcie i~odszedł. Na końcu zaułka odwrócił się i~powiedział głosem, który dał się słyszeć ponad gwar rozmów, które pojawiły się tuż za nim: 

-- Wiem, gdzie mieszkają twoi rodzice, Mala -- po czym odszedł całkowicie w~noc.

Tłum ryknął triumfalnie, gdy zniknął. Ashok pomógł jej wstać, trzymając rękę w~jej dłoni dłużej, niż było to absolutnie konieczne. Chciała go przytulić, ale zadowoliła się uściskiem Yasmin, która płakała radosnymi łzami, jak te, które dzieliły tyle razy wcześniej. Yasmin była cienka jak kartka papieru, ale jej ramiona były silne i~och, dobrze było być przez chwilę przytuloną, zamiast tulić wszystkich innych.

W końcu puściła i~zwróciła się do Ashoka. 

-- Przybyli -- powiedziała.

Zamiast odpowiedzieć, zaprowadził ją do dwóch malutkich starszych pań i~mężczyzny w~jarmułce i~brodzie. 

-- Pan Phadkar, pani Rukmini i~pani Muthappa -- powiedział. -- To jest Mala. Nazywają ją generałem Robotwallah. Jej pracownicy bronią strajku. Są nie do pokonania, dopóki mają gdzie pracować.

Pan Phadkar wyglądał groźnie. 

-- Zawsze będzie pani miała miejsce do pracy, generale -- powiedział głosem, który miał nieść do robotników, którzy zgromadzili się wokół nich, podekscytowani przekazując szeptane relacje z~historycznego spotkania z~powrotem przez ich szeregi.

Staruszki przewróciły oczami, co wywołało uśmiech Mala. Każdy z~nich wziął jedną z~jej dłoni w~swoje zrogowaciałe, suche dłonie i~ścisnął. 

-- Byłaś bardzo odważna -- powiedział jeden z~nich. -- Proszę, przedstaw nas swoim towarzyszom. 

Rozmawiały całą noc, a kobiecy kolektyw papadam przyniósł im jedzenie, był też czaj, a ponieważ było zdecydowanie za dużo ludzi, aby zmieścić się w~małej kawiarni, impreza zajęła całą ulicę, a potem wyszła na ulicę. Mala i~jej wojownicy walczyli przez całą noc na zmiany, wysiadając w~przerwach, żeby się spotkać, nawiązując przyjaźnie, zabierając ich do kawiarni, by wyjaśnić, co zrobili i~jak to zrobili.

Reporterzy zadawali pytania, a gupshup latał w~górę i~w dół ulicami i~alejkami Dharavi, nabierając pary, gdy koguty zaczęły nawoływać, a pierwszy z~rannych ptaszków szedł do toalet i~kranów i~nasłuchiwał. Odwaga dzieci, męstwo robotników, zło złowrogiego Bannerjee w~garniturze i~bandytów, które przywiózł ze sobą -- to była historia prosto z~ekranu filmowego, a każde nowe ucho, w~które weszła, było dołączone do ust, które pragnęły ją rozprzestrzenić.

Rodzice Mali i~Yasmin przyszli zobaczyć się z~nimi następnego ranka, gdy siedzieli otępiali po nocy jak żadna inna na werandzie kawiarni pani Dibyendu. Rodzice nie wiedzieli, co począć ze swoimi dziwnymi córkami, ale byli z~nich wyraźnie dumni, nawet ojciec Yasmin, co wyraźnie zaskoczyło Yasmin, która wyglądała, jakby spodziewała się pobicia.

Kiedy matki przytuliły je do piersi, Mala spojrzała na Yasmin i~zobaczyła nawiedzone spojrzenie w~jej oczach i~wiedziała, po prostu \textit{wiedziała}, że myśli o~małym chłopcu, który zmarł.

Skąd wiedziała? Ponieważ sama Mala nigdy nie przestała o~nim myśleć i~myśleć o~tym, jakie podjęła działania, które doprowadziły do jego śmierci. A ponieważ sama Mala wiedziała, że żadna ilość pokojowego siedzenia z~jej siłą moralną zamiast z~jej armią nigdy nie zetrze plamy śmierci tego chłopca z~jej karmy.

A potem Mamaji pocałowała Mali w~czoło i~szepnęła jej wiele rzeczy do ucha, a jej młodszy brat wyszedł zza jej spódnicy i~zażądał, by mu pokazano, jak to wszystko działa, i~wpatrywał się w~nią z~takim podziwem, że myślała, że pęknie i~przez jakiś czas, wszystko było złote.

Ashok patrzył ze swojego małego biura, spotykając się z~przywódcami związkowymi, rozmawiając z~Big Sister Nor. Wiedziała, że szykuje się z~nim coś wielkiego, coś jeszcze większego niż ten cud, którego dokonał. Namówiła brata na grupę chłopców, którzy chcieli nauczyć go podstaw i~rozkoszować się czystym kultem bohatera, promieniującym z~niego, po czym wślizgnęła się z~powrotem do pokoju Ashoka i~usiadła u jego boku na stołku, przesuwając stos papierów w~pierwszej kolejności.

-- To było niesamowite -- powiedziała. -- Absolutnie niewiarygodne. -- Powiedziała to cicho, z~przekonaniem. -- Jesteś naszym zbawcą. 

Parsknął przez nos, po czym przetarł oczy pięściami. 

-- Mala, mój generale, robisz sto niesamowitych rzeczy każdego dnia. Jedynym powodem, dla którego ci wszyscy ludzie się ujawnili, jest to, że mogłem im pokazać, co zrobiłaś, wyjaśnić, jak zorganizowałaś te dzieci, te szczury ze slumsów, w~zdyscyplinowaną siłę, która została oddana sprawiedliwości.

Wierciła się na swoim siedzeniu. 

-- Jestem po prostu krwiożercza -- powiedziała. -- Jestem po prostu jedną z~tych osób, które cały czas walczą. 

Znowu myśląc o~chłopcu, martwym chłopcu. Jego krew wciąż była pod paznokciami Ashoka.

Odwrócił się i~tylko na chwilę dotknął jej ramienia. Gest był delikatny, czuły. Nikt nigdy nie dotknął jej w~taki sposób. Coś w~niej pękło, jakaś tama powodziowa, która bezpiecznie powstrzymała cały ból, strach i~wstyd, i~musiała skręcić i~pobiec na oślep na alejkę i~za róg, żeby płakać i~płakać, przygryzając wargę, żeby nie wykrzyczeć smutku. Chociaż słyszała, jak inni jej szukali, milczała i~nie pozwoliła im się znaleźć. Potem zdała sobie sprawę, że ukrywa się w~tym samym miejscu, w~którym ukrywała się przed idiotycznym bratankiem pani Dibyendu, a to przerwało kolejną tamę i~minęło sporo czasu, zanim zdołała się opanować i~wrócić na alejkę.

Nie zaszła zbyt daleko. Przed dziesiątkami firm były małe grupki ludzi hałaśliwie wykrzykujących rymowane pieśni o~warunkach pracy i~płacy. Tłumy zbierały się, żeby ze sobą porozmawiać, były kłótnie, śmiechy, walka na pięści. Stała na środku drogi i~myślała: \textit{Jak to możliwe}? 

I w~tym momencie zdała sobie sprawę, że nie jest sama. W całym Dharavi, na całym świecie, byli ludzie tacy jak ona, którzy chcieli więcej, \textit{żądali }więcej, z~tęsknotą, która zawsze była tuż pod skórą i~wystarczyło najlżejsze zadrapanie, by to wyzwolić.

Nie wróciła do kawiarni pani Dibyendu. Zamiast tego wzięła laskę i~pokuśtykała po całym Dharavi, w~górę i~w dół po ulicach, gdzie maleńkie fabryki normalnie byłyby ulami aktywności. Wiele z~nich było, ale wiele nie było, wiele miało robotników i~tłumy przed frontem, i~to było jak wirus, który rozprzestrzeniał się po ulicach, uliczkach i~zaułkach, a teraz było tak, jakby cały płacz ją rozjaśnił, tak że jej stopy ledwo dotykały ziemi, jakby w~każdej chwili mogła odlecieć.

Właśnie odwracała się, żeby wrócić do swojej armii, i~może na kilka godzin snu, kiedy złapali ją, mocno uderzyli ją w~głowę i~wciągnęli do małego, śmierdzącego pokoju.

\bigskip
\threeast

Zaufanie to śmieszna sprawa. Kiedy wielu ludzi wierzy, że coś jest wartościowe, staje się wartościowe. Więc jeśli sprzedajesz złoto z~gier, a ludzie uważają, że jest ono cenne, kupują je.

Ale jest coś lepszego niż to. Jeśli istnieje powszechne przekonanie, że miecze Wojowników Svartalfaheim są wartościowe, to nawet ludzie, którzy \textit{nie sądzą}, że są wartościowe, kupią je, ponieważ wierzą, że mogą je sprzedać ludziom, którzy \textit{wierzą}, że są wartościowe.

A kiedy ludzie, którzy kupują, aby sprzedać innym, zaczynają licytować miecze Svartalfaheim, cena mieczy rośnie. Oczywiście, że tak: im więcej kupujących na coś, tym wyższa cena. A im wyższa cena, tym więcej kupujących, bo hej, jeśli cena jest wysoka, musi być mnóstwo frajerów, którzy za chwilę kupią ci miecze z~rąk za jeszcze wyższą cenę.

Zaufanie tworzy wartość. Wartość tworzy większą wartość, co daje więcej pewności siebie. Co daje większą wartość.

Ale nie jest nieskończona. Pomyśl o~postaci z~kreskówek, która zbiega z~urwiska i~biega jak szalona w~miejscu, pozostając tam, dopóki ktoś nie zauważy, że tańczy w~powietrzu, po czym spada na ostre skały pod sobą.

Tak długo, jak wszyscy będą wierzyć w~wartość miecza Svartalfaheim, miecz będzie cenny i~stanie się jeszcze bardziej wartościowy. W miarę jak rośnie pula ludzi, którzy mogą kupić miecz Svartalfaheim, powiedzmy, ponieważ dostają telefony od swoich brokerów oferujących sprzedaż skomplikowanych, złożonych kontraktów futures na miecze (kontrakt na zakup miecza w~późniejszym terminie) lub ponieważ ich mądre siostrzenice i~siostrzeńcy gadają o~nich -- prawdopodobieństwo, że ktoś powie: ,,\textit{Żartujesz}? To jest \textit{miecz }w \textit{grze wideo}!'' rośnie.

Rzeczywiście, ten wątpiący może mieć inne obserwacje, takie jak: ,,Jeśli \textit{każdy }ma te miecze, czy nie oznacza to, że jest ich więcej, niż ktokolwiek mógłby użyć? Czy nie oznacza to, że nie są one cenne, ale \textit{bezwartościowe}? ''

Lub jeśli wątpiący jest nieprawdopodobnie staromodny, może nawet powiedzieć: ,,A co, jeśli ludzie, którzy prowadzą tę grę Fartenstein, zdecydują się zmienić liczbę dostępnych mieczy, po prostu \textit{usuwając }ich tonę? Lub drukując kazillion więcej? Albo zmienić miecze na wykałaczki?

Och, spekulanci mieczem odpowiedzą, że \textit{nigdy }tego nie zrobią, to zrujnuje grę, nie stać ich na to. A oto sedno: mają w~połowie rację. Dopóki gamerunnerzy wierzą, że majstrowanie przy mieczach wkurzy tych wszystkich ludzi, którzy posiadają, spekulują, kupują i~sprzedają miecze, nie mogą sobie na to pozwolić.

Te postacie z~kreskówek biegną w~powietrzu, krzycząc, że wartość mieczy zawsze będzie rosła, krzycząc, że producenci gier nigdy nie osłabią ani w~żaden inny sposób ich nie zepsują, i~mogą tam pozostać, w~powietrzu, wymachując mieczami, dołączają do nich inni, którzy są przekonani swoimi argumentami i~niepodważalnym faktem, że rzeczywiście nie upadają, dopóki \ldots 

Dopóki \ldots 

Dopóki nie będzie wystarczającej pewności w~propozycji, że upadną. Dopóki prasa nie zacznie publikować z~szeroko otwartymi oczami opowieści o~absurdzie wierzenia w~wartość tych mieczy, wskazując, że upadek jest nieunikniony, że został z~góry przesądzony od momentu, gdy pierwszy spekulant kupił swój pierwszy miecz.

Pomyśl o~wierze w~nieomylne miecze jako o~układzie słonecznym. W centrum znajduje się słońce, gigantyczne i~rozgrzane do białości, wywierające grawitację na planety i~asteroidy, które krążą wokół niego i~wokół niego. Na zewnętrznej krawędzi znajduje się łupież planetozymalów i~asteroid, słabo schwytany przez grawitację, tylko w~połowie zaangażowany w~bycie częścią układu. Gdy słońce zaczyna stygnąć, zaczyna się kurczyć z~siłą niedowierzania, te zewnętrzne wieszaki odlatują. To są degustatorzy, ludzie, którzy kupili jeden lub dwa małe miecze lub miecze-futures lub ,,w pełni zabezpieczone złożone papiery wartościowe pochodzące od mieczy'', ponieważ wszyscy inni to robili. Słyszą, że to jest zbyt piękne, aby mogło być prawdziwe, i~widzą, że ceny zaczynają spadać, więc sprzedają to, co mają, ponoszą niewielką stratę i~mówią o~tym znajomym.

Cóż, teraz jest grupa ludzi, którzy twierdzą, że miecze nie są tak cenne. Mniejsze zaufanie oznacza niższe ceny. A na rynku jest więcej mieczy. Więcej mieczy to niższe ceny. Większe planety, bliżej, inwestorzy z~dużą ilością pieniędzy w~wyimaginowanych sztućcach, ci ludzie widzą, jak ceny spadają i~nadal spadają. Słyszą, jak brokerzy i~analitycy kręcą się wokół, mówiąc: ,,Nie, nie, słońce będzie świecić jasno na zawsze, słońce nigdy nie przygaśnie! Ceny znów wzrosną. To jest tymczasowe''.

Jest taka sprawa: jeśli brokerzy i~analitycy zdołają przekonać tych większych inwestorów, że mają rację, będą mieli rację. Jeśli ci więksi inwestorzy trzymają się swoich mieczy, rynek pozostanie zdrowy jeszcze przez jakiś czas.

Ale jeśli nie są wystarczająco przekonujące, jeśli ci więksi inwestorzy stracą zaufanie i~zaczną sprzedawać, nigdy nie przestaną. To dlatego, że \textit{pierwszy }sprzedawca, który wyjdzie z~rynku miecza, otrzyma najwyższą cenę za swoje towary. Ale kiedy wyjdzie, jego miecze będą na rynku (pamiętaj, więcej mieczy to niższa cena) i~wszyscy inni dostaną niższą cenę. A kiedy \textit{sprzedadzą}, ceny spadną dalej, wywołując panikę kolejnych inwestorów, wprowadzając na rynek więcej mieczy, zmuszając ceny do dalszego spadku.

Gdzieś tam, producenci gier mają skłonność do drobnego szaleństwa, a potem większego. Zaczną majstrować przy dostawie mieczy. Wyjmą miecze z~rynku, włożą miecze, osłabią miecze lub wystrzelą z~nich do diabła, cokolwiek, aby zabawa nie wypadła z~gry.

I to prawdopodobnie pogorszy sprawę, ponieważ to nie jest nauka ścisła, to tylko zgadywanie, a jest dziesięć milionów sposobów, aby to zrobić źle, a tak niewiele sposobów, aby to naprawić.

Słońce staje się mniejsze i~ciemniejsze, a planety znajdujące się w~pobliżu czują teraz szarpnięcie zapomnienia, zew głębokiej przestrzeni, który mówi: ,,Odleć, odleć na zawsze, bo słońce umiera!''

Nie chcą odlecieć. Chcą się trzymać. Mają w~banku tak wiele mieczy, że są praktycznie \textit{zrobione }z mieczy. Zrobiły fortunę, kupując i~sprzedając miecze. Oczywiście wydali fortunę na więcej mieczy. Albo inne miecze. Albo siekiery. Ale na cokolwiek wydali, to w~zasadzie to samo, ponieważ każdy broker wie, że nie będziesz miał kłopotów, polecając ludziom kupowanie rzeczy, które zawsze były opłacalne.

Jeśli załamie się rynek mieczy, te planety -- ci główni, zaangażowani inwestorzy -- umrą. Zostaną wymazani. Poświęcili swoje życie, miłość i~nieśmiertelne dusze magicznym mieczom, a jeśli miecze złamią im serca, nigdy nie wyzdrowieją. W miarę jak rynek na miecze robi się coraz bardziej kruchy, coraz bardziej kruchy, coraz bardziej upierają się, że wszystko jest w~porządku, w~porządku, lada dzień wszystko wróci do ,,normalności''. Nie mogą sobie pozwolić na utratę pewności siebie, ponieważ nie polecą w~kosmos. Spadną w~umierające słońce i~spłoną w~jego płonącym sercu.

Ale zaprzeczanie działa tylko tak długo. Słońce umiera. Nikt nie chce twoich mieczy. Twoje miecze są bezwartościowe. Nawet ludzie, którzy potrzebują miecza do zabicia elfów, orków lub przypadkowych stworzeń dzikich zwierząt, są lekko zakłopotani tym faktem, ponieważ bezwartościowe miecze są teraz przedmiotem licznych żartów na temat idiotycznych planów inwestycyjnych, skorumpowanych domów maklerskich i~szalonych inwestorów, którzy zostali porwani przez chwilę. Ci ludzie chodzą przez chwilę i~zabijają potwory łukami i~maczugami, ponieważ każdy wie, jak bardzo miecze są do niczego.

Jak niska może być wartość miecza? Jak się okazuje, poniżej zera. Miecz nie tylko może stać się bezwartościowy, ale jego pozbycie się może kosztować cię pieniądze. Och, oczywiście nie sam miecz, ale \textit{pochodne }mieczy. Zakłady na miecze. Tam, gdzie ktoś inny postawił zakład na to, czy wartość twojego miecza wzrośnie, czy spadnie, a następnie spakował ten zakład z~kilkoma innymi zakładami, po prostu zastanawiając się, które zakłady dotyczą, które pakiety mogą kosztować tak dużo pieniędzy, że w~końcu przegrywasz pieniądze, nawet na wygrane zakłady.

Zaufanie jest wielkie, ale to nie wszystko. Rzeczywistość w~końcu dopada wszystkich. Wszystkie słońca same zagasną. Wszystkie postacie z~kreskówek w~końcu spadają na dno kanionu. A każdy miecz jest ostatecznie bezwartościowy.

\bigskip
\threeast

Centrala Dowodzenia była domem wariatów. Producenci gry warczeli na siebie jak zdenerwowane dinozaury z~ogromnymi brzuchami i~jedli jak dinozaury, wysyłając po hamburgery, pizzę, wiadra kurczaka, ogromne, gęste koktajle. Wszystko, co mogli przebrać jedną ręką podczas pracy nad ekranami i~wykrzykiwali do siebie obelgi.

Connor prawie tego nie zauważył. Był głęboko pochłonięty jedzeniem. Nowe podprogramy bezpieczeństwa Billa pozwalają mu odtwarzać działania każdego gracza wstecz i~do przodu jak na wideo, rozgałęziając się na osie czasu innych graczy za każdym razem, gdy przekroczą ścieżki w~drużynie, sesji walki PvP, wymianie lub rozmowie. To był ocean informacji, zawierający każdy sekret każdego gracza w~każdej grze prowadzonej przez Coke.

To było za dużo informacji. Szukał czegoś bardzo precyzyjnego -- tożsamości hodowców złota -- ale to, co miał, było każdą cholerną rzeczą, jaką kiedykolwiek wypowiedziano lub zrobiono w~grze. To była cudowna zabawka i~nieskończona rozrywka, a praktycznie każdą wolną chwilę, jaką Connor mógł zebrać, spędzał na pisaniu skryptów i~filtrów, które miały mu pomóc to zrozumieć.

Właśnie teraz obserwował transmisję każdego gracza, który w~PvP zabił innego gracza, gdzie pionek martwego gracza zarobił ponad 1000 monet Mario w~ciągu poprzedniej godziny. Okazało się, że jest to bogata żyła potencjalnych farmerów złota i~Webblies. Próbował tylko wymyślić, jak napisać skrypt, który przechwyci również identyfikatory graczy każdego, kto był w~\textit{pobliżu }podczas jednej z~tych walk, kiedy zdał sobie sprawę, że Centrum Dowodzenia stało się jeszcze głośniejsze niż zwykle, pogrążając się w~surowym chaosie.

Spojrzał w~górę. 

-- Co jest nie tak? -- spytał, nawet gdy jego palce poruszały się, by wywołać ogólne kanały pokazujące ogólny stan gry i~jej systemów. 

I jeszcze zanim ktokolwiek odpowiedział, zobaczył, co się stało. Obciążenie serwerów wzrosło w~każdym kawałku gry, przekraczając czerwoną linię w~klastrach serwerów umieszczone w~klimatyzowanych kontenerach towarowych na całym świecie. Wyglądało na to, że każda metryka obciążenia serwera była na maksimum -- obliczenia na sekundę, zużycie pamięci, użycie dysku. Ale po bliższym przyjrzeniu się, zauważył, że to nie do końca prawda: obciążenie sieci spadło. Mocno. W jakiś sposób te ogromne pokłady mocy obliczeniowej zostały zmuszone do tak ciężkiej pracy, że groziło im zawalenie się, ale wszystko to odbywało się bez mówienia zbyt wiele do serwerów.

Rzeczywiście, obciążenie sieci było \textit{tak }niskie, że wydawało się, że prawie nikt nie może być zalogowany na tych serwerach, i~rzeczywiście, liczba zalogowanych graczy była niska i~spadała, milion graczy, potem 800 000, potem 500 000, potem 300 000, a wreszcie gry ustabilizowały się na około 40 000 sesji. Kolejne kliknięcie ujawniło, dlaczego: system wyrzucał graczy wraz ze wzrostem obciążenia, próbując zrobić miejsce w~pamięci i~na procesorach dla dowolnego procesu potwora, który przedzierał się przez chłodne kontenery transportowe.

-- Co się do cholery dzieje? -- spytał, krzycząc w~ogólnym zgiełku. 

Kaden rozmawiał przez telefon z~operatorami, krzycząc na administratorów systemów, żeby się do tego zabrali, prześledzili każdy proces na skrzynkach, zidentyfikowali gatunki dusicielskich pnączy, które były luźne w~maszynach, zaduszając je na śmierć.

Bill w~międzyczasie wypuścił \textit{swój }specjalny zespół hakerów w~szarych kapeluszach, aby spróbować dowiedzieć się, czy w~systemie nie ma któregoś z~ich braci w~czarnych kapeluszach, włamywaczy, którzy włamali się, aby ukraść tajemnice korporacyjne, zgromadzić wirtualne bogactwo, lub po prostu rozbić rzecz, aby przynieść korzyść konkurentowi, szukać okupu lub po prostu zniszczyć dla przyjemności zniszczenia.

Connor obstawiał hakerów. Każdy klaster został zbudowany i~przetestowany w~siedzibie głównej Coke Games w~Austin, a następnie wypalany przez trzy tygodnie po przykręceniu go do kontenera transportowego. Po zapaleniu na zielono testów był ładowany na ciężarówkę z~platformą i~wysłany do centrum danych w~zimnym miejscu, najlepiej w~pobliżu otworu geotermalnego, farmy pływów lub farmy wiatrowej. W Nowej Fundlandii i~na Alasce było wiele stanowisk, kilka bardzo dobrych w~Islandii i~Norwegii, kilka w~Belgii i~kilka na Syberii. Piękno używania standardowych kontenerów transportowych w~ich systemach polega na tym, że były one łatwe do wysyłki (co nie?). Piękno umieszczania pojemników w~zimnym miejscu polegało na tym, że głównym kosztem eksploatacji systemów było chłodzenie maszyn, które bezlitośnie pocierały o~siebie elektrony, odbijając je przez wnętrzności automatu do gry w~pinball, w~których znajdują się żetony. W chłodny dzień, kiedy wiał wiatr, mogli obniżyć koszty eksploatacji jednego z~tych kontenerów o~połowę.

Coca-Cola kupowała ich gniazda w~centrach danych po trzy, pozostawiając jedno puste. Kiedy przybywał nowy kontener, wsadzano go do pustej zatoki, uruchamiano go przez tydzień, aby upewnić się, że nic nie ucierpiało podczas transportu, a następnie najstarszy kontener w~gnieździe był wyszarpywany z~miejsca i~ładowany z~powrotem na pociąg lub statek, lub ciężarówki z~platformą i~odsyłany z~powrotem do Austin, przez Bombaj, Shenzhen lub Lagos, aby podrzucić komputery ze środka, rozebrane przez ekipy robocze, które wysyłały je na rynki używanych serwerów, aby zostały rozebrane na kawałki i~odzyskane.

Wszystkie kontenery były wyspecjalizowane, obsługiwały tylko ruch lokalny, aby ograniczyć opóźnienia w~sieci. Ale jeśli ktoś był przytłoczony, mógłby zacząć przerzucać się na swoich braci na całej planecie -- lepiej zmierzyć się z~opóźnionym doświadczeniem w~grze, niż zostać całkowicie znokautowanym. Było nie do pomyślenia, że każdy serwer na tej planecie nagle zyskałby wzrost liczby graczy i~osiągnąłby pojemność i~nie był w~stanie zaoferować wsparcia innym. Nie do pomyślenia, chyba że ktoś ich sabotował.

W międzyczasie Connor miał swoje kanały, statystyki kryminalistyczne, gigantyczne stogi siana i~ukryte igły. Niech inni martwią się o~przestój. Miał większe ryby do złapania.

Zanurzył się z~powrotem, pisząc coraz bardziej dopracowane scenariusze, aby spróbować złapać złoczyńców. Miał rosnącą grupę podejrzanych, którym mógł się dokładniej przyjrzeć, używając innego zestawu skryptów i~filtrów, które kreślił w~głębi umysłu. Wiedział już, jak to zrobi: zbuduje swoje pliki złych facetów, zrobi je duże i~głębokie, śledzi ich w~grze, zobaczy, kogo jeszcze znają, zdobędzie tysiące kont, a potem:

Zniszczy ich.

W ciągu jednej sekundy, jednej \textit{chwili }usunie wszystkie ich konta, sprawi, że ich złoto i~elitarne przedmioty znikną, wyrzuci wszystkie za naruszenie warunków świadczenia usług. Ta część byłaby łatwa. Warunki świadczenia usług były tak śmiesznie surowe, a jednocześnie irytująco niejasne, że samo granie w~grę nieuchronnie wiązało się z~ich naruszeniem. Wymazałby ich z~przestrzeni gier i~odesłał z~płaczem do mamusi. Myślenie o~takich rzeczach sprawiało, że czuł się jednocześnie brudny i~dobry.

Był pogrążony w~medytacji, kiedy gruba, włochata ręka sięgnęła mu przez ramię i~trzasnęła pokrywą laptopa tak mocno, że usłyszał pęknięcie ekranu, a potem ręka zmieniła kierunek i~uderzyła go tak mocno w~tył głowy, że odbił się od stołu przed nim.

Centrala Dowodzenia zamilkła, gdy Connor wyprostował się, czując, a następnie smakując krew spływającą mu z~nosa. Dzwoniło mu w~uszach. Powoli odwrócił głowę, ponieważ jego oczy nie skupiały się właściwie, a głowa miała wrażenie, jakby była ledwo przyczepiona do jego szyi. Stojąc nad nim, parskając jak lokomotywa towarowa, stał Kaden, szef operacji, z~dwudniową brodą i~pachnący zjełczałym potem.

-- Co  \ldots 

Mężczyzna ponownie cofnął swoją mięsistą pięść, zadając jej kolejny cios w~głowę Connora, który mimowolnie wzdrygnął się. Nie brał udziału w~bójce od czasów szkolnych i~nie mógł uwierzyć, że ten dorosły mężczyzna rzeczywiście uderzył go pięściami. Coś rosło w~jego klatce piersiowej, bulgocząc, kierując się w~jego ramiona i~nogi. Jego oddech był krótki, każdy wdech napełniał jego usta krwią. Serce mu waliło. Wstał gwałtownie, przewracając krzesło do tyłu i~\ldots 

Skoczył!

Odepchnął się obiema nogami, wrzucając swoje znaczne cielsko w~ogromny, wystający brzuch Kadena. Było jak piłka lekarska, twarda i~nieustępliwa, a on odbił się od niej w~chwili, gdy pięść Kadena uderzyła go ponownie, trafiając go twardym ciosem młota w~kark, który powalił go na ziemię.

Uderzył o~ziemię z~głuchym łoskotem, który poczuł w~każdej kości swojego ciała, jego głowa odbiła się od nogi od stołu. Wsadził pod siebie dłonie i~ponownie zerwał się na nogi, podchodząc do samej góry, jednocześnie wbijając kolano w~jądra Kadena, zginając grubasa. Jego dłonie już zacisnęły się w~niezgrabne pięści i~było naturalne jak cokolwiek, kiedy zaczął uderzać nimi w~głowę mężczyzny, uderzając tak mocno, że pękła mu skóra na knykciach.

Zajęło to tylko kilka sekund, a teraz zareagowała reszta Centrali Dowodzenia. Wielkie ręce chwyciły go za ramiona, talię, nogi i~odciągnęły. Naprzeciw niego czterech producentów gry również przyszpiliło Kadena, krzycząc na niego, żeby się uspokoił, po prostu uspokoił się, do diabła, dobrze?

Uspokoił się, trochę. Ktoś podał Connorowi zwitek serwetek z~pizzerii do przyciśnięcia go do nosa, a ktoś inny podał mu lodowatą puszkę coli z~ogromnej lodówki z~boku pokoju, by przycisnął ją do obolałej szyi.

-- Co do diabła jest z~tobą nie tak? -- zakrztusił się, wpatrując się w~Kadena, wciąż trzymanego przez czterech potężnych biegaczy.

-- Ty cholerny \textit{idioto}! Zwaliłeś całą tę cholerną sieć. Ty i~twoje głupie skrypty! Czy masz \textit{pojęcie}, ile nas kosztowała twoja mała wyprawa na ryby?

Gniew i~szok Connora przekształciły się w~strach.

-- O czym mówisz? 

-- Ktokolwiek napisał te przeklęte programy kryminalistyczne, nie miał \textit{pojęcia}. Tak mocno obciążyły serwery, mając pierwszeństwo przed każdą inną pracą, aż system musiał wyrzucić wszystkich graczy z~gier, aby mógł \textit{powiedzieć}, kim oni są. Powiem ci, co robili, Connor, \textit{próbowali połączyć się z~serwerem}.

Connor spojrzał na Billa, który napisał skrypty, i~zobaczył, że szef ochrony zbladł. Connor niewyraźnie pamiętał, jak mówił mu, że skrypty były eksperymentalne i~używał ich oszczędnie, ale były tak satysfakcjonujące, że dało mu taką radość siedzieć jak nagrywający anioł nad światami, jak Święty Mikołaj wykrywający każdego, kto jest niegrzeczny i~wszystkich, którzy byli mili \ldots 

Ogrom tego, co zrobił, uderzył go prawie tak mocno, jak pięść Kadena. Na kilka godzin zamknął trzy z~dwudziestu największych gospodarek świata. Coke prowadził gry, które przyniosły więcej pieniędzy niż Portugalia, Polska czy Peru. To były tylko P. Gdyby gry Coca-Coli były prawdziwymi krajami, byłby to akt wojny lub zdrady.

To była największa wpadka w~jego karierze. Jego życia. Prawdopodobnie największa wpadka \textit{w całej historii korporacji Coca Cola}.

Centrala Dowodzenia wydawała się oddalać, jakby pokój oddalał się od niego. Z oddali usłyszał, jak producenci gry syczą między sobą wyjaśnienia, wyjaśniając skalę jego wszechogarniającego legendarnego FAIL, który pokonuje świat.

Connor nigdy wcześniej nie doznał takiej porażki. Schrzanił tu i~tam po drodze. Ale on nigdy, nigdy, nigdy, nigdy \ldots 

Potrząsnął głową. Przytrzymujące go ręce rozluźniły się. Pochylił się sztywno, by podnieść laptopa. Kiedy go podniósł, posypały się kawałki plastiku i~szkła. Nie mógł spojrzeć nikomu w~oczy, gdy wyszedł z~pokoju.

Nie był pewien, jak dotarł do domu. Jego samochód stał na podjeździe, co sugerowało, że prowadził sam, ale nie pamiętał tego. I oto on siedział przy swoim stole w~jadalni -- wielkim i~zakurzonym -- jadł nad zlewem, kiedy w~ogóle zawracał sobie głowę jedzeniem w~domu, a jego telefon dzwonił z~daleka.

Z roztargnieniem poklepał się, zauważając, że trzyma kluczyki do samochodu, co potwierdza jego hipotezę, że sam pojechał do domu. Znalazł swój telefon i~odebrał.

-- Connor -- powiedział Ira -- Connor, nie wiem, jak ci to powiedzieć \ldots 

Connor chrząknął. To były słowa, których nigdy nie chciałeś usłyszeć od swojego brokera.

-- Connorze, jesteś tam?

Znowu chrząknął. Gdzieś jego mózg znajdował przestrzeń, w~której mógłby być jeszcze bardziej zaniepokojony.

-- Connor, posłuchaj. Czy ty słuchasz? Connor, to jest tak. Złoto z~Mushroom Kingdom \textit{zapada się}, spada na podłogę. Nie widać dna.

-- Och -- powiedział Connor. Wyszło z~zapierającym dech piskiem.

Pośrednik westchnął. Brzmiał na wpół histerycznie. 

-- Ale jest gorzej. Ten książę w~Dubaju? Okazuje się, że pisał papiery, których nie mógł honorować. On też jest spłukany.

-- Jest -- powiedział Connor.

 Milion kilometrów dalej wściekły goryl trzymał zęby i~uderzał owłosionymi pięściami o~wnętrze swojej czaszki, skrzecząc coś, co brzmiało tak, jak \textit{powiedziałeś, że to bez ryzyka}! 

-- Oczywiście, że tak nie powiedział. 

 Teraz makler brzmiał bardziej niż na wpół histerycznie. Zachichotał, a jego śmiech przebiegał w~górę i~w dół o~kilka oktaw, jak pijak przesuwający palcami w~górę i~w dół po klawiaturze pianina. 
 
 -- Mówił rzeczy w~stylu: ,,Mamy przejściowe problemy z~płynnością finansową, które spowodowały, że z~powodu ogólnej niestabilności na rynku przesunęliśmy niektóre zobowiązania finansowe''. Ale Connor \ldots  -- Znowu zachichotał. -- Byłem w~okolicy. Wiem, jak brzmi finansowe Gie. Książę jest spłukany.

-- Jest -- powiedział Connor. \textit{Powiedziałeś, że to bez ryzyka! Powiedziałeś, że to bez ryzyka!}

-- I jest coś jeszcze.

Connor wydał z~siebie cichy dźwięk przypominający skowyt. Broker rzucił się dalej. 

-- To mój ostatni dzień w~Paglia \& Kennedy. Właściwie to może być ostatni dzień Paglia \& Kennedy. Właśnie otrzymaliśmy nasze zawiadomienia. Paglia \& Kennedy utopili \textit{dużo }pieniędzy w~tych obligacjach i~ich instrumentach pochodnych.

-- Wszyscy pobiegli ukraść trochę materiałów biurowych, ale pomyślałem, że stanę tu na pokładzie Titanica i~wykonam kilka telefonów do moich najlepszych klientów. Włożyłem prawie wszystko w~złoto z~Mushroom Kingdom. Nie na początku, rozumiesz. Ale z~biegiem czasu, krok po kroku, zwroty były po prostu tak dobre \ldots 

-- To było wolne od ryzyka -- powiedział Connor głośniej, niż planował.

-- Tak -- powiedział Ira. -- OK, Connor, kolego, OK. Mam inne telefony do wykonania. 

 Connor wiedział, że biedak oczekiwał, że będzie mu wdzięczny. Myślał, że rewanżuje się Connorowi za kosztowanie -- ile? Sto osiemdziesiąt tysięcy? Dwieście tysięcy? Connor już nawet nie wiedział.

-- Dzięki za telefon -- powiedział. -- Dzięki, Ira. Dbaj o~siebie. -- Ledwo mógł wykrztusić słowa, ale kiedy już to zrobił, poczuł się trochę lepiej.

Odłożył słuchawkę i~rzucił ją na stół, pozwalając klekotać. Gdzieś tam światy gry Coca-Coli wracały do życia, gracze logowali się ponownie, wraz z~farmerami złota, Webbliesami, Pinkertonami i~całą załogą. Ale nie Connor. Connor żył w~takim czy innym świecie gier, odkąd miał siedem lat, a teraz był skłonny wierzyć, że nigdy więcej go nie odwiedzi.

Był całkiem pewien, że lada chwila zostanie zwolniony. A może aresztowany. I był spłukany. Gorzej niż spłukany, ostatnią rundę papierów wartościowych od Paglii i~Kennedy'ego kupił na marginesie, za pożyczone pieniądze i~był je dłużny. Chociaż z~upadkiem pośrednictwa mogą nigdy nie przyjść i~poprosić o~to.

Wziął głęboki oddech i~zamknął oczy. Jakiś zapach -- pot, który przesiąkł jego koszulę, krew, która oblepiała twarz, stęchły zapach domu -- przywołały silne wspomnienie jego miejsca w~Palo Alto, w~pobliżu kampusu Stanford, i~długiego, długiego czasu, który spędził tam, kupując wirtualne aktywa, balansując na krawędzi finansowej ruiny, a nawet głodu. I tak po prostu był wolny.

Wolny od strachu przed utratą pracy. Wolny od terroru bycia spłukanym. Wolny od wściekłości na farmerów złota. Wolny od krzyku, wzburzonego gniewu, który był w~Centrali Dowodzenia i~wreszcie wolny od jego fingerspitzengefuhl. Świat kręcił się wolny i~niekontrolowany i~nie mógł nic z~tym zrobić i~czy to nie było w~\textit{porządku}?

Była taka stara piosenka, która szła \textit{wolność to tylko kolejne słowo oznaczające, że nie ma nic do stracenia }i Connor nagle zrozumiał, co to wszystko znaczy.

Kiedy miał osiem lat, postanowił pracować nad grami wideo. To była jedna z~tych niedorzecznych dziecięcych rzeczy jak decyzja o~zostaniu astronautą, baletnicą, kowbojem lub nurkiem głębinowym. Większość dzieciaków wyrasta ze swoich marzeń, robi coś normalnego i~nudnego. Ale Connor trzymał się tego, znajdując drogę do przestrzeni gry za pomocą najdziwniejszych środków i~sam się tam uwięził. Do dzisiaj.

Teraz ośmiolatek, który wysłał go na wyprawę, w~końcu go z~niej uwolnił.

Wziął prysznic i~jeszcze trochę zmroził nos, włożył t-shirt i~luźne szorty, które kupił rok wcześniej na wakacjach na Bahamach (większość podróży spędził w~swoim pokoju, w~Internecie, logując się do przestrzeni gry, utrzymując fingerpitzengefuhl przy życiu) i~otworzył drzwi.

Na zewnątrz była Atlanta. Mieszkał w~mieście od siedmiu lat, chodził do jego kin i~jadał w~restauracjach, zabierał rodziców do miejsc turystycznych, kiedy je odwiedzali, ale tak naprawdę nigdy tu nie \textit{mieszkał}. To było tak, jakby był z~dłuższą, siedmioletnią wizytą. Kopnął w~klapki, które normalnie nosił, kiedy musiał wyjść na zewnątrz, by odebrać pocztę, i~wyszedł za drzwi.

Wszedł w~piekące popołudniowe słońce Atlanty, wdychając wilgotne powietrze, które było tak wilgotne, że wydawało się, że mogłoby się skroplić na podniebieniu i~kapać na język. Dotarł do końca podjazdu i~rozejrzał się w~górę i~w dół ulicy, na której mieszkał przez te wszystkie lata, z~jej gigantycznymi domami, rozłożystymi drzewami i~nieużywanymi obręczami do koszykówki, i~zaczął chodzić. Nikt poza pokojówkami i~ogrodnikami nie chodził nigdzie w~tej okolicy. Connor nie mógł zrozumieć dlaczego. Rozłożyste drzewa pachniały wspaniale, śpiewały ptaki, nawet ślimak przedzierał się po chodniku. W ciągu pół godziny Connor zobaczył więcej interesujących nowych rzeczy niż przez miesiąc.

Och, uczucie tego wszystkiego! Lekkość w~głowie, otwartość w~piersi. Stare bóle w~plecach i~ramionach, które były tam tak długo, że o~nich zapomniał, zniknęły, pozostawiając przyjemne uczucie tak uderzające, jak cisza po wyłączeniu sprężarki lodówki, pozostawiając nieoczekiwaną ciszę.

Pocił się swobodnie, ale nie przeszkadzało mu to. To po prostu sprawiało, że od czasu do czasu powiew wiatru był znacznie lepszy.

W końcu jego pęcherz zażądał powrotu do domu, więc ruszył z~powrotem, machając do podejrzliwych sąsiadów, którzy patrzyli na niego zza zasłon ogromnych okien salonu. Gdy otworzył drzwi, usłyszał dzwonek telefonu. Chwilowe uczucie niepokoju przesunęło się z~jego gardła do jąder, jak błyskawica, ale zmusił się do ponownego odprężenia i~skierował się do łazienki. Ktokolwiek dzwonił, zostawiał wiadomość. Tam odebrałaby je poczta głosowa. Musiał się wysikać.

Sikał.

Telefon znów zaczął dzwonić.

Poszedł do kuchni i~zaczął grzebać w~zamrażarce. Był tam bochenek ciemnego chleba, nigdy nie był w~stanie przebić się przez cały bochenek, zanim spleśniał, więc kiedyś kupił tuzin bochenków na raz i~zamroził je. Odkroił dwie kromki i~włożył je do tostera. Było masło orzechowe ze sklepu ze zdrową żywnością, chrupiące, bez żadnych dodatków. Podczas gdy chleb się opiekał, mieszał masło orzechowe nożem, mieszając olej, który unosił się na wierzchu, z~mielonymi orzeszkami ziemnymi poniżej. Miał miód, ale się skrystalizował. Żaden problem, dwadzieścia sekund w~kuchence mikrofalowej i~znowu był płynny. To, czego naprawdę chciał, to banany, ale ich nie było (znów dzwonił telefon), a był głodny i~chciał teraz kanapkę. Banany dostanie później.

Kanapka była (znów dzwonił telefon) pyszna. Potrzebował jednak świeżego chleba, dostałby trochę tam, kiedy kupował banany. Wyrzuciłby zamrożony (znowu był) chleb. Od teraz będzie jadł świeżo i~delektował się (i znowu) każdym kęsem.

Aż do momentu, w~którym jego palec nacisnął zielony przycisk, wierzył, że zamierza wyłączyć telefon. Ale jego palec nacisnął zielony przycisk, a niepokój skwierczał w~górę jego ramienia i~rozprzestrzenił się z~ramienia na całe jego ciało, gdy odległy głos w~słuchawce telefonu powiedział: 

-- Halo? Connor?

Connor patrzył, jak jego dłoń owija się wokół telefonu i~unosi go do ucha.

-- Tak? -- powiedziały jego usta starym, napiętym głosem Connora.

-- Tu Bill -- powiedział szef ochrony. -- Możesz przyjść do biura? 

Connor westchnął. 

-- Prześlę moją kartę kurierem. Możesz spakować moje biurko i~odesłać. Jeśli chcesz mnie pozwać, będziesz musiał wynająć prawnika kazać mu tu przyjechać.

Śmiech Billa był gorzki i~niewesoły. 

-- Nie pozywamy cię, Connor. Nie zwalniamy cię. Potrzebujemy twojej pomocy.

Connor przełknął ślinę. To była jedyna rzecz, której się nie spodziewał: że jego życie może wrócić i~znów go w~to wciągnąć.

 -- Co ty do cholery mówisz? 

-- Uważamy, że to twoi farmerzy złota -- powiedział Bill. -- Trzymają nas za jaja i~ściskają.

Connor przebrał się w~robocze ubranie jak skazaniec ubierający się na własne powieszenie. Modlił się, żeby jego samochód się nie zapalił, ale był to nowy samochód -- co roku kupował nowy, tak jak wszyscy w~centrum dowodzenia -- a jego silnik elektryczny zabuczał do życia, gdy patrzył na skaner siatkówki w~osłonie przeciwsłonecznej.

Znów przejechał swoją ulicą, widząc to przez przydymione szyby samochodu, zagięte szyby i~klimatyzację zagłuszającą śpiew ptaków i~odcinającą zapachy drzew i~kiwających się kwiatów. Za szybko, żeby dostrzec ślimaka lub ptaka.

Wrócił do pracy.

\bigskip
\threeast

Przyszli po Big Sister Nor, Mighty Kranga i~Justbob w~środku nocy i~tym razem przyprowadzili policję. Cała trójka obserwowała, jak policja wyłamuje drzwi w~towarzystwie pary zgryźliwych Chińczyków o~wyglądzie gangsterów z~kontynentu, takich, którzy przybyli do Singapuru na łatwych dwutygodniowych wizach turystycznych. Nor i~jej przyjaciele patrzyli, jak drzwi zostały wyłamane z~dwóch Lorongów -- bocznych uliczek, używając kamery internetowej i~przesyłając wideo na żywo do sieci Webblies i~grupy dziennikarzy, których obudzili, gdy tylko się wyrwali się ze starego miejsca, ostrzeżeni przez sympatycznego sklepikarza na szczycie Geylang Road.

Dom zastępczy nie był oczywiście ani trochę tak ładny jak ten, który opuścili, ale oboje szybko doszli do równowagi, gdy policja metodycznie rozbijała każdy mebel w~tym miejscu na drzazgi. Mighty Krang rysował na ekranie adnotacje w~czasie rzeczywistym, gdy policja pracowała, czasami wpisując w~dolarach wartość rozbijanych mebli, a czasami po prostu rysując wąsy i~opaski na oczy policji w~filmie. Kiedy Chińczycy wyjęli swoje kutasy i~zaczęli sikać na wrak, wskoczył do swojego gładzika, okrążył członków, o~których mowa, narysował wskazujące na nich strzałki i~napisał ,,MAŁE!'' w~trzech językach, zanim skończyli.

Obserwowali, jak jeden z~policjantów odebrał jego telefon, podsłuchiwali, jak powiedział: ,,Halo?'' i~,,co?'' i~,,gdzie?'', a potem ,,Tutaj?'' ,,Tutaj?'' obmacując miejsce, w~którym ściana stykała się z~sufitem, aż znalazł kamerę wideo. Wyraz jego twarzy -- mieszanka przerażenia i~wściekłości -- gdy się rozłączył, była bezcenna.

-- Bezcenna -- powiedział Mighty Krang i~zwrócił się do swoich towarzyszy, którzy byli znacznie mniej rozbawieni niż on.

-- Och, rozchmurzcie się -- powiedział. -- Nie złapali nas. Strajkujący strajkują. Bombaj i~Guangdong szaleją. New York Times wysyła nam około dziesięciu e-maili na minutę. Financial Times też. I Times of London. To tylko angielskie gazety. Niemcy, Francuzi \ldots  I The Times of India, oczywiście, mają reportera w~Dharavi, podobnie jak brukowce w~Bombaju. Mamy sześć z~dwudziestu najlepszych filmów na YouTube. Mam \ldots  -- spojrzał w~dół i~nacisnął myszką na kilka -- 82 361 e-maili od ludzi na adresy członkowskie.

Justbob spojrzał na niego zdrowym okiem. 

-- Matthew jest uwięziony w~Dafen. Czterdziestu dwóch nie żyje. Nie wiemy, gdzie są Jie i~biały chłopiec, Wei-Dong.

Big Sister Nor wyciągnęła ręce i~każda z~nich wzięła jedną z~jej. 

-- Towarzysze -- powiedziała -- towarzysze. To jest ten moment, który planowaliśmy. Zostaliśmy zranieni. Nasi przyjaciele zostali zranieni. Więcej będzie rannych, kiedy to się skończy.

-- Ale ludzie tacy jak my są ranni \textit{każdego dnia}. Dają się złapać w~maszyny, wdychamy opary trucizny, jesteśmy bici, odurzani lub gwałceni. Nie zapominajmy o~tym. Nie zapominajmy, przez co przechodzimy, przez co przeszliśmy. Będziemy walczyć w~tej bitwie ze wszystkim, co mamy, i~prawdopodobnie przegramy. Ale potem będziemy walczyć ponownie i~stracimy trochę mniej, bo ta bitwa zdobędzie wielu zwolenników. \textit{Znowu }przegramy. I \textit{znowu}. I będziemy walczyć dalej. Bo choć trudno jest wygrać walcząc, nie da się wygrać, nic nie robiąc.

Na ekranie Kranga pojawił się komunikat, przypominający mu, aby włożył nową kartę SIM do swojego telefonu komórkowego. Sekundę później ten sam alert pojawił się na ekranach Big Sister Nor i~Justboba.

Big Sister Nor uśmiechnęła się. 

-- Dobrze -- powiedziała. -- Wracamy do pracy.

Wymienili karty SIM, wyciągając nowe z~przestarzałych kopert, które nosili pod ubraniem w~pasach z~pieniędzmi. Włączyli swoje telefony. Telefony Justboba i~The Mighty Krang zadzwoniły, gdy tylko włączyły zasilanie.

Mighty Krang spojrzał na numer. 

-- To Wei-Dong -- powiedział. -- Mówiłam ci, że jest bezpieczny.

Justbob spojrzała na jej telefon. 

-- Ashok -- powiedziała.

Oboje odebrali telefony.

\bigskip
\threeast

Ashok wiedział, że ten czas nadejdzie. Miesiącami pracował niewolniczo nad modelami ekonomicznej destrukcji: ile inwestycji w~śmieciowe zabezpieczenia gier potrzeba, aby postawić prowadzących grę na pozycji totalnej bezbronności? Modelował to na tysiąc sposobów, wypróbowywał wiele zmiennych w~swoich równaniach, pocił się nad tym, obudził w~nocy, by chodzić lub jeździć na motocyklu, dopóki wątpliwości nie opuściły jego umysłu.

Gdzieś tam jakiś daleki zwolennik Big Sister Nor przekonał Mechanicznych Turków, by poszli do pracy, sprzedając jego śmieszne papiery wartościowe. Spakowanie ich było dość łatwe, było tak wiele firm, które pozwalały łączyć własne niestandardowe pakiety zabezpieczeń i~sprzedawać je, a wystarczyło tylko dowiedzieć się, która z~nich jest najbardziej niedbała w~procedurach weryfikacji i~stworzyć konta i~wymyślić za ich pośrednictwem mnóstwo wirtualnego bogactwa. Potem zalogował się do mniej niechlujnych konkurentów i~przepakował śmieci, które stworzył, tworząc coś, co wydawało się bardziej uzasadnione. Wspinając się w~górę łańcucha pokarmowego, przechodził od pakowacza do pakowacza, stale gromadząc szelak szacunku ponad swoimi finansowymi łajdakami.

Kiedy już nabyli ten błysk, brokerzy zaczęli polować na jego śmieszne pieniądze. A ponieważ Webblies kierowały sporą część bogactwa gry do leżącej pod nim puli, był w~stanie sprawić, by wszystko wydawało się, że rosło z~zawrotną prędkością -- i~tak się stało. W końcu wszyscy ci inwestorzy wymieniający instrumenty pochodne podbijali ceny za każdym razem, gdy kończyli sprzedaż.

Pewnego razu, około drugiej nad ranem, kiedy Ashok obserwował postęp wymiany, zdał sobie sprawę, że może po prostu zrezygnować z~Webblies, sprzedać ostatnią partię zabawnych pieniędzy i~przejść na emeryturę. Ale nigdy się nie skusił. Zawsze wiedział, że można się wzbogacić, depcząc ludzi wokół siebie, traktując ich jak frajerów, których można oszukać. Nie mógł tego zrobić.

Oczywiście, właśnie stał i~\textit{to robił}, ale to było inne. Jego mała gra finansowa mogłaby się dobrze zakończyć, gdyby wszystko poszło zgodnie z~planem, a teraz nadszedł czas, aby sprawdzić, czy plan pójdzie tak, jak powinien.

Justbob odebrała telefon jej łamanym angielskim, lepszym niż jej hindi, ograniczonym do rozkazów bitewnych i~wojskowych przekleństw. Powiedział jej, że musi porozmawiać z~Big Sister Nor, a ona poprosiła go, żeby chwilę poczekał, ponieważ BSN rozmawiało w~tym czasie z~kimś innym.

W tle usłyszał Big Sister Nor, rozmawiającą mieszanką chińskiego i~angielskiego, przeskakując tam i~z powrotem w~sposób, który przypominał mu jego kumpli na uniwersytecie i~sposób, w~jaki się bawili, mieszając angielskie i~hindi słowa, kalambury i~niejasno brudne frazy, które mimo to brzmiały niewinnie.

Spojrzał na zegar w~rogu ekranu. Była piąta rano i~na zewnątrz słyszał śpiew ptaków. W sąsiednim pokoju armia Mali walczyła na niestrudzonych zmianach, broniąc strajku. Spali teraz na zmianach na podłodze, a na ulicy przed wejściem kręciło się pięćdziesięciu lub sześćdziesięciu robotników hutniczych i~odzieżowych, którzy odwiedzali inne strajkujące miejsca wokół Dharavi z~kartami rejestracyjnymi, próbując zorganizować robotników w~małych, pięcio-- lub dziesięcioosobowych warsztatach w~związki.

Zdał sobie sprawę, że zasypia. Ile czasu minęło, odkąd ostatnio spał ponad godzinę? Dnie. Poderwał głowę i~zmusił się do otwarcia oczu, a przed nim stała Yasmin z~oczami szopa pracza pod hidżabem na czole. Marszczyła brwi, jej usta były otoczone głębokimi zmarszczkami zmartwienia, kolejna nad grzbietem nosa. Trzymała lathi.

-- Yasmin? -- spytał.

Przygryzła wargę. 

-- Mali nie ma -- powiedziała. -- Od godzin nikt jej nie widział. Dwanaście, może czternaście.

Zaczął coś mówić, ale wtedy Big Sister Nor odezwała się przez telefon: 

-- Ashok, przepraszam, że czekasz.

Spojrzał na Yasmin, a potem z~powrotem na swój ekran. 

-- Sekundę -- powiedział do telefonu.

-- Yasmin, prawdopodobnie poszła do domu spać \ldots 

Yasmin potrząsnęła głową z~naciskiem. Poczuł wstrząs strachu. 

-- Ashok? -- Głos Big Sister Nor w~jego uchu.

-- Wejdź -- powiedział do Yasmin -- chodź tutaj. Zamknij drzwi.

Wstał, podał krzesło Yasmin i~przysiadł obok niej, z~obcasami na ziemi. Nacisnął przycisk głośnika w~telefonie.

-- Nor -- powiedział. 

Zawsze czuł się trochę śmieszny, nazywając tę kobietę ,,Starszą Siostrą'', chociaż Webblies wydawali się rozkoszować tym w~taki sam sposób, w~jaki lubili mówić \textit{Generał Robotwallah}.

 -- Mam tu ze sobą Yasmin. Mówi mi, że Mala zaginęła, zaginęła od kilku godzin.

Nastąpiła chwilowa pauza. 

-- Ashok -- powiedział Nor -- to straszne wieści. Ale myślałam, że dzwonisz w~innej sprawie \ldots 

Spojrzał na Yasmin, której wzrok był utkwiony w~nim. Nigdy nie mówił o~pracy, którą wykonywał dla Big Sister Nor, ale wszyscy wiedzieli, że tu coś kombinuje.

-- Tak -- powiedział. -- Druga sprawa. Muszę z~tobą o~tym porozmawiać. Ale Yasmin jest tutaj i~mówi mi, że Mala zaginęła.

Big Sister Nor wydawała się słyszeć powagę w~jego głosie. Wzięła głęboki oddech i~powiedziała cierpliwym głosem: 

-- Znasz Dharavi lepiej niż ja. Jak myślisz, co się stało?

Skinął głową Yasmin. 

-- Myślę, że Bannerjee ją ma -- powiedziała. -- Myślę, że ją skrzywdzi, jeśli już tego nie zrobił.

Z telefonu dobiegł głos The Mighty Krang. 

-- Mam numer telefonu Bannerjee -- powiedział. -- Od jednego z~naszych ludzi w~Guzhen. Wysłał nam e-mailem listę wszystkich osób z~książki adresowej swojego szefa.

Ashok stwierdził, że jego dłonie są zaciśnięte w~pięści. Poznał Bannerjee tylko raz, ale to wystarczyło. Mężczyzna wyglądał, jakby był zdolny do wszystkiego, był jednym z~tych kosmitów, którzy potrafili patrzeć na bliźniego jako na okazję do zarobienia pieniędzy. Oczy Yasmin były szeroko otwarte.

-- Chcesz do niego zadzwonić?

-- Jasne -- The Mighty Krang brzmiał spokojnie, a nawet nonszalancko, tak jak w~inspirujących filmach, które zamieszczał na forach Webblies i~na YouTube. -- Warto spróbować. Może chce ją wykupić.

-- Żartujesz? 

Wesoły ton opuścił jego głos. 

-- Nie, Yasmin, nie żartuję. Słuchaj, Webblies są potężni. Tacy ludzie jak Bannerjee to rozumieją. Kiedy już zdobyłem numer Bannerjee, użyłem go, żeby go dokładnie zbadać. Mamy nad nim pewną przewagę. To możliwe, że możemy sprawić, by zobaczył powód. A jeśli nie możemy \ldots  -- Urwał.

-- Nie jesteśmy gorsi niż wcześniej -- zakończyła Big Sister Nor.

-- Kiedy do niego zadzwonimy?

-- Och, teraz byłoby dobrze. Negocjacje są zawsze najlepsze w~późnych godzinach. Poczekaj, odszukam numer. -- Mighty Krang wpisał trochę. -- OK, zróbmy to. 

-- OK -- powiedziała Yasmin cichym głosem.

-- Dobrze -- powiedział Ashok.

-- Będę was wyciszał dla niego, ale live dla mnie. Pamiętajcie o~tym \ldots  jeśli będziecie mówić razem z~nim, usłyszę obydwa połączenia, co może mnie zdezorientować.

-- Wyciszymy nasz koniec -- powiedział Ashok. 

Zauważył, że jego bateria jest słaba, więc poszukał na biurku kabla zasilającego i~podłączył go. Potem wyciszył telefon. On i~Yasmin nieświadomie pochylili się nad nim razem, tak że mógł poczuć swój kwaśny oddech i~jej, pachnący wymiocinami. Była chora. Zamknął oczy i~poczuł, jakby na wewnętrznej stronie jego powiek był papier ścierny.

Po kilku dzwonkach zaspany głos wymamrotał ,,Zwycięstwo dla Ramy'' w~hindi, co było tradycyjnym pozdrowieniem telefonicznym. To sprawiło, że Ashok parsknął szyderczo. Człowiek taki jak Bannerjee był pobożny jak rzepa. Jak szakal.

-- Panie Bannerjee -- odezwała się Big Sister Nor w~hindi z~akcentem. -- Dzień dobry. 

-- Kto to jest? -- Przeszedł na angielski.

-- Webblies -- powiedziała Big Sister Nor.

-- Jak na Webbly -- mruknął Bannerjee, wciąż brzmiąc na wpół śpiącą -- brzmisz jak nieletnia chińska dziwka. Skąd dzwonisz, Chino-Lalko? Brat w~Hongkongu?

-- Właściwie 2500 kilometrów od HK. A ja jestem Indonezyjką.

Bannerjee znów chrząknął. 

-- Ale \textit{jesteś }dziwką, prawda?

-- Panie Bannerjee, jestem zajętą kobietą \ldots 

-- \textit{Popularna }dziwka! 

Yasmin syknęła na telefon, a Ashok dwukrotnie sprawdził, czy jest włączone wyciszenie. Było.

--  \ldots  zajętą kobietą. Zadzwoniłam, żeby złożyć ofertę. 

-- Mam wszystkie kurwy, których potrzebuję -- powiedział. -- Do widzenia. 

-- Panie Bannerjee! Dzwonię, żeby zorganizować uwolnienie Mali -- odezwała się szybko Big Sister Nor. -- I jestem pewna, że jeśli pomyślisz o~tym choć przez chwilę, zrozumiesz, że mogę ci zaoferować wiele, by zapewnić jej bezpieczny powrót.

Bannerjee powiedział: 

-- Mala zaginęła? -- w~tonie, który mógłby zdobyć medal na  igrzyskach olimpijskich w nieprzekonywaniu.

-- Przestań grać, proszę. Wiesz, że nie jesteśmy policją. Nie będziemy cię aresztować. Chcemy, żeby wróciła.

-- Jestem pewien, że tak. To urocza dziewczyna.

Yasmin chwyciła się za przeciwległe łokcie tak mocno, że jej knykcie były białe. Ashok zacisnął pięści w~materiale nogawek. Zmusił się do ich poluzowania. Ale Big Sister Nor po prostu kontynuowała, jakby nie słyszała.

-- Jestem pewna, że widziałeś, co się stało z~rynkami złota. Ceny płoną. Dzięki moim Webblies nikt nie może wydobyć złota z~farm złota. Gdybyś mógł obiecać farmerowi dostęp do jednego miejsca, bez nękania, pomyśl tylko, co możesz narzucić.

Bannerjee zachichotał. 

-- A wszystko, co muszę zrobić, to znaleźć dla ciebie Mali i~dać ci ją, a ty mi to zagwarantujesz, zgadza się?

-- Taki jest tego kształt.

-- Oczywiście dotrzymasz swojej części umowy, kiedy ją dla ciebie znajdę.

-- Oczywiście. 

Zapadła długa cisza. W końcu Big Sister Nor odezwała się ponownie.

-- Rozumiem twój sceptycyzm. Mogę dać ci słowo honoru.

Bannerjee wydał niegrzeczny dźwięk jak mokry pierdnięcie. 

-- A co powiesz na to: wyjmę złoto z~gry, a potem znajdę dla ciebie Malę.

Ashok nienawidził tej gry, w~którą grał, udając, że nie ma Mali, ale jakoś mógł ją znaleźć. Chciał przeczołgać się przez telefon i~udusić mężczyznę.

-- A co powiesz na to, że po prostu damy ci trochę złota? -- przemawiał The Mighty Krang.

-- Och, jest was więcej? Czy też jesteś indonezyjską dziwką 2500 kilometrów od Hongkongu, czy też dzwonisz z~jakiegoś innego egzotycznego miejsca? 

-- Możemy wyciągnąć złoto z~gry szybciej niż ktokolwiek, kto by cię zatrudnił. Wszyscy najlepsi farmerzy złota są w~związku. Łamistrajki, którzy pracują teraz w~warsztatach, pewnie są słabi, że prawdopodobnie zrobią błędy i~zostaną zbanowani. -- Ashokowi podobało się, że Krang też nie grał w~szyderczą grę Bannerjee.

Bannerjee prychnął. 

-- To nie jest złe -- powiedział.

-- Możemy skorzystać z~usługi depozytowej, na którą się zgodzimy. 

 Rynki złota opierały się na usługach depozytowych, godnych zaufania stronach, które trzymały złoto i~gotówkę podczas zamykania transakcji, pracując za niewielki procent.

-- A ty oddasz nam Malę?

-- Zrobiłbym wszystko, co w~mojej mocy, aby znaleźć biedną dziewczynę i~oddać ją w~twoje ręce. -- Złote, srebrne i~brązowe medale w~szlamie na 100 metrów.

Kłócili się o~cenę i~czas, Nor ostatecznie obiecała mu 300 000 kamieni runicznych Svartalfaheim i~Krang odłączył Bannerjee.

-- Wspaniale -- powiedział Ashok, próbując wymusić na głosie trochę entuzjazmu, podczas gdy w~środku drżał na myśl o~Mali w~rękach Bannerjee.

-- Bardzo dobrze -- powiedziała Yasmin.

-- Tak, tak -- powiedziała Big Sister Nor. -- A twoja drużyna zdobędzie dla nas kamienie runiczne i~jestem pewna, że zrobisz to szybko i~dobrze, ponieważ ona jest twoim generałem. Wszystkie nasze problemy powinny być tak łatwe do rozwiązania. Dobrze, Ashok, jak poradziłeś sobie ze swoim skomplikowanym problemem? 

Ashok spojrzał na Yasmin, która nie zdradzała żadnych oznak odejścia.

-- Myślę, że tam jesteśmy. Sztuczka polegała na tym, by stworzyć sytuację, w~której \textit{nie mogą }złożyć wszystkiego z~powrotem bez naszej pomocy. Nasze konta kontrolują złoto pod tak wieloma tymi papierami wartościowymi, że jeśli nas wszystkich wyrzucą, to powodują poważne awarie, zarówno w~grze, jak i~poza nią. Jednocześnie nie mogą sobie pozwolić na to, abyśmy mogli swobodnie biegać, ponieważ istnieje setki sposobów, w~jakie moglibyśmy zawiesić system, rezygnując w~ogromnych grupach naraz do powtórzenia zadania z~Mushroom Kingdom.

Zniszczenie zabezpieczeń Mushroom Kingdom było łatwe -- Mushroom Kingdom było już pełne oszustw, które krążyły pod radarem niekompetentnych ekonomistów i~zespołów bezpieczeństwa Nintendo. Ashok wykorzystał Webblies i~kilku Mechanicznych Turków, których Big Sister Nor dostarczyła dzięki swoim tajemniczym kontaktom w~środku, tworząc katalog wszystkich innych oszustw, a następnie dając im szturchnięcie tu i~tam, używając Webbliesów do produkcji złota na żądanie w~razie potrzeby.

Wszedł w~to, myśląc, że nigdy nie zdoła zmierzyć się z~ekonomią Mushroom Kingdom, wierząc, że bezpieczeństwo będzie wszechwiedzące i~wszechmocne. Ale tak naprawdę wszystko to było połączone sznurkiem i~myśleniem życzeniowym, napinającym się w~szwach, i~wystarczyło tylko trochę pchać i~ciągnąć, aby najpierw napuchło do niesłychanej wysokości, a potem eksplodowało chwalebnie.

-- Ale nie mogliśmy sobie pozwolić na powtórzenie roboty z~Mushroom Kingdom. Nie było mowy, abyśmy mogli wyciągnąć to z~nurkowania, kiedy już się zaczęło. Od początku było skazane na zagładę. Z grami Coca-Cola, jesteśmy w~stanie obiecać, że wszystko złożymy z~powrotem, jeśli zagrają z~nami w~krykieta. -- Rozmowa o~swojej pracy sprawiła, że na chwilę zapomniał o~Mali, pozwolił, by żelazne opaski wokół jego klatki piersiowej poluzowały się, tylko trochę.

-- Gdybyśmy trzymali się harmonogramu, byłoby znacznie łatwiej. Ale wiesz, gdy wszystko jest chaotyczne, musiałem się spieszyć. Od wielu godzin wyrzucam nasze rezerwy złota na rynek, na co rynek absolutnie oszalał, zwłaszcza po tym, jak mieli krach. Jak u licha udało ci się to zrobić?

Big Sister Nor parsknęła. 

-- To nie ja. Nie jesteśmy pewni, czy zostali zhakowani, czy jakaś poważna katastrofa. To \textit{było} dobrze zgrane.

-- Powiesz mi, czy to ty to \textit{spowodowałaś}?

Yasmin wyglądała na lekko zszokowaną.

-- Ashok -- powiedziała BSN z~udawaną surowością-- Mówię wszystkim wszystko, co myślę, że powinni wiedzieć, i~zwykle mówię im wszystko, co myślą, że powinni wiedzieć. Nie zajmujemy się tutaj tajemnicami.

To sprawiło, że Ashok się zatrzymał. Zawsze myślał o~operacji jako okrytej tajemnicą. Z pewnością Big Sister Nor nigdy nie podała na ochotnika żadnych szczegółów dotyczących jej kontaktu z~Mechanicznymi Turkami, ale przecież nigdy nie pytał, prawda? Nigdy też nie zapytał, czy mógłby omówić swój projekt z~armią Mali. Potrząsnął głową. Co by było, gdyby cały jego umysł tkwił w~tajemnicy?

-- Dobrze -- powiedział. -- Dobrze. Problem polega na tym: gdybym miał wystarczająco dużo czasu, gdybym miał czas, który zaplanowaliśmy, byłbym w~stanie zabrać Svartalfaheim na skraj upadku, a następnie albo go uratować, albo pozwolić się rozpaść. Wszystko sprowadza się do tego, ile złota mamy w~naszych rezerwach i~jak dużą część handlu kontrolujemy.

-- Ale musiałem spieszyć się z~harmonogramem, co oznacza, że nie mogę dać wam obu. Mogę doprowadzić gospodarkę na skraj ruiny, ale kiedy to zrobię, muszę z~góry wiedzieć, czy zamierzamy pozwolić jej polecieć, czy pozwolimy jej się zregenerować. Później nie mogę decydować. -- Przełknął. -- Myślę, że to oznacza, że musimy ją zniszczyć. Nadal mam Zombie Mecha i~Clankers w~drodze. Możemy pokazać im nasze siły, likwidując Svartalfaheim, a następnie grożąc, że zniszczymy pozostałe dwie.

-- Dlaczego chcesz to zrobić w~ten sposób? 

Pokręcił głową, zdając sobie sprawę, że go nie widzi. 

-- Słuchaj, oni ci się nie poddadzą. Wejdziesz tam i~zaczniesz wydawać im rozkazy, a oni założą, że jesteś jakąś śmieszną oszustką z~trzeciego świata. Powiedzą, żebyś spadała. Jeśli grozisz i~nie możesz tego zrobić dobrze, to będzie ostatni raz, kiedy się od nich usłyszysz. Po tym nigdy nie będą cię traktować poważnie.

Big Sister Nor cmoknęła językiem. 

-- Czy tak łatwo nas odrzucić? 

-- Tak -- powiedział Ashok. -- \textit{Wiem}, co potrafią Webblies. Ale oni nie wiedzą. I nie zrobią tego, dopóki im nie pokażemy.

-- Mamy do tego Mushroom Kingdom.

To go powstrzymało. 

-- Tak, to prawda, oczywiście. Ale to było takie \textit{proste} \ldots 

-- Oni tego nie wiedzą. Nic o~nas nie wiedzą, jak wskazujesz. Więc tak, może założą, że jesteśmy słabi, a może założą, że jesteśmy silni. Ale jedno wiem, jeśli dadzą nam to, czego chcemy, a \textit{potem }zniszczymy ich grę, nigdy więcej nam nie zaufają.

-- Więc mówisz, że chcesz, żebym to wszystko zaaranżował, żebyśmy nie mogli spełnić naszej groźby?

-- Jeśli będziemy musieli wybierać \ldots 

-- Musimy.

-- W takim razie tak, właśnie tego chcę, Ashok. Muszę się tylko upewnić, że cokolwiek się stanie, nie spełnimy naszej groźby.

-- Dobrze -- powiedział Ashok. -- Mogę to zrobić. 

-- Dobrze. I Ashok?

-- Tak?

-- Potrzebuję, żebyś z~nimi porozmawiał -- powiedziała. -- Z kimkolwiek, kto będzie z~nami porozmawiać. Ja też oczywiście będę pod telefonem. Ale musisz z~nimi porozmawiać, wyjaśnić im, co zrobiliśmy i~co możemy zrobić.

Ashok przełknął ślinę. 

-- Nie jestem dobry w~tego rodzaju rozmowach \ldots 

Yasmin wydała niegrzeczny dźwięk. 

-- Nie słuchaj go -- powiedziała. -- Namówiłeś hutników i~szwaczki, by przybyli do Dharavi!

-- Tak -- powiedział. -- Nie sądziłem, że to zadziała, nigdy wcześniej nie słuchali. Ale kiedy wyjaśniłem, w~jakiej sytuacji się znaleźliście, bandyci, przemoc, powiedzieli im, że wszyscy z~Dharavi będą wiedzieć, jeśli upadną \ldots  

-- Kiedy naprawdę w~to wierzyłeś -- powiedziała Big Sister Nor. -- To jest różnica. Słyszałam, jak mówiłeś o~rzeczach, które kochasz, Ashok. Jesteś bardzo przekonujący, jeśli chodzi o~takie rzeczy. Różnica między wszystkimi rozmowami, które z~nimi prowadziłeś wcześniej, a tą ostatnią polega na tym, że do nich przyszedłeś jako Webblies ostatnim razem, a nie jako ktoś, kto grał w~grę, aby poczuć, że robi coś ważnego. 

Krytyka zbiła go z~tropu i~przeszyła. Na początku \textit{grał }w grę, zachwycony własną sprytną wizją dzieciaków z~całego świata biegających w~kółko wokół zmęczonych starych związków, z~którymi kręcił się przez całe życie. Ale teraz to już nie była gra. A raczej \textit{była }to gra, ale traktował ją śmiertelnie poważnie.

-- Dobrze -- powiedział. -- Porozmawiam z~nimi.

\bigskip
\threeast

Teraz przyszła kolej na Jie, aby obejrzeć Wei-Dong, gdy wściekle pisał na klawiaturze, docierając do setek Mechanicznych Turków, którzy mówili: 

-- Tak, tak, jesteśmy po twojej stronie; tak, jesteśmy zmęczeni marnym wynagrodzeniem i~groźbami wyrzucenia z~pracy nad naszymi głowami. -- Wyciągnął do nich rękę i~powiedział im wszystkim:

\textit{Teraz}

Teraz się zaczyna, teraz jesteśmy gotowi, teraz ruszamy. Wysłał im linki do filmów na YouTube z~protestów w~Chinach, linii pikiet w~Indiach, robotników, którzy zaczęli odchodzić z~pracy w~Indonezji, Wietnamie i~Kambodży, mówiąc: 

-- My też, my wszyscy razem, my też.

Tyle że to nie działało tak, jak powinno. Mechaniczni Turcy byli na tyle szczęśliwi, że zasiali trochę dezinformacji, przekazali jakieś dziwnie brzmiące wskazówki giełdowe lub odwrócili wzrok, gdy Webblies walczyli z~Pinkertonami, ale nie chcieli iść do Coca-Coli i~mówić: ,,Żądamy, chcemy, wszyscy jesteśmy jednym''. Już od pisania na maszynie wyczuwał ich strach, przerażenie, że w~przyszłym miesiącu mogą znaleźć się bez pracy, że mogą być jedynymi, którzy staną.

Ale nie wszyscy. Najpierw jeden, potem pięciu, potem pięćdziesięciu, a w~końcu ponad setka Turków była z~nim, gotowa umieścić swoje nazwiska na liście płacących składki Webblies, którzy jako grupa chcieli targować się z~Colą o~lepszą ofertę. To tylko 20 procent tego, o~co się targował, ale nadal stanowili 35 z~pięćdziesięciu najlepszych wykonawców w~rankingach Webblies.

Prowadził konto dla Jie, mrucząc do niej po chińsku między wiadomościami i~szybkimi rozmowami głosowymi.

-- Co teraz? -- powiedziała. Utknęła w~kącie pokoju, spoczywając na swetrze, który rozłożyła na brudnym materacu, z~ledwo otwartymi oczami.

-- Teraz dzwonię do Coca Coli -- powiedział.

 Kilkanaście razy rozmawiał o~tym z~Big Sister Nor, powtarzając plan, a nawet rozgrywając z~Mighty Krang, który grał kierownictwo po drugiej stronie. Ale to nie znaczyło, że był spokojny, wcale nie, ale czuł, że w~każdej chwili może zwymiotować.

-- Jak to ma działać?

Zamknął oczy, które płonęły z~wyczerpania i~zaschniętych łez. 

-- Czy jesteś głodna? 

Skinęła głową. 

-- Myślałam o~tym, żeby pójść na górę po pierogi -- powiedziała.

-- Przyniesiesz mi trochę? 

Wstała i~podeszła chwiejnie do drzwi. Wyciągnęła z~torebki puderniczkę i~spojrzała na siebie, skrzywiła się, a potem powiedziała: 

-- Herbata?

Od lat pił herbatę, ale w~tej chwili potrzebował kawy, bez względu na to, jak bardzo poczuł się amerykański.

 -- Kawa -- powiedział. -- Dwie kawy.

Uśmiechnęła się smutnym uśmiechem. 

-- Oczywiście. Przyniosę też strzykawkę.

Ale był już z~powrotem przy komputerze, wkręcał pożyczoną słuchawkę i~wybierał numer alarmowy tylko dla pracowników.

-- Wsparcie drugiego poziomu Coca Cola Games, mówi Brianna -- głos był płaski, amerykański, znudzony, kobiecy, latynoski.

-- Muszę porozmawiać z~kimś w~operacjach -- powiedział. -- To jest Leonard Goldberg, numer Turka 4446E764.

-- Cześć, Leonardzie. Czy mogę dostać piątą literę twojego kodu bezpieczeństwa?

Musiał przez chwilę mocno się zastanowić. Podobnie jak imię Leonard Goldberg, jak całe jego amerykańskie życie, kod bezpieczeństwa, którego używał do komunikowania się ze swoimi pracodawcami, wydawał się w~odległej baśniowej krainie. 

-- K jak kilogram, powiedział. -- Nie, czekaj, Z jak Zulu. 

-- A druga litera?

-- A jak alfa.

-- OK, Leonardzie, co mogę dla ciebie zrobić? 

-- Muszę porozmawiać z~kimś w~operacjach -- powiedział. -- Poziom czwarty, proszę. 

-- O czym musisz rozmawiać z~operacjami, proszę? 

Słyszał, jak klika w~ekran, sprawdzając procedury eskalacji. Technicznie rzecz biorąc, nie było możliwe przejście ze wsparcia poziomu drugiego do poziomu czwartego bez przejścia przez poziom trzeci. Ale cały podręcznik eskalacji był dostępny na prywatnych forach dyskusyjnych na nieoficjalnych grupach tureckich, jeśli wiedziałeś, gdzie ich szukać.

-- Ja, hm, myślę, że znalazłem kogoś, kto był, jakby, pedofilem? Jakby próbował nakłonić jakieś dzieci, by dały mu swoje adresy RL? 

Dzieciaki, mafia, terroryści lub piraci, cztery bilety ekspresowe do czwartego poziomu wsparcia. Cokolwiek, co oznaczało wezwanie policji federalnej lub międzynarodowej. Doszedł do wniosku, że potencjalny pedofil miałby akurat odpowiednią ilość wstrętu, by doprowadzić go do eskalacji bez wysyłania telefonu bezpośrednio do glin.

Brianna coś napisała, coś przeczytała, wymamrotała: 

-- Chwileczkę, kochanie -- przeczytała trochę więcej. 

-- OK, to poziom czwarty. -- Przełączyła go na poczekalnię.

Jie wróciła ze styropianową tacką wypełnioną parującymi kluskami i~butelką gorącego wietnamskiego sosu i~parą pałeczek. Podniosła jedną kluskę, dmuchnęła, zanurzyła w~sosie i~podała mu. Włożył go do ust i~przeżuł, jednocześnie wypuszczając powietrze, próbując schłodzić parzoną wieprzowinę w~środku. Wymienili uśmiech, po czym połączenie rozpoczęło się ponownie.

-- Witam, Coca Cola Games, operacje poziomu czwartego, mówi Gordon, proszę o~imię.

Leonard ponownie przeszedł procedurę uwierzytelniania z~Gordonem, tym razem jego hasło przyszło mu łatwiej.

-- W porządku, Leonardzie, słyszałem, że znalazłeś pedofila? Chwileczkę, kiedy przeglądam historię twoich kontaktów \ldots 

-- Nie kłopocz się -- powiedział Wei-Dong, a jego puls przyspieszył tak szybko, że miał wrażenie, że zaraz eksploduje. -- Wymyśliłem to. 

-- Czy ty. -- To nie było pytanie.

-- Muszę porozmawiać z~Centralą Dowodzenia -- powiedział. -- To pilne.

-- Rozumiem. 

Wei-Dong czekał. Ten Gordon miał być zły lub sarkastyczny, a nie cichy. Przerwa przeciągnęła się, aż poczuł, że musi ją wypełnić. 

-- Chodzi o~Webblies, mam wiadomość dla centrali.

-- Aha.

Och, na litość boską. 

-- Gordon, słuchaj. Wiem, że myślisz, że jestem tylko dzieckiem i~prawdopodobnie myślisz, że jestem pełen bzdur, ale \textit{muszę teraz porozmawiać z~Centrum Dowodzenia}. Obiecuję ci, jeśli mnie z~nimi nie połączysz, pożałujesz. 

-- Będę, prawda? Cóż, posłuchaj, Leonardzie, przyglądałem się twojej historii interakcji i~na pewno wydajesz się wydajnym pracownikiem, więc potraktuję cię łagodnie. Nie \textit{możesz }rozmawiać z~centrum dowodzenia. Kropka. Powiedz mi, czego chcesz, a dopilnuję, żeby ktoś do ciebie wrócił.

\textit{To} było coś, na co Wei-Dong był gotów. 

-- Gordon, proszę, przekaż, co następuje do Centrali. Masz długopis?

-- Och, to \textit{wszystko }jest nagrywane. -- To był sarkazm, na który czekał. Wchodził mu pod skórę. Prawidłowo.

-- Powiedz im, że reprezentuję Industrial Workers of the World Wide Web, Komórka 56, i~że musimy natychmiast porozmawiać z~głównym ekonomistą Coca Cola Games, aby zapobiec załamaniu się Mushroom Kingdom na skalę katastrofy. Powiedz im, że mają dwie godziny na działanie, zanim nastąpi katastrofa. Rozumiesz?

-- Co? Żartujesz \ldots 

-- Mówię poważnie. Poczekam, a ty im powiesz. -- Wyciszył połączenie i~natychmiast zadzwonił z~powrotem do Singapuru i~powiedział Justbob, co się stało. Zapewniła go, że jak najszybciej sprowadzą swojego ekonomistę na linię i~wstrzymają go. Połączył oba połączenia w~słuchawce, ale odizolował je, aby nie mogły go słyszeć, po czym powiedział Jie, co się właśnie wydarzyło.

-- Kiedy mogę przeprowadzić z~tobą wywiad na ten temat dla audycji radiowej?

Przełknął. 

-- Myślę, że może nigdy. Część tej historii prawdopodobnie nigdy nie zostanie opowiedziana publicznie. Zapytamy BSN, dobrze?

Zrobiła minę, ale skinęła głową. A teraz był Gordon.

-- Leonard, jesteś tam, kolego? 

-- Jestem tutaj -- powiedział.

-- Ostatnio logujesz się z~wielu serwerów proxy. Gdzie dokładnie się znajdujesz? Mamy cię w~LA.

-- Nie jestem w~LA -- powiedział Wei-Dong z~uśmiechem. -- Jestem trochę dalej. Nie musisz wiedzieć gdzie. Gordon, jak ci idzie z~Centralą Dowodzenia? Czas się marnuje. -- Utrzymuj ciśnienie, to była krytyczna część planu. Nie dawaj im czasu na myślenie. Spraw, by biegały jak bezgłowe kurczaki.

-- Pracuję nad tym -- powiedział Gordon. Głośno przełknął ślinę. -- Słuchaj, nie mówisz poważnie, prawda?

-- Widziałeś, co się stało z~Mushroom Kingdom, prawda? 

-- Widziałem. 

-- W porządku -- powiedział Wei-Dong. Ostrzeżono go, by osobiście nie przyznawał się do jakichkolwiek wykroczeń.

-- Mówisz poważnie? 

-- Wiesz, minęło już 15 minut.

Kolejna przełknięcie. 

-- Zaraz wracam. 

Wcięła się nowa linia, inny hałas w~tle, chaotyczny, dużo gadania. Gordon prawdopodobnie był telepracownikiem siedzącym w~bieliźnie w~swoim salonie. To było inne. To był pokój pełen wściekłych, kłócących się ludzi, którzy pisali na klawiaturach jak karabiny maszynowe.

-- To jest William Vaughan, szef ochrony w~Coca Cola Games. Witaj Leonard.

-- Witam, panie Vaughan. -- powiedział Leonard. Bądź uprzejmy. To też była część planu. Prawdziwi operatorzy byli dorośli, uprzejmi, rzeczowi. -- Czy mogę porozmawiać z~Connorem Prikkelem, proszę? 

 Nazwisko Prikkela łatwo było wygooglować. Wei-Dong spędził trochę czasu, oglądając filmy z~mężczyzną na konferencjach. Wydawał się niezręcznym, supermózgiem typem akademickim, który przybrał na wadze. Wpisał szybką wiadomość jedną ręką do Justboba: \textit{Mam cntr, gdzie}? 

-- Pan Prikkel jest poza biurem. Poproszono mnie, abym w~jego zastępstwie porozmawiał z~tobą.

Na to też się przygotował. 

-- Obawiam się, że muszę osobiście porozmawiać z~Connorem Prikkelem.

-- To niemożliwe -- powiedział Vaughan, brzmiąc, jakby ledwo trzymał panowanie nad sobą.

-- Panie Vaughan -- powiedział Wei-Dong. Od tygodni nie mówił tak dużo po angielsku. To było dziwne. Zaczął myśleć po chińsku, śnić w~nim. -- Nie wiem, czy \ldots  Gordon powiedział ci to, co mu powiedziałem \ldots 

-- Tak, zrobił. Dlatego teraz ze mną rozmawiasz.

-- Pan Prikkel ma kwalifikacje do oceny tego, co mam mu do powiedzenia. Nie mam kwalifikacji, aby to zrozumieć. I bez obrazy, nie sądzę, że ty masz.

-- Sam to ocenię.

Justbob odesłał mu wiadomość zwrotną: \textit{5 min.}

-- Mam lepszy pomysł -- powiedział Wei-Dong. -- Wezwij pana Prikkela i~oddzwoń do mnie. Zostawię ci identyfikator czatu głosowego. Możesz podsłuchiwać rozmowę.

-- A może po prostu wyśledzę, skąd do \textit{nas }dzwonisz, a my zadzwonimy na policję? Leonard, dzieciaku, pracujesz na granicy moich nerwów i~zaraz się stracę. Uczciwe ostrzeżenie.

Wei-Dong cmoknął. Zaczynało mu się to podobać. 

-- Panie Vaughan, chodzi o~to. Za \ldots  -- spojrzał na zegarek -- za około dziesięć minut zobaczysz totalny chaos na swoich rynkach złota. Wszystkie te kontrakty, które Coke Games podpisała na futures złota, zaczną ześlizgiwać się w~niepamięć. Możesz spędzić następne dziesięć minut, próbując mnie namierzyć, ale mnie nie znajdziesz, a nawet jeśli to zrobisz, nie będziesz w~stanie nic z~tym zrobić, ponieważ jestem ocean od najbliższej policji, która cię posłucha. -- Facet od ochrony zaczął dusić się w~odpowiedzi, ale Wei-Dong mówił dalej. -- Wolałbym \textit{nie }niszczyć tej gry. Uwielbiam ją. Uwielbiam grać we wszystkie te gry. Masz tam moje dane, wiesz o~tym. Wszyscy tak czujemy, wszyscy Webblies. To, gdzie codziennie chodzimy do pracy. \textit{Chcemy}, żeby się udało. Ale chcemy, aby stało się to na warunkach, które są dla nas uczciwe. Więc uwierz mi, kiedy powiem ci, że wzywam do zawarcia umowy, na którą cię stać, z~którą możemy żyć i~która uratuje grę i~przywróci wszystko do normy do końca dnia. -- Spojrzał na zegar znowu, wykonał jakąś mentalną arytmetykę. -- To znaczy do jutra rano, do twojego czasu.

Niemal słyszał, jak biegi obracają się w~głowie Vaughana. 

-- Jesteś gdzieś w~Azji?

-- Czy to jedyna rzecz, którą z~tego zrozumiałeś?

Wydał małe pojednawcze parsknięcie.

 -- Jesteś daleko od domu, dzieciaku. Dziesięć minut, co?

Wei-Dong powiedział: 

-- Osiem, teraz. Bierz albo nie.

-- To całkiem imponujące prognozy gospodarcze.

-- Kiedy masz 400 000 farmerów złota pracujących z~kilkoma tysiącami Mechanicznych Turków, możesz robić naprawdę imponujące rzeczy. -- Wszystkie liczby były zawyżone. Ale Vaughan założyłby, że tak. Gdyby Wei-Dong podał mu prawdziwe liczby, nie doceniłby ich siły. Podobało mu się, jak to szło.

\textit{jeszcze 2 min} od Justbob.

-- OK, Vaughan, oto jak pan Prikkel może się ze mną skontaktować. Raczej wcześniej niż później. 

Powiedział identyfikator i~usługę, która została wyprowadzona ze Specjalnej Strefy Ekonomicznej Mangalore. Rejestracja była dość niezawodna i~łatwa, a ponadto chaty obsługiwały silne krypto i~nie rejestrowały połączeń. Słyszał, że są ulubieńcem dyplomatów z~biednych krajów, które nie mogą prowadzić własnych serwerów.

-- Czekaj \ldots  

-- Zadzwoń! -- powiedział i~jeszcze raz podał mu szczegóły.

\textit{Oddzwonią do mnie}, napisał do Justbob. \textit{Naszego faceta tam nie było}.

Justbob zadzwoniła do niego od razu i~usłyszał, że The Mighty Krang i~Big Sister Nor prowadzą kolejną rozmowę w~tle. 

-- Odłożyłeś słuchawkę? 

-- To nie był właściwy facet. Myślę, że wyjechał, może na wakacjach czy coś. Znajdą go telefonicznie. Bez nerwów. 

Ale Justbob brzmiała zmartwiona, a on tego nie lubił. Wzruszył w~myślach ramionami. Zrobił, co mógł, kierując się swoim najlepszym osądem. Strzelano do niego, widział śmierć swojego przyjaciela. Przemycił się przez pół świata. Zasłużył sobie na pewną autonomię.

Zjadł trochę zimnych teraz pierogów i~starał się nie martwić, gdy czas się przedłużył. Dziesięć minut, piętnaście minut. Justbob wysyłała coraz więcej niecierpliwych smsów. Jie zasnęła na obrzydliwym materacu, sweter rozciągnął pod głową, twarz miała dziewczęcą i~smutną w~spoczynku.

Wtedy zadzwonił jego komputer.

-- Halo? -- smsując \textit{telefon}.

-- Tu Connor Prikkel. Rozumiem, że chciałeś ze mną porozmawiać?

\textit{Teraz}, napisał i~kliknął przycisk, który dołączył Justbob i~ekonomistę do rozmowy.

\bigskip
\threeast

Nikt w~Centrali Dowodzenia nie spojrzałby Connorowi w~oczy, kiedy wracał do biura ze spuchniętym nosem i~czerwonymi i~podpuchniętymi oczami. Wziął zapasowy komputer z~półek przy drzwiach -- rozwalone laptopy nie były czymś niespotykanym w~środowisku wysokiego napięcia Centrum Dowodzenia -- i~podłączył go do zasilania i~włączył.

-- Rynki szaleją -- powiedział cicho Bill, podczas gdy wokół nich mieszkańcy Centrum Dowodzenia, z~wyjątkiem Kadena, którego usunięto dla jego własnego dobra, udawali, że nie słuchają. -- Ogromne ilości złota trafiły na rynek w~ciągu ostatnich dziesięciu minut, a cena spada.

Connor skinął głową. 

-- Oczywiście, nasza normalna polityka monetarna musiała zakładać, że pewna ilość złota wpłynie do systemu od tych postaci. Kiedy zatrzymali przepływ kilka tygodni temu, musieliśmy podnieść produkcję, aby utrzymać inflację na niskim poziomie. Założyłem, że byli zbyt zajęci walką, by wydobywać więcej złota, ale wygląda na to, że spędzili ten czas na gromadzeniu swoich rezerw. Teraz gdy je zrzucają \ldots 

-- Czy możesz zrobić coś na ten temat? 

Connor myślał. Cały spokój i~spokój, które osiągnął zaledwie godzinę temu, kiedy był mężczyzną, który nie miał nic do stracenia, topniał. Miał dziwne wrażenie, że jego mięśnie wracają do swoich zwyczajowych, zawęźlonych stanów. Ale ogarnęła go nowa jasność. Myślał o~Webblies jako o~paczce dzieciaków z~gangu, toczącej wojnę gangów ze swoimi byłymi szefami. Ten biznes był jednak wyrafinowany poza wszystko, co niektórzy gangsterzy podnieśliby. Był to akt wyrafinowanego sabotażu ekonomicznego.

-- Lepiej porozmawiam z~tym dzieciakiem -- powiedział, szybko przeglądając dane, ustawiając kanały, czując powrót fingerspitzengefuhl.

Bill skrzywił się. 

-- Myślisz, że są prawdziwi?

-- Myślę, że nie możemy sobie pozwolić na założenie, że tak nie jest. -- Głos należał do kogoś innego. Rozpoznał to: głos człowieka firmy, który prowadzi interesy firmy.

Kilka minut później powiedział: 

-- Tu Connor Prikkel. Rozumiem, że chciałeś ze mną porozmawiać?

-- Panie Prikkel, bardzo dobrze z~panem rozmawiać. -- Głos miał ciężki indyjski akcent, a tło było doprawione charakterystycznymi dźwiękami graczy podczas ich gier, strzelających, krzyczących.

Bill, nasłuchując przez własną słuchawkę, potrząsnął głową. 

-- To nie ten dzieciak.

-- Też tu jestem. -- Ten głos był młody, niewątpliwie amerykański. Kiedy się włączył, zmieniło się tło, żadnych graczy, żadnych krzyków. Ci dwaj byli w~różnych pokojach. Miał intuicję, że mogą być w~różnych \textit{krajach}, i~pamiętał wszystkie bitwy, które szpiegował, w~których strony były z~całej Azji, a nawet Europy Wschodniej, Ameryki Południowej i~Afryki.

-- Panie Prikkel \ldots  doktorze Prikkel -- Connor stłumił śmiech. Doktorat był czysto honorowy i~nigdy go nie używał. -- Nazywam się Ashok Balgangadhar Tilak. Pozwólcie, że zacznę od stwierdzenia, że po przeczytaniu Twoich publikacji i~obejrzeniu dziesiątek prezentacji uważam cię za jednego z~wielkich myślicieli ekonomicznych naszych czasów.

-- Dziękuję, panie Tilak -- powiedział Connor. -- Ale  \ldots 

-- Więc to, co mam do powiedzenia, jest nieco bezczelne. Niemniej jednak powiem to: jesteśmy właścicielami twoich gier. Kontrolujemy aktywa bazowe, na które zapisano masę krytyczną papierów wartościowych; ponadto kontrolujemy znaczną liczbę tych papierów wartościowych i~możemy je sprzedać według własnego uznania, poprzez bardzo dużą liczbę fałszywych kont. Wreszcie, mamy zlecenia dla wielu poręczeń, których użyłeś do zabezpieczenia tej transakcji, zlecenia, które zostaną automatycznie wykonane, gdy będziesz starać się unosić więcej, aby wchłonąć nadwyżkę.

Connor pisał wściekle. 

-- Nie oczekujesz, że uwierzę ci na słowo?

-- Naturalnie nie. Spodziewam się, że spojrzysz na przykład z~Mushroom Kingdom. I na zamieszanie w~Wojownikach Svartalfaheim. W takim razie sugerowałbym, abyś uważnie sprawdził księgowość pod kątem Zombie Mecha i~Clankers. 

-- Sprawdzę. 

Znowu głos tego towarzystwa, z~tak daleka. Kanały to jednak potwierdzały, wolumen obrotu był szalony, ale pod tym wszystkim było poczucie \textit{kierunkowości}, jakby ktoś to wszystko robił.

-- Bardzo dobrze

-- A teraz przypuszczam, że coś tu się szykuje. Chyba szantaż. Gotówka.

-- Nic w~tym rodzaju -- powiedział Hindus z~oburzeniem. -- Wszystko, czego szukamy, to pokój.

-- Pokój. 

-- Dokładnie. Mogę cofnąć wszystko, co zrobiliśmy, ponownie połączyć rynki, zatrzymać krwawienie, bardzo ostrożnie i~bardzo delikatnie odkręcając transakcje, współpracując z~tobą, aby uzyskać miękkie lądowanie dla wszystkich. Rynki opadną, ale odbudują się, zwłaszcza po ogłoszeniu.

-- Ogłoszeniu, że zawarliśmy z~wami pokój.

-- O tak -- powiedział Ashok. -- Oczywiście. Twoi pracodawcy oczekują, że będziesz mógł prowadzić swoją gospodarkę jak zabawkowy zestaw pociągów na zgrabnych torach. Ale my wiemy lepiej. Uprawianie złota jest nieuniknioną konsekwencją twojego rynku, a to wypycha pociąg z~torów. Ale wyobraź sobie: co by było, gdyby twój pracodawca uznał uprawę złota jako praktykę, pozwalającą naszym pracownikom na uczestnictwo jako legalni aktorzy w~dużej i~złożonej gospodarce. Nasze wymiany stałyby się legalne, gdzie mógłbyś je monitorować, a my spotykalibyśmy się regularnie z~wami, aby omawiać obawy naszych członków, a ty opowiadałbyś nam o~obawach twoich pracodawców. Oczywiście nadal byliby podziemni handlarze, ale zostaliby zepchnięci na margines. Każdy porządny farmer na świecie chce dołączyć do Webblies, bo reprezentujemy najlepszych graczy i~wszyscy o~tym wiedzą. I będziemy na każdej farmie niezrzeszonej w~każdej grze, rozmawiając z~pracownikami o~umowie, którą dostaną, jeśli zwiążą się z~nami.

-- A wszystko, co musimy zrobić, to \ldots  co?

-- Współpracować. Złoto Związku, które pochodzi z~gier Coca-Coli, będzie legalne i~można je swobodnie wykorzystywać. Będziemy mieli spółdzielnię, która kupuje i~sprzedaje, tak jak dzisiejsze rynki giełdowe, ale wszystko będzie jawnie zarządzane przez wybranych menedżerów, którzy będą podlegać odwołaniu, jeśli będą się źle zachowywać. 

-- Więc zastępujemy jeden kartel innym?

-- Dr Prikkel, nigdy bym o~to nie prosił. Nie, oczywiście, że nie. Nie sprzeciwiamy się innym uzwiązkowionym operacjom w~przestrzeni. Mam tu kolegów ze Związku Pracowników Transportu i~Doków, którzy są zainteresowani w~zorganizowaniu niektórych z~tych pracowników. Niech będzie tyle wymiany złota, ile rynek może znieść, wszystkie poświadczone przez was, wszystkie kierowane przez pracowników, którzy je tworzą.

-- A co z~\textit{graczami}, panie Tilak? Czy mają w~tym coś do powiedzenia?

-- Och, myślę, że gracze już się wypowiedzieli. W końcu, jak myślisz, kto \textit{kupuje }to całe złoto?

-- I oczekujesz, że wszystko to wydarzy się w~godzinę?

Włamał się amerykański dzieciak. 

-- Teraz 45 minut.

-- Oczywiście, że nie. Dziś wszystko, czego szukamy, to \textit{zasadniczo }porozumienia. Oczywiście jest to coś, co będzie musiał zatwierdzić zarząd Coca Cola Games. Mamy jednak wrażenie, że zarząd prawdopodobnie zapłaci baczną uwagę na wszelkie zalecenia przedstawione przez jej głównego ekonomistę, zwłaszcza te oparte na twojej pozycji.

Connor odkrył, że się uśmiecha. Te dzieciaki -- nie tylko dzieciaki, przypomniał sobie -- są odważne. Co więcej, byli \textit{graczami}, co zdecydowanie \textit{nie }było prawdą w~przypadku zarządu CCG, którzy byli tak nudną grupą potężnych kapitanów przemysłu, jak tylko można było znaleźć. 

-- Czy to wszystko? 

-- Nie. -- To znowu był amerykański dzieciak. Zajrzał do swoich notatek. Leonarda Goldberga. W Los Angeles. Tyle że Bill był prawie pewien, że ten dzieciak był gdzieś w~Azji. Podejrzewał, że jest tam cała historia do opowiedzenia.

-- Cześć, Leonardzie.

-- Cześć, Connor. Wysyłam ci teraz listę nazwisk. 

-- Widzę. -- Wiadomość pojawiła się na jego koncie publicznym, tym, które stażysta zwykle filtrował, zanim ją zobaczył. Chwycił je, zobaczył, że został zaszyfrowany jego kluczem publicznym, odszyfrował go. Była to lista nazwisk z~numerami obok nich. -- Dobra, dawaj. 

-- To imiona Turków, którzy dołączyli do Webblies.

-- Masz Turków, którzy chcą dorabiać jako farmerzy złota?

-- Nie. -- powiedział chłopiec, mówiąc jak do idioty. -- Mam Turków, którzy chcą wstąpić do związku.

-- Webblies.

-- Webblies.

Connor prychnął. 

-- Rozumiem. A czy ten związek jest certyfikowany zgodnie z~amerykańskim prawem pracy? Czy brałeś pod uwagę fakt, że wszyscy jesteście niezależnymi wykonawcami, a nie pracownikami?

Chłopak wtrącił się. 

-- Tak, tak, to wszystko. Ale to są twoi najlepsi Turcy, są Webblies i~wszyscy jesteśmy w~tym razem.

-- Wiesz, nigdy na to nie pójdą.

-- Twoi kierowcy są uzwiązkowieni. Twoi \textit{dozorcy }są uzwiązkowieni. Teraz twoi Mechaniczni Turcy są \ldots 

-- Synu, nie jesteś związkiem. Zgodnie z~prawem Stanów Zjednoczonych jesteś niczym.

Hindus odchrząknął. 

-- To wszystko prawda, ale dotyczy to również członków IWWWW na całym świecie we wszystkich ich krajach. Wiele krajów zakazuje \textit{wszelkich }związków zawodowych. I prosimy o~uznanie praw tych pracowników.

-- Nie jesteśmy pracodawcami tych pracowników.

-- Twierdzisz, że też nie jesteś \textit{naszym }pracodawcą -- powiedział chłopiec z~oszałamiającą nutą triumfu w~głosie. -- Pamiętasz? Jesteśmy ,,niezależnymi wykonawcami'', prawda?

-- Dokładnie. 

-- Dr Prikkel, pozwól mi wyjaśnić. IWWWW jest otwarta dla wszystkich pracowników, bez względu na narodowość czy zatrudnienie, i~będzie działać solidarnie na rzecz wszystkich tych praw pracowniczych. Nasi farmerzy złota staną w~obronie naszych Mechanicznych Turków i~vice versa.

-- Cholerna racja -- powiedział chłopiec. -- Zniewaga dla jednego \ldots 

-- To zniewaga dla wszystkich. Farmerzy złota mają skromny zestaw żądań: skromne świadczenia, bezpieczeństwo pracy, plan emerytalny. Wszystko, o~co planujemy poprosić pracodawców naszych farmerów. Nic, na co nie może sobie pozwolić twój oddział. 

-- Czy chcesz powiedzieć, że twoje żądania są uzależnione od uznania żądań przyjaciół pana Goldberga?

-- Dokładnie. 

-- I zniszczysz gospodarkę Wojowników Svartalfaheim w~45 minut \ldots  

-- 38 minut -- powiedział dzieciak.

-- Chyba że \textit{zasadniczo }zgodzę się, że to zrobimy?

-- Podsumowałeś to wszystko w~sposób godny podziwu -- powiedział indyjski ekonomista. -- Bardzo dobrze. 

-- Czy możesz dać mi minutę?

-- Mogę dać ci 38 minut. 

-- 37 -- powiedział dzieciak.

Wyciszył ich, a on i~Bill wpatrywali się w~siebie przez długi czas.

-- Czy to jest tak szalone, jak się wydaje? 

-- Właściwie szalone jest to, że nie jest to aż tak szalone. Niemożliwe, ale nie szalone. Już teraz pozwalamy wielu stronom trzecim bawić się naszymi gospodarkami, niezależnym brokerom, ludziom, którzy kupują i~sprzedają swoje instrumenty. Nie ma technicznego powodu, dla którego te osoby nie mogą stać się częścią naszego planowania. Do diabła, jeśli potrafią robić to, co mówią, będziemy znacznie bardziej dochodowi niż teraz.

-- Po pierwsze, nie będziemy musieli zawieszać serwerów śledzących ich wszystkich.

Connor skrzywił się. 

-- Zgadza się. Ale jest też niemożliwa część. Pomijając całą sprawę o~Turkach, która jest po prostu \textit{szalona}, zarząd nigdy, przenigdy, nigdy, nigdy \ldots 

Bill uniósł rękę. 

-- Teraz tutaj się z~tobą nie zgadzam. Kiedy spotykasz się z~zarządem, zawsze próbujesz sprzedać im jakiś dziwaczny pomysł finansowy, który sprawia, że martwią się, że stracą oszczędności swojego życia. Kiedy jak idę do nich, to żeby poprosić ich o~trochę swobody w~walce z~oszustami i~hakerami. Rozumieją oszustów i~hakerów i~mówią tak. Gdybyśmy mieli zapytać ich razem \ldots 

-- Myślisz, że to dobry pomysł?

-- To lepszy pomysł niż ściganie tych dzieciaków w~przestrzeni gier jak kapitan Ahab ścigający białego wieloryba. Formalna definicja szaleństwa polega na powtarzaniu tego samego, ale oczekiwaniu innego wyniku. Nadszedł czas, abyśmy spróbowali czegoś innego.

-- A co z~Turkami?

-- Co z~nimi? 

-- Oni szukają \ldots  

-- Chcą w~takim razie zmniejszyć o~pół procent zysków firmy. Co roku wydajemy na wasze bilety lotnicze pierwszej klasy na konferencje ekonomiczne więcej, niż by chcieli. Ale straszna umowa.

-- Ale jeśli się poddamy, poproszą o~więcej.

-- A jeśli nie poddamy się temu, spędzimy następne sto lat na ściganiu chińskich i~indyjskich dzieci w~przestrzeni gier, zamiast poświęcać naszą energię na walkę z~\textit{prawdziwymi }oszustami i~hakerami. Bezpieczeństwo zawsze polega na wyborze bitew. Każdy złożony ekosystem ma pasożyty. W twoim ciele jest dziesięć razy więcej komórek bakterii niż krwinek. Sztuczka z~pasożytami polega na wymyśleniu, jak z~nimi współistnieć.

-- Nie mogę uwierzyć, że to mówisz.

-- To dlatego, że nie jestem graczem. Nie obchodzi mnie, kto wygra. Nie obchodzi mnie, kto przegra. Jestem ekspertem od bezpieczeństwa. Zależy mi na kosztach zabezpieczenia systemów, którymi zarządzam. Możemy pozwolić tym dzieciom ,,wygrać'' kilka małych bitew, ponieść koszty i~zaoszczędzić dziesięć razy więcej, nie musząc ich gonić.

Connor pokręcił głową. 

-- Co z~nimi? -- powiedział, przewracając oczami po pokoju, by objąć resztę Centrali Dowodzenia, z~których większość otwarcie teraz podsłuchiwała.

Bill zwrócił się do nich. 

-- Ręce do góry: kto chce tworzyć i~prowadzić totalnie zajebiste gry, które czynią nas bogatszymi niż piekło? 

 Każda ręka wystrzeliła w~górę. 
 
 -- Kto chce spędzać czas na gonitwie za chudymi biednymi dzieciakami, zamiast po prostu szukać sposobu na ich zneutralizowanie? 
 
 Kilka rąk unosiło się wyzywająco w~powietrzu, wśród nich Kaden, który wrócił do pokoju, gdy Connor rozmawiał przez telefon i~teraz wpatrywał się w~nich obu. Bill odwrócił się do Connora.
 
-- Myślę, że wszystko będzie w~porządku -- powiedział. Wskazał głową przez ramię i~powiedział głośno: -- Te zbiry są tak zadziorni, że odmówią, jeśli ich zapytasz, czy chcą darmowe lody do końca życia.

\bigskip
\threeast

300 000 kamieni runicznych nie wydawało się dużo, kiedy Yasmin zaczynała. W końcu złoto było dla Mali, a Mala była wszystkim, o~czym mogła myśleć. I miała po swojej stronie armię Mali, wszyscy pracowali razem.

Ale minęły już dni, odkąd spała porządnie, a co kilka minut pojawiali się reporterzy, którzy wciskali się do kawiarni pani Dibyendu z~aparatami, magnetofonami i~notesami, i~zadawali jej różne szalone pytania, a ona musiała opanować się i~mówić skromnie i~spokojnie, kiedy każdy nerw w~jej ciele krzyczał \textit{Nie widzisz, jaka jestem zajęta? Nie widzisz, co mam zrobić? }Ale armia okryła się chwałą i~żaden żołnierz nie stracił panowania nad sobą, a cała prasa podziwiała ich i~ich ciekawą pracę.

Przynajmniej hutnicy i~pracownicy przemysłu odzieżowego byli rozsądni, by im nie przeszkadzać, a poza tym byli głównie zajęci organizowaniem przygód w~Dharavi, żeby im przeszkadzać. Historia o~tym, jak uratowali ten gang dzieci Dharavi przed złymi ludźmi z~bronią, rozeszła się po każdym rogu, a pracownicy, których zainspirowali do odejścia z~pracy, byli w~połowie zachwyceni.

Kawałek po kawałku udało im się jednak zbudować fortunę. Yasmin znalazła dla nich misję z~instancją z~przyzwoitą wypłatą, taką, którą może wykonać trzech lub czterech graczy naraz, i~skierowała ich wszystkich do niej, wysyłając ich do jaskiń za krasnoludami i~ogrami poniżej w~gangach, grasujących w~górę i~w dół przez wąskie, piekielnie gorące przejścia między maszynami, wskazujące sposoby szybszego wykonania pracy, odnotowujące sumę każdego gracza, aż po pozornej wieczności mieli wszystko.

-- Ashok -- powiedziała, wpadając bez zapowiedzi do jego biura. 

Pochylony nad klawiaturą, na słuchawce, mruczał po angielsku do swojego doktora Prikkela w~Ameryce. Podniósł rękę i~poprosił mężczyznę, żeby mu wybaczył -- nienawidziła tego, jak służalczy brzmiał, ale musiała przyznać, że był bardzo opanowany, kiedy negocjacje trwały -- i~wyciszył go.

-- Yasmin? 

-- Mamy okup za Mali -- powiedziała.

-- Tak, powiedział, oczywiście. 

Wysłał szybką wiadomość do centralnej komórki w~Singapurze i~uzyskał numer Bannerjee, po czym szybko wykręcił go na głośniku. Bannerjee odpowiedział, tym razem znacznie mniej rozmytym i~zaspanym głosem.

-- Zwycięstwo dla Ramy!

-- Mamy twoje pieniądze -- powiedział Ashok. -- Nasz zespół dostarcza go teraz do chaty depozytu. Możesz sam to sprawdzić.

-- Tak poważnie, tak rzeczowo. To tylko gra, przyjacielu  \ldots  odpręż się! 

Yasmin miała wrażenie, że może zwymiotować. Ten człowiek był taki \ldots  \textit{zły}. Co uczyniło człowieka tak złym? Zrozumiała, naprawdę rozumiała, jak Mala musi się cały czas czuć. Poczucie, że są ludzie, których \textit{trzeba ukarać}, a ona była osobą, która musi to zrobić. Zepchnęła to uczucie w~dół.

-- W porządku, dobrze. Widzę, że tam jest. Powiem ci, gdzie znaleźć przyjaciółkę, kiedy powiesz agentowi depozytowemu, żeby wypuścił pieniądze, tak?

Ashok pokiwał brodą na telefon, intensywnie myśląc. Yasmin nagle zdała sobie sprawę z~czegoś, co powinna była zrozumieć od samego początku: agent depozytu lub nie, albo będą musieli zaufać Bannerjee, że wypuści Mala po wydaniu pieniędzy, albo Bannerjee będzie musiał im zaufać, że uwolni pieniądze po tym, jak przekaże im pieniądze. im Mala. Usługi depozytowe działały przy transakcjach gotówkowych, ale nie dla okupów. Czuła się jeszcze bardziej chora.

-- Najpierw wypuszczasz Malę i~\ldots 

-- Och, daj spokój. Dlaczego miałbym to zrobić? Traktujesz mnie z~taką pogardą, że nie ma mowy, abyś dał mi to, co obiecałeś. W końcu zawsze możesz wydać 300 000 kamieni runicznych. Ja, z~drugiej strony, nie mam szczególnego pożytku z~lekceważącej dziewczynki. Dlaczego miałbym ci powiedzieć, gdzie ją znaleźć?

Ashok i~Yasmin spojrzeli w~oczy. Przypomniała sobie, kiedy ostatni raz widziała Malę, jak bardzo była zmęczona, jak chuda, jak boleśnie utykała. 

-- Zrób to -- powiedziała, zakrywając mikrofon dłonią.

-- Hasło do depozytu brzmi ,,Zwycięstwo dla Ramy'' -- powiedział Ashok drewnianym tonem.

Bannerjee roześmiał się głośno, po czym zakończył połączenie. Po chwili Ashok spojrzał na ekran, obserwując alarmy. 

-- Zabrał pieniądze. Czekali jeszcze minutę. Jeszcze minuta.

Ashok ponownie wybrał numer Bannerjee. Yasmin od razu wiedziała, że nie odda im Mali.

-- Mala -- powiedział Ashok.

-- Odwal się -- powiedział Bannerjee.

-- Mala -- powiedział Ashok.

-- Milion kamieni runicznych -- powiedział Bannerjee.

-- Mala -- powiedział Ashok. -- Albo \ldots 

-- Albo co ?

-- Albo zabiorę wszystko.

-- O tak? 

-- Wezmę teraz 30 000. I będę brał 30 000 co pięć minut, dopóki nie oddasz nam Mali.

Bannerjee znów zaczął się śmiać, a Ashok znowu mu przerwał, po czym przeniósł się z~powrotem do swojego Amerykanina w~Coca Coli.

-- Doktorze Prikkel -- powiedział. -- Wiem, że jesteśmy zajęci ratowaniem gospodarki przed ruiną, ale mam do ciebie małą, ale ważną prośbę.

Głos Amerykanina był zdezorientowany. 

-- Mów.

Ashok podał mu imię plutonowego, którego Bannerjee wysłał do domu powierniczego. 

-- Porwał naszą przyjaciółkę i~nie chce jej oddać.

-- Porwana?

-- Wzięta do niewoli. 

-- W grze? 

-- W świecie. 

-- Jezu.

-- I Rama też. Zapłaciliśmy okup, ale \ldots 

Yasmin przestała słuchać. Ashok najwyraźniej myślał, że jest najmądrzejszym człowiekiem, jaki kiedykolwiek chodził po Bożej Ziemi, ale miała już dość gier. Opadła na pięty i~przyjrzała się brudnej podłodze, jej oczy traciły ostrość z~powodu braku snu i~jedzenia.

Stopniowo uświadomiła sobie, że Ashok znowu rozmawia z~Bannerjee.

-- Jest w~Generalnym Komunalnym Lokmanya Tilak. Dziś została przewieziona na oddział ratunkowy, bez imienia i~nazwiska. Powinna tam nadal być.

-- Skąd wiesz, że nie odeszła?

-- Ona nie odejdzie -- powiedział Bannerjee. -- Teraz wynoś się z~mojego konta bankowego albo przyjdę tam i~zdmuchnę ci jaja.

Chwilę zajęło Yasmin zrozumienie, jak Bannerjee mógł być tak pewny, że Mala nie opuściła szpitala, musiała być tak ciężko ranna, że nie mogła wyjść. Odkryła, że zawodzi, wydaje dźwięk jak kot w~nocy, okropny dźwięk, którego nie mogła powstrzymać. Armia Mali nadbiegła, a ona próbowała się zatrzymać, żeby im to wyjaśnić, ale nie mogła.

W końcu wszyscy razem poszli do szpitala LT, uroczysta procesja ulicami Dharavi. Kilka osób podbiegło, by zapytać, co się dzieje, a kiedy im powiedziano, przyłączyli się. Coraz więcej ludzi przyłączało się, aż w~ogromnym tłumie setek milczących ludzi przybyło do szpitala. Ashok, Yasmin i~Sushant podeszli do lady i~powiedzieli zszokowanej siostrze oddziałowej, dlaczego tam są. Przez całą wieczność kartkowała swój notatnik, zanim powiedziała: 

-- To musi być ta. -- Spojrzała na nich surowo. -- Ale wszyscy nie możecie iść. Kim jest matka dziewczynki?

Ashok i~Yasmin spojrzeli z~powrotem na tłum. Żadne z~nich nie pomyślało o~sprowadzeniu matki Mali. Byli rodziną Mali. Była ich generałem. 

-- Zabierz nas do niej, proszę -- powiedziała Yasmin. -- Przyprowadzimy jej matkę. 

Wyglądało na to, że siostra nie pozwoli im przejść, ale Ashok przekręcił głowę przez ramię. 

-- Wiesz, że nie odejdą, dopóki jej nie zobaczymy. 

Poruszył dobrodusznie brodą, uśmiechnął się i~przez chwilę Yasmin przypomniała sobie, jaki był przystojny, kiedy spotkała go po raz pierwszy na jego motocyklu.

Siostra westchnęła z~irytacją. 

-- Chodźcie ze mną -- powiedziała.

Nie rozpoznaliby Mali, gdyby nie powiedziała im, które łóżko jest jej. Jej głowa była ogolona i~zabandażowana, a po jednej stronie twarzy była masa siniaków. Jej lewa ręka była na temblaku.

Yasmin mimowolnie jęknęła, kiedy ją zobaczyła, a siostra oddziałowa obok niej ścisnęła ją za ramię. 

-- Nie została zgwałcona -- wyszeptała jej do ucha kobieta. -- A lekarz mówi, że nie było uszkodzenia mózgu.

Yasmin płakała teraz, naprawdę płakała, jak wcześniej nie pozwalała sobie płakać, płacz z~jej duszy i~żołądka, płacz, który nie chciał odejść, płacz, który rzucił ją na kolana, jakby była bita z~lathi. Zwinęła się w~kłębek, płakała i~płakała, a siostra oddziałowa poprowadziła ją na miejsce i~próbowała włożyć pigułkę do ust, ale nie chciała jej pozwolić. Musiała być czujna i~rozbudzona, musiała przestać płakać, potrzebowała \ldots 

Ashok przykucnął pod ścianą obok niej, zaciskając i~rozluźniając pięści. 

-- Zniszczę go -- mruczał raz za razem, ignorując spojrzenia innych pacjentów na oddziale z~ich gośćmi. -- \textit{Zniszczę }go.

To dotarło do Yasmin. 

-- Jak? 

-- Każdy piastr, każdy kamień runiczny, każdą sztukę złota, którą ten człowiek wyciągnie z~gry, zabierzemy mu. Jest skończony. 

-- Znajdzie inny sposób na przetrwanie, inny sposób na zranienie ludzi, aby przetrwać.

Ashok potrząsnął głową. 

-- Dobrze. Znajdę sposób, żeby to też zrujnować. Jest potężny, silny i~bezwzględny, ale my jesteśmy sprytni i~szybcy i~jest nas \textit{tak wielu}.

\bigskip
\threeast

Dafen był pełen duszącego dymu. Matthew przepchał się przez tłum. Próbował zabrać ze sobą dziewczynę malarkę, Mei, ale wpadła na grupę swoich przyjaciół i~poszła z~nimi, zatrzymując się, by mocno pocałować go w~usta, a potem śmiejąc się z~jego zdziwionego wyrazu twarzy, pocałowała go znowu. Za drugim razem miał przytomność umysłu, by też ją pocałować, i~na sekundę zdołał zapomnieć, że był w~środku zamieszek. Przyjaciele Mei zahuczeli i~zawołali na nich, a ona ścisnęła jego pośladki, wyjęła mu telefon z~palców i~wpisała w~niego swój numer, naciskając ZAPISZ. Sieć telefoniczna zgasła godzinę wcześniej, kiedy policja wycofała się z~Dafen i~wróciła do obronnego kordonu otaczającego cały obszar.

A potem został sam, wracając do ogromnego posągu ręki trzymającej pędzel, wejścia do Dafen. Malarze tłoczyli się po ulicach, niosąc pięknie wykonane znaki, śpiewając piosenki, pijąc ogniste, tanie baijiu, którego zapach mieszał się z~dymem, farbą olejną i~terpentyną.

Linia policji zjeżyła się, gdy wyjrzał zza rogu kawiarni na skraju Dafen. Nie tylko on przyglądał im się nerwowo, w~kawiarni kuliła się grupka białych turystów, ściskając w~dłoniach aparaty i~wpatrując się z~niedowierzaniem w~martwe telefony. Matthew przysłuchiwał się ich rozmowie, starając się zrozumieć szybki angielski, i~doszedł do wniosku, że przywiózł ich tu kierowca z~ich hotelu Hilton przy ulicy Jiabin.

-- Witam -- powiedział, wypróbowując swój angielski. Żałował, że gweilo, Wei-Dong, nie pozwolił mu więcej ćwiczyć. -- Potrzebujecie pomocy? 

Był bardzo skrępowany tym, jak źle musi brzmieć, jego akcent i~gramatyka okropne. Matthew był dumny z~tego, jak dobrze mówi po chińsku.

Najstarsza turystka, kobieta z~widocznymi pod topem na cienkich ramiączkach pomarszczonymi ramionami i~szyją, spojrzała na niego twardo. Zdjęła za duże okulary przeciwsłoneczne i~spróbowała trochę chińskiego. 

-- Nic nam nie jest -- powiedziała, jej akcent nie był lepszy niż akcent Matthew, co dziwnie go pocieszyło. Była z~trzema innymi, mężczyzną, którego wziął za jej męża, i~dwoma młodymi mężczyznami, mniej więcej w~wieku Matthew, którzy wyglądali jak skrzyżowanie jej z~mężem: synami.

-- Proszę -- powiedział. -- Zabiorę was, znajdę taksówkę. Powiecie  \ldots  -- próbował znaleźć słowo określające policjantów, nie pamiętając go, znalazł się w~swoim słowniku gier. -- Rycerze? Paladyni? Żołnierze. Powiecie żołnierzom, że jestem przewodnikiem. Wszyscy idziemy. 

Chłopcy uśmiechnęli się do niego, a on pomyślał, że muszą być graczami, ponieważ naprawdę ożywili się na \textit{paladynach}, a on próbował się do nich uśmiechnąć, choć prawdę mówiąc, nie miał ochoty na nic. Naradzali się ściszonymi głosami.

-- Nie, dziękuję -- powiedział starszy mężczyzna. -- Wszystko w~porządku.

Zacisnął powieki. Musiał się gdzieś dostać, żeby jego telefon zadziałał, musiał skontaktować się z~Big Sister Nor i~dowiedzieć się, gdzie są inni, jaki był plan. Będzie musiał zdobyć nowe papiery, może udać się do jednej z~prowincji albo spróbować wkraść się do Hongkongu. 

-- Pomóżcie mi -- zdołał. -- Nie pójdę bez was. Bez \ldots  obcokrajowców. -- Wskazał na policję, na ich tarcze. -- Nie krzywdzą cudzoziemców.

Oczy starszego mężczyzny rozszerzyły się w~zrozumieniu. Znowu rozmawiali między sobą. Złapał słowo ,,przestępca''.

-- Nie jestem przestępcą -- powiedział. 

Ale wiedział, że to kłamstwo i~czuł, że oni też muszą to wiedzieć. Był przestępcą i~byłym więźniem, i~nigdy nie będzie niczym innym, jak przez całe życie; tak jak jego dziadek.

Wszyscy spojrzeli na niego, a potem odwrócili wzrok.

-- Proszę -- powiedział, patrząc na każdego po kolei. Wskazał głową na policję. -- Wkrótce zranią ludzi.

Kobieta wzięła głęboki oddech, odwróciła się do mężczyzny i~powiedziała: 

-- I tak musimy się stąd wydostać. Dobrze będzie mieć miejscowego.

Wyższy z~chłopców spytał: 

-- W co grasz?

-- Wojownicy Svartalfaheim, Zombie Mecha, Mushroom Kingdom, Clankers, Big Smoke, Toon -- powiedział, zaznaczając je na palcach.

-- We wszystkie ? -- Chłopcy mu się przypatrywali.

Pokiwał głową.

 -- Wszystkie 

Śmiali się i~on też się śmiał, ciche dźwięki w~ryku tłumu i~huku helikopterów nad głową.

-- Jesteś tego pewien? -- powiedziała kobieta. Dodając ,,pewien?'' po chińsku. Skinął dwukrotnie głową.

-- Chodźcie ze mną -- powiedział, odetchnął głęboko i~poprowadził ich w~stronę linii policyjnych.

\bigskip
\threeast

Wei-Dong nie chciał budzić Jie, ale musiał spać. W końcu zwinął się na podłodze obok materaca, używając torby na ramię jako poduszki, aby zdjąć twarz z~brudnego dywanu. Na początku leżał sztywno w~jasno oświetlonym pokoju, a jego umysł wirował od wszystkiego, co widział i~robił, ale potem musiał zasnąć i~zapaść mocno, ponieważ następną rzeczą, jaką poznał, było wypłynięcie z~otchłani totalności. zapomnienie, gdy Jie potrząsnął jego ramieniem i~zawołał jego imię. Otworzył oczy na szparki i~spojrzał na nią. 

-- Coo? -- zdołał, po czym zdał sobie sprawę, że mówi po angielsku i~zapytał: ,,Co?'' po chińsku.

-- Czas iść -- powiedziała. -- Big Sister Nor mówi, że musimy się przeprowadzić.

Usiadł. W ustach miał paskudną słoną pastę, nieświeże resztki klusek i~snu. Świadomie oddychał przez nos.

-- Gdzie? 

-- Hongkong -- powiedziała. -- W takim razie \ldots  -- Wzruszyła ramionami. -- Może Tajwan? Gdzieś, gdzie możemy opowiedzieć historię zmarłych, nie będąc aresztowanym. To najważniejsze.

-- Jak przekroczymy granicę? Nie mam w~paszporcie chińskiej wizy.

Uśmiechnęła się. 

-- Ta część jest łatwa. Idziemy do mojego fałszerza.

To był tak dobry plan, jak każdy inny. Wei-Dong obserwował, jak Webblies raz po raz zmieniają dokumenty. Shenzhen było pełne fałszerzy. Znowu jechał metrem oddzielnie od niej, wpatrując się w~swoją głupią mapę przewodnika i~starając się wyglądać jak głupi turysta, niewidzialny. Tym razem było łatwiej, ponieważ działo się o~wiele więcej, dziewczyny z~fabryki rozmawiające o~audycji radiowej Jie i~,,42'', policjanci krążących po samochodach i~żądający dokumentów od dowolnej grupy trzech lub więcej osób, przeszukujący torby i~raz konfiskując transparent namalowany na prześcieradle. Wei-Dong nie widział, co było napisane, ale policja zabrała cztery krzyczące kopiące dziewczyny z~pociągu na następnej stacji. Shenzhen pogrążył się w~chaosie.

Wysiedli z~pociągu na stacji giełdowej, a on poszedł za Jie, pozostawiając między nimi sto metrów. Ale natknął się na nią, kiedy wyszli na powierzchnię. Kiedy był tu po raz ostatni, było zatłoczone fałszerzami i~naganiaczami rozdającymi ulotki reklamujące ich usługi, skupami złomu z~wagami ustawionymi na chodnikach, straganiarzami sprzedającymi owoce i~lody. Teraz była to policja od ściany do ściany, kordon uformowany wokół wejścia na giełdę. Co kilka metrów na ulicy stacjonowali też funkcjonariusze, którzy sprawdzali papiery.

Jie podniosła telefon i~udawała, że rozmawia z~nim, ale Wei-Dong widział, że po prostu nie chciała wyglądać podejrzanie. Wyjął swoją mapę turystyczną i~udawał, że ją studiuje. Stopniowo oboje wrócili na stację. Dołączyła do niego przy dużej mapie okolicy.

-- Co teraz? -- wyszeptał, starając się nie poruszać ustami.

-- Jak zamierzałeś się stąd wydostać? -- powiedziała.

Jego żołądek zacisnął się. 

-- Tak naprawdę nie myślałem o~tym zbyt wiele -- powiedział.

Syknęła z~frustracji. 

-- Musiałeś mieć jakiś pomysł. A w~jaki sposób tutaj się dostałeś?

Nie powiedział nikomu szczegółów swojej transoceanicznej podróży. Byłoby dziwnie przyznać, że był współwłaścicielem gigantycznej firmy żeglugowej. Poza tym tak naprawdę nie \textit{czuł}, że to jego. To było jego ojca.

Przeszło obok dwóch policjantów z~ponurymi twarzami, poruszając się szybko, z~ich słuchawek dobiegało naglące, owadzie brzęczenie.

-- Naprawdę? 

-- Gdybyśmy mogli dostać się do portu -- powiedział. -- Myślę, że mogę nas zabrać wszędzie. 

Uśmiechnęła się i~był to pierwszy prawdziwy uśmiech, jaki zobaczył na jej twarzy od \ldots  od czasu, gdy zaczęła się strzelanina.

-- Ale muszę zadzwonić do mamy.

\bigskip
\threeast

Policjanci, którzy przesłuchiwali Matthew, byli tak spięci, że praktycznie wibrowali, ale turystka zrobiła wielki pokaz, że jest urażona, że zostali zatrzymani i~zażądała, aby pozwolono im iść, praktycznie krzycząc po angielsku. Matthew tłumaczył każde słowo, przemawiając przez policjantów, którzy próbowali zadać mu więcej pytań o~to, jak się tam znalazł i~co się stało, że jego ubranie tak pobrudziło się farbą i~błotem.

Turystka wyjęła aparat i~skierowała go w~policjantów, co zakończyło przyjacielską dyskusję. Zanim zdążyła podnieść ekran do twarzy, dłoń policjanta w~rękawiczce zacisnęła się na obiektywie. Obaj chłopcy ruszyli do przodu i~wyglądało na to, że ktoś wkrótce zacznie się przepychać, a mężczyzna krzyczał po angielsku, a cały hałas wystarczył, by przyciągnąć uwagę policjanta, który ostro dogadał glinom za marnowanie czasu i~przepuścił ich dalej surowym gestem.

Matthew nie mógł uwierzyć, że jest wolny. Turyści wydawali się myśleć, że to tylko zabawa, kiedy popychał ich drogą, poza zasięg kordonu policji i~z dala od krzyków. Podeszli poboczem autostrady Shenhui, trzymając się samej krawędzi, gdy ogromne ciężarówki przemykały obok nich tak szybko, że zaparło im dech w~piersiach.

-- Taxi? -- zapytała go kobieta.

Potrząsnął głową. 

-- Nie sądzę, żeby dziś taksówka -- powiedział. -- Może prywatny samochód.

Wydawała się rozumieć. Zaczął machać do każdego samochodu, który ich mijał, aż w~końcu jeden się zatrzymał, Chang'an sedan, który widział lepsze czasy, z~bagażnikiem zamkniętym liną bungee, która pozwalała trzasnąć pokrywą, gdy samochód staczał się i~zatrzymywał. Prowadził go mężczyzna w~brudnym mundurze szofera. Matthew pochylił się i~powiedział: 

-- 100 RMB, aby zabrać nas na Jiabin Road. -- Było to dużo, ale był pewien, że turystów było na to stać.

-- Nie, za daleko -- powiedział mężczyzna. -- Mam inną pracę \ldots  

-- 200 -- powiedział Matthew.

Mężczyzna uśmiechnął się, pokazując usta pełne stalowych zębów.

-- OK, wszyscy do środka

Byli w~drodze zaledwie pięć minut, zanim jego telefon zadzwonił, informując go, że czeka na niego poczta głosowa. To była Justbob od Big Sister Nor.

\bigskip
\threeast

-- Mamo? 

-- Lawrence?

-- Cześć, mamo. 

 Próbował zignorować Jie, która patrzyła na niego z~wyrazem pomieszanej wesołości i~podziwu. Miała encyklopedyczną wiedzę na temat kafejek dla graczy z~prywatnymi pokojami i~przyprowadziła je do tej znajdującej się na parterze schroniska młodzieżowego, które obsługiwało obcokrajowców i~miało pokój przeznaczony do karaoke i~dostępu do sieci.

-- Dawno nie słyszałam twojego głosu, Lawrence. 

-- Wiem, mamo.

-- Jak twoja podróż? 

-- Um, dobrze. -- Próbował sobie przypomnieć, gdzie powiedział jej, że będzie. Portland? San Francisco?

-- Och, Lawrence -- powiedziała, a on usłyszał, że płacze. To było co, 8 wieczorem w~LA, a ona płakała i~była sama. Tak tęsknił za domem w~tym momencie, że myślał, że podzieli się na pół i~poczuł łzy spływające po jego policzkach.

-- Kocham cię, mamo -- wybełkotał.

I oboje płakali przez długi czas, a kiedy zaryzykował spojrzenie na Jie, ona też płakała.

-- Mamo -- powiedział, dławiąc smarki. -- Mam do ciebie prośbę. Wielką przysługę.

-- Masz kłopoty. 

-- Tak -- Nie było sensu zaprzeczać. -- Mam kłopoty. I nie mogę tego teraz wyjaśnić. 

-- Jesteś w~Chinach, prawda?

Nie wiedział, co powiedzieć.

 -- Wiedziałaś. 

-- Podejrzewałam. Chodzi o~te sprawy z~grami, prawda? Wykonałam obliczenia, kiedy odpowiadałeś na moje wiadomości, kiedy dzwoniłeś.

-- Wiedziałaś? 

-- Nie jestem głupia, Lawrence. -- Już nie płakała. -- Myślałam, że wiem, ale nie chciałam nic mówić, dopóki mi nie powiesz.

-- Przepraszam mamo. 

Nic nie powiedziała.

-- Wracasz do domu?

Spojrzał na Jie.

 -- Nie wiem. W końcu. Najpierw muszę coś zrobić. 

-- I potrzebujesz w~tym mojej pomocy.

-- Mamo, muszę zamówić przesyłkę z~Shenzhen do Bombaju. 

 Zasugerowała to Big Sister Nor, a Jie wzruszyła ramionami i~powiedziała, że to dla niej w~porządku, jedno miejsce jest równie dobre jak każde inne. 
 
 -- Podam ci numer kontenera. I musisz poprosić pana Alforda, żeby zadzwonił do tutejszych władz portowych i~powiedział, że mam do niego dostęp.

-- Nie, Leonardzie. Zadzwonię do ambasady, odwiozę cię do domu, ale to jest \ldots  -- Wyobraził sobie, jak jej ręka trzepocze wokół głowy. -- To szaleństwo, tak właśnie jest.

-- Mamo \ldots 

-- Nie. 

-- Mamo, \textit{posłuchaj}. Tu chodzi o~dużo więcej niż tylko o~mnie. Są tu ludzie, przyjaciele, których życie jest zagrożone. Możesz dzwonić do ambasady, ile chcesz, ale ja tam nie pójdę. Jeżeli nie pomożesz mi, będę musiał to zrobić sam i~muszę być z~tobą szczery, mamo, nie sądzę, żebym był w~stanie to zrobić. Ale nie mogę porzucić moich przyjaciół.

Znowu płakała.

-- Będę w~porcie za  \ldots  -- sprawdził ekran swojego telefonu -- za trzy godziny. Mam ze sobą paszport, dzięki któremu wejdę do środka, \textit{jeśli }to załatwisz to z~władzami portowymi. Numer konteneru to WENU432134. To zachodni port. Zapisałaś to?

-- Leonardzie, nie zrobię tego.

-- WENU432134 -- powiedział bardzo powoli i~odłożył słuchawkę.

\bigskip
\threeast

W sumie było ich pięciu. Matthew, Jie, Cheng, Shirong i~Wei-Dong. Zatrzymali się przy 7-11 w~drodze na stację kolejową i~kupili tyle jedzenia, ile mogli unieść, prosząc oszołomionego pracownika, aby zapakował je w~pudła i~zakleił taśmą do pakowania.

Gdy zbliżyli się do portu, przestali rozmawiać, idąc powoli i~z rozmysłem. Wei-Dong zebrał siły i~podszedł do budki strażnika. Nie oddzwonił do matki. Nie było czasu. W Shenzhen panował chaos, wszędzie kontrole policyjne i~demonstracje, jakieś zamieszki, w~niebo wzbijały się spirale czarnego dymu.

Wskazał Chengowi, żeby do niego dołączył. Zgodzili się, że będzie grał tłumacza, żeby Wei-Dong wydawał się bardziej beznadziejnym gweilo, poza podejrzeniami. Znaleźli mu jakiś tani podrabiany chiński sprzęt Nike, absurdalny dres, który przypominał mu rosyjskich gangsterów, których widywał w~Santee Alley.

Bez słowa wręczył swój paszport -- swój prawdziwy, trzymany przez cały czas bezpiecznie -- młodemu mężczyźnie stojącemu na bramce. 

-- WENU432134 -- powiedział. -- Kontener Goldberg Logistics.

Czekał, aż Cheng przetłumaczy, patrzył, jak rysuje na dłoni angielskie litery.

Ochroniarz spojrzał przez ramię na dwóch policjantów w~budce z~nim. Wziął podrapany tablet i~szturchnął go tępym palcem, mrużąc oczy na paszport Wei-Donga. Wei-Dong miał nadzieję, że nie spróbuje czegoś sprytnego, jak przerzucanie jej stron w~poszukiwaniu chińskiej wizy.

Zaczął kręcić głową i~powiedział: 

-- Nie widzę tego \ldots 

Wei-Dong poczuł, jak pot spływa mu po pęknięciu pośladków i~na udach. Wyciągnął szyję, żeby zobaczyć ekran. Tak było, ale numer został wprowadzony błędnie, WENU432144. Wskazał na to i~powiedział: 

-- Powiedz mu, że to ten. 

Nieme podziękowanie wysłał matce. Strażnik porównał numer z~tym, do którego wszedł, a potem wydawało się, że zamierza ich przepuścić. Wtedy jeden z~policjantów powiedział: 

-- Czekaj.

Policjant odsunął ochroniarza ramię w~ramię, wziął od niego paszport, obejrzał go uważnie, podnosząc stronę do światła, żeby zobaczyć znak wodny. 

-- Co niesiesz? 

Wei-Dong czekał, aż Cheng przetłumaczy.

-- Próbki -- powiedział. -- Odzież. 

Otworzył pudełko u swoich stóp i~wyciągnął złożoną koszulkę ozdobioną kilkoma chińskimi znakami, które mówiły: ,,Jestem na tyle głupi, by myśleć, że ta koszulka wygląda fajnie''. Jie znalazła je od jednego z~nielicznych upartych handlarzy pozostawionych na ulicy przed wejściem do metra w~pobliżu dworca kolejowego. Gliniarz prychnął i~powiedział: 

-- Czy on wie, co to mówi?

Cheng skinął głową. 

-- Tak -- powiedział. -- Ale on myśli, że inni Amerykanie nie. Jeśli im się spodoba, zamówią u nas dwadzieścia tysięcy! -- Roześmiał się, a po chwili dołączyli do niego gliniarz i~ochroniarz. Policjant klepnął Wei-Donga po ramieniu, a Wei-Dong również zmusił się do śmiechu.

-- OK -- powiedział gliniarz, oddając mu dokumenty. Ochroniarz dał im wskazówki. -- Ale będziesz musiał skorzystać z~północnej bramy, żeby wyjść. Za pół godziny zamykamy tę na wieczór.

Cheng popisał się tłumaczeniem dla Wei-Dong, który miał przytomność umysłu, by udawać, że słucha, ale kołysał się na piętach, prawie w~momencie załamania się z~braku snu i~jedzenia.

Szli do kontenera w~całkowitej ciszy, a Wei-Dongowi udało się tylko raz spojrzeć przez ramię. Jie zauważył to, kiedy to zrobił i~pogroziła mu palcem. Uśmiechnął się krzywo i~spojrzał przed siebie, podążając za wskazówkami.

Kontener był dokładnie taki, jak go zostawił, a jego klucz pasował do kłódki. Cała czwórka zachwycała się sprytem jego pracy w~środku, gdy sprawnie rozpakowywali jedzenie.

-- Trzy noce, co? -- powiedziała Jie, gdy zamknął za nimi drzwi.

-- Po tym, jak nas załadują.

-- Kiedy to będzie? 

Westchnął. 

-- Muszę zadzwonić do mamy, żeby się dowiedzieć. 

Wyciągnął telefon i~Jie wręczyła mu swoją ostatnią kartę SIM i~kartę telefoniczną.

\bigskip
\threeast

Big Sister Nor, The Mighty Krang i~Justbob tym razem nie otrzymali żadnego ostrzeżenia. Trzech mężczyzn, drobnych oszustów pracujących na kontrakcie dla człowieka z~Dongguan, który był właścicielem jednej z~wielkich giełd złota, pracowało cicho i~wydajnie. Podążyli za Justbob z~malezyjskiej restauracji satay, którą często odwiedzali, do ostatniej kryjówki, pokoju nad salonem masażu na Changi Road, gdzie Webblies mogli podłączyć się do sieci wifi z~pobliskiego biurowca. Cierpliwie czekali na zewnątrz, aż wszystkie okna zgasną.

Następnie metodycznie przymocowali zamki rowerowe do każdych drzwi. Była prawie piąta rano i~nieliczni przechodnie nie zwracali na nich szczególnej uwagi. Po zamknięciu wszystkich drzwi wrzucili bomby benzynowe przez okna na parterze. Pozostali tylko na tyle długo, by upewnić się, że pożar radośnie płonie, zanim wsiedli do dwóch samochodów zaparkowanych za rogiem i~odjechali. Następnego ranka przejechali do Kuala Lumpur i~nie wrócili do Singapuru przez osiem miesięcy, pobierając niewielką pensję od mężczyzny w~Dongguan, gdy się nie wychylali.

Big Sister Nor była pierwszą, która się obudziła, obudził odgłos trzech rozbijanych okien w~bliskiej kolejności. Chwilę później poczuła tłusty dym i~zaczęła krzyczeć swoim najgłośniejszym głosem: 

-- Ogień! Ogień! -- tak jak ćwiczyła w~tysiącu snów.

Justbob i~The Mighty Krang wstali chwilę później. Justbob podeszła do schodów i~zaryzykowała wyjrzenie do połowy w~kierunku salonu masażu, zanim płomienie zmusiły ją do schowania. Mighty Krang wybił okno krzesłem -- zostało zamalowane -- i~wychylił się na tyle, by zobaczyć zamek, który został założony na drzwi. Zdyszany, ale spokojnie poinformował o~tym Big Sister Nor, która już wyjęła napędy ze swoich maszyn kontrolnych. Podała mu je, wysłuchała oceny Justbob co do schodów i~skinęła głową.

Słyszeli krzyki z~podłogi pod sobą, gdy dziewczyny z~salonu masażu wybiły własne okna i~wzywały pomoc. Z jednego z~małych, wysokich okien salonu masażu pojawiła się dziewczyna nogami do przodu. Krzyczała, płonąc, tocząc się po ziemi. Kilka osób było na ulicy i~rozmawiało przez telefony -- niedługo miała tu przyjechać straż pożarna. To nie było wystarczająco szybko. Duszący dym wypełniał już pomieszczenie i~zostali zmuszeni do klęknięcia.

-- Oknem -- wydyszała Big Sister Nor. -- Prawdopodobnie złamiesz nogę, ale to lepsze niż zostawanie tutaj. 

-- Ty pierwsza -- powiedział Mighty Krang.

-- Ja ostatnia -- powiedziała głosem, który nie znosił sprzeciwu. -- Po tym, jak wy dwoje wyjdziecie. -- Zdołała się uśmiechnąć. -- Spróbujcie mnie złapać, dobrze? 

Justbob chwyciła ramię The Mighty Krang i~pociągnęła go do okna. Dotarł do parapetu, a potem wzdrygnął się. 

-- Za daleko! -- powiedział, opadając z~powrotem na brzuch. 

Justbob posłała mu miażdżące spojrzenie, po czym przeciągnęła się przez parapet, opadła, tak że wisiała na rękach, a potem pozwoliła spaść resztę drogi. Jeśli wydała dźwięk, zginął w~ryku płomieni, które były teraz tuż za drzwiami. Podłoga była zbyt gorąca, by jej dotknąć.

-- IDŹ! -- powiedziała Big Sister Nor.

-- Jesteś naszą przywódczynią, naszą Starszą Siostrą Nor -- powiedział i~złapał ją za ramię. -- Bez ciebie wszyscy jesteśmy niczym! -- Strząsnęła jego rękę.

-- Nie, idioto -- powiedziała. -- Jestem niczym więcej niż centralą. Sami sobą kierujecie. Pamiętajcie o~tym! 

Chwyciła pasek jego dżinsów, tuż nad jego tyłkiem i~praktycznie wyrzuciła go przez okno. Powietrze przeleciało obok niego przez chwilę, a potem nastąpiło ogromne, wstrząsające uderzenie, a potem ciemność.

Big Sister Nor płonęła, jej luźne spodnie z~indyjskiej bawełny, jej długie czarne włosy. Pokój był teraz cały zadymiony, a każdy oddech też był ogniem. Poczuła przypalone własne włosy w~nosie, gdy oddech wrzącego powietrza przedostał się do jej płuc, które zamarzły i~odmówiły dalszej pracy. Wstała i~zrobiła krok do okna, stojąc przez chwilę jak płonący awatar jakiegoś tragicznego boga w~oknie, zanim zachwiała się, upadła na jedno kolano, a potem ogarnęły ją płomienie.

A poniżej tłum na ulicy zaczął płakać. Justbob też płakała z~chodnika, na którym opiekował się nią przechodzień, który znał jakąś pierwszą pomoc i~naciskał na jej lewą nogę. Mighty Krang był nieprzytomny, miał złamaną rękę i~trzy połamane żebra.

Pamiętał jednak, co powiedziała mu Big Sister Nor, i~zapisał te słowa, wpisując je lewą ręką w~języku angielskim, malajskim, hindi i~chińskim, nagrywając je swoim zrujnowanym od dymu głosem ze szpitalnego łóżka.

Jej słowa -- słowa Big  Sister Nor -- rozeszły się po całym świecie, rozprzestrzeniając się od telefonu do tablicy dyskusyjnej, od strony do strony. \textit{Sami sobą kierujecie}.

Słowa te usłyszały dziewczyny z~fabryki w~całych południowych Chinach, które wróciły do pracy po kilku krótkich dniach energetycznego chaosu, masowych zwolnień i~masowych aresztowań. Wysłuchali ich chłopcy z~fabryki w~całej Kambodży i~Wietnamie. Słychać je było w~zaułkach Dharavi oraz w~salonach Mechanicznych Turków w~całej Europie, Stanach Zjednoczonych i~Kanadzie. Były one publikowane w~wielu językach na okładkach wielu gazet i~emitowane w~wielu audycjach.

Ci ostatni potraktowali te słowa jako relację z~odległego świata -- ,,Czy wiesz, że te dziwne gry i~ludzie, którzy w~nie grali, potraktowali to wszystko tak poważnie?'' -- Ale dla ludzi, którzy potrzebowali ich usłyszeć, słowa zostały wysłuchane.

Wysłuchało ich pięciu przyjaciół, którzy pobrali je przez boleśnie powolne połączenie sieciowe na statku kontenerowym, dzień poza portem w~Shenzhen. Pięciu przyjaciół, którzy zapłakali, gdy je usłyszeli. Pięciu przyjaciół, którzy czerpali z~nich siłę.

\bigskip
\threeast

Ukryli się w~wewnętrznym kontenerze, gdy statek wpłynął do portu w~Bombaju, kierując się do Mumbai Port Trust. Wei-Dong sprawdził w~Google procedury bezpieczeństwa w~porcie w~Bombaju i~nie sądził, że używają chromatografu gazowego do wykrywania przemycanych ludzi, ale nie chcieli ryzykować. Było tłoczno, toaleta przestała działać, a wody udało się zebrać tylko na jeden krótki prysznic podczas trzydniowego przejścia.

Oparli się o~siebie, a potem przywarli do podłogi, gdy kontener został podniesiony na dźwigu i~ponownie postawiony. Usłyszeli, że zewnętrzne drzwi otwierają się, a potem zamykają, i~stłumione rozmowy. Potem się toczyli.

Ostrożnie otworzyli wewnętrzne drzwi. Pojemnik wypełnił zapach Bombaju -- pikantny, zakurzony, gorący i~wilgotny. Światło wpadało z~małych otworów, które Wei-Dong wywiercił wieki temu w~przejściu do Shenzhen.

Teraz usłyszeli dźwięk rogów, wiele, wiele rogów. Dużo silników motocyklowych, głośno. Spaliny Diesla. Ogromna, rycząca trąbka ciężarówki, na której został umieszczony ich kontener. Ciężarówka zatrzymywała się i~ruszała wiele razy, wykonała kilka powolnych, niezgrabnych zakrętów, po czym się zatrzymała. Chwilę później zatrzymały się też silniki.

Cała piątka wstrzymała oddech, nasłuchiwała kroków na zewnątrz, przysłuchiwała się rozmowie w~języku hindi, głosami dorosłych mężczyzn. Nasłuchiwała zgrzytu zapadki na dużych tylnych drzwiach kontenera.

A potem światło słoneczne -- zakurzone, gorące, z~kłębiącymi się tumanami kurzu i~smrodem ludzkiego moczu -- wlało się do pojemnika. Zasłonili oczy i~spojrzeli w~twarze dwóch uśmiechniętych Hindusów, z~groźnymi wąsami i~schludnie wyprasowanymi koszulami. Mężczyźni wyciągnęli ręce i~pomogli im zejść, jeden po drugim, do wąskiej alejki, która była całkowicie wypełniona przez ciężarówkę, która zgrabnie osłaniała ich przed wzrokiem. Wei-Dong nie mógł sobie wyobrazić cofania ciężarówki w~tak wąskiej przestrzeni.

Mężczyźni gestykulowali na wnętrze pojemnika, gestykulując: \textit{Masz wszystko}? Wei-Dong i~Jie upewnili się, że wszystko było zabrane, a potem skinęli głową. Mężczyźni pomachali na nich brodami, uścisnęli dłonie Wei-Dong i~Jie, krótko i~sucho, i~przesunęli się z~powrotem wzdłuż przestrzeni między ciężarówką a ścianami zaułka. Silnik ożył, chmura oleju napędowego wpadła im w~twarz i~ciężarówka odjechała, a nad ręcznie malowaną tabliczką na zderzaku z~napisem KLAKSON PROSZĘ.

Ciężarówka zadęła raz w~klakson, gdy wyjechała z~alejki i~skręciła w~niemożliwie ciasny zakręt w~prawo. Aleja była zalana światłem i~hałasem z~ulicy, a potem zobaczyli mężczyznę i~dziewczynę idących nią w~ich stronę.

Zbliżyli się. Dziewczyna miała na głowie jakąś chustę z~welonem, która zakrywała większość jej twarzy. Mężczyzna miał krótkie, nażelowane włosy i~był ubrany w~wyprasowaną białą koszulę wpuszczoną w~czarne spodnie. Dwie grupy stały i~patrzyły na siebie przez dłuższą chwilę, po czym mężczyzna wyciągnął rękę.

-- Ashok Balgangadhar Tilak -- powiedział.

-- Leonard Goldberg -- powiedział Leonard. 

Uścisnęli dłonie. To był kolejny krótki, suchy uścisk dłoni.

Dziewczyna wyciągnęła rękę. 

-- Yasmin Gardez -- powiedziała.

Ledwo chwyciła go za rękę, a wstrząs był krótki.

-- Sami sobą kierujemy -- powiedział Leonard. 

Nie planował tego powiedzieć, ale wyszło to samo, a Cheng zrozumiał to i~przetłumaczył na chiński i~przez chwilę nikt nie musiał nic więcej mówić.

-- Mamy dla was miejsca, w~których możesz się zatrzymać w~Dharavi -- powiedziała Yasmin. Leonard przetłumaczył. -- Wszyscy chcemy usłyszeć, co masz nam do powiedzenia. A jeśli chcesz, mamy dla ciebie pracę.

-- Chcemy pracować -- powiedział Cheng.

-- To dobrze -- powiedział Ashok i~ruszyli.

Wyszli obok hotelu. Ulica przed nimi była zatłoczona ludźmi, więcej niż mogli pojąć, samochodami, trójkołowcami, rowerami i~ciężarówkami wszelkich rozmiarów. Był to rój aktywności, który sprawiał, że nawet Shenzhen wydawał się stateczny. Przez chwilę żaden z~nich nic nie powiedział.

-- Bombaj to ruchliwe miejsce -- powiedziała Yasmin.

-- Mamy przyjaciół w~Związku Pracowników Transportu i~Doków -- powiedział od niechcenia Ashok, idąc zatłoczonym chodnikiem, ignorując dzieci, które podchodziły do nich, błagały, wyciągały ręce i~ciągnęły za rękawy. Leonard miał wrażenie, że przechodzi przez szalony sen. -- Cieszyli się, że mogą pomóc.

Ulica kończyła się na oceanie, ogromnym, lśniącym portem usianym promami i~innymi statkami. Przed nimi rozciągał się ogromny plac, wielkości kilku połączonych ze sobą boisk piłkarskich, pokryty ogrodami, a tam, gdzie stykał się z~oceanem, ogromny łuk zwieńczony minaretami i~pokryty misternymi rzeźbami, a wokół nich tysiące ludzi, rozmawiało, spacerowało, sprzedawało, błagało, spało, biegało, jeździło konno.

Cała piątka zatrzymała się i~gapiła. Trzy dni zamknięcia w~pojemniku i~nic do zobaczenia, co było oddalone o~więcej niż kilka metrów, pozbawiło ich umiejętności łatwego skupienia się na dużych, odległych obiektach i~zajęło im to dużo czasu, zanim zrozumieli. Yasmin i~Ashok pobłażali im, uśmiechając się lekko.

-- Brama Indii -- powiedziała Yasmin, a Leonard przetłumaczył z~roztargnieniem.

Po jednej stronie stał hotel wielkości gigantycznego centrum konferencyjnego w~pobliżu Disneylandu, zbudowany jak gigantyczna świątynia, ogromna i~niezgrabna. Leonard przyglądał mu się przez chwilę, po czym przegonił żebraków, którzy do nich podeszli. Yasmin zbeształa ich w~hindi, a oni uśmiechnęli się do niej i~cofnęli o~kilka kroków, mówiąc coś wyraźnie obraźliwego, co Yasmin zignorowała.

-- To niesamowite -- powiedział Leonard.

-- Bombaj jest \ldots  -- Ashok machnął ręką. -- To niesamowite. Nawet tam, dokąd jedziemy  \ldots  drugi koniec Harbour Line, nasz skromny dom, jest niesamowity. Uwielbiam to miejsce.

-- Uwielbiałem to w~Chinach -- powiedział Cheng. Wyglądał poważnie.

-- Mam nadzieję, że pewnego dnia wrócisz -- powiedział Ashok. -- Wszyscy. Wszyscy. Gdziekolwiek chcemy. 

-- Tłumią strajki w~Chinach -- powiedziała Jie. Leonard przetłumaczył.

Yasmin i~Ashok poważnie skinęli głowami. 

-- Będą inne strajki -- powiedziała Yasmin.

Zbliżał się do nich mężczyzna. Biały mężczyzna, blady i~oczywisty wśród tłumu, ciągnący za sobą komety żebraków. Leonard zobaczył go pierwszy, potem Ashok odwrócił się, by podążyć za jego wzrokiem i~wyszeptał: 

-- O rany, to \textit{jest }interesujące.

Mężczyzna podszedł do nich. Był gruby, miał oczy szopa, włosy w~dzikim nieładzie wokół głowy. Miał na sobie koszulkę polo ozdobioną logo Coca-Cola Games i~parę niebieskich dżinsów, które nie pasowały do niego, oraz Birkenstocks. Nie wyglądałby bardziej na Amerykanina, gdyby trzymał pochodnię Statui Wolności i~śpiewał ,,Star Spangled Banner''.

Ashok wyciągnął rękę. 

-- Doktor Prikkel, jak sądzę.

-- Pan Tilak. -- Potrząsnęli. Zwrócił się do Leonarda. -- Wydaje mi się, że Leonard.

Leonard przełknął ślinę i~wziął mężczyznę za rękę. Miał stanowczy, amerykański uścisk dłoni. Czterech chińskich Webblies rozmawiało między sobą. Leonard szepnął im, wyjaśniając, kim był ten mężczyzna, wyjaśniając, że nie ma pojęcia, co tam robi.

-- Musisz mi wybaczyć dramaturgię -- powiedział Connor Prikkel. -- Wiedziałem, że będę musiał przyjechać do Bombaju, aby spotkać się z~tobą i~twoimi niezwykłymi przyjaciółmi, wymagała tego ciekawość. Ale kiedy umieściliśmy w~twojej organizacji naszych pracowników wywiadu konkurencyjnego, nie było trudno znaleźć dziurę w~twoim serwerze pocztowym, i~stamtąd przechwyciliśmy szczegóły tego spotkania. Pomyślałem, że zrobię wrażenie, jeśli przyjadę osobiście.

-- Zadzwonisz na policję? -- powiedział Cheng łamaną angielszczyzną.

Prikkel uśmiechnął się. 

-- Cholera, nie, synu. Co by to dało? Jest was tysiące drani Webbly. Nie, myślę, że jeśli Coca Cola Games będzie robić z~tobą interesy, warto usiąść i~porozmawiać. Poza tym miałem kilka dni urlopu, które musiałem wykorzystać przed końcem roku, co oznaczało, że nie musiałem przekonywać szefa, by pozwolił mi tu przyjechać.

Blokowali chodnik i~potrącali się co kilka sekund, gdy ktoś przepychał się obok nich. Jeden z~nich prawie wepchnął Prikkela do zawrotnej trójkołowej taksówki, a Ashok złapał go za ramię i~przytrzymał.

-- Zwolnisz mnie? -- powiedział Leonard.

Prikkel skrzywił się. 

-- Nie mój wydział, ale szczerze mówiąc, myślę, że to prawdopodobnie dobry zakład. Ty i~inni, którzy podpisali twoją małą petycję. -- Wzruszył ramionami. -- Mogę robić takie rzeczy, jak wyciąganie pieniędzy z~konta tego drania, kiedy zagrożone jest życie twojego przyjaciela, to nie tak, że będzie narzekał, prawda? Ale jak Coke Games zawiera kontrakty ze swoimi pracownikami? Nie mój dział.

Oczy Yasmin płonęły. 

-- Nie możesz \ldots  nie pozwolimy ci.

-- To dość interesujące twierdzenie -- powiedział, a dwaj mężczyźni trzymający trzymetrową tacę wypełnioną okrągłymi blaszanymi wiaderkami na lunch przecisnęli się obok niego, wbijając go w~Jie. -- Myślę, że z~pewnością moglibyśmy miło spędzić czas, dyskutując. -- Wskazał na ogromny hotel z~tortami weselnymi. -- Zostaję w~Taj. Chcecie dołączyć do mnie na lunch?

Ashok spojrzał na Yasmin i~przeszło między nimi coś niewypowiedzianego. 

-- Zabierzemy \textit{cię }na lunch -- powiedział Ashok. -- Jako naszego gościa. Znamy cudowne miejsce w~Dharavi. To tylko krótka podróż pociągiem. 

Prikkel spojrzał po kolei na każdego z~nich, po czym wzruszył ramionami. 

-- Wiesz co? Byłbym zaszczycony.

Ruszyli na stację kolejową. 

-- Nie mogę się \textit{doczekać}, kiedy to wyemituję -- prychnęła Jie.

Leonard się uśmiechnął. On też nie mógł się doczekać. 

\chapter*{Podziękowania}  


Dziękuję Russellowi Galenowi, Patrickowi Nielsenowi Haydenowi i~mojej pięknej i~niezwykle cierpliwej żonie Alice -- nie mógłbym tego napisać bez was trojga. 


Podziękowania dla Silklisters, Rishab Ghosh i~Ashok Banker i~Yoda, Keyan Bowes, Rajeev Suri, Sachin Janghel, Vishal Gondal, Sushant Bhalerao i~Menyu Singh za wszelką pomoc w~Bombaju. 


Dzięki dla LemonED, Andrew Lih, Paul Denlinger, Bunnie Huang, Kaiser Kuo, Anne Stevenson-Yang, Leslie Chang, Ethan Zuckerman, John Kennedy, Marilyn Terrell, Peter Hessler, Christine Lu, Jon Phillips, Henry Oh za nieocenioną pomoc w~Chinach. 


Dziękuję Julianowi Dibbellowi, Ge Jinowi, Matthew Chewowi, Jamesowi Sengu, Jonasowi Lusterowi, Stevenowi Davisowi, Danowi Kelly i~Victorowi Pineiro za pomoc z~farmerami złota. 


Podziękowania dla Maxa Keisera, Alana Wexelblata i~Marka Soderstroma za porady ekonomiczne. 


Podziękowania dla Thomasa ,,CmdLn'' Gideona, Dana McDonalda, Kurta Von Fincka, Canonical, Inc i~Kena Snidera za wsparcie techniczne! 


Podziękowania dla MrBrown i~blogerów z~Singapuru za niezapomniane uliczne kolacje. 


Dziękuję również JP Rangaswami i~Marilyn Tyrell. 


Dziękuję także Kenowi MacLeodowi za umożliwienie mi korzystania z~IWWWW i~,,Webbly''. 

\clearpage 


\chapter*{O Autorze}

Odcisk palca klucza GPG: \\ 0BC4 700A 06E2 072D 3A77 F8E2 9026 DBBE 1FC2 37AF 

\href{https://www.flickr.com/photos/doctorow/sets/72157622138315932/}{Galeria zdjęć reklamowych} 


\textit{Cory Doctorow ( doctorow@craphound.com /\textit{\textit{craphound.com}}}) jest autorem kilku powieści science fiction. Niektóre są dla dorosłych, inne dla młodzieży i~dorosłych. Jest także autorem tomiku esejów (\textit{Content}, Tachyon Books), powieści graficznej (\textit{Cory Doctorow's Futuristic Tales of the Here and Now}, IDW) oraz dwóch zbiorów opowiadań, które są obecnie drukowane w~Thunder's Mouth Press.


Urodzony w~1971 w~Toronto w~Kanadzie, obecnie mieszka w~Londynie ze swoją cudowną żoną Alice i~klawą dwuletnią córką Poesy. Wcześniej pełnił funkcję dyrektora europejskiego Electronic Frontier Foundation i~jest członkiem tej organizacji. Jest również związany z~Wydziałem Informatyki Uniwersytetu Otwartego (Wielka Brytania) oraz Programem Niezależnych Studiów Uniwersytetu Waterloo (Kanada). 


Jest współredaktorem i~współwłaścicielem poczytnego bloga Boing Boing (boingboing.net) i~pisze felietony dla gazety \textit{The Guardian}, \textit{Publishers Weekly}, \textit{Locus Magazine }i \textit{Make Magazine}. 

\chapter*{Licencja Creative Commons}

Creative Commons Corporation (,,Creative Commons'') nie jest kancelarią prawną i~nie świadczy doradztwa ani usług prawnych. Rozpowszechnianie wzorca niniejszej licencji nie prowadzi do powstania stosunku zaufania w~rodzaju relacji pomiędzy klientem i~jego doradcą prawnym. Creative Commons udostępnia licencje i~związane z~nimi informacje w~stanie, w~jakim się znajdują (as-is). Creative Commons nie udziela żadnych gwarancji dotyczących licencji, licencjonowanych materiałów bądź jakichkolwiek związanych z~nimi informacji. Creative Commons w~najszerszym możliwym zakresie wyłącza swoją odpowiedzialność z~tytułu szkód mogących powstać w~wyniku użycia licencji. 

\smallskip

\textbf{Korzystanie z~Licencji Publicznych Creative Commons} 

Licencje Publiczne Creative Commons stanowią standardowy zestaw warunków, na których twórcy i~inni posiadacze praw mogą udostępniać oryginalne utwory objęte prawem autorskim lub niektórymi innymi prawami określonymi w~poniższej licencji. Niniejsze uwagi służą jedynie celom informacyjnym, nie są wyczerpujące i~nie są częścią licencji. 

\textbf{{\textmd{Uwagi dla licencjodawców:}}}{ Nasze Licencje Publiczne przeznaczone są do wykorzystania przez osoby uprawnione, w~celu udzielenia publicznego zezwolenia na wykorzystanie materiałów, których wykorzystanie w~innym wypadku byłoby ograniczone prawami autorskimi i~niektórymi innymi prawami. Nasze licencją są nieodwołalne. Licencjodawcy powinni poznać i~zrozumieć warunki wybranej licencji, zanim zdecydują się na jej użycie. Licencjodawca, zanim użyje naszych licencji, powinien także zabezpieczyć wszystkie niezbędne prawa, aby inni mogli korzystać z~udostępnionego materiału w~oczekiwany sposób. Licencjodawcy powinni wyraźnie }{oznaczyć wszelkie materiały nieobjęte niniejszą licencją. Dotyczy to zarówno odrębnych materiałów na licencji CC, jak i~materiału wykorzystywanego w~ramach wyjątku lub ograniczenia prawa autorskiego. }
\href{https://wiki.creativecommons.org/Considerations_for_licensors_and_licensees#Considerations_for_licensors}{Więcej uwag dla licencjodawców.}

\textbf{{\textmd{Uwagi dla użytkowników:}}}{ Poprzez użycie jednej z~naszych Licencji Publicznych, licencjodawca udziela publicznego zezwolenia na korzystanie z~licencjonowanego materiału na określonych warunkach. W przypadku, gdy zezwolenie licencjodawcy na określone użycie z~jakichkolwiek powodów nie jest konieczne – np. ze względu na wyjątek lub ograniczenie prawa autorskiego – użycie takie nie jest regulowane niniejszą licencją. Nasze licencje zapewniają jedynie zezwolenie, do udzielenia którego licencjodawca jest uprawniony na podstawie prawa autorskiego i~niektórych innych praw. Wykorzystanie licencjonowanego materiału może być ograniczone z~innych powodów, w~tym ze względu na prawa osób trzecich. Licencjodawca może mieć szczególne żądania, takie jak oznaczenie lub opisanie dokonanych zmian. Mimo, iż na podstawie naszych licencji nie są Państwo do tego zobowiązani, zachęcamy do uwzględniania owych życzeń w~rozsądnym zakresie. }\href{https://wiki.creativecommons.org/Considerations_for_licensors_and_licensees#Considerations_for_licensees}{Więcej uwag dla użytkowników.}

\bigskip
Creative Commons Uznanie autorstwa -- Użycie niekomercyjne -- Na tych samych warunkach 4.0 Międzynarodowa Licencja Publiczna 
\bigskip

Przystąpienie do wykonywania Uprawnień Licencyjnych oznacza akceptację i~zgodę Licencjodawcy na związanie się warunkami niniejszej licencji Creative Commons Uznanie autorstwa -- Użycie niekomercyjne -- Na tych samych warunkach 4.0 Międzynarodowa Licencja Publiczna (“Licencja Publiczna”). W zakresie, w~jakim niniejsza Licencja Publiczna może być interpretowana jako umowa, Uprawnienia Licencyjne przyznawane są Licencjobiorcy w~zamian za zgodę Licencjobiorcy na niniejsze warunki, a Licencjodawca udziela Licencjobiorcy owe uprawnienia w~zamian za korzyści płynące dla Licencjodawcy z~udostępnienia Utworu Licencjonowanego na niniejszych warunkach. 

\textbf{Paragraf 1 – Definicje.} 

\begin{enumerate}
\item  \textbf{{\textmd{Utwór Zależny}}}{ oznacza materiał objęty Prawami Autorskimi i~Prawami Podobnymi do Praw Autorskich, pochodzący od, lub opracowany na podstawie Utworu Licencjonowanego, w~którym następuje tłumaczenie Utworu Licencjonowanego, jego zmiana, aranżacja, przetworzenie lub inna modyfikacja w~taki sposób, że wymagane jest zezwolenie na podstawie Prawa Autorskiego i~Praw Podobnych do Praw Autorskich należących do Licencjodawcy. Dla celów niniejszej Licencji Publicznej, jeśli Utwór Licencjonowany jest utworem muzycznym, wykonaniem lub nagraniem dźwiękowym, Utwór Zależny powstaje zawsze, gdy Utwór Licencjonowany zostaje zsynchronizowany w~relacji czasowej z~ruchomym obrazem. } 
\item  \textbf{{\textmd{Licencja Twórcy Utworu Zależnego}}}{ oznacza licencję, którą Licencjobiorca stosuje do Praw Autorskich i~Praw Podobnych do Praw Autorskich przysługujących mu w~odniesieniu do Utworu Zależnego, zgodnie z~warunkami niniejszej Licencji Publicznej. } 
\item  \textbf{{\textmd{Licencja Kompatybilna z~BY-NC-SA}}}{ oznacza jedną z~licencji wymienionych na }\href{https://creativecommons.org/compatiblelicenses}{creativecommons.org/compatiblelicenses}{, zatwierdzoną przez Creative Commons jako istotnie odpowiadającą niniejszej Licencji Publicznej. } 
\item  \textbf{{\textmd{Prawa Autorskie i~Prawa Podobne do Praw Autorskich}}}{ oznaczają prawo autorskie i/lub prawa ściśle związane z~prawem autorskim, włącznie z, bez ograniczeń, wykonywaniem, nadawaniem, nagrywaniem dźwięku, oraz Prawami Sui Generis do Baz Danych, bez względu na to, w~jaki sposób prawa te są nazywane i~kategoryzowane. Dla }{celów niniejszej Licencji Publicznej, prawa wymienione w~Paragrafie nie są Prawami Autorskimi i~Prawami Podobnymi do Praw Autorskich. } 
\item  \textbf{{\textmd{Skuteczne Zabezpieczenia Techniczne}}}{ oznaczają środki, które, w~braku odpowiedniego zezwolenia, nie mogą być obchodzone na podstawie praw wykonujących zobowiązanie przyjęte na podstawie art. 11 Traktatu Światowej Organizacji Własności Intelektualnej o~Prawie Autorskim z~dn. 20 grudnia 1996, lub podobnych umów międzynarodowych. } 
\item  \textbf{{\textmd{Wyjątki i~Ograniczenia}}}{ oznaczają dozwolony użytek lub inne wyjątki lub ograniczenia Praw Autorskich i~Praw Podobnych do Praw Autorskich, mające zastosowanie do korzystania z~Utworu Licencjonowanego przez Licencjobiorcę. } 
\item  \textbf{{\textmd{Elementy Licencji}}}{ oznaczają atrybuty wymienione w~nazwie Licencji Publicznej Creative Commons. Elementami Licencji w~przypadku niniejszej Licencji Publicznej jest Uznanie autorstwa, Użycie Niekomercyjne, Na tych samych warunkach. }
 \item  \textbf{{\textmd{Utwór Licencjonowany}}}{ oznacza dzieło artystyczne lub literackie, bazę danych, lub inny materiał, do którego Licencjodawca zastosował niniejszą Licencję Publiczną. } 
 \item  \textbf{{\textmd{Uprawnienia Licencyjne}}}{ oznaczają uprawnienia przyznane Licencjobiorcy na warunkach niniejszej Licencji Publicznej, ograniczające się do wszystkich Praw Autorskich i~Praw Podobnych do Praw Autorskich regulujących korzystanie z~Utworu Licencjonowanego przez Licencjobiorcę, i~które Licencjodawca jest upoważniony licencjonować. } 
 \item  \textbf{{\textmd{Licencjodawca}}}{ oznacza osobę(y) fizyczną(e) lub podmiot(y) udzielający(e) praw na warunkach niniejszej Licencji Publicznej. } 
 \item  \textbf{{\textmd{Użycie Niekomercyjne}}}{ oznacza użycie nieukierunkowane na komercyjny zysk lub wynagrodzenie pieniężne. Dla celów niniejszej Licencji Publicznej, wymiana Utworu Licencjonowanego na inny materiał poddany Prawom Autorskim lub Prawom Podobnym do Prawa Autorskiego za pośrednictwem cyfrowej wymiany plików lub podobnego środka jest Użyciem Niekomercyjnym, jeśli nie jest związana z~zapłatą wynagrodzenia pieniężnego. } 
 \item  \textbf{{\textmd{Dzielenie się}}}{ oznacza udostępniane publiczne utworu jakimikolwiek środkami, wymagające zgody na podstawie Uprawnień Licencyjnych, takie jak reprodukcja, publiczne wystawianie, publiczne wykonanie, rozprowadzanie, rozpowszechnianie, komunikowanie, importowanie oraz publiczne udostępnianie utworu w~taki sposób, aby każdy mógł mieć do niego dostęp w~miejscu i~czasie przez siebie wybranym. } 
 \item  \textbf{{\textmd{Prawa Sui Generis do Baz Danych}}}{ oznaczają prawa inne niż prawo autorskie, wynikające z~Dyrektywy Parlamentu Europejskiego i~Rady Nr 96/9/WE z~11 marca 1996 r. o~prawnej ochronie baz danych, z~późniejszymi zmianami/lub zastępowanej nową, jak również inne prawa odpowiadające im istotnie, obowiązujące gdziekolwiek na świecie. }
  \item  \textbf{{\textmd{Licencjobiorca}}}{ oznacza podmiot wykonujący Uprawnienia Licencyjne na warunkach niniejszej Licencji Publicznej. } \end{enumerate}

\bigskip
\textbf{Paragraf 2 – Zakres.} 
\textbf{{\textmd{Udzielenie licencji}}}

\begin{enumerate}
\item  Zgodnie z~postanowieniami niniejszej Licencji Publicznej, Licencjodawca udziela niniejszym Licencjobiorcy, nieodpłatnej, nieobejmującej prawa do udzielania sublicencji, niewyłącznej, nieodwołalnej licencji na korzystanie z~Utworu na terytorium całego świata w~odniesieniu Utworu Licencjonowanego Uprawnień Licencyjnych do:  
\begin{enumerate} \item  zwielokrotniania i~Dzielenia się Utworem Licencjonowanym, w~całości lub w~części, wyłącznie dla celów Użycia Niekomercyjnego; i~ 
\item  tworzenia, zwielokrotniania i~Dzielenia się Utworem Zależnym wyłącznie dla celów Użycia Niekomercyjnego.  
\end{enumerate} 
\item  Wyjątki i~Ograniczenia. W razie wątpliwości, w~przypadku, gdy korzystanie przez Licencjobiorcę odbywa się w~ramach Wyjątków i~Ograniczeń, niniejsza Licencja Publiczna nie ma zastosowania, a Licencjobiorca nie jest zobowiązany do przestrzegania jej warunków.  
\item  Czas trwania. Czas trwania niniejszej Licencji Publicznej określony jest w~Paragrafie \href{about:reader?url=https%3A%2F%2Fcreativecommons.org%2Flicenses%2Fby-nc-sa%2F4.0%2Flegalcode.pl#s6a}{6(a)}.  
\item  Środki przekazu i~formaty; zezwolenie na modyfikacje techniczne. Licencjodawca upoważnia Licencjobiorcę do wykonywania Uprawnień Licencyjnych za pośrednictwem wszystkich znanych lub mających powstać w~przyszłości środków przekazu oraz we wszystkich znanych lub mających powstać w~przyszłości formatach, a także do dokonywania w~tym celu koniecznych modyfikacji technicznych. Licencjodawca zrzeka się i/lub zobowiązuje do niewykonywania prawa do zabronienia Licencjobiorcy dokonywania modyfikacji technicznych koniecznych dla wykonywania Uprawnień Licencyjnych, włącznie z~modyfikacjami technicznymi koniecznymi do obejścia Skutecznych Zabezpieczeń Technicznych. Dla celów niniejszej Licencji Publicznej, proste dokonanie modyfikacji objętych niniejszym Paragrafem \href{about:reader?url=https%3A%2F%2Fcreativecommons.org%2Flicenses%2Fby-nc-sa%2F4.0%2Flegalcode.pl#s2a4}{2(a)(4)} 
nie prowadzi do powstania Utworu Zależnego.  

\item  Dalsi odbiorcy.  
\begin{enumerate}
\item  Oferta Licencjodawcy – Utwór Licencjonowany. Każdy odbiorca Utworu Licencjonowanego automatycznie otrzymuje ofertę Licencjodawcy przystąpienia do wykonywania Uprawnień Licencyjnych zgodnie z~warunkami niniejszej Licencji Publicznej.  \item  Dodatkowa oferta Licencjodawcy – Utwór Zależny. Każdy dalszy odbiorca Utworu Zależnego otrzymanego od Licencjobiorcy automatycznie otrzymuje ofertę Licencjodawcy przystąpienia do wykonywania Uprawnień Licencyjnych do korzystania z~Utworu Zależnego zgodnie z~warunkami Licencji Twórcy Utworu Zależnego.  
\item  Żadnych ograniczeń dla dalszych odbiorców. Licencjobiorca nie może oferować ani nakładać dodatkowych lub zmienionych warunków wykorzystania Utworu Licencjonowanego, ani stosować jakichkolwiek Skutecznych Zabezpieczeń Technicznych, jeśli działania te ograniczają wykonywanie Uprawnień Licencyjnych przez jakiegokolwiek odbiorcę Utworu Licencjonowanego.  
\end{enumerate}
\item  Brak upoważnienia. Żaden element niniejszej Licencji Publicznej nie stanowi zezwolenia, ani nie może być interpretowany jako zezwolenie na sugerowanie lub stwierdzanie, że Licencjobiorca lub wykorzystanie przez Licencjobiorcę Utworu Licencjonowanego są powiązane, sponsorowane, upoważnione lub oficjalnie uznane przez Licencjodawcę lub inne podmioty wskazane w~celu uznania autorstwa, zgodnie z~postanowieniem Paragrafu \href{about:reader?url=https%3A%2F%2Fcreativecommons.org%2Flicenses%2Fby-nc-sa%2F4.0%2Flegalcode.pl#s3a1Ai}{3(a)(1)(A)(i)}.  


\item \textbf{{\textmd{Inne prawa.}}} 

\begin{enumerate}
\item  Autorskie prawa osobiste, takie jak prawo do nienaruszalności formy i~treści utworu, ani prawo do kontroli komercyjnego wykorzystania wizerunku, prawo do prywatności, lub inne podobne prawa osobiste nie są objęte niniejszą Licencją Publiczną. Jednakże, w~najszerszym możliwym zakresie, lub w~zakresie ograniczonym, pozwalającym na wykonywanie przez Licencjobiorcę Uprawnień Licencyjnych (jednak nie w~żaden inny sposób), Licencjodawca zrzeka się lub zobowiązuje do niewykonywania powyższych przysługujących Mu praw.  
\item  Prawa patentowe i~prawa do znaków towarowych nie są objęte niniejszą Licencją Publiczną.  
\item  W najszerszym możliwym zakresie, Licencjodawca zrzeka się wszelkich praw do wynagrodzenia od Licencjobiorcy za wykonywanie Uprawnień Licencyjnych, pobieranego bezpośrednio bądź za pośrednictwem organizacji zbiorowego zarządzania, na podstawie jakiegokolwiek, dobrowolnego lub zbywalnego, ustawowego lub obowiązkowego, systemu licencyjnego. W pozostałych przypadkach, Licencjodawca wyraźnie zastrzega prawo do pobierania wynagrodzenia, zwłaszcza w~razie korzystania z~Utworu Licencjonowanego w~celach innych niż Użycie Niekomercyjne.  
\end{enumerate} 

\end{enumerate}

\textbf{Paragraf 3 – Warunki Licencji.} 

Wykonywanie Uprawnień Licencyjnych przez Licencjodawcę jest wyraźnie uzależnione od przestrzegania następujących warunków. 
\begin{enumerate}
\item  
\textbf{{\textmd{Uznanie autorstwa}}}{.} 
	\begin{enumerate}
\item  W przypadku, gdy Licencjobiorca Dzieli się Utworem Licencjonowanym (również w~zmodyfikowanej formie), jest obowiązany: 
	\begin{enumerate}
\item  zachować następujące elementy, jeśli są one wskazane przez Licencjodawcę w~Utworze Licencjonowanym:  
	\begin{enumerate}
\item  identyfikację twórcy Utworu Licencjonowanego oraz jakichkolwiek innych osób, wskazanych do uzyskania atrybucji, w~rozsądny sposób, oznaczony przez Licencjodawcę (włącznie z~podaniem wskazanego pseudonimu);  
\item  informacji o~prawach autorskich;  
\item  oznaczenie niniejszej Licencji Publicznej;  
\item  oznaczenie wyłączenia gwarancji;  
\item  URI lub hiperłącze do Utworu Licencjonowanego, w~rozsądnym zakresie wyznaczonym przez możliwości techniczne;  
	\end{enumerate}
\item  oznaczyć czy Licencjobiorca wprowadził modyfikacje w~Utworze Licencjonowanym, oraz zachować oznaczenia poprzednich modyfikacji; oraz  
\item  oznaczyć Utwór Licencjonowany jako dostępny na niniejszej Licencji Publicznej, a także załączyć tekst, adres URI lub hiperłącze do niniejszej Licencji Publicznej.  			\end{enumerate} 
\item  Warunków określonych w~Paragrafie \href{about:reader?url=https%3A%2F%2Fcreativecommons.org%2Flicenses%2Fby-nc-sa%2F4.0%2Flegalcode.pl#s3a1}{3(a)(1)} 
Licencjobiorca może dochować w~każdy rozsądny sposób, stosownie do nośnika, środka przekazu i~kontekstu, w~jakim Licencjobiorca Dzieli się Utworem Licencjonowanym. Na przykład, rozsądnym dochowaniem wyżej określonych warunków może być załączenie adresu URI lub hiperłącza do źródła zawierającego wymagane informacje.  
\item  Jeśli zażąda tego Licencjodawca, Licencjobiorca obowiązany jest usunąć wszelkie informacje określone w~Paragrafie \href{about:reader?url=https%3A%2F%2Fcreativecommons.org%2Flicenses%2Fby-nc-sa%2F4.0%2Flegalcode.pl#s3a1A}{3(a)(1)(A)} 
w uzasadnionym zakresie wyznaczonym przez możliwości techniczne.  
	\end{enumerate} 
\item  \textbf{{\textmd{Na tych samych warunkach. }}} 
Oprócz obowiązku dochowania warunków określonych w~Paragrafie \href{about:reader?url=https%3A%2F%2Fcreativecommons.org%2Flicenses%2Fby-nc-sa%2F4.0%2Flegalcode.pl#s3a}{3(a)}, 
w przypadku, gdy Licencjobiorca Dzieli się stworzonym przez siebie Utworem Zależnym, Licencjobiorca obowiązany jest ponadto dochować następujących warunków. 

	\begin{enumerate}
\item  Licencja Twórcy Utworu Zależnego, stosowana przez Licencjobiorcę, musi być licencją Creative Commons składającą się z~tych samych Elementów Licencji, odpowiadających niniejszej wersji lub późniejszej, lub Licencją Kompatybilną z~BY-NC-SA.  
\item  Licencjobiorca obowiązany jest uwidocznić tekst, adres URI lub hiperłącze prowadzące do Licencji Twórcy Utworu Zależnego, stosowanej przez Licencjobiorcę. Należy dokonać tego w~odpowiedni sposób w~zależności of formatu, kontekstu i~sposobu w~jaki Licencjobiorca udostępnia Utwór Zależny. 
\item  Licencjobiorca nie może oferować lub nakładać żadnych dodatkowych lub zmienionych warunków korzystania z~Utworu Zależnego, ani stosować Skutecznych Zabezpieczeń Technicznych, jeśli działania te ograniczałyby wykonywanie praw przyznanych przez Licencję Twórcy Utworu Zależnego, stosowaną przez Licencjobiorcę.  			\end{enumerate} 
\end{enumerate}


\textbf{Paragraf 4 – Prawa Sui Generis do Baz Danych.} 
W przypadku, gdy Uprawnienia Licencyjne obejmują Prawa Sui Generis do Baz Danych, mające zastosowanie do określonego korzystania z~Utworu Licencjonowanego przez Licencjobiorcę: 
\begin{enumerate}
\item  w~razie wątpliwości, Paragraf \href{about:reader?url=https%3A%2F%2Fcreativecommons.org%2Flicenses%2Fby-nc-sa%2F4.0%2Flegalcode.pl#s2a1}{2(a)(1)} 
przyznaje Licencjobiorcy prawo do wyodrębniania, ponownego wykorzystania, zwielokrotniania i~Dzielenia się całością lub istotną częścią bazy danych, jedynie dla celów Użycia Niekomercyjnego;  
\item  w~przypadku, gdy Licencjobiorca zawiera całość lub istotną część bazy danych w~bazie danych, do której Prawa Sui Generis do Baz Danych przysługują Licencjobiorcy, baza danych, do której Prawa Sui Generis do Baz Danych przysługują Licencjobiorcy, stanowi Utwór Zależny (jednakże nie dotyczy to jej samodzielnych części), w~tym także dla celów Paragrafu \href{about:reader?url=https%3A%2F%2Fcreativecommons.org%2Flicenses%2Fby-nc-sa%2F4.0%2Flegalcode.pl#s3b}{3(b)}; 
oraz  
\item  w~przypadku, gdy Licencjobiorca Dzieli się całością lub istotną częścią bazy danych, obowiązany jest przestrzegać warunków Paragrafu \href{about:reader?url=https%3A%2F%2Fcreativecommons.org%2Flicenses%2Fby-nc-sa%2F4.0%2Flegalcode.pl#s3a}{3(a)}  
\end{enumerate}

W razie wątpliwości, niniejszy Paragraf \href{about:reader?url=https%3A%2F%2Fcreativecommons.org%2Flicenses%2Fby-nc-sa%2F4.0%2Flegalcode.pl#s4}{4} 
uzupełnia, lecz nie zastępuje obowiązków Licencjodawcy wynikających z~niniejszej Licencji Publicznej, w~przypadku, gdy Uprawnienia Licencyjne obejmują inne Prawa Autorskie i~Prawa Podobne do Praw Autorskich.  

\textbf{Paragraf 5 – Wyłączenie Gwarancji i~Ograniczenie Odpowiedzialności.} 
\begin{enumerate}
\item  \textbf{{\textmd{Jeżeli Licencjodawca oddzielnie nie postanowił inaczej, Licencjodawca, w~możliwie najszerszym zakresie, oferuje Utwór Licencjonowany w~takiej formie, w~jakiej zapoznał się z~nim Licencjobiorca i~nie udziela żadnych zapewnień, ani jakiegokolwiek rodzaju gwarancji, dotyczących Utworu Licencjonowanego, ani wynikających z~wyraźnego postanowienia, dorozumianych, ustawowych, ani jakichkolwiek innych. Obejmuje to, bez ograniczeń, rękojmię, zbywalność, przydatność do konkretnego celu, brak naruszeń praw innych osób, brak ukrytych lub innych wad, dokładność, występowanie lub niewystępowanie wad widocznych jak i~ukrytych. W przypadku, gdy wyłączenie }}}\textbf{{\textmd{gwarancji nie jest dozwolone w~całości lub w~części, niniejsze wyłączenie może nie mieć zastosowania do Licencjobiorcy.}}}{ } 
\item  \textbf{{\textmd{W najdalej idącym stopniu, w~żadnym wypadku Licencjodawca nie odpowiada wobec Licencjobiorcy na żadnej podstawie prawnej (włączając w~to, bez ograniczeń, niedochowanie należytej staranności) za bezpośrednie, specjalne, pośrednie, przypadkowe, następcze, karne, ani żadne inne straty, koszty, utracone korzyści, wydatki, ani szkody wynikające z~zastosowania niniejszej Licencji Publicznej lub korzystania z~Utworu Licencjonowanego, nawet w~przypadku, gdy Licencjodawca był powiadomiony o~możliwości poniesienia takich strat, kosztów, wydatków, lub szkód. W przypadku, gdy ograniczenie odpowiedzialności nie jest dozwolone w~całości lub w~części, takie ograniczenie nie ma zastosowania do Licencjobiorcy.}}}{ } \end{enumerate}
\begin{enumerate}
\item  Powyższe wyłączenie gwarancji i~ograniczenie odpowiedzialności będzie interpretowane w~sposób zapewniający wyłączenie i~zrzeczenie się odpowiedzialności w~zakresie możliwie najszerszym.  \end{enumerate}
\textbf{Paragraf 6 – Termin i~Wygaśnięcie.} 
\begin{enumerate}
\item  Niniejsza Licencja Publiczna jest udzielona na czas trwania licencjonowanych Praw Autorskich i~Praw Podobnych do Praw Autorskich. Jednakże, jeśli Licencjobiorca naruszy postanowienia niniejszej Licencji Publicznej, prawa Licencjobiorcy wynikające z~niniejszej Licencji Publicznej wygasają automatycznie.  
\item  Jeżeli prawo Licencjobiorcy do korzystania z~Utworu Licencjonowanego wygasło na podstawie Paragrafu \href{about:reader?url=https%3A%2F%2Fcreativecommons.org%2Flicenses%2Fby-nc-sa%2F4.0%2Flegalcode.pl#s6a}{6(a)}, 
zostaje ono przywrócone: 
\begin{enumerate}
\item  automatycznie z~datą usunięcia naruszenia, jeśli nastąpiło ono w~ciągu 30 dni od odkrycia naruszenia przez Licencjobiorcę; lub  
\item  w~razie wyraźnego przywrócenia przez Licencjodawcę.  \end{enumerate}
W razie wątpliwości, niniejszy Paragraf \href{about:reader?url=https%3A%2F%2Fcreativecommons.org%2Flicenses%2Fby-nc-sa%2F4.0%2Flegalcode.pl#s6b}{6(b)} 
nie ogranicza żadnych uprawnień Licencjodawcy przysługujących mu w~razie naruszenia przez Licencjobiorcę niniejszej Licencji Publicznej.  
\item  W razie wątpliwości, Licencjodawca może także oferować Utwór Licencjonowany na odrębnych warunkach, bądź przestać dystrybuować Utwór Licencjonowany w~każdej chwili; nie prowadzi to jednak do wygaśnięcia niniejszej Licencji Publicznej.  
\item  Paragrafy \href{about:reader?url=https%3A%2F%2Fcreativecommons.org%2Flicenses%2Fby-nc-sa%2F4.0%2Flegalcode.pl#s1}{1}, \href{about:reader?url=https%3A%2F%2Fcreativecommons.org%2Flicenses%2Fby-nc-sa%2F4.0%2Flegalcode.pl#s5}{5}, \href{about:reader?url=https%3A%2F%2Fcreativecommons.org%2Flicenses%2Fby-nc-sa%2F4.0%2Flegalcode.pl#s6}{6}, \href{about:reader?url=https%3A%2F%2Fcreativecommons.org%2Flicenses%2Fby-nc-sa%2F4.0%2Flegalcode.pl#s7}{7} i~\href{about:reader?url=https%3A%2F%2Fcreativecommons.org%2Flicenses%2Fby-nc-sa%2F4.0%2Flegalcode.pl#s8}{8} 
trwają również po wygaśnięciu niniejszej Licencji Publicznej.  \end{enumerate}
\textbf{Paragraf 7 – Inne Warunki.} 
\begin{enumerate}
\item  Licencjodawca nie jest związany żadnymi dodatkowymi lub zmienionymi warunkami zakomunikowanymi przez Licencjobiorcę, chyba że dosłownie wyrazi na to swoją zgodę.  \item  Jakiekolwiek ustalenia, porozumienia, umowy odnoszące się do Utworu Licencjonowanego, nie wyrażone w~niniejszej Licencji Publicznej, nie stanowią jej postanowień i~są od niej niezależne.  
\end{enumerate}
\textbf{Paragraf 8 – Wykładnia.} 
\begin{enumerate}
\item  W razie wątpliwości, niniejsza Licencja Publiczna nie zawęża, nie ogranicza, ani nie warunkuje jakiegokolwiek dozwolonego wykorzystania Utworu Licencjonowanego, mogącego odbywać się zgodnie z~prawem bez zezwolenia wynikającego z~niniejszej Licencji Publicznej, ani nie może być w~ten sposób interpretowana.  
\item  Jeżeli jakiekolwiek postanowienie niniejszej Licencji Publicznej uważa się za nieskuteczne, będzie ono automatycznie, na ile to tylko możliwe, w~jak najmniejszym, koniecznym stopniu przekształcone tak, aby stało się skuteczne. Jeżeli postanowienie takie nie może zostać przekształcone, zostanie ono unieważnione bez wpływu na skuteczność pozostałych postanowień niniejszej Licencji Publicznej.  
\item  Żadne postanowienie niniejszej Licencji Publicznej nie będzie uchylone, ani żadne naruszenie jej postanowień dozwolone, bez wyrażonej dosłownie zgody Licencjodawcy.  
\item  Niniejsza Licencja Publiczna nie stanowi, ani nie może być interpretowana jako ograniczenie lub zrzeczenie się jakichkolwiek przywilejów Licencjodawcy lub Licencjobiorcy, w~tym immunitetów procesowych względem jakiejkolwiek władzy jurysdykcyjnej.  
\end{enumerate}

Creative Commons nie jest stroną udostępnianych przez siebie Licencji Publicznych. Jednakże, Creative Commons może zastosować jedną ze swych Licencji Publicznych do publikowanego przez siebie materiału, i~w tych przypadkach będzie występować jako Licencjodawca. Tekst Licencji Publicznych Creative Commons jest przekazany do domeny publicznej na podstawie \href{https://creativecommons.org/publicdomain/zero/1.0/legalcode.pl}{CC0 Public Domain Dedication}. 

Za wyjątkiem ograniczonych celów oznaczania materiału jako dostępnego na Licencji Publicznej Creative Commons lub innego użycia dozwolonego przez polityki Creative Commons udostępnione na \href{https://creativecommons.org/policies}{creativecommons.org/policies}, Creative Commons nie zezwala na korzystanie ze znaku towarowego „Creative Commons”, ani żadnego innego znaku towarowego, lub logo Creative Commons, bez uprzedniej pisemnej zgody, w~tym, bez ograniczeń, w~związku z~jakimikolwiek nieautoryzowanymi modyfikacjami którejkolwiek z~Licencji Publicznych Creative Commons, lub innych ustaleń, porozumień lub umów dotyczących wykorzystania licencjonowanego materiału. W razie wątpliwości, niniejszy akapit nie stanowi części Licencji Publicznych. 

Z Creative Commons można skontaktować się pod adresem \href{https://creativecommons.org/}{creativecommons.org}. 

\chapter*{Seria ,,Czarny Kot''}



\begin{center}
\includegraphics[width=0.2\textwidth]{Anarchist_black_cat.png}

\begin{large}
W serii \textit{Czarny Kot} opublikowano online:
\end{large} 
\end{center}

\begin{enumerate}
\item \href{https://archive.org/details/errico-malatesta-w-kawiarni}{W Kawiarni}, Errico Malatesta
\item \href{https://archive.org/details/david-graeber-akcja-bezposrednia}{Akcja Bezpośrednia:Etnografia}, David Graeber
\item Dla Wygranej, Cory Doctorow
\end{enumerate}


\begin{center}

\begin{large}W planach:\end{large}\end{center}

\begin{enumerate}

\item Triton, Samuel R. Delany  
\item Dhalgren, Samuel R. Delany
\end{enumerate}


\newpage

Projekt serii został przygotowywany dzięki Wolnemu Oprogramowaniu. Zestaw narzędzi składa się z:
\begin{itemize}
\item \href{https://ubuntu.com/}{Ubuntu 23.04 Lunar Lobster} --- system operacyjny
\item \href{https://omegat.org/}{OmegaT} --- narzędzie wspomagające tłumaczenie (CAT)
\item \href{https://github.com/soimort/translate-shell}{translate-shell} --- narzędzie do tłumaczenia w~\href{https://translate.google.pl}{Google Translate} przez terminal 
\item \href{https://glosbe.com/en/pl}{Glosbe} --- największy słownik online
\item \href{https://www.wikipedia.org/}{Wikipedia} --- podstawowe źródło tłumaczeń pojęć technicznych, politycznych i~ekonomicznych czy not biograficznych
\item \href{https://www.libreoffice.org/}{LibreOffice} --- przetwarzanie dokumentów 
\item \href{http://pandoc.org}{pandoc} --- uniwersalny konwerter dokumentów 
\item \href{https://www.latex-project.org/}{LaTeX} --- redakcja, skład i~łamanie dokumentu
\item \href{https://sigil-ebook.com/}{sigil} --- przetwarzanie plików ebook
\item \href{https://calibre-ebook.com/}{calibre} --- konwersja plików ebook
\end{itemize}


\end{document}
