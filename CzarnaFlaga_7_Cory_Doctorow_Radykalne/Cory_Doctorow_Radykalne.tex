\documentclass[oneside,polish,11pt,sfheadings]{mwbk}
%polonizacja
\usepackage[T1]{fontenc}
\usepackage[polish]{babel}
\usepackage[utf8]{inputenc}
\usepackage{polski} 
\frenchspacing 
\usepackage{indentfirst} 
%koniec polonizacja
%grafika
\usepackage{graphicx}
%pakiet czcionki
\usepackage{times}
\usepackage[a5paper]{geometry} %wielkość papieru (148x210-book w~PL)
%gwiazdki
\newcommand{\threeast}{\bigskip\par\centerline{*\,*\,*}\medskip\par}

%EPUB
%\usepackage[hyperfootnotes=true]{hyperref} 
%move footnotes to endnotes
%\usepackage{enotez}
%\let\footnote=\endnote
%\setenotez{
%  list-name = Przypisy,
%  backref = true
%}

%pdf anonimize
%dla EPUB wykomentować
\pdfsuppressptexinfo=-1 %Suppress PTEX.Fullbanner and info of imported PDFs

%pakiet odnośników i~pdf metadata
\usepackage[unicode, pdftex]{hyperref}
\hypersetup{pdfauthor={Cory Doctorow},
            pdftitle={Radykalne},
            pdfsubject={Radicalized},
            pdfkeywords={tłum. Jacek Hummel, Creative Commons, tłumaczenie CC BY 4.0, antologia, science fiction},
            pdfcreator={pdfLaTeX}}
%dla EPUB koniec wykomentowania


\begin{document}

\title{Radykalne}
\author{Cory Doctorow}


%--titlepage start
\DeclareRobustCommand{\cs}[1]{\texttt{\char`\\#1}}
\newlength{\tpheight}\setlength{\tpheight}{0.9\textheight}
\newlength{\txtheight}\setlength{\txtheight}{0.9\tpheight}
\newlength{\tpwidth}\setlength{\tpwidth}{0.9\textwidth}
\newlength{\txtwidth}\setlength{\txtwidth}{0.9\tpwidth}
\newlength{\drop}
\newcommand*{\titleSI}{\begingroup% Sagas
\drop = 0.13\txtheight
\centering
{\Huge \textsf{~}}\\[1\baselineskip]
{\huge \textsf{~}}\\[1\baselineskip]
%{\LARGE  \textsf{~}}\\[4\baselineskip]
{\Huge \textsc{Radykalne}}\\[1\baselineskip]
{\LARGE \textsc{Radicalized}}\\[2\baselineskip]
{\huge \textsc{Cory Doctorow}}\\[4\baselineskip]
{\large Na podstawie wydania TOR, Nowy Jork, 2019 \\ przetłumaczył i~opracował:}\\
{\Large Jacek Hummel}\\[1\baselineskip]
{\normalsize \textit{Tłumaczenie jest dostępne na licencji\\
\href{https://creativecommons.org/licenses/by/4.0/deed.pl}{Creative Commons Uznanie autorstwa 4.0 Międzynarodowe}} \\ [1\baselineskip] \par}
\includegraphics[scale=0.3]{CC.png}

~

\vfill
{\Large {Warszawa, 2021}}\\
%\vspace*{\drop}
\endgroup}
\titleSI
\thispagestyle{empty}
%--titlepage end

\begin{figure}[p]
    \vspace*{-1cm}
    \makebox[\linewidth]{
        \includegraphics[width=1.1\linewidth]{doctorow-radicalized.jpeg}
    }
\end{figure}
\thispagestyle{empty}

\newpage

\thispagestyle{empty}
\vspace*{\fill}
\textit{Moim rodzicom: Roz i~Gord Doctorow, którzy nauczyli mnie, dlaczego
walczymy i, by się nie poddawać.}\\

\textit{To nie jest rodzaj walki, którą wygramy, to rodzaj walki, w~której walczymy.}

\vspace*{\fill}


\chapter*{Nieautoryzowany Chleb}

Sposób, w~jaki Salima dowiedziała się, że Boulangism\footnote{ od franc.
boulangerie -- piekarnia -- przyp.tłum.} zbankrutował: jej piekarnik\footnote{
oryg. toaster czyli toster, jednak w~dalszej części autor pisze o~,,toaster oven'' czyli minipiekarniku (ok. 20 l) elektrycznym, stąd
minipiekarnik -- przyp.tłum.} przestał akceptować jej chleb. Trzymała
kromkę przed nim i~czekała, żeby pokazał jej na ekranie emoji ,,kciuk do
góry'', ale zamiast tego, pokazał ,,drapanie się po głowie'' i~zrobił
miękkie \textit{brrt}. Znowu pomachała chlebem. \textit{Brrt}.

-- \textit{Dawaj}. 

-- \textit{Brrt}.

Wyłączyła i~włączyła minipiekarnik. Potem odłączyła go z~gniazdka,
policzyła do dziesięciu i~znowu podłączyła. Potem przeszła przez menu,
aż pojawił się ekran ,,PRZYWRÓCENIE USTAWIEŃ FABRYCZNYCH'', poczekała
trzy minuty i~wbiła znowu hasło do swojego wi-fi.

\textit{Brrt}.

Już wcześniej, zanim doszła do tego punktu, doszła do wniosku, że to
była przegrana sprawa. Jednak to były te kroki, które człowiek robi,
kiedy elektronika przestaje działać, żeby móc zadzwonić na infolinię i~powiedzieć: 

-- Wyłączyłam i~włączyłam, odłączyłam w~ogóle od gniazdka,
zresetowałam ustawień fabrycznych i\ldots 

Na ekranie dotykowym minipiekarnika widniała opcja telefonu do wsparcia,
ale nie działała, więc użyła lodówki, żeby znaleźć numer i~zadzwonić.
Zadzwoniło siedemnaście razy, potem zakończyło połączenie. Ciężko
westchnęła. \textit{Kolejny poszedł do piachu}\footnote{ w~oryg. słowa piosenki
Queen, ,,Another one bites the dust'' -- przyp.tłum.}

Minipiekarnik nie był pierwszym urządzeniem, które zamarło (ten honor
należał do zmywarki, które przestała sprawdzać, czy talerze pochodzą z~innych źródeł tydzień przed bankructwem Disher), ale to \textit{była}
kropla, która przepełniła czarę. Mogła myć naczynia w~zlewie, ale jak do
cholery miała zrobić sobie tosta, nad świeczką?

Żeby się upewnić, zapytała lodówkę o~nagłówki o~Boulangism i~oto były,
ich chmura wybuchła w~nocy. Serwisy społecznościowe zapełnione
wściekłymi ludźmi na ich codzienny chleb. Przesunęła nagłówki i~dowiedziała się, że Boulangism była statkiem widmem przez co najmniej
sześć miesięcy, ponieważ tak długo śledczy bezpieczeństwa próbowali
skontaktować się z~firmą, żeby jej przekazać, że wszystkie dane
użytkowników -- hasła, loginy, szczegóły zamówień i~faktur -- wisiały,
niezaszyfrowane i~bez hasła, w~publicznym internecie. W~bazie danych
były \textit{żądania okupu}, rekordy wprowadzone przez hakerów żądających
wypłat w~kryptowalutach w~zamian za zachowanie w~tajemnicy gównianego
przetwarzania danych przez Boulangism. Nikt ich nawet nie widział.

Cena akcji Boulangism przez ostatni rok spadła o~98 procent. Już więcej
mogło nawet nie \textit{być} Boulangism. Kiedy Salima wyobrażała sobie
Boulangism, wyobrażała sobie francuską piekarnię, które była na
przedstawiana na ekranie gotowości minipiekarnika, omączone, drewniane
stoły ze zwartymi szeregami chrupiących bochenków chleba. Wyobrażała
sobie chwiejne schody prowadzące z~piekarni do zatłoczonych biur
wyglądających na brukowaną drogę. Wyobrażała sobie lampy gazowe.

Artykuł zawierał zdjęcie centrali Boulangism z~ulicy, czteropiętrowy
biurowiec w~Pune, koło Mumbaju, ogrodzony z~niestrzeżoną budką
ochroniarza na poziomie ulicy.

Chmura Boulangism wybuchła i~to oznaczało, że nikt nie odpowiadał
minipiekarnikowi Salimy, kiedy pytał, czy chleb, który miała podpiec,
pochodził od piekarza (pochodził) autoryzowanego przez Boulangism. Z~braku odpowiedzi, paranoidalny mały gadżet założyłby, że Salima należała
do klasy nikczemnych oszustów, którzy kupowali przecenione
minipiekarniki Boulangism i~próbowali wycofać się po swojej stronie
umowy przez wkładanie nieautoryzowanego chleba, czego konsekwencje mogły
zawierać się od pożaru w~kuchni do nieoptymalnego tosta (Boulangism był
w stanie dopasować procedurę podpiekania w~czasie rzeczywistym, żeby
skorygować względną wilgotność kuchni i~wiek chleba, oczywiście
odmówiłby zagrzania chleba, który był czerstwy nie do uratowania), nie
mówiąc o~utracie zysków przedsiębiorstwa i~jego udziałowców. Bez tych
zysków, nie byłoby nadwyżek kapitału przesuniętych na B+R\footnote{oryg. R\&D,
badania i~rozwój -- przyp.tłum.}, tworzących ciągłe ulepszenia, które
oznaczały, że rzadko który dzień mijał bez przebudzenia się Salimy i~milionów innych interesariuszy (nigdy po prostu ,,klientów'')
ekscytującym nowym firmwarem na ich ukochanych urządzeniach.

A co z~piekarniami, partnerami Boulangism? Zrobili właściwą rzecz,
podpisując licencję Boulangism, poddając się procesowi inspekcji i~kontroli jakości, która oznaczała, że ich chleb miał dokładnie właściwy
skład, żeby tostować go \textit{doskonale} w~precyzyjnie zaprojektowanych
urządzeniach Boulangism, miękisz i~porowatość w~doskonałej równowadze,
żeby wchłonąć masło i~inne pasty. Ci cenieni partnerzy zasługiwali, by
ich zaangażowanie w~doskonalenie było uhonorowane, a~nie porzucone przez
oszustów polujących na okazje, którzy lekkomyślnie chcieli tostować
każdy stary chleb.

Salima znała te argumenty, nawet zanim jej głupi minipiekarnik
wyświetlił jej wyjaśniające wideo, co robił po trzech nieudanych próbach
autoryzacji chleba, odtwarzając je bez przycisku pauzy czy wyciszenia
jako kombinacja kary i~kampanii reedukacyjnej.

Próbowała wyszukać na lodówce ,,hakowanie boulangism'' i~,,kody anty
boulangism'', ale urządzenia trzymały się razem. Filtry sieciowe
KitchenAid wchłaniały jej zapytania i~wypluwały sarkastyczne ekrany
,,brak wyników'', mimo że Salima wiedziała doskonale, że istniała cała
podziemna ekonomia poświęcona nieautoryzowanemu pieczywu.

Musiała wyjść do pracy za pół godziny i~jeszcze nie wzięła prysznica,
ale, cholera, najpierw zmywarka, a~teraz toster. Znalazła laptop,
używany, kiedy go dostała, teraz ledwie działający. Bateria dawno się
zużyła i~musiała odłączyć szczoteczkę do zębów, żeby zwolnić kabel
ładujący, ale kiedy już włączyła i~pozwoliła uruchomić się kilkunastu
aktualizacjom oprogramowania, była w~stanie uruchomić przeglądarkę
darknetu\footnote{ część sieci internetowej, która jest niepubliczna i~wymaga
dodatkowych działań w~celu podłączenia
zob.~\url{https://www.comparitech.com/blog/vpn-privacy/access-dark-web-safely-vpn/}, \url{https://en.wikipedia.org/wiki/Darknet} lub \href{https://www.webhostingsecretrevealed.net/pl/blog/web-tools/tourist-guide-to-dark-web-accessing-the-dark-web-tor-browser-and-onion-websites/}{Przewodnik po Darkweb}
 -- przyp.tłum.}, którą ciągle miała i~rozsądnie poszukać.

Tego dnia spóźniła się do pracy czterdzieści pięć minut, ale miała tosta
na śniadanie. Cholera.

Zmywarka była następna. Kiedy Salima znalazła właściwe forum, byłoby
szaleństwem \textit{nie} odblokowanie rzeczy. W~końcu, musiała z~niej
korzystać i~teraz była faktycznie zepsuta\footnote{ w~oryg. bricked, od cegły,
czyli tego, w~co zamienia się urządzenie, termin oznaczający
nieaktywności stanu urządzenia elektronicznego (brak zasilania, brak
reakcji na przyciski) -- przyp.tłum.}. Nie była też jedyną, której
dotyczyła podwójna klątwa Disher-Boulangism. Niektórzy biedni frajerzy
mieli nieszczęście posiadać jeden z~zestawu urządzeń produkowanych przez
HP-NewsCorp -- lodówki, szczoteczki do zębów, nawet seks-zabawki -- które
wszystkie przestały działać dzięki awarii dostawcy usług chmurowych,
Tata. Choć ta awaria była niezwiązane z~dublem Disher-Boulangism,
wszyscy się zgadzali, że było to dość niefortunne wyczucie czasu.

Wspólny upadek Disher i~Boulangism \textit{miały} wspólną przyczynę, jak
odkryła Salima. Obie firmy były spółkami notowanymi na giełdzie i~więcej
niż dwadzieścia procent ich udziałów zostało nabytych przez Summerstream
Funds Management, największy fundusz hedgingowy na ziemi, zarządzający
184 miliardami dolarów. Summerstream był ,,aktywnym udziałowcem'' i~był
gigantem skupowania akcji. Kiedy fundusz otrzymał prawo do zasiadania w~radach obu firm -- w~obu przypadkach zajmowanych przez Galta
Baumgardnera, młodszego partnera firmy, ale z~bardzo dobrej rodziny w~Kansas -- obie firmy wynajmowały tego samego konsultanta z~Deloitte, żeby
sprawdził konta firm i~zarekomendował program wykupu akcji, który
oznaczałby, że akcjonariusze otrzymają swoją należność, ale bez
grzebania w~kapitale operacyjnym przedsiębiorstw, co mogło im zagrozić.

Oczywiście, wszystko to było matematycznie do udowodnienia. Firmy mogły
spokojnie przesunąć miliardy z~bilansów księgowych do akcjonariuszy.
Kiedy to było przesądzane, obowiązkiem powierniczym członków zarządu,
było głosowanie za tym (co było poręczne, skoro wszyscy posiadali grube
pliki akcji spółki) i~kilka miliardów dolarów później, spółki były
szczupłe, twarde i~gotowe do walki, i~w ogóle nie żałowały tych środków.

Ojej.

Summerstream opublikował komunikat prasowy (często cytowany na forach,
które Salima teraz obsesyjnie przeszukiwała) obwiniając
,,nieprzewidywalność'' i~nazywając to ,,niefortunnym'' i~,,niezadowalającym''. Byli przekonani, że oba przedsiębiorstwa
dokonałyby restrukturyzacji przy upadłości, może po szybkiej sprzedaży
do konkurencji, i~wszyscy znowu zaczęliby robić tosty i~zmywać naczynia
w ciągu kilku miesięcy.

Salima nie zamierzała czekać. Jej Boulangism nie puścił łatwo. Po
ściągnięciu nowego firmware'u z~darknetu, musiała zdjąć obudowę
(zrywając trzy oddzielne plomby zabezpieczające przed manipulacją i~wielką naklejkę ostrzegającą przed porażeniem prądem i~dochodzeniem
sądowym, może jednocześnie, każdą osobę dostatecznie głupią, żeby to
zignorować) i~odszukać określoną część, potem zewrzeć dwa piny pincetą w~trakcie uruchamiania. To działanie włączyło w~minipiekarniku tryb
testowy, który deweloperzy wyłączyli, ale nie usunęli. Natychmiast, gdy
pojawił się ekran testowy, musiała wcisnąć pendrive USB (zdjęcie obudowy
minipiekarnika odsłoniło zestaw gniazd USB, łącze do monitora i~nawet
mały port LAN, wszystkie upchnięte na jednej płytce PC, które je
kontrolowała) w~dokładnie właściwym czasie, potem użyć klawiatury na
ekranie, żeby wprowadzić login i~hasło, które brzmiały ,,admin'' i~,,admin'' (oczywiście).

Zajęło to jej trzy próby, żeby trafić właściwie w~czas, ale przy
trzeciej próbie, ekran logowania został zastąpiony przez kiepską
tekstową animację pirackiego firmware w~kształcie czaszki 3D na widok,
której Salima się uśmiechnęła, a~potem wybuchła śmiechem, gdy tekstowy
tost pojawił się w~kadrze i~był radośnie schrupany przez czaszkę,
okruchy spadające na dół ekranu i~tworzące przesuwające się małe stosy.
Ktoś się mocno wysilił w~symulacji fizyki tej śmiesznej animacji. To
sprawiła, że Salima poczuła się dobrze, jakby powierzała minipiekarnik,
poważnym i~doświadczonym rzemieślnikom, a~nie byle komu, kto lubił
stawać w~szranki przeciwko anonimowym programistom z~wielkich, głupich
firm.

Okruchy się zbierały, gdy czaszka chrupała, a~wskaźnik postępu pokazywał
od 12 procent, potem 26 procent, potem 34 procent (gdzie zatrzymał się
na całe dziesięć minut, póki nie była gotowa zaryzykować naprawdę
zbrickowania tej przeklętej rzeczy przez odłączenie jej, ale wtedy\ldots )
58 procent i~tak dalej, aż do dręczącego czekania przy 99 procentach, a~potem wszystkie okruchy ruszyły z~dna ekranu i~wyszły przez usta
czaszki, zamieniając się \textit{z powrotem} w~tosta, każdy powtórnie
zbudowany w~rzędach, który szybko wymazał czaszkę, słowa ,,ALL DONE''
wypaliły się na powierzchni tosta, lśniąc od masła, które spłynęło w~dół
strumykami. Właśnie chwytała telefon, żeby zrobić zdjęcie tego
niesamowitego pirackiego ekranu, kiedy minipiekarnik błysnął diodami i~się zrestartował.

Kilka sekund później trzymała kawałek chleba przed czujnikiem
minipiekarnika i~obserwowała, jak dioda zaświeca się na zielono i~drzwiczki się otwierają. W~połowie chrupania tosta, wpadła na dziwną
myśl. Wystawiła dłoń przed tosterem, palce płasko, jakby, też, były
kawałkiem chleba. Dioda minipiekarnika zmieniła się na zielony i~otworzyły się drzwiczki. Natychmiast poczuła pokusę spróbowania
podpieczenia widelca, ręcznika papierowego lub kawałka jabłka, tylko
żeby sprawdzić, czy toster by to zrobił, ale oczywiście, że by to
zrobił.

To był nowy rodzaj minipiekarnika, który wykonywał polecenia, a~nie je
wydawał. Minipiekarnik, który sprzedałby jej linę, na której mogła się
powiesić, pozwoliłby jej upiec baterię litową lub puszkę lakieru do
włosów, lub cokolwiek innego, co chciała piec: nieautoryzowany chleb.
Nawet chleb domowej roboty. Pomysł sprawił, że poczuła lekkie mdłości i~lekkie drżenie. Chleb domowej roboty był czymś, o~czym czytała w~książkach, widziała w~starych serialach, ale nie znała nikogo, kto
właściwie wypiekał chleb. To byłoby jak robienie własnych mebli z~całych
kłód czy coś.

Składniki okazały się niesamowicie proste i~choć jej pierwszy bochenek
wyszedł, jakby był emoji ,,kupa'', smakował \textit{niesamowicie}, ciągle
ciepły od małego tostera, i~jeżeli cokolwiek, kromka (ok, kawał) który
zachowała i~opiekła następnego ranka smakował jeszcze lepiej,
szczególnie z~masłem. Wyszła do pracy tego dnia z~magicznym, ciepłym,
\textit{tostowym} uczuciem w~żołądku.

Tej nocy przerobiła zmywarkę. Hakerzy Disher byli w~podejściu bardziej
utylitarni, ale również byli Szwedami, sądząc z~URL w~ich pliku README,
co mogło tłumaczyć minimalizm. Była w~IKEI, rozumiała to. Disher nie
wymagał niczego takich tańców jak Boulangism: zdjęła osłonę, wysunęła
gumową zatyczkę z~portu USB, włożyła pendrive i~zrestartowała zmywarkę.
Ekran pokazał dużo przesuwającego się tekstu i~jakieś tajemne informacje
o błędach, potem znowu się zrestartował w~coś, co wyglądało jak zmywarka
w normalnym trybie, prócz bez czerwonych ostrzeżeń o~nieosiągalnym
serwerze, które prześladowały ją przez tydzień. Ułożyła talerze ze zlewu
w zmywarce, czując lekki dreszczyk za każdym razem, gdy zmywarka
odgrywała melodyjkę ,,Rozpoznano nowy talerz''.

Pomyślała, żeby następnie umyć ceramikę.

Jej doświadczenie ze zmywarką i~minipiekarnikiem zmieniło ją, choć z~początku nie potrafiła powiedzieć jak. Następnego dnia wychodząc z~mieszkania, zauważyła, że przygląda się windzie, patrząc na płytkę
sterowania ręcznego straży pożarnej pod ekranem wzywania, myśląc o~tym,
że lokatorzy na dotowanych piętrach musieli czekać trzy razy dłużej na
windę, ponieważ mieli prawo jeździć tylko windami, które miały drzwi
otwierające się także do tyłu na tylny hol prowadzący do drzwi dla
biedaków. Nawet te windy nie zatrzymałyby się na jej piętrze, jeżeli
zabrały kogoś z~pełnopłatnych mieszkańców po drodze, ponieważ niebo
niech broni, żeby ci ludzie musieli oddychać tym samym powietrzem co
plugawy plebs.

~

Salima były uszczęśliwiona, gdy dostała mieszkanie w~tym budynku,
Wieżowcach Dorchester, ponieważ lista oczekujących na dotowane
mieszkania, które były wymogiem Departamentu Architektury wobec
dewelopera, była długa na lata. W~tym czasie była w~kraju około
dziesięciu lat, spędzając pierwsze pięć lat w~obozie w~Arizonie, gdzie
obserwowali śmierć jednej po drugiej osoby w~miażdżącym gorącu. Kiedy
Departament Stanu w~końcu zakończył lustrowanie i~ją wypuścił, pracownik
socjalny spotkał się z~nią, przekazując torbę rzeczy, przedpłaconą kartą
debetową i~wiadomość, że jej rodzice zmarli, kiedy była w~obozie.

Przyjęła wiadomości w~ciszy i~nie pozwoliła sobie na okazanie
jakiegokolwiek znaku cierpienia. Zakładała, że jej rodzice nie żyją,
ponieważ obiecali jej dołączyć do niej w~Arizonie w~ciągu miesiąca od
jej przybycia, gdy tylko jej ojciec upomniałby się o~stare długi i~zapłacił za dokumenty i~korektę bazy danych, która umożliwiłaby przelot
samolotem do Punktu Kontrolnego Imigracji USA, gdzie mogliby wnieść o~azyl. Była wtedy nastolatką, a~teraz była młodą kobietą, z~doświadczeniem pięciu lat trudnego życia w~obozie. Wiedziała, jak
kontrolować płacz. Podziękowała pracownicy i~spytała, co się stało z~ciałami.

-- Zaginęli na~morzu -- powiedziała kobieta i~przywdziała współczującą
minę. -- Statek i~wszyscy pasażerowie. Nikt nie ocalał. Włosi przeszukiwali
obszar tygodniami i~nic nie znaleźli. Wrak poszedł prosto na dno.
Powiedzieli, że zła informatyka. -- Statek był komputerem, do którego
wsadzałeś zdesperowanych ludzi, a~kiedy komputer się zepsuł, statek był
grobem, do którego wsadziłeś zdesperowanych ludzi.

Skinęła głową, jakby rozumiała, choć dźwięk krwi w~uszach był tak
głośny, że nie słyszała własnych myśli. Pracownica socjalna mówiła
więcej, dała jej dokumenty, które zawierały tani bilet autobusowy do
Bostonu, gdzie znalazłaby łóżko w~ośrodku dla uchodźców.

Przeczytała plan podróży trzy razy. Nauczyła się czytać po angielsku w~obozie, nauczona przez kobietę, która była profesorem filologii, zanim
stała się uchodźczynią. Nauczyła się geografii z~obowiązkowych lekcji
wiedzy o~społeczeństwie, na które uczęszczała co dwa tygodnie, oglądając
filmy o~życiu w~Ameryce, które były szczególnie ubogie we wskazówki co
do przeżycia w~tej części Ameryki, gdzie spali w~trzypiętrowych łóżkach
na ognistej pustyni, otoczeni przez drony i~drut kolczasty. Jednak
dowiedziała się, gdzie był Boston. Daleko.

-- Boston?

-- Dwa dni i~siedemnaście godzin -- powiedziała pracownica. -- Zobaczysz
całą Amerykę. -- To niesamowite doświadczenie. -- Jej maska spadła na
chwilę i~wyglądała na bardzo zmęczoną. Potem znowu wkleiła swój uśmiech.

-- Idź najpierw do sklepu spożywczego, to moja rada. Będziesz chciała
zjeść jakieś prawdziwe jedzenie.

Salima stała się bardzo dobra w~nudzeniu się przez pięć lat w~obozie,
stając się mistrzynią w~rodzaju drzemce na jawie, gdzie jej umysł
uciekał, czas biegł jak karaluchy czepiające się listwy przypodłogowej,
ledwie widzialne kątem oka. Jednak w~autobusie Greyhound, ta umiejętność
ją zawiodła. Nawet kiedy usiadła przy oknie -- dwudziesta druga godzina w~podróży -- odkryła, że jej umysł wracał, ciągle, do jej rodziców, statku,
głębin Morza Śródziemnego. Wiedziała, że jej rodzice nie żyją, ale była
wiedza, i~była wiedza.

Wysiadła w~Bostonie dwa dni i~siedemnaście godzin później, zauważając
przy tym, że autobus nie miał kierowcy, coś, co przeoczyła, wsiadając i~wysiadając przez tylne drzwi. Kolejny komputer, w~który wkładałaś swoje
ciało. Przy złym oprogramowaniu, autobus mógłby spaść z~klifu lub wbić
się w~nadjeżdżające pojazdy.

W podłokietniku było gniazdko ładowania, dzieliła się nim ze
współpasażerami, którzy przychodzili i~wychodzili z~autobusu, ale
upewniła się, że naładowała telefon, gdy wychodziła z~autobusu i~dobrze
zrobiła, ponieważ zużyła prawie całą baterię na tłumaczenia i~wskazówki,
jak znaleźć ośrodek dla uchodźców, do którego była przypisana, którego
nawet nie było w~Bostonie, ale na przedmieściach zwanych Worcester,
czego wymowy nauczyła się dopiero po sześciu miesiącach.

Wszystkie jej zapasy zostały zjedzone, wszystko, co posiadała, mieściło
się w~torbie, której pasek pękł, gdy taszczyła ją w~górę uszkodzonych
schodów, zmieniając pociągi sieci ,,T''\footnote{ ,,T'' -- od litery T w~logo
Massachusetts Bay Transportation Authority, tj.~Agencji
Transportu Zatoki Massachusetts,
por.~\url{https://pl.wikipedia.org/wiki/Massachusetts\_Bay\_Transportation\_Authority}
-- przyp.tłum.} w~drodze do Worcester. Wydała połowę funduszy na karcie
debetowej na jedzenie i~jadła jak myszka, jak ptaszek, jak pędzący
karaluch. Zaczynała prawie z~niczym i~teraz miała prawie nic.

Najtrudniejszą częścią odszukania ośrodka było to, że to było martwe
centrum handlowe, jedenaście sklepów przekształconych przy pomocy łóżek,
pryszniców, pokoi dla dzieci, sklepów ustawionych wzdłuż dalszej części
pustego parkingu, który był kilometr od najbliższego przystanku
autobusowego. Salima minęła to centrum handlowe trzy razy, wpatrując się
w telefon -- którego bateria była znowu prawie pusta, była tak stara, że
ledwie utrzymywała napięcie -- zanim zrozumiała, że ten rząd sklepów jest
jej nowym domem.

Recepcja była w~starej aptece, która była krańcem galerii handlowej.
Była to nienadzorowana, przepastna powierzchnia ograniczona opuszczaną
bramą z~rzędem ekranów dotykowych, gdzie były kiedyś kasy. Pachniało
moczem, podłoga była brudna, tego rodzaju starym wbitym brudem, który
jest w~miejscach, gdzie ludzie ciągle chodzą.

Tylko jeden ekran dotykowy działał, wiele prób jej zabrało, zanim
odkryła, że musiała uderzać około półtora centymetra niżej, trochę na
lewo od przycisków, które naciskała. Kiedy to zrozumiała, sprawy poszły
szybciej. Przełączyła ekran na arabski, pozwoliła kamerze nad ekranem
przeskanować jej siatkówkę i~wielokrotnie przyciskała palce do
podkładki, aż maszyna sczytała odciski. Kiedy maszyna ją zatwierdziła,
musiała przestukać przez osiem ekranów rzeczy, które obiecywała: że nie
będzie pić, narkotyzować się lub kraść, że nie ma żadnych przewlekłych
lub zakaźnych chorób, że nie popiera terroryzmu, że rozumie, że na tym
etapie, nie ma prawa pracować za wynagrodzeniem, ale również i~paradoksalnie, jest zobowiązana pracować w~Worcester, żeby spłacić
narodowi Stanów Zjednoczonych łóżko w~ośrodku, które miała mieć
przyznane.

Przeczytała drobny druk. To było coś, co nauczyła się robić, wcześnie w~czasach uchodźców. Czasem funkcjonariusze imigracji zadawali pytania w~sprawach, które właśnie przeklikałaś, a~jeżeli nie potrafiłaś
odpowiedzieć na pytania poprawnie, wysyłali cię na koniec kolejki, lub
przenosili rozprawę o~kolejny miesiąc, ponieważ nie doceniłaś w~pełni
wagi umowy, którą wykuwałaś z~USA.

Następnie odnalazła, w~którym z~dawnych sklepów miała zamieszkać, była
poproszona o~wsunięcie karty debetowej, która została doładowana
kredytami, które mogła wymienić na jedzenie w~określonych sklepach
zaopatrujących ludzi na zasiłku. Gdy stukała w~kolejne ekrany, podając
numer telefonu, wybierając terminy badań lekarskich, zaczęła być
świadoma niskiego brzęczenia, narastającego coraz bliżej. Odwróciła się
i zobaczyła niski wózek z~tekturowym pudełkiem toczący się alejkami
opuszczonej apteki. Wózek kierował się pracowicie dookoła rogów, potem
przesunął się do bramy wstawionej w~opuszczanej klatce, która ze
zgrzytem się otworzyła. Ekran poinformował ją, żeby odebrała pudło,
które zawierało pościel, ręcznik, kilka paczek białej, bawełnianej
bielizny, t-shirty, pudełko tamponów i~torebkę z~szamponami, mydłami i~dezodorantami. Było to najbardziej \textit{funkcjonalna} wymiana, którą
miała od\ldots  \textit{lat}\ldots  i~chciała pocałować tego głupiego, małego,
niekochanego robota.

Nie mogła w~tym samym czasie nieść pudła i~torby, nie chciała żadnego z~nich zostawiać bez dozoru, zatem ustawiła je przed frontem centrum,
przesuwając pudło o~dziesięć kroków, stawiając je na ziemi, wracając po
torbę, przenosząc ją o~dziesięć kroków dalej niż pudło, potem przenosząc
pudło dalej niż torbę. Jej stos dokumentów z~kiosku zawierał mapę
pokazującą położenie jej witryny sklepowej, blisko końca galerii
(oczywiście), zatem była to długa droga. W~połowie drogi, kobieta wyszła
ze sklepu, który właśnie minęła i~ją obejrzała z~rękoma na biodrach,
głowa przechylona, mały uśmiech na twarzy.

Kobieta była Somalijką -- było ich wiele w~obozie -- i~nie była starsza
niż Salima, choć miała małe dziecko czepiające się nóg, płci nieznanej.
Miała na sobie ogrodniczki i~bluzę ,,Boston University'', włosy w~chuście, i~mimo wszystko, wyglądała jakoś stylowo. Później, Salima
dowiedziała się, że kobieta, która miała na imię Nadifa, pochodziła ze
starego rodu krawcowych i~potrafiła rozpruć dowolne ubranie, które
wpadło w~jej ręce i~dopasować je do swoich wymiarów.

-- Jesteś nowa?

-- Nazywam się Salima. Jestem nowa.

Kobieta przechyliła głową na drugą stronę. 

-- Gdzie mieszkasz? Pokaż mi. -- Podeszła do Salimy i~wyciągnęła dłoń po mapę. Salima pokazała, a~ona
zacmokała zębami. 

-- To nie dobrze, to ma złe ogrzewanie i~toaleta nie
przestaje cieknąć. Och\ldots  dobra, naprawmy to.

Bez pytania, kobieta podniosła pudło i~zaprowadziła ją do biura, Salima
idąc koło niej razem z~dzieckiem, które ciągle ukradkiem się jej
przyglądało. Kobieta wiedziała, który ekran działa i~potrafiła położyć
palec w~dokładnym miejscu niżej, trochę w~lewo, żeby trafić w~klawisze.
Jej palce pływały nad ekranem, a~potem przesunęła Salimę przed kamerę do
siatkówek, położyła jej palec na skanerze i~nowy dokument pojawił się na
tacy kiosku.

-- Znacznie lepiej -- powiedziała kobieta. Salima była zmieszana i~nieco
przestraszona. Czy ta kobieta właśnie przeniosła ją do swojej rodziny?
Czy miała być niańką dla dziecka, które znowu się na nią gapiło?

Ale nie musiała się martwić. Samotna kobieta mieszkała w~jednej z~trzech
jednostek, a~rodziny w~dwóch pozostałych. Nowy dom Salimy -- dzięki
kobiecie, która w~końcu się przedstawiła -- był kiedyś salonem
kosmetycznym, ale teraz był wyłożony ciężkimi, pochłaniającymi dźwięk
kocami zrobionymi z~jakiegoś rodzaju syntetycznego włókna, które okazało
się zaskakująco dobre blokowaniu kurzu i~dźwięku. Kobieta i~jej dziecko
zostawiły ją tam, a~ona zaciągnęła narożniki tkanin i~spięła je razem,
spędziła chwilę w~dzwoniącej ciszy małego, przegrodzonego pokoiku, który
naprawdę będzie jej, niedzielony z~nikim, przez jakiś nieokreślony czas.

Później, odkryła wszystkie sposoby, w~jaki inni mieszkańcy ośrodka
ozdobili ich małe przestrzenie, które większość z~nich nazywała celami,
z dużą dozą ironii, ponieważ każdy z~nich spędził miesiące dosłownie w~celi, tego rodzaju z~betonowymi ścianami i~żelaznymi kratami.
Udekorowała własny pokój, a~dzieci Nadify wtykały głowy bez ostrzeżenia
i żądały historyjek lub kogoś do zabawy lub pomysłów na obrazki do
narysowania. Nie była właściwie wciągnięta w~bycie niańką, ale również
nie była właściwie \textit{nie }wciągnięta w~bycie niańką, lubiła
dzieciaki Nadify, które były tak odważne i~nieustraszone jak ich matka,
która była źródłem zabawy, szczególnie gdy znalazła butelkę wina i~wysłała dzieci bawić się we wspólnym pokoju, a~one siedziały po
przeciwnych stronach wąskiego łóżka Salimy, opowiadając kłamstwa o~mężczyznach i~czasem by się wślizgnęły dziwne prawdy o~wcześniejszych
życiach przed ośrodkiem, byłyby łza lub dwie, ale to też było w~porządku.

Nadifa już miała swoje pozwolenie na pracę i~pokazała Salimie jak dostać
dokumenty dla niej, co zabrało miesiące cierpliwego dźgania w~jedyny
działający ekran, żeby wydrukował papier, który musiała zanieść do
urzędu i~zasilić inne kioski, wykradając czas na wyjścia pomiędzy jej
pracą. Ironia bycia zbyt zapracowaną, żeby otrzymać pozwolenie na pracę,
nie umknęła jej, a~och, jak śmiała się z~tej ironii, gdy zmywała
graffiti, zbierała śmieci w~parku, czyściła miejskie autobusy w~wielkich
halach autobusowych w~miejscach jeszcze dalej niż jej centrum handlowe w~Worcester.

Otrzymanie pozwolenia nie było tym samym co znalezienie pracy, ale
Salima była inteligentna i~spędziła lata w~obozie, nabywając różne
kwalifikacje w~kursach online -- zaplatanie włosów, księgowość, usuwanie
wirusów i~pielęgnacja kotów -- i~czuła pewność, że będzie coś, co mogłaby
robić. Z~pomocą Nadify przeszukiwała tablice z~ogłoszeniami pracy,
zapisała się do agencji pracy tymczasowej, poddała się ich poniżającym
testom, które obejmowały podanie dostępu do jej kont w~mediach
społecznościowych i~historii email, napastowanie, które się pogorszyło,
kiedy była potem przepytana o~wiadomości, które zachowała od rodziców,
video i~zdjęcia, wiadomości wysłane, kiedy byli rozdzieleni, ale zanim
oboje zmarli.

Praca sączyła się, kilka godzin tu i~tam, zmiany krótkie w~porównaniu z~długimi dojazdami autobusami, ale miała nadzieję, że przyjmowanie tych
gównianych prac zbuduje jej opinię w~agencjach, które ją wysyłały, że
spłaci swoje długi i~zacznie dostawać prawdziwe zmiany, za prawdziwe
pieniądze. Kupiła kilka zewnętrznych baterii dla jej schorowanego
telefonu tak, że mogła pracować w~trakcie dojazdów. Ona i~Nadifa
podzieliły całą Nową Anglię i~każdego dnia wysyłały setki zapytań,
poszukując decyzji budowlanych na wieżowce, które miały dotowane
mieszkania, wtedy notowały dzień, kiedy lista zainteresowanych byłaby
otwarta. Wiedziały, że szanse dla którejkolwiek z~nich bycia przyjętą
były znikomo małe, a~nawet jeżeli obie zostałyby naraz zaakceptowane,
było dość niemożliwe, żeby skończyły w~tym samym miejscu.

Dlatego też Wieżowce Dorchester były takim cudem. To był gorzki
grudzień, ośrodek nie otrzymał jeszcze obiecanej dostawy płaszczy
zimowych, więc wszyscy nosili wiele warstw swetrów i~podkoszulek, co nie
było postrzegane jako ,,profesjonalne'' i~kosztowało Salimę bardzo
dobre, tygodniowe zlecenie księgowości dla think-tanku, który zamykał
kwartał. Bardzo się zmartwiła utratą pracy i, jeszcze bardziej,
otrzymaniem czarnej oceny w~agencji pracy tymczasowej, która dała jej
kilka innych wspaniałych zleceń księgowości, co podtuczyło jej małe
konto oszczędnościowe bardziej niż dziesiątka prac w~sprzątaniu.

Stukając dookoła centrum handlowego z~innymi mieszkańcami uwięzionymi
przez pogodę i~niedostosowane ubrania, rozważała wydanie oszczędności na
płaszcz, próbując wymyślić, ile robót by straciła, zanim wyszłaby na
czysto i~oceniając prawdopodobieństwo, że mocno opóźniona dostawa
płaszczy zimowych w~końcu dotrze, zanim zbyt dużo zleceń będzie
straconych. Jej telefon dał jej znać, że otrzymała wiadomość od rządu -- ten rodzaj, który musiała odebrać z~kiosku w~biurze ośrodka -- zatem
założyła trzy swetry i~wcisnęła dłonie w~trzy grube skarpetki i~walczyła
z wichurą aż do biura.

Stojąc w~kałuży rozpuszczającej się wody, zalogowała się do kiosku -- naprawili je wszystkie, w~tym ten, który raczej działał, i~teraz
wszystkie były równie niepewne i~skłonne do wpadnięcia w~niekończący się
cykl restartów -- i~odebrała wiadomość. Właśnie wchłaniała niemożliwie
dobre wiadomości, kiedy Nadifa wtoczyła się z~zimna, trzymając jej
najmniejsze blisko siebie w~cieple ciała.

-- Czy ten działa? -- Wskazała na ekran Salimy i~Salima uśmiechnęła się do
siebie, gdy zamknęła program i~odeszła od niego.

-- Działa! -- Jej radość była słyszalna w~jego głosie i~Nadifa spojrzała
się na nią dziwnie. Salima stłumiła uśmiech. Chciała powiedzieć Nadifie,
kiedy\ldots 

-- O mój Boże. -- Nadifa właśnie gapiła się na ekran, szeroko otwarte
usta. Salima zajrzała i~roześmiała się głośno.

-- Ja też, ja też!

Wiadomość informowała, że Wieżowce Dorchester zaakceptowały pobyt Nadify
w dwupokojowym mieszkaniu na czterdziestym drugim piętrze, które byłoby
gotowe do zamieszkania za osiemnaście miesięcy, zakładając brak opóźnień
w budowie. Czynsz był indeksowany dochodem, co znaczyło, że Nadifa i~jej
dzieci stać byłoby tam na życie niezależnie od tego, co się im przytrafi
w przyszłości. Nadifa była czasem głośna i~nachalna, ale nigdy nie była
piskliwa, zatem całkiem mocno rozbawiła Salimę, gdy wyrzuciła ręce w~powietrze i~podskakiwała w~górę i~w dół na palcach, wydając podniecone
dźwięki tak wysokie, że ogłuszyłyby delfina.

Nie przestała podskakiwać, kiedy przytulała Salimę, pociągając ją do
podskakiwania w~górę, śmiejąc się z~radości, i~Salima śmiała się jeszcze
bardziej z~tego, co wiedziała.

Wylogowała Nadifę ze stanowiska i~zalogowała się sama, szybko
przestukała aż do oficjalnej rządowej skrzynki, i~po prostu wskazała bez
słowa na ekran, póki Nadifa nie pochyliła się i~przeczytała. Jej szczęka
opadła jeszcze bardziej.

-- Jesteś na trzydziestym piątym piętrze! To tylko siedem pięter poniżej
nas! Możemy przechodzić do siebie po schodach! -- Najmniejsze dziecko
Nadify zmieszane całym tym krzyczeniem i~skakaniem wybrało tę chwilę,
żeby zacząć płakać i~Nadifa wyjęła je z~nosidełka i~zakręciła nim
dookoła nad głową. 

-- Mamy mieszkanie, własne mieszkanie! I~ciotka Salima
też tam będzie! Będziemy mieć kuchnię, \textit{sypialnie}, będziemy
mieć\ldots  -- Przerwała i~przytuliła chłopca na ramieniu, użyła wolnej ręki,
żeby złapać Salimę i~potrząsnąć ją za ramię. -- Będziemy mieć łazienki.
Własne łazienki! Własne wanny! Własne toalety!

-- Nasze własne toalety! -- krzyknęła Salima i~maleństwo powiedziało coś,
co brzmiało prawie \textit{toalety}, to rozśmieszyło je obie do rozpuku,
śmiały się, aż łzy pociekły po ich twarzach, a~dziecko śmiało się razem
z nimi.

Płaszcze przybyły tego wieczoru po kolacji.

~

Salima i~Nadifa złożyły się na wynajem busa w~dniu, w~którym się
wyprowadzały. Wypełniły go do sufitu z~odpadami lat Nadify i~miesiącami
Salimy w~ośrodku, zabawkami dzieci, ubraniami, butelkami z~szamponem na
trzy ostrożne umycia, rysunkami, książkami z~obrazkami, skrawkami
papieru do rysowania, lalkami papierowymi mozolnie wycięte ze starych
wydruków z~kiosku. Samochód posuwał się powoli w~ruchu Bostonu, który
mogły dojrzeć sporadycznie przez małe kawałki szyby, które nie były
zakryte torbami zakupowymi pełnymi ich własności.

Dwie godziny później bus zatrzymał się przy tylnej alejce Wieżowców
Dorchester. Był gorący czerwcowy dzień i~dzieci potrzebowały dwóch
przerw na toaletę, kilku przerw na wodę, które rozwaliły plany ominięcia
godzin korków, wrzucając ich dokładnie w~trakcie. Jednak obie kobiety
zachowywały stoicki spokój. Odbywały podróże, które były znacznie,
znacznie dłuższe i~znacznie trudniejsze.

Drzwi dla biednych w~Wieżowcach Dorchester jeszcze nie były skończone,
zatem musiały przejść przez tymczasowy tunel z~dykty, żeby wejść do
budynku. Hol był w~tym samym stanie co drzwi, surowe płyty
gipsowo-kartonowe, otwarte gniazdka elektryczne, surowe betonowe ściany
z korytami pozostawionymi dla przewodów. Przytaszczyły swoje rzeczy do
holu etapami, zostawiając najstarsze Nadify do pilnowania i~zajmowania
się dziećmi, gdy krążyły tam i~z powrotem do busa, próbując zabrać
wszystko przed upływem jednej godziny, gdy zostaną obciążone kosztem
wynajmu kolejnej godziny. Ledwie się zmieściły.

Potem w~holu, spocone i~mokre, poznały windy Wieżowców Dorchester. Ekran
dotykowy pytał o~Twoje piętro, potem śledził przejazd kabin w~górę i~w dół szybów. Kabiny dotarłyby do holu, usłyszałyby drzwi otwierające się
z westchnieniem i~zamykające, ale drzwi naprzeciwko nich nigdy się nie
otworzyły.

Zastanawiały się, co robić. W~końcu, zdecydowały, że drzwi po ich
stronie w~ogóle nie działają, że to była kolejna rzecz, która miała być
skończona, razem z~holem, drzwiami i~proszę, boże, klimatyzacją.

Dzieci, dobytek, one same, jakoś znowu wyszły przez drzwi, wzdłuż
alejki, dookoła obwodu budynku do drzwi do holu, które, nie mogły nie
zauważyć, były gotowe, chromowane, błyszczące, bez smug i~strzeżone.

Strażnik po drugiej stronie drzwi włączył interkom, kiedy próbowały
nacisnąć klamkę. Był biały i~nosił strój na wzór policji, jakiś rodzaj
prywatnej ochrony, co było niezwykłe, ponieważ to był rodzaj pracy, w~której zwykle widywałaś, w~większości przypadków, brązowych ludzi. To
też zauważyły.

-- Tak?

-- Mieszkamy tutaj, wprowadzamy się dzisiaj. Po\ldots  -- Salima machnęła
dłonią na drogę. -- Po drugiej stronie? Ale windy tam jeszcze nie
działają. Możemy korzystać ze schodów, kiedy się wprowadzimy, ale ona
jest na czterdziestym drugim piętrze, a~ja na trzydziestym piątym, i~mamy całe to wszystko\ldots  -- Stos toreb, ubrań, rysunków, dzieci i~ich
samych, wszystko haniebne, szczególnie na tle błyszczącego chromu,
niepomazanych szyb, na których teraz powoli dwoje dzieci Nadify
rozmazywało twarze i~dłonie. Ojej.

Ochroniarz stuknął w~ekran. 

-- Windy działają.

-- Nie po tamtej stronie. Windy zjeżdżają, ale drzwi się nie otwierają.

-- Proszę się odsunąć. -- Powiedział to tak ostro, że nawet dzieci Nadify
stanęły na baczność. Jacyś ludzie próbowali wejść, przyciskając kciuki
do matowego miejsca na futrynie, które nie przyjmowało plam. Drzwi się
otworzyły i~wypuściły błogosławiony powiew klimatyzowanego powietrza,
które prawie rzucił je na kolana. Krople potu na plecach, nogach,
twarzach i~włosach straciły tyle ciepła, ile mogły w~trakcie krótkiego
powiewu wiatru. Potem dobrzy ludzie przeszli przez drzwi, nie patrząc na
nich ani razu. Byli eleganccy, wygląd, który Salima w~końcu zrozumiała,
kiedy przeniosła się do tego miasta uniwersytetów i~szkół wyższych, z~miękkimi blond włosami, ostrożnie muśniętymi strojami tenisowymi i~mokrymi, błyszczącymi twarzami. Ochroniarz przywitał ich i~gawędził z~nimi, słowa niezrozumiałe przez zamknięte drzwi. Byli wystarczająco
sympatyczni i~pomachali na ,,do widzenia'', gdy wchodzili do windy. Gdy
drzwi się zamykały, Salima zobaczyła drzwi po przeciwnej stronie, drzwi,
które wychodziły na drugą stronę.

Ochroniarz spojrzał się na nie zirytowany i~pokręcił głową, jakby nie
wierzył, że ciągle tam stoją, blokując wejście. 

-- Wasze wejście jest od
tyłu.

-- Windy nie działają -- przypomniała mu Salima. -- Czekałyśmy i~czekałyśmy\ldots 

-- Windy działają. Po prostu dają pierwszeństwo stronie czynszu
rynkowego. Dostaniecie windę, kiedy nikt z~tych ludzi nie będzie ich
potrzebował.

Salima w~mgnieniu oka pojęła system i~jego logikę. Jedynym powodem, dla
którego była w~stanie wynająć w~tym budynku było to, że deweloper musiał
obiecać, że zbuduje tanie mieszkania w~zamian za pozwolenie wybudowania
pięćdziesięciu pięter zamiast trzydziestu, do których sięgały okoliczne
budynki. Było dużo tego rodzaju rzeczy i~wiedziała, że były zasady
dotyczące taniego mieszkalnictwa, co właściciele musieli zapewnić i~co
było jej zabronione.

Ale teraz zrozumiała ważną prawdę: nawet najdrobniejsze udogodnienie
było złośliwie odmówione mieszkaniom dotowanym, chyba że kamienicznik
był zmuszony przez prawo, żeby je zapewnić. Spędziła wystarczająco czasu
jako ciocia Salima, pomagając wychować troje dzieci Nadify, że rozpoznać
logikę upartego dziecka, które chciało, żeby jego niezadowolenie było
widoczne.

-- Chodźcie -- powiedziała, gdy podnosiła podwójny ładunek toreb i~cofała
się dookoła budynku do drzwi dla biednych.

Mieszkanie było \textit{cudowne}. Obiecany luksus prywatnego prysznica i~wanny, w~której mogła się położyć, jeżeli ścisnęła nogi i~położyła brodę
na piersi (ale była to \textit{jej} wanna!) i~pojedyncze łóżko z~dobrym
materacem, na którym nikt inny wcześniej nie spał, a~kiedy je złożyła,
mogła rozłożyć sofę z~jasnymi poduszkami. Unieś płytę nieco wyżej i~obróć, zmieniając ją w~stolik do kawy, którego nogi się rozciągały i~stolik stawał się stołem do jedzenia, przy którym mogliście siedzieć w~trójkę lub czwórkę, jeżeli wszyscy byli dobrymi przyjaciółmi. Ściany
miały dobrą izolację, a~ten drobny hałas, który przesączał się z~lokali
naokoło, był niesłyszalny, tak długo, jak uruchamiała wentylator w~jej
klimakonwektorze\footnote{ oryg. HVAC od ,,heating, ventilation, air
conditioning'', rodzaj urządzenia wentylacyjnego umożliwiającego
podgrzewanie lub chłodzenie pomieszczenia,
zob.~\url{https://en.wikipedia.org/wiki/Heating,\_ventilation,\_and\_air\_conditioning}
-- przyp.tłum.} na najniższym ustawieniu, coś, co mogła zautomatyzować
czujnikami mieszkania tak, aby zdarzało się za każdym razem, gdy była w~domu.

Kuchnia miała ,,wszystkie artykuły gospodarstwa domowego'', tak jak
reklamowano: minipiekarnik, zmywarkę -- taką małą, że mogła zmyć naczynia
z posiłku dla jednej osoby, razem z~miską do mieszania czy formą do
pieczenia z~minipiekarnika -- lodówkę. Wszystkie zaczęły działać od razu,
gdy wprowadziła numer karty debetowej, oraz przedstawiły jej menu
akceptowanych materiałów eksploatacyjnych: talerze, które działałyby w~zmywarce, wybór jedzenia, który działałby w~tosterze, od chleba do
gotowych posiłków. Pralka wyprałaby ręczniki i~pościel, lub ubrania z~dwóch dni, i~istniało dziesiątki zgodnych proszków do prania, które
mogła zamówić przez ekran. Wszystkie ceny zawierały dostawę, ale mogła
sama je kupić w~atestowanych sklepach, ale zawsze istniało ryzyko, że
wybrałaby coś niezgodnego, niekompatybilnego z~jej modelem, zatem byłoby
lepiej dla wszystkich zainteresowanych, gdyby robiła zakupy właśnie w~kuchni, gdzie było to najbardziej wygodne dla wszystkich.

Do przystanku sieci ,,T'' był krótki spacer, a~schody nie były takie złe
przy schodzeniu rano. Wracanie na górę było inną sprawą: trzydzieści
pięć pięter oznaczało siedemdziesiąt biegów schodów. Pokonywała je raz
na tydzień i~wmawiała sobie, że taki intensywny trening aerobowy to dla
jej zdrowia.

Posiadanie miejsca do życia, które było jej, wprowadziło dużą różnicę w~jej życiu. Coś o~stabilności, pewności siebie -- cholera, po prostu
posiadanie solidnego miejsca do zrobienia prania wieczorem -- to wszystko
składało się na poczucie, że w~końcu wychodzi z~niekończącej się
otchłani, w~której żyła całe życie. Jej najwcześniejsze wspomnienia
dotyczyły przeprowadzek z~jej rodzicami, jeden obóz, potem kolejny,
potem przez jakiś czas dom wujka, potem kolejny obóz, potem tymczasowe
mieszkania, podróż do Ameryki, obóz, ośrodek dla uchodźców. Cały czas
miała poczucie, że jej życie jest na wstrzymaniu, że unosi się dookoła
jak liść na wietrze, czasem złapana przez gałąź, czasem podniesiona do
chmur, ale nigdy nie lądująca, nigdy nie odpoczywająca. Oznaczało to, że
nigdy naprawdę nie planowała swojego życia na więcej niż kilka dni z~góry. Teraz, w~swoim własnym domu, myślała o~tym, co ją czekało w~przyszłości.

Pewna kombinacja szczęścia i~pewności siebie była przy niej, w~ciągu
miesiąca dostała pracę na pełny etat jako księgowa dla firmy, która
obsługiwała rodzinne sklepiki. Miała kilkoro klientów i~próbowała
spotkać się z~nimi raz w~tygodniu, nawet jeżeli mogła wykonać większość
pracy z~domu. Wolała zorganizować pracę w~biurze na tyłach pralni,
sklepu spożywczego czy małej lodziarni i~opracować zapisy z~kasy i~faktury, planując opłaty i~rozmawiając z~pracownikami i~właścicielami.
Nauczyła się, że ludzie lubili być ostrzeżeni o~zbliżających się
kryzysach z~gotówką i~innych potencjalnych przeszkodach, które mogła
wyczytać z~ich ksiąg, w~ciągu kilku miesięcy stała się bardziej zaufanym
doradcą niż wykonawcą. Pamiętała o~ich urodzinach, przynosiła im kartki,
a kiedy minęły jej dwudzieste piąte urodziny, właściciel sklepu ze
stylową odzieżą zaskoczył ją wspaniałą japońską, satynową kurtkę z~poprzedniego wieku, na plecach wyszyty tygrys, który pięknie wyblakł
przez lata, patynowany niczym perski dywan.

Dzieciaki Nadify dostały się do szkoły i~nawet Nadifa zaczęła się
dopasowywać, czas na oddech dla niej oznaczał, że mogła zjeść porządny
posiłek, zająć się włosami i~ubraniami. Zawsze miała królewską postawę,
ale walczyła z~naturalnym, przeciążonym, opadaniem ramion matki,
zmarszczkami z~wyczerpania, rękoma pełnymi dzieci, zabawek lub prania,
plamami na jej pięknie szytych ubraniach. Pewna stabilność w~życiu
ujawniła się natura Nadify. Jej ubrania były nienaganne, zmarszczki na
twarzy sprawiały, że wyglądała poważniej, a~potem kiedy opowiedziała
niegrzeczny żarcik i~jej oczy zabłysnęły, różnica pomiędzy powagą a~humorem była czymś, co widziałaś zwykle na starych obrazach.

Dzieci Nadify nigdy nie przestały psocić, ale szkoła była dla nich
dobra, dała im pewną strukturę do wypracowania i~zwalczenia. Byli w~tyle, szczególnie Abdirahim, najstarszy dwunastolatek, i~Nadifa męczyła
go, zmuszając odrabiania dodatkowej pracy domowej na jego telefonie, lub
nawet na wielkim ekranie w~salonie, zakładając, że małe mogły zostać
uspokojone do zwykłego chaosu. Mieszkanie Nadify miało dwa pokoje, jeden
dla dzieci i~salon, który wyglądał dokładnie jak Salimy, zamieniając się
w sypialnię poprzez złożenie stołu i~rozłożenie sofy w~łóżko, magiczny
trik, który musiał być wykonany we właściwym porządku lub wszystko
łapało wszystko inne w~środku pokoju w~piekielnym splątaniu, który
musiał być ostrożnie luzowany.

Rzeczy były dobre, póki sprzęt AGD\footnote{ AGD -- artykuły gospodarstwa
domowego -- przyp.tłum.} nie przestał jej słuchać.

Naprawiła pralkę, bo \textit{mogła}. Kiedy masz kuchnię pełną urządzeń,
które Cię posłuchały, te, które nie były posłuszne, objawiały się jako
coraz większe i~coraz mniej tolerowane. Poza tym była samotna, nie była
zainteresowana w~przesuwaniu w~prawo nieznajomych, którzy zawsze
zawodzili. Stała się obsesyjnym widzem filmów wideo o~hakowaniu\footnote{oryg.
jailbreak tj. łamanie zabezpieczeń własności intelektualnej nałożonych
przez producenta niż w~klasycznym znaczeniu hakowania
tj.~nieuprawnionego dostępu do danych rządowych lub
firmowych -- przyp.tłum.}, szczególnie, od kiedy podążyła po śladach do
jeszcze odważniejszych filmów, póki nie znalazła tego, który powiedział
jej jak ściągnąć programy darknetu, które umożliwiły jej dostęp do
poważnych stron, gdzie mogła znaleźć nowe obrazy firmware, wymienić się
wskazówkami i~skargami i~bawić się wesoło z~tysiącami samowolnych
anarchistów takich jak ona sama, którzy tostowali każdą rzecz wedle
uznania.

Pralka była dotychczas najtrudniejsza, wymagała odłączenia kilku
przewodów z~wodą. Ciągle psuła zaciski pierścieniowe, których nigdy
wcześniej nie musiała używać, ale przeszła przez to tak jak księgowa,
metodycznie próbując jednej kombinacji po drugiej, trzymając patelnię
pod połączeniem, żeby przechwycić wyciek, kiedy testowała je przy
odkręcaniu wody. Kiedy pracowała, przełączała się na filmy o~robieniu
chleba, ponieważ to była jej nowa pasja, i~teraz za każdym razem, gdy
widziała się z~dziećmi Nadify, błagały ją o~cokolwiek, co było ostatnim
wytworem. Teraz robiła zaplatane bochenki, chleb z~jajkiem zwany chałką,
pokrywając wyrastające ciasto białkiem jaj, żeby nadać skorupie połysk.

Tydzień później odkryła dwie ważne rzeczy: pierwsza, kupowanie środków
do prania w~sklepie spożywczym było \textit{znacznie} tańsze niż kupowanie
go przez ekran maszyny, i~drugie, jej ciągła egzema była właściwie
reakcją alergiczną na coś w~autoryzowanym mydle do prania. Wiosna trwała
w najlepsze i~ona bała się pocenia w~gorące dni w~długich rękawach,
które zasłaniały jej łuszczące, swędzące ramiona. Kupiła trzy stylowe
bluzki na krótki rękaw i~poprosiła Nadifę, żeby je dopasowała do niej
tak pięknie, jak wszystkie ubrania Nadify.

Zatrzymała się przed hakowaniem termostatu. Na razie. Termostat był
połączony z~siatką czujników budynku, włączając w~to kamerę nad jej
drzwiami, kamery rozrzucone po jej mieszkaniu. Rozpoznawał ją, zanim
jeszcze weszła, rozkręcał system HVAC, zanim zamknęła drzwi za sobą,
dając jej moment klaustrofobicznego, chwytającego powietrze mieszkania,
zanim biały szum wentylatora uniósł prąd miękkiego powietrza dookoła. Co
więcej, budynek mógł obserwować jej mieszkanie, kiedy była w~pracy,
wysyłać wideo, jeżeli wykrył kogoś w~jej mieszkaniu, kiedy jej tam nie
było. Lubiła to, uspokajało ją to. W~obozie w~Arizonie była okradziona
dwa razy i~przyzwyczaiła się do noszenia ze sobą wszystkiego cennego
przez cały czas. To była taka ulga, móc zgromadzić więcej cennych
towarów niż mogła łatwo nosić.

Winda była inną kwestią.

Kiedy przeprowadzała się do Wieżowców Dorchester, budynek był użytkowany
przez kilka tygodni i~miał mniej niż połowę mieszkańców. Gdy mieszkania
się wypełniały, liczba osób płacących stawkę rynkową używających wind
wzrastała do momentu, kiedy mogło zabrać całe czterdzieści pięć minut,
żeby wjechać na trzydzieste piąte piętro, a~kiedy winda w~końcu wjechała
na hol dla biednych, było tak wiele osób czekających, że kończyła, jadąc
ściśnięta ludźmi, twarzą przy zapoconej pasze, a~jeżeli miała farta,
była przyciśnięta plecami do ściany, a~nie do jakiegoś obcego mężczyzny.
Była prawie pewna, że w~przypadkach, kiedy wydawało się, żeby
napastowana, to był przypadkowy nacisk ciał, a~nie prawdziwe
napastowanie, ale nie mogła być pewna, a~zresztą to było \textit{nadal}
odrażające.

Pewnego dnia siedziała w~salonie Nadify, pijąc herbatę i~oglądając
najstarszego Nadify odrabiającego swoją dodatkową pracę domową. Ona i~Nadifa narzekały na windy przez dobre dwadzieścia minut -- to był stały
temat pomiędzy wszystkimi dotowanymi mieszkańcami i~naprawdę mogli nad
tym popracować -- kiedy Abdirahim, najstarszy syn Nadify, podniósł wzrok
znad matematyki.

-- Mama, dlaczego po prostu nie użyjemy kapitanów windy?

-- Rób pracę domowę. -- Nadifa działała na czystym odruchu, kiedy sprawy
dotyczyły jej dzieci i~pracy domowej, ale po chwili dodała: 

-- Co to jest kapitan windy?

Uśmiech Abdirahima był ogromny. 

-- To jest takie \textit{fajne}. Jeżeli
jesteś pierwszą osobą w~windzie w~Japonii, stajesz się kapitanem windy.
Musisz trzymać przycisk otwarcia drzwi, aż wszyscy wejdą, a~potem musisz
naciskać przyciski do zamykania drzwi i~wszystkie przyciski pięter.
Jeżeli kapitan windy wychodzi, zanim winda opustoszeje, następna
najbliższa osoba musi to robić.

-- Gdzie się o~tym dowiedziałeś?

-- Mieliśmy lekcję o~niepisanych zasadach na ,,wiedzy o~społeczeństwie''.
Robię dodatkową pracę o~niepisanych zasadach w~ośrodkach dla uchodźców w~Ameryce. Nauczycielka to uwielbia, robi się taka poważna, kiedy o~tym
mówię. Inne dzieciaki z~ośrodków myślą, że to śmieszne.

-- Nie sądzę, że to jest śmieszne. -- Nadifa była śmiertelnie poważna. -- Myślę, że to pełne szacunku. -- Odwróciła się do Salimy, otworzyła usta
do dalszej rozmowy, a~potem odwróciła się do Abdirahima. 

-- Dlaczego
mielibyśmy mieć kapitana windy?

Jego uśmiech był jeszcze większy. 

-- Moglibyśmy się zmieniać w~windzie
rano i~wieczorem, podczas dużego ruchu. Windy nie zatrzymują się dla
biedaków, jeżeli są bogacze, którzy ich potrzebują, jasne, ale jeżeli w~środku jest biedak, nie zatrzymają się dla bogaczy, póki biedak nie
wysiądzie.

Nadifa wyszeptała \textit{biedak} do Salimy i~przewróciła oczami. Salima
zasłoniła uśmiech. Dzieci wiedziały, co jest co i~mówiły tak, jak jest.
W międzyczasie Salimie i~Nadifie zaświtał plan. Był dziwnie elegancki:
prosty, zatem nie było dużo rzeczy, które mogły źle pójść. Do tego,
wykorzystywał fakt, że bogacze nie chcieli widzieć jednego ze swoich
wykorzystanych przeciwko nim.

-- Rób pracę domową. -- Nadifa użyła surowego głosu, ale jej uśmiech do
Salimy był dwa razy większy niż Abdirahima. Salima powiedziała
bezgłośnie \textit{bystry dzieciak}, Nadifa skinęła głową.

Dwa tygodnie kapitanów windy były najlepszymi w~krótkiej historii
budynku. Od 7:30 do 8:45 rano i~od 17:15 do 18:30, była efektywnie jedna
winda wyłącznie zarezerwowana do wykorzystania przez biedną stronę
budynku, obsługując dziesięć pięter z~pięćdziesięciu sześciu. To
zostawiało do wykorzystania piętnaście wind dla bogaczy i~z początku
mogli nie zauważyć, że niewidziani sąsiedzi żyjący w~niedostępnych
przestrzeniach budynku poruszali się w~górę i~w dół w~ciągu minut,
zamiast czekać godzinę lub męcząc się na schodach.

Niemniej jednak ktoś zauważył. Nadifa zadzwoniła do Salimy w~pracy,
używając gniewu, żeby ukryć zmartwienie. 

-- Czekali w~holu. Trzech
ochroniarzy! Trzech! I~oczywiście to musiał być Abdirahim jako kapitan
windy. -- To był pomysł Abdirahima i~był najbardziej entuzjastycznym
kapitanem w~budynku. Salima nawet kupiła mu małą wojskową czaszkę z~daszkiem ze sklepu jednego z~jej klientów ze stylową odzieżą, a~on nosił
ją pod wesołym kątem w~trakcie jego zmian, wyglądając prawie
nieprzyzwoicie słodko.

Właśnie sprowadził windę na parter i~naciskał w~przycisk zamknięcia
drzwi z~błyskawicznymi odruchami trzynastolatka wychowanego na grach
video, kiedy \textit{drugie} drzwi się otworzyły, drzwi dla bogatych osób,
drzwi, które nigdy, przenigdy nie otworzyły się, kiedy ktokolwiek z~nich
był w~kabinie.

Trzech ochroniarzy zażądało nazwiska i~dokumentów Abdirahima, a~kiedy
powiedział im, że są w~jego domu, na czterdziestym drugim piętrze,
odmówili puszczenia go, żeby je przyniósł. Zamiast tego, zabrali go na
dół do piwnicy, wykorzystując swoje piloty ochroniarzy, żeby przejąć
kontrolę nad windą. Zamknęli go w~pokoju bez okien ze wzmocnionymi
drzwiami i~rzucającymi się w~oczy kamerami w~każdym rogu pod sufitem,
trzaskając drzwiami.

Po dłuższym czasie, przyszli go przepytać. Wiedział, że spóźnia się do
domu i~że jego mama będzie się martwiła, choć nie szalała, ponieważ
Nadifa nigdy nie szalała. Była wściekła, tak. Szalona, nigdy. Szczerze,
wściekłość martwiła bardziej. To była myśl, którą miał w~głowie, kiedy
wyjaśniał sprawę kapitanów windy ochroniarzom, którzy przesłuchiwali go
i nie dali wody lub skorzystać z~toalety, póki nie wydawało mu się, że
każda kropla wody w~jego ciele jest w~pęcherzu, próbując desperacko
uciec.

Powtarzali informacje raz po razie, zaczął płakać, ponieważ przypomniało
mu to przesłuchania, które mieli w~obozie, kiedy był maleńki, i~znowu,
kiedy dostali się do Ameryki, kiedy był mały. To były trudne chwile,
jego ojciec umierający i~odmawiający pokazania tego, siedzący prosto jak
słup, odpowiadający pytanie za pytaniem, modlący, żeby nie odkryli jego
choroby i~nie wykorzystali jej jako wymówki, żeby zawrócić jego rodzinę.

Wspomnienie tego czasu przezwyciężyło i~szlochał, nie mógł odpowiedzieć
na ich pytania, i~wtedy zadzwonili do Nadify, w~końcu, i~była
\textit{wściekła}, ale nie na niego. Krzyczała na nich o~traumatyzowaniu
dzieci i~żądała ich nazwisk i~numerów odznak, nawet wyjęła telefon i~nagrywała to wszystko, nawet jego płacz, co go zawstydziło, ale ciągle,
nie mógł przestać.

Nadejdzie dzień sądu, przysięgała Nadifa, a~Salima pomyślała, że
prawdopodobnie miała rację, ale że one najpierw dostałyby najgorsze.

Dostały.

Windy miały kamery, oczywiście, i~oprogramowaniu rozpoznawania twarzy
zabrało całe dziesięć sekund stworzenie listy wszystkich mieszkańców
biednych drzwi, którzy używali systemu kapitanów wind, z~czasami, datami
każdej jazdy. Zarządowi budynku zabrało dzień, żeby złożyć te zapisy w~formę listu informującego wszystkich, którzy uczestniczyli, że naruszali
warunki umowy, które zabraniały ,,manipulowania, analizowania,
wyłączania, omijania, odłączania, fałszowania, podważania, uszkadzania
lub usuwania'' jakiegokolwiek z~systemów budynku. Następne naruszenia
zakończyłyby się postępowaniem eksmisyjnym. Bądźcie ostrzeżeni. Bądźcie
powiadomieni.

Traktowanie w~ten sposób było poniżające. Nawet dla Salimy, która była
poddana najbardziej poniżającym grabieżom przez lata -- przeszukania z~rozbieraniem, konfiskaty, grupowe kary i~badanie najbardziej prywatnych
danych i~wspomnień, żeby znaleźć powód zaprzeczenie jej człowieczeństwa
-- to było bolesne. Po latach kręcenia się w~kółko, w~końcu zaczęła swoje
życie na serio, z~mieszkaniem, pracą i~przyjaciółmi, którzy byli prawie
rodziną. To było przypomnienie, że jej obecne życie ledwie przykrywało
świat, w~którym żyła wcześniej.

Przez całe jej życie, świat był podzielony na ludzi dookoła niej, ludzi,
którzy ją znali i~kim była. Większość z~nich była ludźmi, którzy jej
życzyli dobrze i~wspierali ją, a~ona ich wspierała. Niektórzy z~tych
ludzi byli źli i~chcieli ją skrzywdzić -- obozy nie były rajem -- ale
nawet dla nich, to było osobiste.

Jednak tutaj był inny świat, zdecydowanie poza jej wiedzą, ludzi, którzy
jej w~ogóle nie znali, ale którzy trzymali jej życie w~swoich dłoniach.
Tych, którzy tłoczyli się na demonstracjach przeciwko uchodźcom.
Polityków, którzy wściekali się na plagę terrorystów ukrytych pomiędzy
uchodźcami, tych, którzy mówili kodem o~,,asymilacji'' i~,,zbyt dużo,
zbyt szybko''. Żołnierzy, policjantów i~strażników, którzy celowali w~nią z~broni, rozkazywali jej. Biurokratów, których nigdy nie widziała,
którzy odrzucali jej dokumenty z~tajemniczych powodów, których mogła się
tylko domyślać, i~biurokratów, którzy patrzyli jej w~oczy i~odrzucali
jej dokumenty, odmawiając jakichkolwiek wyjaśnień.

Teraz była nowa grupa w~tej klasie, odległa jak przyczyny pogody: firma
zarządzająca budynkiem i~jej laserowa drukarka, strzelająca groźbami o~eksmisji ludziom, których nazwisk nawet nie znali i~twarzy, których nie
widzieli za przekroczenia tak drobne i~zasady tak poniżające.

Kapitanowie wind byli dobrym żartem, sposobem, żeby wszyscy z~drzwi i~pięter dla biednych poczuli się jak mysz, która przechytrza koty. Listy
ustawiły ich na ich miejscu: karaluchy, twarzą w~twarz przed
tępicielami.

Windy nie były lepiej zaprogramowane niż Disher czy Boulangism,
termostat czy cokolwiek innego, naprawdę. Największą różnicą był dostęp.
Mogła rozebrać zmywarkę na części w~swojej kuchni bez tłumaczenia się
komukolwiek, ale gdyby spróbowała tego na korytarzu, na widoku kamer i~sąsiadów, to sytuacja wyglądałaby zupełnie inaczej.

Żeby załatwić windy, musiałaby załatwić kamery, a~potem pracować w~bardzo krótkich okresach i~ciągle ryzykować odkrycie, zatem potrzebowała
przebrania, ubrania roboczego, i~może też powinna wyłączyć światła i~zastąpić je oświetleniem roboczym, zaczęła wizualizować widok: ona na
jednym kolanie w~bezkształtnym kombinezonie z~naciągniętym kaskiem,
światła wyłączone i~podświetlenie od lamp, które świeciłyby prosto w~oczy każdemu, kto próbowałby cokolwiek podejrzeć. To było zabawne
marzenie i~bawiła ją gra obmyślania, jak ktoś mógłby ją złapać i~jak ona
mogłaby uniknąć złapania. Był to szczególnie dobry sposób odpoczywania
po wysiłku wspinania się trzydzieści pięć biegów schodów lub frustracji
czekania przez czterdzieści pięć minut ze stygnącym garnkiem tadżin z~delikatesów na rogu w~rękach, zapach doprowadzający ją do szału.

Popchnęła ruiny tadżinu kieliszkiem taniej i~pysznej retsiny -- ulubionej
pośród tak wielu uchodźców, którzy przemierzali Grecję, a~teraz oznaką
uchodźców, którzy tamtędy nie uciekali, dzięki jej popularności w~obozach -- i~wyjrzała przez jej małe okno na Boston daleko poniżej, rzeka
Charles spuchnięta aż do wałów, ludzie jak mrówki rojący się do domu pod
światłami ulicznymi, gdy gładko zapadał jesienny wczesny wieczór.
Marzyła o~kamizelce odblaskowej i~światłach roboczych, o~narzędziach,
których by użyła, żeby zdemontować panel dla straży pożarnej, żeby
odkryć USB pod nim, subtelnych sposobach, w~jakie zmieniłaby algorytmy
budynku, tak, że anonimowi ludzie nigdy by nie odkryli jej wtargnięcia.

Dzwonek zabrzmiał przy drzwiach i~ekran pokazał Abdirahima, dziwnie
zniekształconego przez autofocus kamery na jego twarz, która była dobre
trzydzieści centymetrów poniżej uchwytu kamery na wysokości dorosłego.
Pomachała otwarcie drzwi i~wpuścił się, spoglądając na nią, potem na
tadżin z~winem i~znowu na nią.

-- Jadłeś? -- To była fraza, którą zapamiętała, że jej matka mówiła do
każdego, kto przeszedł przez drzwi, nawet jeżeli nie było jedzenie do
podzielenia. Kiedyś ją denerwowało, a~teraz mówiła to automatycznie przy
tych rzadkich okazjach, kiedy ktoś przeszedł przez \textit{jej} drzwi.

-- Tak. -- Abdirahim powiedział to zbyt szybko.

-- Ale nadal jesteś głodny. -- To nie było pytanie. Pamiętała swoje
trzynaście lat: głodna cały czas. Dała mu talerz i~położyła łyżką nieco
tadżinu, potem znalazła pitę i~włożyła do minipiekarnika, żeby ją
podgrzać. Kiedy piekarnik skończył, Abdirahim spojrzał na nią szeroko
otwartymi oczami.

-- Twój działa?

Zajęło jej chwilę domyślenie, co miał na myśli. Minipiekarnik. 

-- Działa
-- powiedziała. -- Naprawiłam go. -- Potem, z~lekką dumą. -- Oraz zmywarkę.
I termostat. I~lodówkę.

-- Pokaż.

-- Najpierw jedz.

Jedzenie ledwie dotknęło jego gardła w~drodze na dół. Czuła, że powinna
prawdopodobnie zmusić go do jedzenia wolniej jako in loco parentis, ale
była tak chętna, żeby się pochwalić, jak on, żeby zobaczyć. Kiedy
powiedział \textit{pokaż}, sprawiło to, że zrozumiała, że pękała od tajnej
wiedzy, którą chciała się bardziej niż cokolwiek podzielić.

Kiedy skończył, włożyła jego talerze do zmywarki i~wskazała mu miejsce
koło niej, żeby mogła pokazać mu ekran startowy, gdy wybierała program,
z dziwacznymi grafikami, które doinstalowała, antropomorficznych talerzy
o złym charakterze myjących się z~gniewem w~prysznicu ze znakiem
handlowym Disher. Klasnął, roześmiał się i~zażądał, żeby zobaczyć
resztę, a~potem, żeby pokazać mu jak to zrobić.

Mówią, że nie wiesz naprawdę jak coś zrobić, póki kogoś tego nie
nauczysz. Gdy Salima znowu wyszukała instrukcje, zrozumiała ile z~nich
było wykonaniem przepisu za pierwszym razem i~jak dużo zrozumiała od
tego czasu tak, że teraz kroki miały \textit{sens}. Była w~stanie wyjaśnić
Abdirahimowi \textit{powód} każdego kroku, prawie tak samo jak \textit{co} i~\textit{jak}, jej serce biło, jej krew śpiewała od doświadczenia
biegłości.

To było antidotum, zrozumiała, na poczucie odległych ludzi, których
nigdy nie poznała, a~którzy mieli władzę wszystkiego nad nią. Być w~stanie sterować komputerami dookoła niej, zamiast być sterowanymi przez
nie.

-- Widzisz -- powiedziała w~końcu, gdy zrozumienie w~końcu pojawiło się
znikąd i~pozostawiło ją w~osłupieniu i~oszołomieniu, uczucie jak u
prorokini objawienia. -- Widzisz, jeżeli ktoś chce Tobą władać przy
pomocy komputera, musi postawić komputer tam, gdzie \textit{jesteś}, i~gdzie \textit{oni} nie są, i~tak masz dostęp do komputera bez nadzoru.
Komputer, do którego masz dostęp bez nadzoru, jest komputerem, który
możesz zmienić, ponieważ wszystkie te komputery, w~głębi, są takie same.
Kiedy dotrzesz do programów pod skórą, tostera, zmywarki, termostatu,
wszystkie są takim samym komputerem w~innych skrzynkach. Kiedy
zdobędziesz władzę nad tym komputerem, wszystkie są Twoje.

Gdy słowa opuściły jej usta, jej mesjański ferwor został zastąpiony
przez dokuczliwe wątpliwości, wiedza, którą wykrzykiwała triumfująco do
małego chłopca, który kilka miesięcy temu właśnie wydostał się z~tymczasowego ośrodka dla uchodźców, i~poczuła się głupio i~mało. Ale
wtedy ujrzała błysk w~oczach Abdirahima, i~to był ten sam błysk co w~\textit{jej} oczach, i~zrozumiała, że oboje mają tę samą wizję.

-- Nasza zmywarka i~kuchenka nie działają od tygodni -- powiedział.

-- Och, jej. -- Nawet o~tym nie pomyślała. Odruchowo ukryła swoją pracę
przed Nadifą, ponieważ robiła coś potencjalnie niebezpiecznego i~nie
chciała, żeby Nadifa to pokazała. Jednak to było przed jej wizją. 

-- Powinniśmy coś z~tym zrobić. -- Wyciągnęła dłoń. -- Chodźmy pogadać z~Twoją mamą. -- Pamiętała, żeby zabrać retsinę ze sobą w~drodze do drzwi.

Może Nadifa byłaby zdenerwowana przez pomysł łamania zabezpieczeń jej
urządzeń AGD, ale tak byłoby, zanim spędziła dwa tygodnie, prowadząc dom
bez minipiekarnika lub zmywarki. Doświadczenie jedzenia wszystkiego
zimnego lub wyrzucania pieniędzy, których nie miała, na dania na wynos,
zmiękczyły jej obawy, które mogłaby mieć, i~gdy obserwowała Abdirahim
pokazującego jego siostrom jak włamać się do wszystkich głównych
urządzeń AGD, promieniowała matczyną dumą.

-- Jest w~tym dobry -- powiedziała Salima. -- Pokazałam mu tylko raz i~teraz\ldots  -- Wskazała dłonią.

-- Sądzisz, że będziemy mieć kłopoty? Z~budynkiem, wiesz? Są
właścicielami sprzętu AGD.

Salima wzruszyła ramionami. 

-- Dostawali część pieniędzy, które wcześniej
wydawaliśmy, na specjalny chleb, proszek do prania i~tak dalej. Ale po
bankructwie obu firm, nie będą oczekiwać nowych wpłat. Jednak jeżeli
firmy odbiją się po bankructwie i~ciągle nikt tutaj nie będzie używał
ich produktów\ldots 

Nadifa skinęła głową. 

-- To zdecydowanie byłby kłopot. -- Obserwowała
swoje dzieci, które zdjęły osłonę z~termostatu i~chciwie oglądały film
wideo na wielkim ekranie koło niego, gdzie osoba, której ciało było
animacją wielkiego królika, wyjaśniała jak włączyć tryb debugowania, w~którym termostat przyjmowałby komendy, które pozwoliłyby nadpisać główny
program. 

-- Ale jeżeli to tylko nas dwoje, czy kiedykolwiek się dowiedzą?

Salima znowu wzruszyła ramionami. 

-- Jeżeli system jest dobrze
zaprojektowany, to tak. Byłoby bardzo dziwne, gdyby nasze mieszkania
generowały znacznie niższe przychody niż wszystkie inne. Kiedyś miałam
robotę, gdzie odkryłam, że dwie kasy samoobsługowe w~aptece przynosiły
dwadzieścia procent mniej pieniędzy niż reszta. Z~początku myślałam, że
były uszkodzone, ale nawet kiedy oni je naprawili, a~nawet przenieśli,
ciągle były dwadzieścia procent niżej od przeciętnej. Wysłali je do
analizy i~okazało się, że są zhakowane i~były okradane.

-- Ale Ty jesteś dobra w~swojej pracy i~dbasz o~tego rodzaju sprawy. -- Niewypowiedziane: nikt dobry w~swojej pracy, kto troszczy się o~cokolwiek, nie był zamieszany w~sprawy pięter dla biednych w~Wieżowcach
Dorchester.

-- Jestem pewna, że dbają o~swoje pieniądze. Ale może nie są tak dobrze
zaprojektowane. -- Zastanawiała się nad tym. -- Nie wiem. Zdecydowanie
troszczą się bardzo o~robienie pieniędzy, zatem te części, które
przynoszą im pieniądze, prawdopodobnie otrzymywałyby dużo uwagi.
Poszukam o~tym. Muszą być inni ludzie w~tej sytuacji.

Dzieciaki z~powodzeniem przetestowały swoje zmiany w~termostacie i~zamknęły go, rozglądając się tu i~tam za czymś innym do zaatakowania.

-- Nie zapomnijcie o~lodówce -- powiedziała Nadifa. -- To będzie zabawa.
Bardzo zawiła.

Pobiegły do ekranu i~zaczęły pisać na klawiaturze, wysyłając Idil,
najstarszą dziewczynkę, żeby przeczytała numer modelu z~nalepki wewnątrz
drzwi lodówki.

Salima wiedziała, że dzieci nie zatrzymają się na swoich mieszkaniach,
oczywiście. Mijała je na korytarzu, czasem, śpieszących się z~jednego
mieszkania do następnego. Poczucie tego dało jej coś trudnego do
określenia: dumę i~podniecenie, ale także drżenie i~mdłości, bezsilne
uczucia, echo wolnego przewracania się w~żołądku, kiedy wróciła do domu,
żeby znaleźć wydrukowane groźby przyczepione do jej drzwi, do wszystkich
drzwi.

W windzie, Salima słyszała szepty: 

-- \textit{Załatwili już Twoje
mieszkanie? To takie proste. Znowu piekę chleb. Znaleźliśmy piękne,
używane talerze, co za przyjemność móc je umyć w~maszynie. Mój synek
zrobił to tak łatwo!}

Wtedy pewnej nocy, stukanie w~drzwi, pilne i~niskie. Otworzyła, to był
Abdirahim, szeroko otwarte oczy, zdenerwowany, i~jej puls przyśpieszył,
spociła się pod pachami i~pomyślała \textit{O nie, to jest to}.

-- Mów -- powiedziała, wprowadzając go do środka.

-- Zrobiłem to samo, co zawsze -- powiedział. -- Minipiekarnik. Zrobiłem
ich tak wiele. Ale coś poszło źle i~teraz nie chce się włączyć.

Westchnęła z~ulgą. Zepsuł toster jakiejś biednej osoby, ale nie
spowodował eksmisji wszystkich. 

-- Chodźmy zobaczyć.

Oczywiście, istniały wątki na forach dyskusyjnych na taką możliwość.
Abdirahim nie był pierwszą osobą, która zamieniła urządzenie w~cegłę.
Były triki, żeby wystartować je w~awaryjnym shellu, z~którego oryginalny
fabryczny system operacyjny mógłby zostać pracowicie reinstalowany, a~wtedy mogli zacząć od nowa. Wykorzystała Abdirahima jako pomoc, czytał
jej instrukcję i~szukał pomocy, kiedy otrzymali niespodziewane
komunikaty o~błędach.

Właścicielem minipiekarnika była stary Serb, który nigdy nie
wypowiedział słowa do niej, choć dzielili windy i~czekali w~holu razem
dostatecznie często. Zakładała, że był rasistą, ponieważ zwykle to był
powód, dla którego biali nie chcieli z~nią rozmawiać. Nie brała tego do
siebie. Niektórzy ludzie byli po prostu prostaccy.

Jednak wydawało się, że był boleśnie wstydliwy i~niezgrabny, a~nie
(koniecznie) rasistowski. Zaproponował im herbatę i~podał im herbatniki,
które ostrożnie odliczył z~paczki, którą wyjął z~prawie pustej szafki, w~której dojrzała tylko wielki słoik masła orzechowego z~banku żywności.
Przepraszał ich, żeby cztery razy skorzystać z~toalety, kiedy pracowali,
słyszała bolesne kapanie starca walczącego z~przerośniętą prostatą.

W pewnej chwili, była gotowa się poddać. Minipiekarnik nawet nie
pokazywał ekranu startowego, był w~gorszym stanie niż kiedy zaczynała.
Jednak starzec wyglądał na tak zatroskanego i~wiedziała, że nie byłoby
go stać na nowy toster, wyobrażała sobie zimne posiłki z~herbatniki i~masła orzechowego, na których żył, od kiedy Boulangism się zamknął.

Zatem ona i~Abdirahim wrócili do kroku pierwszego, sprawdzając wszystko,
\textit{wszystko} bardzo metodycznie. W~końcu, zauważyli, że minipiekarnik
właściwie był wcześniejszym modelem niż te, które wszyscy inni mieli w~budynku, zewnętrznie identyczny, ale z~numerem modelu, któremu brakowało
jednej litery, a~kiedy wyszukała na tej podstawie, otrzymała kompletnie
odmienny zbiór instrukcji, i~te zadziałały. Było po pierwszej w~nocy,
kiedy skończyli, ale nadal zeszła do swojego mieszkania i~wróciła ze
składnikami kanapek z~grilowanego sera na domowym chlebie i~zrobili
sobie ucztę w~środku nocy, która spowodowała u niej zgagę, ale było
warto.

Następnym razem, był to dzieciak, którego nie poznawała, stukający w~jej
drzwi i~proszący o~pomoc z~zbrickowanym ekranem. Abdirahim najwidoczniej
powiedział swojej armii Nieregularnych Dorchester, że była wiarygodnym
źródłem pomocy technicznej drugiej linii.

Za trzecim razem, gdy to się zdarzyło, zrozumiała, że musi wyprzedzić
zjawisko, jeżeli chce kiedykolwiek mieć chwilę dla siebie.

-- Abdirahim. -- Patrzyła intensywnie na chłopaka, aż na nią spojrzał.
Obecność Nadify pomogła.

-- Tak, ciociu? -- Nazywał ją tak tylko, kiedy miał kłopoty. Zwykle to
była ,,Salima'', lub z~nauczoną amerykańską zażyłością ,,Sally'', co
było czymś dla niej nowym, z~czego nie była do końca zadowolona.

-- Sposób, w~jaki to robicie, Ty i~Twoi przyjaciele, jest niebezpieczny.
Zostaniecie złapani i~ludzie, którzy tutaj mieszkają, też będą w~to
wciągnięci. Pamiętasz kapitanów windy i~co się zdarzyło.

-- Tak, ciociu.

-- Nie chcemy znowu tego, prawda?

-- Prawda, ciociu.

-- Nie chcemy, żeby wszyscy tutaj zostali wyrzuceni na ulice, prawda?

-- Prawda, ciociu.

-- Zatem chcę, że wszyscy Twoi przyjaciele przyszli do mnie, jutro, po
szkole. O siedemnastej. Powiedz im, każdy, kto nie przyjdzie, nie może
już więcej niczego hakować.

Był zaskoczony. 

-- Znaczy, jeżeli przyjdziemy, to nadal możemy hakować?

Uśmiechnęła się i~spojrzała na Nadifę. 

-- Och, tak, chłopcze. Nie
przestaniemy łamać reguł. Zaczniemy być po prostu w~tym
\textit{inteligentniejsi}.

Kupiła przekąski dla dzieci, więcej niż myślała, że potrzebuje, ale nie
wystarczyło. One ciągle przychodziły, drapiąc w~drzwi, aż po prostu
zostawiła je otwarte, a~one ciągle przychodziły, pakując się do jej
mieszkania. Dziesięcioro. Dwadzieścioro. Czterdzieścioro. Potem
przestała liczyć. Stały w~łazience i~podawały sobie torebki słodyczy i~precli, wypełniały kuchnię i~stały na ladzie. Jedno z~nich usiadło w~zlewie.

-- Cisza proszę, cisza. -- Uciszały się wzajemnie. Odwróciła się do
Abdirahima. -- Czy to wszyscy?

Zrobił pokaz wykręcania szyi dookoła pokoju, skinął głową. 

-- Chyba tak.

Pokręciła głową na niego. 

-- Zamknijcie, proszę, drzwi i~niech ktoś
włączy klimatyzację? -- Zapach był tym kozim, piżmowym zapachem wielu
dzieci w~różnym stopniu dojrzewania w~zamkniętej przestrzeni, zapach,
który przypomniał jej sypialnie dla dzieci w~obozach, w~których żyła.

-- Chciałabym zacząć od powiedzenia, że jestem z~was bardzo dumna. Sami
nauczyliście się ważnych umiejętności, wiecie, i~pomogliście sąsiadom,
kiedy tego potrzebowali. Ale wiecie, że to, co zrobiliście, co i~ja
zrobiłam, jest niezgodne z~przepisami i~teraz gdy to się zdarza to tak
często, będzie coraz trudniej nie zostać złapanym. Bycie przyłapanym nie
jest opcją. Czy ktoś może mi powiedzieć dlaczego?

Las ramion wystrzelił do góry i~zapach znowu ją sponiewierał. Wskazała
na młodą dziewczynę, pyzatą, z~miną pełną sprytnej psoty. 

-- Ponieważ
wszyscy zostaniemy wyrzuceni z~budynku?

Salima skinęła głową. 

-- Co jeszcze?

Dziewczyna myślała przez chwilę. 

-- Do tego wszyscy, którym pomagaliśmy,
zostaną wyrzuceni?

Salima znowu pokiwała głową. 

-- Bardzo dobrze. -- Dobrze było usłyszeć, że
dzieci znają stawkę. Było również przerażające usłyszeć je głośno i~jeszcze bardziej przerażająca była myśl o~poziomie beztroski, którą
dzieci pokazywały, nawet gdy znały tę stawkę.

-- Szukałam informacji o~tym, ponieważ nie wiem więcej na ten temat niż
któreś z~was. Nauczyłam się, oglądając te same filmy i~czytając te same
fora dyskusyjne co wy. Ale nie jesteśmy jedynymi, którzy przez to
przechodzą i~jest dużo o~tym dyskusji. Firmy, które produkowały zmywarki
i tostery, zostały wykupione przez nowych właścicieli, którzy mówią, że
wkrótce znowu będą online, a~zatem musimy wymyślić, jak się zachować
bezpiecznie, kiedy to się stanie.

Sprawdziła, czy nadal ją rozumieją. To było całkiem ezoteryczne, idea
bankructwa odległych przedsiębiorstw. Co te dzieci myślały o~sprzęcie,
który zhakowały? Czy postrzegały je jako tylko dziwnie niedziałające
gadżety, nad którymi musiały popracować, jak uszkodzone touchpady w~ośrodku? Czy raczej postrzegały je jako wrogów, coś, z~czym walczyli,
broń odległych przeciwników, którzy chcieli podporządkować ich swojej
woli?

-- Prawda jest taka, że nikt nie wie dokładnie, jak korporacje będą
monitorować to, co robimy, szczególnie gdy mają nowych właścicieli.
Wielu pierwotnych programistów zostało zwolnionych i~niektórzy z~nich są
na tych samych forach, lub przynajmniej twierdzą, że są autorami
oprogramowania, jest cały wyścig, żeby ustalić, kto pierwszy przedstawi
pewny sposób na oszukanie monitoringu, żeby myślały, że nie zhakowaliśmy
niczego. Zdecydowanie będziemy musieli zmienić sprzęt AGD ludzi, żeby
generowały \textit{jakieś} rachunki dla firm, ponieważ zauważą, że nie ma
pieniędzy z~Wieżowców Dorchester, prawda? -- Były przytaknięcia. Nadążali
za nią. To były bystre dzieciaki, pomyślała, dzieciaki, które spędziły
całe życie na przechytrzaniu urządzeń zaprojektowanych do kontroli.

-- Oto zadanie: przeczytamy \textit{wszystkie} te fora i~domyślimy się,
którzy ludzie wiedzą, o~czym mówią, potem pójdziemy do każdego
mieszkania i~poprawimy wszystkie urządzenia tak, żeby były bezpieczne.

-- To zakładając, że wymyślimy rozsądny plan. Jeżeli zdecydujemy, że nikt
nie wie, o~czym mówi, przywrócimy wszystkie urządzenia do ustawień
fabrycznych, odhakujemy wszystko. -- To wywołało stękanie i~nawet kilka
okrzyków protestu, a~ona uniosła dłoń. -- Wiem, wiem. Ale lepiej mieć
zepsuty sprzęt niż nie mieć domu. Miliony ludzi dookoła świata są w~tej
samej sytuacji co my, tak czy inaczej, i~wszyscy próbują rozwiązać ten
problem, więc może to nie będzie takie trudne. Wiem, że czytanie forów
dyskusyjnych nie jest taką zabawą jak bawienie się gadżetami, ale jeżeli
chcecie hakować, też będziecie musieli zrobić badania.

Było zbyt ciasno na pytania i~odpowiedzi, ale Abdirahim i~tak podniósł
dłoń. Wywołała go. 

-- Ciociu, jest jedna sprawa, której nie mogę
zrozumieć.

-- Tylko jedna? -- Uśmiechnęła się, a~on odwzajemnił uśmiech.

-- Na razie. Kiedy przynoszę zeszyt do szkoły, mogę w~nim napisać, co
chcę. Nie muszę pytać się firmy, która zrobiła długopis, czy sklepu,
który sprzedał mi zeszyt, jak go używać. Mogę wyrwać strony i~zrobić
papierowe samoloty, lub gryzmolić, lub zapisać to, co mówi nauczyciel.
Kiedy wkładam buty, mogę nosić takie skarpetki, jakie chcę. Mogę chodzić
w moim butach, gdzie chcę. Mogę podcierać się każdym rodzajem papieru
toaletowego\ldots  -- To wywołało śmiech. -- Ale nie mogę zrobić tosta, tak
jak chcę, w~moim tosterze.

Czekała. Walczył, żeby znaleźć słowa. 

-- Jak brzmi pytanie, Abdirahim?

Pokręcił głową, wzruszył ramionami. 

-- Chyba nie wiem. Po prostu chcę
zrozumieć, jak wybór chleba może być niezgodny z~prawem, ale wybór
skarpetek już nie? Co odróżnia toster od butów?

Otworzyła usta, żeby odpowiedzieć, ale odkryła, że nie zna odpowiedzi.
Aż do tej chwili, miała jakieś intuicyjne poczucie, które rzeczy były
,,przedmiotami z~prawem'', a~które były ,,przedmiotami bez praw'', czuła
to tak łatwo, że nawet nie nazwała tych kategorii, aż do tej chwili.
Teraz próbowała wymyślić regułę, która wyjaśniłaby różnicę pomiędzy tymi
dwoma kategoriami, okazało się, że nie potrafi.

-- To jest doskonałe pytanie -- powiedziała w~końcu. -- Dlaczego nie
poszukasz odpowiedzi i~nie opowiesz nam, czego się dowiedziałeś.

Przewrócił oczami i~wydał efektowny jęk, ale wyglądał na podnieconego
wyzwaniem. Naprawdę był bystrym człowieczkiem. Wygoniła dzieciaki z~mieszkania, brzęczących rozmowami, żartujących, popychających się i~pozujących jak to dzieci. Zapach się utrzymywał, kiedy poszły, włączyła
klimakonwektor na 140 procent nominalnej wydajności, funkcję, której cel
dla niej był tajemnicą, kiedy hakowała termostat. Jednak teraz była za
nią bardzo wdzięczna.

Próbowała pójść spać, ale pytanie Abdirahima nękało ją, więc podniosła
łóżko, póki nie stało się sofą i~pracowała na wielkim ekranie przez
kilka godzin, odkrywając ku jej uldze, że czytanie prawa technicznego
było lepsze niż tabletka na sen.

~

Było coś w~kobiecie stojącej nad nią w~pociągu, co przyciągnęło jej
uwagę. Nie była żadną znajomą Salimy, ale \textit{coś} w~niej wydawało się
\textit{tak }znajome. Rzucała ukradkowe spojrzenia i~wtedy zrozumiała:
logo na identyfikatorze dziewczyny, wiszące na smyczy na wysokości oczu.
Salima znała to logo, choć nie widziała go przez miesiące: to było logo
Boulangism, stylizowana kromka chleba, jedną ciągłą linią, trzy
falowane, promieniujące linie ciepła.

Kobieta złapała jej spojrzenie i~nawiązały kontakt wzrokowy. Była
młoda, biała, miała niechlujne, brązowe włosy i~te soczewki kontaktowe,
które ludzie od komputerów nosili, żeby pomóc im wpatrywać się cały
dzień w~ekrany bez psucia cyklu dobowego snu. Światło z~reklam
wyświetlających się na górnej jednej trzeciej wagonu rzucało na nie
tęczowy połysk.

-- Boulangism? -- spytała Salima.

Kobieta pokiwała entuzjastycznie głową. 

-- Zgadza się.

-- Są w~Bostonie?

-- Teraz już tak. Zostali wykupieni przez fundusz przy Drodze 128\footnote{
nazwa klastra firm high-tech w~okolicach Bostonu,
więcej~\url{https://pl.wikipedia.org/wiki/Droga\_128} -- przyp.tłum.}, a~potem zrekrutowali nas, dzieciaki MIT, żeby przywrócić
firmę do porządku i~postawić ją na nogi. -- Kobieta była młodsza niż
myślała Salima, studia inżynierskie lub może młoda studentka studiów
magisterskich. Salima próbowała wyobrazić sobie dzieciaki, które
tłoczyły się w~jej mieszkaniu jako znajomych z~klasy tej młodej kobiety,
oderwanej od studiów i~skuszonej przez bardzo bogatą firmę. Były bystre,
ale czy były dostatecznie bystre? Jak bystrym trzeba być? Nagle chciała
dowiedzieć się więcej o~kobiecie.

-- Masz Boulangism?

Salima skinęła głową. 

-- A Ty?

Kobieta parsknęła. 

-- Boże, nie. Znaczy, ten model biznesowy,
autoryzowany chleb? Nie wstawiłabym żadnego do domu, nawet jeżeli byś mi
zapłaciła. Dlaczego je kupiłaś?

-- Nie kupiłam. Był na wyposażeniu mieszkania.

-- Odłącz i~wstaw do szafy. Załatw sobie prawdziwy piekarnik.

To dosłownie nigdy nie przyszło Salimie przez myśl. Nie miała pojęcia,
ile minipiekarnik kosztuje. Prawdopodobnie było ją stać. Nie, żeby miała
jakąś przestrzeń w~szafie. A jeszcze byli wszyscy ci ludzie na piętrach
dla biednych, starsi, ci z~dziećmi, ci, którzy nie mieli jej
umiejętności lub nie potrafili rozmawiać po angielsku. Nie mogła kupić
im wszystkim piekarników, nie mówiąc już o~zmywarkach i~innych sprzętach
AGD.

-- Nie wydaje się, żebyś ich mocno ceniła.

Kobieta przewróciła oczami. 

-- To dobra praca, a~wyzwania techniczne są
interesujące. Cholera, nawet podstawowe urządzenia robią dobrą robotę,
jeżeli o~to chodzi. Jednak blokada jest ob-rzyd-liwa.

Salima nie mogła się powstrzymać. 

-- Też tak myślę.

-- Stworzyli cały zespół, żeby znaleźć ludzi, którzy hakowali ich
urządzenia, drużyna kapusi. No to \textit{jest} praca, której bym się nie
podjęła. Dziewczyna musi mieć jakieś standardy.

Minęły przystanek Salimy. Nie dbała o~to. Zawsze mogła pojechać w~drugą
stronę. Rozmowa była zbyt dobra, żeby ją przerwać. 

-- Nazywam się Salima.

-- Wyoming -- powiedziała dziewczyna i~potrząsnęła dłonią. Jej dłonie były
szczupłe i~zręczne, dłonie do klawiatury.

Choć były w~bardzo publicznym miejscu, Salima poczuła bańkę braku
zainteresowania dookoła nich, taką miejską, gdzie udawałaś, że nie
widzisz ludzi tłoczących się wokół Ciebie. To było coś, w~czym była
dobra w~obozach, gdzie było tak wiele sytuacji, kiedy ta umiejętność
była konieczna, żeby zachować zdrowy rozsądek.

-- Jak myślisz, kiedy znowu uruchomią serwery? -- spytała Salima. Facet
koło drzwi wstał, żeby wysiąść, gdy wjeżdżali na stację. Jak wiele
przystanków jeszcze miały? Salima pomyślała, że zostanie jeden
przystanek za Wyoming, wyjdzie, przejdzie na przeciwny peron i~wróci.
Jeżeli wyszłyby na tym samym przystanku, wyglądałaby jak obleśna
stalkerka.

Wyoming była zaskoczona pytaniem. 

-- Och, gówno! Nie pomyślałam o~tym. To
musiałaś mieć \textit{przechlapane}. Ile to było? Cztery miesiące? Pięć?
Bez piekarnika? Musisz być \textit{tak} stęskniona za nami, co?

-- Pięć miesięcy -- powiedziała Salima. -- Zdecydowanie to długi czas.

-- Biedactwo. Ja bym po prostu swój zhakowała. Jest tak wiele tego,
szczerze, miną miesiące, zanim to ogarną. Jednak nie winię użytkowników.
Znaczy, \textit{błe}.

-- Zgadzam się. -- Obie się roześmiały. Wagon był prawie pusty i~zostały
jeszcze dwa przystanki, zanim dotarłyby na koniec linii. Gdzie ta
dziewczyna mieszkała? Z~fajną pensją programistki, mogła pozwolić sobie
na piękne miejsce w~środku Cambridge, a~nie miejsce tutaj w~Needham.

Salima podjęła ryzyko. Polubiła tę dziewczynę. 

-- Czy to trudne,
hakowanie?

-- Nie, to nie takie trudne, znaczy, jak mogłoby być? Widzisz, to
podstawowe zabezpieczenia, podstawy matematyki. Chcesz uruchomić program
na tosterze, który pozwoli Ci przejąć kontrolę nad weryfikacją chleba.
Nie chcemy, żebyś go uruchamiała. Zatem mamy coś w~systemie operacyjnym,
co sprawdza, czy program, który uruchamiasz, jest tym, który
zatwierdziliśmy. Żeby to sprawdzić, patrzymy na podpis cyfrowy,
sprawdzamy, czy program został podpisany prywatnym kluczem, który
trzymamy w~tajemnicy.

-- Trochę się cofnę. W~kryptografii, mamy to pojęcie kluczy prywatnych i~publicznych\footnote{ por.
\url{https://pl.wikipedia.org/wiki/Kryptografia\_klucza\_publicznego} -- przyp.tłum.}. Działają w~parach. Wszystko, co zaszyfruje prywatny
klucz, tylko klucz publiczny odszyfruje, i~na odwrót. Jeżeli klucz
publiczny to odszyfrował, to musiało zostać zaszyfrowane kluczem
prywatnym. Jeżeli klucz prywatny może to odszyfrować, to musiało być
zaszyfrowane kluczem publicznym. Rozumiesz?

Nagle, tak dużo z~tego, co czytała na forach, nabrało sensu. Klucze
publiczne i~prywatne, działające parami. Co jeden skręci, drugi odkręci.

-- Mogę zakodować wiadomość Twoim kluczem publicznym i~moim prywatnym i~tylko Ty możesz ją przeczytać i~dowiedzieć się, co napisałam. -- Nadchodziło powoli, ale zrozumienie ją zalewało. To było takie
eleganckie.

-- Dokładnie tak. Cóż, Boulangism jest wysyłany z~publicznym kluczem
firmy, a~wszystkie aktualizacje, które są wysyłane, są podpisane
prywatnym kluczem. Jeżeli publiczny klucz odszyfruje podpis, wtedy
minipiekarnik wie, że może zaufać aktualizacji, ponieważ została
podpisana przez kogoś używającego publicznych kluczy w~korporacji. -- Więcej zrozumienia, jak ostatni akt powieści kryminalnej, gdzie
wszystkie wskazówki razem się łączą i~dezorientacja zmienia się w~uporządkowany ciąg zdarzeń. Prawie podskoczyła, żeby coś powiedzieć, ale
powstrzymała się, ponieważ pomyślała, że wie, dokąd to zmierza i~nie
chciała ostrzec tej nieznajomej, że wie więcej. Była miłą kobietą,
jasne, ale pracowała dla wroga.

-- To działałoby w~stu procentach, ponieważ matematyka naprawdę działa.
Coś, co zostało zakodowane jedną połową pary kluczy, może być tylko
odkodowane drugą połową. Miliardy lat trwałoby, żeby coś takiego
sfałszować, nawet z~wszystkimi komputerami na świecie pracującymi tylko
nad tym. Ale jest jedna słabość.

Serce Salimy waliło. \textit{Wiedziała}, co Wyoming powie dalej, ponieważ
zagadka była w~ostatnim akcie i~ona wyprzedzała detektywa, gdy odkrywał
zabójcę i~szczegóły zbrodni.

-- Klucz publiczny, który wykorzystuje Boulangism, jest przechowywany na
samym urządzeniu. Jest zakopany w~zabezpieczonym chipie, który nie
powinien dawać się zmienić, ale jest tak, tak wiele sposobów obejścia.
Czasem to błąd w~chipie, coś, co pozwala podmienić klucz. Znacznie
częściej, jest to sekwencja bootowania, gdzie komputer w~minipiekarniku
dowiaduje się, jakim jest komputerem i~gdzie ma szukać kluczy
publicznych. To jest także w~bezpiecznym magazynie, ale istnieją
sposoby, żeby je zmienić, ponieważ programiści robią błędy, a~kiedy je
robimy, nasze programy mogą być zhakowane przez złych facetów, dlatego
chcemy być w~stanie wysłać nowy kod na Twoje urządzenia.

-- Zatem właściciel Boulangism włącza internet, dowiaduje się, jak
zmienić klucze lub zmienić, gdzie komputer szuka kluczy, i~podmienia
klucze, tak, że właściciel ma też prywatny klucz, więc może teraz
podpisać dowolny program i~uruchomić go na tosterze. Boulangism wynajął
wykwalifikowanych inżynierów, żeby spędzili lata, blokując ich produkty
i zostali pokonani w~kilka godzin przez nastolatków z~amatorskim
sprzętem. To nie tak, że programiści byli głupi, ale z~pewnością
\textit{robili} coś głupiego.

Salima się uśmiechnęła.

 -- Ale Ty nie robisz głupot? Pracujesz nad czymś
innym?

Wyoming odwzajemniła uśmiech. 

-- Dokładnie tak. Raczej bym zjadła szkło
niż robiła to głupie gówno. Pracuję nad kuchnią adaptacyjną, wiesz,
wykorzystanie czujników, żeby się upewnić, że jedzenie jest doskonałe.
To jest super satysfakcjonujące i~smaczne, ponieważ jestem testerem,
kiedy od czasu do czasu coś ugotuję.

-- To niezły dodatek.

-- Tak i~jest jeszcze siłownia, co jest dobre, ponieważ minęły tylko dwa
tygodnie i~przytyłam półtora kilo.

Pociąg jechał na stację i~nastąpiło niezrozumiałe zakłócenie z~głośników, gdy konduktor coś ogłaszał. Zaskoczona, Salima zrozumiała, że
mówił im, że dotarli do końca linii i~wszyscy mają wysiąść. Wstała i~próbowała wymyślić, jak ma zostać na peronie i~wrócić, nie ujawniając
przed Wyoming, że została w~pociągu, tylko żeby wyciągnąć informacje.

Wyoming założyła plecak, Salima zawiesiła torebkę i~włożyła ją pod
ramię, i~wyszły z~pociągu i~podryfowały ku ruchomym schodom na poziom
gruntu. Salima pogodziła się ze stratą biletu do metra, żeby wrócić na
poziom ulicy i~przejść dookoła bloku i~potem wrócić prosto na stację.

Przy wejściu na schody ruchome, Wyoming położyła swoją mądrą dłoń z~długimi palcami na jej ramieniu i~pociągnęła ją na stronę. 

-- Muszę Ci
coś wyznać. -- Rumieniła się.

-- Och?

-- Mój przystanek był jakieś osiem wcześniej. Miło mi się z~Tobą
rozmawiało, więc zostałam w~pociągu. Ja, hm, tak naprawdę nie rozmawiam
zbyt dużo o~mojej pracy. A Ty jesteś dobrą słuchaczką.

Salima nie mogła się powstrzymać. Roześmiała się. 

-- Mój przystanek też
był kilka przystanków wcześniej. Green Street. Podobała mi się nasza
rozmowa tak bardzo, że\ldots 

Oczy Wyoming się rozszerzyły i~położyła dłoń na ustach.

 -- Nie mówisz
serio\ldots  -- Zachichotała, i~to sprowokowało Salimę, ich śmiech napędzał
drugiej, aż oboje sapały, gdy konduktor ogłosił, że pociąg, z~którego
właśnie wyszły, jest gotowy do drogi i~wsiadły z~powrotem na miejsca,
które właśnie opuściły.

W drodze powrotnej, Salima dowiedziała się, że Wyoming przybyła z~Cincinnati, że zrobiła inżyniera z~elektrotechniki i~informatyki w~MIT,
że zaczęła magisterskie w~matematyce stosowanej, kiedy rekruter
Boulangism zaoferował jej potężną premię za podpisanie umowy, udziały w~nowej firmie, którą rozkręcał fundusz hedgingowy i~wielką miesięczną
wypłatę.

Salima powiedziała jej nieco o~sobie, ostrożnie, ponieważ spotkała wiele
bardzo miłych białych osób, które nie były tak miłe, kiedy powiedziała
słowo ,,uchodźczyni''. Do czasu, kiedy dotarły do jej przystanku,
Wyoming przesłała jej smsy z~adresami do wiadomości, numerem telefonu i~zaoferowała pomóc Salimie kiedykolwiek jej toster by ,,grymasił''.
Podały sobie ręce, kiedy wstała do wyjścia, i~Salima spojrzała na
Wyoming -- Wye, powiedziała, nazywaj ją Wye -- i~ona patrzyła na nią, i~uśmiechnęły się po raz ostatni i~pomachały na pożegnanie.

Wjeżdżając schodami ruchomymi, miała krótki atak paranoi. Ich rozmowa
była taka luźna, ciepła i~przyjemna, czy to była pułapka? Czy Boulangism
miał szpiegów, którzy próbowali usidlić podejrzanych piratów,
zaprzyjaźnić się z~nimi w~pociągu przez pokazanie im swojego
identyfikatora na poziomie oczu?

Pokręciła głową i~ustawiła się w~ścisku do bramek. To był absurd. Świat
był zaledwie małym i~dziwnym miejscem i~nikt tego nie kwestionował.

Czekając w~holu tej nocy, wpadła na troje dzieci, którym wykładała w~tłoku w~mieszkaniu, razem z~mieszanymi rodzicami, pocącymi się w~niespodziewanie wczesnej wiosennej odwilży, płaszcze przełożone przez
ramię z~wełnianymi czapkami i~szalikami wysuwającymi się z~kieszeni.

Dzieciaki zawołały do niej ,,ciociu'', uśmiechnęła się do nich i~przedstawiła się formalnie dorosłym, dwóm matkom, babce, znajome twarze,
które widziała w~windzie i~korytarzach, ale nigdy nie miał powodu do
porozmawiania. Dorośli wiedzieli, kim jest, to było jasne, i~traktowali
ją z~dziwnym szacunkiem, jako dostawcę technologicznej wolności.
Rozpoznała relację z~obozu, gdzie zawsze były osoby techniczne i~energiczne, które mogły załatwić rzeczy, dostarczyć specjalne jedzenie i~alkohol, załatwić kartę do telefonu lub naprawić telefon, który oddał
ducha. Nigdy nie myślała o~sobie w~tych terminach, ale z~jej
nowoodkrytym zrozumieniem, miało to doskonały sens, nawet dla niej.

Pogawędka z~tymi sąsiadami wydawała się kontynuacją rozmowy z~Wye w~pociągu, częścią długiej rozmowy o~temacie, który ją zaangażował:
przejęcie kontroli nad technologią wokół niej, tą, która była używana
przez odległe i~anonimowe siły, które chciały \textit{ją kontrolować.} Wye
mówiła bardzo dużo o~beznadziejności próbowania kontrolowania kogoś przy
pomocy urządzenia, które ktoś może zabrać do domu i~wykorzystać
prywatnie, teraz to było całkowicie sensowne i~było objawieniem.
Poczucie beznadziejności w~byciu otoczonym czujnikami i~urządzeniami,
które były zaprojektowane, żeby nią pomiatać, przekształciło się w~poczucie nieuniknionego triumfu nad głupcami, którzy myśleli, że mogliby
tak z~nią postępować. Jak to powiedziała Wye? ,,To nie tak, że
programiści byli głupi, ale z~pewnością robili coś \textit{głupiego}.''

Nowo odkryta pewność siebie śpiewała w~niej, gdy podgrzewała nieco curry
z kozy i~ryżu jollof w~swoim Boulangism, usiadła przed wielkim ekranem
salonu, żeby pogrzebać jak utrzymać jej gadżety pod kontrolą bez
donoszenia do producentów. Nie znalazła żadnego bezpośredniego
rozwiązania, ale to nie miało znaczenia: rozkoszowała się znajomością
rzeczy, czytając ponownie wątki forów dyskusyjnych, które wcześniej ją
zastanawiały, otwierając linki i~dowiadując się więcej, więcej i~więcej\ldots 

Dzwonek przy drzwiach sprowadził ją z~jej marzeń, spojrzała w~róg
ekranu, żeby zobaczyć Abdirahima przy drzwiach. Zegar wskazywał 20:45,
prawie jej pora snu. Wpuściła go i~dała mu jabłko z~miski z~owocami.
Zawsze był głodny.

-- Odkryłem, dlaczego toster nie jest jak skarpetki -- powiedział z~ustami
pełnymi jabłka.

Usiadła i~gestem zachęciła do mówienia.

-- To prawo autorskie, wiesz, jak ostrzeżenia na początku filmów?

To nie brzmiało poprawnie, ale spróbowała tego nie pokazać miną. Jeżeli
się mylił, mogli to rozgryźć razem. W~międzyczasie, był
trzynastolatkiem, który wyruszył i~przeczytał masę nudnych wyjaśnień
prawa technicznego, ponieważ mu kazała, a~to zasługiwało na pełne
szacunku wysłuchanie.

-- Znam je.

-- Dawno temu, w~zeszłym wieku, zrobili przestępstwo z\ldots  -- zmarszczył
czoło, gdy walczył z~dziwnym, zapamiętanym zdaniem -- \ldots  ,,omijania
skutecznych środków zabezpieczenia
technicznego''\footnote{por.~\url{https://euipo.europa.eu/ohimportal/pl/web/observatory/faqs-on-copyright-pl}
-- przyp.tłum.}. Jeżeli coś jest dziełem, ma prawa autorskie, wiesz,
film, czy cokolwiek, i~jest coś innego, co kontroluje do tego dostęp,
nie możesz usunąć tej kontroli lub zrobić czegokolwiek z~tym. Nawet nie
z dobrych powodów. Mogą wysłać Cię do więzienia za to, pięć lat i~pół
miliona dolarów kary! Za pierwsze przestępstwo!

-- Ok, to brzmi, jakby mogło być prawdziwe, ale co to ma wspólnego z~minipiekarnikami? Czy jest prawo autorskie dla chleba?

Pokręcił głową. 

-- Też tutaj mi się pomieszało. Ale to nie chleb ma prawa
autorskie, to software w~tosterze, wszystkie te rzeczy, które zmieniamy,
kiedy łamiemy zabezpieczenia. Ta część, kiedy musisz zresetować i~zrobić
coś dziwnego i~skomplikowanego, żeby zaczęło działać, to właśnie jest
ob-cho-dze-nie -- Nie mogła nie zauważyć, że wykuł na pamięć -- prawa
autorskiego, to jest kod, który zmieniamy. Więc jeżeli to ma program w~sobie i~jest tam kontrola dostępu, nie wolno Ci zmienić programu. Nawet
jeżeli należy do Ciebie!

Kolejny duży fragment zrozumienia wsuwał się na miejsce, większość z~połowy zapamiętanych postów z~połowy zapamiętanych wątków ustawiała się
w uporządkowaną serię. Pokiwała energicznie głową. 

-- Abdirahim, myślę,
że masz rację, to dokładnie tak brzmi. Bardzo dobrze zrobiłeś,
powinieneś być z~siebie dumny.

Rozświetlił się z~pełnymi ustami jabłka. Dotarł do rdzenia i~odgryzał
każdy kawałek na tym z~praktyką kogoś, kto dorastał głodny. Co za dziwny
świat, gdzie ten chłopiec uczy ją o~prawie autorskim z~innego wieku.

-- Nawet rozmawianie o~tym jest przestępstwem, dlatego nie znajdziesz
niczego w~zwykłym internecie, dlaczego musisz używać narzędzi darknet,
żeby dowiedzieć się więcej. Mówienie komuś jak złamać zabezpieczenia
urządzenia jest tym samym co sam łamanie ,,jailbreak'', ale jest to
nazywane ,,dilowaniem'' jak z~narkotykami, i~tak samo, pięć lat w~więzieniu za pierwsze przestępstwo!

Jej jelita powoli się przewróciły. Jak wiele osób nauczyła hakować? Jak
wiele wyroków pięciu lat więzienia zgromadziła? Brakowało jej dawnej
niewiedzy, tylko kilka godzin temu, kiedy nic z~tego się jej nie
objawiło, a~fora były pełne tajemnic. Z~radości zrozumienia do strachu,
w minuty.

-- Jak się nazywało to prawo?

Abdirahim skonsultował się z~telefonem. 

-- To Sekcja 1201 Digital
Millennium Copyright Act z~1998 roku\footnote{ ustawa U.S.A. Digital
Millennium Copyright Act, w~skrócie DMCA, wdraża traktat WIPO tj.~Światowej Organizacji Własności Intelektualnej,
\url{https://pl.wikipedia.org/wiki/\%C5\%9Awiatowa\_Organizacja\_W\%C5\%82asno\%C5\%9Bci\_Intelektualnej},
WIPO razem z~WTO zaproponowało i~wprowadziło zasady opisane w~porozumieniach TRIPS,
zob.~\url{https://pl.wikipedia.org/wiki/TRIPS}, te
porozumienie opisane m.in.
w~\url{https//eur-lex.europa.eu/legal-content/PL/TXT/?uri=LEGISSUM:r11013},
obowiązuje w~Rzeczypospolitej Polski}.

Zanotowała sobie. 

-- Dziękuję, Abdirahim. Muszę trochę poczytać.

Pozwolił jej dać sobie kolejne jabłko w~drodze z~mieszkania. Zrobiła
sobie kubek kawy -- zwykły przelew, choć marzyła o~pójściu do Nadify po
dzbanek mocnej, ziarnistej kawy etiopskiej podawanej w~małych
filiżankach -- i~zaczęła czytać o~prawie.

Salima się ucieszyła, gdy następnego wieczoru zeszła Nadifa, kiedy
wróciła do domu po pracy w~małym sklepie z~kanapkami, przeglądając ich
księgi i~wracając myślami, raz po raz, z~nieskończonej pętli martwienia
się o~niebezpieczeństwa, na które naraziła jej sąsiadów.

Nadifa przyniosła butelkę retsiny i~talerz małych baklaw z~nadzieniem
kajmakowym, ociekających miodem. Najwcześniejsze wspomnienie Salimy
dotyczyło tego smaku, czysta pamięć wrażenia z~czasów, kiedy była
dziewczynką, może nawet niemowlęciem dopiero zaczynającym jadać stałe
pokarmy. To był smak, który zapomniała w~latach w~obozach, pierwszy raz,
gdy natrafiła na niego w~Bostonie, był objawieniem, szokiem, od którego
się popłakała i~pomyślała o~rodzicach, tak dawno utraconych. Było tak
łatwo dla dawnej jej zamienić się w~nową osobę, tak łatwo wymazać
wspomnienia tej dawnej normalności. To była umiejętność, która jej
dobrze służyła, ale tym kawałkiem i~jego miodzie policzyła koszt
pierwszego razu.

Nadifa wiedziała o~tym, ale wiedziała również, jak bardzo Salima lubiła
ciasta, ponieważ rozumiała, jak to jest wchłonąć przeszłość nie do
pomyślenia, która sprawia, że jesteś, kim jesteś. To była więź, którą
dzieliły obie kobiety.

-- Dawno Cię nie widziałam, musiałam się pytać Abdirahima.

Wykorzystała łyk wina, żeby spłukać miód, cierpkie i~słodkie się
mieszające. 

-- Wiesz, jest takim bystrym chłopcem.

-- Wiem. Zbyt bystrym. Zawsze, od małego. Pytającym dlaczego, dlaczego,
dlaczego cały czas i~nigdy niezadowolonym z~odpowiedzi ,,bo tak mówię''.

-- To znak dobrego charakteru. Byłam kiedyś taka, kiedyś.

-- Kiedyś? -- parsknęła Nadifa. -- Spędziłaś ostatnie pół roku
przeprogramowując nasze mieszkania, Salima. Nigdy nie przestałaś,
dziewczyno.

Pokręciła głową. 

-- Myślę, że zrobiłam błąd.

-- Tak, wiem, że zrobiłaś. To właśnie powiedział mi Abdirahim. Myśli, że
jesteś przerażona.

-- Nigdy mu tego nie powiedziałam.

-- Tak. 

-- Jest bystrym chłopcem, pamiętasz?

-- Jestem przerażona. -- Stuknęła w~ekran, pokazując Nadifie artykuł. -- Dwa tygodnie temu, Boulangism się zrestartował. Dzisiaj był Disher. -- Stuknęła znowu. -- Popatrz na to: Zapewnienie Zgodności\footnote{ oryg.
compliance assurance -- przyp.tłum.} Sp. z~o.o., nowa firma, otrzymała dwadzieścia osiem
milionów na inwestycje w~produkt, który odkryje zhakowane urządzenia
AGD. To tylko \textit{dzisiejsze} wiadomości.

Nadifa skinęła głową i~spojrzała z~namysłem. 

-- Nie będę udawać, że
rozumiem wszystko z~tego, ale wiem co nieco, wystarczająco, żeby
rozumieć, dlaczego jesteś przerażona. Pytanie brzmi, co z~tym wszystkim
zrobisz?

Salima patrzyła się w~ekran. Nie chciała spojrzeć w~oczy Nadifie. 

-- Próbowałam znaleźć rozwiązanie. Jest tak wiele osób w~tej samej sytuacji
co my, ludzi, którzy złamali zabezpieczenia po tym, gdy korporacje
zbankrutowały i~teraz nie wiedzą, co dalej robić. Mogłabym powiedzieć
dzieciom, żeby poszły i~przywróciły wszystko do wcześniejszego stanu.

Nadifa znowu prychnęła. 

-- Nie, nie mogłabyś.

-- Mogłabym im kazać. Ale one mogłyby tego nie zrobić.

-- Nie zrobiłyby. To dzieci. Jeżeli rozumiałyby ryzyka, nie dołączałyby
do powstań i~marszy na ulicach i~świat byłby prostszym miejscem. Nie
lepszym, oczywiście. Ale prostszym.

-- Zatem będzie lepiej, jak nadal będę szukać rozwiązania. Ci ludzie z~Zapewnienia Zgodności idą po nas.

Nadifa poklepała ją po ramieniu. 

-- Coś wymyślisz.

Wymyśliła, godzinę później, i~nie było to dobre.

-- Halo?

-- Czy to Wye?

-- Tak. -- Brzmiała zmęczona, choć była dopiero dwudziesta pierwsza. Czy
ci techniczni nie mieli być nocnymi sowami? -- Kto to?

-- Tu Salima. Poznałyśmy się w~pociągu?

-- Och! Hm, cześć. Przepraszam. Długi dzień i~właśnie drzemałam na sofie.
Co jest?

Salima wiedziała, że gdyby zaplanowała tę rozmową, to by stchórzyła,
zanim mogłaby ją wykonać. Otworzyła usta i~nic nie powiedziała.

-- Salima?

-- To\ldots  to rodzaj nagłego wypadku.

-- Wszystko ok? -- Była teraz bardziej obudzona, przestraszona.

-- Ok. Ale\ldots  -- Przerwała. -- Czy możemy się gdzieś spotkać? Mogę
przyjechać do Ciebie. Może kawiarnia?

Przerwa. Wydłużała się. Potem: 

-- Tak, dobra. Prześlę Ci adres, dobra?

-- Dobra. Dam Ci znać, jak wyjdę z~pociągu.

-- Super. To jest tuż za rogiem koło mnie.

Salima nie wierzyła, żeby agenci z~Boulangism lub Dishera podsłuchiwali
jej telefon, ale ciągle instynktownie myślała, że byłoby lepiej odbyć tę
rozmowę z~tak małą ilością technologii w~użyciu, jak to możliwe.
Denerwowała się w~pociągu i~wysłała smsa ze schodów ruchomych.

Restauracja była całkowicie zautomatyzowana, co oznaczało, że nie było w~pobliżu podsłuchujących ludzi, ale oznaczało to, że wszędzie były
mikrofony i~kamery. Skrzywiła się i~próbowała się nie gapić, gdy
czekała, żeby przyszła Wye.

Przyszła ubrana w~dżinsową kurtkę na spłowiałym t-shircie z~obrazem
antropomorfizowanej piłki do baseballu z~ramionami i~nogami i~wielkim
,,C'' na kurtce, z~czego Salima wywnioskowała, że musiało to być coś
wspólnego z~Cincinnati. Salima nie miała podkoszulek z~logo w~dawnym
domu. Czy istniała drużyna baseballu w~Libanie? Tygrysy Bengalskie
Benghazi?

-- Cześć -- powiedziała Wye i~wbiła szybkie zamówienie na stole z~łatwością, a~taśmociąg, który wił się z~tyłu kabin, uruchomił się i~zaczął warkotać w~ich kierunku, piszcząc.

-- Dzięki za przyjście.

Gorące pudełko na taśmociągu dotarło do ich stołu i~sapnęło przy
otwarciu, pokrywka szeroko się odchyliła. Wye wyjęła kubek herbaty z~paczką ciastek położonych na spodku z~prasowanej skrobi kukurydzianej. 

-- Pewnie. Zamówiłaś coś? Grilowane kanapki są dobre, jeżeli jesteś głodna.
Mają też bardzo dobre boba\footnote{ bubble tea,
zob.~\url{https://en.wikipedia.org/wiki/Bubble\_tea} -- przyp.tłum.}. -- Czy tak się odżywiali studenci MIT?

-- Dzięki -- powiedziała Salima. Szybko przejrzała menu i~stuknęła w~sodę
selerową\footnote{ dosł. celery soda -- przyp.tłum.}, która brzmiała ohydnie,
ale kosztowała dokładnie wartość ,,minimum dla jednej osoby'' określoną
na dole menu. Gorące pudełko samo się zamknęło i~taśmociąg znowu zaczął
się poruszać. -- I~jeszcze raz dziękuję za przyjście, serio.

-- Jak mówiłam, to nie kłopot. Normalnie byłabym przytomna o~tej porze,
ale ostatnio mój sen został spieprzony przez stres w~pracy i~źle spałam
ostatniej nocy. Szczerze, miło było Cię znowu usłyszeć. -- Uśmiechnęła
się. To był piękny uśmiech, z~dobrymi białym zębami i~dołkiem. Była tak
młoda, nawet jeżeli była rok starsza niż Salima.

-- Zadzwoniłam do Ciebie, ponieważ nie wiem z~kim rozmawiać. Ja\ldots  -- Westchnęła. -- Pomogłam moim sąsiadom zhakować ich rzeczy.

-- Ich rzeczy?

-- Wszystko. Zmywarki Dishera. Minipiekarniki Boulangismu. Termostaty,
lodówki, telewizory, telefony. Wszystko. -- Pomyślała. -- Jednak nie
windy.

Ona się roześmiała\ldots  roześmiała! 

-- \textit{Windy}?

-- Żyję na biednym piętrze z~niskim czynszem. Windy zatrzymują się dla
nas tylko, gdy nikt z~normalnych pięter nie czeka lub nie jedzie. To
może zająć bardzo dużo czasu.

Wye przestała się uśmiechać. 

-- I~dlatego wszystkie sprzęty AGD wymagały
hakowania?

-- Były na wyposażeniu mieszkań. Nie mamy prawa ich dotykać, to jest w~umowie. Mogłybyśmy zostać wyrzucone. Mam na myśli, wszyscy.

Wye patrzyła na Salimę szeroko otwartymi oczami. Otworzyła herbatniki i~zjadła jednego, ciągle patrząc na Salimę. 

-- Do. \textit{Kitu}.

Teraz Salima się uśmiechnęła. 

-- Też tak myślimy.

-- Zatem, co, chcesz przywrócić wszystko do poprzedniego stanu, zanim Cię
złapią?

-- Nie chcemy być złapani.

-- Och. \textit{Ochhh}. Chcesz nadal to robić. Ha. Wiem, że jest tam trochę
rzeczy, które powinny zadziałać.

-- Szukałam. Jest dużo porad i~wszystkie sobie zaprzeczają. Nie umiem
powiedzieć, co zadziała.

-- Tak, to brzmi jak pomoc techniczna w~sieci. -- Popiła herbatą. Salima
spróbowała sody selerowej, przygotowana do dyskretnego wyplucia jej
przez słomkę, ale nie była taka straszna. Wye miała nieobecne spojrzenie
i powiedziała: 

-- Daj mi sekundę. -- Wyjęła ekran z~kieszeni i~przez
chwilę w~niego stukała, przesyłając obraz na siatkówkę po kilku
sekundach, potem stukając jeszcze trochę. -- Myślę, że mam pomysł.

-- Pomożesz nam?

Prychnęła. 

-- Dlaczego miałabym Ci nie pomóc?

-- Mogłabyś stracić pracę.

Wye wzruszyła ramionami. 

-- Tak czy inaczej, i~tak bym tam nie
przetrwała, tyle Ci mogę powiedzieć. Są inne prace.

Salima nigdy nie była zwolniona w~życiu. Nie mogła sobie tego wyobrazić.
Kiedy sprawy w~pracy były ciężkie, pracowała dwa razy ciężej. Czuła
wdzięczność do Wye, która ryzykowała jej pracę dla niej, ale też
pełznący wąż dezaprobaty wobec tego lekkiego nastawienia Wye wobec
kariery zawodowej. Czy ona w~ogóle rozumiała, jakie miała szczęście?
Może gdyby rozumiała, nie byłaby taka prędka do pomocy. Salima zachowała
ostrożnie niewinną minę.

-- A co z~aresztowaniem? To wyrok pięciu lat za łamanie zabezpieczeń.

-- Jeżeli w~celach handlowych. Jeżeli nie pobieram opłat, wszystko, co
mogą zrobić, to mnie pozwać. Zatrę moje ślady\ldots 

-- W~każdym razie, mam pomysł. Czy wiesz, co to jest maszyna wirtualna?

Salima nie wiedziała, ale mogła powiedzieć, że gdyby powiedziała po
prostu nie, wtedy Wye wyjaśniłaby coś jej i~to poprowadziłoby Wye do
zrobienia czegoś, żeby jej pomóc i~czy to nie było to, na co miała
nadzieję? Ale nie mogła po prostu się na to zgodzić.

-- Proszę, przestań -- powiedziała Salima. -- To był błąd. Nie mogę Cię
prosić, żebyś dla nas ryzykowała. Jest\ldots  -- Mrugnęła mocno na łzy, które
pojawiły się w~jej oczach. -- Dzieciaki w~moim budynku, pokazałam im, co
robiłam i~one się za to wzięły, a~teraz wszystkie mogą zostać ukarane,
razem z~rodzinami, ponieważ ja nie byłam ostrożna, komu pozwalałam na
moje ryzyko. To zła sytuacja, ale ,,Zatrę moje ślady'' nie wystarczy do
ochronienia Ciebie. Wszyscy, którzy kiedykolwiek wpadli w~kłopoty,
myśleli, że ukryli swoje tropy. \textit{Ja} myślałam, że zatarłam swoje
ślady. Przepraszam, Wye, naprawdę nie powinnam dzwonić do Ciebie.

Wzięła kurtkę i~zaczęła wstawać. Wye położyła dłoń na jej ramieniu,
ścisnęła, zatrzymując ją. 

-- Proszę, Salimo, nie wychodź. \textit{Umiem}
zatrzeć moje ślady. To nie będzie trudne. Przynajmniej pomogę Ci
wymyślić sposób, żeby odzyskać wszystko tak, żebyś nie wpadła w~kłopoty.
Jestem dorosła i~potrafię zdecydować, jakie podejmuję ryzyka.

W każdej chwili mogły popłynąć łzy. Salima delikatnie uwolniła swoje
ramię. 

-- Dziękuję, Wyoming. Zadzwonię, kiedy to wszystko ułożę i~będziemy mogły pójść na drinka dla uczczenia, dobrze?

Nie dała jej szansy odpowiedzieć. Łzy napłynęły, gdy wyszła na ulicę i~żałowała, że nie zabrała serwetki przy wyjściu z~jadalni. Nigdy nie
płakała, nigdy. W~każdym razie nie od czasu śmierci rodziców. Dlaczego
zatem teraz płakała?

Prawie zasypiała, kiedy odpowiedziała na swoje pytanie: płakała,
ponieważ miała coś do stracenia, po raz pierwszy od czasu utraty
rodziców. To była przerażająca świadomość, jakby została zdradzona przez
własne szczęście. Wszystko, co osiągnęła, było czymś, co mogła stracić.

Jeżeli tylko nie próbowała mieć własnego życia, nie miałaby niczego do
stracenia. Gdyby się nie zaprzyjaźniła, nie miałaby przyjaciół do
zdradzenia przez swoją brawurę w~pragnieniu wiedzy. Gdyby nie wygrzewała
się w~zachwycie tych dzieciaków, dzieci nie sprowadziłyby
niebezpieczeństwa na siebie i~swoich rodziców.

Już prawie nie spała. Wróciła do ekranu, zakopała się w~darkwebie,
zdobyła poradniki jak przywrócić wszystko do stanu, w~jakim było.
Przećwiczyła je we własnej kuchni, upewniając się, że potrafi. Kiedy je
opanowała, jej wirujący mózg w~końcu zdecydował się pozwolić jej zasnąć,
ale słońce już wschodziło. Ustawiła alarm na za godzinę i~kiedy się
uruchomił, zrobiła cztery kapsułki wyschniętej, prawie nie do picia
kawy, które znalazła z~tyłu szafki i~wypiła je jako pokutę. Cierpki,
kwaśny płyn sparzył jej język i~zwijał jelita przez całą drogą do jej
pierwszej pracy tego dnia.

Kiedy wróciła wieczorem, w~holu czekał tłum dzieci i~ich dorosłych jak
zawsze. Przywitali ją, a~ona pośpiesznie ich minęła i~wspięła się na
schody, niewyspanie obciążające jej nogi, pot spływający po karku i~plecach, aż do twarzy i~w jej oczy. Zmęczenie było jak ciężar na jej
plecach, gdy zatoczyła się w~mieszkaniu, opadając na sofę i~pozwalając
oczom zamknąć, tylko na moment, który zamienił się w~godzinę, gdy
zasnęła, tylko żeby się obudzić w~poczuciu winy. Miała pracę do
zrobienia.

Weszła na górę do mieszkania Nadify. Po trzydziestu pięciu piętrach,
które wspięła w~drodze do domu, kolejne sześć powinno być łatwe, ale jej
nogi i~tyłek bolały i~były zmęczone, musiała się wciągać po poręczy.
Uświadomiła sobie, że zapomniała zjeść kolacji, a~potem uświadomiła
sobie, że nie mogła sobie przypomnieć, czy jadła lunch. Minęło dużo
czasu, od kiedy pominęła znaczącą ilość snu. Wyszła ze wprawy.

Nadifa tylko spojrzała na nią, poprowadziła do mieszkania i~wepchnęła
jej miętową herbatę i~małe ciastka. Przysięgała, że nie ma apetytu, ale
nie mogła przestać ich jeść. Abdirahim formalnie odrabiał pracę domową,
ale Salima potrafiła stwierdzić, że płonął z~ciekawości i~podsłuchiwania, zatem po drugiej szklance herbaty, zawołała, żeby do
nich dołączyć. Odwrócił się od ekranu na fotelu i~podniósł go do
wysokości stołu.

-- Abdirahim, podjęłam decyzję i~raczej Ci się nie spodoba.

Umiał zachować minę pokerzysty. Dzieciaki, które dorastały w~obozach,
były dobre w~kontrolowaniu informacji, które wysyłały do osób dookoła.
Mimo tego potrafiła powiedzieć, że wiedział, co nadchodzi i~że tego nie
lubi.

-- Kiedy zaczęłam łamać zabezpieczenia, nie wiedziałam, co robię. Nie
rozumiałam niebezpieczeństw. Ale teraz rozumiem, dzięki Tobie i~własnym
lekturom. I~Abdirahim, ryzyko jest po prostu zbyt duże. Nie ma sposobu,
żeby wiedzieć na pewno, co korporacje zrobią, żeby nas złapać, a~jeżeli
złapią, możemy stracić wszystko. Nawet gdybyśmy mogli oszukać firmy,
kierownictwo budynku zauważy, kiedy nie dostanie swojej części
pieniędzy, teraz gdy firmy znowu zaczęły działać. Ludzie, którym Twoi
przyjaciele pomogli, nie zdawali sobie spraw z~tego, na co się pisali,
żadne z~nas nie było świadome, jesteśmy odpowiedzialni za nich.

Jego pokerowa twarz się zmieniała. Jego dolna warga drżała. Jego kostki
były białe, tam gdzie złapał krawędź stołu. Nadifa spojrzała się na
niego ostrzegawczo. Serce Salimy łamało się nad nim. Po wszystkim, co
przeszedł, znalazł sposób na decydowanie w~świecie, który nigdy nie dał
mu najmniejszego prawa do decydowania, a~teraz ona kazała mu to wszystko
odwrócić. Chciała płakać i~mogła tylko podziwiać jego samokontrolę.
Położyła dłoń na ramieniu Nadify.

-- Przepraszam, Abdirahim. Masz prawo się rozgniewać. To nie jest
właściwe, ale jest konieczne. To jedna z~najtrudniejszych sytuacji. -- Wzięła głęboki wdech. -- Nie mogę tego zrobić bez Ciebie. Nie staram się,
żebyś poczuł się lepiej. Ty jesteś tym, który nauczył inne dzieciaki i~tylko one wiedzą, w~których mieszkaniach były, a~w~których nie były. Bez
Twojej pomocy, nie udałoby mi się zmusić innych dzieci nawet do
posłuchania.

-- Abdirahim, pomożesz?

Widziała, że Nadifa chciała kazać mu się zgodzić i~lekko, ale stanowczo
ścisnęła ramię Nadify. \textit{Niech sam zdecyduje}.

Patrzył na dłonie przez długi, długi czas. Jego oddech był nierówny.
Zastanawiała się, czy potem będzie płakał. Ale wtedy podniósł głowę i~mrugnął mokrymi oczami do nich. 

-- Zrobię to, ciociu.

Wiedziała, że poświęcił czas, żeby naprawdę to przemyśleć i~dojść do
właściwych wniosków po tym dokładnym namyśle, a~nie ponieważ dorosły mu
kazał, nawet nie dlatego, że jego matka nigdy by przyjęła jego odmowy.

-- Wiedziałam, że pomógłbyś, Abdirahim. -- Ścisnęła ramię Nadify raz
jeszcze. -- Powinnaś być bardzo dumna z~tego tutaj.

-- Jestem -- powiedziała. Poklepała dłoń syna. Nadifa też widziała, co to
go kosztowało.

Nagle, Salima była tak bardzo zmęczona. Była zmęczona wcześniej, ale
teraz było zupełnie nowa wysokość, a~może głębia. Przez chwilę,
dosłownie nie mogła powstrzymać oczu od zamknięcia. Walczyła, żeby
pozostawić je otwarte. Nadifa patrzyła na nią z~troską.

-- Zabierzemy Cię na dół.

Oboje wsunęli ramiona pod jej pachy, Abdirahim już prawie tak wysoki jak
jego matka, i~zataczając się, poprowadzili ją do windy i~nacisnęli
przycisk. Czas mijał. Wydawało się, że minęła godzina, zanim pojawiła
się kabina i~z sykiem się otworzyła, pachniało w~niej drogimi perfumami
kogoś z~równoległego kosmosu pięter ,,nie dla biednych'', kto zakończył
jazdę i~zwolnił windę dla przejażdżki dla takich jak oni.

Nadifa posłała Abdirahima z~powrotem schodami, gdy byli w~mieszkaniu
Salimy, potem pomogła jej przebrać się w~nocną koszulę i~zmieniła układ
z sofy na łóżko, skręcając stolik w~dół i~opuszczając jego boki, że stał
się stolikiem przy łóżku. Wyciągnęła koc i~ułożyła go na Salimie, a~Salima, na wpół świadoma, przypomniała sobie dawne wspomnienie, czasu
przed czasem, kiedy leżała w~kołysce i~matka ją otulała. Wspomnienie
było tak słodkie, bez tego żalu, który zwykle pojawiał się we
wspomnieniach jej matki, Selima zanurzyła się w~sen z~uśmiechem, który
ciągle był na jej ustach, kiedy wstała kilka godzin później na siku i~umycie zębów, żeby pozbyć się zepsutego smaku, zanim znowu zapadnie w~sen.

Poczucie celu jest cudownym pokrzepieniem na lęk. Teraz gdy Salima
wiedziała, co robić, bezradne martwienie się zostało zastąpione przez
nieskończoną energię. Po wczesnym śniadaniu, zadzwoniła do Nadify, żeby
potwierdzić, że Abdirahim już nie śpi, i~wspięła się po schodach, dwa
schodki naraz. Pokazała mu jak przywrócić ustawienia fabryczne na
wszystkim w~mieszkaniu. Tak jak to zrobiła we własnym miejscu,
przeprowadziła go przez procedurę dwa razy, żeby upewnić się, że
zrozumiał, a~potem kazała mu wypisać czynności z~pamięci na kartce.
Szybko się nauczył, tak jak myślała.

-- Teraz musisz rozpuścić wieści. Czy możesz powiedzieć dzieciom, żeby
przyszły do mojego miejsca dzisiaj wieczorem, po szkole, ale przed
kolacją, powiedzmy osiemnasta?

Nie był szczęśliwy z~tego powodu. 

-- Znienawidzą to.

-- Wiem. Ja tego nienawidzę. To jak poddanie się. Jednak poddanie się
jest mądrzejsze niż bitwa, której nie możesz wygrać. To jest tak ważna
lekcja jak każda inna, wiesz.

Nadifa skinęła głową. 

-- Są znacznie trudniejsze sposoby, żeby tego się
nauczyć. -- Nadifa patrzyła nieobecnym wzrokiem.

Abdirahim wyglądał nieszczęśliwie.

-- Wiem, że to jest bardzo trudne -- powiedziała Salima. -- Byłeś
bohaterem, kiedy uczyłeś przyjaciół, teraz będziesz tym przestraszonym
chłopcem, który każe im to oddać. Wezmę winę na siebie. Wyjaśnię im. Po
prostu sprowadź ich do mojego mieszkania, dobrze?

Nocny sen zrobił jej tyle dobrego. Jej dzień minął tak gładko, że równie
dobrze mógł być posmarowany. Znalazła pewne drobne systematyczne błędy w~księgach pralni, które wyjaśniają, dlaczego jego zyski konsekwentnie
spadały, a~właściciel wyznał jej, że miał zamiar zwolnić jedynego
pracownika za kradzież i~tak mu ulżyło, że ją przytulił. Miała miejsce
siedzące w~pociągu w~obie strony. Wiosna przestała wreszcie przeskakiwać
pomiędzy mrozami a~deszczami i~zatrzymała się na słonecznej, wietrznej
medianie, która pędziła chmurami nad głowami jak na wygaszaczu ekranu, a~nowe pączki na drzewach wydawały się wybuchać przez noc.

Kupiła znacznie większą torbę przekąsek na wieczorne spotkanie,
pamiętając o~niedosycie na poprzednim spotkaniu. Musiała nabyć chleb w~alejce Boulangism w~sklepie na rogu, w~sąsiednich sekcjach kupiła
kapsułki do kawy i~płyn do zmywarki.

To zepsuło jej nastrój, zamieniając ostatnie sto metrów w~wolny marsz.
Myślała o~pierwszym spotkaniu z~Wye, o~przerażeniu Wye, że ktoś używałby
w ogóle Boulangismu. Salima dobrze teraz zarabiała, nawet odkładała
pieniądze co miesiąc, oszczędności, które narosły przez miesiące, kiedy
mogła wybierać produkty z~całego sklepu. Były inne miejsca do życia,
które nie były Wieżowcami Dorchester. Miejsca, gdzie mogła wybrać,
jakich urządzeń AGD używała. Kosztowały więcej, tak wiele ogłoszeń o~wynajmie zawierało drobnym drukiem informacje pouczające ewentualnych
najemców, że ich umowa zabraniała zmieniania urządzeń zarabiających na
dochód właściciela. Niemniej jednak istniały. Z~jednym lub dwoma
współlokatorami, mogła sobie na nie pozwolić.

Jednak Wieżowce Dorchester nie były miejscem gdzie żyła, były małą
społecznością, miejscem, gdzie pasowała, gdzie miała przyjaciół i~ludzi,
którzy byli czymś jak rodzina, jak Nadifa i~dzieciaki, które wołały na
nią ,,ciociu''. Ludzi, którzy rozumieli, przez co przeszła. Wyobraź
sobie życie w~domu pełnym Wye, dziewczyn, które wydawały się tak młode,
z przepaścią nie do pokonania pomiędzy ich doświadczeniami a~jej
własnymi.

Dzieciaki w~holu windy pytały ją podniecone spotkaniem tego wieczoru i~zrozumiała, że Abdirahim nie powiedział nawet słowem o~treści. Nie mogła
go winić.

Stłoczyły się i~wyjadły wszystkie przekąski. Może nie było wystarczająco
jedzenia na świecie, żeby wypełnić te brzuszki.

-- To nie będzie łatwe do wysłuchania. -- Szepty, uśmiechy i~wiercenia się
natychmiast się skończyły i~wszystkie oczy spoczęły na niej. Tyle z~jej
doskonałego dnia.

-- Od ostatniego naszego spotkania, dowiedziałam się czegoś. Ważnego. -- Opowiedziała im, o~prawie, o~wyrokach więzieniach, o~nowych firmach,
które powstały z~pozostałości starych i~zespołach wynajętych do złapania
takich oszustów jak oni. Niebezpieczeństwo dla nich i~dla ich rodzin.
Eksmisja i~gorzej. Obserwowała, jak ich twarze stają się poważne i~jeszcze poważniejsze.

Pyzata dziewczyna, ta bystra, która ostatnim razem też się pytała, była
pierwsza do mówienia, kiedy Salima skończyła. 

-- Jak to rozwiążemy?

To była straszna chwila. Zmartwione twarze rozjaśniły się i~uwaga całego
pokoju skupiła się intensywnie na niej. Te dzieci były wystarczająco
sprytne, żeby zrozumieć ryzyko, ale nie dostatecznie bystre, żeby
zrozumieć, że nic nie mogła zrobić, żeby to naprawić.

-- Nie możemy. Musimy ustawić wszystko, tak jak było. Cofnąć wszystko.
Ustawienia fabryczne wszędzie. -- Zanim ktokolwiek mógł powiedzieć
cokolwiek, dodała: 

-- To koniec.

Twarze dzieci powiedziały wszystko. Szok, niedowierzanie, potem bunt.
Mamrotanie. Słowo \textit{nie}, po cichu, potem głośniej, potem
przebiegające od dziecka do dziecka.

-- Tak! -- krzyknęła, podnosząc ręce. -- Przepraszam\ldots  tak mi przykro\ldots 
ale \textit{tak}. Musimy to zrobić. To był błąd. -- Podniosła ręce wyżej.
Było coraz głośniej. -- Mówię poważnie. Znajdziemy coś innego. Możecie mi
pomóc. Ale najpierw musimy to zrobić.

Niektóre dzieciaki wychodziły. Zobaczyła, że Abdirahim kręci głową.

-- Tak musi być. -- Więcej wyszło.

-- Proszę.

Abdirahim był ostatni.

-- Przepraszam. -- Było wszystkim, co powiedział, zanim on, też, wyszedł.

Oczywiście nie mogła się poddać. Zaczęła od ludzi, którym pomagała,
zaczynając od wyższych pięter w~dół, robiąc jedno, dwa mieszkania przez
noc. Abdirahim czasem przychodził i~pomagał, ale wyraźnie był nastawiony
negatywnie do roboty, robił błędy, które często opóźniały jej pracę mimo
jego pomocy. Nie sądziła, że robi to specjalnie, ale też z~pewnością nie
były to zwykłe wypadki.

Wiele razy kusiło ją, żeby wysłać smsa do Wye i~zaprosić na spotkanie,
żeby wyciągnąć informacje o~stanie Boulangismu i~ile czasu miała, zanim
znowu uruchomią usługi. Domyślała się, że właściciele budynku nie
oczekiwaliby całych dochodów od razu po uruchomieniu znowu
minipiekarników -- chwilę zajęłoby ludziom zauważenie, że ich AGD znowu
działa i~wycofanie się z~już ustalonego stanu rzeczy, który powstał w~trakcie przestoju -- ale nastąpiłby czas, kiedy stałoby się oczywiste, że
we wszystkim ktoś grzebał, chyba że wycofałaby zmiany.

Pracując w~mieszkaniach sąsiadów, widywała dzieciaki, dzieciaki, które
obserwowały łamanie zabezpieczeń w~ich mieszkaniach i~robiły to samo dla
ich sąsiadów. Dzieciaki, które wyszły ze spotkania, kiedy im
powiedziała, że muszą wrócić do stanu, w~jakim wszystko było na
początku. Dzieciaki, które obserwowały ją kątem oka, udając, że
odrabiają lekcje. Poświęcała dodatkowy czas, opisując rodzicom
konsekwencje przyłapania głośno i~bardzo szczegółowo.

Zabrało jej pięć nocy i~cały weekend, żeby przerobić mieszkania, które
samodzielnie zhakowała. W~poniedziałkowy wieczór, wróciła do domu,
wrzuciła poniżający, paczkowany, autoryzowany obiad w~mikrofalę i~ruszyła na górę pięter dla biednych, siedem pięter wyżej. Czekała krótko
na windę, potem przyznała, że jedynym powodem wybrania windy, było to,
że wiedziała, że to okres ruchu i~mogła zabić pół godziny aż do
czterdziestu pięciu minut, czekając na kabinę, odkładając to, co
nastąpiłoby następnie.

Weszła schodami i~zastukała w~pierwsze drzwi koło klatki schodowej po
dokładnym zanotowaniu numeru w~małym notatniku, który przyniosła ze
sobą, ładnym pismem księgowej.

Kobieta, która odpowiedziała, była trochę znajoma, ktoś, z~kim jechała
windą raz lub dwa, Salwadorka czy Honduranka, jak myślała. 

-- Tak? -- Była
nieco starsza niż Salima.

-- Cześć. -- Przećwiczyła to, ale w~ustach jej zaschło i~nie mogła
powiedzieć słowa. -- Cześć. Mieszkam w~tym budynku i\ldots  -- To było błędem. -- Czy ktoś z~dzieci w~budynku pomógł Ci z~urządzeniami w~kuchni? Lub
może termostatem?

Kobieta wyglądała podejrzanie. To nie tak powinno iść. 

-- Nie sądzę.

Policzki i~czubki uszu Salimy płonęły. 

-- Przepraszam, szczerze, za
niepokojenie. Ale, chodzi o~to, dzień, w~którym tostery przestały
działać? Wymyśliłam jak zmusić mój do działania. Potem pokazałam to
niektórym dzieciom, one poszły dalej i~zmieniły to dla wszystkich
innych. Jednak wtedy dowiedziałam się, że producenci potrafią
powiedzieć, kto tak zrobił i~mogą po nas przyjść. Właściciele mieszkań
również, mają udział w~pieniądzach, które wydajemy. Zatem przyszłam
cofnąć zmiany, zanim wpadniesz w~kłopoty i~zanim wszyscy będziemy mieć
kłopoty. -- Uśmiechnęła się swoim najlepszym, najbardziej solidnym
uśmiechem.

Kobieta pokręciła głową. 

-- Nigdy nie wpuściłam tutaj żadnych dzieci.

Salima była pewna, że kłamie, szybkość odpowiedzi, sposób, w~jaki
rozejrzała się dookoła, kiedy to mówiła. 

-- Dobra, jeżeli ja tego nie
zrobię, \textit{złapią} Cię. Mogłabyś stracić dom. Gorzej, mogliby posłać
dziecko, które to zrobiło, do więzienia. -- To nie było do końca prawdą,
na podstawie tego, co Wye jej powiedziała, ale było prawie prawdziwe.
Dziecko mogło wpaść w~kłopoty, była tego pewna, i~oczywiście, także i~ona. -- Proszę.

-- Powiedziałam, żadnych dzieci.

-- Czy mogę zatem spojrzeć? -- Kobieta wyglądała na zdenerwowaną. -- Znaczy, może zapomniałaś. Mogę sprawdzić, tylko żeby się upewnić?

-- Muszę iść. -- Drzwi kliknęły przy zamknięciu, zanim mogła odpowiedzieć
choć słowem. Była świadoma kamery, zatem utrzymała neutralną minę, gdy
robiła notatkę w~notatniku i~potem wzięła głęboki oddech i~poszła do
mieszkania obok.

Zapowiadała się długa noc.

Praca stała się rodzajem snu albo koszmaru na jawie, z~którego wracała
ciągle do prawdziwej pracy, biegania po schodach budynku, pukania do
drzwi, błagania obcych ludzi, żeby ją wpuścili i~pogorszyli sobie życie.
Słowo się rozniosło, napotykała zaciekawione spojrzenia i~czasem wrogość
w windach, a~Abdirahim już nawet nie udawał, że jej pomaga. Nie mogła
się poskarżyć Nadifie, ponieważ nie chciała spowiadać się z~wszystkich
grzechów ostatniej przyjaciółce, jaką miała.

Tak czy inaczej, stała się lepsza w~przekonywaniu w~drzwiach i~prawie
wszyscy wpuszczali ją do domu, żeby zrobiła swoje, w~czym również stała
się lepsza, szybko i~sprawnie przerabiając urządzenia.

W prywatnych chwilach w~nocy, poszukując nieuchwytnego snu, przyznawała
się przed samą sobą, że niektóre dzieci prawdopodobnie chodziły i~odwracały to, co zrobiła, przy współudziale dorosłych, którzy powinni
wiedzieć lepiej.

~

Za każdym razem, gdy jechała do domu z~pralni, wyglądała za Wye w~pociągu, niepewna, czy przy spotkaniu byłaby szczęśliwa, czy
przestraszona. Nigdy jej nie spotkała, ale pewnego popołudnia, gdy
próbowała znaleźć własne błędy w~pracy zamknięcia kwartału dla
lodziarni, liczby pływające przed jej oczami, jej telefon piknął.

\textbf{ Muszę z~Tobą porozmawia, Wye}

Otworzyła mały notatnik i~spojrzała na numery mieszkań. Przeszła przez
prawie trzy czwarte miejsc i~w większości z~nich pozwolili jej pracować.
Niektóre zostałyby znowu przerobione przez dzieci, ale może to było
wystarczająco. Może Wye znalazła trik, który pozwoliłby im w~końcu
zachować ich hakowania, coś niezawodnego.

(Ale tylko głupcy wierzyli w~niezawodne triki).

\textbf{ Kończę o~17. Jestem na Mass Ave, koło Harvard
Square}

\textbf{ Możemy się spotkać 1715 przy dziale na Cambridge
Common?}

\textbf{ ok}

Niemądra nadzieja narosła w~Salimie pomimo jej najlepszych wysiłków. Nie
zapomniała podniecenia, z~jakim Wye mówiła, kiedy zaczęła myśleć, jak
pomóc im pokonać Boulangism, absurdalna pewność, którą promieniowała, że
może przechytrzyć cały przemysł. Może mogłaby, choć oczywiście, nie
mogłaby.

Lato było w~pełni, było gorąco i~lepko, z~kilkoma studentami, którzy
zostali tak późno w~roku. Wachlowała się składanym chińskim wachlarzem,
który kupiła na ulicy kilka dni temu, kiedy pojawił się upał i~wzrosła
wilgotność. Myślała, że w~Arizonie było gorąco, ale ta wilgotność
próbowała ją zadusić od środka, uczucie, które przywoływało mroczne i~ukryte wspomnienia z~przekraczania Morza Śródziemnego, kiedy była małym
dzieckiem, zmysłowe wspomnienia pragnienia, mdłości i~smrodu.

Wachlowała się i~rozglądała, ale nie zauważyła Wye, póki nie była tuż
koło niej, ponieważ Wye przycięła swoje jasnoblond włosy rozsądnie
krótko na ten gorąc i~zafarbowała je nieco na różowo. Była też chudsza
niż ostatnim razem, gdy się widziały, i~bledsza. Długie godziny,
pomyślała Salima.

-- Przyszłaś -- powiedziała Wye.

-- Cześć -- powiedziała Salima. -- Przyszłam. -- Ciągle była zażenowana ich
ostatnim spotkaniem. -- Przepraszam za wcześniej. To było bardzo miłe z~Twojej strony, ale \ldots 

-- Ta. Wiem, ryzyko. Rozumiem. W~pewnym sensie. Znaczy, wiem, że naprawdę
mogę nie rozumieć, przez co przeszłaś, ale\ldots  -- Wytarła pot z~czoła
ramieniem. -- Wiesz, rozumiem. Też przepraszam I~nie martw się. -- Była
naprawdę słodka.

-- Odwracałam wszystko, wszystko resetowałam do ustawień fabrycznych.
Nikt mi jednak nie chce pomóc. Za to dzieciaki mnie nienawidzą.

-- To do chrzanu.

-- Prawda.

-- Słuchaj, chciałam się z~Tobą spotkać, ponieważ coś dzieje się w~Boulangism, o~czym myślałam, że byłabyś zainteresowana, ale nie możesz o~tym nikomu powiedzieć, ponieważ nie powinnam mówić nikomu bez podpisania
klauzuli o~poufności. Czy to jest ok? Znaczy, mogę Ci powiedzieć i~zachowasz to w~tajemnicy?

Salima skinęła głową. 

-- Oczywiście.

-- Zatem, jesteśmy prawie gotowi do uruchomienia i~teraz nowi właściciele
wykupili jeszcze dwóch konkurentów i~złożyli nas razem w~jednej
platformie. Teraz jesteśmy \textit{znacznie }więksi, oni mają te plany, na
przykład, żeby pozwolić kupić ludziom złamanie blokady na dzień lub
tydzień, żeby mogli ugotować to, co chcą. Obserwowali fora dyskusyjne
darknetu, wiedzą, że wszyscy pracowali nad tym, jak zhakować ich
urządzenia, podczas gdy my ponownie się organizowaliśmy, i~wymyślili, że
ci ludzie mogą być klientami, ale zamiast płacić za jedzenie, które im
sprzedajemy, płaciliby nam za wykorzystanie jedzenia, które ktoś inny im
sprzedał.

Salima prawie się roześmiała. To było przestępstwo, jeżeli ona to
robiła, produkt, jeżeli jej sprzedawali. Wszystko mogło być towarem.

-- Wiem, że to dziwne. Ale tutaj pojawiasz się Ty. Oni mają oddział
badawczy, antropologów, specjalistów od danych i~marketingowców, chcą
porozmawiać z~ludźmi takimi jak Ty, dowiedzieć się, ile byś zapłaciła za
różnego rodzaju produkty. Chcą się dowiedzieć, czy sprzedawałabyś takie
pakiety sąsiadom, jeżeli dostałabyś część pieniędzy od nich, jak
prowizja? Mają taki plan, mogłabyś nauczyć te dzieci, z~którymi
pracujesz do sprzedawania płatnego odblokowywania ludziom w~budynku, a~one dostałyby prowizję i~Ty dostałabyś prowizję, ponieważ ich
zrekrutowałaś.

-- To piramida finansowa?

-- To program partnerski. Dzieci nie mogłyby rekrutować ludzi do pracy
dla nich, wybralibyśmy skrupulatnie rekruterów i~tylko oni mogliby
otrzymywać podwójne prowizje. Liderzy. To ciągle tylko idea, ale kiedy o~niej usłyszałam, natychmiast pomyślałam o~Tobie. Mam na myśli, to
rozwiązuje wszystkie Twoje problemy, prawda? Twoje dzieciaki stają się
legalne, nawet zarabiają kieszonkowe. A Ty wykorzystujesz umiejętności i~masz szacunek sąsiadów, że im pomagasz i~zarabiasz nieco pieniędzy dla
siebie. Och, i~oczywiście, mogłabyś na zawsze odblokowywać cokolwiek, co
wyprodukujemy, żebyś mogła przedstawić to sąsiadom. Tak jak mówiłam, to
nie jest pewna sprawa, ale pomyślałam, że mogłabyś przyjść i~poznać
zespół, przegadać, zadzwonić\ldots  -- Przerwała, poszukując w~twarzy Salimy
wskazówki o~jej reakcji. Salima starannie utrzymywała neutralną minę.

-- Wye -- powiedziała. -- To bardzo miłe z~Twojej strony, że pomyślałaś o~mnie. Naprawdę.

-- Ale?

Salima osunęła się. 

-- Nie wiem. Jest \textit{ale}, ale nie potrafię
określić dokładnie jakie.

-- To dziwna idea -- powiedziała Wye. -- Wiem. Ale może mogłabyś o~tym
pomyśleć? Nie potrzebuję odpowiedzi już teraz.

Salima chciała powiedzieć ,,nie'', ale nie zrobiła tego. Nawet gdy coś w~środku niej wzdrygało na ofertę, inna część jej rozumiała, że Wye mogła
mieć rację, to mogła być najlepsza opcja.

-- Jak sądzisz, ile mamy czasu?

-- Póki będziesz musiała się zdecydować?

Salima myślała o~Wye jako o~sojuszniczce, każda część obrażona przez
zamknięty świat Wieżowców Dorchester tak jak ona. Ale Wye pracowała
godzinami dla Boulangism i~jej nowych siostrzanych firm. Myślała, że
problemem było to, że Salima nie chciała mieć kłopotów. Salima też tak
kiedyś myślała. Ale to nie był problem. Boulangism jako taki, to był
problemem. Ten cały zepsuty biznes, to był problem.

-- Póki Boulangism nie odkryje, co zrobiliśmy i~nie spowoduje naszej
eksmisji.

Wye pokręciła głową. 

-- Czy nie słuchałaś mnie? To się nie zdarzy,
próbują obecnie marchewek, a~nie kijów. Chcą traktować ludzi jak
klientów, a~nie jak oszustów.

-- Ok, jasne. Jednak ile czasu minie, zanim do tego dojdzie?

Wye wyglądała na zranioną. 

-- Nie wiem. Jednak wkrótce. Tydzień lub dwa.
Naprawdę byli podekscytowani sprzedażą za zdejmowanie blokad, ale
chcieli zrobić małe badanie dotyczące poziomu cen, zanim to przedstawią,
więc to może trochę opóźnić sprawy. Jednak właściciele nie zamierzają
płacić wszystkim wypłat bez żadnych wpływów.

-- Kilka tygodni. -- Mogła przejść przez mieszkania, które ominęła, potem
znowu zacząć od najwyższego piętra i~znowu przerobić drogę w~dół,
przemawiając do ludzi, że było to istotne, żeby jej powiedzieli, jeżeli
pozwolili dzieciom bawić się ich sprzętem.

-- Ta. Słuchaj, Salima, myślę, że naprawdę powinnaś o~tym pomyśleć. To
dobry plan, dobry dla wszystkich.

-- Pomyślę. -- Jej głos brzmiał nieprzekonująco, nawet dla jej własnych
uszu.

Wye wyglądała na jeszcze bardziej strapioną. Salima poczuła się źle. Wye
chciała tylko pomóc.

-- Salimo, jadłaś już obiad? Jestem szalenie głodna. Lubisz ryby?
Niedaleko stąd jest miejsce z~rybami, które jest niesamowite, rodzaj
miejsca, gdzie odwiedzający rodzice zabierają dzieci na dobry posiłek w~trakcie roku szkolnego. Odkąd zaczęłam dostawać wypłatę, chciałam
wrócić, ale byłam zbyt zajęta. Poszłybyśmy razem? Zapłacę. Nie chcę jeść
samej.

Nie, chciała powiedzieć. Nie, muszę iść do domu i~wrócić do pracy. Nie,
nie stać mnie na Twoją łaskę, ponieważ może musiałabyś zeznawać
przeciwko mnie. Nie, nie muszę się przyjaźnić z~kimkolwiek z~Twojego
świata.

-- Tak -- powiedziała. -- To byłoby bardzo miłe. -- Była zbyt głodna, żeby
powiedzieć nie, i~miała dość drogich, paczkowanych posiłków do
mikrofali.

Wróciła do domu po dwudziestej drugiej, o~wiele za późno, żeby dzwonić
do drzwi kogokolwiek i~prowadzić dziwne konwersacje o~ich AGD. Winda
była cudownie wrażliwa, co było dobre, od kiedy pomiędzy winem a~stresem
ostatnich tygodni nie była w~stanie wspiąć się po schodach lub nawet
przytomnie stać długo w~holu.

Winda pachniała drogim produktem do włosów, co powiedziało jej, że
została ostatnio opuszczona przez kogoś, kto wyszedł przez drugie drzwi
w drodze na miłą wieczorną randkę, lub może w~drodze z~powrotem,
zwalniając niańkę i~przygotowując wieczorną przekąskę w~piekarniku,
który piekł to, co mu kazano.

Zapach utrzymywał się w~jej nozdrzach, cytryny i~tytoń, gdy dotykała
dłonią ściany korytarza w~drodze do mieszkania. Właśnie miała otworzyć
drzwi, gdy zobaczyła słowa nabazgrane grubym, markerem permanentnym:
ODJEB SIĘ. Litery były duże, gniewne i~jednak niepewne, jakby były
napisane przez dziecko, lub może przez kogoś, kto właśnie się uczył
angielskiego.

Była tak zmęczona.

Polizała palec i~potarła atrament. Nawet się nie rozmazał. Weszła do
środka i~sprawdziła zapisy kamery przy drzwiach, odkryła, że były
czyste, doskonale usunięte. Zatem, może to był dzieciak, który to
napisał, dzieciak, który nauczył się szukać w~darknecie sposobów
zapanowania nad technologią zaprojektowaną do panowania nad nim.
Dzieciak, który był oburzony jej prośbą o~zapomnienie jak to zrobić i~powrotem do bycia potulnie kontrolowanym.

Czy taki dzieciak w~ogóle pracowałby na prowizji, instalując oficjalne
kody odblokowujące w~mieszkaniach dla biednych w~Wieżowcach Dorchester?
Czy było wystarczająco dużo pieniędzy na świecie?

A jeżeli były, czy chciała być tą, która używała tych pieniędzy, żeby
przekonać dziecko do zrezygnowania z~tej bezkompromisowej zaciekłości?

Następnego dnia kupiła rozpuszczalnik w~drodze do domu z~pracy.

Wpadła na Nadifę na schodach kolejnego ranka, walczącą z~wózkiem,
popędzającą Idil, starszą dziewczynkę, niosąc Yasmiin, berbecia. Salima
wzięła wózek i~Idil, zostawiając Nadifę, żeby zarzuciła Yasmiin na
biodro i~podtrzymywała się wolną ręką balustrady. Nadifa westchnęła i~jej podziękowała.

-- Wyglądasz strasznie -- powiedziała Nadifa, trzy półpiętra niżej.

-- Nie śpię zbyt dobrze.

-- Już tak długo nas nie odwiedzałaś. Retsina zaczyna się piętrzyć w~lodówce.

Jej oczy wypełniły się łzami i~mocno mrugała. Tęskniła za Nadifą,
tęskniła za dniami, kiedy była przepełniona podnieceniem z~panowania nad
budynkiem. Akcja przywrócenia wszystkiego do starego stanu wypełniła ją
głębokim i~niewymownym wstydem i~nawet myśl o~pokazaniu się Nadifie
wzbudzała mdłości.

Ale tak dobrze było ją zobaczyć znowu.

-- Przepraszam. Było\ldots  trudno. -- Przełknęła. Potem powiedziała Nadifie o~słowach na drzwiach, używając ostrożnie eufemizmów przed dziećmi.
Powiedziała Nadifie o~kamerach. Przestała, zanim powiedziała jej o~Abdirahimie, porzuceniu jej. Chciała przyjaznego ramienia do oparcia,
nie zemsty na trzynastolatku.

-- To straszne. Przyjdę do Ciebie, kiedy wrócisz po pracy i~razem to
umyjemy.

-- W~porządku, mogę to zrobić sama. -- Pomyślała o~powiedzeniu Nadifie o~ofercie od Wye, ale nie zrobiła tego. Nadifa mogłaby jej powiedzieć,
żeby to zrobiła. Lub powiedzieć jej, żeby nie zrobiła.

Jej ramiona i~plecy były w~ogniu, kiedy dotarły na parter.

-- Dziękuję serdecznie -- powiedziała Nadifa, gdy rozłożyła wózek. -- Normalnie nie próbowałabym wyjść, zanim poranny ruch by się nie
skończył, ale Idil musi zobaczyć dentystę. -- Idil uśmiechnęła się do
niej, pokazując słodkie przerwy w~jej zębach. Nadifa przypięła dziecko
do wózka, a~potem objęła Salimę w~mocnym uścisku. -- Będzie dobrze.
Przeszłaś przez znacznie gorsze rzeczy niż to. Jesteś silna.

Salima była zażenowana swoim pociąganiem nosa, ale alternatywą było
wypłakanie wszystkiego na ramieniu Nadify. Nadifa udawała, że nic nie
zauważyła, tak, że była w~stanie przynajmniej uratować odrobinę
godności.

Tego ranka nie miała miejsca siedzącego w~pociągu i~gdy trzymała się
poręczy, poruszając się do tyłu i~do przodu w~ruchu pociągu, patrzyła
się bezczynnie na reklamę nad jej głową, oczy nieskoncentrowane od
senności i~dopiero gdy wychodziła z~pociągu, zrozumiała, że była to
reklama nowego Boulangism.

Dostała smsa od Wye na schodach.

\textbf{ Program partnerski rusza. Mogę zatrzymać miejsce
dla Ciebie. Wchodzisz w~to?}

Zaznaczyła ją jako nieprzeczytaną, żeby zapamiętać i~odpowiedzieć
później. Za każdym razem, gdy patrzyła na ekran tego dnia w~pracy,
widziała powiadomienie. Trudno jej było się skoncentrować. Zrobiła na
początku dnia błąd i~spędziła godzinę na rozplątywaniu go.

To nie był dobry dzień.

-- Myślę, że muszę porozmawiać z~Abdirahimem.

Nadifa popatrzyła ze zdziwieniem. 

-- To porozmawiaj z~nim.

Salima zawirowała winem w~kieliszku. 

-- Problem jest taki, że nie jest
zbyt chętny rozmawiać ze mną. Jest zły.

-- Adi! Chodź tutaj!

Wyszedł z~pokoju z~nieruchomą miną. 

-- Robię pracę domową.

-- Ciocia Salima chciałaby z~Tobą porozmawiać.

-- Dobrze. -- Powiedziane tonem, który jasno mówił, że wolałby cokolwiek
innego.

Szybko złapał pomysł z~,,opłatą za odblokowanie'', gdy mu opowiadała o~tym, szybciej niż ona. 

-- To jak moje książki w~szkole. Mogę je czytać w~szkole lub domu, ale jeżeli chcę się pouczyć w~parku, muszę zapłacić za
ich odblokowanie.

-- Nie wiedziałam, że działają w~ten sposób.

Wzruszył ramionami. 

-- W~porządku, nie muszę się uczyć w~parku.

Powiedziała mu o~programie partnerskim. 

-- Czyli mógłbyś zarabiać
pieniądze dla rodziny, żeby pomóc w~domu. Twoi przyjaciele też.

-- I~ciocia Salima też zarabiałaby pieniądze -- powiedziała Nadifa. -- Zatem mogłaby oszczędzić na swoje własne miejsce.

Salima spojrzała ostro na Nadifę. 

-- Dlaczego miałaby zostawić Wieżowce
Dorchester?

Nadifa prychnęła. 

-- Kto nie chciałby zostawić, gdyby mógł? Przenieść się
gdzieś z~dobrą windą, z~dobrym AGD? Miejsce, gdzie jesteś chciana?

Nadifa, uwięziona w~mieszkaniu codziennie, póki gorączka w~windzie się
nie skończy, lub zmuszona znieść wózek, niemowlę i~małe dziecko
czterdzieści dwa piętra w~dół. Oczywiście, że chciała odejść. Ale w~jaki
sposób mogłoby to się stać? Była krawcową w~Somalii, ale nie pracowała w~szwalni już prawie dziesięć lat, a~kiedy Yasmiin będzie cały dzień w~szkole, będzie to już prawie dwadzieścia lat. Nawet gdyby znalazła
pracę, pensja krawcowej nie opłaci pełnego czynszu w~Bostonie i~kosztów
życia trojga dzieci.

Salima była ostrożna z~pieniędzmi, ostrożna jak księgowa. Żyła na własną
rękę i~oszczędziła trochę pieniędzy, szczególnie kiedy mogła gotować
jedzenie, które chciała i~kupować składniki zamiast posiłków ,,gotowych
do zjedzenia''. Mogła się wyprowadzić już teraz, jeżeli chciałaby
znaleźć współlokatora, a~jeżeli zdobyłaby jednego lub dwóch klientów w~księgowości, byłaby w~stanie znaleźć mieszkanie na własną rękę w~ciągu
roku. Jednak nigdy nie myślała o~zostawieniu Wieżowców Dorchester. Tu
właśnie należała.

-- Nie odeszłabym. Chcę zobaczyć, jak Twoje dzieci dorastają.

-- Nie bądź śmieszna. Przyjeżdżalibyśmy na wizytę. Gdy tylko będziesz
mogła odejść, powinnaś.

Abdirahim obserwował dwie dorosły kobiety uprzejmie dyskutujące i~Salima
zastanawiała się, ile z~podtekstów rozumiał. Zastanawiała, ile
\textit{ona} rozumiała.

-- To nieuczciwe kazać płacić sąsiadom za używanie ich własnych rzeczy -- powiedział, wtrącając się.

Nadifa miała coś powiedzieć mu o~szanowanie cioci, ale Salima się
wcięła.

-- Tak uważasz?

-- Oczywiście. -- Powiedział to tak szybko, tak pewnie, że wiedziała, że
nie było miejsca na kłótnie.

-- Dlaczego?

-- Ponieważ to są \textit{ich }domy. Dlaczego mieliby płacić za używanie
rzeczy w~swoich domach?

-- Zgadzam się z~Tobą, ale firma powiedziałaby, że to dlatego, że wybrali
życie w~miejscu, gdzie czynsz jest niższy, ponieważ właściciel pomyślał,
że zarobi pieniądze na urządzeniach AGD. To było umowa, a~oni są stroną,
i mogą zapłacić więcej gdzieś indziej, jeżeli chcą móc wybierać.

-- Możemy płacić więcej?

Nadifa prychnęła. 

-- Nie, póki nie skończysz studiów i~nie dostaniesz
dobrej pracy, Adi.

Spojrzał na Salimę.

-- Wiem. Nie powiedziałam, że się z~tym zgadzam. To jest to, co oni by
powiedzieli. Jest wiele umów, które możesz zawrzeć, a~umowa tutaj jest
taka, że musisz używać ich produktów w~sposób, który daje im najwięcej
pieniędzy lub zapłacić za ich odblokowanie. Powiedzą, że dostajesz
\textit{więcej} wyborów, ponieważ pozwolą Ci kupić odblokowanie.

-- Ale my mamy już ten wybór.

Salima spojrzała na niego ostro. 

-- Nie, nie masz. Nie, jeżeli
przywróciłeś urządzenia do ustawień domyślnych.

Wyglądał przez chwilę na winnego, potem powiedział: 

-- Ok,
\textit{mieliśmy} ten wybór i~znowu możemy go mieć dla nas. Za darmo.
Pokazałaś nam.

To powolne przewracanie w~żołądku. Odblokował je wszystkie, wszystko w~ich domu, a~teraz zostaną złapani. Wszyscy zostaną złapani. Jeżeli
Abdirahim nie zrobił, jak prosiła, to kto zrobiłby?

Wciągnęła głęboko powietrze. 

-- To, co zamierzam powiedzieć, to nie tak
jak widzę sprawy, ale to jest jak korporacja je widzi. Mówią, że nie
masz tego wyboru, że \textit{oni} mają ten wybór i~mogą Ci go sprzedać.
Jednak jeżeli weźmiesz sobie bez płacenia, to kradzież. Znowu, tak
\textit{oni} myślą.

Był szybki. 

-- Ale byłabyś w~stanie odblokować swoje rzeczy bez płacenia,
prawda? Zatem dlaczego to nie jest kradzież?

Bystry chłopak, z~inteligencją kogoś, kto zawsze musiał myśleć szybko, z~poważnymi konsekwencjami, jeżeli się pomylił. 

-- Ponieważ pracowałabym
dla firmy.

-- Przeciwko sąsiadom. Jednak twierdzisz, że nie opuściłabyś tego
miejsca, ponieważ tu jest Twoje miejsce. Ale jesteś traktowana jakbyś
była lepsza niż my! -- Tracił spokój. Ciągle chłopiec, mimo wszystko. Nie
pozwoliła sobie na gniew. Spoglądając ukradkiem na Nadifę, ujrzała, że
przyjaciółka była bardzo zadumana, zapominając zganić syna za brak
szacunku do starszych.

-- Nie sądzę, że jestem lepsza. Firma po prostu zobaczyła moje
umiejętności i~zaoferowała mi pracę. Tak jak Twoja matka dostaje
pieniądze, kiedy zszyje komuś ubrania. Tobie też by zapłacili,
pamiętasz.

-- Nie wziąłbym ich pieniędzy. -- Spojrzał na matkę. -- Mam pracę domową.

-- Idź odrabiać.

Wstał i~poszedł do drugiego pokoju. Nie patrzyły na siebie. 

-- Co
zamierzasz zrobić? -- spytała Nadifa.

Salima wzruszyła ramionami. 

-- Muszę o~tym pomyśleć.

Wye przysłała jej jeszcze dwa smsy przed snem, na które nie
odpowiedziała. Zasnęła, w~końcu, do szumu klimatyzacji i~agregatu
lodówki.

Jej telefon zadzwonił, gdy myła zęby. Wye. Wypluła, przepłukała i~nie
odpowiedziała. Zadzwonił znowu.

I znowu.

-- Halo?

-- Przepraszam, że jestem taką zrzędą, ale tutaj się wszystko zaczyna.
Zarząd polubił projekt partnerów i~naciskają na to, rzucają wręcz tonami
inżynierów. Chcą zrobić duże otwarcie w~przyszłym tygodniu, konferencja
prasowa i~tak dalej, z~partnerami w~ośmiu krajach. Uwielbiają Twoją
historię i~chcą Cię wyróżnić. Nawet dostałabyś pieniądze za pracę
reklamową, wiesz, żeby pomóc Ci za utraconą pracę. Byłam w~biurze od
szóstej rano. Okazuje się, że jestem tutaj jedyną, która zna prawdziwego
hakera i~to czyni mnie ekspertem na miejscu. -- Zachichotała nerwowo. -- Przepraszam, to właśnie się wydarzyło. Jednak musimy iść dalej, wszystko
czeka na Ciebie.

Salima nie potrafiła wymyślić niczego w~odpowiedzi.

-- Halo? Salima?

-- Wye\ldots 

-- Salima, wiem, że to szalone, ale to rozwiązuje problemy wszystkich.
Proszę, powiedz, że przynajmniej przyjedziesz i~z nimi porozmawiasz?

-- Mam pracę.

-- Gdzie? Możemy przyjechać do Ciebie.

Poczuła się uwięziona. 

-- Pracuję dzisiaj z~domu. -- Musiała tak robić, co
tydzień lub dwa, kiedy było dużo drobnych prac do zrobienia.

-- Doskonale! To jest po prostu idealne! Wyślę smsa, kiedy ustalimy, o~której przyjedziemy, dobra?

-- Wye!

Ale ona już przerwała połączenie.

Potem Salima nie potrafiła się skoncentrować. Słuchała dźwięków sąsiadów
wychodzących do pracy, potem matek z~małymi dziećmi przechodzącymi po
korytarzach, kierującymi się do innych mieszkań w~towarzystwie
piskliwych, dziecięcych głosów o~zabawach i~przyjacielskim współczuciu.

Liczby pływały na jej wielkim ekranie, odmawiając spójności. Chodziła po
małym mieszkaniu, potem po korytarzu. Jej mały notatnik był w~kieszeni,
numery mieszkań, daty i~notatki. Była w~tak wielu miejscach.

\textbf{ ETA 15 minut. Ok?}

Westchnęła.

\textbf{ ok}

Wróciła do mieszkania, żeby poczekać na dzwonek z~wejścia. Przynajmniej
przyjechali po porannym tłoku, więc nie czekaliby długo na windę.
Rzeczywiście, byli przy jej drzwiach minuty po wpuszczeniu ich na dole,
Wye i~dwóch facetów, jeden biały, jeden pochodzenia hinduskiego, obaj w~koszulkach Boulangismu, obaj z~młodzieńczą fryzurą i~wielkimi,
błyszczącymi równymi uśmiechami.

-- Dziękuję za spotkanie -- powiedział Hindus. Mówili do niego Paul, ale
na wizytówce było napisane ,,Pritpaul''. Odmówił herbaty i~wody, tak jak
biały facet (,,Rog''), ale Wye zaakceptowała kawę i~obserwowała uważnie,
jak Salima wkładała kapsułkę w~maszynę i~podstawiała filiżankę z~dołu,
potem wyciągała kapsułkę i~ją wyrzucała.

-- W~porządku -- powiedziała. -- Wye była bardzo tym podekscytowana.

Wye była na tyle skromna, żeby wyglądać na zmartwioną.

Paul nie zauważył. 

-- Była też bardzo podekscytowaną Tobą. Słyszeliśmy o~Tobie i~szczerze, nie mogłabyś być doskonalsza do tego, co mamy na
myśli. Myślimy, że to mogłoby być bardzo duże. -- Podniósł dłonie do
góry, ręce rozłożone tak szeroko, jak mógł w~jej zatłoczonym pokoju. -- \textit{Bardzo} duże. Dobre dla nas, dobre dla Ciebie i~dobre dla ludzi
takich jak Ty.

-- Ludzi takich jak ja?

-- Ludzi, którzy wpadają między szczeliny, którzy nie mogą sobie pozwolić
na płacenie całej ceny za wszystko, ale którzy czasem chcą się popisać
większymi możliwościami przy specjalnych okazjach. To naprawdę najlepsze
dla każdej stron, nowy rodzaj elastyczności. Starzy właściciele
Boulangism byli na to ślepi, ale my jesteśmy kompletnie nakręceni
możliwościami pracy \textit{z }naszymi klientami, a~nie przeciwko nim.
Mamy nadzieję, że staniesz się tego częścią.

Tutaj pojawiła się przestrzeń w~rozmowie, gdzie Salima mogłaby
powiedzieć coś pozytywnego. Wszyscy w~pokoju oczekiwali po niej, że
powie coś pozytywnego. Rozmowa miała kształt, a~może kierunek, i~mogła
ją poklepać w~plecach, dać jej małe pchnięcie w~tym kierunku, a~następny
moment byłby czymś miłym ze strony Paula lub Wye lub białego faceta, a~potem z~powrotem do niej, pchnięcie, pchnięcie i~pchnięcie, aż rozmowa
nabrałaby wystarczającego pędu, żeby nikt nie mógł jej zatrzymać.

Wydawało się to małostkowe, żeby odmawiać, ale widziała, że pozytywna
odpowiedź w~tym miejscu była biletem do ekspresu bez zatrzymywania.

-- To brzmi bardzo miło, ale nie sądzę, żebym była właściwą osobą dla
Was.

Wye wyglądała na zszokowaną. Paul i~biały facet wyglądali na osłupiałych
przez chwilę, a~potem wkleili uśmiechy. 

-- Oczywiście szanujemy Twoją
decyzję, ale zastanawiam się, czy możesz powiedzieć nam dlaczego? W~końcu przyjechaliśmy z~dość daleka, żeby porozmawiać o~tym z~Tobą. Może
jeżeli wyjaśnisz swoje obiekcje, dowiemy się czegoś, co pomoże nam
bardziej na naszym kolejnym spotkaniu?

Nie powiedziała, że w~ogóle ich nie zapraszała. 

-- Po prostu nie czuję
się z~tym dobrze. Rozumiem wasz pomysł, że sprzedajecie nam więcej
wolności. Jednak jest tak dlatego, że nasze urządzenia zabrały nam tak
dużo wolności na początku, a~teraz mogą odsprzedać.

-- Ale nikt nie zmuszał Cię do wyboru Boulangismu. Wybrałaś produkt,
który miał określone ograniczenia, a~w~zamian masz umowę na wynajem.

-- Czy masz minipiekarnik Boulangism?

-- Nie, nie mam.

-- Dlaczego nie?

-- To nie wybór, który dokonaliśmy -- powiedział biały facet. -- Wybraliśmy
inne umowy. To jest wspaniała sprawa z~wolnością: wszyscy otrzymujemy
wybór propozycji, które nam najlepiej pasują.

Salimie udało się lekko uśmiechnąć. 

-- Ciągle mówicie o~wyborze. To jest
jedyne miejsce, które mogłam dostać i~zabrało mi to miesiące. W~jaki
sposób jest to wybór?

-- Żyłaś gdzieś wcześniej przed tym miejscem, prawda?

-- Schronisko dla uchodźców.

-- Mogłaś wybrać, że tam zostaniesz, prawda?

Chciała, żeby ci ludzie sobie poszli. 

-- Nie sądzę, żeby to był rodzaj
wyboru.

Pokręcił głową. 

-- Chodzi o~to, że \textit{miałaś} wybór, a~to z~powodu
urządzeń takich jak nasze, dzięki którym wybudowanie dotowanych mieszkań
przez właścicieli stało się ekonomicznie opłacalne.

Nic nie powiedziała. Zaczynała wpadać w~gniew i~nie lubiła się gniewać,
nie chciała pokazywać tym ludziom, że się gniewa.

-- Chcemy pomóc wam, ludzie, pozwolić wam wyciągnąć więcej z~życia, dać
wam więcej wyborów.

\textit{Co z~wyborem obejścia zabezpieczeń moich rzeczy?} Nie spytała o~to.

-- Szczerze, nie rozumiem Twojej decyzji.

\textit{Wybór jest dobry tak, długo, dopóki nie wybiorę niepomagania wam?}
Nie powiedziała tego.

-- Nie widzisz, że chcemy wam pomóc?

\textit{Widzę, że chcecie, żebym wam pomogła wyciągnąć więcej pieniędzy od
,,ludzi takich jak ja''.}

Ciągle nic nie mówiła.

-- Może powinniśmy pójść -- powiedziała Wye. Odmiennie od mężczyzn,
zwracała uwagę na reakcje Salimy.

-- Po prostu przyjaźnie sobie dyskutujemy -- powiedział biały facet. -- Tak
czy inaczej, nie musimy być w~biurze jeszcze przez godzinę. Salima, czy
możesz mi po prostu powiedzieć, w~czym jest problem?

Usłyszała siebie mówiącą: 

-- Wolałabym pomóc moim sąsiadom oszczędzić ich
pieniądze niż je wydać.

-- Co to miało znaczyć? Odblokowanie minipiekarników mogłoby oszczędzić
mnóstwo, jeżeli rozsądnie kupisz duże zakupy i~zaplanujesz posiłki.

Znowu, jej głos powiedział: 

-- Zaoszczędzilibyśmy więcej, gdybyśmy nie
musieli płacić za odblokowanie naszych minipiekarników.

-- Nie rozumiem, co to ma\ldots  -- Przerwał. -- Och. Ta, jasne, ale wiesz, co
się stanie, jeżeli zostaniesz na tym złapana.

-- Wolałabym raczej im pomóc niż być złapaną.

Prychnął. 

-- Wszyscy są łapani.

-- Skąd mógłbyś wiedzieć? Ludzie, których nigdy nie złapałeś, byliby
ludźmi, o~których byś się nie dowiedział. -- Spojrzała mu w~oczy. Był
teraz rozjuszony, czerwony, żyła pokazała się czole.

-- Tak, może tak, ale to nie będziesz Ty, paniusiu. Wiemy, co tutaj się
dzieje, wiesz. Jesteś na naszym radarze. Znaczy, mam nadzieję, że masz
swój sprzęt w~porządku, ponieważ jeżeli cokolwiek jest nie w~porządku,
zobaczymy to. Będziemy też wiedzieć z~kim najpierw rozmawiać.

Wye otworzyła usta, zamknęła. Spojrzała przepraszająco na Salimę. Też
była zarumieniona. Paul wstał. 

-- Myślę, że będziemy już iść. Dziękuję
bardzo za Twój czas, Salimo.

Patrzyła, jak wychodzą. Potem kiedy drzwi się zamknęły, otworzyła je po
cichu i~obserwowała, jak wzywają windę, chcąc być pewna, że wychodzą bez
rozmawiania z~jej sąsiadami. Chwile po tym jak nacisnęli przycisk
wezwania, drzwi otworzyły się i~była tam w~kabinie kobieta wyglądająca
na zaskoczoną, kobieta, której nigdy nie widziała wcześniej, ubrana w~inteligentny, letni strój, z~inteligentnym makijażem, inteligentną
fryzurą. Ktoś z~drugiej strony, kogo winda nigdy, ale to nigdy nie
powinna otworzyć się na piętra dla biednych.

Trzech pracowników Boulangism skinęło jej głową, jakby nic nie było źle
i weszło do kabiny. Kiedy się odwrócili, złapała spojrzenie Wye, Wye
wzruszyła ramionami i~skrzywiła się w~wymownej minie zagadkowych
przeprosin. Czy przepraszała za sposób, w~jaki jej szef mówi, za groźby,
za fakt, że windy przyszły dla nich, kiedy je wezwali, ale nie dla
Salimy?

Salima wspięła się siedem pięter na piętro Salimy i~nacisnęła dzwonek.

Musiały czekać tylko kilka minut w~biurze szkoły, zanim Abdirahim się
pojawił. 

-- Mama? -- Wyglądał na zmartwionego i~też, kiedy zobaczył
Salimę, na zmieszanego.

-- Chodź, będziemy rozmawiać w~drodze -- powiedziała Nadifa, spoglądając
na niego z~troską, co uciszyło jakiekolwiek pytania. To był wzrok,
którego musieli przez lata używać znacząco, choć nie ostatnio.

Kiedy byli na ulicy i~śpieszyli się na przystanek, Salima powiedziała: 

-- Musimy przywrócić wszystko w~budynku do ustawień fabrycznych.

Pokręcił głową. 

-- Myślałem, że już to zrobiłaś.

-- Tak, a~Ty i~Twoi przyjaciele to odkręciliście.

Zaczął zaprzeczać. Przerwała mu.

-- Nie jestem głupia, Abdirahim. Nawet się nie zgadzam. Ale dzisiaj
odrzuciłam ich ofertę, powiedziałam, że nie pomogę sprzedawać kodów
odblokowującym naszym przyjaciołom. Są wściekli na mnie i~zamierzają
spróbować mnie za to ukarać. Zamierzają obserwować nas wszystkich
\textit{bardzo} dokładnie i~współpracują z~właścicielami Wieżowców
Dorchester. -- To był tylko domysł, ale był dobry. W~końcu, właściciele
dostawali prowizję od nich, zatem musieli mieć jakąś relację. -- Dlatego
musimy wszystko odwrócić, zanim nas złapią.

Abdirahim szedł kilka kroków w~ciszy. Potem: 

-- To nie przetrwa długo.
Zbyt wiele osób wie jak hakować.

-- Wiem -- powiedziała. -- Ale musimy wymyślić lepszy sposób.

Wye mówiła o~maszynach wirtualnych i~tam była wskazówka. Fora dyskusyjne
były, jak zwykle, pełne niezgadzania się, spekulowania, obrażania,
przechwalania, spamu i~nieprzyzwoitości.

Abdirahim niechętnie przejrzał notatnik Salimy i~wskazał mieszkania, o~których wiedział, że są zhakowane, włączając w~to kilka, których nigdy
nie była w~stanie tego ustalić. Pracując razem, przeszli znowu od drzwi
do drzwi, razem z~Nadifą dla wsparcia moralnego, dzwoniąc do drzwi długo
po przyzwoitej porze, pracując, aż byli tak zmęczeni, że zaczęli robić
głupie błędy. Robili zmiany, pracując w~każdym mieszkaniu, jedno z~nich
wycofujące zmiany, a~drugie rozmawiające z~ludźmi, szczególnie z~dziećmi
o ważności pozostawienia wszystkiego tak jak było, tylko na chwilę, póki
nie wymyślą lepszego rozwiązania.

Poszukiwania informacji o~,,maszynach
wirtualnych''\footnote{zob.~\url{https://pl.wikipedia.org/wiki/Maszyna\_wirtualna}
-- przyp.tłum.} na forach dyskusyjnych rozjaśniło niektóre rzeczy, ale
ciągle posyłało ich w~labirynt czytania o~pomysłach informatyki, co do
których żadne nie miało przygotowania teoretycznego. Na szczęście, było
wielu właścicieli Dishera i~Boulangismu dookoła świata, którzy byli tak
samo niewykwalifikowani do zrozumienia maszyn wirtualnych, ale niemniej
jednak nalegający, żeby ktoś im wyjaśnił, stąd i~oni potrafili złożyć
coś podobnego do zrozumienia. Pomogło, że Nadifa, Salima i~Abdirahim
potrafili czytać w~siedmiu językach.

Komputery, jak się zdawało, były w~pewnych ważnych częściach takie same.
Każdy komputer dzielił wspólne dziedzictwo, ,,architekturę'', która
umożliwiała uruchomienie dowolnego programu, który był napisany w~języku
dla komputerów, w~kodzie programu. Niektóre komputery były szybsze lub
miały więcej pamięci niż inne, a~niektóre oczekiwały poleceń napisanych
w inny sposób, ale nawet najwolniejszy komputer mógł uruchomić
najbardziej skomplikowane programy, choć wykonanie mogłoby zająć mu
lata, podczas gdy inny komputer mógł skończyć wykonanie tego programu w~mgnieniu oka.

Jednak nie musiałeś tłumaczyć kodu komputera, żeby go uruchomić na innym
komputerze. Zamiast tego, mogłeś napisać program komputerowy, który był,
w istocie, \textit{samym komputerem}. Mogłaś napisać program komputerowy,
który mógłby być uruchomiony na zmywarce, którego celem byłoby
uruchomienie programów z~minipiekarnika Boulangismu. Tak, mogłaś
przekonać zmywarkę, że jest minipiekarnikiem. Kiedy minipiekarnik
wewnątrz zmywarki wysyłał instrukcje włączenia grzałki lub zrobienia
zdjęcia jedzenia w~komorze, żeby sprawdzić jego gotowość, zmywarka z~działającym komputerem mogłaby wysłać dowolne dane, które chciał toster,
a on by im ślepo zaufał.

To właśnie była ,,maszyna wirtualna'', wyobrażony komputer wewnątrz
innego komputera. A gdyby to nie było dostatecznie dziwne, można było
uruchomić maszynę wirtualną \textit{Boulangism} wewnątrz minipiekarnika
Boulangism, czyli minipiekarnik udawałby \textit{inny minipiekarnik}. Co
wydawało się rodzajem sztuki dla sztuki lub gry dla Salimy, póki
Abdirahim nie złapał tego w~przebłysku i~nie wyjaśnił.

-- Jeżeli masz toster, który był zhakowany, możesz w~nim uruchomić
wirtualny toster. Wtedy możesz uruchomić zwyczajną, fabryczną wersję
programu minipiekarnika na wirtualnym piekarniku. Kiedykolwiek fabryka
łączy się z~urządzeniem, to prześle wiadomości do wirtualnego
minipiekarnika. Wirtualny toster odpowie w~taki sposób, żeby wyglądało,
że jest niezmienione. To jak wzięcie zakładnika na wojnie i~zmuszenie go
do mówienia tego, co trzeba odpowiadać na punktach kontrolnych, żeby nie
wzbudzić podejrzeń. -- Potarł razem dłonie.

-- Czy to działa?

Wzruszył ramionami i~wskazał na monitor. 

-- Oni mówią, że tak. Ale nie
wiedzieliby na pewno, prawda?

-- Chciałabym zapytać Wye -- powiedziała. -- Ona by wiedziała.

-- Powiedziałaś, że mówiła Ci o~maszynach wirtualnych\ldots 

-- Tak, ale ona tylko myślała na głos. Może przemyślała to i~znalazła
błąd w~planie.

Nawet nie podniósł wzroku z~ekranu, gdzie chciwie wczytywał się w~maszyny wirtualne. 

-- Mogłabyś jej zapytać.

-- To nie byłoby wobec niej sprawiedliwe.

-- Nic nie jest sprawiedliwe. -- Powiedział to tak nonszalancko, że ją to
zaszokowało. Taki młody chłopak, takie wielkie myśli.

Dotarła aż do wyszukania numeru Wye, ale nie wybrała. Zamiast tego,
wzięła swojego Boulangism i~pomogła Abdirahimowi zainstalować na nim
wirtualną maszynę. Musieli być w~tym bardzo dobrzy.

~

W dzień, kiedy Boulangism się uruchomił, każda wirtualna maszyna w~każdym minipiekarniku i~zmywarce zadzwoniła i~pokazała nową,
podniecającą ofertę odblokowywania. Te powiadomienia załadowały się w~małe okienka, które trzeba było stuknąć do powiększenia i~przeczytać
przed usunięciem.

Spędzili cały dzień nerwowo, spięci, podskakując każdego dnia, gdy
słyszeli brzęczek u drzwi, pewni, że to ktoś w~domu przekazujący
informację, że policja zrobiła nalot lub wszystko przestało działać lub
właściciele budynku byli przy drzwiach, każąc im się wynosić.

Ale dzień minął, potem następny, i~następny.

Ostrożnie, jeden mięsień na raz, zrelaksowali się.

Salima była bardzo dobra w~pieczeniu. Odkryła chleby skandynawskie i~każdego ranka robiła cztery małe kardamonowe bułeczki posypane
cynamonem. Zjadała jedną i~dawała trzy inne pierwszym trzem osobom,
które spotkała na klatce schodowej w~drodze na dół, ciągle ciepłe,
pozostawiające zapach, który był lepszy niż najlepszy powiew perfum, na
które mogłaś natrafić w~kabinie windy, kiedy została opuszczona przez
kogoś po drugiej stronie.

Czekała na dźwięk Boulangism, kiedy jej ekran zagrał. Prawie nie
odebrała -- numer zablokowany, wczesna godzina, prawie z~pewnością spamer
-- ale przesunęła palcem w~poprzek.

-- Salima. -- Głos Wye był ściśnięty, nieco zadyszana. Ramiona Salimy
spięły się tak mocno jakby rakiety tenisowe. Gdzieś w~głowie, zawsze
czekała na ten telefon.

-- Tak?

-- Maszyny wirtualne, których używasz, już ich nie ogłupiają. Wysłali
aktualizację, która jest zaprojektowana, żeby złamać wirtualki. Właśnie
sprawdziłam Twój budynek. Tylko Ty teraz tam \textit{wisisz}. Nie ma
sposobu, żeby tego nie znaleźli.

-- Och. -- Zacisnęła mocno powieki. Boulangism zadzwonił i~drzwiczki się
otworzyły. Zapach cynamonu, kardamonu i~świeżego chleba. Dźwięk jej
krwi, grzmiący w~uszach. -- Och.

-- Ale jest fix.

-- Fix.

-- VM\footnote{ od ang. virtual machine -- maszyna wirtualna -- przyp.tłum.}, której używam do
testowania. Jest niewykrywalna. Musi być, inaczej nie mogłabym jej
używać do symulacji w~pracy. Spakowałam ją i~\ldots 

-- Nie, Wye.

-- \textit{Tak}, Salimo.

-- Nie będziesz ryzykowała wszystkiego dla mnie.

-- Nie pozwolę im Cię złapać, ponieważ Ci nie pomogłam.

Salima nie myślała o~tym w~ten sposób. Ludzie w~budynku zostaliby
przyłapani, gdyby Salima nie pomogła im. To by ją zniszczyło. Salima
zostałaby przyłapana, gdyby Wye jej nie pomogła. Zatem\ldots 

-- Tak, Wye.

Wye faktycznie się roześmiała, cichy, mocny dźwięk. 

-- Coś Ci wyślę. -- Wymieniła nazwę darknetu, tego, na którym Salima miała konto, choć nigdy
o tym nie wspominała Wye. Zastanawiała się, ile Wye wiedziała o~tym, co
zamierzała, w~tygodniach, od kiedy ostatni raz rozmawiały.

Abdirahim nie wyszedł jeszcze do szkoły i~Nadifa nawet nie mrugnęła
okiem, kiedy Salima powiedziała, że potrzebuję go na cały dzień. Jej
notatnik był zdarty, z~oślimi uszami, od noszenia pomiędzy mieszkaniami
tyle razy.

-- Ufasz jej?

Salima skinęła głową. 

-- Jeżeli chciałaby tylko, żebym miała kłopoty,
mogłaby to zrobić znacznie łatwiej niż to.

Nowe VM i~oprogramowanie sterujące było znacznie elegantsze niż
cokolwiek, co ściągnęli z~darknetu. Chowało się bardzo dobrze, chyba że
znałaś wielodotykowy wzór, żeby przełączyć ekran z~prawdziwego,
zhakowanego sterowania do pozornych opcji VM. Dołączone notatki
obiecywały, że byłoby też doskonale ukryte przed narzędziami sieciowymi
Boulangism. Było też wystarczająco proste w~instalacji, tylko
aktualizacja ustawień, które już były na miejscu, ale też był to także
poranek, kiedy tak wiele osób wychodziło z~domu na cały dzień,
zostawiając urządzenia wystawione na wścibskie sondowanie korporacji.

Opracowali system. Ona pukała do drzwi, szybko wyjaśniała sprawy osobie,
która odpowiedziała i~wysyłała Adiego, żeby zaczął pracę. Wtedy szła do
kolejnych drzwi i~kolejnych, mówiąc szybko, błagając ludzi, żeby dla
nich zostawiali odblokowane drzwi. Adi kończył jedno mieszkanie, a~potem
kolejne, wpuszczając się do niezamkniętych mieszkań. Musieli dotrzeć do
starszych osób, osób z~niepełnosprawnością, matek, które zostały w~domu,
ale musieli dotrzeć do tych mieszkań, zanim ich mieszkańcy wyjdą.
Pędziła od jednego miejsca do drugiego, udało się jej dotrzeć do prawie
wszystkich i~otrzymać numery telefonu do czterech, które ominęła,
wspinając się po schodach z~powrotem do Adiego, gorączkowo wydzwaniając
do nich, docierając do dwojga i~zostawiając wiadomości dla pozostałych
dwóch. Zadzwoniłaby później.

Korzystne było to, że software Wye był taki łatwy w~instalacji, ponieważ
Salima była wrakiem, choć Abdirahim był po nastoletniemu spokojny i~zuchwały, szybko wykonując pracę w~każdym miejscu, pierś wypchnięta i~uśmiech grający na jego ustach, gdy się koncentrował.

Ale robili to. Do lunchu byli dalej niż połowa. Gdy zapukali do drzwi
starego Serba, nalegał na krakersy z~masłem orzechowym, które pożarli,
zaskoczeni, jak bardzo byli głodni. Salima pamiętała, że miała jeszcze
cztery bułki kardamonowe w~tosterze, pobiegła po nie i~zjedli po jednej
i zostawili resztę Serbowi. Uśmiechnął się, pomachał do nich, gdy szli
korytarzem do kolejnego mieszkania.

Gdy schodzili po schodach, ekran Salimy się włączył.

-- Nadifa?

Głos był pilnym szeptem. 

-- Właściciele budynku pukają do drzwi.

-- Gdzie?

-- Zaczęli od czterdziestego pierwszego. Właśnie byli u nas, sprawdzali
sprzęt AGD. -- Jako pierwsze zrobili własne mieszkania, częściowo, żeby
się upewnić, że wiedzą, co robią, częściowo, ponieważ myśleli, że
właściciele mogą zacząć od nich.

-- Ok.

-- Nic nie znaleźli. Cokolwiek zrobiłaś, działa.

-- Ok.

-- Wchodzą do windy.

-- Ok.

Przez chwilę, wszystko, co słyszała, było szumem krwi w~uszach.
Zrozumiała, że patrzy się bezmyślnie na telefon. Abdirahim patrzył się
na nią zaniepokojony.

-- Ile zostało?

Wyciągnął jej notatnik, oberwany, z~notatkami w~dwóch stylach pisma,
przesunął palcem po kolumnach, poruszał ustami, gdy liczył.

-- Dwadzieścia cztery -- powiedział.

Zamknęła oczy i~głęboko wciągnęła powietrze. Dwadzieścia cztery.
Właściciele w~windach. Jej mieszkanie było bezpieczne, a~właściciele by
tam pewnie poszli. Pozostałe mieszkania były rozrzucone po siedmiu
piętrach dla biednych. Były na czterech piętrach nad nimi i~trzech
poniżej.

Wzięła notatnik od Abdirahim, wydarła strony z~czterema piętrami
powyżej, złożyła je i~włożyła je do tylnej kieszeni jej dżinsów. 

-- Zrób
dolne trzy -- powiedziała. -- Ostrożnie. Sprawdzaj, zanim wyjdziesz na
korytarz. Nie pozwól właścicielom zobaczyć Cię. Jeżeli Cię złapią, nie
pozwól im dostać notatnika.

Abdirahim lekko zbladł, jego zuchwałość nastolatka stopniała. Słyszalnie
przełknął. Spojrzał na pozostałe strony notatnika, przejrzał je. 

-- Ok.

-- Ok -- powiedziała. Odwrócił się, ale złapała go i~uściskała go gorąco.

-- Idź -- wyszeptała, potem odwróciła się na pięcie i~pobiegła do schodów.

Spokój spłynął na nią, gdy weszła na kolejne piętro. Na tym piętrze były
trzy niezamknięte mieszkania, z~żadną osobą nie rozmawiała, zatem weszła
do pierwszego, oczy poruszające się, żeby znaleźć sprzęty, dopasowujące
do listy, raz, dwa, trzy. Wyjęła USB z~kieszeni -- nie pamiętała, czy
ostrzegła Adiego, żeby wyrzucił go, gdy znajdzie się w~niebezpieczeństwie, ale wiedziałby o~tym(prawda?) -- i~podeszła do
minipiekarnika, podniosła go pewnym, ekonomicznym ruchem.

Jedno mieszkanie. Drugie. Trzecie. Klatka schodowa. Tutaj dwa. Ludzie w~środku. Zadzwonić do pierwszych, wystawiona i~przestraszona, szum windy
w szybach kilka metrów obok. Serce bijące. Pachy mokre. USB trzymany tak
mocno w~pięści, że bolało.

Kobieta, która odpowiedziała, była stara, ale stała wyprostowana,
patrzyła jasno, chuda, pomarszczona i~brązowa. Salima pamiętała, że była
pediatrą w~Damaszku. Kiedyś grała na skrzypcach, ale reumatyzm, od
którego spuchły jej kłykcie, odebrał jej to.

-- Właściciele budynku są w~budynku -- wyszeptała Salima, gdy wsuwała się
koło kobiety. Kobieta odsunęła się i~pozwoliła jej ruszyć do pracy.
Kiedy Salima skończyła, kobieta zatrzymała ją, złapała za dłoń, palce
kruche, suche i~wygięte.

-- Dziękuję -- powiedziała i~lekko ją ścisnęła. Poczucie utrzymywało się
na dłoni Salimy, gdy patrzyła w~notatki i~dzwoniła do kolejnych drzwi.

Zanurkowała na klatkę schodową, a~jej telefon zabrzęczał. Sprawdziła go.

\textbf{ Właściciele na klatce schodowej}

Och.

Mogła usłyszeć cichy ruch, miękkie stąpanie butów, ktoś schodzący i~próbujący zachować ciszę. Po drugiej stronie wieżowca była druga klatka
schodowa, zarezerwowana na sytuacje awaryjne z~drzwiami pod alarmem.
Kroki były bliżej. Znowu otworzyła drzwi i~weszła z~powrotem na korytarz
na piętrze.

Mogła zapukać do kogoś drzwi, próbować pozbyć się notatnika i~USB, zanim
nadejdą właściciele. A co jeżeli się nie uda? Wszyscy mieliby kłopoty.

Prawie im się udało. Może zostało kilka mieszkań. Zbyt mało, pomyślała,
żeby właściciele byli w~stanie wyrzucić ich wszystkich za konspirację.

Teraz mogła usłyszeć kroki na klatce. Nie przejmowali się, żeby być
cicho. Musieli usłyszeć drzwi zatrzaskujące się za nią.

Winda zaszumiała. Była w~pułapce. Skrzyżowała ręce na piersi, rozstawiła
szerzej stopy i~czekała.

Drzwi windy otworzyły się przed drzwiami na klatkę schodową. Odwróciła
się, żeby stawić im czoło, starając nie pokazać strachu. Zimna mina, to
właśnie im pokaże.

W kabinie był Abdirahim.

-- Wsiadaj! -- syknął.

Drzwi się zamknęły i~winda szarpnęła przy ruszaniu.

Abdirahim nacisnął przycisk parteru. 

-- Są też w~holu -- powiedział. -- Zatem nie możemy tam zjechać.

Winda szybko opadała. Wskaźnik pięter pokazał trzydzieści pięć i~potem
pokazał dwa myślniki ,, - - '', co wskazywało, że mijali pełnopłatne
piętra. Będą na parterze w~sekundy.

-- Adi\ldots 

Uśmiechnął się i~nacisnął szybką sekwencję na panelu windy, palce pewne
i szybkie. Winda zwolniła i~się zatrzymała. Drzwi się otworzyły.

Te \textit{drugie} drzwi.

Drzwi, które otwierały się na piętro dla bogaczy.

-- No \textit{chodź} -- powiedział.

Wyszli na korytarz. Dywan był bogaty, purpurowobrązowy, miał delikatne
linie od automatycznego odkurzacza, który ułożył runo w~jednolitym
kierunku.

-- Adi\ldots 

Znowu się uśmiechnął. 

-- Rozgryzłem windy dzień po tym, gdy zatrzymali
kapitanów windy. To nie było trudne.

Na piętrach dla biednych, wskaźniki windy pokazywały tylko pozycję
kabin, kiedy przejeżdżały przez piętra, na których mogły się zatrzymać,
wydając się znikać pomiędzy czterdziestym drugim piętrem a~poddaszem,
trzydziestym piątym i~holem na parterze. Tutaj, grafika pokazywała
wszystko. Kabina zatrzymała się na trzydziestym siódmym, pojechała na
trzydzieste ósme. Żaden mieszkaniec biednych pięter nie traciłby czasu
na wzywanie windy na przejechanie jednego piętra. To byli właściciele,
ze specjalnymi brelokami, które wezwałyby windę bezpośrednio, nawet na
piętra dla biednych.

-- Teraz co? -- spytała. Ciągle próbowała zrozumieć, co zrobił Adi.

Pokazał jej notatnik ze znaczkami koło każdego wpisu. 

-- Teraz skończymy
Twoją listę.

Jeszcze dwa piętra. Trzydzieste piąte i~trzydzieste szóste. Podała mu
kartkę, jedną sobie zatrzymała, a~wtedy wezwał kolejną windę i~wcisnął
kolejną szybką sekwencję, potem przycisk pięter trzydzieści pięć i~trzydzieści sześć.

Kiedy drzwi się otworzyły, pobiegł w~korytarz, cała lekkomyślność, nawet
nie sprawdzając, czy właściciele nie czekali na niego. Była
ostrożniejsza, kiedy drzwi otworzyły się na trzydziestym szóstym,
wyglądając zza drzwi, zanim weszła na korytarz, potem, też, pobiegła do
ostatnich drzwi na jej liście.

~

\textbf{ Kolejna VM dla Ciebie, aktualizacja wychodzi w~przyszłym tygodniu. Wystarczy czasu?}

Salima dotknęła ekran i~odpowiedziała.

\textbf{ Jesteś bardzo złym pracownikiem, Wyoming.}

\textbf{ Zdaje się, że tak.}

\textbf{ Ale jesteś dobrą przyjaciółką}

\textbf{ Zdaje się, że tak.}

\textbf{ Dziękuję, Wye}

\textbf{ Nie ma sprawy}

Salima rozciągnęła się i~wstała. Musiała iść do pracy. Lodziarnia. Teraz
mieli niesamowite obroty, dzięki wyjątkowo gorącej wiośnie. Mieli też
nowy smak, który uwielbiała: kruche czarne oliwki, kozi ser, który był
najdziwniejszym smakiem lodów, jaki kiedykolwiek próbowała i~jednocześnie najlepszym.

Zebrała bułeczki kardamonowe i~sprawdziła fryzurę w~lustrze. Ekran
zabrzęczał.

\textbf{ Wyślij mi przepis na bułeczki kardamonowe,
proszę!}

Wysłała i~wyszła do pracy.

\chapter*{Modelowa Mniejszość}


Amerykański Orzeł właśnie mijał Drogę Międzystanową I-278, kiedy
usłyszał ostre trzaśnięcie pałki uderzającej w~okno samochodu, potem
głos, głośny, rozkazujący, głos gliny, wściekły, na północ od I-278,
linii Masona-Dixona, która oddzielała białą Staten Island od czarnej
Staten Island.

-- Otwórz to pierdolone okno!

Nawet z~superludzkim słuchem, nie potrafił dokładnie określić, skąd
dochodził głos, nie dopóki uderzenia się nie powtórzyły, trzy uderzenia
pałką policyjną, potem dźwięk pękającej szyby samochodowej. Namierzył
dźwięk, dostroił się do dźwięków bijatyki i~przesunął ciało w~tym
kierunku, tracąc wysokość i~przecinając lekki deszczyk, który zawirował
po jego burzliwym przelocie. Z~ramionami przed nim, peleryną trzaskającą
za nim, był niebiesko-czerwoną rakietą, skierowaną w~dół i~ku dźwiękowi.

Mężczyzna na ziemi był czarny, czterech policjantów dookoła niego było
białych. Czarny człowiek na ziemi był szczupły, po trzydziestce, z~krótko przyciętymi włosami i~małym, zgrabnym wąsikiem. Jego samochód był
tak samo zadbany jak jego wąsy, błyszczący nawet w~tym wilgotnym, szarym
dniu, starszy model samochodu, ale taki, o~którego dobrze dbano.
Mężczyzna był na ziemi, nad jego wąsikiem krwawił z~twarzy. Były w~nią
wbite drobne kostki z~hartowanego szkła, ale także siniak na policzku w~kształcie końca pałki.

Czterech gliniarzy dookoła czarnego mężczyzny na ziemi miało wyjęte
pałki. Uderzali go. Uderzaliby go mocniej, ale został wyciągnięty z~zaparkowanego samochodu w~wąską przestrzeń pomiędzy tym, a~kolejnym
samochodem, dlatego wchodzili sobie w~drogę. Pałka wylądowała na jego
głowie z~trzaskiem, które Amerykański Orzeł poczuł głęboko w~drobnych
kościach jego wewnętrznego ucha kosmity. Skrzywił się i~skorygował kąt,
żeby dotrzeć szybciej, bezpośrednio na miejsce.

Samochody utrudniły gliniarzom dosięgnięcie do ciała czarnego mężczyzny,
ale jego nogi wystawały na parking, gdzie dwóch policjantów naraz mogło
się zamachnąć pałką. Ruszyli do tego jak robotnicy wbijający paliki
ogrodzenia, duże zamachy znad głowy, które były skierowane w~kości,
stawy.

Mężczyzna na ziemi jęczał i~krzyczał, podczas gdy czterej policjanci
krzyczeli. Krzyczeli: ,,Nie stawiaj oporu'', a, po dwóch lub trzech
uderzeniach, odrywali się od bicia, żeby pokazać panoramę na kamerach
osobistych, łapiąc kadry innych krzyczących ,,Nie stawiaj oporu'', gdy
kołysali się i~uderzali.

Amerykański Orzeł zanurkował nad ich głowami dostatecznie blisko, żeby
jego podmuch przesunął czapki, gdy on chwytał dwie pałki, szarpiąc je
tak mocno, że jeden z~policjantów upadł na twarz. Drugi złapał równowagę
akurat, gdy Amerykański Orzeł opadł na ziemię. Żaden z~policjantów już
nie bił czarnego mężczyzny. Patrzyli na niego, wszyscy, nawet czarny
mężczyzna, który krwawił z~nosa i~rany na czaszce.

-- Co? -- powiedział gliniarz, który uderzał mężczyznę w~głowę. -- Skurwysyn miał narkotyki na centralnej konsoli, na widoku. Nie chciał
otworzyć okna. Utrudnianie działania administracji rządowej.

Amerykański Orzeł zajrzał w~policjanta, ujrzał jego ciśnienie krwi,
zmierzył puls. Oko mężczyzny drgnęło, jego wzrok przesunął się po
Amerykańskim Orle, potem jego pałka drgnęła, wszystko zbyt szybko, żeby
ludzkie oko nadążyło.

Amerykański Orzeł nic nie powiedział. Obserwował, w~ciszy. Unosił się,
pięty jego butów tylko kilka centymetrów nad mokrym chodnikiem parkingu.
Sprzeciwianie się komuś, kto mógł się unosić, było onieśmielające.
Gliniarz zmiękł.

-- Skujcie go -- powiedział.

-- Hej, panowie policjanci, te kajdanki na moich kostkach, są zbyt ciasno
-- powiedział czarny mężczyzna. -- Nie czuję lewej stopy. Coś jest nie
tak.

Nie miał kajdanek na kostkach, ale jego lewa stopa i~kostka były
widzialnie złamane, wygięte pod mdlącym kątem z~powodu uderzeń dwóch
pałek.

Policjant, który powiedział ,,Skujcie go'', wezwał karetkę przez radio.
Czarny człowiek powiedział: 

-- Czy ktokolwiek nagrał wideo? Pozwę was
wszystkich aż do samego piekła. 

Gliniarz od poleceń, sierżant, naszywka BIANCHI, napakowany niski facet z~bicepsem, który wybrzuszał
rękawy i~karkiem zbyt grubym dla kołnierza munduru, wyglądał, jakby
chciał kopnąć go w~głowę. Jego stopa właściwie lekko drgnęła, ale
potrzebowałbyś obcego aparatu percepcyjnego Amerykańskiego Orła, żeby to
zauważyć. Niemniej jednak powstrzymał się i~wszedł do karetki z~czarnym
człowiekiem, kiedy przyjechali sanitariusze. Miał nieco hartowanego
szkła na twarzy, kiedy rozbijał szybę, co dało mu minimalne cięcie na
policzku. Sanitariusz sprawdził go bardzo dokładnie i~założył opatrunek.

Inni policjanci patrzyli dookoła nieśmiało, gdy Bianchi i~karetka
odjechały. Przypadły im do wykonania zadania administracyjne, zebranie
dowodów, zrobienie zdjęć, zajęcie pojazdu. W~porównaniu z~biciem, które
trwało, kiedy Amerykański Orzeł przybył na miejsce, było to nudne i~biurokratyczne, ponieważ teraz fotografowali rozbryzgi krwi i~robili
poważne notatki o~narkotykach, które znaleźli na centralnym panelu
czarnego mężczyzny.

-- Co to jest panel centralny? -- spytał Amerykański Orzeł, podsłuchując
szeptaną rozmowę z~odległości dziesięciu metrów.

Dwóch policjantów, którzy rozmawiali, wyglądało na jeszcze bardziej
zażenowanych. 

-- To, to jest panel -- powiedział jeden z~nich. Wskazał na
oparcie pod łokieć, które oddzielało fotel kierowcy od fotela pasażera.

-- On miał tam narkotyki na widoku?

-- W~torbie. -- Policjant podniósł dużą torbę zioła.

-- Ale to jest zaokrąglone na górze. Cokolwiek tam położysz, to spadnie.

-- Może było na fotelu pasażera.

Partner policjanta powiedział: 

-- Nie, konsola centralna. Widziałem.

Amerykański Orzeł wpatrzył się w~niego. Znowu się unosił.

-- Słuchaj -- To był inny policjant, ten, który rozmawiał z~kierowcą
holownika przez telefon. -- Słuchaj, Orzeł. O Gioffim, wiesz, Bianchim?
On nie jest złym facetem. Jest dobry, jego tata zrobił karierę,
Departament Policji Nowego Jorku. Jest wolontariuszem w~Małej Lidze.

-- Facet był formalnym Skautem Orła -- powiedział ,,Centralny Panel''. -- Dosłownie.

Obaj patrzyli znacząco na godło Amerykańskiego Orła.

-- Do którego szpitala pojechali? -- spytał Amerykański Orzeł.

Powiedzieli mu, po ponurej pauzie. Wzbił się i~jego supersłuch
dosłyszał, jak ,,Nie jest zły'' mówi ,,Centralnemu Panelowi'', że tylko
pedał lub zbok latałby dookoła w~ciasnych rajtuzach, to było odrażające.

Wylądował na parkingu dla karetek i~odnalazł sanitariuszy, którzy
zajmowali się tamtym czarnym człowiekiem, sączących kawę, czekających na
kolejne wezwanie. Ostrożnie obserwowali jego podejście. To nie było
normalne. Większość ludzi dorastała z~Amerykańskim Orłem: komiksy,
kroniki filmowe, słuchowiska radiowe, filmy rysunkowe, zabawki,
adaptacje, wywiady, filmy edukacyjne, sztuka teatralna, musical, filmy,
powtórki, tematyczny park rozrywki. Nawet kryminaliści, którzy mieli
skończyć w~więzieniu, ponieważ właśnie pojawił się Amerykański Orzeł,
byli zafascynowani, mimo że przeklinali swój pech.

-- Cześć, chłopaki -- powiedział.

-- Heja. -- Sanitariusz był biały, umięśniony, wysoki, krótkie włosy i~miał naprawdę, naprawdę dobre buty do biegania.

-- Co do tego faceta, co go przywieźliście, tego, nad którym policjanci
popracowali, z~nogą?

-- Ta.

-- Chciałbym się upewnić, że wszystko z~nim w~porządku. Czy moglibyście
mi powiedzieć, czy został przyjęty i~zaklasyfikowany? Znacie jego
nazwisko? Tam było dość źle i~chcę sprawdzić co z~nim, powiedzieć mu, że
ma opcje prawne.

-- Ty jesteś prawnikiem czy superbohaterem?

Amerykański Orzeł zeznawał w~dosłownie setkach procesach karnych, nie
wspominając kilku procesów o~ludobójstwo w~Hadze. Posiadał więcej niż
wiek praktycznego doświadczenia prawnego, wiedział więcej niż połowa
profesury w~Yale. Wiedział również, że ten facet nie był zainteresowany
jego profesjonalnym doświadczeniem.

-- Czyli czy znacie jego nazwisko?

Sanitariusz wypił trochę kawy.

 -- Zasady Szpitala. HIPAA\footnote{ ustawa
U.S.A. dotycząca m.in. dostępu do prywatnych informacji pacjentów,
zob.~\url{https://en.wikipedia.org/wiki/Health\_Insurance\_Portability\_and\_Accountability\_Act}
-- przyp.tłum.}. Nie mogę podać. Sorry, Orzeł. -- Rozłożył ręce. To było
w porządku, ponieważ Amerykański Orzeł już przeczytał nazwisko z~rozmazanej kopii w~cieniach wnętrza karetki: Wilbur Robinson, który miał
silne parametry życiowe, kiedy był przekazywany od ratowników do
pielęgniarki na triaż. Orzeł również przeczytał dokumenty sierżanta
Gioffre Bianchi, które były obok Robinsona. Również miał dobre parametry
życiowe, a~małe cięcie na twarzy opisano jako ,,powierzchowne''.

-- Tak czy inaczej dziękuję za pomoc. Po prostu zapytam w~recepcji SOR. -- Skierował się do drzwi wahadłowych, ale ratownik go zatrzymał.

-- Tylko pracownicy, przez te drzwi. Wejście frontowe, \textit{proszę}. -- Na końcu zadziornym tonem.

Amerykański Orzeł przyciągnął spojrzenia, gdy wszedł do szpitala,
oczywiście, i~poczuł się źle, że przerywał pracę pielęgniarek i~lekarek.
Także dwóch policjantów, którzy stali przy biurku recepcji, intensywnie
się patrzyło na niego: właściwie, gapili się. Jeden z~nich przeklął
szeptem, dostatecznie głośno dla Orła, żeby usłyszał, ale zbyt cicho,
żeby ktokolwiek inny usłyszał. Amerykański Orzeł zapamiętał jego numer
odznaki w~mgnieniu oka.

-- Przyszedłem, żeby sprawdzić co z~pacjentem -- powiedział
recepcjonistce, która była realnie oszołomiona i~uśmiechała się jak
szalona.

-- Ojej -- powiedziała.

-- Przyjechał karetką. Wilbur Robinson. -- Zaczęła wklepywać.

Policjant, który przeklął, powiedział: 

-- Chwila. -- Pochylił się nad
recepcjonistką SOR-u. -- Ten pacjent jest podejrzanym w~sprawie karnej.
Żadna informacja nie może zostać przekazana bez pozwolenia Departamentu
Policji.

Amerykański Orzeł zmroził go. To nie był wzrok, z~którym był często
fotografowany, ale było wielu kryminalistów, którzy pod tym spojrzeniem
marnieli. 

-- Nie jest pan przedstawicielem szpitala, panie policjancie.

-- Jestem przedstawicielem organów prawa -- powiedział i~stuknął w~odznakę.

-- A ja, jak to się zdarza, jestem zastępcą szeryfa federalnego -- powiedział Amerykański Orzeł. -- Jaki jest stan pacjenta, proszę?

Recepcjonistka podjęła stukania w~klawiaturę. Spojrzała na ekran, na
Orła, na policjanta. 

-- Wezwę dozorującego lekarza, żeby powiedział -- powiedziała, co było inteligentnym sposobem zejścia z~widoku bez bycia
świnią wobec prawdziwego superbohatera. Praca na SOR-ze nauczyła ją
szybkiej dyplomacji.

-- Dziękuję -- powiedział Amerykański Orzeł.

Policjant wpatrywał się w~ich oboje.

Lekarka była pochodzenia azjatyckiego, a~jej nazwisko, Faroogi,
pozwoliło zgadnąć Amerykańskiemu Orłowi, że jej rodzicie byli
Pakistańczykami. Policjant zaczął mówić, zanim mógł powiedzieć słowo: 

-- Pacjent został aresztowany przez Departament Policji Nowego Jorku i~nie
ma prawa przyjmować gości.

-- Ne prosiłem o~spotkanie z~nim. Prosiłem o~rozmowę z~lekarzem.

Policjant spiął się do walki, co był odważne i~głupie, skoro Amerykański
Orzeł mógł dosłownie uderzyć go tak mocno, że przebiłby pięścią przez
pierś, pancerz i~resztę. W~Gdziekolwiek były amatorskie klasy boksu,
gdzie zmienił worki do bicia w~porwane łachmany w~żenującej eksplozji
piasku po całej podłodze. Nie było łatwo żyć ośmioletniej istocie
pozaziemskiej z~tajną tożsamością w~Gdziekolwiek w~stanie Ohio.

Policjant miał coś powiedzieć, ale lekarka się wtrąciła. Miała dryg
lekarki SOR, która dosłownie niczym nie była zaskoczona. Policjant
kłócący się z~superbohaterem jest tylko interesujący, jeżeli spałaś w~ciągu ostatnich trzydziestu sześciu godzinach i~jeżeli nie byłaś
wezwana, żeby improwizować środki, żeby usunąć dziwnie ukształtowany
obcy obiekt z~żenującego otworu.

-- Nazywam się doktor Farooqi. -- Drobna, schludna, z~podkrążonymi oczami,
krótkie, lśniące czarne włosy, skóra jaśniejsza niż jakakolwiek
Pakistanka, którą poznał Amerykański Orzeł, ktoś, kto spędza dużo czasu
pod dachem. -- Potwierdzam, że zasady Departamentu Policji nie są istotne
wobec moich możliwości przekazywania lub zatrzymania informacji na temat
stanu pacjenta. -- Rzuciła policjantowi martwe spojrzenie, które
pokazywało dokładnie zero chuja. Orzeł polubił ją. -- Istotne reguły są
zawarte w~HIPAA i~zasadach tego szpitala. Gdzie oba stwierdzają, że mogę
nie przedstawiać stanu pacjenta szeryfowi federalnego, chyba że istnieje
nakaz sądowy lub przekonująca, nagła potrzeba egzekwowania prawa, lub
chyba że pacjent wyrazi zgodę. Panie Orzeł, czy przypadkiem posiada pan
nakaz sądowy?

Amerykański Orzeł pokręcił głową. Policjant wyglądał na zadowolonego
siebie.

-- Rozumiem. I~rozumiem, nie istnieje nagła potrzeba organów ścigania.
Czy to się zgadza?

Orzeł skinął głową.

-- Dziękuję. Teraz zapytam mojego pacjenta, czy wyraża zgodę na
podzielenie się informacjami z~Tobą.

Policjant podskoczył. 

-- Czekaj\ldots  -- Ale doktory już nie było. Policjant
wyglądał, jakby chciał dogonić ją i~powalić, ale Amerykański Orzeł na to
by nie pozwolił.

Kiedy lekarka wróciła, powiedziała: 

-- Proszę za mną -- do Amerykańskiego
Orła i~poprowadziła go dalej korytarzem. Policjant zrobi dwa kroki za
nim, kiedy lekarka odwróciła się i~powiedziała: 

-- Proszę poczekać tutaj.

Lekarka poprowadziła Orła do gabinetu lekarskiego i~zamknęła drzwi.

-- Pan Robinson cierpi na rozległe obrażenia lewej nogi, które będą
wymagać wielu operacji chirurgicznych, lat terapii i~może już nigdy nie
chodzić bez niewygody. Ma lekkie wstrząśnienie mózgu i~pęknięcie
oczodołu lewego oka. Na szczęście, żadne odpryski kości nie zostały
wprowadzone w~oko. Jego nos jest złamany i~stracił dwa zęby. Wszystkie
palce w~prawej dłoni są złamane i~zostały umieszczone w~gipsie. Ma
złamane jedno żebro i~mniej poważne stłuczenia i~otarcia na większych
partiach ciała.

Amerykański Orzeł widział wiele nieludzkich zachowań ludzi wobec ludzi w~różnych strefach wojennych przez dekady, nawet musiał sprzątać, gdy
jeden z~,,dobrych facetów'' stracił głowę i~zrobił coś nie tak dobrego.
Ale to dotknęło go inaczej. To tutaj nie zdarzyło się na polu bitwy w~chaosie wojny, to zdarzyło się na małym prywatnym parkingu na Staten
Island w~świetle dnia, popełnione przez grupę facetów, którzy mogli
zatrzymać jeden drugiego, ale zamiast tego krzyczeli ,,Nie stawiaj
oporu'' na potrzeby kamer osobistych.

Rozluźnił pięści i~lekarka nieco się uspokoiła. Nie zamierzał straszyć
doktorki, która mu pomogła.

-- Czy może Pani przekazać mu wiadomość?

Lekarka skinęła głową.

-- Będę zeznawała na jego korzyść.

Doktorka znowu kiwnęła głową. 

-- To wszystko?

-- Tak. Nie. Chwila. Proszę mu powiedzieć, że to było złe i~że będę
zeznawał na jego korzyść.

-- W~porządku.

Wyszedł innym wyjściem, wychodząc na parking karetek bez mijania
policjanta. Wzniósł się pionowo i~wykorzystał widzenie rentgenowskie,
żeby obserwować wjeżdżanie windy z~lekarką, potem śledził drogę lekarki
przez korytarze siódmego piętra, aż weszła do pokoju Wilbura Robinsona.
Orzeł zapamiętał pokój, a~potem poleciał w~chmury, żeby uciec od
wścibskich oczu ludzi na parkingu, którzy patrzyli w~górę i~pokazywali
go palcem.

Słońce późnej zimy zaszło wcześnie, Orzeł wrócił znad chmur i~unosił się
nad zachodnim skrzydłem szpitala, skanując pokój Wilbura Robinsona,
zanim wylądował na dachu i~użył swojego zimnego oddechu do zamrożenia
mechanizmu zamka drzwi na dachu, a~potem odsunął zapadki przy pomocy
cienkiego kawałka metalu. Już w~środku, ogrzał zamek z~powrotem swoim
gorącym wzrokiem i~upewnił się, że zapadka wpadła na swoje miejsce.
Przeskoczył przez balustradę schodów i~powoli opuścił się na siódme
piętro, unikając kamer ochrony wycelowanych w~same schody. Na siódmym,
rozmył się zbyt szybko, żeby ludzkie oko mogło podążać, do pokoju
Wilbura Robinsona, łapiąc po drodze garść ołówków z~kubka przy biurku
pielęgniarki. Rozrzucił je przed policjantem stojącym na zewnątrz pokoju
Robinsona, wykonał szybki zwrot, który posłał nagły podmuch wiatru
wzdłuż korytarza, a~gdy policjant schylił się, żeby spojrzeć oszołomiony
na ołówki, przemknął korytarzem i~koło niego do pokoju Robinsona.

Robinson patrzył na niego jednym okiem i~uśmiechał się ironicznym,
posiniaczonym uśmiechem. 

-- Ty, co?

-- Czy lekarka przekazała Ci moją wiadomość? -- spytał Orzeł, patrząc na
dziwne skręty eleganckiego wąsika Robinsona, teraz garbatego na
opuchniętych wargach, jak rozgnieciony robak na chodniku po deszczu.

Robinson kiwnął głową. 

-- Bardzo miło z~Twojej strony, dziękuję.

-- To nie w~porządku. -- W~głosie Orła wybrzmiał supeł.

-- Nie, ale też nie jest to niecodzienne. Dlaczego ja, Panie Orzeł?

Amerykański Orzeł wiedział, co Wilbur miał na myśli. Oglądał wiadomości,
czytał Twittera. Widział filmy nagrane na telefonie. 

-- Gdzieś trzeba
zacząć.

-- Czy mogę coś przewidzieć, wiesz, nie zamierzając obrazić?

Amerykański Orzeł kiwnął głową. 

-- Oczywiście.

-- Teraz, jesteś biały, ale to twierdzenie jest całkowicie warunkowe.
Znaczy, bez obrazy, nie jesteś białym człowiekiem, ponieważ nie jesteś
człowiekiem. Nawet nie jesteś mężczyzną, prawda?

To nie był nowy obszar dla Orła: superzłoczyńcy często wyśmiewali go
przy pomocy okrutnych domysłów na temat jego tożsamości płciowej i~gatunku. Jednak, tak czy owak, bycie białym nigdy naprawdę nie było tego
częścią. Wstrząsnęło to nim, podwójnie: najpierw, gdy zrozumiał, że
nigdy o~tym nie myślał, i~znowu -- wstrząs -- to naprawdę było dziwne, że
nigdy nie wyszło na jaw.

-- Nie jestem naprawdę mężczyzną, mniemam, nie w~sensie bycia ziemianinem
czy człowiekiem.

-- Również nie jesteś biały. Kiedykolwiek słyszałeś typów ,,dziedzictwo
zamiast nienawiści'', ten mit, że Irlandczycy byli pierwszymi
niewolnikami w~Ameryce? To interesująca grupka. Są tak blisko,
wiesz, tak blisko zrozumienia, że bycie białym jest czymś, co inni
ludzie wybierają za Ciebie. Ludzie są biali, jeżeli większa grupa
białych jest obdarzona tym byciem białym. Zaczyna się to przed
narodzinami i~trwa po śmierci. A co biali dają, mogą zabrać. Nie
rozumiesz tego, założę się, ale jesteś biały tylko przez grzeczność,
panie Orzeł, i~naprawdę jesteś tylko małym zielonym ludzikiem ubranym w~białość. Nie uwierzysz, jak szybko możesz być odbielony. Jesteś
prawdziwym nielegalnym obcym.

-- Panie Robinson, to, co się wydarzyło, jest niewybaczalne. Zrobię
wszystko, co trzeba, żeby sprawiedliwości stało się zadość.

Wilbur Robinson prychnął, potem się skrzywił. 

-- Muszę być uczciwy, panie
Amerykański Orzeł, czuję, że jeżeli tego spróbujesz, sprowadzisz na nas
znacznie większe kłopoty.

Amerykański Orzeł użył swojego gorącego wzroku, żeby roztopić uszczelkę
dokoła okna, ostrożnie je wysunął, potem stopił ponownie uszczelkę na
miejsce, gdy unosił się na siódmym piętrze. Potem poleciał do domu i~zrobił sobie obiad.

Amerykański Orzeł nie sypiał wiele. Cztery godziny na noc były
wystarczające, a~w~kropce mógł działać pięć lub sześć dni bez spania.
Jego planeta pochodzenia miała znacznie dłuższy cykl dobowy niż Ziemia.
Zwykle w~te długie godziny patrolował z~powietrza, ale dzisiaj wieczorem
usiadł, laptop otwarty, połączenie VPN do anonimizera na Brytyjskich
Wyspach Dziewiczych, które były preferowane przez niesamowicie udanych
oszustów finansowych i~zatem prawdopodobnie całkiem dobre przeciwko
agencjom ścigania prawa i~szpiegowania Stanów Zjednoczonych Ameryki.

Przeczytał, co fora dyskusyjne i~media społeczne mówiły o~nim. To nigdy
nie był dobry pomysł, a~jednak był przyciągany jak ćma do płomienia.
Wkrótce miał otwarte dziesiątki zakładek, przeładowywał je wszystkie i~przeglądał, szukając nowych wiadomości.

Nikt nie był pod jego wrażeniem. Tłum Blue Lives Matter mówił, że się
sprzedał i~stanął po stronie ,,bandytów'' zamiast organów ścigania.
Aktywiści antyrasistowscy mówili, że żadna liczba symbolicznych gestów
nie mogła odwrócić pokoleń pomagania w~sprawach białej przewagi i~amerykańskiego imperium. Nawet ludzie, którzy byli pełni sympatii,
myśleli, że był naiwny, wierząc, że może ,,rozwiązać rasizm poprzez
agresję''.

Potężnie pragnął wejść w~kłótnię i~się anonimowo bronić, ale powstrzymał
pragnienie. Spędził więcej niż wiek utrzymując swoją tożsamość w~tajemnicy od wzrastająco wymyślnego aparatu wywiadu rządu U.S.A. i~jego
przeciwników i~sojuszników, a~w~wieku masowej inwigilacji, był gorąco
świadomy, że było wiele sposobów, żeby narysować strzałkę na wskazującą
na tajną tożsamość i~linię łączącą ją z~Amerykańskim Orłem.

Ciągle restartował strony, ceniąc wysoko ludzi, którzy go bronili, nawet
jeżeli widział, że ich argumenty nie są zbyt dobre: mogły być
podsumowane do ,,Ale on jest \textit{superbohaterem}!''.

Świtało, gdy zobaczył wpis mówiący, że Wilbur Robinson jest przenoszony
karetką do Tombs. Nacisnął przycisk, który wyłączył wszystkie jego
urządzenia i~pobiegł cztery kilometry, unikając kamer CCTV, zanim wzbił
się w~niebo. Nigdy nie startował dwa razy z~tego samego miejsca, miał
program losujący, który wskazywał mu punkt startu tego dnia, unikając
jakiekolwiek stałej odległości od jego kryjówki. Wzbił się z~troposfery
i przekroczył barierę dźwięku z~uderzeniem, które zafalowało chmurami za
nim. Był nad Staten Island pięć minut później, śledząc drogę, którą
policyjna karetka by wybrała, żeby dojechać do Tombs. Znalazł ją
kilkadziesiąt metrów od więzienia i~wylądował bezszelestnie na dachu,
słuchając swoim supersłuchem, jak wózek inwalidzki był prowadzony po
pochylni karetki i~do więzienia. Bianchi pchał, a~Orzeł słyszał go
przeklinającego lekko pod nosem stałym monotonnym recitalem wszystkich
sposobów, w~jakie Wilbur Robinson zapłaci za wykroczenia.

Szept skończył się, gdy zastępca szeryfa wyszedł, żeby ich spotkać i~przejąć pchanie wózka. Twarz Wilbura Robinsona była pustą maską, albo
był odważnie stoicki albo zbyt przerażony, żeby mówić. Amerykański Orzeł
poczuł znowu ten kolec, poczucie pilności i~nieprawości. Wykorzystał
swoje widzenie rentgenowskie, żeby prześledzić przejazd wózka do wejścia
więzienia, obserwując, jak Bianchi wchodzi do jakiegoś rodzaju
poczekalni i~bierze sobie kawę, potem przerzucił uwagę na Wilbura
Robinsona dokładnie w~chwili, gdy został rzucony na podłogę.

Bez świadomej myśli, Amerykański Orzeł skoczył do wejścia i~szarpnął
drzwi tak mocno, że zamek odleciał z~dźwiękiem karabinowego strzału.
Natychmiast, stał nad Wilburem Robinsonem, który leżał na twarzy na
podłodze, ciągle stoicki, noga i~gips wyciągnięte za nim pod dziwnym
skręconym kątem.

Strażnik, który właśnie wyrzucił go na podłogę, był chudy, młody i~biały, czerwony na twarzy od furii, która szybko zmieniała się w~szok.

-- Co robisz? -- ryknął Orzeł, myśląc \textit{głupie pytanie}.

Strażnik gapił się długo, potem znalazł słowa. 

-- Więźniowie muszą być
przeszukani. To zasady bezpieczeństwa. Ten więzień się nie
podporządkował.

-- Nie mogłem wstać -- powiedział łagodnie Wilbur Robinson z~podłogi.
Poklepał znacząco wielki gips na lewej nodze.

-- On ma złamaną nogę -- powiedział Amerykański Orzeł.

-- Więzień się nie podporządkował. -- Strażnik przełknął. -- Nie masz prawa
tu przebywać. Proszę wyjść.

-- Widzisz? -- powiedział Wilbur Robinson. -- Od ,,mogę selfie, żeby
pokazać dzieciom'' do ,,proszę wyjść'' w~minuty. Uważaj, Orzeł,
nie-białość jest zaraźliwa. Stąd to droga w~dół.

Amerykański Orzeł patrzył na strażnika. Strażnik zmiękł.

-- Możesz przeszukać go w~mojej obecności -- powiedział Orzeł.

Strażnik wyglądał, jakby zamierzał iść po pomoc, ale potem zacisnął
usta, pokręcił głową, przyklęknął przy Wilburze Robinsonie i~przeszukał
go.

-- Byłem w~szpitalu -- powiedział Robinson, gdy strażnik go dotykał. -- Mam
na sobie szpitalną koszulę. Jak myślisz, gdzie chowam kontrabandę?

Jakby odpowiadając, strażnik wyciągnął małą latarkę i~włożył ją do ust -- była pogięta od małych śladów zębów -- i~uniósł koszulę Robinsona. Orzeł
opuścił koszulę tak szybko, że podmuch potargał krótkie włosy strażnika.

-- To nie będzie konieczne. Pan Robinson nie jest agresywny.

-- Taka jest procedura -- powiedział strażnik bez wyjmowania latarki.
Orzeł pomyślał o~prawdopodobnych problemach wynikających z~trzymania
narzędzia wykorzystywanego do rewizji różnych jam w~ustach i~skorygował
na minus opinię o~inteligencji strażnika.

-- To nie będzie konieczne -- powiedział Amerykański Orzeł i~podniósł
strażnika. -- Zabierzmy pana Robinsona do szpitala więziennego, dobra?

Kolejny pojedynek na spojrzenia. Orzeł był w~nich dobry. W~końcu, jego
gatunek nie wykorzystywał wzroku w~sposób ludzi, zatem mógł patrzeć się
na sposoby, które były dowolnie niepokojące. Błony migawkowe też
pomagały. Orzeł wygrał.

Strażnik przyklęknął, żeby posadzić Robinsona w~wózku inwalidzkim.
Amerykański Orzeł podniósł Robinsona jak dziecko i~położył go delikatnie
w wózku, potem przeszedł ze strażnikiem całą drogę przez wiele
wewnętrznych zakrętów więzienia. Więźniowie i~strażnicy podobnie
patrzyli za nimi i~Orzeł wysłuchiwał ich szeptanych rozmów, ponieważ
każdy, kto szepcze wokół obcego z~legendarnie superludzkim słuchem, nie
powinien rozsądnie oczekiwać prywatności. To nie tak, że to, co mówili,
było jakoś interesujące.

Obserwował Robinsona w~trakcie badań lekarskich, wychodząc z~pokoju na
część, żeby zapewnić mu prywatność, kiedy był nagi, używając
supersłuchu, żeby określić, kiedy może wrócić. Poczekał, aż Robinson
został położony w~szpitalnym łóżku i~wtedy pożegnał się z~nim i~zapewnił, żeby się nie martwił, że wróci. Robinson przewrócił oczami.

Amerykański Orzeł upewnił się, że wpadnie na Bianchiego w~drodze do
wyjścia z~więzienia. Bianchi szarpnął się na bok, kiedy ujrzał Orła,
jakby unikał uderzenia. Wyglądał strasznie, worki pod oczami,
nieogolony, zmięty.

-- Dlaczego Robinson jest przetrzymywany?

Bianchi udawał, że nie usłyszał, zatem Orzeł zrobił krok do przodu -- kolejne wzdrygnięcie -- i~Bianchi się zebrał się w~sobie.

-- Dlaczego Robinson jest przetrzymywany?

Bianchi minąć obojętnie Orła, który przesunął się, żeby go zablokować. 

-- Nie wyjdziesz, póki nie odpowiesz mi, dlaczego jest przetrzymywany. -- Technicznie to była przeszkadzanie, ale Orzeł był szeryfem federalnym, a~ten facet był lokalnym gliną. Orzeł miał rangę.

Bianchi wpatrywał się w~niego.

Orzeł wzruszył ramionami.

-- Robinson został zatrzymany za utrudnianie działania administracji
rządowej, nielegalne posiadanie marihuany, napaść drugiego stopnia,
napaść trzeciego stopnia, blokowanie ruchu pieszego i~włóczęgostwo.

Kiedyś Amerykański Orzeł wleciał w~serce ziemskiego słońca, gdzie
światło, które powstało pięćdziesiąt tysięcy lat temu, zaczynało powolną
walkę z~ogromną grawitacją, żeby dotrzeć do powierzchni Słońca i~promieniować ku Ziemi. To, co ujrzał tam, było cudem i~dziwem natury, aż
zapomniał słów.

Było mniej zaskakujące niż lista zarzutów przeciwko Robinsonowi. Kiedy
Amerykański Orzeł kogoś łapał, robił to z~dobrego powodu.

-- Przepraszam, czy możesz powtórzyć?

-- Wypierdalaj mi z~drogi, zanim \textit{Cię} zaaresztuję.

\textit{Chciałbym zobaczyć, jak próbujesz}. Zrozumiał, że od samego
początku to \textit{było} to, co chciał, żeby ten facet spróbował.
Amerykański Orzeł spędził całe swoje życie, jego wieki, prowokując
łobuzów, żeby ,,znaleźli kogoś swojego rozmiaru'', co zawsze było
brednią, ponieważ żaden \textit{Homo sapiens} nawet nie dorównywał
Amerykańskiemu Orłowi. Niemniej jednak ogromnie satysfakcjonujące było
ujrzenie miny łobuza, kiedy docierało do niego, że był w~obecności
nieporuszalnego obiektu, który był także niezatrzymywalną siłą.

W końcu, władze lub armia pojawiłyby się i~zabrały łobuza, a~Orzeł
skoczyłby w~niebo i~wrócił do swojej kryjówki, całkowicie pewien, że
zrobił swoją część dla ludzkości, którą uznał za własną. Tym razem, nie
byłoby władz, żeby zabrać łobuza. Nie mógł blokować Bianchiego, póki
Zespół Wydziału Dyscyplinarnego nie pojawiłby się i~nie pozbawił go jego
odznaki, broni i~honoru. To mogło nigdy nie nastąpić. Jedynym wyjście, w~jaki to starcie mogło się zakończyć, było wycofanie się Amerykańskiego
Orła.

Tak zrobił.

Bianchi skrzywił się i~przepchnął się koło niego, a~Amerykański Orzeł
skoczył w~niebo i~wrócił do kryjówki.

Na przesłuchaniu w~sprawie kaucji, odszukał go Bruce.

-- Możemy zamienić słowo? -- spytał playboy-milioner. 

Sala sądowa była
prawie pusta, oczywiście, ale wszyscy patrzyli się na nich. Sędzia
nakazał opróżnienie sali sądowej chwilę temu, a~zatem był tylko obrońca
z urzędu, Wilbur Robinson, prokurator, Bianchi i~prawnik związku
zawodowego policjantów, trzech pozostałych policjantów z~pobicia,
urzędnik sądowy, reporter sądowy i~czterech strażników sądowych, którzy
zajęli nerwowe pozycje dookoła Orła, pytając się po cichu siebie, w~jaki
sposób dokładnie kurwa mają usunąć Amerykańskiego \textit{na litość boską}
Orła, jeżeli nie będzie chciał wyjść.

Wszyscy byli zadowoleni z~przybycia Bruce'a. Playboy-milionerzy
nakręcali Nowy Jork. Przez lata odbyła się pewna liczba aktów
kumplowania pomiędzy nim a~Orłem, publiczne przecinanie wstęg i~tak
dalej. Nie wiedzieli o~tajnej tożsamości (tajnej!) Bruce'a i~nie mogli
wiedzieć, że on i~Orzeł walczyli koło siebie przy kilku okazjach, byli
weteranami tej samej długiej wojny z~przestępcami i~superzłoczyńcami.

-- Wysoki Sądzie, czy mogę mieć chwilę z~moim przyjacielem? -- Bruce był
gładki jak węglowa nanorurka w~smoothie. Sędzia, który prywatnie
zajmował się sanktuarium dla zwierząt na Long Island Sound, którego
działanie zależało od dorocznego czeku od Bruce'a, obdarzył milionera
jednym ze swoich pobłażliwych uśmiechów i~uderzył młotkiem. Orzeł i~Bruce patrzyli sobie w~oczy, gdy wszyscy wychodzili z~sali.

Kiedy drzwi się zamknęły, Amerykański Orzeł wylądował, Bruce opadł w~jedno z~krzeseł prawnika za stołem oskarżonego.

-- To musi się skończyć. -- Głos rozkazujący, człowiek przyzwyczaił się do
wydawania rozkazów. Przynajmniej nie próbował wykorzystać kamiennego
głosu twardziela. Ten głos. Tak poruszony.

-- Zgadzam się -- powiedział Orzeł. -- To kończy się teraz, w~rzeczywistości.

Bruce przewrócił oczami. 

-- No weź, Orzeł, złożone problemy mają złożone
rozwiązania. Nie możesz bić rasizmu, aż sam dojrzy swoje błędy.

-- Nie planują zakończenia rasizmu w~całości. Jednak chcę zagwarantować,
że Wilbur Robinson będzie sprawiedliwie sądzony.

Bruce miał bystrą minę. 

-- A jak Robinson się z~tym czuje?

Amerykański Orzeł się skrzywił. Bruce był dobry w~docierania do istoty
sprawy. 

-- Odnoszę wrażenie, że myśli, że jestem naiwnym idiotą. -- Bruce
prychnął. -- I~może jestem. Jednak również wskazał, że nie jestem
bardziej biały niż on, ale mogę udawać białego, a~to znaczy, że mogę
robić rzeczy, których on nie może robić bez ryzykowania zabicia. Zatem
zamierzam z~tego skorzystać. To zaczyna się od Wilbura Robinsona i~nie
jestem pewien, gdzie się kończy.

Bruce otworzył usta, ale Amerykański Orzeł (naturalnie) był szybszy i~powiedział pierwszy.

-- Ty, Bruce, ty też masz dużo z~tego przywileju. Używasz tego przywileju
do\ldots  -- ściszył głos, świadom magnetofonu reportera sądowego,
dopasowując szept tylko do uszu Bruce'a -- \ldots  ubierania się w~kostium i~bicia superzłoczyńców. -- Wrócił tonem głosu do normalnej głośności. -- Czy kiedykolwiek zastanawiałeś się, że w~tym czasie, są ludzie w~mundurach, którzy są, jako organ, w~takim samym stopniu superzłoczyńcami
jak najgorsi terroryści, których plany psułeś? Ludzie, którzy, jako
grupa, powodują tyle nieszczęść ile ktokolwiek na liście wrogów
publicznych?

Bruce wyglądał na zirytowanego. 

-- A teraz wracamy do pobicia rasizmu w~uległość.

-- Nie zamierzam bić nikogo ani niczego, Bruce. Tylko upewnię się, że ten
jeden facet dostanie uczciwy proces. Jeden niewinny facet, Bruce.

-- A potem co?

Amerykański Orzeł wzruszył ramionami. 

-- A potem zrobię to ponownie.
Namydl, spłucz, powtórz. Mam dużo czasu, Bruce. Setki lat. Pomyśl o~mnie
jako o~losowym audycie. Urząd Skarbowy nie musi sprawdzać wszystkich
zwrotów podatkowych, tylko wyciągnąć dostatecznie przypadków losowo,
żeby wszyscy inni nie przekraczali linii.

-- Co jeśli ktoś zabije Twojego ,,niewinnego'', ponieważ go wybroniłeś? -- Bruce zaczynał się denerwować, Amerykański Orzeł mógł to stwierdzić po
podczerwonym odczycie żył na skroniach. -- Czy Ty się w~ogóle słyszysz?
Myślisz, że Urząd Skarbowy zapewnia uczciwość ludzi? Człowieku, jestem
\textit{miliarderem}. Myślisz, że wypełniam zeznania ze strachu przed
skarbówką? Mogę sobie pozwolić na więcej prawników niż oni. Lepszych
prawników. Więcej i~lepszych. Płacę dokładnie tyle podatków na ile mam
ochoty.

-- Skarbówka nie ma żadnej władzy, Orzeł, ale powiem Ci, kto ma: NSA\footnote{
amerykańska agencja wywiadowcza koordynująca zadania wywiadu
elektronicznego czyli podsłuchiwanie telefoniczne, elektroniczne czy
radiowe, zob.~\url{https://pl.wikipedia.org/wiki/National\_Security\_Agency}
-- przyp.tłum.}. Myślisz, że Twoja tożsamość jest tajna? Myślisz, że
Twoje środki zapobiegawcze i~opsec\footnote{ ang. operations security -- zespół
czynności podejmowanych, aby rozpoznać zagrożenia i~zabezpieczyć ważne
informacje przed udostępnieniem ich niewłaściwym stronom -- przyp.tłum.}
powstrzymały ich przed odkryciem Twojej tożsamości? Nie powstrzymały. -- Pochylił się do Orła, pociągnął go do siebie, żeby wyszeptać nazwisko w~jego ucho. Jeżeli Orzeł pochodziłby z~gatunku, który bladł przy
zdenerwowaniu, zrobiłby się popielaty.

-- Tak, Orzeł, dokładnie. Pomyśl o~tym. Pomyśl o~tym, co jeszcze wiedzą.
\textit{Kogo} jeszcze znają. Zakładałeś, że musisz utrzymać swoją
tożsamość w~tajemnicy, żeby ochronić swoich bliskich przed przestępcami
i zagranicznymi szpiegami. Gdy zechciałeś utrzymać te sprawy w~tajemnicy
przed swoim własnym rządem, twój model oceny ryzyka zmienia się
kompletnie, a~właściwy czas, żeby rozwiązać to zagrożenie, minął
pięćdziesiąt lat temu. Nie każ mi przedstawiać całego obrazka.

Nie potrzebował go. Amerykański Orzeł mógł sobie wyobrazić Lois rzuconą
na podłogę jakiegoś więzienia przez jakiegoś sadystycznego chujka
policjanta, poddaną rodzajowi przeszukania, które powstrzymał. Mógłby ją
strzec, pewnie, może nawet wykraść ją pustkowie, ale \textit{a}) nie
zniosłaby tego, i~\textit{b}) wtedy nie mógłby chronić Wilbura Robinsona.

Przez chwilę był krótko i~potężnie wściekły: w~jaki sposób
najpotężniejszy byt na Ziemi mógł zostać tak łatwo przechytrzony? Ale
oczywiście, znał odpowiedź: ponieważ pozwolił sobie na troskę o~ludzi.
To było źródło \textit{wszystkich} jego problemów. Mógłby spędzić dziesięć
tysięcy lat w~sercu słońca, ale nie, wrócił na Ziemię, wtrącić się w~sprawy inteligentnego życia na tej planecie, która nigdy nie była jego
rodzinną.

Opanował emocje i~myśli. 

-- Nie pozwoliłbyś, żeby tak się zdarzyło,
Bruce.

-- Sądzisz, że mógłbym zatrzymać NSA?

Amerykański Orzeł wpatrzył się w~niego. Bruce był jednym z~najbogatszych
ludzi, jacy żyli, wykonawcą zarówno ,,białego'' jak i~,,czarnego''
budżetu Pentagonu, z~dostępami podobnymi do Kolegium Połączonych Szefów
Sztabów. Był właścicielem tajnej zbrojowni, oraz \textit{jawnej}
zbrojowni, a~korporacje, w~których miał pakiet większościowy,
dostarczały wywiadu jako wykonawcy całej listy agencji federalnych
wywiadu i~policyjnych, nie wspominając większość sił policyjnych
wielkich metropolii. Prawie na pewno dostawał prowizję, za każdym razem,
gdy algorytm uczenia maszynowego radził policjantom w~Staten Island
pojechać na brązową stronę linii Masona-Dixona i~zacząć zmuszać czarnych
ludzi wyjmować zawartość kieszeni i~zdejmować spodnie.

Bruce pierwszy odwrócił wzrok. Coś w~patrzeniu w~te obce oczy pokonywało
nawet najtwardsze ziemskie organizmy. Amerykański Orzeł kiedyś to
sprawdził z~wielorybem.

-- Zrobię, co będę mógł. -- Przełknął ślinę. -- Słuchaj, lubię Cię. Twoja
sprawa jest dobra. Ale nie możesz tego wygrać. Musisz to zrozumieć. Nie
możesz po prostu stanąć pomiędzy tym biednym facetem i~tym, co system
zamierza mu zrobić.

-- Mniemam, że dokładnie, co robię.

-- Zatrudnię mu prawnika. Już stracił pracę w~mieście. Kiedy wyjdzie,
zatrudnię go, dam mu lepszą pracę.

-- Miło mi to słyszeć.

-- Musisz to zatrzymać, zanim to zajdzie za daleko. -- Skrzywił się. -- Słuchaj, wiesz, że znam ludzi. Słyszałem, jak o~tym rozmawiali. Nikomu
nie podoba się pomysł kosmity wypowiadającemu wojnę amerykańskiemu
rządowi. To nie wygląda dobrze, Orzeł.

\textit{Teraz, jesteś biały, ale to twierdzenie jest całkowicie
warunkowe.}

Uderzyło Amerykańskiego Orła, że Bruce nazywał go ,,Orzeł'', ale
ostatnio nie wspominał części ,,Amerykański''. Bruce zwykł żartować z~niego, nazywając go ,,Yankee Doodle'' i~,,U.S.A. Today'', ale dzisiaj to
było tylko ,,Orzeł'', co zapewne coś znaczyło.

-- Po prostu walczę o~te same rzeczy, o~które zawsze walczyłem.

Bruce uśmiechnął się krzywo. 

-- Prawda, sprawiedliwość i~amerykański styl
życia?

-- Równa ochrona na mocy prawa, prawo do obrony osoby przed
nieuzasadnionym przeszukaniem i~zajęciem, okrutne i~niezwykłe kary nie
powinny być wymierzane\ldots 

-- Ok, dobra, Prawo Konstytucyjne, pierwszy semestr. Ale Orzeł, czy
naprawdę \textit{zawsze }o to walczyłeś? Wiem, że to było przed moimi
czasami, ale nigdy nie słyszałem, żebyś stanął pomiędzy psami Bulla
Connora\footnote{ amerykański polityk, który popierał segregację rasową i~negował prawa obywatelskie czarnych osób, znany z~tego, że kazał szczuć
psami w~trakcie demonstracji o~prawa obywatelskie,
zob.~\url{https://en.wikipedia.org/wiki/Bull\_Connor} -- przyp.tłum.} i~NAACP\footnote{ ang. The National Association for the
Advancement of Colored People -- Krajowe Stowarzyszenie na rzecz
Popierania Ludności Kolorowej, walczy o~zniesienie segregacji rasowej w~Stanach Zjednoczonych, poprzez zrównanie praw politycznych,
ekonomicznych, edukacyjnych, prawnych między czarną i~białą ludnością,
zob.~\url{https://pl.wikipedia.org/wiki/Krajowe\_Stowarzyszenie\_na\_rzecz\_Popierania\_Ludno\%C5\%9Bci\_Kolorowej}
-- przyp.tłum.}.

Bruce posiadał niesamowity talent do znajdowania miejsc, których
Amerykański Orzeł nie chciał, żeby odnalazł. Ale Orzeł uniósł policzek.

-- To dlatego, że mnie tam nie było. Ale powinienem być. A teraz jestem
tutaj. Co powiesz swoim wnukom, kiedy spytają, gdzie byłeś dzisiaj?

-- Powiem im, że próbowałem przekonać przyjaciela, żeby nie niszczył
swojego życia na bezużytecznej próbie złagodzenia swojej białej winy.

Orzeł rozejrzał się po pustej sali sądowej. \textit{Całkowicie warunkowe
twierdzenie}.

-- Zatem mniemam, że już tutaj skończyliśmy.

Bruce patrzył na niego dostatecznie długo, że Orzeł poczuł nagłą chęć,
żeby pierwszemu odwrócić wzrok. Potem Bruce opuścił sąd, a~chwilę
później, wszyscy inni wrócili. Sędzia i~strażnicy działali tak, jakby go
tam nie było. Ewidentnie zdecydowało traktować go jako mebel, jak
wyszukany motyw orła na zwieńczeniu.

Słuchał odczytania zarzutów i~zachowywał bierną minę przy każdym
absurdzie. Wydawało się, że to było minimum, które mógł zrobić, biorąc
pod uwagę, jak stoicko zachowywał się Wilbur Robinson. Bianchi był
ubrany w~mundur, który był napięty na bicepsach wielkości piłki nożnej,
a jego krawat widzialnie wbijał się w~byczy kark. Wyraźnie był to facet,
który więcej zastanawiał się nad treningiem na siłowni niż nad wyglądem.
Amerykański Orzeł złapał wzrok Bianchiego i~Bianchi uśmiechnął się do
niego.

-- Kaucja zostaje wyznaczona na pięćset tysięcy dolarów -- powiedział
sędzia. Amerykański Orzeł powstrzymał sapnięcie i~zobaczył, że Wilbur
Robinson zaciska zęby. Orzeł wiedział, że ta astronomiczna suma była
wynikiem jego własnych działań. Zrozumiał, że powinien był poprosić
Bruce'a o~wpłacenie kaucji. Ciągle mógł.

Obrońca publiczny sprzeciwił się połowicznie, został odrzucony na
miejsce, a~potem strażnicy sądowi zaczęli wyprowadzać Wilbura Robinsona.
Orzeł podążył za nimi. Udawali przez chwilę, że go nie widzą, potem
jeden z~nich odwrócił się, zebrał się w~sobie i~powiedział: 

-- Proszę
pana, czy może się Pan odsunąć? Mamy tutaj pracę do wykonania.

-- Tak jak ja -- powiedział Orzeł, ale ich puścił. Robinson, który
zachowywał się, jakby go tam nie było, odwrócił się i~spojrzał na niego
po raz ostatni.

Tego wieczoru Orzeł zadzwonił do kilku znajomych w~Departamencie Stanu,
mając nadzieję, że jeden z~nich dotrze do prokuratorów w~sprawie
Robinsona. Bolało go to, ponieważ chciał, żeby system był po prostu
sprawiedliwy, a~nie robił wyjątek. Przyznał, że raczej niż interweniując
w trakcie bicia, lub w~szpitalu, lub w~więzieniu, lub na rozprawie o~kaucję, mógłby pociągnąć za jeden czy drugi sznurek. Był Amerykańskim
kurwa Orłem i~wiele było osób, które były mu winne jedną lub dwie
przysługi.

Ale jak szybko odkrył, dowolne długi, jakie miał wobec niego lud Ameryki
i rządowi urzędnicy, mogły być łatwo anulowane. Wystarczyło, żeby
publicznie zadeklarował się jako wróg systemu amerykańskiego. Nie
rozumiał, że właśnie to robił, lecz było boleśnie jasne, że nie istniał
sposób oddzielenia systemu amerykańskiego od tego, czy czterech
policjantów na Staten Island mogło pobić na kwaśne jabłko Wilbura
Robinsona, skłamać, a~potem posłać go do więzienia.

Jego telefon zadzwonił: Lois.

-- Hej, kochanie.

-- Ciągle nie śpisz, co? Widziałam Twoje tweety o~modelu kolejek i~powiedziałam do siebie: ,,Lois, ten głupiec dawno powinien spać i~ktoś
musi mu nagadać''.

Roześmiał się. Faktycznie miał model kolejki, naprawdę wyszukany, a~jego
tajna tożsamość była członkiem wszelkiego rodzaju małych forów i~kółek
innych modelarzy, publikujących zdjęcia pod kątem ich stołów, małych
figurek produkowanych w~Niemczech. Autentycznie lubił swoje hobby i~był
w stanie pomalować figury z~dokładnością, która była, oczywiście,
dosłownie nadludzką. Tweetowanie zdjęć jego mini było jego
przyjemnością, nawet jeżeli, robiąc to, opsec stawał się trudniejszy
(miał randomizer, który kolejkował tweety, żeby pojawiły się w~trakcie
godzin dziennych w~Nowym Jorku, tak, że nie było podejrzanych przerw we
wzorcu, który był zbieżny ze znanymi misjami zagranicznymi
Amerykańskiego Orła). Lois przyzwyczaiła się do jego dziwnych
nieobecności związanych z~pracą i~czasem śledziła go na Twitterze, żeby
się dowiedzieć, czy był dostępny, co czasem oznaczało, że próbowała
dodzwonić się do niego, gdy rozbrajał bombę jądrową lub przekierowywał
rakietę w~stratosferę, zanim mogła uderzyć w~obóz dla uchodźców, żeby
nieszkodliwie wybuchła.

-- Jak Twój dzień, kochanie?

-- Uff. Chyba dobry. Kolejny dzień sprawdzania anonimów z~drugiego
źródła, kolejny dolar.

-- Biedactwo. -- Przeszedł do sofy i~wyciągnął się na plecach, ramię
zasłaniające oczy, wyobrażając sobie Lois, przypominając sobie zapach
jej włosów i~dotyk skóry. -- Tęsknię.

-- Ja też tęsknię. Jutro randka!

Zapomniał. 

-- Pamiętałem!

-- Jasne, na pewno. Ale w~porządku, ponieważ \textit{ja} pamiętałam i~załatwiłam nam stolik w~Big Carrot. -- Amerykański Orzeł miał problem z~trawieniem większości ziemskiego jedzenia, ale im niżej w~łańcuchu
żywności było jedzenie, tym łatwiej mógł je metabolizować. Wynalezienie
weganizmu było niesamowitym darem, a~moda na surowe jedzenie była
ratunkiem dla jego jelit.

-- Jesteś wspaniała.

-- Jestem. Wybierz koktail bar, gdzieś w~Flatbush lub nawet Park Slope,
gdziekolwiek z~hipsterem w~skórzanym fartuchu, który robi swoje własne
likiery, proszę.

-- Załatwione. -- Lois uwielbiała rzemieślnicze koktajle, a~on uwielbiał
być z~Lois.

-- Dobrze, dlaczego do cholery nie śpisz tak późno, co?

-- Praca -- powiedział, co powinno zakończyć dyskusję. On i~Lois mieli
umowę, że nie mówił o~swojej pracy. Ten zakaz doprowadzał wścibstwo
reporterki do szaleństwa, ale rozumiała też dyskrecję reporterską,
wiedziała, że mogłaby być zmuszona do ujawnienia rzeczy, o~których nie
wiedziała, i~że dwoje może utrzymać coś w~tajemnicy tylko jeżeli jedno z~nich jest trupem. Była to dziwna, ale wytrwale uczciwa podstawa dla
trwającego związku pomiędzy zawodowo wścibską osobą i~profesjonalnym
posiadaczem tajemnic, ale tak czy inaczej, Amerykański Orzeł
podejrzewał, że sama mogła do tego dojść. Doszło do kilku pamiętnych,
uroczystych wspólnych konferencji prasowych z~gubernatorem, na których
wpadł na nią nadal ubrany w~kostium Orła i~była okropnie poufała i~chytra.

-- Nie możesz pozwolić, żeby to na Ciebie wpływało, Clark. Niech inne
osoby przejmą ładunek. Nie możesz robić wszystkiego. -- To były rzeczy
takie jak ta, które sprawiały, że myślał, że wszystkiego się domyślała,
ale nie miał zamiaru do niczego się przyznawać. Szczególnie nie na
niezabezpieczonej linii telefonicznej.

-- Masz rację, oczywiście, ale łatwiej powiedzieć, niż zrobić.

-- Słuchaj, skoro nie śpisz i~pierdzisz dookoła swoimi pociągami,
dlaczego nie wpadniesz? Mam deadline na zgłoszenia na o, którąś w~nocy,
zatem prawdopodobnie będę na nogach przez godziny i~przydałoby mi się
towarzystwo. Możesz zrobić kawę i~wymasować mi ramiona.

Mógłby tam być w~mniej niż dziesięć minut, nawet korzystając z~randomizowanego biegu do niepowtarzalnego punktu startu. 

-- Do zobaczenia
za godzinę.

-- Dobry chłopak. Przywieź pączki.

Udawał sen na jej sofie, aż była gotowa pójść spać, potem pozwolił jej
,,obudzić się'' i~spędził kilka godzin, drzemiąc z~przytulony do niej,
jego twarz pogrzebana w~jej włosach. Miał innych ludzkich bliskich,
widział, jak się starzeją, stają się starożytni, umierają. Ujawnił się
niektórym z~nich, ale innych zostawił za sobą, fałszując swoją śmierć,
czasem z~pomocą jednego lub dwóch przyjaciół w~rządzie. Amerykański
Orzeł musiał wydzwonić wiele przysług. Zawsze szedł na ich pogrzeby, co
było dużym błędem kultury bezpieczeństwa, co mogło powiązać jego
tożsamości, ale to też była właściwa rzecz do zrobienia.

Obserwowanie Lois napełniło go nieopisywalnymi emocjami: smutkiem z~bycia ostatnim jego rodzaju, innym smutkiem z~wiedzy, że jego związek z~Lois zakończyłby się jej odejściem, podczas gdy on żyłby dalej,
gwałtowne uczucie ochrony wobec niej, o~którym myślał jako ,,miłości'',
ale którą wiedział, że nie mogła się równać z~tym, co Lois i~inni
uważali, gdy myśleli o~tej emocji.

Wysunął się z~sypialni i~dopasował słuch do zakresów FM, koncentrując
się, aż dostroił się do NPR\footnote{ang. National Public Radio -- niekomercyjny związek stacji radiowych,
zob.~\url{https://en.wikipedia.org/wiki/NPR} -- przyp.tłum.}.
Coraz częściej w~ten sposób odbierał swoje wiadomości: odbieranie radia
analogowego nie zostawiało żadnego cyfrowego śladu, który mógł być z~nim
powiązany, w~przeciwieństwie, powiedzmy, do dekodera telewizji kablowej,
który mógł wykryć jego zainteresowanie, gdy kiedykolwiek historie
dotyczące Amerykańskiego Orła były pokazywane w~wiadomościach. Pewnego
dnia radio stałoby się całe cyfrowe i~musiałby się nauczyć demodulować
je w~głowie -- niczym wykonywanie miliona prostych zagadek
matematycznych, ale wszystkie na raz i~w tej samej sekundzie -- albo
osłonić konsumpcję wiadomości przy pomocy narzędzi anonimizujących.

Wiadomości były zwyczajną mieszanką geopolitycznego chaosu, krajowych
rozgrywek i~promocji podcastów pomysłowo opowiedzianych osobistych
wspomnień. Lois zawsze słuchała profesjonalnym uchem, zapewniając plotki
z zaplecza o~innych reporterach, osobowościach, z~którymi mieli wywiady.
Pomimo jego długiego życia, Amerykański Orzeł nie miał jej perspektywy,
więc to, co usłyszał, po prostu wypełniało go pewnego rodzaju
melancholią. Nie było czasów, kiedy sprawy były lepsze?

Teraz rozmawiali o~nim. Kryminolog z~Uniwersytetu Nowego Jorku, historyk
wojny z~West Point i~krajowy reporter policyjny, w~panelu.

-- \ldots  rządy prawa, nie rządy człowieka. Zawsze jest to ryzykowne, jeżeli
pozwalamy komuś działać poza normalną ramą prawną, że zostanie raczej
samosądnikiem \footnote{ oryg. vigilante -- osoba samodzielnie dokonująca sądu,
samozwańczy stróż prawa, w Polsce wsp. ,,szeryf'' -- przyp.tłum.} niż
ochotnikiem. W~tym przypadku on nawet nie jest człowiekiem, więc kto
może powiedzieć, jak on myśli, wobec czego jest lojalny? Wiemy, że żyje
znacznie dłużej niż my, może to jest rodzaj nastoletniej rebelii\ldots 

Gospodarz się wtrącił. 

-- Pomijając spekulacje ksenobiologiczne, czy
możemy wrócić do pytania: co powinien \textit{zrobić} Nowy Jork? Czy
wielka ława przysięgłych powinna zostać wybrana, by oskarżyć ukochaną,
narodową postać? Czy burmistrz powinien zaprosić go do biura na cichą
rozmową jeden na jeden? Czy powinniśmy wszyscy podpisać petycję mówiącą
mu, żeby się wypchał?

Historyk wojny: 

-- Długie doświadczenia pokazują, że zawsze istnieją
niezaspokojone tłumy w~takich starciach, ale pokój zawsze jest osiągany
przez jakiegoś rodzaju pojednanie. Nie mamy możliwości uwięzienia
Amerykańskiego Orła, mimo plotek, że rzadkie metale mogłyby go osłabić.
W końcu będziemy musieli zawrzeć z~nim pokój. Przy tym, przypominam, że
on nie musi być \textit{Amerykańskim} orłem, nie ma prawa lub siły natury,
które mogłyby przeszkodzić mu staniem się Orłem Saudyjskim, Orłem Chin,
czy Orłem ISIS\ldots 

Reporter wydał niegrzeczny dźwięk. 

-- No weź, ten facet nie jest
obojętny, sprzeciwia się policyjnej brutalności. Myślisz, że zamierza
dołączyć do ISIS?

Gospodarz wspomniał o~czasie, mówiąc, że na razie muszą zostawić to w~tym miejscu, a~Amerykański Orzeł się rozstroił od radia. To, co naprawdę
chciał, to wiadomości o~Robinsonie, ale nikt nie opisywał sprawy
Robinsona i~wszyscy pisali o~nim.

Usłyszał poruszenie się Lois w~pokoju obok, zakładającej szlafrok,
korzystającej z~toalety, przychodzącej do niego. Był nieruchomy, udając,
że nie słyszał, tak, że mogła podejść do niego od tyłu, dotknąć w~ramię
i mógłby udawać, że był zaskoczony. To zawsze ją cieszyło i~pozwalało mu
zachować pozory bycia człowiekiem.

Zrobiła im kawy, dałam mu jego czarną, z~całusem w~policzku. Pachniała
wspaniale. Bardzo dużo ludzkiego jedzenia pachniało lepiej, niż
smakowała. Mimo to popijał kawę, zmusił się do uśmiechu i~mlaskania.

-- Nie możesz spać, co? -- Podmuchała na kawę. Zrozumiał, że
prawdopodobnie powinien zrobić to samo. Według standardów ludzkich, to
była bardzo gorąca kawa. Lois mogła go testować. Był żonaty z~kobietą,
która tak robiła, stulecie temu. Lubił je bystre.

-- Chyba staję się coraz starszy. Mówią, że dobry nocny sen jest pierwszą
rzeczą, jaką tracisz, potem włosy -- Złapał swoje faliste włosy i~je
pociągnął.

-- Jeszcze masz kilka dobrych lat przed sobą. -- Dotknęła jego dłoni,
ujęła podbródek i~spojrzała w~jego oczy. -- Chcesz o~tym porozmawiać?

Uciekł wzrokiem.

-- Ta, dobra. Ale pozwól, że opowiem Ci o~moich kłopotach. Może to ukaże
Twoje drobne zmartwienia w~perspektywie.

Zachichotał. 

-- Wal.

-- Mój wydawca przydzielił mi temat Wilbura Robinsona, ale w~rzeczywistości chce, żebym napisała o~Amerykańskim Orle. -- Orzeł nawet
nie mrugnął, choć jej oczy przyglądały się jego twarzy. -- Oczywiście, to
jest to, co wszyscy chcą czytać. Każdego roku w~Nowym Jorku są setki
Wilburów Robinsonów, ale tylko jeden pozaziemski superczłowiek, który
grozi policjantom i~sędziom.

-- Ale naprawdę chcesz napisać o~Robinsonie.

-- Naprawdę chcę.

-- Zatem zrób z~nim wywiad.

-- Nie mogę. Jest dostatecznie inteligentny, żeby nie rozmawiać ze mną
bez prawnika i~nie może pozwolić sobie na prawnika, więc nie rozmawia ze
mną.

-- Ach. Nie możesz go obwiniać.

-- Nie, nie mogę. Zatem skoro on nie chce ze mną rozmawiać i~skoro ja nie
chcę pisać o~Amerykańskim Orle, zdecydowałam się napisać zamiast tego o~patrolowaniu prewencyjnym\footnote{ oryg. predictive policing,
więcej~\url{https://en.wikipedia.org/wiki/Predictive\_policing}
-- przyp.tłum.}.

-- Co to takiego?

Uśmiechnęła się. 

-- Dobra, właśnie o~to chodzi: patrolowanie prewencyjne
jest powodem, dlaczego policjanci zbili Wilbura Robinsona na kwaśne
jabłko. Korporacje sprzedają policji software, AI ,,sztuczną
inteligencję'', która przegryza się przez wszystkie dane o~aresztowaniach od początku czasu i~tworzy prognozy, gdzie mogą zostać
popełnione przestępstwa. Argument brzmi, że matematyka nie kłamie i~matematyka nie jest rasistowska. Zatem rekomendacje powinny być
empiryczne, neutralne.

-- Brzmi, jakbyś w~to nie wierzyła.

-- Nikt z~odrobiną oleju w~mózgu nie uwierzyłby. Wszystko, co musisz
zrobić, to pomyśleć przez dziesięć sekund. Wszystkie dane, które ładują
do systemu, żeby przewidzieć przestępstwa, pochodzą od policjantów,
wobec których zakłada się, że mają swego rodzaju uprzedzenia, po
pierwsze dlatego, że są ludźmi, a~wszyscy ludzie mają uprzedzenia, a~po
drugie, ponieważ nowojorska policja ma długą historię łapania na
dyskryminacji rasowej, zresztą dlatego też w~ogóle kupują te rzeczy.

-- Ponieważ policjanci znajdują przestępstwo tam, gdzie szukają. Jeżeli
sprawisz, że każda czarna osoba, którą zobaczysz, musi opróżnić
kieszenie, znajdziesz wszystkie noże i~wszystkie narkotyki, które
dowolna czarna osoba nosi, ale to nie powie Ci niczego, czy czarni
ludzie są szczególnie skłonni do posiadania noży czy narkotyków,
szczególnie gdy policjanci wyrabiają normy, sami posiadając, żeby
podrzucić, jeżeli trzeba.

-- Co więcej, wiemy, że czarni ludzie są częściej aresztowani za rzeczy,
które białym ludziom uchodzą na sucho, jak ,,blokowanie publicznych
chodników''. Białym facetom, którzy zatrzymują się pod blokiem, żeby
zapalić, lub pomyśleć o~pracy, nikt nie każe iść dalej, nie daje
mandatów, ani nie przeszukuje. A czarnych facetów już tak. Zatem każda
okolica z~czarnymi facetami będzie wyglądać jakby miała epidemię
blokowania chodników, ale tak naprawdę to jest epidemia nadgorliwości
policji.

-- Ale teraz bierzesz te wszystkie zatrzymania i~mandaty, zamieniasz je w~,,dane', które są traktowane jako ekwiwalent ,,statystyk przestępstw''.
Jeżeli dane mówią, że ten adres przed osiedlem mieszkaniowym ma epidemię
,,blokowania publicznych chodników'' i~pod tym samym adresem występuje
epidemia posiadania narkotyków, to są dowody, że jest to gorące miejsce
przestępczości, a~nie dowód, że każdy czarny facet, który zatrzymuje
się, czekając na Ubera lub pogadać z~sąsiadem, zostaje zatrzymany i~przeszukany, a~potem aresztowany za posiadanie drobnej ilości
narkotyków, ponieważ, gdy tylko każą mu opróżnić kieszenie, jego skręt
jest ,,na widoku'' i~policjanci mogą za to nałożyć mandat.

-- Każ komputerowi przyjąć te wszystkie dane tak jak są i~zapytaj potem o~prognozy, gdzie będą przestępstwa i, bez zaskoczenia, komputer
wydedukuje z~niesamowitą intuicją maszyny, że policjanci znajdą zioło,
jeżeli każą każdemu wychodzącemu i~wchodzącemu do tego budynku opróżnić
kieszenie. Nieważne, że znalazłbyś tyle samo lub więcej zioła, jeżeli
poddałbyś takiemu traktowaniu wszystkich wychodzących lub wchodzących do
Trump Tower, ale nie masz ,,dowodów'' problemu z~narkotykami w~Trump
Tower, ponieważ potoczyłyby się głowy, gdyby rozmowa z~portierem na
Piątej Alei o~meczu kończyłaby się mandatem i~przeszukaniem, ale masz
zweryfikowany przez komputer dowód, że istnieje problem z~narkotykami na
południe od linii Masona-Dixona.

-- To dlatego Wilbur Robinson został zbity na różne odcienie gówna. Kiedy
nowojorski glina kogoś zatrzymuje, musi wypełnić formularz, UF-250, z~powodem zatrzymania. Są tam kratki do zaznaczenia jak ,,nieodpowiedni
strój'' i~,,podejrzany ruch'' i~tak dalej, i~jeżeli policjant zaznaczy
jedną z~tych kratek, zatrzymanie jest solidne.

-- Ale istnieje jeszcze większa bzdura, gliny mają kontyngenty, minimalną
liczbę tych ,,250-tek'', które mają zrobić, i~zgadnij, jeżeli nie jesteś
zbyt bystrym policjantem pragnącym wypełnić kwotę i~zdjąć sierżanta z~głowy, model patrolowania prewencyjnego powie Ci, gdzie masz iść i~szukać ,,nieodpowiedniego stroju'' i~,,podejrzanego ruchu''.

-- A Stowarzyszenie Policjantów tak ustawiło sprawy, że ci mało bystrzy
policjanci są nie do zwolnienia, dostają do obejrzenia nagrania ze
swoich kamer, zanim przekażą raporty, znają tożsamości demaskatorów,
którzy na nich skarżą, mogą zdecydować, kto i~jak długo może ich
przesłuchiwać, rzeczy, których inni podejrzani nie dostają. Zatem wariat
trafia na wariatkowa, długo po tym, gdy kosztują miasto setki tysięcy, a~nawet miliony w~ugodach z~ludźmi, których pobili, i~wymaszerowują,
uzbrojeni w~predykcje komputerowe i~bloczek ,,250-tek'', każą brązowym
ludziom wysiąść z~samochodu i~położyć ręce na ścianie.

-- Ten glina, który prowadził pobicie, Bianchi? Znalazłam dziesiątki
przypadków takich jak te w~archiwach gazety. Widać gołym okiem,
kompletny wariat, ale jego akta dyscyplinarne są tajne i~takie są ugody,
które załatwił dla miasta. W~ogóle nigdy nie powinien trafić do tej
pracy, ale ,,250'' i~patrolowanie prewencyjne zamieniły go w~rakietę
szukającą kłopotów. Zamierzał wysłać jakiegoś brązowego do szpitala, a~Wilbur Robinson po prostu przegrał na loterii.

-- Lois\ldots 

Uśmiechnęła się do siebie. 

-- Wiem, narzekam. Chodzi o~to, że jeżeli to
napiszę, wyląduje to jako wstępniak jako opinia. Ale to prawda. To
prawdziwa historia Gioffre Bianchi i~Wilbura Robinsona i~nie zrozumiesz
ich bez zrozumienia tego wszystkiego. -- Spojrzała się na niego bardzo
znacząco. -- Gdybym mogła wyjaśnić jedną rzecz Amerykańskiemu Orłowi, to
byłoby to: powinien wieszać urzędników miasta za ich kostki z~okien
ratusza, póki nie wyjaśnią, w~jaki sposób ta popierdolona, śmieszna
sytuacja, przeszła za ich kadencji. Potem powinien zrobić to ze
Związkiem Policjantów. Potem programistami. Rozumiem, w~jaki sposób taki
facet może skończyć, czując, że tylko poprawia coś na marginesie
problemu i~chce postawić temu granicę, ale stawianie jej przed Wilburem
Robinsonem jest sytuacją bez wyjścia.

Amerykański Orzeł rozważał starannie jej słowa. 

-- Wydaje mi się, że
Amerykański Orzeł powiedziałby, że trudno jest zmusić ludzi, żeby brali
Cię na poważnie, kiedy stoisz pomiędzy jednym z~obywateli, a~niesprawiedliwym systemem, natomiast jest znacznie trudniej, kiedy
biciem wymuszasz wyznania na urzędnikach policji.

Spojrzała się na niego jeszcze bardziej znacząco. Amerykański Orzeł się
nie poruszył, ale chciał.

-- Wydaje mi się, że pytanie brzmi, czy Amerykański Orzeł chce się poczuć
lepiej, czy coś zmienić?

~

Sędzia odrzucił wniosek Wilbura Robinsona o~kaucję, a~kiedy ludzie,
którzy przeczytali o~jego historii -- niektórzy dzięki Lois i~jej
artykułom -- zebrali pieniądze na adwokata do apelacji, znowu przegrał.
Niektórzy ujawnili Lois historię o~furii prokuratora na interwencję
Amerykańskiego Orła, ale nie mogła wyszukać drugiego źródła, zatem nic
nie opublikowała. Zamiast tego skończyło to na głównej stronie
\textit{Post}.

Amerykański Orzeł pojechał promem do Rikers, uśmiechając się do ludzi,
robiąc selfie, rozdając dzieciom autografy. Kiedy dotarł do wartowni i~poprosił o~spotkanie z~Wilburem Robinsonem, czekali na niego: dyrektor,
zastępcy, facet od PR, kilku nerwowo wyglądających strażników w~sprzęcie
SWAT.

-- Czy mógłby pan wejść do mojego biura na prywatną rozmowę? -- Dyrektor
był młody jak na pracę, z~rozbrajającym uśmiechem i~kilkoma wielkimi
bliznami ukazującymi się pod jego krótkimi włosami po IED, który zerwał
mu hełm, kiedy patrolował prowincję Kandahar.

Orzeł zastanowił się. 

-- Myślę, że chciałbym po prostu przejść przez
normalną procedurę odwiedzin, jeżeli to wszystko jedno, panie
dyrektorze. -- Mieli widownię, czarne kobiety z~dziećmi, w~większości,
odwiedzające ukochanych, przygotowujące się do poniżenia przeszukania, a~nawet kontroli osobistych.

-- Proszę pana, myślę, że byłoby lepiej, gdyby\ldots 

Orzeł wiedział, że gdy znajdzie się w~biurze Dyrektora, z~dala o~tych
kamer, zamieniłoby się to w~biurokratyczne bagno, on czekający,
czekający i~czekający, aż właściwi urzędnicy zostaliby wezwani do ich
telefonów, żeby przedyskutować wizytę z~nim, a~to dałoby mnóstwo czasu,
żeby wymyślić wymówki, dlaczego nie może odwiedzić Wilbura Robinsona,
lub nawet przetransportować Robinsona do innego więzienia.

-- Dyrektorze -- powiedział, pochylając się. -- Proszę dać mi zobaczyć pana
Robinsona teraz. Nie róbmy sceny. -- Dwóch facetów w~sprzęcie SWAT
spojrzało się z~lękiem na siebie. Wszyscy inni w~pokoju ucichli, nawet
małe dzieci przed odwiedzinami rodziców.

Dyrektor długo się w~niego wpatrywał. Istniało sławny film
Amerykańskiego Orła atakującego bunkier reduty niesławnego dyktatora,
stworzony z~opancerzonych kamer punktowych, które rzecznik PR w~Departamencie Obrony przekonał, żeby zabrał. Zaczął natarcie z~tuzinem
kamer i~zakończył tylko z~jedną, reszta zniszczona przez odłamki, kule,
miotacze płomieni i~fale uderzeniowe, gdy Orzeł bezlitośnie robił sobie
drogę w~bunkrze, odrzucając na bok strażników, pozwalając pociskom
odbijać się od jego obcej skóry, rozwalając pancerne drzwi gołymi
rękami. Chociaż kamery były niszczone, przesyłały obraz do opancerzonego
dysku twardego na jego pasku, a~ten wysyłał łączem do najbliższego
armijnego transportera, w~którym było studio wideo. Ani jedna klatka nie
została ominięta. Ostateczny wynik był zredagowany tak, aby dołączyć
jedenaście strzałów trafiających w~kamery, z~ostatnimi kadrami przed
zniszczeniem kamery w~maksymalnym zwolnieniu, gdy obraz kamery rozpadał
się i~znikał. Było to tak interesująca praca, że nawet najbardziej
zaciekli przeciwnicy awanturnictwa wojskowego U.S.A. przyznawali
niechętnie, że wzdrygnięcie się widzów współczujących kamerze było
imponujące. Orzeł wyleciał z~dyktatorem w~chwycie strażackim, twarz
mężczyzny odwrócona od czekających zewnętrznych kamer, prócz ostatniego
kadru strachu, którym kończył się film, zbliżeniem, żeby pokazać, że
mężczyzna dosłownie zmoczył swoje spodnie w~drodze od prywatnego pokoju
paniki na powierzchnię. Amerykański Orzeł dowiedział się, że powinien
oczekiwać powołania jako świadek w~rozprawie nad wojennymi zbrodniami
dyktatora, ale minęło osiem lat i~nadal czekał na wezwanie.

W końcu, dyrektor przemówił. 

-- Niech ktoś wpisze tę osobę do dziennika
gości.

Orzeł nie uruchomił wykrywacza metalu -- schował klucze i~telefon na
dachu wieżowca biurowego Jackson Heights w~drodze do Rikers -- ale i~tak
poddał się przeszukaniu. Dowódca, który go obszukiwał, próbował zachować
spokój, ale jego serce biło szybko, a~ręce drżały.

Zaprowadzili go do pokoju spotkań z~dwoma przykręconymi krzesłami po obu
stronach przykręconego stalowego stołu z~porysowanym blatem. Nie było
kamer, nawet tych małych, których człowiek nie mógł dojrzeć gołym okiem.
To był pokój, którego używali przy spotkaniach klienta z~prawnikiem.
Podczas gdy Orzeł czekał na przyprowadzenie Robinsona, eksperymentował
ze stołem, uderzając go, szukając punktów rezonansu. Do czasu przybycia
Robinsona, znalazł właściwe miejsce.

Robinson został wprowadzony na wózku przez strażnika z~kwaśną miną,
strażnik nic nie powiedział, gdy podjeżdżał z~Robinsonem do stołu,
blokując od tyłu hamulce wózka, potem wyszedł, ostentacyjnie zamykając
drzwi za sobą.

Orzeł spojrzał na Robinsona. Był wychudzony i~nieogolony, pachniał
nieświeżo. Nie łatwo było wziąć prysznic w~Rikers z~gipsem. Robinson
odpowiedział na spojrzenie.

Orzeł znalazł miejsce na stole i~zaczął uderzać kciukiem miękko i~szybko, tworząc szum ula, który odbijał się od kamiennych ścian. 

-- Kiedy
to robię, większość mikrofonów nie będzie w~stanie wychwycić naszej mowy
-- powiedział, mówiąc w~taki sposób, że Robinson mógł usłyszeć go ponad
buczeniem.

-- To fajny trick. Masz takich dużo.

-- O ile to warte, przepraszam.

-- Nie uderzyłeś mnie. -- Robinson przesunął się z~bólem w~wózku.

-- Przyznaję, że nie przemyślałem tego przed zadziałaniem. To, co
zobaczyłem, to mnie wkurzyło.

-- Jak myślisz, jak ja się czułem?

Orzeł się zamknął. Wiedział, że osobą z~prawdziwymi problemami tutaj był
Wilbur Robinson, a~nie kosmita, którego ludzie nazywali Amerykańskim
Orłem przez ponad wiek.

-- Pozwól, że opowiem Ci śmieszną historię. Ten glina, który mnie
okaleczył? Bianchi? Pierwszym razem, gdy go zobaczyłem, miałem
dziewiętnaście lat, a~on był w~OPU, Oddziale Przestępczości
Ulicznej.\footnote{
por.~\url{http://users.soc.umn.edu/~samaha/cases/diallo\_commandos.html}
-- przyp.tłum.}. To już nie istnieje, ale oni byli prawdziwymi
chuliganami. Mieli te specjalne t-shirty z~nadrukami, biały na czarnym,
ciasne, żeby pokazać te wszystkie mięśnie, z~cytatem Ernesta Hemingwaya:
,,Żadne polowanie nie przypomina polowania na człowieka, a~ci, którzy
polowali na uzbrojonych ludzi dostatecznie długo i~polubili to, nigdy
nie przejmą się niczym innym''. Te skurwysyny chodziły konkretnymi
ulicami konkretnego Nowego Jorku z~tym konkretnymi słowami na ich
konkretnych t-shirtach i~nikt w~dowództwie nie powiedział słowa. Nikt
nie wierzył w~raporty o~ich brutalności. Nikt nie uważał za stosowne
zapytać, dlaczego policjant w~czasach pokoju, zaprzysiężony, żeby
popierać prawo, mógł chodzić dookoła w~koszulach takich jak ta.

-- Ci panowie prawa, wymyślili nowy rodzaj ,,zatrzymania i~przeszukania'', nazwali go ,,gwałtem społecznym''. Gwałt społeczny jest,
kiedy idziesz ulicą, myśląc o~swoich sprawach, kiedy nagle pojawia się
ekipa tych typów OPU, może z~założonymi literackimi t-shirtami, i~łapią
Cię, rozbierają do naga, dokładnie tam na chodniku, jaja na wierzchu,
chuj na wierzchu, i~pochylają Cię i~smarują dupę i~wkładają palec w~rękawiczce, właśnie tam, na ulicy, przed Twoimi przyjaciółmi, na oczach
żony, na oczach Twoich \textit{dzieci}. Nigdy nie istniała grupa
heteroseksualnych mężczyzn tak mocno zainteresowana zawartością dupy
czarnych ludzi, niż ci policjanci z~OPU. ,,Gwałt społeczny'', uwierzysz
w to?

-- Ten sierżant Bianchi, wsadził palce w~wiele odbytów. Nie mojej, ale na
kwartale, był zawsze tym, który zrobiłby przedstawienie z~zakładania tej
niebieskiej rękawiczki, rozglądając się, upewniając się, że wszyscy
widzieli, co robił, szczególnie facet, które miał przeszukać. Jak w~Hiszpańskiej Inkwizycji, stare ,,okazywanie narzędzi'', upewnienie się,
że ofiara wie co się stanie, dając jej chwilę, żeby mogła o~tym
pomyśleć, \textit{oczekiwać}.

-- Sierżant Bianchi jest sadystycznym skurwysynem pierwszego stopnia i~wierz mi, że na komendzie o~tym wiedzieli. Ale ten matkojebca ciągle się
obnosił z~bronią i~odznaką, \textit{promocją}, a~w~tym tygodniu, okaleczył
mnie na całe życie i~teraz stoję wobec perspektywy więzienia,
prawdziwego więzienia, jeżeli nie zrezygnuję ze skargi przeciwko niemu.

-- Żeby ten skurwysyn robił to, co robi, musi być w~to zamieszane wiele
osób: prokurator, sędzia, oficer dowodzący, ten drugi policjant w~radiowozie. Bianchi nie jest zepsutym jabłkiem, jest prototypem, panie
Orzeł, modelem współczesnego żołnierza wyklętego\footnote{ oryg. bushwhacker -- partyzanci w~okresie amerykańskiej wojny secesyjnej, którzy nie
przerwali akcji wojskowych po zakończeniu wojny oraz byli oskarżani o~akty
bandytyzmu,
por.~\url{https://en.wikipedia.org/wiki/Bushwhacker} -- przyp.tłum.}, a~możesz to stwierdzić, ponieważ za każdym razem, gdy to
robił, dostawał awans. On działa w~ten sposób, jak sprawy \textit{powinny} działać, więc wybacz mi, jeżeli nie jestem aż tak zainteresowany, jak
się czujesz na temat niesprawiedliwości mnie leżącego na ziemi, podczas
gdy sierżant Bianchi i~jego kumple bili mnie prawie do śmierci bez
żadnego powodu.

Puls Robinsona był podwyższony, jego dłonie lekko drżały. Amerykański
Orzeł ciągle uderzał kciukiem w~stół, zastanawiając się, czy jest coś,
co mógłby powiedzieć, żeby nie pogorszyć spraw.

-- Przykro mi. -- W~końcu się odważył.

-- Tak, widzę, że tak jest. Ale ja nie jestem aktem żalu. Jestem ludzką
istotą, a~to jest moje życie, którym się bawisz.

Przemyślał to, naprawdę przemyślał. Amerykański Orzeł znał wiele
tajemnic i~widział rzeczy, których nikt nigdy nie widział, oraz rzeczy,
których bardzo, bardzo mało osób widziało, ale rzadko miał powód, żeby o~nich myśleć, nie w~ten sposób.

-- Naprawdę mi przykro. Nigdy nie spytałem, co chciałbyś, żebym zrobił,
to było głupie i~aroganckie. Czy mogę zapytać o~to teraz?

Robinson chrząknął i~wykrzywił zgrabny wąs w~lekkim uśmiechu. 

-- Właśnie
się domyśliłeś, co? Zdaje się, że bycie człowiekiem ze stali oznacza, że
rdzewiejesz w~mózgu.

Orzeł odwzajemnił uśmiech. 

-- Chyba tak. Przynajmniej, nie mogę z~tym się
nie zgadzać, biorąc pod uwagę, jak sprawy idą.

-- Cóż, sprawa jest taka, czas, żebyś mnie zapytał, czego chcę od Ciebie,
był \textit{przedtem}, zanim stworzyłeś tę sytuację. Powiedziałbym Ci,
żebyś to nagrał, wrzucił video na YouTube, przekazał mojego adwokatowi.
Teraz\ldots 

-- Ta.

Robinson uśmiechał się pięknie, nawet ze złamanym zębem i~opuchniętą
wargą. 

-- Słyszałem, że zbierają pieniądze dla prawnika. Ludzie, którzy
usłyszeli o~mnie, z~powodu tego, co zrobiłeś. Więc jest tak. Oczywiście,
mogłem nie potrzebować tych pieniędzy, gdybyś pomyślał, przed wtrąceniem
się. Ale pieniądze dla prawnika zawsze są lepsze niż brak pieniędzy i~brak prawnika. Obrońca z~urzędu, tylko powie, żeby się przyznać, żeby
zmniejszyć liczbę spraw.

-- Tak właśnie słyszałem. -- Amerykański Orzeł spojrzał na swój stukający
kciuk. -- Słuchaj, czy coś mogę Ci załatwić, coś przyjemnego, gdy czekasz
na apelację po rozprawie o~kaucję? Jedzenie? Pieniądze? Poduszkę?

Robinson pokręcił głową, potem pomyślał przez chwilę. 

-- Czy możesz
zabrać list dla mojej córki? Mogę się z~nią widzieć co drugi weekend i~minie mi nasze następne spotkanie.

-- Oczywiście -- powiedział Orzeł. -- Mogę Ci załatwić papier i~ołówek,
odebrać jutro?

-- Byłoby dobrze -- powiedział Robinson. Patrzyli się dziwnie jeden na
drugiego.

-- Naprawdę mi przykro.

-- Tak, wiem, że tak. Do zobaczenia jutro, panie Orzeł.

-- Do jutra, panie Robinson.

~

Amerykański Orzeł dostał informację, że zbierają się protestujący, gdy
słuchał przez radio raportu o~ruchu drogowym. Celowo uniknął ich, gdy
uczestniczył w~rozprawie o~kaucję, pozostając zbyt wysoko, żeby być
dostrzeżonym przez nieuzbrojone oko, aż mógł wylądować na dachu drapacza
chmur, który górował nad budynkiem sądu, potem przeskakując na dach sądu
w sekundy, cień na niebie, który zatrzymał się nagle z~zapasem
milimetrów, potem skorzystał ze schodów, żeby dojść do sali.

Ale protestujący nigdy naprawdę nie odeszli. Tłum przy budynku sądowym
został przyćmiony przez tłum, który maszerował od ratusza Staten Island
do komendy policji każdego wieczora w~godzinach szczytu, warcząc na ruch
drogowi i~uruchamiając chór klaksonów, który reporterzy od ruchu
drogowego uwielbiali mieć w~tle, gdy nadawali wywiady od wściekłych
kierowców, którzy chcieli potępić całą sprawę, od Wilbura Robinsona
(,,Policja tego nie robi, jeżeli zajmujesz się swoimi sprawami\ldots  kiedy
poznamy \textit{całą} historię o~tym, co się wtedy zdarzyło?''), do
protestujących (,,Te włóczęgi chcą pomóc Czarnym, dlaczego nie nauczą
ich jakichś umiejętności, żeby dostali pracę?'') i~szczególnie o~Orle
(,,Nigdy nie lubiłem gościa. Kto wie, kogo on naprawdę reprezentuje,
tylko się pytam? Pewne, że nie prawo i~porządek, tyle mogę wam
powiedzieć.'').

Ale protesty były każdego dnia coraz większe, nie mniejsze (,,Przysyłają
protestujących autobusami z~całego stanu, agitatorzy. Rosjanie nakręcają
tych ludzi fałszywymi wiadomościami na Facebooku.''), a~im dłużej trwały
protesty, tym większą stawały się koalicją: grupy solidarności z~imigrantami bez dokumentów, anarchiści w~maskach narciarskich, weterani
Occupy niosący hasła antykapitalistyczne, marksiści z~gazetami, których
nikt nie chciał czytać.

Ruch na Staten Island zamienił się w~niekończący się żart. Stojaki na
rowery dookoła portów promowych były zatłoczone, gdy dojeżdżający
przestali zabierać samochód z~domu na dojazd. Stragany z~hot dogami
robiły kokosy na miejscu. Gliniarze byli \textit{wkurzeni} i~aresztowali
dziesiątki ludzi za najbardziej bzdurne wykroczenia: śmiecenie, plucie
i, oczywiście, przeszukanie za ,,publiczne pokazywanie marihuany''.

Byli w~połowie drogi do Komendy, kiedy wylądował na dachu stacji paliw,
żeby się rozejrzeć. Wszędzie byli policjanci. Zauważył snajperów na
niektórych dachach (oni go nie zauważyli). Przenośna wieża strażnicza na
wysięgniku wyłaniała się nad skrzyżowaniem, które protestujący
wypełniali, ostentacyjnie nabita kamerami i~antenami, które bez
wątpienia zbierały informacje o~każdym działającym telefonie w~promieniu
dwóch kilometrów.

Okrzyk brzmiał: ,,BEZ SPRAWIEDLIWOŚCI NIE MA POKOJU'', była tam też
szkolna orkiestra dęta, białe mundury ze srebrnymi lamówkami i~wysokimi
czapkami, wybijający rytm dla krzyczących, mosiężne dzwony na rogach
wznoszące się jednym głosem, by wyrzucić nutę dla każdego wersu.
Dzieciak z~ramionami jak obrońcy ump,ump,umpił na suzafonie, przechodząc
po całej skali durowej, co nadawało maszerującym rytm ich krokom.

Orzeł chciał do nich dołączyć, ale oczywiście tego nie zrobił. Kiedy
jesteś Amerykańskim Orłem, wszystko, co robisz, Coś Znaczy, wiedział o~tym od wielu lat, które spędził pomiędzy ludźmi. Tylko raz, pozwolił
sobie zapomnieć o~tym i~spójrzcie, gdzie go to doprowadziło.

Protestujący wypełnili skrzyżowanie, odbijając się i~skandując, śmiejąc
się, szalejąc i~bezskutecznie próbując sprzedać marksistowskie gazety.

Orzeł mógł usłyszeć wysokie tony zaszyfrowanych jęków policyjnego radia,
wykryć ich wielkość, nawet jeżeli nie potrafił odszyfrować i~zrozumieć
wiadomości. Stopniowo stał się świadomy, że ruch radiowy narastał,
wznosząc się do pisku, który rezonował w~jego czaszce jak wiertło
dentystyczne. Rozejrzał się dookoła, zmieniając ognisko wzroku, żeby
zajrzeć w~długie aleje, ale nie potrafił dojrzeć niczego niezwykłego. W~końcu, wzniósł się w~niebo, tylko żeby dostać się na lepszą pozycję.
Wzniósł się wzdłuż tyłu apartamentowca, podążając za kierunkiem szybów
wentylacyjnych, gdzie nie było okien, potem usiadł na dachu i~zobaczył
źródło tej paplaniny: Departament Policji Nowego Jorku zgromadził armię.

Wzdłuż ulic Staten Island były stłoczone kolumny wozów opancerzonych
MRAP\footnote{ ciężki opancerzony wóz piechoty o~zwiększonej odporności na
miny, zob.~\url{https://pl.wikipedia.org/wiki/MRAP} -- przyp.tłum.}, bojowych wozów piechoty, które były praktycznie czołgami
bez działa. Te miały zamontowane armatki wodne i~wymowne talerze
,,Active Denial Systems''\footnote{ mikrofalowa broń
nieśmiercionośna,
zob.~\url{https://pl.wikipedia.org/wiki/Active\_Denial\_System}
-- przyp.tłum.} -- promienie bólu, które sprawiały, że czułeś się, jakby
Twoja twarz była palona od środka. Widział je używane na tłumach w~Afganistanie, obserwował tłumy protestujących depczących się wzajemnie,
gdy byli przytłoczeni nieznośnym uczuciem bycia pieczonym na żywo. Nie
był szczęśliwy z~bycia w~tej samej drużynie, co ludzie władający tą
maszyną, ale walczyli z~Talibami, a~to była drużyna w~której
\textit{zdecydowanie} nie chciał być.

Obserwowanie tych urządzeń kierowanych na pozycje, żeby użyć na
Nowojorczykach, zrobiło coś Amerykańskiemu Orłowi. Widział strajkujących
Nowojorczyków bitych przez Pinkertonów, obserwował powstanie Stonewall z~pobliskiego punktu widokowego, przyleciał z~Syrii, żeby pomóc wykopać
ocalałych i~ciała z~gruzów World Trace Center. Ratował uwięzione
sekretarki i~dyrektorów z~zablokowanych wind w~trakcie blackoutu i~wynosił ludzi z~fal podczas huraganu Sandy. Widział Nowojorczyków
przerażonych, odważnych, zbuntowanych, pobitych, szlachetnych i~tchórzliwych i~ani razu nie zastanawiał się, po której stronie stoi.
Teraz musiał wybrać stronę pomiędzy Nowym Jorkiem a~Nowojorczykami.

-- Gówno -- wyszeptał. Amerykański Orzeł nie przeklinał.

Obserwował kolumnę wyjeżdżającą zza rogu bloku Stuyvesant Place,
kilkadziesiąt metrów przed protestującymi. Kolejna kolumna wjechała na
pozycje za protestującymi, osaczając ich. Policjanci nowojorscy w~pełnym
ubiorze bitewnym -- pancerze, wizjery, tarcze, hełmy, wyraźne maski
gazowe -- maszerowali oddziałami pomiędzy MRAP-ami i~policyjnymi
autobusami za nimi, uderzając ich pałkami w~tarcze, skandując: 

-- CZYJE ULICE? NASZE ULICE! -- co było dość niesmacznym żartem, w~opinii Orła.
Ledwo ,,Służ i~chroń''.

Protestujący zatrzymali się i~się rozejrzeli. Muzyka umilkła. Zaczęli
wyciągać telefony, kierować ich soczewki w~policjantów, nagrywając i~streamując, wysyłając skowyt hałasu elektromagnetycznego, na który Orzeł
się skrzywił, póki go nie stłumił.

Dziecko zapłakało. Kolejne, a~potem jakieś młodsze dzieci. Policjanci
uderzali w~tarcze. 

-- CZYJE ULICE? NASZE ULICE!

Dzieciak z~basowym bębnem uderzył eksperymentalnie, potem jeszcze raz,
potem rozpoczął rytm uderzeń, BUM BUM BUM i~wszedł bębenek, a~potem
uderzyły cymbały i~\ldots  ,,CZYJE ULICE? NASZE ULICE!'' i~protestujący
zaczęli maszerować w~miejscu, dodając uderzenia stóp do wyzywającej
muzyki.

Kolumna przed nimi szarpnęła do przodu, MRAP-y z~ich wielkimi
kwadratowymi antenami przegazowujące silniki i~powoli ruszające, rzędy
zamaskowanych policjantów w~hełmach idących koło nich i~pomiędzy.

Z megafonu na wieży strażniczej na wysięgniku: 

-- ROZEJŚĆ SIĘ. 

Było to
tak głośne, że aż rezonowało w~zębach i~najwidoczniej śmieszne, ponieważ
protestujący byli otoczeni, bez możliwości pójścia do przodu i~tyłu.
Dźwięk sprawił, że dzieci znowu zaczęły płakać.

Zespół podjął tempo i~skandowanie się podzieliło ,,BEZ SPRAWIEDLIWOŚCI
NIE MA POKOJU'' i~,,CZYJE ULICE? NASZE ULICE!'' w~kontrapunkcie, razem
mieszające na ,,sze'' w~,,nasze'' i~,,nie'' w~''nie ma pokoju'',
przypominając funk.

-- ROZEJŚĆ SIĘ NATYCHMIAST.

Kolumna za protestującymi również zaczęła się posuwać. Skandowanie nie
było tylko buntem, było też odwagą, sposobem, żeby się zebrać, żeby
powstrzymać się od stawania się przerażonym tłumem. Bębnienie stało się
szybsze. Kamery telefoniczne wszędzie, wycelowane w~anonimowe wizjery,
bezlitośnie toczące się pancerze.

Rodzice zgarnęli dzieci. Niektórzy szukali ucieczki głęboko w~tłumie.
Inni stali na krawędzi, gdzie byłoby łatwiej uciekać.

Kolumny znowu się zatrzymały. Amerykański Orzeł mógł dojrzeć autobusy
toczące się na pozycje za policjantami. Wystarczająco autobusów,
pomyślał, żeby zabrać każdą z~tych osób do więzienia.

Pozwoliłby im zabrać Wilbura Robinsona do więzienia, zaufał, że jego
autorytet moralny by rozliczył system, zmusił go do rozpoznania uczciwie
osób, które był przyzwyczajonych do traktowania z~największą
niesprawiedliwością. Czy też powinien pozwolić policji zabrać resztę
tych osób, setki, do więzienia?

Jego ciało odpowiedziało, zanim jego umysł był w~stanie świadomie
sformułować odpowiedź, wylatując i~stawiając się pomiędzy protestującymi
a przednią kolumną. System ,,Active Denial'' nie byłby w~stanie go
zranić, tego był pewien.

Wszyscy przestali robić to, co robili i~gapili się na niego: skandujący,
muzycy, protestujący, gliniarze. Facet na pozycji snajperskiej był na
tyle przytomny (może nieprzytomny), żeby skierować karabin na
Amerykańskiego Orła. Orzeł nie polubił tego, nikt nie powinien celować z~broni w~zatłoczonym mieście. Nie w~amerykańskim mieście. Nie w~Nowym
Jorku.

-- Odłóż to -- warknął. Mógł musztrować lepiej niż najlepsi w~Marines, a~kiedy sterował głosem, mógł go podnieść do takiego poziomu, że ludzkie
uszy brzęczały. Teraz był tak głośny.

Snajper odruchowo szarpnął i~Orzeł dojrzał, jak jego palec zaciska się
na spuście, ale był w~stanie powstrzymać odruch, zanim wystrzelił.
Dobrze.

Orzeł przyglądał się policjantom na linii. Ich przekrzywione twarze za
wizjerem patrzyły beznamiętnie na niego. 

-- Ci ludzie bezpiecznie przejdą
do ich miejsca pokojowego protestu. Rozproszycie swoją kolumnę i~odstąpicie. Jeżeli nastąpi jakikolwiek wybuch niepokojów społecznych,
będzie wam wolno poradzić sobie z~nim w~wymierzony, proporcjonalny
sposób. Nie użyjecie tych broni obezwładniających.

Cisza była przerwana przez łomot helikoptera wiadomości nad głowami,
potem kolejnego, potem trzech helikopterów policyjnych. Kamery w~telefonach na ziemi śledziły kamery pod helikopterami nad głowami. Świat
obserwował. Amerykański Orzeł robił coś, co często robił, kiedy
nieregularne siły talibańskie próbowały powstrzymać dziewczyny od
chodzenia do szkoły lub wojska tajskie rzucały się na powstanie
demokratycznych aktywistów. Amerykanie wiwatowali na filmy z~tych
interwencji. Czy będą wiwatować, gdy zablokował ruch ,,najlepszych w~Nowym Jorku''? Mocno podejrzewał, że nie będą.

Niektóre dzieciaki w~tłumie wiwatowały, a~pośród dorosłych część
napięcia się rozproszyła. Nawet wydawało, że ulżyło kilku policjantom,
oceniając na podstawie niejasnego języka ciała. Tylko wariat mógłby
oczekiwać na to, co wydawało się nieuniknione przed chwilą.

Policjanci patrzyli na niego przez wizjery. Odpowiedział wzrokiem, a~jeden z~policjantów ,,Skinął'' mu głową, pełen szacunku gest
podziękowania. Znowu był Amerykańskim Orłem, bohaterem narodu, obrońcą
wszystkiego tego, co dobre i~prawdziwe.

Sieci informacyjne \textit{zarżnęły} go. Nikt nie miał zdjęć tłumu przed
pojawieniem się policjantów, więc zamiast tego użyli starych taśm z~protestów, które były agresywne, potem długich ujęć opancerzonych kolumn
policjantów kierujących się na protestujących, potem Amerykańskiego Orła
onieśmielającego Departament Policji swoimi nadludzkimi mocami,
pozwalając tym ewidentnym przestępcom odejść wolno.

Twitter aktywistów był nieco bardziej uprzejmy, ale nawet ludzie, którzy
chwalili go za interweniowanie, by poprzeć amerykańskie prawa przeciwko
jednej z~ich sił policyjnych, zawsze robili aluzję, albo mówili otwarcie
,,\textit{w końcu}'' w~ich tweetach.

-- To jest deklaracja \textit{wojny} -- powiedział krętacz w~radio. -- Ten
potwór z~gwiazd stylizuje się na \textit{Amerykańskiego Orła}, ale nie
jest Amerykaninem. Nie jest nawet człowiekiem. Nie jest lojalny wobec
naszej planety, naszego gatunku, nie mówiąc już o~\textit{kraju}.
Weszliśmy w~komitywę z~pozaziemskim stworzenie i~teraz musimy wymyślić,
jak się z~nim rozwieść. To nie jest komiks: nie mamy żadnego kryptonitu
pod ręką. To \textit{musi} stać się naszym priorytetem: nasz naród nie
może żyć bezpiecznie, gdy jest w~sporze z~tym niepowstrzymywalnym,
nieludzkim superbytem.

Twitter konspiratorów był katastrofą. Zawsze istniały ciche teorie
spiskowe o~Orle, od wieków: że był demonem wezwanym przez wolnomularzy,
by ujarzmić Amerykę, że był tajnym eksperymentem laboratoryjnym, który
nie wyszedł (lub w~niektórych wersjach, skończył się zgodnie z~planem),
że był efektem specjalnym stworzonym przez holoprojektory lub algorytmy
zmieniające wideo oparte na AI, a~ludzie, którzy twierdzili, że go
widzieli byli zahipnotyzowani, aktorami w~kryzysie lub w~ogóle sami
efektami specjalnymi.

Potem Bruce poprosił o~spotkanie i~przyniósł ze sobą mały, tablet
odłączony od sieci\footnote{oryg. air-gapped tj. fizycznie pozbawiony
możliwości podłączenia do jakiejkolwiek sieci przykładowo poprzez
wymontowanie komponentów -- przyp.tłum.}, na którym miał pdfy ze wpisów
w Intellipedia o~Orle, razem z~ich historiami edycji, gdy analitycy NSA
i dostawcy prywatni z~Booz Allen, Palantira i~SAIC zastanawiali się nad
ich własnymi teoriami spiskowymi o~jego wykorzystaniu jako tajnej
chińskiej (lub europejskiej, lub rosyjskiej, lub przez mafijnej) broni,
i co niektórzy mogliby zaoferować mu lub jak mogliby mu zagrozić w~celu
przejęcia.

-- To jest szaleństwo -- powiedział Orzeł. 

Bruce przygotował spotkanie w~jednym z~jego bezpiecznych domów, penthouse w~smukłym
dziewięćdziesięciopiętrowym budynku, który był prawie cały pusty,
zawierał tylko niezajęte apartamenty, które służyły jako sejfy dla
zamorskich kryminalistów. To było miejsce Bruce'a, zatem meble salonu
wystawowego rozsuwały się na boki, żeby udostępnić składy broni, mały
operacyjny teatr, oraz loch BDSM, który miał być odkryty przez
kogokolwiek, kto chciałby poszukać ciemnych sekretów kojca bogatego
playboy-milionera.

-- Nie-szalone, tylko zawodowo paranoidalne. Znasz tych ludzi. Spędziłeś
z nimi dekady. Co do kurwy sądziłeś, że się stanie?

Bruce nieczęsto przeklinał. To wzbudziło lęk w~Orle.

-- Bruce\ldots 

Podniósł ręce. 

-- Po prostu przestań. -- Odwrócił się, żeby wyjrzeć przez
okno, pokryte lustrem po drugiej stronie dla prywatności, i~spojrzał w~dal, ponad dachami śródmieścia Manhattanu. -- Wiedzą o~Lois. -- Wyszeptał
to i~normalny człowiek nie usłyszałby. Orzeł doskonale usłyszał.

-- Lois kto?

Bruce odwrócił się i~spojrzał ostro. 

-- Nie pierdol, Orzeł. Wiedzą, gdzie
żyjesz. Wiedzą, kim jesteś, gdy nie jesteś Amerykańskim Orłem. Wiedzą, z~kim spędzasz czas. \textit{Wiedzą} o~\textit{Lois}. -- Wymamrotał liczbę.
Dziesięć cyfr. Numer ubezpieczenia społecznego Amerykańskiego Orła, ten
ostatni, ten, który miał być pancerny. -- Oni są NSA, dupku. Myślisz, że
Twój opsec może ich pokonać? Rozgryźli to gówno \textit{lata} temu. To ich
\textit{praca}. Twoje małe gierki nie mogą pobić ich rozpoznawania twarzy,
ich analizy wzorców.

-- Skąd to wszystko wiesz, Bruce? -- Orzeł był wściekły, coś, do czego nie
był przyzwyczajony, ale co stało się ostatnio bardziej znajome. -- Nie
odpowiadaj na to, wiem: to dlatego, że \textit{Ty sprzedałeś im te
narzędzia}, prawda?

Bruce nie musiał odpowiadać.

Część jego wiedziała to, oczywiście, w~taki sam sposób, w~jaki wiedział,
że NRO\footnote{ National Reconnaissance Office -- jedna z~agencji Departamentu
Obrony Stanów Zjednoczonych, zajmująca się rozpoznaniem strategicznym,
zob.~\url{https://pl.wikipedia.org/wiki/National\_Reconnaissance\_Office}
-- przyp.tłum.} może odtworzyć jego loty i~gdzie były rządowe
laboratoria, gdzie pokolenia naukowców badało jego mieszki włosowe i~zrzucone komórki skóry. Jednak jedna sprawą było wiedzieć, inną
\textit{wyciągnąć wnioski}.

Inną sprawą zupełnie być przedmiotem planu ataku.

-- Słuchaj, Orzeł, to, co próbuję powiedzieć, to, że wszedłeś na drogę,
która nie może się ewentualnie skończyć dobrze. Teraz najlepszy
przypadek jest całkiem okropny, a~dalej jest tylko gorzej. Powiedziałem
Ci, że zatrudnię dobrego prawnika dla Robinsona i~zatrudniłem.
Najlepszego.

-- A co z~tymi protestującymi, którzy będą mieli usmażone twarze przez
system ,,active denial''? Czy też dostaną prawników?

Bruce zrobił zbolałą minę. 

-- Wiesz, że nie tak to działa\ldots 

-- Wyobrażam sobie, że \textit{Ty} wiesz, więc zaufam Ci. Robisz je,
prawda? Po prostu kolejna usługa, którą Twój rodzinny biznes zapewnia
miastom, państwom i~narodom świata.

Bruce pokręcił głową. 

-- Ameryka ma prawo bronić się. Wolałbyś raczej,
żeby żołnierze w~Afganistanie używali AR-15? Czy policja powinna
strzelać gumowymi kulami? A może ostrą amunicją?

-- Może po prostu nie powinni strzelać do nikogo.

-- Czy zgłaszasz się na ochotnika, żeby walczyć w~każdej wojnie Ameryki i~patrolować wszystkie jej ulice? Wiem, że jesteś nadczłowiekiem, Orzeł,
ale nie jesteś wszechmocny. Tak długo, jak amerykańscy policjanci i~żołnierze będą walczyć na wojnie i~z przestępczością, będą potrzebować
do tego narzędzi. Nie zamierzam pozwolić, żebym czuł się jak
superzłoczyńca za dostarczanie im tych narzędzi.

Amerykański Orzeł wiedział, co powinno się zdarzyć dalej. Powinien
zaakceptować, że nikt nie jest odpowiedzialny, jeżeli ktoś inny źle
wykorzystał rzeczy, które zrobił lub sprzedawał, to był argument
talibańskich wojowników, że CIA uzbroiło Afganistan i~to był argument
kolegów sierżanta Bianchi celujących z~promieni smażących twarz.

Słyszał ten argument tak wiele razy. Tylko bardzo nierozsądna osoba
prawdopodobnie mogłaby się nie zgodzić.

Nie był w~nastroju do rozsądku.

-- Wcale mnie nie obchodzi, czy czujesz się jak superzłoczyńca, czy nie.
To, co mnie obchodzi, to, że byli tam uzbrojeni policjanci gotowi
torturować pokojowych protestujących, w~tym dzieci, bronią, którą im
sprzedałeś.

-- Gdybym jej nie sprzedał, ktoś inny\ldots 

Podniósł ręce. 

-- Nie zamierzam nawet zaszczycić tego odpowiedzią, Bruce,
nie bardziej niż Ty, gdyby to był jakiś alfons wyjaśniający, dlaczego
zamienił nastolatkę w~prostytutkę na ulicy.

Gapili się na siebie. W~końcu Bruce westchnął i~odwrócił wzrok.

-- Jesteśmy po tej samej stronie. Dosłownie, staliśmy koło siebie,
walcząc razem. Nie mówię Ci tego, ponieważ chcę być dupkiem. Rozumiem,
skąd przybywasz. To wielki, chaotyczny, niedoskonały świat, a~my
jesteśmy chaotycznym, niedoskonałym gatunkiem. Wiem, że było wiele razy,
kiedy sam chciałem zacząć tłuc te łby, a~jestem człowiekiem. Nie
potrafię sobie wyobrazić, jak frustrujące to musi być dla Ciebie.

Amerykański Orzeł żył pośród ludzi, kiedy dziadek Bruce'a był jeszcze w~pieluchach. Kochał ludzi, ratował ludzi, walczył dla nich. Kiedy
superzłoczyńcy wyśmiewali go za bycie kosmitą, śmiał się na ich
niezgrabny gatunkizm, ich niepoparte przekonanie, że był niepewny co do
swojego statusu jako honorowego \textit{H. sap.} Dlaczego miałby się lękać
w tym zakresie?

-- Jestem całkiem dobry w~radzeniu sobie z~tym, Bruce. Mam dużo
doświadczenia.

Mina Bruce'a mówiła jasno, że ,,doświadczenie'' nie mogło zastąpić
dwudziestu trzech par chromosomowych zwiniętych lewostronnie.

-- Słuchaj, Orzeł, to, co próbuję Ci powiedzieć, to, że jeżeli Twoim
celem jest upewnić się, że Wilbur Robinson dostanie uczciwy proces, to
nie osiągasz tego celu. Jeżeli Twoim zadaniem jest zwrócić uwagę opinii
publicznej na odpowiedzialność policji, \textit{naprawdę} nie osiągasz
tego celu. Zastrzegam, że masz rację tutaj, ale nie chodzi o~posiadanie
racji, czy nie. Możesz mieć rację lub możesz być efektywny. Bycie
efektywnym oznacza nieprzypominanie całej ludzkości, że jesteś
niepowstrzymywalnym kosmitą, który przestrzega zasad, kiedy mu się
podoba. Gdy tylko zaczniesz przypominać im ten fakt, to skupi ich całą
uwagę.

-- Więc jak sądzisz, co powinienem robić?

Bruce znowu odwrócił wzrok. 

-- Musisz się wycofać. Tylko zniknąć. Na
jakiś czas. Lata. Dekady, może. Nie masz \textit{pojęcia}, jak oni są
spanikowani, tam w~świecie wywiadów. Mnóstwo osób nigdy Ci nie ufało.
Nie sądzę, żebyś rozumiał, jak bardzo przez te lata Twoi obrońcy
nadstawiali karki za Ciebie. To jest na razie skończone. Może na zawsze,
Orzeł. Naprawdę tym razem spierdoliłeś.

-- Wycofać się, co?

-- Jeżeli chodzi o~Lois, mogę to naprawić. Porozmawiam z~nimi. Nie jest
stroną walczącą i~jest z~prasy. Istnieje mnóstwo brzydkich możliwości,
jeżeli ruszą na nią. Ochronię ją. Będę też płacił za prawników Wilbura
Robinsona. Odwołujemy się od decyzji o~kaucji, wysłana dzisiaj i~naciskaliśmy na urzędnika, żeby przesunąć rozprawę na dzisiejsze
popołudnie. Prosimy o~pozwolenie na zabranie go do Mayo Clinic, żeby
ortopeda obejrzał jego nogę.

-- Uważasz, że w~taki sposób powinno to być załatwione, prawda?

Bruce skinął uroczyście głową. 

-- Uderzanie głową w~system wcale nie
zmieni systemu, tylko zapewni Ci ból głowy. Jeżeli chcesz
sprawiedliwości dla Wilbura Robinsona, to jest najkorzystniejsze i~najpewniejsze podejście.

-- A co z~sierżantem Bianchi? -- Znał odpowiedź.

Bruce skrzywił się. 

-- Jasne, ten facet jest gównem. Rozmawiałem z~Komendantem, zamierzają zaoferować mu wykup, wystarczający, żeby
przeszedł na wcześniejszą emeryturę. Nazwą to częściową niezdatnością z~powodu PTSD\footnote{ ang. posttraumatic stress disorder -- zespół stresu
pourazowego,
zob.~\url{https://pl.wikipedia.org/wiki/Zesp\%C3\%B3\%C5\%82\_stresu\_pourazowego}
-- przyp.tłum.}, wiesz\ldots  -- Spojrzał znacząco na Orła.

-- Zatem dostanie forsę i~odejdzie jako wolny człowiek?

-- Bez odznaki, bez broni, bez poparcia związku policjantów.

-- Bez wpisu do rejestru karnego. Bez problemu ze znalezieniem pracy jako
glina gdzieś indziej, lub w~prywatnej ochronie. Bez osądzenia, że zrobił
cokolwiek złego.

Bruce pokręcił głową. 

-- Bądź realistyczny. To najlepsze, na co teraz
możemy liczyć. Nie pozwól, aby lepsze stało się wrogiem dobrego.

-- Wezmę to pod uwagę.

Bruce złapał go za ramiona i~spojrzał mu prosto w~oczy. 

-- Nie spierdol
tego, Orzeł. Wiem, że robiłeś to przez długi czas, ale teraz jest
inaczej. Nikogo nie obchodzi, co robiłeś wiek temu, ale będą bardzo
długo pamiętać, co zrobiłeś tydzień temu. Nie pogarszaj tego.

Orzeł nic nie powiedział. Nigdy nie zbierał wycinków swoich dokonań do
albumu -- byłaby to cholerna wskazówka, gdyby ktokolwiek znalazł taki
album -- ale inni ludzie tak robili, i~nawet widział je sprzedawane na
eBay w~ostrej licytacji pomiędzy kolekcjonerami, którzy zapłaciliby
tysiące za album żółknących gazet i~podpisanych portretów, uratowanych
całych pokoleń, zła naprawionego, złoczyńców pokonanych.

-- Pomyśl o~tej historii, Orzeł, ,,Kiedyś był naszym bohaterem, a~potem
coś mu się stało i~stał się naszym największym zagrożeniem''. Jesteś
silniejszy niż ktokolwiek z~nas, możesz latać, widzisz przez ściany, ale
nie masz supermocy przekonywania. Ludzie, którzy są na Ciebie wkurwieni?
To jest \textit{ich} supermoc. Mogą przekonać ludzi do czegokolwiek. A Ty
ich wystraszyłeś. Nie chcesz stawiać swoich supermocy przeciwko nim.
Mają te rodzaje siły, których nie możesz pobić.

Orzeł przez lata spotykał propagandystów, pozwalając im nakłonić do
pozowania w~fotografiach, opóźniając misje, aż kamery mogłyby być
ustawione na pozycjach. Zawsze postrzegał ich jako nierealistycznie
pewnych swoich możliwości, ale z~drugiej strony, trudno było dyskutować
z wynikami.

-- Dlaczego NSA chciałaby stanąć w~obronie brudnych policjantów?

Bruce się uśmiechnął. 

-- To nie tak, że stają w~ich obronie, ale wyobraź
sobie środki zaradcze, którymi walczysz przeciwko ,,najlepszym'' Nowego
Jorku, gdyby były wykorzystane w~walce przeciwko tym samym systemom w~prowincji Kandahar. Jeżeli możesz unieszkodliwić te systemy, to musisz
zostać unieszkodliwiony.

To miało doskonały, straszny sens. Orzeł poczuł nieskrępowane poczucie
strachu i~wyrzutów sumienia, a~ostrożnie tłumiony strach, że spalił złą
stronę mostu, wypączkował.

-- Czy możesz zapewnić bezpieczeństwo Lois?

-- Proszę, nie rób niczego głupiego, Orzeł.

-- Ale, czy jest bezpieczna?

-- Orzeł\ldots 

-- Bruce, muszę wiedzieć, czy możesz zapewnić jej bezpieczeństwo.

Westchnął. 

-- Mogę zapewnić jej bezpieczeństwo.

Amerykański Orzeł skinął w~podziękowaniu i~wystrzelił się z~balkonu.

~

Demonstracja poparcia sierżanta Bianchiego była najbielszym, najbardziej
obskurnym zgromadzeniem, jakie kiedykolwiek widział w~Nowym Jorku:
wściekli faceci, starcy, garstka dzieciaków alt-rightu z~hipsterskimi
fryzurami, kilku podnieconych skinheadów, kilku szalonych zwolenników
noszenia broni, oraz weterani w~mundurach. Z~przodu, rzędy policjantów w~mundurach galowych, wydawali się nie mieć nic przeciwko przebywaniu w~tym samym tłumie co ludzie noszący afisze z~otwarcie rasistowskimi
hasłami. Nie wspominając o~zespole sztuki i~rzemiosła, który
przygotowywał kukłę Amerykańskiego Orła z~pasty mącznej i~farb
temperowych.

Obserwował ich z~dachu, czterdzieści pięter nad nimi, zbyt daleko, żeby
oczy ludzkie go dojrzały, dostatecznie blisko, żeby mógł odczytać z~ruchu warg, co skandowali: ,,HEJ HEJ TO NOWY JORK, OWADZI STWORZE IDŻ
JUŻ STĄD''. Nie słyszał frazy ,,owadzi stwór'' od lat osiemdziesiątych
ubiegłego wieku, a~nawet wtedy to był anachronizm, ale ci faceci
(wszyscy) byli w~drużynie anachronizmu.

Przemaszerowali od Ratusza do siedziby DHS\footnote{ ang. Department of
Homeland Security -- Departament Bezpieczeństwa Krajowego,
zob.~\url{https://pl.wikipedia.org/wiki/Departament\_Bezpiecze\%C5\%84stwa\_Krajowego\_Stan\%C3\%B3w\_Zjednoczonych}
-- przyp.tłum.} na Varick Street, policjanci ustawili się na stopniach,
patrząc na zewnątrz, gwardia honorowa za głośnikami, podczas gdy tłum
stał na placu, patrząc na nich i~słuchając, jak przemawiali przez
megafony. Widzowie patrzyli z~krawędzi, niektórzy drwiąc, inni
wiwatując.

W przeszłości zdarzały się marsze przeciwko Amerykańskiego Orłowi.
Asz-Szabab\footnote{ somalijska radykalna islamistyczna organizacja
terrorystyczna,
zob.~\url{https://pl.wikipedia.org/wiki/Asz-Szabab} -- przyp.tłum.} zorganizował masowe protesty uliczne -- obowiązkowe -- przeciwko jego obecności w~Somalii, wypełniając place tysiącami osób
krzyczących bez entuzjazmu, strzelających w~powietrze dla podkreślenia
ich potępienia.

Przeleciał przez niektóre z~nich, łapiąc kule gołymi rękami i~pozwalając
im opaść na scenę, robiąc w~powietrzu gwiazdy, które rozśmieszały
dzieciaki (i nawet niektórych dorosłych), doprowadzając bojowników do
furii, która byłaby przerażająca dla tłumu, gdyby nie głuchota
Amerykańskiego Orła na ich szał i~pociski oraz jego zdolność ochronienia
tłumu przed odwetem.

Czy tutaj mógłby to samo zrobić? Czy Nowojorska Policja otworzyłaby
ogień do niego tutaj w~Greenwich Village? Wątpił w~to. Nawet Bianchi
miał więcej rozsądku. Sierżant Bianchi lubił robić swoją brudną robotę,
kiedy kamery były wyłączone. Kiedy kamery rejestrowały, był przekonująco
pokrzywdzonym, oplutym żołnierzem, który chciał wypełnić swoje obowiązki
i teraz był za to karany.

Tyle teraz mówił, z~jego pozycji na szczycie schodów, w~megafon, który
otrzymał, i~choć nie był najlepszym mówcą, którego słyszał Orzeł, znał
swój tłum.

Orzeł wykrył kontrdemonstrację, kiedy była jeszcze dwieście metrów
dalej. Była znacznie bardziej brązowa niż tłum Bianchi, ale w~żadnym
wypadku całkowicie brązowa, nowojorski miks: wielorasowa, dobrze ubrana,
funky, z~afiszami, które były znacznie bardziej zabawne niż ,,OWADZI
POTWÓR MUSI ODEJŚĆ''.

Na przodzie, o~lasce, szedł Wilbur Robinson, jego twarz zastygła w~masce, która mogła być strachem lub mogła być determinacją.

Nadawał tempo, więc szli wolno, a~wkrótce radiowozy ich powstrzymywały i~wiadomość o~ich bliskim nadejściu dotarła do demonstracji policyjnej,
gdy jeden po drugim sprawdzali wiadomości tekstowe lub zbliżali głowy do
siebie, żeby porozmawiać.

Ludzie na ulicy rozpoznali Wilbura Robinsona, robili zdjęcia, niektórzy
dołączali do marszu. Masa przesuwała się powoli ku policjantom, a~kiedy
wyszli zza rogu, że oba obozy mogły się zobaczyć, wszyscy zauważalnie
się sprężyli. Bianchi przestał mówić do megafonu i~zmrużył oczy na nich,
zrobił minę, jakby właśnie poczuł pierdnięcie. Próbował podjąć mowę od
miejsca, w~której przerwał, grzebiąc w~notatkach. Tak czy inaczej, nikt
nie zwracał na niego uwagi.

Wilbur Robinson prowadził swoją kolumnę maszerujących ku masie białych
gliniarzy, nazistów, ukrywających się rasistów i~autorytarian. Naziści -- było ich tylko dziesięciu -- skandowali slogan, który zawierał słowo na
,,n'', przyciągając uwagę kamer wiadomości i~telefonów ludzi. Wszyscy
dookoła nich odsuwali się, nie chcąc być złapanym w~tym samym kadrze, co
ktoś otwarcie używający potwarzy, które były zarezerwowane dla
prywatnych sytuacji. Z~pozycji Orła, mała grudka nazistów w~białych
koszulkach polo z~czerwonymi opaskami wyglądała jak źrenica wielkiego
wrzącego oka demonstrantów propolicyjnych.

Dwa obozy były teraz dostatecznie blisko, że wyglądali, jakby mieli się
połączyć. Policjanci na służbie -- który chwilę wcześniej entuzjastycznie
słuchali niezdarnej mowy sierżanta Bianchi o~,,chronieniu uczciwych
ludzi'', ,,społecznych wojownikach sprawiedliwości, którzy puściliby
zwierzęta wolno'', ,,mówieniu jak jest'' -- zaczęli przesuwać się
pomiędzy nimi, żeby służyć jako bufor. Wilbur Robinson szedł powoli
dalej, bezlitośnie, póki nie dotarł do linii umundurowanych policjantów.

-- Przepraszam, proszę pana, mogę przejść? -- Amerykański Orzeł potrafił
przeczytać z~jego ust z~wysokości, ale nawet gdyby nie potrafił, mowa
ciała Robinson była jasna.

Policjant w~mundurze się nie poruszył. Kontrmanifestanci zaczęli
skandować ,,PRZEPUŚĆ GO PRZEPUŚĆ GO''. Bianchi, megafon w~dłoni, stojąc
na schodach, wyglądał na zmieszanego i~zdenerwowanego. Stracił kontrolę
nad sceną i~teraz wyglądał głupio i~tchórzliwie, napakowany mięśniak
wspierany przez rząd policjantów na galowo, chroniony przez kolejny rząd
normalnie umundurowanych policjantów, wszystko, żeby powstrzymać go od
konfrontacji z~chudym, czarnym człowiekiem o~lasce.

Niczym lunatyk, zszedł po schodach, idąc wolno przez tłum, który
rozdzielał się dla niego, oczy na nim, póki nie stał po drugiej stronie
policjantów, którzy blokowali drogę Robinsona.

-- Masz mi coś do powiedzenia? -- Orzeł widział jego plecy, ale jego język
ciała mówił głośno, a~słowa były wyłapane przez mikrofon jego megafon.
Grzebał w~nim, ale zanim mógł go wyłączyć, Robinson odpowiedział.

-- Chcę, żebyś wiedział, że nie zamierzam się poddać. -- Policjant
pomiędzy Robinsonem a~Bianchim poruszył się niespokojnie.

-- Grozisz mi?

Robinson rozważał to przez dłuższą chwilę. 

-- Zdaje się, że tak,
sierżancie Bianchi. Hańbisz ten mundur. Kiedy wyciągnąłeś mnie z~mojego
samochodu i~okaleczyłeś, popełniłeś zbrodnię. Kiedy skłamałeś o~tym,
popełniłeś kolejną. Kiedy Twoi przyjaciele kryli Ciebie, popełnili
przestępstwo. Fakt, że stoisz tutaj, jęcząc, a~ja jestem na kaucji i~mogę dostać trzydzieści pięć lat więzienia, znaczy, że w~tym mieście w~ogóle nie ma sprawiedliwości.

-- Pewnego dnia, ludzie to zrozumieją. Pewnego dnia, wystarczająco
białych, wystarczająco \textit{bogatych} białych zrozumie, że żyją w~mieście bezprawia, w~którym chuligani tacy jak Ty wymierzają karę
pobicia i~potem skazują swoje ofiary za przestępstwo skargi, że takie
miasto nie jest dla nich dobre.

-- Jednak na długo przedtem, wszyscy inni będą świadomi tego faktu.
Brązowi ludzie. Biali ludzie. Chińczycy, Portorykańczycy i~wszyscy inni.
Zrozumieją, że jesteś dowodem, że system jest zepsuty poza możliwość
naprawy, że Ty i~wszyscy Twoi przyjaciele nie mogą być ,,zreformowani''
w coś, co Nowy Jork może tolerować. Kiedy ten dzień nadejdzie, to będzie
Ty i~Twoi kumple chuligani przeciwko całemu Miastu Nowego Jorku i~tę
wojnę przegrasz, sierżancie Bianchi.

-- Zatem tak, zdaje się, że jestem zagrożeniem dla Ciebie. I~nic co
zrobisz, tego nie zmieni. Zamknij mnie, będę dla Ciebie zagrożeniem.
Uwolnij mnie, będę dla Ciebie zagrożeniem. To nie koniec, sierżancie
Bianchi, ale to jest początek końca dla Ciebie, jeżeli nie dla
chuliganerii, którą przedstawiasz. Nie poddam się. Postawię na swoim.

Niewiarygodnie, Bianchi nie pomyślał, żeby wyłączyć megafon podczas tej
wymiany i~dźwięk odbijał się od budynków i~trafiał w~głodne mikrofony
telefonów wycelowanych w~parę.

Bianchi był dobry w~byciu twardzielem, ale nie był dobry w~byciu
spokojnym. Nadął się, żyły na jego czole i~szyi, punkt na jego żuchwie,
gdy poruszał nią i~zaciskał, jego szyja czerwona i~napinanie
kołnierzyka. Wyłączył megafon dłonią, która widzialnie drżała.

Wilbur Robinson uśmiechnął się spokojnie do wściekłego policjanta,
wiedząc, że miał przyzwoitą szansę na bycie uderzonym, wiedząc, że gdyby
taka rzecz nadeszła, nastąpiłby chaos. W~tej chwili Wilbur Robinson miał
w dupie chaos.

Tak jak Amerykański Orzeł. Po cichu wiwatował Wilburowi Robinsonowi,
dołączając do głośnych okrzyków od kontrdemonstrujących, którzy byli
trzymani za liniami policji.

Gdy mikrofon był wyłączony, Bianchi nazwał Wilbura Robinson słowem,
które zawierało rzeczownik na ,,n''. Był bokiem do Amerykańskiego Orła,
ale ruchy ust były niewątpliwe, tak jak reakcja kontrmanifestantów,
którzy to zobaczyli i~prawdopodobnie usłyszeli. Bianchi nie był cichym
człowiekiem.

Uśmiech Robinsona się nie zmienił. Powiedział tylko jedno słowo, a~może
powiedział bezgłośnie: 

-- \textit{Tchórz}.

To wtedy sierżant Bianchi go uderzył.

Amerykański Orzeł nie interweniował w~następujące zamieszki. Na miejscu
byli policjanci i~byli przynajmniej świadomi wycelowanych w~nich kamer,
nawet jeżeli Bianchi nie był.

Obserwował, jak policja rozdzielała walczące ciała, oddzielając je w~dwa
obozy, zmuszając ich do przesunięcia się za szybko ustawiane barierki.
Skinhead rzucił butelkę ponad policyjną linią w~środek
kontrdemonstracji, którzy odskoczyli. Kiedy się rozbiła, część szkła
zraniło małe dziecko -- nie miała więcej niż pięć czy sześć lat -- a~jej
lament zwrócił uwagę wszystkich. Kiedy ich uwaga była gdzie indziej,
skinhead pokazał środkowy palec obiema rękoma i~potem wtopił się w~tłum.
Orzeł śledził go i~rozważał spadnięcie, żeby go złapać.

Wilbur Robinson siedział na ziemi, zaopiekowany przez sanitariusza,
który świecił światło w~jego oczy i~ostrożnie badał go od stóp do
czaszki, szukając złamań i~zwichnięć. Czytając jego usta, Orzeł
zobaczył, że powtarzał: 

-- Wszystko w~porządku, wszystko w~porządku. -- choć Bianchi podbił mu oko, które zaczynało puchnąć mimo paczki z~lodem
ratowników.

Robinson w~końcu odepchnął ratownika i~z wysiłkiem stanął na nogi.
Młodszy mężczyzna, jeden z~kontrmanifestantów, ubrany w~wąskoklapowy,
czarny garnitur, zadbane włosy i~okulary w~drucianych oprawkach, jak
starszy Malcolm X, rzucił się mu na pomoc. Rozmawiali krótko i~on
wywołał kolegę, który przyniósł inny megafon.

Robinson podniósł go do ust.

-- Panowie policjanci, chciałbym zgłosić napaść przez Gioffre Bianchi.
Chciałbym, żebyście go aresztowali i~oskarżyli. Mam wielu świadków
tutaj, którzy widzieli jego atak, włączając w~to wszystkich was. -- Patrzył się na nich, łagodnie, bez mrugnięcia okiem. -- Proszę, czyńcie
swoją powinność.

Żaden policjant się nie poruszył.

Wilbur znowu podniósł megafon do ust. 

-- Sierżancie Gioffre Bianchi,
aresztuję was jako obywatel w~imieniu prawa. -- Podszedł do policjanta,
który stanął mocno i~wystawił brodę. -- Proszę, chodź ze mną.

Bianchi pchnął go, jego laska wyleciała w~tłum, gdy zamachał rękami.

Powoli, Wilbur znowu stanął na nogach. Młodszy mężczyzna przyniósł mu
laskę. Podszedł znowu do Bianchiego. Znowu, Bianchi go pchnął. Tym
razem, Wilbur uderzył głową w~chodnik i~usiadł powoli, ewidentnie
zamroczony, delikatnie pocierając głowę. Młody mężczyzna pomógł mu
stanąć na nogach. Ludzie dookoła -- nie tylko kontrdemonstrancji, ale
widzowie także, a~było ich coraz więcej z~każdą minutą -- krzyczeli na
niego, dopingowali, mówili mu, żeby się nie zranił, wyzywali Bianchiego
od tchórzy, nawoływali policjantów dookoła, żeby coś zrobili.

Policjanci wymieniali nerwowe spojrzenia. Oficer nadjechał radiowozem z~włączonymi syreną i~światłami i~dotarł do Bianchiego. Mówił zbyt cicho,
żeby być usłyszany, nawet dla Amerykańskiego Orła. Bianchi raz pokręcił
głową i~zaczął iść do Robinsona. Zacisnął pięści i~przygotował się do
zamachu. Orzeł widział, że byłoby to piekielne uderzenie, wszystkie
mięśnie Bianchiego twarde jak żelazo, mogło złamać szczękę Robinsona.
Orzeł miał skoczyć z~budynku i~polecieć, by przechwycić uderzenie, kiedy
dłoń w~pancernej rękawicy złapała nadgarstek Bianchiego i~zatrzymała go.
Bianchi próbując machnąć i~udało mu się tylko stracić równowagę,
zwisając z~żelaznego chwytu Bruce'a, gdy Bruce patrzył na niego zza
maski. Pokręcił wolno głową, patrząc w~oczy Bianchiego.

Bianchi zbladł, z~zaskoczenia lub z~bólu kości nadgarstka ściskanych
razem, Orzeł nie potrafił powiedzieć. Bruce powiedział, usta widzialne
pod ochraniaczem nosa w~jego masce.

-- Nie. Rób. Tego.

Puścił nadgarstek Bianchiego z~warknięciem i~pozwolił mu upaść na
ziemię, a~zanim dotknął ziemi, Bruce już szedł do kapitana, jego
peleryna falująca za nim. 

-- Zajmij się swoją cholerną robotą -- powiedział, jego twarz centymetry od oficera, a~potem wskoczył na
motocykl i~rycząc, odjechał aleją. Z~jego punktu, Orzeł widział Bruce'a
pojawiającego się w~innej ulicy cywilnie, motocykl zamieniony w~zardzewiałego gruchota, który zabezpieczył łańcuchem do słupa. Orzeł
spodziewał się, że Bruce wyśle później chłopaka, żeby go odzyskał.

Oficer poprowadził Bianchiego do radiowozu, pozwalając mu usiąść z~przodu. Po cichu powiedział mu prawa Mirandy, kiedy Bianchi siedział,
twarz nieruchoma, wrząc.

Obie strony patrzyły jedna nad drugą, potem, jako masa, się rozeszły.
Przedstawienie na dzisiaj było skończone.

~

Lois nie odpowiadała na jego telefony. Po dwóch dniach wiadomości bez
odpowiedzi, Amerykański Orzeł założył cywilne ubrania i~pojechał metrem
do jej mieszkania. Nie odpowiedziała na dzwonek. Zaczął czuć narastający
strach i~poszedł dookoła, wspiął się na śmietnik, skoczył na wyjście
pożarowe, potem wspiął się z~przekonująco ludzką nieporadnością, znalazł
okno Lois, zajrzał, używając rentgena, żeby przejrzeć całe miejsce, choć
szybkie spojrzenie powiedziało mu, że miejsce było puste, wyczyszczone.

Zmuszając się do spokoju, zadzwonił do gazety i~poprosił o~połączenie na
jej biurko, odsłuchał wiadomość głosową mówiącą, że jest ,,na
zagranicznym przydziale'' i~kierującą go do innego reportera, jeżeli
miał jakiekolwiek dobre informacje.

Ledwie odłożył telefon, kiedy znowu zadzwonił.

-- Cześć, Bruce.

-- Ona jest bezpieczna. Ale chciała wyjechać, więc pomogłem jej wyjechać.

-- Bruce\ldots 

-- To był jej pomysł. Rozgryzła to. Usłyszała od swoich źródeł w~wywiadzie, że rzeczy zaczynają się zdarzać, rzeczy, które mogłyby jej
dotyczyć. Ona ma się dobrze. Znalazłem jej źródło w~wielkiej, rosyjskiej
firmie obrony, który chce ujawnić jej, jak szpiegowali oligarchów dla
Kremla. Uwierz mi, jest tam szczęśliwa jak świnia w~błocie.

-- Jest w~Rosji?

-- Bezpieczniejsza niż tutaj, teraz. Narobiłeś sobie wrogów, rozumiesz?

-- A ona nie chciała o~tym porozmawiać ze mną?

-- Była w~tym zakresie nieugięta. -- Nastąpiła dziwna przerwa. -- Daj temu
czasu. Masz go dużo, prawda? -- Kolejna przerwa. -- Lub nie. Znajdź kogoś
innego. Lub weź urlop. Zasłużyłeś na niego i~Bóg wie, dużo nie załatwisz
tak, jak sprawy stoją. Naprawdę zjebałeś sprawy.

Amerykański Orzeł szukał słów, które nie chciały nadejść.

-- No weź, nie masz fortecy samotności na Arktyce czy co?

-- To był pokaz promocyjny. -- Minęły dekady, od kiedy przeniósł
pozostałości lądownika na ciemną stronę Księżyca.

-- Dobra, daj mi znać, jeżeli będziesz potrzebował kryjówki. Mam miejsca
poza Rosją, gdzie mógłbyś ochłonąć.

-- W~porządku. Potrafię zająć się sobą.

-- Wiem, że potrafisz, bracie. Chciałbym, żebyś się po prostu tym zajął.

Dotknięty, Orzeł się rozłączył.

~

Wilbur Robinson żył w~niskim bloku mieszkalnym po złej stronie linii
Masona-Dixona na Staten Island, z~boiskiem do koszykówki od frontu i~ogrodami działkowymi z~tyłu. Jego mieszkanie było na trzecim piętrze,
był tam sam, kiedy Amerykański Orzeł zadzwonił, dopasowując się tak, że
nie było nikogo na tym piętrze gdziekolwiek przy drzwiach i~windy były
puste.

Wilbur uchylił drzwi, długo spoglądał na Amerykańskiego Orła, potem
zamknął drzwi, zdjął łańcuch i~go wpuścił.

-- Pijesz lub jesz?

-- Nie bardzo.

-- Ja tak. -- Obiad Robinsona był na stole w~jadalni, talerz z~hamburgerem, groszkiem i~marchewką, obok szklanka piwa. Usiadł przy
stole, wziął gryza burgera, przeżuł go zamyślony, patrząc na Orła.

-- Słyszałeś kiedykolwiek o~człowieku nazwiskiem Emmet Till?

Orzeł powoli pokiwał głową.

Robinson odłożył hamburger i~wstał z~bólem, poszedł do regału obok
telewizora, wyjął grubą książkę w~twardej oprawie, najeżoną żółtymi
karteczkami z~drobnymi napisami na ich wystających częściach. Popatrzył
na nie, potem otworzył książkę. Amerykański Orzeł spojrzał odruchowo na
fotografię, choć wiedział, co zobaczy, potem odwrócił wzrok.

-- Wyobrażam sobie, że widziałeś to zdjęcie.

-- Widziałem.

Emmett Till leżał w~trumnie w~pokoju dziennym jego rodziców,
czternastoletni chłopak w~niedzielnym garniturze, tylko rozpoznawalna
ludzka część pod garniturem. Głowa, która była oparta na białym
wykrochmalonym kołnierzu, była zmiażdżoną masą, tylko wskazującą na rysy
twarzy. Został pobity, okaleczony i~wrzucony do rzeki przez dwóch
białych mężczyzn, którzy myśleli, że zagwizdał na białą kobietę. Orzeł
pamiętał tamtą chwilę.

-- Rozumiem, że mogłeś widzieć to kiedy zostało opublikowane.

-- Widziałem. -- Brzmiało to jak przyznanie się. Z~pewnością, myślał,
rozumiał stanowisko Robinsona: gdzie byłeś, kiedy to się zdarzyło? Dwóch
mężczyzn, którzy przyznali się do zabójstwa, zostało uniewinnionych
przez białą ławę przysięgłych.

-- Był chłopcem, ale zeznania były o~jego przerażającym czarnym ciele i~jego niesamowitej sile. To zawsze jest o~przerażających czarnych
ciałach, naszej superludzkiej sile, niesamowitej wytrzymałości. Musieli
nas ciągle bić, ponieważ kiedy uderzysz nas tylko raz, zaraz wstaniemy.
Mówiono mi, że pierwsze prawa przeciwko kokainie zostały przyjęte po
zeznaniu białego szeryfa, który musiał użyć broni większego kalibru,
ponieważ czarni mężczyźni nie padali po pierwszym strzale, kiedy wzięli
ten narkotyk. -- Robinson patrzył się na niego długo.

Orzeł się wiercił. W~końcu: 

-- Superludzka siła nie jest wszystkim, co
sobie można wyobrazić.

-- Nie miałbym nic przeciwko. Założę, że też Emmett Till nie miałby nic
przeciwko.

Obserwował rozprawę dla gazety, w~małym stopniu, robiąc wywiady uliczne
na Times Square z~Nowojorczykami, którzy patrzyli w~strachu na
karykaturę w~Mississippi. Robił wywiady z~tymi białymi, uczciwymi
mieszkańcami Północy i~mówili właściwe słowa, potem jego supersłuch
podsłuchałby co \textit{naprawdę} myśleli, kiedy odeszli dalej, szepcząc
do przyjaciół. \textit{Niech no który spróbuje z~moją siostrą, to pomyślą,
że ten dzieciak Till miał łatwo\ldots }

-- Pewnie nie miałby.

-- Panie Orzeł, muszę zauważyć, że miałeś wiele możliwości polecenia na
Południe Jima Crowa\footnote{ od praw Jima Crowa, które miały za zadanie
ograniczenie praw byłych czarnych niewolników oraz pogłębianie
separacji między białą a~czarną ludnością,
zob.~\url{https://pl.wikipedia.org/wiki/Prawa\_Jima\_Crowa} -- przyp.tłum.} i~siedzenia przy ladzie na lunch. Muszę zauważyć, że nie
zrobiłeś niczego takiego, ale byłeś poruszony, żeby interweniować, kiedy
byłem bity przez sierżanta Bianchiego i~jego mały gang sadystów.

Orzeł skinął głową.

-- Możesz wyjaśnić tę rozbieżność?

Orzeł westchnął. 

-- To jest pytanie, które wielokrotnie sobie zadawałem.
Chyba\ldots  -- Urwał. Mieszkanie Robinsona było przytulne, zamieszkałe, mata
do pokera i~chipsy pod szklanym stolikiem do kawy, składane krzesła
wystające z~szafy, gotowe do wyjęcia i~ustawienia, kiedy będzie większa
kompania, niż jadalnia miała miejsc. Orzeł nigdy nie miał kompanii,
nawet Lois. -- Zdaje się, że dotarłem do granicy.

Robinson długo patrzył na fotografię, potem zamknął książkę. 

-- Długo Ci
to zabrało.

Orzeł nie wiedział, co na to odpowiedzieć. Robinson miał rację.

Sam się wypuścił. Zza zamkniętych drzwi, słyszał Robinsona kulejącego
przez pokój, żeby odłożyć książkę.

~

Bianchi drzemał w~radiowozie, kiedy Orzeł go znalazł. Jego mundur
Departamentu Policji w~Tempe był zapocony i~zmięty, jego twarz opalona
słońcem Arizony, prócz grzbietu nosa, który był poparzony od słońca.

Orzeł pozwolił, żeby jego cień położył się na twarzy Bianchiego,
obserwował, gdy się ruszał, otwierał jedno oko, siadał prosto, sięgając
instynktownie po swoją broń boczną.

Gdy Bianchi się bardziej obudził, zabrał dłoń, potem opuścił szybę.

-- Dobrze się bawisz?

-- Nie do końca. -- Orzeł nie nosił swojej peleryny i~maski od miesięcy,
nie odkąd opuścił Nowy Jork. Nie odkąd po raz trzeci został wyśmiany
przez człowieka każącego mu pójść tam, skąd pochodzi. Od tego czasu,
przede wszystkim chodził, wykorzystując stare drogi, które stały się
niepotrzebne siedemdziesiąt pięć lat temu, kiedy weszły międzystanowe.
Kiedyś chodził tymi drogami, w~innym czasie, w~innej Ameryce. Lub
przynajmniej, Ameryce, której historię selektywnie sobie opowiadał.
Byłoby dziwnie nosić je teraz.

-- Jesteśmy w~zasięgu stu pięćdziesięciu kilometrów od granicy. Co jeżeli
zażądałbym dowodu obywatelstwa? -- Uśmiech Bianchiego był gówniany i~drobny.

-- Cóż, zdaje się, że mielibyśmy problem. Myślałem, że straciłeś swój
apetyt na problemy.

-- Nie cofam się, nie z~walki, gdzie mam rację. Może mógłbyś mnie
wyrzucić z~Tempe? I~co z~tego? To by znaczyło, że dostałbym \textit{dwie}
wcześniejsze emerytury i~kolejną rentę. Słyszałem, że zatrudniają na
Florydzie.

-- Może tym razem trafiłbyś do więzienia.

-- Bądź poważny, tłumoku.

Orzeł zabił setki ludzi. Czym był jeszcze jeden? Nigdy by go nie
złapali. Choć oczywiście kamera na ciele Bianchiego nagrywała. Musiałby
wziąć ją ze sobą, zgnieść na pył. Nie zostawiłby odcisków palców. Nie
\textit{miał} odcisków palców.

-- Nie chcę kłopotów. Chciałem zobaczyć, co ze sobą zrobiłeś. Czy
nauczyłeś się czegokolwiek.

-- Nie. Kurwa. Wiarygodne. -- Bianchi podniósł szybę, włączył klimatyzację,
włączył bieg radiowozu. Orzeł położył dłoń na masce, zanim samochód
ruszył, przytrzymał nieruchomo, podczas gdy silnik zawył i~koła się
kręciły. Bianchi miał zamiar znowu chwycić swoją broń, Orzeł prawie
mógł zobaczyć impuls nerwowy biegnący w~ramieniu, kiedy Orzeł puścił i~samochód szarpnął do przodu, rzucając Bianchi na pas bezpieczeństwa.

Kiedy Bianchi spojrzał do tyłu, Orła już dawno, dawno nie było.

~

Kiedy znowu zobaczył Wilbura Robinsona, to było na YouTube. Robinson
wyglądał na lepiej odżywionego, jego wąs był doskonale przystrzyżony,
gdy prowadził tłum, który trzymał kamery wycelowane w~policjanta, gdy
ten przeszukiwał młodego czarnego mężczyznę. Policjant wyglądał, jakby
chciał uderzyć nich wszystkich, ale nie zrobił tego.

To była okropna scena, a~chłopak w~trakcie przeszukiwania nie cieszył
się z~tych chwil. Jednak gliniarz nic nie znalazł i~odszedł, wyglądając
na wściekłego.

Orzeł pomyślał o~zadzwonieniu do Lois. Tęsknił za nią. Jednak nie
odpowiadała na jego telefony i~rozumiał aluzję.

Bruce odpowiedział po drugim dzwonku. 

-- Nie sądzę, żeby to był dobry
pomysł, żebyśmy więcej rozmawiali.

-- Naprawdę?

-- Mam określone zobowiązania, dzięki moim relacjom z~rządem Stanów
Zjednoczonych, a~one dotyczą pomocy w~namierzaniu potencjalnych wrogich
bojowników.

-- Och.

-- Są pewne rzeczy, które Ameryka jest w~stanie tolerować. Pewnych rzeczy
jednak\ldots 

-- Rozumiem.

Rozłączył się. Zastanawiał, dlaczego nie był bardziej zdenerwowany. Po
przemyśleniu tego, oglądania rozwijających się grzmotów, gdy unosił się
w troposferę, zrozumiał, że to, co Bruce powiedział, już dawno wiedział.
To dlatego nie było go, gdy Bull Connor wypuścił psy. Dlatego zignorował
wszystkich innych Wilburów Robinsonów po drodze. Pewne rzeczy, Ameryka
mogła tolerować, ale innych, nigdy, przenigdy by nie wybaczyła.

\chapter*{Radykalne}

Na trzydzieste szóste urodziny Joego Gormana, jego żona Lacey zadzwoniła
do niego na komórkę trzy razy z~rzędu. Był w~trakcie spotkania z~przełożonym, vice-dyrektorem, który już je dwukrotnie przełożył. Zabrało
to dziesięć dni i~latte dla asystenta dyrektora, żeby dostać czas na
żywo z~tym gościem, zatem Joe przerzucił żonę trzy razy na pocztę
głosową. Prawdopodobnie dzwoniła, żeby złożyć życzenia urodzinowe lub
potwierdzić coś o~wielkiej kolacji, którą tego wieczora mieli zjeść w~jego ulubionej restauracji ze stekami.

-- Musisz to odebrać? -- Wice wcale nie krył swojej irytacji.

-- Przepraszam -- powiedział Joe i~zrobił pokaz wyłączania telefonu i~chowania do kieszeni. Wrócił do swojej propozycji: znalazł zewnętrzną
firmę logistyczną, która mogłaby zająć się wszystkimi zwrotami, których
koszt był uparcie wielką pozycją w~bilansie jego oddziału. Naprawdę
ciężko pracował nad propozycją i~ten wice mógł ją przyjąć lub odrzucić.
Jednak w~ciągu kilku minut, asystent wice zapukała do drzwi i~serce
Joego zamarło. Ktoś ważniejszy niż on musiał wbić się na jego dawno
oczekiwany termin. Ale asystent -- Gloria, stylowa czarna kobieta w~średnim wieku, która była w~firmie dłużej niż on czy wice -- zwróciła się
do Joego, nie szefa.

-- To Twoja żona, Joe.

Zrobiło mu się niedobrze. Nigdy nie był bardziej wściekły na Lacey w~ciągu ich ośmiu lat małżeństwa. \textit{Wiedziała}, że ma to spotkanie. To
było wszystko, o~czym rozmawiał, kiedy w~ogóle mówił, kiedy nie siedział
przy kuchennym stole z~laptopem, pracując nad propozycją. Jezu kurwa
Chryste, czy nie potrafiła sama załatwić tej głupiej rezerwacji do
restauracji lub listy gości czy co? Była dorosłą kobietą. To były jego
\textit{urodziny}. Jakiś urodzinowy prezent.

-- Wygląda na to, że musisz to odebrać -- powiedział wice. Brzmiał
żartobliwie, ale było w~tym także ironiczne przewrócenie oczami, którego
Joe nie mógł nie zauważyć. -- Poproś Glorię, żeby umówiła spotkanie,
dobrze?

Gloria spojrzała na niego współczująco, gdy wypadał z~jego biura, przez
parking, gdzie było ogniście gorąco i~oślepiająco słonecznie. Wyjął
komórkę, odblokował ją i~zadzwonił do Lacey.

-- Lacey, kochanie\ldots  -- Nazywał ją \textit{kochaniem}, kiedy był na nią
wściekły.

-- Joe\ldots  -- było wszystkim, co usłyszał od niej, a~potem załkała.

Jego emocje znikły. Lacey \textit{nie} była płaczliwa. Jej matka była
płaczliwa i~randkował z~dziewczynami, które czuły wszystko tak
dojmująco, że łzy nigdy nie były głęboko pod powierzchnią, ale Lacey w~ogóle taka nie była. Nagle był przestraszony, gniew kręcący się dookoła,
ale niezwiązany.

-- Co się stało?

Więcej szlochów. Potem głęboki wdech. 

-- Widziałam się z~lekarzem. -- Długa pauza. 

Joe chciał się rozłączyć. Więcej niż cokolwiek. Ponieważ
wiedział, że ma przejść przez drzwi, które prowadziły z~jego życia,
jakim było, w~inne nowe, gorsze życie. Były to drzwi, które otwierały
się w~jedną stronę i~kiedy przez nie przeszedłeś, nigdy nie mogłeś
zawrócić. Był ułamek sekundy, kiedy właściwie prawie się rozłączył z~Lacey, ale oczywiście nie zrobił tego.

To był rak piersi czwartego stopnia, przerzutowy, guzy w~wątrobie,
trzustce i~w jednym płucu. Lacey miała przed sobą trzy miesiące życia.
Sześć, jeżeli spróbowaliby skrajnych zabiegów. Lacey przestała płakać po
pierwszych dwóch dniach i~skoncentrowała się na poważnie, stoicki
obrońca, który przeczytał wszystko i~nawet znalazł grupę na Facebooku
,,Umierając z~godnością'', gdzie stała się królową pszczół. Kupiła
wszystkie te książki z~obrazkami dla dzieci, których rodzice umierają, i~czytała je regularnie Madison, przytulając Maddie na kolanach i~czytając
tym samym cichym, śpiewnym rytmem opowiadania, którego zawsze używała
przy czytaniu na dobranoc, jakby nie przygotowywała sześciolatki na
życie bez matki.

Doktora przedstawiła wszystkie sposoby, w~jaki jej trzy miesiące mogły
zostać wydłużone do sześciu, a~Lacey spojrzała jej prosto w~oczy i~spytała: -- Jeżeli miałabyś to, co ja, spróbowałabyś tych terapii?

Doktora zacisnęła usta. 

-- Szczerze? Nie. Nie sądzę, żeby jakikolwiek
lekarz próbował.

-- Dziękuję za szczerość. -- Tylko tyle Lacey odpowiedziała i~Joe
wiedział, że nie zrobiłaby niczego innego.

Tylko dlatego, że decydujesz się umrzeć na raka, to nie powstrzymuje
wszystkich, których znasz od zajmowania Twoich ostatnich miesięcy na
Ziemi informacjami o~cudownych lekach. Kasowali wiadomości i~uprzejmie
powiedzieli wszystkim -- nawet rodzicom -- żeby przestali pieprzyć, ale
ludzie nie potrafią się powstrzymać.

Mama Lacey znalazła link do terapii transferu komórek adoptowanych\footnote{ więcej~\url{https://en.wikipedia.org/wiki/Adoptive\_cell\_transfer}
-- przyp.tłum.}. To nie była pułapka, Narodowy Instytut Raka był częścią
Narodowego Instytutu Zdrowia i~mieli mnóstwo artykułów na temat terapii
opublikowanych w~\textit{Nature} z~wielką liczbą cytowań. Joe i~Lacey
przeczytali artykuły tak dobrze, jak potrafili, Lacey rozmawiała o~nich
z jej umierającymi przyjaciółmi z~Facebooka, wszyscy zdecydowali, że
może jest to warte próby.

Sposób, w~jaki to działało, był następujący: lekarze sekwencjonowali
genom raka, szukali cech, które białe krwinki mogłyby zaatakować, potem
szukali w~Twoich własnych białych krwinkach, takich, które mogły
zaatakować te cechy, hodowali około sto miliardów tych małych żołnierzy
w laboratorium, a~potem je wstrzykiwali. Była to tylko metoda
przyśpieszenia wolnego i~nieefektywnego procesu, w~którym własne ciało
dopasowywało populację komórek białych krwinek, dając im obliczeniowe
przyśpieszenie, które mogło wyprzedzić nawet najszybciej mutujący
nowotwór.

Joe i~Lacey nawet znaleźli prywatnego lekarza, tutaj w~Phoenix, który
wykonałby procedurę. Miał pozycję na Uniwersytecie Stanowym Arizony,
opublikował kilka dobrych artykułów na temat procedury i~wszystko, czego
potrzebował, to półtora miliona dolarów od towarzystwa
ubezpieczeniowego.

Wiecie, co się dalej wydarzyło. Ich ubezpieczyciel powiedział Lacey, że
już czas dla niej na śmierć. Jeżeli chciała chemoterapii czy
radioterapii, czy cokolwiek, zapłaciliby za to (niechętnie, z~wielką
biurokratyczną nieustępliwością), ale terapie ,,eksperymentalne'' nie
były opłacane. Co, wiecie, ok, kto chce wydać półtora miliona na
oczyszczający sok cuda czyniący jakiegoś szarlatana lub terapię
kryształów? Ale transfer komórek adoptywnych nie był leczeniem
kryształami, a~Narodowy Instytut Zdrowia nie był lokalnym szamanem.

Przeszli -- \textit{Joe} przeszedł -- dziwną transformację po ostatniej
rozmowie telefonicznej przełożonego przełożonego przełożonego w~ich
towarzystwie ubezpieczeniowym. Lacey była dobra w~tym wszystkim,
znajdując pokój, spokój i~zdecydowanie, żeby jej śmierć była dobrą
śmiercią. Wyciągnęła Joego z~jego gniewu na raka w~miłość do niej i~wzajemne zrozumienie, że mogli sprawić, że ich ostatnie dni razem były
tymi dobrymi, dla nich i~dla Madison.

Nie mniej po tym, gdy ubezpieczyciel ich odrzucił, wściekłość wróciła.
Może terapia nie podziałałaby, ale była \textit{szansą}, realistyczną, nie
zdesperowaną, prawdziwą możliwością, że ich córka miałaby matkę i~że on
miałby żonę i~najlepszego przyjaciela, żeby się zestarzeć razem.

Chciał sprzedać dom i~pożyczyć pieniądze od przyjaciół i~rodziny, iść na
GoFundMe, ale Lacey nie chciała o~tym słyszeć. Podkreśliła, że wszystko,
co oni -- oraz ich najbliższa rodzina -- mogła zebrać w~ogóle, nie
zbliżyłoby się do 1,5 miliona, a~jedyną rzeczą gorszą niż rodzina
tracąca żonę i~matkę była ta sama rodzina tracąca dodatkowo oszczędności
i dom. Była znacznie inteligentniejsza i~znacznie spokojniejsza niż Joe.

Joe był \textit{wściekły}. Joe nie mógł się gniewać na raka, ale mógł być
zimno, morderczo rozwścieczony na firmę ubezpieczeniową i~ludzi, którzy
tam pracowali. Pracował dla ,,blue chip'', jednej ze stu firm
\textit{Fortune}, kupił ubezpieczenie z~górnej półki, zabierali ponad 1500
dolarów z~jego wypłaty każdego miesiąca, a~jakiś anonimowy, zły chujek
właśnie zdecydował, że nawet nie będą \textit{próbować} uratować jego żony
od bolesnej, groteskowej śmierci.

Gniew pożarł Joego. Nigdy nie umówił się ponownie z~tym wice. Spędzał
cały czas na pisaniu do Kadr i~Głównego Księgowego, a~kiedy tego nie
robił, dosłownie płakał w~toalecie.

A przez cały czas Lacey była coraz bardziej chora.

Madison zaczęła się bać Joego, unikając go, kiedy wracał do domu.
Próbowali się zamienić, żeby on odpowiadał za jej usypianie,
opowiedzenie historii i~zaśpiewanie wymaganych trzech piosenek. Madison
wytrzymywała to, ale nie zmniejszyło to jej oczywistego strachu przed
nim.

-- Kotku -- powiedziała Lacey, jej równoważnik do jego ,,Kochanie'' i~Joe
wiedział, że ma kłopoty. -- Nie możesz tak dalej robić. Padniesz martwy
przede mną. Lub kogoś zastrzelisz. Potrzebujesz pomocy. -- Odmawiając
chemo-- i~radioterapii oznaczało, że Lacey nie schudła, ale ból nie
puszczał w~nocy i~miała ten pusty, nieziemski wygląd ,,jedną nogą w~grobie'', który ledwie mógł znieść.

Wzięła jego twarz w~dłonie. 

-- \textit{Nie} wygłupiam się, Joe. Znajdź
pomoc. Ponieważ jeżeli nie dostaniesz pomocy, nasza mała dziewczynka
będzie miała zero rodziców, ponieważ zmierzasz do instytucji dla
umysłowo chorych, więzienia lub sądu, żeby bronić swojej zdolności do
bycia ojcem. -- Jej oczy płonęły. -- Nie mam dostatecznie dużo siły lub
czasu, żeby z~Tobą nad tym popracować, Joe. Wiem, że normalnie
znalezienie psychiatry czy kogokolwiek byłoby moją pracą w~naszym
podziale pracy, ale będziesz musiał wkroczyć. Czy wyrażam się jasno?

Słowa rozwścieczyły Joego, a~furia go zasmuciła. Popłakał trochę, potem
powiedział: 

-- Masz rację. Znajdę. -- Przytuliła go długo, bardzo długo,
potem poszedł do pokoju dla gości, usiadł na łóżku z~laptopem i~zaczął
googlować.

Miał otwarte osiem zakładek recenzji Yelpa lokalnych terapeutów, kiedy
znalazł Forum. Pozornie, było ono dla ojców, których żony umierały na
raka piersi (było dostatecznie dużo ludzi umierających na raka, że fora
stały się wyspecjalizowane), ale właściwie było dla ojców, których żony
umierały na \textit{wyleczalnego} raka piersi, którym odmówiona opłacenia
z polisy ubezpieczeniowej.

Joe czytał przez godziny, długo poza stan, kiedy jego tyłek stał się
drętwy i~dostał skurczu szyi. Słowa na ekranie wydawały się wychodzić
prosto z~jego własnej głowy. Były tajne rzeczy, rzeczy, których nigdy
nie śmiał powiedzieć do innych ludzi, ponieważ Lacey miała rację, były
tego rodzaju rzeczy, których nie można było powiedzieć głośno bez ryzyka
uwięzienia lub pobytu na psychiatryku.

Tutaj byli mężczyźni mówiący te rzeczy. I~inni mężczyźni, którzy ich
słuchali i~mówili im, że rozumieją, że czuli te same niewypowiadalne
uczucia i~że rozumieją te uczucia. Nawet zanim opublikował pierwszą
wiadomość na forum, samo czytanie ukoiło coś surowego wewnątrz niego, a~może pewnego dnia mógłby nawet uleczyć rany, które powiększały się od
jego trzydziestych szóstych urodzin.

Przestał szukać terapeuty. Nie potrzebował żadnego. Ojcowie na forum
,,Jeb Raka Prosto w~Jebany Ryj'' byli tymi terapeutami, których
potrzebował. Tygodnie mijały i~wszyscy w~domu zrozumieli, że czas Joego
w pokoju gościnnym z~laptopem był powodem, dla którego się zmienił i~nikt nie obrażał się za chwile, które tam spędzał.

Tylko idiota naprawdę wierzy w~spontaniczną remisję, co jest medyczną
nazwą dla ,,twój rak znikł i~nie wiemy dlaczego''. Och, zdarza się, ale
zdarzają się też uderzenia błyskawicą i~wygrane na loterii. Spontaniczna
remisja nie jest planem, jest nierealistycznym marzeniem.

Niektórzy ludzie jednak są trafieni przez błyskawicę. I~niektórzy ludzie
wygrywają na loterii.

A Lacey otrzymała spontaniczną remisję.

Trzy miesiące do życia stały się czterema, potem pięcioma, lekarka
zaczęła robić bardzo ostrożne, wczesne uwagi o~zmniejszających się
guzach, nowych testach, a~potem pewnego dnia, doktora wezwała Lacey do
biura, Joe też przyjechał, ponieważ, kiedy lekarz chce przedyskutować
wyniki testów osobiście, lepiej nie zmierzać się z~tym samotnie.

Lekarka się spóźniała, co sprawiło, że Lacey i~Joe byli zdenerwowani,
napięcie narastało. To była praktyka onkologiczna, zatem poczekalnia
była pełna łysych, umierających ludzi z~zapadniętymi oczami i~ich
nawiedzonymi ukochanymi, właściwie byli w~lepszym stanie, niż ludzie,
którzy ciągle wyglądali dobrze, ponieważ ci ludzie właśnie zostali
zdiagnozowani i~przychodzili do doktora dowiedzieć się co dalej. Ci
ludzie byli wrakami.

Pielęgniarka nie przejmowała się sprawdzaniem parametrów Lacey, zatem
ona i~Joe tylko czekali w~gabinecie na parze pomarańczowych krzeseł z~poczekalni, mocno ściskając dłonie.

Lekarka weszła, zamknęła drzwi, przeprosiła za czekanie, powiedziała
żart o~jednym z~tych dni, usiadła w~obitym fotelu i~poprawiła dokumenty
na biurku. Potem patrzyła na nich oboje przez dłuższą chwilę i,
niespodziewanie, \textit{się rozpromieniła}.

-- Lacey, Joe, prowadzę praktykę od czternastu lat i~wiele razy
przekazywałam złe wiadomości. Rozumiem, to część pracy. Jednak to do
Ciebie dociera. Nawet kiedy mam dobre wiadomości, to ciągle nie są
szczęśliwe wiadomości: zabraliśmy połowę organów, usunęliśmy piersi,
zatruliśmy, napromieniowaliśmy i~teraz, myślimy, że jest lepiej.
Przepraszam.

-- Ale raz na bardzo długi, \textit{długi} czas, lekarz w~mojej pracy może
przekazać \textit{dobre} wiadomości. To jest jeden z~tych dni.

Pozwoliła temu dotrzeć. Lacey i~Joe popatrzyli się na siebie. Były słowa
na końcach ich języków, słowa, których nigdy nie śmieli powiedzieć bez
sarkastycznego spojrzenia. Powiedzieli te słowa teraz, Lacey zaczynając,
Joe dołączając, oboje niepewni i~pytający: 

-- Spontaniczna. Remisja?

Lekarka \textit{uśmiechnęła }się do nich.

Joe płakał, zanim Lacey zaczęła. Zawsze był tym emocjonalnym. Jednak
wtedy Lacey zapłakała. A potem lekarka, a~kiedy Joe i~Lacey tulili się
bardzo długi, lekarka przytuliła Lacey, potem Joe, a~potem wszyscy troje
się przytulili, a~Joe nie mówił nic innego niż \textit{dziękuję dziękuję
dziękuję} i~wszyscy wiedzieli, że właściwie nie dziękował doktorce, ale
nikt nie był pewien, komu właściwie dziękował. Nawet lekarka.

Joe nigdy nie przestał odwiedzać ,,Jeb Raka Prosto w~Jebany Ryj'', co go
zaskoczyło. Tej nocy, on i~Lacey kochali się najwolniej, najczulej w~trakcie całego związku, ruchali się tak wolno, że ledwie się poruszali.
Joe obchodził się z~Lacey jakby była zrobiona z~kruchej porcelany, a~Lacey tuliła się do Joe, jakby on był jedyną rzeczą od spadnięcia z~najwyższego budynku. Po wszystkim, przytulili się, potem odsunęli od
siebie, palce splecione. Wkrótce Lacey spała, lekko pochrapując,
otulając się kocami, a~Joe wysunął się z~łóżka i~wrócił na forum.

Było wiele forum wsparcia online i~najlepsze z~nich dokonywały
niesamowitej, prawie magicznej rzeczy dla ich uczestników, dowodząc, że
aforyzm, że ,,współdzielony ból się zmniejsza, współdzielona radość
zwiększa'' i~sprawiały, że życie wszystkich uczestników było lepsze.

,,Jeb Raka Prosto w~Jebany Ryj'' nie było takim forum.

,,Jeb Raka Prosto w~Jebany Ryj'' było forum dla bardzo rozgniewanych
ludzi, których ukochani umierali lub zmarli. Niektórym użytkownikom
,,JRPWJR'' polepszyło się, może nawet częściowo dzięki szansie
upuszczenia na forum, ale także, ponieważ byli otoczeni przez ludzi,
którzy ich kochali i~sprowadzili znad krawędzi, ludzi, którzy
współdzielili ich smutek, ale potrafili sobie lepiej poradzić.

Na forum byłych alkoholików, istnieją wielkie grupy starszych mężów
stanu, którzy od lat są trzeźwi. Są oni mądrym, uspokajającym głosem
oraz są dowodem na życie po uzależnieniu. Kiedykolwiek ktoś na forum
poszedł w~tango i~się obwiniał, pojawiał się ,,suchy'' Starszy, który
mógł opowiedzieć historię, żeby przebić tamtą, o~byciu wyrzuconym na
ulicę, o~stracie dzieci, utracie kończyn, nawet i~jednak wyjściu z~tego.

,,Jeb Raka Prosto w~Jebany Ryj'' nie miało takich ludzi. Ludzie, którym
minął ich wściekły smutek, opuszczali JRPWJR, przegnani kulturą
wściekłości. Ludzie, którzy zostawali, naprawdę byli \textit{rozgniewani},
czepiając się tego jak pijak odmawiający puszczenia się butelki.

Jeżeli Twój gniew zabrał Cię do miejsca, w~którym nie mogłeś sobie
poradzić, miejsca, które Cię przerażało, Starsi na JRPWJR pomogliby Ci:
wyjaśniliby, że to jest właściwy kierunek, jedyna reakcja i~nie będzie
już nigdy, przenigdy lepiej. To było Twoje życie od teraz.

Kiedy Lacey została ogłoszona zdrową, gniew Joego od razu wysechł.
Sługusy ubezpieczyciela, którzy skazali Lacey na śmierć, mogli dusić się
w swoich sokach i~patrzeć na siebie w~lustro każdego ranka, a~Joe miałby
swoją piękną, cudowną żonę i~niesamowitą, słodką córkę, a~to było
wszystko, co było ważne.

Jednak JRPWJR wezwał Joego. Przy pierwszym zalogowaniu, zrozumiał od
razu, że to wszystko, co mu tak pomogło -- ocaliło -- ujrzenie myśli z~własnych niekończących się pętli w~głowie na ekranie, pochodzących od
innych ludzi, jednocześnie by go zniszczyło. Gdyby Lacey zmarła -- o~Boże, na samą myśl skręcało go w~brzuchu -- to byłby jego system
wsparcia, a~on podjąłby kotwicę, którą mu rzucili i~pozwoliłby się
wciągnąć aż na dno oceanu.

Joe zdecydował, że ma obowiązek wobec JRPWJR.

Wcześnie podjął decyzję o~nierozmawianiu o~tym forum z~Lacey.
Prawdopodobnie zrozumiałaby, ale miała dość martwienia się.

Tak czy inaczej, głównie się czaił. Nigdy nie podjął dyskusji z~Wielkimi
Starszymi, którzy doradzali rozpacz i~gniew. Jednak posyłał prywatne
wiadomości do nowych osób, które pojawiały się skręceni z~bólu i~próbował, jak umiał, im pomóc. Miał pod ręką listę numerów telefonów
zaufania dla samobójców, dawał rutynowo odmierzone dwadzieścia dolarów
na każde GoFundMe, które było zapostowane na forum. Nawet na tym
skromnym poziomie wsparcia, przez jeden miesiąc rachunek na GoFundMe
przekroczył granicę pięciuset dolarów, Lacey chciała się dowiedzieć, co
robi, a~on powiedział jej półprawdę, że to na fundusz rakowy przyjaciela
(ale nie powiedział dla jak wielu przyjaciół).

Lacey nie mogłaby się o~to wściekać, ale przedyskutowała z~nim srogo, o~ich finansach, a~on zgodził się zmniejszyć wydatki na GoFundMe do dwustu
pięćdziesięciu dolarów miesięcznie, a~ona zgodziła się na wspieranie
kwotą trzystu dolarów kandydatów w~każdym roku wyborczym, jeżeli
popierali powszechną opiekę medyczną.

Został na forum.

Był gotów rzucić JRPWJR -- które starzy userzy jak ona nazywali
,,JebRaka'' lub RebJaka, kiedy chcieli brzmieć grzecznie -- kiedy
dołączył LisasDad1990. Jego pierwsza wiadomość:

Lisa ma sześć lat. Oto jak wygląda. Kładłem ją spać każdej nocy, od
kiedy przestała być karmiona piersią. Zwykłem jej czytać książeczki dla
dzieci, potem te trudniejsze, a~teraz czytamy Harry'ego Pottera.
Dokładnie tak, sześciolatka. Ona jest BYSTRA.

W zeszłym roku, Lisa zaczęła dużo upadać, uderzać w~różne rzeczy. Jej
nauczyciele powiedzieli, że nie koncentrowała się w~szkole i~to też
zauważyłem. Jej mama jest nieobecna w~jej życiu. Zabrałem Lisę do
lekarzy i~powiedzieli, że ma nowotwór mózgu. Mogę podać szczegóły dalej,
ale to nie jest dobry rak mózgu. Nie jest mały, czy słodki. To jest
agresywny mały chujek i~się rozrasta.

Lisa teraz widzi tylko na jedno oko, chodzi z~chodzikiem, lub wożę ją na
wózku.

Niemniej dobra wiadomość jest taka, że jest uleczalny. Nie w~stu
procentach, ale onkolog mówi, że może stamtąd wybić gnojka, uderzyć w~niego promieniowaniem, podać Lisie truciznę i~przeżyje. Zawsze będzie
miała jakieś problemy, ale jest młoda, pełna życia, i~jakoś to wszystko
sobie zorganizuje.

Ale nasze ubezpieczenie? Nie za bardzo. Pracowałem jako agent celny,
kiedy uderzył rak, moja pierwsza, prawdziwa praca na pełny etat, z~ubezpieczeniem i~tak dalej. Płaciłem tak dużo na to ubezpieczenie. TAK
DUŻO. Ale powiedzieli, że tego typu operacja, którą chce przeprowadzić
lekarz, ona jest eksperymentalna. Powiedzieli, że nie jest objęta
refundacją.

Ludzie, mam dwadzieścia osiem lat, samotny ojciec. Moi rodzice nie dali
mi grosza, od kiedy powiedziałem im, żeby wypierdalali i~wyprowadziłem
się jako siedemnastolatek. Jeżeli moja ex miała dolara luzem, poszłoby
na oksy\footnote{ oksykodon -- silny opioidowy lek przeciwbólowy, narkotyk,
zob.~\url{https://pl.wikipedia.org/wiki/Oksykodon}}, zanim
windykator długu studenckiego mógłby zabrać.

Mam GoFundMe, ale to zadziała, jeżeli znasz milion ludzi lub jednego
milionera. Moje dziecko jest najwspanialsze na całym świecie, ale
wszyscy myślą tak o~swoich dzieciach, a~na razie z~wszystkich dowodów,
tylko ja jeden to widzę.

Chodzi o~to, moja córka Lisa umrze.

Mam na myśli, mogę się oszukiwać, ale tak właśnie będzie. Moje
sześcioletnie dziecko umrze, chociaż nie musi (lub przynajmniej ma
szansę, z~której nie może skorzystać).

Ponieważ jakiś przypadkowy dupek zarabiający pół miliona dolarów w~biurze na szczycie wieżowca pełnego przypadkowych dupków zarabiających
mniej niż ja zdecydował, że musi umrzeć. Nie zna jej, nigdy jej nie
pozna, ale wie, że żyje wiele tak wiele dzieci jak Lisa, które umrą z~powodu jego wyborów.

Byłem smutny, byłem wściekły, byłem zmartwiony. Przytulam Lisę tak
mocno, że mi mówi, ,,tata przestań'', ale pewnego dnia będę ją przytulał
i ona nic nie powie, ponieważ nie będzie żyła. To moja prawda, moje
życie i~żyję tą prawdą każdego dnia.

Kiedy Lisa odejdzie, też odejdę. Nigdy nie powiedziałem tego głośno, ale
napiszę to tutaj, ponieważ wy wiecie, przez co przechodzę. Jestem kurwa
śmiertelnie poważny. Z~Lisą mam wszystko, co potrzebuję do życia. Teraz
nie mam niczego. Nie stać mnie nawet na pogrzeb, nie po tych wszystkich
wydatkach. Każdego dnia wezwania do zapłaty, każda karta kredytowa chce
wysłać do mnie faceta z~kijem, żeby połamał mi kolana. Może kupię broń,
zastrzelę pierwszego, który przejdzie przez te drzwi, potem wsadzę ją w~usta\ldots 

Joe nie mógł przestać czytać, ale chciał. To było takie ostre,
przypomniało mu ciemne miejsce, o~którym myślał, że zostawił za sobą.

Drżąc, przygotował wiadomość z~numerem do Narodowego Telefonu
Zapobiegania Samobójstwom\footnote{ w~oryg. National Suicide Prevention
Lifeline, taki telefon istnieje
zob.~\url{https://suicidepreventionlifeline.org/}, podany
numer jest poprawny -- przyp.tłum.}, numer, który już pamiętał
1-800-273-8255\footnote{ Telefon zaufania dla osób dorosłych w~kryzysie
emocjonalnym \textbf{116 123} (poniedziałek-piątek od 14:00-22:00, połączenie
bezpłatne), lub ITAKA -- Antydepresyjny Telefon Zaufania \textbf{22 484 88 01}
(godziny działania nieregularne więcej
\url{https://forumprzeciwdepresji.pl/wazne-telefony-antydepresyjne}) -- przyp.tłum.} i~jakieś drętwe słowa pocieszenia. Był najbliżej
powiedzenia o~spontanicznej remisji Lacey na JebRaka, ponieważ nie
potrafił pomyśleć o~niczym innym i~ten biedny gościu potrzebował jakiejś
nadziei.

Ale skasował zdania i~kliknął WYŚLIJ.\\ \textbf{ JEDNA WIADOMOŚĆ ZOSTAŁA OPUBLIKOWANA WCZEŚNIEJ}

Zrób to {[}napisał jeden z~Wielkich Starszych{]}. Serio, zrób
to. Zamierzam tak zrobić, pewnego dnia, kiedy się wypalę. Dlaczego nie
miałbyś tak zrobić? Dlaczego te korporacyjne chujki miałyby żyć, kiedy
moja żona nie żyje? Kiedy Twój dzieciak umrze?

Zamierzałem kupić AR-15 i~zrobić to, ale jebać to. AR-15 są dla ludzi,
którzy chcą się wydostać, strzelając, lub ludzi, którzy chcą wejść
siłowo. Nie potrzebuję tego. Jestem starym, białym facetem ze średniej
klasy. Mógłbym po prostu wejść, wjechać prosto na ostatnie piętro z~małą, fajną bombą z~nawozów i~rozwalić cały, pierdolony pion dyrektorów.
Jacyś ,,niewinni'' umrą, ale nie są tacy znowu niewinni, prawda?

Dorastałem na farmie w~Wyoming, mój tata miał książkę nazywaną
,,Podręcznik Wysadzania'', którym mówił wszystko, co trzeba wiedzieć o~wysadzaniu rzeczy w~miarę potrzeb, przykładowo, jeżeli padnie koń i~zamarznie na kamień, musisz go wysadzić na rozsądne kawały.

Podręcznik Wysadzania jest w~wersji cyfrowej i~żeby oszczędzić Ci
kłopotów szukania i~pojawiania się na jakiś listach, dołączam kopię
tutaj. Będziesz zdziwiony, jak łatwo jest zrobić coś, co zrobi bum.
Cholera, mój stary był prosty jak chuj i~radził sobie.

To, co chcę powiedzieć, jeżeli zamierzasz zrobić coś drastycznego, niech
to nie idzie na marne.

Podpisał się swoim imieniem z~forum, DeathEater, na który Joe patrzył
się przez lata, ale nigdy naprawdę nie pomyślał, aż do tej chwili.
Jezus, \textit{DeathEater}.

Joe nacisnął ODPOWIEDZ, potem się zatrzymał. Co mógłbyś odpowiedzieć na
coś takiego? Czy powinien zadzwonić na policję? Facet, który zaczął
forum dyskusyjne, BigTed, ledwie już kiedykolwiek się logował, ale Joe
miał gdzieś jego adres e-mail z~czasów, kiedy zablokował swoje konto.
Zdecydował, że prawdopodobnie powinien coś zrobić, zatem wkleił
wiadomość DeathEatera w~email. Gdy tylko wysłał, doznał chwili zawrotu
głowy, gdy myślał o~tym jaki boty mogły skanować ten email pomiędzy nim
a BigTedem, na jakiego rodzaju listy obserwowanych właśnie się
dopisywał.

Nie patrzył na wątek, ale inni JebRakowcy się dołączyli. Niektórzy
besztali i~moralizowali, nazywając DeathEatera potworem lub udając, że
tylko żartował, informując go, że robi żarty w~bardzo słabym stylu. Inni
kontynuowali ,,żart'' rozwijając coraz bardziej rozbudowane scenariusze
masowego chaosu. Wielu z~nich zamieszczało zrzuty z~ekranów z~\textit{Podręcznika Wysadzania}, rozmawiało o~sposobach ulepszenia
projektu, opakowania ładunku taśmą, dodania nakrętek i~łożysk kulkowych
jako szrapneli.

Niewielu użytkowników rozmawiało z~LisasDad1990 o~jej córce, oferując mu
rodzaj pocieszenia, który próbował Joe. LisasDad1990 nie odpowiedział
żadnemu z~nich.

Następnego dnia BigTed zamknął wątek i~zablokował DeathEaterowi na trzy
dni. LisasDad1990 nie się odzywał. Joe wysłał mu kilka razy prywatne
wiadomości, ale w~końcu o~tym zapomniał.

Madison skończyła siedem lat. Mieli tort z~lodów, grali w~gry, skończyli
nocowaniem u jej najlepszej przyjaciółki po drugiej stronie ulicy, Rose,
która zmoczyła łóżko o~północy, obudziła je obie potokiem łez poniżenia.
Lacey umyła Rose, Joe załatwił materac, kładąc worek na śmieci na mokrym
miejscu, potem zakładając świeżą pościel.

Lacey od razu poszła spać, tak samo Rose. Joe nie mógł spać. Podreptał
do pokoju gościnnego z~laptopem i~włączył JebRaka.

Umarła dzisiaj {[}napisał LisasDad1990{]}.

Przepraszam za cały bałagan, który spowodowała moja ostatnia wiadomość.
Byłem w~dołku. Ale chciałbym podziękować wam wszystkim za rady, nawet
tym, którzy żartowali i~nie. Nawet na niektóre się roześmiałem. Wszystko
to utrzymywało mnie w~ruchu.

Pogrzeb jest w~przyszłym tygodniu.

Znacznik czasu był tylko sześć minut temu. Joe wysłał prywatną
wiadomość.

\textbf{ Jeżeli chcesz pogadać, nie śpię.}

Patrzył na ekran dłuższy czas, czekając na okienko, uderzając
odświeżenie, mając nadzieję, że LisasDad1990 odpowie.

Minuty się ciągnęły. Przeskoczył na zakładkę Facebooka, przejrzał,
wrócił. Spojrzał na oczekiwany czas dotarcia pudła z~Amazona, o~którym
właśnie sobie przypomniał, że na nie czeka. Spojrzał szybko na folder
Spamu w~Gmailu, co próbował zrobić raz w~miesiącu, przed skasowaniem.

Właśnie miał zamknąć laptop (lub prawdopodobnie zacząć robić to wszystko
znowu), kiedy wyskoczyła odpowiedź.

\textbf{ Dzięki. To nie była dobra noc. Pogrzeb w~przyszłym tygodniu. Wrzuciłem to na ostatnią kartę kredytową. Będzie
prawdziwe pożegnanie}

\textbf{ To brzmi jak godny sposób użycia tej karty. Ale
jak się trzymasz}

\textbf{ Powiem tak: nie planuję spłacać tej karty
kredytowej}

Żołądek Joego się zacisnął.

\textbf{ Wiem, że wysłałem ci ten kryzysowy numer telefonu
dla samobójców, ale znowu go wysyłam. 1-800-273 8255 Lub wyślij smsa
HOME na 741741 i~ktoś tam będzie. Oba są 24 na 7}

\textbf{ Dzwoniłem na ten numer. To miłe z~Twojej strony,
że go wysłałeś. Sądzę, że jesteś typem osoby, która chce, żeby wszyscy
mieli nadzieję. To fajny typ osoby. Lisa była takim typem osoby.}

Długa przerwa. Joe właśnie miał coś napisać. Wtedy LisasDad1990 zaczął
znowu pisać.

\textbf{ To było wszystko, o~czym potrafiła mówić.
Dzieciak właśnie skończył siedem lat i~wszystko, co mówiła, to ,,Tato,
musisz mieć nadzieję, wszystko będzie ok''. Chciała, żebym obejrzał
Annie, jeżeli możesz w~to uwierzyć. Głupi Netflix.}

To właściwie odruchowo rozśmieszyło Joego. Właściwie zachichotał. Czuł,
że on i~LisasDad1990 prawdopodobnie lubili swoje towarzystwo.

\textbf{ Miała rację.}

Wziął głęboki wdech.

\textbf{ Słuchaj, gdzie jesteś? Może Ty i~ja moglibyśmy
się spotkać. Jestem w~Phoenix.}

\textbf{ Nie ja. Żyje w~Cow's Asshole, Karolina
Południowa. Ale miło z~Twojej strony.}

\textbf{ A może telefon? Mógłby otworzyć piwo, Ty
otworzyłbyś piwo, i~moglibyśmy mieć piwo razem.}

\textbf{ Serio, jesteś miłym facetem. Dzięki za to. Innym
razem.}

Potem się wylogował.

Joe nie sądził, żeby mógł spać tej nocy, ale był mężczyzną, człowiekiem
zbliżającym się do wieku średniego, istoty ludzkie potrzebują snu, a~zatem spał.

Następnego dnia była niedziela, jego dzień na przygotowanie śniadania.
Udało mu się przejść przez to z~Madison i~jej przyjaciółką (które nocne
zażenowanie zblakło po kilku godzinach snu i~w świeżo wypranej pidżamie)
bez myślenia o~sprawdzaniu wiadomości z~Karoliny Południowej lub postów
na JebRaku.

Ale kiedy lekko upokorzeni rodzice Rose zabrali ją, Madison była w~swoim
pokoju bawiąc się i~oglądając YouTube, wyciągnął swój telefon i~poszukał
na Google za ,,samobójstwo karolina południowa''.

Wyniki szukania była puste. Oczywiście. LisasDad1990 nie zrobiłby tego
aż do zakończenia pogrzebu. Joe miał cały tydzień, żeby go uspokoić.

Praca nigdy nie była już taka sama po zachorowaniu Lacey. Nigdy nie
umówił się ponownie z~tym wice. Po prostu nie potrafił zebrać się
dostatecznie mocno, żeby popchnąć tę całościową dystrybucję na żądanie.
Ludzie, z~którymi zaczynał pracę, odchodzili pracować w~eksperymentalnych wydziałach zajmujących się partnerstwami z~przedsiębiorstwami autonomicznych wózków widłowych, lub zanurzali się w~samoobsługowe platformy dronów dostawczych opartych na chmurze, lub
wszystkie te inne rzeczy, które pomagało ludziom dostarczyć migające
LED-ami ładowarki USB czy podstawki pod nogi ,,Squatty Potty'' pod same
drzwi z~pewnością rzędu pięciu dziewiątek.

Joe osiadł w~swoim boksie i~robił dokładnie te rzeczy z~opisu pracy i~wychodził każdego dnia o~godzinie siedemnastej. Miał telefony służbowy i~osobisty, odbył wiele szkoleń na temat tego, że ruch z~telefonu
służbowego był zapisywany, nawet ,,bezpieczne'' połączenia, ponieważ
firma zrobiła coś w~systemie operacyjnym, że wszystko, co z~nim robił,
było widoczne dla zespołu bezpieczeństwa. Musieli to zrobić,
powiedzieli, dla ich ubezpieczyciela. Było to ważne, zatem, żeby ściśle
oddzielał czynności osobiste od działalności związanej z~pracą. Nikt nie
chciał czytać jego sex-smsów, czytać historii przeglądania, kiedy radził
sobie z~żenującym swędzeniem.

Na początku była to dla Joe prawdziwa walka, ponieważ był rodzajem
faceta, który lubił pociągnąć za rączkę w~maszynie losującej służbowego
emaila, sprawdzając, czy szef nie chciał, żeby zrobił jakiejś pracy po
godzinach. To znaczyło, że telefon, który najprawdopodobniej miał w~ręku
w dowolnej chwili, był tym telefonem służbowym. Czasem nawet zapominał
naładować telefon prywatny.

Ale wtedy, po tym, jak Kadry jego firmy powiedziały mu, że nie jest ich
pracą powstrzymanie ubezpieczyciela od zamordowania jego żony, stracił
motywację. Teraz to jego telefon służbowy zwykle był rozładowany.

Zatem w~biurze, użył prywatnego telefonu, żeby sprawdzić co z~LisasDad1990. Przeczytał FAQ o~tym, co mówić ludziom z~myślami
samobójczymi i~właśnie przerabiał całą listę. LisasDad1990 nie zawsze
odpowiadał, ale czasem odpowiadał, a~to znaczyło, że słuchał i~to
znaczyło, że Joe nie miał zamiaru się poddać.

Dni mijały do pogrzebu Lisy. LisasDad1990 zamieścił fotografie z~pogrzebu, smutne fotografie płaczącej rodziny, w~tym spustoszonej,
szczerbatej uzależnionej eks. Zdjęcia małej trumny, urny z~prochami.

Mężczyźni JebRaka byli pełni szacunku i~poważni na temat tych zdjęć.
Mieli taki zwyczaj, kiedy chcieli podkreślić powagę, zamieszczanie
wiadomości z~jednym pustym znakiem, więc post wyglądał na pusty, jak
\textit{brak mi słów}. Zdjęcia dostawały potężną liczbę takich odpowiedzi.

Ale post DeathEatera miał kolejną kopię \textit{Podręcznika Wysadzania}
dołączoną do niego.

LisasDad1990 nie odpowiedział więcej na prywatne wiadomości od Joego.
Joe wmówił sobie, że LisasDad1990 po prostu mądrze wziął czas wolny od
uzależniającego, toksycznego środowiska JebRaka i~pocieszał się realnymi
przyjaciółmi, którzy nie mieli obsesji na punkcie przemocy.

Jednak Joe odkrywał, że wyciągał prywatny telefon i~szukał fraz
,,bomba'' i~,,karolina południowa'' prawie automatycznie, gdy tylko
pozwolił myślom powędrować.

Był koniec maja i~Madison kończyła drugą klasę, szkoła robiła z~tego
dużą sprawę. Joe przegapił wersję z~przedszkola i~pierwszej klasy tej
ceremonii, ale już dłużej nie czuł tej palącej ambicji, która sprawiała,
że tak trudno było poprosić o~pół dnia wolnego na sprawy osobiste. Zatem
wsiadł do samochodu w~czasie lunchu, pojechał do szkoły, zatrzymując
się, żeby kupić kwiaty i~balony od Meksykanina, który chodził wzdłuż
kolejki rodziców czekających na zewnątrz audytorium z~wózkiem,
upominając ich, żeby ,,pokazali dzieciom, jak są z~nich dumni i~jak
ważna jest ich edukacja''. Zagrywka była dość oczywista, ale jednak
wciągnęła Joe w~jakieś mroczne instynkty winy rodzicielskiej.

Lacey była pierwsza i~zatrzymała dla niego miejsce, trzymali ręce, kiedy
dzieciaki przemawiały o~swoich snach o~przyszłości, śpiewały piosenki i~składały hołd nauczycielom. Dzieciaki były bardzo urocze w~ich małych
sukienkach i~garniturach, ale stało się to dość powtarzalne, Joe
wyłuskał telefon z~kieszeni i~dyskretnie uruchomił przeglądarkę.

Zwykle, ,,bomba karolina południowa'' zwracało strony o~nuklearnym
incydencie B47 w~Mars Bluff, kiedy Siły Powietrzne U.S.A. zbombardowały
nuklearnie miasto Mars Bluff w~1958 roku. Dzisiaj, na szczycie listy
była seria linków do Google News na temat historii eksplozji w~kwaterze
głównej BlueCross BlueShield w~Columbia w~Karolinie Południowej.

Śpiew na scenie, znudzone szepty i~poruszenia rodziców dookoła niego
natychmiast i~całkowicie znikły, jego pole widzenia skurczyło się do
ekranu. Stuknął w~link informacji z~AP, Lacey syknęła na niego, żeby
odłożył telefon, został zaskoczony winien i~podniósł wzrok, ale nie mógł
skupić wzroku.

-- Idę do toalety -- wyszeptał, przecisnął się do alejki, depcząc po
kolanach rodziców i~plecach krzeseł przed nimi, potykając się, potem w~hol szkoły obwieszony sztuką i~poezją, do łazienki dla chłopców z~miniaturowymi urynałami i~porysowanymi budkami. Wszedł do jednej z~nich
i przeczytał wiadomości z~Associated Press.

Nie było dużo szczegółów, tylko informacja z~ostatniej chwili, ale fakty
były druzgoczące: doszło do eksplozji w~kwaterze głównej BlueCross
BlueShield. Policja podejrzewała terroryzm. Już było potwierdzonych
dziesięć ofiar śmiertelnych oraz silne przypuszczenie, że ta liczba
wzrośnie, gdy zespoły ratunkowe sprawdzą gruzy. Obrazy budynku mdliły,
jak zepsuty ząb, wielka dziura, która wysadziła mury górnych pięter,
rozciągając się na trzy piętra. Obrazy kamer zebrane z~Twittera
pokazywały dym i~ogień oraz potykających, krwawiących ludzi w~ubraniach
biurowych.

Stał w~kabinie przez bóg wie jak długo, wpatrując się w~telefon,
pragnąc, że nigdy go nie wyciągnął, pragnąc, żeby to była bezsensowna
tajemnica, zamiast wydarzenia, które w~najmniejszym stopniu nie było
tajemnicze. Powinien zadzwonić na policję. Naprawdę powinien po prostu
zadzwonić na policję.

Na zewnątrz łazienki pojawił się hałas, dźwięki tłumu, potem drzwi do
toalety uderzyły przy otwarciu i~głosy dzieci wypełniły pomieszczenie.
Schował telefon i~wybiegł z~łazienki, ledwie słysząc ważniakowatego
chłopaka wołającego za nim o~umyciu dłoni.

Lacey odprowadziła go do samochodu po tym, jak zrobili sobie zdjęcia z~Madison i~dali jej wieniec i~balony, zostawiając ją z~mamą jej
przyjaciółki na wieczorne oglądanie filmu i~popcorn.

-- Co się stało?

Oczywiście nie zamierzał jej powiedzieć. Jezus, już było źle, że w~tym
był.

-- Wiesz, że forum dyskusyjne\ldots 

Przewróciła oczami. Nie aprobowała JebRaka.

-- Wiem, wiem. Ale tam jest facet, który przechodzi naprawdę trudne
chwile. Jego córka. W~tym samym wieku co Maddy, a~ja tylko chcę być tam
dla niego. Byli dla mnie, kiedy potrzebowałem. Spłacam ten dług.

Znowu przewróciła oczami, ale przyciągnęła go na długie przytulenie. 

-- Jesteś dobrym człowiekiem, Josephie Gorman. Tylko pamiętaj, nie możesz
nieść całego świata na ramionach, masz też rodzinę, która Cię
potrzebuję, nie wydawaj wszystkiego, co masz na obcych.

-- Dziękuję Lacey. Wiem, to dobra rada. Spróbuję utrzymać to wszystko w~perspektywie, ale, wiesz\ldots 

-- Wiem. Oczywiście, że wiem. Dlatego przypominam. Masz wielkie serce i~potrzebujesz pomocy, żeby się o~nie zatroszczyć. Poślubiłam tę pracę.

-- Kocham Cię.

Przytuliła go jeszcze dłużej, ściskając go ostro z~całej siły, która
wróciła do jej ciała po ich cudzie. Nawet gdy zauważał to, myślał o~telefonie w~kieszeni, zastanawiał się, gdzie mógłby go wyjąć, wejść
online, przeczytać wiadomości i~zalogować się do JebRaki.

LisasDad1990 opublikował video wyjaśniające, co zrobił, ale na szczęście
nie wspomniał o~JebRaku. Mimo tego wszyscy na forum byli zdenerwowani
jak cholera, było wiele cofania po historii i~kasowania, póki policjanci
nie ogłosili, że LisasDad1990 szeroko używał przeglądarki Tor i~nie
pozostawił żadnych informacji w~przeglądarce, ani zapisów w~centrach
danych AT\&T. Nieuchronnie uruchomiło to polowanie na czarownice w~sprawie ,,darkweb'' i~wszyscy zastanawiali się, gdzie ten tajemniczy
mężczyzna z~wideo został ,,zradykalizowany''.

LisasDad1990 był miękkim, lekko ciężkim mężczyzną o~smutnych oczach i~z trzydniową brodą. Na filmie mówił monotonnie, patrząc prosto w~kamerę
zaczerwienionymi oczami. Jego włosy były oklapnięte i~tłuste, kuchnia za
nim była chaosem, puste pudełka po lekach i~pudełka pizzy. Cichym,
spokojnym, dżentelmeńsko południowym głosem mówił o~decyzji, którą
BlueCross BlueShield podjęła, żeby odmówić jego córce pokrycia wydatków
i co to znaczyło. Podniósł fotografię uśmiechającej się dziewczynki,
brązowe włosy na elfa, brakujący ząb, rozsypane piegi na małym zadartym
nosie. Mówił o~historyjkach Lisy, rysunkach, które zrobiła, żeby im
towarzyszyły, kotku, którego uratowała i~pielęgnowała do wyzdrowienia,
jej nieutulonym żalu, kiedy kot został potrącony przez samochód. Mówił o~jej chorobie, odwadze, bólu i~obietnicy.

Mówił na żywo, żadnych notatek, a~kiedy skończył, przerwał na długo,
potem wytarł oczy kciukiem, otworzył i~zamknął usta kilka razy, nie
mogąc nic powiedzieć, wciągnął drżąco, głęboko powietrze i~odzyskał
równowagę.

-- Zatem dlatego to robię. To nie zemsta. Nie mam naturalnych skłonności
do zemsty. Nic, co zrobię, nie zwróci jej życia, więc dlaczego miałbym
chcieć się mścić?

-- To jest służba publiczna. Gdzieś jest inny ojciec taki jak ja i~inna
dziewczynka tak jak\ldots  -- Kolejna chwila na odzyskanie równowagi. -- Taka
jak Lisa. A właśnie teraz, ten tata rozmawia z~kimś w~Cigna, czy Humana,
czy BlueCross BlueShield i~osoba na telefonie mówi temu tacie, że jego
dziewczynka Musi. Umrzeć.

-- Ktoś w~tym budynku zdecydował się zabić moją dziewczynkę, a~wszyscy
inni w~tym budynku pogodzili się z~tym. Żaden z~nich nie jest niewinny i~nikt z~nich się nie boi. Zaczną się bać, po tym. Po dzisiaj, każdy z~tych ludzi spędzi resztę swojego życia, oglądając się nad ramieniem za
ludźmi takimi jak ja. Wyglądającymi zwyczajnie. Nieszkodliwymi. Może
trochę smutnymi.

-- Ponieważ musieli wiedzieć w~swoich sercach. Oni, ich lobbyści, ich
ludzie w~Kongresie, którzy to umożliwili. Są rodzicami. Wiedzą. Każdy,
kto skrzywdziłby ich drogie dzieci, dopadliby taką osobę jak pies.
Jedyną niesamowitą rzeczą w~tym jest to, że nikt jeszcze tego nie
zrobił.

-- Teraz dokonam prognozy, że chociaż jestem pierwszy, jestem pewien jak
cholera, że nie będę ostatni. Nadejdzie więcej. Do tych, którzy nadejdą
po mnie, ojców i~mężów, matek i~żon, dziadków i~kochanków, pozdrawiam
Was. Zamierzamy ich przestraszyć, zamierzamy ich \textit{tak przerazić},
że już nigdy nie prześpią nocy. Naprawią to zło, tę plamę na naszym
kraju, nie dlatego, że kochają nasze dzieci tak, jak my je kochamy, ale
ponieważ \textit{przerazimy} ich do tego.

Patrzył w~kamerę jeszcze przez chwilę, jego oczy płonące i~błyszczące.
Potem skinął głową, wstał i~wyszedł z~kadru. Kiedy wrócił chwilę
później, miał paczkę oklejoną taśmą wielkości garnka, który ostrożnie
uniósł i~włożył do plecaka. Kiwnął głową raz jeszcze, założył plecak i~wyłączył kamerę.

Przyszło Joemu do głowy, że nawet LisasDad1990 -- na imię miał Saul, ale
Joe myślał o~nim ciągle jako LisasDad1990 -- mówił tyle o~,,strachu''
nikt w~telewizji czy w~wiadomościach nie wspominał o~,,terroryzmie''.
Były jakieś typki na Twitterze, którzy podkreślali, że LisasDad1990 miał
zły kolor skóry, żeby być terrorystą i~zamiast tego ,,nie był w~porządku
w głowie'' lub ,,umysłowo chory'' lub nawet ,,straumatyzowany''. Nawet
rodziny ludzi zabitych w~eksplozji w~Columbia nie używały słowa na
,,T'', choć wielu nazywało go mordercą, potworem i~gorzej.

Z całą pewnością nikt spośród JebRaków nie chciał szeptać słowa na
,,T''.

Nastąpiła gorączkowa debata, która krążyła wokół słowa na ,,T'', wokół
tego, kiedy jest ok, żeby ,,wygłaszać'' lub ,,fantazjować'' o~przemocy,
a kiedy to przekraczało linię. BigTed mógł mieć własne zdanie na ten
temat, ale dwadzieścia cztery godziny po tym jak LisasDad1990 zabił
siebie i~wszystkich tych ludzi, BigTed ogłosił, że zamierza przestać
administrować JebRakiem i~że przekazuje pochodnię DeathEaterowi, która
ma ,,czas i~energię'', żeby ,,zwrócić uwagę na forum, której
potrzebuje''.

Kiedy DeathEater został panem i~władcą JebRaka, debata się skończyła.
DeathEater zadeklarował, że był tylko jeden sposób, żeby uczestniczyć w~JebRaku i~było to stanąć wobec smutku we wszystkich jego przebraniach, w~tym gniew, być autentycznym i~wiernym sobie.

DeathEater również ogłosił, że większość forum na JebRaku przenosi się
na TOR, że będą potrzebować przeglądarki Tora do czytania i~ustanowił
skomplikowany protokół, żeby umożliwić im odzyskać swoje nazwy
użytkowników na JebRaka w~DarkWebie, jak się nazwał nowy serwer.

Joe ignorował to przez chwilę. Poczuł nawet potajemnie ulgę. Jeżeli
wszyscy najgorsi z~najgorszych złamanych JebRaków zniknęli w~Darkwebie,
to mógłby siedzieć po stronie, która była ciągle widoczna w~otwartej
sieci, działać jako wróżka chrzestna dla ludzi, którzy jeszcze nie byli
poza pomocą.

Jednak DeathEater był zdeterminowany, by wciągnąć tak wiele dusz, ile
mógł. Za każdym razem, kiedy wątek zaczynał cieszyć się popularnością na
JebRaku, zamykał go i~ogłaszał, że przenosi go do Darkwebu. Nowi
podążyli za nim. Za trzecim razem, kiedy to się zdarzyło, Joe zrozumiał,
że DeathEater utrzymywał JebRaka jako narkotyk Ciemności. Ciemność
\textit{była} JebRaką.

Kiedy to zrozumiał, Joe zerwał z~JebRakiem. Zaparkował konto i~odstąpił
od niego.

Drugi zamachowiec uderzył w~republikańskiego senatora stanu Tennessee,
który głosował przeciwko rozszerzeniu Medicare, pomimo obietnicy
złożonej w~kampanii, że ,,każdy mieszkaniec stanu Tennessee, który chce
ubezpieczenia, otrzyma ubezpieczenie''.

Zamachowiec nazywał się Logan Lents, a~jego rodzina była w~Tennessee, od
kiedy stało się stanem w~1796 roku. Mimo że pochodził z~bogatej rodziny,
był bankrutem, jego rodzice stracili fortunę rodzinną, kiedy był
chłopcem, a~on miał stypendium na TSU, gdzie immatrykulowało się sześć
pokoleń jego przodków, był przewodniczącym Phi Beta Sigma, jak jego
ojciec, dziadek i~pradziadek.

Logan Lents był wdowcem po Patricii Lents, też z~długiej i~upadłej linii
Tennessee, której rak macicy był uleczalny (według jej ginekologa) lub
nie (według ubezpieczyciela z~Cigna). Wcześnie w~chorobie Patricia
straciła dziecko, które wszyscy w~tajemnicy traktowali jako
błogosławieństwo, przy jej chorobie i~tak dalej, a~kiedy zmarła rok
później, w~wieku dwudziestu sześć lat, Logan był załamany.

Logan wysadził biuro senatora Williama Blounta w~piękną sobotę, po
całodniowych spotkaniach z~wyborcami podczas przerwy wiosennej. Logan
był na tyle rozsądny, że poczekał, aby wyborcy i~pracownicy, którzy tam
byli w~pracy, opuścili biuro i~zabił tylko siebie, senatora i~policjanta
z Policji Stanu Tennessee. Kolejny policjant i~woźny zostali okaleczeni,
ale przeżyli.

Mężczyźni w~Dark JebRaku uznali to za ,,chirurgiczną'' misję i~chwalili
,,czyste'' wykonanie.

Film Logana był naprawdę dobrze zrobiony, elokwentny na sposób kogoś ze
szlachetnej, starej rodziny, który kiedyś przemowami zapewnił sobie
urząd prezydenta ekskluzywnego bractwa. Posiadał te cechy amerykańskiego
bramina, jak południowy Kennedy, i~łatwo można było zapomnieć, że
zamierza wysadzić się razem z~każdym stojącym w~pobliżu, żeby coś
podkreślić, to znaczy, że opieka zdrowotna jest prawem człowieka i~że
źli ludzi spiskowali, żeby zabrać ją wielu Amerykanom, co oznaczało, że
umrą.

Trzecim zamachowcem był DeathEater.

Joe i~DeathEater wkroczyli na drogę wielu otwartych działań wojennych w~tygodniach po śmierci Logana. DeathEater miał dostęp do logów JebRaka,
zatem mógł zobaczyć, jak Joe wysyła prywatne wiadomości do wszystkich,
który DeathEater zachęcał, a, będąc adminem i~tak dalej, był w~stanie
przeczytać te wiadomości i~zrozumieć, że Joe zamienił się w~anioła-powiernika, by szeptać antidotum na całą truciznę DeathEatera.

DeathEater skonfrontował to z~nim w~prywatnym czacie, który zaczął się
ostro, ale szybko wygasł, gdy DeathEater opowiedział swoją historię,
dorosły syn zmarły od choroby metabolicznej, której terapia
farmaceutyczna była uznawana za ,,eksperymentalną'', żona zmarła od
,,złamania serca'' w~ciągu roku, on siedemdziesięcioczterolatek,
oszukany na zmianie z~publicznego Medicare na prywatne HMO\footnote{ rodzaj
instytucji ubezpieczeniowo-finansowych, których celem jest zapewnienie
finansowania opieki zdrowotnej członkom,
zob.~\url{https://en.wikipedia.org/wiki/Health\_maintenance\_organization}
-- przyp.tłum.}, które powiedziało, że dializy, które robił przez
ostatnie dwadzieścia lat, są ,,poza siecią'' i~skierowało go na
alternatywną terapię, która sprawiała mu ciągły ból.

\textbf{ Stary biały gostek w~wózku inwalidzkim, nikt nie
przeszukuje takich}

Joe odgrywał wersję tej rozmowy tak wiele razy i~zwykle wychodziła
całkiem dobrze. W~końcu, były tylko na razie dwie bomby, a~Joe nie
sądził, żeby kiedykolwiek rozmawiał z~Loganem.

\textbf{ Wszystko, co zrobisz, to sprawisz, że cała grupa
kobiet i~mężczyzn będzie nieszczęśliwa przez całe życie, zostawisz
mnóstwo dzieci bez mamy i~taty}

\textbf{ Ta, właśnie taki mam zamiar. Wydaje się pasować.}

Teraz co? \textit{Dwa złe czyny nie robią dobrego?} To jest Mroczny
JebRak, nie Pinokio.

\textbf{ Zadzwonisz wcześniej? Jeżeli nie potrafisz
powiedzieć żywej osobie, co zamierzasz zrobić, jak możesz mówić, że sam
jesteś dostatecznie pewny }

\textbf{ Pomyślę o~tym}

\textbf{ Zadzwoń}

\textbf{ Zadzwonię}

I potem telefon Joego zadzwonił, o~szóstej rano, Joe powiedział: 

-- Halo, halo? -- póki nie domyślił się, że hałasy w~tle są wózkiem inwalidzkim i~ktoś ustawia laptop, żeby nagrać wideo prosto z~kamery. Głos był
chrapliwy, ale silny od emocji.

-- Zamierzam unieszczęśliwić grupę kobiet i~mężczyzn, sprawić, żeby
rozpaczali przez resztę swoich żyć za mężami i~żonami, których dzisiaj
stracą. Wiele dzieci nigdy już nie ujrzy swoich rodziców.

-- Nie jestem z~tego dumny. Szczerze wam współczuję. Ale muszę to zrobić.
Dość tego. Ludzie, których zamierzam dzisiaj zabić, są częścią maszyny,
która każdego dnia, każdego roku, kosztuje tak wielu z~nas nasze żony,
mężów, rodziców i~dzieci. Obserwujemy, jak umierają złą, wolną, bolesną
śmiercią i~dlaczego? Ponieważ zawsze ktoś pracuje, żeby pilnować
pieniędzy, a~nikt nie pracuje, żeby utrzymać tych ludzi przy życiu,
tych, którzy jeszcze nie muszą umierać.

-- Gdzieś po drodze, muszą być konsekwencje. Dla dilerów metamfetaminy
pracuje wielu dobrych ludzi, tylko próbując przetrwać. Sądzę, że są
dobrzy ludzie, którzy muszą zarobić na życie, którzy pracują dla
handlarzy ludzi. Aresztujemy tych ludzi, wysyłamy ich do więzienia na
resztę ich życia, mimo tego, że po prostu próbują żyć jak reszta z~nas.
Dlaczego praca dla legalnego mordercy oznacza, że jesteś niewinny? Że
wyjdziesz bez szwanku?

-- Jeżeli pracujesz dla korporacji ubezpieczeń na zdrowie, lub ich
lobbystów, lub senatora czy kongresmena, którzy głosują przeciwko opiece
zdrowotnej dla wszystkich, chcę, żebyś się bał. Przeraził się wyjścia z~domu. Zbyt przerażony, by spać. Chcę, żebyście leżeli przebudzeni w~nocy, czując fale strachu, za każdym razem, gdy usłyszycie trzask. Chcę,
żebyście zdobyli pozwolenie na noszenie ukrytej broni, strzelbę przy
łóżku i~każdego ranka ciągle przyłapywali się na zastanawianiu, czy to
dzisiaj nadejdzie ten dzień.

-- Jeżeli nie możecie tego znieść, zmieńcie pracę. Powiedzcie szefowi, że
nie godziliście się na wysadzenie przez jakiegoś oszalałego z~rozpaczy
zamachowca samobójcę. W~końcu, ci dyrektorzy ubezpieczalni, lobbyści i~politycy będą musieli przejść do planu B, czyli opieki zdrowotnej dla
wszystkich.

-- Mówią, że przemoc nic nie rozwiązuje niczego, ale cytując \textit{The
Onion}\footnote{ amerykańskie czasopismo satyryczne,
zob.~\url{https://en.wikipedia.org/wiki/The\_Onion} -- przyp.tłum.}: ,,To prawda, o~ile zignorujesz całą historię ludzkości''.
Przemoc jest jedynym sposobem, żeby zwrócić uwagę niektórych ludzi.
Wiecie, których mam na myśli.

-- Brady, przepraszam, synu. Zasługujesz na lepsze dziedzictwo. Marla, Ty
też. Oboje zasługujecie na lepsze, ale to wszystko, co mam. Kocham was.
Wkrótce was zobaczę.

DeathEater kliknął coś i~wideo się skończyło, ale nadal mówił do Joego,
monologując, gdy jeździł wózkiem po ciasnym domu, przygotowując się do
wysadzenia siebie w~powietrze.

-- Mam nadzieję, że wszystko to słyszałeś, Joe. Przepraszam za
zostawienie śladu wskazującego na Ciebie, ale prosiłeś, żebym do Ciebie
zadzwonił. Wydrukowałem wszystkie logi z~czatów, żeby mogli zobaczyć, że
próbowałeś mnie cały czas zatrzymać. -- Szum, gdy podnosił telefon, jego
głos głośniejszy i~wyraźniejszy. -- Dzięki za próbowanie, Joe.
Tikitiki6538 będzie od teraz kierował sprawami.

Joe próbował coś mu powiedzieć, ale nie był pewien, czy wideo przestało
nagrywać, wiedział, że pewnego dnia, wkrótce, będzie musiał wyjaśnić
to połączenie pewnym, bardzo poważnym ludziom z~federalnych agencji
śledczych, a~cokolwiek by im powiedział, musiałoby być do pogodzenia z~tym fragmentem głosu, jaki był na nagraniu.

I zresztą DeathEater się rozłączył.

Joe zaczął wybierać ,,91''\ldots  i~jego palec zawisł nad ,,1''. Była
trzecia w~nocy w~Phoenix, szósta rano w~dowolnym mieście na Wschodnim
Wybrzeżu, w~którym żył DeathEater (Clearwater, dowiedział się później).
Lacey i~Maddy mocno spały, oddech klimatyzacji zagłuszający ich
miękkie,lekkie chrapanie. Próbował wyobrazić sobie, co wydarzyłoby się,
gdyby nacisnął ,,1'' i~porozmawiał z~operatorem w~Phoenix, wyjaśnił, że
ktoś, kogo nazwiska nie zna w~mieście, którego nie zna, planuje zamach
bombowy. Kolejny zamach bombowy. Próbował sobie wyobrazić policjantów z~Departamentu Policji w~Tempe, którzy przyszliby do niego, rozmowę, którą
by odbyli, rozmowę, którą musieliby odbyć z~Lacey.

Nie mógłby tego zrobić.

Zajrzał do Maddy, poprawił jej kołdrę, pogładził włosy, pocałował w~czoło. Potem położył się koło Lacey, jej zapach unoszący się spod
pościeli, gdy się wczołgiwał i~przez długi czas patrzył w~sufit. Musiał
w końcu zapaść w~sen, ponieważ kolejną rzeczą, jaką zapamiętał, pokój
był wypełniony światłem słonecznym, Lacey i~Maddy kłócące, czy Maddy
może obejrzeć YouTube przez śniadaniem (nie mogła) i~Joe z~telefonem w~dłoni, przeglądający nagłówki.

DeathEater wjechał na wózku na konferencję ubezpieczycieli zdrowotnych w~Sheratonie, wielkie targi, które wystawiły dodatkową ochronę, ponieważ
ludzie, którzy tam byli, już się trochę bali. Jednak DeathEater
zarezerwował pokój tygodnie wcześniej, zapłacił za parkowanie samochodu,
boy pomógł mu usiąść w~wózku i~powiesić jego plecak z~tyłu wózka, potem
DeathEater pojechał samodzielnie, zameldował się, pojechał ku windom,
pokazując klucz do pokoju przy prywatnych ochroniarzach, którzy
zatrzymywali wszystkich, którzy próbowali przejść koło sal bankietowych.
Nikt nie chciał przeszukiwać starego, białego mężczyzny z~kluczem do
pokoju, ubranego w~wesoło kolorowaną koszulę hawajską i~zniszczony
kapelusz słomkowy, blade, chude nogi wystające z~szortów.

Wymierzył swoje przybycie dziesięć minut przed poranną sesją, kiedy
wszyscy uczestnicy konferencji kręcili się przed wielką salą, pijąc
kawę, jedząc muffiny i~dyskutując. Wjechał w~sam środek tłumu i~\ldots 

Liczba ofiar śmiertelnych była widowiskowa.

Lacey nie wiedziała, co jest nie tak z~Joe, ale wiedziała, że \textit{coś}
się dzieje.

-- Żadnych więcej forów dyskusyjnych, Joseph -- powiedziała surowo.

-- Zbyt dużo czasu przy ekranie, tata -- powiedziała Maddy, doskonałe
wrażenie jej matki, niesamowite z~powodu ich narastającego podobieństwa
i ostatniej pasującej mama-córka fryzury z~niebieskimi rozświetleniami.

Podniósł Maddy, mocno ją przytulił, podczas gdy ona piszczała, kopała i~śmiała. 

-- Ok, dzieciaku -- powiedział i~napotkał spojrzenie Lacey. Lacey
wyglądała na zmartwioną.

Joe odwoził do szkoły. Lacey wróciła do pracy, lądując w~pracy w~call
center, która załatwiała problemy z~rezerwacjami dla wielkiej sieci
hoteli, a~Joe przesunął swoje godziny w~taki sposób, że mógł odwozić
córkę do szkoły rano, a~Lacey mogła ją odbierać wieczorem. To nie był
dobry ruch zawodowy -- istniała bezpośrednia korelacja pomiędzy byciem
przy biurku punkt ósma rano a~awansowaniem -- ale od dnia diagnozy Lacey,
cała jego pasja do kariery wypłynęła z~niego i~została zastąpiona przez
równie silne poczucie, że jego czas z~rodziną był przelotną chwilę do
smakowania.

W drodze do szkoły, Maddy chciała porozmawiać o~tym, kiedy mama
zachorowała, co było tematem, który się często pojawiał. Ani Joe ani
Lacey nie byli religijni, ale nieuchronnie, Maddy miała przyjaciółkę w~szkole, która chciała, żeby wiedziała, że Bóg uratował jej mamę i~że
powinna dziękować Bogu. Ciągle do tego wracała i~była widzialnie,
oczywiście przestraszona, że nie robili wystarczająco dużo w~tej sprawie
i może Bóg zmieniłby zdanie.

-- Zatem wiesz, że nie wierzę w~Boga, tak, córciu?

Rzucił szybkie spojrzenie od powolnego ruchu przed sobą i~spojrzał na
nią. Pokiwała głową uroczyście.

-- Myślę, że mama miała tylko bardzo, bardzo dużo szczęścia. Jak
wyrzucenie dwóch szóstek w~Monopolu, dziesięć razy z~rzędu. Ale to nie
znaczy, że nie powinniśmy być wdzięczni, że jest z~nami. Budzę się
każdego dnia i~jestem wdzięczny. A Ty?

-- Też. -- Jej głos był cichy.

-- Ta. Zatem jeżeli jest Bóg, to Ona lub On -- \textit{Ona lub On} było
frazą, na którą nalegała Lacey, kiedy rozmawiali o~Bogu z~Maddy -- wie,
jak się czujesz.

Usłyszał jej płacz z~tylnego siedzenia i~zjechał na pobocze. 

-- Co jest,
kochanie? Dlaczego płaczesz?

Pociągnęła nosem i~dał jej chusteczkę. 

-- Czasem nie jestem wdzięczna.
Czasem jestem wściekła na nią, ponieważ nie pozwoli mi nosić tego, co
chcę lub mówi, że nie umyłam dobrze zębów. Co jeżeli Bóg widzi, że nie
jestem wdzięczna i~ją zabierze?

Po raz dziesięciomilionowy Joe przeklął ewangelizującego dzieciaka w~klasie Maddy, które wsadzał te głupoty w~jej niewinną, straumatyzowaną
głowę.

-- Cóż, córciu, jestem niewłaściwym człowiekiem, żeby na to odpowiedzieć,
ponieważ wiesz, że nie wierzę, że Bóg istnieje. Jednak nie oczekuję, że
nie będziesz się denerwować na mamę. Czasem ja też się na nią denerwuję.
To się zdarza, nawet gdy ludzie się kochają. Szczególnie kiedy ludzie
się kochają! Kochanie kogoś to trudna sprawa. Ważne, że mówisz o~rzeczach, kiedy jesteś zdenerwowana i~nad nimi pracujesz. Wiesz, że
nawet kiedy jesteśmy naprawdę rozzłoszczeni na siebie, to zawsze jakoś
nam wychodzi i~zawsze kończymy, znowu się kochając, prawda? Zatem to, co
musimy zrobić, to tylko skoncentrować się na minięciu szalonej części i~powrocie do szczęśliwej części. Jestem pewien, że jeżeli jest Bóg, to
jest to wszystko, czego Ona lub On oczekuje po Tobie.

Maddy wydawała się zadowolona z~takiej odpowiedzi, więc Joe włączył bieg
i dołączył do ruchu. Kiedy dojechali do kolejnego zestawu świateł,
śpiewała chór z~,,Yellow Submarine'' raz po raz, co zawsze było dobrym
znakiem. Zostawił ją przed domem, posłała mu całusa i~potem wróciła po
zapomniany lunch i~znowu posłała mu całusa.

Joe pojechał do pracy. Próbował skupić się na córce i~co mógłby jej
powiedzieć następnym razem, gdy będzie miała takie pytania. Próbował nie
myśleć o~FBI pozyskującej listę telefonów i~prowadzących badania
kryminalne na jego komputerze. Próbował nie myśleć o~rodzicach w~Clearwater, Florida, którzy będą musieli wyjaśniać dzieciom przez
kolejne lata, dlaczego dziwny, gniewny, biały starzec w~wózku
inwalidzkim zamordował ich matkę. Nie udało się.

-- Zamierzasz powiedzieć mi, co jest nie tak, Joseph, czy będę musiała
zaciągnąć Cię na terapię dla par?

Tydzień od czasu ataku DeathEatera stawał się coraz gorszy. Każdy
dzwoniący telefon, każda dziwna twarz sprawiała, że podskakiwał, pocił,
myślał o~jakimś typku z~DHS, szeryfie federalnym czy agencie FBI. Nie
zalogował się na JebRaka, co było trudne, ponieważ to była jego
zwyczajowa terapia na sen o~drugiej w~nocy i~przeżywał wiele nocy
przebudzony o~drugiej w~nocy.

Lacey zapytała się go kilka razy, co się dzieje, a~on wydał niejasne
dźwięki o~stresie w~pracy, co było śmieszne, biorąc pod uwagę, jak mało
dbał o~pracę. W~końcu, doszło do tego. Byli w~łóżku, Maddy spała, a~on
leżał na boku, próbując nie zgrzytać zębami, jego umysł krążący od
DeathEatera do FBI, do martwych ludzi w~Clearwater i~znowu do Lacey.

Przewrócił się na boku. 

-- Przepraszam, Lace. To tylko jakieś gówna w~moim życiu, z~których nie byłem w~stanie się otrząsnąć i~nad którymi nie
chcę się rozwodzić. Ostatnią rzeczą, jaką chcę to powtarzać to z~Tobą. -- \textit{I żebyś stała się współwinną przestępstwu} -- Po prostu chcę to
wyrzucić z~umysłu.

Spojrzała się na niego z~czystym sceptycyzmem. 

-- Naprawdę robisz z~tego
gównianą robotę, jeżeli nie masz nic przeciwko, że mówię. Robienie tej
samej rzeczy raz po raz i~oczekiwanie innych wyników to definicja
szaleństwa. Kocham Cię bardzo, mój mężu, ale jeżeli nie możesz rozgryźć
jak poradzić sobie z~tym sam i~nie chcesz, żebym Ci pomogła i~nie chcesz
pomocy od nikogo innego, to Ty i~ja będziemy mieć poważny, jebany
problem. -- Lacey rzadko przeklinała.

-- Rozumiem -- powiedział. -- Daj mi czterdzieści osiem godzin?
Prawdopodobnie muszę ściągnąć tę apkę do medytacji lub iść pobiegać, czy
coś.

-- Czterdzieści osiem godzin i~potem albo wszystko mi mówisz albo
zaciągnę Cię do psychiatry.

Joe zrobił najodważniejszą minę.

-- Umowa stoi.

Medytacja okazała się niemożliwa. Nie było sposobu, żeby przestał myśleć
o filmie DeathEatera, słowach, które wymienili, myśli o~wszystkich tych
rodzinach rozdartych. Jednak poszedł pobiegać, przypinając telefon do
bicepsa i~zakładając słuchawki.

Była późna jesień i~gorąco, choć nie ogniście gorąco, zagubił się w~rytmie uderzeń stóp i~Creedence Clearwater Revival w~słuchawkach.

Efekt stracenia siebie był prawie cudem, ciężar zniknął z~jego klatki i~ramion. Dźwięki były ostrzejsze, zapachy lepsze. Przyśpieszył.

Nieuchronnie, przesadził. Zbyt dużo biegania po zbyt długiej przerwie.
Potknął się i~prawie przewrócił, a~kiedy się wyprostował, kręciło mu się
w głowie i~znowu prawie stracił równowagę. Był w~miłej strefie handlowej
dla pieszych, znalazł ławkę i~usiadł na niej, nogi drżące, równowaga
wirująca.

Oparł głowę na rękach, aż wirowanie się zatrzymało, potem oparł się, pot
spływający po jego twarzy i~z powrotem. Jego playlista się skończyła,
wyjął telefon, zaczął dotykać za kolejną serią muzyki na wolniejszy bieg
do domu, i~automatycznie spojrzał na newsy.

Strzelec ostrzelał senatora Grahama -- z~okręgu Joego -- na schodach
Kapitolu. Graham wygrał miejsce obietnicami ,,rozważenia wszystkich
opcji'' i~,,naprawienia bałaganu'' w~opiece zdrowotnej, potem dołączył
do bloku obstrukcjonistów w~Senacie, głosując zgodnie z~zaleceniami
kierownictwa partii, która blokowała każdą propozycję, nieważne, jak
skromną. Po pierwszym roku kadencji, zrezygnował z~wszystkich spotkań w~ratuszach i~Politico zrobiło reportaż z~danymi, który pokazywał, że był
senatorem, z~którym najtrudniej było się wyborcom spotkać czy to w~domu,
czy w~Waszyngtonie.

Strzelec postrzelił dwóch policjantów D.C., szeryfa federalnego, zranił
dziewięć postronnych osób, trafił w~środek szefa sztabu senatora i~zerwał kawałek płatka ucha senatora przy pomocy zmodyfikowanego
automatycznego AR-15. To wszystko stało się w~osiemnaście sekund, czyli
tyle ile zajęło hordzie policjantów z~sześciu agencji wystrzelić do
niego czterdzieści siedem razy. Senator początkowo był zgłoszony jako
poważnie ranny, ale okazało się, że był tylko pokryty w~litrach krwi
innych ludzi i~w histerii.

Strzelec załadował wideo samobójcy do YouTube przed wyjściem z~pokoju
hotelu w~Adams-Morgan i~choć Google zdjął je w~ciągu minut przekazywania
linków na Twitterze i~Facebooku, już zostało skopiowane na kilka
zagranicznych serwerów mniejszych platform i~było wielokrotnie
odtwarzane z~hashtagiem \#MacieSieBac, który zaczął się przy drugim
zamachu bombowym.

Joe czytał i~czytał, a~jego drżenie nie znikało. Czy strzelec miał konto
na JebRaku? Czy żył niedaleko Joego? Kiedy federalni pojawią się u jego
drzwi? Przerażona, zakrwawiona twarz senatora była idealnym contentem,
to leciało po sieci jeszcze szybciej niż wideo samobójcy.

Pot, który pojawił się od biegania, zamienił się w~lód.

Gadające głowy spędziły tygodnie, zastanawiając się, czy prezydent był
świadom tego zabijania. Oprócz kilku chłodnych zdań sympatii dla rodzin
ofiar rozgrywał to bardzo tajemniczo. Jego przedstawiciele chwalili go
za powściągliwość i~sprytną odmowę pochwalenia tych ,,żałosnych, chorych
ludzi'', którzy popełnili te okrucieństwa, aby nie zainspirować innych
do podążania ich śladami.

Ale przy strzelaninie na progu, Prezydent nie mógł sobie pozwolić na
ciszę. Był w~drodze do domu ze szczytu w~Cape Town w~Air Force One,
kiedy pojawiły się wiadomości, po naradach z~zespołem komunikacji,
usiadł z~korpusem prasy i~przedstawił niezdecydowane, chaotyczne
stanowisko, które wędrowało od ,,najlepszej opieki zdrowotnej na
świecie'' w~Ameryce, zła ,,kultury dostawania za darmo'', trwającym
projekcie ,,posprzątania po katastrofie Obamacare'' przed przejściem do
głównej atrakcji: nazwania strzelca ,,terrorystą''.

Słowo zadzwoniło w~uszach Joego. Jego słuchawki stały się mokre i~swędzące. Był nastolatkiem w~trakcie ,,9/11'' i~jego najsilniejszym
wspomnieniem z~tego czasu było to nowe słowo ,,terrorysta'' i~jego
dziwny oddźwięk, jakby opisywało orka, wygłodniałego i~bezlitosnego, na
raz nadludzkiego i~podludzkiego. To było słowo kurczące jaja. Słyszał
wcześniej o~białych ludzi nazywanych terrorystami, ale były to tylko
dziwne dzieciaki z~Berkeley, które uciekły z~domu, żeby dołączyć do
guerrila dżihadu na Bliskim Wschodzie.

To było trzaśnięcie drzwi, na które czekał, moment, z~którego nigdy już
nie mógł wrócić. Wiedział, że raz ,,terroryzm'' przyklei się do Ciebie,
nie może być odlepione. Rozprzestrzenia się od osoby, na którą trafiło,
na osoby dookoła niej. Nawet podróżowało w~czasie i~przylepiało się do
ludzi, których terroryści znali. Na przykład, jeżeli zdarzyło Ci się
spędzić dużo czasu na forum dyskusyjnym, gdzie był planowany terroryzm,
to prawdopodobnie też byłeś terrorystą. A jeżeli to forum było w~darkwebie? Zapomnij.

Jego bieg do domu był osobliwie beztroski. Bał się, czekając, żeby to
się zdarzyło. Teraz to się zdarzyło. Dzwon zadzwonił i~on nie mógł tego
zmienić. Nic, co zrobił na JebRaku, żadna ilość cierpliwej rozmowy lub
wyraźnych wezwań do rozsądku nie zmieniłaby tego. Był, w~pewnym dziwnym
znaczeniu, wolny. Co więcej, to mogła być ostatnie chwile wolności,
ponieważ teraz zdecydowanie to byliby federalni.

Lacey robiła jogę przed telewizorem, kiedy wrócił do domu. Zapytała go
sceptycznie, czy wszystko ok, musiał być przekonujący, ponieważ skazała
go na prysznic i~zapowiedziała, że idą na targ farmerów na sobotni
brunch.

To była słodka sobota. Nawet Maddy, która nienawidziła targu, bawiła się
dobrze, dzięki rozluźnieniu zasad rodziny w~sprawie ciastek i~placków.
Czerpał niesamowitą przyjemność z~obserwowania jej wpychającej usta w~pieczone dobra i~potem biegnącej na malowanie twarzy. Lacey burczała
trochę o~odpuszczenie, ale wyraźnie była bardzo zadowolona widokiem
powrotu starego dobrego Joe.

Kiedy Joe obudził się o~drugiej w~nocy na siku, bez zastanowienia
poszedł do pokoju i~uruchomił przeglądarkę Tor, kierując się prosto na
JebRaka. Jego lęk o~logowanie zniknął. Tutaj był po drugiej stronie bez
powrotu, nie było powodu, żeby nie odwiedzić swoich starych przyjaciół.
Rzeczy były takie, jakie były i~nic, co zrobiłby od teraz, nie
poprawiłoby ich ani nie pogorszyłoby.

Jednak JebRaka nie było, obie wersje, jawna i~dark, były 404, strona nie
odnaleziona.

I to go zdenerwowało.

Ponieważ, zrozumiał, JebRacy byli jego ludźmi. Byli jego społecznością.
Część z~nich przerażała go do szpiku kości i~niektórzy z~nich sprawiali,
że chciał uderzyć w~ścianę, ale to było jego miejsce w~świecie cyfrowym,
miejsce, gdzie prawda, którą czuł głęboko w~kościach, była powszechnie
przyjmowana. Większość Amerykanów wiedziała, że system opieki zdrowotnej
był zjebany, nawet wiedzieli, że dyrektorzy przemysłu zdrowotnej i~politycy, których kupili tanio, stali za tym. Jednak tylko JebRaki
zrozumieli, jak \textit{istotne }były te dwa fakty i~jak \textit{złe} były.
Joe nie chciał nikogo zabijać, ale głęboko w~środku, wiedział, że było
mnóstwo osób, które uzasadniały zabijanie.

W następnym tygodniu były dwa zamachy bombowe i~strzelanina. Mili, biały
faceci z~klasy średniej obserwowali śmierć żon i~dzieci, lub sami żyli z~wyrokiem śmierci.

Wiadomości były pełne historii o~punktach kontrolnych ustawianych na
zewnątrz szpitali i~budynków korporacji ubezpieczeniowych, biur
polityków, a~wtedy trzech facetów w~trzech miastach bez oczywistej
powiązania z~jednego z~drugim poszło i~wysadziło na trzech tych
punktach, wysadzając wielkie tłumy miłych ludzi z~dobrymi pracami,
którzy czekali na przejście skanerów, żeby mogli zarobić na utrzymanie
tych rodzin. Bomby wybuchły w~ciągu kilku minut każda i~zabiły więcej
osób niż wszystkie pozostałe ataki razem.

Było jasne, że gdzieś tam były inne JebRaki. Też byli w~gazetach, a~wtedy prezydent wspomniał o~nich -- ,,mroczne jaskinie, gdzie dobrzy
ludzie zmieniają się w~złych'' -- i~ciągle federalni nie przyszli po
Joego. Przyszło mu do głowy, że mogli mieć więcej śladów tej epidemii
miłych, szanowanych białych facetów się rozwalających niż mogli śledzić,
mogło minąć dużo czasu, zanim rzeczywiście przyszliby po niego.

-- Dlaczego ci ludzie zabijają tamtych ludzi? -- spytała Maddy nad
śniadaniem. Kolejny ktoś z~klasy mówił jej, że nie powinna iść do
lekarza, ponieważ szaleńcy zabijają wszystkich wokół szpitala.

-- Są szaleni -- powiedziała Lacey, przesuwając miskę jagód i~jogurtu pod
jej nos. -- Chorzy w~głowie.

-- Ale dlaczego?

-- Ponieważ niektórzy ludzie po prostu wariują, kochanie.

-- Ale \textit{dlaczego}?

-- Joe? -- Lacey wrzuciła tabletkę zmywarki do maszyny i~poszła do
łazienki, zbierając pranie, gdy szła.

-- Tata?

-- Dlaczego ja dostaję te trudne?

-- Ponieważ w~nich jesteś tak dobry -- zawołała Lacey z~łazienki.

Joe uśmiechnął, a~to sprawiło, że Maddy się uśmiechnęła, ale nie
zapomniała. 

-- Dlaczego, tato?

-- Ponieważ -- powiedział i~pokręcił głową. -- Są bardzo źli ludzie, którzy
zdecydowali, że mogą być bogaci, jeżeli sprawią, że pójście do lekarza
będzie bardzo drogie. Zrobili to tak drogie, że ludzie umierają,
ponieważ ich na to nie stać. To właśnie prawie przytrafiło się mamie. -- Twarz Madison się zachmurzyła, w~ten sam sposób, gdy ten temat się
pojawiał. -- Zatem są inni ludzi, którzy byli zmuszenie oglądać umieranie
dzieci, żony i~przyjaciół, to sprawia, że wariują, zatem idą i~zabijają
ludzi, którzy pracują dla złych ludzi, którzy sprawili, że lekarz jest
tak drogi.

Maddy wyglądała na sfrustrowaną. 

-- Ale dlaczego?

Podzielił to na mniejsze kawałki, słysząc siebie używającego terminów
jak ,,źli ludzie'' i~,,dobrzy ludzie'', myśląc, jak Lacey nie
zaakceptowała tej konstrukcji, ale zdecydowanie JebRaki by się zgodzili.

-- Myślę, że zabijanie ludzi jest złe.

-- Masz rację, kochanie.

-- Jednak powiedziałeś, że źli ludzie zabijali ludzi, nie dając im
zobaczyć lekarza.

-- Tak, to prawda.

-- Zatem oni zabijali ludzi.

-- Tak.

-- Zatem są tacy sami, jak ludzie, którzy strzelają do ludzi.

-- Nie bardzo.

To był czas, żeby się ubrać i~pójść do samochodu. Maddy nie pozwoliła mu
włączyć muzyki w~trakcie jazdy. Chciała wykrzykiwać bardziej moralnie
obarczone pytania z~tylnego siedzenia. To był ,,moment nauki'', wiedział
Joe, a~jego odpowiedzialnością jako dorosłego było podążyć za Maddy tam,
gdzie chciała prowadzić.

-- Dlaczego źli ludzie chcą zabijać ludzi?

-- Nie chcą \ldots  -- Przerwał sobie. Nie przekombinuj, głupku. -- Ponieważ
robią się bogaci, mogą sobie kupić ładne domy, samochody i~wakacje,
jeżeli lekarze są drodzy.

-- Dlaczego ludzie z~bombami chcą zabić ludzi?

-- Ponieważ są wściekli na złych ludzi.

Długa przerwa z~tylnego siedzenia. Wjechał w~pętle parkingu przy szkole.

-- Dobra, córciu, wrócimy do tego wieczorem po szkole.

Stojąc na chodniku z~lunchem i~tornistrem, zastukała w~okno pasażera, on
je opuścił. 

-- Myślę, że źli ludzie są gorsi niż ludzie z~bombami. Ludzie
z bombami tylko karzą złych ludzi za zabijanie ich dzieci.

\textit{I przyszły JebRak się narodził}. Joe próbował wyobrazić sobie te
rozmowy, które przyjaciele Maddy odbyliby z~rodzicami wieczorem, po tym,
gdy Maddy z~nimi porozmawia.

FBI zrealizowało koordynowane najazdy na cztery prywatne domy i~dwa
centra danych, twierdząc, że zajęli trzy różne fora dyskusyjne, gdzie
,,obłąkani zabójcy'' planowali ataki i~aresztowali właścicieli tych
forów.

Joe oglądał wyprowadzanie sprawców na social media, czterech białych
facetów w~średnim wieku w~pidżamach, skute ręce z~tyłu, przerażone
spojrzenia i~ciekawscy sąsiedzi. Policjanci weszli z~nakazem do
wszystkich czterech, weszli z~bronią w~ręku i~wsparciem SWAT, ale w~trakcie jakoś udało im się żadnego nie zastrzelić. Joe czerpał z~tego
spokój.

Media społecznościowe wybuchły życiem osobistym tych czterech facetów,
którzy byli, łagodnie mówiąc, prości jak cholera. Mieli gównianie prace,
prace, których nikt, nawet oni, nie uważali za warte robienia. Jeden był
konsultantem zarządzania. Kolejny kierownikiem obsługi klienta w~call
center. Kolejny programistą reklam. Ostatni był specjalistą do spraw
marketingu dla start-upów w~kryptowalutach. Wszyscy dzielili jedną
cechę: obserwowali powolną śmierć ubezpieczonych ukochanych, którym
odmówiono pokrycia wydatków. Kiedyś, dusiliby się w~osobistej nędzy,
stali się alkoholikami, zastrzelili się. Zamiast tego, podążyli za
prostymi instrukcjami online dla otwierających fora dyskusyjne i~utrzymujących je na bezpiecznych serwerach dostępnych tylko z~sieci TOR.
Nie zdetonowali bomby, nie zaczęli strzelać, nawet nie zachęcali ludzi,
którzy tak zrobili. Jednak zapewnili miejsce dla tego, obserwowali to
wszystko i~nie zamknęli forów. To wystarczyło.

Jeden z~nich -- konsultant zarządzania -- był emigrantem z~Kanady, jego
Twitter był pełen porównań pomiędzy system ochrony zdrowia w~Stanach i~Kanadzie, a~to uruchomiło burzę w~social media o~Kanadzie i~co Ameryka
mogłaby się nauczyć, a~także czy Kanadyjczycy są potajemnie
terrorystami, co było żartem, ale nie do końca. Memy z~\textit{South Park}
były epickie i~wieczorne programy rozrywkowe świetnie się bawiły.

Premiera Kanady zabrała głos na temat i~powiedział, że mimo bycia
konserwatystką, rozumie, że istnieją miejsca, gdzie rynki nie robią
swojej pracy, a~opieka zdrowotna jest jednym z~takich miejsc. Zdobyła
dużo poparcia wśród swoich wyborców, zagrało to również całkiem dobrze z~niezdecydowanymi wyborcami, którzy ogólnie traktowali Torysów jak zły
zapach, ale którzy przechylili szalę dzięki tej demonstracji
współczującego konserwatyzmu, że wybrali w~kolejnym miesiącu pierwszy
kiedykolwiek konserwatywny rząd prowincji Quebec. Jeżeli była taka
sprawa, która zmotywowała Kanadyjczyków, to było poczucie, że polityka
Ameryki była tak pomylona, że Kanada lśniła w~porównaniu.

Żaden z~czwórki nie wyszedł za kaucją.

Maddy przeziębiła się i~zabierało wieczność, żeby zasnęła, kaszląc. Joe
zobaczył, że Lacey siedzi na łóżku, z~lampką nocną zapaloną, czytającą
telefon.

-- Pamiętasz te forum dyskusyjne, które używałeś, kiedy byłam chora?

Próbował zachowywać się swobodnie, gdy odwiesił szlafrok i~wsunął się na
łóżko koło niej. 

-- Ta.

-- To było jedno z~tych, prawda? -- Pokazała mu telefon. Zamach bombowy
się nie udał, facet został postrzelony, gdy zbliżał się do uzbrojonego
punktu kontrolnego na końcu podjazdu do szpitala Kaisera. Wysadził się,
szrapnele poraniły niektórych ludzi, ale żadnych poważnie, i~przynajmniej byli blisko szpitala. Kaiser nawet odstąpił od ich opłat.

-- Właściwie nie -- skłamał.

-- Co to znaczy ,,właściwie nie'' właściwie? -- Brzmiała prawie
zdenerwowanie, jakby była w~nastroju, żeby z~nią nie zaczynać.

-- Lace\ldots 

-- Nie, Joe. Po prostu mi powiedz. Ci ludzie, z~którymi spędzałeś te
godziny na dyskusji, czy to są Ci, którzy latają i~popełniają masowe
morderstwa?

-- Szczerze?

Tylko się patrzyła.

-- Nie wiem. -- Co było kłamstwem. -- Znaczy, mogli być. Byli na to
wystarczająco wkurzeni. I~nie znam ich prawdziwych nazwisk, wszyscy
używali pseudonimów. Przestałem się logować i~tak czy inaczej, wtedy
zostali zamknięci.

-- Skąd wiesz, że zostali zamknięci, jeżeli już więcej się nie logowaś?

Cholera. 

-- Po niektórych zamachach, zdecydowałem się zajrzeć. Byłem też
ciekawy. Ale tam już nic nie było.

-- Sądzisz, że to było jedno z~tych, które zamknęło FBI?

-- Nie, to było wcześniej.

-- Dobrze. Ponieważ, jeżeli by tak nie było, to oznaczałoby, że jest tam
serwer w~szafce na dowody FBI ze szczegółami, które wskazują na nasz dom
i naszą rodzinę.

-- Ta. To coś, o~czym też myślałem.

Kiedy Lacey w~końcu zasnęła, Joe wymknął się z~sypialni i~znalazł numer
telefonu Arizona ACLU\footnote{ American Civil Liberties Union -- amerykańska
organizacja non-profit, której celem jest ochrona praw obywatelskich,
zob.~\url{https://pl.wikipedia.org/wiki/American\_Civil\_Liberties\_Union}
-- przyp.tłum.} i~spróbował jak najlepiej go zapamiętać. Na wszelki
wypadek.

FBI zaaresztowało czterdzieści dwie osoby w~następnym tygodniu,
działając na podstawie tropów z~przejętych serwerów. Każdy z~nich,
podobno, planował akt masowego mordu.

Joe próbował nie oglądać wiadomości o~aresztowaniach, przeklikiwał
doxxing na social mediach o~facetach w~więzieniu. Zauważył, że Lacey
coraz częściej przygląda się telefonowi zmartwiona.

Dwupartyjna grupa senatorów przedstawiła ustawę Medicare-for-All z~wielkimi fanfarami, połowa Ameryki wiwatowała, podczas gdy druga połowa
powtarzała za Fox News potępienia ich jako mediatorów, którzy pękli od
terrorystów.

Następnego dnia facet strzelił do senatora z~Wielkiego Stanu Maine.
Został zabity przez Secret Service i~natychmiast zdoksowany\footnote{ oryg.
doxxed -- akt ujawnienia publicznie prywatnych informacji o~osobie,
przykładowo adresu zamieszkania,
zob.~\url{https://en.wikipedia.org/wiki/Doxing} -- przyp.tłum.} online. Facet startował do Kongresu jedenaście razy jako
jedyny republikanin oczekujący na nominację w~niezmiennie demokratycznym
obwodzie na przedmieściach Chicago. Każda z~jego kampanii zawierała
tyrady o~kartelach żydowskich bankierów i~George Sorosie. Pomiędzy
memami ``I hate Illinois Nazis'' z~The Blues Brothers, pojawiły się
posty od ludzi, którzy mówili, że to było nieuniknione, będzie coraz
gorzej i~obwiniali ,,ekstremistów zdrowia publicznego'' za
zapoczątkowanie.

Joe przestał sypiać w~nocy. Leżał w~łóżku, słuchawki w~uszach, słuchając
starych komediowych podcastów, póki Lacey nie zasnęła, potem chodził do
piwnicy i~biegał na bieżni, podnosił stare sztangi, próbując wymazać wir
myśli krążących w~mózgu. Cały dzień pił kawę przy biurku aż do mdłości,
prawie się zabił, zasypiając jednego wieczoru za kierownicą.

Lacey załatwiał spotkanie z~lekarzem i~powiedziała, żeby zgłosił chorobę
następnego dnia, nie chciała słyszeć żadnych argumentów.

Lekarz słuchał z~uwagą, o~ile bezosobowo, potem pokręcił głową.

-- Brzmi, jakbyś miał ciężki czas, Joe.

Joe poczuł wzbierające się łzy. 

-- Tak -- powiedział, ledwie szeptem.

-- Jak wszyscy w~rodzinie? -- Lekarz spojrzał na ekran. -- Jak Lacey?

Joe rozmawiał szeroko z~lekarzem o~zdrowiu Lacey, wtedy, gdy wszystko
było tak straszne. Pamiętał, że doktor współczuł nad bękartami w~firmie
ubezpieczeniowej, którzy odrzucili jej terapię. Z~miny lekarza -- zaskoczenia, podejrzliwości -- to też zapamiętał.

-- Lacey jest super. -- Poczuł łzy spływające po policzkach, ale nie
wiedział, co tam robiły. -- Pełne wyleczenie. Jej włosy są długie aż
dotąd. -- Dotknął ramion.

Lekarz podał mu pudełko chusteczek. 

-- To wspaniałe wiadomości, naprawdę
wspaniałe. -- Poprawił się w~krześle. -- Cud, naprawdę.

Teraz Joe już naprawdę płakał.

Lekarz stukał w~komputer przez chwilę. 

-- Słuchaj, chciałbym skierować
Cię do psychiatry, ale wygląda, że masz tylko pokrycie wydatków do
dwudziestu pięciu procent. Jest taka kobieta, którą lubię, była lekarzem
na SORze, a~potem stała się psychiatrą. Myślę, że naprawdę ją polubisz.
Nie boi się przepisywać stabilizatorów nastroju, ale też nie są jej
pierwszym wyborem. Tylko nie jest tania. Wiem, że to trudne pytanie, ale
sądzisz, że Cię stać?

Joe zaczął się śmiać, ciągle płacząc, potem szlochając. Lekarz wyglądał
niekomfortowo, potem na zaniepokojonego. Joe nie wiedział, ile dokładnie
czasu mieli na wizytę, ale był całkiem pewny, że przekroczył i~że inni
pacjenci czekają. Uspokoił się, wydmuchał nosa, wytarł oczy.

-- Przepraszam -- powiedział. -- To tylko\ldots  -- Pomachał dłońmi i~upuścił
pobrudzone chusteczki. Podniósł je. -- Moje ubezpieczenie nie zapłaci za
mnie, żebym poszedł do psychiatry pogadać, jak spieprzone jest moje
ubezpieczenie.

-- Tak -- powiedział lekarz. -- Słuchaj, my lekarze nienawidzimy tego nawet
bardziej niż wy. Ty masz z~nimi do czynienia tylko, kiedy jesteś chory.
My musimy radzić sobie z~nimi każdego przeklętego dnia. Mam dwie osoby w~biurze, których jedyną pracą jest gonienie za ich płatnością.

-- Słyszałem o~tym.

-- Nie słyszałeś nawet połowy tego, uwierz mi. -- Lekarz odepchnął wózek z~komputerem, zaczął pocierać oczy, potem odsunął dłonie od twarzy i~sięgnął po ręczny odkażacz u pojemnika zamontowanego na ścianie.
Spojrzał na Joego. -- Przeszedłeś przez dużo, Joe. Prawie utrata żony, to
jest trudna sprawa. To cud, że przeżyła, ale również to oznacza, że nie
dostałeś żadnej pomocy czy opieki, tego rodzaju, która włączyłaby się
automatycznie, gdyby jej się nie udało. Więc po prostu tam się trzymasz.
Wczoraj dzwonił do mnie przedstawiciel ubezpieczyciela, mówiąc, że mogę
skierować na nielimitowaną opiekę psychologiczną każdego, kto stracił
kogokolwiek, doszli do tego, że najtańszym sposobem zachowania głowy
przed zamachami jest zniesienie barier do leczenia psychologicznego. -- Uśmiechnął się niewesoło.

-- Nie jestem profesjonalnym psychiatrą, Joe, niemniej widziałem
dostatecznie dużo przypadków PTSD, żeby je rozpoznać. Potrzebujesz
leczenia. Ze względu na rodzinę i~dla własnego dobra. Wiem, że chcesz
być ojcem, którego Twoja córka potrzebuje.

-- Istnieje powód, dla którego w~tym kraju jest tak duży medyczny dług,
Twoje zdrowie jest na tyle ważne. Ważniejsze niż saldo na karcie
kredytowej, zdolność kredytowa lub nawet kredyt hipoteczny. Jeżeli
wypiszę Ci skierowanie do doktor Haddid, czy poszukasz sposobu, żeby się
z nią spotkać? Możesz porozmawiać o~płatnościach gotówką. Wielu lekarzy
oferuje zniżki za gotówkę z~góry.

Joe wydmuchał nos. 

-- Tak -- powiedział -- dzięki.

Stawka gotówki doktor Haddid było 125 dolarów za godzinę, plus 75
dolarów za pierwszą konsultację. Dwieście dolarów później, Joe wiedział,
że nigdy do niej nie wróci. Zamarł w~gabinecie, nie mogąc mówić,
przerażony, co mogłoby wyjść. Przerażony, że mógłby zacząć wygłaszać
mowy jak najgorszego typu JebRak, zainspirować ją do zadzwonienia na
policję.

Zamiast tego zatrzymał się w~aptece i~kupił cztery różne tabletki na sen
bez recepty, narysował diagram na kawałku papieru drukarki w~pokoju
gościnnym i~śledził zmiany, szukają tabletek, które dały mu najwięcej
snu z~najmniejszym kacem.

To, co naprawdę chciał, to odrobina Zolpidemu\footnote{oryg. Ambien,
zob.~\url{https://pl.wikipedia.org/wiki/Zolpidem} -- przyp.tłum.}, co próbował kilka razy w~szkole wyższej, kiedy był zbyt
nakręcony, żeby zasnąć, a~kolega mu pomógł. Było wiele miejsc, gdzie
mógłby dostać tanio i~bez recepty Zolpidem, korzystają z~rynków w~darkweb. Nawet miał trochę kryptowalut, na której spekulował, kiedy
wydawało się, że nigdy nie przestanie rosnąć. Równie dobrze mógłby wydać
ją teraz, zanim stanie się kompletnie bezwartościowa.

Ale oczywiście, nie potrafił zapamiętać niejasnych adresów ,,.onion''\footnote{ adres stron internetowych dostępnych tylko przez przeglądarkę TOR, przykładowo adres 	\url{https://theguardian.com} to \url{33y6fjyhs3phzfjj.onion}, adres przeglądarki internetowej DuckDuckGo -- \url{https://3g2upl4pq6kufc4m.onion/} -- przyp.tłum. }
dla witryn, jedyny, jaki znał doskonale to JebRaka, a~zanim mógł się
powstrzymać lub nawet przypomnieć, że strona była wyłączona, już się
logował.

Informacja na stronie logowania poinformowała go, że strona została
odtworzona z~kopii zapasowej, wszystkie konta nienaruszone, ale
wszystkie archiwa bezpiecznie skasowane. Strona poprosiła go o~skasowanie wszystkich zrzutów z~ekranu lub zapamiętanych wiadomości,
które mógł gdzieś zgromadzić, powitała go ponownie w~imieniu nowego
kierownika, kogoś o~nazwie Deadzone874755, którego nie mógł sobie
przypomnieć.

Po raz pierwszy od ponad miesiąca, poczuł się zrelaksowany. Napięcie
opuściło go, gdy przerzucał strony, śmiał się na sesje bzdur i~bon moty,
czytając aktualizacje od starych przyjaciół. Czy to się podoba czy nie,
to byli jego ludzie, to było jego miejsce. Jego duchowy dom. A jeżeli
były tam elementy skrajne w~jego społeczności, które robiły złe rzeczy,
horrendalne rzeczy, cóż, co z~tego? Nikt nie obwiniał żołnierzy za
pozostanie w~armii, tylko dlatego, że ktoś w~tym samym mundurze
rozstrzelał wioskę lub torturował więźnia. Koleżeństwo i~zrozumienie,
które dostawał od JebRaków, więź wspólnego doświadczenia, to było
niezastępowalne. Miał prawo do tego, a~nikt nie miał prawa go
powstrzymać. Nigdy nikogo nie zachęcał do aktów przemocy. W~istocie,
zrobił wszystko, żeby powstrzymać przemoc.

Nie wspominając, przyczyna była sprawiedliwa. Najsprawiedliwsza.
Pozwalanie umierać ludziom, ponieważ uratowanie ich życia zmniejszyłoby
zyski, było niegodziwym czynem, a~ludzie podpisujący się pod takimi
czynami byli podli. Wysadzanie ich lub strzelanie do nich nie było w~porządku, ale świat, w~którym podli żyli przerażeni zemstą, był bardziej
sprawiedliwym światem, niż ten, gdzie podli trzymali wysoko głowy.

Departament Sprawiedliwości wysłał surowo sformułowane memorandum do
pięciu największych firm internetowych, prosząc ich, by na poważnie
rozważyli stłumienie hasztagu \#MacieSieBac. Wkrótce przestali tego
żądać. Później tego dnia, ktoś wrzucił zbiór emaili pomiędzy głównymi
adwokatami firm internetowych i~starszym prawnikiem Departamentu, który
złożył taką prośbą prywatnie i~dostał odpowiedź, żeby spadał na drzewo.
Konsensus był taki, że Departament Sprawiedliwości nie sądził, żeby mógł
dostać nakaz sądu, a~przy porażce zabezpieczenia cichej, dobrowolnej
współpracy, próbowali przekonać opinię publiczną, żeby obwiniała
platformy za podżeganie do morderstw.

Zadziałało, w~pewien sposób. Przyjaciele Joego byli po równo podzieleni
pomiędzy ludzi, którzy myśleli, że zakaz hasztaga był śmieszny i~ludzi,
którzy myśleli, że firmy internetowe były kompletnymi dupkami za
niepodporządkowanie się. Ludzie używali tego hasztaga, żeby rozmawiać o~problemie i~taktyce, dzielić się poradami w~sprawie bezpieczeństwa,
dyskutować motywacje. Z~drugiej strony, ludzie używali tego hasztaga, by
pochwalać i~propagować masowe morderstwa. Z~innej strony, zrobienie
nowego taga nie było właściwie skomplikowaną operacją: \#BaćSięMacie,
\#MASIBA. Oczywiście, w~JebRaku się nie przejmowali, mieli wszystkie te
rozmowy o~\#MacieSięBać, których potrzebowali na prywatnym forum.

BlueCross BlueShield z~Minneapolis rozpoczęło prace nad nowym budynku w~rejonie przemysłowym na końcu własnej ślepej uliczki za bramą, z~wysokimi wieżami strażniczymi, automatycznym systemem rozpoznawania
tablic w~promieniu dwóch kilometrów, schronami na każdym piętrze i~dużym
personelem uzbrojonej straży 24/7. Materiały dla udziałowców tłumaczyły
ten koszt i~amortyzowały go w~ciągu pięciu lat, wyjaśniając, jak koszty
działania zostaną pokryte z~połączenia ,,dopłaty za bezpieczeństwo'' na
wszystkich premium i~małemu zmniejszeniu dywidendy za akcji. Po krótkim
skoku, kiedy kontrakty terminowe pojawiły się na rynku, cena akcji się
zamknęła, a~niektóre kontrakty mocno straciły.

\textbf{ Chcą udowodnić, że się nie boją}

DamFool był najnowszy na JebRaku. JebRaka był znacznie trudniejszy do
znalezienia, nowi rekruci byli dodawani przez starych userów, z~poręczeniem. Jego syn, Tommy, miał czternaście lat, gwiazda bieżni,
rozszerzona matematyka, Skaut Orła.

Chłoniak nieziarniczy\footnote{ grupa chorób nowotworowych, która zajmuje 6
miejsce pod względem częstości występowania u dorosłych i~stanowią
5--7\% wszystkich nowotworów u dzieci,
zob.~\url{https://pl.wikipedia.org/wiki/Ch\%C5\%82oniaki\_nieziarnicze}
-- przyp.tłum.}.

Rak Tommy'ego nie odpowiedział na dwie sesje chemoterapii. Był silny,
młody, żywotny, a~ta sama młodość, która dawała mu siły do wytrzymania
całej tej strasznej medycyny, również powodowało, że jego komórki
dzieliły się ze straszną, stałą szybkością, nawet te rakowe.

Jego lekarz sądził, że Tommy był dobrym kandydatem na terapię
chimerycznymi antygenowymi T-komórkami, terapię, która była
eksperymentalna z~każdej strony. Jednak jeżeli ktokolwiek mógłby ją
przeżyć, to był to Tommy. Ludzie, którzy przeżyli tę terapię, mieli
dobrą szansę na trzy lata bez raka, a~potem świat stał otworem. Tommy
był gotowy. Nawet chciał zamrozić nieco spermy, na wypadek, gdyby
później w~życiu chciał mieć dzieci. To był rodzaj wyboru życiowego,
który planowali ludzie w~trybie przetrwania.

Ubezpieczyciel, Cigna, miał inną koncepcję.

Sedno sprawy: nie wydadzą trzech czwartych miliona (podstawowa pula,
suma mogła być wyższa), żeby zabić Tommy'ego przy pomocy
eksperymentalnej terapii. Mimo Tommy był wojownikiem, optymistą, młody i~zdrowy (prócz umierania na raka) i~chciał zamrozić swoją spermę.

Żona DamFool była nieukojona. Jego starszy syn, Rhett, zmarł od
przedawkowania opioidów w~ostatniej klasie liceum, pięć lat temu, a~oboje włożyli wszystko, co mieli, w~Tommy'ego, poświęcając się dla jego
życia i~wychowując z~wigorem, które wydawało się odpowiadać Tommy'emu.
Wszyscy byli zranieni tym, co zdarzyło się Rhettowi.

Po swojej stronie DamFool ledwie się trzymał. Spędzał tyle czasu z~dala
od Tommy'ego ile mógł znieść, nie chcąc przesłonić chłopca, a~kiedy nie
był z~Tommy'm, próbował pocieszyć żonę, która nie była w~nastroju do
pocieszania, lub próbował nie wypić całego piwa w~lodówce. Tommy'ego
bardzo bolało, częściowo ponieważ nie chciał brać środków
przeciwbólowych, bo to były te same, od których był uzależniony Rhett.
To była zła sytuacja.

Wielcy Starsi JebRaka pielęgnowali DamFoola. Były te listy celów, które
ludzie przygotowywali, doxując dyrektorów, analityków, inwestorów,
lobbystów opieki zdrowotnej, jak również prawodawców w~stanie i~na
poziomie krajowym, którzy mieli swoje za uszami. Zrzuty miały adresy
domowe, znane trasy, nawet rysunki architektoniczne z~zapisów
publicznych pokazujące plany pięter ich domów i~biur. Te listy nie były
trudne do natrafienia, krążyły radośnie pod hasztagiem \#MacieSięBać. Za
pierwszym razem, gdy Joe zobaczył te pliki, pomyślał, jak musi to być
mrożące, zobaczyć swój dom, rodzinę, zdjęcie, tablice rejestracyjne na
takiej liście. To dało mu tyle satysfakcji, że próbował nie myśleć o~tym
za dużo.

DamFool mieszkał w~Montanie i, jak na tak słabo zaludniony stan, Montana
miała cholernie wiele bardzo ważnych celów. Małe populacje były
sieciami starych kumpli, a~to ułatwiało rozszerzanie się korupcji,
zasługi za zasługę, przyjaciel przyjacielowi. Można było dojrzeć wzór
dyfuzji na liście celów, która była rozległa.

Wielcy Starsi byli dobrzy na temat. Teraz tylko nadmieniali, nie
pokazywali DamFool celu. To pojawiłoby po odejściu Tommy'ego. Teraz
tylko ustawiali go na pozycji. Joe widział takie zagrywki wcześniej. To
była wskazówka dla niego, żeby zanurkować z~wysoce symboliczną i~bardzo
ozdobną próbą, żeby uratować jego duszę.

Joe po prostu nie potrafił. Czytywał te artykuły wstępne aktywistów
Black Lives Matter na temat oficjalnego zaniedbania, które znosili
ludzie z~niedokrwistością sierpowatokrwinkową\footnote{
	zob.~\url{https://pl.wikipedia.org/wiki/Niedokrwisto\%C5\%9B\%C4\%87\_sierpowatokrwinkowa}
-- przyp.tłum.}, historie o~męczących się nastolatkach przywiązywanych
do szpitalnych łóżek, którym zapowiadano odwiązanie, jeżeli przestaną
krzyczeć. Ogólna treść była taka, że biali, którzy nagle zdecydowali, że
system ochrony zdrowia był zbyt chory, żeby korzystać, po prostu byli
spóźnieni na zabawę, a~przy okazji, pozwól, że opowiem Ci historię o~lotnikach z~Tuskegee\footnote{ oddział pilotów myśliwskich z~okresu Drugiej
Wojny Światowej, złożony z~czarnych żołnierzy,
por.~\url{https://pl.wikipedia.org/wiki/332.\_Grupa\_My\%C5\%9Bliwska}
-- przyp.tłum.}.

Joe zaczął wyobrażać sobie, jak jego życie wyglądałoby, gdyby Lacey
zmarła. Gdyby był sam z~Maddy, wielka dziura w~sercach ich obojga.
Wyobrażał sobie, jak życie musiało wyglądać dla rodzica samemu przez to
przechodzącemu, obserwującemu ich dzieci przeżywające to, obserwującego
ich kuzynów oraz przyjaciół przeżywających to. Nie potrafił sobie tego
wyobrazić. Nie potrafił wyobrazić sobie tolerowania tego. Jak oni mogli
to tolerować? Nie był głupi. Słyszał o~przywileju białych. To był
problem. Rozumiał to.

Im bardziej myślał o~tym, co gdyby Lacey zmarła, co z~tymi ludźmi,
którzy to tolerowali i~gorzej, co z~tymi wszystkim, co mu mogło ujść na
sucho, a~innym już nie?

Kiedy myślałeś o~tym w~ten sposób, to praktycznie on miał obowiązek
zabić jednego lub dwóch dyrektorów firmy.

Tommy zmarł 17 lipca.

Maddy była w~Wisconsin z~Lacey, odwiedzając rodzinę Lacey, z~którą
zaczęła rozmawiać po wyleczeniu z~raka, świadomie wybierając anulowanie
starych sporów w~rodzinie.

Bez rodziny w~polu widzenia, Joe odkrył, że znowu w~pracy uważa,
odpowiada na emaile do późna w~nocy, potem gładko przeskakuje na długą
sesję z~JebRaką. Mimo że Lacey nie było w~pobliżu, ciągle zabierał swój
laptop do pokoju gościnnego, kiedy nadchodził czas na JebRaka.

DamFool był niespójny z~początku ze smutku, potem na zimno gwałtowny,
opisujący bardzo szczegółowe fantazje morderstw i~chaosu. Potem zniknął.

Joe nie włączał się w~ogóle, pozwalając Starszym zachęcać DamFoola,
podczas gdy Joe obserwował z~boku. Ale kiedy zniknął, Joe miał wyrzuty
sumienia i~posłał mu serię prywatnych wiadomości, błagając go o~telefon,
zanim zrobi coś głupiego.

DamFool nie odpowiedział na wiadomości Joego, ale następnego ranka szedł
zaspany do toalety w~bokserkach, trzymając telefon i~myśląc o~pierwszej
filiżance kawy, kiedy zobaczył coś za oknem. Zatrzymał się i~przyjrzał
się kształtowi, próbując zrozumieć, czym jest coś, co wyglądało jak
futurystyczny robot, przynajmniej w~tej krótkiej chwili, którą zajęło mu
zrozumienie, że jest to człowiek w~pancerzu SWAT z~wizjerem, wielką
bronią, którą obracał, żeby skierować w~kierunku Joego.

Joe otworzył usta, żeby powiedzieć cokolwiek -- ,,nie'' lub ,,co do
kurwy?'' -- kiedy odkrył, że leży na podłodze i~nie może oddychać.
Próbował wstać, ponieważ coś było na jego piersi i~musiał wyjść spod
tego, ale nie tylko jego płuca nie pracowały, nie mógł też ruszyć nogami
i ramionami. Znienacka było też bardzo głośno i~w końcu, gdy został
otoczony przez więcej robotów w~kamizelkach kuloodpornych z~bardzo,
bardzo wielkimi karabinami, zrozumiał, że został postrzelony w~pierś.

Policjant, który postrzelił Joe, był weteranem SWAT ze stażem dwudziestu
dwu lat w~Policji Tempe, w~trakcie których postrzelił razem
dziewiętnastu podejrzanych, ale Joe był pierwszą białą osobą, którą ten
policjant Connor postrzelił. Joe był również pierwszą osobą, która
przeżyła, i~znaleźli się eksperci w~social media, którzy podejrzewali,
że utajona wyższość białych Connora -- ta sama siła, która poruszyła jego
palcem tyle razy wcześniej -- zepsuła jego cel, ratując życie Joe. Został
trafiony ,,centralnie'', ale tylko płuco się zapadło i~nie było
uszkodzeń serca.

Policjant powiedział, że pomylił telefon Joe z~bronią.

Joe dowiedział się o~tym szczególe od agenta FBI, który przesłuchiwał
go, gdy był trzeźwy od narkozy, przypięty kajdankami do łóżka i~wracając
do zmysłów. (Dowiedział się później o~osiemnastu innych strzelaninach).

Agent FBI przeczytał Joemu całą serię wiadomości z~DamFoolem i~chciał,
żeby Joe wiedział, że w~jego, agenta Sebolda, opinii, Joe próbował
postąpić dobrze. Agent Sebold mógł powiedzieć, że on i~Joe są tutaj w~tej samej drużynie, obaj próbują pomóc tym biednym, zdezorientowanym
mężczyznom ze złamanym sercami skierować swój gniew w~mniej chorobliwe
działania.

Dlatego agent Sebold wpadł na osobliwie wspaniały pomysł pomocy Joe w~uniknięciu jakiejkolwiek potencjalnej kryminalnej odpowiedzialności w~zamian za pomoc Joe w~infiltracji JebRaka i~złapania Wielkich Starszych.
Agent Sebold silnie podkreślił, że byli inni kolaboranci już pracujący z~nim na JebRaku, jednak spryt Wielkich Starszych udaremnił ich wysiłki.
Pracując z~Biurem, Joe nie przecierałby szlaków idei tajniaków w~systemie, ale używałby swoich unikalnych umiejętności i~długoletniego
dostępu dla dalszych połączonych misji z~Biurem, żeby uratować tych
biednych, naiwnych mężczyzn, nie mówiąc już o~ofiarach.

Joe był bardzo oszołomiony. Narkoza była ciągle w~jego krwi i~była
jeszcze ta fantastyczna plaga wrażeń z~klatki piersiowej, kości i~napuchniętej, zszytej skóry.

Przytłumiony głos z~tyłu głowy gadał intensywnie, mówiąc mu, że ASAP
musi porozmawiać z~Lacey, ponieważ będzie wściekła, mówiąc mu, że
\textit{nie} powinien rozmawiać z~gliniarzem bez obecnego prawnika.

Joe słuchał mówiącego policjanta. Agent Sebold był dobry, spokojny,
uspokajający, przyjazny. Chciał tego samego, co chciał Joe. Chciał pomóc
Joe. Wiedział, że Joe też chciał pomóc.

Joe próbował przemówić, ale nie mógł. Jego usta były zaklejone gęstą,
suchą śliną, jego język tak gruby jak pięść. Zdecydowanie agent Sebold
znał się na swojej pracy, delikatnie zanurzył duży kawałek waty w~wodzie
i przetarł usta Joego, potem przód jego zębów. Strużka zwykłej ciepłej
wody nawilżyła język Joego. To było niesamowite uczucie.

-- Prawnik -- powiedział Joe. Agent wyglądał na wściekłego przez sekundę,
potem się opanował i~przełączył na rozczarowanego. 

-- Przepraszam -- dodał Joe. Potem -- Prawnik. -- To wszystko, co zrobił, zanim jego język znowu
wysechł. Agent FBI wyszedł.

Minęło dziesięć dni, nim Joe zobaczył prawnika. W~międzyczasie, cały
czas był w~kajdankach, prócz wizyt lekarza i~fizjoterapii. Te były w~obecności strażników -- kamiennych policjantów z~Komendy w~Phoenix -- a~za
każdym razem, gdy któryś z~policjantów wchodził do pokoju, prosił o~dostęp do prawnika. To musiał być rodzaj gry. Niezbyt dobrej.

Nie zobaczył Lacey. Kiedy pytał się o~nią policjantów, lekarzy,
pielęgniarki, fizjoterapeutę, zachowywali się tak, jakby go nie
słyszeli. Był dość pewien, że mógłby spotkać się z~Lacey kiedykolwiek, o~ile wcześniej poprosiłby o~rozmowę z~agentem Seboldem.

Każdego dnia prawie się załamał. Ale tego nie zrobił.

Wiecie, co go trzymało? Wizyta faceta z~rachunkiem szpitalnym.

Jego ubezpieczenie nie obejmowało pobytu. Był wyjątek w~jego polisie w~sprawie ,,aktów terroryzmu'' i~się na nią powoływali. Wydział
księgowości szpitala chciał poznać jego aktywa. Byli jedynymi, którzy
chcieli rozmawiać z~nim o~Lacey, a~to, co chcieli wiedzieć, to czy
prowadziła oddzielne finanse od niego.

Facet od rachunku wydawał się rozsądnym gościem tkwiącym w~gównianej
pracy. Joe wiedział, jakoś, że zarabiał mniej niż Joe. To były tanie
buty i~fryzjer za siedem dolarów. Był zażenowany i~pokorny, ale miał
pracę do wykonania. To nie była jego wina, że system był kompletnie
pojebany. Ten jeden facet nic nie mógł z~tym zrobić.

Kiedy księgowy odwiedził go po raz trzeci, Joe zdecydował, że poszedłby
do więzienia na sto lat, zanim zdradziłby JebRaka i~wszystkich, którzy w~nim byli.

Agent Sebold odwiedził go tuż przed prawnikiem, wyglądając na wkurzonego
i znękanego. Ostatnim razem przekonywał delikatnie, ale tym razem
wdzięczył się i~to było wskazówką dla Joego, że sprawy agenta nie idą w~dobrym kierunku.

Prawnik był starym Amerykaninem pochodzenia chińskiego, który robił
pracę pro-bono dla Arizona ACLU od czasów studiów. Pokazał się godzinę
po wyjściu agenta Sebolda i~się przedstawił.

-- Twoja żona ma się dobrze. Chciała, żebym się upewnił, że to usłyszysz
pierwsze. Była bardzo nieugięta. Nie byłbym tutaj, gdyby to nie było dla
niej. Wychodziła tę sprawę, przekonała ACLU, że to było coś, na co
powinniśmy zwrócić uwagę. Było cholernie dużo potencjalnych klientów,
którzy pasowali do profilu i~musieliśmy ustalić priorytety spraw, które
mogłyby stworzyć dobre prawo lub zapobiec stworzeniu złego prawa. Jesteś
w tej drugiej grupie. Nikt nie zarzuca, że kogokolwiek nakłaniałeś,
nawet FBI mówi, że robiłeś, co mogłeś, żeby zatrzymać sprawy. Ale
wiedziałeś o~poważnych przestępstwach, które mają się wydarzyć i~nie
poszedłeś na policję. Mówią, że to czyni z~Ciebie współwinnym
terroryzmu. Kongresmen z~Arizony nazwał Cię ,,wrogim bojownikiem''.

Joe powstrzymał jęk.

Leonard, jego prawnik, poklepał go po ręce, zmarszczył brwi na kajdanki.

-- Słuchaj, przed nami długi proces. Chciałbym zacząć od prostych spraw:
zdjąć Ci te kajdanki, umożliwić widzenie z~rodziną, wypuszczenie za
kaucją. \textit{Potem} możemy porozmawiać, jak zamierzamy trzymać Cię z~dala od więzienia.

-- Przynajmniej nie wspomniałeś o~karze śmierci -- zażartował Joe. Leonard
się nie uśmiechnął.

-- Chowałem to na chwilę, kiedy odzyskasz siły -- powiedział Leonard. -- To
nie jest coś, czym jestem nadmiernie przejęty, ale kiedykolwiek jest
nawet mała szansa, że sąd wyda wyrok egzekucji, to jest to czymś, co
musimy wziąć pod uwagę w~planowaniu. -- Dał chwilę Joemu, żeby to
dotarło. -- Teraz, chcę wiedzieć wszystko. -- Wyjął żółty notatnik. -- Zacznij od początku.

~

Ostatnie potwierdzone zabójstwo \#MacieSieBac odbyło się półtora roku
później, kiedy Joe był w~celi odosobnienia w~superwięzieniu w~Tucson,
gdzie został przeniesiony po pobiciu w~więzieniu Hrabstwa Maricopa,
kiedy czekał na rozprawę w~sprawie apelacji Leonarda w~sprawie jego
kaucji. Dyrektor więzienia spojrzał na pobitą twarz Joego, bandaże na
żebrach i~skierował go do ,,ochronnej izolacji''. To było tydzień temu.

Zabójca był z~Jacksonville, Missouri, strażak, któremu BlueCross
BlueShield odmówił pokrycia wydatków na dializę po poważnej ranie nerek,
której doznał w~pracy, kiedy belka z~sufitu na niego upadła. Nie
zgodzili się z~analizą lekarską jego stanu. Jego lekarz prywatnie
powiedział mu, że lepiej, żeby zebrał forsę na dializy lub może
spodziewać się krótkiego i~nieszczęśliwego życia.

Strażak nie miał żony i~dzieci, co było inne w~stosunku do innych. Miał
starszych rodziców i~wpadł w~poważne długi, płacąc za ich opiekę domową
lata temu. Teraz oboje odeszli, wszystko, co miał to praca i~zdrowie,
żadnego z~nich już nie miał.

Ubezpieczyciel zgłosił strażaka do szeryfa, co było standardową
procedurą, kiedy odmawiano roszczeniom takim jak to, ale szeryf miał
dużo pracy i~mało zastępców, a~zastępca, który miał zajrzeć do strażaka,
przesunął spotkanie na późniejszy termin, ponieważ otrzymał ważniejsze
zadania do wykonania.

Strażak wiedział dużo o~materiałach wybuchowych, jak się okazuje.

Joe dostał słowo o~przejściu Americare podczas rzadkiej wizyty od Lacey.
Leonard doradził mu przeciwko przyznaniu do winy, twierdząc, że ACLU
miało nadzieję, że mogliby dostać dobry precedens z~jego sprawy. Ale
wtedy spotkali się z~prokuratorem, który przedyskutował szczegółowo, co
trzydzieści pięć lat więzienia robi człowiekowi. Leonard nalegał, że
trzydzieści pięć lat jest kompletną bzdurą, że wymagałoby, żeby sędzia
naruszył zasady Komisji ds. Wyroków w~takim rażącym zakresie, że
apelacja byłaby praktycznie automatyczna.

Jednak trzydzieści pięć lat ciąży na umyśle. Joe wyobraził sobie
oddzielenie od Lacey i~Maddy, podczas gdy Maddy dorasta, Lacey się
starzeje, wychodząc z~bram więzienia, żeby poznać piętnastoletnią
wnuczkę po raz pierwszy.

Joe i~Lacey dużo płakali, Maddy się przestraszyła, ale w~końcu,
prokurator zaoferował pięć lat, warunkowe zwolnienie po dwóch. Już był
przetrzymywany przez większość roku, podczas gdy rozprawa się
przeciągała, podczas gdy inni z~JebRaka byli aresztowani, podczas gdy
świadkowie przedstawiali dowody.

Zatem Joe zgodził się na kolejny rok, może cztery lata i~przestępstwo
kryminalne i~szansę na zestarzenie się z~rodziną. Dyrektor zdecydował,
że był rodzajem więźnia, który dobrze tolerowałby izolatkę i~stało się
to stałym warunkiem, w~wyniku czego wizyty były niemożliwe do
zorganizowania. Była jedna na początku, potem trzy miesiące, potem
kolejna.

-- Przyjęli Americare -- powiedziała Lacey. Wyglądała strasznie,
wyczerpana i~emocjonalnie wyciśnięta, jej trądzik straszny, na sposób,
jaki się pojawiał, kiedy się stresowała. Joe był ostro świadomy brzucha,
którego rozwój dookoła paska obserwował, sześć dni minęło od
cotygodniowego prysznica.

Chciał powiedzieć coś jak: 

-- Minęło trzy miesiące i~chcesz tylko
rozmawiać o~\textit{Americare} -- ale zrozumiał, że tematów bezpiecznych
było niewiele i~daleko pomiędzy nimi, a~oczy Maddy były czerwone i~wielkie jak talerze i~wszyscy starali się jak najlepiej.

-- To dobre wiadomości.

-- Nie ma wszystkiego, co chcielibyśmy, ale jest całkiem niesamowite.
Nikt nie wierzył, że przeszłoby. Nikt nie wierzył, że prezydent
podpisałby. Zaskarżenie też umarło na miejscu, mimo że mogli wybrać swój
sąd. Sędziowie apelacyjni Obwodu Federalnego w~ciągu dziesięciu sekund
powiedzieli im, żeby się odjebali. Nikt poważnie nie wierzy, że
Najwyższy przyjmie sprawę, a~ceny akcji \ldots 

-- Tęskniłem, Lacey.

Przerwała. Jej oczy były jasne od łez.

-- Wszystko ok?

Spojrzał na Maddy. 

-- Tak -- powiedział. Kiedy Maddy odwróciła się,
powiedział bezgłośnie ,,nie''. Lacey położyła dłoń na szkle i~zrobił to
samo, świadom, że była to taka klisza, ale także czując
psychosomatycznego ducha ciepła jej skóry przez grube plexi jak grzejnik
w kosmosie promieniujący bezpośrednio w~jego dłoń.

Lacey teraz płakała. Tak jak i~on.

-- Naprawdę przegłosowali, co? -- powiedział przez smark.

-- Kto powiedział, że przemoc niczego nie rozwiązuje? -- powiedziała.

Śmiech Joego był tak niespodziewany, że zapryskał szkło smarkiem niczym
Jackson Pollock, a~to rozśmieszyło Maddy, Maddy rozśmieszyła Lacey i~tama pękła między nimi.

Tylko dziewięć miesięcy więcej, zakładając dobre zachowanie.

\chapter*{Maska Czerwonej Śmierci}

Przed Wydarzeniem, Martin Mars spędził mnóstwo czasu, próbując to
rozgryźć. Czy upadek będzie nagły, zaskakując go nieprzygotowanego,
zmuszając do walki w~drodze do jego fortecy, gdy będzie uciekał z~Paradise Valley i~na pustynne wzgórza? Czy będzie rodzaj znaku, ciągły
wzrost niepokojów społecznych i~porażek oficjalnej władzy, które
odliczały do dnia, dając mu szansę na zaplanowanie uporządkowanego
wycofania się do Fortu Zagłada?

To było ważne. Gdyby Martin się przestraszył i~udał się do Fortu Zagłada
zbyt wcześnie, musiałby cofnąć się do miasta i~pracy po wielu dniach
spędzonych w~bunkrze. Nie tylko byłoby to upokarzające, ale kosztowałoby
go to wiarygodność wobec Trzydziestki, ludzi, których zaprosił do
przetrwania apokalipsy razem z~nim. Kiedy Fort Zagłada zostałby
zamknięty na czas trwania, musiałby być wiarygodnym liderem lub
straciłby panowanie. Kto kurwa wiedziałby co się wtedy stanie? Miał
wizje odległych potomków odsuwających wielkie drzwi i~odnajdujących
zmumifikowane ciała rozciągnięte w~pozycjach, w~których zmarli, kiedy
wygłodniały tłum zwrócił się ku sobie.

Za każdym razem, gdy ogłaszał alarm i~wyprowadzał Trzydziestkę,
zwiększał szansę, że jedno lub więcej ujawni istnienie Fortu Zagłady.
Wszyscy zobowiązali się zachować w~tajemnicy, ale kiedy gówno trafia w~wentylator, ktoś puści parę z~ust do tej fajnej dziewczyny, z~którą
próbował się spotkać lub zostawi ślady, za którymi podąży jakiś
ciekawski włóczęga. Nadszedłby wtedy dzień, kiedy Trzydziestka musiałaby
bronić tego, co należało do nich przed ludźmi, którym zabrakło
przewidywania, by zbudować własne fortece, ale Martin preferował, żeby
ten dzień był odległy, po tym, gdy mieliby szansę stworzyć wspólną
jednostkę przetrwania, której potrzebowali, żeby przeczekać Wydarzenie.

Zatem istniało wiele powodów do niesygnalizowanie i~wzywania wszystkich
przedwcześnie do Fortu, ale co z~kosztami czekania zbyt długo? Paradise
Valley była w~porządku, pewnie, ale Phoenix było niedaleko, na progu, i~było tak wiele osób, którzy nie mieli nic do stracenia, ludzi, którzy
obecnie już się wzajemnie zabijali i~gwałcili. Co zrobiliby, kiedy
policja była zbyt zajęta walką o~swoje własne przetrwanie, żeby bronić
solidnych mieszczan Paradise Valley? Czekać zbyt długo i~to mógł być
czas \textit{Mad Maxa}, walki o~przejście koło bóg wie jak wielu złych
facetów, żeby wydostać się na wzgórza. A co jeżeli podążyliby za nim?
Mógłby zabezpieczyć Fort, nawet gdyby walczył w~straży tylnej -- obrona
Fortu była wyraźnie do tego zaprojektowana -- ale każdy z~Trzydziestki,
który by się pojawił, zostałby rozdarty na strzępy przez oblegającą
armię, zanim dotarłby do drzwi Fortu lub gorzej, dotarliby do drzwi, ale
ich otwarcie byłoby zbyt niebezpieczne. To byłoby prawdziwe wzmocnienie
pozycji, kiedy stałoby się to pod jego przywództwem, obserwowanie
bezradnie z~wnętrza fortu, podczas gdy wszystkie gnojki i~zombie
rozdzieraliby kolegów z~pracy na kawałki na progu.

To był trudny problem. Nigdy o~nim nie zapominał, gdy siedział przy jego
Terminalu Bloomberga \footnote{ specjalistyczne oprogramowanie obsługujące
transakcje finansowe, zob.~\url{https://en.wikipedia.org/wiki/Bloomberg\_Terminal} -- przyp.tłum. } i~obserwował ruch rynków, wzrosty i~upadki walut, aktywa
narastając i~tonące.

Martin wiedział, że Wydarzenie nadchodzi. Fakt był taki, że świat po
prostu już nie \textit{potrzebował} wszystkich ludzi i~Rynek ujawnił ten
fakt, ściskając ich w~małe, mniej wygodne miejsca. Nie tolerowałby tego,
gdyby był na ich miejscu. Byłby pierwszy do zbudowania gilotyny, a~ponieważ był Martinem, byłaby to \textit{niesamowita} gilotyna, tak
całkowicie zła i~przekombinowana, z~turbosprężarką i~samo-ostrzącym się
ostrzem. Ponieważ taki był Martin, cały czas. To dlatego Fort Zagłada
był takim niesamowitym miejscem.

Chodziło o~asymetrie w~informacjach. Rynki je korygowały. Jeżeli
znajomość tajemnicy mogłoby zamienić głupią, niewartą osobą w~bogacza,
to wcześniej czy później, inteligentni, lepsi ludzie odkryliby tę
tajemnicę i~usunęli źle przydzielone bogactwo wszystkich tych
półgłówków. W~tym rynki były niesamowite.

Dawniej, kiedy wszyscy żyli w~wioskach na zadupiu, ci lepsi ludzie nie
mieli wyboru, ale musieli krzyżować się kimkolwiek, kto był w~pobliżu.
Nawet jeden na dziesięć milionów geniusz skończyłby złapany przez jakąś
krowiooką mleczarkę, rozpuszczając swoje niesamowite geny, które
otrzymał w~loterii natury.

Jednak krok po kroku, ludzie stawali się inteligentniejsi, gdy geniusze
spotykali bystrzejsze mleczarki, aż mogli wybudować rynki, a~potem
systemy informacji, na których rynki prosperowały, a~potem asymetria
informacji zaczęła się zapadać, początkowo wolno, potem wszystko na raz,
jak klif, który został podmyty przez nieubłagane pływy milionów ludzkich
działań, sprzężonych we wspólnym celu polepszenia gatunku i~jego
dominacji nad Ziemią.

Gdy informacja o~tym, gdzie najlepsi ludzie spędzają czas, stanie się
wiedzą powszechną, wtedy każdy, kto miał farta do dobrych genów i~wysokie IQ dokładnie wiedział, gdzie przynależy. Stopniowo miasta świata
wypełniały się potężnymi, samorządnymi ludźmi, których dyscyplina,
ciężka praca i~błyskotliwość oznaczały, że stają się coraz bogatsi.

Należy podkreślić, że socjaliści też to pojęli. Wiedzieli, że świat
zmierza do stanu, gdzie liczba osobników beta i~gamma, których alfa
potrzebują, żeby systemy działały, przekracza zapotrzebowanie, oraz że
ci niepotrzebni ludzie zostaliby wypchnięci, stopniowo i~wtedy, wszyscy
na raz. Nie odeszliby oczywiście bez walki. Oczywiście! Kto nie?

To było jak ,,gentryfikacja'', ale na wielką skalę. Rozczochrane
lewackie beksy myślały, że gentryfikacja jest jakiegoś typu konspiracją,
żeby wyruchać biedaków, ale to znowu był rynek. Położenie to kapitał:
jeżeli miałeś farta z~domem, który był niedaleko od ważnego miejsca,
powiedzmy, centrum finansowego czy pięknego krajobrazu, a~wszystko, co
zrobiłeś z~tym położeniem to wybudowanie gównianego domku, na którego
utrzymanie nie było cię stać na tym kawałku ziemi, wtedy rynek
rozwiązywał problem i~łączył Ciebie z~kimś inteligentniejszym i~lepszym
niż Ty z~kapitałem, żeby zapłacić Ci więcej za ten dom, niż myślałeś, że
będzie wart, a~potem nabywcy zrobiliby coś, że miejsce byłoby warto
jeszcze więcej. O wiele więcej. To właśnie robiły rynki: przesuwały
nieużyteczne, niewykorzystane aktywa z~rąk niekompetentnych i~przesuwały
je do rąk lepszych, którzy zaprzęgali kapitał do pracy. Jak gracz w~Monopol, który nie mógł domyślić się, w~jaki sposób zająć świetne
nieruchomości i~budował domy i~hotele, pół-zatrudnieni, pół-zatrudnialni
półgłówki, którym poszczęściło się trafić na dobre pozycje, wkrótce
odkryliby delikatnie wprowadzeni na miejsce, które lepiej pasowało do
ich wartości w~społeczeństwie i~w ludzkiej rasie.

Namydl, spłucz, powtórz: najbystrzejsi i~najlepsi zorientowali się jak
ulepszyć nawet najbardziej marginalne aktywa, dobrze poza możliwościami
99 procent, a~teraz 99 procent odkryło, że zostało uwolnionych od
wszystkich, ziemskich dóbr i~zostało bez pieniędzy na wynajem
jakiegokolwiek miejsca, czekając na śmierć.

To był ,,okres dopasowania'', dwa słowa, które brzmiały bezkrwawo i~biurokratycznie, ale opisywały chaos, który będzie królował, podczas gdy
niepotrzebny proletariat był zwalniany z~istnienia i~ludzkość ustawiała
się wokół najsilniejszych i~najmądrzejszych, których ewolucja mogła
dobrać.

(Martin nie był rasistą: dowolna para mogła wygrać na loterii
genetycznej i~dlatego było kilka czarnych facetów i~wszystkie te
hinduskie kobiety pracujące z~nim w~funduszu, bystre jak cholera, ale w~tym samym czasie, nie mógł nie zauważyć, jak wiele
najinteligentniejszych osób, których znał, miało ten sam kolor skóry co
on).

Ekonomiści nazywali to ,,okresem dopasowania'', ale ludzie jak Martin
nazywali to Wydarzeniem.

Martin zmagał się z~liczbami i~osobowościami, zanim wybrał Trzydziestkę.
Najpierw, dlaczego trzydzieści? Fort został zbudowany z~piętnastoma
sypialniami, zrobił obliczenia na osobę jedzenia, wody i~innych
wykorzystywanych rezerwy, a~każda dodana osoba mnożyła te liczby,
dodając dodatkowe pół metra regałów na papier toaletowy, racje
żywnościowe typu MRE\footnote{ ang. Meal, Ready-to-Eat -- ,,posiłek, gotowy do
jedzenia'', wojskowe racje żywnościowe -- przyp.tłum.} i~butelkowaną
wodę. Każda osoba przekładała się na więcej paneli słonecznych, więcej
baterii, więcej paliwa do generatorów.

Ale każda była także silną parą rąk do noszenia, budowania i~walki.
Połowa z~nich byłaby kobietami, połowa mężczyznami. Połowa singlami,
połowa w~związku. Martin był singlem, oczywiście myślał o~tym, kiedy
wybierał samotne kobiety do zaproszenia. Upewnił się, że wszyscy single
wiedzą, że to nie było zaproszenie ,,plus jeden'', to było zaproszenie
tylko i~wyłącznie dla nich. Jeżeli ich warunki się zmieniłyby, jeżeli
zaczęli coś na poważnie z~kimś, to powinni powiedzieć Martinowi, a~on
wybrałby kogoś innego na ich miejsce. To nie było tak, że płacili. Jedną
z decyzji, które Martin podjął wcześnie, było to, że wszystko było na
jego koszt. To był ruch gwiazdy, posunięcie alfa, rodzaj, który dałby mu
wiarygodność przywódcy, kiedy będzie to miało znaczenie. I~oczywiście
była dość duża szansa, że gotówka byłaby bezwartościowa po Wydarzeniu.
Zbudował ukryty skarbiec w~Forcie Zagłada ze sztabkami złota,
zaszyfrowanymi pendrivami z~BitCoinem, trochę dolarów, szafirów i~kilka
schludnych paczek renminbi, co wydawało się słuszne. Nigdy nie wiesz.
Mógł sobie na to pozwolić.

Mógł sobie pozwolić na to wszystko. Martin i~rynek rozumieli się
wzajemnie. Był gwiazdą w~firmie, osobiście zarządzając portfolio
wartości 11 miliardów, które osobiście rozwinął z~9 miliardów. Była
mowa, żeby dać mu własny fundusz, aby bezpośrednio z~niego korzystali
najwięksi gracze od wysokiej wartości netto, żeby dać mu tony wolności,
żeby robił, jak uważa. Kiedy myślał o~tym, jego mózg jakby dzielił się
na dwa różne, wzajemnie wykluczające się procesy myślowe: pierwszy
zaczynał dodawać bonusy, prowizję i~nową podstawę, drugi zastanawiał,
ile z~tego dostanie przed Wydarzeniem.

Może Wydarzenie nigdy się nie zdarzy. Rozumował w~oparciu o~Bayesa, a~nie o~szklaną kulę. Dawał szanse, nie pewność, jak dobry gracz w~pokera.
Kiedy zaczynał planowanie Fortu Zagłada, myślał o~szansie 40 procent.
Póki nie doszedł do pięćdziesięciu procent pewności, kiedy wbił łopatę,
używając tych dyskretnych wykonawców, którzy specjalizowali się w~schronach. Firma, którą wybrał, miała długą listę oczekujących, ale
kiedy opisał wielkość planowanej pracy, sprowadzili zaufanych
podwykonawców, którzy pracowali przy innych wielkich projektach i~zaczęli od razu. Teraz po zakończeniu, oceniał szanse Wydarzenia na
75\%, i~oczywiście, było możliwe, że myślał w~ten sposób, ponieważ Fort
Zagłada był tak zajebiście \textit{fajny}, byłoby \textit{niesamowicie}
ukryć się tam i~przeczekać chaotyczne miesiące lub lata, aż równowaga
zostałaby przywrócona.

Wezwał wszystkich zbyt wcześnie. Kiedy giełdy otworzyły się drugiego
stycznia, nastąpił nagły spadek, najwidoczniej przyśpieszony przez
pożary prerii w~Kanadzie, które powinny być pod takim lodem, że nic nie
powinno się tam palić o~tej porze roku. Sucha jesień, nieudane zbiory,
potem nietypowo gorąca zima, która zepsuła rzeczy w~wielkim rurociągu z~piasków roponośnych, wywołując przeciek w~rejonie leśnym, który zamienił
się w~pożar w~lesie, a~potem w~dziki pożar. Wybrzmiewało to przez cały
tydzień Świąt Bożego Narodzenia, stały temat w~wiadomościach: \textit{Oto
jak zjebane są rzeczy w~Saskatchewan, a~teraz historia żołnierza, który
wrócił do domu na Święta}. Martin przesunął pewne aktywa, kupił trochę
swapów, założył terminowe jak zawodowiec, nawet został wywołany do
prowadzenia spotkania dla całej firmy na temat ryzyk systemowych w~obecnej Sytuacji Kanady.

Ale inne firmy nie były tak dobrze zarządzane jak Martina. Niemniej
jednak, mieli dużo pieniędzy w~zarządzie, które zaczęło znikać jak złoto
głupców, gdy wyprzedaż rozpoczęła więcej wyprzedaży i~wszystko zaczęło
płonąć.

Więcej paniki sprzedażowej, więcej zarażania się, a~Martin ciągle
zachowywał spokój, przesuwając rzeczy z~jednego rynku przed płomieniami,
nawet zgarniając pewne okazje, kiedy przerażeni głupcy sprzedawali
aktywa, które powinny zatrzymywać. Też nie tylko instrumenty finansowe:
nabył na własność kontener pełen wczesnych Picasso w~obiekcie z~kontrolowaną atmosferą niedaleko Amsterdamu, a~nawet użył osobistego
konta, żeby kupić legendarne Chateau Mouton Rothschild, które przysłał
kurierem prosto do biura. Butelki przyjechały we własnych lodówkach
piezoelektrycznych zasilanych bateriami litowo-jonowymi, które
utrzymywały je w~optymalnej temperaturze i~wilgotności, a~on wpiął je
obok swojej torby. Cztery butelki, każda warta tyle, co nowym Mercedes,
i jedna dodatkowa butelka warta tyle, co rozsądny dom z~dwoma
sypialniami w~miłej części Phoenix. Przyprawiała go o~zawroty myśl, żeby
zostawić je tam, przy drzwiach, zamiast na dole schodów w~piwnicy, w~specjalnym zamykanym sejfie.

Może to była myśl o~wznoszeniu toastu apokalipsy jego winami
inwestycyjnymi w~Forcie Zagłada, ale znajdował się na granicy, prawie
gotowy ogłosić Wydarzenie i~udać się do fortu sześć razy na dzień.

Druga Wiosna Arabska była znacznie brzydsza niż pierwsza. Była zima, z~jednej strony, a~zamieszki żywnościowe były tak samo poza kontrolą jak
kanadyjskie pożary. Każdy gówniany dyktator w~regionie nauczył się
lekcji Baszszar al-Asada: kiedy ludzie ruszają na ulice, Ty ruszasz na
ludzi. Mocno. Krew była niesamowita.

Potem wybuchły bomby w~Houston, \textit{bum bum bum}, wycelowane w~policję, kwaterę główną policji stanowej i~budynek FBI przy Justice Park
Drive nr 1. Prezydent zapowiedział zemstę, Federalni zalali Houston i~wtedy wybuchła Atlanta \textit{bum bum bum}, potem nastąpiło powstanie na
Wyspie Rikers, którego nikt nie mógł powstrzymać.

Trzydziestka była na prywatnej grupie Signal, było dużo dyskutowania o~tym, kiedy i~jeżeli powinni się zbierać. Rzecz w~tym, nikt nie mógł
pojechać do Fortu Zagłada póki i~aż Martina tam nie było. Fort miał
poważne systemy obronne, a~tylko on wiedział, jak je wyłączyć, nie
zamierzał nikomu innemu mówić jak to zrobić, nie póki nie będzie
bezpiecznie w~środku. Ufał Trzydziestce wystarczająco, żeby spędzić z~nimi w~zamknięciu dłuższy czas, ale nie chciał kusić nikogo władzą
wyrzucenia go i~zmiany listy gości. Fort Zagłada był \textit{jego}.

Dom koło Martina został obrabowany. Każdy z~domów na jego ulicy miał
własną bramę i~ochronę, jak również ochronę miał cały kompleks,
przejawiany przez wynajętych policjantów, najczęściej eks-wojskowych,
poruszających się w~wózkach golfowych i~obserwujących wielkie rzędy
monitorów zasilanych przez kamery podczerwieni dookoła i~za płotem.

Zatem oczywiście, to była robota od środka. Frank Patel nie był subtelny
o upłynnieniu większości aktywów w~wymienne, przenośne bogactwo w~postaci drogich metali i~kamieni szlachetnych. Zagadałby Cię na śmierć o~sprytnych sposobach, żeby ukryć łup: ukryte woreczki w~ubraniach dla
niego i~rodziny, w~ukrytych panelach samochodu, ukrytym sejfie w~podłodze piwnicy, za cegłą w~kominku w~jego wielkim pokoju.

Słowo się rozeszło i~dwóch strażników wyłączyło kamery przy drzwiach,
zapukało, jak przy rutynowym wezwaniu, potem wpadło do środka, związało
jego i~jego dzieciaki i~zabrało wszystko. Ponad dziesięć milionów w~anonimowym, wymiennym majątku. Strażnicy zniknęli z~powierzchni Ziemi, a~G4S, które ich zatrudniało, sprowadziło poważnych prawników,
ubezpieczycieli i~detektywów, którzy zaczęli przebąkiwać poważnie, że
Patel był zamieszany w~rabunek. To było brzydkie. Pogorszyło się, gdy
Patel wywalił sobie mózg. Był poważnie zadłużony i~właśnie stracił
życiowe oszczędności. Szczerze mówiąc, Martin był zdziwiony, że nie
zabrał żony i~dzieci ze sobą.

To ich obraz, następnego dnia, ich twarze puste maski, gdy wjeżdżali
przez bramę Patela ku domowi z~garażem, to wystraszyło Martina tak, że
podjął decyzję, wysyłając masowo smsa do Trzydziestki, a~potem kierując
się automatycznie przez algorytm zabrania swojej torby, innych rzeczy (w
tym Rotschilds w~ich przenośnej maszynie podtrzymywania życia) do
opancerzonej Toyoty Tacoma z~bagażnikiem na dachu, zakamuflowanym, żeby
wyglądał jak rzeczy na biwak, ale właściwie tak opancerzonej jak sama
Tacoma, z~dodatkowym paliwem, oponami, bronią, amunicją, jedzeniem,
nawet piłą łańcuchową.

Ćwiczył to raz na tydzień, korzystając z~wewnętrznych drzwi łączących
dom z~garażem. Miał także randomizer, który nakazywał mu jeździć Tacomą
co jakiś czas, żeby nie było niezwykłe, gdyby z~niej skorzystał.
Zazwyczaj zatrzymywał się w~składzie drewna w~drodze do domu i~przywoził
jakieś czterometrowe deski cztery centymetry na dziesięć lub plastikowe
rury wystając z~tyłu bagażnika, na wypadek, gdyby ktoś patrzył.

Zatem wyruszył o~5:30 następnego ranka, co było w~zakresie parametrów
jego normalnego dnia pracy (były dni, kiedy wydarzenia poruszające
giełdami były przewidywane dla Londynu lub Frankfurtu i~wyjeżdżał nawet
wcześniej). Włączył tryb ,,lockdown'' w~domu i~pomachał strażnikowi w~bramce, gdy przejeżdżał, potem wrzucił nadajniki do bramy i~do opłat
drogowych do torby Faradaya i~wsunął je w~schowek. Na opuszczonym
czerwonym świetle zerwał górną warstwę ubrań roboczych, żeby odsłonić
dobrze zużyte, wysokiej jakości ubranie rybackie. Ostentacyjnie położył
pudełko z~przyborami na miejscu pasażera, żeby każdy, kto zajrzy,
zobaczył sprawnego faceta w~średnim wieku wyjeżdżającego z~miasta na
weekend łowienia na muchę, a~nie multimilionera z~interesującymi
ilościami łatwo wymienialnych dóbr i~rodzajem pojazdu, w~którym mógłbyś
żyć przez tydzień, bez zaopatrzenia, w~trudnym terenie w~obliczu wrogich
stron walczących.

Zatem zakamuflowany, przełączył telefon w~tryb samolotu i~jechał jeden
kilometr na godzinę poniżej ograniczenia prędkości przez następne cztery
godziny.

Tylko szesnaścioro z~Trzydziestki pojawiło się w~pierwszych czterdziestu
ośmiu godzinach. Z~trudem ukrył, jak był wściekły, ponieważ Szesnastka
musiała go postrzegać jako opanowanego, przewidującego lidera, nie
dziecko, które się denerwowało, gdy połowa gości nie przyszła na jego
urodziny. Brakujące osoby składały się z~czterech singli, wszystkie
kobiety, w~tym dwie najseksowniejsze, współlokatorzy, które zdecydowanie
nie były lesbijkami na podstawie ich historii randek (choć czasem
zastanawiał się nad możliwością, że nie były wyłącznie hetero). (Miał
jedyne królewskie łóżko w~Forcie Zagłada).

Szesnastka przesączała się, utrzymując cały czas ciszę radiową. Martin
posiadał satelitarne łącze do internetu w~Forcie i~osobiście zbadał
każde urządzenie na wycieki danych o~położeniu, zanim pozwolił podłączyć
do sieci Wi-Fi z~ukrytej anteny.

Witał przyjeżdżających z~doskonałymi, lodowato zimnymi Manhattanami,
które wcześniej przygotował i~zamknął w~plastikowych woreczkach do
,,sous vide'', razem z~wisienkami. Gdy przechodzili przez śluzę w~Forcie, wyciągał woreczek z~lodówki, otwierał, wlewał do lodu, dodawał
skórkę z~cytryny, ustawiał na małej, antycznej srebrnej tacy, którą
kupił na wycieczce do Wersalu.

Smakował te momenty, kiedy ktoś spięty i~zmęczony wchodził do
apartamentu gościnnego Fortu Zagłada -- pierwszy pokój koło śluzy
powietrznej -- z~grubymi perskimi dywanami, uspokajającymi akwarelami,
dostrzegając go, lidera, w~komfortowych, taktycznych spodniach i~koszuli, uśmiechającego się szeroko, błyszczące oczy, oferującego
koktajl. Semiotyka tego brzmiała: \textit{Witamy w~apokalipsie! To będzie
zabawa!}

Apartament gościnny był tak wspaniały częściowo, żeby zrównoważyć
nieuniknione rozczarowanie spartańskimi ,,luksusowymi pokojami'',
pojedynczymi i~podwójnymi z~dzielonymi toaletami chemicznymi i~wspólnym
prysznicem na końcu korytarza. Było mnóstwo świeżej wody w~Forcie,
zakładając, że byliby w~stanie zaopatrzyć się ponownie w~ciągu
pierwszego miesiąca, ale póki rzeczy się układały, większość
,,pryszniców'' byłoby wykonanych przez duży zapas bezzapachowych
chusteczek higienicznych Fortu.

Były też inne obszary wspólne: pokój gier, jadalnia, kuchnia, schron
schowany pod garażem. Rozrysował grafik: gotowanie, sprzątanie,
inwentaryzacja, wszystko, żeby zająć wszystkich. ,,Zaproponował'' -- to
jest nakazał -- godzinę oglądania wiadomości w~pokoju gościnnym, gdzie
wszyscy zeszliby się na raz, razem przeglądając kanały i~rekomendacje,
znajdując warte historie, które wydawały się zawierać i~podsumowywać
chaos na zewnątrz, a~potem dyskutując je w~kręgu z~moderacją Martina.
Ciężko było powiedzieć, czy rzeczy się pogarszały, czy było to zwykły
poziom szaleństwa, bardziej sugestywny przez to, że chowali się w~fortecy, czekając na upadek cywilizacji.

Czy gorzej czy nie, wydarzenia uderzały mocno w~świat na zewnątrz. Bomby
w Nowym Jorku. Zamieszki na przedmieściach Atlanty. Teoria spiskowa o~pożarach rurociągów w~Kanadzie, która twierdziła, że pożary były
przykrywką, choć czego, nikt nie potrafił zrozumieć. Wszyscy żartowali,
że to była przykrywka dla pedofili, ponieważ teorii takie były zawsze o~ukrywaniu pedofilów.

Statek z~uchodźcami dotarł do Miami, flotylla z~dziesiątek tonących wysp
Karaibów, żaglówek, płaskodennych łódek i~pontonów związanych razem i~lądujących na brzegu zaśmieconej plaży, pędzących na linie
pograniczników z~kulami gumowymi i~dryfującym chmurami gazu.

Ale Phoenix? Nie za bardzo. Spokojnie. Burmistrz i~gubernator
uczestniczyli w~odbiorze systemu dla biblioteki, potem poprowadzili
uczestników na ,,wieczór konstytucyjny'' świętować koniec fali gorąca i~kiedy tam dotarli, wiadomość się rozeszła i~tysiące ludzi do nich
dołączyło. Drogi się zakorkowały, gdy ludzie przeczytali na social
mediach, wsiedli w~samochody i~pojechali, a~kiedy korki stały się zbyt
duże, ludzie zatrzymywali się poboczu i~urządzali imprezy, wykupując
sklepy spożywcze i~rozdając piwa i~przekąski. Żadnych walk. Media
wypełniły się wspaniałymi obrazami, zespoły grające na trawnikach, gdy
dzieci tańczyły w~świetle lamp ulicznych.

Następnego dnia, Szesnastka stała się czternastką, kiedy jedna z~par
pojechała do domu. Prywatnie, Martin był wściekły, ale nie chciał tego
pokazywać. Zrobił przedstawienie pełnych uczucia uścisków, długiego
potrząsania dłońmi, zapewniania, że mogą wrócić, jeżeli i~kiedy
nadejdzie czas.

Zastanawiał się ile czasu minie, zanim Czternastka stanie się
Dziesiątką, Dziewiątką, aż będzie tylko Martin, czekający na apokalipsę
w prywatnej fortecy?

To nie trwało długo.

Pomimo oficjalnej zasady ,,jedna godzina wiadomości dziennie'', miał
(oczywiście) grupę botów, która sprawdzała nagłówki i~powiadamiała,
kiedy pewne słowa kluczowe zaczną się zbierać.

Uliczna impreza Phoenix napompowało dużo dobrych wibracji w~miasto, ale
to nie wystarczyło. Zatrzymanie ruchu na żywo zakończyło się, kiedy
glina Phoenix przestraszył się czegoś, co było bronią (to mógł być
telefon) i~zaczął strzelać w~samochód. Rodzina w~nim była tak szanowana,
jak mógłbyś pragnąć, dosłownie w~drodze z~kościoła do domu w~niedzielę,
a chłopczyk, który zmarł w~ramionach matki, miał na sobie maleńki
garnitur, muszkę i~białą koszulę, która pokazała krew.

Ulice wybuchły. Protesty, kontrprotesty z~białymi nacjonalistami
niosącymi kije i~drągi. Bójki. Zamieszki. Podpalenia.

Martin zauważył, że jedna z~singielek po cichu rozpakowała walizkę,
którą wcześniej widział, że po cichu pakuje. Czternastka naprzemiennie
wyglądała na zadowolonych z~siebie i~na zmartwionych. Zauważył, że
siedmioro więcej było w drodze do Fortu, a~wtedy jedna z~dwóch naprawdę
gorących singielek zapytała się emailem, czy mogłaby przywieźć podobnie
gorącą koleżankę z~pracy. Martin wyszukał koleżankę na Facebooku, zrobił
wyjątek w~swoich zasadach i~wysłał jej informację, żeby przyjeżdżała.
Natychmiast tego pożałował. To nie było dobre przywództwo. Ale co się
stało, to się nie odstanie. I~cholera, może to by ucichło.

Pierwszy kryzys pojawił się dwa tygodnie później. Internet pojawiał się
i znikał, i~w końcu zniknął dzień wcześniej. Było osiem stacji
naziemnych, których mógł użyć subskrybowany satelita, sześć z~nich
pojawiało się sporadycznie, a~potem przestało odpowiadać, a~pozostałe
dwa, w~schronach w~górach, daleko od szalonych tłumów, zniknęło.

Uruchomił alarm na całym perymetrze i~zamknął Fort. Fort miał
wzniesienie obok, a~kamery na szczycie obserwowały każdą drogę pod kątem
podchodzenia czegokolwiek większego niż zające wielkouche i~pokazywały
na ekrany w~całym forcie. Mieli dużo fałszywych alarmów. Żartowali jak
niespodziewana wielka gra była w~tej części Arizony, i~pytali, ile byków
zmieściłoby się w~zamrażarkach. Dowcipnie nie dyskutowali, dlaczego woły
wędrowały, nieprzywiązane i~bez towarzyszących hodowców.

A wtedy pojawił się Jeep, jadący wprost na Fort Zagłada, w~sposób, który
jasno pokazywał, że kierowca wiedział, gdzie zmierza. Kilkoro z~Trzydziestki dołączyło, kiedy ogłosili alarm, ale nie był to żaden z~pojazdów, których profile posiadał. Ponuro, odblokował broń. Z~dwudziestu dwóch w~Forcie, ośmioro było naprawdę dobrych z~karabinem, a~czworo zaledwie kompetentnych. Fort miał pozycje snajperskie ze
szczelinami strzeleckimi, specjalista od walut z~prawnikiem od przejęć
wspięli się po drabinach z~karabinami przypiętymi do pleców.

Przedsionek był strefą śmierci. Kiedy znalazłeś się w~środku, byłeś
martwy, zakładając, że Martin chciał Twojej śmierci. Nawet pełne
pancerze nie ochroniłby Cię przed modyfikowanym AR-15 w~pełnym automacie
strzelającym w~tym pomieszczeniu. Nawet jeżeli padłeś na ziemię.
Szczeliny strzeleckie były dobrze zaprojektowane.

Dżip zatrzymał się we wskazanym polu czekania i~pięcioro osób wysiadło.
Jedna z~nich zamachała na frontowe drzwi. Martin zbliżył obraz i~przeklął. To był Albert, którego znał z~siłowni. Zaczęli się kumplować
poza siłownią, kiedyś razem pojechali na polowanie i~Martin był pod
wrażeniem, jak dobrym strzelcem był Albert. Był dziesięć lat młodszy od
Martina i~przebiegał co roku maraton. Martin myślał, jak byłoby
użyteczne mieć kogoś, kto mógłby rąbać drewno przez osiem godzin bez
przerwy, a~potem zdmuchnąć mózg łani z~dwustu metrów, i~wtedy zaprosił
Alberta do dołączenia do Trzydziestki, robiąc z~tego małą ceremonię w~miłym barze, który obaj lubili, w~oddzielnej loży. Wznieśli toasty za
siebie pod koniec nocy i~Martin pozwolił Albertowi zapłacić rachunek.

Albert sprowadził przyjaciół.

Pomachał raz jeszcze, potem zrobił ruch dłonią, żeby wskazać na
przyjaciół, trzech facetów, w~jego wieku, wszyscy latynoamerykańscy jak
Albert, jedna dziewczyna, nie, kobieta czterdzieści plus, i~wzruszył
ramionami widocznie i~przepraszająco. Wskazał na innych, żeby zostali i~pobiegł do przedsionka, wchodząc do niego, w~strefę śmierci.

-- Martin, jesteś tam? -- powiedział w~siatkę mikrofonu.

-- Co się dzieje, Albert? -- Martin zachował ton głosu zimny i~biznesowy.
Wszyscy w~Forcie ich obserwowali.

-- Udało Ci się! -- Uśmiechnął się do kamery. -- Dzięki bogu. Możesz nas
wpuścić?

-- Albert, znasz zasady.

Uśmiech Alberta się załamał. Ściszył głos. 

-- Przepraszam, człowieku. Moi
kuzyni i~ciotka. Mój wujek\ldots  -- Jego głos opadł jeszcze niżej. -- Został
postrzelony. To było tylko ramię, ale poszedł do szpitala i~nigdy nie
wyszedł. Teraz jest tam milicja kontrolująca szpital. Nie mogłem
zostawić cioci i~kuzynów. Ona jest niesamowitą kucharką, a~chłopcy będą
pracować jak muły. Silni, wiesz?

-- Znasz zasady. -- Martin zastanawiał się, czy miał w~sobie to, co było
potrzebne do zamordowania Alberta, jego kuzynów, jego ciotki. Co gdyby
odeszli i~wrócili z~przyjaciółmi? Fort mógłby wytrzymać większość broni
małokalibrowej, ale oblężenie mogłoby trwać dłużej niż zapasy jedzenia i~wody. Przełknął i~po cichu przeklął Alberta. Co za dupek. Co za jebany
idiota. Zamierzał zmienić Martin w~mordercę.

Albert teraz się przypochlebiał. 

-- No dawaj, nie bądź taki. Zapłacę za
wszystko, co zjedzą i~tak dalej, jak tylko to wszystko się skończy. Z~odsetkami. \textit{No} weź, Martin.

Martin nacisnął przycisk na mikrofonie, który odtworzył przerażający
dźwięk przeładowywanej broni z~dodatkowymi basami. 

-- Idź, Albert. Idź i~nie wracaj.

Albert podskoczył na dźwięk broni, zrobił mimowolnie krok do tyłu. 

-- Nie wierzę, Martin. Jezu kurwa Chryste, czy ludzie, których tam zamknąłeś,
wiedzą w~co się wpakowali?

Prawie wtedy zastrzelił Alberta. Martin mógł wytrzymać wiele rzeczy, ale
wyzwania jego przywództwa były ryzykiem egzystencjalnym dla całego
projektu Fortu Zagłada. Wewnętrzna rebelia mogłaby wszystkich pozabijać
lub osłabić do punktu, gdzie byliby łatwi do zdobycia, kiedy maruderzy
zaczęliby się pojawiać.

Nie nacisnął spustu. Ale zdetonował jeden z~drogich ładunków na poboczu,
wysadzając fragment ścieżki prowadzącej do Fortu. Wybrał ten niżej, żeby
szrapnel przeciwpiechotny nie trafił w~kuzynów dupka Alberta, ponieważ
chciał, żeby odeszli, a~nie ociągali się i~umierali na jego progu. Co za
pieprzony burdel.

Albert wpatrywał się w~kamerę przez dziką chwilę, potem sięgnął do tyłu,
tam, gdzie miałbyś pistolet, Martin prawie go zdmuchnął, ale wtedy
Albert wystawił ręce z~powrotem przed siebie.

-- Pierdol się Martin. Jezu Chryste, ty psycholu, dupku, nie wierzę to
gówno. Mam nadzieję, że ludzie tam w~środku, wiedzą, że zamknęli się z~psycholem! -- Pochylił się do mikrofonu, żeby wykrzyczeć ostatnie słowa,
zniekształcając je w~głośnikach po drugiej stronie. Martin mrugnął.

-- Idę, Martin. Miłego dobrego życia. -- Poszedł z~powrotem do samochodu,
odbył krótką rozmowę z~rodziną, przepełnioną brutalnymi gestami dłoni.
Wszyscy wsiedli do dżipa, potem Albert wsiadł, pokazując długo Martinowi
środkowy palec przed przegazowaniem i~odjechaniem.

Tej nocy przez długie godziny Martin oskarżał siebie, próbując zasnąć.
Ciągle miał powracającą wizję Alberta pokazującego się za dni lub
tygodnie, ciągle wściekłego, ciężko uzbrojonego, z~przyjaciółmi. Lub
kuzynami. Albert pochodził z~jednej z~tych wielkich katolickich
hiszpańskojęzycznych rodzin. Miał wielu kuzynów. Martin zakopał bomby w~drodze, w~tym całkiem dużą, tuż przy przyczółku. Mógłby je wszystkie
uruchomić na jeden przycisk. W~końcu odpłynął po północy, ciągle
niepewny, czy skazał ich wszystkich na zagładę.

Jak się okazało, nigdy już więcej nie zobaczył Alberta. Niemniej jednak
spotkał jego ciotkę.

Miesiąc później pojawiło się trochę wiadomości na falach krótkich,
mieszanka BBC World Service, RT i~apokaliptycznych
kaznodziejów-apostatów mormońskich z~wielkomocowymi nadajnikami na
Arizona Strip, krzyczących o~,,proroctwie Białego Konia''. Mieszkańcy
Fortu pili wodę ze studni, żeby zaoszczędzić butelkowanej, na wszelki
wypadek przegotowaną, a~ich dietą była połowa racji MRE, w~połowie łoś i~jeleń zestrzelony z~wież.

Było ich teraz dwadzieścia troje, nabyli zwyczajów dzielenia się
sprzątaniem, gotowaniem, polowaniem i~tak dalej. Nie było wiele do
robienia, więc grali epickie mecze szachowe, mieli maratony filmów i~niektórzy z~nich dobrali się w~pary i~wymykali się na seks na świeżym
powietrzu lub zajmowali jedną z~sypialni i~zawieszali chusteczkę na
gałce.

Zawsze była co najmniej jedna osoba obserwująca ekrany z~CCTV szukając
intruzów, a~kilka raz musieli śpieszyć z~powrotem i~zamykać fort, kiedy
konwoje lub pojedyncze samochody przejeżdżały. To było opuszczone
terytorium i~nie było tego dużo. Kilka razy Martin przyłapał członków
Dwudziestu-Trojga wychodzących na zewnątrz bez obowiązkowych
krótkofalówek i~dobrze ich za to opieprzył, wbijając im do głów, że choć
sprawy były spokojne i~pokojowe za Fortem Zagłady, ciągle było dużo
chaosu, mordów tam w~świecie i~stawiali wszystkich w~niebezpieczeństwie,
wychodząc bez kontaktu. Jeżeli intruzi nadeszliby drogą, Martin nie
miałby sposobu, żeby ich przyzwać, ktoś mógłby zostać wykryty i~mógłby
sprowadzić całą grupę złych facetów. Krew i~śmierć byłyby następne.

Dżip przyjechał ścieżką i~uruchomił wszystkie dzwonki w~Forcie, gdy
wspinał się coraz bliżej. To był pierwszy raz, gdy pojazd podjechał tak
blisko Fortu od przyjazdu ostatnich maruderów.

Kiedy Martin rozpoznał dżipa, jego serce zaczęło mocno bić w~piersi. To
było to, inwazja, której się obawiał, konsekwencje jego głupiej
sentymentalności wróciły, żeby zakończyć ich życia. Posłał dwóch
najlepszych strzelców -- jednym z~nich była kobieta, którą odkrył, gdy
zorganizowali zawody strzeleckie -- na wieże strzeleckie i~poszedł do
przedsionka.

Dżip nie był w~dobrym stanie. Miał wielkie, brzydkie wgięcie po stronie
pasażera, jakby został uderzony i~przednia szyba była popękana. Wydech
wypluwał ciemny, brzydki dym. Dżip podjechał do parkingu i~przez dłuższy
czas stał nieruchomo, a~Martin spinał się coraz bardziej. Prawie dał
rozkaz wypełnienia go kulami.

Potem drzwi kierowcy się otworzyły i~ciotka Alberta wysiadła. Też nie
wyglądała dobrze. Chora, może, a~na pewno poruszająca się wolno. Była
ubrana w~kurtkę polarową, dżinsy i~ciężkie buty, które klapały na jej
stopach, tak na niej duże, że mógł dostrzec, że nie pasowały z~odległości stu metrów.

Wstrzymał oddech, gdy czekał, żeby pozostałe drzwi się otworzyły, ale
tak się nie stało. Kobieta obeszła dżipa i~się zatrzymała, opierając
dłoń na masce, oddychając i~patrząc na przedsionek. Wzięła głęboki
wdech, zebrała się w~sobie i~ruszyła w~ich kierunku, nieco kulejąc, buty
klapały.

-- Zatrzymaj się tam -- powiedział Martin, podkręcając głośność głośników
w przedsionku, dostatecznie głośno, żeby być usłyszanym na trawniku.
Zatrzymała się, zachwiała się lekko, rozglądając, potem znowu ruszyła.

-- Stój -- powiedział ponownie Martin. Szła dalej. Jej włosy wysuwały się
z kucyka, jej twarz lśniła od potu, podszyta zmęczeniem. Położyła dłoń
na żołądku i~jęknęła.

-- Stój! -- powiedział Martin i~odtworzył mp3 z~ładowaniem karabinu.
Wpatrzyła się w~ciemność przedsionka smutnymi oczami. Potem zachwiała
się.

Martin nie potrafił strzelić do chorej, zranionej kobiety w~średnim
wieku. Częściowo była to zwykła ludzka przyzwoitość -- Martin nie był złą
osobą, wiedział to -- a~częściowo był to całkowicie racjonalny rachunek
podtrzymania jego autorytetu moralnego w~Forcie Zagłada.

-- Czego chcesz?

-- Albert nie żyje -- powiedziała. -- Jest tak dużo chorób. Zabrała go,
potem moich chłopców. -- Jej głos był pusty i~płaski. -- Mówią, że to
cholera. -- Twarz jej się wykrzywiła w~minie smutku. -- Tak wiele osób
jest chorych.

Martin poczuł, jak jego jelita się zaciskają, wnętrzności w nich się
rozpuszczają. Miał Cipro i~inne antybiotyki ostatniej pomocy o~szerokim
spektrum, ale nie wymienił ich jeszcze tego roku i~niektóre z~nich
niedługo traciły ważność. Czytał, że ich efektywność zaczyna szybko
spadać, kiedy tak się zdarzy. Po planowanym ataku biologicznym, cholera
była jednym ze strachów, które nie dawały mu spać. Miał dużo broni, ale
nie mógł zastrzelić bakterii.

-- Jeżeli jesteś chora, to musisz odejść. -- Jego głos był twardy.
Wpatrzyła się w~kamerę przedsionka, zaciskając jelita, drżąc, mokra od
potu. Musiałby otworzyć bioskafander, ciśnieniowo umyć cały przedsionek
roztworem wybielacza. Nie poruszała się. 

-- Idź -- powtórzył.

Popatrzyła się, prawie zgięta w~pół, broda podniesiona, ciągle patrząc,
lub może gapiąc. Martin nie pozwolił sobie na empatię. To był
,,pasikonik i~mrówki'': wykonał całą pracę, żeby zapewnić sobie wygodne
schronienie przed zimą, a~ona zrobiła cokolwiek, co robiła w~zamian,
typu urodziła dzieci, oglądała reality tv, brała narkotyki, dziergała.
Ludzie, którzy nie myślą w~przyszłość, są samokorygującym się problemem.
Właśnie się korygowała, nie było to ładne, ale też nie był to jego
problem.

Jej broda drżała. Wstrzymywała łzy. 

-- Proszę -- powiedziała, słabym głosem, który ledwie dotarł do mikrofonu.

Za nim wyczuł ruch, ludzie dryfowali do apartamentu gościnnego,
obserwując interakcję. Obserwując \textit{go}. To był test przywództwa,
moment prawdy. Mógłby ją zastrzelić na miejscu. Dodatkowe płyny nie
stanowiły zagrożenia. Niezależnie od tego i~tak musiałby zmyć pod
ciśnieniem i~odkazić. Czy jego ludzie uznaliby to za pokaz siły, czy za
pokaz słabości?

-- Jeżeli wrócisz za godzinę, zostawię trochę zapasów na przyczółku. Nie
mamy dużo, ale możemy dać Ci jedzenie, trochę Tylenolu, ciepłe koce. Ale
nie mogę tutaj wpuścić chorej osoby. To nie jest szpital. Wprowadzisz
tutaj chorobę, wszyscy możemy jej dostać. Możemy wszyscy umrzeć. -- Ostatnie słowa szarpnęły jej głową, jakby strzelił pomiędzy oczy. Te
oczy, oczy martwej kobiety, która wiedziała.

Powoli się wyprostowała. Zrobiła krok do tyłu, ręce ciągle przyciśnięte
do jej żołądka jakby była postrzelona w~brzuch. Wycofała się z~przedsionka, potem odwróciła się i~odeszła powoli, ostrożne, pełne bólu,
kroki do samochodu. Udało się jej minąć zderzak, zanim skurcz ją zgiął w~połowie. Zobaczył jej kostki zbielałe na masce samochodu, gdy walczyła,
żeby pozostać w~pionie, potem znowu się wyprostowała i~weszła do
uszkodzonego pojazdu, odjechała powoli.

Poczekał, aż będzie poza zasięgiem najwyższych kamer, potem odetchnął
kilka razy, odłożył karabin, który ułożył w~linii szczeliny strzeleckiej
w drzwiach przedsionka, wyszedł z~apartamentu gościnnego z~ostrożnie
przybraną miną zatroskania i~pewności.

-- Co za katastrofa. Naprawdę jej współczuję -- powiedział, upewniając
się, że spojrzał w~oczy Lloydowi, fizycznie najsilniejszemu mężczyźnie w~Forcie, potem Saleha, najstarszej, i~Giorgii, najpiękniejszej kobiecie,
którzy wszyscy tworzyli rodzaj ośrodka władzy w~Forcie Zagłady.

Giorgia powiedziała: 

-- Nie sądzę, że potrafiłabym to zrobić, Martin. -- Nie potrafił określić, czy to był podziw czy oskarżenie.

-- To było trudne -- powiedział. -- Ale zbiorę tyle rzeczy, ile możemy
sobie pozwolić. Chciałbym, żebyśmy mogli zrobić więcej, ale wiecie,
choroba tutaj\ldots  -- Zrobił bolesną, zniesmaczoną minę. -- To byłoby
okropne. Nasze bezpieczeństwo jest najważniejsze. Ona będzie musiała
znaleźć własne przeznaczenie.

Saleha powiedziała: 

-- To marnotrawstwo zaopatrzenia. Jej się nie uda.
To, co robisz, to zachęcanie do zależności. Jeżeli przeżyje, wróci po
więcej, gwarantuję to.

Martin prywatnie się zgadzał, zanotował w~głowie, kto przytakiwał, a~kto
wyglądał, jakby myślał, że Saleha jest potworem przez takie opinie.

-- Nie dam jej niczego, czego nie możemy oddać. To jest właściwy czyn.

Potem zrobił pokaz z~otwierania skafandra spod jednej z~ławek,
dopasowywania maski i~ochraniaczy na oczy, wyciągania myjki
ciśnieniowej. To było słuszne, ale jednocześnie było to bardzo
efektowna, poprawna rzecz do zrobienia. Rodzaj rzeczy, którą kompetentny
lider by zrobił, żeby ochronić swoich ludzi.

Wepchnął skafander do worka na materiały niebezpieczne, spakował paczkę
z jedzeniem -- myślał o~dodaniu kropli do oczu do wody, które by
załatwiły ciotkę Alberta, ale w~końcu, nie mógł się zmusić do zostania
trucicielem, to był najtchórzliwszy sposób zabijania -- i~zawiózł je na
przyczółek w~ATV. Lloyd osłaniał go z~karabinem z~jednej z~wież i~cały
czas trzymał krótkofalówkę.

Następnego dnia, paczka zniknęła.

Minął tydzień. Drugi. Trzeci. Radio było pełen wiadomości, wszystkie
sprzeczne i~niemożliwe do zrozumienia. Istniała ,,Tymczasowa Republika
Arizony'', ale jeżeli istniała, a~nie byli to tylko jacyś dupki z~nadajnikiem, to była bałaganem. Facet, który stylizował się na
,,Sekretarza ds Komunikacji'' Republiki, sprawiał, że Alex Jones\footnote{
prawicowy gospodarz programów radiowych,
zob.~\url{https://en.wikipedia.org/wiki/Alex\_Jones} -- przyp.tłum.} brzmiał spokojnie i~spójnie, sam przemawiał po szesnaście
godzin ciągiem, co \textit{musiało} być napędzane amfetaminą. Z~pewnością
tak brzmieli.

Martin wyznaczył Gene jako kwatermistrza. Gene był młodym człowiekiem
bez wielkiej ambicji. Był młodszym kupcem i~nie miał planów, ale był
łatwy do manipulowania i~nie umiał kłamać, Martin cenił obie te cechy. W~jego doświadczeniu, tworzyły przystojniaków najwyższego kalibru. Gene
nie potrafiłby podebrać zaopatrzenia, a~nawet gdyby, ujawniłby się przez
zamienienie się w~jąkający bałagan przy pierwszych oskarżeniach.

Gene zrobił arkusze z~legendą kolorów, pokazujący ich poziom zużycia
wszystkich środków, z~doskonałymi wykresami słupkowymi i~kołowymi dla
łatwego odczytu. Wynik był taki, że mogliby żyć z~zasobów przez kolejne
dwa miesiące w~miarę potrzeby, ale to wymagałoby racjonowania i~mnóstwo
obiadów z~suszonego mięsa, tabletek witaminowych i~suplementowania
błonnika, żeby to wszystko się poruszało.

Nikt tego nie chciał, a~poza tym, wszyscy byli cholernie znudzeni,
zaczynali tworzyć grupy, ruchać się, rozstawać, ogólnie dziczeć. Zatem
Martin zadekretował, że będą polować i~plądrować zaopatrzenie, żeby
mniej zużyć magazyny, celując w~minimum pięćdziesiąt procent wszystkich
zużywalnych pochodzących spoza Fortu. Oczywiście grali w~polowanie,
Lloyd mógł udawać woła czy sarnę, a~reszta się uczyła, Giorgia okazała
się doskonałą kucharką, która robiła wszystko od steków kowboja do
tadżinu z~wątroby ze zwierząt, które przynosili.

Wyprawy łupieżcze były nowe. Fort był bardzo oddalony, ale Martin
rozrysował wszystkie bliskie osady: kilka ranch, stację paliw z~dołączonym sklepem, podstacja Policji Stanowej Nevady. Łupieżcy
wyjeżdżali elektrycznymi quadami w~ciszy w~świetle księżyca, ubrania w~szarość, z~twarzami poczernionymi i~dłoniami w~czarnych rękawiczkach,
robili rekonesans przy pomocy noktowizorów przed podejściem do pustych:
stacji paliw i~dwóch farm.

Stacja paliw była zniszczona, splądrowana i~ktoś próbował wysadzić
zbiorniki, ale systemy przeciwpożarowe zadziałały, pozostawiając
wszystkie powierzchnie pokryte rakotwórczym osadem. Ale rancza zawierały
wszelkiego typu skarby: jedzenie w~puszkach, warzywa korzeniowe, koce,
ubiory, którymi się bawili, starożytne gry planszowe, pudełko kart do
grania i~inkrustowane żetony pokerowych, albumy fotograficzne, nad
którymi się śmiali i~spekulowali.

To zabiło kolejny tydzień, a~w~trzy dni byli w~stu procentach
uzależnieni od zewnętrznych źródeł żywności. Wprawdzie wiele z~tego
pochodziło z~farm, nieodnawialnych zasobów, ale jak Andreas wskazał,
jeżeli gówno nadal trafiałoby w~wentylator, byłoby znacznie mniej osób i~to znaczyło więcej łosi i~tym podobnych. Andreas miał również coś do
powiedzenia na temat delikatesów, które mogłyby być przygotowane z~jadalnych kaktusów i~pomysł jak określić, które kaktusy się
kwalifikowały.

Były inne rancza, znacznie dalej, a~nawet małe miasteczko z~sklepem z~pamiątkami Nawaho oraz sklepem z~bronią. Martin obliczył, że gdyby
zebrali razem oddział i~wyposażyli jedno z~opuszczonych farm jako bazę
operacji, mogliby robić najazdy na te dalsze cele bez ryzykowania
baterii pojazdów czy oderwania się daleko od Fortu, że nie mogliby
wycofać do niego, gdyby ktoś potrzebował pierwszej pomocy.

Nie było żadnego dobrego powodu, żeby ruszyć na sklep z~bronią, ale
wszyscy byli podnieceni perspektywą. Życie od czasu wizyty ciotki
Alberta i~potem oglądanie ruin sklepu i~stacji paliw przekonało ich, że
byli inteligentniejsi i~lepsi niż idioci tam w~świecie, którzy nie
przygotowali się w~ten sposób co oni. Kiedy próbowali wyobrazić sobie,
co by robili, gdyby utknęli tam w~szerokim świecie, podczas gdy grupa
bystrzejszych osób żyłaby przytulnie w~Forcie Zagłada, łatwo było sobie
wyobrazić siły maruderów próbujące przejąć Fort. Broń byłaby poręczna, a~poza tym jeżeli by jej nie zabrali, ktoś inny by to zrobił i~wtedy broń
mogłaby skończyć wycelowana w~Fort. Kiedy myślałeś o~tym w~ten sposób,
to był praktycznie imperatyw moralny.

Użyli dronów, żeby sprawdzić drogę, robiąc kilka wysokich przelotów, a~potem kilka niskich przelotów, żeby poczuć, kto może być w~rejonie.
Natknęli się na Meksykanów na zewnątrz schludnie utrzymanego domu,
pielęgnujących ogród, wskazujących na drona, kiedy brzęczał koło nich,
ale poza tym, nic. Martin wybrał naprawdę odległe miejsce od jego Fortu,
odległe nawet według standardów Arizony, gdzie było tyle pustego
miejsca, że mógłbyś zejść ze ścieżki i~nie ujrzeć nikogo przez dni.
Oczywiście, gdyby pojechali nie tak daleko w~przeciwnym kierunku, byliby
na przedmieściach Phoenix i~bóg wiedział, jakiego rodzaju chaos by tam
odnaleźli. Jeszcze więcej powodów, żeby zdobyć broń.

Planowanie skoku na broń było niesamowitą zabawą. Martin zamienił
apartament gościnny w~pokój wojenny, wyciągnął białą tablicę, przypiął
fotografie z~powietrza z~dronów. Zaczęli nazywać ranczo ,,Bazą
Operacyjną Alfa'', co było \textit{naprawdę fajne} do mówienia. Zbliżenie
się dostatecznie blisko do sklep -- Big Tom, jak go nazywali -- żeby
sprawdzić warunki szczegółowo, było trudne: drony miały zasięg nieco
ponad kilometr, a~teren dookoła Big Toma był płaski, z~niewielkimi
zaroślami, więc nie było żadnej osłony, gdy drużyny zwiadowcze
pilotowały drony. Mieli tylko nadzieję, że będzie nieruszony, gotowy do
otwarcia przy pomocy palnika acetylonotlenowego z~Fortu. Alan i~Crispin
zostali wybrani do kierowania pojazdem towarowym -- opancerzonym pickupem
Martina -- żeby zabrać łup, gdy miejsce będzie otwarte.

Martin wyznaczył poniedziałek na misję -- używali wielkiego papierowego
kalendarza, zawieszonego w~widocznym miejscu apartamentu gościnnego,
żeby śledzić następowanie dni i~nocy -- a~wieczór przed zrobili grilla ze
stekami i~wypili trochę jego najlepszego burbona, w~tym butelkę
dwudziestoletniego Pappy, którą Gene starannie odnotował w~arkusza
kwatermistrza. Było wystarczająco dużo na trzy kieliszki dla każdego, a~Martin nie potrafił się powstrzymać od przechwalania, że każdy z~tych
kieliszków zawierał alkoholu za sto dolarów według cen przed upadkiem.
Kto wiedział, ile osiągnęłoby na rynkach nowego świata, kiedy powstanie?
Zabawiał się krótką fantazją przewodniczenia w~jakimś kramie na rynku w~jakimś miasteczku, otoczonym przez najemników, targując się o~dobra z~książątkami pobliskiej fortecy. To było śmieszne, jakby u Frazetty\footnote{
amerykański artysta plastyk pochodzenia włoskiego, tworzący w~dziedzinie
fantasy i~science fiction,
zob.~\url{https://pl.wikipedia.org/wiki/Frank\_Frazetta} -- przyp.tłum.}, ale trudno było się otrząsnąć z~tego obrazu.


Misja okazała się katastrofą.

Brett nie wrócił. Był na szpicy, kiedy pędzili pod osłoną ciemności do
Big Toma, z~noktowizorami. Jego ubezpieczenie, Alan i~Crispin, okrążyli
szeroko i~nadjeżdżali z~drugiej strony, czekając na Bretta, żeby wszedł.

Crispinowi też się nie udało.

Alan był tym, który opowiadał historię, drżąc i~płacząc. Zdjęcia z~jego
GoPro były gówniane, nawet z~włączonymi filtrami nocnymi, niezrozumiały,
rozmazany bigos.

Brett podkradł się na sto metrów, potem pełznął, potem jeszcze niżej,
posuwając się powoli przez parking Big Toma, próbując utrzymać te kilka
samochodów pomiędzy nim a~sklepem, poruszając się wolno i~gładko. Alan
trzymał karabin wycelowany we frontowe drzwi, Crispin poruszał lufą,
obracając pomiędzy różnymi celami: okna, dach, ślepy kąt. Za każdym
razem, gdy poruszał lufą, szeptał ,,czysto'' w~krótkofalówkę.

-- Czy teraz? -- spytał Brett. Oddychał ciężko.

-- Teraz -- powiedział Crispin.

-- Teraz -- powiedział Alan.

Brett podniósł się, jak sprinter do startu, i~pobiegł do drzwi.
Rozpłaszczył się przy murze. Usłyszeli jego oddech chrypiący w~słuchawkach. Alan wolałby, żeby był tam razem z~Brettem. To było
zajebiście \textit{fajne}.

Ujrzeli włączające się małe, czerwone światełko, przesuwające się nad
drzwiami, potem gasnące. 

-- Zamknięte. Dobry zamek, też. Stalowy szyld.
Ok, zapukam.

Zastukał w~drzwi. Dyskutowali o~tym, ale w~końcu zdecydowali, że tak
będzie najlepiej. Jeżeli ktoś był w~środku, lepiej nawiązać kontakt w~ten sposób, niż wejść przez zaskoczenie. W~końcu to był \textit{sklep z~bronią}. Uderzył. 

-- Halo? Jest ktoś w~domu?

Wszyscy wstrzymali oddech. Oddech Bretta zaskrzypiał. 

-- Ok. -- Odległy
dźwięk rzepów, gdy wyjmował wiertarkę Makity z~kieszeni i~montował
wysokoobrotowe wiertło do stali. Ujrzeli stożek światła z~wiertarki, gdy
nacisnął przycisk, jednocześnie podtrzymując główkę, żeby zacisnąć
wiertło, potem przykucnął i~wsunął wiertło we wpust zamka. Usłyszeli
odległy jęk wiertarki, lekko poza fazą wobec dźwięku z~krótkofalówek. 

-- No dawaj mały gnojku -- wyszeptał Brett. Potem: 

-- Gówno. -- Przerwał
wiercenie, dźwięk złamanego wiertła upadającego na ziemię, grzebanie się
z wymianą, potem jęk wiertarki znowu się zaczął.

-- Przeszło na wylot. Zmieniam na większy rozmiar. -- Dźwięki pracującej
nasady, stuk wierteł, chrypienie oddechu Bretta, dźwięk Crispina
mówiącego ,,czysto, czysto, czysto'', gdy poruszał karabinem.

-- Ok -- powiedział Brett, dokładnie, gdy dookoła budynku włączyły się
oślepiające światła. Crispin wystrzelił, odruchowo (pomyślał Alan),
potem były jeszcze dwa strzały i~Brett padł do przodu. Przez lunetę,
Alan zobaczył dym unoszący się z~dwóch dziur po kulach w~drzwiach i~natychmiast dojrzał postrzępione krawędzie cienkiego metalu, który
zasłaniał szczeliny strzeleckie. Więcej strzałów i~obłoków dymów uniosło
się z~parkingu. Jeden z~zepsutych SUV stracił szybę w~roztrzaskanych
kostkach szkła, które błyszczały w~niesamowitym świetle.

-- Brett? Brett? -- spytał Alan. Krew wypływając spod dookoła niego była
niewątpliwa, Alan poczuł zimno, gdy zrozumiał, jak był wystawiony.

Miał właśnie wstać i~pobiec do tyłu, gdy włazy na dachu sklepu
odskoczyły jednocześnie i~czterech snajperów wyskoczyło, kładąc ogień
zaporowy, gdy poruszali się na stanowiska za ceglanym murkiem na rogach
dachu. Mieli okulary strzeleckie, ubiór taktyczny, twarze srogie, ale
usta otwarte w~szerokich uśmiechach, niesłyszalne okrzyki, gdy strzelali
w noc.

Usłyszał, jak Crispin umiera przez krótkofalówkę, mięsny odgłos, krzyk,
drugi, westchnięcie, potem nic. Dookoła nóg Alana pojawiła się lepkość. Zsikał
się. To było idiotyczne, nieporozumienie. Nie chcieli walczyć. Nie było
\textit{powodu} do walki. Fort Zagłada już miał mnóstwo broni, więcej niż
kiedykolwiek mogliby wystrzelić. To była przerwa od nudy, gra. Zajęczał.

Czy go namierzyli? Snajperzy skanowali teren powoli i~metodycznie.
Chciał do nich zawołać, poddać się, ale wiedział, że zostałby
zastrzelony na miejscu. Rozumiał, że to jest dokładnie to, co zrobiliby
w Forcie Zagłada, gdyby ktoś pojawił się tam, uzbrojony, rozwiercając
zamki, osłaniany przez innych. Umrze tutaj.

Alan odnosił sukcesy jako deweloper nieruchomości, zaczynał od obrotu
domami w~trudnym położeniu, potem nawiązał znajomości w~biurze
planistycznym miasta, zaczął otrzymywać wewnętrzne przecieki, które
miejsca stawiłyby najmniejszy opór, gdyby chciał je wyburzyć i~zbudować
wieżowce. Zebrał trochę gotówki i~zaczął kupować budynki za gotówkę, dwa
lub trzy w~rzędzie, na starych wielkich działkach, siedziby rodzinne
sięgające pokoleń wstecz. Specjalizował się w~ich rozbiórce, zanim
ktokolwiek wiedział, że coś się działo, rozwalając je do ław
fundamentowych tego samego dnia, wejście i~wyjście, zanim pojawią się
pierwsze skargi na hałas. Nazywał swoich ludzi Załogą Wyburzeniową,
płacił dodatkowo, jeżeli skończyli przed zachodem słońca. Mieszkania,
które najpierw stawiał, płaciły za to wszystko, potem kamienice, potem
wieżowce, potem kompleksy. Rozumiał siłę nieubłaganej szybkości. Czasem,
wyobrażał sobie siebie jako rekina, nigdy nie zatrzymującego,
rzucającego się pochwycić ofiarę, poruszając się od długu do płatności
do długu do płatności, zbierając zysk.

Już nie czuł się rekinem.

Próbował być ślimakiem. Robakiem. Nisko położony, wolno poruszający. Tak
wolny, żeby go nie widzieli. Wsadził twarz w~szorstki, zimny od nocy,
piasek, próbując leżeć tak płasko, jak się da. Używając czubków palcy i~czubków butów, cofał się centymetr za centymetrem do tyłu, w~kierunku
quada, który zostawił zaparkowany dwadzieścia metrów wcześniej. Nikt do
niego nie strzelał. W~słuchawce słyszał dźwięk z~radia Bretta, cicha
rozmowa snajperów, jedno słowo na trzy: \textit{dostałem widziałem dobry
strzał skurwysyn}.

Kolejny centymetr. Wstrzymując oddech, potem oddychając płytko.
Robakiem. Ślimakiem. Kolejny centymetr.

Kolejny. Policzyć do dziesięciu. Kolejny. Skrobanie otwierających się
drzwi Big Toma, uderzenie, gdy uderzyły w~ciało Bretta, wolne sunięcie i~chrząknięcie, gdy ktoś pchał razem drzwi z~ciałem Bretta. \textit{Jebane
dupki} w~radio.

Buty na betonie. Szuranie, gdy krótkofalówka była podnoszona. 

-- Hej, któryś z~dupków tam mnie słyszy? -- Alan wstrzymał oddech, przesunął dłoń
do krótkofalówki przy pasku, wyczuł na ślepo przycisk wyciszenia. Czy to
było to? Pragnął odetchnąć, a~to zdecydowanie byłoby sapnięciem.
Ślimakiem. Robakiem. Nacisnął przycisk. Ciche kliknięcie.

-- Halo? Założę się, że te skurwysyny mają tam wsparcie. Greg, widzisz
kogoś? -- Stłumiona odpowiedź. -- Tak, ok. -- Szum. -- Lizzy, zgaś światła,
co? Noktowizory są bezużyteczne przy włączonych. Zobaczmy, co my tutaj
mamy.

Krew Alana zamarzła. Siki pokrywające nogi zamarzły. Kurwa. Posuwał się
szybciej. Szybciej. Rozmowa w~słuchawce, niezrozumiała przez skrobanie
jego ubrań, chrapliwy oddech. Szybciej.

Klik.

Światła zniknęły. Zielone plamy zatańczyły mu przed oczami, mimo
długiego czasu, gdy przyciskał je do ziemi. Zerwał się na nogi i~\textit{pobiegł}, sprintem do miejsca, gdzie myślał, że zostawił pojazd,
nie widząc, potykając, a~za nim rozległy się krzyki. Strzał, dwa
kolejne. Padł na ziemię, szlochając, jednak nietrafiony. Na nogi,
próbując zygzakować, jego spodnie przylepione do nóg sikami, nie
współpracujące, a~potem uderzył twarzą w~pałąk pojazdu, gwiazdy przed
oczami. Upadł na quada, rozpłaszczając się nisko. Kula przeszła przez
pojazd, kolejna zrykoszetowała w~pustynię. Na niego opadły pyłki kurzu.
Klucze były w~stacyjce, zostawił je, zwracając na to uwagę, część
ostrożnego planowania, które było taką zabawą na tablicy.

Przekręcił kluczyk, schylił się w~pojeździe, głowa w~dole. Dziwacznie
ułożona stopa na gazie i~elektryczny silnik komutatorowy się włączył,
pojazd skoczył do przodu, prawie go zrzucając, stopa zsunęła się z~pedału gazu i~pojazd zatrzymał się nagle, znowu prawie go zrzucając.
Wgramolił się na siedzenie i~dziura po kuli pojawiła się na szybie nad
nim. Zamknął oczy, jęcząc, gdy ostrożnie siadał w~siedzeniu, wyżej,
głowa odsłonięta, ręce na kierownicy, stopa znowu na pedale gazu, a~tym
razem przygotował się na przyśpieszenie i~quad \textit{skoczył} do przodu,
otworzył oczy, włączył światła, ponieważ musiał wiedzieć, gdzie jedzie,
a oni mieli noktowizory, szarpnął na boki, próbując poruszać się losowo
i nieprzewidywalnie (w dużej części z~sukcesem, ponieważ był zbyt
wstrząśnięty, by sterować w~jakikolwiek sposób, tylko chaotycznie).

Więcej strzałów wpadło w~ziemię dookoła niego, jeden dostatecznie
blisko, żeby obrzucić go twardym żwirem, który rozciął jego policzek.
Gdy krótkofalówka wychodziła z~zasięgu, usłyszał ich pohukujących i~klnących.

Jechał, jechał i~jechał, zatrzymując się na opuszczonym odcinku drogi
dobrze widocznym w~każdym kierunku, gdzie rzygał, póki już nie miał
niczego do oddania, prócz śluzowatej plwociny. Na siedzeniu pasażera
miał lodówkę, otworzył ją, wypił puszkę Coca coli, połknął dwadzieścia
miligramów Adderallu, a~potem znowu uruchomił pojazd.

Pojechał najpierw do Bazy Operacyjnej Alfa, drżącymi dłońmi otworzył
zamek, który założyli na drzwi. Chodzenie po znajomym domu farmerskim z~półkami zaopatrzenia i~stosami książek, gier planszowych, jak również
torbą na noc, którą spakował, dało mu poczucie kompletnej nierealności.
Tylko kilka godzin wcześniej, był tutaj z~Brettem i~Crispinem, śmiejąc
się, powtarzając plany, drżąc z~podniecenia. Teraz drżał ze strachu w~pokoju, krwawiąc, w~głowie zawroty, żołądek jednocześnie pusty i~mdłości, wzrok niezogniskowany, nogi spierzchnięte od własnych sików.

Zrzucił spodnie na podłogę, wyjął garść chusteczek higienicznych z~pojemnika i~wyszorował nogi, dupę i~jaja, rzucając brudne chusteczki na
podłogę. Postawił torbę i~znalazł dodatkową parę spodni, założył je,
potem przeniósł zawartość kieszeni brudnych spodni. Zmienił koszulę.
Miał kluczyki do quada. Rozejrzał się po pokoju. Tu było jedzenie.
Powinien zabrać jedzenie. Garść pepperoni, pudełko batonów
energetycznych, zgrzewka Gatorade. Adderall śpiewał w~żyłach, sprawiał,
że czuł się, jak za pierwszym razem wypił szklankę kawy, jakby
nadchodziły supermoce.

Poczuł szalone, prawie niepowstrzymywalne pragnienie, żeby spalić
ranczo. Coś w~tym, nawiedzanego przez duchy ludzi, którzy zmarli tej
nocy, pełna dowodów ich idiotyzmu. Mógłby to zrobić. Przywieźli trochę
benzyny do rancza, żeby uruchomić zapasowy generator, kiedy panele
słoneczne nie zdołałyby doładować baterii. Chlup chlup, jedna zapałka,
\textit{trzask}.

Jednak to posłałoby pióropusz dymu widoczny na kilometry dookoła. Jeżeli
snajperzy z~Big Tom jechali nocą, szukając go, zobaczyliby to. Ranczo
było w~połowie drogi do Fortu Zagłada. Zatem nie spali tego.

Wsiadł na quada i~pojechał.

Żyjący mieszkańcy Fortu Zagłada obserwowali, jak Alan drży, łka, gdy
opowiada historię, Martin obserwował ich obserwujących Alana, próbując
zapanować nad wściekłością. Jebana \textit{nieudolność} i~\textit{panika}.
Alan wrócił do Fortu Zagłada, nie wiedząc, czy był śledzony, czy był nad
nim dron, czy samochody na ogonie. Gang Big Toma był zorganizowany,
bezlitosny i, oczywiście, fantastycznie dobrze uzbrojony. Martin mógłby
zamknąć fort i~zestrzelić ich z~wież, ale co jeżeli mieli ciężką broń,
granatniki, rakiety, tego typu rzeczy, które armia mogła przywieźć na
pole bitwy?

Szanse były marne. Prawdopodobnie przybili piątki, pochowali ciała i~świętowali wielkim śniadaniem z~naleśnikami. Ale co jeżeli nie? Jedną z~rzeczy, które Martin nauczył się jako kupiec: małe ryzyka z~olbrzymimi
kosztami nie mogły być ignorowane.

Zatem czekał, aż Alan opowie historię, wszyscy okazali współczucie,
powiedzieli miłe rzeczy o~Crispinie i~Brettcie, a~potem zabrał Alana do
swojego spokoju, ,,na drinka''.

Alan wyglądał drobno, pobity, skulił się w~jednym z~foteli w~pokoju
Martina. Martin nalał mu więcej niż normalnie whisky single malt na
wielkich kostkach lodu, rodzaj drinka, którym Alan zwykł świętować
triumfy w~interesach. Alan miał tych wiele, razem cieszyli się wieloma
pijackimi nocami w~jakiś miłych barach w~Phoenix, była nawet wycieczka
na ryby w~Bermudach, ze śmiertelnymi drinkami, które barman nazywał
,,Ataki Zaki''. Barman miał na imię Zak. Miły dzieciak. Porzucił Yale.

-- Wszystko będzie z~Tobą ok? -- Zmusił się do tonu obawy w~głosie.

Alan wypił. 

-- Chyba tak. Znaczy, tak, pewnie. Ludzie przeszli gorsze
rzeczy. Wielu ludzi. Szczególnie w~tych czasach.

Martin wziął łyka, obserwował go. Nie przewidział, że Alan tak by się
zwinął. Panika. Naprawdę, ześwirował. 

-- To dobrze. Ponieważ musisz coś
zrobić.

-- Co? -- Sięgnął po butelkę i~dolał do szklanki. Whisky przelała się po
dużych kostkach. Zastukał nimi o~krawędzie ciężkiej szklanki.

-- Musisz to załatwić.

Zaskoczona mina. Martin widział, że adrenalina i~Adderall przestawały
działać. Jego powieki opadały, ramiona coraz bardziej opadały.

-- Musisz wrócić i~upewnić się, że nie zostawiłeś ścieżki.

-- Co?

-- Słyszałeś mnie, Alan. Przyjadą tutaj za Tobą, to będzie całe gówno,
które przeszedłeś, razy tysiąc. Może wojna. Spierdoliłeś to\ldots 

-- Ja\ldots 

Martin podniósł głos. Jego pokój był jako jedyny dźwiękoszczelny w~Forcie. 

-- \textit{Kurwa} zamknij się, Alan. Zjebałeś, dupku. Spanikowałeś.
Jezu kurwa Chryste, nie pamiętasz tych wszystkich dyskusji, które
mieliśmy, żeby nie zabierać niczego, co mogłoby doprowadzić tutaj lub do
Bazy Operacyjnej Alfa, na wypadek, gdybyście coś zgubili, zostali
złapani lub zginęli? Pamiętasz je, Alan? Po co to było, jeżeli
zamierzałeś spierdalać jak zając, zostawiając bóg wie jakie ślady
wskazujące tutaj? Po co to było, Alan, jeżeli zamierzałeś narazić
wszystkich nas swoją paniką, lekkomyślnością, \textit{tchórzostwem}, Alan?

Alan oklapł pod słowami. Martin naciskał. 

-- Cóż, Alan, co masz kurwa do
powiedzenia na swoją obronę?

Czerwone oczy Alana wypełniły się łzami. 

-- Martin, przepraszam\ldots 

-- To doprawdy wspaniale, napiszemy to na naszych jebanych nagrobkach.
,,Alan przeprasza.''

-- Czego \textit{chcesz} ode mnie, Martin?

-- Jedź tam i~napraw to, głupi chuju. Jedź po śladach swojej trasy,
ostrożnie. Usuń każdy ślad, który zostawiłeś za sobą. Przejdź piechotą,
jeżeli będziesz musiał. Każde skrzyżowanie i~przynajmniej dwa kilometry
w każdym kierunku. Zabierz miotłę, grabki, cokolwiek potrzeba. Zajmij
się tym.

-- Nie spałem od\ldots 

Martin zagrzechotał buteleczką Adderall przed nim. 

-- Trzydzieści
miligramów powinno pomóc. Możesz spać, kiedy skończysz. Zabierz śpiwór,
namiot, jeden z~tych małych jednoosobowych Shiftpodów. Nie spierdol
tego.

Alan na sucho okrągłą niebieską pigułkę i~podźwignął się na nogi. Martin
pomógł mu się spakować, zaprowadził go do magazynu, obudził Gene'a, żeby
odegrał kwatermistrza, wyszczekując listę zaopatrzenia na Gene'a i~zmuszając Gene'a do załadowania ich do jednego z~nieruszonych ATV, razem
z dodatkową baterią dla pojazdu.

Odprowadził Alana przez duże drzwi towarowe po drugiej stronie wzgórza,
odsłaniając najpierw kamuflaż. 

-- Idź -- powiedział. -- Wróć z~tarczą lub
na tarczy.

Prawie świtało, kiedy usłyszeli strzał. Martin ruszył do monitorów,
przejrzał nagrania. Pokazywały pojazd Alana wspinający się na wzgórze
przed frontową bramą, potem Alana chodzącego dookoła, siedzącego w~siedzeniu kierowcy przez dłuższą chwilę, kontemplującego pustkę, pustynię i~noc. Potem, bez wahania czy strachu, Alan wyciągnął pistolet, włożył
lufę w~usta i~\textit{puk}. Strzał.

Martin zaczekał do świtu, potem zabrał myjkę ciśnieniową i~wyszedł
posprzątać.

Później tego samego dnia, zabrał quada, w~którym był Alan. Jego ślady
były oczywiste: zjechał na dół wzgórza, zrobił wielkie koło w~kierunku
autostrady, potem zawrócił i~wrócił. W~ogóle nawet nie próbował usunąć
swoich śladów.

Martin połączył się z~Fortem Zagłada, powiedział im, że zamierza wrócić
i sprawdzić ślady Alana pod kątem wskazywania na dom. Ciągle miał
wszystkie materiały Alana.

Okazało się, że Alan nie był taki oczywisty w~drodze do Fortu. Lub może
Martin nie był taki dobry w~czytaniu znaków. W~każdym razie nie odkrył
niczego pomiędzy Fortem a~Bazą Operacyjną Alfa, a~potem przekradł się
powoli i~ostrożnie od Bazy do Big Toma i, nie znajdując żadnych znaków,
wrócił do Fortu Zagłady, docierając po zachodzie, zmęczony i~obolały.

Nikt, prócz Gene'a, nie widział żadnej części prywatnej rozmowy pomiędzy
Alanem i~Martinem dzień wcześniej, a~Gene nie gadałby. Martin wydał
ponure dźwięki o~traumie, odwadze i~poświęceniu, i~potrzebie upewnienia
się, że zawsze mogą z~nim porozmawiać, z~każdym, ponieważ są w~tym
razem, byli teraz jednostką, naprawdę, rodziną i~mieli obowiązki wobec
siebie.

Użył koparki do wykopania płytkiego grobu dla Alana i~potem go wypełnił.
Nie oznaczył grobu, a~kiedy Giorgia zapytała go o~to, powiedział jej, że
nie chciał dawać znaków, że byli jacyś mieszkańcy w~okolicy Fortu
Zagłada, pokiwała głową poważnie, spojrzała na niego z~podziwem, który
wspominał tej nocy, gdy zasypiał.

Nuda była bezsprzecznie najgorszą częścią Końca Świata. Tam w~świecie,
rzeczy \textit{bynajmniej} nie były nudne, sądząc po transmisjach
krótkofalowych. Tymczasowa Republika Arizony wydawała się zniknąć z~eteru, ale CB radio odbierało grupy zwiadowcze, byli amatorzy, którzy co
jakiś czas wracali do życia, opowiadając historie okupacji wojskowej i~epidemii w~odległych miastach, wojen i~bombardowań. W~Provo była stacja
Gwardii Narodowej, która wysyłała codziennie przestrogę uchodźcom, żeby
trzymali się kurwa z~dala od Utah, było też tak\ldots  Wielu\ldots 
Kaznodziei\ldots  Grzech, grzech, grzech. Odkupienie, odkupienia,
odkupienie. Jeden z~nich ciągle twierdził, że Nowy Jork został
zbombardowany jądrówkami, aż gruz podskakiwał, ale to nie pojawiło się w~transmisjach innych, więc ludzie z~Fortu doszli do wniosku, że kłamał.
To uraziło ich uczucia, kilkoro z~nich dorastało w~Nowym Jorku,
większość z~nich pracowała dla funduszy i~banków z~biurami w~Nowym
Jorku, które wymagała ich obecności co jakiś czas, z~wieczorami
wypełnionymi drogim sushi i~bezcenną whisky.

Nikt nie miał chęci na więcej ,,misji'' i~po krótkim namyśle,
zdecydowali się zlikwidować Bazę Operacyjną Alfa. Opróżnienie rancza i~przywiezienie wszystkiego użytecznego do Fortu Zagłada zabrało kilka
tygodni, ale było wykonane, Gene wszystko zinwentaryzował, nie było
niczego do robienia.

Stworzyli kilka różnych typów par, a~nawet jedną lub dwie tria, a~pomiędzy cienkimi ściankami i~klaustrofobicznymi plotkami, wszyscy mieli
całkiem dobre poczucie, jakby to było przelecieć dowolną inną osobę.
Wszyscy byli skazani na przymusową izolatkę, w~towarzystwie innych. Izzy
był magistrem filologii angielskiej, zmusił ich do obejrzenia wideo na
swoim laptopie, produkcję z~Brodwayu sztuki \textit{Przy drzwiach
zamkniętych}\footnote{na podstawie sztuki egzystencjalisty J.P. Sartre'a,
zob.~\url{https://en.wikipedia.org/wiki/No\_Exit} -- przyp.tłum.}, wszyscy się śmieli na ,,Piekło to inni'', potem Izzy
wyhaftował to i~powiesił w~ramce w~apartamencie gościnnym. Wtedy Martin
zdecydował, że musi wkroczyć.

-- Będziemy musieli znowu zacząć plądrować. Wiem, że to niebezpieczne,
ale nauczyliśmy się pewnych rzeczy o~byciu ostrożnym. Dobre wieści są
takie, że nie musimy się śpieszyć, nie jesteśmy jeszcze zdesperowani.
Znajdziemy miejsce, które chcemy sprawdzić, możemy sobie pozwolić na
obserwację, przez kilka dni, nawet tydzień, upewnić się, że jest
bezpiecznie, zanim ruszymy. To luksus, który mamy, ponieważ spiżarnia
nie jest pusta. Ale jeżeli pozwolimy na opróżnienie spiżarni, stracimy
ten bufor, będziemy musieli działać w~sytuacji takiej, jaką będzie, albo
zacząć głodować. Ten sposób jest niebezpieczniejszy niż wyjście teraz.
Jedyną gorszą rzeczą, jaką było przeżycie tragedii utraty Crispina,
Alana i~Bretta, byłoby, gdyby ich śmierć była na próżno, że nie
nauczyliśmy się niczego z~nich.

Nikt nie lubił tego pomysłu, ale wszyscy się zgodzili, że miał rację.

Drobiazgowe mapy Martina zawierały wiele możliwych pozycji zwiadowczych,
dalej niż domy farmerskie. Mieli szczęście z~rodzinną farmą, była
zamieszkana, ale ludzie tam byli skłonni wymienić jaja i~ser za kamienie
szlachetne. Martin uwielbiał to: zachował dużo kamieni szlachetnych, w~niektóre noce zastanawiał się, czy nie żartował sobie, czy ktokolwiek by
ich chciał. Nie możesz zjeść klejnotów, a~wymiana ich na użyteczne dobra
wymagała czegoś jak funkcjonująca cywilizacja. Fakt, że ci stoiccy
farmerzy latino byli skłonni wziąć woreczek rubinów i~szafirów za pięć
wielkich kręgów sera i~obietnicę czterech tuzinów jaj na tydzień przez
następny miesiąc oznaczała, że oni także myśleli, że cywilizacja w~końcu
by wróciła.

W końcu to był plan Martina: poczekać, aż porządek się umocni, potem
pojawić się z~wszystkimi rzeczami koniecznymi do zabezpieczenia sobie w~nim pozycji: dobra handlowe, akcje na okaziciela, gotówka, i~jego umysł,
wszystko poparte lojalną grupą zwolenników, którzy wiedzieli jak
wystrzelić z~każdego karabinu w~obfitej zbrojowni Fortu Zagłada.

Cywilizacje już wcześniej powstawały i~upadały. Ludzkość potrzebowała
współpracy, ale piekło to inni ludzie. Kiedy najlepsi ludzie byli na
szczycie, rzeczy działały: przekonywali racjonalnych, pochlebiali
upartym i, szczerze, zmuszali resztę. Dla większego dobra. Postaw
jednego z~takich nieudaczników, pasożytów, na szczycie i~poprowadzą
resztę ku katastrofie. Jedna rzecz była zawsze jasna dla Martina przez
całe życie: pasożyci sterowali statkiem i~oni go rozwalą.

Kolejna farma była złupiona i~spalona, co było przerażające. Potem
znaleźli miasteczko, najdziwniejsze z~najdziwniejszych rzeczy! Elfrida
miała dwa sklepy, gdzie sklepikarze prowadzili rejestr na tablecie,
zapisując, co każdy zabrał. Wszyscy zgodzili się, że uregulują, ,,gdy
sprawy wrócą do normalności''. Martin był z~grupą, która odwiedzała
miasteczko po początkowym zwiadzie i~zmuszał się, żeby się nie roześmiać
na ,,powrót do normalność''. Tablety sklepikarzy miały listy rzeczy,
które ludzie w~miasteczku potrzebowali: leki, materiały konstrukcyjne,
części silnika, kontrolery i~inne części do paneli słonecznych. Tablety
wypisywały dobra handlowe wszystkich ludzi w~mieście: wszystko od
produktów z~ogrodów przez zamrożone mięso, ubrania do kobiecych
produktów sanitarnych.

Z dyskusji ze sklepikarzami, Martin dowiedział się, że były inne małe
miasteczka jak to, nawet większe miejsca, gdzie używali energii
słonecznej, lokalnych sieci, baz danych na tabletach, żeby śledzić, co
kto ma, kto co potrzebuje i~co kto zabrał. Miasta trochę handlowały i~zapisywały, co się stało, kogo poznali i~na co uważać.

-- Co z~ochroną? -- spytał Martin.

Sklepikarz, Pakistańczyk lub Hindus, w~średnim wieku, ubrany w~czystą
białą koszulę zapiętą pod szyję, i~spodnie, wzruszył ramionami. 

-- Jaką
ochronę?

-- Wiesz. Gangi, milicje, tego rodzaju rzeczy. -- Pomyślał za słowem,
którego szukał. -- Maruderzy.

Sklepikarz uśmiechnął się. 

-- Nie mamy tego rodzaju rzeczy. Dzięki bogu!
-- Uśmiechnął się do Martina. Martin powstrzymał zniesmaczone pokręcenie
głową. Za jakiś miesiąc lub dwa, ten facet będzie martwy, wszystko, co
posiada, będzie posiadane przez kogoś silniejszego i~cholernie mniej
naiwnego. Martin sam prawie chciał strzelić faceta. Lepiej, żeby dobry
facet jak Martin dostał wszystkie towary tego faceta niż jakiś
\textit{maruder}.

-- Dobra, jeżeli to stanie się problemem, możemy być w~stanie pomóc -- powiedział Martin. -- Sprawdzimy co tydzień czy dwa, daj nam znać, jeżeli
ktoś będzie sprawiał problemy. -- W~ten sposób zaczynasz odbudowę,
pomyślał Martin: silni chronią słabych, słabi oddają silnym hołd i~szacunek.

Sklepikarz spojrzał się dziwnie na Martina i~powiedział, że tak zrobi,
potem przehandlował trochę suchego mango za nieco mleka w~proszku
Martina. W~miasteczku były dzieci i~piły więcej mleka niż lokalne krowy
i kozy dawały.

Radiostacja zaczęła nadawać, rzekomo w~Phoenix, dla wszystkich młodych
ludzi, ze specjalnymi gościami, którzy przyszli porozmawiać o~opuszczonych domach, które ,,uwolnili''. Niektóre były w~rejonach, gdzie
mieszkańcy Fortu Zagłada posiadali własność, miejsca, gdzie były bramy,
strażnicy i~systemy bezpieczeństwa. Wszyscy zjeżyli się na myśl, że ich
domy mogły zostać najechane. Tak wiele strasznych rzeczy zdarzyło się,
od kiedy wyjechali do Fortu Zagłada, ale to było to, co sprawiało, że
Wydarzenie było realne: jeżeli prawa własności ludzi takich jak oni
wymykały się, to wszystko było do zagarnięcia.

DJ-e grali nagraną muzykę i~nadawali zespoły na żywo, wymieniali
informacjami na temat położenia klinik, szpitali, garkuchnie i~przedszkola. To było surrealistyczne. Pokazywali obraz radykalnie
wyludnionego Phoenix, pełnego pustych domów i~zamkniętych biznesów,
zimne popioły z~dawnej przemocy. Phoenix, zaludniony tylko przez ludzi,
którzy nie mogli wyjechać.

Zbliżało się lato. Jeden z~DJ-ów zrobił wywiad z~grupą roboczą awaryjnej
klimatyzacji, sześciu gorliwych hakerów HVAC, którzy rozmawiali o~sposobach wybudowania schronów w~sąsiedztwie, gdzie mógłbyś uciec przed
gorącem lata, które miało nastąpić za kilka miesięcy, plany dla
wentylatorów chłodzących, które działały na tanich panelach słonecznych,
z listą części i~magazynów, gdzie można je było dostać lub oddać części.

Przez handel i~racjonowanie, mieszkańcy Fortu Zagłady mogli żyć we
względnym komforcie, ale nie oznaczało to, że życie było proste.
Niektóre zapasy się wyczerpywały, masło orzechowe, świeże owoce, płatki
owsiane, a~nuda była jak żywy organizm, gryząc mocno od momentu wstania
z łóżka do momentu pójścia spać. Z~tego całego żartowania o~byciu razem
w izolatkach, było w~tym ziarno prawdy, przejawiające się w~zgryźliwych
kłótniach i~długich dąsach, fakcjach oraz epickich sesjach obżerania i~picia. Gene spisał zapasy medyczne i~odkrył, że większość opioidalnych
środków przeciwbólowych zniknęła, nastąpiło polowanie na czarownice, kto
miał odjazdy, ale nikt nie został przekonująco obwiniony za
splądrowanie.

I było jeszcze Phoenix, transmisje radiowe, wiadomości z~ich dawnych
sąsiedztw, był jeszcze James i~Ray, którzy jednocześnie byli ich
najlepszymi zwiadowcami i~największymi mąciwodami, żądając prawa do
sprawdzenia na żywo, zobaczenia, jak sobie radzą. Giorgia zdecydowała
się pojechać z~nimi. Mieszkańcy Fortu Zagłada mieli dość Końca Świata i~byli gotowi do powrotu do masaży, steków, koktajl barów, gier w~squasha,
ciężkiej pracy, wielkich zysków, kłótni na social media, nierozważnego
seksu z~interesującymi obcymi, wszystkich wygód współczesności. Kiedy
opuszczali Phoenix, było to kwitnące miasto, wypełnione TaskRabbit i~7-Eleven, Uberami i~elitarnymi krawcami na miarę czy butikami mody. Na
pewno jakieś pozostałości tego zostały.

Zatem zezwolił na misję, słuchali DJ-ów cały czas, jakby to była sztuka
radiowa o~znacznie ciekawszej apokalipsie niż ta, którą właśnie
przeżywali.

Warren zażartował: ,,Armageddonowi znudziło się czekanie'' i~powtarzałby
to cały dzień. Na początku jęczeli, potem mówili razem z~nim, potem go
wyciszyli.

Ale gdy dni po wyjeździe Jamesa, Raya, Giorgii do Phoenix zamieniły się
w tydzień, i~gdy DJ-e w~radio informowali o~fali choroby przechodzącej
przez miasto, myśleli, że jest to legioneloza\footnote{zob.~\url{https://pl.wikipedia.org/wiki/Choroba\_legionist\%C3\%B3w}
-- przyp.tłum.} lub listerioza\footnote{ zob.~\url{https://pl.wikipedia.org/wiki/Listerioza} -- przyp.tłum.},
radzili jak leczyć ludzi bez antybiotyków, o~nawadnianiu i~spokoju,
okazało się, że sami powtarzają ten sam głupi żart. 

-- Armageddon ma dość
tego gówna -- mówili. 

-- Armageddon też jest zmęczony. -- Nikt się nie
roześmiał, nawet Warren.

Kiedy Trójka w~końcu powróciła, dziesięć dni później, wrócili z~łupem:
puszkowane owoce, czekolada, opiaty, stos komiksów i~starego porno z~jakiegoś antykwariatu w~miasteczku. Wymienili klejnoty, trochę amunicji
-- kalibru, którego Martin nie chciał trzymać w~Forcie, zapominając
pozbyć się zapasów -- za wielką torbę słabego zioła.

Zostali powitani jak powracający bohaterowie i~uraczyli ich heroicznymi
opowieściami przygód i~szybkich ucieczek, opisując Phoenix, które
zmieniło się kwartał po kwartale: najpierw długa, spalona blizna ciągle
pachnąca dymem, potem doskonale utrzymane rejony z~rzędami przenośnych
WC co kilka domów, zbiornikami na wodę podgrzewaną słońcem na dachu,
ludzi machających z~dachów i~dzieci bawiących się w~zastępczych
szkółkach w~cieniu na trawniku budynku apartamentowca. Próbowali
odwiedzić stary kompleks Martina, gdzie niektórzy z~nich (w tym Ray)
mieli domy, ale ktoś wystrzelił ostrzegawczo nad ich głowami z~budki
strażniczej. Próbowali krzyczeć do niewidzialnych obrońców, ale kiedy
kolejna kula świsnęła nad głowami (tym razem niżej), odebrali wiadomość
i odjechali.

Ciągle, wszyscy byli podnieceni wiadomościami, obrazem miasta wstającego
z kolan, rejonami normalności. A Martin był zadowolony, że kamienie
szlachetne ciągle były akceptowalnym towarem wymiennym. Czekał na dzień,
kiedy mógłby wyjąć pendrive z~sejfu w~podłodze i~zacząć obracać swoimi
aktywami w~kryptowalucie.

Giorgia i~Martin tej nocy mieli drinka w~jego pokoju, po którym wzięła
gorący prysznic i~whisky. We wczesnych dniach mieli Fortu relację
tak/nie i~najwyraźniej teraz była znowu na tak. Martin zasnął wtulony w~nią, twarz zatopiona w~jej pachnących szamponem włosach, zmęczony, lepki
i głęboko nasycony.

Przywództwo nie było łatwe, ale miało swoje zalety.

Następnego dnia Giorgia zachorowała. Martin obudził się na odgłos jej
wymiotów w~swoim pokoju, jęczącą pomiędzy salwami. Z~niepokojem zajrzał
do łazienki, odkrywając ją nagą, pochyloną nad toaletą, tyły ud pokryte
wodnymi stolcami, które zbierały się przy jej kolanach. Zapach był
niewiarygodny. Zasłonił usta i~nos, wycofał się, poszedł i~przyniósł
wiadro, napełnił je słabym roztworem wybielacza. Założył rękawiczki i~maskę, otworzył świeże opakowanie kuchennych gąbek, umył dookoła Giorgi,
związał jej włosy gumką, nawet zmył gąbką tyły jej ud. Opróżnił wiadro
do toalety pomiędzy jej spazmami, rozpuścił sole mineralne w~destylowanej wodzie dla niej. Obrócił ją, posadził na toalecie, z~pustym, pachnącym wybielaczem wiadrem na to, co pozostało śluzowatego w~jej jelitach. Wyglądała źle i~rosła jej temperatura.

Martin nie chciał alarmować innych, ale poszedł i~sprawdził Raya i~Jamesa czy mają jakiekolwiek objawy. Wszyscy się zgodzili, nawet Giorgia
pomiędzy skurczami, że powinni zamienić pokój Martina w~izolatkę i~przynieśli jakieś łóżka. Gene spakował kilka pudełek zapasów i~przyniósł
mini-lodówkę, jakieś antybiotyki o~szerokim spektrum, w~tym kilka dawek
w płynie, razem ze strzykawką, żeby mogli podać je Giorgii, która nie
była w~stanie przyjąć niczego ustnie lub nawet jako czopek. Domieszali
trochę dimenhydrynatu\footnote{ oryg. dramamine -- lek zapobiegający nudnościom
i chorobie lokomocyjnej,
zob.~\url{https://pl.wikipedia.org/wiki/Dimenhydrynat} -- przyp.tłum.}, który ją powalił. Martin poinstruował innych, żeby na
zmianę budzili ją co dwadzieścia minut na łyk wody, oraz zmieniali jej
nocnik, gdy trzeba.

Następnego ranka zachorowali James i~Ray. Martin pozwolił im się zająć
sobą, przynosząc więcej wiader i~toalet chemicznych, żeby nie musieli
rywalizować o~łazienkę. Martin spędził noc w~jednym z~pustych podwójnych
pokoi -- Gene pomógł mu sprzątnąć nadmiarowe zapasy -- i~zaczął nosić
jeden z~ostatnich kombinezonów, gdy odwiedzał kwarantannę, sprawdzając
co godzinę.

Strach zawisł nad Fortem, wszyscy rozmawiali szeptem, inni mieszkańcy
subtelnie unikali Martina, oddalając się, gdy wchodził do pokoju.
Spędzał popołudnia na zewnątrz, chodząc dookoła granicy, sprawdzając
kable w~panelu słonecznym i~czujniki, naprawiając kruszącą się zaprawę
przy antenie krótkofalowej. Normalnie, lubił samotność, ale tego dnia
jego myśli biegły dookoła w~pętli, która krążyły od strachu czy sam
zachoruje, to nieracjonalnej niechęci do Jamesa, Raya i~Giorgia za
sprowadzenie patogenów do ich domu, to jeszcze bardziej nieracjonalnego
resentymentu wobec pozostałych mieszkańców Fortu za ich oczywiste
podejrzenia, że też zachoruje.

To była prawie ulga, kiedy poczuł skurcz w~jelitach. Nie dosłownie
oczywiście. Ból zgiął go jak kopnięcie w~żołądek, wiedział natychmiast,
że nie dojdzie do Fortu, więc ściągnął spodnie do kostek na czas, żeby
uratować je przed potokiem, który wyleciał z~dupy. Chwilę później,
zaczął wymiotować.

Kiedy pierwsza fala minęła, umył się chusteczkami z~paczki, wypił trochę
elektrolitów, potem włożył chusteczki to torebki i~wyciągnął trochę
Cipro\footnote{ Cyprofloksacyna -- organiczny związek chemiczny o~działaniu
bakteriobójczym,
zob.~\url{https://pl.wikipedia.org/wiki/Cyprofloksacyna} -- przyp.tłum.} z~prywatnych zapasów, które nosił ze sobą od momentu
zachorowania Giorgi.

Dołączył do pozostałych w~kwarantannie. James był teraz zbyt słaby, żeby
dojść do toalety, co znaczyło, że teraz były dwa nocniki do opróżniania.
Z drugiej strony, to również znaczyło, że nie było wojny o~dwie toalety.
Martin jasno postawił sprawę Rayowi, że łazienka była jego. Ray mógł
używać toalety chemicznej. Ray nie sprzeciwiał się. Był tak słaby i~gorączkował, ledwie mógł mówić.

Giorgia zmarła ósmego dnia, jej gorączka tak wysoka, że praktycznie
skwierczała. Martin domyślił się, że nie żyje, kiedy poszedł nakarmić ją
łyżką wody i~odkrył, że jej gorąca sucha skóra jest zimna i~lepka. Gene
zostawił worek na zwłoki na zewnątrz pokoju (Martin przeczytał kilka
przewodników online o~wyposażeniu i~worki na zwłoki były we wszystkich),
on i~Ray włożyli ciało Giorgii do worka, potem zmienili pościel.

Następnego dnia zmarł James. Martin nie miał sił, żeby wyjść z~łóżka,
pomóc przy worku, potem jednak się zebrał i~dwa ciała zostały
wyciągnięte przez zdrowych mieszkańców. Spalili ciała, razem z~torbami
pościeli i~ubraniami, które Martin i~Ray przekazali.

Martin zaczął się zbierać kolejnego dnia, a~potem, zadziwiająco, także
Ray. Leżeli wyczerpani na łóżkach, pijąc rosół i~elektrolity, potem na
zmianę wzięli prysznic, spakowali brudne ubrania, pościel i~przebrali
się w~czyste pidżamy, umyli zęby, ogolili. Zgodzili się, że najlepiej
będzie spędzić kolejny dzień w~kwarantannie, na wszelki wypadek, a~potem
obaj stwierdzili, jebać to, i~wrócili do reszty, słabi i~drżący. Martin
stracił pięć kilogramów, jego skóra była szara i~luźna na twarzy,a ale
wiedział, że przeżyje. Ocalał.

Następni byli Saleha, Lloyd i~Izzy. Żadne z~nich nie przeżyło.

Pochowali ich na małym zaimprowizowanym cmentarzyku na stoku wzgórza,
które nie było widoczne z~drogi. Nie oznaczyli grobów, ale zanotowali
koordynaty GPS, także mogliby postawić nagrobki później, kiedy
przywrócona zostałaby normalność.

Nie było jasne, jak infekcja wróciła do Fortu Zagłada, lub nawet, czy to
była ta sama infekcja. Szambo było pełne, musieli wykorzystywać ubikacje
chemiczne i~opróżniać je do jamy, którą wykonali do tego celu, cuchnącą
i pełną much.

Martin przemówił nad grobami, przypominając sobie najlepsze słowa.

W nadawaniu programów z~Phoenix była przerwa, trwająca długi miesiąc,
kiedy infekcja paliła miasto, ale potem transmisje wróciły, z~nowinami o~dostawach antybiotyków, które były produkowane w~odzyskanej fabryce w~Chandler. Oryginalni DJ-e nie byli na antenie, ale odbył się dla nich
program pamiątkowy.

Do Świąt Bożego Narodzenia, sprawy wpadły w~rutynę. Przynajmniej nie
było tyle czasu, żeby się nudzić. Pomiędzy polowaniami, ostrożnymi
wypadami handlowymi i~kopaniem nowych latryn, mieszkańcy Fortu Zagłada
mieli pełne ręce. Byłoby znacznie gorzej, gdyby farmy Elfridy nie miały
takiego szczęśliwego roku, hojne zbiory, które przehandlowali z~Fortem
Zagłada za więcej klejnotów, a~kiedy te się skończyły, poniżające i~wyczerpujące roboty zbierania w~polu. Martin nakreślił przy tym linię i~zniechęcał innych w~Forcie od tego, ale nie wszyscy go posłuchali.
Zbieracze dostawali zapłatę w~owocach, wszyscy inni tylko dostawali
warzywa korzeniowe, jedyną rzecz, którą farmerzy godzili się wymieniać.

Sara zbierała przez całe dwa tygodnie, wracając szczupła, brązowa i~umięśniona, skarżąc się na fakt, że Fort Zagłada nie dawał im żadnych
możliwości agrykulturalnych. Martin próbował ją powstrzymać, mówiąc o~wewnętrznie podatnych okolicznościach rolnictwa, a~ona odpowiedziała
zgryźliwie, przypominając im o~podatności kurwa śmierci głodowej.

Zatem bez nudy w~Forcie Zagłada, ale ciągle: mnóstwo napięć. Martin
zdecydował, że powinni zrobić imprezę. Święto Dziękczynienia nadeszło i~minęło nieobchodzone, ale ciągle mogli świętować nadzwyczajne
okoliczności przeżycia Wydarzenia poprzez przygotowanie drzewka, ucztę
czy wymianę prezencików.

Martin wypakował plastikowe drzewko z~zapasów Fortu, myśląc o~tym, ile
zabawy byłoby w~zaskoczeniu jego gości w~ich pierwszą Gwiazdkę razem.
Choinka przyszła w~paczce z~błyskotkami i~ozdobami. Obudził się
wcześniej dziewiętnastego grudnia, żeby złożyć drzewko w~apartamencie
gościnnym, wykorzystując okazję do sprzątnięcia zasp śmieci i~zamiecenia
wniesionego brudu. Kiedy skończył, odsunął się i~obejrzał pokój,
oczekując, że będzie wyglądał radośnie i~świątecznie. Jednak wyglądał
smutno i~desperacko, plastikowe drzewko i~mrugające łańcuchy LED małe i~sztuczne, pokój ciągle brudny, poduszki na sofie poplamione i~wyświechtane.

Po raz milionowy marzył, żeby ten jebany kryzys się wreszcie skończył.
\textit{Armageddon ma dość tego gówna}. Co przypomniało mu, że Warren nie
żyje, w~worku w~nieoznaczonym grobie na zboczu.

Sara obudziła się pierwsza po ustawieniu drzewka przez Martina,
przewróciła oczami na to i~powiedziała, że została zaproszona na obiad
świąteczny z~innymi ,,robotnikami'', którzy zbierali tej jesieni. Jednak
pomogła przygotować paski z~imionami na prezenty i~przesunęła nieco
ozdoby dla lepszego efektu. Martin świadomie pozostawił ozdoby zbite
razem, żeby dać innym coś do poprawienia, co było pewną drogą do
stworzenia poczucia własności nad projektem.

Kuchnia miała ciągle dobry zapas przypraw, Martin świadomie schował trzy
skrzynki wina do gotowania, poprosił Sarę, żeby znalazła rozsądne
przepisy na grzane wino z~posiadanych materiałów i~wszyscy temu
kibicowali, a~potem Fort Zagłada wypełnił się smakowitymi zapachami,
dniem picia i~chwiejnym dobrym nastrojem.

To był obiad. Przy kolacji jednak mieli kaca od taniego wina, humor im
się zepsuł, przypominając im o~członkach rodziny, które zostawili przy
Wydarzeniu, zmarłych z~Fortu, chaosu na zewnątrz pancernych drzwi.
Kolacja była ponura i~szybka, nikt potem nie chciał spędzić czasu w~apartamencie gościnnym.

Gene wymyślił plan, żeby otworzyć trochę racji i~zrobić prawdziwą ucztę
z kolacji świątecznej, dodając świeże składniki i~soloną dziczyznę,
ustawiając półmiski, używając prawdziwych lnianych obrusów i~zapalając
świece. Martin pobłogosławił planowi i~wrócił do pracy nad swoim
prezentem: skórzane pudełko na cygara, z~dwoma cygarami z~jego
zmniejszających się zapasów Cohibas dla Setha, jedna dla niego i~jedna
dla przyjaciela. Palenie cygara na grani, gdy słońce zachodziło nad
pustynią, było jedną z~prawdziwych przyjemności w~trakcie Końca Świata.

W końcu, dwudziestego czwartego, jeden dzień przed imprezą, nastrój się
zmienił. Idea ,,ducha świąt'' była głupia i~sentymentalna, ale nie
czyniło to jej mniej rzeczywistej. Mieli to! Wszyscy chowali się w~prywatnych przestrzeniach, robiąc prezenty, a~kiedy tego nie robili,
robili papierowe płatki śniegu, lub wycinali ubrania ze szmat,
przygotowując stroje na imprezę. Ktoś zawiesił papierową jemiołę w~drzwiach pomiędzy kuchnią a~apartamentem gościnnym, bawili się w~zasadzki na innych i~całowanie w~policzek. Gra były znacząco ożywiona
przez teraz wieczny dzbanek taniego wina z~przyprawami.

Sara zrobiła Ciastka Depresji jej praprababki, które nie zawierały mąki
i używały jaj w~proszku i~mleka w~proszku, ciągle pachniały i~wyglądały
wspaniale w~piekarniku. Udekorowali je lukrem i~wyłożyli na
demonstracyjnej tacy w~kuchni z~napisem ,,NIE JEŚĆ, DOTYCZY CIEBIE'',
potem przystawali, żeby podziwiać i~podpuszczać się, żeby naruszyć
zakaz. To była największa zabawa od tygodni i~najlepszy nastrój, jaki
mieli od miesięcy, Martin ciągle zaglądał do nich, upewniając się, że
plan działa, że jest takim przywódcą, jak mu się wydawało.

Dzień po Gwiazdce, rozwalali się po apartamencie gościnnym w~pidżamach,
ciągle pełni od uczty z~poprzedniego dnia. Seth wyszedł na zewnątrz,
żeby cieszyć się jednym z~cygar i~wrócił, zarumienione policzki,
promieniując zimnem i~pachnąc łagodnym, kubańskim tytoniem.

Mieli kaca, oczywiście, ostre bóle głowy, które sprawiały, że mówili po
cichu. Gdy się pocił i~lekko go mdliło, Martin żałował, że wypił tyle
grzanego wina dzień wcześniej, ale pocieszał się, przypominając sobie,
że przepili cały zapas taniego wina i~nie byliby teraz kuszeni powtórką.

Seth był pierwszy, który zaczął wymiotować. Wymioty pojawiły się nagle,
zbyt szybko, żeby dotarł do toalety, zostawił ślady na podłodze całą
drogę do łazienki, okropnie pachnące i~wszechobecne.

Gdy Martin wstał, żeby przynieść rękawiczki i~chusteczki sanitarne,
zakręciło mu się w~głowie i~wtedy zrozumiał, że nie ma bólu głowy, ma
gorączkę. Delikatnie posprzątał, potem wyjął termometr do ucha. Pokazał
39,4 stopnia Celsjusza.

Wszyscy byli chorzy. Może to było to, co prawie zabiło ostatnim razem
Martina. Może to było coś nowego. W~Forcie Zagłada nie było
antybiotyków.

Do nowego roku, ośmioro z~pozostałej dwunastki zmarło. Reszta była zbyt
słaba, żeby ich pochować, choć udało im się wyciągnąć ciała przez
frontowe drzwi.

Fort Zagłada stał się kostnicą. Szambo znowu wybiło, więc nawet gdyby
mieszkańcy zebrali siły, żeby dostać się do toalety, nie spuściliby
wody. Toalety chemiczne się przepełniły. Wypełnili wiadra, przewracali
wiadra, gdy zataczali się w~gorączce, przynosząc wodę, elektrolity i~tabletki na ból głowy.

Gorączka Martina skoczyła do czterdziestu dwóch stopni i~potem spał przez długi
czas, budząc się tylko, gdy wnętrzności bolały i~próbował wycisnąć
cokolwiek, co w~nich zostało.

Kiedy gorączka spadła czwartego stycznia, odkrył, że był ostatnią żywą
osobą w~Forcie Zagłada.

Martin pielęgnował siebie, wracając do zdrowia, najpierw czołgając się,
potem zataczając, pijąc wysokokaloryczne proszki zastępujące dania i~kostki bulionowe w~ciepłej wodzie. Wyciągnął mniejsze ciała na zewnątrz,
odpoczywał cały dzień i~noc, potem większe ciała. Posprzątał.

Dni mijały i~jego gorączka wróciła, cofnęła się. Elektryczność się
skończyła i~uruchomił zapasowy generator. Sprawdzał kamery, myślał o~odejściu, myślał, gdzie mógłby pojechać. Był zbyt słaby, żeby iść daleko
i byłby bezbronny.

Teraz gdy uruchomił generator, musiał oszczędzać zapasy propanu, zatem
przestał gotować wodę przed piciem. Wkraplał jodynę i~wysysał ją przez
LifeStraw\footnote{ słomka zawierająca filtry umożliwiająca picie nieodkażonej
wody, zob.~\url{https://pl.wikipedia.org/wiki/LifeStraw} -- przyp.tłum.}, mając nadzieję, że to wystarczy. Przepełnione szambo
powodowało, że nie był pewien picia wody ze studni Fortu bez dodatkowego
jej potraktowania. Nie wiedział, dlaczego choroba wybuchła w~Forcie
Zagłada, ale to był główny podejrzany.

Wtedy, pewnego dnia, gorączka wróciła i~zaczęła go palić, odmawiając
zmiany. Jego stawy bolały, jelita ścisnęły i~wykręciły. Majaczył,
dryfował w~trzeźwość, słaby i~samotny. I~przerażony. Tak, tak, tak
przerażony.

Czujniki zbliżania wytrąciły go z~gorączkowego półsnu. Drżąc, podczołgał
się do monitora, zobaczył samochody otaczające wejście do Fortu.
Zobaczył otwierające się drzwi pasażera w~pierwszym samochodzie. Inne
drzwi. Cztery postaci. Trzech mężczyzn. Kobieta.

Ciotka Alberta.

Strach przeważył nad drżeniem gorączki. Stało jeszcze gorzej, gdy
uświadomił sobie, że nie pamiętał ryglowania Fortu ostatnim razem, gdy
wyszedł na zewnątrz wyrzucić śmieci.

Na ekranie, ciotka Alberta wskazała palcem na drzwi. Coś w~sposobie
pokazywania powiedziało mu, że ich nie zamknął. Chwilę potem, usłyszał
głosy, odległe i~zbliżające. Z~początku nie mógł zrozumieć słów, ale
potem zrozumiał, że to dlatego, że mówili po hiszpańsku.

Graciela nauczyła się dużo, od kiedy wróciła do Phoenix, tak chora, że
myślała, że umrze. Nauczyła się pierwszej pomocy, podstaw higieny i~rodzaju higieny żywności wykorzystywanej w~pandemii. Nauczyła się słowa
,,pandemia''.

Nauczyła się analizować antybiotyki pochodzące z~małych linii
produkcyjnych, które zbudowali jako część procesu kontroli jakości.
Nauczyła się przyjmować wskazówki od techników w~wielkim zakładzie
oczyszczania wody, który uruchomili ponownie. Nauczyła się czytać mapy
topograficzne, nauczyła się prowadzić zespoły, które szukały chorych,
żeby im pomóc.

Nauczyła się, że lubi te rzeczy. Nauczyła się, że odbudowanie, troska i~naprawa trzymały z~dala nagłe ataki lęku i~napływające łzy. Nauczyła
się, że choć nic nie sprowadzi z~powrotem jej synów, jej męża, jej
siostrzeńca Alberta, to wspomnienie ludzi, którym pielęgnowała znad
krawędzi śmierci, złagodzi jej smutek, który czuła, kiedy myślała o~własnej śmierci. Zachowywała rachubę: jeden uratowany na jednego,
którego straciła, potem dwóch, potem pięciu żywych na każdego zmarłego.
Przestała liczyć. Teraz razem z~nią żyło dwoje dzieci, nikt nie
wiedział, gdzie są ich rodzice. Lanae miała sześć lat, a~Darnell
dziewięć. Kiedy się uśmiechali, to jakby jej zmarli uśmiechali się do
niej.

Mężczyzna, który zostawił ją na śmierć, zostawił drzwi otwarte.
Wiedziała, że ma broń. Wszyscy tutaj mieli broń. Była całkiem pewna, że
jeden z~jej kolegów też nosił, chociaż żaden z~nich nie powinien nosić
broni. Myśląc o~sposobie, w~jaki z~nią rozmawiał, jak rozmawiała z~Albertem, jak strzelił do nich, skrycie cieszyła się, że też mieli broń.

-- Proszę pana? Halo? 

Wąchała te zapachy wcześniej, ale wcale nie było
łatwiej. Gówno. Rzygi. Choroba. Psujące się śmieci. Trupy. Jeden z~mężczyzn wrócił od tylnego wejścia z~wymiocinami na przodzie koszuli.
Tak wielu zmarłych tam. Upewniła się, że jej maska była na miejscu i~posmarowała odrobinę VapoRub\footnote{ maść z~mentolem,
zob.~\url{https://pl.wikipedia.org/wiki/Mentol} -- przyp.tłum.} pod nosem.

-- Proszę pana? -- Zrobiła ostrożny krok do przedsionka ponad stosami
worków ze śmieciami, niektóre rozdarte przez zwierzęta. Cieszyła się, że
ciała nie były widoczne. To było naturalne, co zwierzęta robiły
szczątkom ludzkim, ale było trudne do zniesienia.

Tak długo zajęło im odejście. Martin gryzł własny pasek, kiedy łapały go
skurcze, nie wydając więcej niż jęk. Jego bezpieczny pokój był nieco
większy niż trumna, tylko przestrzeń na wczołganie pod garażem, pełna
starych oparów benzyny. Miał jednak naładowany telefon i~lokalne Wifi,
mógł obserwować, jak przeszukiwali Fort, obserwował zewnątrz, gdy
budowali stos i~palili zmarłych. To zabrało dużo czasu. Jęczał w~skórę
paska.

Jednak w~końcu odjechali.

Do tego czasu był zbyt słaby, żeby poruszyć ciężką płytę nad głową.
Pchał tak mocno, jak mógł, tak silnie, że stracił kontrolę nad
zwieraczem i~poczuł gorące gówno spływające po nodze. Pchnął mocniej,
panika narastała. Gorączka zgrzytała w~stawach.

W końcu zamknął oczy. Prześpi się i~znowu spróbuje. Odzyska siły po
drzemce.


\chapter*{Podziękowania}

Mój agent, Russell Galen, zrobił bardzo dużo, żeby ta książka wyszła,
jak zawsze ponad i~poza. To znaczy dużo, Russ, dziękuję. 

Tor Books wspierało mnie na tyle różnych sposobów, a~osobiste zachęty od mojego
redaktora Patricka Nielsen Haydena i~wydawcy Fritza Foya były
szczególnie mile widziane, a~każde podziękowania dla Tor byłyby niedbałe
bez szczerych podziękowań dla Toma Doherty za jego długą służbę w~polu i~wsparcie dla mnie, osobiście. 

Dziękuję również zespołowi produkcji Tor,
a szczególnie ludziom z~wydziału grafiki za wybranie mnie jako jednego
ze specjalnych, szczęśliwych zwycięzców loterii, gdzie nagrodą są
okładki Will Staehle. Motyla noga, te OKŁADKI. 

Dziękuję Macmilla Audio
(Robert Allen) i~Google Play (Chris Palma), którzy byli niesamowicie
pomocni w~mojej podróży do doskonałości audiobooków, oraz Skyboat Media,
z którymi praca jest prawdziwą przyjemnością, oraz moim czytelnikom.
Dziękuję równiej Jigsaw/First Look, Head of Zeus i~ludziom w~Heyne.

Napisanie książki zajmuje wielu oddanych i~utalentowanych ludzi, a~to
zawsze akt wiary. Dziękuję za inspirację Matt Taibbi, Electronic
Frontier Foundation, Alex Steffen i~przejęciu, oraz wszystkich, którzy
walczą o~sprawiedliwość: \#blacklivesmatter, Alexandria Ocasio-Cortez,
Erica Garner, Bernie Sanders i~milionom na ulicach. 

To nie jest rodzaj walki, którą wygrasz, to rodzaj walki, w~której walczysz.

\chapter*{O Autorze}

Cory Doctorow jest współredaktorem \textit{Boing Boing} i~stałym
współpracownikiem \textit{Guardian}, \textit{Locus} i~wielu innych
czasopism. Jego powieści \textit{Mały Brat} i~\textit{Homeland} były na
liście bestselerów \textit{New York Times}. Mieszka z~rodziną w~Los
Angeles.


\chapter*{Posłowie od tłumacza}

Uważne czytelniczki serii \textit{Czarna Flaga} mogą sobie postawić pytanie, dlaczego po powieści feministycznej Joanny Russ, lewicowo zaangażowanych powieściach Kena MacLeoda, seria przedstawia dwie powieści liberała Cory Doctorowa\footnote{więcej o autorze, zob.~\url{https://pl.wikipedia.org/wiki/Cory_Doctorow}}.

Twórczość Cory Doctorowa jest warta uwzględnienia w~tej serii z~co najmniej dwóch powodów. Po pierwsze, Cory Doctorow opisuje niedaleką przyszłość, odległą o kilkadziesiąt lat. Oznacza to, że nie znajdziemy tutaj ani lotów ponadświetlnych ani międzygalaktycznych imperiów czy wojen pomiędzy cywilizacjami. 

Natomiast w jego powieściach odnajdziemy ekstrapolacje współczesnych problemów m.in. walki z~korporacjami nad możliwościami korzystania z dóbr kultury, rozwoju inwigilacji państwowej, problemów z molochami mediów społecznościowych, wykorzystaniem drukarek 3D, wpływu mediów społecznych czy kultury gier na społeczeństwa.

Po drugie, mimo oczywistego liberalizmu Doctorowa, Doctorow często opisuje bohaterki i bohaterów, którzy nie czekają na ratunek Państwa/Boga/Systemu, ale sami się organizują, często niehierarchiczne, i działają bezpośrednie bez względu na to, czy chodzi o~korzystanie z~kultury czy walce o prawa pracownicze czy zachowanie zwyczajnej prywatności.

Słowo o~tytule, tłumaczę tytuł ,,Radicalized'' jako ,,Radykalne'', ponieważ sądzę, że wspólnym wątkiem czterech minipowieści C.~Doctorowa są byty radykalne: osoby, takie jak ,,Superman'', działania jak czyny podejmowane przez protagonistów, przykładowo łamanie zabezpieczeń ograniczających wolność oraz idee, przykładowo kooperacji osób dotkniętych ,,końcem świata'' czy idee akcji bezpośredniej.

~

Jestem przekonany, że uważna czytelniczka, czy uważny czytelnik odnajdzie wiele błędów w~tym tłumaczeniu. Ponoszę za to całkowitą odpowiedzialność.\\

\href{mailto:theskymyladythesky@zoho.eu}{Jacek Hummel}\\

Warszawa, sierpień -- październik 2021 roku.


\chapter*{Seria ,,Czarna Flaga''}

\begin{center}
\begin{large}
W serii \textit{Czarna Flaga} dotychczas opublikowano online:
\end{large} 
\end{center}


\begin{enumerate}
\item \href{https://archive.org/details/joanna-russ-mezczyzna-rodzaju-zenskiego/Joanna_Russ_M\%C4\%99\%C5\%BCczyzna_rodzaju_\%C5\%BCe\%C5\%84skiego}{Mężczyzna rodzaju żeńskiego}, Joanna Russ
\item Engines of Light t. 1 -- \href{https://archive.org/details/ken-macleod-wieza-kosmonauty}{Wieża Kosmonauty}, Ken MacLeod
\item Jesienna Rewolucja t. 1 -- \href{https://archive.org/details/ken-mac-leod-jesienna-rewolucja-gwiezdna-frakcja}{Gwiezdna Frakcja}, Ken MacLeod
\item Jesienna Rewolucja t. 2 -- \href{https://archive.org/details/ken-mac-leod-jesienna-rewolucja-kamienny-kanal}{Kamienny Kanał}, Ken MacLeod
\item Jesienna Rewolucja t. 4 --  \href{https://archive.org/details/ken-mac-leod-jesienna-rewolucja-droga-do-gwiazd}{Droga do Gwiazd}, Ken MacLeod
\item Jesienna Rewolucja t. 3 -- \href{https://archive.org/details/ken-mac-leod-jesienna-rewolucja-oddzial-cassini}{Oddział Cassini}, Ken MacLeod
\item Radykalne, Cory Doctorow
\end{enumerate}

\begin{center}

\begin{large}W planach:\end{large}\end{center}

\begin{enumerate}
\item Odchodząc (Walkaway), Cory Doctorow
\end{enumerate}

\newpage

Projekt serii jest przygotowywany dzięki Wolnemu Oprogramowaniu. Zestaw narzędzi składa się z:
\begin{itemize}
\item \href{https://ubuntu.com/}{Ubuntu 20.04 Ogniskowa Fossa} -- system operacyjny
\item \href{https://omegat.org/}{OmegaT} -- narzędzie wspomagające tłumaczenie (CAT)
\item \href{https://github.com/soimort/translate-shell}{translate-shell} -- narzędzie do tłumaczenia w~\href{https://translate.google.pl}{Google Translate} przez terminal 
\item \href{https://glosbe.com/en/pl}{Glosbe} -- największy słownik online
\item \href{https://www.wikipedia.org/}{Wikipedia} -- podstawowe źródło tłumaczeń pojęć technicznych, politycznych i~ekonomicznych czy not biograficznych
\item \href{https://www.libreoffice.org/}{LibreOffice} -- przetwarzanie dokumentów 
\item \href{http://pandoc.org}{pandoc} -- uniwersalny konwerter dokumentów 
\item \href{https://www.latex-project.org/}{LaTeX} -- redakcja, skład i~łamanie dokumentu
\item \href{https://sigil-ebook.com/}{sigil} -- przetwarzanie plików ebook
\item \href{https://calibre-ebook.com/}{calibre} -- konwersja plików ebook
\end{itemize}



%EPUB
%\newpage
%\printendnotes
%EPUB

\tableofcontents{}
\end{document}
