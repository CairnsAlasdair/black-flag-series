\documentclass[oneside,polish,11pt,sfheadings]{mwbk}
%polonizacja
\usepackage[T1]{fontenc}
\usepackage[polish]{babel}
\usepackage[utf8]{inputenc}
\usepackage{polski} 
\frenchspacing 
\usepackage{indentfirst} 
%koniec polonizacja
%grafika
\usepackage{graphicx}
%pakiet czcionki
\usepackage{times}
\usepackage[a5paper]{geometry} %wielkość papieru (148x210-book w~PL)
%gwiazdki
\newcommand{\threeast}{\bigskip\par\centerline{*\,*\,*}\medskip\par}
\babelhyphenation{wszyst-ko}

%EPUB
\usepackage[hyperfootnotes=true]{hyperref} 
%move footnotes to endnotes
\usepackage{enotez}
\let\footnote=\endnote
\setenotez{
  list-name = Przypisy,
  backref = true
}

%pdf anonimize
%dla EPUB wykomentować
%\pdfsuppressptexinfo=-1 %Suppress PTEX.Fullbanner and info of imported PDFs

%pakiet odnośników i~pdf metadata
%\usepackage[unicode, pdftex]{hyperref}
%\hypersetup{pdfauthor={Ken MacLeod},
%            pdftitle={Jesienna Rewolucja 4 Droga do Gwiazd},
%            pdfsubject={Fall Revolution -- Sky Road},
%            pdfkeywords={tłum. Jacek Hummel, Creative Commmons, tłumaczenie CC BY 4.0, powieść, science fiction, Jesienna Rewolucja},
%            pdfcreator={pdfLaTeX}}
%dla EPUB koniec wykomentowania


\begin{document}

\title{Droga do Gwiazd}
\author{Ken Macleod}


%-----titlepage start
\DeclareRobustCommand{\cs}[1]{\texttt{\char`\\#1}}
\newlength{\tpheight}\setlength{\tpheight}{0.9\textheight}
\newlength{\txtheight}\setlength{\txtheight}{0.9\tpheight}
\newlength{\tpwidth}\setlength{\tpwidth}{0.9\textwidth}
\newlength{\txtwidth}\setlength{\txtwidth}{0.9\tpwidth}
\newlength{\drop}
\newcommand*{\titleSI}{\begingroup% Sagas
\drop = 0.13\txtheight
\centering
{\Huge \textsf{JESIENNA REWOLUCJA}}\\[1\baselineskip]
{\huge \textsf{FALL REVOLUTION}}\\[1\baselineskip]
{\LARGE  \textsf{TOM 4}}\\[4\baselineskip]
{\Huge \textsc{Droga do Gwiazd}}\\[1\baselineskip]
{\LARGE \textsc{The Sky Road}}\\[2\baselineskip]
{\huge \textsc{Ken MacLeod}}\\[4\baselineskip]
{\large Na podstawie wydania TOR, Nowy Jork, 2009 \\ przetłumaczył i~opracował:}\\
{\Large Jacek Hummel}\\[2\baselineskip]
{\normalsize \textit{Tłumaczenie jest dostępne na licencji\\
\href{https://creativecommons.org/licenses/by/4.0/deed.pl}{Creative Commons Uznanie autorstwa 4.0 Międzynarodowe}} \\ \par}
\includegraphics[scale=0.4]{CC.png}
\vfill
{\Large {Warszawa, 2021}}\\
%\vspace*{\drop}
\endgroup}
\titleSI
\thispagestyle{empty}
%-----titlepage end

\begin{figure}[p]
    \vspace*{-1cm}
    \makebox[\linewidth]{
        \includegraphics[width=1.1\linewidth]{SkyRoad.jpeg}
    }
\end{figure}
\thispagestyle{empty}


\newpage

\vspace*{2cm}


\textit{Dla Mic'a}

\chapter{Światło i Jarmark}

\textit{Zatem stało się tak, że Merrial odnalazła go na placu w~Carron
Town.}

Szła przez Jarmark w~wieczornym świetle północnego lata, szukając mnie.
Z setek osób wokół niej, tysięcy w~mieście i~tysięcy przy projekcie,
tylko ja mógłby posłużyć jej celom. Mój głos i~oblicze, umysł i~ciało
były zawarte w~parametrach pozyskania celu.

Siedziałem na cokole pomnika Wyzwolicielki, wysuszyłem butelkę piwa i~odstawiłem ją ostrożnie na ziemie, rozejrzałem się, mrużąc oczy w~zachodzącym słońcu. Muzyka na chwilę przycichła, potem weszła kolejna
grupa, coś hucznego i~głośnego, co odbijało się echem od wysokich
budynków dookoła trzech stron placu i~huczało w~otwartej przestrzeni nad
brzegiem i~nad wodą. Cichy fiord\footnote{oryg. sea-loch, loch oznacza
przestrzeń wodną, która może być jeziorem lub formacją morską zwaną też
,,sea lochs'' taką jak fiord, estuarium lub zatoka,
por.~\url{https://pl.wikipedia.org/wiki/Loch\_(geografia)} -- przyp.tłum.} był kilometrami złota, odległe wzgórza i~wyspy stosami
czerni. Powietrze było ciepłe, drżące od muzyki i~ciężkie od zapachu i~potu, oddechów z~alkoholu i~dymu zioła. Ludzie już tańczyli, kołysząc
się i~obracając dookoła straganów pozostałych po dziennym targu.
Zauważyłem spojrzenia i~ukłony od moich różnych współpracowników, Jondo,
Druin, Machard, i~reszty, gdy wirowali obok w~tłumie z~kimś, kto mógłby
być ich partnerem na godzinę, noc lub dłużej.

Przez chwilę, czułem się bardzo samotny i~miałem już się zerwać,
zanurzyć się i~poszukać kogoś, kogokolwiek, kto przyjąłby mnie nawet na
jeden taniec. Nie było to normalne, przez całe lato zwykle przy takich
okazjach miałem szczęście. Jak większość moich współpracowników, byłem
młody i~-- z~konieczności -- silny, a~moja próżność nie potrzebowała
pochlebstw. W~większości byliśmy hojnymi obcymi, a~zatem mile
widzianymi. Jednak byłem w~poważnym i~abstrakcyjnym nastroju,
nadchodzące jesienne studia już kładły swój długi cień, a~przy całej
radości wieczoru, nie rozśmieszyłem ani razu kobiety i~moje powodzenie
uciekło.

Ona szła przez ten gęsty tłum, jakby go tam nie było. Zobaczyłem ją,
zanim mnie ujrzała. Jej długie czarne włosy były zaplecione na skroniach
w dwa wąskie warkocze, a~unoszące się fale niezaplecionych pokazywały
ślady kasztanów w~późnym słońcu. To złote światło i~rumiany cień
określały jej opaloną i~zarumienioną twarz: wielkie jasne oczy, wysokie
kości policzkowe, krzywiznę policzka i~brody, czerwone usta. Była ubrana
w suknię z~zielonego aksamitu, która wydawała się, i~prawdopodobnie
była, tak skrojona, żeby ukazać jej silną i~dobrze ufundowaną postać.
Jej wzrok napotkał mój i~zastygł. Jej oczy były duże i~lekko skośne, i~chwyciły moje spojrzenie jak pułapka.

Bez wątpienia istnieje jakiś cielesny odpowiednik takich momentów
,,strzały w~serce'', przedstawianych w~kreskówkach. Może nagłe
zapotrzebowanie na zapasy cukru w~komórkach. To było bardziej jak kolec
niż strzała i~minęło w~mniej niż sekundę, ale istnieje, to ostre, słodkie
pchnięcie.

Chwilę później stanęła przede mną, patrząc na mnie z~góry zagadkowo,
zaciekawiona, potem podjęła jakąś decyzję i~usiadła koło mnie na zimnym
czarnym marmurze. Kopyta konia Wyzwolicielki unosiły się ponad nami.
Patrzyliśmy na siebie przez chwilę. Serce mi waliło. Wydawała się
młodsza, bardziej niezdecydowana, niż wydawałoby się z~pierwszego,
śmiałego spojrzenia. Jej tęczówki były złotobrązowe, otoczone na
zielono\dywiz niebiesko. Mogłem dojrzeć delikatne rozpryskane piegi pod
opalenizną. Drobny złoty łańcuszek dookoła jej szyi podtrzymywał prostą
siateczkę zawierającą kryształ widzenia wielkości gołębiego jaja. Wisiał
pomiędzy jej piersiami, jego mały świat błyskający losowo od delikatnego
tarcia. Jeszcze delikatniejszy srebrny łańcuch wskazywał na jakąś inną
ozdobę, ale wisiała niżej, niż mogłem dojrzeć. Sztylet, derringer i~torebka na jej wąskim pasku biodrowym były tak eleganckie i~delikatne,
że niemal tylko z~nazwy. W~jej zapachu był jakiś mocny półton, nie
wiedziałem, czy naturalny, czy sztuczny.

-- Cóż, oto i~jesteś -- powiedziała, jakbyśmy umówili się spotkać w~tym
dokładnie miejscu. Przez kilka uderzeń serca, bawiłem się myślą, że to
mogła być prawda, że była kimś, kogo naprawdę znałem i~niewytłumaczalnie, niewybaczalnie zapomniałem, ale nie, nie
przypominałem sobie, żebym kiedykolwiek ją spotkał wcześniej. W~tym
samym czasie nie mogłem się pozbyć przekonania, że już ją znałem, i~zawsze znałem.

-- Cześć -- powiedziałem z~braku niczego mniej banalnego. -- Jak się
nazywasz?

-- Merrial -- odpowiedziała. -- A Ty jesteś...?

-- Clovis -- powiedziałem. -- Clovis colha Gree.

Kiwnęła do siebie, jak gdyby pewne dane zostały potwierdzone, i~uśmiechnęła się do mnie.

-- Więc, colha Gree, czy zamierzasz mnie zaprosić do tańca?

Zerwałem się na stopy, zdumiony. 

-- Tak, oczywiście. Czy uczynisz mi ten
honor?

-- Oczywiście -- powiedziała. Wzięła moją dłoń w~ciepły, suchy chwyt i~wstała wdzięcznie, łącząc ten ruch z~pierwszym krokiem. To był szybki
taniec do tradycyjnej pieśni ,,Taktyczni chłopcy''. Rozmawianie było
niemożliwe, ale mimo wszystko bardzo dużo się komunikowaliśmy. Kolejny
taniec nastąpił, a~potem wolniejszy.

Skończyliśmy daleko od miejsca, gdzie zaczęliśmy, przeniesieni blisko do
zewnętrznych stołów największego pubu na placu, Karonady. Niektórzy z~chłopaków z~pracy już byli przy jednym ze stołów, z~lokalnymi
dziewczynami. Moi kumple spojrzeli się na mnie dziwnie, z~zazdrością i~ukrytym rozbawieniem. Ich kobiece partnerki patrzyły laserami na
Merrial, z~powodu, którego nie mogłem pojąć. Oczywiście była atrakcyjna
i wyglądała piękniej w~moich oczach z~każdą mijającą sekundą, ale inne
dziewczyny nie były oczywiście mniej pobłogosławione. Nie była
nierządnicą, chyba że była nierozsądna (nierząd był szanowanym, ale
regulowanym rzemiosłem w~tym mieście, jego wykonywanie było niedozwolone
na placu).

Przedstawienia się były dziwne.

-- Co chciałabyś, Merrial? -- spytałem.

Uśmiechnęła się do mnie. Była, w~rzeczywistości, tak wysoka jak ja, ale
moje buty miały wysokie obcasy.

-- Piwo proszę.

-- Dobrze. Poczekasz tutaj?

Wskazałem puste miejsce na najbliższej ławce, koło Jondo i~jego obecnej
dziewczyny.

-- Tak zrobię -- powiedziała Merrial.

Jondo rzucił mi kolejne dziwne spojrzenie, uśmiech z~opuszczonym
kącikiem ust, i~wzniesionymi brwiami. Wzruszyłem ramionami, poszedłem do
baru, wracając kilka minut później z~trzylitrowym dzbankiem i~kilkoma
wysokimi szklankami. Merrial siedziała tam, gdzie ją zostawiłem,
ignorując fakt, że była ignorowana. Złożyłem tę niezwyczajną
niegrzeczność na karb jakiejś lokalnej kłótni, których Carron Town -- i~Stocznia, w~rzeczy samej, oraz Projekt -- miał wiele. Jeżeli jeden z~przodków Merrial obraził jednego z~przodków Jondo (lub kogokolwiek), na
razie nie był to mój problem.

Stół był zbyt szeroki, żeby prowadzić jakąkolwiek intymną konwersację w~poprzek, więc usiadłem koło niej, uruchamiając newtonowską kolizję
bioder na całej ławie, gdy moi przyjaciele i~ich dziewczyny przesuwali
swoje zadki od nas. Napełniłem szklanki i~wzniosłem moją.

-- \textit{Slàinte}\footnote{gael. zdrowie -- przyp.tłum.} -- powiedziałem.

-- \textit{Slàinte, mo chridhe}\footnote{gael. zdrowie, moje serce -- przyp.tłum.} -- powiedziała, cicho, ale pewnie, jej oczy nad przechyloną
krawędzią.

I Twoje zdrowie, moja droga, pomyślałem. Znowu jej cała postawa nie była
wstydliwa ani bezczelna, ale jakbyśmy byli ze sobą od miesięcy lub lat.
Nie wiedziałem, co powiedzieć, więc to powiedziałem.

-- Czuję, jakbyśmy się już znali -- powiedziałem. -- Ale się nie znamy. -- Roześmiałem się. -- Chyba że kiedy byliśmy dziećmi?

Merrial potrząsnęła głową. 

-- Nie mieszkałam tutaj jako dziecko -- powiedziała niejasnym tonem. -- Może widziałeś mnie przy projekcie.

-- Myślę, żebym zapamiętał -- powiedziałem. Uśmiechnęła się, przyjmując
komplement, więc dodałem: -- Pracujesz przy \textit{Projekcie}? -- Brzmiałem
bardziej zaskoczony, niż powinienem był, w~końcu wiele kobiet przy nim
pracowało w~kantynie i~w administracji.

-- Aye -- powiedziała -- tak. -- Bawiła się naszyjnikiem, rozgrzewając ogień
w jego obrębie, i~nie tylko tam. -- Przy systemie naprowadzania.

-- Och -- powiedział, nagle zrozumiałem. -- Jesteś \textit{inżynierką}.

-- Jestem majsterką\footnote{ oryg. tinker -- m.in. dawniej, osoba, która,
wędrując, trudniła się naprawą metalowych garnków, w~Polsce
kotlarz, jednocześnie sam źródłosłów oznacza wprowadzanie drobnych zmian,
majsterkowanie, stąd majsterek jako nazwa zawodu (nie używam słowa ,,majster'', bo 
oznacza to bezpośrednio mistrza w danym rzemiośle, gdzie ,,tinker'' ma, zdaje się, znaczenie bardziej partacza), por.~\url{https://dictionary.cambridge.org/dictionary/english/tinker} -- przyp.tłum.} -- powiedziała równym tonem, używając słowa, którego tak
niezdarnie unikałem. Wypowiedziała je z~dumą tak oczywistą i~dostatecznie głośną, żeby być usłyszana. Chichoty okrążyły stół.
Spojrzałem nad ramieniem Merrial na Jondo i~Macharda. Potrząsnęli lekko
głową, niepewnie, potem wrócili do swoich rozmów.

Sprawiedliwość ich osądzi. Jako człowiek z~miasta, czułem się wyżej
ponad takimi wiejskimi bzdurami, jednak uświadamiając sobie, że jej
zawód nawet mną lekko wstrząsnął. Cokolwiek się działo pomiędzy nami,
byłoby mniej lub bardziej poważne niż jakakolwiek zabawa z~lokalną
dziewczyną. Pochyliłem się do przodu tak, że ramiona moje i~Merrial
zdefiniowały nasz własny krąg społeczny.

-- Brzmi jak interesująca praca -- powiedziałem.

Skinęła głową. 

-- Dużo matematyki, dużo -- i~tym razem zniżyła głos -- programowania.

-- Ach -- powiedziałem, próbując wymyślić jakąś odpowiedź, która nie
ujawniłaby mnie jako tak uprzedzonego jak koledzy. -- Czy to bardzo
niebezpieczne? -- Powstrzymałem impuls, by spojrzeć nad ramieniem, ale
nagle, ostro byłem świadom masywnej obecności wzgórz dookoła Miasta, ich
zalesionych stoków jak najeżone plecy wielkich bestii Lasów Kaledonu.

-- \textit{Biała} logika -- wyjaśniła Merrial. -- Prawa droga, wiesz? Ścieżka
światła. -- Nie brzmiała, jakby to rozróżnienie wiele dla niej znaczyło.

-- Rozum nas prowadzi -- odpowiedziałem z~odruchową pobożnością. -- Ale, to
musi być kuszące. Skróty, tak?

-- Ścieżka mocy zawsze jest pokusą -- powiedziała, ze zwykłą zażyłością. -- Szczególnie gdy pracujesz nad systemem naprowadzania! -- Roześmiała się,
przyznaję, że wzdrygnąłem. Wskazała na swój talizman. -- Dość o~tym.
Wiem, co robię, więc to nie jest niebezpieczne. Przynajmniej, nie tak
niebezpieczne jak to wygląda z~zewnątrz.

-- Dobra. -- Pomimo elektrycznego dreszczyku, które jej słowa wzbudziły,
byłem skłonny tak jak ona, zmienić temat. -- Mogłabyś powiedzieć to samo
o tym, czym ja się zajmuję.

-- A czym się zajmujesz? -- spytała przez uprzejmość. Już wiedziała. Byłem
tego pewien, nie do końca wiedząc dlaczego.

-- Pracuję w~Stoczni -- powiedziałem.

-- Na Statku?

-- Och, nie na Statku! -- Samokrytyczny śmiech, niezbyt szczery. -- Na
Platformie. Przez lato, jestem spawaczem.

Przełknęła trochę piwa. 

-- A resztę czasu?

-- Jestem uczonym -- powiedziałem. -- Historii. W~Glaschu\footnote{ gael. Glasgow
-- przyp.tłum.}

Było to lekkie wyolbrzymienie. Właśnie osiągnąłem stopień Mistrza
Sztuk\footnote{ oryg. master of arts -- zapewne mowa o tytule magistra nauk
humanistycznych -- przyp.tłum.}, a~moja letnia praca była szalonym,
skromnym wysiłkiem, żeby zarobić dostatecznie dużo na utrzymanie siebie
przy próbie doktoratu. Studia wyższe były moją ambicją, nie moim
zawodem. Jednak odmawiałem nazywania siebie studentem. Merrial
popatrzyła na mnie ze swego rodzaju wymuszoną empatią podobną do tej,
którą ja obdarzyłem jej ujawnienie. 

-- To brzmi\ldots interesująco -- powiedziała. -- Jaką \textit{częścią} historii?

Wskazałem na plac, na czarną sylwetkę posągu. Za nią, na wschodzie
pojawiały się na niebie pierwsze widzialne gwiazdy wieczoru.

-- Życie Wyzwolicielki -- powiedziałem.

-- I~czego się dowiedziałeś? -- Pochyliła się bliżej, widocznie bardziej
zainteresowana. Jej czarne brwi uniosły się odrobinę, jej ciemne oczy
się rozszerzyły. Bez myślenia, zapaliłem papierosa. Pamiętałem o~manierach i~zaoferowałem jej jednego. Wzięła go, uśmiechając się, i~obsłużyła się dzbankiem piwa, potem napełniła moją szklankę. 

-- Nie myślałam, że jest tam dużo nowego do poznania -- dodała, patrząc na
mnie spode brwi.

Złapałem przynętę. 

-- Ach, ale jest! -- odpowiedziałem jej. -- Wyzwolicielka żyła w~Glasgow, wiesz. Przez jakiś czas.

-- W~wielu miejscach powiedzą ci, że tam żyła, przez jakiś czas! -- Merrial się roześmiała.

-- Aye, ale mamy dowody -- powiedziałem. -- Widziałem dokumenty pisane jej
własną dłonią i~podpisane. Nie ma dyskusji, że to ona je napisała. Co
one znaczą, cóż, to zupełnie inna kwestia. A wielka część innych pism,
druków to jest, i~materiałów, jest ciągle w\ldots wiesz.

-- W~Mrocznym Magazynie?

-- Tak -- powiedziałem. -- Mroczny Magazyn. Chciałbym\ldots -- Nawet tutaj,
nawet teraz, było niemożliwością powiedzenie, co chciałbym. Jednak
Merrial zrozumiała.

-- Proszę bardzo, colha Gree -- powiedziała. -- Ścieżka mocy zawsze jest
pokusą!

-- Tak, tak właśnie jest -- przyznałem ponuro. -- Możesz na nie spojrzeć,
opisane jej własną dłonią, zastanawiasz się, co w~nich jest, i \ldots cóż.

-- Prawdopodobnie skażone -- powiedziała energicznie. -- Niewarte
zajmowania się.

-- Oczywiście, że skażone\ldots

Potrząsnęła głową, z~lekkim, krótkim zmarszczeniem brwi. 

-- W~znaczeniu technicznym -- wytłumaczyła. -- Śmieciowe dane, nieczytelne.

Śmieciowe dane? Co to znaczyło?

-- Rozumiem -- powiedziałem, rozumiejąc tylko tyle, że próbowała wyjaśnić
część gwary jej profesji, kolejna nierozsądna intymność.

-- Mimo wszystko -- kontynuowała -- to musi być dziwna praca, historia. Nie
wiem, jak możesz to znieść, przekopywanie się przez martwą przeszłość.

Słyszałem wariacje tej samej opinii od tak wielu osób, zaczynając od
mojej matki tak, że irytacja wezbrała się we mnie i~jestem pewien, że
pokazała się to na mojej twarzy. Uśmiechnęła się, jakby zapewniając
mnie, że nie miała tego przeciwko mnie osobiście, i~dodała: 

-- Wiesz, Posiadacze działają nie tylko przez czarną logikę. Mogą się dostać do
Twojego umysłu również przez słowa na papierze.

-- Mówisz bardzo swobodnie -- powiedziałem. Jak na kobietę, nie dodałem.

Potraktowała to jako komplement, a~tym samym odpłaciła się, nie
rozpoznając arogancji na sztywnych nogach, którą przedstawiała moja
uwaga.

-- To sposób majsterków. -- powiedziała, wzbudzając we mnie kolejny szok.
-- Mówimy, jak chcemy.

Nie mogłem na to odpowiedzieć, więc ciągnąłem dalej.

-- Musimy zrozumieć Posiadanie -- wyjaśniłem świętoszkowato -- żeby
zrozumieć Wyzwolenie.

-- Ale czy rozumiemy Wyzwolenie? -- spytała, dokuczając mi nieubłaganie. -- Czy Ty rozumiesz, Clovis colha Gree?

-- Nie mogą powiedzieć -- powiedziałem, co było wystarczającą prawdą, choć
dość ekologicznie z~prawdą.

-- Dobrze -- powiedziała Merrial. -- \textit{My} nie twierdzilibyśmy, że je
rozumiemy, a~znaliśmy Wyzwolicielkę lepiej niż wielu. -- Chytry uśmiech.
-- Jak wiesz.

Skinąłem powoli głową. Wiedziałem dobrze. Pogardzani i~przerażający,
jak czasem są, nie na darmo Majsterkowie są znani jako Dzieci
Wyzwolicielki. Dawno temu pracowali nad jej testamentem, w~czasach
niepokojów, a~błogosławieństwo tej pracy chroniło ich przez pokolenia.
To i, bardziej cynicznie, ich niejasna i~niezastępowalna wiedza.

Słyszałem plotki -- zawsze dyskredytowane przez historyków Uniwersytetu -- o~silnej ciągłości, mrocznych arkanach, które łączyły dzisiejszych
majsterków i~Wyzwolicielkę, i~które sięgały do czasów jeszcze bardziej
odległych, kiedy Posiadanie było tylko drzewkiem, jego cień jeszcze nie
zasłaniał Ziemi.

Jej dłoń przykryła moją, na krótko.

-- Nie rozmawiajmy o~tym -- powiedziała.

Zatem rozmawialiśmy o~innych rzeczach, jej pracy, mojej pracy,
dzieciństwie. Szklanki zostały uzupełnione dwa razy. Wstała, ważąc teraz
pusty dzbanek. 

-- To samo jeszcze raz?

Wstałem, mówiąc: 

-- Załatwię \ldots

-- Nalegam -- powiedziała i~poszła. Patrzyłem na kołysanie jej bioder,
sposób jak się przenosiło na kołysanie jej ciężkiej sukni i~kręcenie potoku jej włosów na plecach, gdy przechodziła przez tłum i~znikała
w szerokich drzwiach Karonady. Moi przyjaciele obserwowali tę uwagę,
ironicznie się uśmiechając.

-- Nadeszły dla Ciebie ciekawe czasy, Clovis -- zauważył Jondo. Potarł
sugestywnie swój długi, czerwony kucyk, sprawiając, że jego dziewczyna
się zaśmiała. -- Mi to wygląda na urok.

Machard uśmiechnął się drwiąco. 

-- Poważnie, człowieku -- powiedział mi -- uważaj. Nie znasz majsterków tak jak my. Są niewierni, bezbożni, klanowi i~się nie osiedlają. W~najlepszym przypadku złamie ci serce, w~najgorszym\ldots

-- O co wam chodzi? -- syknąłem, pochylając się bokiem, żeby trzymać
dziewczyny z~dala od mojego gniewu. -- No weźcie, dajcie jej szansę.

Miny moich dwóch przyjaciół wyrażały bezczelną niewinność.

-- Spokojnie, Clovis -- powiedział Machard. -- Tylko rada. Zignoruj ją,
jeżeli chcesz, to Twoja sprawa.

-- Cholerna racja, że moja -- powiedziałem. -- Więc pilnujcie swoich. -- Powiedziałem lekko ostre słowa, nie kłótliwe, ale stanowcze. Obaj
chłopaki wzruszyli ramionami i~wrócili do rozmowy z~ich dziewczynami.
Zostałem zignorowany tak jak Merrial.

Ostatni pociąg z~Inverness zsunął się w~dolinę\footnote{ oryg. glen -- rodzaj
doliny, która jest długa i~ograniczona łagodnie nachylonymi stokami,
zob.~\url{https://en.wikipedia.org/wiki/Glen} -- przyp.tłum.},
iskry z~kabla nad nim jaskrawe w~zmierzchu, i~zniknął za pierwszymi
domami. Minutę później mogłem usłyszeć krótkie zamieszanie, gdy
zatrzymał się na stacji, kilka ulic dalej. Chmury i~szczyty wzgórz były
oświetlone na różowo, to samo światło odbijało się od samotnego
sterowca, kierującego się na zachód. Kilka świateł było włączonych w~mieście -- wpół do jedenastej wieczorem było znacznie za wcześnie na to -- ale domy, które rozciągały się na zboczu doliny i~wzdłuż brzegu
zaczynały być tak ciemne, jak sosnowy las, który zaczynał się tam, gdzie
kończyły się osady.

Wyżej na zboczu doliny, boczne i~tylne światła pojazdów wyznaczały
meandry drogi, a~ciemna zieleń zalesionych wzgórz spotykała jasną zieleń
niższych stoków, pole połączone z~polem, pastwisko z~pastwiskiem aż do
miejsca gdzie stoki wzgórz ukrywały widok, a~ziemia była ciemna. Gdzieś
daleko, ale brzmiąc niesamowicie blisko, zawył wilk, jego przewlekła,
złowieszcza nuta czysto słyszalna ponad dźwiękami miasta i~hałasem
Jarmarku.

Z każdą chwilą plac stawał się coraz bardziej zapełniony i~głośny. Picie
i tańce miały trwać godzinami. Żonglerzy i~akrobaci, połykacze ognia i~muzycy współzawodniczyli o~uwagę i~drobne, ze sobą i~kramarzami. Targi w~letnie czwartki były lokalnie nazywane ,,Jarmarkami'', ale tylko raz w~miesiącu wypełniały obietnicę, z~bardziej imponującym kontyngentem
artystów niż był dzisiaj tutaj, jak również z~podróżującymi muzykami,
wirującymi mechanicznym przejażdżkami i, oczywiście, majsterkami. Ci
ostatni zajmujący się ich prawowitymi zajęciami inżynierii i~ich mniej
szanowanym, ale znacznie częściej dochodową, sztuką wróżenia.

Pociąg odjechał, ciągnąc iskry wzdłuż równiny ujścia rzeki Carron i~dookoła południowego brzegu Loch Carron.

Merrial wróciła z~pełnym dzbankiem, butelką whisky i~tacą małych
kieliszków. Bez słowa postawiła tacę i~butelkę na środku stołu i~usiadła, tym razem naprzeciwko mnie. Napełniła nasze wysokie szklanki,
odstawiła dzbanek i~wskazał butelkę whisky. 

-- Częstujcie się -- powiedziała.

Moi przyjaciele po tym stali się bardziej przyjaźni. Wszyscy
rozmawialiśmy ze sobą, o~pracy, nieuniknionych plotkach i~sarkaniach
Projektu, o~tym skandalu i~tamtym brygadziście i~innych głupotach.
Ironicznie, dziewczyny wydawały się wyłączone, i~zaczęły rozmawiać
pomiędzy sobą. Merrial, pokazując takt wystarczający za nas oboje,
zauważyła to i~stopniowo, teraz, gdy lody zostały przełamane, wróciła do
rozmowy ze mną. Jondo i~Machard zabrali się do swoich porzuconych zadań
uwodzenia i~flirtu. Kiedy, kilka godzin później, zaprosiła mnie do
swojego domu, ich grubiaństwo było relatywnie ograniczone.

Plac był głośniejszy niż kiedykolwiek. Jedynymi ludźmi idącymi do domu
lub do łóżka, byli tacy jak my, robotnicy projektu, którzy, w~odróżnieniu od tubylców, musieli pracować następnego dnia, w~piątek.
Szliśmy ciemną ulicą na północ od placu i~przez most nad rzeką Carron ku
przedmieściom New Kelso. Merrial zatrzymała się pośrodku mostu. Jednym
ramieniem obejmowała moje biodra. Drugim, wskazała dookoła.

-- Spójrz -- powiedziała. -- Co widzisz?

Po naszej prawej, maszyneria miejskiej elektrowni atomowej złowrogo
mruczała w~ciemnościach, po naszej lewej hodowle ryb, ogrzewane przez
odpływ wody z~reaktora, rozciągały się na brzegu. Spojrzałem się na
lewo, na prawo, potem do tyłu na miasto, na wprost na New Kelso, przez
fiord na inne miasteczka.

Uśmiechnęła się, widząc moją zdumioną ciszę.

-- W~górę.

Nad nami płonęła Droga Mleczna, mrugała zorza polarna, aerostat
komunikacyjny świecił się na różowo w~słońcu, które dla nas dawno temu
zaszło. Wielki Wóz wisiał nad wzgórzami na północy. Meteor zabłysł
krótko, moje westchnienie efekt dźwiękowy dla jego cichego przejścia.
Na zachodzie niebo ciągle było podświetlone, słońce wzejdzie za cztery
godziny.

-- Widzę gwiazdy -- powiedziałem.

-- Dokładnie -- powiedziała, zadowolona z~mojej spostrzegawczości. -- Możesz. Jesteśmy w~środku miasta z~dziesięcioma tysiącami mieszkańców i~możesz zobaczyć Drogę Mleczną. Nie tak dobrze, jak zobaczyłbyś ją ze
szczytu Glas Bhein, oczywiście, ale możesz ją zobaczyć. Dlaczego?

Wzruszyłem ramionami, patrząc tam i~z powrotem. Nigdy o~tym nie
myślałem.

-- Nie ma chmur? -- zasugerowałem bystrze.

Roześmiała się, złapała moją dłoń i~pociągnęła mnie do przodu. 

-- I~ty jesteś uczonym historii!

-- Co to ma do rzeczy?

Wskazała na lampę uliczną na końcu balustrady mostu. Słup lampy miał
jakieś trzy metry wysokości, wewnętrzna odbijająca powierzchnia jej
stożkowej oprawy ostro blokowała prawie całą iluminację do góry. 

-- Czy kiedykolwiek widziałeś lampy jak ta na obrazach dawnych czasów? -- spytała.

-- Teraz gdy o~tym myślę -- odpowiedziałem -- to nie.

-- Miasto tej wielkości miałoby lampy wszędzie, świecące światłem w~niebo. Od lamp ulicznych, okien i~witryn sklepowych. Samo powietrze by
świeciło. Mógłbyś zobaczyć garść gwiazd w~najczystszą noc.

Myślałem o~starych zdjęciach, na które patrzyłem przez szkło. 

-- Wiesz, masz rację -- powiedziałem. -- Tak to właśnie wyglądało.

-- Niektórzy ludzie -- kontynuowała Merrial, w~nagłym przypływie gniewu -- żyli całe życie ani razu nie oglądając Drogi Mlecznej!

-- Bardzo smutne -- powiedziałem. W~rzeczywistości ta myśl spowodowała
ucisk w~piersi, jakbym walczył o~oddech. -- Jak oni to znosili?

-- Ta, cóż, to jest pytanie, które równie dobrze mógłbyś zadać. -- Spojrzała na mnie. -- Myślałam, że wiedziałeś.

-- Szczerze mówiąc, nigdy nie zauważyłem.

-- A dlaczego tak nie robimy? -- Znowu wskazała na elektryczny półmrok
otaczającego miasto.

-- Ponieważ byłoby to marnotrawstwem -- powiedziałem. Gdy tylko
powiedziałem, zrozumiałem, że powiedziałem to bez myślenia i~nie była to
odpowiedź.

Merrial się roześmiała. 

-- Mamy niepotrzebną moc!

Teraz ja się nagle zatrzymałem. Skręciliśmy w~prawo i~szliśmy ścieżką
koło elektrowni. Wiedziałem na pewno, że mogłaby, kiedy potrzeba w~nagłym przypadku -- gdy dodatkowe ogrzewanie było potrzebne, żeby
usunąć śnieg po zamieci -- wyprodukować dostatecznie dużo elektryczności,
żeby kilka razy oświetlić Carron Town.

-- Masz rację -- powiedziałem. -- Więc dlaczego tego nie robimy? Widziałem
zdjęcia wielkich starożytnych miast i, masz rację, one się świeciły.
Wyglądały\ldots imponująco. Może było tak jasno, że nie potrzebowali
gwiazd, w~zamian mieli światła miast. Sami sobie stworzyli gwiazdy!

Merrial powoli kręciła głową.

-- Może to byłoby dobre dla nich -- powiedziała. -- Ale nie byłoby dla nas.
Wszyscy stajemy się\ldots niespokojni, kiedy nie możemy zobaczyć nocnego
nieba. A Ty nie, tylko o~tym myśląc?

Wziąłem głęboki wdech i~wypuściłem go z~westchnieniem. 

-- Ta, masz rację! -- Szliśmy, jej kroki nadawały tempo mojej wolniejszej
przechadzce.

-- Jesteś dziwną kobietą -- powiedziałem.

Uśmiechnęła się, przytrzymała moje biodra bardziej stanowczo i~oparła
głowę o~moje ramię. Zauważyłem, że patrzę na jej włosy, i~niżej w~wycięcie dekoltu jej sukni i~jasny kamień pomiędzy jej piersiami.

-- Pewnie, że jestem -- powiedziała. -- Ale wszyscy jesteśmy, to chcę
powiedzieć. Jesteśmy inni od ludzi, którzy byli przed nami, lub przed
czasami Wyzwolicielki, i~nikt się nie zastanawia, jak i~dlaczego.
Uczucia wobec nieba są tylko tego częścią. Żyjemy dłużej i~mniej się
rozmnażamy, chorujemy mało, czasem myślę, że nawet nasz wzrok jest
lepszy. Wszystkie te zmiany są wpisane na stałe w~nasze geny odporne na
promieniowanie\ldots

-- Nasze co?

Poczułem wzruszenie jej ramion.

-- Tylko żargon majsterków, colha Gree. Nie martw się. Podłapiesz.

-- Och, na pewno, co?

-- Tak. Jeżeli ze mną zostaniesz.

Na to była tylko jedna odpowiedź. Odwróciłem ją i~pocałowałem.
Przycisnęła swoje usta do moich i~wsunęła ręce pod mój rozpięty płaszcz
i zaczęła wędrować nimi po bokach i~plecach. Czułem je przez moją
jedwabną koszulę jako małe gorące zwierzęta. Pocałunek trwał przez jakiś
czas i~skończył się z~naszymi językami trzepoczącymi jak ryby na dnie
głębokiego stawu. Potem odchyliła się, chwyciła moje ramiona, spojrzała
na mnie i~powiedziała: 

-- Zdaje się, że to oznacza, że zostajesz, colha
Gree.

Nagle oboje się śmialiśmy. Złapała moją dłoń, machnęła ją i~zaczęliśmy
znowu iść, rozmawiając nie wiadomo o~czym. Na skraju miasta skręciliśmy
na rogu w~małe osiedle około tuzina jednopiętrowych drewnianych domów z~kominami. Niektóry domy były oddzielne, każdy z~własną grządką ogrodu.
Inne, mniejsze, były ustawione w~nie do końca uporządkowanych rzędach.
Nawet latem, nawet z~kablami elektrycznymi wiszącymi wszędzie, zapach
dymu drzewnego utrzymywał się w~powietrzu. Żółte światła świeciły się
zza słomianych żaluzji. Pies zaszczekał i~został uciszony zirytowanym
okrzykiem.

-- Hej, no chodź -- powiedziała Merrial z~psotnym uśmiechem.

Nie zauważyłem, że moje stopy zawahały się, gdy ścieżka zmieniła się z~bruku na udeptany żwir.

-- Nigdy wcześniej nie byłem w~obozie majsterków -- przeprosiłem.

-- Nie gryziemy. -- Kolejny bezczelny uśmiech. -- Cóż, to znaczy\ldots

-- Jesteś okropną kobietą.

-- Och, jestem, naprawdę. Okrutną, tak mi powiedziano.

-- Trzymam Cię za słowo.

-- Trzymam Cię za więcej.

Trzymała mnie, gdy zatrzymała się przed jednym z~tych małych domów w~środku rzędu, i~wygrzebała kluczyk długości pięciu centymetrów na
rzemyku przyczepionym do jej pasa, ale schowanym w~rozcięciu z~boku jej
sukni. Zamek też wydawał się niedorzecznie mały, mosiężna okrągła plama
na białych drzwiach na poziomie oczu.

-- Więc wchodzisz czy co?

Pożądanie i~rozum walczyły ze strachem i~przesądem, i~wygrały. Wszedłem
za nią nad progiem z~polerowanego drewna, gdy włączyła elektryczne
światło. Stałem przez chwilę, mrugając na nagłą, czterdziestowatową
powódź. Główny pokój miał jakieś cztery metry na sześć. Przy
przeciwległej ścianie stał piecyk na drewno, pochylony nisko. Nad nim
był szeroki gzyms, na którym głośno cykał wielki zegar. Było wpół do
pierwszej. Po obu stronach piecyka stały rzędy półek z~setkami książek.
W rogu po lewej wystawał ze ściany stół warsztatowy, z~mikroskopem, i~piekielnym bałaganem lutownic, kabli i~narzędzi. Szorstkie,
niepolerowane kryształy widzenia różnych wielkości leżały pomiędzy nimi.
Główny stół domu był wielkim, dębowym kwadratem około półtorametrowego
boku na rzeźbionych nogach. Na nim był położony szydełkowany bawełniany
obrus, przyciśnięty na środku przez półkulę kryształu widzenia
przynajmniej średnicy trzydziestu centymetrów, tak delikatnie
wykończony, że wyglądał jak kopuła ze szkła. W~środku, przesuwały się
wzgórza i~chmury.

Merrial chwilę stała przy stole, sięgnęła za głowę i~wyjęła zapinkę z~włosów tak, że dwa wąskie warkocze opadły do przodu i~obramowały jej
twarz. Potem uniosła łańcuch z~talizmanem oraz inny, delikatniejszy
srebrny łańcuszek, z~szyi i~odłożyła je na stół.

Miejsce pachniało dymem z~drewna, ziołami i~kwitnącymi roślinami
wciśniętymi w~niedbale wybrane pojemniki w~każdym możliwym kącie.
Drewniane ściany były polakierowane, wisiała na nich bezsensowna
różnorodność starych druków i~obrazów -- krajobrazy, damy, lisy, koty,
tego rodzaju rzeczy -- i~przypięte rysunki związane z~projektem. Otwarte
drzwi prowadziły do małej pomywalni, zasłonięta alkowa koło niej
zabierała resztę tego końca pokoju. Domniemywałem, że zawierała łóżko.

Jednak poprowadziła mnie najpierw na starą, wielką skórzaną sofę przed
kominkiem. Na wpół oparła, na wpół usiadła na niej i~zaczęła rozpinać
moją koszulę, potem zbadała moją pierś ustami i~językiem -- i~zębami -- gdy ja zająłem się rozpinaniem zapięć na plecach jej sukienki, i~zdejmowaniem swoich butów. Gdy skopałem prawy but, \textit{sgean dhu}\footnote{
gael. sgian-dubh -- mały nożyk noszony jako część
tradycyjnego szkockiego ubioru,
więcej~\url{https://en.wikipedia.org/wiki/Sgian-dubh} -- przyp.tłum.} zastukał na podłodze. Do tego czasu rozpięła mój pas i~z poruszeniem i~krokiem oboje zrzuciliśmy nasze ubrania, które upadły na
podłogę we własnym pozamałżeńskim połączeniu. Merrial stała przez chwilę
w niczym, prócz jej długiej jedwabnej halki. Chwyciłem ją w~ramiona, jej
sutki twarde, pierś miękka i~ciepła o~moją pierś, i~znowu się
pocałowaliśmy.

Poruszaliśmy się, tańczyliśmy, Merrial prowadziła, ku zasłoniętej
alkowie. Odsunęła zasłonę, aby pokazać wielkie i~uspokajająco solidnie
wyglądające łóżko. Ukląkłem przed nią i~ściągnąłem jej halkę i~majtki,
pocałowałem ją pomiędzy nogami aż delikatnie podniosła mnie. Udało mi
się zostawić moją bieliznę na podłodze.

Patrzyliśmy na siebie nadzy jak Mężczyzna i~Kobieta w~Ogrodzie z~tamtej
historii. Merrial na wpół odwrócona, odrzuciła narzutę i~podniosła z~łóżka długą, białą, bawełnianą koszulę nocną, którą strząsnęła i~przez
chwilę trzymała w~dłoniach.

-- \textit{Tego} dzisiaj nie będę potrzebować. -- Uśmiechnęła się i~rzuciła
ją na podłogę, a~mnie na łóżko.

Obudziłem się w~świetle dnia i~leżałem kilka chwil, wygrzewając się w~poświacie i~gorących powidokach miłości i~seksu. Przetaczając się i~sięgając ramieniem, odkryłem, że byłem sam w~łóżku. Było jeszcze ciepłe,
tam, gdzie spała Merrial. Powietrze było wypełnione zapachem kawy i~stałym tykaniem zegara\ldots

Czas! Usiadłem w~pośpiechu i~pochyliłem się, żeby zobaczyć tarczę
zegara, odkryłem z~ulgą, że była dopiero piąta. Dzięki Opatrzności
spaliśmy tylko półtorej godziny. Tym samym ruchem odkryłem przyczynę
małych dolegliwości: ugryzień na moich ramionach i~karku, zadrapaniach
na plecach i~pośladkach, bolących mięśni, otartej skóry\ldots

Zwierzę, którego ataki spowodowały te rany, wyszło pomywalni.

-- Dzień dobry -- powiedziała.

Wydałem swego rodzaju skrzeczące odgłosy. Merrial się uśmiechnęła i~podała mi jeden z~dwóch parujących kubków, które niosła. Usiadła w~nogach łóżka, podciągając kolana pod brodę, żeby schować się w~koszuli,
jej wysoki kołnierz, długie rękawy i~zawiły haft dawały jej absurdalny
wygląd skromności.

Łyknąłem z~wdzięcznością kawy, nie mogąc oderwać od niej oczu. Patrzyła
spokojnie na mnie, uśmiechając się jak zadowolony kot.

-- Dzień dobry -- powiedziałem, w~końcu odnajdując słowa. -- I~dziękuję.

-- Nie tylko za kawę, mam nadzieję -- powiedziała Merrial.

Uśmiechałem się tak mocno, że moje policzki też bolały.

-- Nie, nie tylko za kawę. Boże, Merrial, ja nigdy\ldots

Nie wiedziałem jak to powiedzieć.

-- Nie robiłeś tego wcześniej -- zapytała niewinnie.

Kawa wpadła mi do nosa, gdy prychnąłem ze śmiechu.

-- W~porównaniu z~ostatnią nocą, równie dobrze mógłbym nigdy -- przyznałem
smutno. -- Jesteś\ldots jesteś niesamowita!

Jej spokojny wzrok mnie podtrzymywał. Nie okazywała najmniejszego
zażenowania. 

-- Och, nie jesteś taki zły, colha Gree -- powiedziała
oceniającym tonem. -- Ale musisz się jeszcze dużo nauczyć.

-- Mam nadzieję, że mnie nauczysz.

-- Jestem pewna, że tak -- powiedziała. -- Jeżeli będziesz chciał ze mną
zostać, znaczy. -- Machnęła dłonią, jakby tak kwestia nie była jeszcze
ustalona.

-- Zostać z~Tobą? Och, Merrial! -- Nie mogłem mówić.

-- Co?

-- Nic nie sprawi, żebym cię opuścił. Nigdy.

Byłem prawie przerażony tym, co mówiłem. Nie sądziłem, że usłyszę siebie
mówiącego takie słowa, nie jeszcze przez długi czas.

-- Jak miło, że to mówisz -- powiedział, bardzo poważnie, ale z~uśmiechem.
-- Ale\ldots

-- Ale nic! -- Sięgnąłem w~bok i~położyłem kubek na podłodze, przesunąłem
się wzdłuż łóżka do niej. Bez odwracania wzroku, też odłożyła swój
kubek, na kufer na końcu łóżka, i~przesunęła się do przodu na kolanach.
Klęczeliśmy objęcia oboje.

-- Kocham Cię -- powiedziałem. Musiałem powiedzieć to wcześniej,
powiedzieć to wiele razy przez tę noc, ale teraz za tymi słowami kryła
się cała powaga na świecie.

-- Ja też Ciebie kocham -- powiedziała. Przylgnęła do mnie z~nagłą
zawziętość i~oparła twarz na moim ramieniu. Mokre, słone łzy ukąsiły tam
w ślady po ugryzieniu. Pociągnęła nosem i~podniosła głowę, mrugając
teraz nawet jaśniejszymi oczami.

-- Co się stało? -- spytałem.

-- Jestem szczęśliwa -- powiedziała.

-- To ja też.

Spojrzała na mnie poważnie. 

-- Muszę to powiedzieć -- powiedziała, z~kolejnym niekobiecym pociągnięciem nosem. -- Kochanie mnie nie zawsze cię
uszczęśliwi.

Nie mogłem sobie wyobrazić, co miała na myśli, i~też nie chciałem. 

-- Dlaczego to mówisz?

-- Ponieważ muszę -- powiedziała. Jej głos był spięty. -- Ponieważ muszę
być z~Tobą uczciwa.

-- Ta, pewnie -- powiedziałem. -- Cóż, już mnie ostrzegłaś, czy mogę się
zająć kochaniem Ciebie?

Natychmiast się rozjaśniła, jakby jakiś ciężki obowiązek został zdjęty z~jej ramion.

-- Och tak! -- powiedziała, znowu mnie przytulając. -- Kochaj mnie, ile
chcesz, kochaj mnie na zawsze! -- Odsunęła się lekko, spojrzała w~dół,
potem znowu spojrzała na mnie.

-- Ale nie teraz -- dodała z~żalem. -- Musisz iść.

-- Teraz?! -- Wypadliśmy z~naszego wspólnego snu w~codzienny świat, gdzie
byliśmy dwoma osobami, które, tak naprawdę, nie znały się wcale tak
dobrze.

-- Tak -- nalegała. -- Musisz wrócić do miasta, wziąć \ldots prysznic,
przygotować się do pracy i~złapać autobus o~wpół do siódmej.

-- Mogę go złapać stąd.

-- Piekło, nie możesz. Ludzie będą gadać.

-- I~tak będą gadać.

-- Mam na myśli, ludzie tutaj.

Zszedłem niechętnie z~łóżka. Merrial wsunęła się zwinnie pod kołdrę i~podciągnęła ją aż do brody.

-- A co z~Tobą? -- spytałem, gdy szukałem i~układałem ubrania.

-- Jestem robotnikiem umysłowym -- powiedziała zadowolona z~siebie, gdy
się opatulała. -- Zaczynamy o~dziewiątej.

Patrzyła na mnie ubierającego się z~rodzajem tkliwej ciekawości. 

-- Co
masz przy swoim pasku?

Poklepałem twardą skórzaną sakiewkę i~zapiąłem pas. 

-- Narzędzia
rzemieślnika -- odpowiedziałem jej -- i~broń dżentelmena.

-- Rozumiem -- powiedział z~aprobatą.

-- Zatem kiedy Cię znowu zobaczę? -- spytałem, gdy odnalazłem \textit{sgean
dhu} i~wsunąłem go z~powrotem z~boku buta.

-- Wieczorem, ósma, przy pomniku? Pójdziemy coś zjeść?

Udawałem, że namyślam się nad tym pomysłem, potem się roześmialiśmy, a~ona usiadła i~sięgnęła ku mnie. Przytuliliśmy się i~pocałowaliśmy na
pożegnanie. Gdy cofałem się do drzwi, niechętny każdej chwili bez niej
na widoku, migotanie od dużego kryształu widzenia złapało moje oko.
Zatrzymałem się przy stole i~schyliłem się obejrzeć. Gdy to robiłem,
zauważyłem oba naszyjniki Merrial: talizman -- mały kryształ widzenia -- teraz pokazywał niejasno organiczne ślady zieleni, a~na srebrnym
łańcuchu był srebrny element około centymetra w~średnicy, gdzie ukazywał
się monogram złożony z~liter ,,G'' i~,,T'' i~cyfry ,,4''.

Rzecz na środku stołu była cała czarna, prócz dwóch układu punktów
światła, które mogły być pochodniami, miastami lub gwiazdami. Błyskały
sporadycznie, a~jasne kropki literowały jedno słowo: POMOC.

Spojrzałem na Merrial. 

-- To dotarło do końca swojego biegu -- zauważyłem.

-- Zresetuj więc -- powiedział zaspana z~poduszki.

Potarłem chłodną powierzchnię kamienia o~rękaw, przywracając jego chaos,
i z~ostatnim uśmiechem do Merrial, otworzyłem drzwi i~wyszedłem na
słońce o~pierwszym pianiu koguta.\\
\textit{i zarzuciła ramiona na niego tej samej nocy, gdy pociągnęła go w~dół.}

\chapter{Dawne Czasy}

\textit{Śmierć za mną podąża}, pomyślała, gdy jechała do obozu pracy. Było
coś w~tym nieubłaganego, niczym logika: podąża, podąża \ldots 

Pojawienie się
myśli nie miało nic wspólnego z~logiką, pojawiła się w~tle na
powierzchni jej umysłu, kiedykolwiek umysł był czysty. Trochę ją to
męczyło, tak jak inna myśl, która dryfowała w~takich chwilach:
\textit{gdzie jest chyża konnica?}

Brama zasunęła się za nią, piszcząc na zardzewiałych rowkach. Wiatr ze
stepu szumiał na ogrodzeniu z~drutu kolczastego i~zamiatał wzbity kurz,
gdy kierowała czarnym koniem. Strażnik pośpieszył. W~jakiś sposób udało
mu się ukazać energiczny żołnierski krok jako uniżony, nawet gdy jego
mundur, zrobiony z~ciemnoniebieskiej tkaniny, wyglądał na wojskowy.
Uchylił czapki bejsbolówki z~logo i~literami ,,Ochrony Wzajemnej''.

-- Dzień dobry, Obywatelko.

Ten tytuł był już grzecznościowy. Myra Godwin-Dawidowa uśmiechnęła się i~podała mu wodze.

-- Dzień dobry -- odpowiedziała, zsiadając z~konia. Mogłaby usłyszeć
skrzypienie w~kolanach. Uniosła sakwy przy siodle i~zarzuciła je na
ramię. Waga prawie nią zachwiała i~ramię strażnika szarpnęło ku niej.
Jednak nie zamierzała zaakceptować żadnej pomocy z~tej strony. -- To
będzie wszystko, dziękuję.

-- Jak sobie życzycie, Obywatelko. -- Strażnik zasalutował i~założył
czapkę. Ciągle patrzyła na niego z~góry, jej buty jeździeckie dodawały
siedem centymetrów do jej metra osiemdziesięciu.

Poklepała zad wielkiej klaczy i~patrzyła, jak strażnik odprowadzał
bestię, potem ruszyła w~kierunku baraków noclegowych. Gdy szła, zdjęła
skórzane rękawice, wepchnęła je niezręcznie w~głębokich kieszeniach jej
długiego futrzanego płaszcza i~schowała dzikie pasmo pod uszankę \footnote{
oryg. sable hat -- dosł. czapka sobolowa, prawdopodobnie uszanka tj.~czapka futrzana -- przyp.tłum.}. Dłonie cętkowane, pokazujące się żyły,
prążkowane paznokcie: twarde pazury starego ptaka, ciągle elastyczne,
ale lepsza wskazówka co do jej prawdziwego wieku niż szorstko poryta,
ale jędrna twarz, proste plecy i~gibki krok. Jej kolana bolały, ale
próbowała tego nie pokazywać, lub, żeby ją to zwolniło.

Obóz miał około jednego kilometra na dwa kilometry. Ponad odległym
płotem mogła patrzeć aż do horyzontu, ponad którymi wyrastało wiele
platform startowych i~kilka pozostałych wysokich statków starego portu.
Kiedyś była to dumna flota. Ile czasu minie, zanim będzie musiała
powiedzieć, \textit{wszystko moje statki odeszły, moi ludzie nie żyją?}.

Jakby drwiąc z~jej myśli, mały statek przeleciał, rycząc. Dostrzegła
fragment: kanciasty, wielościanowy, półprzezroczysty, widmowy
niewykrywalny bombowiec wrzeszczał prosto w~niebo z~Bajkonuru na
odrzucie z~pary podgrzewanej laserowo. Ślady powidoków unosiły się
irytująco przed nią, gdy zdecydowanie odwróciła wzrok na ziemię.

Jedna z~fabryk obozu była kilkaset metrów dalej, kompleks aluminiowych
rur i~kabli światłowodowych w~mdląco wyglądającej organicznej masie
około pięćdziesięciu metrów szerokości i~dwudziestu wysokości, w~której
kabiny sterowania i~ludzkie chodniki były wybrzuszeniami i~gwintowaniem
jak jaja i~wysięki jakiegoś gargantuicznego owada. Nazwa firmy, która
była tego właścicielem, Kosmiczni Kupcy, była wypisana krętym neonem na
dachu.

Gdy zbliżała się do najbliższego osiedla robotników, uderzyło Myrę, nie
po raz pierwszy, że baraki były bardziej nowoczesne i~komfortowe niż
betonowy blok mieszkalny, w~którym mieszkała. Każdy barak był
półcylindryczny, jego zaokrąglone końce opływowe wobec dominujących
wiatrów, pokrycie z~smoło\dywiz czarnego poliwęglanu z~rzędami okien
oszklonych diamentowymi płytami.

Ta szczególna grupa baraków była dwoma rzędami po dziesięć, z~wyjeżdżonymi resztkami wybrukowanej, szerokiej na dwadzieścia metrów,
drogi pomiędzy nimi. Grupa kilkunastu mężczyzn była zaangażowana w~naprawę drogi, bryza niosła podmuchy potu i~smoły. Mężczyźni używali
łopat, palnika gazowego w~urządzeniu rozgarniająco\dywiz wysypywującym i~kaszlącego walca dieslowego: prymitywny, ciężki sprzęt. Na poboczu
wylegiwał się strażnik Ochrony Wzajemnej w~niebieskim kombinezonie,
grzebiąc w~zębach i~widocznie oglądając przedstawienie na siatkówkach
oczu, słuchając muzyki lub komentarzy w~uszach.

Wyłonienie się cienia Myry sprawiło, że podskoczył, mrugnął i~potrząsnął
głową z~drobnym dreszczem. Zaczął się podnosić.

-- Nie ma potrzeby wstawać -- powiedziała niegrzecznie Myra. -- Chciałam
porozmawiać z~kilkoma ludźmi.

-- Są na przerwie, Obywatelko -- powiedział, mrużąc na nią oczy. -- Więc to
zależy od nich, dobra?

-- Dobra -- powiedziała Myra. Praca fizyczna liczyła się jako wypoczynek.
To umysłowa praca projektowania i~monitorowania wyczerpywała nerwy
skazanych.

Odwróciła się do mężczyzn, którzy jej pomachali, krzyknęli powitania i~wyjaśnienia: musiała poczekać kilka minut, które zajęłoby im zakończenie
układania i~walcowania jakiegoś świeżo wylanego asfaltu. Nie oferując
strażnikowi, zapaliła Marleya i~pozwoliła mężczyznom zakończyć przerwę w~swoim czasie. Zawsze nalegała, żeby jej przyjazdy i~inspekcje liczyły
się jako czas pracy dla pracowników.

Jej nastrój się polepszył, gdy zadziałały Wirginia i~Maroko. Pracownicy
mieli swoje żółte kombinezony zwinięte do pasa i~pocili się mimo tego,
że temperatura była nieco powyżej punktu zamarzania. Większość z~nich
była młodsza -- spójrzmy prawdzie w~oczy, \textit{znacznie} młodsza -- niż
ona. Ciemno\dywiz opaleni Koreańczycy i~Japończycy, umięśnieni jak uczniowie
sztuk walki, którym rzeczywiście niektórzy byli. Lubiła ich obserwować,
efekt dymu wzmacniający podstawowy ton pożądania, szczęścia, radosny
szum hormonów\ldots

Jednak to przypomniało jej Georgiego i~jej nastrój znowu się zepsuł.
Georgi nie żył. Czasem wydawało się, że każdy mężczyzna, którego
wyruchała, zmarł. To jakby przenosiła chorobę: Niall MacCallum zmarł w~wypadku samochodowym, Jaime Gonzalez zmarł -- kiedy? -- \textit{siedemdziesiąt} lat temu w~wojnie z~Contras\footnote{ wojna w~Nikaragui
pomiędzy prawicową partyzantką Contras, a~lewicowym Frontem Wyzwolenia
Narodowego, zob.~\url{https://pl.wikipedia.org/wiki/Contras} -- przyp.tłum.}, Jon Wilde zmarł w~jej ramionach na poboczu drogi do
Karagandy (na śniegu, który stawał się czerwony, gdy on stawał się biały),
a teraz Georgi Dawidow zmarł na atak serca w~konsulacie w~Ałma-Acie.
(Uważali, że \textit{ona} w~\textit{to} uwierzy?)

Byli też inni, przypomniała sobie. Całkiem niedawno inni. To nie tak, że
każdy mężczyzna, którego ruchała, był skazany, to był każdy mężczyzna,
którego kiedykolwiek \textit{kochała}. Był tylko jeden wyjątek, o~którym
wiedziała. Wszyscy jej mężczyźni nie żyli, prócz jednego, a~on był
zabójcą.

Nawet, prawdopodobnie, zabójcą Georgiego. \textit{Pierdolony atak serca,
blada dupa!} To był jeden z~ich ruchów, musiał być, ruch w~końcówce.

Drzwi gdzieś uderzyły i~ulica nagle zaroiła się od dzieci gnających i~wrzeszczących, ich języki i~akcenty tak zróżnicowane jak kolory ich
skóry. Kilka z~przymusowych pracowników było kobietami, ale wielu
mężczyzn zabrało kobiety. Zachęcano nawet więźniów, żeby sprowadzali
swoje rodziny. Było to humanitarne, ale też polityczne: mężczyzna z~kobietą i~dziećmi raczej nie ryzykowałby ucieczki lub buntu.

Otoczona przez dzieci wołające ojców, wciskające palce w~gorący
asfalt, zbierające się wokół maszyn i~głośno dociekające, grupa dorosłych w~końcu
skończyła, zostawiając strażnika, żeby uważał na świeżo zrobioną drogę.
Myra rozkoszowała się jego niezadowolonym spojrzeniem, gdy zmiażdżyła
niedopałek pod obcasem i~wyszła na środek niewyasfaltowanej części
ulicy.

-- Cześć, chłopaki.

Wszyscy wiedzieli kim była, ale jedynymi pośród nich, których
rozpoznała, było członkami komitetu obozu, Kim Nok-Yung i~Shin Se-Ha.
Pierwszy był młodym koreańskim stoczniowcem, krępym i~twardym. Drugi był
japońskim matematykiem smukłej budowy i~o czujnym obliczu. Kim złapał
jej dłoń, uśmiechając się szeroko.

-- Witaj, Myro.

-- Dobrze Cię widzieć, Nok-Yung. I~Ciebie, Se-Ha.

Japończyk pochylił głowę. 

-- Cześć. -- Nalegał na zabranie jej sakw. Cała
grupa ją otoczyła, błyskając oczami i~zębami, rozmawiając ze sobą i~z nią bez przykładania wagi do wzajemnego zrozumienia. Odpędzili dzieci i~poprowadzili ją do najbliższego baraku. Błona wejściowa otarła się o~nią, wybuchła prysznicem kropelek o~zapachu antyseptyków i~wróciła do
kształtu za nią. Mrugnęła szybko i~zrzuciła jej ciężki płaszcz,
zakładając go na jeden z~rzędu haków, które wyrastały z~zakrzywionej
ściany.

Jej pierwszy głęboki oddech był wystarczającym dowodem jak efektywna
była błona filtrująca we wstrzymywaniu pyłu. W~tym samym czasie,
zarumieniła się jej skóra, gdy jej system odpornościowy rzucił się badać
to nano, którym oddychała, występujące właściwie tylko we wnętrzu tego
budynku. Poszła za Kimem do jadalni, przewiewnej przestrzeni płaskich
mebli, niektórych ostrzegawczo czerwonych, wskazujących, że są do
podgrzewania, innych białych do jedzenia. Krzesła były wyłożone czarnym
plastikiem z~poliwęglanu. Dookoła na ścianach, ustawione na półkach, lub
złożone na podłodze, były tysiące książek: wieki klasyki, bestselerów,
przebojów i~podręczników, jak gdyby zdmuchnięte z~czterech stron świata
i zatrzymane na tej zaporze. Tak samo byłoby w~każdym innym baraku.
Kolejnym wspólnym elementem bałaganu były instrumenty muzyczne, sprzęt
rzemieślniczy i~wyniki: statki kosmiczne w~butelkach, wyszukanie
wyrzeźbione, drewniane zabawki.

Gdy usiedli dookoła stołu, Myra poczuła się nieswojo i~na krawędzi. Wyjęła
,,opaskę na oczy'', półksiężyc szeroki na pół centymetra z~półprzezroczystego plastiku z~włosów i~nasunęła je na skronie, przed
oczy. Wiadomość dryfowała po jej siatkówce. 

-- Nanoprotect56 wykrył w~pokoju następujące znane molekuły inwigilacyjne: Datafag, Hackendice,
Reportback, Merkury, Mołdawskie. Czy chcesz usunąć?

Mrugnęła, gdy kursor zatrzymał się na opcji ,,Kontynuuj'', wzięła
głęboki wdech i~przytrzymała, aż zaczęły boleć jej płuca, potem
wypuściła powietrze. Twarze dookoła stołu były niezaciekawione i~rozbawione.

-- Czyszczenie w~trakcie -- zaraportował wyświetlacz na siatkówce. Myra
wzięła głęboki wdech. Tym razem czuła chłód, jak również gładkość.

-- Więc mamy prywatność -- powiedział jeden z~Koreańczyków, z~silną
ironią.

-- Och, jebać to -- powiedziała Myra. -- Zdarza się za każdym razem. Musisz
założyć, że słuchają. -- Musiało być coś jeszcze, co jej programy nie
wyłapywały: wyobrażała sobie maleńkie maszyny Turinga\footnote{ abstrakcyjny
model urządzenia służącego do wykonywania algorytmów,
zob~\url{https://pl.wikipedia.org/wiki/Maszyna\_Turinga} -- przyp.tłum.} tykające, wszywające wibracje dźwięku w~długie łańcuchy
cząsteczek w~kurzu. Wyjęła z~kieszeni rejestrator, większy i~mniej
zaawansowany niż ten w~jej wyobraźni, i~położyła go na stole. 

-- I~słucham. Zatem, co dla mnie macie?

Szybka wymiana spojrzeń dookoła stołu zakończyła się jak zwykle
zaakceptowaniem Kim Nok-Yunga jako rzecznika. Zaszeleścił papierem z~wewnętrznej kieszeni i~przesunął palcem po protokole. Nowe Sprawy
zaczynały się od rutynowych pierwszych pytań.

-- Jakiś postęp w~uznaniu jeńców wojennych?

Myra była dotknięta nutą nadziei, z~którą zadawał to pytanie, setny raz,
bez zmiany od pierwszego. Zacisnęła usta i~potrząsnęła głową. 

-- Przepraszam, chłopaki. Czerwony Krzyż i~Półksiężyc pracują nad tym, oraz
Amnesty. Nadal nic.

Nok-Yung wzruszył ramionami. 

-- Och, dobra. Proszę, standardowo
zaprotestuj.

-- Oczywiście.

Gdy przeszli przez listę skarg, warunków, przydziałów i~wypłat, Myra
zauważyła, że cały wzorzec produkcji w~obozie się zmienił. Intensywność
pracy i~wielkość produkcji drastycznie wzrosły. Dwadzieścia silników i~sto modułów habitatów ukończonych przez Kosmicznych Kupców w~zeszłym
miesiącu! Nok-Yung i~S-Ha subtelnie podkreślali zmiany ostrożnymi
spojrzeniami i~zmianami w~tonie, ale wyraźnie nie komentowali.

Myra rozejrzała się po stole, kiedy dotarli do końca agendy. Nikt nie
skarżył się na przyśpieszenie. Nie wydawali się zmartwieni. W~powietrzu
była aura tłumionego podniecenia, prawie radości, gdy czekali na jej
przemowę. Sprawdziła raz jeszcze liczby w~głowie i~zrozumiała
wstrząśnięta, że przy tym tempie większość z~tych mężczyzn odpracuje
swoje grzywny -- lub ,,długi'' -- w~ciągu miesięcy zamiast lat.

Kolejny ruch w~końcówce. Myra skinęła lekko głową i~się uśmiechnęła. 

-- Cóż, to wszystko -- powiedziała. -- Nie przepracowujcie się, chłopaki.
Naprawdę. Upewnijcie się, że macie dużo dróg do naprawienia, ok?

Więźniowie tylko się uśmiechnęli ze współdzielonej tajemnicy. Sięgnęła do
sakw, jakby coś właśnie sobie przypominając. 

-- Przywiozłam trochę
książek dla was.

Mężczyźni pochylili się z~zapałem, gdy rozpakowywała. Nie mieli prawa do
jakiegokolwiek interfejsu z~siecią i~nic nie mogłoby wykorzystane do
zbudowania takiego: żadnych telewizorów, komputerów, czytników, czy
sprzętu do VR, czy do muzyki. Nic nie mogło powstrzymać Myry od
przyniesienia czegokolwiek, by chciała -- sakwy były prawnie pocztą
dyplomatyczną -- ale każdy elektroniczny lub molekularny przemyt zostałby
skonfiskowany tuż po jej wyjściu. Zatem musiały to być książki.
Więźniowie i~ich rodziny pragnęli ich nienasycenie. Każda wizyta Myry
dodawała więcej do tej tendencji.

Tym razem miała tuziny książek w~miękkiej oprawie z~gustownymi okładkami
sztuki nowoczesnej i~szarymi grzbietami, Klasyka 20 wieku -- Harold
Robbins, Stephen King, Dean Koontz i~tak dalej -- które przesunęła przez
stół ku mężczyznom, których nazwisk nie znała. Dla jej przyjaciół
Nok-Yunga i~Se-Ha zachowała najlepsze na koniec: książki w~twardej
oprawie tak stare, że tylko zaawansowane środki ochrony zachowywały je
od rozpadnięcia się w~kurz\ldots

Prawie tak jak ją, pomyślała, gdy książki przechodziły jedna po drugiej
z jej sękatych dłoni: niesamowicie rzadka, prawdopodobnie unikalna,
kopia Stirnera w~redakcji Tuckera, \textit{Przenośny Nietzsche} Vikinga i~zniszczona edycja \textit{Pierwszych Zasad} Spencera z~Biblioteki
Myślicieli.

Kim Nok-Yung spojrzał na nie z~szacunkiem, potem na nią. Shin Se-Ha był
w jakimś transie. Nok-Yung pokręcił głową.

-- To za wiele -- powiedział, prawie gniewnie. -- Myra, nie możesz\ldots

-- Och tak, mogę.

-- Skąd je dostałaś? -- spytał Se-Ha.

Myra wzruszyła ramionami. 

-- Od Reida, dość zabawne.

Wszyscy ludzie patrzyli teraz na nią, z~kwaśnymi uśmiechami.

-- Od \textit{Davida} Reid? Właściciela? -- Kim machnął ręką, wskazując
wszystko w~zasięg wzroku.

-- Ta -- powiedziała Myra. -- Tego samego.

Nastąpiła chwila trzeźwej ciszy.

-- Dobra -- powiedział w~końcu Nok-Yung -- mam nadzieję, że wykorzystamy je
lepiej niż on, bękart.

Wszyscy się roześmiali, nawet Myra.

-- Ja też -- powiedziała.

Rozsiadła się na krześle, puściła wokół paczkę Marleyów i~zaakceptowała
ofertę kawy.

-- Dobra, chłopaki -- powiedziała. -- Wiadomości. Wszystko ciągle zmierza
do piekła. -- Skrzywiła się. -- Tak jak w~zeszłym tygodniu. Kilka zmian na
frontach, to wszystko. Zaufajcie mi, niedużo was mija.

-- Kilka zmian na \textit{których }frontach? -- zapytał podejrzliwie Se-Ha.

-- Och -- powiedziała Myra. -- Jeżeli musicie wiedzieć\ldots front
północno-wschodni jest\ldots aktywny.

Kolejna cicha wymiana spojrzeń i~uśmiechów. Myra nie dzieliła ich
przyjemności, ale nie mogła ich za to winić. Dwa wkraczające wydarzenia,
które ją przerażały najbardziej, były dla nich, każde w~inny sposób,
zadatkami ich wcześniejszego wyzwolenia.

Pożegnała się, zastanawiając się, czy nie po raz ostatni, zabrała teraz
puste sakwy i~wyszła przez ulice obozu rekompensaty, wsiadła na konia i~wyjechała bramą, w~kierunku miasta.

Myśląc o~Reidzie, próbując pomyśleć spokojnie i~destrukcyjnie o~Reidzie,
jej myśli zdryfowały. Nie zawsze był takim chujem. Był pierwszą osobą,
która powiedziała jej, że nigdy nie będzie musiała umrzeć. To było
osiemdziesiąt trzy lata temu, kiedy miała dwadzieścia lat. Nie uwierzyła
mu\ldots

\textit{Śmierć za mną podąża\ldots}

~

-- Nie musisz umierać -- powiedział jej.

Czarne włosy okalały jego twarz, czarne brwi jego intensywnego,
brązowookiego spojrzenia. Dave Reid był ciemny i~przystojny, ale nie,
niestety, wysoki. Nosił dżinsową kurtkę z~cynowym guzikiem -- odznaką,
jak je nazywali Brytole -- przypiętą do wyłogu. Odznaka była czerwona z~czarnym ,,młotem, sierpem i~czwórką'' Międzynarodówki.

-- Co! -- Myra się roześmiała. -- Wiem, że tak to się czuje, wszyscy w~naszym wieku czują się w~ten sposób, prawda? Jednak do każdego to
dotrze, człowieku, nie oszukujmy się.

Oparła się na łokciach w~trawie i~spojrzała w~niebieskie, wiosenne
niebo. Było zbyt cholernie zimno na to, ale słońce wyjrzało zza chmur,
ziemia była sucha i~to w~Szkocji wystarczało na kąpiele słoneczne.
Trawiasty stok za budynkiem Boyd Orra był pokryty grupkami i~parami
studentów, pijących, palących, dyskutujących. Prawdopodobnie
opuszczających wykłady, była już druga po południu.

-- Poważnie -- powiedział Dave, z~tym akcentem z~Highland, który niósł
dźwięk wiatru w~trawie, fal na brzegu -- jeżeli uda ci się dożyć
dwudziestego pierwszego wieku, masz bardzo dużą szansę żyć wiecznie.

-- Kto tak mówi? L. Ron Hubbard\footnote{założyciel Kościoła Scjentologicznego
i twórca dianetyki, autor opowiadań science fiction,
więcej~\url{https://pl.wikipedia.org/wiki/L.\_Ron\_Hubbard} -- przyp.tłum.}?

Dave parsknął. 

-- Właściwie Arthur C. Clarke\footnote{ brytyjski prozaik,
pisarz fantastycznonaukowy, propagator kosmonautyki, m.in. na podstawie
jego koncepcji powstały stacje orbitalne,
zob.~\url{https://pl.wikipedia.org/wiki/Arthur\_C.\_Clarke} -- przyp.tłum.}.

-- Kto?

Zmarszczył brwi. 

-- Wiesz, naukowiec, futurolog. Człowiek, który wymyślił
satelity komunikacyjne.

-- Ach, \textit{ten} -- powiedziała pogardliwie Myra. -- Science-fiction.
\textit{2001} i~tak dalej. -- Zobaczyła lekkie skrzywienie na twarzy Davida
i kontynuowała: -- Och, nie zrozum mnie źle. Nie mówię, że to nie
możliwe. Może za sto lat od teraz, może w~komunizmie. Jednak nie za
naszego życia. Trudna sprawa.

Dave wzruszył ramionami i~skręcił kolejnego papierosa.

-- Zobaczymy.

-- Pewnie tak. A w~tym tempie Twoje palenia tych rzeczy, będziesz miał
farta, jeżeli dożyjesz dwudziestego pierwszego wieku. Nawet nie dotrzesz
do pierwszej bazy.

-- Och, wytrzymam kolejne dwadzieścia cztery lata. -- Westchnął,
wydmuchując dym na lekko ciepłą bryzę, potem uśmiechnął się złośliwie. -- Chyba że zostanę męczennikiem rewolucji, oczywiście.

-- ,,Ze Śmiercią mam dziś rendez-vous Na jakiejś spornej
barykadzie,''\footnote{ Ze śmiercią mam dziś rendez-vous, Alan Seeger,
tłum.~Anna Bańkowska, Czytelnia, NowyNapis.eu, 2020,
zob.~\url{https://nowynapis.eu/czytelnia/artykul/ze-smiercia-mam-dzis-rendez-vous}
-- przyp.tłum.} -- zacytowała Myra. -- Nie martw się. To kolejna rzecz,
która się nie zdarzy w~naszych czasach.

Cień wysokiego budynku wkradł się na twarz Dave'a. Przesunął się
zręcznie, znowu na słońce.

-- Tak właśnie myślisz, co?

-- Ta, tak właśnie myślę. -- Uśmiechnęła się i~dodała z~ironicznym
zapewnieniem -- To jest w~naszych \textit{naturalnych życiach}.

Dave podniósł torbę wypchaną kopiami rewolucyjnych gazet i~czasopism. 

-- To jaki jest sens tego wszystkiego? Dlaczego po prostu nie jemy, pijemy
i nie bawimy się?

Myra łyknęła z~puszki MacEwana, opuściła ją i~spojrzała na niego znad
jej krawędzi. 

-- To właśnie \textit{robię}, kochany.

Zrozumiał, sięgnął i~pogłaskał krzywiznę jej policzków. 

-- Ale jednak -- nalegał. -- Dlaczego przejmować się polityką, jeżeli nie sądzisz, że
wygramy?

-- Dave -- powiedziała -- nie jestem socjalistką, ponieważ spodziewam się
pewnego dnia kierować jakąś własną demokracją ludową. Robię to, co
robię, ponieważ uważam, że to właściwe. Ok?

-- Ok -- powiedział Reid, uśmiechając się, ale jego uśmiech był
rozbawiony, jak również czuły, jakby była naiwna. Zirytowana nie wiedząc
do końca dlaczego, odwróciła się.

~

Miasto nazywało się Kapica\footnote{od Piotr Kapica -- 
fizyk radziecki, zajmował się m.in. fizyką plazmy i kontrolowaną fuzją jądrową. Odkrywca zjawiska nadciekłości helu, za co otrzymał
Nagrodę Nobla w~1978,
więcej~\url{https://pl.wikipedia.org/wiki/Piotr\_Kapica} -- przyp.tłum.} i~było stolicą Międzynarodowej Republiki
Robotników Naukowo \dywiz Technicznych, która nie miała innego miasta,
a właściwie, oprócz obozów, żadnego innego ludzkiego osiedla. MRRNT była
niezależną enklawą na krawędzi Poligonu\footnote{dosł. Teren Testowy w~Semipałatyńsku, Semipałatyńsk-21 -- pierwsze miejsce przeprowadzania
testów w~Związku Radzieckim,
zob.~\url{https://en.wikipedia.org/wiki/Semipalatinsk\_Test\_Site}
-- przyp.tłum.}, nieużytków pomiędzy Karagandą a~Semejem, odpad po dziedzictwie testów jądrowych w~Kazachstanie. Dawno
temu, Kapica wyglądała nowocześnie, z~centrum wysokich budynków
biurowych, wewnętrznym pierścieniem automatycznych fabryk, peryferiami
zakurzonych, ale zadrzewionych ulic i~nieruchomościami niskich budynków
mieszkalnych, ruchliwym portem lotniczym tuż za granicą i~pracowitym
kosmodromem na horyzoncie, z~którego dzień za dniem wznosiły się potężne
statki. Teraz to był Pas Rdzy\footnote{ potoczne określenie terenów w~północno-wschodniej części Stanów Zjednoczonych, które uległy
dezindustrializacji,
zob.~\url{https://pl.wikipedia.org/wiki/Pas\_rdzy} -- przyp.tłum.}, osobliwie przestarzały jak japońskie fabryki samochodów,
stocznie Szkocji lub równiny pszenicy na Ukrainie.

Myra, jednakże, jakoś się cieszyła, gdy klacz prowadziła ją w~lekkim
ruchu ulicznym w~środku dnia. Jabłonie kwitły, wszystkie ściany wyglądały
świeżo, kolorowe murale kwiatów, gwiazd, statków, ludzi, dzieci,
bohaterek lub bohaterów. Naprawdę starożytne rzeczy wieku kosmosu, efekt
wzmocniony przez młodszych, naprawdę młodych, ludzi cieszących się
chłodnym słońcem w~modnych skąpych ubraniach, które przywoływały lata
sześćdziesiąte dwudziestego wieku w~ich wesołym futuryzmie. Patrzyła na
dziewczyny w~wąskich trykotach, błyszczących jaskrawych minispódnicach i~zaczęła się zastanawiać czy było im zimno\ldots prawdopodobnie nie, ubrania
były tylko imitacją nylonu czy poliestrów oryginałów, nanoprodukowane
tkaniny pożyłkowane wymiennikami ciepła ze wplecionymi maszynami
molekularnymi.

Jasne ubrania nadawały ludziom na ulicach wygląd pomyślności, ale Myra
była zbyt świadoma, że było to sztuczne. Ubrania były tańsze niż papier,
łatwo osiągalne nawet dla Opieki Społecznej. Przez ostatnie kilka lat, z~nadejściem diamentowych statków, rynek ciężkich rakiet upadł, a~bezrobocie wystrzeliło. Zasiłek był wypłacany przez jej ministerstwo z~czynszu Ochrony Wzajemnej i~nie mógł trwać wiecznie. Turystyka
nostalgiczna -- stary kosmodrom był teraz na Liście Światowego
Dziedzictwa, o~ile to było warte -- wyglądała jak jedyne obiecujące
źródło zatrudnienia.

Zanim się zorientowała, koń się zatrzymał, z~przyzwyczajenia, na
zewnątrz skromnego dziesięciopiętrowego budynku biurowego z~betonu
należącego do rządu Republiki przy Placu Rewolucji. Myra przez chwilę
siedziała nieruchomo, patrząc krzywo na wywieszony w~tym tygodniu na
billboardzie plakat podnoszący morale: wielkie czarno-białe zbliżenie na
klasyczną fotografię Gagarina z~TASS, kiedy uśmiechał się z~hełmu
kosmonauty. Pamiętała tamten czas, w~klasie szkoły podstawowej na Lower
East Side, kiedy pierwszy raz zobaczyła tę ludzką twarz i~utworzyła
jakieś połączenie synaptyczne pomiędzy uśmiechem Gagarina i~spojrzeniem
Guevary.

Kosmos i~socjalizm. Co to było za oszustwo. Potrząsnęła lejcami,
wprowadziła klacz wolnym krokiem na tyły, rozjuczyła, wytarła gnój z~butów i~weszła po schodach. Korytarze do jej biura -- z~frontu budynku,
jak przystało na Ludowego Komisarza do spraw Polityki Społecznej, p.o.
Premiera i~(gdy teraz o~tym pomyślała) p.o. Prezydenta -- były wypełnione
szmerem pośpiesznych kroków i~szybko zanikających szeptów. Myra
spojrzała ostro na grupę, którą mijała, ale niewielu było skłonnych
odwzajemnić spojrzenie.

Zamknęła drzwi do biura z~marnym, ale satysfakcjonującym uderzeniem.
Niech aparatczycy\footnote{ etatowy, zawodowy funkcjonariusz Komunistycznej
Partii Związku Radzieckiego lub aparatu rządowego Związku Radzieckiego, ktokolwiek, kto sprawował jakąkolwiek funkcję
biurokratyczną lub polityczną,
więcej~\url{https://pl.wikipedia.org/wiki/Aparatczyk} -- przyp.tłum.} martwią się o~nastrój, jakby ona miała się martwić o~ich
nastrój. Ostatnim razem, gdy czuła taką wymijającą atmosferę w~korytarzach to było tuż przed pierwszym i~ostatnim razem, kiedy straciła
władzę, w~2046. Wtedy, podejrzewała nadchodzące działanie ze strony
firmy Ochrony Wzajemnej i~ich podopiecznych w~obrębie aparatu państwa:
\textit{coup d'état.}\footnote{ fr. zamach stanu -- przyp.tłum.} Teraz,
podejrzewała, że Ochrona Wzajemna i~sojusznicy rozpoczynali ostatnie
działania w~znacznie większym planie, tak wielkim, jak tylko mógłby być:
\textit{coup du monde.}\footnote{ fr. zamach na świat -- przyp.tłum.} Lub
\textit{coup d'étoile}\footnote{ fr. zamach na gwiazdy -- przyp.tłum.}.

Podeszła do okna, zrzucając płaszcz, czapkę i~rękawiczki szybkimi,
gwałtownymi ruchami, oparła się pięściami o~parapet i~rozejrzała się po
otoczeniu w~ataku poczucia terytorium w~zagrożeniu. Żadne czołgi czy tupiące
nogi nie rozbrzmiewały na ulicach jej miasta, żadne czarne helikoptery
nie łoskotały na niebie jej kraju. Czego oczekiwała? Były jeszcze dni na
ucieczkę, zanim cokolwiek się wydarzy, i, kiedy to się stanie,
otwierające uderzenie ujawniłoby się w~większych stolicach niż jej, nowe
dźwięki CNN ukąsiłyby ją w~pierwszych sekundach nowego porządku.

Westchnęła i~się odwróciła, podniosła upuszczone ubrania i~odwiesiła je
ostrożnie na właściwe wieszaki na chromowanym stojaku. Biuro było
nieśmiało retromodernistyczne jak styl na ulicy, nieco bardziej
wyrafinowane boazerie sosnowe i~podłogi, skórzane warstwy na obu,
ornamenty ze stali i~srebra, hebanowe i~plastikowe globy i~pojazdy
międzyplanetarne.

Opadła na fotel biurowy i~odchyliła się, pozwalając mu wymasować jej
ramiona i~kark. Nasunęła ,,opaskę'' na oczy, włączyła wyświetlacz i~przesunęła oczy, żeby go przestudiować. Oprogramowanie antywirusowe
migotało na jej siatkówce, ale nie było nic niestosownego w~powiadomieniach. Tutaj, jak we wszystkich biurach, ściany miały zęby.
Jej własny software otoczył ją, jego lojalność tak intymna, i~tak trudna
do obalenia, jak wzmocniony system odpornościowy w~jej krwi. Był to
,,personal'', był to \textit{jej },,personal'', unikalna konfiguracja
programów, która skanowała świat i~odpowiedzi Myry na świat, i~z tej
interakcji budowała przebiegłą ocenę jej potrzeb i~zainteresowań. Wyszukiwał informacje dla niej i~pilnował jej inwestycji. Robiło to
sieciom światowym to samo, co silnik wyszukiwarki Sterlinga robił dla
jej Biblioteki, wyszukiwał i~wyciągał to, co istotne z~przepastnego i~wzburzonego morza danych, w~których większość ludzi pływała lub,
znacznie częściej, tonęła.

Posiadanie dobrego zestawu osobistego oprogramowania było nieco
ważniejsze dla nowoczesnego polityka niż tradycyjne sieci osobistego
wpływu i~wywiadu. W~dekadzie od odzyskania władzy, Myra upewniła się, że
jej sieci -- obu typów, wirtualne i~faktyczne -- były silne i~splecione,
wystarczająco silne, żeby ponieść ją, jeżeli struktura państwa
kiedykolwiek znowu ją zawiedzie. Choć nawet było to mało prawdopodobne,
jej czystki, choć bezkrwawe, były tak bezlitosne jak Tito. Żaden
urzędnik MRRNT nigdy nie popełni najmniejszego błędu, gdzie leżą jego
najlepsze interesy, i~żaden pracodawca lub przedstawiciel Ochrony
Wzajemnej nie będzie próbował tego zmienić.

Wkrótce będzie musiała się skonsultować z~resztą SowNarKomu -- spotkanie
było zaplanowane na godzinę piętnastą -- i~zebrać biegających podwładnych
z korytarzy, żeby je przygotowali, ale najpierw chciała samodzielnie
uchwycić sytuację.

Personal Myry nie miał osobowości, o~ile wiedziała, ale był osobą:
rewolucjonistą, giełdowym spekulantem, handlarzem bronią, szpiegiem,
wolnym, ważnym oszustem, konspiratorem, komunisto\dywiz kapitalistą prosto z~koszmaru nazistów. Miał imię.

-- Parvus\footnote{ działacz rosyjskiego i~niemieckiego ruchu socjalistycznego,
zob.~\url{https://en.wikipedia.org/wiki/Alexander\_Parvus} -- przyp.tłum.} -- wyszeptała. Projektory siatkówkowe w jej ,,opasce''
zbudowały obraz wielkiego mężczyzny w~luźnym garniturze i~koszuli
rozciągniętej na jego brzuchu jak wypełniony żagiel, pędzący na wichurze
informacji. Ruszył ku niej, uśmiechając się, jego kieszenie wypchane
dokumentami, jego dłoń z~papierosem machający, gdy przygotowywał się,
żeby jej coś powiedzieć. Nigdy nie trafiła na nagranie prawdziwego
Parvusa w~działaniu, ale nadała temu obraz historycznego lidera
trockistów i~manieryzmy szalonego naukowca od innego, którego kiedyś
całą mowę wysiedziała, dawno temu w~Związku Studentów w~Glasgow.

-- Pokaż mi cały obraz.

Parvus skinął głową. Przebiegł palcami po kłakach białych włosów,
zmarszczył czoło, maniakalnie się uśmiechnął.

-- \textit{Jane's}\footnote{ współcześnie międzynarodowe przedsiębiorstwo
specjalizujące się w~przetwarzaniu publicznie dostępnych informacji
(biały wywiad) w~zakresie militarnym, bezpieczeństwa narodowego czy
transportu,
zob.~\url{https://en.wikipedia.org/wiki/Janes\_Information\_Services}
-- przyp.tłum.}, zdaje się. -- Strzepał centymetr popiołu, wyczarował
ekran. Jej wzrok skupił się na opcji, mrugnęła i~pokój zniknął z~jej
widoku, znowu, a~Ziemia się odsunęła.

Jej pierwszy wirtualny widok, kręcący się na orbicie, był z~\textit{Siły
Rynkowe Jane} -- jawnie dostępną, ale zaporowo drogą analizą ruchów
wojskowych na całym świecie w~czasie rzeczywistym. Pracowała na
przedostatnim wydaniu, obecnie w~fazie testów. Kosztowało to skromny
budżet obronny Republiki niewiele więcej niż stypendium dla 
patriotycznego kazachskiego absolwenta do równie ubogiego wydziału
informatycznego Sztokholmskiego International Peace Research
Institute\footnote{ instytut założony w~1966 roku przez szwedzki parlament dla
uczczenia 150. rocznicy nieprzerwanego pokoju w~Szwecji,
zob.~\url{https://pl.wikipedia.org/wiki/Stockholm\_International\_Peace\_Research\_Institute}
-- przyp.tłum.}. (To oraz niewykrywalna karta kredytowa do jej konta
łącznościowego). Myra, dawno zaznajomiona z~konwencjonalnymi symbolami i~ideogramami, oglądała to wszystko na poziomie abstrakcyjnym: zakodowane
kolorami, grafy ze strzałką w~przestrzeni 3D, inne wymiary wskazywane
subtelnymi cieniami i~okresem pulsowania. Ten świetlny filigran wisiał
jak skomplikowany system chmur ponad relatywnie nieruchomymi
histogramami opisującymi sprzęt i~zasoby ludzkie. Fizyczne lokalizacje i~liczby personelu czy \textit{materiel} były określane jedynie na
podstawowym poziomie rozumienia bilansu światowych wojsk, tak jak
pozycja fizycznej fabryki było jedynie zgrubną oceną stanu światowych
rynków. Sekunda po sekundzie, rynek i~siły wojskowe przesuwały się
nieprzewidywalnie, ich wzajemne przenikanie bardziej złożone niż
jakakolwiek ideologia kiedykolwiek przewidziała. Przy większości
światowych oficjalnych armii będących rewolucyjnymi lub najemniczymi,
lub oboma naraz, a~większością konfliktów załatwianych, zanim się jeszcze zaczęły, przez
bezdyskusyjne symulacje, wszyscy od bankierów
przez generałów aż do szeregowych w~polu wzruszyliby ramionami i~zaakceptowali wirtualny werdykt i~odpowiednio zmienili strony, wsparliby
lub uciekli wraz z~ich software'owym cieniem -- wszyscy prócz Zielonych i~Czerwonych. \textit{Oni} naprawdę walczyli, brali to na poważnie.

To było jak w~tej starej grze ,,Cywilizacja''\footnote{ seria strategicznych
gier turowych polegających na rozwijaniu wybranych cywilizacji,
zob.~\url{https://pl.wikipedia.org/wiki/Civilization\_(seria)}
-- przyp.tłum.}, czasem myślała Myra, z~nowym scenariuszem:
Barbarzyństwo II. Nikt nie zamierzał zejść z~tablicy, nikt nie leciał na
Alpha Centauri. Wszyscy zmierzali razem, w~mrok\ldots Gdy tylko
wystarczająco wielu dużych graczy zdecyduje się zakwestionować bezsporne
i uruchomi symulacje dla audytu wojny.

Niemniej jednak, teraz, mrok był pełen zakrzywianego światła. A w~realnym świecie, mrugającym jako tło, jeden front był bardziej niż
wirtualny i~bliżej niż by chciała. Za północną granicą Kazachstanu,
która była setki kilometrów na północ od MRRNT, poszarpana linia
frontowa Związku Chińsko\dywiz Radzieckiego posuwała się w~błyskach
prawdziwego ognia: potyczki guerilli i~sabotaże po jednej stronie,
ostrzał artyleryjski dalekiego zasięgu bez przekonania i~bezcelowe
bombardowania dywanowe po drugiej.

Szinosowy -- nazwa była subtelnie uwłaczająca, jak Wietkong dla
Narodowego Frontu Wyzwolenia Wietnamu Południowego czy Jankesi dla
Narodów Zjednoczonych -- byli pierwszym autentycznym zagrożeniem
komunistycznym w~tym wieku, którzy naprawdę wierzyli w~ich
zaktualizowaną wersję ideologii, którą demoludy takie jak MRRNT
parodiowały jako postfuturystyczny pastisz. Opierając się na
zapomnianych prowincjach zbuntowanych kołchozów i~fabrykach zajętych
przez robotników, uparcie przeżywający dekady kontrrewolucji i~wojny,
uzbrojeni przez popierające ich oddziały dezerterów (nazywających
samych siebie, nieuchronnie, ,,lojalistami'') z~armii posowieckiego Wschodu i~po-chińskiej Północy, utrzymywali większość Mongolii, Syberii i~nawet
części północno\dywiz zachodnich Chin od Jesiennej Rewolucji z~2045 roku, a~w~następnych latach rozprzestrzenili się po stepie jak porost. Myra
zarówno nie cierpiała ich, jak i~podziwiała.

Z bardziej bezpośrednich, i~frustrujących, problemów: Szinosowiety byli
poza wirtualnym światem, wyrwana czarna dziura w~sieci. Ich komputery
zawsze były offline. Ich aktyw nie handlował kontraktami terminowymi na
walkę. Odrzucali wszystkie symulowane konfrontacje lub negocjacje. Jak
margines Zielonych na Zachodzie i~Zieloni Khmerzy na Południu, Czerwoni
na Wschodzie poddawali wszystko praktycznej próbie, krytyce broni. Nawet
w \textit{Jane's} mogli tylko zgadywać ich obecny układ.

Jednak ich ząbkowana południowo\dywiz zachodnia krawędź była dostatecznie
określona, jak zawsze wcinała się bliżej jej dziedziny niż ostatnim
razem, gdy sprawdzała. Przykładowo, wczoraj o~tej porze\ldots

Westchnęła i~odwróciła uwagę od komunistów do śledzenia mroczniejszych
czynów prawdziwej, międzynarodowej konspiracji: Ruchu Kosmicznego.
Gdzieś w~tej punktowanej ciemności, czytając pomiędzy liniami światła,
musiała odszukać ślady większej i~bardziej obszarpanej armii,
niecierpliwej do zajęcia świata.

Jej pierwszy krok -- przyjęty przez system z~zaskoczoną wdzięcznością -- była aktualizacją wyników obozów pracy Ochrony Wzajemnej. Kiedy były
zintegrowane i~wiarygodnie zmapowane na cały globalny archipelag
przedsiębiorstw, pierwsze przybliżenie re-ewaluacji wagi zależności
wojskowo\dywiz przemysłowych posłało fale przez całą sieć. Dobrze, że
pracowała z~osobistą kopią, pomyślała krzywo Myra. To była informacja,
za którą można by zabić (choć jednocześnie, przypuszczalnie, przeceniona
przez samą Ochronę Wzajemną, która musiała wiedzieć, że ona wiedziała).

Przesłała spekulacyjną aktualizację z~tagowaniem błyskającym ,,ważne''
do Ludowego Komisarza Finansów, oraz jako mniej ważne podsumowanie do
towarzysza od Obrony. Potem uruchomiła akta ,,aktywności w~toku ruchu''
kosmicznego, połączyła je z~nowymi liczbami wyjściowymi, i~wysłała je
wszystkie do komisarzy z~jej własną interpretacją.

,,Zamach Ruchu Kosmicznego'' był omawiany, otwarcie, tak długo, że
stawał się nierealny, tak nierealny jak Rewolucja, póki w~końcu się nie
wydarzyła. Myra sama wieszczyła kasandrycznie o~zamachu, kiedyś. Jednak
teraz czuła się zrehabilitowana. I, znowu, David Reid był zamieszany.

Jej były kochanek rozbudował Ochronę Wzajemną ze spółki usług ochrony
pomocniczej dla firmy ubezpieczeniowej do globalnego biznesu, który
działał w~odszkodowaniach: kryminaliści pracowali, żeby zrekompensować
straty, jakie wyrządzili. Oryginalnie, reklamowana jako ludzka,
kierowana rynkiem, reforma i~zastąpienie dawnego barbarzyńskiego systemu
więziennego, jej rozszerzenie ze zwykłych kryminalistów do więźniów
politycznych i~jeńców wojskowych po Jesiennej Rewolucji nadało temu
przerażającą, niepowstrzymaną logikę niekontrolowanej ekspansji, w~podobny sposób co użycie pracy więziennej w~Pierwszym Planie
Pięcioletnim zrobiło dla oryginalnego GUŁag\footnote{ podmiot zarządzania
obozami pracy przymusowej w~ZSRR, w~którym więźniami byli zarówno
przestępcy kryminalni, jak i~osoby uznawane za społecznie niepożądane
lub politycznie podejrzane,
zob.~\url{https://pl.wikipedia.org/wiki/Gu\%C5\%82ag} -- przyp.tłum.}.

Teraz przez ponad dekadę, ci po przegranej stronie małych wojen i~narastająco drobnych przestępstw zapewniali siłę roboczą dla
gigantycznego boomu osadnictwa kosmicznego, stosując dowolne
umiejętności, które posiadali -- lub mogli szybko się nauczyć -- żeby
spłacić swoje długi przestępcze tak szybko, jak to możliwe. W~tym samym
czasie, rozmnożenie enklaw ruchu kosmicznego, każda podżegała hordy
oblegających barbarzyńców lub roje wściekłych biurokratów, zapewniało
niekończący się napływ nowych skazanych. Dość duży odsetek więźniów, po
ukończeniu spłaty, łapał liczne możliwości zatrudnienia, które oferowały
projekty kosmiczne.

Ochrona Wzajemna była teraz zwornikiem globalnej koalicji
przedsiębiorstw obrony, agencji kosmicznych, programów osadnictwa w~kosmosie, kampanii politycznych i~gospodarzem mniejszych rządów, wiele z~nich stworzonych przez te same firmy. Koalicja Ruchu Kosmicznego była w~punkcie zebrania wystarczających sił, żeby odtworzyć stabilny światowy
rząd i~przywrócić kontrolę Narodów Zjednoczonych nad stacjami
orbitalnymi byłej Obrony Kosmicznej. Ich celem, dawno dyskutowanym, było
wycofanie opozycyjnych ruchów środowiskowych i~antytechnologicznych i~przesunięcie wystarczającej siły roboczej i~kapitału na orbitę Ziemi,
żeby stworzyć samopodtrzymującą się obecność w~kosmosie, która mogłaby
uciec dowolnej oczekiwanej katastrofie poniżej, z~których, Bóg wie, było
wiele do wyboru.

Sam zamach miał się odbyć na dwóch poziomach. Jeden był politycznym
posunięciem, żeby przejąć kierowanie ReONZ, głosami licznych minipaństw,
które mogłyby obalone, podporządkowane lub przekonane. Innym było
posunięcie wojskowe, w~ten sposób usprawiedliwione, żeby zdobyć stare
stacje orbitalne USA/ONZ. To, Myra liczyła, było przyczyną
przyśpieszenia w~obozach pracy. Bez wątpienia masowa detronizacja
odbywała się pośród personelu wojskowego na orbicie, ale z~powodu natury
sprawy nie było wiele, o~czym mogłaby się dowiedzieć.

Patrzyła na wirtualny ekran długo, póki zaciskanie jej pięści, drgające
grymasy twarzy i~mruganie od łez nie zmyliło programu na tyle, że się
wyłączył i~zostawił ją patrzącą na ścianę.

SowNarKom -- Rada Ludowych Komisarzy lub, w~bardziej konwencjonalnej
terminologii, Rząd -- był odpowiednio małym rządem dla prawie
niemożliwego małego państwa (populacja 99 854, ostatnim razem, gdy komuś
się chciało policzyć, i~spadała z~dnia na dzień). Struktury MRRNT były
ćwiczeniem z~obozu socjalistycznego, modelowanym na tych starych
sowieckich republikach, ale bez kierującej roli Partii. Wynikiem tego
strategicznego pominięcia była demokracja tak prawdziwa jak jej
inspiracja była fałszywą. Lub tak się wydawało, w~dostatniejszych dniach
republiki.

Myra przybyła wcześnie i~skorzystała z~przywileju pierwszego przyjazdu,
siedzenie przewodniczącego, na szczycie długiego, pustego stołu
pokrytego bliznami z~mahoniu, z~niezgrabnym, odpornym na wybuch,
urządzeniem sekretarskim w~centrum. Obok stało kolejny tuzin siedzeń, po
sześć po obu stronach stołu, każda z~jej tradycyjną wodą mineralną i~notatnikiem przed krzesłem. Pokój był goły, bez okien, ale oświetlony
płytami pełnowidmowymi w~suficie. Jedyną dekoracją białych ścian była
oprawiona fotografia dawno zmarłego fizyka nuklearnego, po którym zostało
nazwane miasto.

Walentyna Kozlowa weszła, jej wojskowy mundur jak zawsze elegancja, jej
włosy niechlujne, jej ręce pełne wydruków. Była po pięćdziesiątce,
ciągle młode dziecko wieku, wystarczająco młode i~wystarczająco
szczęśliwe, żeby otrzymać terapię przeciw starzeniu, zanim stała się
stara. Uśmiechnęła się spięta i~usiadła. Potem Andriej Muchartow, krótko
obcięty blondyn, około czterdziestki, tak wyglądający -- prawdopodobnie
umyślnie -- trzeźwo konwencjonalny w~trzyczęściowym garniturze z~elektrycznie błękitnego surowego jedwabiu. Denis Gubanow, młodszy niż
inni, ostentacyjnie nieformalny, nieogolony, wyglądał, jakby właśnie
przyszedł z~sondowania informatora w~jakimś marnym barze kosmodromu.
Aleksander Sherman przyszedł ostatni, sprawiając wrażenie, że został
oderwany od ważniejszych spraw. Jego modny, pseudoplastikowy kombinezon
był bez wątpienia tylko pracą dla stanowiska, ale Myra nie lubiła go
nawet bardziej niż jego. Usiadł i~się rozejrzał, jakby oczekując, że
spotkanie natychmiast się rozpocznie, potem zacisnął usta i~przesunął
dwie kartki papieru do Myry.

-- Obawiam się, że więcej rezygnacji -- powiedział. -- Tatiana i~Michael
właśnie\ldots

-- Odeszli na bogatsze pastwiska -- powiedziała Myra. -- Słyszałam. -- Spojrzała na puste miejsca dookoła zubożonego stołu i~wzruszyła
ramionami. -- Dobra, zgodnie z~rewolucyjną konwencją nie ma czegoś
takiego jak spotkanie bez większości, więc\ldots

-- Naprawdę musimy dokooptować nowych członków! -- powiedział Sherman.

-- Tak -- powiedziała sucho Myra. -- Naprawdę musimy.

Jej ton odrzucił Aleksandra. 

-- To hańba, nie mamy Komisarza do spraw
prawa, spraw wewnętrznych czy\ldots

-- Tak, tak -- przerwała Myra. -- I~połowa jebanych członków Rady
Najwyższej spierdoliła, przypadkowo ta gorsza połowa. \textit{Ja} nie
mogłam znaleźć kompetentnego komisarza dla \textit{czegokolwiek} pośród
reszty. W~tym tempie, nie będziemy mieć wystarczającego
\textit{elektoratu}, żeby uzupełnić ranki. Więc co sugerujesz?

Aleksander Sherman otworzył usta, zamknął je i~wzruszył ramionami. Jego
buntownicze spojrzenie przekonało Myrę, że będzie następny, jako
Komisarz do spraw Przemysłu już miał właściwe powiązania.

-- Ok, towarzysze -- powiedziała Myra -- rozpocznijmy spotkanie. -- Zdjęła
swoją opaskę i~położyła ją formalnie na stole, a~ci, którzy jeszcze tego
nie zrobili, podążyli za jej przykładem. Nie była to do końca reguła,
ale był to zwyczaj -- gest uprzejmości jak również zapewnienie, że każdy
będzie uważał -- żeby odłożyć personale na bok na czas spotkania. Myra
nigdy nie mogła się zdecydować, czy to było wzajemne zaufanie, czy
wzajemne podejrzenie, które stało za zwyczajem nierobienia tego samego z~bronią osobistą. Nikt nigdy nie wyciągnął broni na spotkaniu SowNarKomu,
ale były precedensy\ldots

-- Nagrywanie: start. Zwyczajne spotkanie Rady, piątek, dziewiątego maja
dwa tysiące pięćdziesiątego dziewiątego roku, Myra Godwin-Dawidowa
przewodniczy, pięciu członków obecnych. -- Rozejrzała się i~spojrzała się
na stalową siatkę rejestratora. -- Wnoszę, żebyśmy odłożyli porządek
obrad i~przeszli prosto do posiedzenia nadzwyczajnego. Zaczynając od
śmierci Obywatela Dawidowa.

Brak sprzeciwu. Minęły sekundy ciszy.

-- Nie wszyscy naraz -- powiedziała.

Walentyna Kozlowa (Obrona) przemówiła pierwsza. 

-- Dobra, Myra\ldots
Towarzysz Przewodnicząca, wszyscy rozmawialiśmy z~Tobą o~śmierci
Georgiego. Było nam bardzo przykro usłyszeć o~tym.

Myra skinęła głową. 

-- Dziękuję.

-- To powiedziawszy, musimy zdecydować o~naszej politycznej odpowiedzi.
Teraz, oczywiście policja w~Ałma-Acie prowadzi śledztwo i~jak na razie
nie wykryto oznak przestępstwa. -- Wzruszyła ramionami. -- To, oczywiście,
jest trudne do wykazania, w~dzisiejszych czasach. Jednakże\ldots Georgi
Jefrimowicz odpowiadał za bardzo dużo -- wskazała niejasno na Andrieja
Muchartowa, Komisarza do spraw Międzynarodowych -- \ldots i~w tych
okolicznościach, przyczyny naturalne wydają się prawdopodobne.

Myra westchnęła. 

-- Tak, doceniam to. I~doceniam, co wszyscy mi
powiedzieliście. Do protokołu powiem, że osobiście nie akceptuję, że
śmierć Georgiego nie była zamachem.

Zmierzyła się z~wynikłym zamieszaniem.

-- \textit{Jednak} -- kontynuowała -- nie proszę i~nie oczekuję od żadnego z~was, żebyście traktowali to więcej niż podejrzenie. W~tej chwili, nawet
pytanie, kto mógłby skorzystać z~tego, jest bardzo niejasne, jeżeli
Georgi był zamordowany, to mógł być przez jedną lub drugą stronę.
Możliwe, że pewne elementy w~Ruchu Kosmicznym widziały w~nim przeszkodę
w ich\ldots dyplomacji. Możliwe, że pewne siły przeciwne Ruchowi
Kosmicznemu myślały, że tak myślimy i~zamordowały go jako prowokację.
Lub może, tylko może, jego serce się poddało. Cokolwiek\ldots stało się to
w złym okresie.

Muchartow chrząknął w~zgodzie.

Po chwili ponurej ciszy, Walentyna znowu przemówiła. 

-- Wszyscy
zapoznaliśmy się z~Twoją wiadomością -- powiedziała. -- Jaki jest Twój
sugerowany kierunek działania?

-- Próbujemy ich powstrzymać, oczywiście. Cholera, jeżeli chcę pierdolony
ONZ z~powrotem nad nami, nie mówiąc już o~kontrolowaniu przez przeklęty
ruch kosmiczny i~jego pełnomocników.

Walentyna pochyliła się do przodu. 

-- Dla mojej części -- powiedziała -- zgadzam się z~Twoją oceną. Musimy być gotowi na nową sytuację, w~której
Ruch Kosmiczny kontroluje ReONZ, a~z~tym stacje orbitalne Obrony Ziemi.
Jednak \ldots -- zawahała się chwilę, westchnęła prawie niezauważalnie i~kontynuowała -- \ldots myślę, że śmierć Georgiego, zrozumiałe podejrzenia,
które wzbudził i\ldots hm\ldots nieoczekiwany i~nieautoryzowany wzrost
produkcji obozów pracy mogły nadać Twojej odpowiedzi\ldots subiektywny
element. -- Kozlowa rozejrzała się dookoła stołu. -- Nadchodzące
przesunięcie w~równowadze sił nie może być zatrzymane przez nas lub
przez kogokolwiek innego. Większość tego, co możemy zrobić, dzięki
dyplomacji Georgiego, to pomóc Kazachstanowi pozostać neutralnym, z~naciskami przeciwko wrogiemu przejęciu. Nawet oni nie podjęliby
bezpośredniego działania przeciwko temu, choć Bóg wie, że Georgi
próbował ich przekonać. Zapewnili nas, że nie mieli siły politycznej, ja
im wierzę. Teraz wydajesz się sugerować, że \textit{my} rzucimy naszą
wagę, taką, jaką jest, przeciwko im. Moje osobiste zdanie jest takie, że
uzyskamy więcej, pozostając neutralnymi. To mogłoby działać na naszą
korzyść, \textit{jeżeli} dopasujemy się do nowej rzeczywistości we
właściwym czasie.

Myra rozluźniła twarz. 

-- Dostać się na zwycięską stronę, to masz na
myśli? -- zasugerowała lekko.

-- Tak, dokładnie -- powiedziała Kozlowa. Wydawała się zachęcona
odpowiedzią Myry, lub brakiem odpowiedzi. -- W~końcu -- orała dalej -- wszyscy jesteśmy częścią Ruchu Kosmicznego, współpracujemy z~nimi od
dawna, a~Szinosowiety są takim samym zagrożeniem dla nas jak barbarzyńcy
i jak wsteczne rządy są dla niektórych innych enklaw. Szczerze, myślę,
że powinniśmy wysłać jakieś dyplomatyczne czujki na drugą stronę przed
przełomem, który, jak poprawnie wskazujesz, jest kwestią dni lub
tygodni. A my nie jesteśmy obecnie na pozycji siły w~tej chwili. Więc
nasza decyzja w~istocie jest dość pilna.

-- Interesujące -- wymamrotała Myra. -- Ktoś jeszcze?

Denis Gubanow (Bezpieczeństwo Wewnętrzne) ostro przerwał. 

-- Przewodnicząca mówiła w~wiadomości o~państwach podporządkowanych lub
obalonych. Myślę, że nie powinniśmy pozwolić sobie na zostanie jednym z~nich! Niezależnie od retoryki, i~propagandy nieuchronności, jest
oczywiste, co się dzieje. Imperializm otrzymał potężne uderzenie po
upadku Jankesów, ale uderzenie nie było ostateczne, pechowo. Kapitał
monopolistyczny zawsze znajdzie nowe polityczne narzędzia, a~Ruch
Kosmiczny, tak zwany, dowiódł, że jest wspaniałym narzędziem. -- Krótko
parsknął. -- Dosłownie, pojazd startowy! Przez nie, bogaci opuszczają
Ziemię. Dlaczego mielibyśmy pomagać im na ich drodze?

-- Bardziej do \textit{rzeczy} -- powiedział Sherman (Handel i~Przemysł) okazując dobitnie jasno pogardę na retorykę Denisa -- jest
pytanie, z~czego będziemy się utrzymywać, kiedy obozy się wyczerpią.

-- Moglibyśmy zawsze\ldots -- zaczęła Kozlowa, jakby chciała coś powiedzieć
żartem, potem spojrzała na Myrę i~się zamknęła.

-- Co?

-- Nie. Zapomnij. Sprawa bieżąca to, co robimy teraz, w~sprawie zamachu.
-- Myra pozwoliła biec dalej dyskusji. Racja była, przyznała samej
siebie, po obu stronach. Niemniej jednak Walentyna miała rację,
odpowiedź Myry była subiektywna. Głównym elementem Ruchu Kosmicznego
była Ochrona Wzajemna, a~głównym elementem Ochrony Wzajemnej był David
Reid. Jeżeli Ruchowi Kosmicznemu się uda, byłby najpotężniejszym
człowiekiem na świecie.

Nie ma mowy, żeby pozwoliła bękartowi wygrać.


\chapter{Szif w Stoczni}

Godzinę później, po biegu przez miasto, który był cholernie trudny w~(i
na) butach, szybkim prysznicu i~przebraniu się w~ubrania roboczowe,
stałem na przystanku autobusowym z~moim stalowym kaskiem ochronnym w~jednej dłoni i~aluminiowym pudełkiem z~jedzeniem w~drugiej. Pakowanie
mojego lunchu było jedyną dodatkową usługą, którą właścicielka
mieszkania zapewniała, ale mnie to wystarczało, żeby wybaczyć brak
śniadań, kolacji, prania i~solidnej gorącej wody.

Narastające ciepło słońca wypalało poranną mgłę nad loch i~pomiędzy
wzgórzami. Czułem się, jakbym mógł w~każdej chwili unieść się i~odpłynąć. W~oczach miałem piasek, w~mózgu czułem gorąc, ale te niewygody
nie pomniejszały rodzaju blasku uniesienia gdzieś w~mojej piersi i~brzuchu. W~dziwny sposób ledwo mogłem znieść myśli o~Merrial, za każdym
razem sprowadzało to taką eksplozję radości, że drżały mi kolana i~prawie bałem się oddawać się temu w~nadmiarze. Chciałem to zachować,
zgromadzić, wydzielać to sobie, kiedy naprawdę potrzebowałbym, a~nie
połknąć to na raz. (Co oczywiście jest błędem, ta szczególna studnia,
jak wiele innych, nie ma dna).

Zamiast tego myślałem o~innej kobiecie, Wyzwolicielce, pod której
pomnikiem poznałem Merrial, i~pod której odległą i~starożytną ochroną
ona i~jej lud żyli. (Chronieni od prześladowań, w~każdym razie, jeżeli
nie uprzedzeń).

Przez ostatnie cztery lata, Historia była jedną ze sztuk, którą starałem
się opanować. Nie było to łatwe, nawet w~Glaschu, gdzie historia zalewa
to miejsce, jak to mówią. Zdumiona awersja wyrażona przez Merrial była
dostatecznie powszechną reakcją. W~czasach tak wielu możliwości, w~miejscu pulsującym innowacjami na tak wielu polach, które mogły być
zastosowane do oczywistego ulepszenia ludzkości, wydawało się przewrotne
(czasem nawet dla mnie), by energiczny, inteligentny młodzieniec
odrzucił takie sztuki jak Literatura, Muzyka, Kinematografia lub nauki:
Astronomię, Medycynę, wiele działów Naturalnej Teologii, odszedł od
doskonalących zajęć Praktycznej Filozofii, Inżynierii Mechanicznej i~Lądowej, odwrócił się od tych wszystkich użytecznych prac intelektu,
nawet nie dla zrozumiałego, lub, w~obrębie rozumu, chwalebnego czaru
biznesu i~przyjemności, ale ku przeszukiwaniu butwiejących dokumentów i~walących się ruin, oraz wypełnianiu umysłu krwawymi obrazami i~otępiającymi liczbami milionów śmierci w~przeszłości.

Była to niesmaczna i~lekko haniebna fascynacja, z~odrobiną nekrofilii,
nawet nekromancji. Niemniej, czy chcemy, czy nie, wszyscy jesteśmy
historykami, każdy z~własnym zarysem historii w~głowie. To było zdanie,
które często przedstawiałem sceptycznym słuchaczom, od rodziców i~rodzeństwa przez komitety mecenatu aż do przyjaciół i~kolegów w~debatach
napędzanych alkoholem. Podejmujemy zarys od rodziców, nauczycieli,
kaznodziejów, z~pieśni, pomników i~historii.

Na początku, Bóg zrobił Wielki Wybuch i~stało się światło. Po pierwszych
czterech minutach, stała się materia. Po miliardach lat, pojawiły się
gwiazdy, planety i~Ziemia została uformowana. Wody ponad niebem
rozdzieliły się od wód poniżej nieba, co wydało wszelkiego rodzaju
pełzające stworzenia. Po milionach lat, zostały ukształtowane
niewidzialną ręką Boga, Doborem Naturalnym, w~wielkie potwory ziemi i~morza. Ziemia wypełniona była przemocą i~Bóg posłał asteroidę, Granicę
KaTe \footnote{ oryg. Katy Boundary -- od K-T Boundary czyli Granica K-T -- warstwa rozgraniczająca osady kredy od paleogenu, odznaczająca się
skokowym spadkiem bioróżnorodności i~anomalną zawartością irydu. Jest
ona śladem wielkiego wymierania, które zakończyło erę mezozoiczną, ok.
66 milionów lat temu,
zob.~\url{https://pl.wikipedia.org/wiki/Granica\_K-T} -- przyp.tłum.}, żeby ją zniszczyła. Niebo było ciemne w~południe przez
czterdzieści dni i~prawie wszystkie stworzenia zostały zniszczone.
Pośród tych, które przetrwały były małe bestie jak myszy i~uzupełniły
Ziemię, zagrzebały się w~niej i~stały się królikami, wspięły na drzewa i~stały się małpami, zeszły z~drzew i~stały się ludźmi \ldots

\ldots małpoludami, jaskiniowcami, Egipcjanami, Babilończykami, Grekami,
Rzymianami, Amerykanami, Chińczykami i~Rosjanami. Amerykanie upadli, ale
ich imperium przetrwało jako Posiadanie, aż Wyzwolicielka powstała na
Wschodzie i~je obaliła. Nastąpiły ciężkie czasy, a~potem pokój.

\textit{Więc dlaczego je zakłócać, odpowiedz mi na to chłopie!}

Ponieważ prawda jest bardziej interesująca i~ostatecznie bardziej
pouczająca niż mikstura bajek? Nabyłem apetytu nie tylko na prawdę, ale
i na szczegół. Na szczególną przyjemność, która pojawia się w~ujrzeniu
prawdziwej relacji pomiędzy wydarzeniami w~terminach przyczyny i~skutku
niż konwencji narracyjnej. To satysfakcja, którą będę bronił jako
prawdziwie naukową.

\textit{Ale jaki} z~tego \textit{pożytek}, co?

Na to nie miałem gotowej odpowiedzi, oprócz określenia wyniku jako
sztuki, w~ten sam sposób, jak metoda mogła być zdefiniowana jako nauka.
Argument, że ci, którzy nie uczą się z~historii, są skazani na jej
powtarzanie, zawodził wobec większości ludzi, przekonanych, że nie
istnieje ryzyko powtórzenia żadnych bardziej niszczycielskich błędów
historii. Więc musiałem sięgnąć do argumentu, że prawdziwa historia
opowiadała lepsze historie, ponieważ były prawdziwsze, że rzeczywistość
ma swoje własne piękno, surowsze i~wyższe niż mit.

Szczególną historią, którą chciałem opowiedzieć, była ta o~życiu
Wyzwolicielki. Moja propozycja pracy dyplomowej o~jej wczesnych latach
jako studentki i~akademiczki w~Glasgow, długo, zanim stała się postacią
znaną historii, było tylko początkiem mojej ambicji podbicia świata: by
zrekonstruować, o~ile ktoś może przez przepaść czasu, umysł, osobowość i~okoliczności, które ukształtowały przyszłość, które były teraz naszą
przeszłością.

To mogło zabrać dekady badań, lata pisania. Cokolwiek innego robiłem, ta
biografia zdefiniowała moje: życie za Życie. Może to było nieświadome
uchylanie się od tej ceny, może wymyślona, samo usprawiedliwiająca
się próba opłacenia mojego długu wobec czegoś, co moi bardziej
praktycznie nastawieni współcześni nazywali ,,prawdziwą pracą'' lub coś
bardziej pozytywnego, mętne uczucie przyciągania do świata materialnych
dążeń i~mierzalnych sukcesów, zwrócenie się ku przyszłości i~od
przeszłości, które doprowadziły mnie tego lata do Carron Town i~Stoczni
w Kishorn\footnote{ stocznia taka istnieje
zob.~\url{https://en.wikipedia.org/wiki/Loch\_Kishorn} -- przyp.tłum.}.

-- Dzięki Bogu już czwartek -- powiedział wesoły głos za mną. Odwróciłem
się i~uśmiechnąłem do Jondo, który opierał się o~znak przystanka i~zajadał kanapkę z~czarnym puddingiem i~smażonym jajkiem. Za nim liczba
pracowników teraz ustawiała się w~kolejce. Sprzedawcy przegryzek,
napojów i~gazet pracowali wzdłuż tej linii.

-- Jest piątek -- wskazałem.

-- To właśnie miałem na myśli -- powiedział z~pełnymi ustami, machając
pozostałością swojego śniadania. -- Siła przyzwyczajenia. -- Przełknął. -- W~każdym razie, dzień wypłaty.

Skinąłem głową entuzjastycznie. Pół mojej wypłaty przesyłałem
telegraficznie na moje konto w~Kaledońskim Banku Spółdzielczym. Z~reszty
musiałem opłacić kwaterę, jedzenie i~drinki, oraz odrobinę hulanki na
cotygodniowym jarmarku. W~piątkowe poranki miałem wystarczająco
pieniędzy, żeby przetrwać dzień. Wypłata była wysoka, ale też koszty
życia. Projekt podniósł ceny na kilometry dookoła.

Jondo był mężczyzną w~moim wieku, jego brzuch piwny już tak samo
imponujący jak jego mięśnie. Jego długie czerwone włosy, teraz jak
zwykle w~kucyku, i~jego jasne oczy i~brwi dawały mu wygląd paradoksalnie
niewinnego pirata. Odziedziczone prawdopodobnie po jego przodkach
wikingach, którzy przybyli na tę ziemię, żeby grabić i~osiąść na farmie,
a ku którym chrześcijaństwo przyszło jako w~istocie dobra nowina, mile
widziana ulga od pogańskich nieubłaganych kodeksów honoru i~zemsty.
Mówił z~miękkim akcentem z~Inverness, gdzie -- była taka plotka -- ciągle
jeszcze żyli chrześcijanie.

Próbowałem sobie wyobrazić Jondo pijącego wino w~jakiejś mrocznej
ceremonii. Chwilowy absurdalny obraz musiał sprowadzić uśmieszek na moją
twarz.

-- Co jest takiego śmiesznego, Clovis? -- warknął. Potem się uśmiechnął,
zgniatając papier śniadaniowy i~wyrzucając, wycierając tłuszcz z~rąk w~tłuste uda jego fartucha. -- Ach, wiem. Dobra noc z~Twoją dziewczyną
majsterków, co?

-- Można tak powiedzieć.

-- Aye, dobra, każdemu wedle jego, zdaje się -- powiedział, tonem
wypowiadania ważnej i~oryginalnej obserwacji. -- Jest autobus.

Autobus, już w~połowie pełny, zatrzymał się koło nas w~chmurze wydechu z~gazu drzewnego, hamulce piszczące i~jęczące koło zamachowe. Wszedłem,
zapłaciłem szeląga\footnote{oryg. groat -- archaiczna nazwa na monety w~Szkocji i~Anglii -- przyp.tłum.} kierowcy i~usiadłem na siedzeniu przy
oknie. Jondo rzucił swoje ciało koło mnie, posłał mi kolejny sprośny
uśmiech i~mrugnięcie, wypuścił najwidoczniej satysfakcjonujące
pierdnięcie i~natychmiast zasnął.

Niektórzy pasażerowie zajęli się gazetami lub rozmową, ale większość
drzemała jak Jondo, lub patrzyła zaczerwienionymi oczami jak ja.
Rozbieżność pomiędzy tradycyjnym czterodniowym tygodniem i~bardziej
wymagającym harmonogramem Projektu zredukowała piątkową pracę do
rozwiązywania problemów zostawionych przez cały tydzień i~przygotowania
się do następnego. Nawet zachęta podwójnej stawki nie mogła sprawić,
żeby więcej niż garstka siły roboczej wkroczyła w~świętość soboty i~niedzieli, choć mogła sprawić, że większość z~nas pracowała po godzinach
przez cały tydzień. Żadna ilość cierpliwego nakłaniania przez
kierowników z~podkładkami i~zbędnymi kaskami nie mogła nas przekonać do
przyjęcia czegoś, co uważali za bardziej racjonalne tempo pracy.

Autobus szarpnął i~ruszył. Zapaliłem papierosa, by rozwiać metan z~kiszek Jondo i~oparłem czoło o~przyjazny pulsujący chłód okna. Gdy
przekraczaliśmy Carron i~mijaliśmy New Kelso, przyglądałem się poza
schludne domki przedmieść, gdzie poranny dym wznosił się z~obozu
majsterków. Barwny obraz śpiącej Merrial -- rozrzucone czarne włosy,
ramię w~białym rękawie na poduszce -- rozpalił moją wyobraźnię.
Zastanawiałem się, jakie są moje szanse na zobaczenie jej przez dzień.
Nawet nie wiedziałem, w~którym biurze pracuje i~w bezładnej fantazji
znajdowałem jakieś fantastyczne wytłumaczenie, żeby odwiedzić je
wszystkie: wypracowując drogę przez budynki administracyjne i~biura
projektowe, odrzucając flirty chichoczących dziewcząt i~zamyślonych
starszych kobiet z~plakatami przystojniaków nad biurkami, aż w~końcu
wchodziłem do laboratorium inżynierii spotkać samą Merrial rozmarzoną o~mnie, przy czym moje prawdziwe przyjście byłoby namiętnie przywitanym
spotkaniem\ldots

Prawdopodobnie nie.

Moja głowa odsunęła się od okna, gdy autobus skręcił w~lewo na główną
drogę wzdłuż północnego brzegu. Wyprostowałem się, upewniając się, że
moja głowa nie uderzy w~taflę. Nawet o~tej godzinie poranka droga była
pełna ruchem dojeżdżających i~ciężkimi ciężarówkami. Autobus sapał
powoli, podejmując jeszcze więcej pasażerów w~Jeantown, innej wiosce,
którą Projekt rozbudował, jego napakowane budynki balansujące
niebezpiecznie na stokach wzgórz. We fiordzie bawiło się stado delfinów,
ich skoki powodowały westchnięcia i~sapnięcia od mniej znużonych i~sennych współpasażerów.

Potem, z~większym ścieraniem się biegów i~piskiem koła zamachowego, gdy
dodatkowe silniki elektryczne były uruchamiane, autobus skręcił w~prawo w~drogę w~górę wzgórz pomiędzy dwoma górami, An Sgurr i~Glas Bhein, które
dominowały na północnej panoramie miast na brzegu loch. Dla mnie to
zapewniało niewyczerpalny fascynujący widok na dalsze pasma wzgórz i~wody. Wszyscy inni w~autobusie kompletnie to ignorowali. Ktoś otworzył
okno, żeby wypuścić dym i~wpuścić świeżego powietrza. Wpadła pszczoła,
powodując falę podniecenia i~wymachiwania zrolowaną gazetą, zanim
wyleciała.

Ponad ostatnimi domami, ponad łąkami, zaczynały się drzewa:
dwudziestometrowe buki, potem sosny, jarzębina i~brzozy, na samą górę,
aż do skał i~piargów. Wieki temu te wzgórza były puste, prócz
naturalnych pastwisk i~wrzosowisk, przycinanych przez niesławne owce o~czarnym pysku\footnote{ rasa owcy rodzima dla Wysp Brytyjskich,
zob.~\url{https://en.wikipedia.org/wiki/Scottish\_Blackface} -- przyp.tłum.}. Jednak te same nagie wzgórza w~jakiś sposób podtrzymały
rzadkie siły guerilli jakobitów, Ligi Ziemskiej i~republikanów. Daleko
poniżej mogłem zobaczyć skalisty półwysep znany jako Wyspa, ochronne
ramię dookoła zatoki, ciągle z~małym bunkrem na szczycie. W~trakcie
Pierwszej Światowej Rewolucji, trzynastolatka zapisała się w~lokalnej
legendzie, zestrzeliwując myśliwca stealth przy pomocy granatnika
przeciwpancernego z~ładunkiem nuklearnym. W~ciasnym muzeum Jeantown
można zobaczyć jej dawną fotografię: brudny, uśmiechnięty członek kadry
Celtyckiego Wietkongu, w~pozie z~rurą rakiety wiszącą na jej ramieniu,
koło nierozpoznawalnego wraku na porytym zboczu wzgórza, gdzie do dziś
nic nie wyrasta.

Nad przełęczą i~potem w~dół w~ciemną długą dolinę, gdzie Pretendent
unikał żołnierzy Cumberlanda, gdzie Wolny Kościół nauczał
wydziedziczonych, i~gdzie, później, Armia Nowej Republiki przechowywała
swoje komputery, hardware dla wojny softwarowej z~ostatnim imperium.
Ponura dolina otwierała się na kolejną płodną równinę lasów, pól i~ostatnio wybudowanego miasta, Courthill. Za nim, na skraju fiordu,
leżała wielka blizna Stoczni w~Kishorn. Interpretacja widoku była
sztuczką dla oka -- wszystko tam, żurawie, platformy i~Statek, były
znacznie większe niż ich normalne równoważniki, jak plejstoceńscy kuzyni
dla rodzimych ssaków.

Autobus zatrzymał się przy bramie zakładu. Palisada dookoła stoczni była
skonstruowana bardziej, by chronić nieostrożnych lub lekkomyślnych od
zabłądzenia niż by chronić cokolwiek, co otaczała. Trąciłem Jondo i~wysiedliśmy na niebezpieczne, szybko poruszające się zbiegowisko
autobusów, samochodów i~rowerów. Przeszliśmy przez bramę akurat, gdy
syrena o~siódmej zaryczała. Setki, potem tysiące, robotników płynęło
przez bramę i~roiło się w~całej stoczni. Miejsce wyglądało jak łagodne
pole bitwy, pokryte kraterami, usłane pojazdami, zaśmiecone żyjącymi.
Założyłem ciężki kask na głowę i~z Jondo sapiącym koło mnie, zanurzyłem
się. Schylając i~mijając na ścieżkach, nad wykopami, pod kablami,
przeskakując ryzykowne tory kolejki wąskotorowej i~ponad zalanymi
wykopami i~wyschniętymi przepustami (odwodnienie tutaj zawsze było na
chybił trafił). Koło wywrotek i~koparek, kompresorów powietrza i~generatorów prądu, przenośnych kabin i~toalet ustawionych jakby losowo w~błocie, aż dotarliśmy do ogromnego suchego doku, który był ogniskiem
całej tej chwalebnej awantury.

Suchy dok był wielkim okrągłym wyżłobieniem w~stoku wzgórza, który
schodził do morza, setki metrów wszerz, dziesiątki metrów wzwyż. Jego
skaliste klify były stare i~zwietrzałe. Wyglądały jak jakaś robota
Natury lub Opatrzności, nawet Sprawiedliwości bijącej Ziemię przez
gniewnego Boga, ale w~rzeczywistości była to wiekowa praca Człowieka.
(Największe wrażenie robią prace lądowe starożytnych, ale może to
dlatego, że tak dużo przetrwało, wspanialsze prace niż ta rozpadły się w~rdzy i~zgniły). Żelazne śluzy, w~odpowiednio gigantycznej skali\footnote{ w~oryg. brobdingnagian, co jest odniesienie do Kraju Brobdingnag, Gigantów
z powieści Podróże Guliwera Jonathana Swifta,
por.~\url{https://en.wikipedia.org/wiki/Brobdingnag} -- przyp.tłum.}, wytrzymywały morze, choć pompy pracowały dzień i~noc,
żeby zapobiec nieuniknionemu przesiąkaniu i~zalewaniu.

W obrębie wznosiła się platforma, pewnego dnia wkrótce pływający bastion
betonu i~malowanej stali, a~na niej wznosił się Statek. \textit{Morski
Orzeł} (\textit{Iolair} -- wymawianej jakoś
,,Joleire'' w~gaelickim) wyglądała jak granat od RPG zakopany nosem
w platformie. Cztery kołnierze jak płetwy pochylały się z~centralnej
wieży, żeby przeciąć jajowatą powierzchnię kadłuba reaktora i~zbiornika
masy reakcyjnej, który miał czterdzieści metrów średnicy w~najszerszym
miejscu. Część tego ukryta przez platformę zwężała się od środka do
silnika powietrznostożkowego\footnote{w oryg. aerospike,
zob.~\url{https://pl.wikipedia.org/wiki/Aerospike} -- przyp.tłum.} głównej rakiety, dookoła której rozszerzone dysze rakiet
wysokości tworzyły zapiekły układ.

Do tej pory człapałem w~środku mojej grupy roboczej, do mnie i~Jondo
dołączył Machard, Druin, chłopaki z~Lewis -- Murdo pierwszy i~Murdo drugi
-- oraz Angelo i~Trike. Zeszliśmy zygzakiem żelaznych schodów, w~dół i~znowu w~dół, i~poszliśmy przez dno doku, rozchlapując kałuże wody deszczowej i~morskiej (niektóre były tutaj tak długo, że miały własne ekosystemy) do
drzwi u podstawy południowo-zachodniej nogi Platformy. To było jak
wchodzenie do latarni morskiej: w~górę i~dookoła wijących się schodów.
Powietrze pachniało mokrym metalem, gorącym olejem, wilgotnym betonem.
Każda powierzchnia ociekała, każdy dźwięk odbijał się echem.

Po dwóch minutach wspinania dotarliśmy na poziom wewnętrznego
rusztowania, gdzie pracowaliśmy. Schyliłem się w~drzwiach serwisowych
wewnętrznej strony nogi i~pojawiłem się na chodniku naprzeciwko turbiny
Platformy, która była za dwudziestometrową przerwą. Na naszym obecnym
miejscu pracy, dziesięć metrów wzdłuż chodnika, drabiny, więcej
rusztowań i~oszalowań znikało w~-- w~istocie wydawało się scalać z~-- nieskończoną strukturą rozpór łączących nogę wsparcia z~mocowaniem
silnika Platformy.

Nasz kontrakt na ten miesiąc polegał na skończeniu tej konstrukcji.
Kontrakt nie był elastyczny, został tylko miesiąc, zanim Platforma
wypłynie. Angus Grizzlyback, brygadzista, siedział na drewnianej palecie
postawionej na skrzyniach, żeby stworzyć stół, na którym były rozłożone
jakieś rozmontowane palniki spawalnicze, mała puszka nafty i~kilka teraz
bardzo brudnych mewich piór. Wstał i~spojrzał się na nas, odruchowo
pochylając głowę, żeby nie uderzyć czubkiem głowy o~kolejny poziom.
Można było zobaczyć białe włosy na klatce piersiowej i~przedramieniach,
które zainspirowały jego przezwisko (lub, z~tego, co wiem, nazwisko,
lokalny zwyczaj będący tym, czym był). Miał prawie dwa metry wzrostu i~sto pięćdziesiąt lat.

-- Och, dobrego popołudnia, panowie -- powiedział. -- Ufam, że dobrze się
bawiliśmy na długim wypoczynku? Zobaczmy, czy możemy wymyślić coś, żeby
zająć nasz wolny czas przez resztę dnia.

Wyciągnął plik poplamionych dokumentów z~kieszeni, gdy zbieraliśmy się
dookoła palety. Jego jasne szare oczy, pod białymi brwiami, skupiły się
na mnie.

-- I~możesz już zacząć od razu, colha Gree -- dodał.

Skinąłem szybko głową, skrzywiłem się na efekt tego nagłego, brutalnego
ruchu i~poszedłem zrobić herbatę.

Poranne spotkanie -- dwadzieścia minut siedzenia, picia herbaty i~palenia
-- było zwykłym początkiem dnia. Praca w~Projekcie była zorganizowana
przez rodzaj ekologicznej piramidy wykonawców i~podwykonawców, od
wielkiego krakena Międzynarodowego Towarzystwa Naukowego aż do samego
dołu do gorączkowo skrobiącego krylu jak ja. Angus Grizzlyback łączył
funkcje przedsiębiorcy i~brygadzisty, które częściowo się nakładały, a~częściowo uzupełniały, pracę przedstawiciela załogi (w naszym przypadku,
Jondo), który zajmował równoważne stanowisko w~równoległej piramidzie
Związku.

Rozmowy na spotkaniach, według mojego dwumiesięcznego doświadczenia,
obracały się wokół plotek, nowin dnia i~sportu. Na koniec, każdy dopijał
kubek, składał gazetę, gasił papierosa, patrzył na jakiś kawałek papieru
lub malunek w~rozlanej kawie, kiwał głową do Angusa i~szedł załatwić
jakąś złożoną pracę, która była tylko wskazana przez niezrozumiałą
aluzję. Wtedy miałem sprzątnąć bałagan, wymyć kubki, jeżeli bylibyśmy
blisko kranu i~wysłuchać Angusa literującego moje zadanie na dzień w~pojęciach odpowiednich dla nieskomplikowanych umysłów.

Dzisiejszy program był zdominowany sprawą wniosku przed Radą Dystryktu
Strathcarron, opisaną w~\textit{Wolnym Dzienniku West Highland}, że
lokalni powinni oddelegować bicie monet do regionalnej rady w~Inverfefforan. Ta niebezpieczna propozycja centralizacji nie została
dobrze przyjęta wokół palety. Została detektywistycznie przeanalizowana
przez Angusa, wulgarnie wyśmiana przez ludzi z~Lewis, gniewnie odrzucona
przez część z~Carron. Sam podkreśliłem niedawną lekcję z~historii. Kilka
lat wcześniej, podobna propozycja przeszła w~Strathclyde. Marka
glasgowiańska utraciła zaufanie publiczne, a~plan upadł, gdy roczna
inflacja sięgnęła niszczących dwóch procent. Dyskusja przeniosła się na
Narodową Ligę Piłki Nożnej, a~moja uwaga się błąkała.

Możecie zgadnąć gdzie. Tym razem, jednak, moje myśli były bardziej
racjonalne i~niepokojące, niż moje wcześniejsze radosne wspomnienia,
gorliwe przewidywania i~słodkie fantazje. Choć moja opinia o~mnie była
wysoka, nie mogłem otrząsnąć się z~wrażenia, że Merrial spodziewała się
mnie znaleźć, że znała mnie lub wiedziała o~mnie, że jej pierwsze
spojrzenie wyrażało rozpoznanie. Miłość i~pożądanie było w~tym
spojrzeniu, po obu stronach, byłem pewien, ale również byłem, choć
bardziej niejasno, pewny, że nie było to pierwsze spojrzenie. Też ją
rozpoznałem, ale nie miałem pojęcia skąd. Dla niej, było to jasne od
początku, nieukrywane, ale niewyjaśnione.

Przez chwilę -- przyznaję ze wstydem -- rozważałem sytuację, że mogliśmy
\textit{znać się w~poprzednim życiu}, cokolwiek to mogłoby znaczyć.
Natychmiast odrzuciłem idee jako głupi, babski, orientalny przesąd.
Metempsychoza (choć bez wątpienia w~obrębie władzy Wszechmocy) nie miała
miejsca w~naturalnej i~racjonalnej religii.

Zatem wylegiwałem się, łokcie na twardym drewnie surowego stołu, i~piłem
herbatę i~paliłem liście, podczas gdy moi towarzysze kłócili się o~finanse lub piłkę nożną, i~próbowałem zastosować infinitezymalną część
Rozumu do problemu, w~który moje pasje były w~pełni, i~burzliwie,
zaangażowane. Racjonalny wniosek był taki, że skoro się rozpoznaliśmy,
musieliśmy się poznać wcześniej, nie w~wyobrażonym wcześniejszym życiu,
ale wcześniejszym w~tym.

Istniała pewna liczba możliwości po mojej stronie tego równania. (Na bok
odłożyłem Merrial -- istniało wiele sposobów, na które ona, z~jej
uprzywilejowanego widoku, mogła mnie obserwować, nie będąc samą
obserwowaną i~śledzić mnie, bez wykrycia). Czy było to wyobrażalne, że
jedna z~setek twarzy, które prawie każdego dnia widziałem była jej,
wówczas niezauważoną? Wydawało się to mało prawdopodobne, jej twarz była
rodzaju, którego nie umiałbym nie zauważyć. Spojrzałbym drugi raz, i~więcej, w~tłumie tysięcy.

Czy widziałem ją, zatem, w~innym kontekście, może nawet nie na żywo? Na,
przykładowo, jakimś plakacie lub ruchomych obrazkach o~projekcie (z
których wszystkie, ze zrozumiałych powodów rekrutacji, kłamały na temat
kompletu pięknych dziewczyn)? Zastosowane miały te same obiekcje,
pamiętałbym film, \textit{miałbym} plakat.

W dalszych rozważaniach szybko wróciłem do pierwszego wytłumaczenia,
które mnie uderzyło: że się poznaliśmy, lub przynajmniej widzieliśmy, w~naszych wczesnych latach, w~dzieciństwie. Merrial, jak teraz
przypominałem sobie z~powtórnym zainteresowanie, nie wyrzekła się
wyraźnie tej możliwości, jedynie ją odrzuciła, mówiąc, że nie jest stąd.

Ani, oczywiście, i~ja. Nie było powodu, dla którego nie mogłem jej znać.
Nie mógłbym zapamiętać takiego spotkania, ale już wiedziałem, że nasze
dziecięce wspomnienia błądzą tak jak nasze dziecięce jaźnie i~tak
zawodne, i~tak możliwe niewinnego, bezwstydnego podstępu.

Podejście siłowe\footnote{ w~oryg. brute-force,
zob.~\url{https://pl.wikipedia.org/wiki/Wyszukiwanie\_wyczerpuj\%C4\%85ce}
-- przyp.tłum.} samo się narzucało: przesłuchać rodziców, braci i~siostry, splądrować rodzinne fotografie\ldots jeszcze nie. Już, świadoma
myśl, że szukałem wspomnienia, uruchomiłaby nieczułą byt, o~którym
prywatnie myślałem jako o~Bibliotekarzu. Ta część mnie zrobiłaby resztę
i sprowadziła z~powrotem zapisy, jeżeli miałyby być znalezione, bez
wątpienia za jakiś czas tak niespodziewany, jak byłby nie na miejscu,
ale mimo to mile widziany.

-- \ldots części palników? -- powiedział Angus.

Zrozumiałem, że coś mnie ominęło. Angus westchnął.

-- Wiesz, jak je złożyć, przetestować i~wyregulować?

-- Pewnie -- powiedziałem, kiwając się z~większą pewnością niż czułem.

-- Dobra, dobra -- powiedział Angus, wstając i~raźno pocierając dłonie
razem. -- Bierzmy się do tego, panowie.

Inni uśmiechali się do mnie.

-- Jaka noc musiała to być -- powiedział Murdo drugi, uruchamiając kolejną
rundę rubasznego dokuczania. Przyjąłem to za dobrą monetę, ale poczułem
ulgę, kiedy wspięli się w~strukturę konstrukcji, zostawiając mnie do
zajęcia się pracą bez korzyści nieusłyszanych instrukcji Angusa. Kilka
godzin minęło całkiem przyjemnie, jeśli niebezpiecznie, i~na porannej
przerwie na herbatę, Angus był dostatecznie zadowolony z~wyników, żeby
przerzucić mnie na jakąś blachę, dziesięć metrów wgłąb i~dziesięć w~górę. Siedziałem na wypełnionej hałasem otwartej przestrzeni konstrukcji
z tylko widocznym, gdy pracowałem, tym, co oświetlił mój własny
strumieniem z~palnika i~niczym innym na głowie.

Około dwunastej zdecydowałem się przerwać na lunch. Skręciłem palnik i~podniosłem maskę. Gdy zbierałem elementy sprzętu, żeby je zanieść,
usłyszałem głos Merrial. Mrugnąłem i~spojrzałem w~dół. Oto i~była,
patrząc na mnie spod kasku.

-- Cześć Clovis! -- krzyknęła, machając pudełkiem na obiad.

Odmachałem i~wróciłem do rusztowania, rzuciłem narzędzia, złapałem moje
pudełko i~zbiegłem na poziom doku tak szybko, że schody dzwoniły od
moich butów. Kiedy dotarłem na dno, Merrial podeszła bliżej i~czekała na
mnie. Miała na sobie standardowy kombinezon i~buty, zestaw, w~którym -- z~jej włosami związanymi z~tyłu -- wyglądała chłopięco. Jej uściski i~pocałunki na powitanie były słodkie i~ciepłe. Krawędzie naszych kasków
uderzył i~oderwaliśmy się, śmiejąc.

-- To piękna niespodzianka -- powiedziałem.

Złapała moją dłoń. 

-- Chodź -- powiedziała. -- Znam dobre miejsce.

Ruszyliśmy przez dok, przy przewidywalnych gwizdach i~zaczepkach kumpli,
wysoko nad nami. Poszliśmy dookoła szerokiej granicy platformy i~na
zewnątrz w~światło od strony morskiej. Gdy tylko minęliśmy potężne drzwi
śluzy, Merrial skręciła w~kierunku klifu, gdzie szereg półek i~stopni
utworzył niebezpiecznie wyglądające naturalne schody, po których
wskoczyła i~zwinnie weszła. Szedłem za nią, nie patrząc w~dół, aż
zatrzymała się na szerszej, trawiastej i~pełnej wrzosu półce dobre
trzydzieści metrów w~górę.

Usiedliśmy. Merrial oparła się o~skałę, a~ja, nie myśląc, zrobiłem to
samo, potem szarpnąłem do przodu, gdy znowu odkryłem podrapania i~siniaki na plecach. Z~wyciągniętymi nogami, nasze stopy były prawie na
krawędzi. Czułem się bardziej zaniepokojony na tej solidnej skale niż
kiedykolwiek na wielkich wysokościach Platformy. Po drugiej stronie
bram, przez fiord, toridońskie\footnote{geol. nieformalna nazwa grupy skał
osadowych z~okresu mezoproterozoiku, w~większości koloru czerwonego i~brązowego,
więcej.~\url{https://en.wikipedia.org/wiki/Torridon\_Group} -- przyp.tłum.} blanki Applecross rzucały wyzwanie niebu. Skala tych
sędziwych gór przyćmiewała sam Statek niczym metalową rzeźbę jakiegoś
ekscentrycznego artysty wykonaną w~ogródku w~wolnym czasie.

-- Moje miejsce -- powiedziała Merrial.

-- Jakieś miejsce -- przyznałem. -- To Ty powinnaś pracować na Platformie,
z taką głową do wysokości jak to.

-- Zostanę przy swoim wygodnym laboratorium i~późnym wstawaniu, dzięki.

Otworzyliśmy pudełka, rozłożyliśmy i~podzieliliśmy zawartość, potem
zajęliśmy się jedzeniem, oboje głodni. Przez kilka minut jedliśmy, bez
mówienia, potem Merrial dolała do kubków, zapaliła papierosa, podała mi
i oparła się o~skały.

-- Clovis, jest coś, o~co chcę się zapytać\ldots

Przerwała. Patrzyła prosto przed siebie, jakby chciała rozmawiać bez
patrzenia na mnie.

-- Co to jest?

-- Coś, co może możesz mi powiedzieć. Coś, co może nie powinieneś. Ma to
związek ze Statkiem.

To stawało się poważniejsze niż miłość.

-- Chcesz nauczyć się spawania? -- spytałem, próbując być nonszalancki.

Roześmiała się. 

-- Nie, o~historii.

-- Och. -- Machnąłem dłonią. -- W~każdej chwili. Jednak musi być wielu
bardziej wykwalifikowanych niż ja, wszystko, co znam bardzo dobrze to\ldots

Obserwowała mnie, gdy nagle do mnie dotarło.

-- Życie Wyzwolicielki?

-- Dokładnie to -- zgodziła się radośnie.

-- Jesteś pewna?

-- Jestem pewna -- powiedziała. Teraz nie odwróciła wzroku, patrzyła
prosto na mnie spojrzeniem stałym i~intensywnym, co wydało się
alarmujące.

-- Dobrze -- powiedziałem, mój umysł przygotowany. -- Naprawdę chcesz
czegoś się dowiedzieć o~Wyzwolicielce? Mogę Ci powiedzieć, co tylko
chcesz wiedzieć. Niemniej co ma to wspólnego ze Statkiem, na litość
boską?

Wzięła głęboki wdech, odwracając wzrok ode mnie i~patrząc na wysoki
statek. 

-- To dobry statek, colha Gree, i~jestem dumna, że mogę przy nim
pracować. Jednak rozważ to: to będzie pierwszy statek wyniesiony z~Ziemi
od wieluset lat. Pierwszy do czasu Wyzwolenia. Nie wiemy za dużo, co
wtedy się wydarzyło, ale wiemy, że w~kosmosie przed Wyzwoleniem byli
ludzi i~maszyny, oraz nie usłyszeliśmy od nich nic więcej od tego czasu.
Bez wątpienia wszyscy nie żyją. Dlaczego sądzisz, że tak jest?

-- Była wojna -- powiedziałem cierpliwie -- i~rewolucja. Rewolucja Drugiego
Świata, lub Wyzwolenie, jak je nazywamy. Ludzie poza Ziemią podążyli
ścieżką władzy i~upadli od Posiadania. Wyczerpali zapasy lub się
wzajemnie pozabijali, najprawdopodobniej.

-- Tak mówi opowieść -- powiedziała, tonem zmęczonym od spierania. -- Ale
co, jeżeli jest błędna? A co, jeżeli to, co wyczyściło niebo ludzi,
maszyn tak samo jak diabłów, \textit{nadal tam }jest?

-- Ach -- powiedziałem, patrząc odruchowo na czyste błękitne niebo. -- Ale
to ma Sens, ludzie kierujący projektem na pewno się tym zajmą. Dlaczego
im tego nie przedstawisz?

-- Zajęli się tym, racja -- powiedziała -- i~odrzucili. Nie ma dowodów
czegokolwiek tam górze, co mogłoby wyrządzić krzywdę Statkowi. Nie ma
dowodów, że utrata siedlisk w~kosmosie była spowodowana czymś innym niż
wspominałeś.

-- Więc dlaczego myślisz, że mógłbym wiedzieć cokolwiek o~tym \ldots -- machnąłem lekceważąco dłonią -- \ldots \textit{domniemanym}
niebezpieczeństwie?

-- Ponieważ\ldots -- W~tym momencie, przysięgam, rozejrzała się i~pochyliła
bliżej, prawie szepcząc mi do ucha. -- Istnieje stara tradycja
majsterków, lub plotka, lub wskazówka, wiesz, jak to jest ze starymi
ludźmi, że cokolwiek, co \textit{zniszczyło} w~kosmosie osady i~satelity i~tak dalej ciągle tam może być, i~że to była\ldots sprawka samej
Wyzwolicielki.

Moje usta musiały być otwarte. Mogłem poczuć, jak wysychają i~poczułem
przez chwilę mdłości i~zawroty głowy. Moje palce wbiły się w~twardą
trawę, gdy świat niebezpiecznie zawirował. Spojrzałem na nią, zemdlony,
a jednak zafascynowany wbrew sobie. Religia naturalna nie rozpoznawała
grzechu bluźnierstwa, ale to było bluźnierstwo tak blisko jak cholera. 

-- To niebezpieczne wody, Merrial.

-- Mów \textit{mi} jeszcze! -- parsknęła. -- Miałam wystarczające problemy
tylko za samą sugestię. Wszyscy myślą, że Wyzwolicielka była doskonałym
żołnierzem Boga, jak Chomeini\footnote{ szyicki przywódca duchowny i~polityczny, ajatollah, polityczny przywódca Iranu (jako Najwyższy
Przywódca) w~latach 1979--1989,
więcej~\url{https://pl.wikipedia.org/wiki/Ruhollah\_Chomejni}
-- przyp.tłum.} czy ktoś. Och, wśród mojego ludu istnieje bardziej
realistyczne nastawienie, przyznają, że popełniała błędy, ale to tylko
pośród nas. Publicznie, nie spotkasz majsterka wypowiadającego się
przeciwko niej.

Uśmiechnąłem się krzywo. 

-- Prócz Ciebie.

-- To nie jest publiczne, colha Gree. -- Przebiegła palcami po mojej
twarzy i~po ustach.

-- Musisz być tego bardzo pewna -- powiedziałem. -- Żeby mi powiedzieć.

-- Jestem dostatecznie pewna -- powiedziała. -- Jestem pewna Ciebie.

Żeby odwrócić uwagę od kotłowaniny mieszanych uczuć, które to
zapewnienie wywołało, zapytałem: 

-- Więc co to jest to, co ci mogę
powiedzieć?

-- To, co wiesz -- powiedziała. -- Zawsze myślałam, że uczeni mogą wiedzieć
więcej o~Wyzwolicielce, niż nam mówią.

Roześmiałem się. 

-- Nie ma tajemnic wśród uczonych, nie są tacy jak
majsterkowie. Wszystko, co odkryjemy, jest publikowane. Jeżeli nie
zgadza się z~tym, co większość ludzi wierzy, to ich sprawa. Ale tak czy
siak, większość ludzi nie czyta uczonych prac. I, cóż, zdaje się, że są
w tym podobni do majsterków, utrzymują własne realistyczne nastawienie
pośród swoich. To prawda, Wyzwolicielka nie była doskonałą świętą.
Jednak nie widziałem niczego, co sugerowałoby, że kiedykolwiek zrobiła
coś tak strasznego \ldots jak twierdzisz.

Skrzywiła się z~rozczarowania. 

-- Och, dobra. Może to zbyt wiele nadziei,
że coś takiego byłoby zapisane w~papierze. -- Zerwała różową koniczynę i~zaczęła odrywać zwinięte płatki jeden po drugim i~wysysać je, podała
jeden mi. Wziąłem go pomiędzy zęby, uwalniając maleńką kroplę nektaru na
język.

-- Na papierze -- powiedział zamyślony. -- Mogły być inne informacje, do
których nie możemy dotrzeć.

-- W~mrocznym magazynie?

-- Tak, cóż, jak mówiłem ostatniej nocy, jest tam, ale nie możemy
dotrzeć.

-- Ja mogę się dostać -- powiedziała Merrial od niechcenia.

-- Och, mogłabyś, co?

-- Tak -- odparła. -- Mogę załatwić ekwipunek, żeby wydostać dane z~mrocznego magazynu i~włożyć je do bezpiecznego magazynu.

-- Bezpieczny magazyn? -- spytałem, zbyt zdumiony, żeby pytać głębiej w~tamtej chwili.

-- Wiesz -- powiedziała. -- Kryształy widzenia.

-- I~skąd o~tym wiesz?

Kolejne oddalone spojrzenie. 

-- Widziałam, jak to jest robione. Przez\ldots
inżynierów idących skrótem.

-- Istnieje dobry powód, dlaczego unikamy ścieżki lewej dłoni -- powiedziałem.

-- ,,Konieczność jest własnym prawem'' -- powiedziała, jakby cytując, ale
wyrażenie pochodziło od mędrca, którego nigdy nie czytałem. -- Tak czy
inaczej, Clovis, to nie jest tak niebezpieczne, jak sądzisz.

Ciekawość napędzała mnie jak lubieżność. 

-- Jak robią to bezpiecznie?
Rysują pentagramy solą, czy co?

-- Nie -- odparła, całkiem poważnie. -- Przygotowują linie z~izolowanych
obwodów, wiesz? To właśnie ogranicza cokolwiek, co mogłoby czekać, żeby
się wydostać. Dla wzroku istnieją inne proste zabezpieczenia \ldots -- zrobiła gest odcinania w~odpowiedzi na moje zaskoczone spojrzenie -- \ldots
ale dziewięćdziesiąt dziewięć razy na sto, nie ma niczego, czym
należałoby się martwić. Tylko słowa i~obrazy. -- Roześmiała się mrocznie.
-- Czasem \textit{obce} słowa i~obrazy, muszę przyznać.

-- A za setnym razem?

-- Spotykasz demona -- powiedziała, bardzo cicho, ale stanowczo. -- W~większości przypadków, możesz go wyłączyć, zanim narobi szkód.

-- A w~innych przypadkach? -- naciskałem.

-- Wydostaje się i~zjada Twoją duszę.

Gapiłem się na nią. 

-- Znaczy, to rzeczywiście jest \textit{prawda}?

Roześmiała się. 

-- Oczywiście, że nie. Chociaż sprawia, że sprzęt wybucha
w płomieniach lub eksploduje.

-- Rozumiem, w~jaki sposób może to być ryzykowne.

Sięgnęła i~dotknęła moich ust. 

-- Ciii, chłopie, nie przeżywaj tego jak
stara kobieta. Większość rzeczy w~mrocznym magazynie jest dla nas
bezużyteczna lub zła na wiele innych sposobów od tych, o~których
myślisz. Diabelskie idee z~dawnych czasów, możesz się rozchorować od
nich, lub zechcieć się nimi podzielić, więc rozprzestrzeniają się jak
choroba.

Odchyliła się znowu, zamknęła oczy i~cieszyła się słońcem jak kot. 

-- Rozumiem, że Ty i~ja jesteśmy dostatecznie silni i~zdrowi na umyśle,
żeby być bezpiecznie od tego rodzaju rzeczy. -- Znowu otworzyła oczy i~rzuciła mi wyzywające spojrzenie.

Ścieżka mocy zawsze jest pokusą, jak powiedziała Merrial tak lekko
ostatniej nocy. Dotąd nigdy naprawdę mnie nie kusiła. Znałem
niebezpieczeństwa i~wiedziałem, że nie ma sposobu dotarcia do
niewątpliwych nagród. Teraz oferowano mi taki sposób. Mógłby skrócić o~lata czas wymagany na badania w~mojej pracy doktorskiej, mógłby nawet
dać mi przewagę w~Życiu. Pożądanie straconej wiedzy sprawiło, że głowa
zaczęła mnie łupać.

Zadałem pytanie, zanim wiedziałem, o~czym mówię. 

-- Czy chcesz, żebym ci
z tym pomógł?

Jej oczy rozszerzyły się i~rozjaśniły. 

-- \textit{Mógłbyś}? To byłoby po
prostu\ldots cudowne!

Patrzyła na mnie z~takim podziwem i~szacunkiem, że nie mógłbym wyobrazić
sobie niezrobienia tego, żeby na to zasłużyć. Jednak nawet w~moim
zakochanym zapale, żeby ją zadowolić, moje prawdziwe wątpliwości o~problemie, o~którym ona myślała, że odkryła, i~moje własne pragnienie
wiedzy i~przygody, żeby je pozyskać, nawet z~tym wszystkim, moja cała
edukacja i~naturalna ostrożność powróciła i~się zawahałem.

-- Och, Boże -- powiedziałem. -- Muszę o~tym pomyśleć.

-- Możesz zastanowić się nad tym do ósmej wieczorem? -- spytała sucho
Merrial.

-- Może. A co jeżeli powiem nie?

Spojrzała na mnie zdeterminowana. 

-- Nie pomyślałabym o~Tobie gorzej. To
nic w~tym nie zmieni.

-- Naprawdę? -- spytałem, nie z~niepokojem, ale psotnie. Już się
zdecydowałem. Zbałamuciła mnie do takiego stanu umysłu, który nie bał
się ani Boga, ani ludzi ani diabłów. -- Więc co zrobisz?

Potrząsnęła głową. 

-- Znajdę inny sposób lub, w~najgorszym przypadku,
złożę oficjalny protest do dokumentacji i~będę kontynuowała moją pracę,
jak mi kazano.

-- To przede wszystkim brzmi jak najrozsądniejszy kierunek działań.

-- Prawda -- powiedziała. -- Ale wolałabym poczuć satysfakcję z~wiedzy, że
Statek jest bezpieczny, w~ten lub inny sposób, niż mówić ,,Mówiłam wam''
po fakcie.

Nie mógłbym się z~tym nie zgodzić i~nie chciałem. To, co powiedziała,
musiało mieć jakiś głębszy wpływ na mnie, ponieważ kiedy zeszliśmy po
niebezpiecznych schodach z~wrzosowego gniazda, każde z~nas jedno
potknięcie od powitalnych ramion Darwina, w~ogóle się nie bałem.

Mój pokój był wąski i~długi, na strychu. Po gorącym dniu, pachniał mocno
starym lakierem i~ciepłą rdzą, słychać było dźwięki skrzypiącego drewna.
Świetlik skierowany na zachód wpuszczał wystarczająco światła do
patrzenia i~wystarczająco powietrza do oddychania.

Przyszedłem z~pracy i~zrzuciłem kombinezon i~koszulę, rzuciłem
tymczasowo ciężką sakiewkę na łóżko i~otworzyłem schłodzoną butelkę
piwa, którą kupiłem na przystanku autobusowym. Otworzyłem świetlik na
maksimum i~usiadłem pod nim na jedynym krześle w~pokoju, oparłem łokieć
o ramę okna, jakbym siedział przy barze. Przy moim przedramieniu małe
czerwone pajęczaki poruszały się po szarym i~żółtym poroście jak kropki
przed moimi oczami.

Merrial i~ja mieliśmy się spotkać za dwie godziny. Dużo czasu na umycie
się, ogolenie i~ubranie, na myślenie i~rozmyślanie. Prawie się skusiłem
na drzemkę, ale zdecydowałem się na nie, pomimo atrakcyjności ledwo
posłanego łóżka w~tamtej chwili. Po wypiciu piwa, napiłbym się kawy.
Zapaliłem mojego piątego papierosa tego dnia i~wyjrzałem ponad dachami w~kierunku loch, moje wysuszone ciało z~wdzięcznością absorbujące napój,
mój mózg jadący na podnieceniu z~liści.

Niepokojąca, ale pociągająca propozycja Merrial zajmowała mnie całe
popołudnie i~choć podjąłem decyzję, miałem mnóstwo wątpliwości i~lęków.
Nie byłbym pierwszym kopiącym w~mrocznych archiwach w~celach
historycznych lub inżynierskich, jeżeli o~to chodzi. Nie było to ani
przestępstwo ani grzech, ale zawsze przedstawiano mi to jako
niebezpieczny kaprys. Oraz, dla pewności, nie mogłem wymyślić dobrego
powodu na zrobienie tego, innego niż te, które motywowały mnie i~Merrial. Bez wątpienia, każdy, kto choć raz wszedł na tę ścieżkę, czuł
się tak samo o~\textit{swoich} powodach. Racjonalnie, było oczywiste,
dlaczego niebezpieczeństwa były częściej publikowane niż korzyści, ci,
którzy odnaleźli jedynie szaleństwo i~śmierć w~czarnej logice nie mogli
być niezauważeni, natomiast ci, którzy odkryli wiedzę, bogactwo lub
przyjemność dyskretnie zachowywali złowieszcze źródło dla siebie.

Jaką obłudę, zastanawiałem się, praktykowali majsterkowie, jeżeli sami
okazyjni przykładali rękę do leworęcznej ścieżki? Póki Merrial o~tym nie
wspomniała, nie podejrzewałem takich rzeczy: ale wtedy, z~wirtualnym
monopolem majsterków rozumienia białej logiki, było w~ich interesie
publicznie deprecjonować czarną. Optyczne i~mechaniczne przetwarzanie
danych i~szczególnie delikatny interfejs pomiędzy nimi -- kryształy
widzenia jak klejnoty w~błyszczącym mosiądzu maszyn obliczeniowych -- były ich specjalnością i~tajną umiejętnością. Co by się zdarzyło, gdyby
ludzie spoza ich gildii zaczęli badać ścieżkę lewej ręki na serio, jako
przedsięwzięcie publiczne niż prywatny występek, tylko niebiosa by
wiedziały. Nowe Posiadanie, prawdopodobnie, w~takim przypadku
majsterkowie mogliby zaprojektować nowe Wyzwolenie. Nie była to
uspokajająca myśl.

Zgasiłem papierosa i~posłałem niedopałek wirujący po łupkowych
dachówkach do suchej rynny. Dźwięki ludzi wracających do domu, silników,
kopyt, stóp unosiły się z~ulicy poniżej. Odwróciłem się do pokoju i~skończyłem piwo, potem rozebrałem się i~wszedłem do kabiny prysznicowej
i się umyłem. Woda stała się zimna tuż przed spłukaniem ostatnich
mydlin. Zacisnąłem zęby i~wytrzymałem, potem wyskoczyłem i~wytarłem się,
podczas gdy elektryczny czajnik gotował wodę. Napełniłem dzbanek
mieszanką ciepłej i~zimnej wody i~ogoliłem się ostrożnie, potem
nastawiłem kawę, podczas gdy się ubierałem. W~te same spodnie i~kamizelkę, jakie miałem poprzedniej nocy, ale pomyślałem, że okazja
wymaga czystej koszuli.

Łóżko było dostatecznie blisko stołu, żeby te dwa meble tworzyły w~pewien sposób nieergonomiczne biurko. Usiadłem z~kawą i~spojrzałem na
stos książek i~papierów, które przywiozłem ze sobą do czytania na lato.
Sięgnąłem i~wyciągnąłem tom ze stosu, przekląłem, wstałem i~znalazłem
szmatkę do przetarcia kurzu i~pajęczyn z~wszystkich książek, umyłem ręce
i znowu zasiadłem. Sącząc stygnącą kawę, przewracałem strony, próbowałem
się skupić na kwestiach, które zawierały.

Kiedy się obudziłem po raz trzeci od czoła uderzającego w~stół, poddałem
się, nalałem sobie kolejnej kawy i~zwróciłem się do mojego prawdziwego
zmartwienia, tego, o~którym nie chciałem myśleć: co, jeżeli Merrial
zwyczajnie mnie wykorzystuje? Że przede wszystkim znalazła mnie,
ponieważ chciała, żebym wykonał dla niej pracę?

Chodziłem tam i~z powrotem po wąskim pokoju, obracając pytanie w~głowie
prawie tak często jak zawracałem. Po kilku powtórzeniach, zdecydowałem,
że nie mógłbym być oszukany w~sprawie jej uczuć, że jej namiętność była
prawdziwa, i~jeżeli chciała mną manipulować, zrobiłaby to bardziej
subtelnie\ldots

Jednak w~tym przypadku, może to było dowodem jak subtelnie działa. W~tym
miejscu się zatrzymałem. Podejrzewać manipulację tak subtelną -- najwidoczniej niezdarne i~oczywiste podejście, które maskuje te bardziej
przebiegłe i~eleganckie -- było podważaniem wiary w~mój osąd, na którym
takie odróżnienia musiałyby z~konieczności polegać.

Zatem zapomniałem o~podejrzeniach i~zajrzałem raz jeszcze do książek, a~kwadrans przed ósmą wyszedłem w~wieczór spotkać ją, i~mój los.


\chapter{Papierowe Tygrysy}

Trzy flagi wisiały za trumną: sowiecka, czerwona ze złotym młotem i~sierpem, kazachstańska, niebieska z~żółtym słońcem i~orłem, i~MRRNT,
żółta z~czarną, trójlistną koniczyną.

Około dwustu osób stłoczyło się w~sali krematorium. Pogrzeb był
najbliższą uroczystościom państwowym, które republika odbyła od czasu
stulecia Sputnika. Cały zubożony aparat był tutaj i~dobry odsetek
robotników, chłopów i~inteligencji prawdopodobnie oglądał w~telewizji.
Wyróżnieni zagraniczni goście obejmowali konsula Kazachstanu, dyrektora
Sekcji Interesów Zachodnich Stanów Zjednoczonych i~Davida Reida, który
był wciśnięty pomiędzy kilku agentów Ochrony Wzajemnej. Myra siedziała z~resztą SowNarKomu w~pierwszy rzędzie, suche oczy, gdy jedna ze starych
towarzyszek Georgiego -- kolejna weteranka afgańska -- wygłaszała
eulogię\footnote{ mowa pogrzebowa -- przyp.tłum.}.

-- Major Georgi Jefrimowicz Dawidow urodził się w~Ałma-Acie w~1956 roku.
W szkole, w~Pionierach i~Komsomole, szybko wyróżnił się jako wzorowa
jednostka, pilna, obywatelska ze wspaniałą atletyczną sprawnością. Po
otrzymaniu stopnia na Uniwersytecie Kazachstanu, gdzie dołączył do
Komunistycznej Partii Związku Radzieckiego, ukończył służbę wojskową i~wybrał karierę militarną. W~1979 roku zakwalifikował się jako pilot
helikoptera i~później tego samego roku był pierwszym wśród ograniczonego
kontyngentu radzieckich sił zbrojnych, które wypełniły
internacjonalistyczny obowiązek wobec ludu Afganistanu.

Fala różnicy w~poglądach, wyrażona we wdechach lub westchnięciach lub
przesuwaniu stóp, przebiegła przez pokój. Myra sama pociągnęła nosem,
zacisnęła usta, spojrzała w~dół. Wszystkie te noce, kiedy budził ją,
chwytając ją, trzymając ją, gadając o~koszmarach, wszystkie te poranki,
kiedy nie mówił słowa, nie dawał znaku, że pamiętał jakąkolwiek przerwę
we śnie swoim lub jej.

Mówczyni podniosła nieco głos i~kontynuowała nieposkromiona.

-- Jego służba zapewniła mu awans i~honor Bohatera Związku Radzieckiego.
W 1985 wniósł o~przeniesienie do programu kosmicznego i~po treningu w~Bajkonurze otrzymał dumny tytuł Kosmonauty Związku Radzieckiego.
Jednakże, wiele dziesięcioleci musiało minąć, zanim mógł spełnić tę
część swojego przeznaczenia.

Kiedy to już była jebana rutyna i~nie było już jebanego Związku
Radzieckiego, więc do rzeczy\ldots

-- Podczas burzliwej drugiej połowy lat osiemdziesiątych, major Dawidow
zajął pewne stanowiska polityczne, co do których jego przyjaciele i~towarzysze mogli się uczciwie róż \ldots

Dobre, był za jebanym Jelcynem, do rzeczy\ldots

-- \ldots ale które zaświadczają o~jego prawdziwym radzieckim i~kazachskim
patriotyzmie i~powadze, z~jaką podejmował się obywatelskich obowiązków i~leninowskich wartości sił zbrojnych, które według niego zakazywały
użycia przemocy wobec Ludu.

Myra nie była jedyną, która musiała powstrzymać śmiech.

-- Po uznaniu niepodległości Republiki Kazachstanu, doświadczenie majora
Dawidowa w~kwestiach broni jądrowej i~rozbrojenia nuklearnego zapewniło
mu nowe pole dla jego wspaniałych umiejętności politycznych i~osobistego
uroku\ldots

Myra przygryzła wargę.

~

Był przed nią w~kolejce do taksówki na zewnątrz lotniska w~Ałma-Acie.
Wysoki, nawet wyższy niż ona, bardzo ciemny. Odrzucone do tyłu czarne
włosy, brwi prawie tak gęste jak jego czarny wąs. Zrelaksowany w~sztywnym oliwkowozielonym mundurze. Paląc Marlboro i~spoglądając co
chwilę na sfałszowanego Rolexa.

Myra, właśnie przyjechała, zagubiona i~niespokojna, nie mogła oderwać od
niego oczu. Jednak to żółta plastikowa torba u jego stóp dodała jej
odwagi do rozmowy. Wydrukowany na niej w~czerwieni był obraz papugi i~słowa:\\
\\ THE PET SHOP\\
992 Pollockshaws Road\\
Glasgow G41 2HA\\

Pochyliła się, w~jego pole widzenia.

-- Przyleciałeś z~Glasgow? -- spytała po rosyjsku.

Odwrócił się, wytrącony z~jakiegoś transu i~spojrzał na nią z~oszołomioną miną, która nagle zmieniła się w~uśmiech.

-- Ach, torba. -- Dotknął ją stopą, odsłaniając, że była wypchana
kartonami papierosów i~butelkami Johnny Walker Black. -- Zatem jesteś
tutaj obca.

-- Och?

-- Te plastikowe torby nie mają nic wspólnego z~Glasgow. Są używane w~każdym sklepie stąd do Chin, Bóg wie dlaczego. -- Roześmiał się,
pokazując mocne zęby poplamione nikotyną. -- Byłaś w~Glasgow?

-- Tak -- powiedziała Myra. -- Mieszkałam tam przez kilka lat, w~latach
siedemdziesiątych.

Coś ostygło w~jego spojrzeniu. 

-- Co tam robiłaś?

-- Pisałam pracę dyplomową -- powiedziała Myra -- o~ekonomii Związku
Radzieckiego.

Zarechotał. 

-- Miałaś pozwolenia na \textit{to}?

-- To nie był problem\ldots -- zaczęła, potem przerwała. Zrozumiała, że wziął
ją za obywatel byłego Związku Radzieckiego. Byłą \textit{nomenklaturę}\footnote{
system obsadzania stanowisk kierowniczych w~ZSRR i~krajach bloku
wschodniego oparty na rekomendacji partii komunistycznej,
więcej~\url{https://pl.wikipedia.org/wiki/Nomenklatura\_(polityka)}
-- przyp.tłum.}, skoro miała pozwolenie na tak niebezpieczne badania.

-- Nie jestem Rosjanką, jestem ze Stanów Zjednoczonych!

Uniósł brwi.

-- Twój akcent jest bardzo dobry -- powiedział, po angielsku. Jego akcent
był bardzo dobry. Rozmawiali, póki nie doszli na początek kolejki, a~potem dalej rozmawiali, ponieważ dzielili taksówkę do miasta, ciągle
rozmawiali\ldots

~

Czy kiedykolwiek zaczęłaby rozmowę, zastanawiała się Myra, gdy nie ta
żółta torba? A gdyby nie zaczęła rozmowy, czy kiedykolwiek by go jeszcze
zobaczyła? Może, ale może nie, lub nie w~takiej chwili, kiedy oboje byli
wolni, lub na odskoczni od kochanków, a~w~takim przypadku\ldots

Nie byłaby tutaj, z~jednej strony, a~Georgi nie byłby w~trumnie i\ldots
konsekwencje biegły i~biegły, eskalując, póki nie wiedziała, czy się
śmiać, czy płakać. Przez jeden gwóźdź upadło królestwo\footnote{ przysłowie,
które nie występuje w~Polsce,
zob.~\url{https://en.wikipedia.org/wiki/For\_Want\_of\_a\_Nail}
-- przyp.tłum.}, a~wynik tego banału, tego fikcyjnego adresu sklepu
zoologicznego na torbie dał jej republikę, i~obłożył innych stratami,
których nie śmiała rozważać. Lub tak mogło się wydawać, jeżeli
ktokolwiek kiedykolwiek dowiedział się wystarczająco o~niej, żeby
dostrzec jej wpływ na historię.

Jednak wtedy znowu, może nie, może stary Engels i~Plechanow mieli w~końcu rację na temat roli jednostki w~historii: może to wszystko
wychodziło w~praniu, pod koniec Rewolucji Francuskiej \textit{ktoś}, ale
oczywiście ha ha ,,niekoniecznie ten szczególny Korsykańczyk'' wszedłby
w buty, które okoliczności, jak dobry lokaj, nałożyłyby na człowieka na
koniu.

Nigdy nie uważała tej teorii za szczególnie przekonującą i~rozważanie
jej teraz dawało jej niewielkie pocieszenie. Nie, utknęła, jak wszyscy,
w czynach i~ich konsekwencjach.

-- \ldots w~ostatnich latach Georgi Jefrymowicz odegrał główną rolę w~służbie dyplomatycznej MRRNT, służbie, w~której napotkał śmierć. -- Mówczyni przerwała na chwilę, żeby skierować przeszywające spojrzenie na
honorowych zagranicznych gości. -- Pozostawił byłą żonę i~lojalną
przyjaciółkę, Myrę Godwin-Dawidową, ich dzieci i~wnuki\ldots

Zbyt wiele do wyczytania i~żadne na miejscu, do rzeczy\ldots

Jednak zostały odczytane na głos wiadomości od wszystkich nieobecnych
potomków, innych kuzynów, starych przyjaciół. Mówczyni w~końcu odłożyła
wiązkę dokumentów i~uniosła dłoń. Krematorium wypełniło się dziwnie
cichym i~skromnym dźwiękiem narodowego hymnu Kazachstanu. Trumna po
cichu potoczyła się przez dyskretny właz. Wszyscy wstali i~zaśpiewali,
lub udawali, ,,Międzynarodówkę''. I~to było wszystko. Kolejny dobry
materialista zamienił się w~popiół.

Myra odwróciła się i~wyszła z~krematorium, a~rząd za rzędem, od przodu,
wszyscy ruszali i~wychodzili za nią.

Jej dłonie się trzęsły, gdy grzebała przy czarnej futrzanej czapce i~próbowała zapalić papierosa na podjeździe. Na ulicy, samochody się
przesuwały na pozycję, by zabrać dygnitarzy na po-pogrzebowe spotkanie
lunchowe. Ktoś uspokoił jej dłoń, pomógł z~papierosem. Zapalił i~spojrzała w~górę, zobaczyła Davida Reida. Ciemne brwi, ciemne oczy,
białe włosy aż do postawionego kołnierza futrzanego płaszcza. Wyglądał
na mniej niż połowę jego lat, tylko z~białymi włosami -- własnym
upodobaniem -- wskazującym na cokolwiek innego. Bez jej ujawniających
wiek wad. Była całkiem pewna, że jego stawy nie trzeszczą, jego kości
nie bolą. Mieli lepsze terapie na Zachodzie. Jego opiekunowie trzymali
się kilka kroków dalej, wzrokiem przeszukiwali otoczenie. Ludzi mieszali
się dookoła, dryfując w~kierunku czekających samochodów.

-- Wszystko w~porządku? -- spytał Reid.

-- W~porządku, Dave.

Zaszurał stopą na żwirze, podrapał się po karku.

-- Nie zrobiliśmy tego, Myra.

-- Tak, cóż\ldots -- Wzruszyła ramionami. -- Czytałam wyniki autopsji. Wierzę
w to.

\textit{Byłbyś martwy, gdybym nie uwierzyła}, wzgardziła dodać. Wierzyła
autopsji, nie miała wyboru. Wierzyła Reidowi, też. Ciągle miała swoje
wątpliwości o~orzeczeniu: przyczyny naturalne, to mógł być jeden z~tych
mrocznych epizodów, gdzie nigdy nie mogła być pewna prawdy, jak wkład
Stalina w~sprawie Kirowa\footnote{ Siergiej Kirow -- 1 grudnia 1934 został
zastrzelony przez bolszewika Leonida Nikołajewa. Wielu historyków uważa,
że zabójstwo Kirowa było zlecone przez Stalina i~zorganizowane przez
NKWD. Zarzut uczestnictwa w~spisku postawiono wielu dawnym działaczom
podczas wielkiej czystki.,
zob.~\url{https://pl.wikipedia.org/wiki/Siergiej\_Kirow} -- przyp.tłum.} lub śmierci Roberta Harte'a \ldots Ale Reid zajął pozycję, którą
chciała, żeby zajął. Wydawał się lekko uspokoić i~zapalił papierosa.
Jego wzrok przeskakiwał od palącego końca do komina krematorium, potem
na nią.

-- Och, cholera. Wydaje się taką stratą.

Myra skinęła głową. Wiedziała, co ma na myśli. Palenie zmarłych,
grzebanie ich \textit{w jebanej dziurze w~ziemi}, już to zaczynało się
wydawać barbarzyństwem.

-- Nawet nie chciał krio -- powiedziała. -- Ani tym bardziej tego
kalifornijskiego komputerowego skanu, oszukanego.

-- Dlaczego nie? -- spytał Reid. -- Mógł sobie na to pozwolić.

-- Och, pewnie -- powiedziała Myra. -- Po prostu w~to nie wierzył, to
wszystko.

Reid uśmiechnął się cienko. 

-- Tak jak ja.

-- Och?

Rozłożył ręce. 

-- Ja tylko sprzedaję polisy.

-- Czy są \textit{jakieś} sprawy, w~których nie maczasz palców?

Reid potarł nos palcem. 

-- Dywersyfikacja, Myro. Nazwa tej gry.
Zmniejszanie ryzyka. Nauczyłem się tego w~ubezpieczeniach, dawno temu. -- Sięgnął, czekając na niewypowiedziane pozwolenie, żeby wziąć ją pod
ramię. -- Musimy porozmawiać o~interesach.

-- Samochód -- powiedziała, chwytając stanowczo jego łokieć i~kierując go
na chrzęszczący żwir. Szli koło siebie do opancerzonej limuzyny. Myra,
kątem oka, obserwowała obserwujących ludzi. Dobrze: niech będzie jasne,
że już nie podejrzewa Reida. Nie publicznie, nie politycznie, nie nawet,
na pewnym poziomie, prywatnie. Tylko osobiście, tylko w~zazdrosnych
starych kościach. Jednak to było więcej niż przedstawienie
dyplomatyczne. Ciągle istniało pomiędzy nimi uczucie, choć osłabione
przez lata, podirytowane ich wrogością. Reid nigdy nie był człowiekiem
pozwalającym wrogości przeszkodzić przyjaźni.

Myra spojrzała na zegarek, gdy drzwi samochodu się zamknęły z~dobrze
zaprojektowanym trzaśnięciem. Mieli około pięciu minut prywatnej
rozmowy, gdy wielki Ził toczył się ulicami centrum Kapicy do jedynego
wytwornego hotelu, Sheratona. Rozparła się na skórzanym siedzeniu i~spojrzała ostrożnie na Reida.

-- Ok -- powiedziała. -- Do rzeczy.

Reid sięgnął po wielką popielniczkę, zgasił jednego papierosa i~zapalił
kolejnego. Myra postąpiła tak samo. Ich wydychane dymy spotkały się w~obliczu wzajemnych zakłóceń. Reid podrapał brew, spojrzał w~bok,
spojrzał na nią.

-- Cóż -- powiedział. -- Chcę Ci złożyć ofertę. Wiemy, że ciągle posiadasz
stare -- zawahał się, nawet tutaj, były słowa, których nikt nie
wypowiadał -- strategiczne zasoby, i~chcielibyśmy je od Ciebie odkupić.

Mógł blefować.

-- Nie mam\ldots -- zaczęła. Reid odchylił głowę i~dmuchnął małym strumienie
dymu, który po kilku centymetrach, zawinął się w~sobie w~miniaturowy
obłok w~kształcie grzyba.

-- Nie marnuj czasu zaprzeczając temu -- powiedział.

-- Dobrze -- powiedziała Myra. Przełknęła narastające mdłości, uspokoiła
się wobec oszałamiającego, chłodnego zaciemniania w~oczach. To jakby
była przyłapana ze wstydliwą tajemnicą, ale takiej, której nie
wiedziała, że jest tajemnicą. Jednak, wiedziała zbyt dobrze, jeżeli nie
wiedziała, to dlatego, że nigdy nie próbowała, i~nie chciała, się
dowiedzieć.

-- Załóżmy, że mamy. Nie sprzedalibyśmy nikomu, tym bardziej Tobie.
Jesteśmy przeciwko Twojemu zamachowi\ldots

Teraz była kolej Reida na pozorowanie niewiedzy, a~Myry na pokaz
niecierpliwości.

-- Nie \textit{użylibyśmy }ich -- powiedział. -- Dobry Boże, za kogo nas
bierzesz? Chcielibyśmy je\ldots zdjąć ze szachownicy, żeby tak powiedzieć.
Z gry. I~całkiem szczerze, jedynym sposobem, jakiego możemy być pewni,
to posiadanie nad nimi kontroli.

Myra potrząsnęła głową. 

-- Nie ma mowy. Żadnej umowy.

Reid podniósł dłoń. 

-- Pozwól mi powiedzieć, co oferujemy, zanim
odrzucisz. Możemy Cię wykupić, wolną i~czystą. Dać każdemu w~tym
Państwie, każdemu z~obywateli, wystarczająco pieniędzy, by osiąść
gdziekolwiek i~żyć bardziej komfortowo. Pomyśl o~tym. Obozy będą
zlikwidowane i~\textit{ktokolwiek} wygra kolejną rundę ruszy na Ciebie.
Twoje zasoby nie będą miały znaczenia, kiedy Obrona Kosmiczna wróci do
biznesu.

-- To groźba, tak?

-- Ani trochę. Stwierdzenie faktu. Sprzedaj je teraz lub strać je
później, zależy to od Ciebie.

-- Strać je\ldots lub użyj ich!

Reid rzucił jej spojrzenie ,,to nie jest zabawne''.

-- Nie oszukuję -- powiedziała mu Myra. -- Najlepsze, co mogę przewidzieć
po Twoim zamachu, to więcej chaosu, w~takim wypadku będziemy potrzebować
wszystkich cholernych \textit{zasobów}, jakie mamy!

Reid wziął głęboki wdech. 

-- Nie, Myro. Jeżeli będzie chaos, to dlatego,
że nie wygraliśmy. Ten zamach, jak go nazywasz, jest ostatnią, najlepszą
szansą na stabilność. Jeżeli zawiedziemy, świat pójdzie do piekła na
swój sposób. Twój osobisty wkład do tego nie będzie wtedy moim
problemem, będę martwy lub w~kosmosie, ale Ty \textit{możesz }móc
zapewnić, że się nie \textit{zdarzy} i~odnieść korzyści, Ty i~Twoi ludzie,
z procesu. -- Włożył cały swój niezaprzeczalny urok w~głos i~wyrażenie,
gdy kończył. -- Przemyśl to, Myro. To wszystko, o~co proszę.

-- Pomyślę o~tym -- powiedziała, przyznając mu przynajmniej to zwycięstwo,
na tyle, o~ile było warte. Rozejrzała się. -- Dojechaliśmy.

Ozdobnie umeblowany apartament na specjalne okazje był wypełniony ludźmi
w ciemnych ubraniach, stojących w~grupkach i~rozmawiających po cichu.
Już zaczęli się relaksować po uroczystości pogrzebowej, uśmiechać się i~lekko śmiać: życie toczy się dalej. Dobrze.

Myra i~Reid podeszli razem do długich stołów, na których był rozłożony
bufet i~zdołali stracić się z~oczu w~losowych ruchach ludzi
wybierających jedzenie i~napoje. Z~talerzem przysmaków w~jednej dłoni i~dużą szklanką whisky w~drugiej, Myra się rozejrzała. Tam w~jednym rogu
Andriej Muchartow był zatopiony w~rozmowie z~damą w~ciemnej sukni i~dużym kapeluszu. Odpowiadała głośno na jego ciche pytania. Myra miała
nadzieję, że ta reprezentantka poszarpanych zachodnich obrzeży byłych
Stanów Zjednoczonych nie mówiła o~niczym poufnym. Możliwe, że o~to
chodziło. Zauważyła, że Walentyna stoi sama, w~oliwkowozielonym ubiorze,
którego czarna opaska na ramieniu raczej została zakrzyczana z~powodu
zadziwiającej ilości złotych warkoczy. Myra skierowała się mało
subtelnie prosto na nią.

-- Och, tutaj jesteś -- powiedziała, gdy Walentyna się odwróciła. Trąciła
jej ministra obrony w~kierunku najbliższego z~wielu małych stołów
stojących w~obszernym pomieszczeniu. Usiadły.

-- Nowy mundur? -- spytała Myra.

Sztywne epolety Walentyny poruszyły się do góry i~dołu. 

-- Nigdy nie
miałam wcześniej okazji -- powiedziała.

-- Nie wiedziałam też, że zebrałaś tyle medali.

Walentyna musiała się roześmiać. 

-- Tak, jest trochę\ldots breżniewowski,
co?

-- Wszystko jest zbyt właściwe, dla nas. Okres stagnacji.

Walentyna pożarła kanapeczkę, nie odwracając wzroku od Myry. 

-- Dokładnie. Widzę, że odbyłaś małą rozmowę z~naszym głównym inwestorem
wewnętrznym.

-- Tak. Przedstawił mi interesującą ofertę. -- Myra spojrzała na swój
talerz, wybrała coś z~nogami. -- Mam nadzieję, że to jest syntetycznie.
Nie zniosłabym myśli o~poziomie promieniowania, gdyby nie było.

-- Myślę, że powinnyśmy polegać na kogoś etyce biznesowej w~sprawie
promieniowania -- powiedziała Walentyna.

-- Ach, racja. -- Myra popatrzyła na skorupkę krewetki. Miała znak
handlowy ICI\footnote{ ICI -- prawdopodobnie Imperial Chemical Industries,
brytyjska korporacja, jeden z~największych chemicznych producentów na
świecie,
więcej~\url{https://en.wikipedia.org/wiki/Imperial\_Chemical\_Industries}
-- przyp.tłum.}. Pełne sztucznej dobroci. Wyciągnęła jasnoróżowe mięso
zębami. -- Tak czy inaczej, Pani Towarzysz Komisarz Ludowa do spraw
Obrony, moja droga, nasz wewnętrzny inwestor dał mi do zrozumienia, że
wie, że zrobiliśmy nieco mniej \ldots widocznego zrzeczenia, niż byłam
skłonna uwierzyć.

Walentyna, raczej na jej korzyść, pomyślała Myra, wyglądała na
zażenowaną.

-- Odziedziczyłam zasoby od poprzedników \ldots i~nigdy nie wspominałam o~nich, ponieważ myślałam, że już wiesz, lub nie wiesz, a~potrzebujesz
możliwości zaprzeczenia.

Zatem to była prawda. Potwierdzenie było mniejszym szokiem niż
oryginalne twierdzenie Reida. Zajęłoby chwilę zaakceptowanie całej tego
potworności.

Myra skinęła głową, jej usta pełne. Połknęła z~łykiem whisky. 

-- To
drugie, właściwie. Nie wiedziałam. Myślałem, że po wojnie wszystkie
zostały zajęte przez Jankesów.

-- Większość z~nich. Jednak był jeden wyjątek. Duże portfolio zasobów
przetrwało represje tak, że USA/ONZ nie mogło przejąć kontroli. Jedna umowa,
która była ciągle odnawiana. Oczywiście aż do Jesiennej Rewolucji. Potem
przedawniła się i~zostałam z~tymi maleństwami. Zostały wysłane do nas
dużymi partiami w~wielkich przesyłkach dyplomatycznych z~różnych miejsc,
wszystkich kontrolowanych przez\ldots

-- Rozumiem, że teraz możesz mi powiedzieć?

Walentyna się rozejrzała i~wzruszyła ramionami.

-- Pierwotne minipaństwo, z~oryginalnymi, najemnymi siłami obronnymi.

Myra musiała pomyśleć przez chwilę, zanim zrozumiała, o~którym państwie
mówiła Walentyna.

-- Płaczący Jezu!

-- Całkiem możliwe -- powiedziała Walentyna -- całkiem możliwe, że płakał.

Były chwile, kiedy wszystko, co mogłeś zrobić to być cynicznym,
przygotować się do ataku, nie pozwolić temu zdominować\ldots Myra dołączyła
do mrocznego śmiechu Walentyny.

-- Więc co się stało z~zasobami i~dlaczego nasz inwestor jest nimi
zainteresowany?

-- Ach -- powiedziała Walentyna. -- Pamiętasz kilka lat temu stulecie
Sputnika. Raczej ekstrawagancko wystrzeliliśmy niepotrzebny booster,
żeby je uczcić. W~tym czasie wykorzystałam możliwość, by umieścić
większość naszego żenującego dziedzictwa na orbicie.

-- Na orbicie \textit{Ziemi}? -- Myra powstrzymała irracjonalną chęć, by
skryć głowę pomiędzy ramionami.

-- Część z~nich -- powiedziała Walentyna. -- Te zaprojektowane specjalnie
do wykorzystania na orbicie, prawda? Są na wysokiej orbicie, całkiem
bezpieczne. -- Zmarszczyła brwi i~wbrew jakiemuś wewnętrznemu oporowi
dodała -- Cóż, dość bezpieczne. Niemniej resztę wysłaliśmy do jeszcze
bezpieczniejszego miejsca: Lagrange.

Myra wyobraziła sobie w~głowie, żywo jak na wirtualnym ekranie, punkt
Lagrange\footnote{więcej~\url{https://pl.wikipedia.org/wiki/Punkt\_libracyjny}
-- przyp.tłum.}: L5, jedno z~tych miejsc, gdzie ziemska i~księżycowa
grawitacja razem tworzyły region orbitalnej stabilności, i~który, przez
pół wieku, zgromadził zagraconą grupę stacji badawczych, satelitów
wojskowych, oficjalnych i~nieoficjalnych habitatów, zapuszkowanych
utopii, opuszczonych rakiet, zaskłotowanych modułów, losowych śmieci\ldots
Była to ziemia obiecana Ruchu Kosmicznego, a~z~nowo nanofabrykowanymi
ultralekkimi rakietami startowymi populacja wzrastała tam tak szybko,
jak populacja Kapicy spadała.

-- Och, kurwa -- powiedziała Myra.

-- Nie martw się -- zapewniła ją Walentyna. -- Są prawie niewykrywalne
pomiędzy śmieciami.

Myra nie miała serca powiedzieć jej, jak bardzo nie rozumiała sytuacji.

-- Dlaczego do cholery zostawiłaś je \textit{tam}? -- zażądała odpowiedzi. -- Bezpiecznie, w~pewny sensie, tak, to mogę zrozumieć, ale czy wpadło Ci
na myśl, że jeżeli kiedykolwiek to wyjdzie na jaw, to nasze intencje
mogą być\ldots źle zrozumiane?

Walentyna wygląda jeszcze bardziej zażenowaną. 

-- To była\ldots cóż, to
sprawa Partii, Myro. Żądanie.

-- Och, racja. A niech mnie. Ciągle jesteś \textit{w }jebanej Partii?

Walentyna zachichotała. 

-- Ja jestem Partią. Przynajmniej sekcją MRRNT.

-- Teraz gdy Georgi odszedł. 

-- Kurde, zapomniałam.

Nie wystawili czwartej flagi, Flagi Czwartej, na jego trumnie. Gówno.
Nie, żeby to teraz było ważne. Nie dla Georgiego, w~każdym razie. I~nie
dla tych, którzy się zebrali uczcić go, jedynym obecnym, który
zrozumiałby wagę, był Reid.

-- Nie martw się -- powiedziała Walentyna.

-- Co chce Międzynarodówka zrobić z\ldots och, kurwa. Mogę wymyślić dowolną
liczbę rzeczy, które może chcieć z~tym.

Walentyna skinęła głową. 

-- Niektóre z~nich mogłyby być naszą przewagą.

-- Ha. Ja to osądzę. Zachowałaś sobie kody dostępu?

-- Oczywiście!

-- Dobra, to już coś.

-- Więc nasz człowiek proponuje wykup, prawda? -- kontynuowała Walentyna.
-- Może warto się zastanowić.

-- Ta. -- Myra wstała, zabierając szklankę. -- Zamierzam z~nim jeszcze
porozmawiać. Dzięki za info, Walu.

Napełniła szklankę, tym razem wódką, i~wyruszyła na ostrożnie
przypadkową wędrówkę do miejsca, gdzie Reid stał, rozmawiając z~przerażonym stadem urzędników niskiego stopnia. Denis Gubanow i~jeden z~łapaczy Reida krążyli dyskretnie, zachowując przezorny dystans od grupy
i samych siebie, każdy w~punkcie Lagrange drugiego. Nie mogła usłyszeć
rozmowy. Po drodze, została przechwycona przez Aleksandra Shermana.
Komisarz Przemysłu był ubrany w~ten sam plastikowy, sztywny garnitur,
jego kolor dostosowany do czerni. Wyglądał chytrzej niż zwykle, zły
znak.

-- Ach, Myro. Smutny dzień dla nas wszystkich. -- Pokręcił wolno głową. -- Smutny dzień.

-- Tak -- powiedziała Myra. Fraza \textit{do rzeczy} jeszcze raz pojawiła
się w~jej umyśle.

Aleks wziął głęboki wdech i, jakby telepatycznie, ogłosił: 

-- Muszę Ci
coś powiedzieć. To nie najlepszy czas, ale\ldots Dobra, otrzymałem ofertę
od Pana Reida.

-- Żeby wykupić nasze zasoby?

-- Nie, nie! -- Aleks wyglądał na zaskoczonego sugestią. -- Ofertę
\textit{zatrudnienia.}

-- Och, racja -- powiedziała lekceważąco. Machnęła rękę, gdy go omijała. -- Skorzystaj.

Gdy szła, mogła ujrzeć siebie w~wielkich lustrach w~pozłacanych
oprawach. Były ustawione naprzeciw siebie z~innymi lustrami na drugim
końcu pokoju i~przez chwilę widziała siebie pomnożoną, potencjalna
nieskończoność różnych wersji samej siebie: wizualny, wirtualny obraz
interpretacji wieloświatowej\footnote{ interpretacja mechaniki kwantowej,
więcej~\url{https://en.wikipedia.org/wiki/Many-worlds\_interpretation}
-- przyp.tłum.}. Kiedyś bawiła się dziecinną myślą, że obrazy lustrzane
mogą być oknami w~te inne światy. Czy foton kiedykolwiek decydował,
zastanawiała się, czy kiedykolwiek odchylał się przy odbiciu?

To, co zobaczyła w~nieskończenie powtarzalnych obrazach, to wysoka chuda
kobieta w~długiej czarnej sukni poruszająca się w~kierunku ciągle
nieświadomego Reid jak jakaś nemezis typu MIRV\footnote{ MIRV ang. Multiple
independently targetable reentry vehicle -- rodzaj pocisku rakietowego,
jedna z~wielu głowic umieszczonych w~tym samym pojeździe pocisku klasy
ICBM lub SLBM, z~których po zakończeniu fazy startowej każda może być
niezależnie umieszczona na kursie balistycznym prowadzącym do odrębnych
celów,
zob.~\url{https://pl.wikipedia.org/wiki/Multiple\_independently\_targetable\_reentry\_vehicle}
-- przyp.tłum.}. Zobaczyła przeskakujące spojrzenia wymieniające
wiadomości pomiędzy jej Komisarzem Bezpieczeństwa, ochroną Reida, Reidem
i nią samą, aż odbite oczy Reida spotkały jej rzeczywiste oczy i~się
rozszerzyły.

Napotkała rodzaj znieruchomienia w~powietrzu i~zrozumiała, że
ochroniarze, pomiędzy sobą, organizowali środki przeciwpodsłuchowe,
nakładając płaszcz ciszy dookoła grupy. Potem przeszła rejon martwego
powietrza, gdzie głosy były zniekształcone i~dziwne i~nagle rozmowy były
wyraźne, przez chwilę, zanim zostały przerwane przez tych, którzy
zauważyli jej nadejście.

-- Cóż, cześć ponownie -- powiedziała. Jej wzrok przesunął się po kilku
jej pracownikach zebranych wokół Reida. Wszyscy na raz komicznie
próbowali uciec, cofając się tak dyskretnie, o~ile to możliwe. -- Szukasz
pracowników w~kadrze średniej jak również wśród moich komisarzy?

-- Tak -- powiedział Reid, całkiem niespeszony. Ruszył minimalnie końcami
palców i~brwiami, a~jego petenci, lub aplikanci, rozproszyli się jak dym
w przeciągu. Łapacz Reida i~Gubanow kontynuowali swoje uważne, wzajemne
okrążanie. Kelner przeszedł z~tacami szklanek i~bieługi na chlebie
żytnim. Myra i~Reid zaopatrzyli się z~obu, potem stanęli twarzą do
siebie, z~lekką niezręcznością, jak onieśmieleni nastolatkowie po tańcu.

-- Mogłabym sama poszukać pracowników, wiesz -- powiedziała Myra. -- Może
powinnam kupić szpiega od Ciebie. Okazuje się, że jesteś lepiej
poinformowany na temat naszego portfolio inwestycji niż ja. Szczególnie
o jego, hm, zasięgu.

Reid przyjął to z~lekkim skinieniem.

-- Stawia to nas w~trudnej pozycji -- powiedział. -- Masz nas na celowniku,
szczerze. W~końcu orbity Ziemi są lepszą pozycją.

\textit{Och?}, pomyślała do siebie. Więc nie wie o~Lagrange'u? Lub nie
chce, żeby ona wiedziała, że wie.

-- Jednak -- kontynuował Reid -- jestem dość pewny, że nie, hm,
zlikwidujesz. Z~oczywistych powodów.

-- Skąd zatem oferta?

-- Spokój umysłu\ldots nie, poważnie. Pomiędzy nami, Ty i~ja znamy
wszystkich, którzy wiedzą o~obecnym poziomie ekspozycji. Jednak żadne z~nas nie może zagwarantować, że to przetrwa. Słowo w~złym miejscu i~po
mojej stronie mogą być poważne wahania rynku. Które też, spieszę dodać,
nie byłyby dla Ciebie korzystne, więc mamy wzajemnie\ldots

-- Zagwarantowane odstraszanie?

Reid rzucił jej spojrzenie typu \textit{zamknij ryja}. 

-- Możesz tak
powiedzieć\ldots ale wolałbym, żebyś nie mówiła.

Myra uśmiechnęła się nikczemnie. 

-- Ok -- powiedziała. -- Ciągle nie mamy
umowy, Dave.

Spojrzał na nią, bez wyrazu, ale nie potrafił ukryć prośby w~głosie. 

-- Czy przynajmniej zgodzisz się nie zrzucać zasobów podczas oferty
przejęcia? Nie oferować konkurencji?

Och, Matko. To było podchwytliwe. Nie miała intencji robienia rzeczy,
których się bał. Z~drugiej strony, jeżeli bałby się ich (nawet jeżeli
teoretycznie i~tylko na marginesie, ale jednak\ldots), mogłoby go to
powstrzymać. Mogłoby to trzymać, jego i~jego sojuszników, od
przekroczenia tej niewidzialnej granicy, tego terminatora pomiędzy dniem
a mrokiem. Niech nienawidzą, tak długo, jak się boją.

Potrząsnęła głową i~zobaczyła swoje wielokrotne odbicia robiące to samo,
w poważnym powtórzeniu. Tak, akt obserwacji załamuje funkcję falową:
kości rzucone, kot umiera.

-- Przepraszam, Dave -- powiedziała mu. -- Nie mogę niczego obiecać.

Jego wzrok mierzył je przez chwilę, potem wzruszył ramionami.

-- Czasem wygrywasz, czasem przegrywasz -- powiedział lekko. -- Do
zobaczenia, Myra.

Ujrzała, jak odchodzi, jak często musiała. Jego ochroniarz podążył za
nim w~bezpiecznej odległości. Denis uniósł brew, przewrócił oczami,
podszedł.

-- O co chodziło?

-- Och, tylko stare rzeczy pomiędzy nami -- powiedziała Myra. -- Nie
widujemy się oko w~oko, to wszystko. -- Wzięła go pod ramię. -- Zobaczmy
jak sobie radzi Andriej z~tą kobietą z~Zachodnich Stanów Zjednoczonych,
dobrze?

Niezbyt dobrze, jak się okazało. To nie było miejsce na tajną
dyplomację, nawet jeżeli używali tarcz prywatności, których nie używali.
Juniper Bear, nieoficjalna konsula zachodnioamerykańska, nie ukrywała
swojej pozycji dyplomatycznej. Jej czarny kapelusz z~szerokim rondem, z~czarnymi woskowymi owocami dookoła główki wydawał się wzmacniać jej
głos, nawet jeśli jej poza wskazywała na ważną, poufną komunikację.

-- \ldots właśnie w~zeszłym miesiącu uderzyliśmy na najazd Zielonej guerilli
z SoCal, a~w~tym samym czasie Narody Białych Aryjczyków naciskały przez
Rockies, i~uwierzyłbyś, że Federacja Pierwszych Narodów, cholerni
\textit{Indianie}, przerzucali znaczący sprzęt konwencjonalny na nasze,
północne kolonie po stronie Kanuka na starej granicy? Powiem wam,
Towarzyszu Muchartow, skorzystalibyśmy ze wsparcia orbitalnego, tym
razem dla odmiany po naszej stronie. -- Roześmiała się, uśmiechając do
Myry i~Walentyny, gdy dołączały do rozmowy. -- Czy \textit{uwierzylibyście}
-- powtórzyła -- że przeklęci Zieloni obecnie lobbują starą gwardię, żeby
utrzymała stacje orbitalne jako obronę przed asteroidami? Gdybyśmy
kiedykolwiek tego potrzebowali, a~teraz mamy te większe i~mniejsze
zmapowane i~śledzone, równie dobrze moglibyśmy się martwić nową epoką
lodowcową!

-- Cóż, nadchodzi -- powiedziała Walentyna.

Kapelusz Juniper Bear się przechylił. 

-- Pewnie, cykle
Milankovicia\footnote{periodyczne zmiany parametrów orbity ziemskiej mogące
wpłynąć na nasłonecznienie,
zob.~\url{https://pl.wikipedia.org/wiki/Cykle\_Milankovicia} -- przyp.tłum.}, tak, ale to nie zmartwienie, prawda? -- Roześmiała się. -- Hej, pamiętam globalne ocieplenie!

-- A \textit{to} się dzieje -- powiedziała Myra. -- Ale, jak mówisz, nie jest
to \textit{zmartwienie}, już nie. I~dziury ozonowe, poziomy promieniowania
tła, syntetyczne polimery w~organizmach, przeskakujące geny i~to
wszystko, tak, nie \textit{martwimy} się. -- Poczuła się zaskoczona
brzmieniem głosu, jak była na to wszystko zła, teraz gdy to wypowiadała.
Wyglądało to, jakby miała zaangażowanego Zielonego głęboko w~sobie,
tylko czekającego na szansę. -- Szczerze mówiąc, Pani Bear, martwimy się
czymś innym. Planami rewitalizacji Narodów Zjednoczonych. Nawet jeżeli
\textit{będą} wrogami naszych wrogów, w~pierwszej kolejności. Nie chcemy
tego rodzaju władzy obróconej przeciwko komukolwiek na Ziemi, nigdy
więcej. -- Zdjęła swój kapelusz, przesunęła po gładkich włosach, kciukiem
po czerwonej gwieździe i~złotej pieczęci, zrozumiała, że stała tam,
dosłownie z~czapką w~ręku, prosząc o~pomoc.

Juniper Bear potrząsnęła głową. Była starą kobietą, nie tak starą jak
Myra, wyglądała na trzydziestkę, według rachuby przed\dywiz odmładzaniem,
kiedy jej twarz była wypoczęta, ale waga lat ukazywała się w~każdej
minie, jeżeli byłeś dostatecznie stara, żeby zauważyć takie sprawy.
Uczyłaś się wysyłać i~odbierać te niewerbalne tiki, w~równoległych
procesach wzrastającej mądrości.

-- To właśnie mówi nasza opozycja -- powiedziała kobieta. -- ,,Nigdy więcej
Porządku Nowego Świata!''. Cóż, przepraszam, ale potrzebujemy
prawdziwego porządku nowego świata, tym razem po naszej stronie. To
będzie tylko tymczasowy, kiedy tylko będziemy mieć dostatecznie dużo sił
tam w~górze, nie będzie sposobu, żeby ktokolwiek zachował władzę
centralną. Kiedy zagrożenie minie, to po prostu\ldots -- Zrobiła gest
opadania w~dół.

-- Obumrze?

Zmarszczone oczy Juniper zarejestrowały ironię, jej zaciśnięte usta jej
odmowę na ugięcie się. 

-- Mówiąc o~państwach, które obumrą -- powiedziała,
zmieniając zręcznie temat -- jeżeli którekolwiek z~was będzie szukać
nowych możliwości, kiedy to wszystko w~ten czy inny sposób się
skończy\ldots

Walentyna i~Andriej nic nie powiedzieli, przynajmniej nie w~obecności
Myry, ale Myra się uśmiechnęła i~skinęła głową, powiedziała, że będzie
pamiętała.

-- Dobrze! -- powiedział Andriej Muchartow, kiedy spotkanie było
zakończone i~goście wyjechali, dyplomaci, aparatczycy i~kapitanowie
przemysłu. Andriej, Walentyna, Denis i~Myra spoczęli w~jednym z~mniejszych i~cichszych barów w~hotelu. Drewno i~lustra, skóra i~szkło,
dywany pluszowe i~cicha muzyka. W~barze było mnóstwo osób, która nie
miała nic wspólnego bezpośrednio z~pogrzebem. To wpłynęło na poziom
bezpieczeństwa czterech pozostałych komisarzy, skulonych dookoła butelki
wódki przy narożnym stole, jak dysydenci. 

-- Dziękuję za interwencję
wcześniej, towarzysze. Myślałem, że gdzieś docieram, póki się nie
pojawiliście.

-- Źle myślałeś -- powiedziała Myra. Nie chciała się kłócić. -- Znam
Juniper, wydaje się zgadzać z~Tobą, a~potem zacznie mówić o~wojnie.
Kiedy się pojawiliśmy. Nic nie straciłeś.

-- Ha -- chrząknął Andriej. Stuknął kciukiem w~szkło. -- Powiedz mi,
dlaczego w~ogóle potrzebujesz Sekretarza do spraw Zagranicznych.

-- Ponieważ nie mogę robić wszystkiego sama -- powiedziała mu Myra. -- Nawet jeżeli mogę zrobić każdą pracę lepiej niż inni. Podział pracy, nie
krytykuj. To wszystko jest u Ricardo.

Andriej i~Walentyna patrzyli na siebie, przewracając oczami, przesadnie
zakłopotani.

-- Megalomania -- powiedział ze smutkiem Andriej. -- Dotyka wszystkich
dyktatorów proletariatu, tuż przed końcem.

-- Uważasz, że powinniśmy obalić ją, zanim będzie za późno? -- Walentyna
wyprostowała się i~nakreśliła salut. -- Dobierzmy do tego Denisa i~możemy
uformować trójkę. Zrzućmy winę na Myrę i~ogłośmy czystą kartę.

-- To nie jest śmieszne -- powiedziała Myra. Nalała kolejkę, obserwując
czysty spirytus uderzający w~kryształy, cztery razy. -- Dokładnie tak
to się odbędzie. Pewnego dnia, wszystkie problemy świata będą moją winą.
-- To nie było śmieszne, pomyślała. To było jej najgłębsze podejrzenie, w~najmroczniejszych chwilach. Uśmiechnęła się do towarzyszy. -- Za
wspaniałą przyszłość!

Przechylili kieliszki wódki i~uderzyli w~stół pustymi kieliszkami. Myra
odmówiła ofercie Marley czy Moskiewskich Złotych, zapaliła Dunhilla z~jej ostatniej podróży. Podwójna folia w~paczce, czerwone i~złote na
zewnątrz, ciągle dla niej było coś dziwnego i~obfitego w~marce, którą
paliła pierwszy raz, kiedy strefa wolnocłowa coś znaczyła.

-- Zatem, jaki jest wynik, Andriej? Oprócz dzisiejszych subtelnych prób.

-- Ach. -- Andriej wypuścił aromatyczny dym nosem. -- Nie dobrze, muszę
powiedzieć. Kazachstan ciągle się trzyma z~dala, w~końcu mają Bajkonur
do rozważenia i~zagrożenia Szinosowskie. Gdyby nie było wcześniej złej
krwi pomiędzy nimi a~Ruchem Kosmicznym, myślę, że mogliby się skusić do
zajęcia pozycji po ich stronie. Więc ich neutralność jest czymś, kiedy
wszystko powiedziano i~zrobiono. Co do reszty, agitowałem każdy kraj,
sprawdziłem naszych delegatów w~Nowym Jorku, i~szczerze wygląda, że
przyszłotygodniowe głosowanie przejdzie.

-- Walentyna?

Myra nie musiała niczego literować. Kozlowa spędzała noce i~dnie śledząc
raporty od agentów w~koloniach i~na stacjach orbitalnych. Odpowiedziała,
wystawiając rozłożoną dłoń i~machając ją.

-- Niedużo możemy zrobić tam na górze -- powiedziała. -- Druga strona ma
wszystkie zasoby, żeby przechylić szalę na ich stronę, niezależnie od
kierunku sporu.

-- Nie \textit{wszystkie} zasoby -- powiedziała Myra.

-- Och, nie -- powiedziała Walentyna, ostrożnie spokojnie. -- Nie
moglibyśmy. -- Mogła mówić o~oszukiwaniu w~kartach.

-- Ale oni nie wiedzą, że nie moglibyśmy -- powiedziała Myra. -- Mamy w~końcu ciężką reputację. Większość z~nowych krajów, nie wspominając osad,
prawdopodobnie myśli, że jesteśmy jakimiś bezlitosnymi Bolszewikami.

Roześmiali się cynicznie.

-- Jestem pewny, że Reid wyprowadza ich właśnie z~błędu -- powiedział
Andriej. Wydawało się, że wychwycił, o~czym była rozmowa. I~jak dla
Denisa Gubanowa, odchylał się z~zadowolonym z~siebie uśmiechem, jakby
wiedział o~tym od lat. Prawdopodobnie tak było.

-- Och, nie wiem -- powiedziała Myra. -- To przebiegły skurwysyn. Mówi, że
jego strona nie wie, co mamy, i~ciągle może mieć nadzieję, że nas
przekona, lub wykorzysta nas jako groźbę, żeby zachować porządek we
własnych szeregach.

Odetchnęła głęboko.

-- Poza tym -- dodała -- nie wie o~wszystkim, co mamy. Lub tak zrozumiałam.
Myśli, że jest to na orbicie ziemskiej.

-- \textit{Nie} jest? -- Uśmiech Denisa natychmiast zblakł. -- Więc gdzie
jest?

-- Dobre pytanie -- powiedziała Myra. -- Zobaczymy, czy potrafisz odnaleźć.

Walentyna intensywnie przyglądała się odbiciu żyrandola w~lustrze nad
barem.

-- To żart, czy co? -- zażądał Denis.

Myra pokręciła głową, położyła dłoń na jego dłoni. 

-- Spokojnie,
człowieku. Nie trać zbyt dużo czasu na to, potraktuj to tylko jako
ćwiczenie, zobacz, co możesz odkryć, co ludzie wiedzą lub
podejrzewają\ldots

-- A sam mam nie wiedzieć?

-- Podwójnie ślepy -- powiedziała Myra stanowczo. -- I~podwójny blef.
Powiem Ci, gdy będziesz miał jakieś wyniki, ale nie chcę, żeby Twoje
śledztwo nieumyślnie zostawiło jakieś wskazówki.

Denis spojrzał spode łba. 

-- Ok -- zgodził się -- rozumiem tego cel. -- Spojrzał na zegarek, westchnął i~wstał. -- Trzecia piętnaście -- powiedział. -- Czas wrócić do biura.

-- Bezsenny miecz CZeKa -- powiedziała Myra. -- Zdaje się, czas, żebyśmy
wszyscy wrócili.

-- Nie -- powiedział Andriej. -- Ty i~Walentyna zostańcie tu i~się upijcie.
-- Odepchnął krzesło i~wstał ociężale na nogi. -- My rosyjscy
\textit{mężczyźni} zajmiemy się resztę tego dnia pracy.

-- Tak?

-- Tak. -- Położył dłoń na jej ramieniu. -- Spokojnie, Dawidowa. Zamach nie
nastąpi dzisiaj ani jutro.

-- Wiem o~tym -- powiedziała. -- Ale właśnie straciliśmy dzisiaj jeszcze
jednego komisarza\ldots

-- Aleks, ha, skurwysyn. To nie strata. Sprzątnąłem jego biurko i~zablokowałem go w~chwili, gdy wspomniał, że nas opuszcza.

-- Był dobry w~tej pracy i~nie mamy zastępstwa.

-- Przez jakiś czas, ekonomia poradzi sobie bez komisarza -- powiedział
Andriej. -- Wolny rynek, nie krytykuj. To wszystko jest u Ricardo.

Dwóch mężczyzn poszło do baru. Andriej szarmancko położył na ladzie
zwitek pieniędzy, wskazując Myrę i~Walentynę wzrokiem, skinął im głową i~wyszedł z~Denisem.

-- Zatem -- powiedziała Walentyna, patrząc za nimi -- jak myślisz, co
kombinują?

-- Wszystko, prócz wracania do pracy, mam nadzieję. -- Myra się
roześmiała. -- Picie w~barach kosmodromu, planowanie naszego upadku.
Nieważne. Co, kurwa. -- Wychyliła kolejną wódkę, zapatrzyła się w~czubek
papierosa, który spalał się, ignorowany. Zapaliła kolejnego.

-- Już jesteś pijana -- oskarżyła Walentyna.

-- I~gorzka i~skrzywiona. Tak, wiem.

-- Powiem Ci, dlaczego wyszli -- powiedziała Walentyna. -- To jest, oprócz
atrakcji portu kosmicznego.

-- Ta?

-- Dają nam miejsce, moja droga. Na zebranie.

-- Zebranie kobiet? Trochę to przestarzałe.

Walentyna rozluźniła kurtkę munduru, zdjęła krawat i~ostrożnie go
zwinęła. 

-- Nie, jak to nazywali, feminizmu, Myra. Socjalizmu. Zebranie
Partii.

-- Ale nawet nie jestem w~Partii!

-- Jesteś tego pewna? -- spytała Walentyna. -- Nigdy nie \textit{widziałam}
Twojej rezygnacji. A wiesz, że widziałabym. Jestem pewna, że jesteś
przynajmniej zwolenniczką, nawet jeżeli\ldots -- Zachichotała. -- \ldots
ostatnio nie bywałaś na zebrania Oddziału.

Myra musiała pomyśleć o~tym. Mniemała, że ciągle tam \textit{było}
upoważnienie opłacania jej składek na jakieś konto w~karaibskim raju
informacji. Ciągle dostawała pocztę, nieprzeczytaną. Ciągle pisała dla
\textit{Analysis}, czasopiśmie Międzynarodówki online o~teorii.
(Współpracownicy nazywali je \textit{Dialysis}, ponieważ jego ciągłym
motywem było, że wszystko i~tak spłynie kanalizacją.)

Myra zmarszczyła brwi na Walentynę. Hałas w~barze stał się głośniejszy
niż wcześniej. Ludzie dryfowali z~innych imprez odbywających się w~hotelu: konferencji biznesowej, konwentu anime, i~przynajmniej dwóch
ślubów.

-- Jakie to ma znaczenie? -- spytała. -- Jesteśmy niczym, prawdopodobnie
jesteśmy ostatnimi z~Międzynarodówki na całym jebanym \textit{świecie}.

-- W~istocie jesteśmy -- powiedziała Walentyna. -- Ale ciągle jest kilka
rzeczy, które możemy zrobić. Jedno do podarować naszemu towarzyszowi
dobre pożegnanie poprzez absolutne uchlanie się ku jego pamięci.

Stuknęły kieliszkami, wypiły.

-- A inna sprawa?

-- Och, tak. Możemy sprawdzić, czy jest cokolwiek, co Międzynarodówka
planuje zrobić w~sprawie zamachu.

-- Chyba, kurwa, żartujesz.

-- Nie żartuję. Jeżeli chcesz mojej opinii, to po to chcieli zasoby.

-- Ktokolwiek o~tym myślał, musiał być chory w~swojej małej, jebanej
głowie. Mówimy o~ryzykanctwie.

-- Nie jestem pewna. Pamiętaj, może nie jest już zbyt wiele nas na
świecie, ale\ldots -- Walentyna się pochyliła bliżej -- \ldots to nie jest
jedyny świat.

-- Och, nie \ldots -- Myra przemyślała to jeszcze raz. -- Och -- powiedziała. -- Nasi przyjaciele na niebie.

-- Tak -- powiedziała Walentyna. -- Frakcja Kosmiczna.

-- Nie chcę o~tym rozmawiać teraz -- powiedziała Myra. Rozejrzała się
dookoła, dziko. Miejsce podskakiwało. Piękna kazachska dziewczyna, o~której Myra myślała, że jest panną młodą, krzyknęła coś, co brzmiało jak
japoński. Jej wielka biała suknia skurczyła się jak folia termokurczliwa
na ciele, zmieniając kolor, twardniejąc w~pastelowy kostium z~plastikowej zbroi. Inteligentny skafander -- zrobiony z, raczej niż
przez, nanotechnologię -- był haniebnie drogą nowinką, oferującą
limitowane menu zaprogramowanych transformacji. Myra zastanawiała się,
ile czasu zajmie, zanim cena spadnie, repertuar eksploduje, ile czasu
minie, zanim ludzie będą mogli tak skwapliwie transformować swoje ciała.
Świat superbohaterów z~komiksów, nie dało się o~tym myśleć. Dziewczyna
przyjęła pozę bojową, ku rozproszonemu aplauzowi ze strony innych fanów
anime.

-- Napijmy się -- powiedziała Myra.


\chapter{Kościół Człowieka}

Merrial, tak jak obiecywała, czekała. Siedziała na cokole tak jak ja
wtedy, pod pomnikiem Wyzwolicielki na koniu. Była ubrana w~luźną, letnią
sukienkę i~spódnicę z~kolorowych warstw. Coś poruszyło się w~mojej
pamięci, potem zniknęło jak sen o~poranku. Była pogrążona w~ożywionej
rozmowie z~mężczyzną siedzącym koło niej. Oboje spojrzeli do góry, gdy
podszedłem.

-- Cześć -- powiedziałem ostrożnie.

Mężczyzna był wysokim, chudym mężczyzną, około trzydziestki, jak
sądziłem. Całkiem brązowy, z~ostrymi rysami i~ciemnymi oczami, które
patrzyły w~dziwny, powątpiewający sposób. Jego ciemne włosy kręciły się
na szczycie głowy, krótkie z~boku i~tyłu. Ubrany był w~skórzane spodnie,
kurtkę, biały bawełniany podkoszulek i~czerwoną bandanę. Cienki łańcuch
wisiał na szyi pod bandaną, jego wisior, jeżeli w~ogóle, był pod okrągłym
kołnierzykiem podkoszulka.

-- Cześć -- powiedział ciepło Merrial. -- Clovis, to jest Fergal.

Mężczyzna wyciągnął prawą dłoń i~ją potrząsnąłem, zauważając, że jeden z~jego kciuków nacisnął wierzch dłoni i~że przytrzymał go, jakby czekając
na jakąś odpowiedź, około sekundy dłużej niż podświadomie oczekiwałem,
zanim puścił moją dłoń.

-- Miło mi cię poznać, Clovis -- powiedział. Jego głos był niski i~głęboki, akcent trudny do umiejscowienia: poprawny, ale przez tę
poprawną intonację każdej sylaby, jakoś obcy. Przypominał mi księcia
Zanu\footnote{ ZANU -- ang. Zimbabwe African National Union Afrykański Narodowy
Związek Zimbabwe,
więcej~\url{https://pl.wikipedia.org/wiki/Afryka\%C5\%84ski\_Narodowy\_Zwi\%C4\%85zek\_Zimbabwe}
-- przyp.tłum.}, którego kiedyś słyszałem przemawiającego na
Uniwersytecie.

-- Napijmy się -- powiedział, wstając. Poszliśmy do najbliższego wolnego
stolika na zewnątrz Karonady. Fergal przyjął nasze zamówienia i~zniknął
w środku.

-- Kim jest ten facet? -- spytałem.

Merrial potraktowała mnie uśmiechem. 

-- Brzmisz zazdrośnie -- droczyła
się.

-- Och, przestań. Tylko jestem ciekawy.

-- Znam go od bardzo dawna -- powiedziała. -- Nic osobistego. Po prostu\ldots
jeden z~nas.

-- Cóż, jakby pomyślałem, że jest majsterkiem.

Oczy Merrial lekko się zwęziły. 

-- Tak, to właśnie to -- powiedziała.

Fergal powrócił po kilku chwilach, zajmując miejsce koło mnie i~naprzeciw Merrial. Zaproponowałem mu papierosa, co zaakceptował z~dziwnie ironicznym uśmiechem.

-- Dobra -- powiedział, zapalając -- wiesz o~\ldots wątpliwościach, ze
Statkiem?

Skinąłem głową. -- Tak, ale Merrial nic nie mówiła, że są podzielane.

Uśmiechnął się. 

-- Och, są całkiem szeroko podzielane, mogę ci to
powiedzieć. Oferta, którą przedstawiłeś, jest bardzo odważna, -- rozłożył ręce -- i~wszystko, co mogę powiedzieć, to dziękuję.

Byłem bardziej zdziwiony niż skromny na temat tego wspomnienia odwagi
mojej oferty, więc po prostu wzruszyłem na to ramionami.

-- Też jesteś na Projekcie?

Wydawał się rozbawiony. 

-- Nie jestem na miejscu, ale jestem na liście
płac, jeżeli o~to ci chodzi -- powiedział. -- Wszystkie -- spojrzał na
Merrial -- nasze zawody są bardzo mocno zaangażowane w~projekt jako
całość. -- Wziął dużego łyka piwa i~zaciągnął się papierosem, stając się
widocznie bardziej zrelaksowany i~wylewny, gdy to robił. -- Jego sukces
ma dla nas duże znaczenie. Bardzo pragniemy ujrzeć znowu wytyczoną drogę
do gwiazd.

-- Podoba mi się -- powiedziałem. -- ,,Droga do Gwiazd''.

-- Tak -- odpowiedział. -- Cóż, zabrało to wam ludzie bardzo długo, żeby na
nią wrócić.

-- Wrócić?

-- Kiedyś na niej byliście. -- Kolejne spojrzenie na Merrial, potem
uśmiech do mnie. -- Lub my byliśmy.

-- Nasi przodkowie byli -- powiedziałem.

-- To właśnie miałem na myśli -- powiedział leniwie. -- Ale wracając do
biznesu. Muszę załatwić sprzęt dla was, lub raczej Merrial, którego
będziecie potrzebować. To zabierze trochę czasu, ale poradzę sobie z~tym
w ten weekend. Będziecie musieli zarezerwować czas wolny i~miejsca w~poniedziałkowym pociągu. -- Uśmiechnął się krzywo. -- Tak czy inaczej, nie
ma powodu próbować jechać w~sobotę lub niedzielę. Brak pociągów i~cholernie wolny ruch, nawet gdybyś chciał pojechać.

Skinąłem głową. 

-- A Uniwersytet i~tak miałby wszystkie okna zabite.

-- Tak, racja. Ciągle, nie mogę narzekać, wolny weekend jest jedną ze
zdobyczy klasy pracującej, co?

-- Mógłbyś tak to określić -- powiedziałem. -- Nieważne, to, co się dzieje
na Uniwersytecie powinno się liczyć jako praca\ldots

Kontynuowaliśmy rozmowę przez jakiś czas. Fergal był ostrożny na swój
temat, a~ja nie naciskałem, a~po kolejnych kilku piwach wstał i~nas
opuścił. Mieliśmy wieczór, i~weekend, dla siebie.

~

Merrial spała, opierając się o~moje ramię, całą drogę z~Carron Town do
Inverness. Wydawało się, że to wstyd dla niej przegapić podróż, ale
podejrzewałem, że musiała widzieć sławnie widowiskowy i~zmienny
krajobraz wcześniej, wiele razy częściej niż ja. Poza tym lubiłem
obserwować jej sen, doświadczenie, którym, z~natury naszych trzech
ostatnich nocy, nie mogłem się do tej pory delektować.

Złapaliśmy wczesny pociąg, o~5:15 rankiem w~poniedziałek. Każde z~nas
oddzielnie zorganizowało dwa pierwsze dni tygodnia wolne, szukając
naszych różnych przełożonych w~barach Carron w~piątkowy wieczór.
Należało mieć nadzieję, że Angus Grizzlyback będzie pamiętał, że nie
przychodziłem tego ranka, ale gdyby zapomniał, byłem pewien, że moi
lojalni przyjaciele przypomnieliby mu z~przewidywalnym i, jak to się
zdarza, nietrafionymi spekulacjami, jak zamierzałem spędzić dzień.

Mieliśmy, w~istocie, spędzić sobotę i~niedzielę w~ten właśnie sposób,
bardzo przyjemnie, w~łóżku lub na wzgórzach. W~sobotnie popołudnie
Merrial wyciągnęła pstrąga z~ciemnych, głębokich wód górskiej doliny Alt
na Chuirn. Wyskoczyła z~rzucającą się rybą w~zaciśniętych rękach i~tańczyła pewnie na śliskich kamieniach. Znowu, coś poruszyło się w~moim
umyśle, jak przelotne mignięcie ogona w~wodzie, które, gdy tylko cień
moich myśli na nich spoczął, odpłynął.

Słońce było coraz wyżej, cienie krótsze, najwidoczniej naprzeciw
postępowi pociągu. Zatrzymywaliśmy się w~małych, ruchliwych miastach
wybudowanych dla przemysłu leśnego i~świetlnego, oraz, coraz częściej na
wschód, rolniczego: Achnasheen, Achnashellach, Achanalt, Garve \ldots Silnik
elektryczny prawie bezszelestnie ślizgał się pomiędzy zaskoczonymi
owcami o~krótkiej pamięci, królikami i~sarnami koło torów i~uruchamiał
ciągłą falę zwierząt spacerujących, liżących lub uciekających. Widziałem
szary kształt wilka na Achanalt. Gdy objechaliśmy klif w~Garve,
zobaczyłem dziką kozicę na półce. Zauważyłem orła patrolującego na
wznoszącym się wietrze ponad stokami Moruisg.

Nie budziłem Merrial przy żadnym.

Zapaliłem, raz, przy kawie przyniesionej na terkoczącym wózku przez
dziewczynę w~spodniach w~tartan. Ani dźwięk, ani zapach, ani dym nie
poruszyły w~ogóle Merrial, prócz kilku głębszych wdechów, długich fal w~potokach jej włosów przez jej pierś i~moją klatkę piersiową. Pozwoliłem
jej głowie oprzeć się w~teraz dziwnym zgięciu mojego lewego ramienia i~na zmianę piłem i~paliłem prawą dłonią. To był cichy pociąg, choć było
tłoczno, od urzędników i~handlowców w~ich cotygodniowym ruchu z~domów na
brzegu do pracy w~Inverfefforan lub Inverness.

Na kolanach Merrial, z~lewą ręką -- zakrzywioną jak moja -- ochronnie
wokół tego, leżał nieporęczny worek z~polerowanej skóry, zapięty
rzemykiem do ściągania. Mógł się wydawać nieco bardziej wybrzuszony i~ważyć nieco więcej, niż ten rodzaj torb, które dziewczęta miały zwyczaj
taszczyć ze sobą, ale zauważenie tego wymagało dodatkowej i~szczegółowej
inspekcji. Wewnątrz niego, ukryta pod warstwami różnych dziwactw, które
można oczekiwać w~takim worku -- batystowa chusteczka, kosmetyki,
małokalibrowa amunicja i~tym podobne -- była skomplikowana aparatura,
którą Fergal dostarczył do jej domu wczesnym wieczorem w~niedzielę. Była
zbudowana dookoła kryształu widzenia średnicy około piętnastu
centymetrów, ułożonej w~gnieździe ze zgrabnych zwojów izolowanego
przewodu miedzianego. Najdziwniejszym aspektem, dla mnie, tego
urządzenia był układ delikatnych dźwigni, każda oznaczona literą
alfabetu w~dziwnym porządku:

QWERTYUIOP\ldots\\
Prawdopodobnie, pomyślałem, zaklęcie.

-- Ponure stare miejsce -- powiedziała Merrial, trąc twarz rękawem i~rozglądając się po wilgotnej, brukowanej hali dworca w~Inverness. Jej
policzki zarumieniły się, jej oczy rozszerzyły się pod delikatnym
naciskiem dłoni. Jej suknia, tym razem niebieski aksamit, wyglądała na
nieco zmiętą. Staliśmy w~barze, czekając dwadzieścia minut na 8:30 do
Glasgow.

Spojrzałem na kreozotowany\footnote{ kreozot -- składnik smoły drzewnej o~temp.
wrzenia ok. 200--215 st. C; produkt suchej destylacji drewna, używany do
zabezpieczania drewna przykładowo podkładów kolejowych, rakotwórczy,
zob.~\url{https://pl.wikipedia.org/wiki/Kreozot} -- przyp.tłum.} dach z~szerokimi świetlikami i~zawieszonymi lampami
elektrycznymi. 

-- Przynajmniej nie ma gołębi.

-- Nie mogę powiedzieć, żeby mewy\footnote{ oryg. herring-gulls, mewa
srebrzysta,
zob.~\url{https://pl.wikipedia.org/wiki/Mewa\_srebrzysta} -- przyp.tłum.} były jakimś ulepszeniem. -- Kopnęła stopą, posyłając
głodnego, czerwonookiego ptaka skrzeczącego do tyłu. Jedna strona stacji
wychodziła na dworce, druga na główną ulicę. Układ wydawał się
specjalnie dostosowany dla przeciągów, zimnych, ale nieodświeżających.
Pomimo omszałych ścian i~bruku, stacja była nowsza niż budynki na
zewnątrz, których większość pochodziła z~okresu przed Wyzwoleniem,
jeżeli nie wszystkich trzech wojen światowych.

Skończyłem kanapkę z~bekonem, uśmiechnąłem się do Merrial -- która
mamrotała, częściowo do siebie, z~pełnymi ustami, jakieś zdenerwowane
spekulacje o~ewolucji wstecznej ptaków morskich -- i~ruszyłem do stoiska
z gazetami. Tam uzupełniłem zapasy papierosów i~kupiłem kopię
\textit{Press and Journal}, gazety, która przewyższa nawet \textit{West
Highland Free Press} w~niepoprawnej zaściankowości i~czcigodnej
starożytności. Większość z~jej stron zawierała drobne ogłoszenia o~rybołówstwie, rolnictwie, wydobyciu uranu i~ropy naftowej i, oczywiście,
Narodziny, Małżeństwa i~Pogrzeby. Ostatnia z~tych mogła zająć pół
długiej kolumny drobnym drukiem: ,,Dolleen Starholm, spokojnie we śnie,
w wieku 251 lat, ukochana prapraprababka \ldots '' tu następowała lista
imion. I~czasem (jak w~tym przypadku) dyskretne wskazanie przynależności
do kultu: ,,RIP''\footnote{ RIP od łac. requiescat in pace -- spoczywaj w~pokoju
-- fraza wykorzystywana w~obrządkach chrześcijańskich,
zob.~\url{https://en.wikipedia.org/wiki/Rest\_in\_peace} -- przyp.tłum.} lub ,,IHS''\footnote{ IHS -- chrystogram symbolizujący imię Jezusa,
więcej~\url{https://en.wikipedia.org/wiki/Christogram\#IHS} -- przyp.tłum.}. Znacznie częściej, i~bardziej widoczne, były dumne
oświadczenia ortodoksyjnej nadziei: ,,Powrócił przez Ogień (lub Niebo,
Słońce lub Morze) do Jedności''.

Wróciłem do lady i, gdy Merrial kończyła śniadanie, przejrzałem rzadkie
skrawki narodowych i~międzynarodowych wiadomości, którym udało się
dostać pomiędzy straszliwie ważne wiadomości o~hokeju na trawie i~piłce
nożnej, spory rybackie i~dyskusje Rady.

Kongres Paryża uroczyście otworzył dziewięćdziesiąty siódmy rok obrad i~natychmiast pogrążył się w~gorzkiej kontrowersji w~sprawie propozycji
upoważnienia Sądu Kontynentalnego do orzekania w~sprawach sporów
granicznych pomiędzy kantonami i~gminami. Najwidoczniej znacznie
trudniejsze kwestie nieporozumień pomiędzy krajami zostały już dawno
temu rozwiązane przez Kongres, sukces widocznie uderzył do kolektywnej
głowy.

Westchnąłem i~przewróciłem stronę. Kolejna amerykańska republika
przegłosowała wkład z~dochodów celnych na rzecz Projektu Statku
Kosmicznego, co było satysfakcjonujące i~tajemnicze, był nawet komentarz
redaktora o~tym, pełen mamrotania mędrca o~tym, że ich drogi nie są
nasze, że nie powinniśmy gardzić taką pomocą, niemoralną jakby się mogła
wydawać. Nie byłem pewien, dla mnie, pachniało to kradzieżą pieniędzy,
ale Amerykanie znacznie bardziej szanują swoje rządy niż ludzie w~bardziej cywilizowanych krajach. Gdyby jakiś afrykański król, azjatycki
magnat czy południowoamerykański kacyk zaoferował jakiś łup, mam
nadzieję, że Międzynarodowe Towarzystwo Naukowe grzecznie by odmówiło, a~ta sprawa wyglądała nieco inaczej. Jednak z~tego wszystkiego, w~tym
momencie, całkiem teoretycznie, nie było żadnej takiej oferty i~właściwie żadnych wiadomości z~Azji lub w~Afryki w~dzisiejszym wydaniu.
Złożyłem gazetę i~zdecydowałem się zostawić wiadomości krajowe na
później.

Merrial starła okruszki z~ust i~spojrzała na mnie ubawiona. 

-- Naprawdę wyglądasz, jakbyś traktował to wszystko poważnie -- powiedziała,
podnosząc skórzany worek. Zarzuciłem płócienną torbę na ramię i~poszliśmy do pociągu do Glasgow.

-- Cóż, śledzę wiadomości -- powiedziałem, jakoś obronnie, gdy zajęliśmy
swoje miejsca, tym razem patrząc na siebie ponad stolikiem. -- Co w~tym
złego?

Merrial wzruszyła ramionami. 

-- Są takie\ldots efemeryczne -- powiedziała. -- I~niesolidne.

-- W~porównaniu do?

-- Nie zrozum mnie źle -- powiedziała. -- Jestem pewna, co tu mamy \ldots -- sięgnęła po gazetę i~rozłożyła -- \ldots że Kongres jest prawdziwy, i~naprawdę
zrobił to, co artykuł mówi, że zrobił. Niemniej to tylko drobna część
prawdy i~może nie najważniejsza część tego, co się tam w~Paryżu dzieje.
Nie mówiąc już, co dzieje się poza Paryżem. Więc to, i~wszystkie inne
takie artykuły, przedstawiają ci, w~rzeczywistości, fałszywy obraz
świata.

Mógłby się obrazić, ale nie zrobiłem tego. 

-- Jestem uczonym historii,
pamiętasz? -- powiedziałem. -- Rozumiem, jak gazeta przedstawia, nawet
dokumenty to nie wszystko\ldots

-- Och, nie chcesz usłyszeć, co myślę o~\textit{dokumentach historycznych}!

-- Więc co jeszcze możesz zrobić?

Zmarszczyła brwi, zdziwiona. 

-- Podróżujesz i~poznajesz samodzielnie.

-- Tak, jeżeli jeszcze mielibyśmy czas.

Dotknęła czubka nosa czubkiem palca. 

-- Tak właśnie majsterki robią i~na
tym spędzają cały czas swojego życia.

Pociąg ruszył, Moray Firth na początku w~zasięgu wzroku, z~jego polami
wodorostów i~gospodarstwami rybackimi, potem nic prócz blisko rosnących
sosen w~Drumossie Wood, gdy pociąg skręcił i~silniki przejęły obciążenie
w długim, wolnym wspinaniu się na Slochd.

Kilka godzin później, może, po Speyside słodów i~posępnym Drumochter,
byliśmy w~długich i~pięknych dolinach pomiędzy Blair Atholl i~Dunkeld. Z~jednej strony linii były strumienie pełne pstrągów i~turbin, po drugiej
stronie wzgórza brzęczały piłami i~wiertłami warsztatów. Pociąg
zatrzymał się na pięć minut w~Dunkeld. Małe, stare kamienne miasto,
ciągle z~chrześcijańską katedrą.

Merrial wyjrzała przez okna, po krajobrazie i~usiadła z~powrotem z~lekkim dreszczem.

-- Dziwne miejsce -- powiedziała -- z~tymi wzgórzami dookoła jak w~zasadzce.

-- Ale dlatego to jest wspaniałe miejsce -- powiedziałem i~opowiedziałem
jej historię jak Cameronianie\footnote{ zwolennicy Richarda Camerona, którzy
utworzyli radykalną fakcję w~II połowie XVII wieku,
zob.~\url{https://en.wikipedia.org/wiki/Cameronian} -- przyp.tłum.} powstrzymali gospodarza Highland i~ochronili Rewolucję,
której zawdzięczali własną wolność. Słuchała z~większym
zainteresowaniem, nawet, niż moje opowiadanie zasługiwało, na koniec
oparła się i~powiedziała: 

-- Aye, dobra, może jest jakiś w~końcu pożytek
z historii. Nigdy nie będę już się bać tych wzgórz.

Była druga po południu, gdy pociąg dotarł do stacji Queen Street w~Glasgow i~byliśmy zadowoleni z~możliwości wyjścia. Czasem dwoje osób,
które mogą się nieskończenie fascynować, kiedy są same razem, i~które
mogą rozgrzać się wzajemnie w~biesiadnym towarzystwie, odkrywa siebie
zahamowanych pośród obcych, którzy są nie do zignorowania w~zasięgu
słuchu, i~okazuje się, że są wstydliwi, cisi i~zastali. Tak właśnie było
z nami, ku końcowi tej podróży. Nie mogłem się nawet zebrać opowiedzieć
o Bitwie pod Stirling, kiedy przechodziliśmy przez miasto.

Oboje jednak rozweseliliśmy się, schodząc na dworzec. Znajomy zapach
dworca w~Glasgow -- czyszczonych ryb, gnijących liści, ozonowanego
powietrza, starego żelaza, metanolu, gorącego oleju i~spalonej wanilii -- uderzył w~moje zatoki jak kieliszek bimbru. Merrial, także, wydawała się
tym ożywiona, biorąc głębokie wdechy zepsutego zapachu z~wyrazem twarzy
satysfakcji i~nostalgii.

-- Och, dobrze wrócić -- powiedziała.

Spojrzałem z~boku, gdy schodziliśmy z~peronu. 

-- Kiedy byłaś w~Glasgow? I~jak mogłem na Ciebie nie trafić?

Uśmiechnęła się i~ścisnęła dłoń. 

-- Och, zapomniałam. Wieki temu. Jednak
zapach przypomina.

-- To i~hałas.

-- Co?

-- HA\ldots

Jednak się ze mnie śmiała.

Przeszliśmy halę dworcową, zgadzając się, że gołębie są większym
kłopotem niż mewy (choć, jak Merrial poważnie to zauważyła, lepszym
jedzeniem). Ten komentarz i~pewne bardziej apetyczne składniki zapachu,
przypomniały nam, że jesteśmy wygłodniali, więc kupiliśmy sandwiche i~butelki piwa ze straganu na stacji i~zanieśliśmy je na George Square.

Usiedliśmy na ławce przy trawiastym pagórku pod posągiem Wyzwolicielki.

-- Po kąt -- powiedziała Merrial, wskazując w~górę, gdy żuła. -- Nimłe.

-- Co?

Przełknęła. 

-- Posąg. Starzy ojcowie miasta musieli być nieco chytrzy.

Spojrzałem w~górę. 

-- Nic nowego o~ojcach miasta -- powiedziałem. -- Nadal
są skąpi. Ale posąg wygląda dobrze.

-- Koń jest czarny -- wskazała Merrial. Postukała rączką noża o~pęcinę. -- Odlany z~brązu. Sama pani jest zielona, tylko miedź. Wzięli palniki
acetylenowo-tlenowe i~odcięli pierwotnego jeźdźca, króla, generała,
cokolwiek i~na jego miejsce wsadzili Wyzwolicielkę!

Wstałem, obszedłem dookoła, przyglądając się.

-- Masz rację -- powiedziałem. -- Widzę połączenia. Musiałem z~tysiąc razy
patrzeć na statuę i~nie zauważyłem w~niej niczego dziwnego. -- Spojrzałem
na głowę posągu. -- I~ma inną twarz niż ta w~Carron Town i~obie są różne
od zdjęć Wyzwolicielki, które widziałem.

-- Cóż, proszę bardzo, colha Gree -- powiedziała. -- Są rzeczy, które
majsterek może nauczyć uczonego, co?

-- Och tak -- powiedziałem. Usiadłem z~powrotem. -- Uważasz, ciężko, żeby
to była oszczędność, to jednak dobry egzemplarz, i~nawet zrobili jej
włosy w~złocie.

-- To złota \textit{farba} -- powiedziała pogardliwie. -- A co do rzemiosła,
rasa i~kropierz konia są błędne dla tych czasów i~okoliczności.

Znowu miała rację, kiedy spojrzałem. To nie był stepowy koń, prawie na
oklep, z~grubsza osiodłany, jak to było pokazane całkiem autentycznie na
placu w~Carron. Zamiast tego, to było siodło husarza, z~wyszukanym
rzędem. Jednak wtedy myślałem, i~ciągle myślę, że przedstawienie samej
Wyzwolicielki było dobrze oddane. Dobry przykład stylu Glasgow, który,
może, naprawiał fuszerkę z~koniem i~częściowo pokazywał punkt widzenia
artysty.

Wyrzuciliśmy śmieci do kosza i~poszliśmy na najbliższy przystanek
tramwajowy, na Buchanan Street. System transportu był jednym z~najwspanialszych osiągnięć Rady Miasta Glasgow, bardziej niż
dostatecznym zastąpieniem wielkiego kręgu Metra, które było, jak mówią,
jednym z~cudów starożytnego świata. Sądząc po jego pozostałościach,
które tu i~tam przetrwały wieki zalewania i~osiadania, można się
zgodzić, że tak rzeczywiście było.

Nadjechał tramwaj, dzwoniąc, i~weszliśmy, zapłaciliśmy monetą i~wbiegliśmy jak dzieci po spiralnych schodach na górny pokład. Dzwonek
znowu zadzwonił i~tramwaj szarpnął do przodu, skrzecząc po Buchanan
Street i~kołysząc, gdy skręcał na rogu w~Sauchiehall.

Główna ulica Glasgow wydawała się zatkana ruchem, ale wszystkie pojazdy
-- z~silnikiem parowym, samochody, wozy konne i~rowery -- robiły miejsce
dla nieubłaganego ruchu tramwaju. Piesi, o~tej porze dnia, były
kobietami na zakupach. Jednak one wszystkie, czy to młode dziewczyny
prosto ze szkoły, czy matki z~dziećmi, czy emerytowane damy w~wolnym
czasie, musiały łapać swoje spódnice, worki czy dzieciaki i~uciekać
panicznie, kiedy tramwaj przewalał się przez skrzyżowanie. Sklepy i~biura z~ostatnich wieków były zbudowane z~kłód i~desek i~rzadko były
wyższe niż dwa piętra. Starsze, budynki sprzed Wyzwolenia, były budowane
z kamienia, niektóre miały nawet pięć pięter. W~dawnych czasach były
znacznie wyższe budynku, ale większość z~nich była zbudowana z~betonu,
który nie radzi sobie dobrze i, choć to może być bolesne dla
archeologii, prawie wszystkie takie struktury zostały już dawno temu
splądrowane ze stali i~szkła. Ich fundamenty dawały prostokątny wzór
wzrostu drzew w~lasach dookoła Glasgow: Pollock Fields, Possil Wood,
Partick Thorn.

Jeszcze dalej, na zachodzie, mogliśmy tylko dojrzeć mgłę i~dym ze
Stoczni w~Clydeside, od której zależała pomyślność Glasgow. Stocznie
były rozsadnikiem umiejętności, które -- razem z~mokrym dokiem w~Kishorn,
prawie unikalnym po tej stronie Atlantyku -- były odpowiedzialne za
logiczny wybór Szkocji jako budowy platformy startowej.

Na szczycie Sauchiehall był nowy kamienny most, zastępujący oryginalny
betonowy, który się rozsypał. Przeprowadził nas ponad Ósmą Autostradą i~w Woodlands Road, która biegnie koło Kelvin Woods. (Ta, i~rzeka, która
biegnie przez nie, zostały nazwane po Lordzie Kelwinie, który wynalazł
termometr).

Wyszliśmy z~tramwaju na szczycie University Avenue i~staliśmy przez
chwilę, obserwując główny budynek, wielką i~starożytną kupę zwaną
Gilmore-hill. Wyglądał jak przykład architektury religijnej, który
zdziczał, ale był poświęcony wyłącznie świeckiej wiedzy, kościół
Człowieka.

-- Nie jest tak stary, jak wygląda -- powiedziała Merrial, jakby
zdeterminowana nie być pod wrażeniem. -- To wiktoriański gotyk.

Nie wierzyłem jej, ale się nie spierałem. Czułem w~jego chłodnych
kamieniach i~ciepłym drewnie, cienie Dunsa Szkota\footnote{szkocki filozof i~teolog,
zob.~\url{https://pl.wikipedia.org/wiki/Jan\_Duns\_Szkot} -- przyp.tłum.}, Knoxa\footnote{ prawd. John Knox -- duchowny kalwiński,
pierwotnie kapłan katolicki, przywódca reformacji w~Szkocji, czołowy
twórca narodowego protestanckiego Kościoła Szkocji,
zob.~\url{https://pl.wikipedia.org/wiki/John\_Knox} -- przyp.tłum.}, Kelvina\footnote{ William Thomson, lord Kelvin -- brytyjski fizyk, matematyk i~przyrodnik,
zob.~\url{https://pl.wikipedia.org/wiki/William\_Thomson} -- przyp.tłum.}, Watta\footnote{James Watt -- szkocki inżynier i~wynalazca, twórca
kilku kluczowych ulepszeń konstrukcji maszyny parowej,
zob.~\url{https://pl.wikipedia.org/wiki/James\_Watt} -- przyp.tłum.}, Millara\footnote{ John Millar of Glasgow -- szkocki filozof,
historyk,
zob.~\url{https://en.wikipedia.org/wiki/John\_Millar\_(philosopher)}
-- przyp.tłum.} i~Fergusona, i~żadna kwestionowana data nie mogłaby mnie
przekonać, że to miejsce było niemal tak stare jak naród, którego umysł
tak mocno został ukształtowany przez to miejsce.

-- Nieważne -- powiedziałem. -- Tak czy inaczej, Wydział, na który idziemy,
nie jest tutaj.

-- Bardzo dobrze -- powiedziała Merrial.

To była właściwie jedna z~mniejszych bocznych ulic przy University
Avenue, gdzie budynki datowały się przynajmniej na dwudziesty wiek.
Drzewa, które ją wyznaczały, były prawdopodobnie tak stare, gigantyczne
wieże gałęzi i~liści, wyższe niż budynki. Ich cień zaciemniał ulicę,
pierwsze opadłe liście tworzyły śliskie śmieci pod stopami.

-- Więc po prostu podejdziemy i~zastukamy do drzwi? -- spytała Merrial.

-- Nie -- odpowiedziałem. -- Mam klucz.

Spojrzała na swoją skórzaną torbę. 

-- I~jesteś pewien, że nie będziemy
sprawdzeni?

-- Ta, jestem pewien -- powiedziałem. Przerobiliśmy to wcześniej. Jako
potencjalny student, już z~zaakceptowanym projektem, nawet jeżeli nie
sfinansowanym, miałem prawo tutaj być, w~istocie, powinienem bywać tutaj
znacznie częściej, przez lato. Więc nikt nie powinien podawać w~wątpliwość nas, lub naszą obecność w~starym archiwum. Planowaliśmy, jak
wykonamy pracę, ale bliskość wydawał się denerwować Merrial bardziej niż
mnie.

-- Dobra -- powiedziała.

Klucz obrócił się gładko w~nasmarowanym zamku i~pstryknęła zasuwa.
Pchnąłem ciężkie drzwi na bok i~weszliśmy. Zamknąłem za nami. Miejsce
było ciche i,o ile potrafiłem powiedzieć, puste. Hol był ciemny i~chłodny, jasnożółta farba przyciemniona pokoleniami nikotyny. Dzielił
się co kilka metrów na węższe korytarze prowadzące głębiej w~Instytut i~klatkę schodową prowadzącą na wyższe piętra. Miejsce miało ciekawy
stęchły zapach starego papieru, zakurzonych elektrycznych żarówek i~delikatny powiew dymu z~fajki. Sprawdziłem stos nieotwartej poczty na
długim drewnianym stole z~boku. Kilka wiadomości do mnie, które po
krótkim sprawdzeniu były odmowami różnych wniosków o~mecenat. Wepchnąłem
je do kieszeni kurtki i~ruszyłem na górę po schodach do biblioteki,
włączając syczące lampy elektryczne, jak szliśmy.

Merrial zmarszczyła nos, gdy otworzyłem drzwi biblioteki i~włączyłem
światła.

-- Stary papier -- powiedziała.

Uśmiechnęła się. 

-- Martwe muchy.

Ruszyłem zamykać drzwi, gdy weszliśmy do pokoju, ale Merrial dotknęła
mojej dłoni i~potrząsnęła głową.

-- Nie mogłabym tego znieść -- powiedziała.

-- Racja, ja też. -- Czułem zadyszkę w~nieruchomym, martwym powietrzu.
Przytrzymałem jej dłoń, dla jej i~swojego uspokojenia, gdy szliśmy przez
labirynt wysokich do sufitu regałów z~książkami. Merrial, ku mojemu
zaskoczeniu, raz czy drugi szarpała, żebym się zatrzymał, podczas gdy
ona przeglądała tytuły i~nazwy na złamanych i~wyblakłych grzbietach z~miną rozpoznania i~przyjemności.

-- \textit{Protokoły sądowe antysowieckiego bloku prawicy i~trockistów}!\footnote{ ostatni z~trzech procesów moskiewskich, które
zorganizowano w~1938 roku przed Kolegium Wojskowym Sądu Najwyższego ZSRR
przeciwko dawnym przywódcom bolszewickim; kulminacyjny publiczny punkt
wielkiego terroru.
zob.~\url{https://pl.wikipedia.org/wiki/III\_proces\_moskiewski}
-- przyp.tłum.} -- westchnęła. -- Niesamowite! Wiesz cokolwiek o~tym?

-- To był jakiś rodzaj publicznych egzorcyzmów -- powiedział, pośpieszając
ją. Kiedyś spojrzałem do tego ponurego \textit{grymuaru} i~wspomnienie
lekko mnie zemdliło. -- Ludzie twierdzili, że zmieniali się we wściekłe
psy, które niszczyły maszynerię. Straszne. Jak przesądni musieli być
komuniści.

Merrial zachichotała, ale spojrzała na mnie dziwnie zadowolonym
wzrokiem.

Na samym końcu biblioteki rzędy regałów się skończyły. Było tam
ustawionych kilka tablic i~krzeseł, najwidoczniej do studiowania, ale
nikt, według mojej wiedzy, nigdy tu nie studiował. Większość, co
ktokolwiek mógł zrobić, to położyć stos książek i~dokumentów na szybkie
przejrzenie ich zawartości pod lampkami do czytania, przed wybiegnięciem
z biblioteki. Przypomniałem sobie komentarz Merrial, że ludzie teraz
byli bardziej klaustrofobiczni niż ich przodkowie.

Poza tymi stołami były kolejne drzwi, żelazne, z~klamką, ale bez zamka.
Sama myśl możliwości, że drzwi mają zamek, wystarczyła, żeby oblały mnie
zimne poty.

-- Oto jesteśmy -- powiedziałem i~dodałem, żeby wyjaśnić -- mroczne
archiwum.

-- Co jest w~środku?

-- Nie wiem -- powiedziałem. -- Nigdy tam nie byłem.

Zmarszczyła brwi. 

-- Czy to jest niedostępne, czy co?

-- Nie, nie. -- Potrząsnąłem głową. -- Nie jest zakazane, czy coś. Rzadko
kto chce tam wejść.

-- Nie ma powodu do wahania -- powiedziała Merrial. -- Miejmy to za sobą.

Nacisnąłem klamkę i~pociągnąłem drzwi. Dopasowując się do moich uczuć,
powinny wydać nadprzyrodzony pisk, ale ciężkie zawiasy były dobrze
nasmarowane. Kilka razy nacisnąłem klamkę od wewnątrz. Wydawała się w~porządku, ale przyciągnąłem jedno z~krzeseł i~użyłem go do podparcia
otwartych drzwi, na wypadek, gdyby przypadkowo się zamknęły.

Włączyłem światło nad głową i~wkroczyłem śmiało przez drzwi. Mały pokój
z tyłu wydawał się wystarczająco niewinny. Stało tam biurko, kilka
krzeseł przed nim, a~na biurku stos skrzynkowych, masywnych struktur jak
modele starożytnej architektury. Aluminiowe regały były ustawione po obu
stronach. Powietrze miało inny, subtelniejszy zapach, prawie jak zapach
umytych włosów lub polerowanego rogu, z~ostrą nutą acetonów.

Merrial powąchała. 

-- Jak gnijący plaster miodu -- zauważyła wesoło.
Zwalczyłem falę.

-- Czy papieros pozbędzie się miazmatów? -- zasugerowałem.

-- Tak, ale mógłby uszkodzić dyski.

Podczas gdy ja ciągle się rozglądałem za czymś, co trochę przypominałoby
dysk, Merrial zaczęła grzebać na półkach. Ustawione tam pudełka były
półprzezroczyste, koloru owczej skóry, z~zakurzonymi, ciasnymi
pokrywkami. Zawierały płaskie czarne talerze około dziewięciu
centymetrów kwadratowych i~dwóch milimetrów grubości. Wyciągnęła kilka
losowo, podniosła i~lekko je potrząsnęła. Z~każdego uniósł się sadzowy,
czarny pył. Kryształy utleniania pokryły skorupą małe, metalowe płytki
na krawędziach. Potrząsnęła głową. 

-- Beznadziejne -- powiedziała.

W innych, mniejszych pudełkach, były mniejsze, błyszczące wafle. Te,
kiedy podniosła, po prostu skruszyły się od dotyku.

-- Tyle dla nich -- powiedziała. -- Musimy po prostu zobaczyć, czy coś
zostało na twardych dyskach. -- Wysunęła siedzenie przed maszynami.
Największa, przed którą usiadła, miała rodzaj szybki na przodzie.
Merrial otworzyła worek, przegrzebała bałagan na górze i~ostrożnie
wyciągnęła dziwne urządzenie. Położyła je na stole: kryształ widzenia
świecący losowymi tęczowymi falami, mała ciasna skrzynka i~ramka dźwigni
z literami, wszystko połączone zwojami izolowanego drutu miedzianego.

-- Och, popatrz, ta rzecz tam ma te same\ldots

-- Nie dotykaj!

-- Dobra.

Spojrzała na mnie. 

-- Przepraszam za naskoczenie. Jestem trochę nerwowa.

-- Aye, dobra, ja też.

-- Do tego jestem w~trybie majsterkowania. -- Uśmiechnęła się. -- Uprzejmość
do niego nie należy. Jeżeli chcesz pomóc, zobacz, czy jest jakieś źródło
elektryczne dla tej rzeczy, podczas gdy ja przygotuję system. -- Pomachała niejasno ręką w~ciemność pod stołem.

Powstrzymując wątpliwości, pochyliłem się w~ciemność, a~po chwili, gdy
moje oczy dopasowały się, zobaczyłem zakurzone gniazdko elektryczne, z~trzema otworami. Kabel grubości centymetra wisiał z~tyłu stołu i~kończył
się wtyczką z~trzema kolcami. Wydedukowanie jak wtyczka i~gniazdko
pasowały do siebie, było kwestią chwili tak jak wsunięcie jednego w~drugie.

Światło dookoła mnie nagle się rozjaśniło. But Merrial uderzył mnie w~żebra, a~ona jednocześnie wypowiedziała dziwne przekleństwo.

-- Co?

-- Chryste, nie \textit{rób} tak!

Kolejna dziwna modlitwa. Wyczołgałem się do tyłu spod stołu. Merrial
spojrzała na mnie ostro.

-- Myślałem, że chciałaś, żebym to zrobił -- zaprotestowałem.

-- Och. -- Pomyślała o~tym. -- Zdaje się, że mogłeś to w~ten sposób
zrozumieć, tak. Wybaczam Ci. Teraz chodź i~siadaj. -- Poklepała siedzenie
koło siebie.

Gdy wstałem, zauważyłem, co się stało z~maszyną i~skąd brało się
dodatkowe światło. Okno na przodzie pudła świeciło się perłową szarością
z ciemniejszymi i~jaśniejszymi plamami wirującymi po nim, jak niebo na
portem w~śnieżny dzień. Cofnąłem się o~krok. Temperatura w~pokoju
zdawała się spaść o~kilka kelwinów. Teraz rozumiałem, dlaczego
wypowiadała te inwokacje. W~momentach takich jak ten nawet najbardziej
racjonalna osoba będzie wypowiadać dowolne imię bóstwa, które przyjdzie
na myśl.

-- To nie gryzie -- powiedziała.

Ostrożnie podszedłem, przezornie patrząc na rzecz, jak ktoś mógłby to
robić z~psem, o~którym właśnie usłyszał takie zapewnienie. Ręką, której
Merrial nie mogła zobaczyć, zrobiłem znak Rogów, potem pomyślałem, że
było to haniebnie przesądne i~zamiast tego zacząłem w~głowie recytować
kilka Imion Jednego i~jego Proroków: Allah, Budda, Chrystus, Bóg,
Jordan, Sprawiedliwość\ldots

-- Czy ja to zrobiłem? -- spytałem.

Chomeini, Kryszna, Łaska, Maria, Odyn, Konieczność, Natura\ldots

-- Tak, kiedy włączyłeś zasilanie.

Paine, Opatrzność, Quine, Rozum, Jahwe, Zoroaster. To powinna
wystarczyć.

Patrzyła w~moje oczy z~figlarną uciechą, sięgnęła i~pogłaskała twarz.
Szuranie mojej brody zabrzmiało niesamowicie głośno.

-- Wszystko w~porządku, \textit{mo gràidh} -- powiedziała. -- Jestem majsterkiem. Wiem, co robię. Ta
rzecz tutaj \ldots -- poklepała po górze -- to tylko maszyna, która robi te
same rzeczy co kryształy widzenia, tylko nie tak dobrze. -- To nie
diabeł, wiesz. To komputer.

-- Tak, wiem to\ldots

-- Dobra, to zacznij się zachowywać, jakbyś w~to wierzył -- powiedziała.

-- Ale czy to \textit{telewizja}? -- Wewnętrznie zadrżałem od nazwania tego
mrocznego instrumentu Posiadania.

Potrząsnęła głową. 

-- Nie. To tutaj to klawiatura, a~to tu to ekran.
Ekran, lub monitor, pracuje na podobnej zasadzie co telewizja, ale nie
jest telewizją. A nawet gdyby, nie mógłby ci wyrządzić krzywdy.

Łatwo jej to powiedzieć, pomyślałem, ale mądrze nic nie powiedziałem.

-- Zakładając, że w~ogóle jeszcze pracuje -- dodała wesoło. -- Scalaki
usmażyły się w~Wyzwoleniu, w~przeważającej wielkości.

(Ja też nie, ale tak właśnie powiedziała).

Postukała w~kilka klawiszy. Szary ekran w~ogóle się nie zmienił.

-- Control Alt Delete -- powiedziała do siebie i~nacisnęła na raz trzy
przyciski.

Znowu nic się nie wydarzyło.

-- Hmm -- powiedziała. Sięgnęła do przodu i~szturchnęła słupek na
maszynie. Ekran zmienił się w~czarny.

-- Tyle z~tego -- powiedziała. Wstała, pochyliła się nad stołem i~zaczęła
przyglądać się bliżej niektórym pudłom.

-- Hej! -- powiedziała. -- Mam to! Jeden z~tych chyba jest zabezpieczony
przed promieniowaniem! -- Sięgnęła pomiędzy pudłami i~zaczęła
niebezpiecznie ruszać przewodami pod napięciem, zdejmując przewód z~tyłu
skrzynki, której używaliśmy i~wkładając go z~tyłu innej. Co wydawało się
jedynie pustym przodem tej skrzynki, nagle się rozświetliło, gładko
lśniąca szarość, okazując się być ekranem.

-- Taaak! -- powiedział Merrial, uderzając w~powietrze.

W tym momencie zacząłem brać się w~garść, choć muszę przyznać, że prawie
wymiękłem, kiedy Merrial się odwróciła i~szturchnęła litery na
klawiaturze, a~na ekranie błysnął napis ,,Demon Internet Software''\footnote{
Demon Internet -- brytyjski dostawca internetu,
por.~\url{https://en.wikipedia.org/wiki/Demon\_Internet} -- przyp.tłum.}.

Allah, Budda, Chrystus\ldots

-- Dobra -- powiedziała energicznie Merrial, gdy ekran z~trzema
złowieszczymi nazwami zniknął i~został zastąpiony obrazem z~wieloma
małymi obrazami rozrzuconymi po nim. -- Wreszcie ten gnojek wstał i~chodzi, ale Chrystus wie, jak długo wytrzyma. -- (Mówiła w~ten sposób,
zacząłem zauważać, z~tą interesującą kombinacją odniesień niejasno
seksualnych i~religijnych, kiedy była w~tym, co sama nazwała jej
,,trybem majsterkowania''). -- Więc lepiej wyduśmy to z~niego ej es ej
pi.

-- Z~niego co?

-- As. Soon. As. Possible\footnote{ w~oryg. ASAP -- tak szybko jak to możliwe -- przyp.tłum.}.

-- Och, racja. Jak strzała.

-- Co?

Machnąłem dłonią. 

-- Zabierzmy się do tego, jak mówisz.

-- Dobra.

Ostrożnie odwinęła jedno pasmo drutu miedzianego i~przyłączyła mały
kołek z~miedzianą szpilką na końcu. Potem wsunęła to w~okrągłą dziurę
(która, jak wyjaśniła, nie musiała być jebana okrągła kurwa wtedy, ale
kurwa była) na froncie komputera.

-- Dobra -- powiedziała. Czubek języka pomiędzy ustami, gdy wstukała słowa
,,Myra Godwin'', nazwisko Wyzwolicielki, na klawiaturze. Napisy
jednocześnie pojawiły się na ekranie i~na teraz czarnym krysztale
widzenia.

-- Go -- powiedziała, naciskając kolejny klawisz.

Minęło kilka sekund (język znowu pomiędzy jej zębami) i~ekran oraz
kryształ wypełniły się listą tytułów, które powoli pięły się w~górę, ich
góra znikająca z~widoku i~tak trwało to przez kilka minut.

Kiedy lista przestała pełznąć, powiedziała: 

-- Ok, kopiuj. -- I~znowu
zastukała na klawiaturze. Obraz klepsydry pojawił się na ekranie i~piasek zaczął się przesypywać. Kryształ widzenia, w~międzyczasie,
pokazał drzewo, rozgałęziające się, kwitnące i~hodujące liście.

Po około półtora minucie i~połowie piasku wszystko przepłynęło z~górnej
połowy klepsydry i~kryształ był wypełniony zielenią. Oba wyświetlacze
zniknęły.

-- To jest to -- powiedziała Merrial.

-- To wszystko?

-- Tak. -- Uśmiechnęła się. -- Wszystkie pliki, które wymieniają Myrę
Godwin, przetransferowane z~mrocznego magazynu do kamienia. Poszło
nieźle, co?

-- Cudownie -- odpowiedziałem. Wstała, pochyliła się znowu za komputerem,
odłączyła swój przewód i~zwinęła go szybko dookoła dłoni. Potem dźgnęła
kilka klawiszy na obu klawiaturach. Ekran znowu stał się błyszczącą
szarością, a~kryształ stał się czarny.

Uśmiechnęła się do mnie. 

-- Masz moje pozwolenie na wyłączenie zasilania.

Zostawiliśmy mały pokój i~większą bibliotekę, dokładnie tak jak je
znaleźliśmy, i~poszliśmy po cichu po schodach z~Instytutu. Kiedy byliśmy
kilka metrów na ulicy, przytuliliśmy się i~krzyknęliśmy.

-- Zrobiliśmy to! -- chełpiła się Merrial. -- Rzeczywiście to kurwa
zrobiliśmy!

-- Tak, ciągle nie mogę w~to uwierzyć -- powiedziałem. Złapałem jej dłoń.
-- A teraz co robimy?

-- Przeglądamy to, co mamy -- powiedziała. -- Gdzieś, gdzie nikt nas nie
zobaczy, ani nie będzie przeszkadzał.

Znałem takie miejsce.

~

Z powodu wakacji, mało było wokoło studentów, więc moja właścicielka
mieszkania była szczęśliwa z~możliwości wynajmu mojego zwykłego małego
pokoju nad księgarnią na Southpark Avenue na jedną noc. Nie podniosła
brwi, gdy brała moje pięć marek i~podawała klucz, chociaż było około
wpół do piątej popołudniem. Mniemam, że założyła, że chcemy użyć pokoju
do seksu.

Dała nam szybko po kubku kawy, wypaliła papierosa i~podzieliła się
lokalnymi plotkami z~ostatnich kilku miesięcy, z~tyłu jej kuchni, potem
machnęła na nas z~mrugnięciem do mnie. Pokój był dość hojny, choć
teoretycznie pojedynczy, łóżko, krzesło, stół i~gniazdko. Okno było
zostawione otwarte, ale jedynym widokiem było podwórko. Jednak, każdy
mógłby wyjrzeć i~zobaczyć niebo w~każdej chwili.

-- Doskonale -- powiedziała Merrial.

Wyładowała kryształ widzenia i~elementy peryferyjne, rozłożyła je na
stole, prowadząc mały przewód od czarnego pudełka do gniazdka w~ścianie.
Małe pudełko zaczęło delikatnie szumieć i~w tym samym momencie ludzka
twarz wyłoniła się z~ciemności kryształu widzenia, poruszając ustami w~zmartwieniu.

-- Och, jebać to -- powiedziała Merrial. Potarła kamień mankietem i~twarz
rozpadła się w~drobinki koloru. -- Teraz -- powiedziała -- zajmijmy się
sortowaniem i~szukaniem. Szukamy rzeczy sprzed Wyzwolenia, ale
znalezienie ich w~tej działce nie koniecznie będzie proste. Miejmy
nadzieję, że pliki są podatowane.

Usiadła na krześle, kierując mnie do przycupnięcia na stole, i~zaczęła stukać
na jej wersji klawiatury. 

-- Ach, dobrze, możemy posortować po dacie.

Lista pojawiła się w~głębokościach szklanego kamienia, tym razem ze
stosem artykułów na szczycie z~pojedynczą datą 28 maja 2059 roku.
Merrial musnęła palcem delikatnie i~powoli wzdłuż małej listwy na
klawiaturze, potem uderzyła kolejny klawisz. 

-- Zobaczmy, co to jest.

Razem zajrzeliśmy w~szkło i~zaczęliśmy czytać.

\textit{Bankructwo dowolnej perspektywy dla przezwyciężenia kryzysu, rządząca
elita może jedynie siedzieć i~obserwować, jak społeczeństwo rozpada się
pod nimi. Fabrykom nie udało się wypełnić zobowiązań, korupcja jest
rozpowszechniona, a~wartość rynkowa wytwarzana w~gospodarce ciągle
spada. Wiele sektorów przemysłu w~istocie produkuje ujemną wartość: ich
wynik jest wart mniej -- w~pojęciach rynkowych lub innych -- niż surowce,
które wykorzystują. W~istocie są potężnymi organizmami psującymi zasoby.
Przy braku jakiegoś prawdziwego ruchu w~kierunku rynku, lub -- z~drugiej
strony -- inicjatywy ze strony robotników, system może się tylko dalej
rozpadać.}

-- Brzmi jak 2059 -- powiedziała Merrial. -- Od \textit{tego} właśnie
Wyzwolenie nas wyzwoliło.

Skinąłem ostrożnie głową. 

-- Spójrzmy niżej\ldots

\textit{Nie może być całkowicie wykluczone, że moskiewska oligarchia mogłaby
zorganizować jakąś dywersyjną sprawę wojskową, ale to również zbyt
szybko rozwinęłoby swoje własne problemy i~zintensyfikowało te w~centrum.}


-- Cholera! -- powiedziałem.

-- Co?

-- To nie 2059, bardziej 1999!

\textit{Inwazja w~Afganistanie musiała być postrzegana w~tym kontekście.}

-- Nie, to 1979! -- Cóż, -- Zmarszczyłem brwi na datę w~stopce artykułu. -- Właściwie 1980, ale było napisane o~sytuacji w~79. W~Związku
Radzieckim. -- Gorzko się roześmiałem. -- Powodem, dlaczego było trudno
rozróżnić z~początku okres, o~którym mówi, jest to, że to był Związek
Radziecki, którego upadek się rozpoczął, właśnie wtedy w~latach
siedemdziesiątych. Po rozpadzie Związku Sowieckiego, było tylko gorzej i~to się rozszerzało.

Tyle było dość dobrze przyjętą relacją historyczną, którą poznałem na
studiach licencjackich w~Starożytnej Historii.

-- Więc dlaczego jest datowane 2059? -- spytała Merrial. Musnęła pasek i~przesunęła listę w~dół. -- Ha! -- powiedziała. -- Ten plik, i~wiele innych
według tego jak wyglądają, było wgranych \textit{na} ten komputer tego
dnia. Co nie znaczy, że wtedy były stworzone. Nie wiem też, czy mogę
wyciągnąć oryginalną datę utworzenia.

-- Poczekaj chwilę -- powiedziałem. -- Może tutaj będę mógł pomóc.
Powinienem być w~stanie powiedzieć zgrubne daty na podstawie tytułów
plików lub, może, szybkiego zajrzenia w~ich zawartość.

-- Tu są tysiące plików -- wytknęła. -- Jeżeli datowanie każdego z~nich
zabierze tyle, ile nam zajęło przy tym jednym, będziemy tutaj całą noc.

Uśmiechnąłem się. 

-- Dlaczego miałby to być problem?

Okazało się, że nie był to problem. Choć większość miała tę samą datę w~kolumnie ,,data'' na maszynie Merrial, a~ona zrezygnowała z~odszukania
czegoś, co nazywała ,,data utworzenia'', całkiem duża liczba plików
miała odniesienia do daty swego rodzaju w~tytule. To były najwidoczniej
artykuły z~czasopism i~gazet, napisane przez Myrę Godwin lub o~nie.
Całkiem szybko wpadliśmy w~tryb pracy, który pozwalał mi zidentyfikować
takie pliki, a~Merrial poradzić sobie poprzez skopiowanie daty z~tytułu
do inne kolumny o~nazwie ,,data''. Po dziesięciu minutach tego, uderzyła
się w~czoło ręką i~krzyknęła: 

-- Stop!

-- Co jest?

-- Tracimy czas. \textit{Ja} tracę nasz czas, mam na myśli. -- Roztarła
dłonie. -- Potrzebujemy tutaj maleńkiego programu, żeby przejrzał tytuły
za datami, wyciągnął daty, przeformatował i~potem posortował po dacie\ldots

-- Wierzę Ci na słowo -- powiedziałem, nie zrozumiawszy ani słowa.
Machnęła na mnie, z~miną oderwanej koncentracji na twarzy.

-- To powinno być proste -- powiedziała. -- Zaoszczędzi nam godzin.

Usiadłem na parapecie, paląc papierosa, podczas gdy jej palce migały nad
małą klawiaturą, tworząc stukający hałas jak deszcz na dachu. Uderzyło
mnie, że chyba nie ma dostrzegalnej różnicy pomiędzy białą a~czarną
logiką, ale bez wątpienia to tylko pokazywało moją niewiedzę.

-- Taaak! -- powiedziała. -- Żaden kłopot.

Uderzyła klawisz i~się rozsiadła. Potem znowu pochyliła się, patrząc na
kamień.

-- Och kurwa!

Spojrzałem ostrożnie.

-- Użyłam, kurwa, dwucyfrowych lat. Siła nawyku. Pierdolony program
wypieprza się na roku 2000.

Stukanie znowu się zaczęło.

Pół godziny później Merrial miała pliki częściowo posortowane po dacie i~mogliśmy się w~nie zagłębić z~nieco większą pewnością co do ich ważności
dla naszych rozważań.

-- ,,Umowa polisy obrony (wygaśnięcie), Miasto Watykanu, 11 grudnia
2046''. To wygląda interesująco.

Nacisnęła jeden z~przycisków i~plik, jak to nazywała, się otworzył,
zamiast tytułu rozjarzonego nieco jaśniej pomiędzy innymi, mogliśmy
zobaczyć cały dokument. Część z~niego była nieprzenikalnym językiem
prawniczy (część, w~rzeczywistości, była po łacinie), ale było
dostatecznie dużo, żeby zrozumieć, o~czym był.

Merrial zatrzymała się przed otwarciem kolejnego pliku, opisanego
,,Ochrona Wzajemna/Kosmiczni Kupcy/2058''.

Spojrzeliśmy na siebie, oboje lekko bladzi, każde czekające na drugie,
żeby się pierwsze odezwało.

Merrial przełknęła głośno i~sięgnęła po jeden z~moich papierosów.

-- Wiesz -- powiedziała wolno -- co musiała Wyzwolicielka robić, żeby się
utrzymać, pod Posiadaniem?

-- Cóż\ldots -- Czułem dolną wargę ruszającą się po krawędzi moich zębów i~przestałem. -- Tak. To jeden z~aspektów historii, o~której historycy
starają się nie mówić. To znaczy, w~pracach popularnonaukowych.

-- Och! -- Merrial odetchnęła z~ulgą. -- Zatem wiesz o~obozach niewolników.

-- Co? -- Przez przelotną chwilę, dosłownie zobaczyłem czarny cień przed
oczami. Wskazałem na pismo kryształu widzenia. -- Myślałem, że mówisz o~nuklearnym szantażu!

Merrial spojrzała zdziwiona. 

-- Nuklearny szantaż? Wiem, że miała broń
jądrową od \textit{Papanich}, to jest tutaj. Jaki to ma związek z~tym, jak
się utrzymywała?

-- Och, wielki Rozumie! -- Chwyciłem się za głowę. -- Wyjaśnijmy to sobie.
\textit{Ty} uważasz, że wstydliwym sekretem jest to, że prowadziła
\textit{obozy niewolników}. Ja myślę, że tym było handlowanie groźbami
nuklearnymi.

Merrial westchnęła. 

-- Tak, otóż to. -- Rozłożyła dłonie i~przedramiona w~parodii uprzejmości. -- Ty pierwszy.

-- Dobra. -- Zauważyłem, że moje lewe kolano drga w~górę i~dół. Wstałem i~przemierzyłem pokój, gdy mówiłem. -- Wiesz o~odstraszaniu jądrowym?

-- Och, ta -- powiedziała, grymasząc.

-- Dobra, tak, dla nas polisa zagrożenia wypalenia wielu wspaniałych
miast i~ich mieszkańców wydaje się ohydna, ale starożytni tak tego nie
postrzegali. W~rzeczywistości, niektórzy z~nich zaczęli postrzegać
odstraszanie jądrowe jako dobro, które jak wszystkie dobra byłoby
lepiej, gdyby były sprzedawane i~kupowane przez biznesy niż zapewniane
przez rządy. Problem był, wszystkie bronie jądrowe były własnością
rządów i~niemożliwym było je kupić i~trudno było je ukraść.

-- Zatem Myra Godwin i~jej mąż, Georgi Dawidow, ukradli rząd. Dawidow był
wojskowym i~przeprowadził wojskowy zamach stanu w~części Kazachstanu, w~region był bardzo nieprzyjemny i~jałowy, ale który akurat miał duże
zapasy broni jądrowej. W~pewien sposób, to, co się wydarzyło, to
żołnierze, którzy obsługiwali bronie jądrowe, zdecydowali się zająć
terytorium i~nikt nie śmiał im się sprzeciwić.

-- Lokalni ludzie cierpieli poważnie pod rządami komunistów. Stalin
zagłodził przynajmniej milion z~nich w~latach trzydziestych\footnote{ głód w~Kazachstanie 1932-1933,
zob.~\url{https://en.wikipedia.org/wiki/Kazakh\_famine\_of\_1931\%E2\%80\%931933}
-- przyp.tłum.}. Potem sprawy się polepszyły, ale po upadku
komunistów znaleźli się w~gorszej sytuacji pod lichwiarzami, baronami i~właścicielami ziemskimi. Prawdziwa odpowiedź na ich problemy nie była
wtedy znana, lub nie była znana dostatecznie szeroko, i~zaczęli tęsknić
za bezpiecznym, choć ograniczonym, życiem, jakie prowadzili wcześniej.

-- To wtedy Myra i~Georgi mieli przypływ geniuszu. Kiedy Myra studiowała
tutaj, była zwolenniczką człowieka nazwiskiem Trocki, który został
zabity przez Stalina i~stał się sztandarem dla innego rodzaju komunizmu,
oczyszczonego ze zbrodni Stalina. Jakby coś takiego mogło istnieć!

-- Co masz na myśli? -- spytała Merrial, mrużąc oczy.

-- Och, weź, wiesz, \textit{komunizm}\ldots -- Słowo wywoływało u mnie fizyczne
mdłości, jakby obmacywały mnie brudne ręce. -- Wszyscy pilnują
wszystkich, wszyscy \textit{posiadają} wszystkich, a~to po prostu ideał!
Czym to jest jak nie złem? Nie mówiąc już o~rzeczywistości, małej grupie
rządzącej pilnującej i~posiadającej.

-- Jak to pomogło Wyzwolicielce?

Wzruszyłem ramionami. 

-- Mogła w~to wierzyć, kiedy była młoda. Nikt nie
jest doskonały. Jednak kiedy Dawidowowie założyli swoje państwo, zrobili
to w~imię Trockiego, nawet jeżeli już więcej w~niego nie wierzyli.
Zachowali dostatecznie dużo komunizmu, żeby ludzie czuli się bezpiecznie
i dostatecznie dużo wolności, żeby ludzie mogli być szczęśliwi i~bogaci.

Twarz Merrial wyrażała zainteresowanie, ale ostrożnie neutralne.

-- A sposób na bogactwo -- kontynuowałem -- był taki. Zaczęli sprzedawać
opcje użycia posiadanej broni jądrowej. W~ten sposób, państwa, które nie
miały własnych broni jądrowych, mogłyby odstraszać nuklearnie. Byli w~tym całkiem otwarci, ale musieli przestać po Trzeciej Wojnie Światowej,
kiedy ostatnie imperium wzmocniło władzę.

Westchnąłem i~wzruszyłem ramionami. 

-- Przyznaję, to plama na jej
życiorysie. Jednak nigdy ich właściwie nie użyli.

Merrial spojrzała lekko wstrząśnięta. 

-- Zatem uczeni wiedzieli o~tym
cały czas? 

-- Cóż, wiem, co ludzie Godwin zrobili, kiedy stracili ich mały
biznesik z~groźbami nuklearnymi. 

-- Wydaje mi się, że nie wiesz.

Otworzyła kolejny plik. Ten, który czytałem z~narastającym przerażeniem,
był o~zupełnie innej umowie. To był miesięczny raport pracy wykonanej
przez więźniów, strzeżonych przez firmę o~nazwie Ochrona Wzajemna, dla
kolejnego przedsiębiorstwa o~nazwie Kosmiczni Kupcy.

-- Praca więzienna była kolejnym \textit{dobrem} -- powiedziała Merrial -- które nasza Wyzwolicielka uważała, że najlepiej dostarczać na wolnym
rynku.

-- Ale to jest niewolnictwo!

-- W~rzeczy samej -- powiedziała Merrial. -- To dlatego o~tym nie
rozmawiamy. Nie byłabym zaskoczona, gdyby niektórzy uczeni już to
opisali. -- Jej oczy się zwęziły. -- Może niektórzy starsi majsterkowie
wiedzą o~tym biznesie nuklearnym i~to wszystko. Jednak nie mówią o~tym.

Usiedliśmy, patrząc na siebie z~nagłą pasją ludzi, którzy stracili coś,
w co wierzyli i~zostali tylko sami sobie. Było to jeszcze bardziej
gorzkie, ponieważ każde z~nas oddzielnie myślało, że powiedziano nam
najgorsze o~wielkiej kobiecie, byliśmy zadowoleni z~siebie, myśląc, że
jesteśmy dostatecznie dojrzali, żeby wiedzieć to i~zachować to w~tajemnicy przed łatwowierną ludnością, a~oboje odkryliśmy, że zostaliśmy
oszukani przez nasze własne gildie, że była jeszcze mroczniejsza
niewypowiedziana historia. Mój umysł biegł i~czułem nadchodzący ból
głowy. W~tym samym czasie poczułem rodzaj ulgi, drobnego wyzwolenia, gdy
obraz Wyzwolicielki został obalony w~moim umyśle.

Po krótkiej przerwie, kiedy wyszliśmy w~ciepły wieczór na kolację w~restauracji rybnej przy Kelvin, pracowaliśmy z~plikami. Znaleźliśmy dużo
o dziwnej karierze Myry Godwin -- więcej niż potrzeba, by napisać całkiem
sensacyjną biografię -- ale nic o~tym, co zdarzyło się w~okresie samego
Wyzwolenia. Było już po dziewiątej, kiedy Merrial podskoczyła i~syknęła:

-- Kurde! Kurde!

-- Co się dzieje?

-- Znalazłam plik katalogów. Żadnego znaczącego tytułu, możesz w~to kurwa
uwierzyć. I~ma znacznie, znacznie więcej wpisów niż my mamy plików. Mamy
tylko rzeczy niskiej tajemnicy! Reszta jest ciągle w~mrocznym magazynie
Uniwersytetu.

Potarłem moje obolałe oczy i~sięgnąłem po dłoń Merrial. 

-- Więc to, co
tam jest, może być \textit{gorsze}?

-- Ty to powiedziałeś. Może nawet zawierać rzeczy, których szukamy.
Musimy wrócić.


\chapter{Lekkie Bronie}

Dawno temu istniała pewna ojczyzna o~nazwie ,,Międzynarodówka''. Był to kraj
umysłu, kraj nadziei i~obejmował cały świat. Póki pewnego dnia, w~sierpniu 1914, jego obywatele nie poszli na wojnę ze sobą i~świat się
skończył. Wszystko umarło w~tej wojnie, Bóg, Ojczyzna, Międzynarodówka i~Cywilizacja. Umarło i~poszło do piekła. Wszyscy umarli. Ocaleńcy
myśleli, że żyją, ale tak nie było. Po sierpniu 1914 roku nie było
żywych ludzi na świecie, tylko martwi ludzie na urlopie, przeklęci i~demony.

Ostatni moralnie odpowiedzialni ludzie na świecie byli frakcją
Socjaldemokratycznej Partii Niemiec\footnote{
więcej~\url{https://pl.wikipedia.org/wiki/Socjaldemokratyczna\_Partia\_Niemiec}
-- przyp.tłum.} w~Reichstagu. Zagłosowali za kredytami na wojnę Cesarza
wbrew każdej, swojej uchwale. Wiedzieli, co powinni zrobić, i~wybrali
zło. Cała, następująca po tym, historia dotyczyła tych przeklętych, biednych
diabłów walczących w~piekle, do którego ci ludzie ich wepchnęli. I~nikt
nie mógł być osądzony za to, jak się zachowali w~piekle.

Ta myśl, posępna mieszanka herezji marksistowskich i~chrześcijańskich,
pierwotnie została jej przedstawiona przez Davida Reida, pewnej nocy
wiele dziesięcioleci temu, kiedy był bardzo pijany. Pomogła przetrwać
Myrze przez wiele złych nocy. W~innych czasach -- w~dzień czy dobre noce
-- wydawała się bezczelnym nihilizmem studentów, płytkim, niedobrym i~absurdalnym. Jednak w~złe noce docierało do niej jako głębokie i~prawdziwe i, na swój sposób, afirmujące życie. Jeżeli myślałeś o~ludziach jako o~\textit{żywych} i~\textit{mających życie do przeżycia},
wpadłbyś w~takie przygnębienie wobec tego, co tak wielu dostało w~zamian,
te ostatnie półtora wieku, że w~złe noce mógłbyś być skuszony dodać
swoje życie do ich, i~przez to dodać niewykrywalny przyrost do tej już
niewyobrażalnej, nie do pomyślenia liczby.

Liczby, którą Myra, w~swoje złe noce, podejrzewała, że już zwiększyła
całkiem znacznie. Nie bezpośrednio -- jeżeli w~ogóle zgrzeszyła, to był
grzech zaniedbania -- i~nikt nigdy nie mógłby jej o~to winić, ale winiła
się sama. Jeżeli sprzedałaby polisę odstraszania niemieckim
imperialistom, kiedy tego potrzebowali, podarła wszystkie istniejące
kontrakty i~potem je pozałatwiała, jak wielu ludzi byłoby żywych spośród
tych, którzy umarli? W~złe noce odpowiedzią wydawały się miliony. W~innych czasach, przy bardziej trzeźwej refleksji, rozumiała, że nie była
w tej lidzie. Nie była tam z~Wielką Trójką. W~tym roszczeniu było prawie
coś z~nastoletniej autodramatyzacji. Jeżeli w~ogóle należała do tej
kompanii, to była w~drugim lub trzecim rzędzie, poniżej wielkich
rewolucjonistów, ale na górze z~bardziej niszczycielskimi z~wielkich
imperialistów, Churchillem, Mountbattenem i~Johnsonem czy ludźmi tego
pokroju.

Jej buty zostały kopnięte pod stół, czarna, krepowa suknia devore 
leżała na oparciu krzesła, uszanka poleciała w~róg, czarny futrzany
płaszcz na podłodze, butelka whisky otwarta na stole, a~mroczne słowa
Leonarda Cohena zakłócały zadymione powietrze: Manhattan, potem Berlin,
w istocie.

Myra miała jedną ze swoich złych nocy.

Późnowiosenna noc za cienkimi, starymi zasłonami była zimna, a~grzejnik
centralnego ogrzewania nie robił dużo, żeby powstrzymać chłód. Główny
pokój mieszkania wydawał się mały, prawie ciasny, jak studencki pokój.
Miała kuchnię, łazienkę, sypialnię. Ale większość, co definiowało jej
życie, było wepchnięte w~ten salon. Regały wypełnione książkami, dwa lub
trzy rzędy w~głąb, choć miała całą edycję (ostatnią) Biblioteki Kongresu
z 2045 roku, na jakimś gratisowym dysku razem z~przeglądarką Sterlinga
gdzieś w~bałaganie. Jej muzyka, programy i~sprzęt komputerowy, jej
obrazy, wszystkie były spiętrzone w~podobnie zamulonych pokładach
pokoleń technologicznych, z~najnowszymi rzeczami na górze lub na
zewnątrz, i~wszystkim aż do CD i~PC i~nawet, na jakimś poziomie
prekambryjskim, winylem, w~warstwach niżej. Miała, w~opasce, dostęp do
dowolnego krajobrazu na Ziemi lub poza nią, ale ciągle miała plakaty na
ścianach.

Kiedyś, te plakaty zawierały głównie stare reklamy z~eksportu MRRNT. A
ostatnimi latami, jeden po drugim, przyklejone ujęcia startów rakiet,
rakiet czy eksplozji były zrywane w~chwilach wstydu i~furii, pogniecione
i wyrzucone, a~potem zastąpione krajobrazami kazachskiej natury i~tradycji. Góry i~łąki, jeźdźcy i~chłopi, tancerze w~wyszywanych
kostiumach, cała orientalna Szwajcaria atrakcji turystycznych.
Kazachstan radził sobie, nawet dzisiaj. Odszedł od zanieczyszczającego,
katastroficznego przemysłu ery sowieckiej i~wydobywczej monokultury,
wykorzystał prerie do bardziej produktywnej i~naturalnej hodowli bydła.
Kazachscy jeźdźcy znowu byli w~siodle.

Myra odchyliła się i~rozciągnęła. Była prawie północ. Wypiła znacznie za
dużo. Jej kilka godzin w~barze z~Walentyną poprzedzały godzinę lub dwie
picia samemu. Była tak pijana, że była przejrzysta, ,,czmychająca'', jak
zwykł nazywać to Dave. Lub możliwe, że trzeźwiała, gładko i~stopniowo, i~była w~stanie, gdzie powtarzalne aplikacje klinów odraczały nieuniknione
uderzenie kaca. Jednak pijana czy trzeźwa, z~lub bez sprzecznego
uzasadnienia Reida, musiała działać. Musiała skontaktować się z~Międzynarodówką.

Były dwie Międzynarodówki (,,dla wielkich wartości dwójki'', jak określił
to Reid, nawiązując do licznych podziałów): Druga i~Czwarta. Kiedy
większość ludzi mówiła o~\textit{Międzynarodówce}, myślała o~drugiej,
następcy tej, która sama się rozdarła w~1914 roku i~boleśnie przyłączyła
swoje odcięte kończyny w~trakcie trzech wojen światowych, pięć
światowych krachów i~jednej udanej światowej rewolucji. Nawet dzisiaj
była solidna: partie stowarzyszone z~Socjalistyczną Międzynarodówką,
związki zawodowe, spółdzielnie i~milicje ciągle skupiały członków
idących w~dziesiątki milionów.

To, co Myra miała na myśli, i~Walentyna czy Georgi, mówiąc o~\textit{Międzynarodówce} było mniej imponującą instytucją, pozostałością
części, większość z~niej włączona do wielkiego ciała Drugiej, odłamek
powoli poruszający się w~żyłach tamtej. Członkostwo w~Czwartej
Międzynarodówce liczyło tylko tysiące, rozrzucone po świecie, i~jak
Walentyna przypomniała jej, poza światem, dzięki pionierskim wysiłkom w~uzwiązkowianiu platform kosmicznych jeszcze latach dwudziestych
dwudziestego pierwszego wieku. Była to teraz prawie bezczynna, wątła
sieć starych towarzyszy, którzy nie potrafili się pożegnać ze sobą, lub
z marzeniami z~żarliwych dni młodości.

Radykalne sekty Rewolucji Angielskiej, Muggletonianie i~Cameronianie,
Ludzie Piątego Monarchy przetrwali jako malejące, marginesowe
kongregacje przez wieki po tym, gdy ich Królestwo nie przyszło. Tak
byłoby, myślała Myra, dla dawnych partyzantów Czwartej. Wiedziała to,
ale ciągłe płaciła składki.

Teraz nadszedł czas otrzymać coś w~zamian. Na początek, mogłaby ustalić,
co jej towarzysze zrobili z~bombami jądrowymi jej kraju.

Myra leciała przez przestrzeń wirtualną, pijana napędem danych. ,,New View''
unosił się przed nią, jego obraz wypełniający pole opaski. Habitat był
rodzajem orbitalnej komuny -- światowego socjalizmu, w~bardzo małym
świecie -- który był zorganizowany przez lewe skrzydło Ruchu Kosmicznego,
wtedy, gdy takie idee wydawały się mieć znaczenie. Siatka pokazywała, że
ma setki metrów w~poprzek, okrągły przyrost habitatów, uratowanych
zbiorników paliwa, kanibalizowanych statków kosmicznych. Sięgnęła i~obróciła to dłońmi w~datarękawiczkach, lekko rozbawiona chłodną,
kolczastą odpowiedzią w~palcach i~spojrzała na małe druki adresów na
kadłubie, póki nie znalazła imienia, którego szukała.

Logan, czy imię, czy nazwisko, prawdziwe imię, czy partyjne nie
wiedziała, nigdy nie słyszała go nazywanego inaczej. Oto i~było, napisane
na panelu kadłuba z~ciężkiego dopalacza starego SSTO\footnote{ Single Stage to
Orbit -- pojazd, który dociera na orbitę z~powierzchni, używając jedynie
zabranego paliwa bez utraty zbiorników, silników lub dopalaczy, zwykle
są to pojazdy wielokrotnego użytku,
por.~\url{https://en.wikipedia.org/wiki/Single-stage-to-orbit}
-- przyp.tłum.} McDonnell Douglasa. Stuknęła w~to i~widok się
przybliżył, pokazując okno z~wyglądającą twarzą mężczyzny. To był
ujmująco trafny interfejs. Myra wbiła kody powitalny i~twarz w~oknie
odpowiedziała.

-- Och, cześć? Myra Godwin? Chwilę proszę. -- Sobowtór zamigotał i~prawdziwa twarz Logana, subtelnie inna, płynnie się pojawiła, wycofując
się, gdy ikona okna rozszerzyła się do widoku wewnątrz rzeczywistego
bezokiennego pokoju.

Kajuta była oświetlona pełnym spektrum światła, jarzące się tuby jak
wały słońca pomiędzy splecionymi winoroślami, gałęziami, kablami i~rurami. Logan unosił się w~centrum pokoju. Jego przycięte białe włosy
pasowały do białego zarostu. Miał na sobie zblakły niebieski podkoszulek
i workowate spodnie. Na czole miał opaskę z~narzędziami, na której były
zamontowane lupa i~światło. Standardowa opaska była pchnięta wyżej na
czole. Był zgięty wokół otwartej pokrywy panelu sterowania, który
trzymał pomiędzy stopami i~pracował na nim z~ręcznym laserem i~zestawem
śrubokrętów jubilerskich.

Odsunął lupę znad oka i~uśmiechnął się do niej.

-- Cóż, Myra, dawno się nie widzieliśmy. -- Ciągle miał londyński akcent
pokryty przeciąganiem osadników kosmicznych. Jego kosmiczna frakcja
wyłowiła wielu ludzi, których ona i~Georgi znali w~Kazachstanie,
twardych bojowców związków zawodowych skrwawionych w~latach Nazarbajewa.

-- Ta, też za Tobą tęskniłam, Logan. Jak życie na New View?

Logan wskazał jedną ręką, automatycznie wykonując kompensujący ruch
drugą. 

-- Ok. Prawie mamy kompletną liczbę populacji, prawie tysiąc
ostatnim razem, gdy sprawdzałem. Jednak żyjemy dobrze, mamy dużo
produktów i~umiejętności, których potrzebują biali osadnicy. A stary
projekt Marsa się telepie.

-- Ciągle \textit{to} robicie?

Logan podniósł kciuk. 

-- Przygotowujemy ekspedycję, krok po kroku. Nie
mamy intencji kręcenia się tutaj na zawsze, w~każdym razie nie z~białymi
osadnikami obstawiającymi Księżyc. Nikt nawet nie jest naukowo
zainteresowany Marsem, szczególnie gdy wyszła ta sprawa ze skażeniem.

Myra skinęła głową ponuro. Zdecydowanie było to rozczarowanie, że Mars
ma całą biosferę pracowicie ewoluujących mikroorganizmów, pochodzenia
niedawnego. W~latach siedemdziesiątych Sowieci z~dumą złożyli kartkę
papieru podpisaną przez Leonida Breżniewa na Czerwonej Planecie, która
teraz była bardzo powoli terraformowana przez potomków bakterii z~potu
Sekretarza Generalnego.

-- Więc będziemy próbować -- kontynuował Logan. -- Za jakiś czas w~ciągu
następnych kilku lat, wyprowadzamy się.

-- Zamierzacie zabrać \textit{New View}? -- Myra uśmiechnęła się do Logana i~z siebie, na razie każde pytanie kończyło się pełnym zaskoczeniem.

-- Minus kilkaset ton rzeczy, których nie będziemy potrzebować, ale
właściwie, tak. Wypełnić ją, hm, wypełnić kilka zbiorników, wodą z~biegunów Księżyca, kupić silnik termojądrowy od białych osadników i~odepchnąć się z~orbity Hohmanna\footnote{ manewr ekonomicznej zmiany orbity
kołowej statku kosmicznego na wyższą lub niższą, przez dwukrotne użycie
silników,
zob.~\url{https://pl.wikipedia.org/wiki/Manewr\_transferowy\_Hohmanna}
-- przyp.tłum.}. Mamy dostatecznie dużo starych pojazdów przywiązanych
do tego stosu śmieci, żeby zbudować lądowniki, a~potem habitaty na
powierzchni.

-- Widzę, że wszystko sobie opracowaliście -- powiedziała Myra. -- Cóż,
powodzenia z~tym. -- Projekt kolonii marsjańskiej był w~toku, Już
Wkrótce, w~planach Logan tak długo, jak go znała. 

-- Jednakże, mam pilne
zapytanie. Ci biali osadnicy, o~których mówisz, nie są przez przypadek
ludźmi, na których zarobiłam fortunę, wsadzając ich na szczyty
Protonów\footnote{ rosyjska rakieta nośna, po raz pierwszy zastosowana w~1965
roku,
zob.~\url{https://pl.wikipedia.org/wiki/Proton\_(rakieta)}
-- przyp.tłum} i~Energii\footnote{ największa radziecką rakieta nośną,
budowana i~testowana w~II połowie lat osiemdziesiątych, zdolna wynieść
na orbitę ładunek o~masie około 100 ton, czy też wysłać ładunek o~masie
32 ton w~drogę na Księżyc,
więcej~\url{https://pl.wikipedia.org/wiki/Energia\_(rakieta)}
-- przyp.tłum.} i~wysyłając ich tam?

-- To ci sami -- powiedział Logan. -- I~mnóstwo nowych wychodzi z~diamentowych statków, oczywiście. -- Roześmiał się. -- Kolonialna
burżuazja!

-- Dobra, jakkolwiek ich chcesz nazywać -- powiedziała Myra -- wiesz, że
planują przejąć władzę przez ReONZ i~stacje orbitalne?

-- Och, pewnie -- powiedział Logan. -- Wszyscy to wiedzą. -- Wzruszył
ramionami. -- Co możemy zrobić? I~tak czy inaczej, jaka to będzie dla nas
różnica? -- Uruchomił swój mały laser. -- Jesteśmy bezpieczni.

-- Nie, nie jesteście -- powiedziała Myra. Rzuciła wzrokiem do góry,
sprawdzając firewall. Był włączony. -- Właśnie się dowiedziałam, od
mojego Ministra Obrony, że \textit{mamy} stos ,,niszczycieli miast''
gdzieś pochowanych w~bałaganie wokół Ciebie.

-- Czy to problem? -- spytał Logan. -- Z~pewnością to najlepsze dla nich
miejsce.

Musiała docenić jego spokój.

-- Jakoś nie sądzę, że to dlatego Międzynarodówka prosiła, żeby je tam
umieścić.

-- Ach -- powiedział Logan. -- Więc o~tym wiesz.

-- Ta -- powiedziała Myra. -- Wielkie dzięki za ,,niepowiedzenie'' mi.

Logan wymamrotał coś całkowicie przewidywalnie o~,,minimum wiedzy''.
Myra przerwała jego tłumaczenia ostrym cięciem dłoni.

-- Daj mi kurwa spokój -- powiedziała, zirytowana. -- Tyle mogę sama się
domyślić. Atomówki są elementem sytuacji, ale teraz nie są moim głównym
problemem. Po prostu pomyślałam, że dam ci znać, że wiem o~nich, z~tych
samych powodów, dla których powinieneś mi powiedzieć: ze względu na
grzeczność, jeżeli nic więcej. Ok?

-- Cóż, tak, ok -- zgodził się Logan, niechętnie. -- Więc co jest Twoim
głównym problemem?

-- Zastanawiałam się -- powiedziała Myra -- czy je przechwyciłeś, ponieważ
zamierzałeś coś zrobić z~zamachem. Jak, na przykład, zatrzymać go.

Logan się roześmiał. 

-- Ja sam?

-- Nie. Międzynarodówka. I~nie mów mi, że Ty \textit{osobiście} jesteś
jedynym członkiem, który jest tam w~górze.

-- Och, nie, ani trochę. -- Logan spojrzał na nią, w~oczywisty sposób
zdziwiony. -- Mamy wielu towarzyszy, mam na myśli, New View jest
praktycznie nasze, ale minęło dużo czasu, od kiedy Partia miała armię,
Myra, wiesz to tak samo, jak i~ja. Mamy organizację bojową\footnote{ w~oryg.
military org, w~historii partii w~Polsce znane są organizacje bojowe,
zajmujące się np. ochroną demonstracji lub przygotowaniem czynów
zbrojnych, stąd tłumaczenie,
por.~\url{https://pl.wikipedia.org/wiki/Organizacja\_Bojowa\_Polskiej\_Partii\_Socjalistycznej}
czy
\url{https://pl.wikipedia.org/wiki/Grupa\_Rewolucjonist\%C3\%B3w\_M\%C5\%9Bcicieli}
-- przyp.tłum.}, ale to tylko\ldots mały aktyw.

-- Oczywiście, że wiem o~tym. Jednak również wiem \textit{do} czego jest
mały zespół bojowy. Jest po to, że kiedy potrzebujesz armii, możesz
zwerbować żołnierzy z~\textit{innych} armii. Chcesz mi powiedzieć, że
frakcja kosmiczna nie wykonała żadnej pracy partyjnej na stacjach
orbitalnych? Przez te lata?

Logan wyglądał niekomfortowo. 

-- Nie do końca, nie, nie mówię tego.
Mamy\ldots cóż, naturalnie mamy sympatyków, otrzymujemy raporty\ldots

-- Tak jak i~my -- powiedziała. -- Niektóre z~nich od tych samych
towarzyszy co wy. -- Nie była całkowicie pewna tego ,,minimum wiedzy'',
ale to dałoby mu do pomyślenia. -- Kto właściwie wie o~atomówkach?

-- Walentyna Kozlowa -- powiedział Logan. -- I~Twój eksmąż, Georgi Dawidow.
-- Jeżeli Logan zauważył mimowolne drgnięcie Myry na te wieści, nie dał
po sobie poznać. -- I~oczywiście ja. To wszystko. Jedyni ludzie, którzy
wiedzą. Chyba że był przeciek.

-- Hmmm -- powiedziała Myra. -- Reid wydaje się nie wiedzieć o~nich,
wiedział, że mamy broń jądrową w~kosmosie, ale myśli, że wszystkie są na
orbicie Ziemi. -- Przerwała. -- Kurwa, poczekaj chwilę. Jeżeli jesteś
jedyną osobą tam, która wie o~nich, zatem prośba od Partii kilka lat
temu była w~rzeczywistości prośbą od Ciebie. Ciebie, osobiście.

-- Cóż, tak -- powiedział Logan. Wcale mu to nie przeszkadzało. -- W~moim
zakresie obowiązków jako sekretarza partii Frakcji Kosmicznej.

-- Wziąłeś na siebie zrobienie \textit{tego}? Co, do kurwy, miałeś w~\textit{głowie}? -- Boże, pomyślała, znowu niedowierzająco piszczę. Dodała,
równym, spokojnym głosem: -- Poza tym, kto dał ci prawo ingerować w~moją
sekcję, i~państwo mojej sekcji?

Logan poruszył się jak ktoś poruszający się niekomfortowo w~niewidzialnym krześle. 

-- Miałem ważne instrukcje. Od organizacji
bojowej.

-- Ach! Wiec \textit{jest} ktoś jeszcze, kto o~tym wie!

-- Nie jako taki -- powiedział Logan. -- Organizacja bojowa jest\ldots -- Zawahał się.

-- Jak mówiłeś, małym aktywem? -- podpowiedziała Myra.

-- W~pewnym sensie -- powiedział Logan. Spojrzał, jakby zbierał się do
przyznania. -- To AI, sztuczna inteligencja.

Myra poczuła jej plecy uderzające w~oparcie krzesła, dosłownie była
odrzucona tą informacją. Wzięła głęboki wdech.

-- Podsumujmy to jeszcze raz, dobrze? Powiesz mi, czy dobrze to
zrozumiałam. Dwa lata temu, na stulecie Sputnika, Wala otrzymuje
wiadomość od Ciebie, z~prośbą o~część naszych zachowanych atomówek. To
uzasadniony wniosek Partii, zatem decyduje, że nie muszę wiedzieć i~beztrosko wykonuje. A powodem, dlaczego to się zdarzyło, to ponieważ
\textit{Ty} dostałeś żądanie od jebanego \textit{komputera}?

-- Wojskowego systemu eksperckiego -- powiedział pedantycznie Logan. -- Ale
tak, to mniej więcej podsumowuje.

Myra po omacku znalazła papierosa, zapaliła go roztrzęsiona.

-- I~jak długo Czwarta Międzynarodówka przyjmuje wojskowe rady od AI?

Logan wykonał jakieś obliczenia w~umyśle.

-- Około czterdzieści lat -- powiedział.

To nie była wielka tajemnica, dowiedziała się Myra. Jedna z~tych rzeczy,
których nigdy nie musiała wiedzieć. AI powstała jako ekonomiczny i~logistyczny system planowania opracowany przez trockistowskiego
inżyniera oprogramowania w~Brytyjskiej Partii Pracy. Ten mechanizm
logistyczny został użyty przez Zjednoczoną Republikę Wielkiej Brytanii i~odziedziczony przez samozwańczego następcę, podziemną Armię Nowej
Republiki, kiedy Brytania została zajęta, a~monarchia przywrócona, przez
Jankesów w~Trzeciej Wojnie Światowej. AI otrzymała znaczące
aktualizacje, nie wszystkie zamierzone, podczas dwudziestu lat wojny
partyzanckiej, które nastąpiły, i~odegrała dyskusyjną rolę w~narodowym
powstaniu Brytyjczyków podczas Jesiennej Rewolucji 2045 roku. Główne
programy zostały przeszmuglowane w~kosmos przez uciekinierów z~konsolidacji Nowej Republiki po zwycięstwie. Od tego czasu AI rozwijała
swoje możliwości i~działania.

-- Większość ludzi nazywa to Generałem -- powiedział jej Logan. -- Bez
kłopotu bije Turinga.

-- Ale co robi? -- spytała Myra. -- Jeżeli to taki super doradca, dlaczego
nie wygrywamy?

-- Zależy, co masz na myśli, mówiąc ,,my'' -- powiedział Logan. -- I~co to
jest ,,wygrywanie''.

Myra, zdała sobie sprawę, nie miała na to odpowiedzi. Może doradca AI
uczył się z~prac \textit{Analysis} i~zgodził się, że sytuacja jest
beznadziejna.

Logan patrzył na nią ze współczującą ciekawością, rodzajem odwróconego
lustrzanego wrogiego zakłopotania, które kierowała w~jego stronę. Musiał
stać się tam tubylcem. Przyzwyczaił się do tej sytuacji i~tego stylu
pracy przez dekady i~zapomniał o~zwykłych uprzejmościach nawet wobec
swoich teoretycznie towarzyszy.

-- Tak czy inaczej -- mówił -- możesz sama tego zapytać. -- Stuknął, w~roztargnieniu, w~panel sterowania pomiędzy stopami, spojrzał,
powiedział: 

-- Łączę.

Zanim Myra mogła chociażby otworzyć usta, Logan zniknął i~został
zastąpiony przez wojskową AI. Wyobrażała sobie takie, od kiedy Logan po
raz pierwszy wspomniał o~tym: coś w~rodzaju oprogramowania
\textit{Jane's}, VR linii i~świateł. W~najlepszym przypadku symulowany
automat jak Parvus.

AI była młodym mężczyzną w~poplamionym potem kamuflażu na skale na
polanie lesie klimatu umiarkowanego: porosty, kora brzozowa, dźwięk
wody, trele ptasie, cienie liści, wstęgi dymu drzewnego. Wyglądało,
jakby tam się zatrzymał, może zastanawiając się nad założeniem obozu.
Mężczyzna wyglądał niczym commandante, długie, faliste czarne włosy,
czarna broda i~ciemne oczy rzucały coś z~uroku Guevary, arogancji
Trockiego. Przypominał równie, niepokojąco, Georgiego, wystarczająco,
żeby podejrzewała, że obraz był dopasowany do jej osobowości, że był
precyzyjnie dostrojony, żeby przekazać jej przytłaczające wrażenie
obecności, charyzmy.

-- Cześć -- powiedział. -- Od dawna chciałem cię poznać Myro.

Rozłożyła ręce. 

-- Mogłeś zadzwonić.

-- Bez wątpienia tak bym zrobił, już wkrótce. -- Byt się uśmiechnął. -- Wolę, żeby ludzie przychodzili do mnie. Unikamy kolejnych nieporozumień.
Tak czy inaczej, rozumiem, że masz dwie sprawy: atomówki w~Lagrange i~zamach Ruchu Kosmicznego. Co do pierwszego, broń jądrowa ciągle jest pod
Twoją kontrolą. Twój minister obrony nadal ma kody dostępu. Zażądałem,
żeby sama broń była przesunięta tutaj dla bezpieczeństwa. -- Wzruszył
ramionami i~znowu się uśmiechnął. -- Są Twoje. Tak jak broń na orbicie
Ziemi, która jest, oczywiście, łatwiej dostępna i~użyteczna. To prowadzi
mnie do Twojej drugiej troski, zamachu. Jest nieuchronny.

-- Jak nieuchronny?

-- W~ciągu następnych kilku dni. Przepchnęli głosowanie o~reorganizacji
ReONZ i~nowa Rada Bezpieczeństwa wyda rozkaz zajęcia stacji orbitalnych.
Mają siły, żeby to zrobić.

Przerwał, patrząc na nią, albo przez nią. 

-- Ale my mamy siły, żeby ich
powstrzymać. Mogę Cię zapewnić, Myro, wszystko pod kontrolą.

Potrząsnęła głową. 

-- To nie to, co mój wywiad wskazuje. Sprawdziłam,
moje ministerstwa obrony i~spraw zagranicznych sprawdziły. Mamy agentów
na stacjach orbitalnych, jak zapewne wiesz, kurde, niektórzy z~nich
muszą być w~Twojej organizacji bojowej. \textit{Jeżeli} taka rzecz
istnieje. -- Żałowała, że nie czytała tej poczty.

-- Z~całą pewnością istnieje -- powiedział stanowczo Generał. -- I~przekazywała ci dezinformację.

-- \textit{Co?}

Byt wstał i~ruszył ku niej w~swojej wirtualnej przestrzeni. Rozłożył
ręce i~przyjął przepraszającą minę, ale z~lekkim konspiracyjnym blaskiem
w oczach.

-- Wybaczcie mi, towarzyszko Dawidowa. Nie robiliśmy tego przeciwko
Tobie. Robiliśmy to przeciw naszemu wspólnemu wrogowi: fakcji Reida w~Ruchu Kosmicznym.

-- Jak\ldots -- zaczęła, ale zrozumiała, zrozumiała.

-- Mówię Ci teraz -- powiedział Generał -- ponieważ dzisiaj straciłaś
ostatniego nielojalnego komisarza. Aleksander Sherman od miesięcy
przekazywał informacje Reidowi. Nie był pierwszy, ale był ostatni.

-- Kim byli inni?

General poruszył dłoniami w~uspokajającym geście. 

-- Nie mogę ci
powiedzieć bez zagrożenia obecnych operacji. Zresztą ta szczególna
informacja nie ma dla Ciebie użytku.

-- Zdaje się, że nie. -- Myra zgodziła się niechętnie. Chciałaby wiedzieć
kim byli zdrajcy, mimo wszystko. Miała nadzieję, że Tatiana i~Michael
nie byli nimi. Całkiem polubiła tę dwójkę \ldots

-- Zatem użyłeś ich, i~\textit{nas}, jako kanału dezinformacji?

Generał skinął głową. 

-- A co do informacji zwrotnych, Twoje
aktualizacje w~\textit{Jane's} były bardzo pomocne.

-- Jezu. -- Jej reakcje na to były interesująco skomplikowane, pomyślała
odlegle. Z~jednej strony czuła rozdrażnienie, że była użyta, że jej
kłamano. Z~drugiej, potrafiła docenić scenografię podstępu. Ponad tym
czuła ulgę, że ponure negatywne oceny, którymi się martwiła, były
wszystkie błędne.

Póki sytuacja nie była jeszcze \textit{gorsza} niż sądziła\ldots

-- Zdecydowanie sytuacja jest lepsza, niż sądzisz -- powiedział Generał. -- Mamy naszych ludzi na miejscu, stacje orbitalne mogą być przejęte bez
walki, którą w~większości przypadków oczekujemy, że wygramy.

-- \textit{Większość przypadków} nie jest wystarczająca. Nawet jedna
stacja\ldots

-- Dokładnie. Czyli gdzie wchodzi w~grę Twoja broń orbitalna. Lasery,
ładunki EMP, inteligentne kamyki, myśliwce, bronie energii
kinetycznej\ldots

Myra nie wiedziała, że jej arsenał był tak rozległy. (Boże, pomyśleć, że
te zapasy kiedyś należały do papieża! Cóż, do Gwardii Szwajcarskiej, tak
czy inaczej, całkiem możliwe, że Jego Świątobliwość został dyskretnie
pominięty w~tej sprawie). Zadrżała w~szalu, poprawiła na ramionach,
zapaliła kolejnego papierosa. Nie wiedziała, co powiedzieć. Czuła, że
jej policzki płoną pod narastająco zagadkowym spojrzeniem Generała.

-- Co chcesz, żebyśmy z~nimi zrobili? -- spytała w~końcu.

-- Jestem pewne, że możesz się domyślić -- odparł. -- Będę w~kontakcie.

-- Ale\ldots

Uśmiechnął się do niej, boleśnie, szatańsko.

-- Mam nadzieję, że się jeszcze zobaczymy -- powiedział. Sięgnął dłonią i~wykonał drobne poprawki w~powietrzu. Łącze zgasło.

Myra zdjęła opaskę i~potarła oczy. Potem niepewnie poszła do kuchni i~zrobiła herbaty, usiadła, pijąc, paląc przez około kwadransa, patrząc
bezmyślnie na wirtualne przestrzenie w~głowie. Wydawało jej się, że
powinna coś zrobić, powiedzieć komuś, ale nie potrafiła pomyśleć co
robić, lub komu powiedzieć.

Jutro będzie czas, zdecydowała.

Jej sypialnia była mała, kilka metrów luzu po trzech stronach podwójnego
łóżka dawało ledwie miejsce na garderobę i~toaletkę. Przez lata pokój
zgromadził przytłaczający opad miękkich mebli, robótek ręcznych i~ozdób.
Ładne rzeczy, które kupiła pod wpływem impulsu i~nigdy nie miała serca
ich wyrzucić. Proces naturalnej selekcji dla żenująco dużej kolekcji
babcinego nieładu. Od czasu do czasu, jak teraz, wściekała się na tę
rozbieżność z~resztą jej życia, jej stylem, jej postawą. A potem, po
refleksji, zrozumiała, że bezsensowność wyglądu pokoju była tym, co
sprawiało, że było to miejsce, gdzie mogła zapomnieć o~troskach i~spać.

Rano to wydawało się snem.

Tym bardziej, Myra zrozumiała, gdy walczyła ze świadomością przez
warstwy snu, kaca i~splątanej, wilgotnej od potu pościeli, że
\textit{śniła} o~Generale. Czuła się niejasno tym zawstydzona, zażenowana
przed sobą budzącą się, nie dlatego, że sen był erotyczny -- choć był -- ale dlatego, że był zakochany, oddany, \textit{służalczy}, jak te sny,
które Brytole zwykli mieć o~Rodzinie Królewskiej. Usiadła w~łóżku i~położyła poduszkę, oparła się i~próbowała pomyśleć o~tym racjonalnie.

Byt, wojskowa AI, mógł mieć, Bóg jeden wie, ile pokoleń software'u, by
wyewoluować intymną znajomość ludzkości. Miał czas, żeby stać się tym,
co Japończycy nazywali \textit{idoru}, przedstawienie softwarowe, które
było lepsze niż rzeczywistość, mądrzejsze i~bardziej seksowne niż
jakikolwiek potencjalny umysł lub forma ludzka mogłyby być, jak te
wielkookie, fałszywie niewinne bachory anime lub symulowane gwiazdy
pornografii i~romansu. Seks nie był tego połową, były inne kody, inne
klucze, w~semiotyce uroku: subtelna sugestia mądrości, przypadkowa
wskazówka możliwości przemocy, założona gotowość do rządzenia, lustrzane
spojrzenie empatii. Wszystkie te elementy, które tworzyły obraz
mężczyzny, za którego mężczyźni by umierali, a~w~którym kobiety by się
zakochiwały.

Zatem, powiedziała sobie, nie była takim żałosnym przypadkiem, jednak.
Zdarza się najlepszym z~nas. Gdy sięgnęła po zestaw medyczny i~wycisnęła
tabletki na kaca, przyłapała się na uśmiechu na wspomnienie
uśmiechającego się Generała. Znowu sobą zirytowana, wstała z~łóżka i~podreptała do kuchni w~puszystych klapkach i~pidżamie, łyknęła zimnej
wody, kiedy kawa się sączyła. Dodała tabletkę PoprawNastrój do jej dawki
Rozum i~codziennego spożycia suplementów przed starzeniem, i~połknęła je
naraz. Poczuła się lepiej.

Była godzina ósma. Założyła kontakty i~spojrzała na kafelek z~telewizją,
oglądała, gdy jadła muesli i~jogurt, słuchając mamrotanej, porannej
odprawy Parvusa. Wiadomości, jak zwykle, były złe, ale nie bardziej niż
zwykle. Żadnej muzyki wojskowej czy baletu na wszystkich kanałach, to
liczyło się jako dobre wiadomości. Po kawie i~papierosie, czuła się
prawie jak człowiek. Wydawało jej się, że mogła równie dobrze wstać i~pójść do pracy.

Spacer do budynku rządowego obudził ją jeszcze bardziej, podniósł jej
nastrój lepiej niż tabletki. Powietrze było rześkie, poranne niebo
niespodziewanie kolorowe, czerwone, pomarańczowe i~żółte przechodzące w~zieleń na horyzoncie. Zauważyła, że ludzie patrzą w~niebo.

Jego kolory zmieniały się wyraźnie, płynąc, nagle zrozumiała, że patrzy
na zorzę polarną, tysiące kilometrów na południe od miejsca, gdzie mogła
być obserwowana. Gdy zatrzymała się i~spojrzała, z~otwartymi ustami,
niebo pojaśniało na kilka sekund od jakiegoś wielkiego doświetlenia
poniżej horyzontu.

Pobiegła. Pobiegła sprintem ulicami, wtargnęła przez drzwi, krzyknęła na
ochronę i~wbiegła na schody. Gdy wkraczała do biura, jej słuchawka
piknęła i~bełkot metalicznych głosów rywalizował o~jej uwagę. Usiadła
ciężko na krawędzi biurka i~opuściła opaskę, włączyła wiadomości.

Czołgi ruszyły, na całym świecie.

Bez spuszczania oczu z~kanałów informacyjnych, Myra wsunęła się za
biurko i~usiadła w~fotelu. Wstukała komendy na klawiaturach w~podłokietnikach, przekształcając ściany biura w~ekrany dla awaryjnego
centrum dowodzenia. Pierwszą rzeczą, którą zrobiła, było zabezpieczenie
budynku. Potem uderzyła przycisk nadzwyczajnego zebrania dla SowNarKomu.
Rzucone sobowtóry Andrieja, Denisa i~Walentyny stanęły na baczność na
ekranach. Czy ich fizyczne ciała były w~biurach, w~drodze do nich lub
ciągle w~łóżkach nie miało znaczenia, o~ile ich opaski były online.

Myra rozejrzała się po ich wirtualnych obecnościach.

-- Ok, towarzysze, to wielka sprawa -- powiedziała. -- Pierwsze, wszystko
jasne u nas?

Było małoprawdopodobne, by małe siły Milicji Robotniczej i~jeszcze
mniejsza Armia Ludowa MRRNT dołączyły do zamachu, ale nieprawdopodobne
rzeczy zdarzały się już przed jej oczami co kilka sekund. (Nowy Jork -- Nocny desant na South Street Seaport! Waszyngton -- Czołgi na
Pennsylvania Avenue! Londyn -- Śmigłowce szturmowe ostrzeliwują
Westminster Bridge!).

-- Jesteśmy solidni -- powiedział Denis. Nawet jego sobowtór wyglądał na
wyczerpanego i~na kacu. -- Tak jak Kazachstan, trzymają się z~dala od
tego. Oczywiście, armia w~gotowości. Kosmodrom w~Bajkonurze jest pod
kontrolą rządu. Tak jak pas startowy ,,Jubileiny''\footnote{ lotnisko
wybudowane na potrzeby obsługi radzieckiego promu Buran,
zob.~\url{https://en.wikipedia.org/wiki/Yubileyniy\_Airport} -- 
przyp.tłum.}. Ałma-Ata zmobilizowana, milicja na ulicach, ale są
lojalni.

Masz nadzieję, pomyślała Myra. Fajną rzeczą o~zamachu wojskowym było to,
że mobilizacja przeciwko niemu mogła całkiem łatwo stać się
\textit{częścią} niego, gdy łańcuch dowodzenia wykręcał się, łamał i~ponownie łączył.

-- Dobrze, wspaniale. Front północno-wschodni? Wala, żyjesz?

-- Tak, jestem z~wami. Na razie żadnych ruchów ze strony Szinosowietów. -- Walentyna wkleiła łącze satelitarne, aktualizowane co sekundę, step był
nieruchomy.

-- Co z~Ochroną Wzajemną?

-- Nie ruszyli się z~obozu, a~obóz jest cichy.

Myra trochę się rozluźniła. 

-- Wygląda na to, że nasze najbliższe
otoczenie jest bezpieczne. Cokolwiek z~orbity, Walu?

Walentyna pokręciła głową. 

-- Łączność jest słaba, nie można odebrać
niczego spójnego z~osad, fabryk czy stacji orbitalnych\ldots

-- To niemożliwe! -- Pomyślała o~tym, w~jaki sposób mogło to być możliwe.
-- O mój Boże, niebo\ldots

-- Około dziesięć minut temu -- zawiadomił Andriej z~jakiegoś szklanego
transu -- ktoś zdetonował broń jądrową w~warstwie Kennelly-Heaviside\footnote{
jonosfera 90-120 km na powierzchnią,
zob~\url{https://en.wikipedia.org/wiki/Kennelly\%E2\%80\%93Heaviside\_layer}
-- przyp.tłum.}. Około dziesięciu wybuchów, niedużo impulsów
elektromagnetycznych, ale na tyle dużo oraz~naładowanych cząsteczek,
że przez kilka godzin zakłócą sygnały radiowe.

-- Więc jak w~ogóle dostajemy wiadomości? -- spytała się Myra.

-- Kabel -- powiedział Andriej. -- Łącza światłowodowe nie zostały
dotknięte. I~część spraw dostajemy przez lasery, oczywiście, jak łącze z~satelity szpiegowskiego Wali. Powinno się zwiększyć, jak ludzie się
przełączą lub zaimprowizują. Jednak na tę chwilę to kurz w~oczach u
wszystkich.

-- Nie wiedziałem, że Ruch Kosmiczny miał atomówki na orbicie -- powiedział Denis. -- W~istocie, nie wiedziałem, że ktokolwiek ma
\textit{jakieś} konkretne atomówki.

To miało sens. Rozbrojenie jądrowe było jedyną powszechnie popularną, i~(prawie) powszechnie udaną, polityką USA/ONZ po Trzeciej Wojnie
Światowej. Nawet Myra, w~tamtym czasie, nie oburzała się, ani nie
żałowała, konfiskaty kompletu MRRNT, wraz z~całą resztą. Tylko przez
czysty przypadek przetrwały niezależne zapasy, w~rękach politycznie
nietykalnych instytucji, które liczyły zwolenników w~miliardach, wiek w~mileniach, a~linie polityczną w~wiekach. Reszta strategicznych broni
atomowych została zdemontowana. Nadal istniały tysiące taktycznych
atomówek, oczywiście, ale nikt się nigdy o~\textit{nie} nie martwił:
konsekwencje ich użycia nigdy nie były pokazywane na żywo w~telewizji.

(Obrazy przeleciały znowu w~jej głowie i~nazwy miast: Kijów, Frankfurt,
Berlin. Potrząsnęła głową, odgraniczając się).

Walentyna patrzyła na nią twardo. 

-- To nie były \textit{nasze}, co?

-- Nie, o~ile mi wiadomo -- powiedziała Myra. -- Chyba że zdarzyło Ci się
przekazać komuś kody dostępowe?

Walentyna pokręciła głową, zaciśnięte usta. 

-- Nie. Nigdy.

-- Dobra, to tyle z~tej teorii -- powiedział Myra raźno, żeby zapewnić
Walę, że nie była podejrzana. -- Andriej, jakieś pomysły?

-- Wybaczcie mi -- powiedział Andriej. -- Ciągle próbuję wejść przez
frontowe drzwi.

-- O kurwa! -- Myra wbiła kod, żeby go wpuścić.

-- Dzięki\ldots Ok, myślę, że atomówki były ze strony \textit{ONZ}, przeciwko
zamachowi.

-- A skąd oni je wzięli?

-- Myślę, że ONZ zatrzymała jakieś atomówki dla siebie, tajemnica, która
była utrzymywana w~wewnętrznej kadrze biurokratów, która przetrwała
Rewolucję i~czystki, a~które oddali do dyspozycji obecnego Sekretarza
Generalnego.

-- Wydaje się, że to ma sens -- powiedział Denis. -- Też bym tak zrobił.

-- Jaka jest za tym polityka, Andriej? -- spytała Myra. -- Byliśmy tak
pewni, że poczekają na głosowanie ReONZ \ldots -- Przerwała i~się roześmiała.
Sam Trocki skorzystał z~takiego fortelu. -- Zrobić zamach przed
głosowaniem, zastanawiam się, skąd wzięli idee. Jednak, to raczej
podważa odwołanie się do prawowitości.

Ciągle obserwowała jednym okiem na ekranach wirtualnych wiadomości
kablowych. 

-- Och, czekajcie, coś się pojawiło\ldots

Siedzieli w~ciszy, gdy prezenter odczytał komunikat dużej grupy małych
rządów nazywających się Sojuszem Większości Zgromadzenia. Sensem
komunikatu było to, że obecna Rada Bezpieczeństwa naruszyła zmienioną
,,Kartę 2046 roku'' planowaniem użycia broni jądrowej w~kosmosie, oraz
zawierała wezwanie do natychmiastowych działań, by usunąć konspiratorów
i samozwańców. Siły rządów Sojuszu i~Ochrony Wzajemnej oferowały
natychmiastową, skoordynowaną akcję w~tym celu. Przewidywane było
szybkie rozwiązanie stanu nadzwyczajnego. Wzywano populację do
zachowania spokoju i~pozostania w~domach przez dzień.

-- Boże, to jest tak cyniczne -- powiedziała Wala. -- Musieli mieć
dziesiątki antydatowanych oświadczeń, przygotowanych na wszelki wypadek,
tak, żeby mogli twierdzić, że działają, by zapobiec czemukolwiek, co
planowana Rada Bezpieczeństwa.

-- Tak, tak -- powiedziała Myra. -- Wszystkie standardowe procedury dla
zamachu. Oraz dywersja, w~każdym razie. To w~kosmosie odbywają się
prawdziwe bitwy. Może właśnie w~tej chwili! Wszystko zostanie określone
z prędkością światła. Chodźcie, przejdźmy w~tryb poleceń.

Pozostali pokiwali, ucichli, odwrócili się do ekranów i~zaczęli wyciągać
wszystkie dostępne dane i~uruchamiać na nich oprogramowanie analityczne.
Po minucie lub dwóch, zaczęli zazębiać się jako zespół w~ich wspólnej,
wirtualnej przestrzeni roboczej. Informacje błyskały tam i~z powrotem
pomiędzy ich osobistymi sieciami, siecią rządową, systemem
\textit{Jane's}, kanałami informacyjnymi i~raportami terenowy od ich
własnych żołnierzy i~agentów.

Zaczynał się ukazywać duży obraz, choć ujawniana sytuacja była
chaotyczna. Myra przeskakiwała przez większość ważnych stolic: Pekin,
Pjongjang, Tokio, Władywostok, Seattle, Los Angeles, Waszyngton DC, Nowy
Jork, Londyn, Paryż, Nowy Berlin, Gdańsk, Moskwa. Wszystkie zgłaszały
różnego rodzaju ataki wojskowe, ale wszystkie miały wygląd \textit{puczy},
krótkoterminowego przejęcia budynków publicznych lub miejskich twierdz,
które mogły być utrzymane bardziej dzięki niechęci sił rządowych do
destrukcji niż siłą okupujących. Wszystkie były podejrzanie dywersyjne.

Wszystkie wierne rządy technofobiczne, od Khmer Vertes rządzących
Bangkokiem, przez Muzułmańskich Republikanów Arabii do Białych
Nacjonalistów Dallas, mieli swoje siły w~gotowości i~ich media
wrzeszczały przekleństwa przeciwko wrogom Boga, Człowieka lub Gai
(zależnie od lokalnego smaku ideologicznego). Ale Myra sądziła, że są
świadomi, że nie są, sami, bezpośrednim celem, takim były bardziej
liberalne rządy, które szły na kompromis pomiędzy siłami pro i~przeciw
technologii, które teraz były pod ostrzałem.

Poważniejsze działania miały miejsce w~zagmatwanych, globalnych
enklawach w~głębi kraju, minipaństwach i~krajach korporacyjnych. Wzdłuż
ich fraktalnych granic lokalne siły obronne były gromadzone i~mobilizowane w~postawie, która była agresywna wobec państewek Sojuszu
Większości Zgromadzenia, ogólnie defensywna wobec reszty. W~międzyczasie, na niedookreślonych ziemiach poza i~nawet w~tle tych
anarchicznych ustrojów, w~lasach, na równinach i~nieużytkach, w~slumsach
pełnych Zielonych neobarbarzyńców, margines i~plemiona budziły się wobec
nieoczekiwanego, ku możliwościom tego nowego dnia.

\textit{Jane's Market Forces} rejestrowały nieoczekiwane przesunięcia w~równowadze sił, drobne potyczki miały poważne wpływy, stawiając
żołnierzy, taktykę i~broń przed testem w~nowych warunkach, lub w~prawdziwej, a~nie symulowanej walce. Niezbyt dużo krwi było rozlane, ale
fortuny powstawały i~upadały, sojusze i~antagonizmy się aktualizowały.
Proces miał swój własny krwawy urok. Myra czuła, że mogłaby siedzieć i~patrzeć na to godzinami.

Jednak to była Ziemia, to nie tu, gdzie się naprawdę działo. Bitwy
tutaj, prawdziwe lub wirtualne, były zasadniczo dywersją i~została w~porę odciągnięta. Zdecydowanie skierowała swoją uwagą w~niebo.

Z dobrze wyćwiczoną pomocą Wali, rozwinęła neonowe planetarium
przestrzeni orbitalnych Ziemi, oddzielając ważne wątki od pasm
komercyjnych i~wojskowych orbit. Sama planeta ukazała się jak
przezroczysty glob, z~zarysowanymi liniami politycznymi i~geograficznymi, zachmurzona od wzorców klimatycznych, zakreskowana
konfrontacjami, nakłuta błyskami. Znowu ten wewnętrzny wzorzec zajął jej
uwagę, znowu odwróciła się od tego.

Ich własny kosmiczny \textit{matériel} -- bronie jądrowe i~kinetyczne -- były przedstawione jako czarne pręty i~stożki, głęboko w~stale rosnącym
pierścieniu śmieci kosmicznych, które śledziły główne orbitalne arterie
komunikacyjne.

-- Cokolwiek jeszcze nadchodzi od stacji orbitalnych?

-- Trochę -- powiedziała Wala, brzmiąc rozproszona. -- Ściągam łączność
laserową przez różne stacje naziemne. Cholera, to trudne, czekaj,
czekaj\ldots och!

Lokalizacje stacji orbitalnych podświetliły się jedna po drugiej. Te, z~którymi łączność została ustanowiona, błyskały zachęcająco. Myra
przybliżyła na jedną z~nich. Klasyczna stacja kosmiczna von Brauna, z~obracającym się rurowym pierścieniem połączonym cieńszymi cylindrycznymi
szprychami do wewnętrznego pierścienia otaczającego przeciwobrotową
wieżę osiową z~kompensacją wirowania. Kwatery mieszkalne i~hydroponika
były dookoła pierścienia, w~sztucznej grawitacji obrotów. Działa
laserowe, stacje rakietowe, bronie wiązek cząstek i~wojskowe centrum
dowodzenia były w~piaście w~nieważkości. Cała olbrzymia mandala miała
majestat obozów nazistowskich, zepsutych jedynie przez niezgrabne układy
paneli słonecznych, które kiełkowały, kiedy reaktory jądrowe się
wyczerpywały.

To był jeden z~dziesiątków na różnych orbitach. Obrona Kosmiczna
wymusiła Pax Americana Imperium USA/ONZ, dwudziestoletnią Rzeszę,
pomiędzy Trzecią Wojną Światową a~Jesienną Rewolucją. W~tej rewolucji
stacje przeszły w~ręce ich personelu -- sowiety żołnierskie w~kosmosie -- i, od tego czasu, szukały roli, żeby zastąpić ich upadłe imperium.
Wszystko od transmisji mocy do obrony przeciwko asteroidom było
spróbowane, z~małym zyskiem. Stacje przetrwały na kroplach dotacji -- lub
,,opłat użytkowników'' -- od podobnie zmniejszonej ONZ, płaconych
głównie, by zapobiec buntowi czystej desperacji stacji.

Teraz siły zamachu oferowały im nowe imperium, takie bardziej
usprawiedliwione i~wykonywalne niż stare.

-- Więc jaki jest stan tego? -- spytała Myra.

-- Ciągle lojalna -- odpowiedziała Wala. -- Właśnie się zgłosili, żeby
powiedzieć, że nie idą z~Sojuszem.

-- Jakiś sposób sprawdzenia tego?

-- Nie wiem, wywołuję ich, och! Wpuszczają nas.

-- Pójdę -- powiedziała Myra -- zostań z~dużym obrazem.

W niezgrabnym, dezorientującym przejściu, znalazła się, stojąc w~przedstawieniu w~czasie rzeczywistym, na mostku stacji orbitalnej. Miał
około piętnastu metrów szerokości i~był zatłoczony. Wnętrze pasowało do
zewnątrz: serie błyskających świateł na chromowanych i~czarnych
powierzchniach, zagracony przerost zmodernizowanego nowoczesnego sprzętu
pośród obfitości roślin, jak w~cywilnej kolonii kosmicznej. Układ był
zoptymalizowany pod nieważkość, członkowie załogi przypięci do siedzeń i~łoży pod nieoczekiwanymi kątami wobec siebie. W~tej sekcji wału były
prawdziwe okna, przez które mogła zobaczyć wielkie koła obracające się w~świetle, i~wirujące chmury Ziemi poniżej. Mrugnęła i~nadpisała prawdziwy
widok obrazem software'u.

Załoga nosiła opaski na oczach, a~niektórzy z~nich mogli zobaczyć
sobowtóra Myry w~ich własnych wirtualnych palimpsestach sceny, ale nawet
na nią nie spojrzeli. Kolejna, widmowa obecność zajmowała ich całą
uwagę.

Generał siedział na parapecie okna, przyglądając się mostkowi przez
zmrużone oczy. Coś mówił, jego słowa wydawały się unosić w~powietrzu,
rezonując w~układach ekranu. Przerwał i~odwrócił się do niej.

-- Och, towarzyszko Dawidowa, dziękuję za przybycie.

-- Nie byłam świadoma, że prosiłeś -- powiedziała.

-- Och, byłaś -- powiedział konstrukt. -- To nie jest, jak mówią,
przypadek.

Myra skinęła głową. Bez wątpienia to nie był przypadek, że pierwsza
stacja, która wpuściła ją do systemów wewnętrznych była tą, na której
Generał zwracał się do swoich żołnierzy.

Pomachał dłonią. 

-- Witamy na szybkiej sesji nadzwyczajnej lokalnej
komórki organizacji bojowej. -- Uśmiechnął się. -- Czyli praktycznie
dowództwo tej stacji. -- Obserwujący członkowie załogi spojrzeli na nią
przeciągle, niektórzy się nawet uśmiechnęli.

-- Potrzebujemy Twojej pomocy -- powiedział jej kategorycznie Generał. -- Ładny pokaz -- dodał. -- Mogę?

Sięgnął, łapiąc kciukiem i~palcem wskazującym jej przezroczysty glob i~z przerażającą brawurą pominął jej wszystkie protokoły, przesunął jej
wirtualny widok Ziemi i~okołoziemskiej przestrzeni kosmicznej na centrum
mostka.

Patrzyła na obracające się kształty, wściekła. Nie powinien móc tego
\textit{zrobić} \ldots

-- Ciągle utrzymujemy większość stacji orbitalnych. -- Szybkie, ostre
spojrzenie. -- To znaczy, siły przeciwne zamachowi utrzymują, niezależnie
od ich innych sojuszy. Jednak walka jest ciągle w~równowadze. Mamy jedną
szóstą stacji zabezpieczonych po naszej stronie, wróg podobnie, a~pozostali są niezdecydowani.

Myra chwilowo była oszołomiona. Pomimo tego, co Generał wcześniej do
niej powiedział, nie miała pojęcia, żadnych oczekiwań, że przeniknięcie
organizacji bojowej w~Obronę Kosmiczną było takie kompletne. To musiało
zabrać lata pracy. Jednak Generał nie dał jej czasu na pytania lub
gratulacje.

-- Tutaj, tutaj i~tutaj. -- Dźgnął palcem trzy stacje, których ślady
pomiędzy nimi pokrywały większość planety. -- Te są w~rękach wroga. Nie
możemy uderzyć w~nie ze stacji, która mamy, ponieważ to mogłoby
zakończyć się odwetem. Nie mniej musimy uderzyć szybko, ostrzec innych,
który mają zamiar przejść na stronę wroga. Załatw je.

Przesunął palcem lekko po orbitalnych skrytkach republiki zawierających
inteligentne kamienie, lasery, broń kinetyczną.

-- Nie mogę -- powiedziała Myra. -- Nie mam umiejętności, nie mam
automatyzacji. Żadne z~nas nie ma.

Generał pstryknął palcami. 

-- Klucz, towarzyszko, klucz. To wszystko,
czego potrzebuję. Kody dostępu.

-- Skonsultuję z~moim ministrem obrony -- powiedziała Myra i~prędko się
wycofała. To była ulga, nawet z~nagłą, przełkniętą falą mdłości
cyberprzestrzennych, znaleźć się z~powrotem w~biurze,
patrząc na ekrany.

-- Wala\ldots -- zaczęła.

-- Mam to -- powiedziała Walentyna. -- Obserwowałam Ciebie jednym okiem,
jako odwód. Kim \textit{jest }ten facet?

Myra spojrzała z~boku na nią. 

-- Brawo -- powiedziała. -- To była głowa
organizacji bojowej Czwartej Międzynarodówki. Sztuczna inteligencja.
Nasz własny elektryczny Trocki.

-- Jeb Twoją matkę -- powiedziała Wala, po rosyjsku.

-- Tak. Zamierzamy dać mu kody?

-- Twoja decyzja -- powiedziała Wala. -- Ty jesteś premierem.

-- Co -- powiedziała Myra przez zaciśnięte zęby -- byś \textit{poradziła}?

Wala zwilżyła usta. Inni albo specjalnie je ignorowali, albo
koncentrowali się na własnych sprawach.

-- Cóż, cóż. Powiedziałabym, działaj z~doradcą wojskowym. Daj temu kody.

-- Czy to zadziała? Czy naprawdę mamy środki tam, które mogą zniszczyć
stacje?

-- Ciężko powiedzieć -- powiedziała Walentyna. -- Stare, nigdy nieużyte w~walce, nie serwisowane, ale tak samo stacje! W~teorii, tak, mogą
zmiażdżyć obronę stacji.

Myra próbowała szybko pomyśleć. Uderzyło ją, że stacje same mogą być
dywersją, stare, potężne, ale nieelastyczne i~podatne: orbitująca Linia
Maginota. Może Generał walczył w~ostatniej wojnie i~ją \textit{wygrywał},
podczas gdy prawdziwe bitwy szalały gdzie indziej.

Zawahała się, potem zdecydowała.

-- Daj mi kody do inteligentnych kamyków -- powiedziała. Wala je
przesłała, Myra przełączyła się na stację i~podała je Generałowi. Czekał
na nią, z~zaskoczonym zniecierpliwieniem.

-- Dziękuję -- powiedział ciężko, potem zniknął. Myra rozejrzała się po
teraz gorączkowo działającej załodze, pomachała im dziwnie, zachęcająco
i wróciła do własnego centrum dowodzenia.

-- To było szybkie. -- Walentyna wskazała na ekran. Już, niektóre z~ich
broni orbitalnych zostały uruchomione. Myra pobożnie miała nadzieję, że
to, co widziała jako przedstawienie, nie pokazywało się monitorach czasu
rzeczywistego wroga. W~trzech miejscach chmury ostrych obiektów wybuchły
z osłony i~poruszały się w~na tych samych orbitach co trzy wrogie
stacje, ale w~przeciwnym kierunku. Miały się zderzyć ze stacjami za
dziesięć, osiemnaście i~dwadzieścia siedem minut.

To, co zdarzyło się dalej, skończyło się w~mniej niż sekundę, migotanie
laserów w~pustce. Powtórka nastąpiła automatycznie, cierpliwie
powtarzając wyniki dla powolnych czopków, pręcików i~nerwów ludzkiego
oka.

Myra obserwowała promienie radarów dalekiego zasięgu stacji ocierające
się o~zbliżające salwy broni kinetycznej. Jej drony platformy laserowej
odpowiedziały na to wykrycie igłami światła, dźgając i~oślepiając
stacje, które, w~chwilowym międzyczasie, wypuściły chmury dipoli
odbijających\footnote{ system przeciwdziałania namiarom radarowym,
zob.~\url{https://pl.wikipedia.org/wiki/Dipole\_odbijaj\%C4\%85ce}
-- przyp.tłum.}, żeby zablokować ten manewr. Potem stacje orbitalne
odpowiedziały z~nadal oszałamiającą prędkością, nawet w~zwolnionym
tempie. Każda z~nich wysłała tysiące impulsów laserowych, błyskających
jak szpada szermierza, krojąc kamyki i~ich eskortę laserowych platform.

-- O! -- powiedziała, podziwiając wbrew sobie.

-- Tak, to jest system obrony -- powiedziała Walentyna. -- Niestandardowe
wyposażenie stacji, mogę ci powiedzieć.

Myra powiększyła widok. Każda atakująca chmura ciągle tam była, jako
znacznie większa chmura znacznie mniejszych obiektów. Zbombardowałyby
stacje, na pewno, zrobiłyby jakieś uszkodzenia, ale byłoby to bardziej
piaskowaniem niż uderzaniem.

Była godzina 9:25. Czterdzieści minut minęło od atomówek Heaviside'a.
Zakłócenia, które spowodowały, łagodniały. Łączność radiowa była ciągle
pogmatwana, ale coraz więcej centrów wracało online dzięki obejściom i~doraźnym poprawkom. Wynik tej pierwszej realnej wymiany ognia był już
analizowany. Myra rzuciła krótkie spojrzenie na \textit{Jane's}. Akcje
zamachu zmieniały się dziko.

-- Gówno\ldots

Miała znowu przenieść się swoją przestrzeń roboczą na stację, ale
Generał był szybszy. On, lub to, nagle pojawiło się w~centrum
dowodzenia, jako rozpoznawalna, ale niezbyt solidna postać. Andriej i~Denis, tym razem najwyraźniej powiadomieni przez Walę, nie zareagowali
na tę zjawę, prócz zdziwienia i~otwartych ust.

-- Szkoda -- powiedział Generał, patrząc się smutno na ekran. -- Te obrony
są przenośne, niewmontowane na stacje, ale sprowadzone przez
konspiratorów.

-- Czy inne stacje je mają?

Szkic wzruszenia ramion. 

-- My nie mamy. Może już są rozdzielane pomiędzy
niezdecydowane stacje. Moje podejrzenie to nanofabryki Ochrony
Wzajemnej.

Lepsze niż podejrzenie, uznała Myra.

-- Chcesz kolejnego uderzenia?

-- Nie. Na to jest tylko jedna rzecz. Atak jądrowy.

Myra spojrzała na Walentynę. 

-- Poczekaj. Wala, daj nam symulację
pierwszego uderzenia.

Walentyna wysłała lokalizacje ich orbitalnych broni jądrowych i~uruchomiła symulację natychmiastowego ataku, w~świetle nowych informacji
o możliwościach stacji orbitalnych. Przerwała. Uruchomiła ponownie i~znowu. Wszystko w~ciągu paru sekund, ale mimo to strata czasu. Odpowiedź
była oczywista. Atomówki mogłyby dostać się dostatecznie blisko stacji,
żeby je zniszczyć, ale przestrzeń okołoziemska była bardziej zatłoczona
niż kiedy doktryna takiego rozwoju wypadków była tworzona po raz pierwszy. Nie
było sposobu, żeby uniknąć tysięcy niewinnych ofiar i~biliardów dolarów
w zniszczeniach kolonii kosmicznych i~przemysłu.

-- To jest znacznie gorsze -- podkreśliła Walentyna. -- Bezpośredni efekt
eksplozji i~EMP byłby początkiem, istnieje możliwość, że odłamki
uruchomiłyby kaskadę ablacyjną\footnote{ sytuacja opisana przez amerykańskiego
astrofizyka i~pracownika NASA Donalda J. Kesslera, w~której kosmiczne
śmieci nagromadzone na niskiej orbicie okołoziemskiej zderzają się ze
sobą, generując w~ten sposób nowe szczątki, które z~kolei ponownie się
ze sobą zderzając generują jeszcze więcej odłamków. Ten samonapędzający
się proces może doprowadzić do takiego zagęszczenia kosmicznych odpadów,
że nowe satelity będą narażone na bardzo wysokie ryzyko kolizji,
więcej~\url{https://pl.wikipedia.org/wiki/Syndrom\_Kesslera} -- przyp.tłum.}, każde zderzenie tworzyłoby więcej odłamków, aż w~ciągu
kilku dni oczyścilibyśmy niebo.

Kaskada Kesslera była znanym koszmarem, jednym z~najbardziej
śmiertelnych zagrożeń dla habitatów w~kosmosie lub nawet badań. Myra
znała dyskusje i~obliczenia, które sugerowały, że całkowita kaskada
otoczyłaby Ziemię pierścieniem odłamków, która sprawiłaby, że podróże
kosmiczne byłyby nad wyraz niebezpieczne przez \textit{stulecia}.

Generał wyglądał, jakby oceniał wagę. Mogłaby to teraz dojrzeć, te
obliczenia, nawet z~kaskadą, było możliwe, że nowe diamentowe statki
mogłyby przebić się przez odłamki, bariera w~ogóle mogła być
przenikalna, a~w~międzyczasie\ldots

-- Zapomnij -- powiedziała Myra. -- Nie zamierzamy użyć atomówek. -- Jej
palce pracowały, kody błyskały przed jej oczami, próbowała znaleźć
kanał, na którym wjechał sobowtór Generała.

Coś w~jej tonie powiedziało Generałowi, że nie będzie dyskusji. Zamiast
tego, odwrócił się do innych i~powiedział, całkiem przyjemnie: 

-- Towarzyszka nie myśli obiektywnie. Czy chcecie ją zwolnić od
odpowiedzialności?

-- Nie -- powiedzieli mu, satysfakcjonującym jednym głosem.

-- Bardzo dobrze. -- Uśmiechnął się do nich, jakby chciał powiedzieć, że
jest mu przykro, ale musiał spróbować.

-- A \textit{teraz }możesz wypierdalać -- powiedziała Myra. Stuknęła palcem,
triumfując, na przycisk kanału wejściowego i~go wyłączyła.

\chapter{Bar u Roszczeniowca}

Wyszliśmy w~letni zmierzch. Ćmy, zaskoczone, szukały słońca w~lampach
ulicznych. Kilka cichych ulic pomiędzy domem a~Instytutem było teraz
zatłoczonych przez lokalnych mieszkańców wykorzystujących sezon letni w~barach, które normalnie byłyby zapchane studentami. Chłopcy dumnie
kroczyli w~ciemnych, wąskich spodniach, dziewczyny kołysały wielkimi,
jasnymi sukienkami. Musieliśmy wyglądać na mniej szczęśliwą parę,
zabieganą i~się śpieszącą.

Kilka świateł paliło się w~Instytucie, jedno z~nich na korytarzu. Gdy
weszliśmy i~zamknęliśmy drzwi, zapach dymu fajczanego był silniejszy niż
wcześniej i~znajomy.

-- Ktoś jest w~środku -- wyszeptała Merrial.

-- Tak -- odparłem -- to\ldots

Jak na zawołanie, drzwi biura w~korytarzu otworzyły się i~wyszedł Anders
Gantry. Niski mężczyzna z~mocnymi ramionami i~brzuchem jak beczułka
piwa, włosy w~srebrnych lokach jak dym z~jego nieodłącznej fajki. Jego
koszula była zaledwie brudna -- jego żonie udawało się narzucić świeże
płótno na niego co jakiś tydzień -- ale jego kurtka nie była czyszczona
od lat. Pachniała jakby była używana do gaszenia pożaru, co zresztą
robiła.

Był najlepszym uczonym historii na Uniwersytecie i~całkiem możliwe, że
na całych Wyspach Brytyjskich. I~najuprzejmiejszym i~najskromniejszym
człowiekiem, jakiego kiedykolwiek poznałem.

-- O, witaj, Clovis -- zagrzmiał. -- Jak dobrze Cię widzieć! -- Podszedł i~potrząsnął dłoń. -- A kim jest Twoja przyjaciółka?

-- Merrial, doktor Anders Gantry -- powiedziałem.

Przytrzymał jej dłoń i~pochylił głowę nad jej knykciami. 

-- Oczarowany. -- Przez chwilę patrzył na nią w~niejasno zdziwiony sposób, potem odwrócił się do mnie. -- Dobrze, colha Gree, co mogę dla Ciebie zrobić?

Gantry zgodził się nadzorować mój projekt. Moje sumienie uporczywie
irytowało, że nie napisałem, ani nie widziałem go, całe lato.

-- Och, nic w~tym momencie, doktorze Gantry. Robiłem uczciwą część
wstępnych badań na Północy i~prawie skończyłem standardowe odniesienia.
-- Potarłem ucho, niespokojnie przypominając sobie kurz na książkach. -- I~pomyślałem, że skorzystam z~możliwości małej wizyty w~Glasgow, żeby
wpaść do biblioteki.

-- To bardzo chwalebne -- powiedział. Nie byłem pewny dokładnego poziomu
ironii w~jego głosie, ale tam była. -- Raczej brakowało nam Ciebie tutaj.

-- Bardzo ciężko pracuje -- wtrąciła się Merrial. -- Projekt platformy
startowej ma napięty harmonogram.

-- Och, to tam jesteś. Kishorn. Hmmm. Słyszałem, że można tam zarobić
dobre pieniądze. A panna?

-- Pracuję tam w~biurze -- powiedział słodko Merrial. Uśmiechnęła się do
mnie. -- Stąd wiem, że ciężko pracuje. Oszczędza pieniądze na kolejny
rok.

-- Cóż, mniemam, że są sposoby i~sposoby przygotowania się na projekt -- powiedział Gantry bardziej pobłażliwym tonem. -- Żadnych sukcesów z~mecenatem, jak rozumiem?

-- Żadnych na razie, nie.

Klepnął mnie w~ramię. 

-- Może powinieneś spróbować wyciągnąć pieniądze na
badania od badaczy kosmosu -- powiedział. -- Nasza wielka Wyzwolicielka
miała dużo wspólnego z~lotami kosmicznymi. Ciągle mogą być lekcje w~tej
historii, co?

Twarz Merrial zamarła i~poczułem, że moje kolana zmieniają się w~gumę.

-- No to jest myśl -- powiedziałem, tak spokojnie, jak możliwe.

Gantry zarżał. 

-- Tak, możesz nawet ich nabrać takim pomysłem! -- powiedział. -- Powodzenia, jeżeli ci się uda. Teraz gdy zaczynasz utykać,
Clovis, mam ci coś do pokazania. -- Uśmiechnął się, pokazując zęby, żółte
jak u psa. -- To jest w~bibliotece.

Po tym odwrócił się i~skierował ku schodom. Podążyłem, ruszając ustami i~pokazując Merrial swoją bezsilność. Ku mojej uldze, wydawała się
bardziej rozbawiona niż zaniepokojona.

Do czasu, gdy dotarli do otwartych drzwi do biblioteki, zniknął im w~cieniach.

-- Co zrobimy? -- wyszeptałem do Merrial.

-- Jeżeli zostanie w~pobliżu, zajmij go czymś -- powiedziała. -- Ja zdobędę
towary.

Właśnie miałem jej powiedzieć, jak mało prawdopodobne jest, żeby jej się
z tym udało, gdy Gantry wyszedł, sapiąc, niosąc ładunek tekturowych
teczek, które sięgały od jego złożonych dłoni przy pasku do dolnej
szczęki.

-- Oto i~jesteśmy -- powiedział, opuszczając chwiejny stos na stół.
Kichnął. -- Niestety paskudne od kurzu. -- Wytarł nos i~ręce w~jeszcze
brudniejszą chustkę. -- Ale nadszedł czas, żebyś na to spojrzał: osobiste
archiwum Myry Godwin.

-- To naprawdę jest niesamowite -- powiedziałem. Mój głos brzmiał jak u
dwunastolatka widzącego po raz pierwszy nagą dziewczynę. Wziąłem je i~rozłożyłem, jedno koło drugiego. Razem osiem: wypchane tekturowe
portfolio w~porządku co dekadę, od lat siedemdziesiątych dwudziestego
wieku do lat pięćdziesiątych dwudziestego pierwszego wieku.

Ledwie ośmieliłem oddychać na nie, gdy otworzyłem pierwszy z~nich i~spojrzałem na dokument na szczycie stosu, tandetna kopia z~powielacza,
pakiet papierów zszyty zardzewiałymi zszywkami o~dziwnym tytule
\textit{Budowa rewolucyjnej partii w~kapitalistycznej Ameryce.
Opublikowane jako braterska uprzejmość wobec prądu kosmicznego.}

-- Dlaczego wcześniej ich nie widziałem? -- spytałem.

Gantry poruszył się niewygodnie. Spojrzał na Merrial, potarł brodę i~powiedział: 

-- Mam rację, myśląc, że jesteś majsterkiem?

-- Racja, jestem -- powiedziała Merrial bez wahania.

Gantry uśmiechnął się, patrząc z~ulgą. 

-- Um, dobra. Pomiędzy nami i~tak
dalej. Uczeni i~majsterkowie wiedzą, jestem pewien, że musimy być\ldots
dyskretni o~Wyzwolicielki\ldots bardziej niechwalebnych czynach i, hm,
młodzieńczych głupotach. Tak więc, choć wcześniejsi biografowie widzieli
te dokumenty, staramy się nie pokazywać ich studentom. Mam nadzieję,
Clovis, że dojrzysz sposób wyjścia poza, hm, powiedzmy hagiograficzne
podejście przeszłości, bez\ldots -- Zatrzymał, ssąc dolną wargę. -- Och,
dobra, nie ma potrzeby tego mówić.

-- Oczywiście, że nie -- powiedziałem.

Spojrzałem na głównego uczonego z~czym, co byłem pewien, musiało być
wyrazem wdzięcznego szacunku. 

-- Możemy przejrzeć je teraz?

Gantry cofnął się i~wyrzucił ręce w~górę w~udawanym strachu. 

-- Nie, nie!
Nie mogę ci patrzyć nad ramieniem na surowe dokumenty, Clovis.
Oryginalna praca bez pomocy i~takie tam. To jest Twoje i~jest tam praca
doktorska, jeżeli kiedykolwiek gdzieś widziałem taką. Nie, to czas,
żebym sobie poszedł i~cię z~tym zostawił. -- Zawahał się. -- Och, nie
powinienem tego mówić, colha Gree, ale nikomu o~tym ani słowa, ani
żadnej strony, na zewnątrz, dobra?

Odbyłem szybką, intensywną bójkę z~moim sumienie, które gładko
zadziałało i~naskoczyło na mnie. 

-- Nic dla gminu, oczywiście -- powiedziałem ostrożnie. -- Ale co do zasady, mógłbym, cóż, pokazać to lub
przedyskutować z~innymi uczonymi?

-- Rozumie się samo przez się -- potwierdził jowialnie Gantry. Dotknął
boku nosa. -- Jeżeli znajdziesz kogoś, komu zaufasz, że nie stwierdzi, to
jego własne. -- Mrugnął do Merrial. -- Niegodni zaufania, ci uczeni,
myślę, że to odkryjesz. -- Lekko mnie stuknął, myśląc, że figlarnie, w~żebra. -- Zaufanie, człowieku, zaufanie! Jestem pewien, że masz rozum,
żeby zrozumieć i~wyjaśnić to sobie, i~to załatwi Ci nazwisko, zapamiętaj
moje słowa!

-- Dziękuję -- powiedziałem, po bolesnym wdechu. -- Cóż\ldots myślę, że zacznę
od razu.

-- Tak, w~rzeczy samej. Doskonały pomysł. Nie zostawaj długo. -- Jego
współwinny uśmiech był oczywisty, że pomyślał, że to niemożliwe, żebyśmy
zarwali noc. -- Najlepiej jak sobie pójdę -- powiedział, jakby do siebie,
potem wycofał się drzwi i~odwrócił.

-- Dobrej nocy panu, proszę pana! -- zawołała Merrial za nim.

-- Dobranoc -- dobiegło słabo z~klatki schodowej.

Merrial wypuściła długi oddech.

-- Co za dziwny, mały człowiek -- powiedziała, na sposób kogoś, kto
właśnie spotkał kogoś z~krasnoludków.

-- Nie jest całkowicie typowym uczonym -- powiedziałem.

-- Mam nadzieję, że nie -- powiedziała Merrial. -- Nie chciałabym, żebyś
się zamienił w~kogoś takiego jak to.

-- Nie daj Boże! -- powiedziałem, dodając lojalnie -- Ale jest dobrym
człowiek na wszystkie śmieszne sposoby. -- Spojrzałem na stos folderów. -- Może to byłby dobry pomysł -- powiedziałem wolno -- gdybyś poszła zrobić
te rzeczy z~komputerem, a~ja zostałbym tutaj, na wypadek, gdyby wrócił.

-- Och, i~zostawił mnie, bym zmierzyła się z~diabłem samotnie? -- zakpiła
Merrial, potem roześmiała, ustępując. -- Tak, to nie jest taki zły
pomysł. Jeżeli on, lub ktokolwiek inny, wejdzie, zajmij ich czymś. Nie
będę długo i~sobie poradzę.

-- A co z~zaporą bezpieczeństwa?

Machnęła ręką i~wydawał niegrzeczny dźwięk. 

-- Fuj! To małe narzędzie ma
procedury, które mogą usmażyć zapory bezpieczeństwa nad ,,ścianą ogniową'' i~zjeść je na śniadanie.

Rozważając, ile musiała zaprogramować coś o~wiele prostszego niż to,
żeby posortować daty, nie wierzyłem jej, ale zdaje się, że to była
czarna logika.

Uśmiechnęła się i~wyszła. Po lękliwej minucie słuchania, usłyszałem
dźwięk otwierania wewnętrznych drzwi i~szuranie krzesła przeciąganego po
podłodze i~stawianego. Uspokoiłem się nieco i~zwróciłem się do plików,
papierowych plików, mentalnie się poprawiłem, po raz pierwszy tworząc
powiązanie pomiędzy ,,plikami'' użytymi przez Merrial, i~jak zakładałem,
majsterków i~moim własnym.

Byłem chętny zacząć od wczesnych dekad, ale wiedziałem, że byłoby to w~pewien sposób dogodzenie sobie i, że miałbym jeszcze na to dużo czasu.
To późne lata, bliżej czasu Wyzwolenia, były ukryte przed historią.
Wziąłem teczkę dla ostatniej dekady, lat pięćdziesiątych, i~miałem
właśnie ją otworzyć, kiedy usłyszałem krzyk Merrial.

Nie pamiętałem drogi do drzwi mrocznego archiwum. Pamiętam tylko
zatrzymanie się, mój pęd do przodu zblokowany szokiem przerażenia, które
wstrzymało mnie jak wróbla uderzającego w~okno. Teczka plików,
dostatecznie niedorzeczna, ciągle była w~moich rękach i~trzymałem tę
ciężką masę wątłego papieru i~delikatnego kartonu jak broń, lub jak
tarczę.

Merrial również trzymała broń, krzesło, na którym siedziała i~najwidoczniej właśnie się z~niego zerwała. Przed nią, nad komputerem, w~kratownicy rubinowego światła stała postać mężczyzny. Był to wysoki i~tęgi mężczyzna, jego antyczny strój kremowej kurtki i~trzepoczących
spodni, jego wstrząsane białe włosy przepływające w~tej samej
niewidzialnej wichurze, która zerwała jego kapelusz gdzieś w~jakiś długi
korytarz, którego zmniejszająca się perspektywa prowadziły dalej niż
ściany pokoju. Jego twarz była czerwona i~gniewna, jego pięści drżące,
usta krzyczące coś, czego nie mogliśmy usłyszeć.

Trzymając krzesło nad głową, jej przedramię przed oczami, śpiewając
jakieś tajemne abrakadabra, Merrial podchodziła jak ktoś walczący z~ogniem, i~złapała kryształ widzenia i~maszynerię ze stołu. Przewody,
szarpnięte z~niewygodnie umieszczonego gniazda, smagnęły jak zerwana
żyłka wędkarska. Mały czop na końcu, teraz wygięty jak haczyk, poleciał
do mnie i~puknął w~teczkę. Merrial w~tej samej chwili zawirowała i~zobaczyła mnie. Spojrzała się na mnie wzrokiem, za który warto umrzeć, a~potem spokojnie się uśmiechnęła.

-- Czas iść -- powiedziała. Pozwoliła krzesłu uderzyć o~podłogę, odwróciła
się, żeby stanąć twarzą w~twarz z~cicho krzyczącym bytem, który
wzbudziła. Gdy wycofywała się od tej rzeczy, to zaniknęło. Mechanizm
gdzieś w~komputerze warczał, potem przestał. Światło na jego przodzie
błysnęło krótko, potem zgasło.

Wszystkie światła pogasły. Z~dołu schodów usłyszeliśmy cichy oburzony
krzyk. Słyszałem Merrial chowającą aparaturę w~torbie. Uderzyła we mnie,
ciągle się cofając.

Trzymając ręce jakby przez przepaść, przeszliśmy przez duszącą ciemność
biblioteki. Mogłem poczuć suche stare papiery, kruchy klej i~postrzępione nici i~skórę okładek. Z~tych włókien, starożytni mogli
odtworzyć wiele utraconych gatunków drzew i~ras bydła, pomyślałem
szaleńczo. Szkoda, że tak nie zrobili.

Po długiej minucie nasze oczy zaczęły się dostosowywać do słabego
światła, które wpadało mimo żaluzji i~z innych części budynku. Szliśmy
pewniej przez labirynt do drzwi. Na parterze budynku mogliśmy usłyszeć
Gantry'ego błądzącego i~się odbijającego.

Potem, za nami, usłyszałem cichy krok. Merrial też usłyszała i~zamarła,
jej dłoń w~mojej nagle wilgotna. Kolejny krok i~dźwięk czegoś
\textit{ciągniętego}. Prawie rzuciłem się w~bieg, krzycząc.

-- W~porządku -- powiedziała Merrial, jej głos zaskakująco głośny. -- To
jest projekcja dźwięku, tylko kolejna rzecz, by nas wystraszyć.

Za nami, niski, głęboki śmiech.

-- Spokojnie -- powiedziała Merrial.

Moje udo uderzyło krawędź stołu przy drzwiach. 

-- Tylko chwilę -- powiedziałem. Puściłem jej rękę, wziąłem jeszcze jedną teczkę, złapałem
znowu rękę Merrial.

Dotarliśmy do drzwi biblioteki, zatrzasnęliśmy je za sobą i~zeszliśmy po
schodach tak szybko, jak bezpiecznie mogliśmy, lub szybciej. Potem
przestaliśmy być ostrożni i~zwyczajnie uciekliśmy, biegnąc na oślep koło
zagniewanej i~zaskoczonej twarzy Gantry'ego, niesamowitej w~świetle
zapalniczki, którą trzymał nad głową, a~potem w~noc.

Była noc, dookoła na setki metrów nie było zasilania. Przestaliśmy biec,
kiedy dotarliśmy do pierwszych działających lamp ulicznych, przy Great
Western Road.

Spojrzałem na twarz Merrial, błyszczącą od potu, żółtą w~sodowym potoku.

-- Co to, na Rozum, było?

Merrial potrząsnęła głową. 

-- Mam sucho w~ustach -- wyskrzeczała. -- Potrzebuję drinka.

Moje stopy bezbłędnie zaprowadziły mnie do najbliższego baru, ,,U
Roszczeniowca''. Tego wieczoru był cichy i~Merrial mogła zająć miejsce w~rogu, kiedy kupowałem kilka kufli piwa i~kilka whisky. Przy pustym
kominku grał skrzypek i~śpiewała kobieta, bolesny gaelicki tren utraty.

Merrial wychyliła whisky zwinnym ruchem i~lato wróciło na jej twarz.

-- Jezu! -- przeklęła. -- Potrzebowałam tego. Daj mi papierosa.

Podałem, patrząc na nią, gdy go zapalała, rozglądając się dyskretnie,
gdy zapalałem swojego. Pub, do którego uczęszczałem przez moje lata
studiów, był przyjaznym i~wygodnym miejscem, choć dekoracja ścian mogła
nieco zmrozić, jeżeli dumałeś nad nimi: oprawione reprodukcje
starożytnych plakatów, zawiadomień i~rozporządzeń o~,,aktywnym szukaniu
zatrudnienia'' i~,,otrzymywaniu zasiłku''. Miało to coś wspólnego z~życiem z~pomocy publicznej, do czego musiało się uciekać wielu całkiem
zdrowych i~zdolnych ludzi, znanych jako ,,roszczeniowcy'', w~czasach
Posiadania, kiedy ziemia była własnością dziedziców, a~kapitał -- lichwiarzy.

Zwyczajni dwaj staruszkowie przypominali sobie pierwsze kilka wieków
głosami podniesionymi, by poradzić sobie z~lekkim upośledzeniem słuchu w~wyniku wieku, grupa chłopaków dookoła wielkiego stołu grała na grosze,
kilka par kochanków była pochłonięta sobą, a~wysokie nuty pieśni
piosenkarki unosiły się ponad tym.

-- Chciałaś powiedzieć? -- spytałem. Mój głos drżał bardziej niż Merrial
kiedykolwiek w~trakcie tego incydentu. W~tym samym czasie, czułem
zawroty głowy z~powodu ulgi z~naszej ucieczki i~dziwną ekscytującą
mieszankę strachu i~egzaltacji na pewną wiedzę, że moje życie będzie
odtąd nieprzewidywalne.

-- Nie mówiłam -- powiedziała Merrial -- ale i~tak ci powiem. Ta rzecz,
którą widzieliśmy, była diabłem, która strzeże pliki. Jednak -- dodała z~uśmiechem -- wywalenie bezpieczników na kilka przecznic wokół było
najgorszym, co potrafiła.

-- Hej, to pocieszające.

-- Tak, tak jest -- powiedziała bardzo zdecydowanym tonem. -- Lepiej to niż
porażenie elektryczne, które spali ręce lub ogień, który spali cały
budynek. Lub\ldots

-- Co?

-- Słyszałam o~gorszym. Atakowaniu umysłu przez oczy.

-- A śmiałaś się na samą myśl, tam w~podwórzu.

-- Aye, dobra -- powiedziała. -- To tylko ja miałam się z~nimi zmierzyć.
Nie było powodu, żebyś się martwił.

-- Och, dzięki.

Wzięła moją dłoń. 

-- Nie, byłeś tam bardzo odważny.

-- Ach, ani trochę -- zgodziłem się.

-- Więc, w~końcu, dużo nie znaleźliśmy -- powiedziałem, wracając do stołu
z napełnionymi szklankami jakieś dwie minuty później. Na zewnątrz mogłem
usłyszeć narastający rwetes biegnącej milicji, gwizdów i~dzwonków straży
pożarnej. Gdzieś po drugiej stronie ulicy, pojazd z~błyskającym światłem
przetoczył się wolno.

Merrial spojrzała znad przekartkowywanych teczek.

-- Cóż, masz lata dwa tysiące pięćdziesiąt i~tysiąc dziewięćset
dziewięćdziesiąt -- powiedziała. -- To coś. Co \textit{ja} mam -- poklepała
torbę, uśmiechając się -- to znacznie więcej. Może wszystko, jeszcze nie
wiem.

Postawiłem ostrożnie szklanki.

-- Hm\ldots bariera\ldots nie zadziałała, zatem?

-- Do pewnego momentu. Jak mówiłam, moja maszyna, i~logika w~niej, są
silniejsze niż inne. Po prostu nie mogły zatrzymać tej rzeczy od
zrobienia, co to ciągle ostrzegało, że \textit{zrobiłoby}. Można ukraść
kość psu, jeżeli zignorujesz szczekanie i~poradzisz sobie z~ugryzieniami. -- Mniej zadowolonym tonem dodała -- Ale to wszystko zależy,
jak dużo wyciągnęłam, zanim musiałam\ldots

-- Wyciągnąć!

-- Tak.

-- Więc co teraz robimy? -- Spojrzałem na teczki. -- Zdaje się, że będę
musiał spróbować i~wyprostować sprawy z~doktorem Gantry. -- Zmieszane
myśli walczyły w~moim umyśle, jak te programy, o~których mówiła Merrial.
Jedna sekwencja impulsów sprawiała, że myślałem o~planie uniżonych
przeprosin, tuszowania i~wygładzania. Inna sprawiła, że zrozumiałem, że
prawie na pewno miałem poważne kłopoty z~władzami Uniwersytetu, i~całkiem możliwe obraziłem Gantry'ego na sposoby, które trudno byłoby mu
wybaczyć.

-- Och, a~jak zamierzasz to zrobić? -- spytała Merrial. -- Sądzę, że nie
będzie zadowolony z~Twojej ucieczki z~tymi materiałami.

-- To na pewno -- powiedziałem ponuro. -- Ale mógłbym zawsze powiedzieć, że
złapałem je, żeby je uratować, czy coś, i~że zwrócę je za kilka dni. Po
skserowaniu ich, oczywiście. Nie, to ta druga rzecz, która go wkurzy.
Niebiosa wiedzą, jakie zniszczenia ta rzecz zrobiła, wątpię, żeby to
było tylko odcięcie elektryczności. Bardziej wysadzone bezpieczniki w~całym budynki, może gorzej. To zostanie przebadane, nie tylko przez
Uniwersytet. I~będzie chciał się dowiedzieć, kim jesteś i~co robiliśmy.

-- Hmm. -- Merrial wydmuchała cienki strumień dymu, obserwując go, jakby
był wróżbą. -- Dobra, wiedząc, że zna moje nazwisko, i~gdzie pracuję\ldots
powiem Ci coś, colha Gree. Zakładając, że zrobi zamieszanie, lub ktoś
inny zada pytania. To, czego nie chcę, żeby się wydało, to, że miało to
cokolwiek wspólnego ze Statkiem lub z~moim\ldots ludem. Możemy powiedzieć,
z odrobiną prawdy, że byłeś kierowany nadgorliwością, żeby pogrzebać
w\ldots mrocznych miejscach. Że omotałeś mnie do pomocy. Że jest ci bardzo
przykro, sparzyłeś się, i~więcej już tego nie zrobisz. Że oczywiście
dokumenty, które zabrałeś nie będą obejrzane przez nikogo poza wspólnotą
uczonych. Ich \textit{fotokopie}, teraz, mogą być pokazane, ale nie musisz
nic o~tym mówić.

Myślałem o~zaliczeniu Merrial jako honorowego uczonego w~mojej własnej
wersji kazuistyki, ale jej wersja poradziłaby sobie w~razie potrzeby.
Moje dwa przeciwstawne programy się zlały: miałem kłopoty, tak, ale
mogłem się z~nich wydostać, przez wyżej wymienione uniżenie i~ukrywanie.

Zegar nad barem pokazywał, że jest kwadrans po dziesiątej.

-- Wątpię, żeby Gantry był na miejscu -- powiedziałem. -- Nie wiem gdzie
mieszka ani nie znam numeru telefonu, jeżeli ma jakiś. Myślę, że
najlepszą rzeczą będzie spotkanie się z~nim rano, zanim wyjedziemy. -- Wyjąłem bilet powrotny z~kieszeni. -- Pociąg wyjeżdża za dwadzieścia
pierwsza. Będę w~pobliżu, żeby go spotkać około dziewiątej i~spróbować
wyprostować sprawy.

Merrial skinęła głową. 

-- Niezły pomysł. Nadstawiła ucho. -- Sprawy wydają
się przycichać, ale nie sądzę, żeby wędrowanie z~powrotem mogłoby teraz
dobrym pomysłem.

-- Chcesz wrócić i~sprawdzić co mamy?

-- \textit{Dhia}, nie! Wystarczająco się napatrzyłam na jeden dzień. Chcę
tutaj zostać, pić z~Tobą i~może z~Tobą potańczyć, jeżeli odrobina srebra
mogłaby pomóc skrzypkowi zmienić melodię, potem wrócić do mieszkania i~przetestować wytrzymałość tego łóżka z~Tobą.

To nie to, co powinniśmy zrobić, przyznaję, ale czy w~ogóle jesteś
zaskoczony, że tak właśnie zrobiliśmy?

~

Siedziałem na stopniach na zewnątrz Instytutu, w~nieruchomym, chłodnym
poranku pod cieniami wielkich drzew i~patrzyłem na zegarek. Dziesięć do
dziewiątej. Westchnąłem i~zapaliłem kolejnego papierosa. Kilkaset metrów
dalej młot pneumatyczny zaczął kucie. Jasno pomalowane stojaki,
poprzeczki i~stosy zerwanego asfaltu wskazywały, że podobne prace były
już wykonane nocą.

Ścieżka mocy, w~rzeczy samej. Jednym z~powodów, dlaczego to tak było
nazwane, jest to, że obliczenia elektroniczne są nierozerwalnie i~nieprzewidywalnie powiązane z~generowaniem mocy elektrycznej i~można je
zakłócić w~drogi i~niebezpieczny sposób. Miałem nieprzyjemne
podejrzenie, że koszt tego wszystkiego, w~ten czy inny sposób, będzie
się wił jakimś długim systemem Rady Miasta i~księgowości Senatu
Uniwersytetu i~wyląduje u moich stóp.

-- Dzień dobry, Clovis. -- Podniosłem wzrok na Gantry'ego. W~jednym ręku
miał fajkę, klucz w~drugim. -- Wejdź.

Jego biuro miało okno, które zajmowało większość jednej ze ścian,
pokazując kojący widok zaduszonego chwastami podwórka, oraz półki na
książki na innych. Każda pionowa powierzchnia w~pokoju była zaplamiona
lekko na żółto, a~każda pozioma była pod delikatną warstwą popiołu
tytoniowego. Wytarłem bez efektu drewniane krzesło przed jego biurkiem,
podczas gdy siadał w~skórzanym za biurkiem.

Patrzył na mnie przez chwilę, mrugając. Przebiegł palcami przez krótkie
włosy. Westchnął i~zaczął napełniać fajkę.

-- Dobra, colha Gree -- powiedział po minucie onieśmielającej ciszy -- nie
masz pojęcia, o~ile bardziej cię szanuję za to, że przyszedłeś tutaj.
Kiedy zobaczyłem cię chwilę temu, gaszącego papierosa o~chodnik,
pomyślałem ,,Cóż, to jest facet, który wie jak się zachować
przyzwoicie''. Znacząca poprawa po Twojej panice wczoraj, znacząca.

Chrząknąłem, niejasno myśląc, że cokolwiek lekarze mogą powiedzieć,
\textit{musi} być coś szkodliwego w~nałogu, który sprawia, że płuca są
takie szorstkie o~poranku. 

-- Aye, dobrze, doktorze Gantry, obawiam się,
że nie byłem sobą.

-- Och -- powiedziałem sucho -- a~co to było zatem, hmm?

Bez sensu, mój wzrok powędrował ku górze. 

-- To był, hm, demon internet
software, który obawiam się ja i~moja przyjaciółka, hm, niechcący
wywołaliśmy.

Gantry zapalił fajkę i~posłał chmurę dymu.

-- Tak, tyle wywnioskowałem. A co na Ziemię opętało cię, powiedzmy, żeby
grzebać w~mrocznym magazynie, kiedy właśnie dałem ci więcej niż potrzeba
materiałów na lata studiów?

Znowu spojrzałem mu w~oczy. 

-- To był mój pomysł -- powiedziałem. -- Nazwijmy to nadgorliwością. Miałem pomysł, zanim dałeś mi dokumenty,
oczywiście, ale nawet potem, myślałem, że moglibyśmy równie dobrze to
zrobić. Obawiam się, że byłem\ldots raczej oślepiony żądzą wiedzy.

-- I~innym rodzajem żądzy, nie zdziwiłbym się -- powiedział Gantry. -- Ta
Twoja \textit{przyjaciółka}, to coś więcej niż to, prawda?

Nie było powodu, żebym zaprzeczał, więc nie zaprzeczyłem.

-- Dobra -- powiedział. Dźgnął ustnikiem na mnie, potarł kciukiem brodę i~przygryzł dolną wargę przez chwilę. -- Dobrze. Pierwsze, pozwól mi
powiedzieć, że administracja Uniwersytetu ma pracę, które jest odmienna
od autoadministracji wspólnotą akademicką. Musi utrzymywać fizyczną
strukturę tego miejsca, zaopatrzenia i~usługi i~tak dalej, a~mimo
najlepszej woli na świecie nie mogę ingerować w~środki śledztwa \textit{i
dyscypliny}, które mogą być właściwe do przedsięwzięcia w~tej
niefortunnej sprawie. Doceniasz to, prawda?

-- Tak, oczywiście.

-- Dobrze. Cóż\ldots co do następstw akademickich, tu mogę wstawić się za
Tobą, mogę\ldots powstrzymać się od informowania o~tym, jak doszło do tego
demonicznego wybuchu. Niemniej jednak nie mogę kłamać w~Twoim imieniu,
stary. Zrobię, co mogę dla Ciebie, ponieważ myślę, że byłoby to wstydem
odrzucić kogoś z~takimi możliwościami z~powodu, jak to powiedziałeś,
uczonej nadgorliwości. Bardzo zrozumiałej pokusy i~tak dalej. Niektórzy
z Senatu mogliby równie dobrze pomyśleć ,,byłem tam, zrobiłem to, będąc
młodym, oparzony, nauczony, nie mówmy już o~tym'' i~tego rodzaju rzeczy.

Lekko się rozluźniłem na twardym krześle. Bawiłem się papierosem przez
chwilę, niepewny, czy mam pozwolenie na palenie, Gantry pochylił się z~zapalniczką, w~roztargnieniu prawie wypalając mi brwi płomieniem nafty.

-- Dziękuję.

-- Jednakże -- kontynuował, opierając się w~fotelu -- istnieją większe
problemy. -- Machnął fajką, niejasno wskazując otaczające regały ciężko
zdobytej wiedzy. -- My, Brytyjczycy, zaczynamy brać udział w~tej grze w~cywilizację. Kiedy odeszli Rzymianie, nie było publicznej biblioteki,
spłukiwanej toalety, dobrej drogi czy listonosza przez tysiąc lat. Kiedy
upadło Imperium Amerykańskie, myślę, że mogę uczciwie powiedzieć, że
postaraliśmy się lepiej, i~w istocie lepiej niż większość. Straciliśmy
biblioteki elektroniczne, oczywiście, i~bardzo dużo wiedzy, ale
infrastruktura cywilizacji dała sobie radę w~trudnych czasach rozsądnie
nienaruszona. Pod niektórymi względami, nawet się polepszyła. Dużo tego
zawdzięczamy faktowi, że zapisy elektroniczne zostały stracone, a~razem
z nimi łańcuchy lichwy i~czynszu i~inne\ldots mroczne moce, które trzymały
świat w~czymś, co nawet oni mieli tupet nazywać ,,Siecią''.

Wstał i~podszedł spokojnie do rogu i~oparł łokieć o~półkę. 

-- To, co mamy
zamiast sieci, to majsterkowie. -- Machnął znowu ręką. -- I~telefony,
telegrafy, biblioteki i~tak dalej, oczywiście, ale to poza tą kwestią.
Majsterkowie opiekują się naszymi obliczeniami, które nawet na ścieżce
światła większość z~nas\ldots niechętnie wykonuje, z~powodu tego, co
wydarzyło się w~przeszłości, ale jest wdzięczna, że jest ktoś, kto je
robi. To sprawia, że\ldots nie są pariasami, ale zdecydowanie lekko
stygmatyzowanym zawodem. I~ten właśnie stygmat, widzisz, paradoksalnie
zapewnia, lub daje pewne ubezpieczenie, czystość ich produktu. Utrzymuje
to dwie ścieżki, jasną i~mroczną, rozdzielonymi. Widzisz, do czego
zmierzam?

-- Nie -- odpowiedziałem. -- Obawiam się, że nie.

-- Och. -- Wyglądał na nieco zawiedzionego powolnością zrozumienia. -- Cóż,
mówiąc wprost, jedna sprawa to uczeni ryzykujący ciała lub dusze w~mrocznych magazynach. Nie zdarza się, że tak powiem, ale pomiędzy Tobą,
mną i~bramą, się \textit{zdarza}. To całkiem inna, gdy robi to majsterek.
Może zanieczyścić kryształy widzenia, rozumiesz. Zły biznes.

Podszedł i~wpatrzył się we mnie. 

-- Efekt, mój przyjacielu, jest taki, że
lepiej przyprowadź tę swoją koleżankę majsterkę z~czymkolwiek, co
zabrała, i~przynieś te teczki, które \textit{pożyczyłeś} razem z~nią,
jeżeli chcesz, by ten epizod był przeoczony. Jasne?

-- Tak, ale\ldots

-- Żadnych ,,ale'', Clovis. Nie masz czasu. Wyjdź i~wróć, zanim
ktokolwiek cokolwiek zauważy, to jest szansa.

-- Zrobię, co będę mógł -- powiedziałem, wystarczająco szczerze, i~wyszedłem.

Gdy śpieszyłem się do mieszkania, zastanawiałem się, co do cholery
moglibyśmy zrobić. Miałem nadzieję trzymać pliki papierowe co najmniej
tydzień, co dałoby mi wystarczająco dużo czasu, żeby określić, czy było
w nich coś ważnego. Nie było jednakże sposobu, żeby Merrial ,,zwróciła''
jakiekolwiek pliki komputerowe, które udało jej się pozyskać. Mogłaby
udawać, że je skasowała z~pamięci jej kryształu widzenia, ale wątpiłem,
czy to by oszukało Gantry'ego. Chciałby sam kamień, a~ona prawie na
pewno by go nie oddała.

Gospodyni wpuściła mnie, ponieważ zostawiłem klucz do drzwi Merrial.
Uśmiechnąłem się z~przymusem i~pobiegłem po schodach, zapukałem do drzwi
pokoju, gdzie zostawiłem Merrial drzemiącą. Nie było odpowiedzi, więc po
cichu otworzyłem drzwi.

Merrial tam nie było. Ani niczego, co należało do niej. Ani dwóch teczek
z dokumentami. Rozejrzałem się, oszołomiony przez chwilę, a~potem
przypomniałem sobie, co Merrial powiedziała o~kserowaniu dokumentów. Z~ulgi poczułem słabość. Zebrałem swoje własne rzeczy, sprawdziłem, czy
nic naszego nie zostało w~pokoju i~poszedłem na dół.

-- Aye -- powiedziała gospodyni -- dzioucha wyszła chwilkę po Ciebie.
Ostawiyła mi klucz.

-- Czy pytała się o~punkty kserokopiowania w~okolicy?

-- Nie. Ale jest tylko jeden, za rogiem. Niy możesz ominąć.

-- Och, dzięki!

Wybiegłem na zewnątrz, wzdłuż ulicy i~za róg. Punkt tam był, na pewno,
ale nie Merrial. Nikt odpowiadający jej, dość jednoznacznemu, opisowi
nie dzwonił.

Powędrowałem Great Western Road w~rodzaju oszołomienia i~zatrzymałem się
przy balustradzie mostu nad Kelvin. Kolejny most, który przekroczyliśmy
tramwajem, był kilkaset metrów w~górę rzeki, ruiny stacji metra, zabite
deskami i~opatrzone srogimi ostrzeżeniami, były na drugim brzegu.
Restauracje rybne przy nabrzeżu, gdzie jedliśmy ostatniej nocy, wysyłały
zapachy głęboko smażonego ciasta. Rzeka kłębiła się niżej, popiół mojego
zatroskanego papierosa nie zakłócał najmniejszych z~jej fal.

Nie mogła tak po prostu uciec z~rzeczami. Byłem dostatecznie lojalny
wobec niej, żeby być pewny jej lojalności, nawet nie rozważałem -- prócz
chwilowej, hipotetycznie -- że po prostu wykorzystała mnie, żeby dostać
informacje, których poszukiwała. Najdrastyczniejsza, pozostała możliwość
była taka, że jakoś została odkryta i~musiała wyjechać z~jakiegoś
ważnego powodu lub przymusu. Jednak gospodyni z~pewnością zauważyłaby
taką rzecz, więc nie mogło to się zdarzyć w~mieszkaniu.

Zatem pomiędzy tym miejscem a~punktem ksero. Stworzyłem dziki plan
przejścia chodnikiem, szukania wskazówek, przepytania przechodniów.
Wydawało się to melodramatyczne.

Bardziej prawdopodobne, powiedziałem sobie, że po prostu gdzieś
pojechała z~jakiegoś własnego powodu. Miała własny bilet powrotny.
Oczekiwała po mnie, żebym się domyślił i~spotkał ją na dworcu. Mogłem
sobie wyobrazić nas śmiejących nad nieporozumieniem, nawet jeżeli
szalone telefony musiałyby zostać wykonane do Gantry'ego.

Lub nawet, mogła pójść do kolejnego punktu ksero!

Milicjant przeszedł obok, jego wzrok mimochodem mnie zauważył.
Zaczekałem tam, gdzie byłem, póki nie zniknął z~widoku, świadom, że
odejście od razu tylko wyglądałoby dziwnie, i~również świadom, że
patrzenie zmartwiony nad balustradą na dwudziestometrowy upadek w~rzekę
mogło wzbudzić zainteresowanie nawet najmniej podejrzliwego milicjanta.

Wtedy, naturalnie, zastanawiałem się, czy została aresztowana, za
nieautoryzowany dostęp do Uniwersytetu, nekromancję, lub po prostu na
ogólnych zasadach. Ale wtedy znowu, jeżeli byłaby, było to moim
zmartwieniem tylko na poziomie osobistym: jako majsterka miała dostęp do
dobrego prawnika, tak jak i~ja miałbym, jako uczony.

Tak więc to koniec mojego podnieconego myślenia oraz spojrzenie na
zegarek, który pokazywał kwadrans po dziesiątej, zdecydowało, bym
poszedł na stację i~na nią poczekał.

Pociąg miał odjechać o~jedenastej dwadzieścia. Pięć po jedenastej
odstawiłem pusty kubek po kawie, zgasiłem papierosa i~poszedłem do
publicznego telegrafu. Tam wstukałem wiadomość:\\ GANTRY UNIW INST HIST
ZALUJE OPOZ W~ZWROCIE STOP KONTAKT Z~CARRON STOP SZACUNEK CLOVIS.

Byłem w~momencie naciskania przycisku wysłania, kiedy poczułem zapach i~pot Merrial za mną. Potem pochyliła się koło mojego policzka i~powiedziała, ciepłym, rozbawionym głosem: 

-- Bardzo lojalne, wobec niego
i wobec mnie.

Odwróciłem się i~złapałem ją w~ramiona. 

-- Gdzie, do cholery, byłaś?

-- Wyślij tę wiadomość -- powiedziała. -- Opowiem Ci w~pociągu. -- Uśmiechała się do mnie i~poczułem, jak wszystkie zmartwienia blakną, gdy
przytuliłem ją mocno, potem odstąpiłem, trzymając na odległość ramienia,
jakby upewniając się, że naprawdę była. Jej torba wyglądał na większą i~cięższą niż wcześniej.

-- Masz papierowe pliki?

-- Tak -- powiedziała, ważąc torbę. -- No dawaj.

Wysłałem wiadomość, potem pośpieszyliśmy ręka w~rękę na peron. Pociąg
nie był mocno wykorzystany, znaleźliśmy przedział -- pół wagonu -- dla
samych siebie, wsunęliśmy się na siedzenia i~spojrzeliśmy na siebie
ponad stołem, śmiejąc się.

-- Dobra -- powiedziałem. -- Opowiadaj. Trochę się zmartwiłem, muszę się
przyznać.

Zacisnęła palce na mojej dłoni. 

-- Przepraszam -- powiedziała. -- Ale
pomyślałam, że zniknięcie było dobrym pomysłem. W~ten sposób, jeżeli
Gantry, czy ktokolwiek, naciskałby na Ciebie w~sprawie plików, lub,
wiesz, innych plików, mógłbyś uczciwie powiedzieć, że nie mógłbyś,
naprawdę nie wiedziałbyś, gdzie jestem i~wyglądałbyś na naprawdę
zakłopotanego, kiedy doszłoby do tego, że wróciliby z~Tobą do pokoju.

-- Och, racja. Naprawdę byłem zakłopotany, muszę Ci to przyznać. Ale
jeżeli byliby ze mną, mogliby tak samo się domyślić, jak ja i~przyjść na
dworzec.

Wzruszyła ramionami. 

-- Musiałabym się ukrywać. -- Przeczesała palcami
przez włosy, uśmiechając się nieśmiało. -- Nie jestem w~tym zła.

-- I~złapała pociąg w~ostatniej chwili?

-- Lub coś. -- Nie była zainteresowana roztrząsaniem spekulacyjnych planów
awaryjnych. -- Tak czy inaczej, jesteśmy tutaj i~mamy towary. Gantry nic
nie może zrobić, żeby je teraz zabrać.

-- Tak. Jednak, musimy mu posłać wiadomość z~Carron, zapewnić, że są w~bezpiecznym miejscu.

-- Jak mówiłeś. Więc wszystko załatwione.

Pociąg zaczął się poruszać. Wyjrzałem na widzialnie przesuwające się
dworzec i~peron, płynący do tyłu w~ruchu względnym, potem spojrzałem na
nią.

-- Nie -- odpowiedziałem. -- To nie jest tak proste.

Wysłuchała mojej relacji o~tym, co Gantry powiedział o~majsterkach i~mrocznym magazynie. Kiedy skończyłem, powoli pokręciła głową.

-- Powinieneś ukryć moje bycie majsterką -- powiedziała.

To był szok. 

-- Jak mógłbym? -- zaprotestowałem. -- Już się tego domyślił i~byłoby łatwo to sprawdzić. Nie chciałem kłamać. Szczególnie kłamać i~być
przyłapanym, gdy tylko podniesie telefon.

Jej usta się zacisnęły. 

-- Zdaje się, że tak. Słusznie. Zaufanie Twojego
człowieka jest ważniejsze w~dłuższym czasie. A może nawet kręcenie
potwierdziłoby jego podejrzenia. -- Spojrzała, jakby na jej ramionach w~tej chwili pojawił się ciężar.

-- Wykręcałbym się, Prawdo pomóż mi, byłbym \textit{skłamał}, gdybyś mnie
poprosiła!

-- Nie mogłabym tego zrobić -- powiedziała. -- Ach, to jest takie
skomplikowane!

-- Hej, wszystko w~porządku -- powiedziałem. -- Wymyślimy coś. Wyślę
Gantry'emu jakąś wymówkę, da nam to czas sprawdzić pliki, i~oddamy je za
tydzień. Wezmę wolny następny poniedziałek, jeżeli muszę.

Oczy Merrial nagle zaszkliły. Mocno zamrugała.

-- \textit{Dhia}, mam nadzieję, że to będzie tak proste! -- Westchnęła. -- Chciałabym powiedzieć ci więcej. -- Pokręciła głową. -- Ale nie mogę.

-- Dlaczego nie?

-- Och, \textit{mo chridhe}! Jestem majsterką, a~majsterki muszą uważać na
to, co mówią. Nawet gdy, szczególnie gdy, ich języki spędzają czas w~ustach innych!

-- Więc masz tajemnice rzemiosła, -- powiedziałem sucho -- które musisz
utrzymać. Nie mam nic przeciwko.

Spojrzała na mnie, jakby miała powiedzieć coś pilnego, a~potem wszystko,
co powiedziała: 

-- Nie powinnam się tak martwić. Wszystko się
prawdopodobnie ułoży.

-- Tak, na pewno -- powiedziałem, udając, że się z~nią zgadzam. 

-- Och, dobra. 

-- Zatem, czy zaglądamy do teczek?

-- Ok -- powiedziała, wyciągając je. -- Powiem Ci coś. Przejrzyj te
wczesne, a~ja przejrzę te późne. To zwiększy szanse, że któreś z~nas
znajdzie coś, co możemy \textit{zrozumieć}.

-- W~porządku -- powiedziałem.

Otworzyłem teczkę dla lat dziewięćdziesiątych i~przerzuciłem
niecierpliwie przez nudne i~poczciwe rzeczy medycznej charytatywności i~pewne fascynująco nieprawdopodobne statystyki ekonomiczne Kazachstanu.
Przy końcu znalazłem coś bardziej osobistego: strony wyrwane z~notesu na
spirali, najprawdopodobniej dziennik. Ślęczałem nad bazgraniną
Wyzwolicielki:

\textit{Czw 16 czer 1998 Wędrówka po pozostałych NJ księgarniach, chyba
nostalgia. Podjęta na Against the Current u św. Marka, trend miejsce,
zostawia puby na marginesie, wydaje się ok. Stara klika Critique ciągle
wali, Suzi W~w~AtC, itd. Przynajmniej lojalni, nie tak jak moi, ha.
Potem poszłam na Revo Księg Avakiana sprawa, szalona niż zwykle. Mieli
fałszywe elektryczne krzesło w~środku na kampanię Mumii. Przeleciałam
stare debaty na SU itd. Depresyjna myśl ,,Marksizm jest gównem'' ciągle
w głowie. Potem Księgarnia Unity na Zachodniej 23 Nie mogłam się zebrać
na Pathfindera. Po mojej małej przygodzie, nie wiem czy chcę zmierzyć
się z~kadrą Czwartej Międzynarodówki. Lub oni ze mną. Ach.}

\textit{Pią 17 czer 1998. Gorące wilgotne popołudnie, burza potem.
Spotkałam M na promie Staten Island. Oparłam się o~reling i~patrzyłam na
Statuę przez mgłę. M wydaje się wiedzieć, że mówię ludziom o~jego
podejściach. Sprawa jest taka, że wydaje się nie mieć nic przeciwko.
(Dziewczyna, różowe włosy na promie. Chyba ta sama dziewczyna w~Bostonie. Śledzą mnie czy paranoja?)}

Nic z~tego nie potrafiłem zrozumieć, więc przeszedłem do ostatniej
kartki.

\textit{Czw 17 gru 1998. Znowu Ałma-Ata. Telewizor w~holu nastawiony na
CNN. Zielone światło miasta zapada w~ciemność. Szpital się wypełnia.
Jebani Jankesi. Oto tutaj próbuję pomóc w~rozwoju, tam oni próbują go
cofnąć.}

Po tym, nic, prócz plamy i~gniewnych zapisków, gdzie długopis wbił się i~porwał stronę. Może dotarła do końca tego notatnika lub przestała
prowadzić dziennik. Przekartkowałem resztę dokumentów, z~przygniatającym
uczuciem, że przeglądanie ich w~obecnej niejasności zabrałoby mi nawet
więcej, niż myślałem. Potem daremnie odwrócona strona zatrzymała mnie.

Była to kserokopia starego artykułu, który napisała, ale to mała reklama
przypadkowo włączona na marginesie złapała moje oko. Dotyczyła
publicznego spotkania na ,,Pięćdziesiąt lat Czwartej Międzynarodówki'' i~miała w~jednym z~rogów symbol, który był identyczny z~monogramem na
naszyjniku Merrial. Tylko tyle mogłem zrobić, żeby nie uderzyć się w~czoło lub krzyknąć na własną głupotę. To, co myślałem, że było literami
,,G'' i~,,T'' było w~rzeczywistości młotem i~sierpem symbolu
komunistycznego, a~znaczenie ,,4'' było oczywiste. Nie złapałem
powiązania po prostu dlatego, że symbol był obrócony w~przeciwną stronę
niż ten na sowieckiej fladze.

Złowieszcze znaczenie młota i~sierpa sprawiło, że poczułem mdłości.
Następstwa tego samego symbolu pojawiającego się poprzez taką przepaść
czasu wywołała zawroty głowy.

Zamknąłem teczkę i~podniosłem wzrok i~popatrzyłem prosto w~równie zbite
z tropu oczy Merrial.

-- To wszystko albo jest niezbyt interesujące lub kompletnie kurwa
niezrozumiałe -- powiedziała.

-- To samo tutaj -- powiedziałem. -- Zostawmy to.

Całe to długie popołudnie, rozmawialiśmy o~innych rzeczach.

Bitwach, w~większości, jak sobie przypominam.

Pociąg wjechał na stację Carron Town punkt osiemnasta. Słońce było
ciągle wysoko, późne popołudnie ciągle ciepłe. Po raz kolejny zmęczeni i~sterani podróżą, Merrial i~ja wyszliśmy z~pociągu z~energią i~przypływem
głodu. Merrial poprowadziła prosto do Karonady i~usiedliśmy w~ciemnym
kącie dziwnie pachnącym barze z~talerzami pstrąga z~farmy, świeżego
groszku i~młodych ziemniaków w~towarzystwie współdzielonego dzbanka
piwa.

-- Nie mogę doczekać się powrotu do Twojego miejsca -- powiedziałem -- odrobiny prywatności i~wsadzenia twarzy prosto w~\ldots pliki.

Roześmiała się. 

-- Aye, byłoby wspaniale wreszcie spojrzeć na nie, bez
ciągłego oglądania się za siebie.

Jednak gdy to mówiła, oglądała się za siebie, co robiła co około minutę
przez cały posiłek. Plecami opierała się o~ścianę, ja opierałem się o~bar. Pub zaczynał się wypełniać ludźmi z~projektu, na krótkiego drinka w~drodze do domu lub na kwaterę. Jak dotąd nie usłyszałem żadnego głosu,
który bym znał.

-- Wydajesz się nieco na krawędzi -- powiedziałem.

-- Aye, cóż, jak mówiłam w~pociągu\ldots

-- Fergal?

-- Tak.

-- Spodziewasz się go spotkać tutaj? -- spytałem, przypominając sobie, że
byliśmy w~tym barze z~jej, choć mile widzianej, sugestii.

Rozłożyła ręce. 

-- Może. Zależy.

-- Od czego? -- Naskładałem puste talerze i~zapaliłem papierosa.

-- Och, od tego, jak chcą to rozegrać -- powiedziała, brzmiąc niezwykle
gorzko.

-- Tajemnice czy nie -- powiedziałem, próbując zachować lekki ton -- będziesz musiała mnie do nich dopuścić, wcześniej czy później. Zaczynam
być ciągle zmęczony widzeniem Ciebie zmartwionej.

-- Nie \textit{muszę} nic robić! -- wybuchła. -- A Ty nie \textit{musisz}
widzieć mnie wyglądającą jak cokolwiek!

Nic nie powiedziałem, patrząc na nią, zszokowany i~zirytowany, ale już
jej wybaczający. Była pod dużym napięciem z~powodów, o~których
wiedziałem i~powodów, o~których wiedziałem, że nie wiem.

-- Ach -- powiedziała, znowu delikatnie -- nie miałam tego na myśli, colha
Gree. Nie byłeś wychowany jak ja, żeby być twardą.

Na to musiałem się uśmiechnąć. W~tamtej chwili wyglądała na bardziej
wrażliwą niż twardą. Jej oczy się rozszerzyły. Usłyszałem kroki za mną,
a potem Fergal niezaproszony wsunął się na ławkę koło mnie.

-- Cześć -- powiedziała Merrial, nie ciepło. Jej wzrok wrócił do mnie.

-- Och, cześć -- powiedziałem. Spojrzał na nasze drinki. -- Moja kolejka,
chyba. -- Sięgnął do tyłu nad ramieniem i~strzelił palcami. Większości
nie uszłoby to na sucho, ale jemu tak. Pół minuty później, barman
podawał kolejny pełny dzbanek na stolik.

-- Więc Merrial -- powiedział cicho -- masz to?

-- Mamy, -- powiedziała Merrial. -- o ile mogę powiedzieć. Sprawdzałam cały
ranek i~to jest całe archiwum.

-- A gdzie to robiłaś? -- wtrąciłem się, nieco oburzony.

-- Kelvin Wood -- powiedziała Merrial, uśmiechając się rozbrajająco
bezwstydnie. -- W~krzakach.

-- Więc to chciałaś zrobić.

Merrial skinęła, rzucając spojrzenie spode brwi. Fergal spojrzał na nią,
potem na mnie, jak gdyby przypominając nam, że miał ważniejsze rzeczy na
głowie.

-- Dobrze -- powiedziałem.

-- To dobre wieści -- powiedział Fergal do Merrial. Krótko się roześmiał.
-- Mówiąc łagodnie, co?

-- Aye -- powiedziała. -- Tak jest.

-- Tak czy inaczej, Clovis -- powiedział Fergal -- zdajesz sobie sprawę, że
informacje, które pomogłeś pozyskać, wymagają przejrzenia przez
ekspertów. Raczej szybko, w~istocie, biorąc pod uwagę, jak długo może to
zająć.

-- Oczywiście -- powiedziałem. -- Szansa, żebym spojrzał na to pierwszy,
tylko rzut oka?

Pokręcił głową. 

-- Przykro mi, Clovis. Nie masz pojęcia, bez urazy, jak
dużo tam jest. To niesamowita ilość niezbyt dobrze zorganizowanych
informacji. W~czasie, który Tobie by zajęło zrozumienie czegokolwiek, my
moglibyśmy szukać informacji, którą \textit{wiemy} jak zinterpretować.
Każda godzina może się liczyć.

-- Tylko minuta! -- powiedziałem, przerażony i~oburzony. -- Nikt nie
wspominał o~niczym takim. Też chcę na nie spojrzeć, a~nie pozwolić im
zniknąć w\ldots

-- Jakiejś kryjówce majsterków? -- Fergal uniósł brew. -- Zapewniam Cię, że
tak nie będzie. Masz moje słowo, że nie będziemy ich trzymać długo,
maksymalnie tygodnie, i~że zobaczysz je i~przeszukasz w~swoim tempie,
gdy tylko skończymy.

-- Ale -- powiedziałem -- skąd będę wiedział, że nie zostały zmienione,
nawet przypadkowo? Ponieważ muszę być w~stanie na nich polegać.

Merrial wyglądała na rozpaczliwie zakłopotaną. Rzuciła ostre, szybkie
spojrzenia Fergalowi i~pochyliła się ku mnie nad stołem.

-- Pomyśl o~tym, człowieku -- powiedziała cicho. -- Te rzeczy są i~tak
nielegalne, raczej nie mógłbyś ich zacytować w~przypisach, co? Możesz
ich użyć do znalezienia wskazówek do materiałów, do których
\textit{możesz} się odnieść. Więc po prostu musisz nam zaufać, zaufać mi,
że informacja nie będzie zmieniona.

-- Dobrze -- powiedziałem niechętnie.

-- Dobry człowiek! -- Wychylił szklankę i~wstał. -- Dzięki za pomoc. -- Fergal sięgnął nad stołem. Merrial już opróżniała skórzaną torbę z~rzeczy osobistych. Zacisnęła pasek i~podała. Fergal złapał ją, gdy ja
ciągle patrzyłem, zdziwiony, działaniami Merrial.

-- Czekaj! -- powiedziałem. -- Ciągle tam są pliki papierowe. Nie możesz
ich zabrać!

Fergal uniósł brew. 

-- Dlaczego nie?

-- Te dokumenty należą do Uniwersytetu.

-- Obawiam się, że nie -- powiedział Fergal, brzmiąc pełen skruchy. -- Należą do nas.

Spojrzałem się gorączkowo na Merrial, która skinęła lekko, smutno głową.

-- Kim są, kurwa, ,,nas''? -- Zażądałem, choć już podejrzewam odpowiedź. -- No
weź, mogę dać ci kserokopie, jeżeli trzeba.

-- Nie wystarczy, chłopie.

-- To mi je oddajcie.

-- Przykro mi -- powiedział Fergal. -- Nie mogę.

Przesunąłem się na nogach, poruszyłem łokciem, wszystko odruchowo. Oczy
Fergal się zwęziły.

-- Nie -- powiedział bardzo cicho -- nawet nie \textit{myśl} o~zadzieraniu ze
mną.

Właściwie myślałem o~nakrzyczeniu i~wezwaniu innych w~barze, niektórzy
przypatrywali się tej konfrontacji. Jednak coś w~pozycji Fergala i~spojrzeniu sugerowało, że jedynym wynikiem takiej burdy byłaby jego
ucieczka po zadaniu ciężkich uszczerbków naszej stronie, zaczynając ode
mnie. A jakąkolwiek stronę przyjmie Merrial, lub nawet gdyby próbowała
się nie mieszać, prawdopodobnie zostałaby skrzywdzona.

Mój honor nie był stawką w~powstrzymania odejścia Fergala z~dokumentami
-- byłby zagrożony przy ich odzyskiwaniu -- a~na razie nie miałem prawa
ryzykowania życiem lub kończyną swoją lub innych w~tej sprawie.

-- Zabierz to, majsterku -- powiedziałem. -- Mogę zaczekać.

Uśmiechnął się, bez protekcjonalności.

-- Mam nadzieję, że się jeszcze spotkamy -- powiedział i~wyszedł.

Spojrzałem na kilka zaciekawionych, napiętych twarzy przy barze,
wzruszyłem ramionami i~wróciłem do stołu, gdzie Merrial, drżąc, zapalała
jednego z~moich papierosów.

-- Jakieś wyjaśnienie byłoby w~porządku -- powiedziałem, tak normalnie,
jak tylko potrafiłem. Jedno z~moich kolan drżało.

Merrial wzięła głęboki oddech i~głęboko się zaciągnęła.

-- Przepraszam -- powiedziała. -- Nie mogę, naprawdę.

-- Ale spójrz -- powiedziałem. -- Dlaczego po prostu nie powiedziałaś mi,
żebym je schował, albo powiedziała, że musimy je oddać\ldots

Zacząłem się denerwować, byłem zmieszany i~wtedy, w~końcu, do mnie
dotarło.

-- Zgadzasz się z~nim! -- powiedziałem. -- Ty faktycznie się
\textit{zgadzasz}, że ma jakieś \textit{prawo} do tych dokumentów, żeby
pierwszy obejrzeć pliki, że nikt inny nie może nawet na nie spojrzeć,
nie krzywdząc go. Włączając w~to mnie.

Spojrzała beznamiętnie na mnie.

-- A Ty nie zamierzasz mi powiedzieć dlaczego.

Mały ruch jej głowy.

-- I~wiedziałaś od początku, że tak się może zdarzyć.

Mniejsze kiwnięcie.

-- W~porządku -- powiedziałem. W~dzbanku był jeszcze litr.
Rozlałem je nam obojgu i~sam zapaliłem papierosa, pochylając się do
przodu w~dym Merrial, prawie w~jej włosy. Wnętrze mojej
dłoni pocierało pod okiem, zirytowany na siebie, przestałem to robić i~w zamian bawiłem się papierosem. Dźwięk śmiechu i~konwersacji przy barze
był jak dźwięk wodospadu na skale, obmywający nas, ukrywający rozmowę.
Mogliśmy powiedzieć wszystko.

-- Naprawdę jestem w~rozterce -- powiedziałem. -- Nie mogę uwierzyć, że
mnie wystawiłaś, ale póki mi nie powiesz, o~co naprawdę chodzi\ldots

-- Powiedziałam Ci -- powiedziała. -- Nie mogę. Nie możesz mi zaufać?

-- Och, całkowicie mogę ci zaufać -- powiedziałem. -- Ale jeżeli nie zwrócę
tych teczek, tak jak obiecałem, nikt na Uniwersytecie mi już nigdy nie
zaufa.

Wyglądała na spiętą, rozdartą, tak jak się czułem.

-- Bardzo mi przykro z~tego powodu -- powiedziała. -- Ale nic nie mogę z~tym zrobić.

-- No weź -- powiedziałem. -- Musi coś być. Cholera, jeżeli zwrócę pliki,
mogę dać Twoim ludziom kopie wszystkich plików. Czy to nie warte więcej
dla nich niż to, co właśnie dostali?

-- Nie rozumiesz -- powiedziała Merrial. -- Teraz gdy wiemy o~innych
teczkach, zamierzamy zdobyć je wszystkie. Jak powiedział Fergal, są
nasze.

,,Nasze'', w~istocie! Byłem nieskłonny, lub nieprzygotowany, do
zakwestionowania jej społeczeństwa, do którego mogła się odnosić.
Rozłożyłem dłonie. -- Nie możesz oczekiwać mnie, że zaakceptuję tego bez
cholernie dobrego powodu, którego mi nie podajesz.

-- Powiedziałam Ci. Nie mogę. Więc dlaczego po prostu nie zapomnimy o~tym
wszystkim?

-- Merrial -- błagałem, przerażony głębią jej braku zrozumienia -- te pliki
są częścią mojej pracy, moja kariera zależy od nich. Więc, proszę\ldots

Sięgnąłem, dotykając jej włosów.

Jej oczy zaiskrzyły.

-- Och, \textit{odpierdol się}! -- powiedziała mi, nie do końca krzykiem,
ale głośno i~wystarczająco dobitnie, by głowy się odwróciły.

-- Tak zrobię -- odparłem, wstałem i~wymaszerowałem. Spojrzałem do tyłu
przy drzwiach i~zobaczyłem jedynie górę jej głowy, włosy opadłe do
przodu i~jej ręce na jej twarzy. Drzwi zatrzasnęły się za mną.


\chapter{Zachęty Zachodu}

-- Już po wszystkim -- mówiła Walentyna.

-- Po wszystkim? -- spytała Myra. Pokręciła głową, rozglądając się po
biurze. Wala, Andriej i~Denis byli tutaj, usadowieni na biurkach lub
parapetach okien. Ekrany centrum dowodzenia zniknęły jak sen. Parvus
unosił się na skraju jej wizji, wyglądając, jakby miał coś do
powiedzenia.

-- \textit{Pucz} -- wyjaśniła Walentyna.

-- Tak po prostu?

Myra się wpatrzyła, mrugając przez opcje przedstawiane przez Parvusa.
Personal miał swoje analizy i~był zajęty zgadzaniem się z~Walentyną.
Stacje orbitalne zajęte przez Ruch Kosmiczny wystarczyły do strzeżenia
ich obleganych enklaw i~kompleksów startowych, ale nie, by przechylić
równowagę światowych potęg na ich korzyść. Rada Bezpieczeństwa Narodów
zachowała kontrolę nad ReONZ, ale stacje orbitalne, które oparły się
zamachowi, zrobiły tak w~swoim własnym imieniu, nie w~imieniu ReONZ.
Pozostały niebezpiecznie samodzielne.

Na Ziemi lokalne równowagi wszelkiego rodzaju uległy zmianie, prawie
całkowicie przez nagłe, ponowne oceny prawdziwej wagi różnych stron,
którą wywołały cholerne podmuchy rzeczywistej walki. Spory zostały
rozstrzygnięte lub ponownie otwarte, całe armie zostały zmobilizowane
lub rozpuszczone siłą potężnych cieni rzuconych na ekrany analizy dzięki
małym starciom w~polu.

-- Boże -- powiedziała Myra z~niesmakiem. -- To jest \textit{tak}
dekadenckie. -- Przypominało to jej najemników Renesansu, nad którymi
jęczał Machiavelli w~\textit{Discorso}, kalkulujących, kto by wygrał,
gdyby walczyli i~dotrzymali swojej umowy taką decyzją, jednocześnie
unikając krwawych spraw rzeczywistej bitwy. -- Nikt nie chce prawdziwej
walki, raczej podążają za symulacjami. Mówimy o~pornografii przemocy.
Dupki.

-- Jest gorzej -- powiedział grubo Denis. -- Mamy przejebane. -- Rzucił
projekcję przedziału czasowego \textit{Jane's} i~podświetlił laserem
istotne fragmenty. -- Spójrz.

Profil wojskowy i~ogólna wiarygodność MRRNT nie była już czymś, co
ostrożni stratedzy, oceniając po przeszłych działaniach i~obecnych
plotkach, wyceniali wysoko. Było to nieistotne.

-- Zostaliśmy odkryci -- powiedział Denis Gubanow. -- Na precyzyjnie zły
sposób. Zawsze musieli się liczyć przynajmniej z~możliwością, że mamy
atomówki. Ochrona Wzajemna, lub, tak czy inaczej, Reid, wiedzieli, że
mamy. Rzecz w~tym, nie użyliśmy ich, więc założono, że albo ich nie
mamy, albo nie mamy serca ich użyć. Przeszliśmy od bycia Burkina Faso z~atomówkami \footnote{ oryg. Upper Volta istniejące państwo w~latach 1958-1984,
zob.~\url{https://pl.wikipedia.org/wiki/Republika\_G\%C3\%B3rnej\_Wolty}
-- przyp.tłum.} do bycia Burkina Faso bez. A użyte przez nas bronie nie
zadziałały.

-- Zadziałały\ldots -- zaczęła Walentyna, raczej defensywnie.

-- Ha! -- parsknęła Myra. -- Działały bardzo dobrze, tylko nie zniszczyły
celów. Tak, rozumiem, że nasza postawa odstraszania jest mocą dobra.

Gorąca linia telefoniczna -- solidny, staromodny, jednoznacznie czerwony
telefon na biurku Myra -- zaczęła dzwonić. Przez chwilę patrzyła na niego
z powątpiewaniem, potem wzruszyła ramionami i~podniosła słuchawkę.

-- Myra Godwin-Dawidowa.

Przerwa.

-- Witaj, Myro. Tu Dave.

Przez chwilę była cicho, skonsternowana.

-- Myro? Tu \textit{David Reid.}

-- Tak. Witaj -- powiedziała. -- Czego teraz chcesz?

W jego odpowiedzi było sekundowe opóźnienie.

-- Czego chcesz, lubię to. -- Nawet w~trzeszczącym łączu
laser-telefon, mogła usłyszeć jego furię. -- Miałaś całą sytuację w~równowadze, wiesz o~tym? Miałaś jebany \textit{decydujący} głos,
przewodnicząca Dawidowa! Miałaś opcję nuklearną i~ją odrzuciłaś! Prawie
wolałbym, żebyś użyła swoich cholernych atomówek przeciwko nam,
przynajmniej w~ten sposób Rada Bezpieczeństwa przejęłaby władzę i~musiałaby przyjąć odpowiedzialność. Byłaby jakaś szansa zakończenia
chaosu, czego wszyscy naprawdę chcieliśmy. Tak jak jest, to zmieniłaś
coś, co powinno być grą końcową w~kolejny jebany pat.

-- Nie rozumiem, jak to Tobie pogorszyło sytuację.

Usłyszała dźwięk uderzania i~zrozumiała po chwili, że uderzał czymś w~głowę.

-- To pogorszyło sytuację dla nas\textit{ wszystkich}! To jak entropia,
Myro, nie rozumiesz tego? Wszyscy wspięli się kilka schodków,
\textit{eskalowali}, to jest kurwa opis tego. Jesteśmy wszyscy wyżej, ale
względnie nie jesteśmy w~lepszym miejscu, a~straciliśmy energię, pracę w~całym procesie. A wiesz, kto tylko zyskał z~tego wszystkiego? Margines,
jebani barbarzyńcy, oto kto. Włączając to Twoich bezbożnych, lokalnych
komunistów.

-- To Ty powinieneś o~tym pomyśleć. Zanim rozpocząłeś cholerny zamach
stanu.

Reid wziął głęboki wdech, długie westchnienie po drugiej stronie.

-- Tak, masz rację. To moja wina. Nie oczekiwałem kontrzamachu, to
wszystko.

-- Jakiego kontrzamachu?

Ponownie dziwne opóźnienie.

-- Nie graj niewinnej. Ktoś przejął większość satelitów i~na pewno nie
byli to moi. Ani ONZ, jeżeli o~to chodzi.

-- Nie wiesz, kto to był?

-- Nie. Więc kto to był? Musisz wiedzieć.

Myra pomyślała przez chwilę. Ach, cholera, i~tak się dowie.

-- Czwarta Międzynarodówka -- powiedziała mu. -- Frakcja kosmiczna,
organizacja bojowa.

Sekundy mijały, potem usłyszała głośny śmiech Reida. 

-- Hahaha! Ok, Myra,
niech tak będzie. I~tak się dowiem. W~międzyczasie, rzuć okiem na
północno\dywiz wschodnią granicę i~zobacz, czy to wszystko nadal jest takie
zabawne. Wychodzę z~tego, jestem na promie na Lagrange. Cześć.

Zakończył połączenie w~taki sposób, że brzmiało to jak uderzenie
słuchawką w~staromodny telefon, uderzeniem, od którego mrugnęła.

Zanim mogła spojrzeć na północno\dywiz wschodnią granicę, Parvus wszedł w~ramę
i podniósł dłoń. Myra gestem pokazała innym, by czekali.

-- Tak?

Mocny fantom machał dłońmi ekspansywnie. 

-- Ach, Myra, muszę szybko
zmienić Twoje inwestycje. Otrzymałem gorącą wewnętrzną wskazówkę \ldots -- położył żółty palec na rumianym nosie -- \ldots że Ochrona Wzajemna
likwiduje swoje zasoby.

-- Co! -- Myra w~tym czasie tak się przyzwyczaiło do używania ,,zasoby''
jako eufemizmu dla ,,atomówek'', że prawie zanurkowała pod biurko. Jej
zaskoczone spojrzenie przeleciało ostatnie wiadomości, nic.

-- Och, masz na myśli \textit{finansowo}.

-- Oczywiście, że finansowo. Kiedy zacznie się ostatnia wojna, jasno ci to powiem. Nie, Ochrona Wzajemna wszystko wyprzedaje, wycofuje się.

-- Wycofuje skąd?

-- Stąd. Z~Kazachstanu. -- Spojrzał na nią smutno, prawie współczująco. -- Z~Ziemi.

Przez następne kilka dni stało się jasne, że głównymi beneficjantami z~krótkiego rzucenia się w~przemoc był margines, który skorzystał z~własnej przewagi rozproszenia -- oraz szybkiej likwidacji Ochrony
Wzajemnej -- żeby rozszerzyć swoje domeny kraj za krajem, oraz
Szinosowiety.

Ruszyli wzdłuż przełęczy na Zajsan, na południowy\dywiz wschód. Kazachskie
bombowce dalekiego zasięgu biły w~sinosowieckie drony bojowe, urządzenia
niepokojących i~rozmaitych kształtów, kombinacje prawie sowieckiej
mechanicznej niezgrabności z~prawie organicznym nanotechnologicznym
połyskiem. Ich wraki, lub ciała, zaśmiecały drogi i~wzgórza za Buran.
Funkcjonujące komponenty miały niepokojącą tendencję do odbudowywania
się. Naloty bombowe zostały przerwane, gdy zaopatrzenie w~bomby zaczęło
się wyczerpywać. Drużyny szinosowieckiego \textit{specnazu} -- rzucając
hologramowe zwody, duchy radarowe, dźwiękowe sobowtóry -- potykały się
pośród wraków i~okopywały na skrajnych pozycjach swojego natarcia. W~międzyczasie, czołgowa ludzka armia, lub horda, oskrzydlała góry Ałtaju
od północnej strony pasma: tocząc się na południe i~zachód z~basenu
Katun, potem drogą i~koleją z~Barnaul, bez oporu. Pod koniec czwartego
dnia po próbie zamachu, przekroczyli północną granicę Kazachstanu i~się
zatrzymali.

Rada \textit{oblys} w~Semeju -- najwidoczniej zmiękczona przez zastraszanie
lub przewrót -- zaprosiła ich, a~oni radośnie zaakceptowali zaproszenie.
Wjechali jak wyzwoliciele, powitani wiwatującymi tłumem i~osiedlili się
z każdym objawem planowanego pozostania.

Czerwony telefon znowu zadzwonił.

-- Chingiz Sulejmanow -- przedstawił się dzwoniący. Obecny prezydent
Kazachstanu. Jego przezwisko ,,Dżyngis Prezydent'' nie było do końca
sprawiedliwe. -- Mam dla Twojego rządu propozycję, Madame Dawidowa, i~dla
Ciebie osobiście\ldots

Następnego ranka Myra wstała, ubrała i~spakowała. Większość jej bagażu
już wysłała na lotnisko. Załadowała stosy starych plików, w~formatach od
dyskietek do rzeczywistego papieru, do kilku skrzynek, zapieczętowała
jako bagaż dyplomatyczny i~wysłała pod pewien adres docelowy. Potem
zaczęła ogałacać mieszkanie, w~rodzaju wściekłości na siebie. Rozkazała
jakimś dzieciakom z~milicji zabrać rzeczy na dół, fizycznie nie była do
tego gotowa i~wiedziała o~tym.

Zawartość sypialni poszła pierwsza, wszystkie poduszki i~narzuty,
koronki, lakier i~lapis-lazuli, wszystko do wielkich, czarnych,
plastikowych toreb, które pojechały prosto do najbliższego sklepu
rzemieślniczego za szyderczą kwotę. Niech ruszą w~swoją drogę, niech
podróżują po obwodach jak dobra handlowe, jak muszelki kauri i~paczki
marlboro, aż do Camden Locks i~Greenwich Villages świata. Plakaty na
ścianach były następne, do kolejnego sklepu, dla innych kolekcjonerów.
Płyty winylowe i~kompaktowe -- tak one były nazywane, pomyślała z~uśmiechem, gdy podnosiła ułożone luzem -- do trzeciego.

A potem książki. To bolało, ale kontynuowała. Smutno, ponuro wyciągając
je z~półek, sortowała i~układała. Co chwila kusiło ją przebrać,
zbłądzić. Od czasu do czasu zatracona w~książce lub we wspomnieniach,
jakie sprowokowały. Mrugnięcie, otarcie oczu, zatrzaśnięcie okładek,
kichnięcie od kurzu, kontynuowanie. Jej oczy poczerwieniały, palce
ściemniały, a~ramiona bolały.

Większość książek też poszło na bazar. Resztę załadowała na tył małej
ciężarówki. Umyła się i~rozejrzała się po dźwięcznej pustce jej
mieszkania. Ciągle było do zamieszkania, było to miejsce, do którego
mogła wrócić, ale nic jej w~nim nie zostało.

Wsunęła Bibliotekę Kongresu 2045, inne biblioteki, koncerty, galerie
sztuki i~archiwa na górę jej torby podróżnej i~rozmieściła noże i~pistolety na pasku i~kieszeniach. Chłopaki, którzy zataszczyli jej
rzeczy na targ, wracali jeden po drugim, ze zwitkami pieniędzy. Wydawała
więcej niż trzeba, żeby im zapłacić, jednemu po drugim.

Ciężarówka z~książkami ruszyła przed nią, dobrze z~przodu, gdy wsiadła
na konia z~torbą i~wyjechała po raz ostatni do obozu.

-- Otwierać!

Myra krzyczała, uderzając w~żelazną bramę. Ciężarówka zaparkowała z~przodu, czekając z~cierpliwością robota na zniknięcie przeszkody.
Dowolne elektroniczne prośby, jakie wykonała, najwidoczniej zostały
zignorowane.

Myra mogła zobaczyć dlaczego. Prócz płotu, niedużo zostało z~obozu, a~dalej z~jednej strony -- zbyt daleko, żeby mogła skorzystać -- widziała
mężczyzn rozbierających płot i~zwijających go w~wielkie bele i~koła.
Nic, prócz trawy i~drogi, nie rozciągało się przed nią na kilkaset
metrów. Tam, gdzie były chaty, widziała tylko grudki ciemnego materiału
na stepie, z~ludźmi wędrującymi dokoła i~ścigającymi się dziećmi.
Fabryki nie zniknęły, ale wyraźnie wysychały, jakby ich konstrukcja
następowała w~przeciwnym kierunku.

Zsunęła opaskę, poprawiła parametry i~spojrzała na scenę. Nikt nie
słyszał jej krzyków. Cholera. Wyciągnęła starą podróbkę Glocka z~Nowego
Wietkongu z~kabury, uspokoiła i~przytrzymała konia, wystrzeliła, nie w~powietrze, ale ostrożnie w~kępę trawy kilkadziesiąt metrów dalej. Klacz
się spłoszyła, kula i~tak zrykoszetowała, ale strzał dał efekt, jaki
chciała. Postać odłączyła się od kręcącego się tłumu i~pomaszerowała ku
niej. Kim Nok-Yung z karabinem.

-- Cześć, Myra. -- Nie mógł przestać się uśmiechać. Wbił kod w~płytę
zamka. Brama zaskrzypiała, otwierając, zostawił ją otwartą. Myra
wprowadziła konia i~ciężarówka podążyła za nią, potem trzymała tempo
koło niej. Nok-Yung wskoczył na podest i~trzymał się jedną ręką,
wymachując triumfująco karabinem w~drugiej, jakby wjeżdżał na czołgu do
wyzwolonej stolicy.

-- Czy to nie wspaniałe!

Zaraziła się jego entuzjazmem.

-- Tak, to wspaniałe. Cieszę się, że to koniec, Nok-Yung.

Minęli jedną z~fabryk, znikającą przed ich oczami, kruszącą się z~jednej
krawędzi w~osobliwie uporządkowany kurz, kurz, który kapał jak kolumna
mrówek wzdłuż ścieżek do pozostałych maszynerii lub na trawę. Część
kurzu układała się w~toporne stosy, które twardniały w~kostki
określonych kolorów, bezwładne, z~których wiatr niczego nie zwiewał.
Inne linie kurzu łączyły się w~szklane sferoidy, obsydianowe lub
przezroczyste, które leżały w~wysokiej trawie jak lśniące otoczaki,
kamienie i~głazy.

-- Komponenty sterowania, komputery i~tak dalej -- wskazywał Nok-Yung. -- Kostki są materiałem konstrukcyjnym.

-- Zastanawiam się, czy ktokolwiek je będzie zbierał?

Koreańczyk się roześmiał. 

-- Zabierzemy niektóre części sterowania ze
sobą, mogą być cenne, tam gdzie się udajemy.

-- Och?

Spojrzał z~boku na nią, prawie przepraszająco. 

-- Semej -- powiedział. -- Do Szinosowów.

Myra powstrzymała się od ściągnięcia cugli konia. 

-- Co? Dlaczego, na
miłość boską? -- Machnęła ramieniem, wściekle, dookoła i~poza nią. -- Możecie zostać z~nami, jesteście tu mile widziani, w~naszej republice
lub gdziekolwiek w~Kazachstanie. Hej, człowieku, Bajkonur na pewno cię
weźmie, pomyśl o~tym!

Pokręcił głową. 

-- Oczywiście, niektórzy z~więźniów tutaj zostaną. Jednak
ja, Se-Ha i~inni, jedziemy do Szinosowietów. Niektórzy z~nas mają już
tam przyjaciół i~rodzinę. Nie ma innego miejsca dla nas. Nawet bez
Ochrony Wzajemnej \ldots -- odwrócił się i~splunął w~trawę -- \ldots nadal mamy
długi i~czarne listy. Żadnej pracy tam, w~domu, tylko odpracowywanie
długu. Pośród Sinosowietów będziemy wolni. -- Uśmiechnął się, już nie
przepraszająco, ale dziko. -- I~jest tam praca do zrobienia, praca dla
nas. Oni są przyszłością.

-- Ale nie wiesz niczego, jacy oni naprawdę są. Tylko dlatego, że
nazywają siebie komunistami, to nie znaczy, że są \textit{mili},
powinieneś to wiedzieć!

Nok-Yung roześmiał się cierpko. 

-- Nie mają Wielkiego Wodza, albo
Drogiego Przywódcy, możesz ich pokochać lub porzucić, a~my zamierzamy
spróbować.

Do tego czasu dotarli do brzegu tłumu. Myra zatrzymała konia i~zasygnalizowała ciężarówce stop. Nok-Yung zeskoczył ze stopnia. To, co z~daleka wyglądała jak bezcelowe wędrowanie, okazało się poruszającymi się
celowo ludźmi, zabierającymi i~układającymi swoje rzeczy z~samorozbierających się chat. Większość z~nich zignorowała jej przybycie.
Myra nie była zaskoczona lub dotknięta. Korzyści jej nadzoru łatwo było
przeoczyć, a~sam komitet obozowy nie był popularny pośród więźniów,
chociaż był wybrany. Jak związek w~firmie, częściowo reprezentował
interesy pracowników, chociaż wystarczająco często przekazywał wolę
właścicieli.

Zauważyła Shin Se-Ha, eleganckiego w~niestety przestarzałym garniturze
\textit{sararimana}\footnote{ japoński, etatowy pracownik umysłowy, który
okazuje lojalność i~zaangażowanie zatrudniającej go korporacji,
więcej~\url{https://en.wikipedia.org/wiki/Salaryman} -- przyp.tłum.}, który prawdopodobnie założył pierwszy i~ostatni raz na
swoim sądzie, ale który teraz podkreślał nową wolność. Niósł małą
walizkę przez biegające dzieci i~brnących dorosłych. Teraz inne pojazdy
i bestie toczyły się lub człapały na widok, wezwane przez telefony
przywrócone ich właścicielom.

Myra stała, pieszcząc kark klaczy, uciszając ją, gdy japoński matematyk
skierował kroki ku niej. Próbowała przypomnieć sobie, za co został
zesłany: nadużycie zasobów przedsiębiorstwa lub inny pretekst, uruchomił
udoskonalenia Otoh dla neomarksistowskich schematów reprodukcji
kapitału, na danych empirycznych, na komputerach uniwersyteckich.
Prawdziwym powodem były jego wyniki, które niedyskretnie rozesłał w~arkuszach: złowieszcza algebra równań Otoh pokazywała kompletne
załamanie za dwa cykle biznesowe.

To było jeden boom i~jeden krach temu.

-- Witaj, Myro -- powiedział. Odstawił walizkę. Prawdopodobnie zawierała
wszystko, co posiadał, był tego rodzaju człowiekiem. Przerażającym, na
swój sposób.

-- Cześć, Se-Ha. Nok-Yung mówił mi, że wybierasz się\ldots -- skinęła głową -- \ldots na Wschód.

-- Tak. Przepraszam, jeżeli nie akceptujesz.

Bardzo bezpośrednie! Słońce świeciło w~jej twarz jak lampa do
przesłuchań i~wiatr ciągle szumiał. To był czas na powiedzenie prawdy
lub stawienie czoła gorszym mękom.

-- Czy ja to akceptuję, czy nie, to nie jest pytanie -- powiedziała. -- Jesteście wolni i~nie mam nic do powiedzenia w~tej sprawie. Ale powinna
ostrzec was, że Republika Kazachstanu będzie opierać się Szinosowym, tak
jak i~ja. Nie obalicie nas. Byłoby mi przykro być po przeciwnej stronie
w bitwie, ale\ldots

Wzruszyła ramionami.

-- Mnie też byłoby przykro -- powiedział Shin. -- Ale ,,tak to wygląda'',
\textit{ach-tak}?

-- \textit{Ach-tak} w~istocie. -- Uśmiechnęła się i~nagle zrozumiała, jak
Reid był w~stanie utrzymać swoje nieprzyjaźnie typu ,,bez urazy'' tak
długo. -- W~międzyczasie, mam coś dla was. -- Wskazała dłonią ciężarówkę.
-- To i~wszystko w~niej. -- Rzuciła mu panel sterowania ciężarówki, który
zgrabnie złapał. -- No idź, zobacz.

Drzwi kliknęły otwarcie, zatrzasnęły się. Wrócił. Złapał jej dłoń,
pochylił się nad nią, jakby chciał pocałować jej palce i~zrobił krok do
tyłu.

-- Jestem Ci dłużny -- powiedział, sztywno. Potem rozłożył ręce,
wyglądając jak na Zachodzie i~raczej zakłopotany niż na Wschodzie i~zobowiązany. -- Co mogę powiedzieć, Myro? Jesteś bardzo miła.

-- Ach, nie mów głupot, mój przyjacielu -- powiedziała. -- Ty, Nok-Yung i~inni sprawili, że moja praca tutaj była bardziej satysfakcjonująca niż
byłaby w~innych przypadkach. Jestem wam to winna, jeżeli cokolwiek. -- Współdzieliła z~nim konspiracyjny uśmiech. -- A biblioteka rewolucyjnej
teorii może być przydatna tam, gdzie się udajecie, prawda?

-- Tak. Nie wiem, czy mogę przyjąć odpowiedzialność. -- Pokręcił głową,
myśląc o~tym. -- Są w~tym pojeździe książki i~dokumenty, które
\textit{nigdy nie zostały zeskanowane}.

Myra poklepała kieszeń. 

-- Nawet nie dla Biblioteki Kongresu?

-- Nawet tam! -- Wydawał się uznać tę myśl za niesamowitą, naruszenie praw
natury. To powstrzymało decyzję Myry, gdy proste założenie utrzymywane
przez pół życia, że wszystko jest zarchiwizowane, że każda jota i~kreska
była niezmieniona w~krzemowym niebie, zostało nagle skonfrontowane z~rzeczywistością, że niektóre myśli mogły stawić czoła wieczności w~wątłej arce pulpy drzewnej i~że była za nie odpowiedzialna. Jej zaangażowanie wzrosło.

-- Och, dobra. Powinnam już je dawno przeczytać, a~jeżeli nie, to już
zbyt późno dla mnie.

Gwar dookoła nich narastał. Pojazdy wyły, konie i~wielbłądy rżały i~pluły. Niektóre dzieci, nawet dorośli, płakali, zostawiając to miejsce,
które mimo całego przymusu nie narzucało zbyt ciężkich niedostatków, i~które było znajome. Niektórzy wytrwale podnosili szklane kamienie, czy
jako talizmany, czy jako drobiazgi na handel, Myra nie potrafiła
określić. Tysiące byłych więźniów rozchodziło się na całym horyzoncie.

Kilku innych mężczyzn schodziło się tam, gdzie stała, zbierając się
dookoła, rozmawiając po koreańsku lub po japońsku, uśmiechając się do
niej, wspinając się na pakę ciężarówki. Pojawił się Nok-Yung i~potrząsnął dłonie.

-- Będziemy w~kontakcie.

Było tyle do powiedzenia, tak dużo, że nie mogłoby być wypowiedziane.

-- Jeszcze się spotkamy -- powiedziała Myra. -- Wszystkiego najlepszego,
chłopaki. Powodzenia z~komuchami.

-- Ha! -- Nok-Yung podniósł zaciśniętą pięść i~uśmiechnął się do niej. -- Pewnego dnia dołączysz do nas, Myro, zobaczysz. Żegnaj i~dzięki!

Wrzucił torbę na pojazd i~wskoczył na miejsce kierowcy, potem roześmiał
się, gdy Shin Se-Ha wspiął się przez przeciwne drzwi i~pomachał panelem
sterowania przed jego nosem. Ciągle krzycząc i~machając, mężczyźni
odjechali, podskakując na stepie, zdecydowanie na północny-wschód.

Myra obserwowała ich aż do horyzontu, potem wsiadła na konia i~pojechała
do miasta. Tylko raz się obejrzała i~zobaczyła, że nie było nic do
ujrzenia.

~

Lotnisko stolicy Międzynarodowej Republiki Robotników
Naukowo\dywiz Technicznych miało tylko jeden budynek terminala. Była to
wielka, otwarta przestrzeń, poznaczona franczyzami. Nigdy się nie
przejmowali Urzędem Celnym czy Urzędem Imigracji. Pomiędzy oknami od
podłogi do sufitu -- z~czarującymi widokami stepu, pasa startowego,
bloków mieszkalnych, platform startowych i~stepu -- wisiały identycznie
gigantycznie plakaty Trockiego, Korolewa, Kapicy, Gagarina i~Guevary.
Idea, wiele lat temu, była taka, żeby hala wyglądała komunistycznie:
nieco zuchwalstwa macho. Teraz miało wygląd miejsca, które miało
\textit{wpaść} komunistom w~ręce, raczej ku niesmakowi Myry. Zatłoczone
ludźmi siedzącymi na zbyt dużym bagażu, ich miny przeskakujące pomiędzy
zniecierpliwieniem a~rezygnacją przy każdej zmianie na ekranach odlotów.
Na litość boską, myślała Myra, Semej jest dwieście kilometrów dalej,
przesadzają.

Jej własny czas odlotu nie miał nastąpić w~ciągu kolejnej godziny.
Potwierdziła rezerwację w~odprawie, upewniła się, że jej bagaż był na
pokładzie i~odmówiła propozycji oczekiwania w~poczekalni pierwszej
klasy. Zamiast tego przeszła do starej franczyzy Nicafe i~usiadła nad
kawą z~papierosem, by stopy odpoczęły, i~aby dogodzić odrobinie
nostalgii.

W starych, dobrych czasach przed Trzecią Wojną Światową, wypiła tutaj
dużo kawy z~wieloma mężczyznami po drugiej stronie stołu. Zawsze różnymi
i prawie nigdy takimi, których by polubiła: brzydcy, jowialnie wojskowi
w przeważającej części, na jet-lagu, z~zarostem, w~pogniecionych
mundurach z~wieloma orderami, albo dyplomaci, albo \textit{biznesmeni}
lśniący, ogoleni, wyperfumowani w~jedwabnych garniturach. I~zawsze,
trzymając się kilka metrów dalej, poza ponurym pierścieniem ochroniarzy,
byliby fotografowie i~reporterzy, by nagrać podpisanie umowy. MRRNT
nigdy nie korzystał z~tajnej dyplomacji, otwartość była ważną kwestią w~handlu odstraszaniem nuklearnym.

Działało dobrze, aż do wojny jądrowej.

Niemcy rozpoczęli Wojna o~Integrację Europejską bez atomówki po swojej
stronie. To nie było przeoczenie, to był istotny element zaskoczenia.
Kiedy ich pierwsza fala przejechała bezpiecznie przez granicę Polski,
złożyli Myrze bardzo hojną ofertę na zbywalne odstraszanie nuklearne.
Dzikie wydzwanianie Myry po klientach pokazało, że nikt nie był chętny
do umowy: nie za żadną kwotę pieniędzy, na całkowicie racjonalnej
podstawie, że Trzecia Wojna Światowa nie była dobrym okresem na
sprzedaż. Myra rozważała odcięcie ich i, tak czy inaczej, sprzedanie
opcji Niemcom, ale jej lojalność biznesowa z~nią wygrała. Wygrała też
nad niemieckimi okupantami w~Kijowie i~cywilami Frankfurtu i~Berlina.
Ciągle czuła się temu winna.

Dla towarzystwa założyła opaskę i~wezwała Parvusa. Dla śmiechu usadziła
wirtualny obraz na krześle przy stole naprzeciwko niej. Konstrukt
ztriangulował swoją obecną pozycję, dostrzegł żart i~się uśmiechnął.

-- Co mogę dla Ciebie zrobić, Myro?

-- Powiedz mi, co myślisz o~Generale. -- Nie przejmowała się mówieniem do
pustego powietrza. Nie była jedyną osobą w~tej kawiarni rozmawiającej z~personalem lub sobowtórem.

-- To jest podchwytliwe -- powiedział Parvus. Przebiegł palcami po
włosach, pogrzebał w~zmiętej kurtce za papierosami. Zapalił i~zrelaksował się, osobowość uzależniająca była częścią pakietu, aspekt
jak rzeczy się ze sobą trzymały. -- Oczywiście były plotki \ldots -- lekceważąca smuga dymu -- \ldots że Czwarta miała od dawna dostęp do dzikiej
AI. Lub na odwrót, według ich przeciwników. -- Parvus pokazał zęby. -- To
jeszcze z~czasów, gdy AI o~takiej komplikacji były rzadkie, przed
Rewolucją lub Osobliwością.

-- To jest Osobliwość? -- Teraz była kolej Myry na machnięcie papierosem.
-- Nie, żebym zauważyła.

-- To jedna z~tych rzeczy, których nie zauważasz, kiedy jesteś w~środku -- zgodził się Parvus. -- Jak masowe wymieranie, które dzieje się obecnie
wokół nas.

-- Ale to jest powolne, o~to chodzi. Osobliwość miała być szybka na czymś
więcej niż skala geologiczna.

-- Była.

-- Och. -- Nie była pewna, czy chciała prowadzić tę rozmowę dalej. -- Tak
czy inaczej, wracając do Generała, oraz tego, co o~nim sądzisz.

-- Ach, tak. Hmm. Bardzo niebezpieczny, w~mojej opinii. Jego użycie
twarzy i~głosu jest wybitnie skuteczne na dostanie się pod skórę \ldots
ludzi ze skórą. Uważaj się za szczęśliwą, że nie używa feromonów,
przynajmniej nie w~sieci.

-- Rozumiem, że sam jesteś nieczuły na jego uroki.

-- Tak. -- Parvus westchnął. -- Na szczęście dla mnie, brak mi
samoświadomości.

Myra ciągle gapiła się zaskoczona niespodziewaną uwagą personala -- na
pewno ironiczną, choć nie była pewna, na jakim poziomie -- kiedy miejsce
Parvus zostało zajęte przez kazachskiego mężczyznę w~gładkich ubraniach
i pomarszczonej twarzy. Trzymał z~tyłu rozpraszające małe dziecko i~cicho oskarżającą kobietę z~napuchniętymi oczami. Kobieta zajęła kolejne
krzesło, przytrzymała wykręcające się dziecko na kolanach.

Myra wymrugała Parvusa z~widoku, niejasno mając nadzieję, że AI nie się
uraziła, podniosła opaskę i~uśmiechnęła się do mężczyzny i~jego rodziny.
Jego uśmiech w~odpowiedzi był wymuszony.

-- Dzień dobry, Pani Prezydent. Dlaczego nas zostawiasz?

Myra się rozejrzała. Wydawało się, że nikt inny jej nie zauważył. Kult
jednostki był kolejnym strategicznym pominięciem w~ich socjaldemokracji.
I bardzo dobrze, nie chciała być oblegana przy wylocie. 

-- Nie zostawiam was -- powiedziała żarliwie, pochylając się i~mówiąc jakby poufnie. Jej misja nie była jeszcze publicznie ogłoszona, ale nie miała nic przeciwko
prawdziwej plotce zawczasu. Tylko szczegóły były delikatne, a~na tym
poziomie tajności było to bezcelowe, była pewna, że jej plan podróży
krążył w~sieci, pogrzebany pomiędzy setkami nieprawdziwymi wersjami, z~równie autorytatywnych źródeł. 

-- Jadę na Zachód, po pomoc. Wsparcie ekonomiczne i~wojskowe.

Mężczyzna patrzył sceptycznie. 

-- Przeciwko Szinosowom? Ale wobec nich,
nie mamy szans. Nie mamy obronnych granic.

-- Nie, ale Kazachstan ma, i~to w~imieniu Kazachstanu jadę.

-- Dla Chingiza? -- Twarz mężczyzny rozjaśniła się, spojrzał na żonę, jak
gdyby pocieszająco. -- Więc zamierzamy wyrzucić czerwonych z~Semeja?

-- Nie możemy bombardować Semeja -- powiedziała Myra, powtarzając
dokładnie przekazane jej słowa prezydenta Sulejmanowa. -- Ale możemy
utrzymać przełęcz na wschód od jeziora Zajsan i~możemy zatrzymać dalsze
postępy na północny-wschód. Jeżeli szybko otrzymamy pomoc. Siły Związku
Chińsko-Radzieckiego raczej nie będą próbować niczego przez kilka
tygodni, ponieważ są rozciągnięte. I~nie lubią walk frontowych. Tak
długo, jak Republika Kazachstanu pozostania wobec nich wroga, nie wejdą.
-- Uśmiechnęła się zachęcająco. -- I~mogę być pewna, że nasza republika
pozostanie nieprzyjazna.

W ogóle tego nie była pewna. Było wystarczająco dużo społecznego
niezadowolenia, dostatecznie zrozumiałego w~jej zbędnym państwie
robotników, żeby Sinosowieci mieli na czym pracować. Bez wątpienia
pierwsi agitatorzy już dryfowali, pośród pierwszych uciekinierów z~Semeja. Jednak mężczyzna wziął jej słowa na poważnie.

-- Tak -- powiedział, dodając -- jeśli Allach zechce. Ale my wyjeżdżamy, z~wszystkim, co mamy.

-- Nie mogę was winić -- powiedziała Myra. -- Życzę Wam dobrze. Mam
nadzieję, że zobaczycie szansę na powrót, kiedy sprawy będą bardziej\ldots
pewne.

-- Może. -- Mężczyzna wzruszył ramionami, kobieta lekko się uśmiechnęła,
dziecko nagle wrzasnęło. Odeszli, patrząc niepocieszeni na ekrany,
zostawiając Myrę w~depresji.

Mężczyzna wyglądał jak drobny kupiec, jeden z~wielkiej klasy średniej
wyniesiony przez mieszaną ekonomię Republiki. Pomimo wszystkich diabłów,
którymi malowała ściany, MRRNT zawsze była bardziej NEP\footnote{ NEP -- Nowa
Polityka Ekonomiczna -- określenie doktryny polityki gospodarczej w~ZSRR
w latach 1921--1929. Określana jest jako rodzaj gospodarki mieszanej lub
kapitalistycznej,
zob.~\url{https://pl.wikipedia.org/wiki/Nowa\_Polityka\_Ekonomiczna}
-- przyp.tłum.} niż permanentna rewolucja: jedynie obrona i~przemysł
kosmiczny były państwowe, a~oprócz systemu opieki społecznej wszystko
inne (co w~terminach PKB nie dawało zbyt dużo, musiała przyznać) było
mniej lub bardziej \textit{laissez-faire.} Zastanawiała się, czego ta
rodzina bała się ze strony Szinosowietów, którzy według wszystkich
relacji zostawiliby ich własność i~bogobojność w~spokoju. W~pewnym
sensie to nie było zaskakujące: Sinosowiety postępowali dzięki blefowi i~zastraszeniu, przez wyglądanie i~brzmienie na bardziej radykalnych i~komunistycznych niż w~rzeczywistości byli, a~ich nieobecność w~sieciach
łączności pozostawiła wielki pusty ekran na odtwarzanie najbardziej
złowieszczych spekulacji. Zatem może ten rodzaj bezpodstawnego strachu
był ceną ich postępu.

Cóż, sprawiłaby, że zapłacą wyższą cenę w~mocniejszej walucie. Dopiła
kawę i~ruszyła do poczekalni odlotów.

W Ałma-Acie odebrała dokumenty, paszport dyplomatyczny i~kartę do linii
kredytowej w~odlotowej Samsonite Diplock przekazanej przez kuriera, a~w~trakcie lotu do Izmir przejrzała je. Papiery były dosłownie tylko dla
jej oczu, pokryte błoną polaryzacyjną dopasowaną do jej opaski, która
była dopasowana tylko do niej. Mimo tego, nawet siedząc w~sekcji klasy
biznesowej z~przodu odrzutowca, sama prócz stewarda, Myra czuła impuls,
by przygarbić się nad papierami i~zasłonić je nadgarstkami i~łokciami
jak dziecko w~klasie próbujące nie dać ściągać.

Sulejmanow podpisał śmiałą umowę z~MRRNT i~z nią. To była umowa, która
została zaproponowana przez Georgi Dawidowa, który zmarł, zanim mógł z~nią wrócić. Usta Myry zaciskały się, kiedykolwiek o~tym myślała. Jej
podejrzenia zostały wzruszone i~nie chciały być ukojone. Umowy były
naszkicowane w~walizce, która była znaleziona przy jego ciele w~hotelu
pokojowym. Warunki były proste, bezpośrednia oferta unii ekonomicznej i~wojskowego sojuszu. Kazachstan przejąłby szczątkowe zobowiązania
społeczne MRRNT, wchłaniając tych mieszkańców, którzy chcieliby stać się
obywatelami Kazachstanu, dotując resztę. Zapewniłby dla mniejszego
państwa obronę konwencjonalną, pozostawiając Armii Ludowej i~Robotniczej
Milicji te funkcji, do których były rzeczywiście stworzone,
bezpieczeństwo wewnętrzne, patrolowanie granic, głównie ochrona
kosmodromu i~lotniska. W~zamian, rząd Myry zintegrowałby posiadane
bronie w~kosmosie, w~tym atomówki, w~obrębie sił obrony większej
republiki. Zachowaliby ostateczne sterowanie operacyjne -- nawet
Sulejmanow nie mógł oczekiwać od nich poddania tego -- ale dla celów
publicznych, dyplomatycznych i~wojskowych, działaliby razem pod jednym
dowództwem. Za jednym zamachem Kazachstan miałby siły wojskowe
współmierne z~ich obszarem raczej niż populacją.

To nowe mocarstwo mogłoby negocjować pomoc z~Zachodu. Mogłoby stanąć
jako solidny bastion -- możliwe, że nawet klin -- przeciwko Sinosowietom,
czego nie mogły zapewnić rozwijające się reżimy byłego Związku i~podzielonej wojnami Chin. Bronie jądrowe byłyby ich kartą przetargową.
Same w~sobie bezużyteczne -- w~każdej najkrótszej perspektywie -- przeciwko Szinosowom, mogły być dostępne dla USA lub ONZ w~zamian za
wsparcie hardware i~na orbicie i~nawet, na zewnątrz, rozmieszczenie
wojsk, które \textit{mogłoby} powstrzymać nową czerwoną falę.

Myra, jako najstarszy dostępny polityk, z~najdłuższym doświadczenie i~najszerszą wiedzą i~kontaktami na Zachodzie, zainicjowałaby by rozmowy.
W pewien sposób wracała do swojego starego biznesu sprzedawania polis
odstraszania jądrowego. Jedyny różnica była taka, że teraz był tylko
jeden logiczny klient. A ponieważ to byłaby wytężona praca, przy
napiętym harmonogramie, zamierzali dać jej tydzień przerwy, zanim
zacznie, i~bardzo dużo pieniędzy. Miała użyć pieniędzy i~czasu, żeby
znowu stać się młodą.

Odmładzanie było czymś, co powinna zrobić już dawno temu. Teraz, myśląc
nad tym, trudno było dla niej określić powody prokrastynacji. To nie
było to, że nie stać jej było na proces, lub nawet, że była
nieprzyzwoicie uprzywilejowana, wielu jej własnych obywateli i~pracowników robiło wycieczki do zachodnich klinik. Sprytne czarnorynkowe
szczepy stosownego nanoware'u krążyły, gdziekolwiek istniały usługi
medyczne, a~aktualizacje ich niedociągnięć były rozwiniętym i~legalnym
handlem. Jednak Myra nigdy do nich nie dotarła, częściowo, że była
zadowolona z~jej obecnej kondycji, atrakcyjnej na tyle, by przyciągnąć
interesujących i~zainteresowanych mężczyzn, dostatecznie zgrabna do
pracy i~niewymagających zestawów ćwiczeń, ale w~żaden sposób nie
dostatecznie dobra, żeby ktokolwiek był wprowadzony w~błąd, że jest
młoda, kiedy tylko zobaczyliby więcej niż jej twarz lub zbliżenie na
twarz.

Kolejnym aspektem, zrozumiała, była pewna patriotyczna upartość, rodzaju
tego, że ciągle prowadziła swoją starą Skodę Traversera. Nie chciała
kupować młodości od \ldots nie tyle Zachodu co \ldots nowej rasy, pokolenia
post-nano-tech. Chciała radzić sobie z~naprawami, które dotychczas dla
niej działały: szwajcarskie zastrzyki kolagenowe, brytyjskie mikroboty
układu krążenia, gruziński zapasowy układ odpornościowym z~bakteriofagów, wietnamskie fitochemiczne regeneratory nerwowe,
amerykańskie hakowanie telomerów\ldots wszystkie złożone w~postsowieckim
zestawie, które służby zdrowia byłego Związku i~demoludów rozdawały
od dziesięcioleci.

Prezydent Kazachstanu w~ciągu około trzydziestu sekund przekonał ją, że
to było jej osobiste prawo i~obowiązek patriotyczny, by skorzystać z~pełnej roboty, jednego strzału cudownego środka nanotechu na śmierć.
Uwolniona od ciężaru odpowiedzialności za MRRNT, z~misją, na którą nawet
historia może pewnego dnia się uśmiechnąć, że słuszność jakoś uzasadniła
jej samolubną próbę stania się nieśmiertelnym.

Ale ciągle \textit{memento mori}, gdy jej umysł dryfował, słowa się
pojawiały.

\textit{Śmierć za mną podąża\ldots}

Pomyślała, że śmierć mogła ją dopaść na kilka sposobów przez kolejne
kilka godzin. Podróż z~lotniska Izmiru, Adnan Menderes, do Olu Deniz na
wybrzeżu egejskim była przerażająca, nawet w~opancerzonej limuzynie. Nie
były to tylko serpentyny, zatrważające kierowanie samochodem, strome
spadki czy -- po zmroku -- sposób, w~jaki światła huśtały się w~pustą,
czarną przestrzeń. Było to wszystko oraz martwi mężczyźni.

Samochód właśnie z~wysiłkiem pokonał wzniesienie, wyprzedzając kilka
autokarów na centymetry pomiędzy huczącym metalem autobusów z~jednej
strony a~szerokością opony od przepaści po drugiej, i~dwie sekundy na
ucieczkę przed nadjeżdżającą ciężarówką. Na zakręcie, kępa sosen nieco
dalej od głównego lasu. Trzech zakrwawionych mężczyzn wiszących z~gałęzi, za szyję, martwych. Mózg zachował widok szokującego wrażenia
braku: na twarzach, w~kończynach, w~kroczu. Mrugnięcie i~nie
zauważyłaby.

Myra jęknęła. Wzrok kierowcy napotkała jej w~lusterku. Zmarszczki
dookoła jego oczu pogłębiły się w~uśmiechu.

-- Greccy partyzanci.

Zaczął opowiadać jej historię, jak Izmir był kiedyś Smyrną, zanim Kemal
wyzwolił naród, i~był -- tylko trzydzieści pięć lat temu -- znowu 
Smyrną, lotnisko było nazwane po greckim faszyście Grivasie raczej niż
po tureckim demokracie Menderesie, i~jak Grecy zaczęli rekolonizować,
jak powstali Nowi Turcy, żeby wyrzucić helleńskich szowinistyczne pionki
imperializmu, i\ldots i~tak dalej. Myra słuchała uważnie długiej, krętej
opowieści nacjonalistycznych pretensji. Rozpraszało to ją, zajmowało jej
umysł od wszystkich najgorszych atrakcji na poboczach i~większości
wstrzymujących serce zakrętów na drodze. To było miejsce, gdzie małe
wojny były rzeczywiste, bez symulacji i~bez pardonu.

Dlaczego Sulejmanow zarezerwował jej miejsce w~klinice tutaj, z~wszystkich miejsc? Wiedziała, że odpowiedź ma coś wspólnego ze złożoną
dyplomacją reszty jej podróży. Federacja Turecka była jak zwykle w~sporze z~Rosjanami, którzy wspierali Bułgarów, Serbów i~Greków, a~większość reżimów dziedziczących po USA wspierała Turcję, natomiast
burzliwa relacja Kazachstanu z~resztą Byłego Związku była właśnie w~okresie burzy, więc\ldots

Niemniej jednak.

W końcu, w~ciemnościach, ujrzała, że zmierzali długim zjazdem ku
dolinie, która otwierała się na morze. Światła były rozsiane na poboczu
i wzdłuż stoków doliny ze zwiększającą się częstotliwością do grup poza
plażą, które były światłami statków. Gdy droga się wyrównała, kierowca
skręcił w~lewo, potem w~prawo przez wielką żelazną bramę, która się dla
nich otworzyła. Betonowe mury zwieńczone wiązkami drutu ostrzowego,
krótki żwirowany podjazd. Wysiadła i~się rozejrzała. Mogła zobaczyć
basen z~barem i~wielopiętrowe apartamenty. Kierowca podał jej bagaż
kilku chłopakom w~dżinsach i~koszulkach polo. Dała napiwek kierowcy,
zameldowała się, poszła za facetem do jej pokoju, rzuciła jej rzeczy,
dała napiwek chłopakom i~wróciła po schodach i~śliskich płytkach do
baru, gdzie zamówiła Pilsa. Wypiła go w~sekundy. Po klimatyzowanym
wnętrzu samochodu, gorąc był przerażający.

Była w~trakcie trzeciego piwa i~czwartego papierosa, kiedy drobna,
ciemna kobieta w~białym fartuchu laboratoryjnym podeszła do niej.

-- Madame Dawidowa? -- Wyciągnęła dłoń. -- Doktor Selina Masoud.

-- Cześć. Miło mi cię poznać. Szukałaś mnie?

-- Tak. -- Doktor Masoud wycisnęła tabletkę z~dozownika. -- Połknij to.
Popij to \ldots

Myra przełknęła. Doktor Masoud się uśmiechnęła. Miała kręcone włosy i~ładne, białe zęby. 

-- Czymś bezalkoholowym, miałam powiedzieć. Ale w~porządku, po prostu zachce ci się spać, teraz, to wszystko.

-- Dobrze -- powiedziała Myra, zasłaniając ziewnięcie. -- Już jestem
zmęczona. Papieros?

-- Dziękuję. -- Doktor wzięła papierosa i~wyciągnęła złotą zapalniczkę,
wsunęła ją z~powrotem do kieszeni, z~wdzięcznością się zaciągnęła. -- Ach\ldots potrzebowałam tego. -- Usiadła na stołku koło Myry, zamówiła colę.

-- Więc kiedy rozpoczynam kurację? -- spytała Myra.

Doktor Masoud błysnęła brwiami. 

-- To \textit{była} Twoja kuracja -- powiedziała. -- Zostajesz tydzień na wypadek komplikacji, niepożądanych
reakcji. Nie będzie żadnych. Lekka grypa jest normalna.

-- Och -- powiedziała Myra. Wydawało się to jakoś rozczarowujące. -- Więc
co powinnam robić?

-- Relaksować się. Pić dużo, przede wszystkim bezalkoholowo, żeby uniknąć
odwodnienia. Jeżeli chcesz pomóc procesowi, pal i~siedź na słońcu, ile
tylko możesz. To kancerogeny i~też denaturują kolagen, wiesz\ldots

Powiedziała to, jakby przekazywała ostatnie i~kontrowersyjne odkrycie.

-- Tak -- powiedziała Myra. -- I?

-- Katalizują reakcję telomerazy.

Uśmiechnęła się, dopiła colę, zeskoczyła z~wysokiego krzesła. 

-- Muszę iść. Miłego pobytu.

Nagrane na taśmie wezwanie poranne muezzina z~minaretu budziło ją przed
ósmą. Leżała chwilę ciesząc się chłodem pokoju i~szybko narastającym
światłem. Pokój był, w~porównaniu z~jej własnym, orzeźwiająco
niezabałaganiony: pomalowany i~umeblowany w~odcieniach bieli, świeże,
proste linie wystroju i~tkanin poruszały się tu i~tam z~rozstawem oczka
lub pęczkiem koronki, jakby biały nastrój wahał się pomiędzy klinicznym
a ślubnym, niezdecydowany czy oznacza szpital czy hotel. Niezły cel
podróży poślubnej, domyśliła się Myra, zauważyła wiele młodych, głośnych
par w~barze poprzedniego wieczoru, choć nie mogłaby przestać zastanawiać
się, czy konsekwencje bycia ze sobą \textit{na zawsze} mogą nie uderzyć
zbyt mocno, zbyt wcześnie, w~takim miejscu jak to.

Przy basenie siedziała na leżaku i~wcierała krem do opalania na ramiona
i tułów. Jej ręce były jak pazury (ale giętkie), jej mięśnie żylaste
(ale mocne), jej skóra marmurkowa (ale napięta) tak jak były w~ciągu
ostatnich czterdziestu lat.

Po jej lewej, za głównym budynkiem kliniki, grunt się wznosił jako
pogórze rolnicze do wysokiego, bezużytecznego klifu. Po drugiej strony
około kilometra do dna doliny, był niższy klif, który był zarośnięty
krzewami i~drzewami. Ponad głową, niebo było głęboko niebieskie.
Paralotniarze, ich spadochrony w~formie jasnokolorowych paznokci
dryfowali, z~odległego pasma daleko za wysokim klifem do plaży dwa
kilometry dalej. Cykady warczały jak małe elektryczne urządzenia. Reszta
ludzi tutaj wydawała się albo młoda, otrzymując swój fix, albo stara jak
ona, przyjmując swoje odmłodzenie.

Przez dwa dni było to wspaniałe. Słońce wznosiło się nad klifem po
lewej, zachodziło za klifem po prawej, regularnie jak w~zegarku.
Wieczorami puste klify wyglądały na czerwone i~marsjańskie, a~klinika
jak kolonia na Księżycu, nieco sztuczne środowisko, nad którym nurkowały
zaprzeczające grawitacji paralotnie. Myra spędzała dnie w~słońcu,
pływając i~nie umierając. Było to lepsze niż niebo. Obróciła się i~pozwoliła słońcu przypiec jej plecy.

Wielkie, gołe nogi stanęły przed jej twarzą, w~rozszerzającej się plamie
wody na betonowych płytkach. Jej wzrok przesunął się po owłosionych,
brązowych nogach, mokrych wyciągniętych szortach, owłosionej, brązowej
piersi, do twarzy. Dziobaty nos, jasne brązowe oczy, ciemne,
czerwono-brązowe zakręcone włosy odrzucone do tyłu. Mężczyzna uśmiechnął
się do niej, skinął głową nieświadom siebie.

-- Myra Godwin?

-- Tak? -- Jak w, co to Ciebie właściwie obchodzi.

Przykucnął. Wielkie, białe, nierówne zęby.

-- Jason Nikolaides -- przedstawił się. -- Zostałem poproszony, żeby z~Tobą
porozmawiać.

Poczuła się nieco zdezorientowana.

-- Jesteś Grekiem?

Roześmiał się. 

-- Och nie. Nie od pokoleń. Amerykanin. -- Lekko się
ukłonił. Krople wody spadły z~jego włosów. -- CIA. Mamy kilka spraw do
omówienia.

Myra przewróciła się, spuściła nogi, usiadła prosto. Wygrzebała
papierosa. Spojrzała na niego, oczy zmrużone od słońca i~dymu.
Westchnęła.

-- Minęło dużo czasu -- powiedziała.

\chapter{Śpiew Sierpu}

Spojrzałem wstecz na drzwi pubu, pokręciłem głową, ruszyłem wzdłuż boku
placu i~skręciłem za rogiem na ulicę, gdzie kwaterowałem. Poszedłem na
kwaterę, wbiegłem na górę i~rzuciłem torbę, potem na dół i~znowu na
zewnątrz.

Bez namysłu, skręciłem w~prawo, w~przeciwnym kierunku od stacji i~placu.
Przeszedłem kładkę nad torami i~poszedłem wzdłuż drogi z~miasta, koło
zalewisk rzeki Carron i~południowego brzegu Carron Loch. Linia kolejowa
była po mojej prawej, pomiędzy drogą a~morzem. Słońce obniżało się
przede mną, ale jeszcze nie świeciło prosto w~oczy. Na lewo wznosiły się
zalesione wzgórza. Przeszedłem koło przysiółka i~glen Attadale, dalej
obok i~poniżej stoku Carn nan Iomairean.

Szedłem około pięciu kilometrów, zanim się zatrzymałem, przeszedłem tory
i usiadłem na skale na brzegu przy Immer. Przypływ był wysoki, a~jezioro
górskie spokojne. Mogłem czysto usłyszeć skrzypka grającego na jakiejś
biesiadzie w~lasach Strome Carronach. Wzgórza toridońskie, ich skały
starsze niż życie, starsze niż światło z~widzialnych gwiazd, wyłaniały
się czarne za wzgórzami Strome.

Przez całą drogą nikogo nie spotkałem, minęło mnie kilka pojazdów. Cały
krajobraz wydawał się zamykać mnie i~przypominać mi, że byłem tutaj
obcym, wyłączonym ze wszystkiego, prócz straszliwej miłości Boga.
Kilkaset metrów dalej, mężczyzna z~kosą pracował w~wysokiej trawie łąki,
tak jak jego przodkowie to robili, i~jego następcy, bez wątpienia, będą
robić. Merrial wyrecytowała, w~sobotę na wzgórzach, fragment rymowanki
majsterków, która znaczyła więcej dla niej niż dla mnie:

\begin{verse}
\textit{Młot dzwonił w~fabryce}\\
\textit{Sierp w~polu śpiewał}\\
\textit{Rolnik robił opornie}\\
\textit{Sierp młota usłuchał}\footnote{ ,,Fall 1991'', Kena Macleoda, tłum.
własne -- przyp.tłum.}\\
\end{verse}

Żaden młot, żadna fabryka nie zatrzymała kosy tego mężczyzny. Jego
rytmiczne zamachy cięły trawę, jak gdyby wieki się nie wydarzyły.

Potem mężczyzna ostrożnie odłożył ją na bok i~wskoczył na siedzenie
traktora, pierdnięcie silnika metanowego wystraszyło ptaki, gdy opuścił
prasę i~zaczął grabić siano.

Roześmiałem się z~siebie, wstałem i~ruszyłem z~powrotem do miasta.

Wyszła, jak powiedziała mi barmanka, krótko po naszej kłótni.
Podziękowałem dziewczynie, unikałem kolegów i~ruszyłem do miasta
majsterków.

-- Nie ma jej tutaj.

Odwróciłem się od bezowocnego stukania w~białe drzwi Merrial. Chłopczyk
w szortach i~koszulce, obie zbyt duże na niego, patrzył na mnie poważnie
ze ścieżki. Podszedłem.

-- Wiesz, gdzie poszła?

Był bardzo czysty, o~ile mogłem zobaczyć w~słabym świetle, prócz
czerwonej i~najwidoczniej lepkiej plamy na policzku, pokrytej puchem.
Powstrzymałem chęć naplucia na palec i~wytarcia.

-- Nie mogę powiedzieć -- powiedział mi z~bezinteresowną przebiegłością.

-- Cóż, a~możesz mnie zaprowadzić do kogoś, kto może?

Gdy pokręcił głową, uświadomiłem sobie skrzypienie żwiru dookoła mnie i~zrozumiałem, że nie muszę szukać daleko. Kilkunastu majsterków, młodych
i starych, kobiet i~mężczyzn, wydawało się dryfować znikąd. Zebrali się
w luźnym półkręgu wokół mnie, żaden bliżej niż trzy metry. Niektóre z~twarzy widziałem przy wcześniejszych wizytach w~obozie, inni byli
całkowicie mi obcy. Wszyscy byli ubranie w~tę mieszankę prostoty i~sprytu, którą zaczynałem rozpoznawać jako osobliwość stroju majsterków.
Wyglądało to, jakby reszta z~nas nosiła odrzucone ozdoby jakieś upadłej
arystokracji, podczas gdy majsterkowie sami wykrajali swoje eleganckie
tkaniny.

-- Szukam Merrial -- powiedziałem, dość odważnie. W~ciszy mój głos brzmiał
zaskakująco i~cienko jak skowronka na polu.

-- Aye, wiemy to -- powiedział młody mężczyzna. -- Ale nie znajdziesz jej
tutaj.

-- I~to wiem -- odparowałem. -- Więc gdzie mogą ją znaleźć?

Wzruszył ramionami. Ktoś zachichotał. W~końcu, i~jakby z~sympatią,
starszy mężczyzna dodał: 

-- To zależy od niej. Jeżeli nie chce, żebyś ją
znalazł, to znaczy nie dla naszej pomocy. Jeżeli chce, znajdziesz ją
wkrótce.

-- Więc wiecie, gdzie ona jest? -- Brzmiałem, nawet dla siebie, żałośnie
pełen nadziei. Jedyną odpowiedzią było więcej wzruszeń ramionami i~chichot.

-- Jest ktoś jeszcze, kogo chciałbym spotkać -- powiedziałem. -- Fergal.

-- Och -- powiedział starszy mężczyzna z~udawanym zdziwieniem -- jest wielu
mężczyzn o~takim imieniu. Nie wiesz może jak miał na nazwisko, prawda?

-- Wiecie cholernie dobrze, kogo mam na myśli -- powiedziałem. -- Przekażcie mu, że chcę się z~nim spotkać.

Wszyscy zrobili krok bliżej. Półokrąg stał się ciasno upakowaną podkową
ludzi, którzy zaczęli się poruszać tak, że otwarty koniec był skierowany na
drogę. Nigdy nie myślałem o~majsterkach jako zastraszających jednego z~tubylców -- częściej to działało w~drugą stronę -- ale poczułem się wtedy
zastraszony, może z~powodu większej ich liczby. Zdecydowałem się ustąpić
z łaski, gdy mogłem, raczej niż dać im powód do ukrytego -- lub może
wyobrażonego -- zagrożenia. Zatem utrzymywałem dystans, gdy ciągle
napierali.

-- Ach, najlepiej jakbyś sobie poszedł -- powiedział młody mężczyzna.

-- Tak sądzę -- powiedział. -- Dobrej nocy wam wszystkim.

Odwróciłem się na pięcie i~odmaszerowałem z~taką godnością, na jaką mnie
było stać. Kamień odbił się od wybrukowanej ulicy, gdy do niej dotarłem,
ale się nie odwróciłem, ani nie przyśpieszyłem. Wewnętrznie gotowałem
się ze wstydu z~bycia, dwa razy jednego wieczoru, pokonanym przez
majsterków. Byłem jednakże zdeterminowany, żeby nikomu spośród moich
przyjaciół i~znajomych o~tym nie mówić, nie z~powodu swojego
zażenowania, ale ponieważ mogliby poczuć się zobowiązani do jakiegoś
wspólnego zastraszania.

Nie była to ruchliwa noc na placu, a~ja nie czułem chęci spotkania ludzi
i rozmawiania. W~rzeczywistości pragnąłem pić w~samotności. Kupiłem
butelkę whisky w~Karonadzie, za markę, i~wypadłem bez przywitania innego
niż gestem.

W moim pokoju odkryłem kopertę wepchniętą pod moje drzwi. Zawierała
telegram, który rozłożyłem i~przeczytałem przy rumianym świetle zachodu
przy oknie.

CLOVIS PRZEZ CATHERINE FARFARER UL GŁÓWNA CARRON STOP BARDZO ZANIEPOK
ZAGIN DOK NAKAZ ZWROTU POCZTĄ DO JUTRA WTOREK INACZEJ RĘCE ZWIĄZANE MOŻL
DYSCYPLIN ORAZ BLISKIE ŚLEDZTWO STOP POZDR GANTRY.

Spacerując wybrzeżem, doszedłem do wniosku, że byłem głupcem,
zostawiając Merrial, niezależnie od prowokacji, a~teraz czułem się nawet
bardziej gorzko. Ostrzegała mnie od początku, że kochanie jej nie zawsze
by mnie uszczęśliwiło, i~miała rację. Informacja, że mogła być
członkiem tajnego stowarzyszenia, sprawiła, że jej odmowa zaufania była
bardziej zrozumiała, nawet jak podstawy tego stowarzyszenia napełniały
mnie przerażeniem. Moja historyczna erudycja nie wyprowadziła mnie z~błędu popularnych poglądów: że komuniści, na swój niezdarny, cholerny
sposób, dużo zrobili, walcząc z~Posiadaniem, ale że ostateczne
zwycięstwo nie było ich, a~my mogliśmy dziękować Opatrzności, że nie
było więcej komunistów na Ziemi. Nie mogłem się zmusić do uwierzenia, że
Merrial naprawdę, w~sercu, popierała tę diabelską wiarę.

Nie więcej niż Wyzwolicielka. Może Merrial, a~nawet inni majsterkowie w~społeczeństwie, używali rytuałów i~fraz dla swoich własnych potrzeb, tak
jak Wyzwolicielka wykorzystała je do założenia swojej republiki.

Na tę radośniejszą myśl, wypiłem kieliszek lub dwa i~zasnąłem na łóżku.

Następnego ranka Catherine Farfarer, właścicielka, podała mi dwa
telegramy. Pierwszy był od Podkomisji Dyscyplinarnej Senatu
Uniwersytetu, zawieszającej moje członkostwo w~Uniwersytecie \textit{sine
die}\footnote{ łac. na czas nieokreślony -- przyp.tłum.}, wycofującej
wszystkie prawa i~przywileje, prócz reprezentacji przed Sądem
Uniwersytetu, tuż przed rozpoczęciem roku akademickiego. Drugi był od
Gantry'ego, wyrażającego swoją sympatię i~mówiącego, że ODWOŁ OD SKANDAL
DECYZJI.

I to było skandaliczne, w~rzeczywistości byłem ukarany przed rozprawą,
ponieważ moje szanse na sponsora lub mecenasa przestały istnieć. Nawet
gdybym się oczyścił, straciłbym przynajmniej część pierwszego roku
mojego projektu, co równie dobrze oznaczało stratę całego.
Przetelegrafowałem do Gantry'ego, dziękując mu. Ale nie miałem wielkiej
nadziei, że mógłby mi pomóc, lub że ja, z~moimi upartym milczeniem,
zasługiwałem na to.

Nie byłem zaskoczony, że Merrial nie było w~pracy. Przetrwałem przez
większość mojego niebezpiecznego dnia w~ciemności rozświetlanej łukiem
elektrycznym w~nodze platformy bez wypadku i~właśnie czyściłem swoje
narzędzia (i wszystkich innych) kwadrans po czwartej, kiedy pojawił się
Angus Grizzlyback z~niedoświetlonego rusztowania i~usiadł na skrzynce.

-- Clovis -- powiedział. Spojrzałem w~górę. Podrapał się po głowie jedną
dłonią i~odwrócił wzrok ode mnie na mały kawałek papieru, który trzymał
w dłoni.

-- Coś nie tak?

Nawet wtedy, pierwszą myślą było, że był niechętnym posłańcem złych
wieści o~moich rodzicach, lub podobnej sprawie rodzinnej.

-- Tak, obawiam się, że tak -- powiedział. -- Muszę cię zwolnić. Spłacić.

-- Za co? -- spytałem, jednocześnie z~ulgą i~wstrząsem.

-- Nic, co tutaj robiłeś -- zapewnił mnie. -- To jest wbrew moim chęciom,
Clovis. Z~tego wszystkiego, co tutaj pracowałeś, nie jesteś zły w~tym,
co robisz, i~jesteś uczciwy, ale\ldots -- Wzruszył ramionami i~spojrzał
znowu na papier. -- To Towarzystwo. Wycofali Twoje zezwolenie na pracę
przy tym projekcie. -- Spojrzał na mnie bystro, pytanie w~oczach. -- Jakieś kłopoty, w~jakie wpadłeś na Uniwersytecie.

Odłożyłem narzędzia na stół i~zacisnąłem tłuste ręce na mojej głowie. 

-- Jak oni mogli to zrobić? -- spytałem, ale znałem odpowiedź. Uniwersytet
wskazał mnie Towarzystwu, którego był, oczywiście, częścią, jako
ryzyko dla bezpieczeństwa projektu. To wszystko miało sens, choć
wydawało się niesprawiedliwe.

-- Wiesz, możesz się odwołać -- powiedział Angus. -- Poprę Cię.

Przełknąłem żółć. 

-- Dzięki -- powiedziałem. -- Będę o~tym pamiętał.
Oczywiście się odwołam.

Jedynym powodem, dla którego myślałem o~odwołaniu, było to, że brak
takiego wyglądałby jak przyznanie się do winy, a~w~istocie byłem winny
wielu rzeczy, z~których żadnych nie chciałbym przedstawiać przed sądem
pracy. Pewny jednak byłem, że nic, co zrobiłem, nie mogłoby zagrozić
projektowi, inni mogliby nie potraktować bycia szalenie zakochanym w~obcym jako dobrej podstawy do skazania.

-- Ach, dobrze, uruchomię maszynerię -- powiedział Angus. -- Powiem Jondo i~włączy w~to Związek. -- Zmusił się do uśmiechu. -- Szybko wrócisz.

-- Dzięki, Angus -- powiedziałem.

-- Ale teraz -- kontynuował -- muszę cię poprosić o~natychmiastowe wyjście.
Tu pisze, że powinienem odeskortować cię do granicy, ale nie zrobię
tego.

Byłem bardzo wdzięczny, że zaufał mi aż do bramy, ale gdy odwróciłem się
i spojrzałem na przebytą drogę w~stoczni, zauważyłem jego małą postać na
zewnątrz platformy i~zrozumiałem, że dyskretnie obserwował każdy mój
krok.

Wsiadłem do wcześniejszego i~prawie pustego autobusu do Carron Town i~wróciłem do pokoju. Butelka whisky, w~tej chwili, wydawała się jedynym
moim przyjacielem. Rano byłoby to fałszem. Mielibyśmy ciężkie rozstanie,
ale oboje wiedzieliśmy, że była to kwestia czasu, póki byśmy się nie
pogodzili. Wiedziałem to wszystko doskonale dobrze, gdy siadałem pod
świetlikiem i~nalewałem sobie hojną miarkę alkoholu. Pokrzepiający ogień
wpadł w~moje nerwy i~mogłem rozważać moje rozpadające się życie z~pewnym
dystansem.

Myślałem o~tym, co straciłem, i~czego nie straciłem, i~ustaliłem, że to,
co mi zostało, było wystarczające, żebym odzyskał resztę, jeżeli tylko
wymyśliłbym sposób. Więc, zamiast urządzać jakieś smutne, samotne picie,
umyłem się, ogoliłem, zmieniłem ubrania i~poszedłem do Karonady.

Drzwi pubu, ciężkie od szkła i~mosiądzu, zamknęły się za mną. Po słońcu,
światło wydawało się słabe. Gdy wszedłem do baru, moje oczy się
dostosowały. W~tym czasie, około wpół do szóstej, pub był prawie pusty.
Barmanką była ta sama dziewczyna, która nas obsługiwała w~poniedziałkowy
wieczór. Była lokalną dziewczyną, wysoką i~szczupłą, z~długimi, jasnymi,
spiętymi włosami, mocnymi ramionami od pompowania. Jej imię, które
poznałem po kilku minutach rozmowy, gdy pochyliłem się leniwie nad
barem, sącząc pół litra jasnego, było Jeanna Berrymead. Dorastała na
farmie nieco w~górę glen, w~Achnashellach.

Carron Town, zanim zaczął się projekt, było miejscem, gdzie każdy
wiedział wszystko o~każdym, lub przynajmniej rozmawiał jak gdyby
wiedział. Wiedza Jeanny o~moim spotkaniu, i~rozstaniu, z~Merrial była
wystarczająco szczegółowa, co sugerowało, że lokalne plotki szybko
łapały nowości.

-- Ten majsterek, który był tutaj \ldots -- powiedziałem, próbując skierować
ją od oczywistego badania mojej strony historii.

-- Och, aye, Fergal.

-- Znasz go?

Wzruszyła ramionami i~zrobiła minę. 

-- Widuję. Wpada niekiedy. Trochę arogancki gnojek, ale stawia czasem kolejkę.

-- Wiesz, gdzie pracuje?

-- Aye, w~starej elektrowni w~Lochluichart. To już nie jest elektrownia,
wiesz. Ale ludzie ciągle tak ją nazywają.

-- Więc co to jest teraz?

Skrzywiła się. 

-- Miejsce, gdzie nie chciałbyś iść. Mówią, że tam
majsterki robią swoje kryształy widzenia. Słyszałam, że to jakby \ldots
nawiedzone. Przerażające miejsce. Ale, nigdy nie spotkałam nikogo, kto
tam był. Albo chciał pójść -- dodała dosadnie.

-- Znaczy, nikogo kto był majsterkiem -- powiedziałem. -- Przypuszczalnie
Fergal wspomniał, że tam był.

Potrząsnęła głową, marszcząc brwi. 

-- Nigdy nie powiedział słowa, nawet
kiedy jest pijany. Nie, żeby był często pijany! Ten to potrafi się
napić.

-- Więc skąd wiesz, gdzie pracuje?

-- Ach, nie wiem -- powiedział, jakby zniecierpliwiona zejściem z~tematu.
-- To tylko, wiesz, co mówią ludzie.

Miałem właśnie spróbować wyciągnąć więcej z~niej, kiedy kolejny głos
dołączył do rozmowy.

-- Już jesteś na podrywie, Clovis, tak szybko po kłótni z~ostatnią
dziewczyną? -- Mój kolega z~pracy Druin brzmiał rozbawieniem. Odwróciłem
się i~uśmiechnąłem się do niego, gdy barmanka nalewała pół litra. Druin
był tubylcem, żonaty i~w swoich latach trzydziestych, jego kamizelka
odkrywała gołe brązowe ramiona ciągle poplamione olejem od pracy i~pozabliźniane od lat pracy wcześniej.

-- To wcale nie to -- powiedziałem. -- Myślałem o~tym lepiej, kto by nie
myślał? Jednak jej nigdzie nie ma. Więc próbuję dowiedzieć się więcej o~majsterkach.

Roześmiał się. 

-- Jesteś niezły. Czytanie sprawia, że czujesz się śmiesznie
w głowie. -- Powiedział to nie jako obrazę, ale dobroczynne wyjaśnienie.
-- Ale -- dodał -- to jest dziewczyna, której sam bym nie rzucił.

Poprosiłem Jeannę o~kolejne pół litra i, zauważając kusząco tanią
butelkę, powiedziałem: 

-- Och, i~po kieliszku Taliskera, proszę.

Druin uniósł szklankę. 

-- Dzięki, chłopie. -- Łyknął Taliskera i~spytał -- O co chodzi ze Twoim zwolnieniem?

-- Kłopoty na Uniwersytecie -- powiedziałem. -- Pożyczyłem dokumenty i~okazało się, że nie mam żadnego wyboru, prócz oddania ich Fergalowi. MTN
wydaje się, że uznało to za znak, że nie można mi ufać. Uważam to za
obrazę.

-- Może i~tak. -- Spojrzał na mnie z~ciekawością. -- Jednak nie wydajesz
się tym bardzo przejęty.

Skrzywiłem usta, odwróciłem dłoń. 

-- Aye, przejąłem się, ale nie ma sensu
pozwalać czemuś takiego mocno trafić. Odwołam się, Jondo też się ma za
to wziąć. Jakoś to się wyjaśni. Bardziej jestem zmartwiony, dlaczego
Merrial nie ma w~pracy.

-- Ach -- powiedział. -- Nie bierze dnia wolnego, czy zawieszenia, czy coś
takiego. Rozwiązała swoją umowę.

-- Skąd o~tym wiesz?

Dotknął boku nosa. 

-- Jondo mi powiedział, ponieważ oczywiście zapytał
się administracji, czy też została wyrzucona z~pracy.

Westchnąłem. 

-- Zdaje się, że to ulga, w~pewien sposób. Ale nic mi nie
mówiła o~tym, nawet wcześniej.

Druin skinął głową. 

-- Aye, to małomówna grupa, majsterki. Zatem, czego
chcesz się o~nich dowiedzieć?

-- Cóż, jakby przyjmujemy ich za pewnik, racja? Niektórzy ludzie robią
jakiś rodzaj pracy i~nikt inny nie wie o~tym za dużo. Jak to się
zaczęło? Dlaczego wszyscy nie mogą podążać drogą światła? W~ogóle, w~jaki sposób ludzie stają się majsterkami?

Druin spojrzał na Jeannę, potem na swojego drinka. Podrapał policzek.
Jenna niewytłumaczalnie lekko się zarumieniła i~uniosła dłoń,
zasłaniając chichot.

-- To bardzo dużo pytań -- powiedział Druin. -- Odpowiadając na ostatnie
jako pierwsze, większość ludzi, którzy stali się majsterkami, urodzili
się w~nich. Są majsterkami, ponieważ ich rodzice byli majsterkami.

-- Aye -- powiedziałem -- ale spójrz na majsterków. Nie są ludem wsobnym,
choć mogą być czymkolwiek innym. Więc muszą mieć nowych rekrutów, że tak
powiem, ale nigdy o~takich nie słyszałem.

Chichot Jeanny się przedarł. Odwróciła się i~poszła na drugi koniec
baru. Druin spojrzał za nią i~na mnie, uśmiechając się głupawo.

-- Cóż -- powiedział ostrożnie -- jest taka plotka, że ci z~osiadłych,
którzy stali się majsterkami, zrobili to poprzez stosunek seksualny. -- Roześmiał się na minę mojej twarzy. -- Zdaje się, mógłbyś być w~trakcie
procesu stawania się takim.

-- Och, no \textit{weź} -- powiedziałem. -- To śmieszne.

Druin pokręcił głową. 

-- To nie jest śmieszne -- powiedział stanowczo. -- Pomyśl o~tym. Majsterek osiadając, przestanie być majsterkiem, a~cholernie mało tak robi. Więc jeżeli chcesz być z~majsterką, sam musisz
stać się majsterkiem. I~wywędrować, nigdy nie być ponownie ujrzanym.
Majsterki nie zostają w~jednym miejscu dłużej niż rok lub dwa lata,
jeżeli już.

-- Dobra -- powiedziałem -- rozumiem, że coś w~tym może być. -- Mój umysł
obracał to w~różnych następstwach, żadne, z~których byłem w~nastroju
podzielić się z~Druinem. -- Co z~pozostałymi pytaniami?

Wzruszył ramionami. 

-- Co do tego, dlaczego oni i~tylko oni robią to, co
robią? Myślałem nad tym i~jedyną rzeczą, jaką mogę powiedzieć, to,
wywodzi się to z~czasów Wyzwolicielki i~dobrze działa. Co więcej możesz
powiedzieć?

-- Och, dużo -- powiedziałem. -- Jak czy to jest najlepszy sposób na
robienie rzeczy.

-- Aye, cóż, jak powiedziałem. To działa. -- Pochylił się bliżej. -- Oto
kawałek żargonu majsterków, jaki podłapałem: ,,Jeżeli nie jest zepsute,
nie naprawiaj''. Rozsądna rada, niezależnie skąd pochodzi.

Wysuszył kubek i~wychylił whisky, potem uśmiechnął się i~klepnął mnie w~ramię. 

-- Widzę, że dałem ci dużo do myślenia, ale nie mam czasu więcej
gadać. Odpadam. Do domu, do żony i~herbaty, potem na wzgórza z~karabinem.

Gdy zsuwał się ze stołka i~prostował, rzucił mi sprytne spojrzenie i~spytał: 

-- Może masz ochotę się przejść, Clovis?

-- Polowanie na sarny? -- Nagle brzmiało to jak coś, co desperacko
potrzebowałem, żeby rozjaśnić sobie w~głowie. Moje pierwsze badania już
dały mi znacznie za dużo nowych informacji do przyjęcia.

-- Pewnie -- powiedziałem. -- Dzięki.

-- Wspaniale, dobrze, chodź zatem i~na herbatę.

-- Och, nie mógłbym, Twoja żona nie oczekuje nikogo\ldots

-- Ach, człowieku, jeżeli zobaczyłbyś, jak bardzo próbuje mnie nakarmić,
poszedłbyś z~czystej sympatii. Co Ty, będziesz mile widziany.

-- Dzięki wielkie. Na razie, Jeanna.

~

Żona Druina nazywała się Arrianne. Spokojna, solidna, czarna kobieta,
która przyjęła moje przybycie całkowicie bez zaskoczenia. Usiedliśmy
dookoła ciężkiego stołu w~salonie, pod głośno tykającym starożytnym
zegarem, z~dwojgiem dzieci: chłopcem około czternastu lat nazywanego
Hamishem, już pracującym na farmie rybnej i~sześcioletniej dziewczynie
imieniem Ailey, która beztrosko pomagała matce podać obiad.

Obiad -- lub ,,herbata'' jak to nazywali -- składał się ze świeżej
makreli, skałoczepów ugotowanych w~słonej wodzie, młodych ziemniaków,
marchewki i~świeżej fasolki. Musiałem przestać przy trzeciej porcji
jedzenia, ale Druin i~Hamish jedli dalej. Ten rodzaj odżywiania wydawał
się nie dodawać grama tłuszczu na żadnych z~nich. Arriane nalegała, że
wyglądałem na niedożywionego i~mogła mieć rację.

Kiedy kobieta i~dziewczynka sprzątnęły talerze, Druin wstał i~z szacunkiem podniósł dwa karabiny z~półki na ścianie. Podał mi jednego
nad stołem.

-- Wiesz jak się z~tym obchodzić?

Powtarzalny, czterotaktowy, luneta. Zademonstrowałem znajomość i~bezpieczeństwo ku satysfakcji Druina.

-- Ma cholernego kopa -- ostrzegł, podając mi sześć naboi. -- Jednak, nie
będziesz miał szansy na więcej niż jeden strzał, nawet jak będziemy mieć
szczęście.

Pożegnał się, ja podziękowałem całej rodzinie, potem wyprowadził mnie do
tyłu i~z boku domu, gdzie była zaparkowana jego półciężarówka.
Położyliśmy karabiny z~tyłu i~wspięliśmy się do kabiny. Siedzenia były
skórzane, deska rozdzielcza drewniana, stal nierdzewna, wszystko
wypolerowane z~miłością.

-- Silnik fuzyjny -- powiedział z~dumą, gdy przekręcił kluczyk i~w odpowiedzi natychmiast otrzymał niskie brzęczenie. -- Osiemdziesiąt lat i~nic złego się nie dzieje. Był w~rodzinie tak długo. Żadne te wasze
metanolowe czy metanowe smrody dla nas.

Pojazd zamruczał na głównej ulicy i~na drodze koło New Kelso. Druin
przyłapał mnie na wykrzywianiu szyi, by obejrzeć osiedle majsterków i~się roześmiał.

-- Ach, znajdziesz ją -- powiedział.

Na skrzyżowaniu skręcił w~prawo, w~górę jeziora. Wieczorna fala ruchu
się zmniejszyła i~posuwaliśmy się szybko z~prędkością czterdziestu
kilometrów na godzinę.

-- Gdzie się kierujemy? -- spytałem, gdy zwolnił na głównej ulicy
Achnashellach. Małe stado krów rasy Highland było prowadzone przez
miasto, Bóg wie, z~jakich powodów.

-- Ach, zobaczysz, gdy tam dotrzemy. -- Spojrzał na mnie z~boku. -- Możesz
palić, jak chcesz, tylko upewnij się, że popiół idzie za okno, a~niedopałek do popielniczki. -- Nacisnął klakson. -- Ach, ruszcie swoje
jebane dupy -- poradził owłosionej bestii, która wyglądała na niego,
jakby go usłyszała, podrzuciła rogami i~człapała nieświadomie przed nami
przez następne kilka minut.

Po minięciu przeszkody, przyśpieszył na długiej, powoli wznoszącej się
drodze do Achnasheen, które przejechaliśmy dwadzieścia minut później.
Ulice miasta wznosiły się wysoko na zalesione wzgórza i~pomiędzy
szklarnie na dnie doliny.

-- W~dniach mojego pradziadka, wszystko to była pieprzonym bagnem, według
tego, co mówi -- zauważył Druin. -- Stacja, hotel i~wszystko kurwa
dookoła. Aye, odebraliśmy ziemię i~bez pomyłki, jak powiedział Brahan
Widzący.

-- Kto?

-- Och, taki prorok ze starych czasów, powiedział, że ludzie wrócą do
górskich dolin. Nostradamus Północy! -- Roześmiał się. -- Mówią, że
patrzył na przyszłość przez dziurę w~kamieniu, i~ten właśnie kamień jest
gdzieś na dnie jeziora.

-- Kryształ widzenia?

Druin zarechotał. 

-- Masz majsterków w~mózgu, Clovis! Widzący żył i~umarł
dawno temu, jeszcze przed komputerami. Których nie przewidział. Nie, to
był zwykły mały kamień z~dziurą, przez którą patrzył.

-- Wierzysz w~to?

-- Nie sądzę, żeby było coś specjalnego w~kamieniu -- powiedział Druin. -- Ale może było coś specjalnego w~oku lub mózgu za nim.

-- Jasnowidzenie? -- powiedziałem sceptycznie.

-- Nie wiem o~tym -- powiedział Druin. -- Brahan Widzący zobaczył
przyszłość w~wyobraźni, tak jak i~my wszyscy. -- Zachichotał. -- Był po
prostu lepszy w~tym niż większość.

Druin zatrzymał się w~małym miejscu zwanym Dark i, zostawiając samochód
zaparkowany na poboczu, poprowadził mnie przez sosny na lewo.

-- Żadnego palenia -- powiedział cicho. -- I~teraz bez gadania.

Skinąłem głową, koncentrując się na dźwiganiu siebie i~coraz cięższego
karabinu w~górę stoku. Gruba warstwa igieł spowalniała, tyle o~ile,
cichy postęp. Miałem nieco problemów z~dotrzymaniem kroku Druinowi i~zdecydowałem wtedy i~tam, że palenie jest naprawdę niezdrowe. W~tym
samym czasie, czułem napięcie, które tylko dym mógł złagodzić. Coś w~manierze Druina i~coś w~naszej pozycji męczyło mnie, ale nie wiedziałem
co. Stale się wspinaliśmy, coraz dalej od drogi i~w górę wzgórza.

Druin dotarł do szczytu pasma przede mną i~tam się zatrzymał, ręce na
jednym kolanie, gdy nadrabiałem. Wskazał w~dół przez przerwę w~drzewach,
gdzie druga strona grzbietu opadała z~powrotem na drogę. Patrząc w~dół,
mogłem dojrzeć drogę, linię kolejową i~długie, wąskie jezioro górskie.

Loch Luichart. Rozpoznałem miejsce z~nagłym wstrząśnięciem
przypomnienia, że to było gdzie -- jak mówiła mi Jeanna -- Fergal pracował
i majsterkowie robili swoje dziwne kamienne komputery. Stara
elektrownia, na którą wskazywał Druin, była wielkim, ciemnym,
klockowatym budynkiem u stóp stoku poniżej nas.

-- O co chodzi? -- spytałem Druina, tak cicho, jak mogłem.

Uśmiechnął się do mnie i~zaczął powoli iść w~górę grzbietu.

-- Pomyślałem, że mógłbyś chcieć zapolować na coś więcej niż łanię -- powiedział. -- Szukasz tego faceta Fergala i~Twojej dziewczyny Merrial.
Tam na dole może być dobre miejsce do szukania.

Sapnąłem, ale nie z~wysiłku dotrzymywania mu kroku. 

-- Nie możemy po
prostu tam wmaszerować!

-- Dlaczego nie? -- chrząknął. -- Poza tym, nie ,,wmaszerujemy'' tam po
prostu. -- Zatrzymał, zrobił kilka kroków na prawo w~kępę krzaków. -- Ach,
jest tutaj.

Doszedł do cylindrycznej struktury zniszczonej, pokrytej pnączami,
ceramicznej, około metra wysokiej i~na metr szerokiej. Gdy podszedłem,
wskoczył na górę i~zaczął zeskrobywać przerośnięte rośliny bokiem
podeszwy. Po chwili odsłonił się zardzewiały właz.

Nie na tyle zardzewiały, żeby jednak go nie otworzyć. Spojrzałem do
środka i~zobaczyłem serię szczebli znikających w~ciemnościach. Druin
zrzucił kamyk i~nadstawił ucha.

-- To tylko około dwudziestu metrów -- powiedział mi.

-- Wielkie nieba, człowieku, chyba nie mówisz o~schodzeniu tam, prawda?

-- Aye, tak mówię -- powiedział. -- Jest to wystarczająco bezpieczne, jak
długo się trzymasz.

-- Ale wiesz, co jest na dnie? -- Spojrzałem na niego podejrzliwie. -- A
poza tym, skąd o~tym wiesz?

Druin westchnął teatralnie. 

-- To, co jest na dnie, to tunel, nie wiem,
czy był częścią oryginalnej hydroelektrowni, czy czymś, co zostało
dodane potem. To całe wzgórze było pokopane i~pocięte tunelami. Było
używane jako baza podziemna przez Armię Brytyjską, i~przez Republikanów
w trakcie wojny domowej przed Pierwszą Światową Rewolucją, zdaje się, że
zmieniło kilka razy właściciela. A co do tego, skąd o~tym wiem \ldots -- Roześmiał się. -- Schemat i~mapa tego wszystkiego są w~muzeum w~Jeantown!
Uważasz, wydaje mi się, że majsterki zmienili schemat, w~ten sposób lub
inny.

-- Brzmi całkiem mroczno -- powiedziałem.

-- Ach, na dole będzie jakiś rodzaj światła. I~mam latarkę.

-- Czy miałeś to na myśli cały czas?

-- Aye -- przyznał. -- Ale nie chciałem ci mówić wcześniej, na wypadek,
gdybyś spanikował, martwiąc się tym, zanim w~ogóle tutaj dotarliśmy. Tak
jak jest, zaczynam się właściwie zastanawiać, czy miałem rację, myśląc,
że masz w~sobie ducha awanturnika. Przez ostatnie pięć minut nic nie
robiłeś, tylko krytykowałeś. Czy chcesz starać się o~tę kobietę, czy
nie?

-- Oczywiście, że chcę -- powiedziałem, nakręcony do działania, jak bez
wątpienia umyślił, jego aluzją o~tchórzostwie. Przewiesiłem karabin
przez plecy, wdrapałem się i~postawiłem stopę na szczeblu, gdy zsunąłem
się do środka. -- Też idziesz?

-- Będę tuż nad Tobą -- powiedział Druin.

Przez następne kilka minut, koncentrowałem się całkowicie na schodzeniu
po drabinie. Szczeble nie wyglądały na zardzewiałe, tak jak śruby, w~rzeczywistości metal i~ceramika szybu były mi nieznane. Jednak nie
mogłem być pewny, że każdy szczebel przetrwał wieki, więc testowałem
każdy, zanim przyłożyłem mój pełny ciężar. Zawieszony karabin tylko
zwiększał niezręczność. Jedno spojrzenie w~górę potwierdziło, że Druin
schodził za mną. Ponad nim właz był widoczny jako mała, jasna dziura.

Po, co wydawało się długim czasem, moja stopa natrafiła na puste
miejsce, tam, gdzie powinien być szczebel. Po chwili przestrachu
opuściłem stopę niżej, ostrożnie i~dotknąłem podłogi. Chrząknąłem z~ulgi, zszedłem i~odsunąłem się od drabiny, ciągle uważając, gdzie
stawiam stopy. Druin skończył schodzenie chwilę potem i~staliśmy razem w~ciemności i~milczeniu.

W trakcie schodzenia moje oczy zaadaptowały się do słabego światła i~nawet tutaj, na dnie szybu, nie było całkowicie ciemno. Byłem świadom,
bez do końca wiedzy dlaczego, że byliśmy w~rzeczywistości w~tunelu i~że
tunel był dość mocno nachylony. Rozglądając się dookoła, widziałem
jaśniejszą przestrzeń niżej. Popatrzyłem na Druina i~wskazałem w~tym
kierunku. Jasny owal jego twarzy zrobił kołyszący ruch, który
zinterpretowałem jako skinienie. Razem poszliśmy w~dół pochyłości.

Po kilku krokach uderzyłem się palcem w~coś twardego. 

-- Cholera -- wymamrotałem, podskakując. Druin wpadł na moje plecy i~obaj
zakołysaliśmy się niebezpiecznie.

-- Jebać tę zabawę w~żołnierzy -- powiedział Druin. Odpiął latarkę od pasa
i włączył ją. Mocna wiązka białego światła rozświetliła tunel przed
nami. Pokazała, że podłoga była zaśmiecona przeszkodami, dziwnie
ukształtowanymi kryształami widzenia w~różnych wielkościach. Również
pokazała, że tunel był pełen ludzi.

Druin jęknął przekleństwo i~zadziwiająco gładkim i~płynnym ruchem zdjął
karabin. Wiązka latarki w~ogóle się nie zachwiała. Ciągle byłem nieruchomy
od szoku, natychmiast, gdy zebrałem się w~sobie, spojrzałem ponad
ramieniem i~ujrzałem więcej postaci zbierających się za nami,
niewyraźnych w~odbiciu światła latarki. Jedna taka postać była
najwyraźniej w~trakcie sięgania po mnie, uderzyłem mocno w~ramię i~prawie upadłem, ponieważ moja pięść przeszła przez nią. Druin obrócił
się dookoła w~tym samym momencie i~światło rzuciło mój cień groteskowo
na postaci przede mną. Nie zareagowały ani na cień, ani na światło.
Druin wypuścił oddech w~jednym porywie, potem się roześmiał.

-- One są puste, człowieku!

-- Ach. -- Patrzyłem na nie w~zdumieniu. -- Aye, takie jak majsterki
straszą dzieciaki na jarmarkach.

-- Dokładnie tak. Boże, ale mnie przestraszyli.

-- Nic dziwnego, że Jeanna mówiła, że to miejsce jest nawiedzone.

-- Tak mówiła, czy nie? -- dumał Druin. -- Będę musiał kiedyś pogadać z~tą
dziewczyną. Jednak. Ruszajmy. I~trochę ściszmy głos.

Żaden z~nas w~ogóle nie mówił głośno, ale najmniejszy dźwięk wydawał się
wzmocniony przez akustykę tunelu. Odwróciliśmy się i~poszliśmy, światło
z latarki Druina umożliwiało nam uniknąć kamieni na podłodze i~prawie
zignorować zjawy, które były przez nie rzucane. Prawie, ponieważ
nieruchome twarze mężczyzn i~kobiet przedstawione w~tych nienamacalnych
posągach były uchwycone w~momencie udręki i~alarmu, które, gdy ciągle
wyłaniały się z~mroku i~nas mijały -- lub mijały przez nas -- były
wystarczające, by pobudzić, przynajmniej we mnie, pełznące poczucie
niepokoju. Wyglądali niesamowicie jak zagubione dusze, przeklęci z~zabobonów chrześcijańskich i~mahometańskich i~wymagało to silniejszej
wiary w~Rozum niż moja, żeby iść tą mroczną ścieżką nieporuszonym.
Nieracjonalne jak mogłoby się wydawać, pocieszałem się faktem -- znanym
każdemu dziecku dostatecznie dużemu, żeby nie bać się ,,namiotu duchów''
na jarmarku -- że ,,pustaki'' nie istnieją poza światłem i~zatem, że nie
było niewidzialnego tłumu w~ciemnościach za nami.

Obecnie minęliśmy ich niesamowite towarzystwo i~byliśmy bliżej źródła
światła na końcu tunelu (wyrażenie, którego pełną siłę po raz pierwszy
doceniłem). Powietrze pachniało wilgocią i~w tym samym czasie świeżej.
Dotarliśmy do końca nachylenia, skalista podłoga tunelu była płaska.
Druin wyłączył latarkę i~postępowaliśmy bardzo wolno i~cicho, przez
kolejne kilka metrów. Powodem niejasności światła okazał się być ostry
zakręt w~tunelu. Zakradliśmy się dookoła, trzymając się dalszego boku
łuku, karabiny w~garści (choć nie, jak przypomniałem sobie w~tamtym
momencie, załadowane).

Szturchnąłem Druin i, wyjmując nabój z~kieszeni, spróbowałem włożyć go
do karabinu. Pokręcił głową, stanowczo, a~ja odstąpiłem, uspokajając
siebie myślą, że pistolety przy naszych paskach były gotowe do
natychmiastowego użycia. Obeszliśmy łuk i~spoglądaliśmy na jasno
oświetloną wielką przestrzeń, przynajmniej dwadzieścia metrów w~poprzek,
jak mi się wydawało, i~dziesięć w~górę. Światło pochodziło z~paneli nad
głowami i~wydawało się światłem słonecznym. Ściany zakrzywiały się ku
sufitowi, kamienne, zatem grota i~nie taka naturalna. Jej długość nie
była jasna z~miejsca gdzie staliśmy, w~jednym z~rogów.

Jaskinia zawierała rzędy za rzędami kamiennych koryt, połączonych
schodkami otwartych rur, przez które kapały strumyki wody. Niektóre były
zorganizowane do zasilania koryt, inne do odprowadzania ścieków, lub tak
zgadywałem, z~faktu, że żaden kanał, który wychodził z~koryta, nie
wchodził do kolejnego. Mogłem naliczyć kilka osób tam pracujących,
poruszających się od koryta do koryta, dokonujących niewykrywalnych
zmian w~nurcie lub dodających jakiś proszkowy materiał. Wyglądali jak
hydroponiczni ogrodnicy i~pomyślałem na pierwszy rzut oka, że wykonywali
to znajome rzemiosło, może dla jakieś tajemnego składnika żywności
majsterków. Potem zauważyłem zawartość koryt dalej na prawo i, gdy
szybko zrozumiałem, bardziej dojrzałych. Hodowali kryształy widzenia,
mogłem rozróżnić większe z~nich ustawione w~linii, pięć w~korycie.

-- Cóż, dobra -- powiedział Druin, jakby myśląc, tak jak ja: więc
\textit{tak} one są robione! Zawiesił karabin na ramieniu, spojrzał na
mnie i~wzruszył ramionami.

-- Nie ma sensu się teraz skradać -- powiedział.

Potem wymaszerował odważnie w~światło.


\chapter{Zapomnij o~Babilonie}

Wyszli z~kostnicy, schylając się pod łukami i~przez wybite otwory w~ścianach, do kościoła. Poniżej dziobatych, oszpeconych murali
Ortodoksów, turecka kobieta sprzedawał jadeit, srebro i~szydełka.
Zignorowali jej gest zachęty, wyszli na zewnątrz, przeszli koło kilku
stoisk. Po drugiej strony zagłębienia na szczycie budynku, gdzie stał
kościół, zbocze pełne ulic i~pustych, kamiennych domów bez dachów,
walczyło z~powolną zieloną entropią brzozy i~jeżyn. Światło było
oślepiające, gorąc dławiący, cisza intensywna. Cykady ją łamały, ptaki,
skoki jaszczurek.

Jason obszedł front kościoła, odnalazł datę w~kolorowych kamieniach na
bruku.

-- 1912 -- powiedział. -- To wtedy go skończyli. Jak dumni musieli być.
Dziesięć lat później, odeszli. Dobrowolna wymiana ludności, ha.

Myra kucnęła w~słońcu, łyknęła wody, zaciągnęła się papierosem. 

-- Gorsze rzeczy się od tego czasu wydarzyły. -- Suche, starożytne kości żeber i~ud
w kostnicy nie przeszkadzały jej tak bardzo, jak świeże ciała, które
widziała pierwszego wieczoru.

-- Bez wątpienia. -- Jason wzruszył ramionami. -- Ale wiesz, to miejsce,
sprawia, że czuję się jak Grek, po raz pierwszy w~moim życiu. Nawet
cholerny \textit{chrześcijanin}. -- Spojrzał na kramarzy kilkadziesiąt
metrów dalej, skulił się koło niej i~powiedział niskim, szczerym głosem.
-- Jak, wiesz, zachodni. To inna kultura. Nie są tacy jak my.

Myra patrzyła na niego zszokowana. Karmilassos, lub Kaya, lub Kayakoi,
lub jakkolwiek była nazywana (Turcy bezwstydnie nazywali ją ,,wioską
greckich duchów'') też na nią wpłynęła, ale agent CIA wydawał się
wyciągać z~tego kompletnie zły morał.

-- Tak właśnie działa nacjonalizm -- powiedziała. -- I~takie właśnie
myślenie. Nie, dziękuję. Nie przyjmuję tego.

Jason wyglądał jakoś na zranionego. Przechylił kapelusz do tyłu i~zaczął
przygotowywać skręta. Jego wiek -- jak twierdził, i~mu wierzyła, choć kto
mógłby być teraz pewny? -- to dwadzieścia cztery lata. Ostatnim razem,
kiedy była poważnie zaczepiana przez CIA, było to ponad sześćdziesiąt
lat wcześniej. Było coś niesamowitego w~człowieku, który kontynuował
akta tyle starsze niż on sam.

~

Ostatnim razem: mężczyzna z~Agencji rozmawiał z~nią nad latte w~Starbuck's przy Harvard Square, w~lipcu 1998, kiedy nagłaśniała pomoc
medyczną dla ofiar opadu w~Kazachstanie. Dziecko na plakacie kampanii
miało rozszczep podniebienia. Chirurg, którego poznała, ustawił kontakt,
ktoś, kto pracował w~konsulacie w~Ałma-Acie, powiedział, ale się nie
oszukiwała. Przyniosła magnetofon, dyskretnie w~kieszonce bluzki.
Oczekiwała kogoś, kto wygląda jak mormon, Mężczyzna w~Czerni. Był młody,
ciemny, bystry. Ciemnoniebieski podkoszulka, spodnie bojówki. Nazywał
siebie Mike.

Rozmawiali o~Wielkiej Brytanii. Mike był zainteresowany Ulsterem.
Oranżyści maszerowali w~Drumcree. Myra nie powiedziała mu niczego, czego
nie wiedział. Wiedział więcej o~niej, niż ona wiedziała, od niechcenia podane
nazwy demonstracji, na których była w~latach siedemdziesiątych, gdy
leniwie przewracał wiadomości zagraniczne w~\textit{Boston Globe}. Zabrali
swoje kawy na zewnątrz, usiedli na niskim murze, podczas gdy Myra
paliła.

Mike skinął na zaciśniętą czarną pięść na spłowiałym muralu ,,black
power'' wysoko na murze po drugiej stronie ulicy, ponad sklepem z~mapami
na rogu. 

-- Wszystko to skończone -- powiedział. -- Koniec z~kłótniami o~polityce, Myra. Wszystkie układy są nowe, teraz. Nie prosimy, żebyś
kogokolwiek zdradzała, czy cokolwiek. Tylko dzieliła się informacjami.
Mamy wspólne cele. Udajesz się w~końcu w~niebezpieczne miejsca. -- (Ach
oto i~było, groźba). -- Nigdy nie wiesz, kiedy właściwe znajomości mogą
być ważne.

-- W~rzeczy samej -- powiedziała. Patrzyła w~roztargnieniu na nastolatkę z~różowymi włosami, pewna, że już ją wcześniej widziała. Potrząsnęła
głową. -- Będę to mieć na uwadze -- powiedziała. -- Tu mój numer komórki.

Mike dał jej swój i~poszedł. Tej nocy Myra przekazała taśmę całej
rozmowy przez biuro do jednej z~lokalnych sekcji Czwartej, i~reporterowi
\textit{Mother Jones}\footnote{ amerykański dwutygodnik liberalno-progresywny,
zob.~\url{https://en.wikipedia.org/wiki/Mother\_Jones\_(magazine)}
-- przyp.tłum.}. Reporter był niezdecydowany, lokalny aktyw, po krótkiej
panicznej konsultacji, kazał jej współpracować.

Dwa tygodnie później była w~Nowym Jorku i~spotkała znowu Mike'a
opierającego się o~poręcz na promie Staten Island. Ostatnia przeprawa
dnia, który był wilgotny, a~teraz mglisty. Codzienni pasażerowie
siedzieli na ławkach, turyści pozowali do zdjęć siebie ze Statuą
Wolności lub wieżami Manhattanu, \textit{aparat} kapitału, wyłaniający się
z tyłu. Zgodziła się nawiązać kontakt z~konsulatem, kiedy wróci. A w~kolejnych latach, tak zrobiła, od czasu do czasu, gdy ona i~Georgi
wspinali się w~strukturach postsowieckiego Kazachstanu, przez rewolucje
i kontrrewolucje. Głównie informowała o~ludziach, którzy byli tak samo
jej wrogami, jak byli dla CIA. Przemytnicy narkotyków, ludzi, broni,
skorumpowani handlarze koncesji minerałów i~grabieży zasobów. Mówiła
Czwartej o~każdym takim spotkaniu i~nic jej nie odpowiedzieli, w~końcu
to wszystko zblakło. Po Jesiennej Rewolucji archiwa zostały otwarte.
Myra leniwie przeszukała po swoim nazwisku lub nazwach kodowych o~odkryła, że większość osobników i~firm, które podała do CIA, już
pracowała dla CIA.

Jednak ciągle traktowali ją jako aktywną, bękarty, po wszystkich tych
latach i~zmianach.

A dziewczyna z~różowymi włosami też była na promie do Staten Island.
Nigdy tego nie zrozumiała i~w końcu złożyła to na karb przypadku.

~

Jason podał jej skręta i~palili razem, gdy wędrowali w~dół po stromej,
kamiennej ścieżce przez porzucona sady oliwne do podnóża wzgórza, gdzie
zostawili wynajętego jeepa. Obskurne małe osady były spójne z~nowo
wybudowanymi domami z~betonu i~kilkoma odebranymi kamiennymi domami na
pierwszej ulicy dawno opuszczonego greckiego miasta. Wszystkie były
wypatroszone lata temu, tureckie rodziny, które tam żyły, zostały zabite
przez greckich partyzantów w~ostatniej wojnie. Niebieskie i~białe
ceramiczne oczy -- na powodzenie, przeciwko złemu oku -- nad drzwiami były
popękane, belki ściemniałe. Myra ugasiła niedopałek w~popiołach
drzewnych, które ciągle były głębokie na kilka centymetrów. Nie czuła
się zjarana, tylko skupiona, jej wzrok rozszerzony jak przy nakładce VR.
Mogła \textit{zobaczyć} dlaczego ta ziemia była warta walki.

Jason wsiadł na miejsce kierowcy, gdy Myra wsiadała z~drugiej strony.
Patrzył na nią współczująco jakby na wpół z~przykrością za sprowadzenie
jej tutaj.

-- Czasem Bóg jest sprawiedliwy -- powiedział.

-- Ta. W~bardzo starotestamentowy sposób.

Jason uruchomił silnik i~zawrócił dżipa na wąską drogę do Hisaronu.
Droga się wspinała, ocierając się o~drzewa, krawędzią o~urwiska. Sosna,
skała i~suche wąwozy, było jak w~gorący dzień w~Szkocji. Myra pamiętała
dzień z~Davidem Reidem, przy rzece pomiędzy Dunkeld i~Blair Atholl,
który wyglądał prawie tak samo. Rozmawiali o~depopulacji i~wymuszonej
migracji jak również w~pojęciach biblijnych, jak sobie przypominała.

-- \textit{Mene, mene, tekel, upharsin} -- usłyszała siebie mówiącą.

-- Co?

-- Ta rzecz z~Biblii. Wiesz, o~królu Babilonii? ,,jesteś zważony na wadze
i znaleziony lekkim''\footnote{ cyt. z~księgi Daniel 5,27 Biblia Brytyjska -- przyp.tłum.}.

-- Jestem świadom źródła -- powiedział Jason, patrząc się na drogę. -- To
związek z~tematem mi umyka.

-- Tak się czuję -- powiedziała Myra. Wystawiła dłoń na wiatr nad szybą,
czując chłodny pęd pomiędzy palcami.

-- Tak właśnie myślisz o~sobie? To źle.

-- Nie -- odpowiedziała mu. -- O jebanym \textit{świecie}.

-- To gorzej.

Roześmiała się, jej nastrój się poprawił.

-- Tak czy inaczej -- kontynuował Jason -- to tylko wynik odmłodzenia.
Ludzie tak mają.

-- Wiedziałbyś, co?

-- Nie osobiście. Dla mnie, to tylko stabilizowanie, prawda? Dla
Ciebie\ldots -- parsknął, patrząc z~boku na nią -- \ldots to ma znacznie
\textit{więcej} pracy.

-- Dzięki.

-- Powoduje, że czujesz się dziwnie. Euforycznie i~osądzając.

-- Tak, czyli wszystko w~porządku!

Był piąty dzień, od kiedy połknęła nanochirurgów. Nanomaszyny zróżnicowały
się i~namnożyły w~niej, rozpościerając się przez jej układ krążenia jak
armia saperów, zrywając i~odbudowując. Czuła ich ciepło odpadowe jak
gorączkę, wypalającą ją. Jej nastrój zmieniał się od normalnego do
wysokiego, nie miała już depresji, to było jak biologiczny Keynesizm,
prócz tego, że na dłuższą metę miała nie umrzeć. Nie była nieśmiertelna,
nie naprawdę, ale kto potrafiłby to określić? Najlepsze oceny mówiły o~wiekach, a~w~takim czasie coś innego mogłoby się pojawić, ale czuła się
nieśmiertelna, czuła się jak ludzie w~wieku dwudziestu lat, zanim ich
komórki zaczną się wyczerpywać, a~neurony umierać, nic dziwnego, że
pamiętała lata siedemdziesiąte tak żywo, nic dziwnego, że stawała się
tak arogancka!

Seks z~Jasonem był przesądzoną konsekwencją, gdzieś od sekundy, w~której
go zobaczyła. Był agentem imperialistów, strategicznym wrogiem, nawet
jeśli taktycznym sojusznikiem, ale nie dbała o~to, chciała go uwieść,
przekabacić, pokazać triki wyuczone przez całe życie, które zacisnęłyby
jego palce i~posiwiały jego ciemnomiedziane włosy. Jeżeli miał jakieś
zahamowania lub wstręt wobec jej ciągle starego ciała, zostały
rozpuszczone w~pierwszej butelce rakiji pierwszego wieczoru. Ssała go
sztywnego, wyjebała go surowego, nauczyła go dużo i~powiedziała mu
niewiele.

Niewiele tego, co mu powiedziała, było o~Georgim i~okolicznościach jego
śmierci. Z~powodów, których Jason nie wypowiedział, ale które Myra
podejrzewała, że miały napisane na marginesie ,,Aktywa Agencji, możliwe
wykorzystanie?'', CIA prowadziło swoje własne śledztwo tej śmierci,
która była tak zdecydowanie wygodna dla \textit{kogoś}.

We wczesnych godzinach ranków, kiedy myślał, że ona śpi, wychodziłby na
mały balkon pokoju i~rozmawiał długo przez telefon. Udawała, że nie
zauważa i~nie protestowała, zamiast tego używała tych okresów w~mamrotanej pogaduszce do poduszki, używając opaski do konsultowania
Parvusa i~słuchania maili od jej kolegów z~SowNarKomu o~sytuacji w~domu.
Nie była dobra.

Denis Gubanow, w~szczególności, był ponury. Jego podsumowania nastrojów
społecznych -- na podstawie raportów agentów i~listów czytelników
\textit{Prawdy Kapicy} -- wskazywały na to, co dla Myry było zaskakującą
narastającą falą sprzeciwu wobec całej umowy z~Kazachstanem. Wszystko
niezauważone, grube krzaki patriotyzmu wyrosły przez lata na chudej,
bezpłodnej glebie jej małej republiki. Niepodległość stała się ważna dla
obywateli, znacznie bardziej niż kiedykolwiek była dla niej. Każdej nocy
patrzyła na zdjęcia codziennie rosnących pikiet na zewnątrz budynków
rządowych: czerwone flagi, ,,żółte i~czarne'' flagi, zdjęcia Trockiego.
Westchnęła, odwróciła się i~udawała, że śpi, kiedy Jason wrócił.

W Hisaronu, przyjemnym, małym miasteczku rozrzuconym na wzgórzach
otoczonych wyższymi, dalszymi górami, zatrzymali się przy kawiarni na
głównej ulicy. Wypili Amstela i~zjedli kebab Iskander, pod paskowaną
plastikową markizą. Kiedy palili i~sączyli mętną kawę, Myra nachyliła
się nad stołem i~zacisnęła dłoń na dłoni Jasona, pozwalając ich palcom
się zapleść.

-- Czego chcesz ode mnie? -- spytała.

Zacisnął swoją dłoń.

-- Oprócz tego, co mam?

-- Tak.

Rozplątał palce, wyciągnął z~kieszeni i~rozwinął mapę świata w~odwzorowaniu Mercatora wytartej na zagięciach. Odsunął łokciem drinka i~butelkę ketchupu i~rozłożył mapę na metalowym stoliku.

Wskazała. 

-- Jesteśmy tu. -- Otrzepała ręce i~poruszyła się, jakby chciała
wstać. -- Cieszę się, że mogłam pomóc.

-- Siadaj -- powiedział, śmiejąc się. -- Spójrz.

Usiadła. 

-- Kto jeszcze patrzy? Jeżeli chcesz zrobić odprawę, czy VR nie
byłaby lepsza?

Jason machnął dłonią i~się rozejrzał. Turyści, żołnierze i~tubylcy
wędrowali po ulicy. 

-- Nikt nie patrzy. -- Przeczesał palcami włosy. -- I~jak mogłaś zauważyć, nie mam opaski. -- Wzruszył ramionami. -- I~tak
wszystkie sieci są zagrożone, były od lat. Dlatego słucham radia, czytam
gazety, piszę w~notatniku i~noszę papierowe gazety.

-- Słusznie -- powiedziała Myra, lekko, żeby ukryć zimny szok co do tego,
co właśnie powiedział. Jednak uznała, że nie może tego puścić. -- Co masz
na myśli, ,,zagrożone''?

-- Niezabezpieczone, niezależnie od tego, co robisz. Kody, ukrywające
prawdziwą wiadomości w~śmieciach, cokolwiek, są systemy, które złamią
każdy nowy wariant, gdy tylko go wystawisz. Obliczenia kwantowe zabiły
kryptografię, teraz są lepsze metody niż to, uruchomione na rzeczach,
których \textit{nikt} nie rozumie. Są tam, Myra. Widziałem je.

Uśmiechnęła się sceptycznie. 

-- Rzeczy, których człowiek nie miał
poznać? 

Jason pokiwał energicznie głową. 

-- Tak, dokładnie tak -- powiedział, jak
gdyby nigdy nie słyszał wyrażenia wcześniej. Może nie słyszał. Młodzi
dzisiaj. Spojrzał znowu na mapę, odrzucając temat poruszeniem dłoni.
Myra nie drążyła dalej, ale nie odrzuciła tego. Była prawie pewna, że
był w~błędzie, lub kłamał, lub został okłamany. A w~czyim interesie
mógłby być brak zaufania do jej programów? Ha.

Jason dźgnął palcem Północną Amerykę, przesunął dookoła Wielkich Jezior
i częściowo w~dół po Wschodnim Wybrzeżu. 

-- Ok, oto mój kraj, był Twoim.
Stany Zjednoczone, jak ciągle się nazywamy. Już nie do końca ,,od morza
do morza''. ,,Od St Lawrence do Keys'' nigdy nie złapało i~nawet to było
trudno utrzymać. Mam na myśli, potrzebujemy Maine pomiędzy nami a~kanadyjskimi hordami, ale, cholera, powstrzymujemy wielkie powstania
wszędzie pomiędzy Baltimore i~Jacksonville. A jedynym powodem, dla
którego trzymamy Florydę, jest Canaveral, szczerze, i~jedynym powodem,
dla którego się z~nami trzymają, jest to, że się boją \textit{El Barbudo}.
-- Spojrzał spode brwi, skrzywił się w~uśmiechu. -- Powinnaś posłuchać
chłopców w~Langley jak się za to kopią. Gdy komisja Pike'a powstrzymała
zabawy z~eksplodującymi cygarami, pomyśleli, kurde, gnojek musi kiedyś
umrzeć. Nie.

Rozłożył palce jak dzielniki i~rozkraczył je na kontynencie. 

-- Zachodnie Wybrzeże\ldots -- Westchnął. -- La-la Land\footnote{ zwyczajowa nazwa Los Angeles -- przyp.tłum.}. Mają rywalizujące roszczenie, by być jedynym dziedzicem
USA, więc dyplomatycznie się nie dogadujemy, ale pomiędzy Tobą, mną i~\textit{garson} tutaj\ldots -- w~roztargnieniu machnął dłonią, strzelił
palcami, wskazał na ich szklanki -- \ldots jesteśmy najlepszymi
przyjaciółmi. -- Położył dłoń na środku Ameryki, zakrywając większą część
pomiędzy Appalachami i~Górami Skalistymi. -- W~porównaniu z~tym, jak
dogadujemy się z~resztą. Mormoni, milicje, fundamentaliści, Biała
Prawica, Indianie, wymień je, straciliśmy wobec nich.

-- Tak, dobra -- powiedziała Myra. -- Słyszałam.

-- Szczęśliwie dla nas -- kontynuował -- nieco się gryzą z~naukowcami. Mają
ropę i~kopaliny, oczywiście, ale nie znajdą wiele więcej używając
Geologii Potopu. To nie nauka o~rakietach. Wspominając o~tym, my i~nasi
przyjaciele z~La-la mają wszystkich ekspertów od aeronautyki, nauki i~techniki jądrowej. Przynajmniej, mamy tych, którzy nie zginęli, próbując
przekonać jakiegoś chłopskiego inkwizytora z~generatorem i~przewodami
rozruchowymi, że naprawdę, naprawdę nie wiedzą, gdzie są pochowane ciała
kosmitów. Lub gdzie leżą rozbite latające spodki.

-- Żartujesz.

-- Chciałbym. Okazało się, że więcej ludzi wierzy w~przykrywkę o~UFO, niż
kiedykolwiek wierzyło w~żydowskich bankierów. Kiedy dostali w~swoje ręce
rzeczywistego \textit{djabelskiego rzondowego naukowca}\ldots możesz sobie
wyobrazić, jaką mieli zabawę. -- Patrzył w~dal, przez nią, przez chwilę.
-- Niektórzy z~naukowców się przyznali. Z~zadziwiającymi szczegółami.
Nazwiska, daty, miejsca, pliki od A do Z.

Dzieciak obsługujący stoliki postawił kolejne butelki. Myra uśmiechnęła
się do niego, wsunęła mu kilka banknotów gigalirowych, pomachała
papierosem na Jasona.

-- Czy cokolwiek z~tego jest prawdziwe? -- Roześmiała się niespokojnie. -- Czasem się zastanawiam jak na przykład diamentowe statki\ldots

Jason mrugnął, pokręcił głową. 

-- Och, nie. Całkowicie skoroborowana
halucynacja. Jak porwania obcych lub sabaty czarownic. \textit{Oni} też
słyszeli historie, prawda? Cholera, może niektórzy nawet w~nie wierzyli,
kto może powiedzieć. Diamentowe statki, e tam, to była tylko czarna
technologia z~dawna. Twój podstawowy latający spodek nazistów. Zgrabny
pomysł co do zasady, ale nigdy nie był praktyczny, aż właściwe materiały
pojawiły się w~węglowych asemblerach.

Myra odchyliła się, napełniając szklankę, marząc, o~możliwości
skonsultowania Parvusa. 

-- Mówisz mi -- powiedziała -- że Wschodnia Ameryka
ma też problemy z~ochroną granicy? Cóż, uspokoję cię. Nie zamierzam ci
sprawiać kłopotu pytaniami o~\textit{oddziały lądowe}. Lub nawet
teleżołnierzy.

-- Boże, gdyby tylko o~to chodziło\ldots -- Jason znowu się gdzieś wpatrywał.
-- Nie, to nieco bardziej skomplikowane. Potem jedziesz do Ankary, tak?

-- Co?

-- Zamierzasz poprosić Turków o~wojska lądowe.

-- Nie wiem, skąd taki pomysł -- powiedziała Myra, ostrożnie nie
zaprzeczając temu. Ankara nie była w~ogóle w~jej planie podróży, ale
była bardzo ciekawa, dlaczego Jason myślał, że była i~co go w~związku z~tym dręczyło.

-- Źródła. -- powiedział Jason. -- Tak czy inaczej, dlatego tutaj jestem,
żeby ci powiedzieć, że byłby to bardzo zły pomysł. Jeżeli chciałabyś
otrzymać jakąkolwiek pomoc od Stanów, znaczy.

-- Hmmm -- powiedziała Myra. Spojrzała na żołnierza przeglądającego
stragan z~pamiątkami kilka metrów dalej. -- Właśnie patrzę na mundur
wyprodukowany w~USA, pancerz KevlarPlus z~USA, hełm US Robotics z~AI
Raytheonu, karabinek Colt-14 z~USA\ldots

-- Ta, tak, tak -- powiedział Jason niecierpliwie. -- Cenieni klienci.
Starzy przyjaciele. Nie znaczy, że bylibyśmy szczęśliwi patrzeniem na
ich standardowe buty US Army tupiące po całej centralnej Azji.

-- Nawet, żeby deptały po Szinosowach?

Jason pochylił się na łokciach, złożył palce przed twarzą, żeby
zamaskować usta i~powiedział cicho.

-- Spójrz, Myro, to nie są dni chwały komunizmu. Mam nam myśli, w~\textit{naszych} dniach chwały walilibyśmy w~nich z~B-52 całą dobę,
niezależnie co by to dało. Rozumiem, że wasi, ach, braterscy sojusznicy
próbowali tego na swój niepowtarzalny sposób z~Antonowami. Zostałem
upoważniony, żeby przekazać ci, oczywiście poza protokołem i~zaprzeczając, że jeżeli przyjedziesz do Nowego Jorku lub Waszyngtonu,
będziesz mile widziana, i~Twoje prośby zostaną wysłuchane z~sympatią.
Niemniej jednak. Nasza ocena zagrożenia Szinosowów, skąd kurwa się
wzięła ta nazwa, jest całkiem stonowana. Jeżeli zmotoryzowana horda
Mongołów w~plastikowych jurtach chce planować swoją ekonomię komputerami
napędzanymi parą, to ich problem, a~jeżeli okaże się być lubiana w~Twoim
kraju, to Twój.

Myra patrzyła na niego, bujając się. 

-- Jezu. To mi powiedziałeś.

-- Hej, nic osobistego. Musiałem to być ja lub ktoś jak ja, kto ci powie
to, ponieważ na poziomie, na którym będziesz działać w~Nowym Jorku lub
DC byłoby to\ldots niedyplomatyczne, lub niepolityczne, żeby przedstawiać
to tak dosadnie. Nie mówię, że nic nie dostaniesz. Dostaniesz, tylko,
może nie tyle ile byś chciała.

Zmrużyła oczy, pochylając się znów do przodu. Patrzył tak bezpośrednio,
tak uczciwie. Nie mógł wiedzieć o~nuklearnej karcie w~rękawie.

-- Ok, ok -- powiedział, jakby niezbyt przejęta, czym nie była. -- Zatem,
jesteście bardziej zmartwieni rozwojem Federacji Tureckiej niż o~Związkiem Chińsko-Radzieckim?

-- Rozumiesz. I, cóż, są jeszcze większe problemy niż to. Próba zamachu
miała, powiedzmy, że nie ułatwiła nam spraw.

-- Jak to?

Jason zacisnął usta. 

-- Dowiesz się -- powiedział ponuro.

-- Dobrze -- powiedziała Myra. Zawirowała piwem, spojrzała do niego, nie
znalazła wskazówek. Spojrzała w~górę i~uśmiechnęła się do Jasona. -- Nic
osobistego, słuszna uwaga. Więc wróćmy do osobistych.

Jason nagle się rozluźnił. 

-- Tak, ok.

-- A przy okazji, to z~gaelickiego.

-- Co?

-- Nazwa, Szinosowy. Myślę, że ukuł ją David Reid.

-- Cóż, co też nie powiesz.

-- To, co chcę wiedzieć -- powiedziała Myra, dopijając szklankę i~wstając
-- to, ta sprawa o~nich mających komputery napędzane parą?

-- Ach -- powiedział Jason, gdy wrócili do jeepa -- mogę ci wszystko o~tym
powiedzieć.

-- Nie powinieneś prowadzić?

-- Ach, chyba nie. -- Jason przełączył jeepa na autopilota, i~gdy zabrał
ich w~długą drogę do Olu Deniz, powiedział jej wszystko o~dziwnych
maszynach Szinosowów.

~

Dziwna to była maszyna, która zabrała ją do Ameryki.

Ostatniego poranka obudziła się przed Jasonem, leżała przez chwilę,
potem sięgnęła automatycznie po kontakty. Była w~momencie nakładania
jednorazowych, gdy zauważyła, że widzi wszystko ostro, dookoła pokoju.
Szybkie spojrzenie za okno potwierdziło, że nie była już krótkowzroczna.
Podniosła dłoń na pięć centymetrów od twarzy i~nadal pozostała ostra.
Nie miała też dalekowzroczności.

Pod prysznicem spojrzała na swoje ciało, ale oprócz widzenia ostro
swoich palców nie mogła określić różnicy. Wycierając potem głowę,
odkryła luźny włos w~dłoni. Wpatrzyła się w~niego.

-- Jason, popatrz, popatrz na to!

-- Co? -- Usiadł, spojrzał na nią, obejrzał włos.

-- Wygląda na\ldots włos.

-- Nie, spójrz na \textit{koniec}. Nie, \textit{drugi} koniec.

-- Mam coś zobaczyć?

Czy się obudził? Potrząsnęła znowu go za ramię.

-- Tutaj jest pół centymetra blond! Nie szare!

-- Och, Jezus. Wierzę Ci na słowo.

-- Ha -- powiedziała. -- Najwidoczniej fix nie zrobił nic z~\textit{Twoimi}
oczami. Sprawdziłabym je, na Twoim miejscu.

-- Tak czy inaczej, wystarczają na drogę.

Pomógł jej załadować bagaż do dżipa, znikając uprzejmie -- prawdopodobnie
kolejna, ukradkowa rozmowa telefoniczna -- gdy pociła się na ostatnim
badaniu u doktor Masoud, i~czekał za kierownicą, gdy wyszła z~kliniki i~usiadła koło niego.

-- Wszystko gotowe?

-- Tak. Wszystko jasne.

-- Witamy w~wieczności -- powiedział, odpalając silnik i~obracając dżipa
na podjeździe z~rozpryskiem żwiru.

-- Tylko nie \textit{wysyłaj} mnie tam!

-- Ach, będzie ok -- powiedział Jason, skręcając w~prawo na drogę na
wzgórza, ku Fetiye. Wspinali i~wspinali, wyprzedzając taksówki,
ciężarówki i~busiki, będąc ostrożnie uprzejmymi wobec wozów
opancerzonych. Farmy w~dolinie i~kramy przy drodze były prawie wszystkie
obsługiwane przez zadziwiająco starych ludzi, którzy wyglądali, jakby
mieli podstawowe odmłodzenia metaboliczne, ale nie mogli kupić tych
kosmetycznych. Zamiast bycia małymi i~zgarbionymi, byli wysocy i~prości,
ale ich twarze wyglądały jak maski z~Beninu, czarne i~zmarszczone, z~bystrymi, błyszczącymi oczami.

Zatem, jak zauważył Jason, bez zmiany.

Minęli wzniesienie i~Myra mogła znowu zobaczyć przed i~poniżej nich
niemożliwie niebieskie, ciemne morze. Około kilometra od brzegu,
widoczny nawet z~tej odległości, tej wysokości był ekranoplan\footnote{ rodzaj
pojazdu poruszającego się na niewielkiej wysokości nad wodą lub inną
gładką powierzchnią, z~wykorzystaniem efektu przypowierzchniowego,
zob.~\url{https://pl.wikipedia.org/wiki/Ekranoplan} -- przyp.tłum.}. Mniejsze pojazdy brzęczały dookoła jego stu metrów
długości. Dalej za nimi wszystkimi wodoloty i~poduszkowce marynarki
pracowicie patrolowały. Jeszcze dalej, za cieśniną ku Rodos, Myra mogła
dostrzec ich równie pracowite odpowiedniki, patrolowce Greckiego
Zagrożenia.

Pojechali długą, krętą drogą do Fetiye, mijając groby licyjskie w~klifie
i skręcając w~prawo tuż przed meczetem i~potem w~dół brzegiem bazaru do
długiego mola i~esplanady zatoki. Zatrzymali się w~miejscu
zaokrętowania, koło flagi ,,gwiazda i~półksiężyc'' i~rozjarzonej statui
Kemala.

Silnik się zatrzymał. Jason spojrzał na nią.

-- Cóż -- powiedział. -- Czy zobaczę cię kiedykolwiek?

-- Jeżeli oboje będziemy żyć wiecznie -- powiedziała Myra krzywo -- prawdopodobnie tak.

-- Przyjmę to jako nie. -- Jason wyciągnął dłoń. -- Jednak. To było kilka
dobrych dni. Zostańmy w~kontakcie. A jeżeli śledztwo cokolwiek wykaże,
\textit{ja} nawiążę kontakt.

Złapała jego dłoń, jej dopiero co wyostrzone spojrzenie nagle się
zamgliło. 

-- Och, nie traktuj tego jako nie! -- powiedziała, przerażona
jego zwykłą akceptacją jej zwykłych słów jako trwałego rozstania. To
wyglądało jak dojrzewanie jeszcze raz, więcej niż pożądanie, zakochała
się w~nim i~mówiła złe rzeczy. Zaskoczyła go dzikim uściskiem, jej usta
mokre na jego, jej brwi mokre na jego szyi, i~cały czas myślenie, że to
nie była ona, to nie było w~porządku, powinna zachowywać się jak
dyplomata i~zakochiwała się w~jebanym \textit{agencie CIA}, który był
wysłany w~zupełnie innej sprawie. To Nie Uchodziło.

Oderwali się od siebie, trzymając się w~ramionach, patrząc na siebie,
niepomni gadającego tłumu chłopców dookoła pojazdu.

-- Myro, jesteś niesamowita -- powiedział Jason. -- Nigdy Cię nie zapomnę,
będziemy w~kontakcie, spróbuję znowu zobaczyć, ale oboje\ldots

-- Tak -- powiedziała Myra. Wciągnęła powietrze długo, pociągając nosem. -- Jesteśmy oboje dorośli, mamy prace, nie zawsze możemy być po tej samej
stronie i\ldots -- zachichotała -- \ldots mamy tylko czternaście godzin, żeby
uratować Ziemię.

-- Lub coś takiego. Tak.

Jason się oderwał, z~uśmiechem, który dla Myry ciągle wyglądał jak pełne
skruchy pożegnanie. Potem byli dziwnie formalni wobec siebie, gdy Jason
odrzucał niechciane oferty noszenia bagażu chłopców, pomógł jej zabrać
bagaż do łodzi i~potrząsnął dłonie, gdy stała na szczycie drabiny.

Gdy mała łódź sapała przez zatokę do większego pojazdu, Myra obserwowała
Jasona włączającego jeepa, zawracającego i~odjeżdżającego, znikającego
na zakręcie bulwaru.

Westchnęła i~odwróciła się do ekranoplanu. Potężna maszyna wyglądała
nawet nieprawdopodobnie potężniej, gdy wyłaniała się bliżej: samolot
wielkości statku z~przysadzistymi skrzydłami. Statek, który latał.
Regularnie prowadził rejsy od Stambułu do Nowego Jorku, które
zatrzymywały się w~Izmir i~Fetiye, zanim ruszyły w~ostatnią część
podróży. Łódź kierowała się pomiędzy rywalami i~weszła w~cień
lewoburtowego skrzydła, gdzie stopnie wysuwały się aż do platformy
pontonowej. Oficjele oficjalnie oznakowali bagaż do załadowania w~przedziale towarowym, a~pasażerowie weszli na statek.

Myra przeszła do przedniego holu, kupiła gin z~tonikiem przy barze za
pomocą pozostałych tureckich gigalir i~zastosowała się do pilnej
wielojęzykowej rady, żeby usiąść, zanim statek wyruszy.

Nigdy wcześniej nie podróżowała jednym z~tych hybrydowych pojazdów -- wynalazek Związku Radzieckiego ery Chruszczowa, przypomniała sobie ze
szczątkową dumą -- i~była odpowiednio oszołomiona prędkością, a~ponad
wszystkim \textit{wrażeniem} prędkości, gdy wielka maszyna ryczała nad
Morzem Śródziemnym na średniej wysokości dziesięciu metrów i~z maksymalną prędkością pięciuset kilometrów na godzinę. Zostawiła Fetiye
o południu, goniła dzień nad Atlantykiem i~przybyła do Nowego Jorku
czternaście godzin później o~godzinie szóstej po południu lokalnego
czasu.

Myra spędziła większość z~tych czternastu godzin, relaksując się, śpiąc,
oglądając krajobrazy i~zastanawiając się, jak uratować Ziemię.

Od morza, Manhattan miał dziwny, niezrównoważony wygląda, Wieża Dwóch
Mil wyrastała z~Lower East Side, wyrzucając resztę z~perspektywy. Port
South Street ciągle był zniszczony po zamachu i~pachniał więcej niż
zwykle rybami. Myra przeszła osłoniętym tymczasowym nabrzeżem,
nieodróżnialna w~strumieniu schodzących na ląd pasażerów, póki nie
weszła do czekającej limuzyny ambasady z~proporcem ,,słońce i~orzeł'' i~witającym szoferem, który zatrzasnął drzwi, zanim ktokolwiek mógłby się
spojrzeć.

Długi samochód wcisnął się arogancko w~ruch uliczny. Kierowca, krępo
zbudowany Kazach, który wyglądał, jakby dorabiał jako ochroniarz, złapał
jej spojrzenie w~lusterku.

-- Ambasada, obywatelko Dawidowa?

Myra oparła się o~tapicerkę. Na zewnątrz, przez jednokierunkowe szkło
pancerne, mogła ujrzeć ludzi siedzących dookoła ognisk. 

-- Nie, proszę ONZ.

-- Bardzo dobrze, obywatelko.

Samochód szarpnął się z~przodu, potem z~tyłu, zawieszenie poradziło
sobie z~płytkim lejem po pocisku. A może wybojem, finanse miejskie
Nowego Jorku były takie, jakie były.

-- Ale byłabym wdzięczna, gdybyś mógł wyśledzić mój bagaż ze statku do
ambasady, dziękuję.

-- Proszę uprzejmie. -- Zaczął szybko mówić po rosyjsku do telefonu.

Zatrzymali się przy budynku ONZ około dziesięciu minut później, ciężkie
bramy kompleksu odsunęły się dla nich, zamykając się szybko za nimi.
Myra sprawdziła makijaż w~lusterku ręcznym, wyszła z~samochodu i~sprawdziła kurtkę i~spódnicę w~połysku karoserii. Wszystko wyglądało
dobrze. W~rzeczywistości czuła się raczej ubrana za dobrze na to stare,
ponure miejsce. Kałuże na placu, naprawy w~oknach, rdza na stali
konstrukcyjnej i~Wieża Dwóch Mil rzucająca cień na obelisk ze szkła. W~gaju masztów flagowych powiewało w~bryzie dwa tysiące trzysta
dziewięćdziesiąt siedem flag narodów Ziemi i~ich kolonii niczym stado
ptaków przygotowujących się do migracji przed jakąś długą, nadchodzącą
zimą.

Wzięła numer telefonu komórkowego kierowcy i~powiedziała mu, że ma
przynajmniej kilka godzin, zanim go wezwie. Podziękował jej, uśmiechnął
się i~raźno odszedł. Myra przeszła powoli koło starej rzeźby późnego
ZSRR -- Święty Jerzy ubijający Smoka Wojny z~zaoranego metalu rakiet -- ostrożnie w~szpilkach Prady, dookoła kałuż i~przez kruszący się asfalt,
do drzwi. System ekspercki ją rozpoznał, strażnik zasalutował.

W foyer stała zagubiona przez chwilę, póki nie przypomniała sobie, że
całe miejsce było wypatroszone i~odnowione, prawdopodobnie kilka razy,
od kiedy tu była ostatnim razem. Tym razem zostało to wykonane w~modnym
stylu retrofuturystycznym, raczej jak jej własne biuro. Motywem koloru
były liście, od odcieni zieleni przez brązowy do miedzi. Kojące, choć
ludzie w~tym uspokajającym środowisku śpieszyli się, wyglądając
mizernie. Wielka flaga ONZ, niebieskie tło ze stylizowanym globem i~wieńcem oliwnym, wisiała nad biurkiem recepcji. Myra zarejestrowała
chwilowy szok, to było jak zobaczenie swastyki.

Dwóch mężczyzn podeszło, ich kroki lekkie na ciężkim dywanie. Rozpoznała
ich obu: Mustafa Kamadi, ambasador Kazachstanu przy ONZ, niski i~ciemny
oraz Iwan Ibrajew, przedstawiciel MRRNT, wysoki, krótkoobcięty blondyn,
w jego postaci i~cerze objawiały się jakieś geny recesywne
nadwołżańskich osadników niemieckich.

Kamadi potrząsnął jej dłonią, jego uśmiech pokazujący złote sowieckie
zęby, które zachował mimo dwóch odmłodzeń. Ibrajew pochylił się nad
dłonią, prawie ją całując.

-- Cóż, cześć, towarzysze -- powiedziała Myra, chętna zerwać z~formalnościami. -- Dobrze was widzieć.

-- Cóż, wzajemnie -- powiedział Kamadi. -- Przejdziemy do mojego biura?

Iwan Ibrajew rzucił jej spojrzenie.

-- Ach, dziękuję -- powiedziała Myra. -- Ale może z, hmm, dyplomatycznych
powodów, obywatel Ibrajew mógłby\ldots?

-- Doskonale -- powiedział Kamadi.

Gdy czekali na windę, jego język poruszył ustami. 

-- Ach, obywatelko
Dawidowa\ldots

-- Och, Myra, proszę\ldots

-- Myro -- kontynuował w~pośpiechu -- proszę, przyjmij moje spóźnione
kondolencje z~powodu śmierci byłego męża.

-- Dziękuję -- powiedziała.

-- Znałem go niedużo, oczywiście, ale był powszechnie szanowany.

-- W~rzeczy samej był.

Drzwi się otworzyły. Dwóch mężczyzn przepuściło ją, gdy wszyscy
wchodzili. Drzwi się zamknęły.

-- Ciągle myślę, że te zaziemskie\footnote{ oryg. spacist -- zapewne mowa o~sile
polityczno-ekonomicznej, a~nie o~miejscu pobytu ani o~kolonizacji, zatem
korzystam z~tłumaczenia opartego na powieści ,,Pozytonowy Detektyw''
Isaaca Asimova, gdzie ,,spacer worlds'' zostało przetłumaczone jako
Światy Zaziemskie -- przyp.tłum.} bękarty go zabiły -- powiedział nagle
Ibrajew. Spojrzał na mikrokamerę w~rogu. -- I~nie obchodzi mnie, kto wie!

Szum i~pęd, lekki wzrost i~zmniejszenie siły przyciągania. Myra poczuła,
jak jej kolana drżą, gdy wyszła z~windy na długi korytarz.

-- Śledztwa nadal trwają. -- Wzruszył sztywno ramionami. -- Osobiście, nie
sądzę, żeby w~tym maczał ręce Reid, to wszystko, co mogę powiedzieć. -- Uśmiechnęła się szybko do Iwana, potem Mustafy. -- Znałam tego
mężczyznę\ldots intymnie.

Uczciwa twarz Iwana zarumieniła się widocznie. Mustafa pokazał złote
kły.

-- Długie życie, prowadzi do problemów -- powiedział. -- W~końcu,
wszystkich nas zbliża do siebie. Jaka jest teoria, sześć stopni
oddzielenia? -- Roześmiał się chropawo. -- Kiedy byłem bardzo młody,
uścisnąłem dłoń kobiety, która była jedną z~sekretarek Lenina. Pomyślcie
o tym!

Myra pomyślała o~tym. 

-- Jeżeli o~tym pomyśleć -- zaśmiała się ponuro -- to
ja też.

Jednak ciągle ją to dotknęło, ból jak ostrze w~brzuchu:\textit{wszystko
moje statki odeszły, moi ludzie nie żyją.}

Nie, nie. Jeszcze nie. Ciągle ma statki i~ciągle może mieć Jasona.

Biuro Iwana Ibrajewa było małe. Usiedli z~kolanami opartymi o~biurko.
Trójlistna flaga wisiała na jednej ścianie, reklamy rakiet na drugiej.
Okno wychodziło na East River. Drzwi były otwarte. Sługa pojawił się z~kawą i~kubkami, potem dyskretnie zniknął. Iwan zamknął drzwi i~włączył
urządzenia przeciwpodsłuchowe. Myra przełknęła, próbując usunąć dziwne
ciśnienie w~jej bębenkach. Nie znikło.

Przełknęła ponownie, łyknęła kawy. Dwóch mężczyzn pochyliło się do
przodu, spojrzało na siebie. Ibrajew zachęcił gestem, żeby zaczęła
pierwsza.

-- Ok -- powiedziała. -- Wiecie, dlaczego tutaj jestem, tak?

-- Negocjować pomoc wojskową z~USA -- powiedział Ibrajew.

-- Tak, cóż. Tak czy inaczej, Wschodnioamerykanów. -- Roześmiali się. -- Już mi dano do zrozumienia, że niedużo nadejdzie. Co ta osoba
powiedziała mi, że nie wie, czego prawdopodobnie nie wiecie, jest to, co
mamy im do zaoferowania. -- Przerwała. Ich twarze nic nie mówiły. -- MRRNT
ciągle ma działające atomówki.

-- \textit{Broń} jądrową? -- spytał Kamadi. Ibrajew uśmiechnął się, jakby
zawsze podejrzewał, że małe państwo, któremu służył, ciągle osłaniało
ukryte żądło.

-- Bronie. -- Myra skinęła głową. -- Niszczyciele miast, w~większości, ale
rozsądnie wszechstronny zestaw, aż do taktycznych atomówek, które\ldots -- wzruszyła ramionami -- \ldots nie są takie trudne do załatwienia. Jednak
ciągle.

-- Nic o~tym nie wiedzieliśmy -- powiedział Kamadi. Ibrajew kiwnął na
zdecydowaną zgodność.

-- Chingiz Sulejmanow nie powiedział wam?

-- \textit{Niet.}

-- Dobrze -- powiedział raźnie Myra. -- Cóż, to właśnie wam teraz mówię.
Kazachstan jest teraz \textit{de facto} mocarstwem, o~ile jest to warte.

Iwan Ibrajew splótł palce. 

-- Jak je użyć, to jest pytanie. Nie można ich
użyć bezpośrednio przeciwko Szinosowom, nie ma sensu bombardowanie
stepu, co?

Oczy Kamadi pojaśniały, jego usta ułożyły się w~błyszczący węzeł. 

-- Moglibyśmy powiedzieć, że nie muszą być wycelowane we wschodnią\ldots

-- Ha! -- parsknęła Myra. -- Obywatele, \textit{towarzysze}\ldots \textit{jestem}
Amerykanką i~mogę wam powiedzieć jedną rzecz, Amerykanie, Wschodni,
Zachodni, czy Środkowi, nie zniosą szantażu nuklearnego. To są ludzie,
których strategia jądrowa zawierała straty rzędu megaśmierci po
\textit{własnej} stronie. Mogli trochę się ograniczyć w~świecie, ale nie
są zbyt zdemoralizowani, żeby nas rozwalić, zanim dowiemy się, co nas
trafiło, jeżeli tylko tego spróbujemy. Nie. To, co chce Prezydent, jest
prawie przeciwnością: zaoferować broń, pod naszą kontrolą oczywiście,
ale z~publiczną umową nie do złamania, USA lub ONZ, w~zamian za sojusz
wojskowy, który może zatrzymać Szinosowiety w~miejscu.

Obaj mężczyźni rozmyślali nad tą propozycją ze spokojem pokerzystów.
Iwan otworzył paczkę Marlboro i~zaoferował Myrze. Zapaliła z~wdzięcznością.

-- Warte spróbowania -- powiedział Kamadi. -- Muszę powiedzieć, pomiędzy
nami, myślę, że możemy żałować zrezygnowania z~nowej władzy, którą
atomówki włożyły w~nasze ręce.

-- To niewielka władza -- powiedziała Myra. -- W~sensie, proponujemy
szantaż Amerykanów, ale nie możliwe użyciem przeciwko nim, ale z~możliwym użyciem przeciwko komuś bez ich pozwolenia.

Kamadi napełnił kubki, marszcząc brwi. 

-- ONZ ciągle ma jakieś atomówki,
jak właśnie widzieliśmy. Podejrzewam, że ich zapasy zostały znacząco
wyczerpane tym użyciem. Więc mogą być chętni do uzupełnienia.

Iwan wskazał na plakaty na ścianie. 

-- Właśnie do mnie trafiło -- powiedział -- że moglibyśmy wrócić \textit{do początków} do starego
biznesu: sprzedawania odstraszania każdemu, kto chce!

Myra się roześmiała. 

-- Odstraszania przeciwko komu? ONZ? Nie wydaje mi
się, żeby to długo potrwało.

Kamadi skrzywił się, jakby kawa była bardziej gorzka niż oczekiwał. 

-- Tak, rozumiem Twój punkt. Może tak jest najlepiej. Zatem co możemy
zrobić, żeby to ułatwić?

Myra zaciągnęła się mocno papierosem. 

-- Oprócz potwierdzenia mojego
statusu? -- Uśmiechnęła się do nich. -- Możecie zaaranżować, mam nadzieję,
kogoś do reprezentowania drugiej strony. Dużo o~tym myślałam w~drodze i~przejrzałam personel USA tutaj, mam sugestię co do osoby, do której
moglibyśmy podejść.

-- Sadie Rutelli -- powiedział Ibrajew.

-- Dokładnie! Skąd wiedziałeś?

Ibrajew postukał w~opaskę. 

-- Wielkie systemy eksperckie myślą tak samo.

-- Och, cóż -- powiedziała Myra, czując się nieco sflaczała. -- Domyślam
się, że była oczywistym wyborem. Jakie są szanse na spotkanie?

Ibrajew przewróciła oczami i~mrugnął kilka razy. 

-- Zgodnie z~jej
publicznym harmonogramem\ldots całkiem dobre. Ma wolne miejsce pomiędzy
dziesiątą wieczorem a~północą, czyli kiedy zamierza jechać do domu. Czy
chciałabyś, żebym uruchomił program pytający, żeby umówić spotkanie?

-- Oczywiście, że chciałabym -- powiedziała Myra.

-- To późno -- powiedział Kamadi. -- Będzie zmęczona.

-- Złóż jej ofertę spotkania przy kolacji -- zasugerowała Myra. -- Może
wybierać, zapłacę. Tylko nas dwoje, mam nadzieję, że nie macie nic
przeciwko?

Dyplomaci odrzucili sam pomysł, że mogliby minimalnie pomyśleć o~tak
głęboko niegodnej emocji. Myra i~Iwan dopasowali sobowtóry i~ich
elektroniczni sekretarze zaczęli próbować dotrzeć do Rutelli.

-- Dotarcie do niej może zabrać chwilę -- powiedział Ibrajew. -- Jest
zajęta.

Myra wstała. 

-- Zatem wezmę prysznic i~prześpię się w~hotelu. Jeżeli ktoś
powie, że pilnie się skontaktować, dzwońcie na sobowtóra. Jeżeli Rutelli
odpowie, dzwońcie natychmiast, bezpośrednio. W~innych przypadkach,
zadzwońcie do mnie rano!

~

-- Mam nadzieję, że nie jesteś jeszcze wystarczająco ekskomuchem, żeby
być tym zażenowana -- powiedziała Sadie Rutelli. Podała Myrze kieliszek
szampana z~minibaru limuzyny, która podjęła ją z~Waldorfu.

-- W~rzeczy samej nie. -- Myra wzniosła ironicznie toast. Opierała się o~skórzane siedzenie i~cieszyła się z~każdej sekundy. -- Wiem wszystko o~wydatkach na reprezentowanie. To wszystko było w~Marksie. My ekskomuchy
jesteśmy w~takich kwestiach twardymi cynikami.

-- Dobrze Cię widzieć znowu, Myro. Minęło tyle lat.

-- Tak, ile? Trzydzieści cztery lata. Jezu. I~wyglądasz jakbyś się
\textit{urodziła }w 2025.

Sadie, siedząc na miejscu naprzeciwko, wyglądała całkiem oszałamiająco z~jej długimi czarnymi włosami, z~sobolowym bolero i~w wieczorowej sukni
koloru indygo. Myra pamiętała ją właśnie tak olśniewającą w~błękitnym
mundurze. Była jednym z~agentów Komisji Rozbrojenia ONZ, która odebrała
MRRNT jej atomówki po wojnie. Zrobiła to taktownie i~z determinacją i,
pomimo napiętych okoliczności, Myra ją polubiła.

-- Och, schlebiasz mi -- powiedział Sadie. -- Muszę powiedzieć, że
wyglądasz młodziej, niż pamiętam.

-- Ach, ciągle nad tym pracuję. Lub małe maszyny to robią. -- Myra potarła
grzbiety dłoni, rozkoszując się teraz uczuciem gładkości i~miękkości,
tego rodzaju, które kremy kosmetyczne obiecywały, a~nanomaszyny
zrealizowały.

Czuła się również energiczna, nie doświadczała jet \dywiz lagu
(ekranoplan \dywiz lagu \ldots) i~złapane dwie godziny snu odświeżyły ją bardziej,
niż wydawało się to proporcjonalne.

-- Jednak -- powiedziała Sadie -- nie możesz pokonać backupów, jeżeli
chcesz być pewna życia \ldots długi czas.

-- Och, naprawdę? -- Myra próbowała nie drwić. -- Wierzysz, że ta rzecz
działa?

-- Do tego stopnia, że zrobiłam backup, tak.

-- Czy ktokolwiek wrócił \textit{z powrotem} z~backupu?

Sadie zmarszczyła brwi. 

-- Jako taki nie. Nikt nigdy nie był sklonowany i~nie wdrukował zbackupowanych wspomnień w~mózg klona. Choć są plotki, o~pewnych testach, które ludzie Reida wykonali, dawno temu\ldots

-- Z~małpami. Ta, wiem o~nich. Skąd wiesz, że osobowość szympansów,
kurwa, przetrwała?

Sadie się uśmiechnęła. 

-- Ach, Myro. Ciągle jesteś cholerną, dialektyczną
materialistką. Chciałam powiedzieć, były przypadki, kiedy ludzie
\textit{uruchomili }swoje kopie w~środowiskach VR. To drogie. Ostatnie
komputery optyczne w~nanotechnologii, te rzeczy, które wyglądają jak
kryształowe kule. Zawierają \textit{piekielną} moc obliczeniową, ale są
ludzie, których na to stać: gwiazdy rocka, filmu itd.

-- Nie martwią się o~konkurencję?

-- Nie, nie! -- Sadie spojrzała na nią. -- O to chodzi. Kopie występują,
oryginały po prostu są na emeryturze!

-- Brzmi jak niesprawiedliwy układ -- powiedziała Myra. -- Wyobraź sobie,
budzisz się i~odkrywasz, że żyjesz w~kwarcowym scalaku i~musisz pracować
na korzyść swojego egoistycznego oryginału. Jezu. Zastrajkowałabym. -- Upozowała się, jakby trzymała gitarę, zaśpiewała nosowo. -- Nie zagram w~Sim City\ldots

Sadie się roześmiała. 

-- Póki Twój zarząd ciebie nie zrestartuje.

Myra też się śmiała, ale zmroziła ją myśl, że istnieje nowa droga dla
bogatych opuszczenia Ziemi, nie w~kosmos, ale cyberprzestrzeń, z~ich
kontami bankowymi, by żyć na zawsze w~telewizji, gdzie ich twarze zawsze
były. A co za śmiech byłby, gdy, w~ich krzemowym niebie, poznali
Generała\ldots

Ach, cholera. Do interesów.

-- Czy w~tym samochodzie można bezpiecznie rozmawiać? -- spytała, nagle
pewna, że restauracja nie byłaby.

Sadie machnęła powoli dłonią. 

-- Nie ma znaczenia -- powiedziała. -- Wiem,
co chcesz zaoferować, fakt, że prosiłaś o~spotkanie ze \textit{mną} jakoś
to zdradza, tak?

-- Patrząc, jak to przedstawiasz\ldots ale diabeł tkwi w~szczegółach.

-- Nie potrzebujemy się martwić szczegółami -- powiedziała Sadie. -- Nie
dzisiaj. Tylko odrobina dyskrecji i~omówienia i~będziemy w~porządku.

Myra słabo się uśmiechnęła. Prawdopodobnie Sadie znała wiele szczegółów.
Śledzenie dyslokacji atomówek ciągle było jej pracą. Jej opaska -- Myra
domyślała, że delikatnie błyszczący pas dookoła czoła Sadie był opaską -- pokazałby jej każde podejrzewaną atomówkę taktyczną na Ziemi i~poza nią.
I miała przenikliwą idee, gdzie były strategiczne atomówki Myry.

Myra wyjrzała przez okno. Samochód rozsądnie przyśpieszał \ldots Amsterdam
Avenue, dochodząc do wysokich numerów. Stare budynki były pokryte
pęcherzami, chodniki zatłoczone budami skłotersów zbudowanymi przez nano
wyglądającymi jak pajęcze bąble, połączonymi siecią schodów, drabin i~huśtawek. Ich mieszkańcy, oraz ludzie na ulicy, byli w~tej części w~większości biali. Pracownicy biurowi, w~większości czarni i~Latynosi,
przedzierali się przez tłum, ignorując ich natarczywość.

-- Uchodźcy z~Centralnych Stanów -- powiedziała Sadie. -- Oklahoma.

Restauracja, kiedy do niej dotarły kilka minut później, była głęboko w~Harlemie. Migracja czarnych dawno zmieniła charakter obszaru. Myra i~Sadie wyszły na chodnik zajęty kramami pod niezainteresowanymi,
nieodgadnionymi spojrzeniami Peruwiańczyków i~Chilijczyków. Wyglądało to
jak Ameryka, gdzie wygrali Indianie. W~istocie, ci Indianie stracili
wszystko, co mieli na rzecz Gonzalistów, dekadę lub dwie wcześniej.
Gonzaliści zostali pokonani, ale ich ofiary nie miały chęci opuszczać
USA. Teraz drobny handel byłych uchodźców wypełnił biura i~sklepy,
rozlewał się na chodniki, tak jak ich wielkie rodziny wypełniały stare
budynki budownictwa komunalnego.

Niemniej jednak, myślała Myra, ucieczka ogólnie od zdobywania szczytów
była wygraną. Gonzaliści byli okropną bandą, nawet dla komunistów.
Rodzajem, który odrzuciłby Pol Pota jako rewizjonistę.

Restauracja nazywała się Los Malvinas. Wewnątrz było tłoczno, głównie od
młodych Latynosów starych fortun, elegancko ubranych, snobistycznie
pewnych ich społecznej i~rasowej wyższości nad nowszymi imigrantami, ale
wykorzystującymi -- w~swojej modzie, zdaniach jak na inne sposoby -- ich
kulturalne powiązania. Powietrze pachniało mięsem i~dymem, na ścianach
były wielkie plakaty Peron, Eva, Che, Thatcher i~Madonny. Sadie była
przywitana po imieniu przez uważnego głównego kelnera, który poprowadził
je do stołu z~tyłu, na małym podwórcu otoczonym drzewami i~murami
pokrytymi pnączami.

-- Miłe miejsce -- powiedziała Myra. Spojrzała w~menu. -- Nie wygląda, żeby
urwało dużo z~karty firmowej.

-- Wiedziałam, że ci się spodoba -- powiedziała Sadie. Zrzuciła bolero na
oparcie krzesła, pokazując nagie ramiona. -- Dzbanek sangrii?

-- Dobry pomysł. -- Myra stuknęła menu. -- Będziesz musiała mi z~tym
doradzić. Całe szczęście, że nie jestem wegetarianką.

Złożyły zamówienie, które, Sadie zapewniła, będzie zarówno dobre, jak i~duże, sączyły sangrię, paliły skręta i~żuły chleb zanurzony w~oliwie z~czosnkiem, gdy czekały.

-- Ok -- powiedziała Myra. Rozejrzała się dookoła, odruchowo. Kilku
wenezuelskich inżynierów naftowych, w~koszulach i~szortach, rozmawiało
głośno wokół jedynego innego zajętego stolika. Wzruszyła ramionami i~pokręciła głową. 

-- Ok. Rozmawiajmy. Mam nadzieję, że nie masz nic
przeciwko, ale cholera. Masz pełnomocnictwo do negocjowania na poziomie,
o którym mówimy?

-- Pewnie -- powiedziała jej Sadie. -- Nie martw się o~to. Prosta linia na
szczyt. Nie, żeby to był wysoki priorytet u Szefa.

-- A co z~ONZ?

Sadie lekceważąco pomachała kawałkiem chleba. 

-- To już załatwione.

-- Bez zmiany tam, co?

-- Zmiany, tak, ale znowu weszliśmy na szczyt. Na tyle, o~ile jest to
warte.

-- Racja, wiem, co masz na myśli. ,,O ile jest to warte'' wydaje się
często pojawiać w~rozmowach w~tych dniach. Jednak. Oto umowa.
Sprzedajemy wam wyłączne prawa do pakietu, wspieracie nas przeciwko
hordom komunistów. Lista zakupów będzie następna, ale jak mówiłaś,
szczegóły później.

Kelner pojawił się z~gorącym półmiskiem i~talerzami. Dziewczyna przyszła
z miskami sałatek i~ryżu. Głównym daniem była jakby sałatka z~mięsa, w~której większość możliwych typów mięsa krowy było reprezentowanych, razem ze
smakowitszymi wnętrznościami i~kilkoma nieco mniej.

-- Smacznego paniom.

-- Dziękuję -- powiedziała Sadie. Zgasiła niedopałek. -- Och, i~jeszcze
jedną sangrię, proszę.

Myra była wygłodniała, jej apetyt zaostrzony jeszcze bardziej skrętem i~spędziła dwadzieścia minut w~atawistycznej, mięsożernej ekstazie i~okrzykach, zanim znowu mogła podjąć rozmowę.

-- Zatem Sadie. -- Położyła żebro, wytarła palce i~brodę. -- Co powiesz?

Sadie wzięła duży łyk sangrii, lód sentymentalnie stukający.

-- Wiesz, ten facet, którego wysłaliśmy, by z~Tobą porozmawiał? Z~Firmy?

-- Trudno o~nim zapomnieć.

-- Aha. -- Sadie westchnęła. -- Cóż, Myro, przykro nam, ale. -- Podrapała
się w~ucho. -- To ciągle jest umowa, w~zasadzie. Możemy dać wam trochę
sprzętu, pewnie, ale nic takiego, o~co prosicie. Zdecydowanie nie
sojusz.

Myra zakołysała się do tyłu. Usłyszała, jak nogi metalowego krzesła
drapią bruk.

-- To nawet \textit{z} tym, co oferujemy?

-- Nawet z. -- Sadie wybrała coś wyglądającego na wnętrzności,
przeciągnęła to przez zęby. -- Ponieważ nie możemy tego wziąć. Nie ma dla
nas sensu, szczerze.

-- O mój Boże. Och, cholera. -- Myra sięgnęła po papierosy. -- Masz coś\ldots

-- Śmiało. Proszę.

-- Jaki jest problem z~naszym pakietem?

-- Zestaw umiejętności i~system zastany, w~zasadzie. -- Sadie spojrzała na
czubek papierosa, zmarszczyła nos i~wyssała tłuszcz z~ust. -- Spójrz
ponad moją głowę. W~górę. Co widzisz?

Myra spojrzała na południe i~w górę.

-- Szczyt Wieży Dwóch Mili?

-- Tak. Wiesz, co jest w~środku? Skłotersi, w~większości. Cholerna rzecz
prawie się sama zbudowała jak kamienne drzewo. Jednak budowniczy nie
mogli znaleźć biznesów, żeby wynajęli przestrzeń w~niej.

-- Tego rodzaju rzeczy są dość powszechne -- powiedziała Myra. -- Spekulacyjne spektakularne budynki są zwykle oddawane tuż przed recesją
i są puste aż do następnego boomu.

-- Jeżeli będzie kolejny boom\ldots -- powiedziała ponuro Sadie.

Myra pamiętała wersję Shin Se-Ha równań Otoh. 

-- Będzie -- powiedziała. W~każdym razie jeszcze jeden, nie powiedziała. -- O co chodzi?

-- Tracimy ludzi -- powiedziała Sadie. -- To nie tajemnica. Zamach się udał
na więcej sposobów niż zawiódł. Cholernie dużo naszych najlepszych
naukowców i~inżynierów wyemigrowała na kolonie orbitalne i~wspiera
fakcję, której zaopatrzenie dostarczała Ochrona Wzajemna.

-- Zewnętrzni.

-- Ta. Myślą, że cywilizacja na Ziemi jest skazana na zagładę i~się
wydostają. A, bardziej do rzeczy, też dużo wielkich fortun. Większość
korporacji ma siedziby w~orbitalnych rajach podatkowych, od przynajmniej
od Jesiennej Rewolucji. Teraz mają mięśnie, techniczne, wojskowe, żeby
to poprzeć. I~personel na miejscu. Finansują nas, jasne, ale ściśle jako
opłaty użytkowników, jak wynajmowanie agencji obrony, i~tylko tak długo,
jak nie wyjdziemy poza linię. Możesz myśleć o~USA jako starym,
imperialistycznym dręczycielu, ale w~tych czasach jesteśmy tylko kolejną
bananową republiką. Cała Ziemia to Trzeci Świat. Wielkie pieniądze i~zaawansowana praca są w~kosmosie, a~to, co zostało poniżej to w~większości nadwyżka ludności. -- Sadie uśmiechnęła się krzywo. -- I~biurokraci jak Ty czy ja.

-- Zatem mówisz, że imperium USA nadal istnieje -- powiedziała Myra. -- Ale
jego kapitał jest teraz na orbicie.

-- Tak, dokładnie!

-- Słusznie -- powiedziała Myra -- ale jak to wpływa na naszą ofertę?

-- Cóż. -- Sadie oparła się, zaciągnęła się krótko, jak łyk, papierosem. -- Pozwól, że przedstawię ci analogię. Załóżmy, całkiem hipotetycznie, na
potrzeby argumentu, że USA chciałoby wrócić do strategicznej pozycji
jądrowej. Zostawiając na boku fakt, że Trzecia Wojna Światowa zrobiła to
samo z~atomówkami co Pierwsza z~bronią chemiczną. Przynajmniej w~terminach użycia ich na Ziemi, ONZ udało się nie odpowiadać za wybuchy w~warstwie Kennely\dywiz Heaviside'a, ale to był rodzaj farta. Zostawmy z~boku
fakt, że wielkie fortuny na orbicie stają się wirtualnie Zielone od
paranoi o~atomówkach w~kosmosie.

Aha, pomyślała Myra. Nie odkładałaby tego na boku, w~ogóle. To było
sedno sprawy, mimo tego, że reszta punktów Sadie była prawdziwa.

-- Zostawmy z~boku fakt, że po prostu nie ma już tak wiele wielkich
atomówek. Załóżmy, że ktoś przyszedłby do nas, nie wiem, z~magazynem
starych postradzieckich niszczycieli bomb: zapalniki laserowe, długi
czas magazynowania, minimum utrzymania. Nadal nie byłyby dla nas żadnym
pożytkiem, ponieważ nasza cała doktryna wojskowa odsunęła się od
polegania na atomówkach. Utrzymanie wiarygodnego strategicznego
odstraszania nuklearnego wymaga więcej niż samo utrzymanie broni.
Potrzebujesz rakiet, załóg bombowców, taktyków, analityków, ciągłej
praktyki. Cholera, powinnam wiedzieć, pracowałam dostatecznie ciężko na
rozpędzaniu zespołów i~niszczeniu zapisów, w~moich czasach rozbrojenia.
Nie mamy już ludzi z~takimi umiejętnościami i~nie mamy ludzi, żeby
wytrenowali nowych. Potrzebujemy wszystkich naszych umiejętności, żeby
nasze myśliwce stealth latały, nasi teleżołnierze, taktyki i~techniki
inteligentnej bitwy nadal działały.

-- Myślę, że rozumiem Twój punkt widzenia -- powiedział Myra sucho. -- Zatem, na podstawie tego samego rozumowania, nasza oferta, hm, praw do
wydobycia w~Kazachstanie nie jest interesująca.

-- Mogłabyś tak powiedzieć. To jest w~analogii.

Myra wątpiła, czy odwrócenie analogii i~rzeczywistości w~ogóle
oszukałoby jakiegoś podsłuchującego, ale istniał protokół do
zrealizowania w~tego rodzaju rzeczach. Było, jak sobie przypominała,
nielegalnym dla funkcjonariuszy publicznych w~jurysdykcji ONZ -- po
Jesiennej Rewolucji tak samo jak wcześniej -- nawet \textit{dyskutować} o~odstraszaniu jądrowym jako prawdziwej opcji politycznej.

I oczywiście, one nie rozmawiały. Nie w~sposób, który utrzymałby się
przed sądem, co miało znaczenie.

-- Oczywiście istnieje jeden zaawansowany kraj, który nie jest jeszcze
bananową republiką\ldots -- powiedziała Myra. -- Nigdy nawet nie dołączył się
ONZ, jeżeli o~to chodzi.

Sadie wzruszyła ramionami. 

-- Jeżeli chcesz, idź do Brytoli -- powiedziała, lekko, ale zrozumiała ukryte zagrożenie. -- Nie mój problem.
Ale będzie dla kogoś innego.

-- Tylko dopóty, dopóki wiemy, gdzie stoimy -- powiedziała Myra, podobnie
rozumiejąc aluzję. -- Ok. -- Zapomnij o~umowie z~pakietem. Co z~wojskami
lądowymi i~wsparciem powietrznym?

-- To drugie, może. W~razie potrzeby. Oraz hardware. Mamy hardware.
Żołnierzy, nie.

-- Och, no weź. Nawet najemników. Możemy zapłacić dobre stawki.

-- Najemnicy? -- roześmiała się Sadie. -- Najemnicy są najlepszymi, których
\textit{mamy}. Używamy ich jako kręgosłupa w~naszych elitarnych
regimentach. A oddziały elitarne to wszystko, co zostało. Zaciąg
zwykłych żołnierzy po prostu stał się niemożliwy. Pobór do wojska? Nawet
o tym nie myśl.

Myra ciągle patrzyła sceptycznie. 

-- Pokażę Ci -- powiedziała jej Sadie.

Rozmawiali przyjaźnie dłuższy czas, zgadzając się zrzucić na Kamadiego i~Ibrajewa szczegółową pracę negocjowania tej małej pomocy, którą USA
mogły dać, ale w~zasadzie, rozmowa była skończona. Myra uregulowała
rachunek, zostawiła hojny napiwek i~wyszła za Sadie. Gdy przechodzili
zatłoczony chodnik do limuzyny, Sadie zaskoczyła Myrę, podchodząc
odważnie do grupki Andyjczyków stojących dookoła straganu z~nakryciami
głowy. Chłopcy spojrzeli na nią od góry do dołu, leniwie ciekawi.

-- Cześć, chłopaki -- powiedziała. -- Jak się macie?

-- Dobrze, proszę pani, dobrze.

-- Co powiecie na pracę?

-- To jest nasza praca. -- Uśmiechnęli się do właściciela kramu, który
uśmiechnął się z~rezygnacją.

-- Myśleliście o~wstąpieniu do Armii? Dobra płaca, świetne warunki.
Twarde chłopaki jak wy moglibyście być w~tym całkiem nieźli.

Musieli się złapać wzajemnie, śmieli się tak mocno.

-- Nie szedłem dać się zabić, zwalczając wieśniaków i~naukowców -- powiedział jeden z~nich. Machnięcie jego ramienia objęło wszystko od
Wieży Dwóch Mil do straganu nakryć głowy najeżonych wąsami. Splunął
dalej na chodnik.

-- Woleliście technologię od człowieka -- powiedział. -- Niech technologia
was obroni.


\chapter{Przymierze Skały}

Wyszedłem za Druinem z~tunelu, w galerię hodowców kryształów widzenia
bez wiedzy, do czego zmierzał. Tak jak on, miałem przewieszony karabin i~puste ręce. Przemaszerował do centralnej alejki pomiędzy rzędami koryt z~kamieniami i~skręcił, idąc w~mniej więcej tym samym kierunku, jak
dotychczas szliśmy w~dół, w~kierunku starej elektrowni.

-- Hej!

Jeden z~hodowców podszedł w~pośpiechu. Był krępym, ciemnym mężczyzną z~ostrymi, ruchliwymi oczami. Jego fartuch był niebieski, przykurzony
białym proszkiem, który łapał światło jak kryształ. Zatrzymał się kilka
metrów przed nami i~spojrzał.

-- Co tutaj robicie? -- zażądał. -- Jak się dostaliście?

-- Jesteśmy\ldots

Druin skinął na mnie, żebym był cicho.

-- Tylko przechodziliśmy -- powiedział. Rozejrzał się po sali z~miną
zszokowanego zachwytu. Inni majsterkowie przestali pracować i~stali
czujnie. -- Muszę powiedzieć, że macie tutaj fascynujące miejsce.

-- Jak się tutaj dostaliście? -- powtórzył majsterek, robiąc krok bliżej.

Druin wskazał kciukiem do tyłu nad ramieniem. 

-- Och, goniliśmy sarnę -- wyjaśnił zwyczajnie. -- Dotarliśmy do rodzaju\ldots -- spojrzał na mnie,
jakby szukał słowa -- \ldots włazu, tak byś to nazwał? W~lasach tam.
Zeszliśmy na dół dla frajdy, jakby, i~przeszliśmy do waszego tunelu.

Druin wsunął kciuk pod pasek karabinu i~dodał: 

-- Więc jeżeli nie macie
nic przeciwko, to pójdziemy w~swoją stronę.

Majsterek okazał więcej prawdziwego zdumienia, niż Druin udawał.

-- Przyszliście z~tunelu?

-- Tak -- powiedział Druin. -- Tam są naprawdę niesamowite pustaki -- dodał,
z doceniającym mrugnięciem. Zaczął iść do przodu, ja koło niego. Ku
mojej zdziwieniu, majsterek odsunął się, ze spojrzeniem i~drobnym
potrząśnięciem głowy do kolegów. Podejrzewałem, że nikt z~zewnątrz nie
przeszedł widmowych strażników jaskini od dawna i~że majsterki tutaj po
prostu nie wiedzieli co z~nami zrobić.

Po obu stronach były koryta kamieni. Pierwsze, które minęliśmy,
zawierały warstwy małych kamieni, prawie żwir. Kolejne rzędy miały
większe i~mniejsze kamienie, aż dotarliśmy do końca, gdzie koryto, a~raczej w~tym momencie, wielka okrągła wanna mogła zawierać pojedynczy
głaz. Na podłodze pod korytami leżały dziwnie ukształtowane kamienie,
najwidoczniej odrzucone, niektóre z~tych ofiar kontroli jakości
najwyraźniej skończyły w~tunelu. Jednak nie widzieliśmy pustaków w~tej
komnacie i~zastanawiałem się, czy źle zrozumiałem implikowaną sekwencję
wydarzeń, czy może światło było tutaj zbyt jasne dla takich
wyświetlaczy.

W samych kamieniach, dziwnie zniekształcone przez falującą wodę,
straszne, chwilowe sceny odgrywały się same bez spójności, która
wzrastała z~wielkością kamieni. Nie miałem czasu, żeby je oglądać, ale
kilka razy miałem poczucie, że twarze błyskające na tych gładkich
powierzchniach, były twarzami, które widziałem w~tunelu.

Ściany i~sufit sztucznej jaskini zbiegały się do wejścia do innego
przejścia, około dwóch i~pół metra wysokości i~dwóch szerokości.
Prowadziło dalej około trzydziestu metrów przed nami, poza którymi
wyłaniały się ciemniejsze drzwi. Ten korytarz był bez wątpienia
sztuczny, jego kwadratowe ściany i~sufit były wykonane z~tej samej
szklistej substancji co szyb. Jego oświetlenie było subtelnie różne od
tego w~galerii wzrostu, choć pochodziło z~podobnych szklanych paneli,
miało nutę żółtego, który wskazywał na zwykłe elektryczne światło, tylko
mocniejsze niż zwykle napotykane. Nasze kroki dzwoniły na ceramicznej
podłodze, dając ostre echo.

-- Dobrze się tam trzymałeś -- powiedziałem do Druina.

-- Ach, to tylko blef -- powiedział. -- Przyzwyczaili się do ludzi
przestraszonych \textit{ich} blefami. Jednak sądzę, że wkrótce spotkamy
kogoś, kto będzie na nas czekał, nasi przyjaciele z~tyłu na pewno
przekazali do przodu.

-- Nie przejmujesz się?

-- Ani trochę.

Ja tak, ale nie miałem zamiaru tego pokazywać. Moje serce dudniło i~w głowie mi szumiało od oszołamiających obrazów, jak same kryształy
widzenia, a~moja ręka trzymająca pasek karabinu była mokra od potu.

Odpowiedź, której oczekiwał Druin -- lub być może, poważniejsza odpowiedź
-- pojawiła się, kiedy byliśmy w~dwóch trzecich drogi w~korytarzu. Fergal
i dwóch innych mężczyzn pojawili się przy wyjściu, zagradzając nam
drogę. Trzymali karabiny nieznajomego wyglądu, niewymierzone w~nas, ale
gotowe do użycia. Szliśmy do przodu. Fergal wyszedł przed innych i~podniósł dłoń.

-- Zatrzymajcie się tam! -- rozkazał.

Zatrzymaliśmy się.

-- Po co tutaj przyszliście? -- spytał Fergal.

Zdecydowałem, że to był moment, żebym powiedział w~swoim imieniu.

-- Jestem tutaj, żeby się z~Tobą zobaczyć -- powiedziałem. -- I~Merrial.

-- Widzisz mnie -- powiedział Fergal. Machnął lekceważąco ręką. -- Pogadam
z Tobą później. -- Podszedł bliżej, kilka metrów dalej i~wpatrzył się w~Druina. -- Znam Cię -- powiedział jadowicie.

Druin wzruszył ramionami. 

-- Widziałeś mnie w~okolicy.

Broń Fergal błyskawicznie wymierzyła w~brzuch Druina. Mój towarzysz
drgnął ku paskowi karabinu, potem podniósł obie ręce ponad głowę.
Pozostali dwaj majsterkowie w~tym samym momencie wycelowali swoje
karabiny.

-- Wiem, kim Ty \textit{jesteś} -- powiedział wolno Fergal -- i~czym jesteś.
Daj mi powód, dlaczego nie powinienem cię zabić.

Druin wziął głęboki wdech. 

-- Och, człowieku, jeżeli musisz o~to pytać,
to nie ma dla Ciebie nadziei -- powiedział to spokojnym głosem.
Spojrzałem na niego z~boku, zmrożony prócz mocnego drżenia szczęki i~kolan. 

-- Widzisz -- kontynuował Druin konwersacyjnie -- jeżeli mnie
zabijesz, teraz, mój przyjaciel Clovis tutaj musiałby cię za jakiś czas
zabić. Zabiłby, odciął Twoją głowę i~zaniósł do mojej wdowy i~dzieciaków
jako dowód, że nie żyjesz i~sprawa by się skończyła.

Spojrzał na mnie. 

-- Zrobiłbyś to, co?

-- Zrobiłbym -- przysiągłem. Jadłem pod dachem Druina i~nie mógłbym
odmówić takiego zadania, w~potrzebie. Myśl o~tym sprawiła, że poczułem
się chory, ale wcale nie zmieniło to mojego postanowienia. Nie miałem w~ogóle pojęcia, dlaczego Fergal chciałby zabić Druina i~nie dbałem o~to.
To, że zastanawiał się nad morderstwem, powiedziało mi wszystko, co
potrzebowałem wiedzieć o~nim.

-- Cóż, a~nie mówiłem -- powiedział Druin. -- Oczywiście, mniemam, że
mógłbyś też zabić Clovisa, ale to raczej tylko podwoiłoby Twój problem.

Nie uważałem tej ostatniej uwagi tak jasnej i~uspokajającej jak
zabrzmiała w~ustach Druina.

Spojrzenie Fergala przeskakiwało pomiędzy nami, jego język nieświadomie
dotykał jego ust. Cofnął się nieco.

-- Odłóżcie broń -- powiedział, potem dodał, gdy obniżyliśmy karabiny -- całą.

Gdy odpiąłem pas, spojrzałem na Druina. Pokręcił głową, prawie
niezauważalnie. Położyłem nóż, pistolet i~multi-narzędzie koło karabinu.

-- Także \textit{sgean dhu}.

Czułem się nagi, gdy się wyprostowałem. Szybkie ręce przesunęły się lub
poklepały po ciele.

-- Czyści.

Fergal podniósł mój sprzęt, a~jeden z~pozostałych majsterków podniósł
rzeczy Druina. Fergal wskazał policzkiem wyjście i~przesunął się dookoła
za nas.

-- Tędy.

Poszliśmy do wyjścia korytarza. Za nim była otwarta przestrzeń starej
elektrowni. Zeszliśmy po niskich schodach na betonową podłogą i~powiedziano nam, żebyśmy się zatrzymali. Za nami mogłem usłyszeć
szeptaną konsultację. Czekaliśmy na decyzję, ręce na głowach, a~ja się
rozglądałem. Turbina, oczywiście, już dawno była usunięta, jak również
większość oryginalnego wystroju. To, co zostało, to było nawiedzające
wspomnienie zapachów, odłażącej farby, zardzewiałego metalu i~antycznej
roboty murarskiej. Ponad tymi zapachami unosiły się nowsze zapachy
betonu i~cyny. Cały wielki, prostopadłościenny budynek, z~wysokimi
oknami, został zamieniony w~kompleks fabryczny pełen warsztatów i~chodników, głośny i~jasny od piszczenia i~iskier robót ślusarskich. Z~liczby osób, które zauważyłem przy ich ławkach, lub pośpiesznie idących,
domyśliłem się, że około setki majsterków pracowało w~budynku.

Tutaj poczułem się dziwnie na pewniejszym gruncie, pośród tej liczby
zajętych osób i~silniejszy poprzez drogę i~kolej cywilizacji.
Wiedziałem, że ten spokój jest iluzoryczny, ale i~tak się go trzymałem.
Myśl zawołania o~pomoc przeszła mi przez głowę. Potem pomyślałem, że
Fergal i~jego towarzysze nie byliby tak odważni, gdyby ich działania nie
były wiadome innym.

Nagle majsterkowie zbiegli po schodach za nami i~zostaliśmy odepchnięci
szorstko, w~przeciwnych kierunkach. Usłyszałem trzaśnięcie drzwi po
drugiej stronie schodów, tuż przed tym, gdy zostałem wepchnięty w~inne.

Pokój, w~który się potknąłem, miał kilka metrów kwadratowych, światło
nad głową, stół i~kilka krzeseł. Pod ścianami ciężkie stosy miedzianych
rur, zwoje kabli, torby i~tak dalej sugerowały, że ten pokój był jednym
z tych, które obecnie nie miały określonego użycia i~był używany
obojętnie jako skład, miejsce spotkań i, teraz, pokój przesłuchań. Był
nawet, tak jakoś nieuniknienie, zlew, elektryczny czajnik i~brzydko
otwarte torebki kawy, cukru i~herbaty.

Fergal wszedł koło mnie, obrócił krzesło na miejsce po przeciwnej
stronie stołu i~wskazał mi drugie.

-- Usiądź.

Położył bronie, które mi zabrał na ociekaczu, cały czas zachowując swój
karabin wycelowany we mnie. Potem usiadł, nie na stole, ale odchylając
krzesło o~najdalszą ścianę i~tuląc czarny karabin z~jego dziwnym,
zakrzywionym magazynkiem.

-- Ok, człowieku -- powiedział. -- Wygląda, że cię nie doceniłem, Clovis. -- Pozwoliłem pochlebstwu minąć. Zabujał się na krześle do przodu, patrząc
na mnie intensywnie. -- Wpakowałeś się w~małe kłopoty -- kontynuował
poufnym tonem -- i~inni są na Ciebie wkurzeni, ale myślę, że mogę się z~nimi dogadać. Możemy to rozwiązać.

Nic nie powiedziałem.

-- Wiesz, czym jest Druin?

Po chwili czekania na odpowiedź, kontynuował: 

-- On jest szpiegiem
kierownictwa, otóż to. Pracuje dla komitetu bezpieczeństwa MTN w~Kishorn. Donosi na aktywistów związkowych, pomiędzy innymi rzeczami.

Fergal powiedział to takim tonem odrazy, że byłem zaskoczony. Drobne
kłopoty pomiędzy związkami, wykonawcami i~podwykonawcami nie wydawały mi
się kwestią takiego moralnego oburzenia, nie mówiąc już o~grożeniu
śmiercią. Złożyłem ręce i~przechyliłem głowę lekko na bok. Fergal
odchylił się znowu do tyłu.

-- Naciskał, żeby cię wyrzucić, wiesz -- powiedział. -- To dlatego był w~barze w~Karonadzie.

Przyznaję, że poczułem się tym lekko wstrząśnięty, ponieważ było to
całkowicie rozsądnie i~ponieważ to wskazywało, że ktoś w~barze nas
obserwował, ale ciągle nie odpowiedziałem.

-- Nie przyszedł do nas tutaj, z~Tobą, żeby nas szpiegować. Jest tutaj,
żeby szpiegować \textit{Ciebie}, odkryć, jakie są Twoje prawdziwe
powiązania z~nami.

-- Jeżeli to właśnie robi, to brzmi dla mnie wystarczająco rozsądnie -- powiedziałem, w~końcu pobudzony. -- Tak czy inaczej, jestem pewien, że
nic, co robisz, nie jest zagrożeniem dla projektu. Przede wszystkim to
dlatego pomogłem Merrial. Więc jaki jest problem jego tutaj obecności?

-- Och, to nie ma nic wspólnego z~tym. Merrial powiedziała ci prawdę,
myślimy, że istnieje możliwe zagrożenie dla Statku, prowadzimy pilnie
śledztwo i~jeżeli odkryjemy dowody na to, przedstawimy je zarządowi
projektu. Nie. Druin, i~ktokolwiek jest za nim, szukają dowolnego kija,
żeby uderzyć majsterków. Próbuje nas skompromitować i~wzniecić wrogość
wobec nas.

Pokręciłem głową. 

-- Nie, nigdy nie okazał żadnej wrogości wobec
majsterków, o~ile wiem.

-- Naturalnie -- powiedział szyderczo Fergal.

-- Zresztą, dlaczego on, lub ktokolwiek, chciałby czegoś takiego?

-- Boże, ale jesteś kurwa naiwny! -- Fergal machnął ręką, żeby wskazać
wszystkich poza pokojem i~w budynku. -- Jesteśmy nieco uprzywilejowaną
grupą, poprzez nasz monopol umiejętności, które, szczerze, nie są trudne
do opanowania. Dlaczego mielibyście zależeć od nas przy budowie i~uruchamianiu komputerów? -- Roześmiał się. -- Widziałeś, jak je robimy. To
starożytna technologia, zwana \textit{nanotech}. Nie rozumiemy jej, ale
umiemy ją stosować. Rolnik mógłby to robić, tak jak rolnik uprawia
rośliny bez rozumienia, jak działa genetyka molekularna i~replikacja.
Kompetentny mechanik, może z~pomocą wprawnego jubilera lub zegarmistrza
dla skomplikowanych części, mógłby włączyć kryształy widzenia, jak je
nazywacie, w~maszyny.

-- Musieliby znać białą logikę.

-- To też nie jest trudne do nauczenia. Więc co cię powstrzymuje?

-- Mnie?

-- Twoich \textit{ludzi} -- powiedział zniecierpliwiony.

-- Zabawne -- powiedziałem -- to właśnie pytanie zadałem Druinowi.
Powiedział, że to była, cóż, tradycja, tak pewnie byś to nazwał. Działa,
powstała w~czasach Wyzwolenia, nie ma powodu jej kwestionować. Tak
właśnie powiedział.

-- Bez wątpienia. I~nie minęłoby długo, zanimby cię nie komplementował,
mówiąc, że przemyślał to i~że jest to dobre pytanie.

-- Masz coś przeciwko, żebym zapalił? -- spytałem. Chciałem, żeby odniósł
wrażenie osłabienia, moje pragnienie uwiarygodniło to.

-- Jasne, proszę bardzo -- powiedział Fergal.

Wyjąłem składniki z~kieszeni i~zapaliłem.

-- To, czego nie rozumiem -- powiedziałem -- to, dlaczego jesteś tak
przejęty jego pojawieniem tutaj. Nawet zagroziłeś mu śmiercią. Może to
był blef\ldots

-- Nie był!

-- Ale dlaczego? Nawet jeżeli jest tak wrogi, jak mówisz, na pewno ma
ludzi, którzy go poszukają, gdy nie wróci, i~dużo to nie zabierze, żeby
zaczęli szukać tutaj.

Fergal pstryknął palcami. 

-- Moglibyśmy sprawić, że to wygląda jak
wypadek, który nie ma nic wspólnego z~nami. Polowanie to niebezpieczny
sport.

-- A ja potwierdziłbym historię lub dołączył do niego na dnie klifu?

-- Coś w~tym rodzaju.

-- Co -- spytałem, próbując utrzymać głos od zdradzenia mojego gniewu i~strachu -- jest na tyle ważne, że usprawiedliwić zrobienie \textit{czegoś
takiego}, co?

-- Ach. -- Fergal zmarszczył brwi. -- On, i~Ty, przybyliście w~bardzo
dziwnej chwili. Odnaleźliśmy coś w~plikach, które Merrial odzyskała,
coś, co było zaginione przez bardzo długi czas, i~co dopiero ostatnio
zrozumieliśmy, że może być przechowywane na Uniwersytecie, z~wszystkich
miejsc. My\ldots

Przerwał. 

-- Powiedzmy, że stracilibyśmy dużo, gdy ktoś zaczął w~tym
grzebać. Oczywiście trwa śledztwo i~naprawdę nie jesteśmy w~pozycji
oporu wobec siłowego wtargnięcia. -- Otrzepał palce i~wstał, kładąc
karabin ostrożnie w~poprzek zlewu, w~swoim zasięgu i~poza moim. -- W~tym
miejscu wchodzisz Ty, Clovis. Oczywiście nie chcemy zabić Druina lub
Ciebie.

-- Jeżeli istnieje możliwość tego uniknięcia.

-- Dokładnie! -- Uśmiechnął się, tłumiąc uśmiech. -- Nie ma potrzeby czegoś
takiego. Jesteś inteligentnym facetem, Clovis, i~możesz nam pomóc.
Wszystko, co musisz zrobić, to przekonać Druina, że nie ma nic tutaj, co
zagroziłoby projektowi, i~że powinien zostawić to w~spokoju.

-- To nie powinno być trudne -- powiedziałem. -- A Druin nie powinien was
martwić. Nawet jeżeli jest tym, co mówisz, wykonywał tylko swoją pracę.
A mówiąc o~pracy, właśnie straciłem swoją i~chcę wyjaśnień. Jak również
dokumentów, które zabrałeś, i~możliwości rozmowy z~Merrial.

Fergal zmrużył oczy. 

-- Merrial może nie chcieć z~Tobą rozmawiać.

-- To ona mi powinna to powiedzieć.

-- Co do plików\ldots

Zmarszczył brwi, zastanawiając się. Miałem wrażenie, że zaczyna myśleć,
że pliki stawały się większym kłopotem niż były tego warte.

-- Słuchaj -- powiedziałem -- rozumiem, dlaczego uważacie, że one są wasze.
Niemniej nie są moje, żeby wam pozwolić je mieć, lub wasze, żeby je
wziąć. Wyzwolicielka zostawiła je Uniwersytetowi, nie Czwartej
Międzynarodówce.

Fergal zerwał się, jakby użądliła go osa.

-- Kto Ci powiedział o~Czwartej Międzynarodówce?

Wzruszyłem ramionami. 

-- Jestem historykiem -- powiedziałem. -- To wiedza
powszechna pomiędzy uczonymi.

To podwójne kłamstwo osłabiło jakoś Fergala. Usiadł z~powrotem i~ostrożnie zmierzył mnie wzrokiem.

-- Więc co wiecie o~tym?

-- To komunistyczne tajne społeczeństwo, które powstało przed czasami
Wyzwolicielki.

-- Hmm -- powiedział. Potarł powiekę. -- To w~porządku. Choć
,,komunistyczne'' wcale nie mówi ci, o~co chodziło, w~tych dniach. -- Roześmiał się zgryźliwie. -- Boże, czasem czuję, że moglibyśmy odzyskać
\textit{kapitalizm}\ldots

-- Posiadanie? -- spytałem z~niedowierzaniem.

-- Cóż, tak byś to nazwał. Pozwól, że powiem, że byłoby to lepsze niż ten
ciemny wiek, w~którym wy ludzie sami siebie zakopaliście.

-- To jest ciemny wiek? -- Roześmiałem się w~jego twarz. -- Budujemy statek
kosmiczny nie dalej jak pięćdziesiąt kilometrów stąd.

-- Och, Chryste. -- Fergal zacisnął pięści. -- Aye, budujecie to z~płyty od
kotła. Budujecie wszystko, od surowej atomistyki do nawet, kurwa,
\textit{silników fuzji laserowej }umiejętnościami przekazywanymi od
mistrza do ucznia. W~porównaniu, ze starożytnymi, wy jesteście
całkowitymi barbarzyńcami. W~porównaniu z~tym, czym moglibyście być\ldots

Westchnął, wstał, zaczął chodzić po pokoju jak bestia w~klatce. 

-- Moglibyście mieć świat, gdzie nikt nie musi wykonywać pracy, która nie
jest zabawą, gdzie prawie każda choroba lub rana może być wyleczona,
gdzie nikt nie musi umierać, gdzie żyjemy jak bogowie i~wypełniamy
niebiosa naszymi wnukami. Zamiast tego mamy \textit{to}. -- Uderzył w~dłoń
pięścią i~rozejrzał się dookoła z~obrzydzeniem.

-- A kto pracowałby w~tym raju? -- spytałem, może bardziej agresywnie niż
zamierzałem.

-- Maszyny, oczywiście. Każda praca na świecie może być wykonana przez
maszyny, połączone i~skoordynowane.

-- Och, prawda -- powiedziałem rozczarowany. -- Ścieżka mocy.

-- Nie musi tak być, następnym razem\ldots

-- \textit{Następnym }razem?

Fergal pochylił się nad stołem oparty na pięściach, na sposób
jednocześnie onieśmielający i~poufny. 

-- Po to właśnie istnieje
Międzynarodówka: po następny raz. Następna szansa, by ludzkość wyrwała
się z~tego więzienia. Nasz czas nadejdzie, znowu. A następnym razem,
będziemy gotowi.

Pokręciłem głową. 

-- Nie rozumiem.

Spojrzał na mnie z~jakimś żalem, potem wyprostował się i~wrócił na
krzesło. 

-- Nie ma sensu próbowanie wytłumaczenia tego ci teraz -- powiedział. -- Jest tyle rzeczy, które musisz wiedzieć, żeby zrozumieć,
nie ma sposobu, żebyś\ldots

Uderzenia w~drzwi przerwały mu.

-- Kto tam? -- krzyknął.

-- To ja\ldots Merrial! Fergal, musisz\ldots

-- Poczekaj tam!

Jego wykrzyczana komenda dotarła zbyt późno. Drzwi otworzyły się z~trzaskiem i~Merrial wparowała. Pognała obok mnie, położyła coś na stole,
a potem wyrwała ręce do tyłu od tego, jakby to był zbyt gorący talerz.
To była aparatura kryształu widzenia, kamień w~środku jarzył się kolorem
i życiem ruchem, formując małe sceny pod kopułą, bańka życia zaskakująca
w jej rzeczywistości wirtualnej.

Sceną była polana w~lesie, na której mężczyzna siedział niczym elf na
skale. Patrzył na nas, całkiem spokojny i~niesamowity. Przemówił, a~jego
głos dobiegł z~głośnika z~boku otaczającego aparatu. Poziom głosu był
zbyt niski, żeby zrozumieć, co mówił, zdecydowanie nie powyżej krzyków
Merrial.

-- Nigdy nie mówiłeś mi, że jest w~tym diabeł!

Fergal zerwał się i~patrzył się intensywnie na kamień. Podniósł dłoń,
bez odrywania wzroku.

-- Uspokój się Merrial -- powiedział miękko. -- To nie diabeł. Tego właśnie
szukamy.

-- Co to, do cholery, jest? -- spytałem. Również wstałem, patrząc w~transie na piękną, niesamowitą rzecz.

-- To sztuczna inteligencja -- powiedział majsterek, jego głos zachwycony
z podziwu. Pochylił się nad kryształem widzenia, przyłożył ucho do
głośników i~słuchał. Merrial wydawała się zauważyć mnie, dopiero gdy
przemówiłem.

-- Co Ty tutaj robisz? -- spytała. Jej oczy były zaczerwienione, jej
policzki blade od zmęczenia. Wyglądała na przestraszoną i~zdziwioną.

-- Przyszedłem tutaj dla Ciebie -- powiedziałem. -- Miałem nadzieję, że
może będziesz chciała wrócić.

-- Ale ja myślałam\ldots

-- Wy dwoje, proszę wyjść -- powiedział Fergal. Nawet na nas nie spojrzał.
Machnął dłonią w~roztargnieniu na stronę. -- Zabierz swoje bronie i~narzędzia, Clovis, zabierz tę kobietę, jeżeli chcesz i~wynoś się, kurwa,
stąd ze swoim przyjacielem, korporacyjnym szpiegiem.

Merrial odwróciła się i~spojrzała na Fergala.

-- Chcesz, żebym poszła? -- Brzmiała urażona, ale również pełna nadziei.

-- Tak, tak -- powiedział Fergal, niecierpliwie racząc spojrzeć na nią. -- Wykonałaś swoją pracę i~do tego bardzo dobrze. Twoje umiejętności nie
będą potrzebne w \ldots kolejnej fazie. Och, Clovis, zabierz te cholerne
papierowe teczki, skoro już przy tym jesteśmy. Też już ich nie będziemy
potrzebować.

Merrial patrzyła się krzywo na Fergala przez chwilę i~chwyciła moją
dłoń.

-- Clovis, co się dzieje?

-- Myślę, że powinniśmy zrobić, tak jak nam mówi. -- powiedziałem.

Puściłem jej dłoń i~okrążyłem stół, wziąłem karabin i~rzeczy z~mojego
pasa. Przypiąłem je z~powrotem, wsunąłem sztylet w~but i~wziąłem dłoń
Merrial w~lewą dłoń, trzymając karabin prawą. Razem wycofaliśmy się z~pokoju. Fergal nie patrzył, jak wychodziliśmy, lub nawet -- o~ile mogłem
zobaczyć -- zauważył. Rozmawiał cicho z~duchem w~kamieniu. Zamknąłem
drzwi pchnięciem stopy.

-- Czy chcesz iść ze mną?

Merrial mrugnęła. 

-- Oczywiście, że tak.

Przytuliłem ją (raczej dziwnie z~karabinem w~dłoni, ale nie miałem
zamiaru go znowu wypuścić) i~potem powiedziałem: 

-- Lepiej stąd ruszajmy,
zanim ten gnojek zmieni zdanie.

-- Lub coś gorszego się wydarzy. Tak, chodźmy.

Wielka przestrzeń warsztatowa ciągle była zajęta, ze światłami
dochodzącymi stąd i~stamtąd, gdy wieczorne cienie się wydłużały -- godzina, gdy zaskoczony zauważyłem, była tylko dziesiąta wieczorem -- i~otaczające światło poczerwieniało. Kilka ludzi na chodnikach ponad nami
spojrzało na nas z~ciekawością, ale to było wszystko.

Pokój, w~którym Druin był trzymany, był tylko kilka kroków dalej.
Otworzyłem drzwi i~wszedłem, Merrial blisko za mną. Ten pokój miał tylko
krzesło na środku, z~bardzo jasnym światłem nad nim. Druin siedział w~tym krześle ze znudzoną, ponurą i~upartą miną, podczas gdy dwóch
majsterków, którzy towarzyszyli Fergalowi, stało, jeden przed nim, drugi
za nim. Ich podniesione głosy ucichły, gdy weszliśmy. Ich karabiny, i~Druina, były oparte o~dalszą ścianę, mój był skierowany prosto przed
siebie. Ciągle nie był naładowany, ale tego nie wiedzieli.

-- Fergal mówi, żebyście go puścili -- powiedział Merrial.

-- Co oni ci robili? -- spytałem.

Druin wstał i~się rozciągnął. 

-- Och, nic, o~czym warto mówić -- powiedział. -- Zaledwie zanudzali mnie opisem moich grzechów. Nie
odkryłem jednak w~moim sercu chęci spowiedzi. -- Zręcznie odebrał swoją
broń i~sprzęt. -- Dziękuję za odeskortowanie nas na zewnątrz, panowie.

Jeden z~majsterków odezwał się w~końcu. 

-- Chcę to potwierdzić u Fergala.

-- Zrób tak, jeżeli chcesz -- powiedziała Merrial. -- Ale ostrzegam, on nie
jest w~przyjaznym nastroju.

Majsterek otworzył usta i~znowu je zamknął. Uśmiechnął się do Merrial w~zadziwiająco współwinny sposób, że zacząłem podejrzewać, że on i~Merrial
mieli jakieś wspólne doświadczenia nastrojów Fergala. 

-- Och, dobrze, to
Twoja odpowiedzialność -- powiedział.

Wyszliśmy z~pokoju.

-- Poczekajcie chwilę -- powiedziała Merrial.

Pobiegła po schodach i~dalej chodnikiem, jej stopy dzwoniły po metalu.
Czekaliśmy w~niespokojnej ciszy, gdy wróciła, dwie teczki przyciśnięte
do jej piersi.

-- Tylko to -- powiedziała. -- Wszystko gotowe.

Dwóch mężczyzn szło przed nami wzdłuż długiego, centralnego przejścia
przez warsztat do starych, zielonych, miedzianych drzwi budynku, potem
odwrócili się ostro w~lewo i~wyprowadzili nas przez raczej mniej
imponujące drewniane drzwi.

-- Do widzenia -- powiedział Druin złowrogo.

Majsterkowie go zignorowali.

-- Wychodzisz? -- jeden z~nich zapytał się Merrial.

-- Idę do domu -- powiedziała. -- Mam nadzieję, że jeszcze cię zobaczę.

Ciężarówka Druina była około kilometra dalej. Pośpieszyliśmy cichą
drogą, późne słońce w~nasze oczy. Druin szedł raźno przodem. Dłoń
Merrial była zaciśnięta w~mojej, palce splecione. Żadne z~nas nie mówiło
dużo, mieliśmy zbyt dużo do powiedzenia na raz.

W końcu dotarliśmy do ciężarówki. Druin zatrzymał się i~spojrzał na
karabiny.

-- Och, zapomniałem, mamy jakieś sarny do zabicia.

Roześmiał się na moją minę, wziął oba karabiny i~położył je z~tyłu
ciężarówki. Obeszliśmy do kabiny i~wspięliśmy się. Merrial dzieliła
podwójne miejsce pasażerskie z~mną. Było komfortowo zatłoczone. Przez
minutę wszyscy siedzieliśmy z~wdzięcznością. Podałem Merrial papierosa i~zapaliłem dla nas obojga. Pociąg do Kyle zastukał, mijając.

-- Wiesz -- powiedział Druin refleksyjnie -- nigdy wcześniej nie miałem
broni wymierzonej we mnie, dzięki Opatrzności. Nie jest to
doświadczenie, które chciałbym powtórzyć.

-- Nie sądzę, żeby naprawdę chcieli któregokolwiek z~nas zabić -- powiedziałem. -- W~końcu, to my wmaszerowaliśmy z~karabinami.

-- Aye -- powiedział Druin z~oburzeniem -- wiele razy miałem karabin w~Karonadzie i~nikt tego źle nie przyjął.

-- Inna sytuacja\ldots

-- Fergal mógłby was zabić! -- przerwała Merrial. -- Jeżeli byłby w~nastroju. Tylko możliwe konsekwencje go powstrzymały. Zrobiliście tam
coś \textit{głupio} niebezpiecznego.

-- Cóż, poszliśmy tam dla Ciebie i~żeby odzyskać te dokumenty, o~które
Clovis tak się awanturował. -- Druin się uśmiechnął. -- I~z~tym właśnie
wyszliśmy.

-- Co za czarujący sposób powiedzenia tego -- powiedziała Merrial,
nieobrażona. Pochyliłem się koło niej i~spojrzałem na Druina, marszcząc
brwi.

-- Co z~Tobą? Fergal powiedział, że pracujesz dla bezpieczeństwa
projektu, szpiegując związki i~majsterków. I~że żądałeś mojego
zwolnienia. Czy to prawda?

-- Nie \textit{szpieguję} nikogo -- powiedział Druin. -- To sposób
przedstawiania tego przez majsterków, przynajmniej tych trzech, którzy
nas złapali. Diabeł będzie musiał za to zapłacić, wiesz!

-- Jak? -- Zaprzeczenia Druina nie ominęły mnie, ale to było ważniejsze.

Druin włączył silnik i~zaczął kierować ciężarówkę na drogę na zachód. 

-- Bezprawne pozbawienie wolności! -- powiedział. -- I~napad z~bronią w~ręku,
czym jest grożenie komuś bronią. Ty i~ja, Clovis, moglibyśmy pozwać
gnojków. -- Spojrzał na mnie z~boku ostro. -- Nie wiesz może, przypadkiem,
dlaczego nas zatrzymali w~pierwszej kolejności i~dlaczego puścili nas,
kiedy to zrobili? Mam na myśli, przy mnie tylko gadali o~tym, jakim
jestem łamistrajkiem. Co Ci Fergal powiedział? I, jeżeli o~tym pomyśleć,
co oboje kombinujecie? Wiem, że coś kombinujecie i~że to dotyczy Statku.
Co oznacza, dotyczy mnie.

Objąłem ramieniem ramiona Merrial. Uśmiechnęła się do mnie, potem
spojrzała prosto.

-- Powiedz mu -- powiedziała. -- Powiedz mu wszystko.

Więc tak zrobiłem, gdy zjechaliśmy z~Dark i~jechaliśmy przy zachodzie
słońca.

~

-- Aye, cóż -- powiedział Druin -- powiedziałeś mi wszystko, co wiesz,
Clovis. -- Łyknął whisky i~pstryknął komara. -- Co za historia! Niemniej
jednak nie słyszałem części Merrial i~sądzę, że to więcej niż połowa tej
historii.

Siedzieliśmy dookoła szorstko zrobionego, wygładzonego wiekiem, stołu w~szerokiej, wyłożonej kamieniem kuchni w~domu Druina, sami otoczeni
półkami porcelany, błyszczącym elektrycznym piekarnikiem i~zlewem z~kapiącym kranem. Arrianne i~dzieci już dawno poszły spać. Tylne drzwi
były otwarte na ciepłą noc oraz na zapachy i~dźwięki fiordu. Spodek na
stole wypełniał się niedopałkami papierosów. Poza nim butelka whisky i~dzbanek kawy opróżniały się szybko.

Merrial potarła brew, przebiegła palcami po szerokich połaciach włosów i~odrzuciła je na ramiona. Nie rozwinęła żadnego punktu mojej historii,
poza okazjonalnym potwierdzającym komentarzem lub skinięciem głową.

-- Cóż, dobra -- powiedziała. -- Z~mojej strony, cóż, część z~tego
wolałabym omówić z~Clovisem, to naprawdę jest osobiste, naprawdę nie
jest Twoją sprawą, Druin.

Druin przechylił głowę. 

-- Ok. A reszta?

-- Ach, dobra, to sięga nieco wstecz, kiedy zaczęłam się martwić \ldots
historiami, które słyszałam, o~tym, co stało się w~trakcie Wyzwolenia.
Praktycznie, to Wyzwolicielka, sama Myra Godwin, uruchomiła coś, co
fizycznie zniszczyło kolonie i~satelity, robiąc to, nie tylko zabiła,
Bóg wie, ile osób, ale także stworzyła barierę dla czegokolwiek
kiedykolwiek wyruszającego w~kosmos. Każda platforma orbitująca, która
była zniszczona, została rozbita na szybko poruszające się fragmenty,
które w~rezultacie zniszczyły inne, i~tak dalej aż nie było nic, prócz
pasa odłamków dookoła Ziemi, a~cokolwiek, co się wzniesie teraz, skończy
jako więcej odłamków! Fergal jest bardzo szanowanym majsterkiem, prócz
jego bycia \ldots czołowym przedstawicielem Międzynarodówki. -- Rzuciła na
nas okiem. -- Co nie jest tak złowieszcze, jak myślicie! Ale to przy
okazji. Fergal kieruje majsterkami, którzy pracują przy Projekcie, choć
sam nie pracuje na miejscu. Więc po zignorowaniu przez kierownictwo
projektu, przedstawiłam to jemu i~on powiedział, że powinniśmy spróbować
zbadać to sami. To ja zasugerowałam, że moglibyśmy poszukać kogoś, kto
mógłby mieć dostęp do czegokolwiek, co Wyzwolicielka zostawiła w~Glasgow, i~że, cóż, przez lato w~Projekcie pracowali studenci, którzy
mogliby\ldots

-- Zatem przyszłaś szukać mnie?

-- Aye. -- Uśmiechnęła się. -- Ale nie wiedziałam, co znajdę. Mógł to być
ktoś, tylko zainteresowany w~stypendium, lub ktoś, kto nie zgodziłby się
z pomysłem. Tak czy inaczej, miałam uszy otwarte i~nie minęło długo,
zanim usłyszałam o~Tobie.

Druin się roześmiał, nie tyle moim zażenowaniem, ile jej opowieścią.

-- Clovis nie był taki cichy o~jego zainteresowaniach! Truł nam o~Wyzwolicielce i~historii całe cholerne lato. Jednak wracając do Fergala.
Brzmi, jakby potraktował Twoje obawy poważnie.

-- Och, pewnie -- powiedziała Merrial. -- Odniosłam wrażenie, że całkiem
sporo majsterków miało ten sam pomysł i~\ldots przynajmniej niektórzy
ludzie w~Międzynarodówce mieli poważniejsze powody, żeby tak myśleć.

Druin wziął nagle rozrzutny łyk jego dobrej whisky.

-- Dlaczego majsterkowie, lub ta Międzynarodówka, chciałyby to utrzymać w~tajemnicy?

Merrial gapiła się na niego. 

-- Z~powodu reputacji Wyzwolicielki oraz jej
ostatniej wiadomości do świata, która chroni majsterków! Jeżeli zwykli
ludzie, Obcy, bez urazy, zaczną myśleć, że była jakimś masowym mordercą
i potworem jak Stalin, to dlaczego mieliby się przejmować tym, co
powiedziała?

Druin oparł brodę o~dłonie i~spojrzał na nią zagadkowo.

-- Czy to jest coś, co myślisz, czy raczej to, co Fergal ci powiedział?

-- To i~to, ale, cóż, tak. Widzę, co masz na myśli.

-- Więcej niż mogę powiedzieć -- powiedziałem.

Merrial odwróciła się do mnie. 

-- Co on ma na myśli, to, to coś, co
akceptowałam tak długo, jak pamiętam, bez myślenia o~tym, ale kiedy
mówisz to na głos i~myślisz o~tym, to po prostu nie wydaje się bardzo
prawdopodobne.

-- Dokładnie! -- powiedział Druin. -- To prawda do pewnego stopnia, ale
właściwie nie wyjaśnia, dlaczego majsterkowie i~reszta z~nas radzi sobie
całkiem dobrze przez większość czasu. Historia, że są dziećmi
Wyzwolicielki, jak to się mówi, jest tylko symbolem, drogowskazem lub
punktem zwrotnym, jak sama statua. Nie dogadujemy się z~majsterkami,
ponieważ szanujemy Wyzwolicielkę, ale szanujemy Wyzwolicielkę i~utrzymujemy jej posągi, ponieważ dogadujemy się z~majsterkami. A dzieje
się tak, ponieważ potrzebujemy majsterków, a~oni nas.

Spojrzałem na niego, zdumiony. Przez wszystkie lata moich studiów, nigdy
nie czytałem lub słyszałem podpowiedzi czegokolwiek jak to. Zdecydowanie
nigdy samemu nie doszedłem do takiej myśli. To coś tak oczywisty
prawdziwego, kiedy wypowiedziane, jednak tak nieoczywistego i~przeciwnego naturze tego, co Gantry nazwałby ,,prostackim żargonem''
wypowiedziane przez robotnika, a~nie uczonego było jakimś szokiem mojej
oceny naukowości, nie wspominając o~sobie.

Nie było sposobu, żebym powiedział to wszystkiego bez zabrzmienia
protekcjonalnie, więc powiedziałem tylko: 

-- Druin, to genialne. Nigdy o~tym nie pomyślałem.

Uśmiechnął się do mnie lekko, ze zmrużonymi oczami, jakby znał moje
niewypowiedziane myśli. 

-- Aye -- powiedział -- genialne, czy nie, jestem
całkiem pewny, że myśl taka pojawiła się u naszego kolegi Fergala. Więc
jego dyskrecja ma inne cele niż to. Jeżeli Ty, Clovis, opublikowałbyś
swoją wielką pracę o~Wyzwolicielce, kiedy byłbyś starszy i~mądrzejszy, w~której udowodniłbyś poza cieniem wątpliwości, że była najdziwniejszą
osobą, która chodziła po Ziemi, myślisz, że ludzie zaczęliby rzucać
kamienia w~majsterków? -- Roześmiał się. -- Nie, rzucaliby kamienie w~Ciebie!

-- Gdzie to nas prowadzi? -- spytałem, nieco obronnie.

-- Prowadzi nas do tego -- powiedział Druin wolno, stukając w~stół
stępionym paznokciem. -- Jak mówiłem, pragnienie dyskrecji Fergal w~tej
kwestii nie jest z~powodu, o~którym myśleliście Ty i~Merrial. W~rzeczywistości, ze sposobu, jak mówisz, zachowywał się, gdy Merrial
znalazła człowieczka w~kamieniu, powiedziałbym, że odnalezienie tej
rzeczy, czymkolwiek jest, cały czas było jego celem. To po to wysłał was
oboje na poszukiwania w~Glaschu. Teraz gdy to znaleźliście, nic go nie
obchodzą jakieś możliwe śmieci kosmiczne. I~nie zapominaj, Merrial, że
przedstawiłaś sprawę Projektowi i~jedynym powodem, dla którego zostałaś
tak mocno skarcona jest to, że \textit{oczywiście} projektanci pomyśleli o~tym, czy to było winą Wyzwolicielki czy nie, rzeczy, które były na
orbicie w~przeszłości, gdzieś musiały trafić! W~starych zapisach, takie
jak są, można było je zauważyć jak poruszające się gwiazdy nagim okiem,
czyż nie tak, Clovis? -- Kiwnąłem głową.

-- Cóż, nie ma ich tam teraz, a~nasze najlepsze teleskopy, co niedużo
mówi, przyznaję, w~porównaniu z~tymi, którymi starożytni obserwowali
narodziny Wszechświata, ale jednak, nie widzą żadnej plamki. I~teraz nie
ma więcej spadających gwiazd niż w~starożytności, wiemy o~tym na pewno,
ponieważ te zapisy były na papierze i~przetrwały. Więc prawdopodobnie
nie ma chmury odłamków nad Ziemią, choć jeżeli Wyzwolicielka zrobiła,
tak jak mówisz, mogą być jakieś materiały na wysokich orbitach. Ale
nawet to jest małoprawdopodobne. Jest powiedziane, że w~czasach
niespokojnych, niebo spadło, a~najlepsi naukowcy sądzą, że to był sposób
naszych przodków powiedzenia, co zobaczyli, kiedy wielkie kosmiczne
miasta, dawno opuszczone lub wypełnione zmarłymi, były w~końcu
sprowadzone na Ziemię przez lekki opór powietrza w~górze i~upadały na
Ziemię z~własnej woli.

W tym czasie, nie byłem zaskoczony przez Druina. Jego słowa były
kolejnymi gwoździami w~trumnie mojej pychy.

-- Czy znalazłaś cokolwiek o~tym w~plikach komputerowych? -- spytałem
Merrial.

Potrząsnęła głową. 

-- Nie, nie było nic, co się dzieje po dacie samego
Wyzwolenia. To wtedy, gdy szukałam przez niego, otworzyłam plik, który
uwolnił coś, co Fergal nazwał ,,sztuczną inteligencją''. -- Jej oczy
rozszerzyły się na wspomnienie. -- Z~początku myślałam, że to była jedna
z tych twarzy, które pojawiają się w~kamieniach.

-- Przy okazji, czym one są? -- spytał Druin.

Merrial machnęła dłonią. 

-- Nie wiemy. Znaleźliśmy odniesienia do rzeczy
nazywanych programami Pomoc i~wydaje się, że właśnie tym są, tak czy
inaczej, wszystkie wypowiadają słowo ,,pomoc''! Myślę, że to jakieś
stare rzeczy, które zostały odziedziczone. Jednak ta rzecz w~ogóle nie
była jedną z~nich. Patrzyła prosto na mnie i~przemówiła.

-- Co powiedziała?

-- ,,Cześć'' -- powiedziała, nienaturalnie głębokim głosem.

Wszyscy się roześmialiśmy.

Zadrżała przesadnie. 

-- Moją następną myślą, kiedy minął mi trochę szok,
było to, że jest to demon bezpieczeństwa, jak ten, na którego
natrafiliśmy w~Glasgow. Ale to też nie było to. Nie odstraszało mnie,
zapraszało mnie. To wtedy pobiegłam z~tym do Fergala.

-- Który, zdaje się, zaakceptował zaproszenie -- powiedziałem. -- Stracił
zainteresowanie we wszystkim innym, gdy tylko to zobaczył.

-- Hmm -- powiedział Druin. Wstał, wyszedł przez drzwi, może, żeby uciec
od naszego dymu. Niebo, godzina po północy, było ciągle jasne, lub znowu
jaśniejsze, za nim. -- Co raczej sugeruje mi, że to było jego celem cały
czas. Jak w, dlaczego by nie miało być? -- Odwrócił się do nas, jego oczy
błyszczały. -- Kto nie chciałby porozmawiać ze sztuczną inteligencją?
Starożytnie je mieli i~nawet majsterkowie je stracili, mam rację,
Merrial?

-- Och, pewnie -- powiedziała. -- Nigdy nie widziałam, ani nie słyszałam,
posiadania czegoś takiego i~myślę\ldots myślę, że usłyszałabym.

-- Wiesz -- powiedział Druin -- to naprawdę ulga. Dobra, oboje zostaliście
użyci przez Fergala, może posłani przez nieprzyjemności i~udręki, ale
nie wyszła z~tego duża szkoda. I~nie, Clovis, nie zaliczam Twoich
drobnych trudności do wielkiej szkody, będziesz miał większe kłopoty niż
to, zanim będziesz w~moim wieku.

-- Dobra -- powiedziałem, powstrzymując pewną irytację -- widzę, jak to
może nie być dla Ciebie ważne. Jednak Fergal dostał tę rzecz i~co mnie
martwi to, co zamierza z~tym zrobić.

-- Co zamierza z~tym zrobić -- powiedział Druin -- zależy od tego, co to
jest. Jakieś pomysły tutaj, Merrial?

-- Nie -- powiedziała. -- To było w~plikach Myry Godwin, a~my wiemy, że
niektórzy ludzie mieli wtedy te rzeczy, to mógł być doradca lub
konsultant jakiegoś rodzaju. Może Fergal wie, co to jest, ale nie ja.

-- Nienawidzę myśli, co Fergal mógłby zrobić z~doradcą, który ma dostęp
do wiedzy z~przeszłości -- powiedziałem. Druin pokręcił głową.

-- Więc co jeśli Fergal znalazł nową zabawkę, lub nowego przyjaciela, z~tego, co wiem. To nie jest nasz cholerny interes i~zdecydowanie nie jest
mój, to nie ma nic wspólnego z~bezpieczeństwem Statku, prawda?

-- Cholernie szybko poradziłeś sobie z~irytacją z~bycia przetrzymywanym i~rozbrojonym -- powiedziałem kwaśno.

-- Ach! -- powiedział Druin. -- Gorące słowa. Zapomnij. Zresztą, kto
pozwałby majsterka?

Na to, Merrial i~ja się roześmialiśmy. Bezowocność ,,pozwania majsterka
przed sąd'' była przysłowiowa.

-- Jednak to wcale nie rozwiązuje problemu -- powiedziała Merrial.

-- Jakiego problemu?

-- Problemem nie jest sama rzecz w~sobie. Fergal jest problemem. -- Zmarszczyła brwi, najwyraźniej zmartwiona. -- Nie jest zły, jego intencje
są dobre, w~pewien sposób, i~może być bardzo\ldots czarującym mężczyzną, na
poziomie osobistym, ale jest bardzo\ldots prostolinijny, wiecie? Ma
tendencję do skupiania się na jednej rzeczy w~danych czasie i~tratowania
wszystkiego i~wszystkich innych.

Druin parsknął. 

-- Ha! Nie znam Fergala, ale znam ten typ. Bardziej przez
sławę niż doświadczenie, dzięki Opatrzności. -- Zachichotał. -- Ale jeżeli
kiedykolwiek trafię na takiego kierownika, to będzie miał krótką
karierę. Przynajmniej jako kierownik. -- Wszedł i~znowu usiadł. -- Ale
jednak, to problem dla waszej grupy, nie mojej. Ciągle uważam, że
najlepiej byłoby to zostawić. Projekt jest strasznie blisko zakończenia,
jesteśmy właściwie szybciej niż harmonogram, nadchodzą duże nagrody za
wyprowadzenie platformy przed końcem Sierpnia, co jest różnicą pomiędzy
wyprowadzeniem jej przed zimą a~czekaniem na wiosnę. To nie mała rzecz,
a problemy z~majsterkami są jedną z~rzeczy, które mogłyby to zepsuć.

-- Co martwi mnie w~Fergalu -- powiedziałem -- to nie jego osobowość, a~jego przekonania. Wiem, że nie jesteś taką osobą, Merrial, ale komunizm
jest notorycznie podatny na postaci, które są\ldots które mogą okręcić to w~rację do zrobienia czego, co chcieliby i~tak zrobić, czym jest życie
poza przymierzem.

-- Co masz na myśli, mówiąc ,,przymierze''? -- spytał Druin.

-- Och, to, co Ty powiedziałeś, kiedy Fergal wydawał się grozić zabiciem
Ciebie. Krew za krew, śmierć za śmierć, to jest przymierze, skała. Lub
to, co powiedziałeś o~nas i~majsterkach, muszących żyć razem, ta sama
rzecz, po stronie żyjących.

-- Fergal czasem mówi takie rzeczy jak ta. -- Nagle wtrąciła się Merrial.
-- Że, ten i~ten powinni być zastrzeleni, czy cokolwiek. Nie mówi tego na
poważnie, to tylko słowa na gorąco, jak powiedział Druin.

Druin pojednawczo machnął dłonią. 

-- To, co oboje mówicie, może być
wystarczająco prawdziwe -- powiedział łagodnie. -- Przymierze jest w~naszych czasach silne, z~powodów, och, wszyscy znamy powody! Zatem
ludzie jak Fergal mogą gadać i~bredzić, ale nie mogą wyrządzić krzywdy.
Jak wielu majsterków poszłoby za jego przykładem, w~przeciwieństwie do,
powiedzmy, szanowania go jako człowieka i~inżyniera?

-- Niewielu -- powiedziała Merrial ostrożnie.

Druin oparł się i~wziął łyk whisky, potem dolał nam kawy.

-- Cóż, proszę bardzo -- powiedział zrelaksowanym i~wylewnym tonem. -- Jak
mówiłem, to nie moje sprawy. -- Pochylił się do przodu, bardziej
skoncentrowana mina, patrząc na nas oboje. -- A co do moich interesów,
Fergal i~jego dwaj pomocnicy mieli rację w~jednej sprawie, mam miejsce w~komitecie bezpieczeństwa Stoczni. Nie jestem szpiegiem, zostałem tam
powołany przez Związek, do cholery! I~naciskałem na odwołanie Twoich
uprawnień, Clovis. Co jeszcze mógłbym zrobić, na podstawie informacji,
które miałem? Ale mogę równie dobrze naciskać na ich przywrócenie i~będę. Wrócisz do pracy w~ciągu dnia lub dwóch, jeżeli tego chcesz,
cokolwiek Uniwersytet zdecyduje w~Twojej sprawie.

-- To\ldots -- pokręciłem głową -- \ldots to wspaniale, tego właśnie chciałem.
Dzięki.

-- Ale zanim zwrócisz te teczki do Uniwersytetu, przejrzyj je, spróbuj
dowiedzieć się, czy \textit{jest }tam coś o~tym, co zdarzyło się w~Wyzwoleniu. Lub cokolwiek o~tej sztucznej inteligencji. Powiesz mi, co
odkryłeś, nawet jeżeli to nic, tylko dla spokoju umysłu. Wykorzystaj
dobrze te kilka dni, Ty i~Merrial. -- Uśmiechnął się chytrze. -- Nie muszę
wam mówić, że dotyczy to także nocy. A skoro o~tym mowa, idę spać. A w~międzyczasie, nikomu ani słowa o~tym. Zachowajmy pokój z~majsterkami i~uda nam się ruszyć w~drogę.

-- Droga do gwiazd -- powiedziałem, cytując Fergala.

-- Tak. Wszyscy zadowoleni?

Szliśmy do domu Merrial i~w drodze rozmawialiśmy.

-- Myślałam -- powiedziała -- że byłeś zbyt zaangażowany w~Twoją historię,
badania i~stare dokumenty, żeby chcieć zostać ze mną. O to się
zdenerwowałam, a~nie Twoimi pytaniami.

-- Ach. -- powiedziałem. -- A ja myślałem, że jesteś zbyt zaangażowana w~tajniki Twojego społeczeństwa, żeby mi zaufać.

-- Ach. -- powiedzieliśmy oboje jednocześnie.

Powiedziałem jej, co Druin mówił, o~metodach rekrutacji majsterków.

Roześmiała się, wieszając się na moim ramieniu i~kołysząc się na nim,
patrząc na mnie i~odwracając wzrok, chichocząc.

-- To prawda! -- powiedziała. -- To nie to, co planowałam.

-- Więc Ty\ldots

-- Zakochałam się i~miałam nadzieję, że dołączysz, tak.

-- Ha ha ha! Zostać majsterkiem!

-- Dobra, dlaczego nie?

Obróciła się dookoła, złapała oba moje łokcie i~spojrzała mi prosto w~oczy.

-- Dlaczego nie? -- powtórzyła.

Pomyślałem o~tym, co widziałem, poczułem -- i~wąchałem -- w~bibliotece,
kiedy poszedłem tam z~Merrial i~pomyślałem o~tym, co widziałem w~starej
elektrowni. To była historia, to była prawdziwa sprawa, nie martwa, a~żyjąca, ciągłość z~przeszłością i~zadatek przyszłości, rzeczywiście
droga do gwiazd. Jednak kto powiedziałby, że to rozważania zaważyły, a~nie widok Merrial pod gwiazdami, w~drodze do łóżka, które mógłbym z~nią
dzielić przez wszystkie noce mojego życia?

Nie ja, na pewno.

-- Dlaczego nie -- powiedziałem. -- Tak.


\chapter{Ciemna Wyspa}

Taksówka wynajęta przez kazachstański konsulat, żeby zabrać Myrę z~Portu
Lotniczego w~Glasgow, nadjeżdżając z~zachodu na M8, została ostrzelana
małokalibrowym ogniem, gdy właśnie zjeżdżała z~wiaduktu w~Kinning Park.

Myra ujrzała białe, gwiaździste znaki dziobiące przydymione, pancerne
szkło \textit{tk-tk-tk}, usłyszała opony \textit{łiii} przyśpieszenia. Jej
dłoń odruchowo ruszyła się do kabury pod pachą pod płaszczem i~została
złapana pasami bezpieczeństwa. Przez chwilę, gdy patrzyła w~dół na jej
ostatnio, znów gładką i~nagle białą dłoń, myślała, że śmierć ją w~końcu
odnalazła, że umrze staro i~zostawi dobrze wyglądający trup.

Potem byli poza tym, gładko dalej, zakręcając w~górę i~na Kingston
Bridge nad Clyde. Myra obróciła się i~spojrzała do tyłu i~na lewo, gdzie
standardowa taktyka zasłony dymnej z~palonych opon wznosiła się gdzieś
pomiędzy biurowcami i~wieżowcami w~jasnoniebieskie poranne niebo późnego
maja. Helikopter zaryczał nisko i~szybko nad autostradą, kołysząc
wielkim samochodem i~poleciał prosto na jeden z~wysokich budynków.
Przekątna smuga dziurkowanych kwadratowych dziur nagle przeszyła po
lustrzanym szkle na fasadzie budynku. Helikopter zatrzymał się, unosząc.
Samochód zanurkował pod nawisem mostu i~scena zniknęła z~widoku.

-- Jezu -- powiedziała, wstrząśnięta. -- O co chodziło?

Głośnik w~części za siedzeniem kierowcy się włączył. 

-- Zieloni -- powiedział mężczyzna. -- Czasem strzelają do ruchu z~lotniska. -- Zobaczyła jego zmarszczone brwi w~odbiciu, drżenie głowy. Nie nosił
spiczastej czapki. Nosił hełm. Samochód zwolnił, gdy ruch się zagęścił.

-- Przepraszam za to.

-- Zdaje się, że nie da się nic zrobić -- powiedziała Myra. -- Ale\ldots -- wykorzystała swój najlepszy, amerykański ton ignorantki -- myślałam, że
wy macie to wszystko pod kontrolą. W~każdym razie w~miastach.

Nie, żeby to nazywała miastem, na litość boską, w~\textit{Kapicy} były
wyższe budynki. Nawet ze wzgórzami, Glasgow wyglądał na płaski. Mogła
dojrzeć białą wieżę Uniwersytetu ponad krępymi biurowcami. Miejsce
zmieniło się znacznie od lat siedemdziesiątych, ale nie tyle, ile
oczekiwała, biorąc pod uwagą, przez co przeszło: Republika 2015-2025,
Trzecia Wojna Światowa i~Proces Pokojowy, potem Restauracja i~wojna
guerilla przeciwko Reżimowi Hanowerskiemu, Jesienna Rewolucja i~Nowa
Republika, teraz miasto w~czternastym roku (jak zostało to nieuchronnie
nazwane) walką z~terroryzmem. Niebieska, biała i~zielona trzykolorowa
flaga Zjednoczonej Republiki i~Saltire\footnote{flaga Szkocji w~kształcie
białego krzyża na niebieskim tle,
zob.~\url{https://pl.wikipedia.org/wiki/Flaga\_Szkocji} -- przyp.tłum.} Państwa Szkockiego powiewały z~wszystkich rządowych lub
ważnych budynków.

-- Nie, obawiam się, że nic nie jest w~ogóle pod kontrolą -- mówił
kierowca. -- Teraz są już w~miastach i~nic nie możemy z~tym zrobić. Prócz
bombardowania przedmieść i~jeszcze nie jest tak źle.

-- Tylko wystarczająco źle, żeby ostrzeliwać wieżowce?

-- Tak.

Myra zadrżała i~poprawiła się w~siedzeniu. Jej niezbyt produktywna misja
w Nowym Jorku zabrała mniej czasu, niż planowano, zostawiając jej kilka
dni przed jej wpisanymi spotkaniami z~kimś z~Ministerstwa Spraw
Zagranicznych Zjednoczonej Republiki. Zaczynała żałować, że nostalgia -- i~pragnienie osobistego uporządkowania rozdysponowaniem swoim archiwum -- nie pozwoliła jej zdecydować spędzić tej soboty i~niedzieli w~Glasgow.

Zjednoczona Republika, choć nie jej pierwszy wybór dostępnych
sojuszników, była ciągle najlepszym wyborem po Stanach Zjednoczonych.
Była politycznie przeciwna postępom Szinosowym, ale nigdy nie zrobiła
dużo, żeby je zatrzymać, ponieważ miała zdrową niechęć do wikłania się z~Byłym Związkiem. Z~drugiej strony, dzięki dzielonym interesom w~Spratlejach\footnote{ grupa około 100 małych obiektów fizjogeograficznych
(wysepek, skał, łach i~raf) położonych na Morzu Południowochińskim,
zob.~\url{https://pl.wikipedia.org/wiki/Spratly} -- przyp.tłum.} miała silnego wojskowego i~handlowego partnera w~Wietnamie, który dość dobrze się przeciwstawiał Khmer Vertes, którzy\ldots
po tym to stawało się skomplikowane, ale Parvus znał szczegóły tej
historii. Wynikiem było to, że przy obecnym stanie przy ofercie jako
stabilny sojusznik, Republika mogła być zainteresowana umową, z~atomówkami lub bez.

Taksówka zjechała z~autostrady i~wykonała kilka ostrych skrętów, by
dojechać na południowy koniec St Vincent Street, zwalniając tuż
naprzeciwko New Britain Hotel, gdzie miała rezerwację pokoju.

-- Mały problem\ldots -- powiedział kierowca.

Tłum kilkuset osób był na zewnątrz hotelu, prawie blokując chodnik i~wylewając się na ulice. Składał się z~kilku małych i~najwyraźniej
rywalizujących demonstracji, trzy oddzielne przemowy przez megafony
odbywały się z~niebezpiecznych miejsc na poręczach i~stopniach budynków
obok. Linie Straży Republikańskiej oddzielały grupy. Tylne strony afiszy
poruszały się nad kołyszącymi się głowami.

-- Ach, żaden kłopot -- powiedziała Myra. -- Tylko demo lewicy.

Prawdopodobnie protestujące obecności przedstawiciela jakiegoś
opresyjnego reżimu lub prawdopodobnie niepopularnemu ministrowi rządu
przebywającemu w~New Brit. Gdy wielki samochód wykonał ładne i~nielegalne zawrócenie i~podjechał, żeby się zatrzymać kilka metrów z~lewej strony demonstracji, Myra bezczynnie zastanawiała, z~jakim
przedstawicielem sławy lub hańby będzie dzielić miejsce pobytu.

Kierowca wysiadł -- ze złej strony, jak przez chwilę pomyślała -- obszedł
z tyłu, otwierając po drodze bagażnik i~otworzył drzwi dla niej. Dała mu
dobry widok na długie nogi, gdy je wysuwała i~pojawiła się, wysokie
buty, krótka spódnica, czapka uszanka i~płaszcz. Odmłodzenie
zdecydowanie sprawiało, że jej nogi były znowu warte oglądania. Powinna
przemyśleć swoją garderobę\ldots

Kierowca podniósł jej dwie wielkie walizki z~bagażnika, czekała na
moment,gdy postawił je i~zamknął drzwi z~boku, potem poszła ku wejściu
hotelu, patrząc z~ciekawością na demo, gdy śpieszyła się koło niego.
Było jakieś trzy metry przerwy pomiędzy wystawami sklepów i~kilkoma
Strażnikami Republiki rozstawionymi na chodniku, żeby oddzielić przód
demonstracji. Za Strażnikami tłum skakała, krzyczał i~śpiewał.

Spojrzała na afisz, który machał ponad nią i~zobaczyła w~jego centrum
rozmazane, powiększone zdjęcie jej własnej twarzy z~wiadomości. Nagle
rywalizujące zawołania stały się jasne, jak oddzielne rozmowy na
imprezie.

-- Zwycięstwo ZCRu\footnote{Związku Chińsko-Radzieckiego, w~oryg. SSU -- Sino
Soviet Union -- przyp.tłum.}!

Ten okrzyk walczył na głośność z: 

-- Szinosowy ręce precz! Viva
Kazachstan!

Ponad nimi, nie śpiewany, ale wykrzykiwany wielokrotnie przez jeden z~megafonów: 

-- Wspieraj polityczną rewolucję w~MRRNT!

Konkurujący megafon przedstawiał w~bardziej liberalnym, edukowanym i~edukacyjnym tonie zbrodnie reżimu Myry Godwin. Mijając, usłyszała słowa
,,jądrowi najemnicy'' i~,,haniebny wyzysk''.

Na chwilę Myra się zatrzymała, po prostu stała tam, zbyt zszokowana,
żeby iść. Jej wzrok przesuwał się po odbijających przesłonach
strażników, by nawiązać kontakt wzrokowy z~młodą dziewczyną w~szaliku z~tartana. Śpiew dziewczyny zatrzymał się w~połowie i~Myra nie mogła się
odwrócić od jej niedowierzającej twarzy z~otwartymi ustami. Potem
dziewczyna sięgnęła ponad ramieniem strażnika i~wskazała drżącym palcem
na Myrę.

-- To \textit{ona}! -- zapiszczała. -- Jest tutaj!

Myra uśmiechnęła się do dziewczyny, rozejrzała się i~spokojnie ruszyła
ku stopniom do drzwi hotelu, teraz tylko dziesięć metrów dalej. Kierowca
sapał koło niej. Śpiew trwał, wydawało się, że uchodzi jej na sucho.

A potem rozpostarła się cisza, tylko nieco wolniej niż dźwięk, od
dziewczyny, która ją rozpoznała. Śpiewy zamarły, mowy z~megafonów
zostały przerwane. Tłum rzucił się przez szerokie przerwy pomiędzy
strażnikami, blokując chodnik. Młody mężczyzna, nie tak wysoki jak Myra,
ale ciężej zbudowany stanął przed nią, krzycząc niezrozumiale w~twarz.

Jej stare zrozumienie akcentu Glasgow wróciło z~pamięci.

-- Gardza tobom! -- krzyczał mężczyzna. -- Godałaś, iże je żeś trockistką,
a je żeś gorszo aniżeli jebani staliniści! Przedowanie groźb
nuklearnych, a~potym przedowanie roboty niywolnikow. A teroz walczysz
przeciwko Szinosowym! Som nadziejom świata, a~ty walczysz z~niymi dlo
jebanych Jankesów! Ty jebano zdrajczyni, ty jebano kapitalistyczno
dziwko!

Pochylił się jeszcze bardziej groźnie, gdy mówił. Jego dłonie się
zacisnęły, przygotowywał się, żeby zamierzyć się na nią. Trzy metry za
nim ktoś trzymający afisz ,,Brońmy MRRNT!'' pchał się przez naciskające
ciała. Myra zrobiła krok do tyłu, uderzając w~jedną z~walizek, kierowca
ciągle je trzymał, ciągle za nią. Dobrze.

Wsunęła prawą rękę w~płaszcz. Wrzask krzyczącego mężczyzny i~rozpęd się
zatrzymały. Kolejna cisza rozszerzała się dookoła nich. Myra sięgnęła do
kieszeni nad bijącym sercem i~wyciągnęła jej paszport dyplomatyczny
Kazachstanu. Otworzyła go i~podniosła wysoko, potem pomachała przed
twarzą najbliższego strażnika.

-- Oficerze -- powiedziała bez odwracania -- proszę odprowadzić mojego
kierowcę do hotelu.

-- Oczywiście, proszę pani.

-- Dziękuję!

Kierowca minął ją po lewej otoczony przez mundury. Myra wykorzystała
towarzyszącemu poruszenie rozproszenia uwagi, żeby zanurkować koło
mężczyzny, który na nią krzyczał i~przepchać się do małej grupki
demonstracji pro-MRRNT. Spojrzała szybko na pięć zszokowanych, ale
przyjaznych twarzy, zauważając emaliowe przypinki z~uśmiechem
rozpoznania i~dumy, stary symbol ,,młot, sierp i~czwórka'', przypinka
kampanii solidarnościowej z~podpisem znaku radioaktywności MRRNT,
naklejki ,,słońce i~orzeł''\ldots

-- Towarzysze -- powiedziała -- wejdźmy do środka.

Towarzysze stłoczyli się dookoła niej i~razem weszli na chodnik. Gniewny
mężczyzna był ograniczany przez jego własnych towarzyszy, ale ciągle
potępiał Myrę z~całych sił. Grupa Myry wmaszerowała po schodach i~przez
wielkie drzwi obrotowe hotelu w~zatłoczone foyer. Biała, marmurowa
podłoga, żelaza pomalowane na czarno, żłobkowany mahoń na recepcji i~schodach, dużo kwiatów i~witraży. Milicjanci i~kierowca stali po jednej
stronie, jakiś facet z~kierownictwa hotelu śpieszył z~uprzejmie
zatroskaną miną i~telefonem komórkowym i, patrząc wstecz, zobaczyła, że
wszyscy są w~środku, schody są puste, a~drzwi są zabezpieczane.

-- Chryste -- powiedziała. Teraz była całkowicie wstrząśnięta. Sięgnęła
znowu do płaszcza. Wszyscy zamarli.

Zatrzymała dłoń i~rozejrzała się dookoła, uśmiechnęła się ponuro.

-- Ktoś jeszcze chce papierosa?

~

Żelazne drzwi ewakuacyjne miały sprężynę i~brzęknęłyby, gdyby je
puściła, więc zamknęła je powoli, puszczają je przy brzegu w~ostatniej
chwili.

Zabrzęczały.

Myra spojrzała w~górę i~w dół wyjścia pożarowego i~dookoła podwórza
hotelu. Cieknące rury, stukające kanały wentylacyjne, wilgotne kartony,
mchy, porosty i~bruk. Zeszła w~dół po schodach, prawie cicha w~jej
zniszczonych trampkach, starych dżinsach, swetrze i~ocieplanej kurtce.
Na dole przesunęła opaskę pod szczyt czapki, pod którą zebrała swoje
ciągle szare włosy, wbiła pięści w~głębokie kieszenie, czując
zapewnienie od paszportu i~broni, i~przeszła przez podwórze, kolejną
bramę w~jednym kierunku, wzdłuż alei do Pitt Street, potem w~Sauchiehall.

Zobaczyła swoje odbicie w~witrynie i~parsknęła, że wyglądała na
studentkę. Nie było to doskonałe odbicie, więc również wyglądała
pochlebnie młodo, jakby wyglądała za miesiąc lub dwa, miała taką
nadzieję. I~już widziała prezencję, mogła ją dojrzeć, gdy spojrzała z~boku na odbicie jej kroku, żwawe i~pewne. Jej stawy nie bolały, jej
stopy nie zgrzytały, i~miała tyle energii, że czuła jakby chciała
pobiegać, poskakać lub poskakać, tylko żeby jej trochę spalić. Nie
potrafiła przypomnieć sobie, żeby czuła się tak dobrze, kiedy naprawdę
była młoda.

I rzeczy wracały, wspomnienia jej wcześniejszej osoby, wczesnych taktyk
osobowych przykładowo, zanim się odmłodziła, gdyby została złapana w~taką sytuację jak przed hotelem, zwróciłaby się do Straży po ochronę,
jakby odruchowo, i~bez wątpienia wywołała tam zamieszki. Nie teraz, to
była błyskawiczna kalkulacja, że demonstrujący, choć wrodzy wobec siebie
i milicji, nie zaatakowaliby drobnego służącego jak kierowca i~nie
zaatakowaliby jej, gdy była osłonięta towarzyszami. Bez przemocy w~ruchu
pracowników, bez wrogów na lewicy, nie zawsze to działało, ale ogólnie
rozejm był honorowany. Wzajemnie gwarantowane odstraszanie, może, ale
jednak, co nie było?

Sauchiehall, główna ulica handlowa Glasgow, została pozbawiona pieszych,
od ostatniego razu, gdy tu była i~brzęczała ruchem, elektrycznym w~większości, ale z~kilkoma kaszlącymi, starymi silnikami wewnętrznego
spalania, pędzącymi rowerzystami i, Jezu, tak, pomiędzy nimi
galopującymi końmi. Myra przebiegła czerwone światło na końcu ulicy,
utrzymała tempo, gdy przechodziła kładkę dla pieszych ponad wyjącym
skrzyżowaniem ponad M8 i~potem w~Woodlands Road. Tam zwolniła i~zaczęła
iść, rozkoszując się starą częścią, znajomym terytorium, nostalgia
kłująca w~oczy. (Boże, oblepiła ulotkami ten właśnie słup tego wiaduktu
na seminarium \textit{Critique} w~1976!).

Jednak rejon był teraz szykowny, pełen Sikhów w~garniturach -- bankierów,
prawników i~doktorów -- oraz kobiet w~sari z~towarzyszącymi dziećmi i~często szkockimi nianiami. Chodniki były zatłoczone drogimi, ciężkimi
malajskimi samochodami. Nie jak w~starych czasach, ani trochę, prócz
sporadycznego zapachu curry, poczucia wiatru i~widoku uciekających chmur
ponad nimi.

Rozmawianie z~towarzyszami w~Nowej Brytanii, \textit{to} było jak stare
czasy. To było jak kurwa \textit{podróż w~czasie} i~znacznie bardziej jak
powrót do domu niż jakiekolwiek spotkanie, które miała w~Nowym Jorku.
Gdy podziękowała oficerom milicji, kategorycznie odmówiła na zarzuty
prasy, nalegała na duży napiwek dla kierowcy, oddaliła się do hotelowej
kawiarni na kawę i~papierosa z~pięcioma młodymi osobami, które ją
eskortowały: Day, Alison, Mike, Sandra i~Rashid, wszyscy dumni
członkowie oddziału Glasgow w~Partii Władzy Robotniczej, organizacji,
która mocno cofnęła się od znaku popularności w~latach dwudziestych XXI
wieku w~czasach starej Republiki, ale ciągle walczyła, ciągle
rekrutowała i~ciągle była brytyjską sekcją Czwartej Międzynarodówki.

A oni naprawdę byli młodzi, nie-odmłodzeni starcy jak ona. Z~trudem
mogła to zrozumieć, ponieważ myślała o~Międzynarodówce, już od lat, jako
klubie starzejących się weteranów. Jednak wtedy pomyślała, jak bardzo
formujące i~ekscytującym doświadczeniem ich dzieciństwa była rewolucja -- tak! brytyjska część Jesiennej Rewolucji -- i~jak to mogło dać im idee,
jak prawdziwa (czyli idealna, nigdy naprawdę nie zaistniała) Rewolucja
mogłaby wyglądać.

Szanowali ją, oczywiście, jako starą towarzyszkę, weterana
rewolucjonistkę, która rzeczywiście robiła rewolucję i~rzeczywiście
kierowała państwem robotników. Ale wkrótce stracili rezerwę, może
nieświadomie wprowadzenie w~błąd (jak marzyła) jej coraz bardziej
wiarygodną młodością. Powiedzieli jej więcej szczegółów, niż
potrzebowała wiedzieć o~nieuniknionych zawziętych rywalizacjach, które
stawiały ich naprzeciw siebie, i~reszcie lokalnej lewicy, liberalnych
krytykach jej reżimu lub wrogach komunistów sinosowieckich.

Była wdzięczna za ich wsparcie, oczywiście, i~powiedziała im to, ale
pomyślała o~ich zakorzenionej mocno lewicowej frakcyjności, która ich
oślepiała aż do prawdziwej nienawiści i~moralnej obrazy, którą wzbudzała
i oczywiście do uzasadniania tego. Nie było nic w~diatrybie wściekłego
mężczyzny, czego sama by sobie od czasu do czasu nie powiedziała.

\textit{Ty jebana zdrajczyni, ty jebana kapitalistyczna dziwko}. Tak,
towarzyszu, macie rację. Coś może być w~tym, co mówicie.

W tym samym czasie, odkryła, że towarzysze są nadmiernie troskliwi,
pewni, że byłaby zagrożona, gdyby chodziła sama po Glasgow. Nalegali,
żeby skontaktowała się z~konsulatem i~podróżowała oficjalnie. Myra
zakwestionowała to, wskazując, że było dokładnie to, przez co wpadła na
początku w~kłopoty. Nie powiedziała im, co zamierza robić, jednak
\textit{ktoś} musiał donieść prasie o~jej wcześniejszym niezapowiedzianym
przyjeździe, i~nie miała powodu podejrzewać, że to nie było któreś z~nich.

Minęła stary kościół, St Jude, który ciągle wyglądał na zbyt okazały,
zbyt \textit{katolicki} na małą denominację, której służył, i~naprzeciwko
Halt Bar, gdzie piła z~Davidem Reidem i~Jonem Wilde'm, oddzielnie i~razem, podczas i~po krótkich, intensywnych romansach, które pchnęły ich
życia na ich specyficzne ścieżki.

A tym samym, życia i~śmierci niezliczonych innych. Jon wirtualnie
rozpoczął Ruch Kosmiczny i~założył Kosmicznych Kupców. Reid zbudował
Ochronę Wzajemną, a~Myra -- MRRNT. Wszystko z~małych początków,
nieistotnych wtedy, wszystkie w~końcu wpływające na historię w~skali
zwykle przypisywanych Wielkim Ludziom.

Może gdyby nie oni, pojawiłby się jakiś inny Korsykańczyk\ldots ale nie.
Chaos panował, tutaj jak wszędzie.

Na zielonym moście ponad Kelvin zatrzymała się, patrząc w~dół na brązowe
wody i~białe wiry. Jak trywialne były przyczyny ruchów każdej
cząsteczki, każdego bąbla, który płynął. Nie, to było dziksze niż to,
ponieważ woda przynajmniej była ograniczona brzegami. To było bardziej,
jak cały bieg rzeki mógł być odchylony kamykiem, ziarnkiem piasku,
łodygą trawy, przy źródle, gdzie wielkie siły grawitacji, erozji i~cała
reszta walczyły w~małym stopniu, ale chwilowo z~napięciem
powierzchniowym każdej kropli. Historia była rzeką, gdzie każda kropla
była potencjalnie nowym źródłem, fontanną przyszłych Amazonii.

Poszła dalej, koło fontanny Kelvingrove Park na lewo, a~potem po stromym
stoku Gibson Street, skręciła w~prawo wzdłuż ocienionej drzewami alei do
Instytutu. Zadzwoniła, uśmiechając się krzywo na wypolerowany mosiądz
tabliczki. Kiedyś Instytut Studiów Sowieckich i~Wschodnioeuropejskich,
potem Instytut Studiów Rosyjskich i~Wschodnioeuropejskich, potem \ldots

Instytut Badań nad Społeczeństwami Post-Cywilizacji, jak teraz go
nazywali.

Kobieta, która otworzyła drzwi, wyglądała bardzo wschodnioeuropejsko, w~rozmiarze (drobna) i~minie (podejrzliwa). Jej ciemne oczy lekko się
rozszerzyły.

-- Och, to Ty -- powiedziała. -- Godwin.

-- Tak, dzień dobry. -- Myra wyciągnęła dłoń. Kobieta potrząsnęła ją, z~krótkim oporem, wciągając Myra do środka i~zamykając drzwi w~tym samym
czasie.

-- To miejsce jest obserwowane -- powiedziała. Miała krótkie czarne włosy,
jej wiek trudno było określić. Jej ubrania były tak zużyte, jak Myry:
niebieska, dżinsowa bluza, czarne dżinsy, szare na kolanach. 

-- Nazywam
się Irina Guzulescu. Miło mi cię poznać.

Stały, patrząc na siebie w~wąskim przejściu. Linoleum instytucjonalne,
szara farba, zielony pas, czarne schody. Miejsce pachniało starym
papierem i~dymem papierosowym. Plakaty -- błyszczące odbitki lub wyblakłe
oryginały -- ze Związku Radzieckiego i~Byłego Związku: Lenin, Stalin,
Gorbaczow, Antonow, poważni, i~Gagarin, uśmiechnięty. Oblężenie
Jelcyngradu: bohaterskie dzieci celujące Stingerami w~Sterowce
,,Pamyat''. Budynek był kompletnie cichy i~nikogo nie było w~pobliżu.

-- Raczej spodziewałam się więcej osób tutaj -- powiedziała Myra. -- Zostawiłam wiadomość.

-- Jak mówiłam.

-- Och. -- Myra poczuła się zbita z~tropu i~zirytowana.

-- Twoje skrzynie przyjechały bezpiecznie -- powiedziała Irina, jakby
łagodząc. Poprowadziła ją wąskimi, czarnymi schodami z~balustradą do
biblioteki. Wykładzina na schodach była postrzępiona w~stopniu
kryminalnego zaniedbania. Sama biblioteka była ciasna, labirynt regałów,
przez które trzeba była się poruszać bokiem. Kilka pokoleń technologii
informatycznych było ostrożnie złożonych ponad stołem do czytania.
Skrzynie Myry były ustawione z~boku.

-- Zostawiam Cię temu -- powiedziała Irina.

-- Dzięki.

Myra, sama, zsunęła opaskę, poprawiła parametry, rozejrzała się po
skrzyniach i~westchnęła. Ciągle były zabezpieczone metalową taśmą.
Wysunęła starego Leathermana z~kieszeni i~zaczęła otwierać skrzynie,
zwijając zdradliwie ostre pasy ostrożnie do kosza na papier. Potem
musiała wyciągnąć gwoździe jak zęby. W~końcu była w~stanie wyciągnąć
teczki.

Posortowała papierowe dokumenty w~stosy: jej osobiste rzeczy -- pamiętniki, listy i~tak dalej -- i~polityczne posortowane według czasu i~organizacji, sięgające od lat MRRNT przez wewnętrzne dokumenty fakcji aż
do nowojorskiego oddziału Socjalistycznej Partii Robotników\footnote{ ang.
Socialist Workers Party, SWP -- amerykańska komunistyczna partia
polityczna, SWP publikuje tygodnik \textit{The Militant}, którego początki
sięgają 1928,
zob.~\url{https://pl.wikipedia.org/wiki/Socjalistyczna\_Partia\_Robotnicza}
-- przyp.tłum.} w~latach siedemdziesiątych. Te ostatnie ciągle
wywoływały uśmiech: czy naprawdę był ktoś dostatecznie zwariowany, żeby
wybrać jako jego \textit{nomme de guerre} \footnote{ fr. imię wojenne -- przydomek, przybrane nazwisko artysty,
zob.~\url{https://pl.wikisource.org/wiki/M.\_Arcta\_S\%C5\%82owniczek\_wyraz\%C3\%B3w\_obcych/Nom\_de\_guerre}
-- przyp.tłum.} na debatę o~walce zbrojnej jak ,,dr Ahmed Estraguel''?

Przepracowała podobnie formaty i~konwersje z~Hipokryty przez Drzwi do
Linuxa do Windowsa do DOSa i~przez nośniki pamięci od dysków optycznych,
wafli bąbelkowej pamięci\footnote{
zob.~\url{https://en.wikipedia.org/wiki/Bubble\_memory} -- przyp.tłum.} i~CD-RW (,,CD ścier'', jak były nazywane) do dyskietek,
prawie zrywając się z~siedzenia na dźwięk starego PC, kiedy zaczął
odczytywać pierwszą z~nich. W~cichym budynku, brzmiało to jak pralka na
wirowaniu.

Po około półtora godzinie, która minęła jak w~jakimś transie, wszystkie
jej optyczne i~elektroniczne pliki były skopiowane do elektronicznego
archiwum Instytutu. Wymrugała menu w~opasce i~wywołała Parvusa.

-- Cześć -- powiedziała.

-- Cześć -- powiedział.

Czuła się prawie niezręcznie. 

-- Masz coś przeciwko zrobieniu kopii z~Ciebie i~ściągnięciu jej?

Byt roześmiał się. 

-- Przeciwko? Oczywiście, że nie! Dlaczego miałbym
mieć coś przeciwko?

-- Ok -- powiedziała Myra. Rozwinęła kabel światłowodu z~portu terminala i~wpięła go w~opaskę. -- Chcę, żeby Twoja kopia strzegła zbioru plików \ldots
-- przesunęła podświetlającym palcem po nich -- \ldots i~cokolwiek, co masz
ze sobą teraz, stosując rodzaj uznaniowych kryteriów dostępu, na które
pozwalają Twoje obecne parametry. Skala półtrwania, hm, pięćdziesiąt
lat. Zrozumiałeś?

-- Tak. -- Parvus uśmiechnął się, zdublował się, potem jeden z~nich
dramatycznie zniknął jak animowany dżin z~powrotem do butelki.

-- Zrobione -- powiedział. Zabrało jej to więcej niż zakładała. Musiała
mieć więcej plików na osobistym nośniku danych niż sądziła.

-- Dziękuję -- powiedziała Myra. -- Przy okazji, cokolwiek do zgłoszenia?

Parvus ekspansywnie wzruszył ramionami. 

-- Nic, co nie może poczekać.
Prócz tego, że Lotnisko w~Glasgow jest zamknięte.

-- Co?

Na pewno nie zamach stanu, nie tutaj\ldots

-- Walki na perymetrze. Zniszczenia pasów. Tylko partyzanci Zielonych,
nic poważnego, ale nie ma szansy, żebyś dostała się na samolot w~poniedziałek.

-- O gówno. Zarezerwuj mi pociąg. Na jutro, dobra? Do zobaczenia później.

Odłączyła przewód i~pozwoliła mu się zwinąć. Potem wróciła do pracy
opisywania stosów, datowania teczek i~robienia notatek dla archiwistów
Instytutu.

Ktoś zbiegł po schodach, wszedł do biblioteki i~włączył światło. Myra
odwróciła się ostro i~napotkała zaskoczony wzrok dziewczyny, która ją
zidentyfikowała na demonstracji.

-- Och! -- powiedziała dziewczyna. Powoli zsunęła jej tartanowy szal z~szyi i~zarzuciła długimi, gęstymi, czarnymi włosami spod kołnierza jej
dżinsowej kurtki. -- Co, co tutaj robisz?

Myra się wyprostowała, czując się nieracjonalnie zadowolona, że była
marginalnie wyższa niż młodsza kobieta.

-- Miałam zadać ci to samo pytanie -- powiedziała.

-- Pracuję tutaj! Jestem doktorantką.

Powiedziała to tak zmieszana, tak rozszerzające się wielkie brązowe
oczy, że Myra nie mogła się opanować i~się uśmiechnęła.

-- I~rozumiem też aktywistką polityczną.

Dziewczyna skinęła stanowczo głową. 

-- Tak. -- Komentarz wydawał się
umożliwić jej odzyskanie pewności siebie. Przeszła do krzesła i~usiadła,
prostując nogi i~opierając buty o~wózek do książek. Myra obserwowała to
wymyślnie zwyczajne zachowanie niezaangażowana i~ubawiona.

-- Sama byłam aktywistką, kiedy tutaj studiowałam -- powiedziała Myra,
siadając na brzegu stołu.

-- Wiem -- powiedziała zimno dziewczyna. -- Czytałam Twoją pracę
magisterską. \textit{Odprężenie i~kryzys ekonomii sowieckiej.}

Myra się uśmiechnęła. 

-- To ciągle jest całkiem dobre ujęcie, chyba.

-- Ta. Jednak nie mogę tego samego powiedzieć o~Twojej polityce. -- Zmarszczyła brwi, zsuwając stopy na podłogę i~pochylając się do przodu.
-- W~pewnym sensie, to nic\ldots osobistego, rozumiesz? Mam na myśli,
czytałam, co napisałaś, lubię osobę, która to napisała. Czego nie mogę
zrobić to pogodzić się z~tym, kim się stałaś.

To było nieowijanie w~bawełnę! Myra poczuła ból i~winę.

-- Też nie wiem, czy potrafię -- powiedziała. -- Zmieniłam się. Prawdziwa
polityka jest bardziej skomplikowana niż, ach pieprzyć to. Dobra\ldots hm,
jak się nazywasz?

-- Merrial MacClafferty.

-- Ok, Merrial. Prawda jest taka, Rewolucja Rosyjska została pokonana i~nigdy nie została powtórzona, może dlatego, że porażka była tak
niszczycielska, że kolejne próby stały się niemożliwe. -- Roześmiała się
cierpko. -- I~jak Człowiek powiedział, to będzie albo socjalizm albo
barbarzyństwo.\footnote{ cyt. Róża Luksemburg przypisująca Engelsowi,
zob.~\url{https://mronline.org/2014/10/22/angus221014-html-2/}
-- przyp.tłum.}. Socjalizm został zniszczony, był martwy, zanim się
urodziłam. Zatem barbarzyństwo. Mamy przejebane.

Merrial kręciła głową. 

-- Nie, nic nie jest nieuniknione. Sami tworzymy
naszą historię, przyszłość nie jest spisana. ,,Chodzi o~to, aby go
zmienić''\footnote{ część cytatu ,,Filozofowie rozmaicie tylko interpretowali
świat; idzie jednak o~to, aby go zmienić'', z~,,Tezy o~Feuerbachu'' K.
Marksa ,
zob.~\url{https://pl.wikipedia.org/wiki/Tezy\_o\_Feuerbachu} --
przyp.tłum.}. Spójrz na Szinosowy, budują demokrację prawdziwych
robotników, udowodnili, że jest to ciągle możliwe, a~Ty co robisz?
Zwalczasz ich! Po stronie Jankesów i~kazachstańskich kapitalistów.

-- Jak powiedziałam -- westchnęła Myra. -- Prawdziwa polityka jest
skomplikowana. Prawdziwe życia, moje i~tych ludzi, za których jestem
odpowiedzialna. Przyszłość może nie została napisana, ale przeszłość
cholernie tak, nie pozostawia mi wielu możliwości.

-- Czyli, nie opuściłaś \textit{siebie}\ldots

-- Powiem Ci coś -- powiedziała Myra, nagle zirytowana. Pomachała na stos
tektury i~papieru dookoła niej. -- Oto moje życie. Jest więcej na
komputerze. -- Pokazała kciukiem nad ramieniem. -- Hasło to ,,Luksemburg i~Parvus'' dla łatwych rzeczy. Zapraszam do tego. Trudne sprawy, naprawdę
brudne sekrety, nałożyłam stuletnie embargo, a~nawet potem będzie to
diabelska robota, żeby to zhakować. Jeżeli będziesz gdzieś w~pobliżu za
kilka wieków, rzuć okiem.

-- To właśnie robisz? -- spytała Merrial. -- Przekazujesz archiwa
Instytutowi? Dlaczego?

Myra czuła, że jej usta rozciągają się strasznym uśmiechu. 

-- Ponieważ
tutaj mają nieco większe szanse przetrwania następnych kilku tygodni,
nie mówiąc już o~wiekach. Chcesz mojej rady, dzieciaku, przestań się
martwić socjalizmem i~zacznij przygotowywać się na barbarzyństwo,
ponieważ to właśnie nadchodzi, w~ten lub inny sposób.

Merrial wstała i~spiorunowała wzrokiem Myrę. 

-- Może Ty się poddałaś, ale
ja się nie poddam!

-- Dobra, powodzenia -- powiedziała Myra. -- Naprawdę.

Młoda kobieta patrzyła na nią z~nieczytelną miną. 

-- I~Tobie też, jak
mniemam -- powiedziała nieuprzejmie, odwróciła się na pięcie i~wymaszerowała. Czy automatycznie, czy specjalnie, wyłączyła światła przy
wyjściu. Myra mrugnęła, poprawiła opaskę i~wróciła do pracy.

-- Wszystko w~porządku? -- Irina Guzulescu była podświetlona światłem w~drzwiach biblioteki.

Myra wyprostowała się i~otrzepała dłonie.

-- Tak, radzę sobie, dzięki. -- Roześmiała się. -- Przepraszam za ciemność,
używałam opaski do patrzenia, zamiast włączyć światła.

-- Pewnie tak samo dobrze -- powiedziała drobna kobieta. Weszła ostrożnie
do pokoju, obok otwartych skrzyń i~opisanych stosów archiwum Myry. -- Niektóre z~książek tutaj są tak delikatne, boję się, że czasem jeden
foton mógłby\ldots -- Uśmiechnęła się i~podała Myrze kubek kawy.

-- Och, dziękuję. -- Było zimno w~stałym, zatęchłym powietrzu biblioteki.
Objęła dłońmi ciepło porcelany. -- Czy mogę tu gdzieś zapalić? -- spytała.

-- Och, jasne, zejdź do piwnicy.

Piwnica wydawała się niezmieniona. Wielki stół, który zajmował większość
pokoju, sprowadzał wspomnienia, długich dyskusji i~kłótni dookoła niego,
planowanych tutaj przygód, popołudni, gdy rozmawiała z~Jonem i~Davem i~wychodziła z~Jonem.

Po drodze, Irina zabrała własny kubek z~kącika kuchennego. Usiadła
naprzeciwko Myry i~przesunęła popielniczkę po stole. W~niewybaczalnym
świetle, wyglądała starzej. Oczywiście miała terapie, ale waga lat
ciągle była widoczna na twarzy. Nie była obwisła, ale pokazywała
napięcie.

-- Cóż -- powiedziała Myra, zapalając -- hm, ta sprawa, o~której mówiłaś?
Że miejsce jest obserwowane? Dlaczego?

Irina poruszyła dłonią, jakby strzepywała popiół. -

- Mentalność policji -- powiedziała. -- Oczywiście skoro badamy post-cywilizacje, potencjalnie
jesteśmy ich sympatykami i~wrogami wewnętrznymi.

-- Co?

-- Zieloni. -- Irina się roześmiała. -- Była Unia i~Zieloni, to jak było ze
Związkiem Radzieckim i~Czerwonymi. W~starych dobrych czasach Zimnej
Wojny, bycie zainteresowanym drugą stroną było podejrzane, niezależnie
jak mogło to być użyteczne. I~oczywiście to samo po drugiej stronie. -- Uśmiechnęła się. -- Pracowałam w~Instytucie Studiów Amerykańskich w~Bukareszcie. \textit{Securitate} cały czas w~moim przypadku.

-- Jezu. Musisz być niemal tak stara jak ja. -- Myra pomyślała o~niedyskrecji uwagi, gdy tylko to powiedziała, ale Irina pyszniła się
tym.

-- Starsza -- powiedziała dumnie. -- Mam sto dziesięć lat.

-- O! Ja sto pięć lat. Miałam wczesne terapie, oczywiście, ale teraz
dostałam nano-lek.

-- Ach, to dobrze, nie będziesz żałować. -- Uśmiechnęła się z~zadumą. -- Wiesz, Myro Godwin, jesteś częścią historii. Tego Instytutu i~społeczeństw, które miał badać. Nadzorowałam studentów kilka lat temu
piszących doktoraty o~MRRNT.

-- Nigdy nie myślałam, że skończę, rządząc moim własnym zniekształconym
państwem robotniczym. -- Mroczny chichot. -- Nie żebym kiedykolwiek
wierzyła, że to było to, lub jest. -- Myra pośpieszyła dodać. -- Lub, że
taka rzecz mogłaby istnieć. Ticktin\footnote{ teoretyk marksistowski
wykładający w~Glasgow w~latach 1965-2002,
zob.~\url{https://en.wikipedia.org/wiki/Hillel\_Ticktin} -- przyp.tłum.} wyleczył mnie z~iluzji \textit{dawno} temu.

-- Hmm -- powiedziała Irina. -- Dla mnie to był Mises i~Hayek, właściwie.
Ticktin nie cenił ich wysoko. Lub mnie. -- Roześmiała się. -- Zwykł o~mnie
mówić ,,ostatnia ofiara Ceausescu''.

-- Cóż, tak -- powiedziała Myra. -- Szczerze mówiąc, nigdy nie uważałam
liberałów za strasznie przekonujących. Pytanie, które mi ciągle
przychodziło do głowy, to ,,Gdzie jest chyża konnica?''

Irina pokręciła głową. 

-- Przepraszam?

-- Och, to było coś, co powiedział Mises. Jeżeli Europa kiedykolwiek
stanie się socjalistyczna, upadnie, i~wrócą barbarzyńcy, biegnąc po
stepie na chyżych koniach. Cóż, pół Europy była, nie socjalistyczna,
jakbyś to postrzegała, ale dla Misesa była, zatem gdzie jest chyża
konnica?

Irina patrzyła na nią. Jakby nieświadoma tego, co robi -- odruchy nawyku,
który myślała, że pokonała, wracały -- sięgnęła nad stołem po papierosy
Myry i~zapaliła jednego.

-- Och, Myro Godwin-Dawidowa, jesteś tak ślepa. W~rzeczy samej, gdzie
jest chyża konnica. -- Przerwała, mrużąc oczy od strumienia dymu. -- Jaki
sposób produkcji według Ciebie istnieje w~Byłym Związku Radzieckim?

-- Sposób po-cywilizacji?

-- Eufemizm. -- Machnęła dymem. -- Jak Twój Engels nazwałby społeczeństwo,
gdzie miasta są tylko rynkami i~obozami, gdzie większość ludzi je to, co
wyhoduje i~upoluje dla siebie, gdzie prawie cały przemysł jest na
poziomie wioski, gdzie nie istnieje pojęcie narodu?

-- Cóż, ok, to staromodny termin -- powiedziała Myra z~uśmiechem -- ale
mniemam, że technicznie mogłabyś to nazwać barbarzyństwem.
Technologicznie zaawansowanym barbarzyństwem, ale tak, tak właśnie jest.

-- Dokładnie -- powiedziała Irina. Spojrzała na papierosa ze zdziwionym
niesmakiem i~go zgasiła. -- Oto Twoja chyża konnica. Spójrz poza nasze
miasta, na Zielonych. Faktycznie spójrz w~nasze miasta. Oto Twoja chyża
konnica!

Myra naprawdę nigdy tak nie myślała.

-- Jedyną chyżą konnicą, którą się martwię -- powiedziała gorzko -- to
przeklęci Szinosowy.

Ku jej zdziwieniu i~przerażeniu, Irina zaczęła płakać. Wyciągnęła brudną
chusteczkę z~kieszeni, płakała i~pociągała nosem w~nią przez minutę.
Nagłym impulsem, Myra sięgnęła ponad stołem i~chwyciła jej dłoń.

-- Och, Boże -- powiedziała w~końcu Irina. -- Przepraszam. -- Pociągnęła
mocno nosem i~wyrzuciła chusteczkę, zaakceptowała ofertę papierosa od
Myry.

-- Nie, to \textit{ja} przepraszam -- powiedziała Myra. -- Wydaje się, że
powiedziałam coś, co cię zdenerwowało.

Irina mrugnęła kilka razy. 

-- Nie, nie. To moja wina. Och, Boże, gdybyś
tylko wiedziała. Zostałam, żeby się z~Tobą spotkać, nie tylko wpuścić. -- Czubek papierosa żarzył się w~stożek, zaciągała się tak mocno. -- Nikt
inny nie chciał przyjść rano i~spotkać się z~Tobą. Myślą, że jesteś
straszną osobą, potworem, kryminalistką. Ja nie. -- Mrugnęła znowu,
ożywiając się. -- Wrócę tam, wiesz. Do Rumunii, i\ldots innych krajów
,,post-cywilizowanych''. Dobra, do byłego Związku. I~wiesz co? Ludzie są
tam szczęśliwi, na farmach, w~warsztatach, z~ich lokalnymi armiami i~drobnymi lojalnościami. Biurokraci odeszli, dla mafii nie ma zakazów, na
których mogłyby się wzbogacić, więc też odeszły. Prowincje mają swoje
małe wojny i~waśnie, ale\ldots -- Uśmiechnęła się teraz, smutno -- \ldots zabrzmię
jak feministka, jeżeli je pamiętasz, ale faktem jest, to sprawa
testosteronu. Młodzi mężczyźni będą się wzajemnie zabijać, tak to już
jest. Dla kobiet Moskwa, cholera, każde postsowieckie, prowincjonalne
miasto, jest bezpieczniejsze niż Glasgow.

Och, nie kolejna, pomyślała Myra. Zielony podróżnik, polityczny
pielgrzym. Widziałam przeszłość i~to działa.

-- A kiedy widzę coś takiego jak \textit{powracający} komunizm -- Irina
kontynuowała -- kiedy widzę przeklętych Szinosowietów na czołgach, znowu
kolektywizujących, wchłaniających te wszystkie małe społeczności, chcę
ich zatrzymać.

Spojrzała prosto w~oczy Myry. 

-- Możesz to zrobić, możesz ich zatrzymać.
Musisz walczyć, Myro. Jesteś naszą jedyną nadzieją.

Myra sama miała ochotę płakać.

~

Brytole nie tylko \textit{robią} pociągi.

Oni je wynaleźli. Mieli kilka wieków doświadczeń. Mieli więcej
faktycznych entuzjastów pociągów na głowę mieszkańca niż ktokolwiek
inny. Wynaleźli \textit{trainspotting}\footnote{ hobby polegające na
wyczekiwaniu i~rejestrowaniu numerów lokomotyw (czasami także numerów
wagonów) pociągów przyjeżdżających i~odjeżdżających z~danej (bądź wielu)
stacji kolejowej oraz czasu ich zjawienia się,
zob.~\url{https://pl.wikipedia.org/wiki/Trainspotting\_(hobby)}
-- przyp.tłum.}. I~ciągle wydawało się, że nie potrafią wymyślić jak
sprawić, by pociągi \textit{przyjeżdżały o~czasie}.

Zatem jechali w~jasny, zimny niedzielny poranek, gdzieś na południe
Penrith, ciągnięci przez jeden, elektryczny silnik, który brzmiał, jakby
pochodził z~takiego sprzętu, którego używałbyś w~domu. Zalesione wzgórza
powoli się przesuwały. Przynajmniej miała miejsce w~pierwszej klasie. Straż
pociągu tylko chodziła po sąsiedniej drugiej klasie, gdzie były wszystkie
krzyczące dzieci, a~wózek z~napojami toczył się za nim.

Myra zapaliła papierosa i~wyjrzała. Czuła się stosunkowo zadowolona,
nawet z~powodu długiej podróży przed nią, jeszcze dłuższej dzięki
cholernej, typowej niewydolności Brytoli. Miała mnóstwo czytania, tam w~opasce. Parvus przygotował jej wybór ostatnich polityk zagranicznych
Brytanii, ostatnim razem, gdy ściągnęła. Około stu kilobajtów, nie
licząc hiperłączy i~załączników. Stosy v-maili do nadrobienia.

Nie mówiąc o~wiadomościach. Teraz to był regularny program CNN, na
kanale informacyjnym specjalistów światowej polityki, dotyczący MRRNT.
Demonstracje sprzeciwiające się federacji z~Kazachstanem wzrosły do
codziennych zgromadzeń około dwóch tysięcy osób, z~kilkoma setkami
odważnych ludzi spędzających chłodne noce w~namiotach na Placu
Rewolucji. Niektóre z~ich transparenty były tymi, które Myra
spodziewałaby się po ich lokalnej ultralewicy, rodzaj ludzi, których
łączyłaby z~zewnętrznymi Nowymi Brytyjczykami. Inne były liberalne,
pro-ONZ lub libertariańskie z~wydźwiękiem pro-kosmicznym,
pro-Zewnetrznym.

Nikt na ulicy, lub w~sieci, nie odkrył jeszcze nic o~atomówkach. Drobna
łaska, ale Myra podejrzewała, że przynajmniej niektórzy z~tych za
różnymi demonstrantami wiedzieli o~nich. Reid, na przykład, zdecydowanie
wiedział i~myślała o~możliwości, czy on sięga po nie przez bojowców Ruchu
Kosmicznego wyhodowanych lokalnie w~MRRNT.

Spędziła mniej więcej pierwszą godzinę podróży przy wirtualnej
klawiaturze, odpisując na raporty, instrukcje, rady dla jej komisarzy,
szczególnie Denisa Gubanowa. Chciała, żeby każdy czekista, którego mógł
poświęcić, zajął się infiltrowaniem i~śledzeniem tych demonstracji.

Drzwi działowe syknęły i~zabuczały, otwierając się. Strażnik wszedł,
wysoki, pochylony mężczyzna w~mundurze z~pistoletem w~kaburze na
biodrze.

-- Bilety z~Carlisle, proszę. -- Miał lekko zniewieściały głos, delikatny
i miły. Uśmiechnął się, sprawdził bilety dyrektora siedzącego
naprzeciwko Myry.

-- Przepraszam -- powiedział śpiewnie kelner zza niego. Kelner był
drobnym, chudym młodzieńcem w~białej koszuli, muszce w~kolorach tartana
i spodniami. Kolczaste czarne włosy.

Wózek zagrzechotał i~zabrzęczał w przedziale. Strażnik przesunął się,
żeby przepuścić. Gdy to robił, pociąg lekko szarpnął, wprawiając
zawartość wózka w~dzwonienie, a~hamulce zaczęły piszczeć, gdy pociąg
zaczął się zatrzymywać.

Nadeszło trzeszczące ogłoszenie, z~którego Myra mogła tylko zrozumieć
słowa ,,drzewa na torach''.

Fala kpin przebiegła po wagonie. Myra dołączyła się z~gwizdem i~wyjrzała
przez okna. Koło torów były drzewa, na prawo, ale były około stu metrów
dalej za połacią łąki. Po drugiej stronie, ostry stok z~drzewami ponad
osypiskiem.

Usłyszała sapnięcie kelnera i~rodzaj kaszlnięcia strażnika. Duża ilość
jakiegoś czerwonego płynu rozprysnęła się na stole, przy którym
siedziała, a~część przelała się ponad krawędzią na spódnicę na jej
kolanach. Myra wzdrygnęła się, patrząc do góry z~chwilowym błyskiem
cywilizowanej zirytowania, jej pierwsze wrażenie było takie, że jakoś
kelner rozlał na niej butelkę czerwonego wina.

Strażnik upadł bokiem na stół z~szokującym uderzeniem. Jego gardło
rozchylało się i~trzepotało jak skrzela, ciągle pompując. Mogła dojrzeć
krawędź przerwanej tchawicy, białej, jak zepsuty plastik. Jego usta też
były otwarte, język drżał, ślina kapała. Jego oczy szeroko otwarte.
Podniósł dłoń i~spojrzał, jakby próbował coś jej powiedzieć. Potem
przestał próbować. Jego głowa uderzyła w~stół z~drugim stuknięciem,
\textit{diminuendo}\footnote{muz. stopniowy spadek dynamiki, przy opisie
dynamiki utworu muzycznego,
zob.~\url{https://pl.wikipedia.org/wiki/Dynamika\_(muzyka)} -- przyp.tłum.}.

Kelner ciągle stał, trzymając w~jednej dłoni krótki nóż i~pistolet,
najwidoczniej strażnika, w~drugiej. Jego mankiety miały krew na nich tak
jak przód koszuli. Wyglądało, jakby miał krwotok z~nosa, który próbował
zatamować rękawem. Było zaskakujące jak rzadką cieczą była krew, kiedy
była świeżo rozlana, ciągle chlapiąca, ciemnowinnym strumień.

Kelner przesunął językiem po ustach. Pomachał pistoletem w~sposób, który
sugerował, że nie był do końca zaznajomiony z~jego użyciem. Potem,
ruchem jak sztuczka magika, przełożył nóż i~pistolet w~dłoniach i~przesunął zamek. Zablokowany i~załadowany. Dobra, wiedział, jak go
używać.

-- Nie ruszać się, kurwa -- powiedział.

Myra się nie ruszyła. Wsadziła mały awaryjny pistolet w~cholewę buta,
kiedy zdejmowała kaburę z~Glockiem, który teraz leżał pod jej kurtką na
półce na bagaże powyżej. Nie mogłaby sięgnąć po żadną broń na czas. Ani
nie mogła wymrugać menu komunikacyjnego w~opasce, jej telefon też był w~jej kurtce. Drugi pasażer, który siedział naprzeciw i~plecami do
kierunku jazdy, też się nie poruszał. Ktoś, nie dziecko, w~przedziale
drugiej klasy krzyczał. Kelner był obrócony plecami do tego przedziału i~przynajmniej kilka osób tam musiało być świadomych, co się wydarzyło.
Bez poruszania głową, lub nawet jej oczami, Myra mogła zobaczyć białe
twarze, okrągłe oczy i~usta przez szklaną przegrodę.

Myślała, \textit{dlaczego ktoś nie strzeli temu chujkowi w~plecy}. Potem,
kątem oka, zobaczyła ruch na zewnątrz, po obu stronach pociągu.
Mężczyźni i~kobiety na koniach. Długie włosy, pióra i~kapelusze,
skórzane kaftany i~kamizelki, karabiny i~kusze potrząsane lub
zawieszone. Jak kowboje i~indianie. Zieloni partyzanci. Barbarzyńcy.

Daleko za nią, blisko końca pociągu, jak się domyślała, była krótka
wymiana ognia i~odległy, cienki krzyk. Trwał i~trwał jak alarm
samochodowy.

Każde drzwi w~pociągu, wewnętrzne i~zewnętrzne, trzasnęły przy
otwieraniu. Ok, zatem ktoś dostał się do sterowania. Myra poczuła zimny
przeciąg na ciepłym i~teraz lepkim płynie na jej kolanach. Kolory
odpłynęły ze świata. Myra zrozumiała, że miała właśnie wpaść w~szok i~zaczęła oddychać mocno i~głęboko.

Niektórzy z~jeźdźców wskoczyli do pociągu. Na końcu każdego wagonu, para
z nich patrzyła w~przeciwne strony, mierząc do pasażerów z~karabinów.
Mężczyzna, który wylądował twarzą w~twarz z~Myrą, wypełnił drzwi do
przedziału. ,,Barbarzyńca'' nie było epitetem stosownym wobec niego. Był
wysoki i~szeroki, miał brodę i~kucyk błyszczący od tłuszczu, jego kurtka
i czapsy były wyłożone gładkimi, nieregularnie ukształtowanymi płytkami
metalu przyłączonymi do skóry przy pomocy metalowych pierścieni, surowy
i częściowy pancerz.

-- Ręce na głowę! Wszyscy na zewnątrz! Na tory!

Myra położyła ręce na głowie, wstała i~przesunęła się bokiem w~przejście. Kelner-punk, który zamordował strażnika, ciągle ją obserwował
i wycofywał się koło wielkiego faceta, którego najwidoczniej znał.
Biznesmen, stojąc, miał ciekawie intensywną minę. Myra domyśliła się
natychmiast, że miał zamiar zrobić z~siebie bohatera i~w przypadkowej
chwili kontaktu wzrokowego, potrząsnęła głową. Jego ramiona lekko
opadły, nawet z~rękami w~powietrzu. Ale podporządkował się wykrzyczanej
komendzie i~nikle pokazanej sugestii, wyskakując na prawo i~lądując w~trwały sposób na nogach i~rękach, potem zbierając się i~idąc wzdłuż
przylegających torów do niskiego stoku z~płotem przy łące.

Myra uniosła ręce i~przekroczyła ugięte nogi strażnika, minęła
barbarzyńcę i~kelnera i~wyskoczyła. Wylądowała lekko, uderzenie
wstrząsające jej pistoletem niewygodnie, ale uspokajająco wciskająca
głębiej w~bok jej buta i~poszła wzdłuż torów, w~górę stoku, potem
odwróciła się do pociągu.

Wszyscy ludzie robili, to co robiła, lub pomagali dzieciom -- teraz
cichym -- aż do pokruszonych kamieni. Zieloni kroczyli, stali lub
jeździli wzdłuż, krzycząc, cały czas celując karabinami w~pasażerów. Po
obu stronach pociągu była pewna liczba atakujących, pewnie więcej. Około
setki osób, pasażerów i~załogi, wyszło z~pociągu. Ktoś ciągle był w~pociągu i~ciągle krzyczał.

Myra stała z~rękami na głowie i~drżała. Widok tak wielu osób z~rękoma w~górze wywoływał w~niej mdłości. Barbarzyńcy prawdopodobnie zamierzali
ograbić pociąg, musieli wiedzieć, że przynajmniej niektórzy pasażerowie
będą mieć ukrytą broń, ale jak dotąd nie przejmowali się jej szukaniem.
Nadzieja, że mogliby być oszczędzeni była wystarczająca, żeby wstrzymać
prawie każdego od nieuniknionej, skazanej na porażkę, próby walki. Mogło
ich powstrzymać, aż było zbyt późno. Jeżeli Zieloni planowali masakrę,
zrobiliby to, tego była pewna, wtedy, gdy była najmniej spodziewana.
Zieloni manewrowaliby niepozornie, tak, że byliby poza swoimi liniami
ognia i~wtedy rozpoczęłaby się strzelanina. Potem odrobina gwałtów i~rabunku i~kilka ostatnich kończących strzałów w~głowę dla rannych,
jeżeli mieli szczęście.

Jeden wysoki mężczyzna w~futrzanym płaszczu i~bawełnianych legginsach
wzmacnianych skórą przechodził od jednej grupy pasażerów do kolejnej,
przyglądając się i~rozmawiając z~każdą młodą, lub młodo wyglądającą
kobietą. Kiedy dotarł do Myry, zatrzymał się na stoku poniżej jej, oparł
dłoń na kolanie i~spojrzał do góry, uśmiechając się. Był gładko ogolony,
z długimi, wypłowiałymi od słońca, czerwonymi włosami związanymi dookoła
jego czoła rzemieniem z~tyłu. Na kolejnym rzemieniu, dookoła szyi,
wisiał gwizdek. Pod futrzanym płaszczem miał zblakły, zielony
podkoszulek z~napisanym motto Sił Specjalnych ONZ: ,,ROZDZIELMY ICH -- NIECH BÓG ICH POZABIJA''.

-- Ach -- powiedział -- \textit{musisz} być Myrą Godwin!

Mówił z~londyńskim akcentem i~ogólnym nastawieniem świetnej zabawy. Myra
zmierzyła go okiem, wstrząśnięta z~bycia wyróżnioną. Rozpoznał ją i~miała niepokojące uczucie, że widziała go gdzieś wcześniej.

-- Tak -- odparła. -- Co Ci do tego?

-- Masz jakiś dowód tego?

-- Paszport dyplomatyczny, kieszeń kurtki, nad miejscem, gdzie
siedziałam.

-- Sprawdzę -- ostrzegł, mrużąc oczy.

-- Och, i~przynieś też mi mojego cholernego Glocka. Masz Pan poważne
problemy.

-- Przekonamy się o~tym -- powiedział. Odwróciła się i~krzyknął do
wielkiego faceta, który opróżniał jej wagon. Ciągle stał w~drzwiach,
karabin skierowany w~górę.

-- Joł! Fix! Przynieś rzeczy tej pani. Znad jej siedzenia.

Nie spuścił jej z~oka, gdy wielki facet podawał mu złożoną kurtkę, a~on
ją przeszukiwał. Jeden szybki rzut oka na otwarty paszport i~włożył
gwizdek do ust, gwizdnął głośno, trelując, dwa razy.

-- Dobra, Fix, przekaż dalej -- powiedział. -- Mamy ją. Opodatkuj tamtych i~jedziemy. Wynośmy się stąd, zanim nadlecą śmigłowce.

Drugi mężczyzna odbiegł, krzycząc rozkazy. Po chwili, kątem oka, Myra
mogła zobaczyć organizowanie podatku: ludzie z~pociągu byli zagonieni w~jedną grupę, a~mężczyzna ze strzelbą i~kobieta z~workiem szli dookoła,
zabierając pieniądze, biżuterię, drobny sprzęt i~broń osobistą. Ludzie
przekazywali swoje rzeczy z~obrzydliwie chętną uległością.

-- Chcesz swoją kurtkę?

Myra skinęła głową. Rzucił jej, ciągle złożoną, zostawił sobie automat w~kaburze, paszport i~telefon.

-- Dostaniesz je później -- powiedział.

Włożyła kurtkę. To była cienka kurtka i~niedużo robiła, żeby ochronić
przed chłodem.

-- Co to znaczy, ,,później''? -- spytała.

Roześmiał się z~niej.

-- Jedziesz z~nami. Wkrótce Cię wypuścimy.

Wiatr stał się chłodniejszy.

Myra wskazała na jej bluzkę popryskaną krwią i~przesiąkniętą krwią
spódnicę. -- Przepraszam, jeżeli wam nie wierzę.

-- Wojna to piekło, co? -- zgodził się pogodnie. Poruszył dłonią, jakby
odrzucał coś lekkiego. -- Zresztą strażnik był szpiegiem.

Myra nic nie powiedziała.

-- Ok, wy tam! -- krzyczał jakiś facet na koniu. -- Wracajcie do pociągu i~tam zostańcie. Nie próbujcie nas gonić, nie próbujcie do nas strzelać.
Bo jak będziecie, to wrócimy i~zabijemy was wszystkich. I~też nie
opuszczajcie pociągu, jak odjedziemy albo helikoptery wystrzelają was na
polach.

Grupa weszła rzędem do pociągu przez jedne z~drzwi. Myra widziała ich
rozpraszających się po wagonach.

-- To wszystko, co zamierzacie zrobić?

Rudy mężczyzna skinął głową. 

-- Tym razem. -- Pokazał kciukiem nad
ramieniem. -- Znaczy, szkoda mi tych ludzi, ale nie na tyle, żeby ich
zabić. I~nie zamierzam tracić czasu na przeszukiwanie pociągu za
kosztownościami. Nie ma powodu być chciwym, w~przeciwnym przypadku,
pociągi po prostu przestałyby przejeżdżać. Tylko trochę podatku, żeby
opłacić operację, wiesz.

-- Jaką operację?

Popatrzył na nią. 

-- Dotarcia do Ciebie.

Och, gówno. Pomyślała, że to było, do czego zmierzał. Mrugnęła nagle,
zapisując jego obraz i~uruchamiając protokół przeszukiwania w~opasce,
żeby zobaczyć, czy ten znający się na rzeczy bandyta sam był znany.

-- Zrobiliście to wszystko, żeby mnie dopaść? -- Uśmiechnęła się kwaśno,
szczękając zębami. -- Skąd wiedzieliście, że będę w~pociągu?

Mężczyzna spojrzał na nią z~pogardą. 

-- To nie było trudne -- powiedział.
Machnął szeroko ręką, ale wymijająco. -- Jesteśmy wszędzie.

-- Wydaje się trochę przesadne.

-- Pewnych rzeczy lepiej nie mówić przez telefon -- powiedział leniwie.
Potem poruszył się, wyprostował, uśmiechając. -- Poza tym najazd jest
taki fajny. -- Wciągnął długi wdech świeżego powietrza, jakby wdychał
narkotyk. -- To sprawa stylu życia.

Smukła, ciemnoskóra kobieta z~kręconymi, falującymi blond włosami aż do
talii podjechała na czarnym, wielkim koniu, prowadząc podobnego konia i~siwą klacz. Uśmiechnęła się do wysokiego mężczyzny i~odwróciła się z~chłodnym uśmiechem do Myry.

-- Wiesz jak jeździć?

Momentalnie wszyscy byli w~siodłach. Myra podciągnęła jej zakrwawioną
spódnicę, gdy usadawiała się w~siodle. Wysoki mężczyzna machnął i~zagwizdał trzy razy. Nagle Zieloni rozpraszali się od pociągu, ukośnie
na stoku z~osypiskiem do drzew lub, jak ci wokół Myry, prosto przez
wilgotną łąkę. Myra galopował na złamanie karku za Fixem, z~blond
kobietą i~rudym mężczyzną po obu stronach. Nad żywopłotem, ścieżką, w~wąską, zalesioną dolinę.

Gdzieś daleko, dźwięk śmigłowca. Potem krótkie serie z~broni maszynowej,
choć w~kogo były wymierzone, Myra nie miała ochoty zgadywać.

Myra jechała cicho jak inni, ale w~widmowym towarzystwie Parvusa. AI
mamrotała w~jej słuchawkę opartą na przewodnictwie przez kość i~błyskając ekranami Groliera przed oczami. Nic bardziej aktualnego nie
było dostępne bez łącza w~telefonie. Tymczasowo zidentyfikował
mężczyznę, który ją porwał, ale nie było to bardzo oświecające, ostatnie
obrazy były sprzed dwunastu lat i~wtedy nie był lądowym piratem. Był
komentatorem w~sieci, a~-- przed tym -- drobnym agitatorem w~Jesiennej
Rewolucji. Fragmenty telewizyjne jego tyrad wyjaśniały, dlaczego
wyglądał niejasno znajomo, oglądała brytyjską, narodową, demokratyczną
rewolucję w~czasie, który mogła poświęcić od śledzenia natarcia
Syberyjskiego Frontu Ludowego na Władywostok.

Dolinka otwarła się na większą dolinę, mocno zasiedloną. Stare kamienne
domy, kopuły geodezyjne, plecione chaty, kryte strzechą budynki, kilka
konstrukcji z~nanoprodukowanych powłok węglowych. Dużo bydła i~owiec na
polach, dzieciaki wszędzie biegające. Ścieżka stała się żwirową drogą,
która poszerzyła się, w~centrum głównej ulicy, do małego, brukowanego
placu. W~centrum, koło grynszpanowej, miedzianej statui żołnierza z~bagnetem, pomnik ofiarom trzech wojen światowych, była przestarzała, ale
ciągle skuteczna bateria rakiet przeciwlotniczych. Nie wyższa niż sam
pomnik, posiadała stojak z~tuzinem rakiet metrowej długości. Myra mogła
przeczytać drobny druk tego, czym były uzbrojone: taktyczne atomówki z~zapalnikiem laserowym.

Ludzie tłoczyli się dookoła, witając powracających łupieżców. Wołali na
rudego jakoś, co myślała, z~początku jako ,,Czerwony'', co miało sens,
potem zrozumiała, że to było ,,Wielebny'', co w~ogóle nie miało sensu.
Zdecydowanie nie było to nazwisko, które jej kwerenda zwróciła.
Dzieciaki wiwatowały i~wykonywały zuluski taniec wojenny zwany
\textit{toyi-toyi}, wysokie kroki, wysokie skoki.

Fix poprowadził konia do dużego, kamiennego budynku, który miał nisko
frontowym pokój otwarty na ulicę: kawiarnia. Myra podążyła za nim,
zsiadła i~została poprowadzona do tylnego pokoju z~kominkiem i~wysokimi,
skórzanymi fotelami dookoła stołu. Pokój pachniał dymem drzewnym,
alkoholem, niemytymi ludźmi i~wilgotnymi psami.

-- Usiądź.

Myra usiadła, a~dwóch mężczyzn i~kobieta usiedli naprzeciw niej.
Patrzyli na nią przez chwilę w~ciszy. Zdecydowała się zaryzykować
domniemaniem Groliera.

-- Jordan Brown -- powiedziała. -- A Ty musisz być Cat Duvalier. -- To
nazwisko było w~rekordzie uzupełnione drobnym drukiem jako żona Jordana
Browna.

-- Bardzo dobrze -- powiedział mężczyzna, niewzruszony. -- Ładną maszynkę
tam masz.

Myra przesunęła opaskę do góry. 

-- Tak. Zatem proszę mi powiedzieć, panie
Brown, czego chcesz.

-- To \textit{Wielebny }Brown -- powiedział. -- Pierwszy kapłan Ostatniego
Kościoła Nieznanego Boga. -- Uśmiechnął się. -- Ale proszę, mów mi Jordan.
-- Spojrzał nad ramieniem i~krzyknął zamówienie. -- Piwo i~brandy!

Zawiesił swój płaszcz na fotelu, bez niego, pochylając się nad stołem w~podkoszulku i~dzikimi włosami, wyglądał jakoś bardziej onieśmielająco.
Jakaś nieobecność w~jego wzroku przypominała Myrze weteranów
\textit{specnazu}, lub starych Afgańczyków. Slogan Błękitnych Beretów na
jego koszulce mógł nie być ironiczny, pomyślała. Chłopak wszedł, niosąc
szklanki i~butelki.

-- Wszystko co mamy w~tej chwili -- powiedziała kobieta nazywana Cat. -- Co
chcesz?

-- Poproszę piwo.

Przyjęła drink bez dziękowania i~zapaliła papierosa bez pytania o~pozwolenie lub zaoferowaniem innym. Cholera, jeżeli będzie się
zachowywać jakby cieszyła się ich gościnnością.

-- Mówiłeś, \textit{Wielebny}.

Jordan Brown rozłożył dłonie. 

-- Tylko żeby przegadać rzeczy.

-- Poświęciłeś dużo, żeby to zrobić.

-- Z~pewnością -- powiedział. -- Zaryzykowałem życia moich wojowników,
ujawniłem jednego z~moich agentów, kazałem zamordować człowieka jak
świnię, którą był, ale to nie ma nic wspólnego z~Tobą, i~spowodowałem
postrzelenie innego strażnika w~brzuch tylko za próbę wykonywania pracy.
Całkiem możliwie, że niektórzy z~pasażerów polegli w~pomyłkowym
ostrzale. -- Wzruszył ramionami. -- I~zabiłbym więcej, gdybym musiał.
Chodzi o~to, że ujdzie mi to na sucho. -- Machnął ręką nad głową. -- Wszystkim nam. Helikoptery są najgorszym, co Brytyjczycy mogą wysłać
przeciwko nam.

Myra spojrzała prosto na niego. 

-- Jakby mi zależało. Może Ci to nie
ujdzie tak lekko, kiedy to wróci do Republiki Kazachstanu.

Jordan pokiwał trzeźwo głową. 

-- Bez wątpienia depczę po wszystkich
dyplomatycznych uprzejmościach. Ale to Ty przybyłaś do Brytanii po
pomoc, a~nie na odwrót. Więc wybacz mi, że nie będę się zbytnio martwił.

-- Ha!

-- Tak czy inaczej -- kontynuował Jordan -- nie mam ochoty startować w~konkursie sikania na odległość. Mam coś ważnego ci do powiedzenia.
Zatem. Czy chcesz poważnie porozmawiać?

Myra wzruszył ramionami, rozglądając się dookoła teatralnie. 

-- Dlaczego
nie? Nie widzę lepszych rozrywek. -- Nalała brandy, znowu bez fałszywej
uprzejmości.

Jordan Brown oparł się na gołych przedramionach, wziął łyk brandy i~zaczął mówić.

-- Przybyłaś do Wielkiej Brytanii po pomoc wojskową przeciwko Szinosowym.
Możesz nawet ją otrzymać. Chcę ci powiedzieć dwie rzeczy. Pierwsza, nie
rób tego. Nic ci to nie da. Nie możesz zwalczyć komunizmu imperializmem.
To jak wrzucanie napalmu w~ogień.

Myra obdarzyła go spojrzeniem, które mówiło, że słyszała to już
wcześniej. 

-- Jeżeli tak mówisz. A co jeszcze chcesz mi powiedzieć? Może
spróbuj powiedzieć coś, co jest nowością, co?

-- Masz większe kłopoty, niż ci się wydaje -- powiedział Jordan. -- Byt,
który nazywasz Generałem, pracuje dla Szinosowów.

Myra prawie zakrztusiła się łykiem brandy. Przez chwilę kaszlała ogniem.
Poczuła się kompletnie zdezorientowana.

-- \textit{Co}? I~skąd możesz wiedzieć?

-- Precyzyjnie mówiąc -- powiedział Jordan Brown -- Szinosowy pracują dla
\textit{tego}. A co do skąd wiem\ldots

Wyciągnął rękę do Cat. Pochyliła się do przodu, gdy Jordan oparł się.

-- Myra -- powiedziała szczerze -- mogę być teraz dzikuską, ale byłam taka
jak Ty. Byłam w~Międzynarodówce.

-- Och, Jezu! -- wybuchła Myra. -- Połowa jebanego świata jest kierowana
przez eks-trockistów! Powiedz mi coś, czego nie wiem, na przykład skąd
dowiedziałaś się o~organizacji bojowej Czwartej, o~Generale.

-- Zmierzałam do tego -- powiedziała Cat, wystarczająco łagodnie, ale Myra
mogła czytać na twarzy młodszej kobiety jak na ekranie komputerowym,
mogła wyczytać chwilowy spazm niecierpliwej wściekłości. Ta kobieta
barbarzyńca była kimś, kto niebezpiecznie przyzwyczaił się, że jej nie
przerywają. Cat wymusiła uśmiech. -- Ciągle słyszę plotki.

-- Plotki? To na tym się opierasz?

-- Wygląda na to, że właśnie jedną potwierdziłaś -- powiedział Jordan,
sucho.

Myra uznała, że tak. Jednak wyglądało to na sytuację, gdzie blokowanie
byłoby mniej produktywne niż przyznanie, że Generał istnieje i~spróbowanie odkrycia, skąd pojawiła się plotka. Parvus nie napotkał
niczego jak to\ldots

-- Wzięliście to z~sieci, czy co?

Jordan spojrzała na Fixa i~Cat, a~potem wszyscy troje się roześmiali.
Dla Myry brzmiało to jak szyderczy śmiech.

-- Boże, wy ludzie -- powiedział Jordan. Jego ton zmienił się, gdy
kontynuował, stając się wezwaniem lub przekleństwem. -- Cały czas macie
ekran pomiędzy sobą a~światem. My mamy ludzki świat i~świat natury. Mamy
cały świat, który nazywacie marginesem, rozproszone społeczeństwo wolnej
ludzkości. Mamy szept na rynku, gest na drodze, znak kredą na chodniku.
Skręt liścia, obrót gałązki. Mamy zapach niesiony wiatrem. Mamy nocne
niebo i~nazwy wszystkich stałych, poruszających i~spadających gwiazd.
Mamy naszych przyjaciół we wszystkich waszych miastach, obozach i~armiach. Mamy odbiorniki kryształkowe\footnote{wczesna odmiana diody z~prostującym złączem metal-półprzewodnik, używana w~radioodbiornikach,
zob.~\url{https://pl.wikipedia.org/wiki/Detektor\_kryszta\%C5\%82kowy}
-- przyp.tłum.}, które odbierają, i~nadajniki iskrowe\footnote{rodzaj
nadajnika służącego wytwarzaniu fal radiowych używanego w~pierwszych
dwóch dekadach rozwoju radia,
zob.~\url{https://pl.wikipedia.org/wiki/Nadajnik\_iskrowy} -- przyp.tłum.}, które wysyłają w~kodach już dawno zapomnianych na
długościach fal, których już nie monitorujecie, w~językach, którymi
wzgardziliście.

Przechylił głowę i~zaczął naśladować kod Morse'a,\\
\textit{pi-pi-piii-pi-piii-pipipipiiipipipiii}\ldots 

Cat i~Fix przechylili
głowy, słuchając, a~po minucie uśmiechając się i~rechocząc.

Jordan wyglądał nieco arogancko po tej demonstracji. 

-- Widzisz, nawet
mogę żartować w~językach. Mamy nasz własny Internet, i~naszą własną
Międzynarodówkę. Nie zajmuj się szukaniem przecieków od waszej.

-- Poza tym -- powiedział Fix, zabierając głos po raz pierwszy -- wiemy te
rzeczy z~dawna. Jordan i~Cat walczyli w~rewolucji, tak samo jak to. Było
wtedy nazywane Czarnym Planem i~było wykorzystane przez, lub
wykorzystało, Armię Nowej Republiki. Wszyscy się z~tym spotkaliśmy i~wiemy, gdzie się udało. Na New View, waszą komunistyczną komunę w~kosmosie.

-- I~wiemy jak to myśli -- powiedziała Cat. -- Widzimy jego wpływ na to, co
robią Szinosowy, w~taktyce i~w ich strategii. To nie jest nieżyczliwe,
ale jest \ldots ambitne.

-- Zatem? -- Myra wzruszyła ramionami, próbując zachować spokój i~potwierdzić wpływ na rozmowę. -- My, to jest mój kraj, Kazachstan\ldots -- proszę, powiedziała to i~słowa \textit{mój kraj, Kazachstan} nie mogły być
cofnięte -- \ldots my nie opieramy się na tej rzeczy. Nie przyjmujemy od
niej rozkazów, nie od, cóż, powiedzmy, że nastąpiło mi na odcisk. Nie
mówię, że wierzę wam na temat tego stającego stronę Szinosowów, ale
powiedzmy, nie zostawiłabym tego. Jeżeli jesteście tak tym przejęci,
dlaczego sprzeciwiacie się moim poszukiwań pomocy, żeby to zatrzymać?

-- Ponieważ -- powiedział Jordan, podkreślając każde słowo ruchem dłoni -- \textit{to nie może być zatrzymane}. Nie walką. Jeżeli stwierdzi, że jest
po stronie przegrywającej, zmieni strony, lub zacznie pracować dla obu
stron i~wygra. Jej jedynymi prawdziwymi wrogami są rywalizujące AI,
takie jak te Ruchu Kosmicznego i~te dziwne duchy geniuszy, w~które
niektórzy Przestrzeniowcy\footnote{ oryg. spacer -- tu i~dalej mowa o~założycielach kolonii kosmicznych, jednocześnie nie tylko kolonistów,
ale także sił polityczne spierające się z~ziemskimi mocarstwami, stąd
korzystam z~tłumaczenia ,,Pozytonowy Detektyw'' Isaaka Asimova, 1953,
gdzie ,,spacer'' jest przetłumaczone jako przestrzeniowiec -- przyp.tłum.} próbują się zmienić. To pokona je lub je zabsorbuje i~wtedy będzie zadowolone w~swoim\ldots osobliwej boskości,
rozprzestrzeniając się z~ludzkością na wszystkie przyszłe światy. Będzie
dbać o~nasze najlepsze interesy, czy tego chcemy, czy nie.

-- Ha, no weź, nie możesz tego wiedzieć.

Jordan rozsiadł się i~spojrzał na nią ironicznie. 

-- Och, tak, mogą, ale
nazwijmy to hipotezą, jeżeli poczujesz się lepiej. Jeżeli Republika
Brytyjska stanie po Twojej stronie, jestem pewny, że będą zachwyceni
odzyskaniem swojego starego systemu planistycznego. Rzucą się na każdą
ofertę, jaką to przedstawi. Lub przyjmą podobne greckie dary od AI Ruchu
Kosmicznego. Więc ktokolwiek wygrywa, mocarstwa zachodnie, Ruch
Kosmiczny czy Szinosowy, ludzkość będzie żyła w~tej lub innej maszynie,
na zawsze.

-- Czy byłoby to takie złe?

-- Nie -- powiedział Jordan. -- Tego właśnie się boję.

Zerwał się. 

-- Ale co tam. Zrobisz to, co najlepsze. Jeżeli nadal
będziesz chciała sojuszu z~Brytyjczykami, kiedy dotrzesz do Londynu,
śmiało. Dużo ci to pomoże.

Dopił pozostałą brandy i~rozejrzał się po innych, potem spojrzał na
Myrę. 

-- Chodź -- powiedział. -- Zabiorę Cię do drogi.

Jechała koło rudego mężczyzny, zmartwiona, ale nieprzekonana jego
dziwnymi tyradami. Mokre gałęzie buku i~brzozy ocierały się o~nich,
sprawiając, że schylała się i~mrugała. Kamienna ścieżka prowadziła na
stok wzgórza ponad osadą. Myra spojrzała do tyłu na nią, zanim zniknęła
z widoku.

-- Jak wy żyjecie? -- spytała. -- Nie możecie żyć tylko z~najazdów, a~pewnego dnia wkrótce, według Ciebie, nie zostanie nic do rabowania. Na
przykład, kto płaci za te rakiety przeciwlotnicze?

-- My wszyscy -- powiedział Jordan. -- Nie mamy podatków, to byłby śmiech.
My, nie tylko ta wioska, wszyscy wolni ludzie, mamy kilka prostych zasad
ekonomicznych, które były stosowane w~społecznościach jak to przez teraz
prawie sto lat. Jedna z~nich mówi, że nie mamy czynszu, ale ziemia nie
jest darmo, Bóg nie robi jej więcej, ale w~dalszym ciągu robimy więcej
ludzi. Zatem stosujemy równoważnik czynszu na potrzeby społeczności, jak
obrona. Inna jest taka, że dowolna osoba, każda grupa, może wypuszczać
własną walutę, popierając ją na własne ryzyko. Żadnych właścicieli
ziemi, lichwiarzy i~urzędników.

-- Och wspaniale -- powiedziała Myra. -- Chłopska idea utopii. Jeden
podatek i~zabawne pieniądze! Teraz już widziała wszystko!

-- To działa -- powiedział Jordan. -- My, jak widzisz, rozkwitamy. Jesteśmy
przyszłością.

-- Jordan -- powiedziała -- wiesz, że znalazłam dane o~Tobie w~moje
encyklopedii? Cóż, na ich podstawie nigdy bym nie przewidziała, że
przejdziesz do Zielonego Szlamu. Lub że zostaniesz kaznodzieją, jeżeli o~tym pomyśleć.

Jordan się roześmiał, nieobrażony. 

-- Świat przypadnie albo barbarzyńcom
albo maszynom. Wybrałem barbarzyńców i~wybrałem upowszechnianie
oświecenia pośród nich. Stąd kaznodziejstwo, które było, na początku,
rodzajem racjonalizmu. Mogę uczciwie powiedzieć, że wyprowadziłem wielu
z moich ludzi z~mroku, pogańskiego kultu Gai, czarów i~przesądów. Jednak
również odkryłem, jak wielu innych misjonarzy, że wolę ich sposób życia
wobec tego, z~którego pochodzę. I~razem z~kochającą naturą, zacząłem
kochać naturę Boga.

-- Byłeś ateistą.

-- Tak uważałem. Potem zrozumiałem, że byłem agnostykiem.
\textit{Wojującym} agnostykiem, jeżeli wolisz. Teologia to bałwochwalstwo,
pisma to apokryfy. Wszystko, co możemy powiedzieć, to Bóg jest
Jednością. Bóg obejmuje świat, nie ma nic poza nim, i~nic przeciwko
niemu. Jak mogłoby być? Zatem Bóg przyzwala na wszystko, co się zdarza,
ponieważ wszystko, co się wydarza, jest jego wolą. Bóg kocha ten świat,
wszystko na nim, od Hubble'a do Plancka, od Wybuchu do Kolapsu. Bóg jest
w jastrzębiu unoszącym się tam w~górze i~w myszy, która czai się przed
jego szponami na tamtym polu. Bóg jest w~Sierpie i~w Snopku zboża,
Młocie i~rozpalonym Żelazie, Mieczu i~Ranie. Bóg jest w~Ogniu, w~Słońcu
i Holokauście. Bóg był w~szpiegu, którego dzisiaj zabiłem i~w mężczyźnie, który go zabił.

Antynomizm\footnote{pogląd, który odrzuca prawa lub legalizm i~argumentuje
przeciwko normom moralnym, religijnym lub społecznym,
więcej~\url{https://en.wikipedia.org/wiki/Antinomianism} -- przyp.tłum.} był, Myra wiedziała, dość powszechną herezją w~okresach
rewolucji i~załamań społecznych. Czterysta lat temu, te same słowa mogły
być deklamowane dalej na tych samych wzgórzach. Nie było nic nowego w~tym, co Jordan mówił, ale Myra była pewna, że wskazanie na to wcale by
mu nie przeszkodziło. Prawdopodobnie sam czytał Winstanleya i~Christopher Hilla.

-- Zdaje się, wiesz dużo o~tym niepoznaznawalnym Twoim Bogu.

-- Prawda.

-- Czy Bóg jest w~maszynach, w~AI, których się obawiasz?

-- Tam też, tak.

-- Jaka różnica pomiędzy Bogiem, który nie robi różnicy i~nie przyjmuje
strony, a~brakiem Boga?

Dotarli do szczytu wzgórza. Jordan zatrzymał konia. Myra też się
zatrzymała, spojrzała w~dół wzgórza na szarą wstążkę autostrady i~białe
bloki stacji obsługi.

Tak blisko, cały czas.

-- Możesz stąd iść na piechotę -- powiedział Jordan sucho. Wziął lejce
konia, gdy Myra zsiadała. Trzeźwo oddał jej broń z~kaburą, paszport i~telefon.

-- Och, a~odpowiadając na Twoje pytanie. Nie ma różnicy, w~pewnym sensie.
Jednak wiara, że Bóg jest wszędzie, po Twojej stronie niezależnie co
robisz i~co się wydarzy, daje ci olbrzymią energię. -- Uśmiechnął się do
niej z~góry. -- Lub tak odkryłem.

I z~tym, agnostyczny fanatyk odjechał, szybko na koniu.

Myra przebrnęła wzgórze do punktu obsługi pasażerów, umyła się, wykonała
kilka telefonów, gdy jadła na stołówce i~wynajęła samochód, żeby zabrał
ją do Londynu.

Dotarła, przez wszystkie przeszkody rzucone małymi bitwami po drodze,
wieczorem kolejnego dnia. Już dawno minęło jej spotkanie z~Ministerstwem, powiedziała im o~tym z~góry, a~oni poprosili o~telefon,
gdy dotrze, w~sprawie kolejnego.

Jednak, po tym wszystkim, co widziała po drodze, czego nie widziała -- jak dowody, że ludzie jak banda Jordan i~gorsi, działali z~taką
bezkarnością i~bezczelnością, plus minus dziwny atak śmigłowca -- wydawało się, że nie ma to jakiegokolwiek cholernego sensu.


\chapter{Morski Orzeł}

Deszcz bębnił po dachu domu Merrial. Widok na zewnątrz był ponury.
Wyglądałem za okno wcześniej, aż do jeziora i~fiordu. Rzędy chmur
maszerowały znad morza i~jedna po drugiej zrzucały swój ładunek na
wzgórza. Wewnątrz, było ciepło: siedzieliśmy skuleni razem, oparci o~naskładane poduszki, sącząc gorącą, czarną kawę.

-- Dzięki Opatrzności, dziś bez pracy -- powiedziałem.

-- W~każdym razie nie w~Stoczni -- powiedziała Merrial. Machnęła dłonią w~kierunku lutownicy, kryształów widzenia i~bałaganowi w~rogu pokoju. -- Tutaj zaczniesz się uczyć innej pracy.

-- Wspaniale -- powiedziałem.

-- Tak czy inaczej, co to za Opatrzność, o~której mówisz? -- spytała.

-- Hm. -- Popatrzyłem na powolne wirowanie kawy. -- To\ldots pomocna część
Przyrody, można tak powiedzieć. Kiedy rzeczy układają się tak, jakbyś
chciała, bez widocznej przyczyny. -- Spojrzałem na nią. -- Musisz to
wiedzieć.

-- Ale to tylko koincydencja -- powiedziała. -- Wszystko pochodzi z~Przyrody.

-- Niektóre rzeczy są więcej niż zbiegiem okoliczności, a~Natura to
więcej niż\ldots -- Chciałem powiedzieć, że ,,więcej niż natura'', ale
przerwałem i~się roześmiałem. -- Naprawdę nic nie wiesz Teologii
Naturalnej?

-- Nie -- odparła wesoło. -- Zawsze przyjmowałam to za pewnik, że Obcy mają
dziwne wierzenia. Nigdy nie zagłębiałam się w~szczegóły. -- Odstawiła
pusty kubek po swojej stronie łóżka i~przytuliła się do mnie. -- No
dalej. Opowiedz mi szczegóły.

-- Och, Boże. Dobra. Cóż, zwykle początek jest tutaj. -- Dotknąłem
delikatnie jej czoła. -- Tam w~środku. Z~zewnątrz widzimy substancję
szarą, ale od środka myślimy i~czujemy. Wiemy, że są tam miliardy
komórek, przetwarzających informacje. Zatem myślenie i~czucie,
świadomość, to coś, co robi informacja. Tym jest informacja, od środka,
z subiektywnej strony. Gdzie jest informacja, tam jest świadomość.

-- Ale informacja jest wszędzie -- powiedziała. -- Kiedykolwiek, cokolwiek
wpływa na cokolwiek innego, to informacja. Deszcz jest informacją.

-- Dokładnie! -- Objąłem ją ramieniem. -- Złapałaś.

-- Co? Och. -- Przesunęła się trochę i~spojrzała prosto w~oczy. -- Masz na
myśli, że świadomość jest wszędzie?

-- Tak! Właśnie tak!

-- Ale, ale\ldots -- Rozejrzała się dookoła. -- Chcesz powiedzieć, że myślisz,
że ten zegar, na przykład, ma \textit{myśli}?

Tykanie brzmiało głośno w~pokoju, gdy nad tym myślałem.

-- Co najmniej jedną -- powiedziałem ostrożnie.

-- A jak ona by brzmiała?

-- ,,Jest później\ldots jest później\ldots jest później\ldots''

Roześmiała się. 

-- Ale cały wszechświat\ldots

-- Jest nieskończoną maszyną, która implikuje nieskończony umysł. -- Położyłem dłonie na jej głowie, gniazdo zawierające jej skończony umysł.
-- ,,A to wszyscy ludzie nazywają Bogiem'' -- zakończyłem z~zadowoleniem.

Merrial uderzyła mnie.

-- A komputery, mniemam, że powiedziałbyś, że są też świadome?

-- Aye, oczywiście -- powiedziałem.

-- Co za potworna myśl.

-- Mogą nie być świadome z~tego, co widzimy na zewnątrz -- powiedziałem. -- Mogą myśleć zupełnie inne myśli.

Merrial popatrzyła roztargniona za okno.

-- Jakie ma myśli deszcz?

-- Nie słyszysz? -- spytałem. -- Myśli ,,ssssssuper''.

-- Hmm -- powiedziała. -- A \textit{to} jest zbieg okoliczności\ldots

Wykorzystaliśmy kilka dni przed moim przywróceniem do pracy w~Stoczni na
rozpoczęcie edukacji w~lutowaniu precyzyjnym i~programowaniu, ten drugi
przedmiot będący jednocześnie fascynującym i~irytującym. Pracowicie
także przejrzeliśmy dokumenty Wyzwolicielki, co kontynuowaliśmy -- po
tym, jak zwróciłem oryginały Gantry'emu i~powróciłem do pracy w~Stoczni
-- na fotokopiach, ale te nie zawierały żadnej informacji ważnej dla
misji Statku. Teczka z~lat pięćdziesiątych dwudziestego pierwszego wieku
wzmocniła, bardziej swoimi zwyczajnymi odniesieniami i~założeniami niż
wyraźnymi stwierdzeniami, pogląd o~oszałamiającym zakresie aktywności
orbitalnej ludzkości z~czasów przed-Wyzwoleniem. Jednak nie zawierała
wskazówki co do samego Wyzwolenia.

Była chwila, kiedy myślałem, że zdobyłem prawdziwy, historyczny wgląd,
aczkolwiek mało istotny wobec naszych bezpośrednich obaw.

Spojrzałem ponad stosem papierów na szerokim stole Merrial. Każdego
wieczoru po pracy, powoli przesiewałem je, jak teraz, w~późnym słońcu.

-- Merrial? -- spytałem. Odwróciła się od aparatury kryształu widzenia, na
którym pracowała i~odłożyła lutownicę.

-- Znalazłeś coś?

-- Nie, tylko\ldots coś zrozumiałem. Ci Zieloni, o~których pisze w~niektórych artykułach, ludzie marginesu, którzy żyli poza miastami.
Pokazuje tutaj, że znali znacznie więcej praktycznych umiejętności niż
ludzie im przypisywali, że nie byli tylko prostackimi barbarzyńcami, ale
rolnikami, rzemieślnikami, elektrykami i~tak dalej.

-- Tak -- powiedziała z~tajemniczym uśmiechem. -- To prawda.

-- Dobra! Ci ludzie, Zieloni, musieli być przodkami majsterków!

-- Proszę -- powiedziała, podając mi papierosa. -- Będziesz tego
potrzebował.

-- Dlaczego? -- spytałem, zapalając.

-- Ponieważ, och \textit{Dhia}, jak ci to powiedzieć delikatnie?
Zrozumiałeś to zupełnie na opak. Jak myślisz, dlaczego nazywamy
osiadłych ludzi ,,Obcymi''?

-- Co?

-- Aye, Zieloni, barbarzyńcy, to nie są nasi przodkowie, Clovis. Oni są,
miałam powiedzieć, że Twoimi, ale nie mogę tego już powiedzieć, \textit{mo
gràidh}, teraz jesteś jednym z~nas. Oni są przodkami Obcych! My jesteśmy
niedobitkami, potomkami mieszkańców miast!

-- Więc jak to jest, że my, mam na myśli Obcy, żyjemy teraz w~miastach?

Wstała wtedy, chodząc dookoła małego pokoju jak wykładowca, gestykulując
papierosem. 

-- Och, ale Twoja mina to obraz, colha Gree! Teraz oni żyją w~miastach, ponieważ je najechali, wprowadzili się przy Wyzwoleniu, kiedy
stara cywilizacja i~życie miejskie się załamały. A teraz ciągle są
tutaj, błogosławić ich, błądząc dookoła jak barbarzyńcy, którymi są, w~pożyczonych kostiumach przeszłości. Wszyscy ci uczeni, których chciałeś
naśladować, tylko grzebią w~ruinach, czytając książki, które tak źle
rozumieją, że to nawet nie jest śmieszne. Jesteś już poza tym, moja
miłości, nauczysz się z~nami w~ciągu roku więcej niż przez całe życie na
Uniwersytecie!

W rzeczy samej.

~

Podniosły się wielkie wiwaty, prawie zagłuszając wściekły ryk wody, gdy
wrota zostały otwarte. Woda wlewała się nad krawędzią suchego doku w~słonej Niagarze, która trwała i~trwała, aż wydawało się, że sam fiord
obniży się, nim głęboki zbiornik zostanie napełniony. Szybciej niż
przypływ, woda wkradała się na nogi i~pontony Platformy.

Dłoń Merrial ściskała moją, gdy przedzieraliśmy się przez tłum, pchając
się do przodu jak dzieci. Wszystkie dostępne części klifu dookoła doku
były wypełnione ludźmi. Wszyscy, którzy pracowali w~Stoczni, na
Platformie lub Statku, na pewno tu byli, razem z~przypadkowymi gośćmi z~otaczających miast, zapalonymi turystami z~każdego miejsca w~Highlands i~kompletnymi entuzjastami z~dalszych terenów. Kilkaset metrów dookoła
klifu i~dalej, przedstawiciele Międzynarodowego Towarzystwa Naukowego,
kierownicy projektu i~wzorowi robotnicy wygłaszali mowy z~drewnianej
sceny z~podwyższonym podium i~pod płóciennym dachem. Nikt dalej niż
pięćdziesiąt metrów, na zewnątrz, nie potrafił zrozumieć słowa, co
mówili ci dygnitarze, szczególnie nie z~głośników rozwieszonych jak
bajkowe światła na łańcuchach kabli wszędzie. Skrzeki, trzaski i~jęki
warte stacji kolejowej odbijały się echem dookoła zboczy klifu.

Schyliłem się pomiędzy kilkoma robotnikami z~przodu, którzy nieostrożnie
pozwolili na ćwierć metrową przerwę pomiędzy nimi. Merrial podążyła za
mną, bez wątpliwości, uśmiech do nich obu, który sprawił, że poczuli,
jakby wyrządzali przysługę.

I oto tam byliśmy, metr lub dwa od kruszącej, sfatygowanej krawędzi.
Platforma i~Statek wyłaniały się zaskakująco blisko. W~tym momencie
podniósł się kolejny wiwat, jakby oklaskując nasze przybycie i~zauważyłem, że kapsuła na szczycie sondy, delikatnie, ale zauważalnie,
się kołysała. Platforma pływała.

-- Huu-raaa! -- krzyknąłem, dołączając entuzjastycznie do aplauzu. Merrial
krzyknęła coś prawie zbyt wysoko do usłyszenia, ledwo co mogłem siebie
usłyszeć. Pomimo mniej widowiskowej chwili niż zatopienie doku, niosło
to znacznie większe znaczenie: początek podróży \textit{Morskiego
Orła/Iolair}, która zakończy się w~kosmosie.

To był dziwny pojazd startowy, jednocześnie bardziej prymitywny i~bardziej zaawansowany niż cokolwiek wysłane w~kosmos w~pierwszym wieku
eksploracji kosmosu. Starożytni mogliby, bez wątpienia, zbudować statek
z termonuklearnym palnikiem\footnote{ w~oryg. torch-ship, prawdopodobnie
rodzaj wymyślonego napędu,
por.~\url{https://en.wikipedia.org/wiki/Torchship} i~\url{http://www.projectrho.com/public\_html/rocket/torchships.php} -- przyp.tłum.}, ale nie zbudowali. Przeszli od masywnych rakiet na paliwo
ciekłe do nanodiamentowych statków dni ostatnich. W~naszych czasach,
przy dość drogich paliwach chemicznych i~nanotechnologii (prócz
komputerów majsterków) poza naszym zasięgiem i~nadal posiadanym sekretem kontrolowanej
syntezy, termonuklearny palnik był logicznym wyborem.

Jednak, jak wskazywał Fergal, zbudowanie tego z~płyty kotła było nieco
nieeleganckie. Z~drugiej strony, umiejętności istniały, lokalnie
dostępne przy budowie statków. I~waga, przy ogromnej mocy silników, była
nieznaczącym ograniczeniem. I~mówcie co chcecie o~miniowanej\footnote{tetratlenek triołowiu -- nieorganiczny związek chemiczny z~grupy tlenków mieszanych, minia stosowana jest jako pigment
rdzo-ochronny (antykorozyjny) w~farbach podkładowych,
zob.~\url{https://pl.wikipedia.org/wiki/Tetratlenek\_trio\%C5\%82owiu} -- przyp.tłum.
} stalowej płycie, ale jest solidnie odporna na morską wodę. Oczywiście nie
było wątpliwości, że start takiego potwora nie odbędzie się z~ziemi,
która była mniej wybaczająca -- wobec silnego ciepła, wysokoenergetycznych
cząsteczek i~niestabilnych izotopów -- niż morze.

Misja, również, była prymitywna, lub przynajmniej prosta: wysłać na
orbitę eksperymentalnego satelitę komunikacyjnego i~obserwacyjnego. Ten
ładunek wymagał współpracy naukowców i~inżynierów (majsterków lub
innych), szlifierzy szkła, fotografów z~całego cywilizowanego świata.
Systemy elektroniczne i~elektryczne błądziły podejrzliwie blisko ścieżki
mocy, nawet wdrażając, jeżeli chciałbyś być dziwny, system prawie jak
telewizja. Jednak po głębokim zastanowieniu i~złości, większość
najbardziej szanowanych praktyków Teologii Naturalnej, z~pewnym
ociąganiem, skinęła długowłosymi głowami. Telewizja, jak podkreślili
groźnie, była destrukcyjna tylko jako środek masowego przekazu.
Przeciwstawiać się jej jako metodzie komunikacji od satelity do stacji
naziemnej byłoby, jak zaręczali, tępym zabobonem, niewartym tego
oświeconego wieku.

Nie trzeba dodawać, mniejszość ich równie szanowanych, choć (musi to być
powiedziane) zwykle starszych kolegów nalegała, że był to pierwszy krok
na równi pochyłej na końcu, którego czekała populacja zredukowana do
biernie gnijącej masy wraków fizycznych i~psychicznych. Z~równą
nieuchronnością, biorąc pod uwagę naturę Teologii Naturalnej, znacznie
mniejsza (i tak, młodsza) fakcja wskazywała, że ten rodzaj nikczemnego
niewolnictwa opisywany i~potępiany przez ich konserwatywnych kolegów
byli w~istocie ludźmi, lepiej znanymi jak \textit{starożytni}, którzy
wytrwale oglądali telewizję i~dokonali jednej lub dwóch rzeczy, zanim
upadli. Czemu, oczywiście\ldots, ale dalsze iteracje dyskusji byłyby nudne
w opracowaniu.

Merrial znacznie odważniej niż ja poszła do przodu i~usiadła wesoło na
samej krawędzi klifu, nogi zwisające ponad i~jej spódnica elegancko
rozpostarta na wrzosie po obu jej stronach. Usiadłem koło niej i~próbowałem nie patrzeć w~dół na spadek do morza, bezpośredni i~pionowy
prócz tego, gdzie były interesująco urozmaicony wystającymi skałami.
Patrzyliśmy z~punktu lekko z~przodu Platformy, pomiędzy jej głównymi
rozszerzeniami a~otwartymi śluzami doku.

Krzyki i~wiwaty już nie trwały, zastąpione szumem konwersacji, ciągłym
napływem wznoszącego się morza i~głębokim wyciem turbin platformy, gdy
pracowały, żeby poruszyć gigantyczną strukturę. Bardzo wolno, kołysanie
się statku połączyło się z~powolnym ruchem do przodu. Powoli jak to
bywa, to wytworzyło zauważalną falę dziobową z~przodu, zderzającą i~pryskającą wobec napływających fal. Złożone wzorce interferencji
uformowały się, gdy fale odbijały się od boków doku i~samej Platformy, a~słońce, już po zenicie i~zmierzające na zachód, dodawało widma w~rozpryskach.

Nawet przy pięciu kilometrach na godzinę, nie zabrało dużo platformie
minięcie nas, z~dźwiękiem dalszych wiwatów i~machania od i~do załogi
operacyjnej na pokładzie. Kolejny znaczący moment, należycie podkreślony
kolejną falą braw, nadszedł, gdy platforma minęła wrota i~wyszła na
otwarte morze, lub w~każdym razie na fiord Loch Kishorn. Po tym, nie
było nic do oglądania, prócz wolnego odchodzenia platformy i~ludzie
zaczęli się rozchodzić. Platforma miała długą podróż przed sobą, z~Loch,
i w~Inner Sound\footnote{cieśnina w~Szkocji, oddzielająca wyspę Skye oraz
Raasay od lądu Wielkiej Brytanii -- przyp.tłum.}, skąd minęłaby by
przyczółki Rona i~Skye przed wyjściem na Atlantyk. Wyjąwszy poważne
nieszczęście -- a~prognoza pogody była optymistyczna -- podążałaby jeszcze
przez siedem dni, zanim byłaby dostatecznie daleko w~oceanie, by
utrzymać pozycję przy starcie samej rakiety. Załoga na pokładzie
przeniosłaby się do jednostki eskortującej i~odsunęłaby się na horyzont,
uruchamiając start przy pomocy sterowania radiowego, kiedy naukowcy i~inżynierowie określiliby, że warunki były poprawne. Zakładając krzepkość
\textit{Morskiego Orła} i~moc jego silnika, nic mniej niż mocny sztorm nie
mogłoby stanąć na jego drodze. Tylko platforma była, teoretycznie,
podatna na wiatr i~fale, zatem ryzykowna część całego przedsięwzięcia,
część, która dosłownie mogła być utopiona, właśnie się zaczęła.

Póki obawy Merrial o~śmieciach na orbicie nie zostałyby potwierdzone.
Nic więcej nie było usłyszane na ten temat od Fergal lub innego
majsterka, według Druina, a~jemu można było zaufać w~tej kwestii, według
Merrial. Choć jej własna umowa przy Projekcie dobiegła końca, ci z~innych majsterków, którzy pracowali przy systemach krytycznych dla misji
(jak opisywał to żargon), nie przestali, i~ciągle była dobrze
poinformowana o~ostatnich plotkach majsterków, jak również narastająco
ja.

W ciągu tygodni pomiędzy naszym pojednaniem a~pływającą platformą,
spędziliśmy interesująco czas w~trakcie, którego nasza radość z~drugiej
osoby była kontrowana -- choć się w~żaden sposób nie zmniejszyła -- reakcjami innych osób na nią. W~Stoczni, codzienne wytrzymywałem
bezlitosne drwiny, które moi koledzy wydawali się uważać, że są
całkowicie kompatybilne z~ciągle przyjacielskimi relacjami w~innych
kwestiach. W~miększych okolicznościach moich wcześniejszych doświadczeń
-- dzieciństwo, szkoła, Uniwersytet -- niektóre z~ich zniewag i~nadużyć
spowodowały długoletnią, tlącą wrogość, jeżeli nie natychmiastową
przemoc fizyczną. Tutaj przechodziły jako lekkie przekomarzanie i~raczej
ich ignorowanie niż zemsta było uznawane za oznakę męskiego honoru.

Nieprzystępne nastawienie majsterków w~obozie ciężej było znieść, ale
Merrial ciągle zapewniała mnie, że to był podobny test, siły mojego
zaangażowania w~ich metody i~wobec niej. Gdy dni i~tygodnie mijały, ich
reakcje wobec mnie stopniowo ociepliły się do punktu lodowatej,
kolczastej uprzejmości.

Merrial i~ja byliśmy, według zwyczajów majsterków, związkiem, próbując
doświadczenia wspólnego życia, zanim dokonamy publicznego ogłoszenia.
Cieszyłem się z~tego eksperymentu i~byłem tak zaangażowany, jak
kiedykolwiek mógłbym sobie wyobrazić, i~taka była Merrial, ale żadne z~nas nie śpieszyło się rozwinąć naszą relację na bardziej formalnych
gruntach. Małżeństwo majsterków jest poważną sprawą, angażującą oprócz
innych przerażających wydatków -- krawcowe, kucharze, muzykantki -- także
utrzymanie setek ludzi pijanych przez tydzień.

Merrial spojrzała na mnie.

-- Idziemy?

-- Tak.

Wstaliśmy i~przeszliśmy do tyłu, łatwiej teraz, przez przerzedzający się
tłum. Z~oczywistych powodów, alkohol był ściśle zakazany w~Stoczni i~na
wydarzeniu tego dnia. Wszyscy kierowali się do miast, zaczynając od
najbliższego, Courthill. Koniec projektu i~ostateczne wypłaty i~nagrody
były świętowane przez osuszenie zapasów pubów w~trakcie popołudnia i~wieczoru.

Wędrowaliśmy wzdłuż ścieżki do głównej drogi, co jakiś czas witając
ludzi, których znaliśmy. Scena, na której odbywały się przemówienia,
stała pusta i~już była demontowana. Różni dygnitarze ruszali ścieżką w~zwartej grupce i~przyśpieszyłem nieco, żeby ich wyprzedzić trawą, skory
bliższego obejrzenia znanych kobiet i~mężczyzn, którzy podróżowali z~daleka, by uhonorować nasze dokonanie. Merrial obserwowała to zachowanie
z ironiczną tolerancją.

Wskazywałem słynnego rosyjskiego astronoma i~angielskiego inżyniera
statków kosmicznych Merrial, gdy oboje zauważyliśmy Fergala na końcu
pochodu, idącego samotnie pomiędzy nimi wszystkimi. Byłem zaskoczony
jego widokiem, potem pomyślałem, że nie powinienem, w~końcu był
kierownikiem projektu dla systemu naprowadzania. W~tej samej chwili,
zauważył nas. Skinął na nas.

Merrial spojrzała na mnie. Wzruszyłem ramionami. Podeszliśmy i~dołączyliśmy do niego, ja upewniając się, że idę pomiędzy nim a~Merrial.
Czułem się niespokojnie, że nie byliśmy na swoim miejscu, ale reszta
dygnitarzy uprzejmie nie zwracała na nas uwagi, do tego stopnia, że w~ogóle nas nie zauważyli, będąc po prostu zatopionymi w~ich własnych
rozmowach.

Spojrzał na nas z~boku, bez wrogości. Nasza konfrontacja równie dobrze
mogła nigdy nie nastąpić, mimo wszystko nie pokazywał znaku urazy. Dla
mnie było inaczej.

-- Jak sobie radzicie? -- spytał. Najwidoczniej słyszał o~naszym związku.

-- Och, dobrze. Wspaniale!

Merrial chwyciła moją dłoń i~ją zamachała. 

-- Ten już nie jest już więcej
obcym, tyle ci powiem.

-- Dobrze. -- Uśmiechnął się i~zmienił temat. -- To wielki dzień dla nas
wszystkich.

-- Ta -- powiedziałem. -- Ale nie będę pewny tego, póki statek nie będzie
na orbicie.

-- Och, nie martwiłbym się o~to -- powiedział. Jego wzrok przeskoczył na
oczy Merrial. -- Statek jest bezpieczny.

-- Jak \textit{sobie} poradziliście? -- spytałem śmiało. -- Z~naszym nowym
przyjacielem?

-- Kim, och SI!

-- Co?

-- Sztuczna Inteligencja -- Fergal i~Merrial wypowiedzieli to w~tej samej
chwili. Spojrzałem z~jednego na drugie i~się roześmiałem.

-- Kiedyś muszę się nauczyć rzeczy tego rodzaju.

-- Rzeczywiście musisz -- powiedział Fergal wyrozumiale. -- Jednak, masz
kilka wieków przed sobą do nauki.

-- Cóż, myślę, że dwa to dużo -- odpowiedziałem, zaskoczony tą dziwną
uwagą.

Fergal się zatrzymał, potem przyśpieszył, gdy inni nastąpili na nas.

-- Nie powiedziała ci?

Merrial patrzyła na niego i~na mnie z~milczącym apelem, co jakoś
wydawało się znaczyć co innego dla każdego z~nas. Fergal stanowczo
pokręcił głową.

-- Dobra, cholernie powinna.

-- Nie chciałam\ldots -- zaczęła Merrial.

-- Niewłaściwie zachęcać? Lub wystraszyć? -- Fergal uśmiechnął się kwaśno.
-- Czy to lubisz czy nie, Merrial MacClafferty, trochę późno na to, nie
sądzisz?

-- Och, nie jestem pewna, czy jest gotowy\ldots

-- Czy oboje -- powiedziałem -- przestaniecie rozmawiać, jakby mnie tutaj
nie było?

Fergal spojrzał nad ramieniem, spojrzał do przodu, potem spojrzał na
ziemię i~powiedział cicho.

-- Wiesz, dlaczego ludzie dzisiaj żyją dłużej, niż żyli aż do jakiegoś
czasu przed Wyzwoleniem?

-- Ta -- powiedziałem. -- Znalazłem odniesienia do tego w~dokumentach
Wyzwolicielki. Terapie przedłużające życie. Mniemam, że w~pewien sposób
efekty musiały wytrwać i~stały się dziedziczne.

-- Wystarczająco blisko -- powiedział, wyraźnie powstrzymując się od
poprawiania. -- Cóż, ludzie, którzy byli przodkami majsterków, mieli
lepsze terapie.

Moje serce waliło. 

-- Jak lepsze?

Rozejrzał się dookoła. Kilka metrów oddzielało nas od innych na ścieżce,
przed nami i~za nami.

-- O tyle lepsze, że nie wiemy, jak lepsze są.

Spojrzałem na Merrial, czując, jak krew odpływa mi z~twarzy, a~potem
wraca. Ścisnąłem jej dłoń.

-- Dobra, jeżeli będziesz mnie chciała, nie dbam, czy mnie przeżyjesz i~zostaniesz młoda, gdy ja będę się starzał. -- Łatwo to powiedzieć, gdy
masz dwadzieścia dwa lata i~nie wierzysz, że starzenie lub śmierć ma w~ogóle jakiekolwiek osobiste zastosowanie. Jednak ku mojemu zaskoczeniu,
Merrial się roześmiała.

-- Te nie są genetyczne, nie bardziej niż inne -- powiedziała. -- To\ldots

-- Zaraźliwe -- powiedział Fergal. -- Czy to zakaźne? Nigdy nie pamiętam.

-- Nieważne -- powiedziała Merrial. -- To jest, hm, przenoszone drogą
płciową.

Brzmiała prawie zażenowanie.

~

Fergal, jak się zdawało, ciągle był mile widziane w~Karonadzie i~nawet
Druin, gdy mijał go przy barze, był wobec niego uprzejmy. Domyślałem
się, sam, po trzecim litrze i~szóstej whisky, że majsterek z~Międzynarodówki nie chciał pokazać nam przyjaznej strony. Pozostałem
nieprzekonany tym, ale zdecydowałem wykorzystać większość z~tego, póki
trwało. Ciągle nie przyswoiłem wiadomości, że mógłbym żyć dłużej, niż
kiedykolwiek oczekiwałem i~zajęłoby mi to wystarczająco długo.

-- A zatem -- spytałem go, w~rogu stołu bezpieczny przy ochrypłych
dźwiękach dookoła nas -- czym była ta rzecz, którą Merrial odkryła? AI?

-- To\ldots planista -- powiedział. -- Umysł, który może koordynować całą
ekonomię. Coś, co będziemy potrzebować, pewnego dnia.

-- Po waszej wspaniałej rewolucji?

-- Tak, a~może wcześniej. To samo w~sobie jest rewolucyjne.

-- Więc co zamierzacie z~tym zrobić? -- spytałem.

Fergal potrafił, jak powiedziała Jeanna, wytrzymać swoje picie. Mógł
równie dobrze nie robić lub mówić niczego bez skalkulowania wpływu tego
na wektory jego celów. Jednak jestem pewien, że był to lekkomyślny
impuls, który sprawił, że powiedział, co powiedział.

-- To jest na Statku. Dobra, tak czy inaczej, kopia tego.

Patrzył na mnie, nie na Merrial, gdy mówił. Nie widział tego, co ja
zobaczyłem: chwilowego błysku triumfu i~rozkoszy na twarzy Merrial. To
przelotne spojrzenie, tak jak jego słowa, musiały wyssać kolory z~mojej
twarzy. I~wtedy -- widziałem jej obłudę -- gdy Fergal spojrzał na nią,
wyglądała na bardziej zszokowaną niż ja.

-- Dlaczego, do cholery, to zrobiłeś? -- spytała.

Fergal pochylił się i~ściszył głos. 

-- Dowiedziałem się kilku rzeczy od
AI -- powiedział. -- Jego wspomnienia pochodzą z~kilku dni przed
Wyzwoleniem. Nie wie nic o~tym, co się zdarzyło, ale wie, że
Wyzwolicielka kontrolowała bronie jądrowe i~inne w~kosmosie. Zatem
możliwość, że, wiesz, czego się obawialiśmy, jest prawdziwa, była zbyt
duża, żeby ignorować. Jednak na tym etapie, cholera, gdyby misja została
przerwana lub statek zniszczony Bóg wie, ile czasu minęłoby, zanim
powstałby następny. Był tylko jeden sposób, żeby to zrobić, a~to
oznaczało skopiowanie i~wpuszczenie do kryształu widzenia systemu
sterowania Statku. Z~czystego pragnienia zachowania siebie, kopia byłaby
zmuszona przejąć rodzaj szybkoreagującej kontroli nad silnikiem statku,
co umożliwiłoby uniknięcie śmieci, które mogłyby tam być.

-- Czy to by zadziałało? -- spytałem Merrial, która patrzyła na Fergala,
jakby go nie widząc.

-- Och, tak -- powiedziała, bez rozglądania -- nie moglibyśmy sami tego
zrobić, ale AI miałaby, chyba, szansę. Ale co się wydarzy, kiedy już tam
będzie?

Fergal uśmiechnął się. 

-- Będzie siedzieć w~centrum nowej sieci
komunikacyjnej, to wszystko. Pożyteczna rzecz.

-- Cholernie niebezpieczna, raczej! -- powiedziałem.

-- Nie martw się -- powiedział Fergal, rozumiejąc, że posunął się za
daleko. -- To nie będzie mieszać w~satelicie. Tylko\ldots zbierać
informacje. Dla przyszłości.

-- O Boże! -- wykrzyknęła Merrial. -- Zwariowałeś! Ta rzecz to diabeł!
Zaprowadzi świat do nowego Posiadania, zanim się obejrzysz!

-- To będzie nasze Posiadanie -- powiedział Fergal.

-- Masz na myśli Twoje!

Fergal wyprostował nogi.

-- A co z~tym byłoby złego?

Spojrzał na nasze zbulwersowane twarze i~wybuchnął śmiechem.

-- Nie bądźcie głupi -- powiedział. -- Nie ma sposobu, żeby to zrobiło
cokolwiek bez ludzi, z~którymi może współpracować, a~jeszcze nie
istnieją tacy ludzie. -- Położył kciuk na brodzie Merrial na chwilę. -- Jak doskonale wiesz.

Odtrąciła jego dłoń, niezbyt delikatnie.

-- To nie było śmieszne -- powiedziała. Wstała z~chwiejną godnością. -- Idę
się wysikać.

Fergal obserwował mnie obserwującego ją przedzierającą się przez tłum.
Jeżeli wykrył zamęt w~moich myślach, nie dał żadnego znaku.

-- Nie ma szansy na przekonanie Ciebie, Clovis?

-- Nie ma mowy -- powiedziałem, ciągle rozkojarzony. Jego niezobowiązujące
przekomarzanie nie oszukało mnie ani na sekundę. To był człowiek, który
chciał władzy, Posiadanie w~istocie, a~jego obecny plan z~AI nie byłby
jego ostatnim. Był człowiekiem, które musiałbym obserwować, a~może
pewnego dnia zabić.

-- Och, dobrze -- powiedział. -- Nasz dzień nadejdzie i~Ty go zobaczysz.

Miałem to zakwestionować, gdy poczułem dłoń na ramieniu.

-- Och, witaj, Catherine.

Moja była gospodyni uśmiechnęła się do mnie. Jak każdy tutaj, już była
lekko pijana. Skinęła głową Fergalowi i~spojrzała na mnie.

-- Cześć, Clovis. Mam nadzieję, że lubisz swoje nowe mieszkanie.

-- Och, tak.

Sięgnęła do torby na biodrze. 

-- Mam coś dla Ciebie -- powiedziała. -- List, który przybył kilka dni temu, nie przyjechałam\ldots

-- W~porządku -- powiedziałem, zabierając wypchaną kopertę. -- Dzięki.

Fergal, może stonowany odmową, ponuro studiował drinka lub taktownie
szanował moją prywatność, gdy otwierałem paczkę. Z~ręcznie wypisanego
adresu, wiedziałem, że była od Gantry'ego. Zawierała list i~grubą
broszurę. List był ładnie wypisany. Przejrzałem przewidywalne
załamywanie rąk nad moim wydaleniem z~Uniwersytetu (rozprawa była farsą,
nie żebym dbał) i~o moim wyborze kariery jako majsterek. Potem
przeszedłem do kolejnej strony.

\textit{Jednakże, Clovis, jako tylko drobna pamiątka radości badań
historycznych, może przypominasz sobie, że wyglądałem nieco zdziwiony,
gdy przedstawiłeś swoją dziewczynę, Merrial? Powodem było to, że
myślałem, że skądś ją rozpoznałem. Właściwie, oczywiście, nie
rozpoznałem, ale napotkałem zdjęcie kogoś,kto może być jej przodkiem o~tym samym imieniu, w~jednym ze starych roczników instytutu, w~istocie z~2058 roku. Mogłeś go nawet sam kiedyś przeglądać. Spójrz na stronę 35,
podobieństwo jest całkiem uderzające. (Nie muszę dodawać, że oczekuję
zwrotu\ldots }

Prawie upuściłem dokumenty, gdy niezdarnie otwierałem broszurę i~szukałem strony. Pokazywała, w~ostrzejszym szczególe i~lepszym kolorze
niż współczesne fotografie, jakąś imprezę towarzyską. Ludzie siedzieli,
elegancko ubrani, przy długich stołach, klaszcząc, gdy inni w~ich
towarzystwie tańczyli. Na pierwszym planie była dziewczyna, uchwycona w~pół obrotu, jej gęste czarne włosy poruszające się za jej głową, jedną
dłonią kołysząc jej długą spódnicą na bok, jej bose stopy lekko,
precyzyjnie ustawione. Świetna tancerka. Merrial.

Nawet była opisana, małym drukiem w~podpisie.

To mógł być przodek, próbowałem sobie powiedzieć, jak myślał Gantry.
Jednak wiedziałem, że tak nie jest. Jeżeli ktokolwiek mógł być
zidentyfikowany z~fotografii, to tylko Merrial. Wyglądała na zdjęciu nie
inaczej niż dzisiaj.

Myślałem o~niej, od pierwszego momentu, kiedy ją ujrzałem, jako
młodszej, bardziej ognistej, świeższej niż ja i~przypisywałem jej
sporadyczne ironie i~niegodziwie inteligentne uwagi jej własnemu
rozumowi, który szczęśliwy uważałem za większy niż mój własny. Było
szokiem zrozumienie, że była to mądrość wieku. Dobry Boże, ile lat
miała? Żyła od czasów Wyzwolicielki! Ta myśl była wystarczająca, żeby
zakręciło mi się w~głowie.

Gantry miał rację co do jednej rzeczy, widziałem to zdjęcie wcześniej,
przy leniwym przeglądaniu archiwów PR Instytutu. I, tak jak
przewidywałem, wspomnienie wróciło. Była to kilku sekundowa pauza, gdy
przewracałem strony, kilka lat wcześniej, moja uwaga na chwilę złapana
tym pięknym obrazem z~przeszłości.

Głos Fergala wpadł w~moje przerażone myśli.

-- Złe wieści z~domu?

Potrząsnąłem głową, składając list dookoła broszury, wkładając kartki do
koperty i~wsuwając je do kieszeni.

-- Nie, nie -- powiedziałem, wymuszając uśmiech. -- Nic z~tych rzeczy. To
tylko, czuję się słaby, myślę, że za dużo wypiłem na pusty żołądek,
wiesz?

Zasłoniłem dłonią usta.

-- O Boże. -- Przełknąłem. Sceptyczne, ironiczne oczy majsterka oceniały
mnie. Zrozumiałem, że jeszcze muszę zdecydować co zrobić z~drugą
szokującą wiadomością, przekazaną kilka minut temu: że on, zapewne
zgodnie z~oczekiwaniami Merrial, wprowadził AI na Statek. Wszystko to,
co odsłoniłoby go, i~rozwaliło jakiekolwiek plany, jakie on lub oboje
mieli, było słowo do Druina\ldots

-- Jesteś pewny, że wszystko w~porządku?

-- Tak, będzie dobrze. Po prostu potrzebuję świeżego powietrza. Wychodzę.
Czy mógłbyś powiedzieć Merrial, żeby też wyszła?

-- Jasne -- powiedział, już przyglądając się tłumowi za innym
towarzystwem. -- Gdzie będziesz?

-- Na placu -- powiedziałem. -- Pod pomnikiem.


\chapter{Końcowa Analiza}

Zatem do Ałma-Aty, kwitnące jabłonie na ulicach, dym w~powietrzu i~góry
Tien-szan w~oddali, tak wysokie, tak bliskie, że dla oka wydawały się
nieprawdopodobne, jak księżyc na horyzoncie. Myra prawie podskakiwała z~ulgi powrotu do Kazachstanu.

Biuro prezydenta Chingiz Sulejmanowa było znacznie większe niż Myry.
Czuła niepokój, gdy mijała żołnierza, który trzymał dla niej drzwi.
Dziesięciometrowy pas czerwonego dywanu na polerowanym parkiecie na
końcu, którego było małe krzesło przed wielkim biurkiem. Krzesło było
plastikowe. Biurko było mahoniowe, wyłożone zieloną skórą, puste prócz
złotego pióra Mont Blanc i~dziewiczej, wykańczanej czerwoną skórą
suszki. Przeszklone regały na książki po obu stronach pokoju zbiegały
się ku szerokiemu oknu z~widokiem na góry. Centralny żyrandol pokoju,
niezapalony w~tym momencie, wyglądał jak lądownik starożytnej i~imponującej cywilizacji właśnie się przedstawiającej.

Prezydent wstał, gdy się zbliżała i~obszedł onieśmielające biurko.
Powitali się uściskiem dłoni. Sulejmanow był niskim, dobrze zbudowanym
Kazachem z~twarzą, którą troskliwie utrzymywał jako dobrodusznie
wyglądający pięćdziesięciolatek. Obecnie miała pięćdziesiąt osiem lat,
dziecko wieku, jak okazjonalnie wspominał, co oznaczało, że dorastał,
gdy Chwalebna Kontrrewolucja 1991 roku przeszła do historii.
Zjednoczenie Kazachstan w~Jesiennej Rewolucji było jego najwspanialszym
okresem i~zawsze nazywał siebie Kazachstańczykiem, nie Kazachem:
tożsamość narodowa, a~nie etniczna. Nie posiadał żadnych z~dwudziestowiecznych lewicowych blokad jak Myra. Nigdy nie miał
najmniejszej aspiracji bycia socjalistą jakiegokolwiek rodzaju. Jednak,
podążał za sowiecką tradycją noszenia najładniejszego i~najbardziej
konwencjonalnego garnituru, jaki mógł dostać za dolary.

-- Dzień dobry, obywatelko Dawidowa -- powiedział po rosyjsku.
Odpowiedziała podobnie, a~potem gestem wskazał jej krzesło i~zajął
swoje. Żołnierz zamknął drzwi.

-- Ach, Myra, mój przyjacielu -- powiedział Sulejmanow, tym razem
angielskim jak w~BBC World Service -- porzućmy formalności. Czytałem
Twoje raporty o~Misji. -- Wskazał dłońmi, jakby pozwalając książce się
otworzyć. -- Co za bałagan. Choć muszę powiedzieć, że wyglądasz dobrze.

-- Przepraszam, że nie odniosłam sukcesu, prezydencie Sulejmanow\ldots

-- Chingiz, proszę. I~nie ma powodu do przepraszania. -- Ścisnął mostek
nosa, zamykając na chwilą oczy. Wyglądał na zmęczonego. -- Nie widzę, jak
ktokolwiek mógłby zrobić to lepiej. Twoje opuszczenie Wielkiej Brytanii
było może \ldots porywcze, ale nawet z~perspektywy czasu prawdopodobnie
okaże się, że to było najlepsze rozwiązanie. Jak nisko upadli, ci
Anglicy. Co do Amerykanów, cóż, co mogę powiedzieć? -- Zachichotał z~pewną \textit{schadenfreude} i~spojrzał w~górę na kryształowy statek
obcych. -- Piętnaście lat temu wyciskali swoją wolę na całej planecie, a~teraz trochę broni atomowej jest dla nich zbyt kłopotliwe. W~czasach
mojego ojca, byli skłonni rozważać przyjęcie wielu uderzeń nuklearnych.
-- Spojrzał na Myrę, wracając ze wspomnienia. -- Przepraszam -- powiedział,
nagle zawstydzony -- nie zamierzałem obrazić. Zapominam czasem, że byłaś,
\textit{jesteś}, Amerykanką.

-- Bez obrazy -- powiedziała Myra. -- Całkowicie się zgadzam z~Twoją oceną.
Jaką kupą gówna stało się to miejsce! Co za żałośni ludzie! Szansa na
długie życie tylko sprawiła, że bardziej boją się śmierci niż
kiedykolwiek.

Krzaczaste brwi Prezydenta się poruszyły. 

-- Nie stało się tak dla
Ciebie?

Myra potrząsnęła głową. 

-- Widzę racjonalność tego, ludzie myślą, że mają
dłuższe życie do straty, skoro mogą oczekiwać długiego, ale myślę, że
jest to fałszywa logika. Ostatecznie, długie życie ucisku lub wstydu
jest gorsze niż krótkie życie.

Przerwała i~spojrzała na niego zagadkowo. Uśmiechnął się.

-- Prawda, nie jesteśmy tutaj, by rozmawiać o~filozofii -- powiedział. -- Niemniej jednak, cieszę się, że myślisz, że lepiej umrzeć wolnym niż żyć
jako niewolnik. Możemy mieć szansę pewnego dnia, ale spróbujmy trochę
opóźnić nasze, heroiczne śmierci, co?

-- Tak, rzeczywiście. -- Chciała mocno zapalić, ale Prezydent był za
czystym życiem.

-- Bardzo dobrze -- powiedział Chingiz. -- Coś, czego Ci wcześniej nie
powiedziałem\ldots zorganizowałem aktyw z~podobnie istotnymi
doświadczeniami, żeby spróbowali podobnych podejść do rządów Francji,
Turcji, Brazylii i~Guangdong\footnote{ obecnie południowo-wschodnia prowincja
Chińskiej Republiki Ludowej. Jest najliczniej zaludnioną prowincją w~Chinach. Jej przestarzała nazwa pochodząca z~transkrypcji francuskiej
brzmi Prowincja Kanton,
por.~\url{https://pl.wikipedia.org/wiki/Guangdong} -- przyp.tłum.}. Natrafili na podobny brak zainteresowania. Zatem musimy
stawić czoła Szinosowym na własną rękę. Nie muszę Ci mówić, że nie mamy
szans na dłuższą metę.

-- Mam sugestię -- powiedziała Myra. -- Jeżeli Zachód nie chce nam pomóc,
to do diabła z~nimi. Dogadajmy się z~Sinosowietami! To, co chcemy, to
naszej terytorialnej całości, ich wycofania z~Semeju i~dostępu do
rynków, szlaków handlowych i~zasobów Byłego Związku. To, co oni chcą,
przypuszczalnie, jest przejście przez lub północ Kazachstanu, jak będą
kierować się na zachód ku Ukrainie, który jest najbliższym miękkim
celem, ale ciągle takim, którego wchłonięcie zajmie wiele lat, może
dekad. Nie sądzę, żeby byli gotowi zabrać się za Moskwę lub Turcję.
Wydaje mi się, że te cele nie są niezgodne.

-- Tak, tak -- powiedział Chingiz -- opcja naszej zmiany stron przyszła mi
do głowy, a~także Ministrowi spraw zagranicznych. Problem jest taki, że
nikt nigdy nie ,,dogadał się'' z~Szinosowami. Nie mają przywódcy lub
nawet kierownictwa, przynajmniej nikt, kto jest znany światu. Są w~rzeczy samej hordą, bez Wielkiego Khana jak mój imiennik. To sprawia, że
trudno się z~nimi ugodzić, w~każdym sensie.

-- Ach, no weź -- powiedziała Myra, czując się odważniej. -- Nawet
anarchiści mieli swoich Machno. Nie wierzę, że horda bez kierownictwa
mogłaby osiągnąć to, co oni, nawet w~warunkach wojskowych. Stosowanie
taktyk guerilli na poziomie strategii i~konfrontacji głównych sił to
nowość, ale wymaga precyzyjnej koordynacji. Nie dzieje się tutaj nic
losowego.

Chingiz na chwilę zacisnął usta. Pokręcił głową. 

-- System bez centrum
może osiągnąć więcej niż intuicyjnie oczekujemy, Myro. To w~końcu jest
lekcją dwudziestego wieku, tak? Działa w~ekonomii, przyrodzie i~też do
pewnego stopnia w~sprawach wojskowych.

-- Słuszna uwaga -- powiedziała Myra. Nie chciała mówić o~obłąkanej plotce
Zielonych o~Generale na tym poziomie konwersacji. -- Załóżmy, że nie mają
kierownictwa. Aby zachować współpracę, którą pokazują, muszą posiadać
łączność horyzontalną pomiędzy jednostkami i~jakąś metodę dochodzenia do
wspólnej odpowiedzi \ldots nawet jeżeli jest to tylko jakiś społeczny
równoważnik pobudzenia i~hamowania sieci neuronowych. W~tym przypadku
jakakolwiek oferta złożona dostatecznie dużej jednostce zostałaby
upowszechniona wśród innych tak jak odpowiedź. Ciągle byłoby to warte
zachodu skontaktowanie się z~nimi.

-- Hmm -- powiedział Chingiz. Splótł palce. -- Co proponujesz? Wyjść im
naprzeciw, aż zauważą, potem rozmowa z~pierwszą osobą zdolną do
zrozumienia Ciebie?

-- O to chodzi.

-- Brzmi niebezpiecznie, poza wszystkim innym.

-- Właściwie, proponuję wcześniej ogłosić moje intencje, przez
jakiekolwiek kanały, jakie mamy, potem udać się do Semeju.

-- Proszę, proszę -- powiedział Chingiz. -- Nie jest tak źle, jeszcze. Możesz ciągle tam polecieć, bezpośrednio.

-- I~wylecieć?

-- Och, tak. Kontrola ruchu lotniczego ciągle działa. Tak jak radio i~telewizja, na wybranych kanałach. To tylko interfejsy komputerowe są
blokowane albo przez fizyczne przecinanie linii telefonicznych albo
zagłuszanie elektromagnetyczne. To niesamowicie zróżnicowane
rzeczy, bardzo bystre. Nie moglibyśmy tego zrobić.

Popatrzyła na jego spokojną twarz.

-- Jakie dostajemy raporty?

-- O życiu pod Szinosowami? Ha. Pod niektórymi względami życie toczy się
normalnie. Nie ma działań ludobójczych. Są to, co Szinosowieci nazywają
\textit{reformami}. Demokracja w~pracy i~tak dalej. W~tym zakresie są
bardzo natarczywi. Wiele przedsiębiorstw zależnych od sieci upada albo
przeorientowuje się na system wewnętrznej komunikacji Szinosowów,
czymkolwiek to jest, albo zabierają zabawki i~idą albo są wywłaszczani
na podstawie zaniechania. -- Potarł dłonie. -- Nie trzeba dodawać, to daje
naszej republice chwilowy napływ ludzi, kapitału, sprzętu
komunikacyjnego i~komputerów. Niektórzy uchodźcy są w~nędzy, ale
niewielu.

-- Ktoś chce dołączyć do walki?

-- Muszę powiedzieć, że nie ma masowego zaciągu do naszych sił zbrojnych.
Zwykłe \textit{emigranckie} dywersje, czyli planowanie, błaganie,
przygotowywanie ekspedycji sabotażu, dyskretny terroryzm. Nie popieramy
tego. -- Potarł palcem bok nosa. -- Oczywiście, próbujemy zapobiegać\ldots
według naszych możliwości, ale nasze zasoby są całkiem nieodpowiednie do
takiego zadania.

-- Ależ oczywiście. -- Myra się uśmiechnęła. -- Mógłbyś mi załatwić
\textit{mudżahedinów}? Dwóch lub trzech dobrych ludzi, nie fanatyków, nie
samobójców, ale gotowych podjąć ryzyko i~polec, jeżeli trzeba. Jestem
ciągle głęboko niechętna lotowi do Semeja. Zbyt dużo możliwości na
dogodną awarię mechaniczną, szczerze, staję się trochę paranoiczna o~wszystko, co jest kontrolowane przez komputer, po obu stronach. Więc
jeżeli mogę, chciałabym pojechać z~ochroniarzami.

Chingiz uniósł brwi. 

-- Całą drogę?

-- Nie, nie. Polecieć do Karagandy, ogłosić, co robię, potem pojechać do
Semeja, omijając MRRNT.

-- Ach, tak. -- Poprawił niektóre włosy na kosmatej brwi. -- Drobna lokalna
trudność. -- Brzmiał pełen wyrzutu.

-- Sytuacja jest pod kontrolą -- powiedziała Myra.

-- Być może. Jednak, w~sumie, chciałbym zasugerować, żebyś tam nie
wracała, lub omijała dżipem przez Poligon. Znacznie bardziej niebezpieczne niż lot. -- Podniósł
dłoń, uspokajając jej początkowy protest. -- Wiem, co masz na myśli o~komputerach i~kontroli lotu. Również o~tym myślałem. Dostaniesz swoich
ochroniarzy. Ogłosisz zamiary, polecisz do Semeja, potem powłóczysz się,
aż ktoś się skontaktuje, co, jak mówisz, na pewno się wydarzy.
Przekażesz propozycję i~poczekasz na rozwój wydarzeń. Potem polecisz z~Semeja do Kapicy i~albo ogłosisz, że konflikt jest załatwiony lub
zmobilizujesz swoich ludzi do wspólnej obrony. -- Uśmiechnął się lekko. -- Tak czy inaczej, Twoje wewnętrzne problemy polityczne się skończą.
Zewnętrznie, jednak, może się okazać, że Sinosowiety nie są naszym
bezpośrednim problemem\ldots

-- Ach, tak -- powiedziała Myra. -- Kolejny krok. Przypuszczalnie
przynajmniej jeden z~tych krajów, którym złożyliśmy ofertę, zacznie się
martwić, co zrobimy z~atomówkami i~opcja rozbrojenia nas dość szybko
pojawi się w~porządku dziennym.

-- Dokładnie -- powiedział Chingiz. -- Zespoły amerykańskich Przestrzeniowców
są tymi, którymi prawdopodobnie powinniśmy się najbardziej martwić, jak
Twój przyjaciel w~Nowym Jorku powiedział, przemysłowcy i~koloniści w~kosmosie są zrozumiale nerwowi w~tym zakresie.

-- Teraz to Twoje atomówki -- powiedziała Myra. -- Zrobimy tak, jak
powiesz. Przypuszczanie będziesz chciał je powstrzymać i~przekazać kody
operacyjne.

Chingiz uderzył pięścią w~masywne biurko, aż Myra podskoczyła.

-- \textit{Nie!} -- powiedział. -- Nie będziemy pomiatani. Nie zamierzamy
poddać naszych atomówek bez gwarancji pomocy wojskowej. I~jesteśmy
gotowi zagrozić nuklearnym atakiem wobec dowolnego ataku.

-- Więc jesteś gotów pojechać po bandzie?

-- Absolutnie -- powiedział Chingiz. -- Po bandzie. Jednak nie dalej.

-- Dobrze -- powiedziała Myra. -- Poprzemy Cię. Zobaczymy, kto mrugnie
pierwszy.

-- Dziękuję -- powiedział Chingiz. Jego twarz lekko się zrelaksowała. -- Wiem, że to bardzo ryzykowna strategia. Jednak końcówka jest przed nami,
a ja nie mam zamiaru wchodzić w~nią bezbronny.

Myra skinęła głową.

-- Najlepsze, co możesz zrobić -- powiedziała -- to działać jakbyś był
gotów nas zostawić, MRRNT. Potępić i~wyprzeć się nas, prywatnie
oczywiście na gorącej linii, i~przekonać ONZ, USA lub kogokolwiek do
negocjowania bezpośrednio z~nami. To powinno kupić nam trochę czasu.

-- Tylko jeżeli będą wierzyć, że jesteś dostatecznie szalona, żeby to
zrobić.

Myra odsłoniła zęby. 

-- Będą.

~

Semipałatyńsk, lub Semej, był dostatecznie przyjemnym miastem, którego
położenie na stepie pozwoliło mu się rozłożyć tak bardzo, że nawet jego
wysokie budynki wyglądały na niskie, nawet węższe ulice były szerokie.
Na tych szerokich ulicach było miejsce na drzewa, których zakurzone
liście były przedmiotem monitorowania licznikiem Geigera w~trakcie jej
pierwszej wizyty, w~późnych latach osiemdziesiątych. Dobre dawne czasy
Stowarzyszenia Nevada-Semipałatyńsk przeciwko testom nuklearnym. Z~wszystkich zdrad, które popełniła przeciwko jej młodości, ta bolała
najbardziej. Marksizm, trockizm i~socjalizm mogły pójść precz. To był
nieubłagany, naiwny, humanistyczny internacjonalizm tego protestu, jego
niepodważalne medyczne i~statystyczne podstawy, jego czysty, cholerny
skandal zakorzeniony w~biologii raczej niż w~ideologii, które był jej
najczystszym, najokrutniejszym płomieniem. Uważała broń jądrową za
najpodlejsze dzieło człowieka, których samo posiadanie skaziło i~których
samo testowanie było mordercze.

Nurup Kerbajew i~Mustafa Ałtynsarin, jej dumnie kontrrewolucyjni
ochroniarze, szli uprzejmy krok lub dwa za nią, brody i~bandoliery
najeżone, Kałasznikowy zawieszone na ramieniu. Nurup etnicznie był
kazachskim Rosjaninem. Mustafa wyglądał bardziej po azjatycku, prawie jak
Chińczyk Han. Z~ich AK, workowatymi spodniami, zdartymi butami i~wypchanymi kurtkami, wyglądali prawie jak kontrrewolucjonistyczni
bandyci. Wyglądali także jak żołdacy Szinosowów lub tubylcy, których
Szinosowy zachęcały do noszenia broni jako odstraszania
kontrrewolucyjnego bandytyzmu.

Szli wzdłuż ulic i~przez place całkiem niekwestionowani, choć jeden czy
drugi spoglądał na Myrę z~zaciekawieniem, jakby rozpoznawał ją z~jej
wystąpień telewizyjnych poprzedniego wieczoru. Oprócz zaparkowanych
czołgów na rogach ulic, dookoła których ciekawski tłum, w~większości
dzieci i~młodych ludzi, bratał się ze zrelaksowaną załogą, miasto nie
pokazywało oznak złapania przez społeczną rewolucję. To dziwne maszyny
walczące były niepokojące. Chodziły majestatycznie i~przechylały się jak
marsjańscy najeźdźcy, ale tubylcy traktowali je ze zwykłą zażyłością,
jak ruch drogowy czy małą architekturę. Może, myślała krzywo Myra, brak
palących promieni i~wijących się metalowych macek to załatwiał.

Jak również te drony bitewne, wielkie niezdarne maszyny liczące były
instalowane, w~wystawach sklepów, wewnątrz fabryk, na zewnątrz na
placach. Kółka, zęby, sfery kryształowe, tworzyły szalone planetaria
alternatywnych systemów słonecznych, kopernikańskich z~ptolemeuszowskimi
epicyklami. Nanotech wyciekał i~zastygał dookoła mosiądzu i~stali, jak
epoksyd, który nie wysechł do końca. Koło południa Myra i~jej towarzysze
obserwowali jedno będące wyciągnięte z~ciężarówki i~ustawione ostrożnie
na placu przy posągu kosmonauty.

-- Kurwa, dziwne -- powiedziała Myra, częściowo do siebie, gdy aktyw
szinosowiecki wspiął się na posąg i~zaczął wyjaśniającą przemową po
uzbecku, w~jednym z~jej nieznanych języków.

-- Tym będą mogli zastąpić rynek -- zadrwił Nurup, pod nosem. -- Boże, pomóż
nam wszystkim.

Ożywiony rynek napoi bezalkoholowych i~gorącego jedzenia już się tworzył
dookoła dziwnego urządzenia. Nurup i~Mustafa przynieśli jej kolę i~kebaby, a~sobie po hot dogu. Obaj rozmawiali ciszo z~kramarzami.
Zabierając jedzenie, usiedli na ławce i~jedli.

-- Jest dużo niezadowolenia -- powiedział Mustafa skwapliwie.

-- Bazarowe plotki -- powiedział Nurup. -- Straganiarze wszystko ci
powiedzą. Powiedzą Szinosowym, że ich kochają.

Obaj mężczyźni kłócili się pośrednio, ale intensywnie przez kilka minut
na temat prognoz działań terrorystycznych przeciwko Szinosowietom.

-- Nie jesteśmy tutaj po to -- przypomniała im Myra. Poczęstowała
papierosami, potem razem wyszli z~placu. Żaden z~mężczyzn nie pytał o~ich losowe chodzenie po ulicach, póki nie zakończyli na brzegu szerokiej
rzeki Irtysz. Mieszkania na przeciwnym biegu, promenada po tej stronie.
Mały parowiec sapał w~dół rzeki, przeprawiając maszynę kalkulacyjną na
pokładzie widokowym.

Myra wychyliła się nad balustradą, patrząc w~rzekę. Dwóch mężczyzn
oparło się o~barierkę, patrząc w~innym kierunku. Ludzie ich mijali. Po
kilku minutach tego, Mustafa zapytał co się dzieje.

-- Nic -- powiedziała Myra, nie odwracając się. -- Lub może coś. Zakładam,
że byliśmy śledzeni lub obserwowani. Jestem całkiem przygotowana czekać
tutaj przez kolejną godzinę. Rozgośćcie się.

Jednak byli zbyt nerwowi, zbyt czujni, żeby tak robić. Jedynie zapalili
po kolejnym jej Dunhillu. Myra zsunęła opaskę i~od razu została trafiona
poczuciem \textit{deja vu}, gdy cała sceneria wokół niej zamgliła się,
zaśnieżona szarymi pyłami. Po chwili zrozumiała źródło tego poczucia
rozpoznania, przypomniało jej to gdy pierwszy raz widziała miasta takie
jak to, w~latach dziewięćdziesiątych: przez mgłę sowieckiego
zanieczyszczeń przemysłowych. Mrugnęła, poruszyła opaską w~górę i~w dół,
próbowała wykryć sieci. Nic, prócz szarego śniegu. Nawet Parvus, wezwany
z pamięci, wyglądał na sparaliżowanego.

Zagłuszanie szinosowieckie. Gówno.

Właśnie miała skończyć ten eksperyment, kiedy usłyszała swoje imię.
Odwróciła się. Shin Se-Ha i~Kim Nok-Yung szli koło siebie chodnikiem,
machając do niej.

-- W~porządku -- powiedziała nagle spiętym ochroniarzom. -- Znam tych
ludzi.

Przywitała się, uśmiechając, z~Koreańczykiem i~Japończykiem,
przedstawiła im Kazachstańczyków. Dyskretne komplementy o~jej
odmłodzonym wyglądzie zostały wymienione za jej podziw dla ich teraz
zdrowszej powierzchowności. Nawet ich relatywnie ludzkie uwięzienie
naznaczyło ich, przyciskając ich z~czymś, co ich nowa wolność, jeżeli to
była wolność, umożliwiła im zrzucić. Szli wyżsi. Nie speszyli się
spotkaniem z~kazachstańskimi \textit{emigrantami}.

-- Zatem, jesteście \textit{Szinosowami} -- powiedział Mustafa,
zdegustowany.

-- Odpuść sobie -- powiedziała Myra. -- Oni są ok. Musimy pogadać.

-- Tak -- powiedział Nok-Yung. -- Musimy pogadać.

Był to łagodny dzień jak na tę porę roku. Nie pogoda na krótki rękaw,
ale przyjemna, jeżeli ubierzesz się ciepło tak jak oni. Myra wskazała
półkrąg ławek w~wybetonowanym rejonie piknikowym nieco dalej na brzegu.
Obaj ekswięźniowie wzruszyli ramionami, potem skinęli.

Nok-Yung i~Se-Ha usiedli po jej obu stronach, dwaj ochroniarze na
oddzielnych ławkach kilka metrów dalej. Dzieci, ciasno owinięte w~pikowanych kurtkach i~watowanych spodniach, podskakiwały i~krzyczały,
niepomne dorosłych.

-- Zatem jak sobie radzicie, w~tym nowym wspaniałym świecie? -- spytała
Myra.

-- W~porządku -- powiedział Nok-Yung, jego towarzysz kiwając wyraźnie
głową. -- Nasze rodziny wkrótce do nas dołączą, a~w~międzyczasie mamy
dużo do zrobienia.

-- Jesteście zatrudnieni? -- Myra się uśmiechnęła.

-- Nie ma \textit{zatrudnienia} -- powiedział wyszukanie Se-Ha. -- Jest
praca. Zostaliśmy\ldots dokooptowani i~zostaliśmy wysłani porozmawiać z~Tobą.

-- Cóż, domyśliłam się, że nie jest to przypadek -- powiedziała Myra. -- Ale nie oczekiwałam, że ujrzę was już jako aktyw Szinosowów.

-- To otwarty system -- powiedział Nok-Yung. -- Interesujące zasługi są
szybko podejmowane, wzmacniane, dyskutowane.

-- Zatem przeciwieństwo sieci -- powiedziała Myra. Roześmiali się.

-- I~przeciwieństwo systemu leninowskiego -- powiedział Nok-Yung poważnie.
-- Kiedy się dostaniesz, to się \textit{dostałeś}, nie ma \ldots praktyki?
Żadnego kandydowania, wspinania się na szczyty. Przeszłe doświadczenie -- dodał raczej zadowolony -- \textit{się liczy}.

Myra spojrzała spod oka. Bez wątpienia bojowi i~marksistowscy matematycy
szybko znaleźli swoje nisze. 

-- Jestem pewna, że to jest fascynujące -- powiedziała. -- Ale jestem tutaj, aby przedstawić dyplomatyczną ofertę
Związkowi Chińsko\dywiz Radzieckiemu jako całości. Czy mogę, tylko rozmawiając
z Wami?

-- Tak.

-- Doskonale. -- Przedstawiła ją im, prosto: umowa, korytarze
eksterytorialne. Niech rewolucyjna horda opłynie Kazachstan, jak powódź
skałę i~potem zatopi resztę świata, jakby ją to nie obchodziło. (Mogłoby
rozpaść się na kawałki, tego nie powiedziała, ale tego oczekiwała).

Słuchali uprzejmie, co jakiś czas pytając o~uszczegółowienie, notując,
rysując mapy na łupkach trzymanych w~dłoni -- choć oczywiste urządzenia
przetwarzające informacje -- wyglądały jakby były zrobione z\ldots łupków.
Se-Ha wstał.

-- Muszę skonsultować -- powiedział, kiwnął głową i~raźno odszedł.
Nok-Yung przyjął papierosa, odchylił się bujnie, rozwalając się z~łokciami na ławce. Obserwowała Myrę przez zmrużone oczy i~wirujący dym.

-- Dlaczego opierasz się Związkowi, Myro? -- spytał łagodnie. -- To tylko
demokracja. To tylko socjalizm. Środek, i~cel, zgodne w~końcu, po
wszystkich katastrofach i~zbrodniach popełnionych w~imię obu. -- Rozłożył
dłonie. -- Tutaj nie ma tajemnic, nie ma oszustw. Kiedy byłaś tak młoda,
jak wyglądasz -- uśmiechnął się -- myślałabyś o~tej rewolucji, tym
wyzwoleniu cudowniejszym niż najdziksze sny.

-- Nie pozwól usłyszeć tego moim przyjaciołom \textit{mudżahedinom}! -- ostrzegła, częściowo żartem. Spojrzała na Nurupa Kerbajewa. Uśmiechnął
się, oczy i~zęby błysnęły jak noże.

-- Ale masz rację -- kontynuowała. -- Powiedzmy\ldots mogę znów wyglądać
młodo, ale miałam długie, długie życie w~międzyczasie. Zaczęłam wierzyć
w siebie i~w\ldots mój kraj, Kazachstan. I~nie będę wchłonięta ani my nie
będziemy. -- Machnęła dłonią dookoła. -- Ci ludzie, mogą wydawać się\ldots
szczęśliwi, by zaczekać i~zobaczyć. Jednak głęboko w~środku, nie, tuż
pod powierzchnią, wrą od podejrzeń. Nie są waszymi Mongołami czy
Syberyjczykami, którzy Bóg wie, że mieli źle za Stalina, ale odkryli, że
wszystko potem było gorsze. Dla Kazachów socjalizm znaczy ,,tragedia''
lat trzydziestych dwudziestego wieku: przymusowe osiedlanie, głód.
Oznacza testy jądrowe, raki, wady wrodzone. Nie chcą być przedmiotem
kolejnych eksperymentów. A jeżeli chcecie wskazać na MRRNT jako
kontrprzykład, to był wyjątek. Samo wybrana malutka mniejszość. Nasz
socjalizm zawsze był żartem, bardziej czarnym humorem niż czerwonym.
Trockizm w~jednym kraju, śmiech na sali!

Gdy się roześmiała, sama się przestraszyła. Jedno z~biegających dzieci
bawiących się dookoła zatrzymało się, włożyło kciuk do ust i~uciekło.

-- Kierujemy łagodnym kapitalizmem państwowym, nic więcej -- kontynuowała.
-- W~waszym przypadku, mój przyjacielu, to nawet nie jest to. Boże, czuję
się zdegustowana sobą za to, co zrobiliśmy, że kiedykolwiek pozwoliliśmy
być klientami przeklętych, prywatnych Gułagów Reida.

Nok-Yung patrzył przez chwilę w~niebo. 

-- Nie wiem, co powiedzieć, Myro -- powiedział w~końcu. -- Twój żal co do obozów Ochrony Wzajemnej jest \ldots
dobrze odebrany. Jednak co do innych kwestii, musisz z~pewnością
wiedzieć, że nic z~tego, co mówiłaś, ZSRR i~tak dalej, nie jest
socjalizmem takim, jak my go rozumiemy i~jak Ty go rozumiesz. Zatem
przestań mieszać wątki.

-- Och, jestem doskonale świadoma, że jesteście inni. Że możecie być
oryginalnym artykułem: Marks i~Engels, Posiadacze. I~wiesz co? Nie dbam
o to. Nie chcę tego ani dla siebie, ani dla nikogo.

-- Dlaczego nie? -- Nok-Yung brzmiał bardziej zdziwiony niż obrażony.

Myra wskazała ponad rzeką na owadzi kształt maszyny bojowej,
patrolującej brzeg wody krokami niczym czapla.

-- Z~powodu tych cholernych rzeczy -- powiedziała. -- I~maszyn
obliczeniowych.

-- Co! -- Oczy Nok-Yung zmarszczyły się z~zabawy. -- Luddyzm\footnote{angielski
radykalny ruch społeczny z~początkowego okresu rewolucji przemysłowej
(1811--1813), którego przedstawiciele składali się głównie z~wolnych
chałupników, rzemieślników i~tkaczy. Luddyści protestowali przeciwko
zmianom sposobu ich życia i~nowej etyce pracy,
zob.~\url{https://pl.wikipedia.org/wiki/Luddyzm} -- przyp.tłum.} nie jest Twoją prawdziwą ideologią, Myro. Nie wierzę w~to.
Te maszyny są jednym z~najcudowniejszych osiągnięć Szinosowów,
alternatywna nanotechnologia, opracowana całkiem niezależnie od Zachodu.
Czy wiesz, że te maszyny są skalowane, aż do poziomu molekularnego, i~wszystkie są mechaniczne, chemiczne i~optyczne, bez potrzeby interfejsów
elektronicznych? To ich, nasza, tajna broń, jawny sekret. System
komputerowy, którego nieprzyjaciel nie może spenetrować, ale każdy może
zrozumieć i~mieć dostęp. Dopiero zacząłem go używać i, powiem ci, że ma
najbardziej intuicyjny interfejs, na jaki się natknąłem. Kapitaliści
zabiliby za niego. Lub raczej, zabiliby, żeby go zmonopolizować. Ale
jest za darmo, więc nie mogą.

-- Wiem o~waszych dziwnych maszynach -- powiedziała Myra. -- CIA
powiedziało mi o~nich. -- Dotknęła skroni, uśmiechając się ironicznie. -- ,,Mam szczegółowe dane''.

Nok-Yung złapał aluzję. 

-- Wiesz, to nie jest \textit{Terminator}! A nie,
jak to było nazwane w~filmach?, Skynet. To nie jest\ldots nieprzyjazne.

-- Może nie teraz. Jednak co, jeśli jest, kiedy to, lub wy, pokryjecie
cały świat, jak figowiec?

Nok-Yung sapnął powietrzem i~dymem. 

-- Więcej luddyzmu! Maszyny stworzą
życzliwe ludzkie środowisko, drugą naturę, w~obrębie, której ludzka
natura może zakwitnąć, naprawdę, po raz pierwszy. -- Pochylił się do
przodu, mówiąc poufnie. -- Powiem ci, co zrobiliśmy, coś, czego nie
ośmieliłby się żaden inny system. Wyprodukowaliśmy nanowirusowo
rozpowszechnianą, genetycznie naprawialną wersję terapii przeciwko
starzeniu. Rozprzestrzenia się przed naszą migracją jak łagodna plaga.
Sama możesz być już zainfekowana. Dar.

-- Boże, to jest \textit{tak} nieodpowiedzialne! -- Myra był wstrząśnięta. -- Wirusy \textit{mutują}, cholera, gdybyś nie słyszał!

Nok-Yung przesunął dłoń poziomo. 

-- Nie ten. Ma wbudowane
samonaprawianie. Był przetestowany na stabilność w~milionie wirtualnych
pokoleń.

-- \textit{Wirtualne} pokolenia, tak! Człowieku, zaprojektowałeś tyle
rzeczy w~obozie, żeby wiedzieć, ile jest \textit{to} warte w~rzeczywistości!

-- Inny system, inna filozofia projektowania -- powiedział z~irytującym
zadowoleniem. -- Nasze zestawy testujące są same \textit{częścią}
rzeczywistości. To jak różnica pomiędzy działającym modelem w~skali a~symulacją. Po prostu nie ma porównania. A zasoby obliczeniowe są
olbrzymie, większe nawet od tego, co zbudowali Przestrzeniowcy.

Myra poczuła, że zatapia się w~bezdennym basenie ich pewności siebie. To
było naprawdę przerażające, to było, zrozumiała, to, czego najbardziej
się bała w~sobie, być tak pewną. Być całkowicie pewną, że się ma rację,
o ile była zaniepokojona, byłoby jej końcem. Wątpliwości były jej jedyną
nadzieją, jej pocieszeniem i~towarzyszem od dzieciństwa, jej sceptycyzm
jedynym zabezpieczeniem.

Shin Se-Ha wrócił i~usiadł, nie zauważając ich zamrożonej chwili
wzajemnego nieporozumienia. Spojrzał na Myrę, poważnie, i~pokręcił
głową.

-- Żadnej umowy, niestety.

Myra ledwie mogła w~to uwierzyć.

-- Dlaczego nie? Alternatywą jest wywalczenie drogi przez Kazachstan!
Wszystko, co mieliście zrobić w~zamian to nie walczyć z~nami! Czego
można więcej chcieć?

Se-Ha pokręcił głową smutno. 

-- To nie tak, Myro -- powiedział. -- To nie
agresja ani uraza. To po prostu imperatyw naszego sposobu produkcji.
Albo będzie globalny albo niczym, jak Twój Trocki zwykł mawiać. Musimy
działać lub upaść, póki nie spotkamy się po drugiej stronie świata.

Zrozumiał, że to nigdzie nie prowadzi. 

-- Konkretniej -- kontynuował -- nie
możemy mieć\ldots niewchłoniętych obszarów wewnątrz Unii. To byłaby zbyt
duża okazja dla naszych wrogów. I~nie możemy się zatrzymać na długo,
ponieważ to zmusiłoby nas do zaangażowania w~wewnętrzną walkę klas,
zwłaszcza z~właścicielami drobnej własności, której nie chcemy. -- Uśmiechnął się. -- Delikatnie ujmując! Na razie mogliśmy uniknąć tego
całego scenariusza dyktatury proletariatu poprzez zwykłe poniesienie
razem z~nami pozostałych drobnych i~wielkich biznesów. Ekonomia wspólnej
własności opartej na maszynach rozwija się, a~oni rozwijają się w~jej
szczelinach. Mogą żyć jak gnidy w~naszych włosach, tak długo jak się
poruszamy. Jeżeli się zatrzymamy, swędzenie będzie nieznośne. Moglibyśmy
się \textit{podrapać}.

-- Och, przestań -- powiedziała Myra. -- Możesz prowadzić mieszaną
gospodarkę w~nieskończoność. Robimy tak w~Kapicy od lat.

-- Mieszankę państwowego i~prywatnego kapitalizmu, tak -- powiedział
Nok-Yung -- jak sama wspomniałaś. Mieszanka prawdziwej, bez towarowej
gospodarki i~rynku byłaby znacznie bardziej niestabilna. Konflikty
pojawią się bardzo szybko, jeżeli obie byłyby ograniczone do tej samej
przestrzeni gospodarki.

Niestabilny system, który musi się rozprzestrzeniać z~właściwą
prędkością, żeby powstrzymać upadek. Nie za wolno, nie za
\textit{szybko}\ldots istniało wiele naturalnych, sztucznych, społecznych
analogii tego. Myra prawie zachichotała na myśl, co by się im
przytrafiło, gdyby Kazachstan się po prostu poddał, gdyby okazało się,
że Szinosowy otwierają otwarte drzwi i~potykają się na progu na ich
kolektywnych nogach.

Jednak to nie była opcja. Rozejrzała się dookoła, sprawdzając, że jej
strażnicy ciągle byli znudzeni i~czujni, potem obu nowych rekrutów
Szinosowów. Absurdalność sytuacji uderzyła ją, prowadziła dyplomację
poprzez rozmawianie z~dwoma osobami na ulicy. Z~tego, co wiedziała, oni
mogli być omamieni jako kontakt UFO, a~nie prawdziwi ambasadorzy
inteligencji obcych. Znowu poczuła impuls do zachichotania, to była
kolejna głupia idea, czuła się lekko, kapryśnie, jakby jej problem mógł
być rozwiązany. Nie \textit{widziała} żadnego rozwiązania. Wpadła w~większe kłopoty niż kiedykolwiek, ale ciągle czuła ulgę.

-- Jest to pilna sprawa -- mówił Se-Ha, trochę przepraszająco. -- Fakcje
zielonych eksperymentują z~wektorami zarazy. Grupy kolonistów w~kosmosie, Zewnętrzni, mają radykalnie post-ludzką wizję. Pomiędzy nimi,
grożą ludzkości wyginięciem. Nasz postęp jest w~istocie obronny\ldots

Spojrzała ostro na niego. 

-- Powiedz mi, Se-Ha -- powiedziała -- tam,
właściwie z~kim się konsultowałeś.

Wyglądał na zakłopotanego. 

-- To była\ldots decyzja rozproszona. Konsensus.

-- \textit{Bzdury}! -- pękła. -- Nie mów mi tego. Nie widziałam
przeprowadzonego głosowania na tych ulicach dookoła. A Ty? Zatem musi
być gdzieś przywództwo, rada. Chcę z~tym rozmawiać.

-- Rozmawiasz z~tym -- powiedział -- kiedy rozmawiasz z~nami. W~stopniu, w~jakim istnieje. Parametry polityki rzeczywiście zostały ustalone
demokratycznie, ale implementacja, \ldots decyzje administracyjnie, są
przeprowadzane\ldots -- Przygryzł dolną wargę. -- Trudno to powiedzieć -- skończył kulawo.

-- Niech zgadnę -- powiedziała Myra, wstając. -- System ekspercki. AI.

Se-Ha spojrzał na nią, oczy ciemne i~bez wyrazu pod cienkimi, czarnymi
brwiami. 

-- Tak, to jest możliwe.

Myra wyprostowała się i~westchnęła. Była przekonana, może paranoicznie,
że szalony kaznodzieja Jordan miał rację: Generał, Plan, był w~środku
tego wszystkiego, wdrożył się w~ekologię maszyn Szinosowów i~był w~trakcie przejmowania całego świata. Bez wątpienia z~najlepszymi
intencjami.

-- Boże, tak, macie rację -- powiedziała. -- Czyli albo wy albo Zewnętrzni.
Obie strony są jak jebany Borg. ,,Zostaniesz zasymilowana'', czy nie
to właśnie mi mówicie?

Nok-Yung wzruszył ramionami. 

-- To nie jest coś złowieszczego. Wszyscy
żyjemy w~światowej machinie. Dlaczego nie żyć w~machinie, która jest po
naszej stronie?

Myra musiała się uśmiechnąć. 

-- Chcecie, żebym wyobraziła sobie
przyszłość -- powiedziała -- jako socjalizm z~ludzką twarzą, na zawsze?

-- Tak! -- powiedzieli obaj, zadowoleni, że w~końcu zrozumiała.

Naprawdę będzie ciężko zakończyć tę rozmowę uprzejmie, ale spróbuje.

-- Zabiorę waszą wiadomość do Prezydenta Sulejmanowa -- powiedziała. -- Bez
wątpienia będziecie oczekiwać na naszą odpowiedź.

Se-Ha i~Nok-Yung wstali i~uroczyście się pożegnali.

-- Do widzenia -- powiedziała.

-- Do widzenia -- odpowiedzieli obaj.

Se-Ha uśmiechnął się psotnie. 

-- Mam nadzieję, że jeszcze cię zobaczę.

~

Wynajęli samolot, odrzutowiec biznesowy, który pamiętał lepsze czasy, w~Ałma-Acie. Myra nie mogła znieść usunięcia pasażerów z~lotów z~Semeja,
gdzie były tylko miejsca stojące i~ograniczony bagaż.

Gdy tylko byli poza sinosowiecką przestrzenią powietrzną, i~zagłuszaniem
Szinosowów, Parvus priorytetowo przejął kontrolę i~wysunął wirtualną
głowę zza tylnego siedzenia przed nią.

-- Przepraszam za to, Myro -- wymamrotała AI. -- Pilne wiadomości.

-- Prześlij je -- powiedziała.

Kolejka wiadomości zawierała połączenia od Sulejmanowa, Walentyny
Kozlowej i~kogoś z~anonimowym kodem identyfikującym. Odpowiadała na nie
jedno po drugim.

Gdy tylko mrugnęła na identyfikator Prezydenta, połączył się, na żywo z~jego biura. Różni doradcy i~ministrowie unosili się na skraju ujęcia.

-- Cześć -- powiedział. -- Wyniki?

Myra skrzywiła się. 

-- Są nieugięci, żeby tego nie zaakceptować. Byłam
tak zaskoczona jak Ty. W~rzeczywistości, byłam zszokowana. Miałam
podejrzenie, że tajemnicą ich wojskowej i~ekonomicznej koordynacji jest
wojskowa AI i~że to\ldots dyktuje warunki.

Chingiz przyjął to z~nieoczekiwanym spokojem.

-- Warto było spróbować -- powiedział. Machnął dłonią, w~dół. -- Jednak
Szinosowy nie są już dłużej naszym najpoważniejszym problemem.

-- Co się stało?

Uśmiechnął się krzywo. 

-- Tak jak oczekiwaliśmy. Wszystko zostało
upublicznione, wszyscy wiedzą o~atomówkach. Nasza hojna oferta złożona
Stanom Zjednoczonymi, i~innych krajom, została skierowana aż do ONZ, i~przekierowana do Rady Bezpieczeństwa, w~celu natychmiastowej akcji. Mamy
przekazać posiadane bronie jądrowe siłom lojalnym ONZ w~ciągu dwudziestu
czterech godzin, teraz dwudziestu trzech i~pół, lub stawić czoło atakom
lotniczym i~kosmicznym. Konkretnie, wobec Kapicy, którą poprawnie
zidentyfikowali jako ognisko problemu. Po Kapicy, Ałma-Ata.

Myra myślała przez chwilę, że widok wirtualny stał się czarno-biały i~że
samolot zawrócił. Potem wszystko znowu było normalne.

-- Jeżeli chcą zrealizować groźby wobec Kapicy, cóż, mam nadzieję na
wsparcie lotnicze. -- Uśmiechnęła się blado. -- Ale proszę, Chingiz. Nie
pozwól im zrujnować Ałma-Aty.

-- Nie planuję pozwolić im tego zrobić -- powiedział. -- Sugeruję, żebyś
wróciła do Kapicy. Masz własne problemy. Jeżeli możesz, ewakuuj miasto.
Niech uderzą w~pustą skorupę. Wyślemy transport i~kawalerię.

-- Kawalerię?

-- Dla\ldots bezpieczeństwa wewnętrznego. Pat wokół budynków rządowych jest
bardzo napięty. -- Spojrzał w~bok. -- Twój własny minister obrony próbuje
się z~Tobą połączyć. Może Ci wyjaśnić sytuację lepiej niż ja. Do
zobaczenia na razie.

-- Do widzenia, Chingiz.

Zanim odebrała kolejne połączenie, Myra odwróciła się do Nurupa i~Mustafy.

-- Lecimy do Kapicy -- powiedziała. -- Mogą mieć do czynienia z~bardzo
zmienną sytuacją. Przynajmniej przemoc uliczna. Oraz możliwe
bombardowanie, aż do poziomu nuklearnego. To nie jest to, do czego was
wynajęłam. Mogą was zostawić w~Karagandzie, jeżeli chcecie.

Obaj \textit{mudżahedini} wyglądali na głęboko urażonych.

-- Naszą pracą jest zachowanie Twojego bezpieczeństwa, aż wrócisz do
Ałma-Aty lub aż powiesz nam, żebyśmy poszli -- powiedział Nurup.

-- Ok -- powiedziała. -- Zatem mówię wam, żebyście poszli.

Sięgnęła do przełącznika interkomu. Mustafa w~mgnieniu oka wstał i~położył dłoń na przełączniku. Jego mina i~ton były przepraszające. 

-- Zostaniemy -- powiedział. -- To wola Boga.

I też kwestia honoru, domyślała się.

-- Zatem do Kapicy -- powiedziała.

Obaj mężczyźni rozpromienili się jakby \textit{ona} wyświadczyła \textit{im}
przysługę. Może tak było. Prawdopodobnie wierzyli, że właśnie zapewniła
im dwa darmowe bilety do nieba. Były czasy, gdy zazdrościła im
pobożności.

Gdy samolot zakręcał, odebrała telefon od Walentyny. Ten był v-mailem,
zarejestrowanym w~jednym z~biur budynku rządowego. Za Walentyną,
mężczyźni z~AK czaili się przy oknach. Biurokraci zamienili biurka w~prowizoryczne barykady. Ktoś obsługiwał niszczarkę bajtów, czyszcząc
pamięci komputerowe, tworząc zadymkę interferencji.

-- Cześć, Myra, mam nadzieję, że to dotrze. Jezu, słyszałaś, że ta sprawa
z atomówkami jest we wszystkich mediach? Mamy zbieraczy wiadomości,
żywych i~zdalnych, pojawiających się cały czas, a~demonstranci zachowują
się dla nich, jakby chcieli się zobaczyć jako bohaterowie w~CNN. Jebane
wycie sprzężenia zwrotnego klasycznych mediów. Wielu z~nich oszalało
przez te atomówki, we wszystkich fakcjach, lewicowe przygłupy, typki
pro-ONZ i~pieprzeni Przestrzeniowcy. Nie wspominając naszych własnych
patriotów. Nasi agenci w~tłumie, cholera, nawet reporterzy, zbierają
informacje o~szturmie na budynek. Chcemy, żebyś wróciła tak szybko, jak
możesz. Będziemy mieć kierowcę z~milicji w~pogotowiu na lotnisku.

Wiadomość miała znacznik czasu o~13:35, a~teraz była 14:50. Myra
mrugnęła podzielony ekran kanałów wiadomości telewizyjnych, gdy
odbierała trzeci telefon. Pojawił się sygnał zapięcia pasów. Samolot
rozpoczął swoje zejście do Kapicy. Dzięki Bogu za ultradokładne
dostrojenie radia, Myra pamiętała czasy, gdy nie można było rozmawiać na
trakcie lotu. Głos pilota był lekko podniesiony, gdy kłócił się z~kontrolą lotu o~pierwszeństwo, rzucając uwagami dyplomatycznymi na równi
z kazachskimi przekleństwami. Myra wyjrzała przez okno. Więcej samolotów
niż zwykle -- pilnie wynajęte odrzutowce, jak się domyślała -- było
zaparkowanych koło pasów startowych. Cyrk medialny był w~mieście.

Jej anonimowy rozmówca pojawił się na wizji.

-- Jason!

Agent CIA uśmiechnął się z~napięciem, ale ciepłem w~oczach. 

-- Witaj,
Myro. Dobrze Cię widzieć. Oj, wyglądasz niesamowicie. Na czas na
globalne gwiazdorstwo, co?

-- Ha!

-- Prawie tyle emocji co przy zamachu. Tak czy inaczej\ldots dzwonię, żeby
powiedzieć Ci, że gdzieś doszliśmy ze śledztwem.

Podwozie wysunięte, uderzenie.

-- Co, och Georgi\ldots

-- Tak. Przykro mi, że muszę ci to powiedzieć, Myro, ale, cholera,
dostaliśmy to z~tajnych laboratoriów, to supernowoczesne rzeczy.
Zrobiliśmy autopsję cholernej \textit{próbki komórek}, nie pytaj, jak ją
dostaliśmy.

Wstrząs, bujnięcie do przodu, kolejny wstrząs i~powolne zwalnianie.

-- Krótko mówiąc, Myro, odkryliśmy ślady bardzo szczególnej, bardzo
subtelnej nanotechnologii. To nie jest do końca trucizna, to jest
właśnie sprytne. Buduje małe maszyny, potem rozpada się, kiedy wykona
swoje zadanie. Znaleźliśmy kilka przekładni zębatych, ale to było
wystarczające.

Samolot się zatrzymał i~światło pasów bezpieczeństwa zostało wyłączone.
Drzwi otworzyły się z~uderzeniem i~stopnie wysunęły się w~dół. Myra
wstała i~przesunęła się do przodu, za Nurupem i~przed Mustafą, ciągle
mówiąc i~słuchając. Machnęła roztargniona na pilota, zostawiła mu garść
złotych monet jako bonus. Myślała z~wyprzedzeniem.

-- Wystarczające do czego?

-- Wystarczające do zidentyfikowania. To był zamach Przestrzeniowców.
Zatrzymywacz serca.

,,Zatrzymywacz serca''. Tak. To było to.

Odmrugnęła unoszący się obraz Jasona, żeby skoncentrować się na jej
otoczeniu. Żadnych znaków bieżącego ostrzału. Podążyła za Nurupem do
budynku terminala, około sto metrów dalej. Głos Jasona w~jej głowie
kontynuował.

-- Zatem nie ma już wątpliwości, to było morderstwo. Nie ma
\textit{dowodu}, że Ruch Kosmiczny maczał w~tym palce, poza dostarczeniem
broni, ale dowody poszlakowe są raczej silne.

-- Można tak powiedzieć -- zgodziła się Myra, świadomie wysilając się,
żeby rozluźnić szczękę. Potwierdzenie jej własnych podejrzeń po całym
tym czasie oddawania się im, a~potem odrzucenia, było szokiem.

\textit{Jebany atak serca}\ldots

-- Oni raczej nie rozdają tego rodzaju sprzętu -- dumała głośno. -- Zbyt
łatwo od odtworzenie, z~jednej strony. Ale dlaczego mieliby to zrobić?

Długim korytarzem, pozwalając Nurupowi i~Mustafie się rozejrzeć. Kątem
oka mogła zobaczyć przyległy, wychodzący korytarz wypakowany od końca do
końca wolno poruszającą się kolejką.

-- Cóż, oczywistym motywem byłoby zatrzymanie go przed rozmowami z~Kazachstańczykami.

-- A skąd Ty wiesz o~tym?

-- Hm, to tajne.

Myra musiała się roześmiać.

-- Ale jak oni dowiedzieliby się o~tym, znaczy wcześniej\ldots?

-- Ty mi powiedz.

Dotarli do hali dworcowej. Nie było tak zatłoczona i~oszalała, jak
podejrzewała. Większość z~tych, którzy chcieli wyjść, musieli już ją
opuścić lub przynajmniej być w~kolejce do wyjścia. Ku jej uldze, jeszcze
nie napotkała żadnych dziennikarzy czy reporterów, choć zidentyfikowała
jednego lub dwóch dzięki kamizelkom kuloodpornym, wrzawie łączności i~niejasno znajomym twarzom. Przeszukując tłum, ujrzała mężczyznę w~mundurze Milicji Robotniczej, który złapał spojrzenie, zasalutował i~zaczął przepychać się w~jej kierunku.

-- To była taka sama niespodzianka dla wszystkich w~rządzie jak dla mnie
-- powiedziała. -- Domyśliliśmy się, że to był własny pomysł Georgiego,
który nam przedstawił, gdy otrzymał\ldots och!

Mustafa wpadł na nią.

Jason machał do niej, nad głowami.

-- Nigdy nie powiedziałeś, że jesteś \textit{tutaj}!

-- Ta, cóż\ldots pomyślałem, że cię zaskoczę.

To było dziwne widzieć, jak poruszają się jego usta i~słyszeć słowa,
poza zasięgiem głosu. Jak czytanie z~ust, jak telepatia.

-- Kim jest ten facet? -- spytał Nurup podejrzliwie.

-- Jest ok -- powiedziała Myra. Nie była pewna, czy przedstawianie Jasona
jako agenta CIA byłoby dobrym pomysłem.

A potem się spotkali i~ku zaskoczeniu wszystkich, ona i~Jason objęli się
na długo.

-- Jezu, człowieku!

Uwolniła się i~odwróciła do kierowcy Milicji.

-- Dziękuję za przybycie. Miejsce dla tych trzech facetów?

Kierowca skinął głową. 

-- Tędy proszę.

Poprowadził ich do drzwi dla obsługi, o~których Myra wiedziała, że
musiała minąć setki razy i~nigdy nie zauważyć. Ich postęp był mniej
niepozorny, obaj \textit{mudżahedini} nie byli jedynymi uzbrojonymi
pasażerami, ale byli najbardziej zauważalni. Gdy kierowca grzebał przy
zamku drążka, Myra zauważyła ruch głów i~małe brzęczenie kamkopterów
spadających z~krokwi hali.

Śpieszyli się przejściem z~falistej blachy i~surowych, pełnych zadziorów
belek, i~pojawili się przy dżipie w~małej zatoce parkingu.

-- Ach, to rozsądny transport -- powiedziała Myra, gdy wsiadali. Dżip
Milicji miał zamontowany lekki karabin maszynowy na belce. Mustafa zajął
tę pozycję. Nurup usiadł z~przodu z~kierowcą, karabin podparty na ręku,
skierowany do góry. Myra i~Jason usiedli z~tyłu, z~nogami Mustafy i~taśmą nabojową pomiędzy nimi. Gdy dżip pognał przez parking i~skręcił na
główną drogę do miasta, Jason pochylił się i~powiedział, głośno ponad
hałasem i~hukiem powietrza: 

-- Mówiłaś?

-- Tak, o~wielkim planie Georgiego. O ile możemy powiedzieć, nigdy nie
powiedział o~nim nikomu, nawet Walentynie. To było w~jego stylu, był
patriotą kazachstańskim i~\textit{ciągle} miał tendencję do działania,
jakby całe to miejsce było jego osobistym lennem. Którym kiedyś było!

Dżip jechał szybko, większość ruchu było w~przeciwnym kierunku, ku
lotnisku, lub, na podstawie bagażów i~artykułów domowych ułożonych na
samochodach i~ciężarówkach, w~kierunku Karagandy. Jej ulga z~powodu
ewakuacji już w~trakcie została stłumiona przez wspomnienia innych dróg,
innych kolumn pojazdów: droga do Basry, droga z~Warszawy, granice
Atlanty\ldots

Ale nie, nie tutaj! Mieli własną osłonę powietrzną, elitarne siły
powietrzne Kazachstanu na pewno osłonią tych uchodźców. Pomyślała krótko
o zorganizowaniu rozmowy konferencyjnej z~Walentyną i~Chingizem, ale
zdecydowała inaczej. Rozmowa z~Jasonem była najważniejszą z~tych, które
mogłaby mieć teraz, z~powodów, które bardziej niż osobiste.

-- Ok -- mówił Jason -- co do motywu, racja, czy ktokolwiek \textit{inny}
proponował ci podobną lub tego rodzaju umowę, po śmierci Georgiego, ale
przed zamachem stanu?

-- Tylko jebany Ruch Kosmiczny! -- Przełknęła głośno. -- Sam David Reid, na
pogrzebie Georgiego.

-- Jezu, to jakby wskazuje na niego, co nie?

Myra stwierdziła, że wkurza ją pytanie \textit{kto wiedział o~tym}.

-- Cóż, jest z~tym problem -- powiedziała. -- Ktokolwiek zabił Georgiego,
lub kazał go zabić, \textit{musiał} wiedzieć, że podejrzewalibyśmy
Przestrzeniowców. Mam na myśli, nawet zanim znalazłeś dowód, trzymałam
ich w~kadrze. A trochę trudno to zrekonstruować, wiesz, jak to jest, ale
kiedy odmówiłam Dave'owi gwarancji neutralności, nie mówiąc już o\ldots
aktywnym wsparciu, cóż, to podejrzenie musiało być wzięte pod uwagę.
Może nawet ich sprowokowało.

Mustafa krzyknął coś i~przesunął karabin maszynowy dookoła do tyłu. Myra
przesunęła rozsądnie nogi z~drogi taśmy i~obróciła głowę. Pięćset metrów
za nimi była mała, lecąca grupę samochodów i~dżipów, przed chmurą pyłu i~poniżej kręgu kamkopterów. Uderzyła w~udo Mustafy.

-- Zostaw ich! -- krzyknęła.

Odpowiedział jakimś uzbeckim przekleństwem, ale odstąpił, unosząc lufę
karabinu znowu do góry.

-- Zatem mówisz, że zabicie Georgiego było przeciwne do zamierzonego dla
Przestrzeniowców?

-- Cholerna racja!

-- Ok. -- Jason rozparł się w~ciasnym siedzeniu i~zamknął oczy na chwilę.
-- \textit{Cui bono?} Kto skorzystał?

-- Och, cholera -- powiedziała Myra, zdając sobie sprawę, właśnie gdy dżip
wyjechał zza rogu na Plac Rewolucji i~się zatrzymał. Myra złapała pałąk
i się podciągnęła. Długa praktyka w~ocenie wielkości demonstracji sama
się uruchomiła jak software opaski.

Około dziesięciu tysięcy.

-- Och, Jezus -- powiedziała.

Nie było to szczególnie bojowy czy gniewny tłum, w~tej chwili. Namioty,
kramy i~szałasy były ustawiane i~wiele transparentów było opartych o~nie
lub uliczną architekturę, lub wepchnięte w~łaty teraz zadeptanej trawy,
lub kwietniki, które kratkowały plac. Ludzie stali lub siedzieli, w~małych grupach, rozmawiając, pijąc kawę, czytając wiadomości z~gazety,
opasek lub ręcznych czytników, słuchając przemów i~pieśni, kłócąc się ze
sobą lub rozsianymi pojedynczymi czy dwójkami Milicji Robotniczej.
Niektórzy byli ubrani zwyczajnie, inni w~najlepszych ubraniach,
narodowych strojach lub kombinezonach radiacyjnych ulicznych teatrów.

-- Wygląda całkiem niebezpiecznie -- powiedział Jason.

Skinęła mu głową, doceniając \ldots 

-- Tak, to jest \textit{masowa
}demonstracja, jeżeli kiedykolwiek jakąś widziałam. Nie wspominając o~dużym ułamku pozostałej populacji. Gówno.

Dzieciaki w~Glasgow miały rację: jej małe państwo miało wielką,
polityczną rewolucję. Obaj \textit{mudżahedini }patrzyli groźnie
niezrozumiale na wymierzane sztandary Kazachstanu, MRRNT, starego
Związku Radzieckiego, Międzynarodówki, czerwone i~czarne flagi.

Schyliła się i~położyła rękę na ramieniu Nurupa.

-- Wstań -- kazała. -- Wyglądaj wesoło. Machaj karabinem nad głową.
Mustafa, na litość boską, \textit{uśmiechnij się}, człowieku, machaj
ramiona i~ręce z~dala od tego LKM-u. \textit{Nieważne co}, rozumiesz?

Do kierowcy: 

-- Dookoła wewnętrznej krawędzi tłumu, w~kierunku wejścia.
Wolno i~ostrożnie.

Podniosła się, przesunęła się i~oparła o~belkę, stopy na oparciu
siedzenia Nurupa. Kierowca włączył pierwszy bieg, potem drugi. Dżip
potoczył się ku rogowi frontu budynku. Minął około pięćdziesięciu
metrów, potem kolejne pięćdziesiąt, kiedy musiałby skręcić w~prawo i~skierować się do wejścia. Jechali niezauważeni przez około pół minuty.
Potem ludzie schodzący im z~drogi zaczęli wołać i~wskazywać palcami.
Chwilę później ścigający reporterzy nadrobili i~wszystkie szanse na
dyskrecje zniknęły.

Mogła zobaczyć, jak wiadomości o~jej przybyciu, rozprzestrzeniają się
przez tłum, jak podmuch wiatru na polu. Kamkoptery krążyły w~bezpiecznej
odległości, zbliżając na nią i~jej kadry reakcję ludzi patrzących na
nią. Ich jedyną szansą, zdecydowała, było wyglądać pewnie i~triumfująco.
Uśmiechnęła się i~pomachała, w~międzyczasie mrugając połączenie do
Walentyny.

-- Widzisz nas?

-- Tak, osłaniamy was. Otworzymy drzwi dla was, gdy będziecie w~zasięgu.

Wiwaty i~szyderstwa odbijały się echem od szklanych i~betonowych ścian
budynku rządowego. Żadnego zorganizowania wołania lub spójnego nastroju
jeszcze, ludzie ciągle byli niepewni, jak zareagować na jej powrót.
Uśmiechała się desperacko do każdej jednostkowej twarzy, która się
pojawiała, i~całkiem sporo uśmiechnęło się w~odpowiedzi. Unoszące się
kamery miały jej kierunkowe mikrofony wycelowane w~nią, ale nie chciała
mówić do, lub dla, nich.

-- W~porządku, ludzie, towarzysze, jesteśmy w~trakcie załatwiania tego
wszystkiego, mamy silny sojusz z~Kazachstanem, negocjujemy z~ONZ i~powstrzymamy Szinosowów, będę do was przemawiać wkrótce, gdy tylko będę
miała szansa się skonsultować\ldots

Dżip delikatnie zatrzymał się przy głównych drzwiach. Myra spojrzała w~bok, zobaczyła kilku milicjantów trzymających je, gotowych do otwarcia,
karabiny w~rękach.

-- Wejdźcie, chłopaki, wszyscy, będę mówić.

Zawahali się.

-- Już już już!

Jeden po drugim wbiegali po schodach i~znikali w~środku. Myra zeszła z~siedzenia na deskę rozdzielczą, nad szybą i~na maskę, potem zeskoczyła
do tyłu na stopień, cały czas na widoku. Cofnęła się po schodach,
uśmiechając się i~machając, potem przed drzwi.

Ramiona Jason otoczyły ją z~tyłu.

-- Bardzo dobrze.

Oparła się o~niego na chwilę, przechylając głowę na jego ramieniu, potem
wyprostowała się i~odsunęła, uśmiechając się.

-- To było przerażające. -- Roześmiała się. -- To dziwne uczucie bycia
\textit{celem} demonstracji, czuję, że powinnam tam być, pomagając ją
zorganizować.

Oczy Jasona się zwęziły. 

-- To -- powiedział -- może stać się opcją.

-- Och, odpierdol się, ty makiaweliczny szpionie! -- Złapała jego dłoń,
otoczyła ramieniem Nurupa i~Mustafę. -- Chodźcie chłopcy, ogarnijmy ten
bałagan.

Prowadzili nadzwyczajne spotkanie w~biurze Myry, której szerokie okna
wyglądały na plac. Denis Gubanow zasugerował wykorzystanie pokoju
SowNarKomu, ale Myra odrzuciła pomysł bezpiecznika. Nie ma mowy, żeby
chciała być w~pokoju bez okien.

Wszyscy siedzieli lub rozwalali się na niewłaściwych meblach, biurkach,
szafkach na dokumenty i~węzłach łączności. Myra usadowiła się na
najwyższej wygodnej powierzchni, na szczycie regału pełnego
nieprzeczytanych, żółknących wydruków. Położyła Glocka pomiędzy udami.
Jakoś siedzenie w~fotelu wydawało się lekkomyślne. Dwóch milicjantów
stało czujnie po bokach okien, używając opasek do próbkowania informacji
z serwisów wiadomości. Andriej Muchartow, Walentyna Kozlowa i~Denis
Gubanow wyglądali na bezsennych i~zaniedbanych: mężczyźni nieogoleni,
kołnierzyk i~krawat Wali poluźniony, jej mundur wygnieciony.

Myra przedstawiła dwóch \textit{mudżahedinów} i~Jasona. Denis uniósł brwi,
ale nie skomentował. Myra dyskretnie upewniła się, że jej trzej
mężczyźni są na pozycji, żeby ją chronić, nie była w~ogóle pewna, kto,
jeżeli ktokolwiek, z~obecnych miał opuścić pokój żywy, niezależnie od
tego, czy pokój zostałby zaatakowany przez rozgniewany tłum. Kiedyś
rozmawiała z~nieskruszonym starym stalinistą, który był w~biurach Partii
w Budapeszcie w~październiku 1956 roku\footnote{
zob.~\url{https://pl.wikipedia.org/wiki/Powstanie\_w\%C4\%99gierskie\_1956}
-- przyp.tłum.} \ldots

-- Dobra, towarzysze -- zaczęła. -- Pierwsze sprawy. Wiecie, że zachodnie
mocarstwa odrzuciły naszą ofertę. Właśnie dzisiaj byłam na najkrótszej
misji dyplomatycznej \textit{kiedykolwiek} i~mogę wam powiedzieć, że
Szinosowy też nie są zainteresowane dogadaniem się. Zatem to tylko
kwestia czasu, zanim zaczną się toczyć drogą z~Semeja. Jednak to tylko
tło. Mamy ważniejsze kwestie do przedyskutowania.

-- Zamierzam zacząć od czegoś, co może nie wydawać się najważniejszym
punktem programu, ale cierpliwości. -- Machnęła dłonią w~kierunku okna. -- Ci ludzie mogą poczekać. Chodzi o~śmierć Georgiego. Jason Nikolaides
tutaj powiedział mi o~wynikach śledztwa CIA, morderstwo, z~użyciem broni
nanotechnicznej Ruchu Kosmicznego. Trudno wykryć ślady, ale Jason mówi,
że im się udało, a~ja mu wierzę. To, w~co nie wierzę, że to zrobiły te
zaziemskie gnojki. Ktokolwiek to zrobił, chciał dwie rzeczy, jedna, że
oferta Georgiego nie dotrze do Kazachstanu \textit{przed} zamachem stanu.
Dwa, że nie będziemy współpracować z~Ruchem Kosmicznym \textit{w} trakcie
zamachu. Widząc, że nikt, prócz Georgiego, nie wiedział, że planował
przedstawić ofertę, nasz zakres podejrzanych jest ograniczony.
Praktycznie, to musi być ktoś, komu Georgi przedstawiłby pomysł, ktoś
spoza obiegu rządowych informacji, może w~SowNarKomie, może nie.

Spojrzała w~dół, bawiąc się zamkiem Glocka przez chwilę, potem spojrzała
na nich. Myślała na głos, nie miała czasu przejrzeć wszystkich
możliwości.

-- \textit{Wala! }-- krzyknęła. Wszyscy podskoczyli. -- Jeżeli myślałabym, że
to Ty, rozbiłabym Tobą ścianę, aż Twoje zęby wystukałyby prawdę z~Ciebie. Ty i~Georgi byliście w~Partii, w~odróżnieniu od wszystkich
pozostałych tutaj.

Uśmiechnęła się, zadowolona, że jej koledzy zostali wytrąceni z~równowagi. 

-- Ale tak bywa, że ci ufam. Tak samo Andriej, który i~tak nigdy nie był w~tego rodzaju gównie. Denis, co\ldots

Tajny policjant spojrzał na nią i~zwilżył wargi.

-- Przysięgam, Myro\ldots

-- W~porządku -- wtrącił Jason. -- Firma go sprawdziła. Jest czysty. -- Spojrzał na Myrę, potem uśmiechnął się do Denisa Gubanowa. -- Kawał
komuszego skurwysyna, ale jest po Twojej stronie.

-- Dobrze -- powiedziała Myra, improwizując. -- Przechodzę przez to, żeby
\textit{potwierdzić}, że nikt tutaj nie jest podejrzanym. To pozostawia
jedną możliwość. Georgi musiał się podzielić pomysłem z~kimś, a~to może
być tylko organizacja bojowa Czwartej Międzynarodówki. Generał.

Pozwoliła im to przemyśleć, podczas gdy wyjaśniała Jasonowi, Nurupowi i~Mustafie sprawy atomówek i~AI.

-- To ma swój własny program -- podsumowała, znowu zwracając się do
wszystkich. -- I~działa przez Szinosowiety. Chce tych atomówek, poważnie.
Tak samo Przestrzeniowcy. Czy wykorzystali się wzajemnie, informacja w~jedną stronę, broń w~drugą, świadomie, czy nie, morderstwo Georgiego
było ruchem w~tej rywalizacji. Ktokolwiek włada tą bronią trzyma
pistolet przy głowie wszystkich i~wszystkiego na orbicie Ziemi i~w punktach Lagrange'a, które razem składają się na dziewięćdziesiąt pięć
procent obecności człowieka w~kosmosie. A ja chciałabym wam przypomnieć,
dzięki zamachowi i~kontr-zamachowi, Generał kontroluje większość stacji
orbitalnych Obrony Kosmicznej. To musimy mieć na uwadze wobec tego, co
zrobimy z~ultimatum ONZ. Które jest\ldots -- uśmiechnęła się dziko -- \ldots
\textit{drugim} punktem programu.

-- Przepraszam -- powiedział Jason, wstając. -- Właściwie kto w~tej chwili
kontroluje te atomówki?

-- My -- powiedziały Walentyna i~Myra, w~tym samym momencie. 

Myra uśmiechnęła się szczególnie ciepło do Wali, mając nadzieję, że jej
oczywiste -- i~częściowo paranoicznie prawdziwe -- wcześniejsze
podejrzenie nie zraniło śmiertelnie ich przyjaźni.

-- To podwójny klucz -- wyjaśniła Walentyna. -- Minister Obrony i~Premier
muszą udać się do przestrzeni roboczej centrum dowodzenia w~tym samym
czasie.

-- I, cóż, to nie jest zapisane na sztywno, ale teraz oczywiście mamy
zobowiązanie traktatowe, żeby Prezydent Kazachstan miał ostatnie słowo -- dodała Myra. -- A jego strategią, teraz, jest blokowanie, aż do ostatniej
chwili i~próbowanie pozyskania ustępstw w~pomocy zbrojnej mocarstw
zachodnich lub ONZ przeciwko Szinosowietom.

-- Zatem ostatecznie zamierza je przekazać? -- spytał Jason.

Myra się zawahała. 

-- Ok -- powiedziała w~końcu. -- To nie wychodzi z~tego
pokoju i~to dotyczy wszystkich tutaj. Wy przy oknach też, wojskowa
dyscyplina, kara śmierci na podstawie Prawa o~Wolności Informacji,
jeżeli wypowiecie słowo o~tym. Wszyscy rozumieją?

Wszyscy rozumieli.

-- Dobra, zatem, tak, zamierza przekazać je, w~końcu. Co jeszcze możemy
zrobić?

-- Możemy użyć broni -- powiedział Denis. -- W~kosmosie.

Usta Wali zacisnęły się w~cienką kreskę. Myra potrząsnęła głową.

-- Masakra -- powiedziała. -- Nie zrobię tego, chyba że jako ostateczność.

-- Wszyscy pomijacie jedną sprawę -- powiedział Jason. Rozejrzał się
dookoła po nich wszystkich, jak gdyby niepewny, czy ma prawo mówić.

-- Dawaj -- powiedziała Myra.

-- Ok -- powiedział Jason -- mówię tylko za siebie, nie za CIA czy
Wschodnie Stany. Nie wiem, czy kiedykolwiek wrócę do któregoś z~nich.
Tak czy inaczej\ldots sprawa, którą wszyscy pomijacie to: \textit{komu}
zamierzacie poddać waszą broń? Formalnie, bez wątpienia, to będzie ONZ.
Jednak fizycznie, ktoś będzie musiał się do nich zadokować, sprowadzić
je, rozbroić. Obrona Kosmiczna i~może niektórzy z~osadników w~kosmosie
mają sprzęt i~doświadczenie. Muszą być sposoby ominięcia software'u
waszego sterowania, zawsze są. Uwierzcie mi, nie ma już kodów nie do
złamania. Wasza współpraca byłaby użyteczna, ale nie jest kluczowa.

Myra zapaliła papierosa. 

-- Ok -- powiedziała. -- Więc?

Jason przeszedł do okna, wyjrzał. 

-- Ciągle cicho -- powiedział. Spojrzał
na zegarek. -- Jesteśmy tutaj, ile? Pół godziny? Wkrótce trzeba będzie
przemówić do ludu, Myro.

-- To fajnie -- powiedział Denis. -- Mamy agitatorów tam, podtrzymują ludzi
mniej więcej w~temacie. Linia jest taka, że Prezydent negocjuje.

-- Jestem pewien, że tak -- powiedział Jason. -- Ale co obie strony muszą
negocjować? Obie strony dotarły do dna zbiornika. Nie macie nic do
zaoferowania i~Zachód nic nie może wam zaoferować. Nie uratują was przed
Szinosowami. Zatem gdybym był dowolnym innym graczem, w~szczególności
Przestrzeniowcami i~Twoją organizacją Czwartej, dziką AI lub nie,
działałbym bardzo szybko w~kierunku dwóch celów. Jeden to fizyczne
wyłączenie was i~waszego cudownego dwukluczowego centrum dowodzenia.
Drugi to ustawienie spotkania z~atomówkami w~kosmosie. Możecie się
założyć, że podczas gdy myślicie, że jesteście bystrzy, oszukując ich,
\textit{oni} oszukują \textit{was} i~zmierzają do tych samych rzeczy.

Rozejrzał się dookoła, teraz bardziej pewny siebie. 

-- To jest końcówka.
Nie tylko dla nas, ale dla nich. Jedna strona lub druga, albo Zachód łamane
Przestrzeniowcy łamane Zewnętrzni, albo Wschód łamane Generał łamane
przez Sinosowiety, zamierza przechwycić tę broń i~\textit{użyć} jej,
raczej wcześniej niż później.

-- Ale\ldots -- krzyknęła Wala, zszokowana. -- Syndrom Kesslera!

-- To nie problem dla żadnej ze stron, na poziomie, o~którym mówimy.
Horyzont Szinosowów jest ściśle przyziemny, na następne kilka wieków. A
ich komputery są niewrażliwe na ataki elektromagnetyczne, są
mechaniczne, a~nie elektroniczne. Co do Przestrzeniowców i~organizacji
wojskowej, żadnemu z~nich nie zależy na powrocie na Ziemię, lub na
czymkolwiek innym uciekającym z~niej. A każda jednostka tych sił
prawdopodobnie oblicza, że mogą odejść i~wejść na wyższą orbitę lub
Lagrange. Oczywiście, woleliby uniknąć tego, ale jeżeli będą musieli,
przyjmą to na klatę.

-- Zatem moja rada dla was wszystkich -- podsumował -- i~tych ludzi tam na
zewnątrz to wynosić się stąd. I~ostrzec wszystkich, że na pierwszy znak
zadzierania z~wami lub Kazachstanem lub atomówkami, wysadzicie je
wszystkie w~diabły. Użycie atomówek przeciwko stacjom orbitalnym lub
zdetonujecie na miejscu, w~każdym przypadku uruchomicie kaskadę
Kesslera.

-- Chryste -- powiedziała Myra, wstrząśnięta. -- To znaczy koniec
kierowania satelitarnego, globalnego pozycjonowania, komunikacji, sieci,
wszystkiego! To jakby świat oślepł!

-- Tak -- powiedział ponuro Jason. -- Oraz każda armia na świecie. Są tak
zależni od łączności satelitarnej i~symulacji, że będą rozjebani. Prócz
marginesu, Zielonych, barbarzyńców i~Szinosowów. -- Roześmiał się. -- Jeżeli to ich nie wystraszy, to nic nie przestraszy.

Strażnicy przy oknie przesuwali się z~boku na środek, wyglądając z~kompletnym brakiem troski o~osłonę. Jeden z~nich się odwrócił.

-- Przybyła kawaleria -- powiedział.

Przez chwilę Myra myślała, że miał na myśli Szinosowów. Potem
zrozumiała, że Chingiz dotrzymał swojej obietnicy i~że kawaleria była
ich własną.

~

Step w~nocy był poruszającą się masą pojazdów i~koni. O ile Myra
wiedziała, każda osoba w~Kapicy się wynosiła. Jechała gdzieś blisko
przodu. Próbowała jechać na przodzie, ale ciągle była wyprzedzana przez
ludzi w~pojazdach szybszych niż jej czarna klacz. Kadłub SowNarKomu,
Jason i~jej \textit{mudżahedini} jechali w~dżipach koło niej. Przy jej
wzmacniaczach obrazu w~opasce na pełnej mocy, widziała kawalerię
Kazachstanu -- konno i~zmotoryzowaną -- prześcigającą po obu bokach
ewakuacji, lub migracji. Obraz był biblijny, exodus i~apokalipsa w~jednym. Plakaty i~flagi z~demonstracji na Placu Rewolucji unosiły się
ponad tłumem, użyte jako punkty zbiórki i~ruchome punkty orientacyjne.
Zdalni i~realni reporterzy wiadomości podążali proces w~rodzaju
oszołomionym podziwie, niezdecydowanie czy opisywać \textit{Drogę Ludu}
(uchodźcy, wzruszająco) czy \textit{Kazachskie Róże} (zagrożenie,
fanatycy).

Coś podobnego, jeszcze nie tak drastycznego, działo się w~Ałma-Acie i~innych miastach w~wielkiej Republice. Chingiz Sulejmanow przedstawił
wezwanie do ewakuacji jako ostateczny marsz protestacyjny, przeciwko
groźbom i~odmowie pomocy Zachodu przeciwko Szinosowom. Jeżeli mieli być
porzuceni na rzecz komunistów, nie mieli nic do stracenia, uciekając
wcześniej do miejsca, które twierdziło, że będzie bronione. Zagrożenie
zamiany tej lawiny w~niepowstrzymaną migrację już siało panikę w~Zachodniej Europie. Na północy, w~byłym Związku, regionalni i~lokalni
przywódcy naradzali się w~swoich fragmentarycznych sieciach, gadając
podburzająco o~dołączeniu.

-- Dalej, dalej, gnojki -- mamrotała Myra. Jechała w~halucynacyjnej
atmosferze wirtualnych obrazów, niektórych ściągniętych z~CNN i~innych
serwisów, innych włączonych z~centrum dowodzenia, którego hardware
rozebrali z~biur i~prowizorycznie wbudowali w~tył dżipa SowNarKomu.
Mogła zobaczyć widok satelitarny jej samej z~góry, mogła pomachać i~po
sekundowym opóźnieniu widziała jedną z~kropek na ziemi odmachującą.
(Uspokajające było to, że to była zła kropka, sobowtór hologramowy jej i~jej otoczenia bezbłędnie włączony w~obrazy odległe o~kilka kilometrów).
Mogła zobaczyć swoją twarz, przekazywaną na ekrany na całym świecie
przez kamkopter unoszący się kilka metrów przed nią.

Teraz próbował skontaktować się z~Loganem. Szczątkowa lojalność do jej
byłych towarzyszy w~kosmosie pobudziła ją do ostrzeżenia ich przed
prawdopodobnie nadchodzącą katastrofą. Przeszukiwanie grupy w~Lagrange
nie odnajdywało New View. W~końcu, sfrustrowana, przełączyła się na
szerszy zakres i~ku jej zdziwieniu niemal natychmiast się połączyła.

-- Jezu, kurwa, Myra -- powiedział Logan bez wstępnych uprzejmości. -- To
Twój największy rozpierdol od Trzeciej Wojny Światowej. -- Nie brzmiało
to w~jego ustach jak oskarżenie.

-- Dzięki za przypomnienie, towarzyszu -- warknęła Myra. -- Mówiąc ci to,
działam wbrew mojemu lepszemu osądowi, ale pokłóciłam się z~Twoim
Generałem. Ten mały elektryczny gnojek wpadł na bystry pomysł własnej
licytacji na światową rewolucją i~nie zamierzam czekać, żeby zobaczyć,
jak to wyjdzie w~praktyce, dziękuję uprzejmie.

-- Tak, słyszałem -- powiedział Logan ciężko. Opóźnienie wydawało się
dłuższe niż zwykle. Myra domyślała się, ponieważ była nerwowa, biegnąc w~przeciągniętym czasie. -- Zadzwoniłaś mi to powiedzieć? -- Brzmiał
rozkojarzony. Bardzo piękna czarna dziewczyna, która wyglądała na
dziesięć lat, wsadziła twarz koło niego, robiąc miny do kamery,
wypełniając pole jej koroną kędzierzawych włosów w~mikrograwitacji.
Logan popchnął ją.

-- Och, uciekaj, Ellen May -- powiedział, nie nieuprzejmie. -- Idź i~pomęcz
mamę, ok? Lub Janis. Na pewno coś ci da do roboty.

Dziewczyna wystawiła język, potem odleciała jak ryba.

-- Dzieci. -- Logan się uśmiechnął, pobłażliwy wbrew sobie.

-- Tak, są wspaniałe -- powiedziała Myra z~bólem. -- Dzwonię właściwie w~tej sprawie. Jeżeli to dziecko ma mieć przyszłość, to lepiej zabierajcie
dupę w~troki z~Lagrange'a.

-- Zabieramy -- powiedział Logan, pięć sekund później. -- Przyśpieszyliśmy
przygotowania po Zamachu. Nie mamy tyle sprzętu, ile chcielibyśmy, ale
górnicy asteroidów zamierzają skoczyć i~dołączyć do nas tam.
Skończyliśmy ciąg dwanaście godzin temu. -- Rozejrzał się. -- Poważnie
nabałaganiło rzeczami, których nie miałem czasu przywiązać -- dodał
smutno.

-- Jesteście w~drodze na \textit{Marsa}?

-- Tak, w~końcu. -- Jego uśmiech wypełnił ekran. -- W~końcu wolni!

-- Co Generał myśli o~tym?

-- Ach -- powiedział Logan. -- Kiedy odkryłem, że liczył na użycie Twoich
orbitalnych atomówek w~zamachu, domyśliłem się tego samego co Ty.
Niebezpiecznie trzymać się w~pobliżu. Pamiętasz, co powiedziałem, że
musielibyśmy zostawić kilkaset ton? Cóż, jest pośród nich, ciągle w~śmietniku na Lagrange. Zostawiliśmy gnojka. -- Jego triumfujący uśmiech
zblakł do posępnego zamyślonego spojrzenia. -- Mam nadzieję.

-- Czy ciągle kontroluję organizację bojową?

-- Chyba tak. Nie mogliśmy nic z~tym zrobić, poza rozebraniem sekcji, w~której był hardware. Jego software to inna sprawa, dostanie się
wszędzie, ale, cholera\ldots

-- Co masz na myśli ,,dostaje się wszędzie''? Podejrzewałam, że załadował
się do tych dziwnych, sinosowieckich silników Babbage'a, ale\ldots

Logan kiwnął głową. 

-- Ta, i~prawdopodobnie skopiował swoje pliki na
wszystko Twoje, co miało z~nim kontakt, jak telefon, ale to tylko kod
źródłowy, nie może zrobić krzywdy tak długo, jak nie otworzysz pliku\ldots

W tym momencie połączenie się skończyło.

Myra wyjęła telefon z~kieszeni i~miała wyrwać wtyczkę z~opaski, na
wszelki wypadek, kiedy zrozumiała, że środek ostrożności był
nieracjonalny. Jeżeli gnojek właściwie działał na jej telefonie, to już
byli skazani na zagładę. Myślała o~sytuacji, gdy Generał pojawił się
dokładnie w~jej centrum dowodzenia i~miała nadzieję, że Logan miał
rację, że to był tylko kod źródłowy, a~nie jego program, który tam był
pozostawiony. A w~innych miejscach\ldots

Pewnego dnia, ktoś otworzy plik zachowany w~Instytucie w~Glasgow i~odkryje Parvusa, a~za nim Generała. Życzyła tej osobie szczęścia. Potem
przypomniała sobie Merrial MacClafferty i~zrozumiała, że musi zrobić
więcej.

Właśnie skończyła wstukiwać ważną wiadomość, kiedy usłyszała tępy,
odległy huk za nią i~się odwróciła. Przez noktowizor opaski zobaczyła na
horyzoncie rozszerzające się, zielone światło pierwszych rakiet
manewrujących uderzających w~Kapicę.

Ta nie była ostatnią.

Godziny później, o~północy, minus dwadzieścia stopni, kiedy większość
migracji obozowała dookoła ogni z~paliwa, Myra siedziała z~Jasonem przed
przenośnym elektrycznym piecykiem, w~schronie drzemiących koni.
Jednocześnie była w~centrum dowodzenia z~innymi oraz z~Chingizem. ONZ i~USA nigdy nie miały zamiaru negocjować i~nawet pozory zostały porzucone.

Kazachstańskie Siły Powietrzne traciły rakiety, samoloty i~życia ponad
Ałma-Atą. Centrum dowodzenia ściągało obrazy ruchów w~kosmosie ze stacji
orbitalnych. Małe, załogowe sondy łowiecko\dywiz zbierackie szły ostrym
ciągiem, dopasowując orbity i~prędkości do kryjówek broni atomowych.
Miały eskortę myśliwców i~one były oczywiście z~przeciwnej koalicji, ich
wymiana ognia już była powtórzona przez CNN, teraz kiedy bombardowanie
Kapicy zostało przerwane z~powodu braku kolejnych celów.

-- \ldots bez wyboru -- mówił Chingiz. -- Naszą pierwszą odpowiedzialnością
jest bronić naszych ludzi, ludzi, wobec których podjęliśmy obowiązek
ochrony, nawet jeżeli to oznacza zabicie większej liczby niewinnych osób
po drugiej stronie, niż zginęłoby po naszej, gdybyśmy nie działali.

To przemowa, myślała Myra, to jest sposób patrzenia, to jest dobre.
Pieprzyć największe dobro dla jak największej liczby osób. Lub może nie.

-- To koniec świata -- powiedziała Walentyna.

-- I~tak to jest koniec -- powiedziała Myra. Spojrzała znad ognia. -- To
moja ostateczna analiza! Możemy nawet uratować życia w~długim czasie,
jeżeli oślepimy i~okaleczymy siły, które przygotowują się do ostatniej
wojny. -- Zaśmiała się gorzko. -- W~obu znaczeniach tej frazy.

Oficer znalazł się w~kadrze dookoła Chingiza i~powiedział pilnie do jego
ucha. Chingiz skinął głową, raz, potem uniósł dłoń.

-- To jest to -- powiedział. -- Niektóre z~diamentowych statków osadników
kosmosu właśnie weszły w~atmosferę. Kierują się na \ldots

Połączenie utracono.

Myra zerwała się i~ku jej całkowitemu horrorowi i~zdumieniu, zobaczyła
je, trzeszczące i~drgające przez niebo ku jej. Ich sygnatura
promieniowania podczerwonego była arogancko oczywista, nie przejmowały
się osłanianiem, inaczej niż myśliwce stealth, które przypominały. W~jednej chwili były kropkami na horyzoncie, w~następnej dyskami nad
głowami, nurkując nad nimi na tysiącu metrów. Ich laserowe włócznie
cięły ogromny obóz i~sekundy za późno odpowiadała im daremna strzelanina
karabinów maszynowych. Wtedy były po drugiej stronie horyzontu i\ldots

\ldots zawracały po drugie przejście\ldots

\ldots krzyki ludzi i~zwierząt w~nocy, umierających pod promieniami
laserowymi i~brzęczącego deszczu ich własnego, niecelnego, spadającego
\ldots

\textit{Ziemie versus Latające Spodki! Odlotowo!}

Myra otrząsnęła się z~tej szalonej myśli i~sięgnęła do sterowania
centrum dowodzenia jakby przez gęsty muł. Oczy Walentyny świeciły się
przez moment w~świetle ognia i~Myra ujrzała w~nich odbicie jej własnej
decyzji. Potem ona i~Walentyna pochyliły się razem nad ich zadaniem. Gdy
Myra wystukiwała kody, czekała na gorący język lasera na karku.

Diamentowe statki były zbyt szybkie na sterowanie przez ludzi lub nawet
przez ich wzmocnionych, superludzkich użytkowników. Ich główne systemy
naprowadzania były łączami w~czasie rzeczywistym do stacji kosmicznych,
którą kilka dobrych eksplozji mogło zakłócić.

Niebo stało się białe, a~czarne dyski opadły jak liście.

Kaskada Kesslera nie zdarzyła się na raz. Lagrange poszedł do wieczności
natychmiast, w~jednej, przerażającej kuli piekielnie gorącej syntezy
termojądrowej, ale orbita Ziemi była inną sprawą. Godziny, może dni,
minęłyby, zanim ostatni produkt ludzkiej pomysłowości i~przemysłu został
starty z~nieba. Mimo tego satelity komunikacyjne zawiodły pierwsze.
Większość, rzeczywiście, została zdjęta przez same impulsy
elektromagnetyczne. Jadąc w~pierwszy świt nowego świata, Myra wiedziała,
że mały kamkopter tańczący kilka metrów przed nią mógł równie dobrze
przekazywać ostatnie wiadomości telewizyjne, jakie większość widzów
kiedykolwiek zobaczy.

Za nią, w~powolnej grupie, która kończyło się karetkami i~noszami
rannych i~umierających, migracja kazachska rozciągała się aż do
horyzontu. Słońce wznosiło się za nią, rzucając cienie ich
rozproszonych, obszarpanych plakatów. Teraz była tylko jedna widownia,
do której warto było przemawiać: spadkobiercy.

-- Nic nie jest zapisane -- powiedziała. -- Przyszłość należy do nas. Kiedy
zajmiemy Miasta, oszczędźmy naukowców i~inżynierów. Cokolwiek zrobili w~przeszłości, potrzebujemy ich w~przyszłości. Zróbmy to lepiej.

Kamkopter zawirował, wzbił się w~powietrze, dziko śmignął i~zanurkował w~ziemię. Kopyta koni, zużyte opony pojazdów, zniszczyły go w~sekundy.
Myra nie zmartwiła się, widziała swój własny obraz, z~sekundowym
opóźnieniem, pojawiający się w~rogu jej opaski, gdzie CNN ciągle
trajkotało. Reszta pola widzenia była wypełniona dziwnymi halucynacjami,
doświadczeniem śmierci internetu.

Bóg wypełnił horyzont, większy niż wschód słońca.


\chapter{Żniwa Młota}

Siedziałem na cokole pomnika Wyzwolicielki i~paliłem papierosa, żeby
zwalczyć nudności. Stopniowo mój umysł i~moje ciało wróciło do pewnej
równowagi. Zgiełk obchodów startu, światła domów i~pubów, stały się
znowu czymś, na co mogłem spoglądać bez niesmaku i~słyszeć bez
przerażenia. Wstałem, a~ziemia była stabilna pod moimi stopami.
Spojrzałem w~górę, a~niebo było ciemne i~gwiaździste nad głową.

Przeszedłem kilka kroków od pomnika i~się odwróciłem. Wyzwolicielka na
koniu wznosiła się nade mną. Merrial powiedziała mi, kilka tygodni temu,
powód, dlaczego wygląd Wyzwolicielki różnił się na wszystkich pomnikach,
jakie widziałem. Była mitem, mnogością. Jej hordy nigdy nie przejechały
od dalekiego Kazachstanu do starożytnych brzegów Lizbony, jak mówią
pieśni i~historie. Nigdy nie zmiotły niczego przed sobą. Zamiast tego,
każde miasto i~miasteczko zostało najechane przez hordę podniesioną
bliżej domu, w~jej własnej głębi kraju. Jak wiele setek, jak wiele
tysięcy miast napotkało nowy porządek w~formie dzikiej kobiety na koniu,
jadącej na czele obdartej armii, by ogłosić, że internet jest
zniszczony, niebo upadło, a~świat jest wolny?

To była ta ostatnia wiadomość, ostatnia wypowiedziana z~sieci i~ekranów,
która utożsamiła ich z~tą szczególną kobietą, Wyzwolicielką. Pochyliłem
się do przodu, przeczytać słowa wyryta na cokole takie, jak na każdym z~nich, od dalekiego Kazachstanu do starożytnych brzegów Lizbony:

\begin{center}

\textsc{Nic nie jest zapisane.} 

\textsc{Przyszłość należy do nas.}

\textsc{Kiedy zajmiemy miasta, \\ oszczędźmy naukowców i inżynierów. }

\textsc{Cokolwiek zrobili w przeszłości, \\ potrzebujemy ich w przyszłości.}

\textsc{Zróbmy to lepiej.\\}

\end{center}

Ostatnie słowa Starego Świata i~pierwsze Nowego.

Pomyślałem o~Merrial i~zrobiłem kolejny krok do tyłu, ciągle zaciągając
się papierosem. Była starsza niż kiedykolwiek sobie wyobrażałem jako
możliwe. Jednak była również, jak zrozumiałem, ciągle tak młoda, jak
wydawała się, gdy ją pierwszy raz ujrzałem. Nic się nie zmieniło, nic
nie mogło zmienić tej kochanej, chętnej, otwartej osobowości. Nie była
stara, jedynie pozostała\ldots młoda.

Tak jak ja zostałbym.

Na co miałem się skarżyć?

Zaśmiałem się z~siebie, z~młodzieńczych szaleństw. W~długim okresie
historii, w~obietnicy długie życia, różnice w~naszym wieku
chronologicznym, choć duże, mogły być tylko nieistotne.

Krok, świst, zapach. Jej ciepłe, suche dłonie schwyciły moją.

-- Wszystko w~porządku, Clovis?

Odwróciłem się, spojrzałem na nią i~zaciągnąłem ją ku pomnikowi.
Usiedliśmy.

-- Merrial -- powiedziałem -- wiem, kim jesteś.

-- Och -- powiedziała. -- A kim jestem?

Podałem jej broszurę, otwartą na stronie.

Siedziała przez dłuższą chwilę, patrząc w~dół na nią, z~lekkim uśmiechem
i powoli zbierającą się łzą.

-- Och, kurwa -- powiedziała. -- Wszyscy inni tam dawno nie żyją, o~ile
wiem. Jednak może nie wiedziałabym tak, jak oni nie wiedzieliby o~mnie.
-- Pociągnęła nosem i~oddała mi album. -- Zatem teraz już wiesz. Nigdy nie
chciałam być tym, co ludzie by oczekiwali po mnie, gdyby wiedzieli.

-- Ale jesteś -- powiedziałem. -- Wiedziałaś o~AI i~oczekiwałaś, że Fergal
zrobi to, co zrobił. Widziałem Twoją minę, kiedy powiedział i~to było,
jakbyś właśnie poradziła sobie z~przypadkiem białej logiki.

-- Lub czarnej! Tak, wiedziałam. Wyzwolicielka sama mi o~tym powiedziała,
tuż przed końcem. Ostrzegła mnie, że to niebezpieczna rzecz, choć
łagodna w~świetle reflektorów. Jak Fergal!

-- Ale \textit{dlaczego} mu to dałaś?

Merrial oparła się i~spojrzała w~górę. 

-- Ponieważ śmiertelne śmieci są
tam, w~górze, colha Gree. \textit{Wiem}, co zdarzyło się w~Wyzwoleniu,
ponieważ je przeżyłam. Widziałam błyski. Byłam tam, kiedy upadło niebo.
Wiedziałam, że statek nigdy nie przedostanie się bez znacznie lepszego
systemu naprowadzania niż ten, nad którym pracowałam, cóż, wiedziałam,
kiedy skończyłam testy, co nie było tak dawno temu. Potrzebowałam kogoś
do odnalezienia AI pod przykrywką szukania czegoś innego i~potrzebowałam
kogoś, kto umieści ją na statku, z~dobrych lub złych powodów.

Opuściła wzrok i~się uśmiechnęła. 

-- I~oto jesteśmy. A teraz to Ty musisz
zdecydować, \textit{mo gràidh}. Sukces tego statku pobudzi kolejne, jak
również z~innych krajów, orientalnych i~południowych. Rywalizacja
pomiędzy przedsiębiorstwami i~kontynentami, nadejście wielkich rewolucji
i droga do gwiazd przed nami. Gdyby nie wystartował, lub jego nowy umysł
wyrwał się i~by spadł, lub w~istocie AI nie była tak inteligentna, żeby
ochronić, to minęłoby dużo czasu, zanim znowu by próbowano. A następna
próba mogłaby nie być tak życzliwa jak Międzynarodowe Towarzystwo
Naukowe. To mogła być armia lub imperium.

Złapała mnie za ramiona i~wpatrzyła się we mnie. 

-- Jeżeli wejdziesz tam,
powiesz Druinowi i~chłopcom, to ciągle może się wydarzyć.

Zamknąłem oczy. 

-- Wyobrażam sobie -- powiedziałem -- ale jestem bardziej
zatroskany władzą, jaką Fergal, lub ktoś jak on, mógłby mieć.

-- Otwórz oczy -- powiedziała Merrial.

Patrzyła na mnie bardzo poważnie. 

-- Ta rzecz, AI, strateg, może tylko
robić to, co ludzie pozwolą sobie nakazać przez nią. Fergal powiedział,
że nie ma już takich ludzi. Powinien powiedzieć to, że nie ma już takich
ludzi \textit{więcej}. Twój lud, colha Gree, nie jest typem, który pozwoli
sobie na przyjmowanie rozkazów od komunistów, ponieważ nigdy nikt wam
nie rozkazywał!

-- Ach! -- powiedziałem, nagle rozumiejąc. -- Z~powodu Wyzwolenia i~Wyzwolicielki!

Merrial się roześmiała.

-- ,,Nie nam wyglądać zmiłowania Z~wyroków bożych''\footnote{cyt. z~Międzynarodówki,
zob.~\url{https://pl.wikisource.org/wiki/Mi\%C4\%99dzynarod\%C3\%B3wka}
-- przyp.tłum.} -- powiedziała krzywo. -- Twój lud sam się wyzwolił. To
kolejna rzecz, którą widziałam i~opowiem ci o~niej pewnego dnia. Jeżeli
ciągle będziesz ze mną.

-- Och, tak -- powiedziałem. -- Ciągle będę z~Tobą.

-- Dobrze -- powiedziała. -- Mamy dużo do zrobienia i~dużo czasu na to.

Rozejrzała się dowcipnie. Plac podskakiwał.

-- Więc, colha Gree, czy zamierzasz mnie zaprosić do tańca?

-- Oczywiście -- powiedziałem. -- Czy uczynisz mi ten honor?

Przez sekundę za nim zawirowaliśmy, patrzyłem na scenę za mną,
zapamiętując ją w~pamięci. Zza pomnika wschodził Mars, niebiesko-zielona
kropka na wschodzie. Cokolwiek stanie się ze Statkiem, czy wzbije się na
bezpieczną orbitę, czy zostanie rozbity na drzazgi, inne statki jakoś by
się tam dostały, na drogę ku gwiazdom.

Jakakolwiek byłaby prawda o~Wyzwolicielce, pozostanie w~moim umyśle, tak
jak była przedstawiona na statui i~wszystkich innych pomnikach i~muralach, pieśniach i~historiach, na czele jej własnej chyżej konnicy, z~rosnącą migracją za nią i~dekadenckim, wrażliwym, bezbronnym kontynentem
przed nią, a~nad jej głową i~ponad jej armią, dzielnie łopocze czarna
flaga, na które nic nie jest zapisane.


\chapter*{Posłowie od tłumacza}

W związku z tym, że trzeci tom ,,Cassini Division'' jako jedyna pozycja Kena MacLeoda, został wydany w~Polsce jako ,,Dywizja Cassini'' przez Wydawnictwo Amber, zdecydowałem się odwrócić porządek wydawania i najpierw przybliżyć Polkom i Polakom ,,Drogę do Gwiazd'' -- czwarty tom w kolejności wydawania serii.

Z punktu widzenia narracji ,,Jesiennej Rewolucji'', przedmiotowy tom jest częścią alternatywnej przyszłości wobec tej opisanej w ,,Oddziale Cassini'', zatem nie naruszam wewnętrznej chronologii.

Autor przedstawia, po raz pierwszy w swojej karierze pisarskiej, jeden z później dominujących wątków swojej twórczości -- powrót barbarzyńców, czyli upadek znanej nam cywilizacji. Tylko w jego wersji upadek cywilizacji nie jest to tożsamy z apokalipsą.

Wątek barbarzyństwa pojawia się w literaturze socjalistycznej, etatystycznej od blisko 100 lat\footnote{por.~\url{https://en.wikipedia.org/wiki/Barbarian\#Marxist_use_of_\%22Barbarism\%22}}, już w 1892 roku Karl Kautsky pisał:\\
\textit{W obecnym stanie rzeczy cywilizacja kapitalistyczna nie może trwać dalej.  Musimy albo iść naprzód w socjalizm, albo powrócić do barbarzyństwa.
}

Wtórowała mu Róża Luksemburg w 1916 roku: \\ 
\textit{Społeczeństwo burżuazyjne stoi na rozdrożu, albo przejście do socjalizmu, albo cofnięcie do barbarzyństwa.}

Zatem wydaje się, że barbarzyństwo, powrót do barbarzyństwa byłoby zjawiskiem negatywnym, a jak to postrzega Ken MacLeod?

Według niego rewolucja barbarzyńska to nie upadek i koniec świata, ale obalenie porządku politycznego, zaproponowanie w jej miejsce ,,chłopskiej utopii'', między innymi rozproszenie ośrodków władzy i uwolnienie ludzi od przymusu państwa, a zatem usunięcie hierarchii i przymusu w~imię szczęścia większości, a~nie ambicji niewielu.

Idee takiego barbarzyństwa pozostawiam osobom czytającym do rozważenia.

~

Jestem przekonany, że uważna czytelniczka odnajdzie wiele błędów w~tym tłumaczeniu. Ponoszę za to całkowitą odpowiedzialność.\\

\href{mailto:theskymyladythesky@zoho.eu}{Jacek Hummel}\\

Warszawa, maj -- lipiec 2021 roku.

\chapter*{Chronologia ,,Jesiennej Rewolucji''}

W trakcie tłumaczenia serii ,,Jesiennej Rewolucji'', udało mi się skompletować chronologię tej tetralogii, którą przedstawiam poniżej.

\begin{description}
\item[1995] wydanie pierwsze ,,Gwiezdna Frakcja'' (The Star Fraction)
\item[1996] wydanie pierwsze ,,Kamienny Kanał'' (The Stone Canal)
\item[1998] wydanie pierwsze ,,Oddział Cassini'' (The Cassini Division)
\item[1999] wydanie pierwsze ,,Droga do Gwiazd'' (The Sky Road)\\
wydanie polskie ,,Dywizja Cassini'', Wydawnictwo Amber\\ 
\item[2015] powstanie Republiki
\item[2017] ,,Czwarta Międzynarodówka'' zaczyna otrzymywać rady od AI znanej jako Zegarmistrz lub Generał\\
\item[2021] polskie tłumaczenie ,,Jesiennej Rewolucji''\\
\item[2025] upadek Republiki\\
    Wojna niemiecko-polska znana też jako Trzecia Wojna Światowa\\
    Powstanie USA/ONZ
\item[2026] powstanie Norlonto
\item[2028] Traktat o Restytucji i powstanie reżimu hanowerskiego
\item[2032] Moh Kohn zakłada ,,Spółdzielnię Robotników Obrony imienia Feliksa Dzierżyńskiego'' 
\item[2045] Jesienna Rewolucja \\
wydarzenia opisywane w ,,Gwiezdnej Frakcji''
\item[2057] stulecie wystrzelenia Sputnika, 
\item[28 maja 2059] ,,Wyzwolenie'' przez Myrę Godwin-Dawidową\\ 
\begin{enumerate}
\item Wersja przyszłości nr 1 --- ,,Droga do Gwiazd'' --- dzieje się w bliżej nieokreślonej przyszłości co najmniej 200 lat, prawdopodobnie około 2300 roku, w zgodzie z wersją przyszłości nr 2.\\
\item Wersja przyszłości nr 2 --- ,,Oddział Cassini''
\begin{description}
\item[2080] Artykuł I.K. Malleya o tunelach czasoprzestrzennych
\item[3 marca 2093]  przebudzenie się J. Wilde'a w robocie na orbicie Jowisza
\item[2093-2094] budowa Mili Malleya
\item[2094] przejście Statku z Ziemi na Nowy Mars
\item[2296] wydarzenia opisane w ,,Kamiennym Kanale''\\
powrót J.Wilde'a z córki wormhole'a do Układu Słonecznego
\item[2303] wydarzenia opisane w ,,Oddziale Cassini''
\end{description}
\end{enumerate}
\end{description}

\chapter*{Seria ,,Czarna Flaga''}

\begin{center}
\begin{large}
W serii \textit{Czarna Flaga} dotychczas opublikowano online:
\end{large} 
\end{center}


\begin{enumerate}
\item \href{https://archive.org/details/joanna-russ-mezczyzna-rodzaju-zenskiego/Joanna_Russ_M\%C4\%99\%C5\%BCczyzna_rodzaju_\%C5\%BCe\%C5\%84skiego}{Mężczyzna rodzaju żeńskiego}, Joanna Russ
\item Engines of Light t. 1 -- \href{https://archive.org/details/ken-macleod-wieza-kosmonauty}{Wieża Kosmonauty}, Ken MacLeod
\item Jesienna Rewolucja t. 1 -- \href{https://archive.org/details/ken-mac-leod-jesienna-rewolucja-gwiezdna-frakcja}{Gwiezdna Frakcja}, Ken MacLeod
\item Jesienna Rewolucja t. 2 -- \href{https://archive.org/details/ken-mac-leod-jesienna-rewolucja-kamienny-kanal}{Kamienny Kanał}, Ken MacLeod
\item Jesienna Rewolucja t. 4 --  Droga do Gwiazd, Ken MacLeod
\end{enumerate}

\begin{center}

\begin{large}W planach:\end{large}\end{center}

\begin{enumerate}
\item Jesienna Rewolucja t. 3 -- Oddział Cassini\footnote{dostępna, jako ,,Dywizja Cassini'', w~antykwariatach, bibliotekach lub w~wersji ebook na stronach \url{https://doci.pl/} lub \url{https://docer.pl/}}, Ken MacLeod
\item Radicalized\footnote{tytuł roboczy: Zradykalizowane}, Cory Doctorow
\item Walkaway\footnote{tytuł roboczy: Odchodzący}, Cory Doctorow
\item Dhalgren, Samuel R.~Delany
\end{enumerate}

\newpage

Projekt jest przygotowywany dzięki Wolnemu Oprogramowaniu. Zestaw narzędzi składa się z:
\begin{itemize}
\item \href{https://ubuntu.com/}{Ubuntu 20.04 Ogniskowa Fossa} -- system operacyjny
\item \href{https://omegat.org/}{OmegaT} -- narzędzie wspomagające tłumaczenie (CAT)
\item \href{https://github.com/soimort/translate-shell}{translate-shell} -- narzędzie do tłumaczenia w~\href{https://translate.google.pl}{Google Translate} przez terminal 
\item \href{https://glosbe.com/en/pl}{Glosbe} -- największy słownik online
\item \href{https://www.libreoffice.org/}{LibreOffice} -- przetwarzanie dokumentów 
\item \href{http://pandoc.org}{pandoc} -- uniwersalny konwerter dokumentów 
\item \href{https://www.latex-project.org/}{LaTeX} -- redakcja, skład i~łamanie dokumentu
\item \href{https://sigil-ebook.com/}{sigil} -- przetwarzanie plików ebook
\item \href{https://calibre-ebook.com/}{calibre} -- konwersja plików ebook
\end{itemize}



%EPUB
\newpage
\printendnotes
%EPUB

\tableofcontents{}
\end{document}
