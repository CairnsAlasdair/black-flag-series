\documentclass[oneside,polish,11pt,rmheadings]{mwbk}

%polonizacja
\usepackage[T1]{fontenc}
\usepackage[polish]{babel}
\usepackage[utf8]{inputenc}
\usepackage{polski} 
\frenchspacing 
\usepackage{indentfirst} 
%koniec polonizacja
%grafika
\usepackage{graphicx,longfbox,caption}
%\usepackage{garamondlibre}
\usepackage{times}
\usepackage[a5paper]{geometry} %wielkość papieru (148x210-book w~PL)
\newcommand{\threeast}{\bigskip\par\centerline{*\,*\,*}\medskip\par}

\captionsetup{justification   = raggedright,
              singlelinecheck = false}
%EPUB
%\usepackage[hyperfootnotes=true,unicode, pdftex]{hyperref} 
\usepackage[hyperfootnotes=true]{hyperref} %original
%move footnotes to endnotes
\usepackage{enotez}
\let\footnote=\endnote
\setenotez{
  list-name = Przypisy,
  backref = true
}

%pdf anonimize
%dla EPUB wykomentować
%\pdfsuppressptexinfo=-1 %Suppress PTEX.Fullbanner and info of imported PDFs

%pakiet odnośników i~pdf metadata
%\usepackage[unicode, pdftex]{hyperref}
%\hypersetup{pdfauthor={David Graeber},
%            pdftitle={Akcja Bezpośrednia},
%            pdfsubject={Direct Action},
%            pdfkeywords={red. Jacek Hummel, Creative Commons, tłumaczenie CC BY 4.0, etnografia, akcja bezpośrednia},
%            pdfcreator={pdfLaTeX}}
%dla EPUB koniec wykomentowania

\begin{document}

%Direct Action
\begin{titlepage} % Suppresses displaying the page number on the title page and the subsequent page counts as page 1
	\newcommand{\HRule}{\rule{\linewidth}{0.5mm}} % Defines a new command for horizontal lines, change thickness here
	
	\center % Centre everything on the page
	\normalsize 
	\includegraphics[width=0.4\textwidth]{CC.png}\\
	%\\[1cm] 
\href{https://creativecommons.org/licenses/by/4.0/deed.pl}{ Uznanie autorstwa 4.0 Międzynarodowe}\\[1.5cm]
	
	\textsc{\Huge David Graeber}\\[0.5cm] % Author
	
	\rule{\textwidth}{1.6pt}\vspace*{-\baselineskip}\vspace*{2pt} % Thick horizontal rule
	\rule{\textwidth}{0.4pt}\\[0.4cm] % Thin horizontal rule
		
	{\huge\bfseries Direct Action}\\[0.4cm] % Title of your document
    {\huge\bfseries Etnography}\\[0.1cm] % Title of your document

	\rule{\textwidth}{0.4pt}\vspace*{-\baselineskip}\vspace{3.2pt} % Thin horizontal rule
	\rule{\textwidth}{1.6pt}\\[0.5cm] % Thick horizontal rule	

	
    \textsc{\Large Akcja Bezpośrednia}\\[0.4cm] % Major heading such as course name
	\textsc{\large Etnografia}\\[2.5cm] % Minor heading such as course title
		
	{\large\textsc{redakcja:}}\\
	 \href{mailto:theskymyladythesky@zoho.eu}{Jacek \textit{Hummel}}
	\vfill
	{\large Warszawa, 2023} % Date, change the \today to a set date if you want to be precise	
\end{titlepage}

\pagestyle{plain}
\begin{figure}
\centering{\includegraphics[width=\textwidth]{titlepage.jpg}}
\end{figure}

%\setcounter{tocdepth}{9}
\tableofcontents

\chapter*{Przedmowa}

 Książka tej wielkości jest w~dzisiejszych czasach niezwykła. Z~pewnością nie był to mój pierwotny plan. Kiedy po raz pierwszy zdecydowałem się zacząć spisywać niektóre z~moich doświadczeń akcji bezpośredniej z~perspektywy etnograficznej, miałem zamiar napisać dość krótką książkę. Jednak im więcej pisałem, tym bardziej zagadnienie wydawało się rozrastać. Zdałem sobie sprawę, że mam do czynienia z~powszechnym dylematem pisania etnografa: punkty, które wydają się proste i~oczywiste dla każdego, kto spędził lata w~danym uniwersum kulturowym, wymagają dużej ilości atramentu, aby przekazać komuś, kto tego nie doświadczył. Coś podobnego przydarzyło mi się, kiedy wiele lat temu wróciłem do Chicago z~podróży badawczej w mojej pracy doktorskiej na Madagaskarze. Pamiętam, że martwiłem się tym, ile mam do powiedzenia. Czułem, że o społeczności, którą studiowałem, mogę powiedzieć co najwyżej dwie lub trzy naprawdę interesujące uwagi. W chwili, gdy zacząłem pisać, zdałem sobie sprawę, że wyjaśnienie któregokolwiek z~tych punktów komuś, kto nie pochodzi z~wiejskiej społeczności Madagaskaru, wymagałoby kilkuset stron. Zanim skończyłem pisać, zdałem sobie również sprawę, że większość czytelników prawdopodobnie uznałaby tę ekspozycję za znacznie bardziej interesującą, w~sumie, niż to, co początkowo uważałem za ,,uwagę''.

 Nazwijmy zatem tę książkę hołdem dla ciągłej wartości opisu etnograficznego. Przez ,,opis etnograficzny'' rozumiem ten rodzaj, który ma na celu opisanie konturów społecznego i~pojęciowego wszechświata w~sposób, który jest jednocześnie teoretycznie informujący, ale sam w~sobie nie jest przeznaczony do poparcia argumentu lub teorii. Były czasy, kiedy szczegółowy opis systemu politycznego lub ceremonialnego lub systemu wymiany w~Afryce lub Amazonii był uważany za cenny wkład do ludzkiej wiedzy samej w~sobie. Tak już nie jest. Antropologowi z Afryki czy Amazonii, a nawet z~niektórych części Europy, może i~ujdzie na sucho napisanie takiej książki. Obecnie, konwencja akademicka w~Ameryce (którą młody uczony nie powinien ignorować) głosi, że należy udawać, że opis ma na celu przedstawienie jakiejś większej odpowiedzi. Wydaje mi się to niefortunne. Po pierwsze, myślę, że ogranicza to możliwości książki do trwania. Klasyczne etnografie można przecież reinterpretować. Nowe -- jakkolwiek fascynujące -- rzadko zawierają wystarczającą ilość materiału, aby to umożliwić; a to, co jest, jest zwykle ściśle zorganizowane wokół określonego argumentu lub powiązanej serii takich argumentów.

 Pozwolę sobie zatem od razu ostrzec czytelnika: nie ma w~tej książce żadnego konkretnego argumentu -- może taki, że ruch w~niej opisany jest wart przemyślenia. Nie oznacza to, że ta książka nie zawiera argumentów teoretycznych. W jej trakcie przedstawiam ich dowolną liczbę: czy to o ideologicznej roli dużych, ciężkich przedmiotów, politycznych implikacjach słowa ,,opinia'', podobieństwie pisania wiadomości i~homeryckiej kompozycji epickiej, czy kosmologicznej roli policja w~kulturze amerykańskiej. Tym, co sprawia, że jest to dzieło etnograficzne w~klasycznym znaczeniu tego słowa, jest to, że, jak to kiedyś ujął Franz Boas, ogół jest w~służbie konkretu -- być może poza refleksjami końcowymi. Teoria jest przywoływana głównie po to, aby pomóc w~ostatecznym zadaniu opisu. Anarchiści i~kampanie akcji bezpośrednich nie istnieją po to, by pozwolić niektórym naukowcom na przedstawienie teoretycznego punktu lub udowodnienie, że teoria rywala jest błędna (tak samo jak rytuały balijskiego transu czy andyjskie technologie irygacyjne) i~wydaje mi się, że sugerowanie czegoś takiego jest wstrętne. Chciałbym sądzić, że w~rezultacie zainteresowanie tą książką może nie tylko pozostać w~centrum zainteresowania osób kierujących się ciekawością historyczną, którzy chcą zrozumieć, jak to było naprawdę być w~środku tych wydarzeń, ale postawić tego samego rodzaju pytania, jakie stawiali w~nim aktorzy, o naturę demokracji, autonomię i~możliwości -- a właściwie dylematy, ograniczenia -- strategii transformacyjnej akcji politycznej.


\bigskip
\noindent KILKA SŁÓW O KONTEKŚCIE HISTORYCZNYM

 Minęło już wystarczająco dużo czasu od zapierających dech w~piersiach dni 2000 i~2001 roku, aby być może można zacząć patrzeć na ten historyczny moment z~pewnej perspektywy. Jest teraz jasne, że ten okres stanowił pewien przełom dla globalnego neoliberalizmu. Były to lata, w~których ,,konsensus waszyngtoński'', z~lat 90., został rozbity. Stało się to bardzo szybko. W rzeczywistości jest to świadectwo skuteczności akcji bezpośredniej, skoro zajęło zaledwie około trzech lat masowej mobilizacji ludowej, aby tego dokonać.

 W dzisiejszych czasach czasami trudno sobie przypomnieć, jak wyglądały czasy konsensusu waszyngtońskiego. Być może najlepiej byłoby zacząć od słów kontekstu, aby pomóc zrozumieć, dlaczego bunt Zapatystów w~roku 1994 posłużył jako taki katalizator globalnego ruchu przeciwko neoliberalizmowi, który nastąpił po nim, i dlaczego ruch ten przybrał taką formę.

\bigskip
\noindent CHWILOWE ZAWIESZENIE HISTORII

 Lata tuż przed ogłoszeniem się przed światem buntu Zapatystów w~Chiapas były prawdopodobnie najbardziej przygnębiającym okresem dla rewolucjonistów -- a nawet oddanych ideałom lewicy --  ostatnich dekad. To nie upadek stalinowskich reżimów w~Europie Wschodniej był przygnębiający; większość radykałów cieszyła się, że odchodzą. Przygnębiające było to, co wydarzyło się później. Po śmierci stalinizmu większość marksistów spodziewała się renesansu bardziej humanitarnych form marksizmu. Socjaldemokraci wierzyli, że w~końcu wygrali spór z~rewolucyjną lewicą i~spodziewali się pchnąć dawnych poddanych bloku sowieckiego do swojej ,,owczarni''; rozsądne oczekiwanie, ponieważ podczas ankiety większość mieszkańców Europy Środkowo-Wschodniej stwierdziła, że  chce wzorować swoje nowe gospodarki na Szwecji. Zamiast tego, przeszli terapię szokową i~najdzikszą formę nieograniczonego kapitalizmu. Pod każdym względem świat zdawał się zmierzać w~stronę koszmarnego scenariusza. Romantyczny obraz partyzanckiego powstania, który w~latach 60. zawładnął tak wieloma wyobraźniami, przeradzał się w~rodzaj obscenicznej parodii. Już w~latach 80. prawica, która od lat argumentowała, że partyzantki w~miejscach takich jak Wietnam, Zimbabwe czy Salwador nie są spontanicznymi, ale wynikiem szatańskich planów stworzonych przez zagranicznych ideologów, zaczęła wprowadzać w~życie własne teorie przy pomocy amerykańskich i~południowoafrykańskich agencji wywiadowczych tworzących armie partyzanckie, takie jak contras lub RENAMO, by szczuć je na lewicowe reżimy. W tym samym czasie istniejące marksistowskie ruchy partyzanckie od Kolumbii po Angolę, które zaczynały jako pełne szlachetnej retoryki, skłaniały się do zostania królami bandytów lub armiami nihilistycznymi bez żadnego celu poza własnym buntem (te, które trzymały się starego ideału społecznej transformacji, jak Shining Path w~Peru wydawały się jeszcze gorsze). Wszędzie ruchy wyzwoleńcze przekształcały się w~okrutne wojny etniczne. Potem przyszła fala ludobójstwa, z~których Rwanda i~była Jugosławia były tylko najbardziej dramatyczne i~widoczne.

 Na kilkunastu zazębiających się płaszczyznach jednocześnie, wyłaniający się wzór wydawał się katastrofalny. Wydawało się, że będzie mniej więcej tak: na poziomie międzynarodowym kapitalizm przekształci się w~siłę rewolucyjną. Porzuciwszy opiekuńczą wersję kapitalizmu, która faktycznie wygrała zimną wojnę, starzy Wojownicy Zimnej Wojny i~ich korporacyjni sponsorzy domagali się czystej, wolnej od wszelkich ograniczeń, wolnorynkowej wersji, która nigdy nie istniała, i~byli gotowi do spustoszenia wszystkich istniejących instytucjonalnych rozwiązań społecznych, aby to osiągnąć. Wszystko to wiązało się z~pewnym rodzajem dziwnej inwersji. Standardowa linia prawicowa, przynajmniej od lat 90. XVIII wieku, zawsze głosiła, że rewolucyjne marzenia są niebezpieczne właśnie dlatego, że są utopijne: ignorowały prawdziwą złożoność życia społecznego, tradycji, autorytetu i~ludzkiej natury, oraz marzyły o przekształceniu świata zgodnie z~jakimś abstrakcyjnym ideałem. W latach 90. miejsca zostały całkowicie odwrócone. Lewica w~dużej mierze porzuciła utopie (a im bardziej to robiła, tym bardziej się kurczyła i~upadała), a mimo to prawica je podchwyciła. Wolnorynkowi ,,reformatorzy'' z~dnia na dzień zaczęli ogłaszać się rewolucjonistami -- problem polegał na tym, że zrobili to jako najgorsi stalinowcy, zasadniczo mówiąc biednym na świecie, że nauka dowiodła, że  istnieje tylko jeden sposób, by iść naprzód w~historii, że jest to zrozumiałe przez naukowo wyszkoloną elitę i~dlatego musieli się zamknąć i~robić to, co im kazano, ponieważ nawet jeśli ich recepty mogą spowodować ogromne cierpienie, śmierć i~migracje w~teraźniejszości, w~pewnym momencie w~przyszłości (nie byli pewni kiedy) wszystko to doprowadzi do raju pokoju i~dobrobytu. Fakt, że ,,nauka'' przeszła z~pozycji materializmu historycznego do ekonomii wolnorynkowej, był dość drobnym szczegółem; w~każdym razie ułatwiało to wyjaśnienie, w~jaki sposób byłym stalinowcom od Rumunii po Wietnam tak łatwo było po prostu zmienić kapelusz i~ogłosić się neoliberałami. Tymczasem, gdy polityka dostosowania strukturalnego pozbawiła najbiedniejszych mieszkańców planety niewielkich zabezpieczeń społecznych, propaganda i~manipulacje statystyczne stały się tak skuteczne, że większość Amerykanów głównego nurtu, którzy zwracali uwagę na takie sprawy, była przekonana, że  warunki dla najbiedniejszych na świecie są rzeczywiście coraz lepsze, i~to nie tylko w~obszarach takich jak Azja Wschodnia, które w~większości odmówiły przyjęcia neoliberalnej polityki.

 Każde postępowe zwycięstwo wydawało być zagrożone lub odwrócone. W Republice Południowej Afryki pokolenia walk ostatecznie wyeliminowały rasowy apartheid; chwila szczęścia, oczywiście, ale w~skali globalnej tworzył się niemal identyczny system, oparty na coraz bardziej zmilitaryzowanych granicach i~reżimie migracji zarobkowej, gdzie dla uwięzionych w~biednych krajach pobyt w~bogatych, w~większości białych krajach był uzależniony od posiadania dokumentów tożsamości i~chęci pracy w~zawodach, których sami mieszkańcy nie byli skłonni wykonywać. Feminizm był ograniczany. Wcześniejsze zwycięstwa w~niewolniczych fabrykach, wobec pracy dzieci, a nawet niewolnictwem zostały zniszczone lub wręcz wykorzenione.

 Wiele z~problemów wynikało właśnie z~klęski marzenia o rewolucji społecznej i~tych utopijnych fantazji, które zawsze były konieczne, aby inspirować ludzi do pasji i~poświęcenia, niezbędnych do rzeczywistej pracy na rzecz przemiany świata w~kierunku większej wolności i~większej równości. Mam tu na myśli autentyczny, żywy utopizm -- ideę, że radykalne alternatywy są możliwe i~że można zacząć je tworzyć w~teraźniejszości -- w~przeciwieństwie do tego, co można by nazwać ,,utopizmem naukowym'' idei, że rewolucjonista jest sprawcą nieuniknionego marszu historii, który tak łatwo i~katastrofalnie przywłaszczyła sobie prawica. Mordowanie marzeń mogło prowadzić tylko do koszmarów. Uniemożliwiło to utworzenie centrum, z~którego można by walczyć z~najazdami (obecnie supernaładowanej, rewolucyjnej) prawicy. Partie socjaldemokratyczne w~Europie, na przykład, zrodzone z~reformistycznego szczepu marksizmu, wydawały się początkowo raczej zadowolone z~upadku ich rewolucyjnych kuzynów -- w~końcu wygrały spór -- dopóki nie zdały sobie sprawy, że ich własny urok i~chęć angażowanie się kapitalistów, były prawie całkowicie oparte na ich zdolności do pozycjonowania się jako mniej groźna alternatywa. W niedługim czasie ustroje socjaldemokratyczne doświadczyły takiego moralnego i~politycznego upadku, że nieliczni, którzy nadal byli u władzy, zostali zredukowani do roli agentów demontażu państw opiekuńczych, które pierwotnie stworzyli. Działająca lewica w~krajach uprzemysłowionych stawała się coraz bardziej reakcyjna, zdolna mobilizować pasje tylko do obrony rzeczy, które już istniały -- warstwy ozonowej, programów akcji afirmatywnej, drzew -- i~to coraz bardziej nieefektywnie. Gdzie indziej wydawało się, że prawie całkowicie się załamała.

 Potem w~końcu nastąpiła ,,globalizacja''.

 Jak niedawno przypomniała nam Anna Tsing (2002), to ciekawa historia. Pojęcie naprawdę zaczynało jako postępowe. To była silniejsza wersja internacjonalizmu: poczucie, że wszyscy ludzie są nie tylko braćmi, ale że jesteśmy wspólnymi opiekunami jednej, kruchej planety -- idea zawarta w~fotografiach Ziemi zrobionych z~kosmosu przez astronautów w~latach sześćdziesiątych. Retoryka globalizacji z~lat 90. nic z~tego nie zawierała. Zasadniczo opierała się na dwóch nogach: po pierwsze, telekomunikacja, a zwłaszcza Internet, niszczyły odległość i~umożliwiały natychmiastowy kontakt między dowolną częścią planety; po drugie, upadek żelaznej kurtyny i~inne bariery w~handlu jednocześnie tworzyły jednolity, zunifikowany rynek globalny, którego mechanizmy finansowe mogłyby następnie działać za pośrednictwem tych samych natychmiastowych środków elektronicznych. Głównie chodziło tylko o siłę kapitału finansowego. Jednak retoryce towarzyszyła zwykle seria bardzo szerokich uogólnień: że nie tylko pieniądze, ale produkty, idee i~ludzie ,,przepływali'' jak nigdy dotąd, gospodarki narodowe nie mogły już marzyć o byciu autonomicznymi; stare ideologie nacjonalistyczne, a nawet granice państwowe, stawały się coraz bardziej nieistotne i~tak dalej. Wszystko to zostało przedstawione jako coś, co dzieje się samo z~siebie. Zaawansowane technologie sprawiły, że ludzie coraz bardziej się ze sobą kontaktowali: jedynym możliwym językiem, w~jakim mogli się ze sobą kontaktować, był handel -- ponieważ kapitalizm był przecież zakorzeniony w~ludzkiej naturze.

 Dla kogokolwiek, kto naprawdę uważał, oczywiście, rzeczywistość była odmienna. Granice nie były zacierane, ale wzmacniane. Ubogie populacje nadal były zamykane w~swoich krajach pochodzenia (w których istniejące świadczenia socjalne były szybko wycofywane). ,,Globalizacja'' odnosiła się jedynie do zdolności kapitału finansowego do przeskakiwania według własnego uznania i~wykorzystywania tego faktu. Przede wszystkim jednak okres ,,globalizacji'' -- lub neoliberalizmu, jak zaczęto go nazywać wszędzie poza Ameryką -- był świadkiem stworzenia pierwszego w~historii ludzkości prawdziwie planetarnego systemu biurokratycznego.

 Z perspektywy czasu wyobrażam sobie, że tak będą postrzegane ostatnie lata XX wieku. ONZ istniała oczywiście od połowy stulecia, ale ONZ nigdy nie opierała się na niczym więcej niż autorytet moralny. To, co teraz budowano, to był system z~zębami. Na szczycie znajdowali się finansiści -- bankierzy, handlarze walutami, właściciele funduszy hedgingowych i~tym podobni -- wszyscy połączeni elektronicznie. Powstały gigantyczne, zorganizowane biurokratycznie korporacje transnarodowe, które w~tym okresie wchłaniały i~konsolidowały dosłownie miliony wcześniej niezależnych przedsiębiorstw. Powstały organizacje biurokratyczne zajmujący się globalnym handlem -- International Monetary Fund (IMF -- Międzynarodowy Fundusz Walutowy -- IMF), Bank Światowy, World Trade Organization (WTO -- Światowa Organizacja Handlu ) i~tak dalej, ale także instytucje takie jak Rezerwa Federalna USA, organizacje traktatowe, takie jak Unia Europejska (UE) czy Północnoamerykańska Umowa o Wolnym Handlu (NAFTA), których główną rolą wydawała się ochrona interesów dwóch pierwszych. i~wreszcie, powstały różne poziomy organizacji pozarządowych, których rola, od udzielania kredytów na farmy po zaszczepianie niemowląt lub dostarczanie żywności w~czasie głodu, coraz częściej polegała na świadczeniu usług, które kiedyś oczekiwano od państw, ale które teraz zostały skutecznie zakazane przez IMF. Godne uwagi jest to, że osiągnięto to poprzez ideologię radykalnego indywidualizmu: przede wszystkim szerokie odrzucenie roszczeń wspólnoty, a w~szczególności wspólnoty politycznej. Wszyscy mieliśmy być racjonalnymi jednostkami na rynku, dążącymi do nabywania dóbr. Jeżeli byliśmy różni, miała to być kwestia osobistej samorealizacji poprzez konsumpcję, ponieważ z~kolei konsumpcja miała w~dużej mierze polegać na tworzeniu i~wyrażaniu tożsamości. Wtedy, oczywiście, można by powiedzieć, że tożsamość kręciła się w~miejscu: skoro zakładano, że wszystkie kwestie polityczne i~ekonomiczne zostały skutecznie rozwiązane (historia pod tym względem się skończyła), polityka tożsamościowa stała się niemal jedyną polityką, którą można uznać za uprawnioną.

\bigskip
\noindent WTEDY HISTORIA ZNOWU SIĘ TOCZYĆ

 Wszystko to sprawia, że  łatwo zrozumieć, dlaczego bunt Zapatystów -- który rozpoczął się 1 stycznia 1994 roku, w~dniu wejścia w~życie Północnoamerykańskiego Układu Wolnego Handlu (NAFTA) -- stanowił punkt zwrotny. Zapatystom, z~ich odrzuceniem staromodnej partyzanckiej strategii przejmowania kontroli nad państwem poprzez walkę zbrojną, z~ich wezwaniem do stworzenia autonomicznych, demokratycznych, samorządnych społeczności, w~sojuszu z~globalną siecią podobnie myślących demokratycznych rewolucjonistów, udało się skrystalizować, często pięknym poetyckim językiem, wszystkie szczepy sprzeciwu, które w~poprzednich latach powoli się zlewały. Jak członkowie kolektywu Midnight Notes zaczęli słusznie wskazywać nawet w~tamtym czasie, sprzeciw wobec narzuconej przez IMF polityki dostosowania strukturalnego (niezależnie od tego, czy przybiera ona formę kampanii na rzecz praw rdzennych mieszkańców Ameryki Łacińskiej, afrykańskie zamieszki związane z~żywnością czy indonezyjskie ruchy islamistyczne) prawie zawsze opierały się na moralnej obronie jakiegoś zbiorowego zasobu: prawa do traktowania ziemi, żywności, paliw kopalnych, a nawet kultury nie jako towaru rynkowego, ale jako wspólnego dobra administrowana przez jakąś formę wspólnoty moralnej -- nawet jeśli w~coraz mniejszej liczbie przypadków państwo narodowe było postrzegane jako właściwy strażnik takich praw lub ram danej wspólnoty moralnej. Niemal zawsze ich cele były postawione zarówno lokalnie, jak i~w skali planetarnej. Zapatyści, z~ich zręczną umiejętnością wykorzystywania pojawiających się globalnych technologii komunikacyjnych do mobilizowania międzynarodowych sieci do obrony własnych autonomicznych enklaw w~Lacandon Rain Forest, byli nie tylko idealnym symbolem, udało im się wyrazić to, co się wydarzało, poprzez nowe podejście do samej idei rewolucji.

 W związku z~tym to Zapatyści, wraz ze swoimi dwoma międzynarodowymi encuentros ,,Za Ludzkość i~Przeciw Neoliberalizmowi'', zaczęli kłaść fundamenty pod ruch, który stał się znany jako ruch ,,antyglobalistyczny'' Otóż ten termin, jak już wielokrotnie mówiłem, jest trochę mylący. To był w~zasadzie wynalazek mediów. Najbardziej dynamiczne i~najważniejsze elementy ruchu zawsze postrzegały go jako dążenie do autentycznej, demokratycznej formy globalizacji; przynajmniej powrót do tego rodzaju świadomości planetarnej, z~której pojęcie to wyłoniło się po raz pierwszy. W przypadku anarchistów, autonomistów i~innych radykalnych grup oznaczało to całkowite zatarcie wszystkich międzynarodowych granic. To, co wyłoniło się z~encuentros Zapatystów, było luźno zorganizowaną siecią planetarną zwaną Globalną Akcją Ludów (PGA), której jednym z~celów było ponowne umieszczenie akcji bezpośredniej bez użycia przemocy na arenie światowej jako siły prowadzącej do globalnej rewolucji.Z PGA był ważny przede wszystkim dlatego, że wyraźnie odrzucał udział partii politycznych lub jakiejkolwiek grupy, której celem było wejście do rządu. To z~kolei PGA wydało pierwsze ,,wezwania do akcji'', które ostatecznie zakończyły się akcjami w~Seattle w~listopadzie 1999 roku. Zamiast próbować samemu opowiedzieć tę historię -- będzie ona opowiadana wiele razy, na różne sposoby w~trakcie książki -- pozwólcie, że zamiast tego przedstawię czytelnikowi oś czasu zawierającą tylko najważniejsze wydarzenia. Poniżej znajduje się prosty opis, który odzwierciedla perspektywę bardzo północnoamerykańską, ale czytający mogą uznać za przydatne konsultowanie się od czasu do czasu podczas czytania tej pracy: 

\medskip


\noindent \textbf{1 stycznia 1994 roku} Wchodzi w~życie Północnoamerykańska Umowa o Wolnym Handlu (NAFTA -- North American Free Trade Agreement). Powstanie EZLN (lub Ejército Zapatista de Liberación Nacional, lub Zapatistas) w~Chiapas rozpoczyna się niespodziewaną ofensywą militarną, która na krótko prowadzi do zajęcia stolicy stanu Chiapas, San Christobal de las Casas. Zapatyści jednak szybko przekształcają się z~siły ofensywnej w~defensywną, tworząc szereg samorządnych wspólnot autonomicznych, poszukując międzynarodowych sojuszników i~propagując politykę akcji bezpośredniej, demokratycznych eksperymentów i~nowego podejścia do rewolucji, które są zbieżne z~tradycją anarchistyczną odrzucającą tradycyjne próby transformacji poprzez przejęcie władzy państwowej.

\noindent \textbf{Sierpień 1997 roku} Drugie zapatystowskie ,,International Encuentro For Humanity and Against Neoliberalism'' w~Hiszpanii kończy się wezwaniem do stworzenia międzynarodowej sieci, która ostatecznie będzie znana (w języku angielskim) jako Peoples' Global Action (PGA). Poza samymi Zapatystami rdzeń PGA składa się początkowo z~brazylijskiego Ruchu Rolników Bezrolnych (MST), Indyjskiego Stowarzyszenia Rolników Stanowych Karnataka (KRRS, masowego ruchu Gandhyjskiego), anarchistycznych lub zainspirowanych grup, w~tym Ya Basta! we Włoszech i~Reclaim the Streets w~Wielkiej Brytanii oraz różnych ruchów tubylczych i~agrarnych, oraz radykalnych związków zawodowych.

\noindent \textbf{18 czerwca 1999 roku} ,,J18'', pierwszy masowy sponsorowany przez PGA globalny dzień akcji, znany na przemian jako ,,Globalny Dzień Akcji Przeciwko Centrom Finansowym'' lub ,,Karnawał Przeciw Kapitalizmowi'', zbiegający się ze spotkaniami G8 przywódców głównych potęg przemysłowych, ze skoordynowanymi działania w~ponad stu miastach na całym świecie od Australii po Zimbabwe. W Ameryce organizowanych jest kilka demonstracji, głównie pod szyldem nowych amerykańskich wersji Reclaim the Streets.

\noindent \textbf{30 listopada 1999 roku} Akcje ,,N30'' przeciwko spotkaniom ministerialnym WTO w~Seattle, kolejny międzynarodowy dzień akcji zaproponowany przez PGA. Akcja jest od dawna planowana, ale jest całkowitym zaskoczeniem dla mediów głównego nurtu, które postrzegają ją jako narodziny ruchu. Seattle było świadkiem ostrych podziałów co do taktyki między pokojowymi demonstrantami przeprowadzającymi blokady i~blokady hotelu, w~którym odbywa się ministerstwo, zorganizowane przez nowo utworzoną Sieć Akcji Bezpośrednich (DAN), a uczestnikami mniejszego ,,Czarnego Bloku'', składającego się głównie z~anarchistów i~radykalnych ekologów, którzy mają bardziej bojową interpretację niestosowania przemocy i~którzy po tym, jak policja zaczyna atakować blokujących, rozpoczynają kampanię ukierunkowanego niszczenia mienia przeciwko symbolom korporacyjnej władzy (głównie okien) w~centrum miasta. Pierwszego dnia spotkania były właściwie zamykane, a negocjacje zakończyły się fiaskiem. Kolejne dni to masowe represje, których kulminacją było ogłoszenie stanu wojennego i~wezwanie Gwardii Narodowej. Miesiące bezpośrednio po Seattle wypełnione były wybuchem nowych organizacji i~aktywności oraz tworzeniem autonomicznych oddziałów DAN w~miastach w~całych Stanach Zjednoczonych, a nawet w~Kanadzie.

\noindent \textbf{16 kwietnia 2000 roku} Akcje ,,A16'' przeciwko posiedzeniom Banku Światowego i~IMF w~Waszyngtonie. Chociaż nie jest tak udana taktycznie jak w~Seattle (spotkania nie są przerwane), A16 wyznacza początek zbliżenia między organizatorami DAN a autonomicznym Rewolucyjnym Blokiem Antykapitalistycznym (RACB) -- Czarnym Blokiem zebranym na tę okazję -- z~RACB powstrzymującym się od niszczenia własności, a zamiast tego zapewnianiem wsparcia dla demonstrantów i~tych, którzy zostali aresztowani.

\noindent \textbf{1 sierpnia 2000 roku} Działania ,,R2K'' przeciwko Konwencji Republikańskiej w~Filadelfii. W połączeniu z~akcjami D2K przeciwko Konwencji Demokratycznej w~Los Angeles, są one wspólnie znane wśród aktywistów jako R2D2. Podczas gdy LA DAN odrzuca szeroko zakrojoną akcję bezpośrednią na rzecz strategii marszów w~sojuszu z~grupami społeczności, akcje w~Filadelfii, organizowane są przede wszystkim przez DAN w~Nowym Jorku, Filadelfii i~DC, co oznaczało dalszą integrację Czarnych Bloków i~blokad z~,,Rewolucyjnym Blokiem Antyautorytarnym'', co w~tym przypadku stanowiło dywersję, aby odciągnąć policję od blokad. Filadelfia jest również naznaczona próbą stworzenia sojuszy między w~większości białymi DAN a radykalnymi ludźmi z~kolorowych organizacji, z~mieszanym sukcesem. Patrząc wstecz, jest to postrzegane jako punkt, w~którym strategia lockdown/blokad w~dużej mierze dobiegła końca,

\noindent \textbf{26 września 2000 roku} Działania ,,S26'' przeciwko spotkaniom IMF/Banku Światowego w~Pradze, Czechy. To pierwsza duża i~dramatyczna akcja w~Europie po Seattle. Podobnie jak w~przypadku wielu działań europejskich, poziom bojowości był znacznie wyższy niż w~USA. W akcjach widać zaciekłe starcia między anarchistami Czarnego Bloku a policją, pierwsze pojawienie się świątecznego ,,Różowego Bloku'' i~pierwszy międzynarodowy debiut włoskiej taktyki ,,białych kombinezonów'' (,,Tute Bianche'', zorganizowanej przez włoskie Ya Basta!), rodzaj komicznej udawanej armii aktywistów w~hełmach, ochraniaczach, tarczach i~często nadmuchiwanych dętkach, którzy próbują szturmować linie policyjne uzbrojonych m.in. w~balony i~pistolety na wodę.

\noindent \textbf{20 stycznia 2001 roku} Protesty ,,J20'' podczas inauguracji Busha, drugi co do wielkości protest inauguracyjny w~historii Ameryki, choć nie przyciągnęły prawie żadnej uwagi mediów głównego nurtu. Większość członków NYC DAN w~końcu dołącza do innego Rewolucyjnego Bloku Antyautorytarnego. Czarnemu Blokowi udaje się przebić przez policyjne barykady i~tymczasowo zająć Pomnik Marynarki Wojennej, wywieszając czarną flagę i~blokując trasę parady oraz procesję Busha na jakiś czas, zanim w~końcu zostają wypchnięci przez tajne służby i~policję.

\noindent \textbf{25-30 stycznia 2001 roku} Pierwsze Światowe Forum Społeczne (WSF) odbywa się w~Porto Alegre w~Brazylii. Pierwotnie pomyślane jako radykalna alternatywa dla Światowego Forum Ekonomicznego (WEF) -- rodzaj spotkania towarzyskiego i~sesji networkingowej dla globalnych urzędników i~biurokratów, zwykle odbywającej się w~Davos w~Szwajcarii -- WSF szybko staje się intelektualnym centrum globalnego ruchu przeciwko neoliberalizmowi, z~tysiącami różnych organizacji i~osób biorących udział w~setkach sesji.

\noindent \textbf{20-22 kwietnia 2001 roku} Działania przeciwko ,,szczytowi Ameryk'', negocjacjom w~sprawie paktu o strefie wolnego handlu obu Ameryk (FTAA) w~mieście Quebec w~Kanadzie. To pierwsza akcja, w~której władze organizują swoją strategię wokół budowy dużego ogrodzenia (,,muru'' wokół części miasta, w~której ma się odbyć szczyt. Akcje, organizowane głównie przez montrealską Convergence des Luttes Anti-Capitalistes, czyli CLAC, mają na celu głównie ataki na sam mur, jako symbol sprzeczności neoliberalizmu.

\noindent \textbf{19-21 lipca 2001 roku }Kilkaset tysięcy protestujących gromadzi się w~Genui we Włoszech na spotkaniach G8 przywódców krajów uprzemysłowionych. Ponownie stosuje się strategię muru, a włoska policja, która tradycyjnie była stosunkowo tolerancyjna wobec ogólnej taktyki białych, tym razem przyjmuje strategię ekstremalnych represji, odmawiając wszelkich kontaktów z~przywódcami protestów i~stosując systematyczną strategię zachęcania faszystów i~prowokatorów do zapewnienia wymówki, by atakować, aresztować, a następnie systematycznie maltretować, a nawet torturować aktywistów. Genua jest postrzegana jako znak represji w~Europie i~powoduje, że europejskie grupy walczą o sformułowanie nowej strategii.

\noindent \textbf{11 września 2001 roku} Ataki na Pentagon i~World Trade Center. Anarchiści w~Nowym Jorku są jednymi z~pierwszych, którzy mobilizują się przeciwko nadchodzącej wojnie, z~marszami, których kulminacją jest marsz sześciu tysięcy ludzi na Times Square miesiąc po wydarzeniu. Zostały one prawie całkowicie ignorowane w~mediach głównego nurtu. Działania planowane na nadchodzące spotkania Banku Światowego/IMF w~Waszyngtonie są radykalnie ograniczone, ponieważ ruch jest zmuszony do ponownego rozważenia ogólnego strategicznego kierunku.

\noindent \textbf{3-4 lutego 2002 roku} Protesty Światowego Forum Ekonomicznego w~Nowym Jorku. Bezpośrednio po 9/11, WEF ogłasza, że  w~tym roku przeniesie się z~Davos (gdzie stało się obiektem częstych oblężeń aktywistów) do Waldorf Astoria w~Nowym Jorku ,,jako akt solidarności''. Anarchiści w~NYC DAN i~nowo utworzonej NYC Anti-Capitalist Convergence (ACC) są zmuszeni do podjęcia akcji w~ciągu kilku miesięcy, porzuceni przez prawie wszystkich ich zwykłych sojuszników z~organizacji pozarządowych i~Partii Pracy. Akcja jest prowadzona pomyślnie i~bez użycia przemocy, ale spotkała się z~masowym zastraszaniem przez policję i~setkami aresztowań. Stres 9/11 i~przymus stworzenia narodowej mobilizacji z~niczego w~tak krótkim czasie, powodują niekończące się napięcia na scenie nowojorskiej i~ostatecznie prowadzą do upadku i~ostatecznego rozwiązania DAN w~ciągu następnego roku. 

\noindent \textbf{10-14 września 2003 roku} Spotkanie ministerialne WTO w~Cancún w~Meksyku. Masowe akcje meksykańskich i~globalnych aktywistów -- w~tym dramatyczne samobójstwo rolnika z~Korei Południowej -- kończą się ostatecznym zablokowaniem procesu WTO.

\noindent \textbf{17-21 listopada 2003 roku} Negocjacje FTAA w~Miami, które spotkały się z~pierwszą prawdziwie zakrojoną na dużą skalę mobilizacją narodową w~USA od 9/11. Spotkania te to również pierwsze zastosowanie w~USA nowej polityki masowych ataków wyprzedzających i~ekstremalnej przemocy policji wobec protestujących -- podejście, które staje się znany jako ,,model Miami'' po tym, jak Homeland Security ogłasza go jako sposób na radzenie sobie z~takimi działaniami w~przyszłości. Z drugiej strony negocjacje w~sprawie wolnego handlu spełzły na niczym, wyznaczając ostateczny koniec procesu FTAA.

\bigskip

 Zakończę tutaj nie dlatego, że Miami reprezentuje koniec czegokolwiek (chociaż niektórzy twierdzą, że oznacza to koniec jednego cyklu przynajmniej ruchu północnoamerykańskiego), ale raczej dlatego, że oznacza koniec okresu opisanego w~tej książce. 11 września i~,,wojna z~terroryzmem'' z~pewnością stworzyły dramatycznie nowy klimat w~Stanach Zjednoczonych, ale jego skutki w~innych miejscach były mniej głębokie i~z pewnością mniej trwałe. W innych częściach świata represje nigdy nie były tak dotkliwe i~większości udało się uniknąć fali ksenofobii i~militarystycznego nacjonalizmu, które wyrządziły tyle szkód w~USA. Pod wieloma względami Ruch zaczął wchodzić na nową i~szerszą scenę, szczególnie w~Ameryce Łacińskiej, wraz z~falą okupacji fabryk i~lokalnych zgromadzeń w~Argentynie, lub niegdysiejszych organizatorów PGA, takich jak Evo Morales, którzy faktycznie doszli do władzy w~Boliwii, oraz wydarzeniami w~Atenco, Oaxaca i~innych częściach samego Meksyku. Nie chcę uogólniać ani przewidywać: w~chwilach prawdziwych zmian historia robi głupców ze wszystkich, którzy próbują. Jednak przynajmniej powtórzę to, co powiedziałem wcześniej (por. Graeber 2002; Graeber i~Grubacic 2004): że anarchizm jako filozofia polityczna oraz anarchistyczne idee i~imperatywy stają się coraz ważniejsze na całym świecie. Powszechnie zrozumiano, że epoka rewolucji bynajmniej się nie skończyła, ale rewolucja w~XXI wieku przybierze coraz bardziej nieznane formy. Przede wszystkim mam nadzieję, że ta książka będzie źródłem informacji dla tych, którzy chcą pomyśleć o poszerzeniu swojego wyczucia możliwości politycznych, dla każdego, kto jest ciekawy, jakie nowe kierunki mogą obrać radykalne myślenie i~działania.

\bigskip
\noindent PODZIĘKOWANIA

 Trudno jest napisać podziękowania dla książki takiej jak ta. Nikt nie chce nikogo specjalnie wyróżniać, bojąc się sugestii, że ktoś może być mniej godny. Jednak mogę zacząć od uhonorowania miłości i~wsparcia moich przyjaciół i~rodziny oraz moich zwolenników w~Yale podczas niefortunnych wydarzeń, które miały miejsce, do pewnego stopnia, w~wyniku tych badań, na których opiera się ta książka. Okres, w~którym prowadziłem badania, a następnie pisałem tę książkę, był niemal nieustannym stresem i~osobistą tragedią: naznaczony był przedłużającą się chorobą i~ostateczną śmiercią zarówno mojego brata, jak i~mojej matki, a wszystko to w~kontekście konieczności radzenia sobie z~niekończącymi się, dziwacznymi kampaniami tych starszych członków Wydziału w~Yale, którzy najwyraźniej postanowili wypędzić mnie wszelkimi niezbędnymi środkami. Nie będę wchodzić w~szczegóły, ale chciałbym przede wszystkim podziękować moim kolegom z~Yale, którzy zapewnili mi wsparcie i~poczucie wspólnoty, dzięki którym to miejsce stało się do zniesienia: Jennifer Bair, Bernard Bate, Richard Burger, Kamari Clarke, Hal Conklin, Michael Denning, Saroja Dorairajoo, Ilana Gershon, Paul Gilroy, Thomas Blum Hansen, Natalie Jerimijenko, Bun Lai, Enrique Mayer, Sam Messer, Marilda Menezes, John Middleton, Karen Phillips, Dhooleka Raj, Iman Saca, Lidia Santos, Jim Scott, Mary Smith, John Szwed, Thomas Tartaron, Frederic Vandenberge, Immanuel Wallerstein, David Watts i~Eric Worby, żeby wymienić tylko kilkoro. Przyjaciele i~koleżanki spoza Yale, którzy udzielili mi pomocy i~zachęty w~tym projekcie, są zbyt liczni, by ich wymienić. Chciałbym również być w~stanie podziękować imiennie wszystkim, tym którzy mnie poparli, gdy Wydział przegłosował zakończenie mojego kontraktu, ale byłoby to niemożliwe. Prawie pięć tysięcy osób podpisało petycje stworzone przez studentów Yale; kilka Wydziałów (Chicago, Sussex, Glasgow, Manchester) i~organizacje, od Global Studies Association po Kanadyjski Związek Pracowników Pocztowych, napisało zbiorowe listy do Wydziału, żądając wyjaśnień (oczywiście nie otrzymały żadnego), podobnie jak zażądał takich wyjaśnień niekończący się strumień indywidualnych uczonych. Przede wszystkim chcę podziękować studentom z~Yale i~znowu ta lista nie jest w~żadnym wypadku wyczerpująca -- i~mocno skoncentrowana na tych, których poznałem w~ciągu ostatnich kilku lat w~Yale -- ale oni zawsze byli moją największą inspiracją: Mahomet Ikraam Abdu-Noor, Ahmed Afzal, Colleen Asper, Ping-Ann Ado, Omolade Adunbi, Nikhil Anand, Caitlin Barrett, Kalanit Baumhauft, Ben Begleiter, Nina Bhatt, Rebecca Bohrman, Sheridan Booker, Devika Bordia, Lisa Allette Brooks, Elizabeth Busbee, Lucia Cantero, David Carston-Knowles, Durba Chattaraj, Linda Chhay, Kate Clancey, Robert Clark, Seth Curley, Anthony Dalton, Amelia Frank-Vitale, Antonios Finitsis, Thomas Frampton, Emily Friedrichs, Ajay Gandhi, Vladimir Gil, Josh Gordon, Jessica Gussberg, Annie Harper, Joseph Hill, Emily Hitch, Jennifer Jackson, Nazima Kadir, Kristin Kajdzik, Csilla Kalocsai, Brenda Kondo, Adrian LeCesne, Moon-Hee Lee, Kat Lo, Molly Margaretten, Andrew Mathews, Madeleine Meek, Christina Moon, Yancey Orr, Simon Moshenberg, Jason Nesbitt, Nana Okura, Juan Orrantia, Jonathan Padwe, Richard Payne, Anne Rademacher, Mieka Ritsema, Elliot Robson, Phoebe Rounds, Arian Schulze, Colin Smith, Olga Sooudi, Sarah Stillman, Will Tanzman, Jordan Treviño, Karen Warner, Kristina Weaver, i~Tiantian Zhang. 

 Wymienienie moich przyjaciół-aktywistów stwarza jeszcze dziwniejszy problem: bardzo trudno jest określić, do kogo właściwie mogę się odwołać po imieniu -- to znaczy tych, których legalne imiona faktycznie znam. Wymienię tylko kilkoro, głównie dlatego, że wiem, że nie mieliby nic przeciwko: Majeed Balavandi, Autumn Brown, Ayca Cubukcu, Crystal Dubois, Mike Duncan, Todd Eaton, Neala Byrne, Beka Economopolos, Stefan Christoff, Shawn Ewald, Heather Gautney, Andrej Grubacic, Harry Halpin, Eric Laursen, Bob Lederer, Brooke Lehman, Yvonne Liu, Daniel McGowan, Michael Menser, Dyan Neary, Ana Nogueira, Priya Reddy, Ramor Ryan, Mac Scott, Danielle Leah Sered, Ben Pasterz, Stephven Shukaitis, Marina Sitrin, John Tarleton, Lesley Julia Wood. Wszyscy w~nowojorskim DAN i~ACC; wszyscy w~IWW i~nowo założonej SDS; każdy, kto przeglądał szkice lub ich fragmenty, by wskazać na niezliczone punkty, w~których się pomyliłem; tak naprawdę każdy, kogo nazwisko pojawia się w~tym tekście, zasługuje na podziękowania i~wiele więcej. To są ludzie, którzy dali mi nowe poczucie nadziei dla planety, w~okresie, który inaczej byłby najgorszym okresem mojego życia. Mam dla nich tylko miłość.

 Oczywiście jest kilka osób, które muszę szczególnie wyróżnić: przede wszystkim Lauren Leve, Eric Graeber, Ruth Graeber, Andrej Grubacic, Nhu Le i~Stuart Rockefeller. Chciałbym podziękować mojemu redaktorowi Charlesowi Weiglowi i~wszystkim innym w~AK Press.

 Rozpocząłem ten projekt, mając niewiele więcej niż siebie i~optymizm. Dążyłem do końca z~rosnącym zrozumieniem, że bez względu na to, jak ponure i~jak niebezpieczne są niektóre miejsca, przez które trzeba przejść, życie jako buntownik -- ciągle świadomy możliwości rewolucyjnej transformacji i~wśród tych, którzy o tym marzą -- to z~pewnością najlepszy sposób, w~jaki można żyć.

\chapter*{Wstęp}

\begin{flushright}
\texttt{Zaczynasz od wściekłości, przechodzisz do głupich fantazji\ldots}
\end{flushright}
\bigskip
\bigskip


-- \textit{Więc } -- mówi Jaggi. -- Mam pomysł na to, w~jaki sposób Ya Basta! może przyczynić się do akcji w~Quebec City. Prasa kanadyjska wciąż przedstawia to jako rodzaj inwazji obcych. Tysiące amerykańskich anarchistów zamierza najechać Kanadę, aby zakłócić szczyt. Prasa Québécu robi to samo: tym razem to angielska inwazja. Więc moim pomysłem jest to, żebyśmy się tym zabawili. Odtworzymy bitwę o Quebec.

 Zdziwione spojrzenia Amerykanów przy stole.

-- To była bitwa w~1759, w~której Brytyjczycy w~pierwszej kolejności podbili miasto. Zaskoczyli francuski garnizon, wspinając się po tych klifach na zachód od Równin Abrahama, w~pobliżu starego fortu. Zatem oto mój pomysł. Możecie się przebrać w~stroje Ya Basta! i~wspiąć się dokładnie na ten sam klif, z~wyjątkiem, nie, chwila, posłuchajcie! Ta część jest ważna, na ochraniaczach i~kombinezonach chemicznych będziecie nosić koszulki hokejowe Québec Nordiques.

-- Chcesz, żebyśmy wspięli się na urwisko? -- zapytał Moose.

-- Aha.

-- A jak wysoki jest dokładnie ten klif?

-- Och, nie wiem, 60 metrów. Co to jest, jakieś 180 stóp?

-- Więc chcesz, żebyśmy wspięli się na sześćdziesięciometrowy klif w~rękawiczkach, hełmach, maskach przeciwgazowych i~osłonach z~gumy piankowej? -- Moose zachowywał się tak, jakby Jaggi rzeczywiście traktował sprawę poważnie.

-- Pomyśl o tym w~tak: kaski i~ochraniacze byłyby bardzo pomocne, jeśli w~ogóle spadniesz. Prawdopodobnie dlatego, że przypuszczalnie klify \textit{będą }bronione.

-- Och, świetnie. -- Moose -- Więc teraz wspinamy się na sześćdziesięciometrowy klif z~gliniarzami na szczycie.

-- Och, daj spokój, prawdopodobnie wszyscy zostaniecie natychmiast aresztowani tylko za \textit{noszenie }tych kombinezonów. Równie dobrze możesz najpierw coś najpierw zrobić. A symbolika byłaby idealna.

-- Nie chcę być tak pesymistą -- mówię. -- Wyobraźmy sobie, że niektórzy z~nas się przedostają. Wspinamy się po klifach. Nagle jesteśmy w~strefie bezpieczeństwa\ldots

-- Cóż, właściwie nie -- mówi Jaggi, spoglądając na mapę miasta. Mapa miasta jest narysowana pisakiem na dużej rozłożonej serwetce, na stole cukierni w~nowojorskiej Małej Italii, otoczona różnymi solniczkami i~cukiernicami używanymi do reprezentowania wyimaginowanych działaczy i~jednostek policji, wszystkie otoczone przez puste butelki po piwie i~resztki ciasta czekoladowego. Wokół stołu tłoczy się sześciu aktywistów, trzech Kanadyjczyków, trzech przedstawicieli nowojorskiej kolektywu Ya Basta! -- wszystko, co zostało z~tego, co zaczęło się jako znacznie większa grupa. -- Zakładamy, że ogrodzenie faktycznie będzie również biegło wokół krawędzi klifu.

 Jaggi krótko naradza się z~dwoma przyjaciółmi z~Quebecu, którzy przytakują. Jedna, Nicole, dodaje kolejną linię do mapy, aby to uwydatnić.

-- Masz na myśli, że przeskoczyliśmy przez klify i~nadal musimy przejść przez Mur? -- ktoś pyta.

-- Och, daj spokój -- mówi Jaggi. -- Jeśli możesz wspiąć się na sześćdziesięciometrowy klif, trzymetrowe ogrodzeniem z~siatki będzie problemem?

-- Dobrze, jesteśmy w~środku. -- Nalegam na mój scenariusz. -- Pięćdziesięciu aktywistów w~żółtych kombinezonach chemicznych, i~co to było, w~koszulkach hokejowych jakiejś drużyny Québec?, przedostało się przez mur. Jesteśmy wewnątrz strefy bezpieczeństwa. Odwróciliśmy inwazję brytyjską. Co teraz robimy? Zajmujemy cytadelę? Przedstawiamy petycję? 

-- Właściwie to byłoby naprawdę zabawne -- mówi jeden z~Yabbów. -- Walczymy w~górę klifów obok dwóch tysięcy gliniarzy z~prewencji, przechodzimy przez mur, a potem, kiedy tam dotrzemy, po prostu składamy petycję.

-- Do kogo?

-- Oczywiście do Busha.

-- Skąd wiemy, gdzie będzie Bush? -- pyta ktoś inny.

-- Zatrzyma się w~hotelu Concord -- mówi jeden z~anarchistów z~Quebecu. -- Łatwo będzie znaleźć; można go zobaczyć niemal z~każdego miejsca w~mieście. Teraz szczególnie łatwo -- uśmiecha się. -- Wystarczy poszukać budynku z~pociskami ziemia-powietrze na dachu.

-- Plus około dziesięciu tysięcy snajperów i~tajnych służb, prawdopodobnie z~niekończącym się zaawansowanym technologicznie sprzętem inwigilacyjnym\ldots

-- \ldots co z~kolei zostanie zakłócone przez naszą ogromną flotę zdalnie sterowanych modeli samolotów\ldots

 Rozmowa rzeczywiście poważnie degenerowała się przez co najmniej pół godziny.

 Zaczęło się dość poważnie, jako jedna z~tych trzygodzinnych maratonowych rozmów o wszystkim. Kanadyjczycy byli w~mieście w~ramach objazdowej trasy aktywistów, zorganizowanej przez CLAC, grupę anarchistyczną z~Montrealu, której francuski akronim oznaczał Convergence des Luttes Anti-Capitalistes, czyli ,,Konwergencja Walk Antykapitalistycznych''. Był początek marca 2001 roku. Jechali w~trasę, aby mobilizować się przeciwko Szczytowi Ameryk, który miał się odbyć 20 kwietnia w~Quebec City, w~którym miały uczestniczyć wszystkie głowy państwa na półkuli zachodniej (z wyjątkiem Kuby). Na wydarzeniu miało odbyć się podpisanie wstępnego projektu ustawy o strefie wolnego handlu obu Ameryk, będącej zasadniczo próbą rozszerzenia NAFTA na całą półkulę. Wysiłki te, wspierane przez Stany Zjednoczone, zostały w~rzeczywistości ostatecznie udaremnione, a ludzie w~tej cukierni, choć może się to wydawać nieprawdopodobne, odegrali znaczącą rolę w~udaremnieniu. Jednak to trochę inna historia, a mimo to skaczę do przodu. W każdym razie rozmowa zaczęła się w~meksykańskiej restauracji Tres Aztecas na Lower East Side, gdzie kilku aktywistów z~nowojorskiej sieci akcji Direct Action Network zabrało gości -- Jaggi z~Montrealu i~spokojniejszą francuskojęzyczną parę z~samego Quebecu -- na obiad. Właściwie dwoje z~nowojorskich członków DAN było Kanadyjczykami: para o imieniu Mac i~Lesley, pochodząca z~Toronto, obecnie mieszkająca w~Nowym Jorku. Ona była studentką socjologii na Columbii, on obecnie pracował jako malarz pokojowy i~wolontariusz dla National Lawyers Guild. Większość pozostałych była również częścią NYC Ya Basta! kolektywu. Była to nowo utworzona grupa zainspirowana inną o tej samej nazwie we Włoszech, której nazwa wzięła się od hasła (które znaczy ,,Już dość!'' rozsławionego przez zapatystowskich rebeliantów w~Chiapas, którzy z~kolei rozpoczęli swoje powstanie 1 stycznia 1994 roku, w~dniu, w~którym NAFTA po raz pierwszy weszła w~życie.

 W kręgach aktywistów tego roku Ya Basta! miała coś z~jakości Next Big Thing. Prawdopodobnie wynikało to przede wszystkim z~ich spektakularnie innowacyjnej taktyki: członkowie grupy słynęli z~okrywania się wszelkiego rodzaju wyszukanymi ochraniaczami, wykonanymi ze wszystkiego, od arkuszy gumy piankowej po gumowe urządzenia wypornościowe w~kształcie kaczki, łącząc je z~hełmami i~plastikowymi osłonami, więc wyglądało się jak jakiś futurystyczny grecki hoplita, a całość wieńczono maskami przeciwgazowymi i~białymi kombinezonami ochronnymi. Pomysł polegał na tym, przy takim ubiorze, że  stosunkowo niewiele gliniarzy może zrobić coś, aby naprawdę cię skrzywdzić. Oczywiście jesteś tak niezdarny, że prawdopodobnie nie możesz zrobić zbyt wiele, aby zranić kogoś innego; ale o to właśnie chodzi. Przedstawiciele twierdzą, że taktyka ta jest zakorzeniona w~nowej filozofii obywatelskiego nieposłuszeństwa. Tam, gdzie staromodne, masochistyczne podejście Gandhiego zachęca aktywistów do wykazania gotowości do pobicia przez policję na znak wyższości moralnej, ,,białe kombinezony'' proponowały etos ochrony: o ile odmawiasz krzywdzenia innych, całkowicie uzasadnione jest podjęcie wszelkich niezbędnych środków, aby uniknąć krzywdy dla siebie. Kostium też sprawia, że wygląda się dość śmiesznie, ale o to też chodzi. Spotkanie Ya Basta! często nim grały, na przykład atakując linie policyjne balonami lub pistoletami na wodę. Tym, co naprawdę zrobiło wrażenie na wielu aktywistach w~Ameryce, było to, że takie grupy miały prawdziwą bazę społeczną. Spotkanie Ya Basta! wyłoniły się z~niezwykle rozległej sieci włoskich skłotów i~okupowanych ośrodków społecznych ({\textquotedbl}białe kombinezony{\textquotedbl} zaczynały w~rezultacie jako armia skłotów). Mieli też swoich własnych intelektualistów: w~tamtych czasach prace myślicieli Autonomistów jak Toni Negri, Paolo Virno, Bifo Berardi zaczynały być dopiero tłumaczone i~upowszechniane przez Internet oraz zaczynały wpływać na aktywistów w~Ameryce.

 Nie powinienem przesadzać. Wiosną 2001 roku zdecydowana większość amerykańskich anarchistów prawie nic nie wiedziała o włoskiej teorii. Istniały jednak pewne bardzo entuzjastyczne wyjątki. W Nowym Jorku najważniejszym z~nich był mężczyzna o imieniu Moose. Wysoki, chudy młody mężczyzna, który prawie zawsze nosił czapkę rybaka, Moose był z~zawodu retuszerem zdjęć mody. Był również aktywny w~nowojorskim DAN. Zainspirowany tym, co przeczytał o ruchu we Włoszech po dramatycznym występie Ya Basta! na protestach IMF w~Pradze, Moose przeprowadził małe badania i~odkrył, gdzie można kupić tanie kombinezony chemiczne. Zamówił kilka pocztą i~zaczął je od czasu do czasu nosić na marszach. Pewnego dnia, podczas marszu przeciwko brutalności policji, student z~Włoch, który faktycznie pracował z~Ya Bastą! podszedł do niego i~zapytał, co się dzieje. i~tak, narodziła się Ya Basta! w~Nowym Jorku.

 Według jego koncepcji, było to jednocześnie przyjęcie włoskiej taktyki i~niektórych ogólnych zasad opracowanych przez włoski autonomiczny marksizm, który kładł nacisk na odmowę pracy, ,,exodus'' lub zaangażowane wycofanie się z~głównego nurtu i, krytykując, na swobodę przemieszczania się przez granice. We Włoszech ,,białe kombinezony'' podjęły serię dramatycznych działań przeciwko obozom dla imigrantów, aby podkreślić to, że wiele z~tego, co reklamowano jako ,,globalizacja'', w~praktyce oznaczało otwarcie granic dla przepływu pieniędzy, producentów i~niektórych form informacji, radykalnie zwiększając bariery i~kontrolę nad przemieszczaniem się ludzi. Ten pomysł trafił już na podatny grunt w~Ameryce Północnej, gdzie aktywiści lubili wskazywać, że amerykański patrol graniczny trzykrotnie zwiększył swoją liczebność od czasu podpisania NAFTA. Wielu z~nas już przekonywało, że cały sens ,,wolnego handlu'' polegał w~rzeczywistości na zamknięciu większości światowej populacji w~zubożałych globalnych gettach o silnie zmilitaryzowanych granicach, w~których można było usunąć istniejące zabezpieczenia społeczne, a wynikający z~tego terror i~desperację w~pełni wykorzystać przez globalny kapitał. Pytanie brzmiało, jak połączyć te dwie rzeczy -- idee i~taktyki.

 Dodatkowo ta perspektywa ekscytowała ludzi. Kolektyw NYC Ya Basta! szybko się rozwijał, podobnie jak podobne kolektywy (Wombles w~Anglii, Wombats w~Australii) rosły w~całym anglojęzycznym świecie. Większość pierwszej części rozmowy w~Tres Aztecas polegała na tym, że Moose mówił o Ya Basta!. Później, w~cukierni (przyjaciel Jaggiego uparł się, że znajdziemy taką, bo był trochę uzależniony od czekolady), dyskusja przeniosła się na potencjalne działania na granicy, stan anarchii w~Kanadzie, gubernatora dupka w~Ontario, gwiazdy ruchu i~dlaczego są irytujący, filozofia, antropologia, muzyka -- typowa niekończąca się rozmowa aktywistów o wszystkim. Jaggi wyjaśnił, że tak jak w~większości Kanady, anarchiści Québéc byli podzieleni w~dużej mierze na hardkorowych skłoterów i~studentów (,,jak ci dwaj, prawdopodobnie odejdą z~chwilą zakończenia pracy doktorskiej'' -- choć było też trochę staromodnych typów syndykalistycznych. Żadnych anarchistycznych związków zawodowych per se, ale działali w~ramach istniejących związków. Prawdziwy dramatyczny wzrost miał miejsce w~ruchu globalizacji, gdzie, jak w~wielu miejscach, wyłaniał się podział między organizacje pozarządowe i~duże grupy związkowe, które zdominowały dyskusje polityczne, oraz anarchistów, którzy szybko zaczęli dominować w~zakresie działania bezpośredniego. W Montrealu istniały w~zasadzie dwie grupy organizujące akcje: CLAC i~coś, co nazywało się Operacją SalAMI. Oczywiście CLAC nie było oficjalnie anarchistyczne. Oficjalnie było po prostu ,,antyautorytarne'' (no cóż, antyautorytarne, antykapitalistyczne, przeciwne wszystkim formom rasowej i~płciowej opresji, oddane akcji bezpośredniej, niechcące wchodzić w~kompromisy z~wewnętrznie niedemokratycznymi organizacjami, co w~praktyce znaczyło ,,anarchistyczne'').

-- A co z~SalAMI? Nie są anarchistami?

-- Och, jestem pewien, że są tam ludzie, którzy uważają się za takich czy innych anarchistów.

-- Więc czym są?

-- Och, wiesz. -- Mac wtrącił. -- Zwykłe typy antykorporacyjne. Nie antykapitaliści. Pierwotnie wyszli z~kampanii przeciwko Wielostronnemu Porozumieniu Inwestycyjnemu z~1998 roku. W tym czasie zorganizowali naprawdę dobrą akcję w~Ottawie. Ale\ldots cóż, to pacyfiści. Myślę, że byłby to najlepszy sposób na podsumowanie.

-- Czy widziałeś wytyczne, które pierwsi zaproponowali dla działań w~Quebec City? -- zapytał Jaggi. -- Całkowite niestosowanie przemocy. Częścią ich zasad postępowania był brak ,,przemocy słownej'', nikomu nie wolno używać wulgarnego języka. Nie, dosłownie, nie zmyślam tego. Malowanie w~sprayu sloganów to forma przemocy. Żadnego noszenia masek ani innych elementów garderoby zakrywających twarz\ldots

 Inni Kanadyjczycy się przyłączyli. 
 
 -- Co daje im prawo do mikrozarządzania wszystkim.

-- Są całkowicie maniakami kontroli. Szeryfowie, to wszystko.

-- Więc nie rozumiem -- powiedział jeden z~Amerykanów. -- Jakiego procesu używają ci goście?

-- Tak -- zapytał inny Amerykanin. -- Czy są demokratyczni, czy raczej mają formalną strukturę przywódczą? Czy przed akcją organizują rady delegatów?

-- Och, tak, tak, robią to wszystko. A przynajmniej robią to teraz. Kiedy zaczynali, było to całkowicie odgórne, charyzmatyczne przywództwo, rozkazy z~góry. Pozornie wszystko się teraz zmieniło. Ale wszystkie kluczowe decyzje, takie jak kodeks postępowania, są zawsze podejmowane w~wezwaniu do działania, zanim jeszcze pojawisz się w~radzie delegatów. Tak więc jest to w~zasadzie fikcja, ponieważ z~szeryfami kontrolującymi wszystko, jakakolwiek samoorganizacja staje się bezsensowna.

-- Dodatkowo -- powiedział Jaggi -- nadal mają coś w~rodzaju charyzmatycznego przywództwa. Co\ldots  cóż, w~porządku. Czy zauważyłeś, że pacyfiści \textit{zawsze }wydają się rozwijać charyzmatyczne przywództwo? Gandhi, King, Dalajlama. Wydaje się, że jest coś w~etosie pacyfistycznym, że po prostu ich wytwarza. Kiedy byłem na A16, widziałem tych idiotów niosących na sobie tabliczki z~wielkimi zdjęciami Gandhiego, a pod nimi był jakiś cytat z~niego mówiący: ,,ważne nie jest ja, ale moje przesłanie''. Musiałem więc do nich podejść i~zapytać: ,,Nie sądzisz, że jest tu trochę sprzeczności?'' 

 Dyskusja toczy się na temat zasług Gandhiego, w~przeciwieństwie do innych postaci indyjskiego ruchu niepodległościowego. Wydaje się, że konsensus był taki, że Gandhi był postacią wysoce ambiwalentną. Z jednej strony miał wiele bardzo anarchistycznych ideałów. Z drugiej był dziwnym, pokręconym seksualnie patriarchą, który współpracował z~daleką od rewolucji partią Kongresu i~otwarcie pielęgnował wokół siebie kult jednostki. Jeden z~Kanadyjczyków twierdził, że pacyfizm Gandhiego faktycznie opóźnił niepodległość o pokolenie. Jeden z~Amerykanów podkreślał, że  Gandhi powiedział również, że chociaż niestosowanie przemocy było ideałem, ci, którzy sprzeciwiają się brutalnemu uciskowi, są moralnie lepsi od tych, którzy w~ogóle się nie opierają -- sentyment, którego jego bardziej zadufani w~sobie zachodni akolici zawsze wydają się zapominać.

-- To, co mnie martwi w~całej koncepcji pacyfizmu -- powiedział Mac -- to, że jest on z~gruntu elitarny. Biedacy, ludzie, którzy na co dzień muszą żyć z~przemocą policji, którzy są do niej przyzwyczajeni, którzy jej oczekują\ldots  nie zobaczą niczego godnego podziwu, nie mówiąc już o heroizmie, w~\textit{zachęcaniu }do przemocy policyjnej, a następnie biernym stawianiu jej czoła.

 Takie opinie zawsze mnie trochę niepokoją, pochodząc od tego, kim są. Mac był jedną z~najbardziej sympatycznych, niefrasobliwych, dość skromnie głupich ludzi, jakich znałem. Często zastanawiam się, czy w~ogóle jest zdolny do gniewu. Jego żona była prawie taka sama.

-- Co myślisz? -- pytam Lesley.

-- Och, całkowicie się zgadzam. Po pierwsze, cała idea, że  zamierzasz ujawnić prawdziwą przymusową naturę tego stanu, pokazując, jak cię zaatakują, nawet jeśli nie stwarzasz im fizycznego zagrożenia, cóż, daj spokój. Mówisz biednym ludziom coś, czego jeszcze nie wiedzą?

-- Pracowałem z~OCAP, czyli Ontario Coalition Against Poverty (Koalicji Ontario przeciwko Nędzy), przez trzy lata w~Toronto -- powiedział Mac -- i~jedną rzecz, którą odkryliśmy, jest to, że jeśli, powiedzmy, pracujesz z~bezdomnymi lub naprawdę uciskanymi społecznościami, albo oni nic nie zrobią, albo będą chcieli bezpośrednio skonfrontować się z~ludźmi, którzy ich robili w~konia. W ten sposób dochodzi do tych ,,zamieszek'', takich jak te zeszłej wiosny w~Toronto. 

Lesley wyjaśnia, że  Mac odnosi się do marszu z~15 czerwca 2000 roku, zorganizowanego przez OCAP, w~którym ponad tysiąc bezdomnych, wraz z~aktywistami mieszkaniowymi, zostało zaatakowanych przez policję prewencji, gdy uparli się, że mają prawo do przemawiania do parlamentu, co skończyło zażarta bitwa, która trwała kilka godzin. -- Po trzeciej szarży kawalerii na pokojowych demonstrantów wszyscy po prostu eksplodowali. Zaczęli rzucać wszystko w~zasięgu wzroku, wyrywając chodnik, znaki drogowe, rzucając śmietnikami.

-- Poczekaj chwilę -- protestuję. -- Sam Gandhi pracował z~wieloma biednymi ludźmi

-- Prawda -- wtrącił się Jaggi -- ale to mieści się w~bardzo specyficznej tradycji religijnej. Jeżeli jesteś Hindusem, umiejętność wytrwania w~niskiej pozycji w~obrębie hierarchii kastowej jest oznaką cnoty, o to w~tym chodzi.

I~tak dalej. Cała rozmowa wydała mi się trochę oklepana i~jednostronna. Wskazałem, że od Seattle, związki panikowały z~powodu możliwości ,,przemocy'', czy nawet niszczenia własności. Inni odpowiedzieli, że mówiłem o biurokratach związkowych, a nie szeregowych członkach związku. A co z~grupami biednych ludzi, które krytykują wojownicze taktyki jako produkt białych przywilejów klasy średniej, w~sensie że naprawdę uciskanym nigdy nie pozwolono by na takie działania? Ktoś zmienia temat. 

-- A czy zauważyłeś, jak typy SalAMI zawsze uważnie śledzą, którzy politycy, celebryci lub bogaci ludzie ich aprobują. Całe nastawienie jest całkowicie elitarystyczne.

 W każdym razie SalAMI wypuściło swoje pacyfistyczne wezwanie do działania, a następnie CLAC wypuściło własne, wzywając do ,,różnorodności taktyk''. Chodziło im o to, że należy zrobić miejsce dla sztuki i~marionetek, czy dla tradycyjnego Gandhijskiego nieposłuszeństwa obywatelskiego ,,chodź i~aresztuj mnie'', a także należy zrobić miejsce dla bardziej bojowych taktyk. Najważniejsze jest to, aby w~końcu wszyscy solidaryzowali się ze sobą. Jak się okazało, bardzo niewiele osób zarejestrowało się w~radzie delegatów SalAMI, więc odwołali ją i~teraz skoncentrowali się na robieniu czegoś w~Montrealu. Z drugiej strony rada delegatów CLAC poszła na tyle dobrze, że doprowadziła do powstania nowej grupy lokalnej -- CASA, Komitetu Powitalnego Szczytu Ameryk. CASA teraz gorączkowo organizowała lokalne sprawy. Grupki chodziły od drzwi do drzwi w~dzielnicach robotniczych w~pobliżu starej fortecy. Była to wyjątkowa możliwość, ponieważ kanadyjska policja ogłosiła niedawno, że przed szczytem stare miasto i~teren wokół Centrum Kongresowego, w~którym miały się odbyć spotkania, zostaną otoczone czterokilometrowym ogrodzeniem ochronnym. Tylko ci z~dowodami osobistymi potwierdzającymi, że mieszkali wewnątrz, będą mogli wejść do środka. Jednocześnie policja wciąż wypowiadała sprzeczne stwierdzenia, gdzie dokładnie będzie przebiegać ogrodzenie, ale z~pewnością przecięłoby wiele dzielnic na pół. Dzieci musiałyby przechodzić przez silnie zmilitaryzowane posterunki policyjne, aby wrócić do domu ze szkoły. Miejscowi już nazywali to ,,murem'' 

 Trzeba mieć na uwadze, zauważył Jaggi, że jest to ludność, która ze względu na swoją historię już i~tak jest wyjątkowo podejrzliwa wobec rządu centralnego. Nawet nacjonalizm quebecki jest bardzo dziwnym, proletariackim rodzajem nacjonalizmu: francuskojęzyczni widzą siebie jako białą klasę robotniczą wschodniej Kanady, co do pewnego stopnia jest prawdą. To właśnie w~tym momencie -- mniej więcej w~czasie, gdy Mac i~Lesley musieli wyjść -- weszliśmy na politykę muru; o obiecywanej militaryzacji granicy kanadyjskiej (np. podczas rozmów handlowych w~Windsor rok wcześniej dwie trzecie próbujących przekroczyć granicę Amerykanów zostało zawróconych, a sporo aresztowano). Pytanie brzmiało, jak zaplanować akcję graniczną, która zwróciłaby uwagę na hipokryzję militaryzacji granicy i~budowania murów wewnątrz miasta, aby móc chronić przywódców politycznych przed wszelkim niebezpieczeństwem kontaktu z~ich wyborcami -- nie wspominając o retoryce ,,wolnego handlu'' burzącego mury i~jednoczącego planetę, tylko żeby móc je nawet podpisać, trzeba zrobić dokładnie odwrotnie. Reszta z~nas zaczęła rzucać pomysłami. Możliwe działania graniczne. W końcu doprowadziło do pytań o scenariusz, a następnie do klifów Quebec City. Właściwie to było pod koniec rozmowy -- w~tym momencie wszyscy byliśmy już trochę zmęczeni, a niedługo po tym, jak się rozstaliśmy, poszliśmy do domu i~poszliśmy spać.

\bigskip
\noindent O TEJ KSIĄŻCE

 Zacząłem od rozmowy w~cukierni z~kilku powodów. Po pierwsze, jest zabawna. Pomyślałem, że może też przekazać coś z~poczucia ruchu, który, jak zobaczymy, jest szczególnie podatny na formy działania, które są jednocześnie głęboko wyjątkowo głupie i~całkowicie poważne. Taka rozmowa, szczególnie w~zestawieniu z~poważnymi argumentami na temat Gandhiego i~tak dalej, wydawała się najlepszym sposobem, aby dać czytelnikowi natychmiastowe wyobrażenie o tym, co faktycznie jest zaangażowany w~takim ruchu. To także sprawi, że książka jest lepsza.

 Taka rozmowa od razu podnosi również kwestię, z~którą będę się zmagać przez całą książkę: co zrobić z~tożsamościami osób, kiedy omawiamy drażliwe politycznie i~prawnie rozmowy? Na przykład New York Ya Basta! prawie na pewno nadal jest wymieniany w~niektórych policyjnych systemach wywiadowczych jako organizacja terrorystyczna. Na kilka tygodni przed szczytem zarówno amerykańska, jak i~kanadyjska policja zidentyfikowała ją jako jedną z~zasadniczo potencjalnym ,,elementem przemocy'' oraz każdy podejrzany o udział w~Ya Basta! był zatrzymany podczas próby przekroczenia granicy, przetrzymywany przez wiele dni i~intensywnie przesłuchiwany. Wszystko to było śmieszne. Ya Basta!, jak wspomniałem, opierała się na zasadzie tzw. ,,radykalnej'' obrony. Członkowie uzbroili się przeciwko pałkom i~gumowym kulom, ale usprawiedliwiali to właśnie dlatego, że nie chcieli zrobić niczego, co mogłoby skrzywdzić kogokolwiek innego. Jednak w~tym kontekście fakt, że te stwierdzenia są śmieszne, jest w~dużej mierze nieistotny. Reclaim the Streets Nowy Jork, grupa specjalizująca się w~niedozwolonych imprezach ulicznych, została również sklasyfikowana przez niektóre siły policji jako grupa terrorystyczna\footnote{Jednocześnie inne, równie publiczne, grupy, które w~rzeczywistości angażują się w~bardziej bojową działalność, nigdy nie pojawiają się na tych samych listach. Arbitralność sprawia, że  to jest bardzo skuteczną strategią, ponieważ nikt nie czuje się bezpiecznie. Jednak nie ma prawdziwego sposobu, aby dowiedzieć się, czy mamy do czynienia z~zamierzoną strategią, czy z~prostymi skutkami biurokratycznej głupoty.}. Te rzeczy nigdy nie mają sensu. Jedna rzecz, której można się szybko nauczyć jako aktywistka, to to, że ręka represji jest niezwykle przypadkowa. W rezultacie rozmowa, którą rozpocząłem, choć oczywiście żartobliwa, mogłaby zostać zaklasyfikowana jako spisek terrorystyczny.

 Wyobraźmy sobie przez chwilę, że w~cukierni był ukryty mikrofon. Wyobraźmy sobie policjanta lub agenta FBI monitorującego powyższą rozmowę. Nie jest to poza sferą możliwości: być może spodziewali się, że spotkają się tam jacyś mafiosi i~planują prawdziwą zbrodnię. Następnie wyobraź sobie -- całkiem prawdopodobne -- że policjant przysłuchujący się tej rozmowie nie ma absolutnie żadnego poczucia humoru. Co by sobie pomyślał? Oto członkowie organizacji prawdopodobnie terrorystycznej, spotykają się z~Kanadyjczykiem o nazwisku Jaggi Singh i~rozmawiają o wzięciu udziału w~jakimś brutalnym konflikcie z~udziałem prezydenta Busha. Jeśli funkcjonariusz, o którym mowa, przepuściłby nazwiska przez kanadyjską policję, natychmiast otrzymałby informację, że Jaggi Singh jest znanym anarchistą, który był kilkukrotnie aresztowany w~związku z~nielegalnymi protestami.

 Ten ostatni punkt jest technicznie prawdziwy, ale po raz kolejny absurdalny, jeśli znasz odrobinę kontekstu. W Kanadzie Jaggi jest kimś w~rodzaju postaci publicznej. Regularnie pojawia się w~telewizji jako rzecznik CLAC lub innej radykalnej organizacji. W rezultacie ciągle jest aresztowany. Stało się to czymś w~rodzaju powtarzającego się żartu w~radykalnych kręgach w~Kanadzie. Przed każdą wielką akcją lub mobilizacją policja prawie zawsze wejdzie i~aresztuje Jaggi Singh; częściowo, jak się wydaje, tylko dlatego, że jest jedynym wybitnym anarchistą, o którym słyszeli.

-- Znowu przybywają antyamerykańscy protestujący. Wszystko na swoim miejscu?

-- Gaz łzawiący ? 

-- Tak.

-- Tarcze i~pałki? 

-- Tak.

-- Bariery bezpieczeństwa?

-- Są.

-- Aresztowano Jaggi Singha? 

-- Tak\footnote{\url{http://www.snappingturtle.net/flit/archives/2003_07_29.html}\textit{, dostęp 9 czerwca 2004 roku}}.

 Przykłady można by mnożyć. To zawsze jest areszt prewencyjny; Jaggi nigdy nie został o nic oskarżony, nie mówiąc już o skazaniu, przynajmniej po części dlatego, że nigdy nie zrobił nic nielegalnego. Jaggi jest przede wszystkim radykalnym dziennikarzem. W związku z~tym stał się regularnym rzecznikiem publicznym grup rewolucyjnych. Ale cały sens używania tej samej osoby jako swojego rzecznika polega na tym, że w~ten sposób twarze tych, którzy faktycznie planują działania, nigdy nie są widoczne. Pomysł, że Jaggi, który w~rzeczywistości jest na trasie z~przemówieniami publicznymi, występując pod własnym nazwiskiem, przybędzie na spotkanie poświęcone planowaniu działań, jest absurdalny. Jednak znowu, fakt, że jest absurdalny, nie jest istotny. Gdyby policja zdecydowała się oskarżyć nas wszystkich o spisek w~celu popełnienia aktu terroryzmu, zgodnie z~prawem, byłoby to całkiem możliwe. Policja miałaby poważne problemy z~doprowadzeniem do skazania, ale przez kilka lat całkiem łatwo mogłaby sprawiać nam trudności.

 Wszystko to może sprawić, że sam pomysł napisania takiej etnografii będzie raczej wątpliwą propozycją. Trzeba jednak pogodzić prawne możliwości z~faktem, że nic takiego się nigdy nie wydarzyło. Nie sądzę, by w~ciągu ostatnich czterech lat zdarzyło się kiedykolwiek aresztować aktywistę z~powodu czegoś, co powiedzieli lub mieli powiedzieć na spotkaniu, nie mówiąc już o nieformalnej rozmowie. Działacze są regularnie aresztowani za bycie rzecznikami publicznymi, tak jak Jaggi. Aktywiści bywali zatrzymywani na granicach za przynależność do rzekomo brutalnych organizacji -- jak na przykład wielu członków nowojorskiej Ya Basta! w~końcu miało zostać. Setki aktywistów -- a często zwykłych obywateli, którzy akurat stoją obok nich -- zostało zatrzymanych w~masowych aresztowaniach podczas protestów. Kiedy tak się stanie, kilka z~nich zostanie prawie zawsze losowo wybranych pod zarzutem przestępstwa: ,,napaść na oficera'' lub tym podobne. Zarzuty te prawie nigdy się nie utrzymują, ponieważ prawie zawsze są całkowicie zmyślone; jednak udaje im się związać aktywistów niekończącymi się terminami sądowymi i~opłatami prawnymi. Z pewnością miały miejsce dziwaczne i~oburzające akty represji wobec jednostek. Aktywiści trafiali do więzienia za linki, które umieszczali na stronach internetowych lub za posiadanie urządzeń służących do wykrywania genetycznie zmodyfikowanej żywności. Nikt nie został oskarżony o to, co powiedział na spotkaniu. Niemniej jednak strach, że mogą, od lat wywiera dławiący wpływ na życie aktywistów, a strach ten narastał tylko wraz z~rosnącymi represjami państwowymi. Same spotkania stawały się coraz bardziej tajemnicze. Uczestniczący w~nich stawali się bardziej paranoiczni. Myślę, że rezultaty były katastrofalne.

 Moim zdaniem są one szczególnie katastrofalne, ponieważ to, co dzieje się na spotkaniach, struktura podejmowania decyzji, ma kluczowe znaczenie dla ruchu. Być może bardziej niż cokolwiek innego jest to bowiem ruch na rzecz tworzenia nowych form demokracji. Jednym z~powodów, dla których media były w~stanie w~dużej mierze spisać na straty tak zwany ruch ,,antyglobalistyczny'' jako niespójny bełkot bez żadnego głównego tematu lub głównej ideologii, to właśnie to, że jego ideologia jest osadzona w~praktyce. W świadomej kontradykcji do dawnych grup rewolucyjnych, to jest nie wymyślając jakiejś abstrakcyjnej linii partyjnej faworyzującej ,,demokrację'', a następnie nie zamieniając się w~dobrze naoliwioną autorytarną machinę oddaną przejmowaniu władzy tam, gdzie to możliwe, aby w~końcu móc ją wprowadzić, grupy takie jak DAN czy CLAC są zdeterminowane, aby żyć według swoich zasad. W dużym uogólnieniu (jak argumentowałem wcześniej: Graeber 2002), demokratyczna praktyka, którą te grupy rozwinęły \textit{jest} ich ideologią.

 Moim zdaniem jest to niezwykle zdrowe i~niezwykle odświeżające podejście. To w~dużej mierze powód, dla którego zaangażowałem się w~takie grupy. Z drugiej strony stwarza realne dylematy przy reprezentacji. Mamy ruch, który postrzega siebie jako tworzący nowe formy demokracji, ale z~powodu obaw o bezpieczeństwo, jego rzeczywisty proces demokratyczny nie może być przedstawiony nikomu poza ruchem prócz najbardziej abstrakcyjnych terminów. Wszyscy są tak zaniepokojeni niebezpieczeństwami represji prawnych, że nigdy nie można mówić o konkretnych szczegółach tego, co wydarzyło się na jakimkolwiek konkretnym spotkaniu. Jest to szczególnie ironiczne, ponieważ jest to ruch, który poza tym jest niezwykle wyrafinowany w~przedstawianiu siebie. Składa się z~wielu radykalnych filmowców, dziennikarzy internetowych, aktywistów radiowych; obejmuje ogromną sieć niezależnych mediów, która jako pierwsza wyłoniła się z~Seattle i~nadal, podczas każdej większej konwergencji, dostarcza szczegółowych relacji minuta po minucie z~akcji. Potem szybko i~niezmiennie pojawia się film dokumentalny. Jednak żadne z~tych przedstawień zwykle nie zawiera pojedynczego opisu konkretnego aktu zbiorowego podejmowania decyzji. Na przykład każda poważna akcja jest zwykle poprzedzona serią rad, zgromadzeń, w~których setki, a nawet tysiące ludzi zbierają się, aby wspólnie planować działania, bez żadnej formalnej struktury przywódczej. Jednak żadna nigdy nie została sfilmowana. i~to pomimo faktu, że w~pewnym momencie przynajmniej w~połowie większych rad delegatów, w~których brałem udział, jakiś radykalny filmowiec prosił o pozwolenie na sfilmowanie jakiejś części obrad. Byli niezmiennie odrzucani. W zasadzie, rady delegatów są wydarzeniami otwartymi: każdy, kto nie pracuje dla jakiejś agencji informacyjnej lub organów ścigania, może wejść do środka, a uczestnikom często przypomina się, aby nie dyskutowali o niczym, o czym nie chcieliby, aby policja wiedziała. Mimo to, gdy proszą o film, ktoś zawsze blokuje. W rezultacie, o ile mi wiadomo, takie zdarzenie nigdy nie zostało zarejestrowane. Tak więc kończy się to filmami dokumentalnymi, które pokazują aktywistów maszerujących ulicą, skandujących ,,tak wygląda demokracja'', ale nie zawierają zdjęć nikogo, kto faktycznie praktykuje demokrację.

 Wynik jest osobliwym rozłączeniem. Kiedy aktywiści rozmawiają ze sobą, mają tendencję do niekończących się rozmów o ,,procesie'' -- o abecadle demokracji bezpośredniej. Przygotowując się do wielkiej akcji, wydaje się, że jedyne co się robi, to spotkania, szkolenia, więcej spotkań. Ale kiedy czyta się później relacje z~tej samej akcji, prawie wszystko to znika.

 Przede wszystkim więc ta książka ma wypełnić lukę. Zacznę od wykorzystania własnego doświadczenia, aby przekazać poczucie, jak to jest brać udział w~planowaniu i~ostatecznie uczestniczyć w~dużej akcji przeciwko globalnemu szczytowi. Aby zilustrować typy spraw, o które faktycznie spierają się aktywiści, jakie problemy lub wydarzenia stają się zbiorowymi dramatami; aby zorientować się, jak to jest przebrnąć przez maraton, dwudniowe spotkanie i~wyjść z~tego z~uczuciem, jakby właśnie przebrnęło się przez dwudniowy maraton spotkań, ale jednocześnie doświadczyło się czegoś głęboko przemieniającego. Jak czytelnik mógł zauważyć, nie udaję tutaj obiektywności. Nie angażowałem się w~ten ruch po to, żeby napisać etnografię. Zaangażowałem się jako uczestnik. Pochodzę ze starej lewicowej rodziny i~przez większość mojego życia uważałem się za anarchistę. Jeśli przez większość mojego życia rzadko angażowałem się w~politykę anarchistyczną, to głównie dlatego, że w~latach 80. i~większości lat 90. polityka anarchistyczna, na którą byłem narażony, wydała mi się małostkowa, zatomizowana i~bezsensownie kontrowersyjna -- pełna niedoszłych sekciarzy, których sekty składały się tylko z~nich samych. Nagłe odkrycie istnienia ruchu o radykalnie odmiennej wrażliwości, który kładł ogromny nacisk na wzajemny szacunek, współpracę i~egalitarne podejmowanie decyzji, było głęboko radosne. To było tak, jakby ruch, w~którym zawsze chciałem być częścią, nagle powstał. Nawet jeśli jestem krytyczny wobec ruchu, to jestem krytyczny jako osoba wtajemniczona, osoba, której ostatecznym celem jest wspieranie jego celów. Moja ostateczna decyzja o napisaniu etnografii zrodziła się z~tego samego impulsu. Oczywiście do pewnego stopnia jako wyszkolony etnograf nie możesz nie przemóc. Niemal jak tylko się zaangażowałem, zauważyłem, że notatki, które robiłem na spotkaniach, stawały się coraz bardziej szczegółowe. Zaczęły zawierać małe obserwacje na temat fryzur i~stylów butów, postawy, nawyków, refleksje w~nawiasach na temat małych aktywistycznych rytuałów. Jednak moja decyzja, aby opisać to wszystko w~formie etnograficznej, pojawiła się w~dużej mierze dlatego, że jako uczestnik uderzyła mnie ona jako ważny sposób realizacji jednego z~celów ruchu: rozpowszechniania pewnej wizji demokratycznych możliwości. Podczas mojego szkolenia antropologicznego zdobyłem umiejętność, która wydawała się doskonale pasować do przekazywania tego, czego brakowało w~istniejących relacjach o ruchu. Choć również uświadomiłem sobie, że stworzyłoby to niesamowicie interesujące badanie etnograficzne.

 Ale potem pojawił się problem, jak to zrobić, nie narażając nikogo.

 W końcu rozwiązanie, które wymyśliłem, było takie. W kwestiach naprawdę drażliwych (w przeciwieństwie do głupich fantazji) nie cytuję niczego, co nie zostało już powiedziane na jakimś publicznym forum. Przytaczam rzeczy, które pojawiły się na serwerach aktywistów listserv\footnote{system obsługi list dyskusyjnych, pierwotnie zaprojektowany w~1986 roku dla systemu operacyjnego IBM, opatentowany w~1995 roku.}, o których wszyscy wiedzą, że są monitorowane, lub w~radach delegatów lub spotkaniach otwartych dla publiczności, o których należy przypuszczać, że są prawdopodobnie infiltrowane\footnote{Niektóre rady delegatów ds. akcji są zdecydowanie nieotwarte i~unikałem nawet wspominania o nich: w~Quebecu nie było żadnej, o której bym wiedział. Grupy takie jak Direct Action Network w~Nowym Jorku miały otwarte spotkania, choć najczęściej wszyscy na danym spotkaniu znali się nawzajem. Grupy takie jak Ya Basta! nie były właściwie zamknięte, ale bardziej intymne, dlatego starałem się unikać opisywania mniejszych spotkań DAN lub jakichkolwiek spotkań Ya Basta!, w~każdym przypadku, w~którym omawiane były scenariusze działań.} O innych forach byłbym bardziej niejasny. Kiedy mam do czynienia z~rzeczami wypowiadanymi na forach publicznych, które mają jakikolwiek wpływ na działania, unikam używania prawdziwych nazwisk. Nie jest to trudne, ponieważ w~większości nie znam prawdziwych imion ludzi. A przynajmniej nie znam pełnych nazwisk. Wielu aktywistów używa ,,nazwisk w~akcji'', których używają nawet z~najbliższymi przyjaciółmi. W kręgach aktywistów można bardzo blisko współpracować z~kimś przez lata, zostać bliskimi przyjaciółmi, a może nawet kochankami, i~nigdy tak naprawdę nie poznać ich pełnego imienia i~nazwiska. Kiedy znam czyjeś pełne imię i~nazwisko, to prawie zawsze, ponieważ jest to, podobnie jak Jaggi, takie czy inne osoby publiczne, których tożsamość nie musi być chroniona. Wreszcie, czy opisuję spotkania, czy akcje, trzymam się wydarzeń, w~których sam w~pełni uczestniczyłem; oznacza, to, że nie proszę nikogo, by przyjmował, anonimowo, ryzyka, którego nie jestem gotów ponieść w~swoim własnym imieniu.

 Nie musiałem oczywiście zaczynać od opowiadania historii mobilizacji wokół Szczytu Ameryk w~Quebec City. Mogłem wybrać kilka innych. Po części zacząłem od Quebecu właśnie z~powodu tego rodzaju rozważań. Nie tylko dlatego, że wszystkie opisane w~relacji przestępstwa zostały popełnione w~Kanadzie, ale także dlatego, że było to bardzo bojówkowe wydarzenie -- w~rzeczywistości najbardziej bojowe, w~jakim kiedykolwiek byłem zaangażowany -- w~którym, jak się okazuje, najpoważniejszym aktem spisku, o który mógłbym zostać oskarżony, jest spisek mający na celu zburzenie ogrodzenia z~siatki i~porzucenie. Historia Quebec City ma inne oczywiste zalety. Po pierwsze uważam, że to całkiem dobra historia. Jest również przydatna, ponieważ chciałem uniknąć zarówno pokusy idealizowania ruchu, lub (równie irytującego) nawyku wielu aktywistów do mówienia tylko o swoich problemach, co często powoduje, że osoby z~zewnątrz zastanawiają się, dlaczego ktokolwiek miałby na początku angażować się w~taki ruch. Historia Quebecu wydawała się pod tym względem idealna, ponieważ łączy w~sobie to, co najlepsze i~najgorsze. Pozwoliła mi porozmawiać zarówno o grupach, których proces demokratyczny działał wyjątkowo dobrze, jak i~o innych, w~których był on naprawdę okropny; zarówno o grupach, które przetrwały, i~grupach, które się rozpadły; zarówno akcjach, które były niezwykle udane, jak i~innych, które były kompletnymi katastrofami.

\bigskip
\noindent STRUKTURA KSIĄŻKI

 Część i~będzie zatem w~dużej mierze dotyczyła Quebecu. Rozdział 1 będzie zawierał swego rodzaju pamiętnikowe sprawozdanie z~miesiąca następującego bezpośrednio po wizycie CLAC; Rozdział 2 bardziej szczegółowego opisu ,,consulta'' w~Quebec City na około miesiąc przed akcjami; Rozdział 3 opisuje wydarzenia prowadzące do nieudanej akcji na Seaway International Bridge w~Akwesasne; Rozdział 4 opisuje same działania w~Quebecu. Część ta przyjmie formę narracji pierwszoosobowej, ze sporą ilością zrekonstruowanych dialogów tego rodzaju, od którego zacząłem. Znajdą się w~nim również dość obszerne fragmenty moich notatek terenowych, składające się głównie ze szczegółowych rekonstrukcji tego, co każda osoba faktycznie powiedziała na ważnych spotkaniach aktywistów, ale z~okazjonalnymi komentarzami lub refleksjami.

 Część II będzie składać się z~analizy. Rozpoczyna się (rozdział 7) komentarzami na temat społecznej treści ruchu, co do których, jak sądzę, jest wiele nieporozumień. Po tym nastąpi długi rozdział (rozdział 8) o spotkaniach i~eksperymentach w~tworzeniu nowych form demokratycznych; kolejny zarys typologii działań (rozdział 9) i~wreszcie omówienie polityki reprezentacji: media, lalki itd. (rozdział 10). Zakończę wnioskiem teoretycznym (,,Wyobraźnia'', rozdział 10), składającym się z~jednego rozdziału o przemocy i~wyobraźni.

 Napisanie tej książki -- zwłaszcza pierwszej części -- postawiło mnie przed prawdziwymi dylematami reprezentacji. Najpierw próbowałem napisać Część i~prawie całkowicie w~formie pamiętnika, co, jak sądziłem, oddałoby jakiś sens podzielonej i~epizodycznej jakości życia aktywisty. Utrzymanie takiego podejścia okazało się jednak niemożliwe. Po pierwsze, szybko okazało się, że jeśli oddałbym prawdziwą sprawiedliwość bogactwu wydarzeń, wydałbym książkę, której wydanie nawet nie rozważyłaby żadna prasa. To było zdecydowanie za długo. Kondensacja przyniosła jednak niekończący się kompromis. Im więcej trzeba było oszczędzać, tym silniejsza była chęć nadania całości jakiejś ogólnej formy narracyjnej. Z drugiej strony imperatywy narracyjne w~pewnym stopniu kłócą się z~logiką tego, co próbowałem opisać. Oczywiste jest, że dobre narracje nie mają setek bohaterów. Jednak zastosowanie standardowych technik narracyjnych i~przedstawienie niektórych jednostek jako typów dla innych oznaczałoby zastosowanie dokładnie tej logiki reprezentacji, której aktywistyczne struktury decyzyjne, które starałem się opisać, starają się najsilniej uniknąć. Co więcej, umieszczenie zbyt wielu ram narracyjnych wydarzeń przesłoniłoby z~konieczności rzeczywiste doświadczenie akcji bezpośrednich, gdzie spędza się miesiące na przygotowywaniu wydarzenia, o którym ma się nadzieję, że można je opowiedzieć w~określony sposób, przechodzi przez krótką lawinę akcji, w~których ma się bardzo niewielkie pojęcie o tym, co się dzieje, a potem ostatecznie spędza tygodnie, próbując ustalić, co się stało, i~kłócąc się o to, jak w~rzeczywistości należy opowiedzieć tę historię. Mam nadzieję, że wypracowałem rozsądny kompromis, historię, która jest jednocześnie czytelna, nadająca się do publikacji, oraz przynajmniej jakoś uczciwa wobec postaci.

 Mam również nadzieję, że wyniki będą zgodne z~najlepszą tradycją etnografii -- próbą opisania i~uchwycenia czegoś z~faktury, bogactwa i~ukrytego sensu sposobu bycia i~działania, których inaczej nie można by uchwycić na piśmie. Mam nadzieję, że czyniąc to, będę mógł dać czytelnikowi przebłysk jednej małej, północnoamerykańskiej części znacznie większego, rozwijającego się globalnego ruchu społecznego, o którego istnieniu wielu nawet nie zdaje sobie sprawy. 

\chapter{Dziennik Nowy Jork: Marzec 2001}

 Kiedy przyjechała karawana CLAC, większość z~nas w~Nowym Jorku toczyła długą debatę na temat tego, czy w~ogóle powinniśmy próbować dostać się do Quebecu. W tym czasie NYC Direct Action Network koncentrowała swoje wysiłki na pomocy w~zorganizowaniu masowej ,,konwergencji'' aktywistów w~Burlington, żeby odbyła się kilka dni przed akcją. Tam wszyscy zwołaliby radę delegatów, aby zdecydować, co dalej. Spotkanie Ya Basta! w~większości była pozostawiona sobie, aby wymyślić scenariusze działań bezpośrednich. Problem był taki, że z~bardzo małym prawdopodobieństwem udałoby się przekroczyć granicę kilkunastu znanym aktywistom wyposażonym w~maski gazowe, hełmy, ochraniacze i~kombinezony chemiczne. Oznaczało to, że albo musieliśmy zrezygnować ze sprzętu, albo wysłać go do Kanady z~dużym wyprzedzeniem -- żadna z~tych alternatyw z~różnych powodów nie była szczególnie rozsądna. W obliczu podobnego dylematu podczas protestów Światowego Forum Ekonomicznego w~Genewie włoska Ya Basta! przeprowadziła swoje akcje na samej granicy.

 Dla wielu z~nas miało to sens. Przez cały czas koncentrowaliśmy się na kwestiach imigracyjnych. Pojawiliśmy się już w~naszych kolorowych kostiumach na protestach w~dwóch różnych ośrodkach dla imigrantów w~Nowym Jorku. Obszar Nowego Jorku był szczególnie pełen takich obiektów. Nawet w~tych dniach przed 11 września w~Nowym Jorku zamykano setki osób ubiegających się o azyl czy cudzoziemców bez dokumentów, w~tym wielu ubiegających się o status uchodźcy, którzy przez dwadzieścia trzy godziny przebywali w~zamknięciu w~warunkach znacznie gorszych niż wielu morderców i~gwałcicieli. Jeśli ostatecznym celem międzynarodowego systemu imigracji i~kontroli granic było zamknięcie większości ludzkości w~miejscach, o których ludzie w~bogatych krajach nie musieli o nich myśleć, to był to jego ostateczny przejaw: miejsca, w~których istoty ludzkie dosłownie znikały. Prawie nikt w~Ameryce nie wiedział, że coś takiego się dzieje. Jednym z~pomysłów, którym wracał co jakiś czas, była taki, aby w~jakiś sposób udramatyzować sytuację, agresywnie sprawić, by to, co niewidzialne, pojawiło się ponownie: na przykład zdobyć portrety niektórych z~tych zatrzymanych i~umieścić je, być może wraz z~wypowiedziami lub biografiami, na naszych tarczach. Zdawaliśmy sobie również sprawę, że kanadyjski posterunek graniczny w~Champlain, przez który zwykle przechodzą Amerykanie podróżujący do Quebecu, znajduje się tuż obok bardzo dużego ośrodka dla imigrantów. Domagalibyśmy się naszych praw, jako globalnych obywateli, do przemarszu (w formacji) przez granicę. Istniała niewielka możliwość, że moglibyśmy się nawet przedostać. 

 Nie wszyscy byli szczęśliwi z~tego planu, lub z~pomysłem jakiejkolwiek akcji na granicy. Wielu myślało, że to stworzy medialny szum, którego media nawet nie opiszą. ,,Akcja bezpośrednia'', argumentował jeden z~aktywistów DAN, we wpisie na kilku serwerach aktywistów, ,,nie jest symboliczna!''. Chodzi o bezpośrednią konfrontację z~decydentami odpowiedzialnymi za kapitalistyczną globalizację, o bezpośrednie próby powstrzymania ich planów. Naprawdę powinniśmy skoncentrować nasze wysiłki na wymyśleniu sposobu na dostanie się do Kanady (i naprawdę jak trudne może to być ?).

 Śledziłem większość tej debaty online z~New Haven, gdzie uczyłem w~Yale trzy lub cztery dni w~tygodniu. W tym czasie mój harmonogram dla aktywistów zaczynał się od cotygodniowych spotkań Ya Basta! w~czwartek i~kończył się spotkaniem DAN o 18:00 w~niedzielę; potem znów jazda pociągiem do Connecticut. Wydaje mi się, że jednym ze sposobów, aby dać czytelnikowi poczucie, jak wygląda życie aktywisty, jest po prostu przejrzenie moich notatek i~podanie pewnych wskazówek dotyczących spotkań, w~których uczestniczyłem w~ciągu tygodni po wizycie CLAC. Jak się wkrótce okaże, są powody, od których są to szczególnie dobre tygodnie, aby zacząć. To, co nastąpi poniżej, będzie czymś w~rodzaju pamiętnika i~czerpie obszernie z~notatek przypominających pamiętniki, które prowadziłem w~tym czasie -- choć w~dużym stopniu przepisanych. Będzie też zawierało o wiele bardziej dosłowne fragmenty moich notatek terenowych.

\section{Czwartek, 1 marca 2001}

\noindent \textbf{Spotkanie Ya Basta! -- szkolenie formacyjne, Manhattan, 19:00}\medskip


 Co drugi tydzień, zamiast spotkań, Ya Basta! organizowała to, co nazwaliśmy ,,treningami formacyjnymi''. Odbywały się one w~studiu tańca w~Chelsea, udostępnionym nam przez członka kolektywu Betty. Betty była tancerką i~choreografką, w~tym czasie znaną na nowojorskiej scenie artystycznej ze swojego wyjątkowego tańca cieni. Po raz pierwszy została wciągnięta w~aktywizm po fiasku wyborczym na Florydzie w~2000 roku, potem wpadła w~kontakt z~Ya Basta! w~autobusie jadącym na protesty inauguracyjne w~Waszyngtonie. Później wyjaśniła, że  pociągał ją głównie teatralny, performatywny aspekt Ya Basta! -- choć wkrótce stała się również wierną postacią NYC DAN.

 W szkoleniu brało udział może dwadzieścia osób.

 Powinienem zwrócić uwagę, że termin ,,trening'' jest tutaj używany bardzo luźno, ponieważ nikt z~nas, z~wyjątkiem prawdopodobnie samej Betty, nie miał wystarczająco dużo doświadczenia, aby kogokolwiek ,,trenować''. Moose był we Włoszech i~widział prawdziwe taktyki i~sprzęt Ya Basta!, ale nigdy nie brał udziału w~żadnych akcjach. Betty, jako instruktorka tańca, wiedziała bardzo dużo o tym, jak ciała poruszają się w~przestrzeni, ale była nowa w~świecie akcji bezpośredniej. Reszta z~nas w~zasadzie zmyślała, gdy sobie radziliśmy. Niektórzy członkowie kolektywu studiowali starożytne techniki walki obronnej, obejmujące ściany tarcz i~tym podobne, lub wymieniali się pomysłami z~innymi kolektywami w~całym kraju, pracującymi przy podobnych eksperymentach. Niedawno ktoś znalazł broszurę na temat taktyki tarczy, stworzoną przez anarchistyczny kolektyw gdzieś na Środkowym Zachodzie i~umieścił ją na naszym listserv (co miało później wywołać nieoczekiwane skutki, ponieważ listserv był, jak większość serwerów, monitorowany przez policję). Pewien członek był kiedyś członkiem Towarzystwa Twórczego Anachronizmu i~wiedział coś o zbroi. Mimo to pytanie, co kto ,,ćwiczy'' zawsze było nieco arbitralne: rola wydawała się sprowadzać głównie do tego, co ważne. Nie, żeby ktokolwiek robił z~tego duże problemy w~tym momencie, ponieważ wszystko było tak oczywiście dobrą zabawą. ,,Treningi'' były głównie okazją do zakładania kombinezonów chemicznych i~improwizowanych ochraniaczy, zakładania tarcz, które zaczęliśmy składać z~pomarańczowych znaczników autostrad w~kształcie śmietnika na popiół(tych dużych plastikowych -- jeśli przeciąć je na pół, dadzą dwie idealne metrowe tarcze) i~bicia się piankowymi kijami.

 Była to również okazja do debiutu dla nowego sprzętu i~zabawek. Dwa tygodnie temu ktoś przyszedł z~pudełkiem tanich izraelskich masek przeciwgazowych, które kupił w~sklepie wysyłkowym. W tym tygodniu przynoszę pudełko kazoo (rozmawialiśmy z~przerwami o możliwości stworzenia sekcji kazoo Ya Basta!). Emma natychmiast zaczyna śpiewać nam swoje wykonanie ,,Walczyłem z~prawem i~prawo wygrało''. 

-- Niekoniecznie najbardziej inspirująca melodia, jaką wybrałam.

-- Cóż, ktoś stworzył wersję zatytułowaną ,,Walczyłem z~prawem i~wygrałem'', ale muzyka jest taka sama. 

 Odbyliśmy długą dyskusję na temat możliwych taktyk na większą skalę. Jednym z~pomysłów, który krążył, to była działo na pączki jakiegoś typu. Sprawa zaczęła się przed Konwencją Partii Republikańskiej w~Filadelfii w~2000, kiedy gazety doniosły, że dowódcy policji ostrzegali policjantów, aby nie przyjmowali żadnego jedzenia od protestujących jako ,, sposobu na ich przekabacenie''. Jedna z~grup afinicji uznała to za tak zabawne, że zaproponowała ustawienie stołu całkowicie pokrytego pączkami z~napisem ,,Policjo: Dołącz do nas i~wszystko to może być Twoje!''. Stół nigdy się nie zmaterializował. Ale wielu z~nas w~Ya Basta! czuło, że czysto defensywna taktyka wydawała się nieco ograniczająca. Jeśli strzelają do ciebie plastikowymi kulami i~gazem łzawiącym, chcesz jakoś odpowiedzieć -- po prostu nie w~sposób, który mógłby być uznany za szkodliwy. Coś śmiesznego, absurdalnego, ale co jednak sugerowało, że gdyby to była bitwa, dalibyśmy tyle, ile się da. Pączki wydawały się najbardziej oczywistym wyborem pocisku. Zastanawialiśmy się nad możliwościami ich dostarczenia: czy byłaby to katapulta (nawiązująca do starożytnego/średniowiecznego motywu)? A może bardziej coś typu proca? Ktoś zanurkował w~śmietniku za gigantyczną rurą i~jakąś wielką gumkę i~przyniósł ją na trening formacyjny, ale wszyscy doszliśmy do wniosku, że będziemy musieli skonsultować się z~kimś, kto rzeczywiście wiedział coś o inżynierii.

 W każdym razie szkolenie było częścią zabawną. Potem mieliśmy krótkie, formalne spotkanie, które zawsze było pewnym rozczarowaniem. Nie tylko dlatego, że najpierw spociliśmy się i~zmęczyliśmy, a potem musieliśmy przez godzinę siedzieć na ziemi i~rozmawiać. Było tak również dlatego, że zwykle dwie lub trzy osoby rozmawiały. Od początku, spotkania Ya Basta! składały się głównie z~przedłużającej się rozmowy trzech aktywistów: dwudziestokilkuletniego Moose'a i~nieco starszego małżeństwa o imieniu Smokey i~Flamma. Niektórzy mieli określone role: na przykład Laura i~ja tworzyliśmy grupę propagandowo-medialną. Ale większość z~nas została zdegradowana do rzucania od czasu do czasu komentarzy lub pytań. Wszystko to było po części spowodowane nietypowym składem grupy. Moose wyszedł z~DAN, grupy, która bardzo poważnie podchodziła do dynamiki spotkań. DAN zastosowała formalny proces konsensusu z~rotacyjnymi facylitatorami, rozbudowany system ,,składania głosów'' zaprojektowanym tak, aby zapewnić, że rozmowa nie zostanie zdominowana przez małą grupę głosów. Smokey i~Flamma nienawidzili DAN. Podobnie jak wielu innych anarchistów w~Nowym Jorku -- z~braku lepszej nazwy nazwę ich ,,hardcore'', którzy prawdopodobnie mieliby większe doświadczenie w~czarnych blokach, okupacjach drzew, scenie skłoterskiej lub w~każdym razie przyzwyczajeni do pracy w~małych, intymnych kolektywach -- postrzegali formalną strukturę DAN jako duszącą i~przytłaczającą. Ponieważ Ya Basta!, w~przeciwieństwie do szkoleń, rzadko angażowała kilkanaście osób, i~tak nie wydawało się zbytnio potrzebne do prowadzenia formalnego procesu. Zwykle Moose działał jako de facto facylitator. To samo w~sobie byłoby grzechem kardynalnym w~formalnym procesie konsensusu, ponieważ podstawową zasadą jest to, że ci, którzy zamierzają przedstawić propozycje na spotkaniu, nigdy nie powinni go również prowadzić (na spotkaniach formalnych facylitatorzy w~ogóle powinni unikać wyrażania opinii). Ponieważ Ya Basta! pierwotnie była pomysłem Moose'a, zwykle przedstawiał większość propozycji. W tamtym czasie jednak nikt z~nas nie widział w~tym większego problemu -- chociaż spotkania były dość męczące.

 Powodem, dla którego nie uważaliśmy tego za problem, był fakt, że NYC Ya Basta! była wciąż nową grupą. Nie było niczym niezwykłym, że nowe grupy aktywistów wyłaniały się z~wizji jednej osoby i~przez pierwsze kilka miesięcy jedna lub dwie osoby wykonywały prawie całą pracę koordynacyjną. Jednak to nie mogło trwać wiecznie. Jeśli grupa ma stać się prawdziwym, zrównoważonym kolektywem, nieuchronnie przychodzi moment, w~którym inni członkowie przejmują odpowiedzialność. Uczestnicy zaczynają pytać ,,dlaczego spotkanie prowadzi zawsze ta sama osoba? Dlaczego facylitator jest również tym, który przedstawia wszystkie propozycje?''. Następuje rodzaj chłopskiego powstania i~jeśli kolektyw nie rozpływa się w~gorzkich wzajemnych oskarżeniach, staje się prawdziwie demokratyczną grupą.

 W Ya Basta! było to pytanie otwarte, ponieważ, dość nietypowo, były dwa ogniska energii: po jednej stronie Moose, po drugiej Smokey i~Flamma. Można by o nich myśleć jako o różnych tendencjach, być może, typ DAN versus hardkor\footnote{ Chociaż rzeczywistość jest zawsze bardziej skomplikowana: na przykład była też spora liczba osób, które po raz pierwszy zaangażowały się w~aktywizm z~Ya Basta!, jeden włoski i~różne inne nieklasyfikowane.}. W tamtym czasie ta sytuacja mnie fascynowała, ponieważ nie mogłem znaleźć żadnej socjologicznej podstawy do rozłamu: pod względem pochodzenia klasowego lub trajektorii, pochodzenia etnicznego lub wykształcenia, te dwie grupy były nie do odróżnienia. To była całkowicie różnica w~filozofii.

 Powstaje oczywiście pytanie, co by się stało, gdyby rzeczywiście przyszło powstanie chłopskie.

 Przynajmniej w~ostatnich tygodniach spotkania stały się bardziej interesujące. Dwa tygodnie wcześniej Mac, jeden z~Kanadyjczyków w~nowojorskim DAN, przybył na szkolenie, aby nakłonić nas do rozważenia alternatywy dla Champlaina: akcji granicznej w~Cornwall, na moście w~środku rezerwatu Akwesasne Mohawk. Mac był w~kontakcie ze starym przyjacielem, członkiem Mohawk Warrior Society po stronie kanadyjskiej, który był bardzo entuzjastycznie nastawiony do wykorzystania mobilizacji FTAA do zakwestionowania faktu, że granica amerykańsko \dywiz kanadyjska przebiegała przez środek ziemi Mohawk. Pomimo faktu, że zarówno Stany Zjednoczone, jak i~Kanada uznały swoje terytorium za suwerenne na mocy traktatu, miejscowa ludność musiała przechodzić przez granicę międzynarodową i~poddawać się odprawie celnej tylko po to, by odwiedzić swoich bliskich po drugiej stronie. Pomysł Cornwall miał oczywisty urok -- zwłaszcza że  Mac sądził, że mógłby zebrać kilku kanadyjskich związkowców, by wesprzeć nas po drugiej stronie -- ale oznaczał porzucenie całego problemu przetrzymywania imigrantów, na którym się skupialiśmy. To też wydawało się trochę zbyt piękne, aby mogło być prawdziwe. Na pierwszym spotkaniu zgodziliśmy się trzymać z~Champlainem. Następnego dnia kilka osób pomyślało o tym lepiej i~za pośrednictwem listserv postanowiliśmy odłożyć decyzję do następnego spotkania. Ostateczną decyzją było dalsze dochodzenie; więc dzisiejsze spotkanie było w~dużej mierze poświęcone zebraniu grupy wolontariuszy, którzy wybraliby się na weekend do Cornwall i~sami sprawdzili. Shawn, kontakt Maca, już zbierał kilku innych Wojowników na spotkanie. Moose już znalazł samochód.

\section{Sobota, 3 marca}

\noindent \textbf{ Spotkanie z~Mohawkami}\medskip


 Właściwie skończyliśmy w~dwóch samochodach, ponieważ przyjechało też kilka osób z~Filadelfii. Ponadto mieliśmy Moose'a, Smokey'a i~Flammę, Maca reprezentującego DAN i~kilku lokalnych anarchistów mieszkających obecnie w~Independent Media Center (IMC). Mieli wyruszyć w~sobotę rano.

 Też miałem jechać, ale rodzinny kryzys medyczny zmusił mnie do zrezygnowania. Około godziny 9 rano dwa samochody pełne aktywistów wyruszyły. Jeden samochód zepsuł się w~tunelu Holland i~wszyscy musieli rzucać monetami, aby ustalić, kto pojedzie dalej.

 Tego wieczoru w~Internecie pojawił się następujący raport:

 Przedstawiciele NYC DAN, NYC Ya Basta!, IMC NYC, Philly Direct Action Working Group i~People's Law Collective spotkali się w~sobotę w~Cornwall z~Tyendinaga Mohawks, członkami Ontario Coalition Against Poverty (OCAP) i~Guelph Direct Action Group oraz Związek Społeczności Ludowej (ZKP) w~Kingston.

 Mohawkowie ogłosili, że są gotowi otworzyć granicę w~Cornwall dla aktywistów, którzy chcą przedostać się do Kanady 19 kwietnia, aby ci ostatni mogli dołączyć do karawany do Québec City, już organizowanej przez aktywistów z~Kingston.

 Mohawkowie przeznaczają ten ,,Dzień Gniewu'' jako potwierdzenie suwerenności, ponieważ most przekraczający tę granicę znajduje się na Ziemi Mohawków. Obecnie Mohawkowie zezwalają na korzystanie z~przejścia przez 364 dni w~roku i~otwierają je raz w~roku, aby zapewnić sobie suwerenność.

 Informacje te zostały następnie odesłane do odpowiednich grup i~poddane ich własnemu procesowi demokratycznego podejmowania decyzji. Jak dotąd NYC-DAN, NYC-Ya Basta! i~kilka tradycyjnych domów Mohawk publicznie zadeklarowało poparcie dla tej akcji.

 Kiedy czytałem to w~tamtym czasie, wydawało mi się to trochę niejasne. Sprawy stały się jaśniejsze na spotkaniu DAN następnego dnia. Pozwolę sobie dokładniej zdać relację z~tego spotkania, ponieważ było to jedno z~ciekawszych, w~których uczestniczyłem.

\section{Niedziela, 4 marca}

\noindent \textbf{ Spotkanie DAN, Centrum Kultury Charas El Bohio, 18:00}\medskip


 Spotkaliśmy się w~naszym zwykłym pokoju w~Charas, aktywistycznym centrum społecznym w~Lower East Side. Spotkanie zaczęło się skromnie: może dziesięciu lub dwunastu z~nas, chociaż w~ciągu następnej godziny napłynęło znacznie więcej, aż w~szczytowym momencie było nas dwudziestu pięciu lub trzydziestu. Tego dnia mieliśmy również co najmniej trzech zagranicznych gości: Mike i~Corey z~SalAMI oraz Olivier de Marcellus, który współpracował z~Peoples' Global Action w~Szwajcarii. Ludzie z~SalAMI byli na jedenastodniowej trasie po Ameryce, prowadząc treningi akcji w~miastach na północnym wschodzie. Gościła je głównie Międzynarodowa Organizacja Socjalistyczna (International Socialist Organization, ISO) i~towarzyszył im lokalny organizator ISO. Olivier akurat był w~mieście.

 Nicky i~Betty facylitowali. Zgłosiłem się na ochotnika, żeby zająć się protokołem.

 W przeciwieństwie do Ya Basta!, spotkania DAN miały wyraźnie formalny proces. Zawsze zaczynały się w~ten sam sposób. Najpierw opracowujemy agendę. Na ścianie zawsze był już napisany szkielet programu, ale każdy miał możliwość dodania nowych elementów, a potem przeznaczaliśmy czas na każdą z~nich: pięć minut na jedną, piętnaście na drugą, jedną lub dwie bardzo drobne ogłoszenia. Mike i~Corey musieli wyjść wcześniej, więc wystawiliśmy ich jako pierwszych.

 Myślę, że wszyscy byli przynajmniej trochę ciekawi Mike'a i~Coreya, ponieważ do tej pory wszyscy mieliśmy do czynienia tylko z~CLAC, a o ludziach z~SalAMI słyszeliśmy tylko z~drugiej ręki określanych jako irytujących pacyfistów. Większość z~nas była ciekawa, jacy naprawdę będą. Jak się okazało, obaj młodzi mężczyźni byli dość zadbani, w~zapinanych na guziki koszulach i~dokerach -- sympatyczni faceci, mówiący z~lekkim francuskim akcentem.

 Obaj wstali. Mike wyjaśnił, że SalAMI organizowało się w~Quebec City już od trzech lat, ale odkąd rozeszła się wiadomość o płocie bezpieczeństwa, postanowili, że nie będą czysto reagowali i~nie będą stawić czoła wrogowi na własnych warunkach. Więc zamiast Quebecu planowali akcję w~Ottawie, stolicy Kanady. Jak wyjaśnił, kluczową kwestią było to, że wszystkie negocjacje wokół FTAA były prowadzone w~tajemnicy. Najwyraźniej po fiasku rozmów WTO w~Seattle, negocjatorzy handlowi ze Stanów Zjednoczonych zdecydowali, że ich wielkim błędem było dać społeczeństwu pewne pojęcie o tym, o czym negocjują. Tym razem nie popełnią tego samego błędu. Żadne z~projektów czy informacji o tym, co w~nich jest, nie były publikowane, choć wszystkie były udostępniane korporacjom takim jak McDonald, Monsanto czy Citibank.

\noindent Mike: Pomysł jest taki, że 1 kwietnia zorganizujemy masowe demo w~Ottawie. Zarezerwowaliśmy trzy pokoje w~Parlamencie, aby poddać FTAA procesowi\ldots 

\noindent Ktoś: Chwileczkę, udało ci się zarezerwować pokoje w~\textit{Parlamencie}?

\noindent Corey: Cóż, właściwie zarezerwował je jeden z~naszych sojuszniczych związków zawodowych.

\noindent Majeed: Pamiętaj, Kanada to inny kraj. Związki faktycznie mają tam pewne prawa.

\noindent Mike: \ldots także mamy zamiar zaprosić każdego, kto pracuje nad projektami FTAA, aby pozwolił nam je tam przejrzeć, więc następnego dnia, drugiego, możemy przeprowadzić CD bez przemocy, blokadę Biura Spraw Zagranicznych i~Handlu. Zrobimy to, co nazywamy akcją ,,przeszukanie i~zajęcie'', wejdziemy w~poszukiwanie tekstu. Ogłosiliśmy, że zrobimy to, jeśli nie opublikują tekstu do 20 marca. Oczywiście, aby to zrobić, będziemy potrzebować dużej pomocy, aby uświadomić to mediom.

 Różne szczegóły zostały przedstawione na temat prób uzyskania poparcia i~możliwego udziału piosenkarki folkowej Ani DiFranco, zaciemnienia mediów w~sprawie FTAA w~USA (chociaż relacja była całkiem przyzwoita w~Kanadzie) i~innych kwestii. Majeed zapytał o różnorodność taktyk.

\noindent Mike: Cóż, oczywiście nigdy nie przekazalibyśmy ludzi policji, jak mówią niektóre e-maile. A jeśli mówisz o naszych wcześniejszych wytycznych, z~zasadami dotyczącymi masek i~tak dalej: nie, pozbyliśmy się ich. Ale kiedy słyszymy frazę ,,różnorodność taktyk'', cóż, brzmi to dla nas jak eufemizm ,,wszystko wolno''. 

 SalAMI gromadzi to, co nazywamy ,,stołem konwergencji'', z~ponad trzydziestoma różnymi grupami, w~tym związkami zawodowymi oraz grupami studenckimi i~kościelnymi, do których CLAC nigdy nie byłby w~stanie dotrzeć. To właśnie uważamy za \textit{prawdziwą }różnorodność. Ale z~konieczności opiera się na zasadzie działania bez przemocy; te grupy nigdy nawet by z~nami nie rozmawiały, gdyby pomyślały, że poprosimy je o zatwierdzenie akcji bez żadnych parametrów.

\noindent Corey: Co do CLAC\ldots  Jasne, są problemy z~przywództwem. i~problemy z~samcem alfa. Ale wciąż staramy się wszystko połączyć. Nasze szkolenie w~zakresie Kreatywnej Akcji jest przeznaczone dla obu stron i~mamy nadzieję, że gdy w~końcu dojdzie do działania, nie będziemy mieć dwóch różnych rad delegatów. Jeśli przynajmniej zgodzimy się na nie dla koktajli Mołotowa, możemy mieć jedną radę przedstawicieli. W przeciwnym razie, według mojej osobistej opinii, gramy tylko z~ułamkiem procenta ruchu.

\noindent Mike: Zostawię mój e-mail.

\noindent Corey: Jutro mamy szkolenie na Uniwersytecie Nowojorskim, o 19:00. Pomóżcie rozpowszechniać informacje!


\noindent  Brooke: Właściwie, powinnam prawdopodobnie zaznaczyć, że DAN reprezentuje różnorodność opinii, a nasze zasady Kontynentalnego DAN są właściwie trochę niejasne w~kwestii niestosowania przemocy. Myślę, że celowo. Dokładne sformułowanie to DAN wzywa do ,,obywatelskiego nieposłuszeństwa bez przemocy i~działań bezpośrednich''. Więc wspieramy obie. LA DAN jest całkowicie przeciwko przemocy. CLAC próbuje dostać się na rozmowę z~CDAN i~byłoby dobrze, gdybyście też się do tego przyłączyli.



\noindent  Mike: To nie są łatwe pytania, ale myślę, że wszystko się ułoży (\textit{śmiechy}) i~Quebec będzie niesamowite. Może to nie tylko uśmiechy i~przytulanie się, ale kiedy pojawia się opresja, wszyscy jesteśmy w~tym razem.



\noindent  Zoe: Ile czasu minie, zanim zbudują ogrodzenie z~drutu kolczastego?



\noindent  Mike: Cóż, większość betonu została już ułożona, zanim ziemia zamarzła. Ale to tylko fundament. Najwyraźniej będzie miał cztery kilometry wokół, to jest 2,5 mili obwodu, otaczając część miasta z~25 tysiącami mieszkańców. Wszyscy otrzymają specjalne karty, które pozwolą im wejść i~wyjść. Podjęto pewne wysiłki, aby zachęcić ludzi do odmowy, a nawet lepiej, taka była moja sugestia, do ich spalenia.



\noindent  SP: A co z~ludźmi, którzy tam pracują?



\noindent  Mike: Nie jestem pewien, jak sobie z~tym radzą. Przypuszczalnie też dostaną jakiś dowód tożsamości.



\noindent  Majeed: Mam pytanie. CLAC i~CASA (Comité d'accueil du Sommet des Amériques) są wyraźnie antykapitalistyczne. A co z~SalAMI?


\noindent  Mike: Cóż, tak, myślę, że można powiedzieć, że jesteśmy. Osobiście nie lubię używać słowa ,,kapitalizm'', bo to niektórych ludzi odstrasza. Przyjęliśmy wspólne podejście, ale by promować radykalną alternatywną wizję, w~tej chwili mamy komisję, która pracuje nad mapowaniem niektórych z~nich. Z pewnością możesz założyć wszystkie podstawy: jesteśmy przeciwko kapitalizmowi, przeciwko patriarchatowi, przeciwko homofobii.


\noindent  Corey: Musisz zrozumieć, że będzie to jedna z~największych akcji bezpieczeństwa w~historii. Wcześniej zdecydowaliśmy, że zamknięcie w~stylu Seattle jest wysoce nieprawdopodobne. Nie powstrzymamy Szczytu. Tak więc pytanie w~przypadku muru brzmiało: jak wymyślić plan, który można uznać za zwycięstwo? Co dałoby nam prawo do ogłoszenia zwycięstwa? Ludzie nad tym pracują. Jedna z~grup kobiet rozesłała wezwanie do wplecenia w~ogrodzenie obrazów i~haseł oporu. Byłoby to mocne w~mediach, ale z~pewnością nie zadowoliłoby wszystkich. Więc co wtedy? Celować w~lotniska? Zrobić blokadę, zamknąć bramy i~wkurzyć wszystkich mieszkańców? Dlatego apelujemy o strategiczną radę delegatów, aby nasza taktyka była zgodna z~naszymi celami strategicznymi. Naprawdę ważne jest to, w~jaki sposób nasze działania wpłyną na opinię publiczną, co wyniknie z~nich w~tworzeniu długoterminowych sojuszy \ldots 

 Olivier zauważa miękkim, bardzo dostojnym głosem, że wszystko to brzmi bardzo podobnie do tego, co wydarzyło się w~Davos podczas protestów na Światowym Forum Ekonomicznym miesiąc wcześniej. Policja przesadnie zareagowała i~zatrzymywała ludzi kilometry od rzeczywistych spotkań. Represje były tak brutalne -- wysyłali policję na pola, by zbierała odchody bydlęce, które mieszała z~wodą w~armatkach wodnych -- tak, że uderzyło to w~policję, powodując ogromną publiczną reakcję i~całkowite zwycięstwo dla nas. Pod koniec, kiedy w~Genewie wybuchły małe zamieszki (podpalili kilka banków), sondaże wykazały, że opinia publiczna nadal bardziej popierała protestujących niż rząd. A to jest w~Szwajcarii!

 Ludzie z~SalAMI są sceptyczni. 
 
 -- Możesz spróbować przejść przez Mur, jeśli chcesz -- mówi Mike. -- Ale musisz pamiętać, że będzie osiem tysięcy gliniarzy, pięćset Dartha Vaderów, których musiałbyś prześcignąć, gdyby rzeczywiście udało ci się wejść. Dlatego zdecydowaliśmy, że naszą strategią podczas samego Szczytu będzie nie-zbliżanie się do muru w~ogóle, ale stworzenie czegoś, co nazywamy ,,Obszarami Wolności i~Prawdy Ameryk'', może kilometr dalej. SalAMI chce utrzymać takie wydarzenie jako prawdziwie wyzwoloną strefę, a wiecie -- (znaczące spojrzenie na Yabbasów w~pokoju), -- to oznacza przestrzeń dla taktyk dla Ya Basty!, aby nie dopuścić do nich gliniarzy.

 Mike i~Corey muszą uciekać na szkolenie na NYU. Wychodzą ze swoim opiekunem ISO i~spotkanie trwa.

 Następny jest raport Lesleya z~wyprawy do Mohawków, która, jak stwierdziła, poszła bardzo dobrze. Siedem lub osiem osób, które dotarły do  Cornwall, spotkało się nie tylko z~członkami Mohawk Warrior Society, ale także z~członkami Kingston Labour Council po stronie kanadyjskiej (,,wraz z~kilkoma facetami z~Guelph, których nazywamy ,,Sieć Działań Guelph''). Mohawkowie zobowiązali się otworzyć granicę, aby zademonstrować swoją kontrolę nad ziemią w~Akwesasne. Shawn, ich główny rzecznik, przedstawiał tę akcję jako ,,dzień wściekłości'' z~powodu podziału ich ziemi i~deptania praw traktatowych przez oba rządy. Wojownicy sugerowali bardzo bojową taktykę, mówiąc o otwarciu mostu ,,wszelkimi niezbędnymi środkami'' -- co, jak zauważa Lesley, jest naprawdę czymś w~rodzaju blefu, ale może to postawić rząd kanadyjski w~niezwykle delikatnej sytuacji, ponieważ naprawdę nie chciałby użyć zbyt dużej siły na ziemiach Mohawków. To naprawdę zjednoczyłoby społeczność przeciwko nim. W rzeczywistości nie spodziewali się żadnego znaczącego sprzeciwu: Wojownicy mieli zwyczaj przejmowania mostu jeden dzień w~roku, przez ostatnie kilka lat, jako potwierdzenia suwerenności, a rząd nigdy nie próbował ich powstrzymać. Kanadyjscy pracownicy fabryk samochodowych i~poczty planowali już karawanę z~Toronto i~Kingston do Quebec City; chętnie byliby po drugiej stronie granicy i~mogli nas wesprzeć. 
 
 -- O tak -- dodała -- i~Okręgowa Rada Pracy mówi, że będą serwować herbatę.

 Różne radykalne pomysły były pojawiały się dookoła. Niektórzy Kanadyjczycy mówili o możliwości przejęcia śluz St. Lawrence, aby zamknąć ruch statków. Ale pojawiły się też słowa ostrzeżenia. 
 
 -- Pamiętajmy, że samo Akwesasne jest bardzo podzielone, jest bardzo niespójną społecznością. Mieli tam swoją własną małą wojnę domową w~latach 80. o plany budowy kasyna. Ludzie, z~którymi mamy do czynienia, pochodzą z~Domów Wojowników (którzy byli przeciwko kasynu); są z~nami, nawet jeśli wszyscy w~tych społecznościach nie są jednomyślni.

 Pytam, ile z~tego zostało zapisanych w~protokołach, które są publikowane na listserv z~otwartą subskrypcją. 
 
 -- W rzeczywistości Mohawkowie powiedzieli nam wyraźnie, że \textit{chcą}, aby te informacje zostały upublicznione, zwłaszcza zdanie: ,,Wojownicy Mohawków wzywają do Dni Gniewu'' 

 Po raporcie Lesley następuje szereg innych ogłoszeń: o zasiłku dla Casa del Sol, squatu na Bronksie; nadchodzące terminy rozprawy sądowej oskarżonych w~Esperanza Garden (zostali aresztowani w~obronie wspólnego ogrodu przed buldożerami kilka miesięcy wcześniej); przypomnienie o robieniu lalek w~każdą sobotę po południu dla Grupy More Gardens!. Były też raporty z~różnych grup roboczych DAN: Pracy, Policji i~Więziennictwa, Prawnego; kampanii WBAI; zespół WWW; Klubu Kobiet. Brooke ogłosiła, że  Continental DAN (DAN Kontynentalny -- CDAN) otrzymał prośbę od kilku osób w~Santa Cruz o przyłączenie się do sieci CDAN. ( ,,Prawdopodobnie banda hipisów i~głupków, ale i~tak ich kochamy''). Jest też raport nowo utworzonej Grupy Roboczej Banerów, która wydaje się składać z~dwóch zdecydowanie dziwnych osób w~czarnych bluzach z~kapturem, którzy odsłaniają wspaniały plakat, który jeden z~nich namalował dla DAN do niesienia na marszach.

 Dalej jest Nowe Sprawy. Pierwszym punktem porządku obrad jest Konwergencja w~Burlington. To, wyjaśnia Brooke, zaczyna przeradzać się w~problem. Pierwotnym pomysłem było zapewnienie miejsca, w~którym ludzie mogą zacząć się gromadzić w~poniedziałek 16 kwietnia, aby w~czwartek udać się do granicy i~najlepiej dotrzeć do Quebecu na czas parady CLAC ,,Karnawał przeciwko kapitalizmowi'' w~piątek 20-go. W ten sposób każdy miałby kilka dni na zorganizowanie szkoleń, wydarzeń edukacyjnych i~rad delegatów. Jednak w~tej chwili mieliśmy w~Vermont zaledwie cztery lub pięć osób, które próbowały wszystko zorganizować. Ponadto wydarzenie było technicznie organizowane za pośrednictwem NEGAN, New England Global Action Network. (Globalnej Sieci Działań Nowej Anglii). W zasadzie NEGAN było lokalnym odpowiednikiem DAN -- ale niestety na szczycie były typy ,,antykorporacyjne'', liberalni reformatorzy, Zieloni, i~grupy socjalistyczne -- zwłaszcza ISO. ISO miała swoje własne cele, które wydawały się w~niewielkim stopniu pokrywać się z~naszymi.

 Wymagane jest tutaj pewne tło. ISO jest jedną z~niezliczonych sekt trockistowskich, które zostały założone i~oddzieliły się od siebie w~latach 60. i~70., którym udało się przetrwać, a nawet rozwinąć w~międzyczasie. Stało się tak, ponieważ, w~przeciwieństwie do innych, ISO nie koncentrowała swoich wysiłków rekrutacyjnych w~fabrykach, ale na kampusach uniwersyteckich. W 2001 roku ISO była pod wieloma względami anarchistycznym nemezis -- szczególnie dla DAN. Po części było to spowodowane tym, że próbowali robić podobne rzeczy za pomocą radykalnie różnych metod. Obie grupy były rewolucyjnie antykapitalistyczne. Obie wierzyły w~pracę w~szerokich koalicjach i~próbowały zachęcić do podjęcia bardziej radykalnych kierunków. Problem polegał na tym, że dla ISO był to bardzo długotrwały proces, a tymczasem interesowały ich głównie liczby. Zawsze starali się tworzyć jak najszersze koalicje, co oznaczało zabieganie o kierownictwo związków i~głównego nurtu organizacji pozarządowych, które z~kolei prawie niezmiennie pragnęły gwarancji przeciwko przemocy lub, często, przeciwko wszelkim działaniom bezpośrednim. Z anarchistycznego punktu widzenia było to jak próba wystawienia w~pole stutysięcznej armii, ale tylko pod warunkiem, że nikt nic nie zrobi.

 Nie byłoby aż tak irytujące, gdyby ISO była po prostu przeciwna działaniom bezpośrednim. Wtedy można by ich po prostu zignorować. Ich członkowie uczestniczyli w~radach delegatów i~często sami brali udział w~działaniach. Dlatego to oni zawsze starali się namówić wszystkich do złagodzenia sytuacji, zamieniając plan bojowej akcji bezpośredniej w~akt nieposłuszeństwa obywatelskiego bez przemocy, zamieniając plan pokojowego nieposłuszeństwa obywatelskiego w~niedozwolony marsz, zmieniając zabroniony marsz w~legalny. Strategia dążenia do jak największej koalicji zapewniała im skłonność do bycia szczodrymi nawet wobec grup, które wygłaszają zbyt radykalne przesłanie: stało się to rodzajem powtarzającego się żartu wśród anarchistów, że jeśli nazwać organizację ,,antykapitalistyczną'', to gwarantuje, że socjaliści się nie pojawią. Wreszcie, grupy takie jak ISO były wyraźnie awangardowe. Uważali, że dysponują właściwą analizą sytuacji na świecie. Kiedy angażowali się w~szersze koalicje, miało sens tylko to, że to oni powinni zapewniać kierunek i~przywództwo.

 W przeciwieństwie do tego anarchiści mieli tendencję do określania swojej strategii jako ,,skażenie''. Założenie było takie, że działanie bezpośrednie i~demokracja bezpośrednia są zaraźliwe; prawie każdy, kto się z~nimi zetknął, może zostać przemieniony przez to doświadczenie. Zresztą nie chodzi o to, żeby organizować ludzi, ale żeby zachęcić ich do samoorganizowania. Zamiast zawierać układy z~biurokratami związkowymi, grupy takie jak CLAC czy DAN próbowały odwoływać się bezpośrednio do szeregowych członków. Zamiast próbować przejąć duże organizacje, dążyli do stworzenia dramatycznych modeli samoorganizacji, które inni mogliby naśladować, nawet jeśli nieuchronnie zakładano, na własny idiosynkratyczny sposób.

 Wszystko to bez wątpienia ułatwia zrozumienie, dlaczego trasa SalAMI była sponsorowana przez ISO i~dlaczego Mike i~Corey przybyli i~odeszli, eskortowani przez przyzwoitkę ISO.

 Żeby wrócić do NEGAN, to\ldots 

 Tydzień wcześniej Moose i~Marina, długoletni działacz DAN (i były członek ISO), pojechali na spotkanie NEGAN w~Worcester w~stanie Massachusetts. Żaden z~nich nie był dziś na spotkaniu DAN -- wrócili z~grypą -- ale wszyscy wiedzieli, co się stało. Spotkanie było pełne ludzi z~ISO, którzy nalegali na utworzenie komitetu sterującego i~naciskali na głosowanie większościowe zamiast konsensusu. (W końcu nie można próbować zebrać komitetu sterującego, który nie działa większością głosów.) Zaproponowali również, aby NEGAN skoncentrował się na organizowaniu autobusów do Québec City w~sobotę, aby mogli współpracować ze związkami zawodowymi, które były tego dnia będą przewozić swoich ludzi na marsz. Argumentowali, że znacznie ułatwiłoby to przedostanie się przez granicę. Byłoby to jednak również całkowite pominięcie jednego dnia działań bezpośrednich zaplanowanych na ten dzień. To wszystko było ogromnym problemem, ponieważ logiczne powinno być, aby DAN skierowała wszystkie swoje zasoby na Konwergencję w~Burlington -- organizowanie rad przedstawicielskich i~tym podobnych było przecież tym, co robiliśmy najlepiej. Ale wyglądało na to, że anarchiści byli po prostu pomijani:


\noindent Brooke: Mam wiele nazwisk ludzi tam [w Burlington], ale\ldots  mam nadzieję, że nie obrażam nikogo, mówiąc to, ale: Burlington miało kiedyś grupę Direct Action. Ale została opanowana przez ISO i~typy socjalistyczne. Biella i~Native Forest Networks są bardziej anarchistyczne, ta pierwsza to kobieta, która w~zasadzie samodzielnie zajmuje się większością organizacji na rzecz konwergencji. Bardzo się staram zaangażować Instytut Ekologii Społecznej (co zrobią, jeśli będą wiedzieć, co jest dla nich dobre), ale do tej pory też niewiele zrobili, więc na razie sprawy tam naprawdę nie są w~dobrym stanie.

\noindent  Majeed: Wiesz, nie chcę być wulgarny czy sekciarski, ale mówię ,,jebać ISO''. 

\noindent  David: Um, czy powinienem to umieścić w~notatkach?

\noindent  Majeed: Właściwie tak. Zapisz to.

\noindent  Brooke: Pamiętaj, że ludzie czytają protokoły nawet w~Kalifornii.

\noindent  Majeed: Nieważne. Szczerze mówiąc, mam już dość tych facetów. W chwili, gdy pojawia się najmniejsza iluzja lepszej pozycji, przejmują władzę i~natychmiast przerywają wszelką debatę. Robią to przynajmniej od wojny w~Zatoce Perskiej. Mówię, po prostu skontaktujmy się z~,,autentycznymi elementami''.

\noindent  David: (wciąż notując) Autentyczne elementy?

\noindent  Majeed: Wiesz: ludzie, którzy robią to, ponieważ chcą, aby mobilizacja się powiodła, a nie aby posuwać naprzód jakiś pieprzony imperatyw organizacyjny.

\noindent  Majeed, były członek Irańskiej Partii Komunistycznej (która, jak nam wyjaśnił, była w~dużej mierze kurdyjska), obecnie aktywny w~Partii Pracy DAN, odkąd został anarchistą, stał się niezwykle niecierpliwy wobec awangardystów.



 Po podkreśleniu naszej determinacji, by pomóc z~,,autentycznymi elementami'' w~Burlington, rozmawiamy trochę o następnym zaplanowanym cotygodniowym spotkaniu. Tak się składa, że  odbywa się to w~tym samym czasie, co protest Krytycznego Ruchu Oporu przeciwko Horizon Center, aresztowi dla nieletnich w~Midtown. Ktoś sugeruje: dlaczego wszyscy nie pójdziemy na wiec, a jeśli jest jakaś pilna sprawa, którą trzeba omówić, możemy to zrobić na peronie metra, gdzie i~tak wszyscy powinni się zbierać? Wszyscy się zgadzają, chociaż Brooke nalega, abyśmy natychmiast i~w widocznym miejscu zamieścili to na liście.

 Koniec spotkania jest dość nietypowy. Technicznie rzecz biorąc, na końcu każdego spotkania DAN była opcja zorganizowania ,,sesji edukacyjnej''. Myślę, że nigdy nie mieliśmy. Ale wszyscy chcą się dowiedzieć o Peoples’ Global Action (PGA). Wszyscy słyszeliśmy o PGA -- w~rzeczywistości DAN był w~pewnym sensie na nim wzorowany -- ale niewielu z~nas (z wyjątkiem Lesley, która studiowała PGA jako absolwentka studiów magisterskich w~Kolumbii) naprawdę wiedziało o tym tyle, poza fakt, że była to globalna sieć stworzona przez Zapatystów, która wystosowała wezwania do równoczesnych globalnych dni akcji i, co najsłynniejsze, pierwotnie wpadła na pomysł globalnego dnia akcji przeciwko spotkaniom WTO w~Seattle. Poprosiliśmy więc Oliviera -- lub ,,Olivera'', jak sam siebie nazywał -- o przybliżenie nam tła. Olivier to mężczyzna, który wygląda na pięćdziesiątkę lub wczesną sześćdziesiątkę, bardzo arystokratyczny, wyglądający na Europejczyka z~naprawdę niezwykłym nosem. Jesteśmy raczej zaskoczeni, gdy dowiadujemy się, że tak naprawdę jest Amerykaninem, uchodźcą z~lat 60., który uciekł z~kraju przez Wietnam i~od tamtej pory mieszka w~Genewie.

\noindent SESJA EDUKACYJNA

\noindent Olivier: Witam. Nazywam się Oliver de Marcellus i~pochodzę z~Genewy. Mieszkam tam, odkąd opuściłem Stany w~1968 roku. W PGA jestem od jego powstania w~1998 roku; wcześniej pracowałem z~ruchem Zapatystów.

\noindent Brooke: Naprawdę chcielibyśmy usłyszeć więcej o historii.

\noindent Olivier Cóż, więcej na ten temat możesz przeczytać na naszej stronie internetowej czyli www.agp.org. (To z~francuskiego lub hiszpańskiego akronimu. Jeśli wpiszesz ,,pga'', zostaniesz przeniesiony do Professional Golf Association.)

 O PGA\ldots  hmmm. Myślę, że są dwa sposoby mówienia o PGA. Najprościej jest powiedzieć, że możesz być członkiem PGA, niezależnie od tego, czy o tym wiesz, czy nie. Ponieważ PGA to nic innego jak pięć zasad (które, jak sądzę, są również podstawowymi zasadami DAN). Dobrze, to, a także udział w~akcjach zgodnych z~tymi zasadami. Jeśli więc spojrzeć na to w~ten sposób, jedyną definicją PGA to ,,ludzie, którzy zgadzają się z~manifestem''. Zgodnie z~tą definicją w~PGA są miliony ludzi i~większość z~nich nawet o tym nie wie.

 To szeroka definicja. Mniejsza definicja, która prawie nie istnieje, to organizacja. Nie \textit{powinniśmy }być organizacją. Nie mamy funduszy, sekretariatu, nikt nie jest uprawniony do wypowiadania się w~imieniu PGA. Mamy Międzynarodowy Komitet Organizacyjny, składający się z~przedstawicieli grup z~różnych kontynentów, którzy zmieniają się co dwa lata. Wszystko, co ten Komitet może zrobić, to zwoływać międzynarodowe spotkania PGA, decydować, kto przyjedzie i~kto z~Globalnego Południa otrzyma darmowe bilety.

\noindent Maggie: Jak definiujecie ,,Globalne Południe''?

\noindent Olivier: Wszędzie oprócz Europy i~Ameryki Północnej.

 Na początku Organizatorzy mieli też decydować o globalnych dniach akcji, ale jak się okazało, tak trudno było nakłonić część z~nich do odpowiedzi na e-maile itd., że grupy same zaczęły przejmować inicjatywę. Tak więc działa to tak, że działania są w~końcu proponowane przez najbardziej zainteresowane grupy lokalne, wezwanie krąży w~sieci do wszystkich zaangażowanych, a ci, którzy są zainteresowani, biorą udział, ci, którzy nie są, ignorują je. Myślę, że jest to najbardziej demokratyczny sposób. (Zazwyczaj działania są następnie zatwierdzane pięć miesięcy później przez Komitet Organizatorów, z~mocą wsteczną, ale nikt tak naprawdę tego nie zauważa).

 Na przykład demonstracje w~Genewie w~1998 roku zostały zwołane przez Organizatorów. The Reclaim the Streets w~Anglii nazwali dema J18 rok później, zaproponowali to, a ludzie po prostu zaczęli to robić. N30 było tym samym, to była największa rzecz, jaką kiedykolwiek zrobiliśmy, ale zaczęło się od apelu o podjęcie działań przeciwko WTO, gdziekolwiek się zjedzie, nawet zanim wiedzieliśmy, że spotka się w~Seattle. W przypadku [działań przeciwko spotkaniom IMF w] Pradze było to samo, zaproponowała to lokalna grupa i~akcja została podjęta. Więc myślę, że tak to robimy od 1999 roku.

\noindent Brooke: Czy możesz porozmawiać o nadchodzącej konferencji?

\noindent Olivier: Tak. Międzynarodowa konferencja PGA odbędzie się w~Cochabamba od 17 do 24 września, a w~tym tygodniu \textit{naprawdę }zostanie ogłoszone wezwanie (przepraszam, wiem, że ciągle to powtarzamy, ale w~tym tygodniu naprawdę tak się stanie). Dążymy do dwustu delegatów, z~których siedemdziesiąt procent musi pochodzić z~południa lub wschodu; sześćdziesięciu z~Europy Zachodniej i~Ameryki Północnej, a reszta z~,,Globalnego Południa''. Największy kontyngent będzie pochodził z~Ameryki Łacińskiej. Obecnie epicentrum oporu wobec globalizacji stanowią Andy; to jest kluczowe miejsce, dlatego na początek organizujemy je w~Boliwii. No cóż, oczywiście, ponieważ to w~tym mieście odbywała się wielka kampania przeciwko Bechtel, kiedy próbowali sprywatyzować system wodociągowy, która była kierowana przez grupy związane z~PGA.

 Ale naprawdę mamy nadzieję, że nie dojdzie do przewrotu, zanim to się stanie.

\noindent Stuart: Gdyby dokonali zamachu stanu, nie mogłoby to pogorszyć rządu.

\noindent Olivier: Ale jeśli gospodarze będą się ukrywać, będzie to bardzo trudno zorganizować.

 Próbujemy też uruchomić bardziej zdecentralizowany system finansowania, co jest kluczowe dla zdobycia biletów dla delegatów z~Południa, ponieważ bilety lotnicze są po prostu bardzo drogie. Na grudniowym spotkaniu Conveners w~Pradze zdecydowaliśmy, że nie możemy dalej otrzymywać pieniędzy od fundacji, ponieważ im bardziej będziemy skuteczni, tym mniej fundacji będzie chciało nam dawać pieniądze.

\noindent Lesley: A na co właściwie wykorzystuje się te pieniądze?

\noindent Olivier: Tylko po to, żeby przywieźć delegatów na spotkania.

\noindent [następuje pewna dyskusja na temat potencjalnego zaangażowania DAN w~spotkania w~Cochabamba\ldots  Postanawiamy, że naprawdę powinniśmy umieścić manifest PGA na naszej stronie internetowej]

\noindent Olivier: To by było właściwie przydatne, bo jedną z~funkcji konferencji jest zmiana manifestu. Na przykład delegaci europejscy będą chcieli się upewnić, że coś na temat zmian klimatycznych zostanie tam umieszczone, ponieważ nie wydawało się to tak wyraźnie pilne, kiedy po raz pierwszy pisaliśmy w~1998 roku.

\noindent David: Czy możesz nam trochę opowiedzieć o tym, jak to wszystko się zaczęło?

\noindent Olivier: Cóż, PGA z~całą pewnością zostało po raz pierwszy pomyślane jako część ruchu Zapatystów. Można powiedzieć, że została założona podczas Drugiego Międzygalaktycznego Zapatista Encuentro w~Hiszpanii w~1997 roku. Wtedy po raz pierwszy spotkały się grupy, które stały się kręgosłupem PAR: europejscy anarchiści, brazylijski Ruch Chłopów Bezrolnych i, właściwie, prawdopodobnie najważniejszą grupą był KRRS. To jest Karnataka State Farmers’ Association (Stowarzyszenie Farmerów Stanu Karnataka), które jest Gandhijskim socjalistycznym ruchem chłopskim w~Indiach i~ma około dziesięciu milionów członków. Po raz pierwszy zasłynęli z~kampanii ,,Cremate Monsanto'' w~połowie lat 90., w~której systematycznie palili genetycznie zmodyfikowane uprawy. W zeszłym roku KRRS zmobilizował 51 000 ludzi w~wozach dla byków, którzy próbowali przejąć port w~Bombaju, i~dla nich była to tak naprawdę tylko akcja średniej wielkości. W maju 1998 zorganizowali 280 tysięcy ludzi na masową demonstrację anty-WTO. To było prawdopodobnie ich największe. Ale działają na kolosalną skalę.

\noindent Natalie: Wiesz, że naprawdę powinnaś mieć całą tę historię na stronie internetowej.

\noindent Olivier: Wiem. Powinniśmy. Prawdopodobnie mamy najgorszą istniejącą stronę internetową; była to prawdopodobnie pierwsza strona antyglobalistyczna, ale projekt jest przerażający.

\noindent Stuart: Mówisz o propozycjach wyłaniających się z~lokalnych grup, przez ,,grupy lokalne'' masz na myśli ,,wszelkie grupy, które poparły zasady''?

\noindent Olivier: Cóż, innym aspektem tego statusu pozaorganizacyjnego jest brak formalnego członkostwa. Każdy może coś zaproponować.

\noindent DW: Więc teoretycznie my też moglibyśmy?

\noindent Olivier: Och, absolutnie. Dlaczego nie?

 Wróćmy jednak do historii. PGA miało swoje pierwsze spotkanie w~Hiszpanii, w~1998 roku, i~na tym pierwszym spotkaniu było wielu anarchistów z~Anglii, takich jak ludzie z~Reclaim the Streets-London, aktywni w~ruchu anty-drogowym, którzy nie mieli pojęcia, że  podobne rzeczy dzieją się na kontynencie i~odwrotnie. Spotkali się ze skłoterami we Włoszech i~Niemczech, a pomysły zaczęły się rozprzestrzeniać. Na przykład nikt z~nas na kontynencie nie słyszał o idei nielegalnej imprezy ulicznej w~stylu brytyjskim. Miesiąc później organizowaliśmy taką w~Genewie. i~było cudownie. Wkrótce ludzie organizowali je wszędzie.

 Ktoś wymyślił teorię, że rezultatem jest coś w~rodzaju globalnego mózgu: połączenia komunikacyjne są takie, że można sobie wyobrazić ludzi nie tylko komunikujących się, ale działających i~działających cholernie skutecznie, bez przywództwa, sekretariatu, nawet bez formalnych kanałów informacyjnych. To trochę tak, jak mrówki spotykające się w~mrowisku, wszystkie machają do siebie antenami, a informacje po prostu się rozchodzą, nawet jeśli nie ma łańcucha dowodzenia ani nawet hierarchicznej struktury informacji.

 Oczywiście bez internetu byłoby to niemożliwe.

\noindent Ktoś: Oczywiście na początku powiedzieli to także o Zapatystach.

\noindent Olivier: Właściwie na początku było trochę irytujące, jak media zwykły mówić, że Zapatyści byli po prostu fenomenem w~Internecie, irytującym, to znaczy dla ludzi, którzy faktycznie wiedzą, jak trudno jest do nich dotrzeć. Ale w~pewnym sensie to jest prawda. Lista internetowa z~natury nie może być autorytarna, po prostu wystawiasz propozycję i~ludzie o niej dyskutują, ci, którzy ją lubią, idą to zrobić. Jeśli to nie jest tak dobra propozycja, mniej ludzi to zrobi. Jedyną rzeczą, której absolutnie nie możesz zrobić przez Internet, jest głosowanie.

\noindent Stuart: Wszystko to brzmi jak DAN!

\noindent Olivier: Wiesz, kiedy przybyłem do Seattle na akcje WTO, nie wiedziałem nawet, czym jest DAN. Potem wziąłem ulotkę DAN, w~środku były zasady PGA i~zdjęcia działań Genewy i~powiedziałem: ,,Och, to tylko PGA'' To mi się przytrafia cały czas. Spotkałem kogoś z~PGA Korea w~zeszłym tygodniu i~było to: ,,Naprawdę? To istnieje?''. W Pradze pojawiły się dwa autobusy z~Turkami. Okazało się, że w~Turcji jest sieć PGA w~pięciu miastach, pobrali nasze zasady ze strony internetowej i~chodzili po okolicy, pokazując filmy z~Seattle. Nikt z~nas nie miał najmniejszego pojęcia, dopóki ich nie poznaliśmy. To się dzieje cały czas.

 Oczywiście, teraz gdy istnieją Indymedia, informacje docierają do nas częściej niż kiedyś. Kiedy maniacy sieci po raz pierwszy wyjaśnili nam ideę równoczesnych demonstracji, staraliśmy się to koordynować, wysyłając e-maile do Genewy. To nie zadziałało. Ale teraz zlecamy to niejako Indymediom. Tak więc podczas akcji w~Pradze mieliśmy 250 jednoczesnych manif na całym świecie, z~czego 70 było relacjonowanych w~Indymediach. A to z~kolei zmienia nasz stosunek do mediów korporacyjnych, w~zasadzie już ich nawet nie potrzebujemy. Kilka miesięcy temu mieliśmy akcję w~Genewie, gdzie w~solidarności z~ludźmi prowadzącymi tam akcję okupowaliśmy ambasadę Ekwadoru. Po wszystkim zdaliśmy sobie sprawę, że zapomnieliśmy nawet powiedzieć o tym mediom, bo kto ich potrzebuje? To trafia do ludzi w~Ekwadorze, że zrobiliśmy to przez Indymedia, i~to jest dla nas naprawdę ważne.

 To, co dzieje się teraz, jest z~pewnością największym wydarzeniem od maja 1968 roku. Przynajmniej w~Europie. Po raz pierwszy poczułem tak ogromny, globalny wzrost. Praga była po prostu\ldots  wow! Co najmniej osiem różnych krajów wysłało kontyngenty liczące ponad tysiąc osób. Kiedy zaczynaliśmy, nikt z~nas nie miał pojęcia, jak zorganizować masową konwergencję lub radę przedstawicieli, musieliśmy to wszystko wymyślić od zera. Potem nadszedł wrzesień: oto i~to! Zadziałało! Skończyło się na tym, że dzień wcześniej wyrzuciliśmy IMF z~miasta.

 A każde spotkanie musiało być koordynowane w~siedmiu różnych językach: angielskim, francuskim, niemieckim, tureckim, hiszpańskim, włoskim, czeskim\ldots 

\noindent Brooke: Jezu!

\noindent Olivier: Ale zadziałało!

\noindent Betty: Czy mógłbyś opowiedzieć o działaniach w~Davos i~jakich lekcjach możemy się z~nich nauczyć dla Quebec City?

\noindent Olivier: Cóż, na ten temat:

 Niestety, wydaje się, że stanowimy obecnie największe zagrożenie dla Imperium i~wydają być \textit{naprawdę }zaniepokojeni. Przykro mi to mówić, bo tak naprawdę jesteśmy tylko śmieszną bandą klaunów, ale proszę bardzo.

 W Nicei pomyśleliśmy, że spróbują zablokować granice, zanim jeszcze tam dotrzemy, i~faktycznie zrobili to, całkowicie nielegalnie, przynajmniej przeciwko Włochom, którzy teoretycznie mają te same paszporty UE. Użyli także interesującej taktyki ,,dziel i~zwyciężaj'', jak zapewnienie darmowych pociągów dla ludzi związkowych, a następnie próbując pobić przedstawicieli ludów autonomicznych.

 Spodziewaliśmy się, że granice zostaną zablokowane, a dotarcie do samego Davos będzie niemożliwe; więc powiedzieliśmy, że jeśli nie możemy się tam dostać, zrobimy akcje i~blokady, gdziekolwiek będziemy musieli. Gdybyśmy nie mogli dotrzeć dalej niż dno doliny, gdzie pociąg łączy się z~autostradą, poniżej ośrodka narciarskiego, w~którym faktycznie odbywały się spotkania, to w~porządku, zablokowalibyśmy tam trasy dla samochodów. Albo jeśli nie możemy dostać się do kraju, jeśli spróbują zamknąć granicę, sami zamkniemy granicę. Skończyło się na demonstracjach we wszystkich trzech, co było świetne, pięciuset Włochów zatrzymało się na granicy, zablokowało tam autostradę, pięćset innych ludzi wkradło się do samego Davos, co było świetne, a na dnie doliny było coś w~rodzaju pięciu różnych grup \ldots  To było \textit{całkowite zwycięstwo}, pomimo największej mobilizacji sił bezpieczeństwa w~historii Szwajcarii, wszędzie czołgi i~drut kolczasty, strzelanie do nas armatkami wodnymi, gazem łzawiącym i~gumowymi kulami od chwili, gdy się pojawiliśmy, nawet jeśli była to tylko kupa głupich balonów i~ludzie przebrani na szczudłach lub w~kostiumy Ronalda McDonalda. W końcu przesadzili tak bardzo, że nawet wysoce burżuazyjna szwajcarska opinia publiczna była po naszej stronie. Kilka kantonów zagłosowało za usunięciem policji federalnej z~ich terytorium, prezydent zrobił z~siebie głupka na konferencji prasowej tego wieczoru, ponieważ chciał porozmawiać o obradach w~Davos i~ciągle rzucał dziennikarzom: ,,Dlaczego ciągle pytacie o demonstracje?''.

\noindent Natalie: Czy były jakieś aresztowania? 

\noindent Olivier: O tak. Ale musieli ich szybko wypuścić. 

 Jeśli chodzi o FTAA, nie ma powodu, aby ich nie blokować, nawet jeśli jest to ogromne ogrodzenie bezpieczeństwa, nadal muszą być bramy. W każdym miejscu, w~którym możesz ich zablokować, chodzi o\ldots 

 Rozmawiamy przez jakiś czas o problemach koordynowania z~grupami z~niewielkim lub żadnym dostępem do Internetu, o niesamowitej grupie PGA zwanej ,,Siecią Wolnych Czarnych Społeczności Ameryki Południowej'', założonej przez zbiegłych niewolników w~XIX wieku, o kilkunastu innych sprawach. Zanim udaliśmy się do pobliskiej kawiarni, aby kontynuować rozmowę, była już prawie 23:00.

\section{Wtorek, 6 marca}
\noindent \textbf{ Spotkanie koalicji FTAA, godz. 20:00}\medskip

 Właściwie je przegapiłem (wraz z~odbywającym się w~tym samym czasie spotkaniem DAN Labor), ale słyszałem, co się stało.

 Koalicja FTAA to szeroka grupa obejmująca cały Nowy Jork, w~skład której wchodzą DAN, Zieloni, ISO i~różni niezależni aktywiści organizujący się dla miasta Quebec. Kiedy więc Moose i~Marina w~końcu wyszli z~choroby, aby złożyć raport ze spotkania NEGAN, musieli być względnie ostrożni. Najwyraźniej istniała również pewna niejasność co do stopnia, w~jakim Mohawkowie po stronie amerykańskiej są naprawdę na pokładzie, ponieważ do tej pory rozmawialiśmy tylko z~kanadyjskimi. W Domach Wojowników po stronie amerykańskiej zachodził jakiś proces i~nikt nie był do końca pewien, jak się sprawy potoczą. Narastały też napięcia dotyczące samej struktury koalicji.

\section{Czwartek, 8 marca}

\noindent \textbf{ Spotkanie Ya Basta! Manhattan, 19:00\footnote{ Czytelniczka zauważy, że niektóre z~tych wpisów są napisane w~czasie teraźniejszym, inne w~przeszłości. Zasadniczo logika jest następująca: kiedy relacja jest wzięta względnie bezpośrednio z~moich notatek i~zawiera długie fragmenty rzeczywistego dialogu, to zwykle jest opisywana w~czasie teraźniejszym. Kiedy narracja jest rekonstruowana ex post facto, na podstawie zapisów notatek, przyjmuję zwykle czas przeszły. W niektórych wpisach czas zmienia się z~czasu przeszłego, opisując tło, a nawet relacjonując wczesną część spotkania, do teraźniejszości, kiedy narracja staje się bardziej bezpośrednia i~podąża za notatkami w~sposób krok po kroku.}}\medskip


 O wiele lepsze spotkanie niż zwykle, które odbyło się w~mieszkaniu Aladyna na osiedlu mieszkaniowym w~Chelsea. Było około dwudziestu osób. Tym razem spotkanie było nawet facylitowane: nieformalnie, ale dobrze. Co jeszcze bardziej niezwykłe, wszystko zostało utrwalone na taśmie wideo.

 Za tym kryje się długa historia, ale krótka wersja jest taka, że  młody filmowiec imieniem Sasha skontaktował się z~ludźmi ze społeczności aktywistów, ponieważ chciał nakręcić film dokumentalny. Jego pomysł polegał na zestawieniu standardowych wizerunków medialnych przerażających zamaskowanych anarchistów z~portretami prawdziwych ludzi za maskami. Wkrótce zaangażował się w~Ya Basta! i~w ciągu miesiąca lub dwóch stał się faktycznie częścią grupy. Nikt nie miał z~tym większych problemów. Ale to był pierwszy raz, kiedy nakręcił spotkanie. Właściwie to pierwszy raz, jaki znam, żeby ktoś kręcił spotkanie takiej grupy -- po części uszło mu to na sucho tylko dlatego, że obiecał nikomu nie pokazywać twarzy, zawsze trzymając aparat skierowany nisko. Jeden lub dwóch członków faktycznie nosiło maski na spotkanie, głównie, jak podejrzewałem, dla efektu dramatycznego.

 Zresztą, jak potoczyły się spotkania, okazało się, że był to doskonały wybór.

 W związku z~akcją Mohawk, z~powodu przedwczesnego rozgłosu, wybuchł już mały kryzys. Podczas gdy Shawn, nasz główny sojusznik, specjalnie poprosił nas, abyśmy użyli słów ,,Dni Gniewu'' i~podczas gdy w~lokalnym magazynie \textit{Eye News }natychmiast ukazał się artykuł, cytując go, mówiącego, że przejmą most ,,za pomocą wszystkich potrzebnych środków'' i~pokazujące zdjęcia zamaskowanych Mohawk Warriors z~karabinami maszynowymi z~okupacji Oka w~Quebecu w~latach 80., wszystko to miało raczej charakter blefu. Shawn obliczył, że po traumie bliskiego powstania i~długiej rozgrywce z~rządem kanadyjskim w~sprawie Oka w~latach 80. oraz po wcześniejszej niemalże wojnie domowej w~samym Akwesasne, rząd kanadyjski nie zaryzykuje wysłania dużego kontyngentu wojskowego, jeśli myśleli, że prawdziwy konflikt jest prawdopodobny.

 Tym, o co naprawdę martwił się Moose, była przedwczesna reklama. A konkretnie depesza wysłana przez dwóch niezależnych anarchistów IMC, którzy przybyli z~wyprawą, Target i~Warcry. Obaj byli na swój sposób pomniejszymi legendami ruchu. Target był punkowym dzieciakiem słynącym z~wyczynów Czarnego Bloku, który wydawał się zmieniać swoje imię co dwa tygodnie. Warcry, urodzona w~Indiach, była dawną opiekunką drzew, aktywistką ekologiczną i~niezależną dziennikarką, która miała wtedy reputację anarchistycznej dziewczyny z~plakatu, która pojawiała się w~niemal każdym filmie o Seattle -- częściowo dzięki charyzmie, częściowo, ponieważ była jedną z~niewielu osób anarchistycznych Czarnego Bloku, którzy chcieli udzielać wywiadów. Po powrocie z~Kanady natychmiast napisali swoje wezwanie, które zostało przesłane do szeregu anarchistycznych listserwerów. W nim -- przynajmniej według Moose -- rażąco błędnie przedstawiali to, co się działo jako ultra-wojskowe wydarzenie zbrojne i~wzywali anarchistów do udziału. Poza tym, że jesteśmy dziecinni, podkreślił Moose, było to całkowicie poza procesem: obiecaliśmy, że nie będziemy mówić niczego, do czego nie byliśmy specjalnie upoważnieni. Powiedział, że porozmawiamy później z~Warcry, aby sprawdzić, czy nie mogą opublikować jakiejś korekty, a przynajmniej łagodniejszej wersji.

 Właściwie niektórzy ludzie zaczęli się martwić całą sytuacją. Co jest z~obrazem Wojownikiem Mohawk z~M16? Czy ci faceci naprawdę będą nosić broń? Moose zapewnił nas, że nie. W Oka zajmowali most przez dwa miesiące, zanim w~ogóle zaczęli nosić broń, a nawet wtedy nigdy nie użyli jej na nikim, nawet gdy policja do nich strzelała. Powiedział nam, że to trochę sztuczka. Ludzie, którym brakuje przywilejów białych aktywistów, nie mogą twierdzić, że prowadzą pokojowe działania, nawet jeśli w~rzeczywistości tak jest.

 Moose mówi również, że Shawn zapewniał nas, że przedostanie się przez most i~tak nie będzie problemem. Bardziej prawdopodobne jest, że problemem będą przeszkody na drodze.

 Tak więc formalnie zgadzamy się na nasze poparcie dla akcji w~Cornwall. Następnie, po kolejnym raporcie na temat NEGAN, zaczynamy mówić o większej, ogólnonowojorskiej koalicji anty-FTAA, która w~rzeczywistości borykała się z~podobnymi problemami. Koalicja jest na szczycie z~Zielonymi i~ludźmi z~ISO, a napięcia organizacyjne stały się tak wielkie, że zgodziliśmy się na specjalne spotkanie w~piątek, żeby wszystko wyjaśnić. (,,Marina jedzie. Jest królową procesu'' -- zauważa Moose. Jest także niegdysiejszą członkinią ISO, która została anarchistką, która prawdopodobnie wie, jak myślą tacy ludzie.)

 Nową wiadomością jest to, że CLAC ma ,,consultę'', czyli radę delegatów w~Quebec City 23. i~Ya Basta! musi wysłać przedstawicieli -- zwłaszcza, że  podczas ostatniej konsulty nasi ludzie się nie dostali. Zgłaszam się na ochotnika. Podobnie Emma,  artystka pracująca obecnie w~sklepie ze zdrową żywnością na Lower East Side. Emma zaznacza, że  może nie być idealnym wyborem, ponieważ będąc częścią kolektywu, nie zamierza robić Ya Basta!, ale będzie z~Czarnym Blokiem. Wydaje się, że nikomu to nie przeszkadza.

 Wybór delegatów nie jest tak delikatną sprawą, jak mogłoby być, ponieważ delegaci nie są technicznie upoważnieni do podejmowania decyzji za grupę. Tak naprawdę nie są przedstawicielami. Są w~zasadzie kanałami informacyjnymi: wyjaśniają, co ich grupa zamierza zrobić, przedstawiają propozycje i~przekazują informacje i~propozycje z~powrotem do grupy, aby mogła je wspólnie rozważyć. (Na właściwej radzie, gdzie inni członkowie grupy afinicji są obecni w~pokoju, może się to zdarzyć na miejscu. Na konsulcie, w~której ich nie ma, liczba decyzji, które można podjąć, jest znacznie bardziej ograniczona). Mimo to rodzi się pytanie: co Ya Basta! w~rzeczywistości planuje zrobić, jeśli uda nam się dotrzeć do Quebec City? Przez resztę spotkania rozważamy możliwości. Ponieważ nikt nie jest zbytnio zainteresowany ideą ochrony autonomicznej strefy SalAMI w~szczerym polu, sprowadzają się one do: (1) pomocy w~zburzeniu ściany, (2) próby przedostania się przez mur i~wejścia na obwód, lub ( 3) zapewnienie jakiejś dywersji -- ponieważ wiemy, że jeśli ubierzesz się w~jasne, watowane stroje, policja z~pewnością będzie za tobą podążać. Mur jest oczywistym symbolem hipokryzji neoliberalizmu, ale niektórzy z~nas uważają go za zbyt symboliczny. Z drugiej strony, gdybyśmy mogli dostać się do środka, co byśmy tam zrobili? Smokey słyszał historię o schronisku dla bezdomnych, które skutecznie musiało zamknąć działalność z~powodu Szczytu -- może moglibyśmy nakłonić ich, by formalnie zaprosili nas do zapewnienia bezpieczeństwa? Ktoś inny dążył do udramatyzowania losów zaginionych azylantów: Koalicja na rzecz Obrony Praw Imigrantów zasugerowała, abyśmy mogli pomyśleć o umieszczeniu nie ich zdjęć, ale serii konkretnych żądań na tarczach i~transparentach oraz dostarczaniu ich na szczyt. Ale komu? A jak je upowszechnić Amerykańskie media nigdy nie ujawniłyby tej historii. 

 To doprowadziło do długiej debaty na temat zalet i~wad działania przeciwko samym mediom. Czy byłoby możliwe, na przykład, zamknięcie namiotu medialnego przed budynkiem szczytu, a nawet zażądanie odtworzenia taśmy zawierającej głosy nieobecnych na debacie? Wszyscy zgadzają się, że media korporacyjne są uzasadnionym celem, ale jak skuteczne byłyby działania przeciwko nim? Co stanowiłoby sukces? To podstawowe pytanie, które ciągle sobie zadajemy, planując akcję: jak ułożyć wydarzenie w~taki sposób, abyśmy mieli później prawo ogłosić zwycięstwo? A w~przypadku mediów było to szczególnie dotkliwe: nawet gdybyś przeprowadził skuteczną akcję przeciwko mediom, kto by o tym wiedział?

 Nie podejmujemy w~końcu żadnych decyzji. W każdym razie, jak kilka osób wskazuje, jesteśmy tylko jednym kolektywem. Inne grupy Ya Basta! dołączą do Burlington i~nie chcemy podejmować za nich decyzji. Możemy to zachować na radę w~Burlington. Ale kiedy spotkanie się kończy, zarówno Emma,  jak i~ja mamy już całkiem jasne pojęcie o tym, co zamierzamy powiedzieć.

 Ostateczna zapowiedź. 
 
 -- Mam powiedzieć ludziom, że Starhawk jutro będzie w~mieście. -- mówi Moose (wymawia to z~nutą łagodnej kpiny: Staaaarhawk.) -- To znaczy, ja nie jestem zbyt przygnębiony tego rodzaju bzdurami o gwiazdach, ale najwyraźniej chce poznać niektórych z~nowojorskiego kolektywu Ya Basta! więc pomyślałem, że przekażę dalej.

\section{Piątek, 9 marca}
\noindent \textbf{ Spotkanie struktur koalicyjnych w~Amsterdam Pizza przy 111th Street, 18:30}\medskip


 To spotkanie zebrało około dwudziestu aktywistów ustawionych wokół stołu na tyłach pizzerii, udających, że się nie kłócą.

 To, co obecnie nazywa się Koalicją FTAA, zaczęło się jako grupa robocza Sieci Akcji Bezpośrednich DAN. W każdą niedzielę DAN odbywała generalne zebranie i~całą serię spotkań grup roboczych w~inne dni tygodnia. Niektóre z~tych grup roboczych są strukturalne (prawne, medialne, informacyjne), niektóre są zaangażowane w~trwające kampanie (DAN Pracy, Policja i~Więzienia), ale zawsze są takie, które są tworzone tylko po to, aby pracować nad konkretnymi działaniami: na protesty IMF w~Waszyngtonie, konwencję republikanów w~Filadelfii, a teraz FTAA w~Quebecu. Często te ostatnie grupy robocze mogą same zacząć wyglądać jak miniaturowe wersje DAN, z~własnymi grupami roboczymi zajmującymi się zasięgiem, komunikacją, transportem i~tym podobnymi. Stają się pączkującymi strukturami komórkowymi, a następnie odtwarzają te same wewnętrzne relacje między częściami. Mogliśmy sobie pozwolić na elastyczność, bo przecież nie było ustalonego, odgórnego łańcucha dowodzenia; inicjatywy i~tak miały powstawać od dołu; więc każdy mógł swobodnie improwizować, niezależnie od formy organizacyjnej, która wydawała się dla niego odpowiednia.

 Problemy pojawiły się, gdy DAN próbowała pracować z~członkami grup o głęboko odmiennych imperatywach organizacyjnych. Wspomniałem już o wielkim bugaboo DAN, ISO. ISO zaangażowała się w~politykę w~stylu DAN dopiero niedawno. Grali niewielką lub żadną rolę w~Seattle. Jednak jakiś czas później najwyraźniej otrzymali rozkazy z~centralnego dowództwa w~Anglii, aby zaangażować się w~globalny ruch sprawiedliwości. Nagle na spotkaniach DAN zaczęli pojawiać się różnego rodzaju wysokiej rangi organizatorzy ISO. Ich entuzjazm zdawał się słabnąć i~płynąć. Z entuzjazmem uczestniczyli w~pierwszej dużej akcji NYC-DAN -- A16, protestach przeciwko IMF w~Waszyngtonie 16 kwietnia 2000 roku -- ale po protestach na konwencji republikańskiej w~Filadelfii, podczas której kontyngent ISO był powszechnie oskarżany o porzucenie pozycji i~ucieczkę, w~dużej mierze odpadli i~włożyli swoją energię w~kampanię prezydencką Nadera. Teraz wrócili.

 Grupy robocze były w~zasadzie otwarte. Każdy mógł dołączyć. W tym przypadku, kiedy DAN utworzyła grupę roboczą dla mobilizacji FTAA w~styczniu, ludzie z~ISO nagle pojawili się ponownie, wraz z~członkami kilku innych grup, z~którymi współpracowali -- Partii Zielonych, pewnych organizacji pozarządowych -- którzy nigdy, faktycznie, nie byli na spotkaniu DAN. Ponieważ ISO i~Zieloni nie byli tam przynajmniej jako jednostki, ale jako przedstawiciele organizacji, grupa robocza w~efekcie stała się koalicją. Wydawało się więc, że rozsądnym jest ogłosić ją jako taką i~porzucić udawanie, że jest częścią DAN. Nie stanowiło to problemu, ponieważ grupy robocze DAN i~tak były w~dużej mierze autonomiczne.
 
 Więc teraz mieliśmy ogólnomiejską koalicję, która rzekomo działała na zasadach anarchistycznych, a przynajmniej demokracji bezpośredniej. W zasadzie tego właśnie powinna chcieć DAN: wszyscy mieliśmy na celu rozpowszechnianie tego rodzaju modelu podejmowania decyzji. Jednak nastąpiło nieuniknione zderzenie kultur instytucjonalnych. Nowo przybyli natychmiast zaczęli traktować koalicję jak nową organizację: chcieli przyjąć stanowisko co do jedności, stworzyć literaturę i~starać się o przyłączenie innych grup w~mieście, grup imigrantów, związków zawodowych i~tym podobnych. Anarchiści w~ogóle nie myśleli o koalicji jako o ,,grupie''. Postrzegali ją nie jako organ decyzyjny, ale raczej jako forum, sposób dla grup już się organizujących przeciwko FTAA do wymiany informacji i~unikania powielania wysiłków. To było coś na wzór rad delegatów. Z pewnością nie widzieli powodu, by przyjąć jakąkolwiek ideologiczną ,,linię''. Z jakiegoś powodu, wiele dyskusji kończyło się próbami przekonania jednej osoby: młodej kobiety imieniem Julie, która pracowała dla czegoś o nazwie Urban Justice League. Częściowo działo się tak, ponieważ nikt jej naprawdę nie znał, wydawała się nową osobą na scenie, ale bardzo entuzjastyczną, aktywną i~chętną do uczenia. Z drugiej strony Julie okazała się stworzeniem ze świata organizacji pozarządowych i~ostatecznie przeszła zdecydowanie na stanowisko ISO. W istocie wkrótce zaczęła zachowywać się jak jednoosobowy komitet sterujący, dzwoniąc w~naszym imieniu do prezydentów związków, pastorów i~przywódców różnych grup społecznych i~próbując zebrać jak najszerszą koalicję. Teoretycznie trudno było się z~tym spierać. Ale wszyscy wiedzieliśmy, co może nastąpić dalej: te same grupy zaczęłyby domagać się, abyśmy złagodzili akcję bezpośrednią lub przynajmniej przestali o niej otwarcie mówić. Ludzie DAN i~inni anarchiści odpowiedzieli, tworząc własną autonomiczną grupę roboczą akcji bezpośredniej koalicji -- nazywając ją odpowiednio ,,autonomiczną grupą roboczą akcji bezpośredniej'' lub AUTODAWG -- z~własnym listserv i~oddzielnymi spotkaniami. Zdecydowaliśmy, że AUTODAWG wysyłałaby co tydzień jednego przedstawiciela na spotkania koalicyjne, ale poza tym pracowalibyśmy razem, podobnie jak oczekiwaliśmy, że zrobi to pierwotna grupa robocza DAN.

 Problem był taki, że Julie i~ludzie z~ISO natychmiast zaczęli pojawiać się też na spotkaniach AUTODAWG. Technicznie, oczywiście, nie było przeciwwskazań, to były otwarte spotkania, ale powodowało to duży dyskomfort u wszystkich. Julie zaczęła narzekać w~Internecie na wykluczenie i~szybko wszyscy zgodzili się, że powinniśmy się specjalnie spotkać, żeby omówić proces i~dopracować szczegóły.

 W rezultacie około dwudziestu osób siedziało wokół stołu na spotkaniu studenckim w~pobliżu Uniwersytetu Columbia, dzieląc się kawałkami dwóch dużych pizzy z~serem i~starając się zachować rozsądek w~stosunku do siebie. Julie zaproponowała facylitację (co prawdopodobnie nie było dobrym pomysłem). Wkrótce stało się jasne, że głównym problemem był brak zaufania do instynktów drugiej strony, ponieważ w~zasadzie strona ISO przedstawiała kilka bardzo rozsądnych uwag. Przede wszystkim, powiedzieli, weźcie pod uwagę nowych ludzi. Pojawiło się wiele nowych osób, zwłaszcza studentów, którzy chcieli zrobić akcję bezpośrednią. Jak dokładnie mieli się przyłączyć i~zdecydować, do jakiej grupy roboczej chcą dołączyć, jeśli ludzie zajmujący się akcją bezpośrednią spotykają się w~zupełnie innym czasie i~miejscu? Po drugie, jeśli zamierzacie stworzyć koalicję, w~skład której wchodzą związki zawodowe i~zorganizowane grupy społeczne, a wy sięgacie, to oni \textit{będą }chcieli zobaczyć jakąś misję. Nie możesz im po prostu powiedzieć, że jesteś przeciwko FTAA. Oczywiście anarchiści obecni na sali mogliby odpowiedzieć, pytając, jaki był sens w~uzyskiwaniu tych aprobat: żadna z~tych grup nie była zainteresowana udziałem w~akcji, a każda grupa, która mogłaby być zainteresowana wysyłaniem ludzi na Quebec, aby maszerować w~paradzie robotniczej, prawie na pewno już się przygotowywała. Po co więc kolekcjonować nazwiska tylko po to, by mieć je na kartce papieru? Nikt jednak nie chciał wykluczyć możliwości, że \textit{jakaś }nowa grupa zostanie wciągnięta i~zdecyduje się przyjąć bardziej radykalną postawę. Więc zamiast tego skończyliśmy niekończącą się rozmowę. 

\noindent  Julie: Są dwie kwestie: Po pierwsze, jak integrujemy się z~innymi organizacjami, do których docieramy? Podpisanie oświadczenia o misji jest wypróbowaną i~dobrą metodą na to. Tak czy inaczej, ludzi w~ogóle, osoby, które nie są częścią organizacji, nie będą w~stanie dopasować się modelu rad przedstawicielskich. Po drugie, jak uniknąć powielania wysiłków. 

\noindent  Moose: Ale ideą koalicji nie jest posiadanie ideologii; jest to sposób dla ludzi o różnych ideologiach lub perspektywach do wspólnej pracy nad problemem.

\noindent  Zielony: Chcę wiedzieć, co aprobata oznaczałaby w~praktyce. Czy związki rozdadzą naszą ulotkę swoim członkom? Jeśli tak, to ułatwiłoby to jednostkom dołączanie jako jednostki. 

\noindent  Meredith (ISO): Cóż, Grupa Robocza postanowiła już napisać oświadczenie i~przekazać je nam. Wydaje mi się, że pojawia się pytanie, co to znaczy być częścią koalicji?

\noindent  Julie: Tak, dokładnie. Kiedy powstał AUTODAWG, nigdy nie było dla mnie jasne, czy jest częścią koalicji, czy nie. Potem kiedy pojawiłam się na jednym z~ich spotkań, poczułam się, jakbym rozbijała imprezę.

\noindent  Enos: Słuchaj, rozumiem, że mogłaś się tak czuć. Ale myślę, że jednym z~powodów, dla których tak się stało, było to, że za każdym razem, gdy tworzyliśmy autonomiczną grupę roboczą ds. akcji bezpośredniej, wydawało się, że wszyscy w~całej koalicji się pojawiali. Zaczęliśmy więc zadawać sobie pytanie: w~jaki sposób jesteśmy autonomiczni? W jaki sposób jesteśmy inną grupą? Pamiętajcie, wszystko zaczęło się, gdy DAN zdecydowała, że  chce pracować nad FTAA i~stworzyła własną grupę roboczą. Potem kiedy wszyscy ci ludzie się pojawili, ta grupa robocza faktycznie stała się tą koalicją. Tak więc główny DAN był zdezorientowany i~próbowaliśmy stworzyć nową grupę roboczą. i~to się po prostu działo.

\noindent  Marina: Powinnaś coś zrozumieć na temat tego, jak DAN ma tendencję do działania, ponieważ częścią problemu może być po prostu zamieszanie. Ludzie, z~którymi normalnie pracujemy, ludzie z~Lower East Side Collective, Reclaim the Streets, Ya Basta!, wszystkie te grupy postrzegają siebie jako luźną część DAN. Jesteśmy w~połowie drogi między grupą a siecią delegatów. Na przykład ludzie Reclaim the Streets, którzy nigdy nie przychodzą na nasze spotkania, częściowo dlatego, że są bardziej zainteresowani lokalnymi sprawami Nowego Jorku, częściowo dlatego, że nie lubią spotkań, są imprezowiczami kochającymi zabawę, to część ich całego szyku. Ale zawsze pojawiają się na akcjach. Tak naprawdę była to grupa robocza dla tej większej społeczności zorientowanej na bezpośrednie działania. Niektórzy ludzie nie chcieli, aby była to grupa robocza DAN, więc powiedzieliśmy, w~porządku, nazwijmy ją po prostu ,,autonomiczną''.

\noindent  Meredith: Może, żeby przedstawić tutaj propozycję, dlaczego nie zrobić listy grup roboczych, które możemy publikować na ścianie podczas spotkań, aby nowe osoby mogły się podłączyć? Jak czuliby się ludzie?

\noindent  Marina: Myślałam, że przyszliśmy tutaj, aby przeprowadzić burzę mózgów, aby wrócić do naszych grup. Nie czuję się komfortowo, czyniąc z~tego ciało decyzyjne. Nie zrozum mnie źle, to konstruktywna propozycja\ldots 

\noindent  James: Ona po prostu sugeruje, żebyśmy lepiej wyrażali to, co robimy. Nazywanie tego ,,podejmowaniem decyzji'' wydaje się kwestią semantyki, z~miejsca, gdzie słucham.

\noindent  Enos: Nie sądzę, że to tylko semantyka. Myślę, że problemem jest odmienny charakter zaangażowanych grup.

\noindent  Maggie: Chcę tylko wiedzieć, co powiedzieć ludziom, którzy chcą do nas dołączyć\ldots 

 I~tak dalej, najwyraźniej \textit{ad infinitum}. Wyszedłem wcześnie, częściowo dlatego, że chociaż często podnosiłem rękę, Julie ani razu mnie nie zawołała; częściowo dlatego, że kilku z~nas powiedziało Starhawk, że przyjedziemy o ósmej. Mieszkała u przyjaciółki o imieniu Nesta w~Columbia University Housing, zaledwie kilka przecznic dalej.

\noindent \textit{Spotkanie ze Starhawk, godz. 20:00}

 O wiele przyjemniejsze spotkanie. Nawet inspirujące.

 Prawie każdy aktywny w~ruchu przynajmniej słyszał o Starhawk. Była kiedyś pisarką science fiction (jej najsłynniejsza powieść dotyczyła wojny między San Francisco a Los Angeles), kiedyś autorką prac na temat feministycznego pogaństwa, która od końca lat 70. była zaangażowana w~kampanie akcji bezpośrednich. Prawie wszyscy widzieli jej obrazy, jak bije w~mały bębenek, prowadzi spiralne tańce. Praktykująca czarownica, miała reputację swego rodzaju kokoszy dla pogańskiego klastra, wiele wicca po czterdziestce lub pięćdziesiątce, ale w~tym też wiele znacznie młodszych członkiń. Większość z~nas podeszła na spotkanie bardzo sceptycznie. Nie chodziło tylko o automatyczne podejrzenia co do celebrytów ruchu, czy nawet stosunek Wschodniego Wybrzeża do rzekomo niestabilnych Kalifornijczyków. Jedyną rzeczą, którą większość z~nas czytała przez Starhawk, był artykuł, który napisała w~szeroko rozpowszechnionej kolekcji, która ukazała się zaraz po Seattle, zatytułowanej ,,Jak naprawdę zamknęliśmy WTO'', w~której skarciła Czarny Blok za odmowę podjęcia działań, przeciwko zasiadaniu w~radzie delegatów, łamiąc uzgodnione kodeksy postępowania, a nawet wypowiadała się z~aprobatą o pacyfistach, którzy wskazywali policji rozbijających okna. Artykuł, wraz z~jeszcze bardziej gniewnymi wypowiedziami działaczy organizacji pozarządowych, takich jak Medea Benjamin, wywołał prawdziwą eksplozję wściekłości ze strony bardziej bojowych anarchistów. Wściekłość w~końcu doprowadziła do debaty: o solidarność, o taktykę, o to, co aktywiści są sobie nawzajem winni na ulicach. Wiele osób zmieniło zdanie, między innymi Starhawk, ale w~tamtym czasie jej wizerunek utrwalił się w~pamięci wszystkich -- zwłaszcza, że  w~przeciwieństwie do postaci takich jak Medea Benjamin, której opinia mogła być odrzucona, Starhawk uważała się za anarchistkę -- co nadawało tej sprawie odcień osobistej zdrady.

 W każdym razie byliśmy podejrzliwi. Mimo to przyjechaliśmy. Co najmniej pięcioro z~nas: Moose, Marina, Rufus, Warcry i~ja. Mimo to wszyscy byli skłonni przyznać, że grupa afinicji Starhawka, kolektyw RANT, robiła świetną robotę, prowadząc szkolenia w~całym kraju. Pod koniec wieczoru byliśmy prawie całkowicie przekonani.

 Częściowo dlatego, że się tak przeciwstawiła oczekiwaniom. Nie wiem dokładnie, czego się spodziewaliśmy, ale przynajmniej można sobie wyobrazić, że anarchistyczna wiedźma byłaby przynajmniej trochę \textit{outré}. Zamiast tego spotkaliśmy jedną z~najprzyjemniejszych, najbardziej rozsądnych osób, jakie można sobie wyobrazić. Wszystko w~niej było otwarte, przyjazne i~całkowicie przyziemne.

 Starhawk mieszkała ze swoją przyjaciółką Nestą, znaną teoretyczką ekofeministyczną i~okazjonalną pisarką \textit{Nation}, która była mniej więcej w~tym samym wieku, a obecnie poruszała się na niezwykle zaawansowanym technologicznie wózku inwalidzkim. Starhawk była ciekawa sceny akcji bezpośredniej w~Nowym Jorku. Moose mówił o Ya Basta!, Marina o Kolektywie Prawa Ludowego, Rufus o Medykach Akcji.

 Starhawk opowiedziała o swoim własnym doświadczeniu: 
 
 -- Byłam jedną z~tych osób, które pojechały do  Seattle, aby wypełnić swój obywatelski obowiązek, a potem spodziewałam się, że po prostu wrócę ponownie do swojego życia. Oto dwa lata później i~jeszcze nie wróciłem.

 Nesta szybko zauważyła, że  to nie tak, że nie miała żadnego doświadczenia w~tego typu sprawach. Naprawdę, mieli swój początek w~blokadzie Diablo Canyon w~1981 roku.
 
  -- Pamiętasz, jak musieliśmy wymyślać to wszystko od zera? Nie mieliśmy pojęcia, co robimy, jak robić rzeczy, które obecne pokolenie po prostu uważa za oczywiste.

-- Och, nastąpił \textit{ogromny }postęp -- zgodziła się Starhawk. -- Nie potrafię powiedzieć, ile razy widziałem dzieci w~wieku szesnastu, siedemnastu lat, a one już wiedziały, jak robić rzeczy, których zrozumienie zajęło nam piętnaście lat.

-- Cóż, jeśli chcecie poznać historię\ldots  Byłam wtedy w~zasadzie autorką. Napisałam kilka książek o pogaństwie, religii Bogini. Sieć, z~którą jestem, Reclaiming, opiera się na zasadzie magicznego aktywizmu, chcieliśmy użyć magii jako sposobu na przekształcenie świadomości, aby dodać duchowy wymiar, który nie był po prostu chrześcijański. Ponieważ na początku tylko kwakrzy naprawdę wiedzieli, jak to zrobić. Rady rzeczników, grupy afinicji -- wszystko to tak naprawdę zaczęło się od Clamshell Alliance, działającego przeciwko elektrowni jądrowej Seabrook w~New Hampshire. Była taka buntownicza grupa kwakrów zwana Ruchem na rzecz Nowego Społeczeństwa, która prowadziła szkolenia na temat niestosowania przemocy, ale także uczyła tego nowego sposobu organizowania -- konsensusu, rad rzeczników, jak podejmować decyzje demokratycznie w~małych grupach, a następnie umożliwiać im koordynowanie się oddolnie. i~zadziałało tak dobrze, że po prostu wystartowało. Na początku toczyła się swoista bitwa między starymi a nowymi sposobami robienia rzeczy. Większość z~tych kampanii nadal miała opłacanych pracowników, zwykle nielicznych słabo opłacanych pracowników, ale mimo to opłacanych pracowników, i~to, co w~rzeczywistości było komitetami sterującymi, i~zawsze istniały napięcia między zasadą od góry do dołu a od dołu do góry.

 Mamrocze ,,to nie do końca tak, że takie rzeczy nigdy się już nie zdarzają''.

-- I, jak mówiłam, pojawił się problem, no cóż, nazywaliśmy ich czasami ,,kwakierskimi faszystami'', których rodzaj duchowości był prawie całkowicie obcy naszemu.

-- Byłam częścią grupy, którą nazwaliśmy Matrix Collective, która była częścią Reclaiming. Najpierw zaangażowaliśmy się w~blokadę w~Diablo Canyon, co było szalonym pomysłem, że musieli zbudować elektrownię jądrową bezpośrednio na linii uskoku w~Kalifornii, a później w~Lawrence Livermore Group, która była jednym z~głównych laboratoriów broni jądrowej w~Berkeley. Chcieliśmy użyć tej samej poziomej struktury, której używali w~Seabrook, ale chcieliśmy też zrobić rytuał. W pewnym sensie oznaczało to wymyślenie wszystkiego na nowo, ponieważ wkrótce zdaliśmy sobie sprawę, że tradycyjne pokojowe nieposłuszeństwo obywatelskie jest głęboko zakorzenione w~etosie chrześcijaństwa, a przynajmniej w~skrajnie patriarchalnej wersji religii. Istniał powód, dla którego na czele ruchu zawsze stał jakiś męski bohater religijny. Chodzi o karę, samozaparcie, chęć poddania własnego ciała bólowi i~cierpieniu w~imię idei, która prawdopodobnie jest Prawdą lub Miłością lub czymś bardzo miłym w~tym stylu, ale jednak czymś abstrakcyjnym, transcendentnym. Negujesz cielesność w~imię czegoś wyższego. Na tym właśnie polegają wielkie religie świata. Jak więc pogodzić to z~immanentną kosmologią, która celebruje ciało i~postrzega przyjemność, zwłaszcza seksualną, jako boską?

 Zostawiła pytanie otwarte.
 
  -- Nie wiem, czy ktokolwiek z~nas naprawdę się do tego domyślił. Jednym z~pomysłów, jaki mieliśmy, było wydobycie źródeł siły z~pozornej słabości, pokazanie, jak mało pospolite rzeczy, takie jak przędza, mogą, utkane razem, coś w~rodzaju zaklęcia, zatrzymać nawet maszyny wojskowe. Pamiętasz te wszystkie wstęgi przędzy z~A16?

-- Och, masz na myśli te na wszystkich skrzyżowaniach, więc musiałeś się pod nimi czołgać, żeby dostać się tam i~z powrotem między blokadami? -- zapytałem

-- To był wkład Klastra Pogańskiego. Właściwie po raz pierwszy pamiętam, że użyłam włóczki na akcji Bohemian Grove we wczesnych latach osiemdziesiątych. To był ekskluzywny, męski klub, w~skład którego wchodzą prezesi i~wiele osób z~gabinetu Reagana, prawdopodobnie niektórych z~gabinetu Busha, ale w~latach Reagana wydawało się to bardziej pilne. Mają klub w~centrum San Francisco i~fantazyjny letni kurort nad rzeką Russian, gdzie co roku organizują tygodniowy obóz letni dla bogatych i~wpływowych, który rozpoczynają rytuałem zwanym Kremacją Opieki, w~którym palą kukłę kobiety. Ich motto brzmi: ,,Tkające pająki tu nie przychodzą'' (\textit{nie }zmyślam tego!), więc zrobiliśmy akcję bezpośrednią, w~której powiązaliśmy sieć z~całym klubem Boho w~centrum SF i~zablokowaliśmy ich.

-- To wspaniałe.

-- Tak, to zabawne -- powiedziałem. -- Zawsze zakładałem, że to tylko paranoidalny spisek.

-- W takim razie prawdopodobnie najlepszą historią tkacką była akcja w~Livermore w\ldots  1982? 1983, prawda?

-- Och, to było takie zabawne! -- powiedziała Nesta, która właśnie przyjechała z~drugiego pokoju. -- Pamiętam tę historię. Kobiety utkały długą pajęczynę, jak osnowę, prawda?, na dwóch patykach, które mogły rozciągać się w~poprzek drogi, wplatały zdjęcia dzieci, kwiatów, ziół itp. i~używały jej do blokowania autobusów robotniczych\ldots  

-- I~pomyślałyśmy o tym jako o zasadniczo symbolicznym geście, tak naprawdę artystycznej wypowiedzi. Nic, co faktycznie byłoby fizycznie skuteczne. Prawie skończyliśmy i~pamiętam, że trzej policjanci na rowerach szydzili z~nas. Nagle odpalili silniki i~postanowili po prostu przez to przejechać. Następną rzeczą, o której dowiedzieliśmy się, było to, że byłyśmy na ziemi, a na ziemi leżeli trzej gliniarze i~ich motocykle, i~wszyscy byliśmy tak beznadziejnie splątani, że zabrało dziesięć minut, żeby nas odciąć. 

-- Pamiętasz Bork z~RNC w~Filadelfii? -- ktoś zapytał. -- Pamiętasz, ta, która pojawiła się następnego dnia na konferencji prasowej z~podbitymi oczami i~siniakami na twarzy? Powód, dlaczego tak bardzo ją pobili w~Filadelfii\ldots  cóż, byli tam gliniarze na rowerach z~dużymi nożycami. Za każdym razem, gdy widzieli jeden z~naszych banerów, wyciągali nożyczki i~przejeżdżali przez nie rowerami. Z wyjątkiem Bork, najwyraźniej nie miała pojęcia, że  będą to robić, ale wzmocniła swój sztandar drutem fortepianowym. 

-- Au!

-- Gdyby próbowali przejechać na poziomie szyi, mieliby duże kłopoty. A tak, dwóch z~nich miało bardzo nieprzyjemne rany. (To znaczy -- nigdy by ich nie mieli, gdyby nie próbowali nielegalnie niszczyć znaków protestujących). Ale tak czy inaczej, zsiedli z~rowerów i~zaczęli walić jej głową o ziemię.

 Po chwili wszyscy wymieniali się historiami wojennymi. Starhawk, jak się okazała, była wściekła na Black Bloc w~Seattle głównie dlatego, że nie uszanowali procesu kolektywnego, odmówili nawet uczestniczenia w~radach delegatów. Od tamtego czasu, w~końcu zaakceptowała zasadę zróżnicowania taktyk.
 
 -- Prowadziłyśmy treningi niestosowania przemocy -- powiedziała. -- Teraz nawet ich tak nie nazywamy. Prowadzimy coś, co nazywamy treningami akcji bezpośredniej, z~klasycznym niestosowaniem przemocy jako jednym z~elementów znacznie szerszego repertuaru. W końcu to odmowa wyrządzania krzywdy lub cierpienia innym jest moralnym stanowiskiem, szczególnie z~jakiejkolwiek duchowej perspektywy, która ma dla mnie sens. -- Marina starała się delikatnie zasugerować, że może poważnie zastanowić się nad tym, aby ta jej zmiana opinii była szerzej poznana.

 Jednym z~powodów, dla których Starhawk bała się spotkania z~nowojorską Ya Bastą!, przyznała w~końcu, było to, że trochę się martwiła, czy taka taktyka może się przełożyć po drugiej stronie Atlantyku. Po raz pierwszy spotkała włoską Ya Basta! przed akcjami przeciwko spotkaniom IMF/Banku Światowego w~Pradze jesienią 2000 roku. Praga była pod wieloma względami po prostu nadzwyczajna. Wyszła wcześniej, aby przeprowadzić szkolenie na temat procesu konsensusu, i~skończyła jako facylitator jednej z~dużych rad delegatów. 
 
 -- I~to była jedna z~tych sytuacji, w~których\ldots no wiesz, jak to się musi skończyć. Były cztery różne grupy i~dwie propozycje. Albo byłyby cztery marsze, wszystkie zaczynałyby się w~różnych miejscach i~wszystkie się gdzieś zbiegały, albo byłby jeden marsz i~wszystkie rozgałęziały się na cztery. i~naprawdę tylko jeden wynik był możliwy: zaczęlibyśmy razem, rozdzielili się, i~mam nadzieję, że jeśli wszystko pójdzie dobrze, w~końcu się zebrali. Ale, oczywiście, najpierw musieliśmy przeanalizować każdą możliwość, każdą możliwą troskę lub sprzeciw przez cztery, pięć godzin, aż w~końcu doszlibyśmy do wyniku, o którym wszyscy musieli wiedzieć, że w~końcu go wymyślimy. Pod koniec byłam po prostu wykończona i~praktycznie myślałam, po co to wszystko? A potem ten Rumun podszedł do mnie i~,,Nie mogę uwierzyć w~to, co się właśnie stało. Nigdy bym nie pomyślał, że coś takiego jest możliwe, tysiąc ludzi mówiących dwunastoma różnymi językami w~jednym pokoju, podejmujących wspólnie decyzje, bez przywódców''. Był po prostu zachwycony. Może czasami zapominamy, jak wiele z~tego naprawdę jest rewolucyjne. 

-- A więc miałeś wiele wspólnego z~Ya Bastą! w~Pradze? -- ktoś zapytał. 

To był ich debiut na większej europejskiej scenie i~zagrali spektakularnie, kończąc słynną konfrontacją z~policją na moście prowadzącym do Centrum Kongresowego, gdzie spotykał się IMF, co wszyscy wielokrotnie oglądaliśmy na wideo.

-- O tak. Szczerze mówiąc, na początku raczej dawali mi spokój. Po części był to rażący seksizm. Przez trzy dni spotkań jeden facet, Luka, prowadził wszystkie rozmowy. Mówił trochę po angielsku, ale głównie po włosku. Dalej była kobieta, która wykonała całą pracę tłumaczeniową, to znaczy trzy dni tłumaczenia symultanicznego, nie sądziłem, że ktokolwiek mógłby tak pracować przez trzy dni bez szaleństwa, i~trzecia, również kobieta, która po prostu siedziała i~robiła notatki. Nigdy się nie zmieniali, nigdy nie zamieniali ról. Było oczywiste, że obie kobiety mówiły perfekcyjnie po angielsku, ale przez cały czas nie odważyły  się na jedno zdanie. Wewnętrznie, w~Ya Basta!, również nie mogłem dostrzec żadnego wewnętrznego procesu demokratycznego. Może działy się rzeczy, o których nie zdawałam sobie sprawy.

 Potem weszliśmy na temat taktyki. Po trzech dniach spotkań Ya Basta! w~końcu zdecydowali, że ich linia frontu będzie uzbrojona w~kantówki. Starhawk zaczęła mówić bardzo powoli i~precyzyjnie. 
 
 -- Aby pokonać tarcze gliniarzy. Nie, żeby ich naprawdę uderzać. Pomysł polegał na tym, że mogliby w~ten sposób przebić się przez linie policyjne i~tak naprawdę nie atakowaliby policji.

 Oczy zamrugały.

 Usta się otworzyły.

-- Naprawdę przynieśli drewniane drągi?

-- Dojście do porozumienia zajęło nam kilka dni.

-- Jeju -- powiedział Moose. -- To znaczy, w~porządku\ldots  widzę logikę, ale\ldots  nikt z~nas nigdy nie marzył o zrobieniu czegoś takiego.

-- Szczerze mówiąc -- powiedziała Starhawk -- cieszę się, że to słyszę. Ponieważ kiedy po raz pierwszy usłyszałem, że Amerykanie zamierzają zastosować taktykę Tute Bianche, trochę się zmartwiłam, że ludzie mogą zostać poważnie zranieni. Trzeba mieć na uwadze, że zajęło im pięć, może sześć lat, zanim mogli zrobić coś takiego w~Europie. Sześć lat nieustannej pracy w~mediach, wypracowywanie idei zasadności taktyk defensywnych, niekończące się akcje medialne. A trzeba pamiętać, że media we Włoszech są tysiąc razy bardziej przychylne ruchom społecznym niż media tutaj. Nawet w~telewizji, która prawie w~całości należy do Berlusconiego, Luka pojawia się za każdym razem, gdy odbywa się duża akcja, w~talk-show, dyskutując z~policją lub prawicowymi dziennikarzami, rzeczy, które byłyby nie do pomyślenia w~tym kraju.

-- Och. Wiesz, skoro to ja zajmuję się mediami dla nowojorskiej Ya Basta! -- powiedziałem -- właściwie trochę się tym martwiłem. Rozważaliśmy różne metody medialne. Ale w~zasadzie tutejsza prasa zawsze pozwala glinom opowiedzieć historię, i~nie ma sposobu, aby nawet poruszyć temat, powiedzmy, filozofii stojącej za naszymi działaniami. Uwierz mi, próbowałem. Nie ma zainteresowania. Pytają nas tylko, czy będziemy ,,agresywni'', z~ochraniaczami i~tarczami jako dowodem, że szukamy walki. Próbowaliśmy stworzyć ten sam efekt, po prostu będąc przesadnie głupkowatymi, z~zespołem kazoo, głupimi herbami i~kostiumami, tak, że jeśli ludzie po prostu widzą nas w~telewizji i~nazywają nas agresywnymi, to będzie oczywiste, że coś jest nie tak. Ale nawet wtedy wiemy doskonale, nawet jeśli wszyscy przebieramy się za Dinozaura Barneya z~rękami związanymi za plecami, dobry redaktor może znaleźć ujęcie, na którym wyglądamy przerażająco.

-- Dodatkowo w~prawie wszystkich krajach europejskich stosunki z~policją są inne. Wszyscy się znają. Całość przypomina trochę grę.

 I~tak dalej. Odeszliśmy do innych tematów, ale Starhawk wypowiedziała jej obawy. Powtórzyły się te, które z~pewnością przyszły mi do głowy w~takim czy innym czasie. Nie miałem pojęcia, czy cokolwiek z~tego rzeczywiście zadziała.

\section{Wtorek, 13 marca}
\noindent \textbf{ Spotkanie AUTODAWG w~National Lawyer's Guild, 20:00}\medskip


 To było właściwie pierwsze spotkanie z~DAWG, na które wszyscy narzekali.

 Zaczęło się od raportu. Dwóch aktywistów z~Brooklynu właśnie wróciło z~Québec City i~wszyscy byli zachwyceni pięknem miasta, jego starymi wieżami i~anarchistycznym graffiti. Następnie Mac przeszedł do przełomowych wydarzeń. Pozostał na spotkaniu w~pizzerii aż do gorzkiego końca i~zrobił wszystko, co w~jego mocy, aby załatwić sprawę z~ISO, która z~kolei chciała nas teraz zapewnić, że są całkowicie zaangażowani w~sprowadzanie ludzi do Quebecu na akcję bezpośrednią w~piątek, a nie tylko marsz związków zawodowych następnego dnia. Ostatnie wydarzenia z~Akwesasne również były obiecujące: Kanadyjski Związek Pracowników Pocztowych był zainteresowany pomocą, także niektórzy pracownicy fabryk; Warclub, hip-hopowy zespół Mohawk, chciał być w~jakiś sposób zaangażowany; nasi sojusznicy Warrior po stronie kanadyjskiej byli już w~kontakcie z~rodziną Boots,jedną z~najważniejszych rodzin Mohawk po stronie USA i~wydawali się zainteresowani, i~tak dalej.

 Na samym spotkaniu były dwa główne tematy. Pierwszym z~nich była CLAC Consulta: w~rezultacie powierzono mi odpowiedzialność za koordynację całości. Wywiązała się dyskusja na temat najbezpieczniejszych opcji: pociąg, autobus, samochód.

 Drugim była akcja zaplanowana na 1 kwietnia. Enos, lokalny podziemny rysownik, brał w~tym udział z~przyjacielem imieniem Nicky. Udało im się również z~powodzeniem narysować aktywistkę o imieniu Twinkie i~był to mały zamach stanu. Twinkie była androgyniczną młodą kobietą, której rodzice pochodzili z~Tajlandii, może miała dziewiętnaście lub dwadzieścia lat, z~dramatyczną punkową fryzurą, słynącą z~wielu rzeczy, ale prawdopodobnie przede wszystkim z~ogromnej pojemności płuc. Była bardzo poszukiwana na demówkach ze względu na jej niesamowitą moc głosu, nie wspominając już o umiejętności wymyślania piosenek i~sloganów na każdą okazję, na miejscu. Tacy ludzie są, jak można sobie wyobrazić, ogromnym atutem każdej demo. W przeszłości w~dużej mierze unikała DAN, woląc pracować z~grupami bardziej zorientowanymi na społeczność, ale zdecydowała się zaangażować w~organizowanie FTAA. Miała również spore doświadczenie w~projektowaniu graficznym.

\noindent Enos: Pomyśleliśmy, że skoro NEGAN będzie się spotykać 31-go w~Burlington, możemy udać się stamtąd do granicy następnego dnia, co oczywiście jest również dniem Prima Aprilis. Zasadniczo jest to rodzaj akcji reklamowej, rzecz medialna, aby zwrócić uwagę ludzi na problemy, ale także na fakt, że oni systematycznie powstrzymują aktywistów politycznych przed przedostaniem się do Kanady. i~to nie tylko zawracają ludzi z~mołotowami, ale zwykłych działaczy społecznych.

\noindent Mac: W zeszłym tygodniu odmówili wjazdu Lorenzo Komboa Ervin, na podstawie jakiegoś aresztowania trzydzieści lat temu.

\noindent Enos: Jeśli myślą, że działasz politycznie, przejrzą twoje dane i~jedyne, czego potrzebują, to trafienie na jedno aresztowania i~mogą odmówić ci wstępu. To nie musi być nawet przekonanie. W niektórych przypadkach odmawiają ludziom wstępu tylko na podstawie podejrzeń.

 Było to oczywiście częścią policyjnych zwyczajów, by dokonywać masowych aresztowań setek osób naraz podczas protestów. Policja DC była szczególnie znana z~otaczania i~łapania kolumn setek maszerujących, a następnie aresztowania ich za ,,niewykonanie polecenia rozejścia'' Aresztowania nigdy się nie utrzymują, są oczywiście nielegalne, ale w~trakcie tego procesu wszyscy są fotografowani i~pobierane są odciski palców, a te informacje są następnie umieszczane w~międzynarodowych bazach danych.

\noindent Nicky: Tak czy inaczej, chodzi o to, by zrobić coś, co uwydatni hipokryzję, ponieważ FTAA ma polegać na wyeliminowaniu kontroli granicznych, z~wyjątkiem, oczywiście, oni mają na myśli, że kontrole graniczne dotyczą korporacji, a nie tych, które wpływają na ludzi . Pomyśleliśmy więc, że będziemy mieć grupę aktywistów przebranych za produkty, którzy będą przechodzić. Miałem iść przebrany za dolara. Ktoś inny zamierzał przebrać się za genetycznie zmodyfikowanego pomidora\ldots  rozumiecie pomysł. Więc kiedy nas zatrzymują, możemy powiedzieć: ,,Myśleliśmy, że to jedyny sposób, w~jaki możemy przedostać się przez granicę''. 

\noindent Twinkie: Chciałam iść jako Ubezpieczenie Zdrowotne. Chociaż nadal nie jestem do końca pewna, jak ten kostium miałby działać.

\noindent Ktoś: Może sprzedawca ubezpieczeń?

\noindent Enos: W każdym razie, moglibyśmy na tej podstawie zrobić jakiś skecz, zorganizować konferencję prasową, podczas gdy w~tle kanadyjska policja przesłuchuje i~bije garść warzyw. Jesteśmy w~kontakcie z~kilkoma radykalnymi mediami w~Vermont, które z~pewnością przedstawią tę historię, i~mamy nadzieję, że uda nam się uzyskać WBAI, a może nawet \textit{Frontline}, dla relacji telewizyjnych.

\noindent Mandy: Wiesz, technicznie, jeśli wykluczają jakiegokolwiek Amerykanina z~kartoteką aresztowania, to obejmuje to też Busha, czyż nie? Może moglibyśmy namówić kogoś, kto by poszedł jako George W z~dużym ,,DUI'' wypisanym na czole?

\noindent Steve: Czy to wszystko nie zależy od założenia, że  faktycznie zatrzymają nas na granicy? Co się stanie, jeśli po prostu pomachają nam? Tylko ze złości?

\noindent Nicky: Nie musisz się tym martwić. Nie tak pracują gliniarze. Jeśli policja ma rozkaz zatrzymania aktywistów, to właśnie to musi zrobić. Niebezpieczeństwo jest większe, że mogą nawet nie zauważyć, że jesteśmy aktywistami. Właśnie o to się martwię.

\noindent Enos: Cóż, wszyscy wiemy, że gliniarze są głupi, ale\ldots  Myślę, że jeśli zobaczą jakiegoś faceta próbującego przekroczyć granicę przebranego za genetycznie zmodyfikowaną żywność, prawdopodobnie zorientują się, że mają do czynienia z~aktywistą.

 Stopniowo spotkanie staje się czymś w~rodzaju rozmowy. Dwie aktywistki queer, Mandy i~Jen, zastanawiają się, czy nie romantyzujemy tych ,,wojowników mohawk''. Czy tak naprawdę nie mamy do czynienia z~ludźmi, którzy nawet w~najmniejszym stopniu nie są po tej samej stronie co my w~kwestiach takich jak seksizm czy homofobia. Twinkie, Target i~Mac przeszkadzają sobie, aby odpowiedzieć, szczegółowo opisując całą historię rad kobiet i~konstytucyjne niuanse konfederacji Sześciu Narodów. (Wydaje się, że wszyscy o tym czytali). 
 
 -- Właściwie -- mówi Mac -- jednym z~głównych osiągnięć Stowarzyszenia Wojowników podczas impasu w~Oka było ożywienie systemu Matki Klanu jako alternatywy dla sponsorowanego przez rząd zarządu plemiennego. Do tej pory wszystkie kluczowe decyzje po stronie kanadyjskiej są w~rękach rad kobiet. Shawn po Akwesasne ma nadzieję, że podobny proces rozpocznie się po stronie amerykańskiej.

 Mandy jest zaskoczona, ale ostrożna.
 
  -- Brzmi cudownie. Ale czy nie sądzisz czasem, że to wszystko jest zbyt piękne, aby mogło być prawdziwe?

 Ta myśl też przyszła mi do głowy, może wszystkim. Trudno było zaprzeczyć, że z~punktu widzenia typowego nowojorskiego anarchisty, posiadanie bandy Mohawk Warriors obiecującej otworzyć dla ciebie most -- nie mówiąc już o bandzie Mohawk Warriors mających na celu ożywienie matriarchalnej struktury decyzyjnej -- było najfajniejszą rzeczą, jaką można sobie wyobrazić. Możesz się tylko zastanawiać, czy to wszystko nie jest trochę za fajne.

 Godzinę później wszyscy spacerowaliśmy do lokalu Świętego Marka na drinki w~Grassroots Tavern. Tego wieczoru na Brooklynie odbywała się impreza kultu śmieci. Wszyscy dyskutowali, czy warto jechać. Mac i~Moose wdają się w~długą dyskusję na temat tego, czy DAN była w~tym momencie organizacją wyraźnie anarchistyczną. Czy w~DAN są jacyś wyraźnie nie-anarchiści? A przynajmniej, inni niż w~DAN Labor? Nikt nie jest do końca pewien. Twinkie znika i~pojawia się ponownie piętnaście minut później z~kilkoma Radical Cheerleaders i~ogromnym stosem sushi wyciągniętego ze śmietnika. Niewielkie napięcia pojawiły się, gdy część z~nich nie została wpuszczona do baru z~powodu braku dowodu tożsamości. Po krótkiej konsultacji na zewnątrz sprawa została jakoś rozwiązana. Twinkie, odkrywając, że nie jestem wegetarianinem, wciąż podaje mi kawałki sushi z~rybą. Nie przepadam za sushi wyciągniętym ze śmietnika, próbowałem je ukryć. Ona ciągle to zauważa. Rufus delikatnie wyjaśnia, że  tak naprawdę to jest dokładnie to samo, co kupiłoby się w~sklepie dwadzieścia minut wcześniej: istnieją przepisy określające, kiedy sushi należy wyrzucić, a w~połowie przypadków, w~momencie, gdy je wystawiają, jest już aktywista lub miejscowy dzieciak lub więcej czekających, żeby je zabrać.

\section{Czwartek, 15 marca}

 W \textit{Toronto Globe and Mail }pojawia się oburzający artykuł, w~którym pojawiają się pogłoski, że Akwesasne Mohawks będzie nielegalnie ,,przemycać'' przez granicę do Kanady aktywistów z~rejestrami kryminalnymi. Najwyraźniej samo Akwesasne ma w~Kanadzie reputację jaskini przemytników -- głównie alkoholu i~tytoniu -- więc sugeruje się, że te same łodzie będą przewozić nowy eksport przestępczy -- anarchistów -- przypuszczalnie dla pieniędzy. E-maile i~rozmowy telefoniczne natychmiast są prowadzone na temat sposobu odpowiedzi.

\medskip
\noindent \textbf{Formacja YABBA w~studio Betty, 19:00}\medskip
 

 Wesoło walimy się nawzajem. Tym razem Smokey wymyślił kombinezon zrobiony z~pustych plastikowych butelek po coli, który okazuje się wyjątkowo odporny na najpotężniejsze ciosy naszych miękkich kijów. Przechodzimy przez różne scenariusze obronne: Jak utrzymać linię, jeśli gliniarze po prostu próbują przebić się przez ścianę tarczy. Jak obronić konkretną osobę, którą chcą porwać. Jedną rzeczą, która staje się oczywista, jest to, że przy całym tym sprzęcie będziemy potrzebować co najmniej dwudziestu minut na przygotowanie, zanim będziemy mogli ruszyć do akcji.

 Toczy się długa dyskusja na temat herbów: udało nam się zdobyć dość dużą liczbę nadwyżek brytyjskich hełmów dla policji prewencji z~katalogu wysyłkowego (każdy ma dwa darmowe gumowe nagolenniki!) i~większość z~nich pomalowaliśmy na dyniowo-pomarańczowy. Pojawił się plan, aby spersonalizować je poprzez umieszczenie na wierzchu pozorowanych heraldycznych urządzeń: wypchanych pingwinów, lalek kewpie, wiatraków i~tego typu rzeczy. Problem polega na tym, jak zauważa Smokey, że to nas zindywidualizowałoby: policja mogłaby z~łatwością wybrać każdego z~nas do aresztowania, gdyby wszyscy nie wyglądali podobnie. Czy byłoby możliwe umieszczenie jakiegoś gadżetu na szczycie każdego hełmu, aby można było dowolnie podłączać i~odłączać grzebienie? W ten sposób moglibyśmy je ciągle zmieniać? Ale projekt wydaje się bardziej kłopotliwy, niż chcielibyśmy poświęcić na niego czasu.

 Zapowiedzi: W Burlington od 13:00 do 17:00 w~sobotę odbędą się szkolenia prawnicze, prawdopodobnie jedno miniszkolenie tylko dla Yabba.

 Emma i~Moose wyjeżdżają na trening uliczny.

 Smokey i~Flamma zwracają uwagę, że nawet bez ochraniaczy formacja Ya Basta! może służyć jako doskonała dywersja. Podczas rajdu anty-sweatshop dwa tygodnie wcześniej, sześciu z~nas właśnie założyło kombinezony chemiczne. W chwili, gdy zaczęliśmy zakładać kombinezony, podbiegł gliniarz wysokiej szarży i~zapytał, co się dzieje, i~przez cały marsz byliśmy przez cały czas otoczeni z~czterech stron przez policję. Związaliśmy większość ich sił tylko naszą szóstką.

 Zachęcam Saszę, aby dołączył do mnie w~consulcie w~Quebec City.

\section{Piątek, 16 marca}

 Inną rzeczą, która wyłoniła się z~czwartkowego spotkania, było to, że jako ,,minister informacji'' Ya Basta!, moim zadaniem było stworzenie komunikatu prasowego będącego odpowiedzią na artykuł \textit{Globe and Mail}. Po spotkaniu zamknąłem się w~swoim pokoju z~laptopem i~około 2 lub 3 nad ranem wysłałem wersję roboczą na listy Yabba w~celu uzyskania opinii:

\bigskip
\noindent KOMUNIKAT PRASOWY

 Od: \textit{Kolektywu Ya Basta! w~Nowym Jorku i~Nowojorskiej Sieci Akcji Bezpośredniej DAN}

 W czwartek, 15 marca, w~\textit{Toronto Globe and Mail} pojawił się artykuł, który błędnie przedstawiał wyniki historycznego spotkania między aktywistami amerykańskimi i~kanadyjskimi a tradycyjnymi Mohawkami z~Akwesasne na początku tego miesiąca.

 Wbrew twierdzeniom artykułu nigdy nie było planów ,,przemycania'' aktywistów (nie mówiąc już o ,,przestępcach'' przez granicę. Nasze intencje były od początku jawne i~jawne; publiczne oświadczenia zostały opublikowane m.in. za pośrednictwem NYC Independent Media Center (www.nyc.indymedia.org. ) oraz w~Internecie. Nie jest to nasza wina, że  reporterzy i~policja (co do której zakładaliśmy, że monitorują nas dość uważnie!) nie spróbowała sprawdzić tych łatwo dostępnych dokumentów publicznych.

 Po przedstawieniu rzeczywistych faktów, kontynuowano, używając dużo języka, który opracowaliśmy w~poprzednich dyskusjach w~Ya Basta!:

\noindent DLACZEGO BYŁO TO KONIECZNE?

 Chociaż zawsze byliśmy otwarci, sama FTAA od początku była tajnym projektem, stworzonym przez rządowe i~korporacyjne elity przy jak najmniejszym udziale opinii publicznej. Z tego powodu sponsorzy regularnie wykorzystywali granice międzynarodowe, aby uniemożliwić przedstawicielom społeczeństwa zbliżanie się do ich spotkań, mimo że ci protestujący, w~przeciwieństwie do traktatu, po prostu wyrażają poglądy przeważającej większości obywateli krajów, których ci sygnatariusze twierdzą, że reprezentują. Podczas spotkań OAS w~Windsor w~Ontario zeszłego lata, które położyły podwaliny pod FTAA, około dwóch na trzech aktywistów, którzy próbowali przekroczyć granicę z~USA, zostało powstrzymanych siłą fizyczną. W ostatnich miesiącach aktywiści próbujący uczestniczyć w~spotkaniach w~Quebecu zostali zawróceni na granicę, zatrzymani i~poddani nielegalnym przeszukaniom i~konfiskatom. Mamy wszelkie powody, by sądzić, że władze zamierzają użyć siły, aby uniemożliwić ekologom, członkom związków zawodowych i~innym dysydentom politycznym wyrażenie sprzeciwu wobec tajnych negocjacji w~Quebec City w~kwietniu.

\noindent PRZECIW GŁUPOCIE GRANIC

 Wykorzystywanie międzynarodowych kontroli granicznych do tłumienia opozycji politycznej jest kolejnym dowodem na to, że proces określany mianem ,,globalizacji'' w~rzeczywistości nie jest niczym podobnym; jak również absurdalność nazywania ogromnego ruchu międzynarodowego, który powstał, by przeciwstawić się mu w~imię globalnej demokracji, ,,ruchem antyglobalistycznym''. Czas porzucić propagandę i~zacząć szczerze mówić o tych rzeczach. Gdyby ,,globalizacja'' miała cokolwiek znaczyć, oznaczałaby stopniowe znoszenie granic państwowych, aby umożliwić swobodny przepływ ludzi, mienia i~idei. Korporacyjna ,,globalizacja'' oznacza coś dokładnie odwrotnego: oznacza uwięzienie biednych za coraz bardziej ufortyfikowanymi granicami, aby bogaci mogli korzystać z~ich desperacji. Liczba uzbrojonych strażników wzdłuż granicy amerykańsko-meksykańskiej wzrosła ponad dwukrotnie od czasu podpisania NAFTA; uchodźcy i~osoby ubiegające się o azyl marnieją niczym przestępcy w~dwudziestotrzygodzinnej blokadzie; społeczności imigrantów żyją w~ciągłym terrorze. Tego samego możemy spodziewać się tylko wtedy, gdy NAFTA zostanie rozszerzona na całą półkulę zachodnią.

 Zamiast tego, Ya Basta! wzywa do zniesienia granic państwowych i~uznania zasady globalnego obywatelstwa. Wierzymy, że każda istota ludzka urodzona na tej planecie ma prawo żyć tam, gdzie chce, a jej życiowe szanse nie zależą od jakiegoś przypadkowego geograficznego przypadku narodzin. Uważamy, że każdy człowiek ma równe prawo do podstawowych środków egzystencji: powietrza, wody, żywności, schronienia, edukacji i~opieki zdrowotnej. Chcemy, aby autorytet państw narodowych stopniowo się rozpadał, a władza była przekazywana wolnym społecznościom na podstawie prawdziwej demokracji gospodarczej i~politycznej; proces, który doprowadzi do wylania się nowych form bogactwa i~kultury, których zubożałe umysły obecnych władców świata nie były w~stanie sobie wyobrazić. Sieć Akcji Bezpośrednich DAN oferuje swój sukces jako szybko rozwijająca się federacja kontynentalna, oparta na zasadach demokracji bezpośredniej i~zdecentralizowanym podejmowaniu decyzji poprzez konsensus, jako żywy dowód na to, że rządzący -- w~tym wybrani ,,przedstawiciele'' -- są po prostu niepotrzebni. Zwykli ludzie są doskonale zdolni do kierowania własnymi sprawami w~oparciu o równość i~prostą przyzwoitość.

 Granice państwowe zostały stworzone przez przemoc i~są utrzymywane przez przemoc. Są pozostałością barbarzyńskiego wieku, który ludzkość musi ostatecznie przezwyciężyć, jeśli ma przetrwać. Odmawiamy uznania ich zasadności.

\noindent ZA SAMOOKREŚLENIEM WSPÓLNOT I~SUWERENNOŚĆ MOHAWK

 Decydujemy się na podróż przez Cornwall, aby wyrazić naszą solidarność z~narodem Mohawk i~nasze uznanie dla jego suwerenności nad terytoriami, które zajmował na długo przed istnieniem rządów USA i~Kanady. Nic nie ilustruje szaleństwa granic państwowych bardziej niż fakt, że te same rządy, które prowadziły ludobójczą wojnę przeciwko Mohawkom, teraz roszczą sobie prawo do decydowania, kto może przejść z~jednej części terytorium Mohawków na drugą. Nasza solidarność z~naszymi siostrami i~braćmi Tradycyjnymi Mohawkami jest zakorzeniona w~naszym poparciu dla regionalnej autonomii i~wspólnotowego samostanowienia w~obliczu aroganckiej władzy państwa; ale także, z~naszym głębokim szacunkiem i~podziwem dla Narodu, którego wkład polityczny -- stworzenie federacyjnej konstytucji bez scentralizowanego państwa, zbiorowe zarządzanie zasobami, poszanowanie indywidualnej autonomii, rola w~zaprowadzaniu pokoju, polityczne upodmiotowienie kobiet -- dostarcza wielu z~nas wizji tego, jak może funkcjonować przyszłe sprawiedliwe społeczeństwo, co jest o wiele bardziej przekonujące niż Konstytucja Stanów Zjednoczonych, która była częściowo przez nią zainspirowana. Pragniemy podziękować naszym przyjaciołom z~Mohawk za ich hojne zaproszenie i~wyrazić nasze głębokie zaangażowanie w~dalszą walkę o suwerenność, prawa społeczne i~sprawiedliwość społeczną, tak jak uznali nasze prawo, jako obywateli świata, do ujawnienia naszej obecności politykom, którzy ośmielają się występować w~naszych imieniu w~Québec City w~dniach 19-21 kwietnia.

 Komunikat kończył się numerami do Ya Basta! (mój), DAN (Eric), i~Wojowników Mohawk (Shawn).

\section{Sobota, 17 marca}

 21:00, w~sobotni wieczór odbywa się wielka impreza zapatystowska. Miesiąc wcześniej EZLN wkroczyła do Mexico City, by lobbować za ustawą o autonomii tubylców, a film już się ukazał. Pokazowi towarzyszyły relacje od dwóch osób z~DAN, które w~tym czasie tam były.

 Potem imprezy. Pow-wow przed jedną z~nich o komunikacie prasowym. Czas jest z~pewnością najważniejszy, ale (kilka osób się pyta) czy nie powinniśmy wyjaśnić tego z~Mohawkami przed wydaniem? Moose mówi, że zadzwonił do Shawna, a Shawn powiedział tylko: ,,Cóż, nie prosimy cię o zatwierdzenie naszych komunikatów prasowych''. Eric z~DAN Media Collective zgadza się wysłać je przez Blast Fax do każdego większego serwisu informacyjnego w~kraju następnego dnia, a kopia pojawi się na stronie internetowej.

 Nie jest jasne, czy ktoś to kiedykolwiek przeczytał. Na pewno nikt nigdy do nas nie oddzwonił. Wszystkie takie wielkie stwierdzenia po prostu znikają w~eterze, tak jak wszystkie artykuły i~listy, które regularnie wysyłamy do gazet przed ważnymi akcjami. Następnie te same media, które odmawiają ich prowadzenia, skarżą się swoim czytelnikom, że nie można dowiedzieć się, do czego właściwie dążą te antyglobalistyczne typy.

\section{Niedziela, 18 marca}
\noindent \textbf{ Spotkanie DAN w~Charas}\medskip

 Kolejne długie spotkanie. Przedłużona dyskusja na temat aktualnego stanu negocjacji z~Shawnem i~OCAP.

 Mac wzywa DAN do poparcia akcji w~Cornwall: najlepiej, jak mówi, zrobić to tak szybko, jak to możliwe przed następnym spotkaniem NEGAN trzydziestego pierwszego, aby upewnić się, że ludzie udają się na akcję bezpośrednią, zamiast wyjeżdżać autobusami związkowymi następnego dnia. Więc popieramy to.

 Toczy się długa dyskusja na temat zbiórki pieniędzy planowanej w~miejscu zwanym Patelnią, historii \textit{Globe and Mail }i innych podobnych, a zwłaszcza wydarzenia medialnego zaplanowanego na Prima Aprilis. Grupa robocza 1 kwietnia już się utworzyła i~wypracowała szczegóły:

\noindent Enos: Ten ostatni artykuł w~\textit{Globe and Mail }jest właściwie symptomatyczny dla rodzaju prasy, jaki otrzymujemy. Wszystko jest prawie takie samo: będziemy agresywni, destrukcyjni, jesteśmy bandą łobuzów, niereprezentujących nikogo ani niczego, przychodzących, by podpalić miasto. Próbowaliśmy więc wymyślić, jak udostępnić bardziej realistyczne obrazy tego, kim jesteśmy i~o co nam chodzi. W ten sposób wpadliśmy na pomysł zrobienia akcji z~zabawnymi kostiumami, czymś głupim i~nieszkodliwym. Pomysł był taki, że możemy to zrobić 1 kwietnia, czyli nie tylko Prima Aprilis, to dzień, w~którym SalAMI przeprowadza akcję ,,pokaż tekst'' w~Ottawie. Pojawiamy się na granicy, mówimy grzecznie, że dołączymy do protestów w~Ottawie; zostajemy zawróceni; organizujemy konferencję prasową. Wyjaśniamy im, że to jest to, co chcemy zrobić, żeby zainteresować media.

 To prawie wszystko. Jednak aby to zadziałało, będziemy potrzebować dużo więcej osób na spotkaniach. Ostatnim razem dostaliśmy tylko trzy lub cztery. Jadę jako Bush, Nicky będzie banknotem dolarowym. Julie z~Urban Justice League będzie genetycznie zmodyfikowanym pomidorem\ldots 

\noindent Target: Szkoda, że  jest to pierwszego, właściwie, bo to jest dzień, w~którym organizują męskie warsztaty antyseksizmu w~Charas.

\noindent  Nicky: O tak. Ups. Cóż, miejmy nadzieję, że nie będzie to ostatnie.

 Spędzam większość następnego tygodnia, próbując dowiedzieć się dokładnie, jak wygląda wynajmowanie samochodu (nie jeżdżę), przygotowując się do Quebecu. Kilka osób twierdzi, że może być zainteresowanych przyjazdem, ale tylko jedna dociera: Dweisel z~kolektywu Free CUNY. Czyli czworo z~nas w~jednym samochodzie. Pomijam spotkanie Ya Basta! w~przyszłym tygodniu i~ruszam w~następny weekend.

\chapter{Wycieczka do Quebec City}

 Oto historia mojej pierwszej podróży do Quebec City. Jedną dziwną rzeczą w~miesiącach poprzedzających akcje FTAA było to, że nasze krajobrazy wyobraźni nieustannie przeskakiwały tam i~z powrotem. Kiedy Jaggi i~jego przyjaciele byli w~mieście, wszystko dotyczyło Quebec City i~tamtejszego muru. Po miesiącu spotkań w~Nowym Jorku wszystko to stało się upiorne, niematerialne; Cornwall, Mohawkowie, działania na granicy, wszystko to wydawało się namacalne i~realne. W następny weekend wszystko się odwróciło i~wyszedłem z~tego całkowicie, całkowicie zdeterminowany, by dotrzeć na Szczyt. Ta determinacja miała w~pewnych momentach wywołać znaczne napięcie u niektórych moich przyjaciół, ale nigdy jej nie porzuciłem.

\section{Piątek, 23 marca 2001}

 Dzień spędziłem głównie na prowadzeniu samochodu. Ja, Emma,  Sasha, trójka z~Ya Basta! i~Dean z~Free CUNY Collective, wyruszyliśmy z~miasta dość wcześnie rano z~zapasem wegańskiego jedzenia i~dużą kolekcją kaset muzycznych.

 Technicznie rzecz biorąc, Sasha, filmowiec, tak naprawdę nie jechał na consultę, ale na konferencję Independent Media Center, która odbywała się kilka przecznic dalej w~tym samym czasie. Lubił też nieocenioną jakość cieszenia się całodniową jazdą, co było dobre, ponieważ w~ogóle nie prowadziłem. Emma,  która przemawiała w~imieniu Ya Basta! pomimo tego, że zamierzała działać w~Black Bloc, była dobrze zapowiadającą się artystką, także po dwudziestce, znaną z~instalacji w~całym mieście. Oddana weganka, pracowała w~sklepie ze zdrową żywnością na Dolnym Manhattanie. Dean był studentem socjologii, wysokim, schludnym, trochę podobnym do młodego Montgomery Clift. Wyruszył głodny, namówił nas do zatrzymania się na porządne śniadanie, a wkrótce potem zaczął narzekać na chorobę lokomocyjną. Wyciągnąłem Dramamine z~puszki z~lekarstwami. Wziął, prawie natychmiast skłonił głowę i~spędził prawie całą wycieczkę z~Hudson Valley do Montrealu przysypiając na tylnym siedzeniu.

 Przejście graniczne przeszliśmy bez problemu, starając się wyglądać jak najczyściej. (Emma próbowała zakryć zielone części włosów małą czapką z~pończochą i~naciągnęła bluzę z~kapturem na T-shirt Clashu, ale zastanawialiśmy się, czy to w~ogóle było konieczne. Jak zauważyła Sasha, amerykańscy punkrockowcy dość regularnie wjeżdżają do Kanady). Sasha, siedzący za kierownicą, wyjaśnił, że jedziemy na Niezależną Konferencję Mediów, twierdzenie to było nieskończenie bardziej przekonujące, ponieważ obok niego leżała duża droga kamera wideo (od czasu do czasu zatrzymywał się, żeby to zrobić panoramiczne ujęcia wsi). Policjanci na granicy przepuścili nas. Przejechaliśmy przez Montreal, wpatrując się w~gigantyczną składaną mapę do muzyki profesora Longhaira, gubiąc się tylko raz, podziwiając billboardy reklamujące wakacje na Kubie (pierwszy dramatyczny dowód, że naprawdę jesteśmy w~innym kraju) i~rozpoczęliśmy ostatni, płaski, raczej ponury bieg do Quebecu, gdy słońce zaczęło zachodzić. Do samego miasta dotarliśmy wczesnym wieczorem.

 Przybywamy

 Poruszanie się po samym mieście nie jest łatwe. Wydaje się, że urbaniści nie widzieli nic złego w~ułożeniu trzech lub czterech ulic jednokierunkowych pod rząd, wszystkie biegnące w~tym samym kierunku; wydawali się też nie czuć, aby umieszczanie nazwisk na tych ulicach jest bardzo ważne, przynajmniej w~takim miejscu, w~którym można je zobaczyć. Do tego instrukcje CLAC, których używamy, są wyjątkowo złe. Wreszcie udaje nam się zlokalizować nasz pierwszy przystanek: Independent Media Center.

 Właściwie IMC jest dość standardowym pierwszym przystankiem, gdy przyjeżdżasz do nowego miasta, ponieważ miejsce to prawie nigdy nie jest puste i~zawsze pełne informacji. Technicznie rzecz biorąc, budynek, do którego przybyliśmy, nie był dokładnie IMC, ale CMAQ (Centre des medias alternatifs du Québec; wymawiano to ,,smack''), prowadzonym przez finansowaną przez organizację pozarządową grupę medialną zrzeszoną w~SalAMI o nazwie Alternativs. Przynajmniej tego dowiadujemy się od Madhavy, czasami z~nowojorskiego IMC, czasami doradcy obozowego w~Poughkeepsie, który siedzi zgarbiony nad komputerem i~drapie niechlujną blond brodę. 
 
 -- Dobrą rzeczą w~Alternativs -- mówi -- jest to, że mają pieniądze. Mnóstwo. Sprzętu mamy po dziurki w~nosie. Niezbyt miłą rzeczą jest to, że mają niezwykle tradycyjną, odgórną ideę organizacji dziennikarskiej: przydziały tematów, redaktorzy przy biurkach\ldots  tego typu rzeczy. Oczywiście, daj nam trochę czasu -- wskazując na inne stare ręce IMC skulone na małym spotkaniu po drugiej stronie sali. -- Zdemokratyzujemy rzeczy.

 Przedstawia nas drobniutkiej, nieco wróżkowej kobiecie o imieniu Isabel, która następnie daje nam wskazówki. Następne dwadzieścia minut wędrujemy po stromym wzgórzu do Centrum Powitalnego CLAC/CASA, w~pięknym starym budynku z~wyjątkowo ciężkimi drewnianymi drzwiami, tylko po to, by odkryć, że Centrum Powitalne to tak naprawdę tylko miejsce, w~którym można znaleźć mieszkanie, a my już mamy mieszkania w~kolejce (wszystko zostało uzgodnione telefonicznie z~ludźmi z~CLAC, zanim wyruszyliśmy). Wreszcie około godziny 22, po zabezpieczeniu tego, co uważamy za odpowiednią mapę, wracamy do samochodu i~ruszamy na spotkanie z~naszymi gospodarzami.

\noindent  Nasi Gospodarze

 Nasi gospodarze, jak się okazało, mieszkali w~niezwykle pięknej okolicy, same gzymsy i~kominy oraz malutkie sklepiki ustawione w~narożnikach dziewiętnastowiecznych bloków. Wyglądało to trochę jak West Village, ale o wiele mniej pretensjonalnie -- pomyślałem, częściowo dlatego, że znajdowało się na niesamowicie stromym wzgórzu i~nigdy nie było poważnie zgentryfikowane. Później dowiedziałem się, że to serce Jean Baptiste, jednej z~niewielu ,,popularnych'' dzielnic pozostawionych w~górnej części miasta w~pobliżu starego, otoczonego murami miasta, obecnie w~większości pełnego hoteli i~centrów konferencyjnych.

-- Witajcie, moi rewolucyjni przyjaciele! -- rozpromienił się młody człowiek, który przywitał nas w~drzwiach. 

Otaczało go pięciu czy sześciu młodych ludzi, praktycznie stłoczonych, aby pokazać swoje szczęście po naszym przybyciu, ale tylko on rozmawiał prawie przez całą noc -- przypuszczalnie dlatego, że jako jedyny miał jakąkolwiek kontrolę nad konwersacją w~języku angielskim.

 Podsumowując, grupa wyglądała prawie dokładnie tak, jak można by sobie wyobrazić, że powinna wyglądać grupa rewolucjonistów -- przynajmniej jeśli wiedziałeś tylko, że pochodzili z~miejsca, które pod pewnymi względami przypominało Europę, ale pod innymi Amerykę Łacińską. Pierwszy, który nas przywitał, był wysoki, prawie wychudzony, z~beretem i~brodą Mefistofela. Mówił łagodnie w~swoim niepewnym angielskim, wyglądał prawie dokładnie jak Lew Trocki. Jego towarzysz z~ciemną brodą udawał wiarygodnego Che Guevarę; trzeciego mężczyznę imieniem Pascal, z~długim kucykiem i~koszulką z~Che Guevary, było trudniej sklasyfikować. Nie wydawał się odpowiadać żadnemu rewolucyjnemu bohaterowi, jakiego pamiętałem, ale nie mogłem przestać myśleć, że jeśli takiego nie było, to naprawdę powinien być. (Zapytałem siebie: dlaczego zakładamy, że jeśli ktoś poświęcił dużo czasu i~energii na upewnienie się, że wygląda dokładnie tak, jak naszym zdaniem powinien wyglądać rewolucjonista, to samo w~sobie czyni go w~pewnym sensie nieautentycznym? Większość kapitalistów poświęca dużo czasu i~energii na upewnienie się, że wyglądają dokładnie tak, jak zakładamy, że kapitaliści powinni wyglądać. Nikt nie sugeruje, że to czyni ich mniej kapitalistami).

 W salonie były też dwie nastolatki, które wydawały się w~dużej mierze ozdobne: nigdy nie odzywały się w~naszej obecności, nawet po francusku, choć niezmiennie zaczynały mówić, gdy tylko wychodziliśmy z~pokoju. Później dowiadujemy się, że oboje mają około siedemnastu lat i~są zawstydzone nieznajomością angielskiego.

 W mieszkaniu znajdowały się dwie sypialnie, duży hamak i~kilka mat już rozłożonych pod śpiwory. Nasi gospodarze byli oczywiście przyzwyczajeni do wielu gości. Właściwie było to całkiem typowe mieszkanie aktywistów studenckich: niekończące się półki na książki, wszystkie książki po francusku, tomy kreskówek i~poezji porozrzucane na używanych meblach, ironiczne plakaty religijne, lewicowe czasopisma, lodówka w~większości pełna resztek jedzenia na wynos. -- Jesteście głodni? -- zapytał Trocki. Emma spytała, czy mają wegańskie jedzenie, a Trocki, zapewniając ją, że tak, idzie do kuchni, żeby je znaleźć.

-- Nie powinnam była o to pytać -- uświadomiła sobie, gdy stoimy i~uśmiechamy się do naszych cichych towarzyszy. -- Założę się, że to jest jak w~Polsce. Jeśli poprosisz ludzi w~Polsce o wegetariańskie jedzenie, uznają, że oznacza to \textit{mało} mięsa. Jeśli poprosisz o wegańskie, myślą, że oznacza to ,,właściwie jest wegetarianinem''. Prawdziwy weganin, o którym nigdy nie słyszeli.

-- Może powinniśmy byli coś znaleźć po drodze -- powiedział Dean.

-- Prawdopodobnie i~tak nie bylibyśmy w~stanie niczego znaleźć o tej porze.

 Che przynosi wino, Trocki przynosi chleb i~wędliny. To wszystko jest niezwykle smaczne. Emma próbuje trochę chleba, patrzy podejrzliwie na resztę; później, gdy nasi gospodarze nie patrzą, zakrada się do innego pokoju, wyciąga plecak i~prezentuje gigantyczną kadź organicznego masła orzechowego i~trochę chleba pita.

 Przy winie wyjaśniamy, że wszyscy jesteśmy anarchistami, pracującymi z~CLAC/CASA. Trocki -- właściwie nazywa się Sebastien -- wyjaśnia, że  tak, rozumieją, że jesteśmy powiązani z~CLAC. Tutaj wszyscy są trockistami -- ale, jak szybko dodaje, ,,nie należą do żadnej sekty''. Są z~GOMM (Group Opposed to the Globalization of Markets -- Grupa Przeciwna Globalizacji Rynków), a Sebastien jest także z~OQP (Opération Québec Printemps 2001), która organizowała logistykę dla protestów (wymawiano to jako ,,occupee'', odpowiednio, biorąc pod uwagę, że planowali różne okupacje kampusów). GOMM stoi na stanowisku, że kluczowe jest branie udziału w~większych ruchach społecznych, nawet jeśli są reformistyczne, po to, aby je zradykalizować. 
 
 -- Oczywiście -- kontynuuje -- w~Quebecu, ze względu na sytuację polityczną, każda grupa musi zająć określone stanowisko: albo jesteś za natychmiastową niezależnością, albo jesteś za jakąś autonomią w~koalicji z~grupami robotniczymi w~anglojęzycznej Kanadzie. Musieliśmy więc zająć stanowisko. Jesteśmy za całkowitą niezależnością. Ale pracujemy głównie ze związkami studenckimi -- co w~Quebecu, wyjaśnił Sebastien, jest nieco niezwykłą sytuacją ze względu na niezwykle dziwną formę systemu edukacyjnego.
 
  W latach 60. długoletni, staromodny prawicowy gubernator, który rządził Quebekiem przez dwadzieścia lat, został ostatecznie przegłosowany. Miał poczucie, że nie więcej niż dwadzieścia procent potrzebuje wyższego wykształcenia. Nowy gubernator podniósł tę poprzeczkę do sześćdziesięciu lub siedemdziesięciu procent. Wykorzystał jednak model z~Kalifornii: nie ten system, który mają teraz w~Kalifornii, ale dziwaczny model używany w~Kalifornii między 1954 a 1964 rokiem, w~którym jeden rok z~liceum i~jeden z~uniwersytetu tworzą dwuletnią Szkołę Kandydatów. Wyjaśnił, że ci studenci kandydaci są nadal najbardziej radykalni, znacznie bardziej niż studenci Uniwersytetu. i~wszyscy będą strajkować dla FTAA. (Niektórzy zajmą też uczelnie). Mogliby wystawić na protesty nawet do pięćdziesiąt tysięcy, jeśli się w~pełni zmobilizują. Prawdopodobnie nie. Cóż, na pewno wyjdzie co najmniej dwadzieścia tysięcy. (Dwie milczące dziewczyny reprezentują tę warstwę).

 W miarę upływu wieczoru pojawia się więcej jedzenia, a skutki głodu zastępowane są przez skutki wina. Wszyscy odkrywamy, że naprawdę jesteśmy zafascynowani dynamiką socjalistycznej polityki Quebecu. Sebastien jest szczęśliwy i~rozmowny. Inni pojawiają się i~wychodzą. Rozmowa z~Sebastienem jest czasem trochę frustrująca, ze względu na jego typowy trockistowski zwyczaj używania terminu ,,my'' (,,nie lubimy pracować z~tą grupą'', ,,zajmujemy w~tej sprawie mocne stanowisko'', bez mówienia tego, co znaczy ,,my''). Zazwyczaj nie odnosiło się to do GOMM. Wydawało się, że odnosi się do znacznie ściślejszej marksistowskiej organizacji, która postrzega GOMM jako część szerszego frontu ludowego, co oczywiście było dla nich obowiązkiem, aby zbudować jak najszerszy. Dlatego nie chcieli być zbyt radykalni ani zbyt wojowniczy. Ale nigdy nie słyszeliśmy jego nazwy. Nie, żeby nam to naprawdę przeszkadzało. Sebastien wyjaśnił, że GOMM nie współpracował bezpośrednio z~CLAC i~CASA ani nie uczestniczył w~ich radach (wydawało się, że zamiast tego wybierają się na rady SalAMI), ale planuje swoje własne działania, klasyczne pokojowe nieposłuszeństwo obywatelskie w~stylu Seattle, z~blokadami i~ograniczeniami. Chcieli jednak skoordynować te działania z~CLAC, aby mieć pewność, że znajdą odpowiednie miejsce, które może być zarezerwowane dla klasycznego, pokojowego nieposłuszeństwa obywatelskiego. Najlepiej byłoby zablokować jedną autostradę, która prowadzi do muru. Wskazuje na mapę w~broszurze informacyjnej CLAC/CASA, która już leży na rogu stołu. -- Widzisz, tutaj, w~Strefie H. 

-- Masz na myśli podstawę wzgórza tam? -- pyta Dean. -- Nie są strasznie strome?

-- Tak, myślę, że mijaliśmy je pięć, czy sześć razy, kiedy się dzisiaj wieczorem gubiliśmy -- potwierdza Sasha. 

-- Ta. Jest zbyt strome, by nadawała się do czerwonej taktyki. Próba szarży pod górę tam byłaby samobójstwem.

 Sebastien chce, abyśmy przekazali dobre słowo na rad przedstawicielskiej, a my oczywiście się zgadzamy. Rozmowa wraca do złożonej dynamiki budowania koalicji anty-FTAA. Pascal wyciąga ksero z~rodzajem schematu blokowego, ilustrującą trzy lub cztery różne konfederacje robotnicze, grupy parasolowe, pełne kółek, strzałek i~sojuszy. Rzecz w~tym, że w~Kanadzie panuje tak szerokie uzwiązkowienie, w~każdym razie w~porównaniu z~USA, i~wiele związków jest wojowniczych. 
 
 -- Co właściwie -- mówię -- powoduje, że  zastanawiam się nad pewnymi możliwościami. Czy ktoś pomyślał o rozmowie z~pracownikami hotelu w~miejscu, w~którym faktycznie odbędzie się Szczyt?

 Sasha energicznie kiwa głową. 
 
 -- A może bardziej odpowiedni związek dostawców żywności.

 Sebastian się uśmiecha.
 
 -- Tak, właściwie, kilka osób rozmawiało z~organizatorami dla niektórych pracowników Centrum Konferencyjnego o możliwości dodania środków przeczyszczających do wielkiej uczty. Nie była to nawet poważna dyskusja, podobnie jak wygłaszanie głupich pomysłów. Już następnego dnia organizatorzy Szczytu publicznie ogłosili, że będą korzystać z~własnych specjalnych firm cateringowych, a cała żywność zostanie przywieziona z~innej prowincji.



\section{Sobota, 24 marca}

\noindent \textbf{ DZIEŃ 1 CONSULTY}\medskip

 Po śniadaniu zostawiamy Sashę w~IMC i~kierujemy się do rady delegatów, która odbywa się pod starym miastem, w~jakimś budynku edukacji dorosłych wzdłuż szerokiej alei zwanej Réne-Lévesque. Rady dopiero się zaczynają. Przedpokój to długi korytarz z~automatami, mała nisza do picia kawy i~ogromny stół pełen literatury aktywistów.

\noindent \textbf{Stół na zewnątrz}

 Na stole niekończące się stosy papierów. Ułożone w~schludne stosy są wszystkie materiały informacyjne, które zawsze można zobaczyć w~każdej akcji: informacje prawne, informacje medyczne, zasoby dla niezależnych dziennikarzy. Istnieją również różne wezwania do działań granicznych, jeden do akcji feministycznej, liczne gazetki informacyjne na temat samej FTAA i~szkód, jakie wyrządzi ona prawom pracowniczym i~środowiskowym, pełne dramatycznych nagłówków i~ilustracji kreskówek. Większość jest dwujęzyczna; kilka jest tylko po francusku. Dostępne są piękne plakaty ,,Carnival Against Capitalism'' za sugerowaną składkę w~wysokości dziesięciu dolarów, niepilnowane, z~miską przed nimi na pieniądze. Biorę dwa, zostawiam dwadzieścia dolców amerykańskich. Na samym końcu tabeli znajduje się bezcenna dziesięciostronicowa broszura zatytułowana ,,Szczyt Ameryk\ldots  Od A do Z''. Wyjaśnia, czy jest CLAC i~CASA, z~planem działania, przewodnikiem turystycznym dla gości o poglądach politycznych, informacjami dotyczącymi transportu, adresami URL i, co najważniejsze, mapą miasta z~zarysem muru bezpieczeństwa, podzielonego na strefy. To ten, który wczoraj leżał na stole. Biorę dwa.

 Jest też ogromna miska pełna naklejek domowej roboty, najwyraźniej za darmo:

\noindent \textit{FTAA: umowy o wolnym handlu zagrażają naszym lasom. \newline  JEBAĆ samochody. \newline Nie bój się technologii. Bój się tych, którzy ją kontrolują. \newline  Żaden rząd nigdy nie da Ci wolności. \newline  Serce bogacza to pustynia. Serce anarchisty to królestwo. \newline  To nie zaczęło się w~Seattle. To się nie skończy w~Quebec City. }

\noindent CHOLERA JASNA! Lepiej coś zróbmy\ldots  \\ Koniec z~władzą korporacji! \textit{(z kreskówką maski gazowej)}

\noindent NAJLEPSZA ZABAWA OD SEATTLE: QUEBEC CITY. \textit{(z~inną kreskówką maski gazowej)}

\noindent \textit{Ręce precz od naszych ciał.}(ze zdjęciem nagiego kobiecego torsu)

\noindent \textit{Uzbrojony i~niebezpieczny.}(z kreskówkowym wizerunkiem strasznie wyglądającego gliniarza)

\noindent \textit{Bez względu na to, na kogo głosujesz, wciąż tu jestem.}(z kreskówkowym wizerunkiem jeszcze bardziej przerażającego policjanta)

 Wraz z~nimi są różne malutkie kolorowe przyciski, sugerowana darowizna w~wysokości pięćdziesięciu centów, z~sympatyczną maskotką szopa pracza CLAC, z~pięścią w~powietrzu. (Anarchiści mają słabość do małych zwierząt futerkowych, zwłaszcza jeśli mieszkają pod ziemią.) Jednak bez T-shirtów. Dean bierze kilka przypinek. Potem wchodzimy.

\noindent \textbf{Pokój w~środku}

 Wewnątrz znajduje się bardzo duża sala, w~której zwykle odbywają się recitale taneczne, a może gimnastyka. Podłogi są z~polerowanego drewna, a jedna ściana wykonana jest w~całości z~luster. Już teraz siedzi tam około stu pięćdziesięciu do dwustu aktywistów w~gigantycznym kręgu pośród niekończących się stosów płaszczy i~innego sprzętu. Przy drzwiach stoi stolik rejestracyjny, do którego przychodzi młoda kobieta z~pudełkiem pełnym kwadracików kolorowego papieru, która zapewnia nas, że spotkanie trwa najwyżej dwadzieścia minut. Szeptane wyjaśnienia: każdy, kto weźmie udział w~spotkaniu, może zabierać głos, ale tylko delegaci mogą głosować. Każdy kolektyw lub grupa afinicji ma prawo do dwóch głosów, oznaczonych papierowymi kwadratami. Czy nasze grupy upoważniły nas jako delegatów? Tak? Wręcza nam dwie kartki, jedną czerwoną, jedną niebieską. 
 -- O tak -- mówi -- prawie zapomniałam. Żaden z~was jest pracującym dziennikarzem lub w~jakikolwiek sposób jest powiązany z~policją?

-- Jak myślisz?

-- No wiesz, musimy zapytać.

 Jako delegat NYC DAN, Lesley już dołączyła do kręgu, wraz ze swoją podwózką, aktywistką o imieniu Lynn, również z~Nowego Jorku, która pracuje z~Rainforest Relief. Wszędzie wymieniamy się uściskami. Obie zbudowały już na swojej części podłogi małe gniazdo dokumentów, płaszczy, swetrów, termosów i~tym podobnych. Wyjmuję sprzęt terenowy, na który składa się tani notatnik na spirali i~bardzo drogi rapidograf (pióra techniczne: lubię je, bo nie trzeba wywierać żadnego prawdziwego nacisku na pisanie, więc dłoń nie będzie miała skurczy, jeśli trzeba pisać godzinami, co na takich spotkaniach zwykle robiłem). Rozpakowuję kilka kaszmirowych swetrów, które mają służyć jako poduszki, mój wkład w~gniazdo i~wszyscy zaczynają szeptać.

 Pierwsze pytanie jest nieuniknione. 
 
 -- Więc o co chodzi z~głosowaniem? Jakiego rodzaju procesu tu używają?

-- Cóż, to interesujące -- wyjaśnia Lesley. -- CLAC jest pod tym względem trochę dziwny. Co do CASA, to nigdy nawet nie zorganizowali rady. Myślę, że naprawdę dobrze sobie radzą z~ludźmi bez doświadczenia. Zasadniczo, -- dodała --  aktywiści w~Quebec City do niedawna nie mieli w~ogóle żadnego doświadczenia z~procesem konsensusu; uczą się go całkowicie od zera. Ale osiągnęli już ogromny postęp, przechodząc w~ciągu ostatnich kilku miesięcy od systemu głosowania większościowego do pewnego rodzaju systemu pół-konsensusu, w~którym, jeśli nie uda im się znaleźć konsensusu za pierwszym razem, przechodzą do głosowania superwiększością siedemdziesięciu pięć procent głosów. Kończy się to tak samo, jak w~przypadku pełnego konsensusu. Większość osób prowadzących to spotkanie pochodzi jednak z~Montrealu, a niektórzy z~nich to bardzo doświadczeni facylitatorzy. CLAC stosuje również dość nietypowy system na zmianę, jest to trochę kontrowersyjne, w~którym nalegają na ścisłą równość płci. W przypadku każdej kwestionowanej propozycji występują naprzemiennie jedna kobieta opowiadająca się za propozycją, jeden mężczyzna za, jedna kobieta przeciw propozycji, jeden mężczyzna przeciw. W praktyce okazuje się to bardziej praktyczną regułą niż ścisłą praktyką, ale jest to przydatny sposób, aby upewnić się, że nikt nie zapomni o podstawowej zasadzie.

-- Proces -- wyjaśnia dalej -- jest nieco bardziej formalny niż jesteśmy do tego przyzwyczajeni. Dzieje się tak po części dlatego, że jest to technicznie konsultacja, a nie właściwa rada delegatów: lokalni organizatorzy wymyślają szerokie ramy działania, ale chcą, aby grupy afinicji pochodzące spoza prowincji udzieliły im porady. Chcą też dowiedzieć się, co zamierzają zrobić ci z~zewnątrz. Dlatego plan jest taki, aby szybko przejść z~walnego zgromadzenia na sesję breakout, gdzie podzielimy się na małe, łatwe do opanowania grupy i~każda z~nich odpowie na szereg pytań dostarczonych przez organizatorów. Pod koniec sesji wszyscy wyjaśnią, co ich grupa afinicji planowała, że  będzie robiła podczas Szczytu. Stanie się to podstawą, dzięki której facylitatorzy będą mogli skonstruować listę różnego rodzaju działań (blokady, teatr uliczny itp.), co z~kolei pozwoli na kolejną przerwę, umożliwiając konsultowanie się w~małych grupach z~tymi, którzy zamierzają robić mniej więcej to samo. Potem będzie kolacja i~impreza, a następnego ranka zbierzemy się ponownie na końcową sesję plenarną.

 W większości rad jest dwóch facylitatorów: jeden mężczyzna, jedna kobieta. W tej jest ich czworo. Wynika to głównie z~problemu językowego: miejscowi ludzie z~CASA wydają się mówić tylko po francusku; aktywiści z~Montrealu przełączają się tam i~z powrotem zgodnie z~żadną logiką, której nie potrafię rozszyfrować; wszyscy inni mówią po angielsku. Na szczycie koła na krzesłach siedzą więc cztery osoby: dwie najwyraźniej ułatwiające, dwie tylko do tłumaczenia -- chociaż w~praktyce obserwuję (przez większość czasu mam zeszyt, pisząc wściekle obserwacje) wydają się okresowo zmieniać. Z wyjątkiem Jaggiego, który najwyraźniej stara się utrzymać jedynie rolę pomocniczą.

 Kiedy weszliśmy, facylitatorzy odpowiadali na prośbę radykalnego zespołu wideo o nagranie części postępowania; po wysłuchaniu zwykłych zastrzeżeń wniosek zostaje przeformułowany: zaprosimy ich do powrotu późnym popołudniem, kiedy nie będziemy omawiać planów działania, a jedynie logistykę, a następnie ponownie poddamy go pod głosowanie. (Oczywiście, w~końcu, jest dużo sprzeciwu.) Kobieta z~CASA, o której myślę, że nazywała się Celine, zaczęła od podsumowania informacji już wydrukowanych w~materiałach informacyjnych.

\noindent \textit{Celine}\footnote{Dla jasności umieszczę nazwiska lub oznaczenia facylitatorów lub innych oficjalnych mówców kursywą.}: Te kolorowe bloki nie są stałe i~niekoniecznie będą fizycznie oddzielone, chociaż będziemy mieli jeden obszar zarezerwowany dla Zielonego Bloku. To są:

 Zielony Bloc to bardziej artystyczny, świąteczny styl demo, w~którym nie ma ryzyka obrony.

 Żółty Bloc to przeszkadzanie. Klasyczne nieposłuszeństwo obywatelskie bez przemocy. Jest defensywny, pokojowy: na przykład blokady lub próby okupowania terenu, które wiążą się z~określonym ryzykiem aresztowania.

 Czerwony Bloc jest destrukcyjny. To blok zakłócania, który będzie próbował przerwać Szczyt, którego uczestnicy powinni być świadomi wysokiego ryzyka represji i~aresztowań. Spodziewamy się tutaj kreatywnych, różnorodnych stylów działań zakłócających.

 Kładziemy nacisk na ,,zakłócenie'', ponieważ od samego początku CLAC i~CASA doszły do  wniosku, że biorąc pod uwagę ograniczenia muru bezpieczeństwa i~masową mobilizację policji, próba powtórki z~Seattle i~faktycznie próba zamknięcia spotkań jest strategią mało prawdopodobną, by odniosła sukces. Zdecydowaliśmy się na alternatywną strategię, która łączyła wysiłki mające na celu zakłócenie Szczytu z~wysiłkami stworzenia Tymczasowych Stref Autonomicznych, wyzwolonych terytoriów w~całym mieście.

 CLAC i~CASA opracowały szereg propozycji dotyczących samych działań, które chcielibyśmy, abyście rozważyli. [\textit{Zaczyna tłumaczyć ze kartki po francusku}]:

 W czwartek, 19 kwietnia, proponujemy o godzinie 15:00 rad delegatów dla wszystkich, którzy są tu do tego czasu, aby sfinalizować szczegóły akcji. Tej samej nocy proponujemy zorganizowanie parady z~pochodniami. To będzie akcja Zielona, naszym celem nie jest aresztowanie przed 20-tym, ale niejako powitanie Szczytu. Chcemy jeszcze raz sprecyzować: to jest demo, a nie konfrontacja. Zatrzyma się, gdy tylko pojawią się gliniarze. Po prostu sposób, aby powiedzieć ,,cześć'' i~zacząć mobilizować naszych ludzi. To jedyne cele na ten dzień.

\noindent [Różne osoby mają pytania.]

\noindent  \textit{Facylitator}: Czy możemy przejrzeć cały harmonogram i~dopiero wtedy przejść do pytań?

\noindent  \textit{Celine}: W piątek 20-go w~południe na Równinach Abrahama zbierze się Karnawał Przeciw Kapitalizmowi i~wtedy ludzie będą mogli wybrać, gdzie pójdziemy. Około godziny 14 wszyscy rozejdą się do swoich bloków i~rodzajów działań; może być marsz, ale jeszcze go nie zorganizowaliśmy, bo nie wiemy, jak będzie wyglądała sytuacja w~zakresie bezpieczeństwa.

I~pamiętajcie: wszystko, co tu prezentujemy, można modyfikować. To tylko propozycje. W tej chwili proponujemy również, aby o 18:00 w~piątek 20-go zorganizować zgromadzenie, aby omówić wydarzenia dnia i~zaplanować następne.

 W sobotę 21 lipca weźmiemy udział w~wielkiej demonstracji pracowniczej jako kontyngent wyraźnie antykapitalistyczny. Podczas marszu będziemy jednak respektować zasady organizatorów. Nie jest to więc sama w~sobie okazja do akcji bezpośredniej.

 Tego wieczoru mogłoby się odbyć wiele demonstracji i~różnorodnych akcji, no i~oczywiście akcji solidarnościowej z~więźniami.

 Niedziela 22-go będzie taka sama: będzie miejsce na różne działania, ale także na solidarność z~aresztowanymi.

 Więc\ldots  wróćmy do 20-go. CLAC i~CASA w~pewien sposób zorganizowały dwa różne dema, Żółte i~Zielone. Jeśli spojrzycie na ulotkę, zobaczycie, po prawej stronie drugiej, obie propozycje. Obaj zakładają istnienie wolnej strefy, w~której będzie bardzo ograniczone ryzyko aresztowania, 
 
\noindent  [\textit{nieco sceptycznego śmiechu}]
  
  miejsce dla zielonych, kreatywnych dem. Będzie to stała lokalizacja, wolne miejsce, w~którym wszystko będzie pięknie. W tej chwili, zakładając, że zbieramy się na Równinie Abrahama w~południe, mamy dwie możliwości. Jest to trochę niejasne, ponieważ nie wiemy, gdzie dokładnie będzie mur bezpieczeństwa, ale zasadniczo po pierwsze, Żółty Blok wyrwie się z~Równiny i~pomaszeruje bezpośrednio, by przeprowadzić karnawałową akcję pod murem bezpieczeństwa; po drugie, zaczniemy razem z~Zielonym Blokiem na Równinie Abrahama i~poprowadzimy znacznie dłuższy marsz, który przewinie się przez całe miasto, umożliwiając Zielonemu Blokowi oddzielić i~dotrzeć do tego samego miejsca godziny później.

 W obu przypadkach ostatecznym celem jest gigantyczny, cudowny karnawał, zarówno z~małymi grupowymi akcjami afinicji, jak i~większymi zbiorowymi, potrzebujemy was wszystkich!

 O tak, i~na później, marsz, moglibyśmy też przearanżować jego drogę w~zależności od mniejszych akcji, aby być z~nimi solidarnym.

 Ponownie wzywamy ludzi do poszanowania różnych bloków i~decyzji osób biorących udział, aby zapewnić poziom jedności i~solidarności.

 \noindent Pytanie: Czy podczas sesji grupowych, które mamy później, możesz upewnić się, że na każdym warsztacie jest jedna osoba z~komitetu działania CLAC lub CASA, która odpowie na pytania?

\noindent  \textit{Celine}: Tak, już to ustaliliśmy.

\noindent  \textit{Facylitator}: Czy ktoś ma pytania wyjaśniające dotyczące którejś z~tych propozycji? Będziemy wybierać na przemian pomiędzy mężczyznami i~kobietami.

 Pytań oczywiście było wiele: o rzeczywisty zasięg muru, drogi z~lotniska, możliwość prewencyjnych aresztowań podczas czwartkowej parady z~pochodniami. (Odpowiedź: brzmi to jak ważna sprawa, ale teraz zajmujemy się wyjaśnianiem pytań.) Czy komitet organizacyjny wiedział, że oficjalne otwarcie Szczytu może zostać przesunięte na 13:00?

\noindent  Mężczyzna: Jestem zdziwiony. Jakiej solidarności Czerwony Blok może oczekiwać od innych bloków? Wygląda na to, że cały ten problem został pominięty. Muszę zgłosić się do ludzi w~Toronto i~nie mam pojęcia, co im powiedzieć.

\noindent  [Lesley do mnie: To też moje pytanie.]

 \ldots ponieważ to oni będą potrzebować wsparcia. Wydaje mi się, że ten pomysł na cały blok musi zostać nieco bardziej dopracowany.

\noindent  \textit{Celine}: Zgadzam się, że musimy to zrobić. Dlatego tu jesteśmy.

\noindent  \textit{Facylitator}: Nie chcę kastrować [\textit{śmiech}] ale mamy na liście dwanaście osób, to jest czas przeznaczony na pytania techniczne dotyczące planu działania, a nie teoretyczne.

 Problem polegał na tym, że prawie niemożliwe było udzielenie odpowiedzi na żadne z~pytań technicznych bez dokładniejszego wyobrażenia, jak ten schemat kolorów będzie wyglądać na żywo. i~najwyraźniej nie zostało to całkowicie wymieszane.

\noindent  Mężczyzna: Punkt wyjaśnienia. Bloki Zielony i~Żółty mają specyficzne marsze. Czy rozumiem, że Czerwony nie ma?

\noindent  \textit{Celine}: Tak. CLAC i~CASA pracują nad zorganizowaniem Zielonego i~Żółtego Bloku, ale działania Czerwonego Bloku powinny być omawiane w~małych grupach, a nie na zgromadzeniach generalnych liczących dwieście osób jak to.

\noindent  Kobieta: We wstępie określiłaś bloki nie jako byty geograficzne, ale jako postawy. Ale wiele pytań, które słyszę, sprawia wrażenie, jakby naprawdę mieli być osobnymi grupami w~osobnych miejscach. Czy to tylko wynik zamieszania? A może zostało to całkowicie wypracowane? 

\noindent [\textit{Pauza, gdy moderator prosi o bardziej szczegółowe tłumaczenie}.]

 To znaczy, jeśli Czerwony Blok był blisko obwodu w~sensie geograficznym, a Żółty Blok chciał przeprowadzić jakąś bezpośrednią akcję bez przemocy\ldots  cóż, oczywiście, ludzie będą chcieli to zrobić również w~pobliżu muru. Powstaje więc pytanie o strefy. Czy podzielimy mapę miasta według kolorów?

\noindent  \textit{Celine}: Cóż, Zielony Blok \textit{będzie }ograniczony geograficznie. Będzie stosunkowo daleko od muru.

\noindent  \textit{Nicole}[\textit{osoba z~CASA, ta, która była w~Nowym Jorku, wtrąca się, aby wyjaśnić}]: Żółty Blok będzie bardziej mobilny niż Zielony, ograniczony nie tyle w~przestrzeni, ile w~rodzajach działań, w~które może się zaangażować. Najlepszy sposób, jaki odkryliśmy na pomaganie tym, którzy zamierzają być w~Czerwonym Bloku, polega na zorganizowaniu Zielonych i~Żółtych najlepiej jak potrafimy, aby ludzie, którzy będą chcieli zrobić Czerwony, poznali nasze plany i~zorganizowali swoje działania gdzie indziej.

\noindent  Kobieta: Problem, jak ja to widzę, polega na tym, że jeśli czerwone i~żółte bloki są mobilne i~definiowane przez nastawienie, to skąd ludzie będą wiedzieć, w~jakim bloku się znajdują? Czy będą osobne marsze, opaski, jakiś odpowiednik szeryfów, którzy mogą ci powiedzieć?

\noindent  \textit{Nicole}: To zdecydowanie coś, co powinniśmy spróbować wyjaśnić. Pamiętajcie: Żółty nie konfrontuje, ale się broni. Ale to też zależy od nastawienia gliniarzy. Jeśli policja przeprowadzi zmasowany atak, jeśli zacznie atakować wszystkich bezkrytycznie, to prawdopodobnie każdy może znaleźć się w~środku de facto czerwonej strefy.

\noindent  \textit{Celine}: Nie możemy nikomu dawać żadnych absolutnych zapewnień co do tego, co będą robić inni. Chcielibyśmy jednak, aby ludzie opowiedzieli, jakie akcje i~prezentacje zamierzają przeprowadzić, jaki kod koloru najlepiej pasuje, i~oczekujemy, że będą starali się zachować ten kolor tak dobrze, jak tylko mogą. Ale wiemy, że Żółty może wpaść w~Czerwony.

\noindent  \textit{Nicole}: Dodam, że tutaj kluczowe stają się grupy afinicji. Jeśli tak się stanie, twoja grupa może wspólnie zdecydować o opuszczeniu obszaru. Bardzo ważna będzie tutaj komunikacja.

 Rozmowa trwała w~podobnym tonie jeszcze przez piętnaście minut. Nikt nie był do końca pewien, jak to wszystko naprawdę będzie wyglądać, i~wydawało się, że planiści celowo pozostawili duże części obrazu niejasne. Zasadniczo plan CLAC polegał na zwróceniu się o naszą zbiorową poradę w~celu uzupełnienia szczegółów. Stąd struktura spotkania. Po pierwszej sesji plenarnej, na której mieliśmy tylko zadawać pytania wyjaśniające, mieliśmy w~południe podzielić się na losowo wybrane mniejsze grupy, po około dwadzieścia osób każda. Te mniejsze grupy otrzymałyby tę samą listę kwestii do omówienia; każda otrzymała kogoś z~zespołu planowania CLAC lub CASA, który odpowie na pytania informacyjne. Wyniki byłyby zapisywane i~służyły jako źródło informacji dla lokalnych grup roboczych. Wreszcie, wszyscy na sesji wyjaśniliby, jaką rolę planują przyjąć ich grupy afinicji podczas samych akcji: czy przyjeżdżają jako grupy artystyczne, grupy wsparcia, drużyny w~ruchu i~tak dalej. Zostałyby one wykorzystane jako podstawa do drugiej rundy dyskusji, w~której każda osoba mogłaby koordynować sprawy z~przedstawicielami innych grup afinicji, zamierzających zrobić mniej więcej to samo. Potem wrócilibyśmy do domu na wieczór i~odbylibyśmy ostatnią sesję plenarną w~niedzielę po południu.

 Lunch był w~biegu. Złapaliśmy talerze, złapaliśmy jakąś dużą zapiekankę i~sałatkę, kubek cydru i~zabraliśmy to ze sobą do pokoi, w~których na dole odbywały się sesje. Oczywiście przeważnie przydzielano nam różne pokoje, chociaż jakoś Lynn i~ja znaleźliśmy się w~tym samym: grupie piątej.

\noindent \textbf{12:10, pierwsza sesja dyskusyjna}

 Na dole znajdowała się cała grupa małych pokoi, które przypominały sale seminaryjne, duże stoły, lampy fluorescencyjne, przeważnie bez okien.

 Zamieszczę tutaj dość długi fragment z~moich notatek. Miejmy nadzieję, że przekażą coś w~rodzaju tekstury spotkania konsensusu -- w~szczególności nieco zakręconej jakości rozmowy, gdy ustawianie głośników sprawia, że  uczestnicy rzadko odpowiadają bezpośrednio na swoje uwagi, a dyskusja wydaje się krążyć wokół swojego przedmiotu, a nie natychmiast go atakować. To, co następuje, jest dość typowe dla takich dyskusji. Będę oznaczać osoby z~grubsza tak, jak pojawiały się w~moich notatkach, ponieważ w~większości nie zapisywałem ich prawdziwych imion. Poza tym, chociaż rozmowa była dwujęzyczna, z~tłumaczeniami -- ograniczę się tutaj do angielskiego, podając tylko tłumaczenia stwierdzeń oryginalnie wygłoszonych po francusku.

 Zgodnie z~moimi notatkami, Grupa Piąta składała się początkowo z~dwunastu mężczyzn i~dziesięciu kobiet, chociaż później pojawiły się dwie kolejne kobiety. Osoba z~CLAC przydzielona do naszego pokoju miała na imię Radikha, smukła młoda kobieta pochodzenia południowoazjatyckiego. Kiedy wszedłem, już siedziała, rozmawiając z~przyjacielem, który pracował z~Toronto IMC.

\noindent  \textit{Radikha}: Tak więc facylitatorzy poprosili każdą grupę o rozważenie trzech pytań podczas pierwszej sesji grupowej. Po pierwsze, ochrona Centrum Konwergencji. Po drugie, postawy, jakie każdy blok (czerwony, zielony i~żółty) przyjmie wobec policji. Wreszcie, w~jakich działaniach Twoja grupa afinicji myśli, że będzie brała udział.

\noindent  Bob: Cześć, jestem Bob z~Toronto IMC. Czy będzie w~porządku, jeśli będę facylitował, tak aby Radikha mogła odpowiedzieć na pytania?

\noindent  \textit{Radikha}: To by mi odpowiadało. Myślę, że wtedy też będę mogła robić notatki, ponieważ organizatorzy chcą mieć zapis wszystkiego, co wymyśli każda grupa.

\noindent  Meredith: Czy chcemy również ustalić limity czasowe dla każdego punktu porządku obrad?

\noindent  [Wiele kiwnięć głową i~potwierdzających odgłosów]

 Czy powinniśmy zatem wybrać kogoś do pilnowania czasu, czy każdy ma zegarek? 
 
 \noindent [\textit{Różne osoby nie mają zegarków}]

\noindent  Inna kobieta [\textit{do Meredith}]: Czy byłabyś skłonna to zrobić?

\noindent  Meredith: W porządku, będę pilnować czasu.

\noindent  \textit{Facylitator}: Więc co mamy do zrobienia, do 13:00? To czterdzieści pięć minut. Powiemy dziesięć minut na pytanie o Centrum Konwergencji? [\textit{do Radikhi}] Czy jest jakieś tło, o którym powinniśmy wiedzieć?

\noindent  \textit{Radikha}: Cóż, w~ramach CLAC podjęliśmy decyzję o stworzeniu Centrum Konwergencji, miejsca spotkań i~dla ludzi przyjeżdżających spoza miasta. Postanowiliśmy też zorganizować jakąś obronę na wypadek ataku policji. Pytanie brzmi, jak to zorganizować i~jak wypuścić ludzi, którzy chcą odejść. Na przykład, czy na zewnątrz będzie inwigilacja? I\ldots  cóż, myślę, że niektórzy z~nas mówili o jakimś rodzaju nadzoru wewnątrz, aby zapobiec prowokacji policyjnej w~Centrum. Jak to zorganizujemy? Nie mamy dużego doświadczenia z~tymi rzeczami i~mieliśmy nadzieję, że niektórzy z~was mogą pomóc.

\noindent  [Facylitator zbiera stosy, gdy różne osoby przy stole rzucają mu się w~oczy, lekko kiwając głową lub w~inny sposób wskazują, że chcą znaleźć się na liście mówców. Wzywa ludzi, głównie wskazując palcem, ponieważ niewielu z~nas zna się nawzajem.]

\noindent  Kobieta: Więc CLAC podjął decyzję. Teraz potrzebujesz tylko porady?

\noindent  Starszy Facet: Moje pytanie brzmi: zanim zaczniemy mówić o czujności i~ochronie, czy nie powinniśmy również mówić o decentralizacji? Co dokładnie będzie się działo w~Centrum Konwergencji? Czy ludzie będą zajmować się wszystkim, od znajdowania mieszkań, konferencji prasowych, jedzenia lub zapewniania przestrzeni artystycznych? A jeśli tak, to czy taktycznie rozsądnie jest skoncentrować wszystkie te funkcje w~jednym miejscu?

\noindent  Francuz: Kiedy faktycznie powstanie Centrum Konwergencji? 

\noindent [\textit{Wszyscy zaczynamy patrzeć na materiały informacyjne, ale nie ma wskazówek.}]

\noindent  \textit{Radikha}: Odpowiadając na pytanie o centralizację: przez ,,Centrum Konwergencji'' rozumiemy miejsce spotkań, w~którym odbywają się rady, także w~celu przyjmowania ludzi, umieszczania ich w~mieszkaniach, tego typu rzeczy. Nie zdecydowaliśmy, jakie inne funkcje może pełnić to miejsce. Co do daty, jeszcze tego nie wiemy, ale z~pewnością będzie gotowe do środy 18-go.

\noindent  Młodszy Francuz: A co z~gigantycznymi lalkami? Czy powstaną w~tym samym miejscu?

\noindent  \textit{Radikha}: Myślę, że mogą tam powstać mniejsze lalki, ale większe będą gdzie indziej.

\noindent  \textit{Facylitator}: To jest mała grupa, więc nie będę używał tutaj ścisłej zasady jeden mężczyzna, jedna kobieta, ale nadal będę starał się zachować równość płci. Więc pozwól mi teraz przejść na początek listy\ldots  kobieta w~czerwonym szaliku?

\noindent  Czerwony Szalik: Moja grupa afinicji zamierza przeprowadzić szkolenia z~akcji bezpośredniej przed Szczytem: czy będzie to możliwe w~Centrum Konwergencji?

\noindent \textit{Radikha}: Wyobrażam sobie, że Centrum Konwergencji będzie dostępne na szkolenia.

\noindent  Amerykanin: Na J20 [protestach inauguracyjnych] mieliśmy nie jedno, ale serię bardzo zdecentralizowanych Centrów Konwergencji i~to działało naprawdę dobrze. Poza tym wszędzie mieliśmy napisy ,,Zakaz narkotyków i~bomb'', co najwyraźniej, wiem, że to brzmi głupio, sprawia, że  policji jest trochę trudniej po prostu legalnie wpaść. Ponadto byliśmy bardzo ostrożni, ukrywając magazyn lalek.

\noindent \textit{Radikha}: Tak więc słyszę wiele obaw o marionetki. Czy uważasz, że powinniśmy mieć zupełnie oddzielne miejsce do robienia lalek?

\noindent  Lynn: Właściwie obawiam się, że wykorzystam protesty inauguracyjne jako nasz model. Na inauguracji było jasne, że policja nie chce aresztowań; kilku gliniarzy powiedziało mi to po tym, jak mnie zatrzymali.

\noindent  Ktoś: Jeśli nie chcieli aresztowań, dlaczego cię zatrzymali?

\noindent  Lynn: Na balu inauguracyjnym zdjęłam ubranie, miałam hasło na piersiach. Ale nawet wtedy puścili mnie po mniej więcej pół godzinie.

\noindent  Ktoś inny: Jezu, jak zdobyłaś bilety na bal inauguracyjny?

\noindent \textit{Facylitator}: Hm, może powinniśmy wrócić do propozycji: co zrobić z~obroną i~ewakuacją?

\noindent  Anglojęzyczny Facet: Poświęcenie dużej ilości środków na obronę pustego budynku wydaje się trochę głupie. Może to ważne, jeśli naprawdę chcemy bronić tej przestrzeni, żeby mieć pewność, że będzie się tam ciągle coś działo, mam na myśli, gdy rady się nie spotykają. W przeciwnym razie po prostu wiązalibyście ludzi. Być może moglibyśmy na przykład zaoferować ciągłe szkolenia.

\noindent  [Krótkie problemy z~tłumaczeniem. Zatrzymujemy się, aby upewnić się, że francuscy mówcy po jednej stronie pokoju nadgonili.]

\noindent  Francuski facet: Wydaje mi się, że głównym powodem, dla którego organy ścigania wkroczyły na przestrzenie konwergencji w~USA, było zniszczenie sztuki i~marionetek, aby zabić przesłanie, które protestujący chcą przekazać. Nie zadzierali zbytnio z~radami delegatów ani spotkaniami. Wydaje mi się więc, że naprawdę ważna jest obrona przestrzeni lalek, gdziekolwiek to będzie, a jeśli lalki nie są budowane w~Centrum Konwergencji, to może w~ogóle nie powinniśmy jej bronić.

\noindent  \textit{Facylitator}: Czy mogę tylko sprawdzić konsensus: wydaje się, że rozmawiamy o tym, jak i~czego bronić, a nie czy\ldots ? A więc: czy się na to zgadzamy? Jakaś niezgoda, że  tak naprawdę chcemy bronić przestrzeni? Że to nawet priorytet?

\noindent  Suzette: Nazywam się Suzette i~działam w~ruchu studenckim. Będziemy strajkować podczas Szczytu i~potrzebujemy naszych ludzi w~naszej własnej przestrzeni\ldots 

\noindent  \textit{Prowadzący}: Przepraszam, Suzette, wciąż mamy tu listę. Mówisz poza kolejnością.

\noindent  Suzette: Och, przepraszam. Myślę, że mówię tylko, że obrońmy przestrzeń, ale nie oczekujmy, że ruch studencki w~Quebecu będzie w~stanie poświęcić na to jakiekolwiek środki.

\noindent  Drugi Francuz: Podoba mi się pomysł, aby ludzie mogli odejść, jeśli miejsce będzie oblężone. Ale, czy jest to obrona żółta czy czerwona?

\noindent  \textit{Facylitator}: Czy możemy sprawdzić czas?

\noindent  Meredith: Właściwie mamy już piętnaście lub dwadzieścia minut.

\noindent  \textit{Facylitator}: i~zostało nam pięć osób na liście. Czy wysłuchamy tych ostatnich uwag, a potem przejdziemy dalej?

\noindent  [kiwanie głowami]

\noindent  Czerwony szal: Czy nie możemy zrobić niektórych marionetek w~Centrum Konwergencji, a niektórych gdzie indziej? Po prostu, żeby być po bezpiecznej stronie?

\noindent  [Ogólne migotanie dłoniami\footnote{,,Mruganie'' czy ,,migotanie'' to termin używany do podnoszenia rąk i~machania palcami, powszechnie używany w~kręgach anarchistycznych jako sposób na ciche przyzwolenie na wypowiedź lub propozycję na spotkaniu. Podobno wywodzi się ze znaku migowego dla oklasków.}]

\noindent  \textit{Facylitator}: Wygląda więc na to, że mamy w~tej sprawie konsensus.

\noindent  [Więcej migotania. Radikha szybko pisze.]

\noindent  Lynn: W LA wykonaliśmy wcześniej bardzo udany manewr prawny, aby bronić Centrum Konwergencji. Wiedzieliśmy, że kiedy gliniarze zaatakowali nasze lokale w~Filadelfii i~Waszyngtonie, ich wymówką było to, że miejsca te były zagrożone pożarem, więc było to częścią naszej obrony: prosiliśmy ludzi, aby nie przynosili pewnych rzeczy, które mogliby powiedzieć, że są zagrożone pożarem, ale przede wszystkim z~góry dostaliśmy prawne gwarancje, że nie wejdą.

\noindent  Francuskojęzyczny facet z~bokobrodami: Zaczekaj chwilę: czy rzeczywiście sugerujesz, że moglibyśmy uzyskać nakaz ochrony od sędziego, a to uniemożliwiłoby im z~prawnego punktu widzenia przeprowadzenie ataku wyprzedzającego, jak to zrobili, powiedzmy, na kukły w~Filadelfii?

\noindent  Lynn: Był nakaz prawny.

\noindent  Amerykanin: Naprawdę nie rozumiem, jak to mogłoby działać. W końcu na A16 i~Philly gliniarze nie powiedzieli dokładnie: ,,Uważamy, że to jest zagrożenie pożarowe'' i~zamknęli nas. Twierdzili, że w~środku są mołotowy i~bomby. Nie, żeby faktycznie były. Po prostu kłamali. Więc nie rozumiem, dlaczego zakładamy, że to, czy rzeczywiście mamy tam coś niebezpiecznego, ma z~tym coś wspólnego.

\noindent  \textit{Facylitator}: Myślę, że mamy tutaj poważne problemy z~procesem. Ludzie przeskakują listy, a, tak czy inaczej, już przekroczyliśmy czas. Radikha, masz odpowiedź na jego pytanie? Czy ktoś przyjrzał się możliwościom prawnym?

\noindent  \textit{Radikha}: Właściwie nie. Jeszcze się tym nie zajmowaliśmy, ponieważ byliśmy zbyt zajęci lokalizowaniem przestrzeni. Zresztą prawa są tu inne.

\noindent  Meredith: Może powinniśmy mieć pod ręką prawników. W Philly nie było w~pobliżu żadnych prawników, kiedy zaatakowali przestrzeń kukiełkową, a poza tym przestrzeń kukiełkowa była ogromnym magazynem pośrodku pustkowia, bez żadnych innych budynków w~pobliżu, więc nie było sposobu, aby zrobić blokadę. Tak więc, jeśli nadal szukasz przestrzeni, może to być coś do przemyślenia. Możemy również upewnić się, że pod ręką jest materiał na blokadę. A także: mamy sposób na natychmiastowe przekazanie mediom, jeśli coś się wydarzy.

\noindent  Nowoangielka: Zdajesz sobie sprawę, że zostało nam tylko dwadzieścia minut na całą sesję, a wciąż nie rozwiązaliśmy pytania pierwszego? Chciałabym również zasugerować, że użyty tutaj język, czerwony, żółty, ta niejasność, jest prawdziwą przeszkodą w~działaniu. Być może ze względu na czas powinniśmy dojść do konsensusu co do tego, co właściwie zrobimy, jeśli gliniarze nas zaatakują. To może nam pomóc przejść do następnego tematu, czyż nie miał to być stosunek do policji w~różnych blokach? Pomyślmy o tym, naprawdę powinniśmy byli najpierw zająć się tym, a potem przejść do rozmowy o Centrum Konwergencji.

\noindent  \textit{Radikha}: Cóż, organizatorzy uznali za pewnik, że tak naprawdę nie bylibyśmy w~stanie zrobić tego wszystkiego w~godzinę. Dodam, że Żółty ma charakteryzować się ,,postawą obronną'', blokowanie jest Żółte. Jeśli twoja grupa nie zamierza odpowiadać na rozkazy policji, jesteś Żółta. Oczywiście twoja grupa afinicji może sama zdecydować, jak postępować, gdy gliniarze atakują, nie ma kodu mówiącego ,,wszystkie żółte grupy afinicji muszą to zrobić''. Czerwony jest bardziej\ldots  ukierunkowany.

\noindent  Starszy Facet: Choć niekoniecznie brutalny.

\noindent  \textit{Radikha}: Nie, niekoniecznie. 

\noindent  Czerwony szalik: Aby przejść dalej, proponuję zaklasyfikować Obronę Konwergencji jako żółtą. Wiesz, technicznie rzecz biorąc, i~tak nie powinniśmy planować tutaj akcji Czerwonych.

 O 12:45 doszliśmy do wniosku, że doszliśmy tak daleko, jak mogliśmy się posunąć, nie wiedząc nawet, gdzie będzie Centrum i~do czego będzie używane, więc w~końcu przeszliśmy do definiowania bloków. Jedna z~kobiet powiedziała, że  jej grupa afinicji zamierzała zaopatrzyć się w~pleksiglasowe tarcze. Czy to nadal będzie liczone jako żółte? Radikha zapewniła ją, że tak, ponieważ tarcze są z~definicji obronne. Lynn twierdziła, że  w~Ameryce gliniarze byli zdecydowanie znani z~interpretowania sprzętu obronnego jako broni.

 Problem z~blokami, jak się okazało, polegał na tym, czy interpretować je geograficznie. Zielona strefa nie miała sensu, chyba że była fizycznie oddzielona. Trzeba zapewnić ludziom bezpieczną przestrzeń, wystarczająco daleko od akcji, aby nie groziło im pomylenie ich z~walczącymi, wystarczająco blisko, aby byli wyraźnie częścią tego samego wydarzenia. Z drugiej strony wyznaczenie określonej przestrzeni na czerwoną strefę byłoby wyraźnie samobójcze. Równie dobrze możesz wywiesić tabliczkę z~napisem ,,Policjo, tutaj ci, których należy aresztować''. Utknęliśmy więc z~jedną zieloną strefą, w~jakimś konkretnym obszarze poza akcją, a pozostałą częścią miasta ogromną żółtą strefą, której każda część może zmienić kolor na czerwony w~dowolnym momencie. Ale jeśli tak, to jak byłoby możliwe, aby ktokolwiek dokonał klasycznego obywatelskiego nieposłuszeństwa? Nie możesz twierdzić, że bierzesz udział w~pokojowym siedzeniu, jeśli w~dowolnym momencie ktoś inny może przejść obok i~rzucić cegłą ponad głową. Z poczucia obowiązku wobec naszego trockistowskiego przyjaciela zasugerowałem, że być może pewne strefy, może jedna lub dwie przecznice, mogą być zarezerwowane dla czysto żółtych akcji. Byłem trochę zaskoczony, słysząc głośne i~gwałtowne sprzeciwy. Przez kilka minut byłem postrzegany jako reakcjonista, a wielu lokalnych aktywistów -- w~tym kobieta z~Ruchu Studenckiego -- ze złością odrzucało wszelkie poglądy, że taktyka Czerwonych zostanie wszędzie zakazana. Wycofałem sugestię: 
 
 -- Cóż, prawdopodobnie grupy będą się po prostu spontanicznie skupiać. Może nie musimy tego formalizować.

\noindent  \textit{Facylitator}: Przejdźmy do trzeciego pytania: szczegółowe pomysły na akcje. Czy ktoś ma coś przeciwko rundzie na ten temat?

\noindent [\textit{Nikt się nie sprzeciwił}]

\noindent  Suzette: Nie powinniśmy tutaj rozmawiać o czerwonych rzeczach, prawda? 

\noindent  \textit{Facylitator}: Tak, takie jest moje rozumienie sytuacji. Tylko działania, o których moglibyśmy dyskutować w~całkowicie publicznej przestrzeni. 

\noindent  Starszy Facet: Jestem z~Pagan Cluster, który jest skoncentrowany w~Vermont i~przedstawiliśmy propozycję działania oparte o oświadczenie Cochabamby, mówiące o dostępie do wody jako podstawowym prawie człowieka. Chcemy stworzyć Żywą Rzekę ludzi, która będzie mogła przepływać przez różne strefy w~mieście, starając się wywołać jak najwięcej zakłóceń. Może obejmować działania wokół centralnej strefy w~pobliżu muru, gdzie zakładamy, że sprawy przybiorą najbardziej czerwony kolor, ale w~zasadzie mamy tutaj na myśli działanie typu żółtego.

\noindent  \textit{Radikha}: Pominę swoją kolejkę, ponieważ w~zasadzie spędzę weekend, wspierając protestujących. (Wiecie, jestem z~CLAC.)

\noindent  Olive (francuska uczennica z~tęczowymi włosami): Nie wiem, czy moja grupa afinicji będzie robić akcję lub wspierać.

\noindent  Bokobrody: Chcemy jak najbardziej przeszkadzać uczestnikom szczytu. Nie mamy jeszcze nic konkretnego poza tym, ale rozważaliśmy pomysł zablokowania autostrady do miasta.

\noindent  Jane: Nazywam się Jane. Właściwie przemawiam w~imieniu dwóch różnych grup. Jedna grupa jest z~Carleton University i~będzie robić zakłócające teatry uliczne, błazenady i~tak dalej. Pojawimy się i~będziemy wędrować po okolicy i~mamy te małe skecze, które możemy złożyć, gdy tylko coś zobaczymy. Druga grupa to SSSA z~Ontario. To grupa uczniów szkół średnich. Będą grać na perkusji ze znalezionymi instrumentami i~blokować w~pewnym sensie.

\noindent  Anglik: Reprezentuję dwie grupy afinicji z~Uniwersytetu w~Toronto, które również robią skecze teatralne, ale chcą być w~Żółtym Bloku, a nie w~Zielonym. Ponadto w~Toronto mamy Oddział Guerilla Rhythm. Niektórzy z~nich chcą angażować się w~ewentualne akcje na lotnisku, ale nie wiedzą, czy te nadal trwają.

\noindent  David: Jestem z~Ya Basta w~Nowym Jorku! Mamy cztery lub pięć pomysłów na scenariusze działań, z~których żaden nie może być tutaj omawiany. Dobra, myślę, że jest jeden, o którym możemy porozmawiać. Niektórzy z~nas wpadli na pomysł, żeby wyjść, ubrać się w~nasze ochraniacze i~chemiczne kombinezony, zdobyć naprawdę dużą drabinę i~po prostu wędrować z~nią tuż przy ścianie. Jeśli nic więcej, to działałoby jako dywersja. Przekonujemy się, że za każdym razem, gdy pojawiamy się w~strojach, gliniarze podążają za nami, gdziekolwiek się udamy.

\noindent  Młoda kobieta z~Quebec: Reprezentuję popularny komitet sąsiedzki w~dzielnicy St. Jean Baptiste, to dzielnica, która zostanie przecięta na pół przez mur. Planujemy serię działań na 17 i~18, które mają z~tym związek. Czy możemy omówić je tutaj?

\noindent  \textit{Facylitator}: Jasne, czemu nie?

\noindent  Młoda kobieta z~Quebec: Cóż, to wciąż jest na etapie planowania, ale jednym z~pomysłów jest to, że ludzie z~sąsiedztwa będą oszczędzać swoje śmieci przez tydzień, a następnie wyrzucą je na ścianę, aby pokazać, co produkuje społeczeństwo konsumpcyjne. i~są jeszcze dwa. Jednym z~nich jest układanie wzdłuż muru rzędów starych ubrań (znowu temat odpadów), drugim jest hałas. Aby zakłócić Szczyt, dwa razy dziennie wszyscy będą włączać muzykę tak głośno, jak to możliwe, i~jednocześnie coś naprawdę denerwującego, aby spróbować doprowadzić delegatów do szaleństwa.

\noindent  Młody mówiący po francusku: Planujemy wziąć udział w~akcjach granicznych w~Akwesasne, ale poza tym nic konkretnego.

\noindent  Człowiek plexiglas: Mój kolektyw w~Toronto organizuje również społeczności, aby przeprowadzać masowe akcje graniczne. Potem jedziemy do Quebecu z~naszą ścianą tarczy. Możemy rzeczywiście pomóc w~obronie Centrum Konwergencji, jeśli ludzie naprawdę tego potrzebują.

\noindent  Lynn: Jestem z~Rainforest Relief w~Nowym Jorku. Mamy kilka osób pochodzących z~Ekwadoru, Nikaragui, które mogą porozmawiać o potencjalnym wpływie FTAA na ich społeczności. Mamy nadzieję, że przeprowadzimy panel, a następnie zabierzemy je na akcję Mohawk, chociaż martwię się, czy narazimy ich na niebezpieczeństwo, jeśli rzeczywiście spróbujemy przejść. W samym Quebecu\ldots  cóż, mam nadzieję, że w~jakiś sposób ruszymy na mur. Może zupełnie bez użycia przemocy. Mam w~głowie ten bardzo mocny obraz z~filmu \textit{Gandhi}, przedstawiający tych wszystkich ludzi maszerujących na żołnierzy, którzy zostają przez nich pobici pałkami, ale potem, coraz więcej ludzi przychodzi i~chociaż każdy z~nich zostaje uderzony, i~tak po prostu przychodzą\ldots  A może tak, z~wyjątkiem tego, że się wspinamy.

\noindent  Bob: Będę też robił Indymedia, relacjonując cięższe akcje.

\noindent  Mężczyzna w~niebieskiej bandanie: Reprezentuję Quebec Medical i~będziemy udzielać wsparcia przed, w~trakcie i~po akcjach. Staramy się współpracować z~ludźmi, aby upewnić się, że mamy medyków na każdej akcji przy murze, ale to trochę bardziej skomplikowane.

\noindent  Starsza kobieta: Jestem też z~Mobilizacji Vermont. Naszym celem jest przeprawianie ludzi przez granicę, ale staramy się też wymyślać scenariusze, co zrobić z~ludźmi, jeśli im się nie uda.

\noindent  \textit{Facylitator}: Dobra, czas minął.

 Ktoś pyta, czy my też mamy dyskutować o marszu: czy jedziemy prosto z~Równiny Abrahama, czy najpierw przez godzinę przedzieramy się przez miasto? 
 
 -- Cóż, nie -- mówi \textit{Radikha} -- ale wygląda na to, że wciąż trwa wiele sesji (tak, rażąco kłamaliśmy na temat czasu). Ludzie jeszcze nie idą na górę, więc z~pewnością moglibyśmy trochę o tym porozmawiać, jeśli ludzie chcą. 
 
Sentyment wyraźnie skłania się ku dłuższemu marszowi (czy to naprawdę dobry pomysł, żeby wszyscy, którzy mają zrobić akcję bezpośrednią, zebrali się i~po prostu posiedzieli w~jednym miejscu przez kilka godzin, zanim cokolwiek zrobią?), kiedy ktoś schodzi na dół, żeby nam powiedzieć, że sesje się skończyły.

 Na korytarzu wpadam na Lesley. Porównujemy notatki. Większość jej sesji była również zmarnowana na meandrujące dyskusje o Czerwonym i~Żółtym. Dopiero na końcu wyszło coś pożytecznego. Dean miał podobne doświadczenie. Wydaje się, że Emma zniknęła. Kiedy idę na górę, kilka osób wskazuje mnie jako delegata Ya Basta! -- odnoszę wrażenie, że wydaje mi się, że będzie to wielka nowa innowacja w~tej akcji: tarcze, ochraniacze i~taktyka defensywna. (Okazuje się, że się mylą; tak się nie stanie. Ale fajnie było być de facto celebrytą.)

\noindent \textbf{13:45, powrót na posiedzenie plenarne}

 Krótka, nieudana próba znalezienia sobie kubka kawy zakończyła się, gdy przypomniałem sobie, że nadal nie mam żadnych kanadyjskich pieniędzy i~nie widziałem bankomatów. Jednak umożliwiło mi to wyjście na zewnątrz. Po spędzeniu trochę czasu w~przedpokoju, gdzie krążyły pogłoski o reporterze \textit{Montreal Gazette}, wróciłem, by odkryć nowo zrotowanych facylitatorów zajętych syntezą. Po zapoznaniu się z~pisemnymi raportami z~każdej sesji sporządzali teraz listę dziesięciu różnych rodzajów działań, którymi należy się zająć w~następnych sesjach, pisząc je na wielkiej płachcie papieru przyklejonego do jednej ze ścian, wywołując sporadyczne chichoty u niektórych poprzez sugestywne nie do końca angielskie omówienia:

 \textit{1. }Grupy świąteczne i~artystyczne

 \textit{2. }Obrońcy Centrum Konwergencji

 \textit{3. }Blokownicy ulic i~bulwarów

 \textit{4. }Blokownicy zewnętrznych określonych budynków

 \textit{5. }Okupacje budynków

 \textit{6. }Chodzenie/Naciskanie/Zwiedzanie/Ruch w~kierunku Muru

 \textit{7. }Redekoracja przestrzeni miejskiej

 \textit{8. }Jedzenie i~rezawłaszczanie różnych rzeczy

 \textit{9. }Grupy wsparcia/mobilne

 W połowie, kobieta z~Bloku Pogańskiego pyta: 
 
 -- Czy mogę zaproponować jeszcze jedno? Myślę, że słyszałeś o naszej propozycji Żywej Rzeki\ldots 

-- Czy nie byłoby to rodzajem mobilnego oddziału?

-- Nie, to nie jest oddział mobilny. To cały blok sam w~sobie.

-- W porządku. 

Dopisuje:

 \textit{10. }Rzeka Humanistyczna

 Facylitatorzy próbowali uzyskać pewne wyczucie konsensusu w~sprawie Centrum Konwergencji i~pytań dotyczących postawy kolorystycznej; powiedzcie nam, że jeśli ktoś absolutnie przegapił lunch, na stole jest jeszcze trochę jedzenia; a następnie przedstawimy przedstawicieli różnych grup roboczych: Prawno-Medycznej, Mieszkaniowej i~Finansowej.

 Kolektyw prawniczy (wydaje się, że składają się głównie z~anglojęzycznych studentów z~McGill) rozdawał arkusze informacyjne i~wyjaśniał, że każda grupa afinicji powinna wskazać jednego członka, aby służył jako kontakt prawny. Ta osoba powinna dążyć do uniknięcia aresztowania i~przez cały czas śledzić, gdzie są wszyscy. Powiedzieli, że kontakt prawny powinien prawdopodobnie odbyć co najmniej jedno szkolenie prawnicze, zwłaszcza jeśli są z~USA, ponieważ przepisy są tutaj różne. Jest to również osoba, która wie, czym należy się zająć, jeśli któryś z~członków ich grupy znajomych zostanie aresztowany: kto będzie potrzebował kogoś, by nakarmić kota, okłamać szefa itp. Będą wprowadzać system używane w~akcjach masowych w~USA: członkowie każdej grupy afinicji zostaną poproszeni o wypełnienie formularza rejestrującego ich prawdziwe nazwiska -- a przynajmniej, niektóre litery ich prawdziwych imion -- wraz z~nazwami akcji, a te dokumenty będą pilnie strzeżone przez zespół prawników. W ten sposób będą mogli śledzić, kto siedzi w~więzieniu, gdy pojawią się nazwiska, i~udostępniać informacje pod specjalnym numerem telefonu. -- i~niech wszyscy od razu nie dzwonią w~sprawie osób zaginionych, jeśli dojdzie do masowego aresztowania! Tylko twoja osoba kontaktowa powinna zadzwonić pod ten numer.

 -- Czy to oznacza, -- ktoś pyta -- że  nie robimy solidarności z~uwięzionymi? Czy powinniśmy przynieść dokumenty tożsamości, czy wszyscy będą odmawiać podania swoich nazwisk po aresztowaniu? Wiele z~tego nie zostało jeszcze do końca wypracowanych.

 Medycy tłumaczą, że nikt nie powinien zakładać, że w~razie kontuzji będą mogli liczyć na służbowych ratowników medycznych i~karetki pogotowia. Zazwyczaj karetki pogotowia odmówią podjechania w~pobliże akcji. Dlatego podczas akcji zespół medyczny będzie zapewniał trzy poziomy infrastruktury medycznej: klinikę z~przeszkolonymi specjalistami, prawdopodobnie gdzieś w~pobliżu IMC; kilka zespołów ulicznych doświadczonych medyków, biegłych w~udzielaniu pierwszej pomocy, leczeniu hipotermii i~radzeniu sobie z~gazem łzawiącym i~gazem pieprzowym, a na koniec każda grupa afinicji powinna wskazać jedną osobę jako grupowego sanitariusza i~upewnić się, że osoba ta była obecna na co najmniej jednym treningu medycznym.

 Gdy zaczynają się pytania, wychodzę do przedpokoju, robię krótki wywiad z~reporterem w~zamian za filiżankę kawy, idę na spacer na zewnątrz. Spotykamy się od pięciu czy sześciu godzin. Kiedy wracam, Jaggi, reprezentujący zespół finansowy, wyjaśnia, że  organizatorzy mają obecnie około 20 000 dolarów na minusie. Następnie proszą o ochotników, którzy będą facylitować następną rundę sesji. Kończę w~grupie ,,zbliżanie się do ściany'' (upewniam się, że to jest to, co zamierza robić moja grupa afinicji i~nie ma to nic wspólnego z~faktem, że facylitatorka, młoda blondynka, wygląda uderzająco jak punkrockowa wersja Buffy pogromca wampirów). Dołącza do mnie Dean -- razem z~Emmą, która przez większość sesji zaprzyjaźniła się z~niektórymi typami z~Czarnego Bloku po drugiej stronie kręgu. Lesley mówi, że zamierza przejść się z~Lynn, żeby znaleźć miejsce, gdzie powinna mieszkać.

\noindent \textbf{ 16:30, Sesja }

 Ostatnie spotkanie tego dnia było trochę frustrujące. Teoretycznie była to najbardziej wojownicza sesja, choć nadal nie mogliśmy otwarcie dyskutować o taktyce bojowników. Była to też dziwna mieszanka: było nas dwudziestu sześciu (piętnastu mężczyzn, jedenaście kobiet, jak słusznie zapisałem w~notesie), głównie anarchiści, ale także przedstawiciele ISO, IAC i~innych marksistowskich typów, z~którymi anarchiści zwykle nie czują się komfortowo, rozmawiając o działaniach bojowych. Wszyscy wydawali się trochę niepewni co do tego, ile mogą powiedzieć. Rady delegatów z~definicji nie są naprawdę bezpiecznymi środowiskami, większość z~nas się nie znała. Każdy może być gliną.

 Zaczynamy od zbadania naszych map. Miejscowa kobieta po czterdziestce z~zielonymi pasemkami we włosach i~wydatnym kolczykiem w~nosie wyjaśniła niektóre tła dla mieszkańców spoza miasta:

\noindent Kobieta Punk: Nie jestem pewna, jak duży będzie mur. Kiedy po raz pierwszy ogłosili, miało to być 2,8 kilometra, ale teraz wydaje się, że się zmniejszył. Poproszono nas, abyśmy trzymali się z~dala od stref oznaczonych 2 i~6, czyli robotniczej dzielnicy St. John Baptiste, w~której lokalna grupa społeczności mocno nas wspiera, ale ma również nadzieję uniknąć prowokacji, które mogą spowodować, że policja zagazuje ich okolicę.

 Strefy 4, C i~B będą najtrudniejszymi obszarami, ponieważ w~rzeczywistości znajduje się tam naturalna kamienna ściana z~klifami dookoła. O tym, że stamtąd ruszamy w~kierunku Muru, możemy prawie zapomnieć.

 Jeśli jest tu ktoś, kto zna tę część miasta lepiej ode mnie, prawdopodobnie powinien wystąpić z~pomocą. Ale myślę, że wszyscy zgadzamy się, że wchodzenie przez dzielnicę robotniczą, która ucierpi przez FTAA, powinno zostać skreślone. Tak więc, to prawie zostawia nam Strefę 3, podejście od zachodu. Problem w~tym, że w~strefie będzie też najwięcej policji, ponieważ jest to główne wejście.

\noindent Siwa Broda: Tak, tak myślę. Jeśli mamy zaatakować ogrodzenie, zakładam, że będzie to dość duża grupa. Nie tylko wszystkie inne obszary są trudniej dostępne, ale nie ma miejsca na odwrót (nawet gdybyśmy mogli wspiąć się na klify, nie moglibyśmy po nich ponownie zbiec, gdyby policja zaczęła nas odpychać). Obszary C i~B znajdują się pod rzeką, nie ma też miejsca na odwrót, więc może tylko Strefa 3 jest fizycznie możliwa?

\noindent  \textit{Facylitator}: [\textit{również wpatruje się w~swoją mapę}] \ldots która jest tą, gdzie są te wielkie ulice?

\noindent  Siwa Broda: Tak, tak myślę.

\noindent  Ktoś: To północno-zachodnia część muru?

\noindent  Ktoś inny: Czy będzie wiele wejść przez mur, czy tylko jedno czy dwa?

\noindent  \textit{Facylitator}: Powiedzieli, że będzie dziewięć, ale jeszcze nie zapowiedzieli, gdzie będą.

\noindent  Craig [\textit{typ anarchistyczny z~gigantycznymi zatyczkami do uszu}]: Czy wiemy, jaki to będzie płot?

\noindent  Ktoś: Nie na pewno. Wiemy, że będzie to ogrodzenie z~siatki z~betonową podstawą, a potem drutem kolczastym na górze. Niewielki odcinek został już rozłożony na Równinie Abrahama, w~pobliżu klifów, ale nie jestem pewien, czy ktoś go już widział.

\noindent  Suzette: Strefa 3 była miejscem wielkiej demonstracji i~bitwy w~zeszłym roku o tej samej porze roku, wokół reformy szkolnej. Zakończyło się to dla nas całkiem sporym zwycięstwem na otwartym polu. Słyszałem, że Strefa E to Touristville, jeśli coś pójdzie nie tak, to powinno odbyć się tam, być może między dzielnicami mieszkalnymi?

\noindent  Kobieta Punk: Trafienie w~dwa miejsca jednocześnie może być dobrym strategicznym posunięciem, a także, jeśli mówimy o Strefie 3\ldots  czy łatwiej byłoby nadchodzić (jakiego języka możemy tu używać? z~wizytą? Atakiem?) do miejsce, w~którym Mur się otwiera i~zamyka. Inną możliwością może być wcale nie atakowanie ogrodzenia, ale zamknięcie głównego wejścia; może je blokując. To skutecznie odcięłoby gliny od reszty z~nas. Trzecią opcją (może czymś do zrobienia w~innym miejscu) może być zdobycie haków z~liną i~faktyczne ściągniecie ogrodzenia przy pomocy ludzi. Czy będzie to możliwe? Właściwie nie wiem, czy betonowa część zostanie przyklejona do podłoża, ale prawdopodobnie nie będzie.

\noindent  Lesley: Już będzie!!

\noindent  [\textit{Dużo śmiechu i~spoglądanie na niewidzialne mikrofony w~suficie}]

\noindent  Młoda Francuzka: Czy wiemy, czy siły bezpieczeństwa całkowicie otoczą Płot? A może w~ich liniach będą luki?

\noindent  Szara broda: Cóż, wiemy, że będzie pięć tysięcy policjantów do ochrony może dwóch, może trzech kilometrów ogrodzenia. Nie jestem pewien, jak to się tłumaczy.

 Przypuszczalnie nie będą równomiernie rozproszeni, przy bramach będą duże jednostki, tu i~ówdzie małe oddziały.

\noindent \textit{Facylitator}: Czy ktoś ma jakąś propozycję do umieszczenia w~formalnym porządku obrad? Ponieważ, wiecie, właściwie nie mamy jeszcze planu.

 Wydaje się, że nie ma to większego sensu i~postanawiamy rozmawiać nieformalnie. A więc: jaki byłby najlepszy dzień na próbę przełamania muru? CLAC mówi tylko o piątku, 20-tego, ale wielki marsz robotniczy odbył się w~sobotę i~to byłoby co najmniej czterdzieści pięćdziesiąt tysięcy ludzi. Jak zawsze, przywódcy związkowi robili wszystko, co możliwe, aby trzymać swoich ludzi z~dala od akcji. Marsz zaczynał się w~miejscu dość daleko od Szczytu, a następnie kontynuował w~przeciwnym kierunku. Gdyby jednak udało się skierować nawet część maszerujących w~stronę samego muru, zmieniłoby to całkowicie układ sił. Wiele osób zwraca uwagę na to, że coś takiego jest nieprawdopodobne. Od Seattle biurokraci związkowi stali się niezwykle dobrzy w~zapewnianiu, że to się nigdy nie wydarzy. Inni zauważają, że Kanada jest inna. Wreszcie, wszyscy poddajemy się autorytetowi starca w~rybackiej czapce i~postrzępionej brodzie, który do tej pory w~dużej mierze milczał. Wyjaśnia po francusku, że dorastał na starym mieście i~może mieć pewne spostrzeżenia, których nie mają inni. Po chwili, widząc, że przyjezdni zwracają szczególną uwagę, przechodzi na angielski:

\noindent Rybak: To prawda, nie wiemy, gdzie będzie policja, ale możemy założyć, że nie będzie ona przebywać tylko na obwodzie, zbliżenie się do niej może być bitwą samą w~sobie. Jeśli tak, to jeśli mamy być pod ostrzałem gazu łzawiącego itp., kiedy się zbliżamy, myślę, że nie powinniśmy zbliżać się z~naszych własnych dzielnic. Istnieją dwie szerokie arterie: jedna to René Lévesque, druga Grand Allée, która biegnie równolegle do jej południa. To są ulice burżuazji. Obie są ulicami, na których mieszkają wysokie szarże biurokracji i~bogaci ludzie; więc byłby to dobry obszar, z~którego można przejść na Mur.

\noindent David: Nad tym zastanawiała się moja grupa afinicji: jeśli jakimś cudem dostaniemy się do środka 20-go, no cóż, to co dalej? Słyszeliśmy, jak mówiono o zakłócaniu ceremonii otwarcia, choćby przez naszą obecność, albo w~jakiś sposób odcięcie centrum kontroli mediów.

\noindent Dean: Czy w~środku będziemy mogli wmieszać się w~tłum? Czy będą strażnicy sprawdzający osoby z~przepustkami?

\noindent Rybak: Nie jest to jasne. Wiele zależy od tego, jak bardzo zagrażamy według nich 

\noindent [\textit{ponownie zerkamy na wyimaginowany mikrofon}]. 

Jeśli po tej radzie delegatów poczują, że Mur jest niepewny, zmniejszą obszar i~łatwiej będzie go bronić. Dzięki temu w~środku będzie mniej zwykłych obywateli. Dzięki temu będą mogli lepiej widzieć, kto jest kim (czyli będzie więcej garniturów, mniej ludzi ubranych jak my); ale wtedy też będziemy mogli zobaczyć, kto jest kim. Jeśli w~końcu będą musieli zrobić z~tego kapitalistyczne getto, nawet jeśli oznacza to, że mogą robić w~nim, co chcą, to samo w~sobie jest dla nas wielkim zwycięstwem, a atak na tę przestrzeń, nawet czysto symboliczny, byłby również wielkim zwycięstwem.

 Stopniowo zdałem sobie sprawę, co się dzieje. Jak wspomniałem, na każdym takim spotkaniu zakładaliśmy, że ktoś w~pokoju jest gliną (odniesienia do mikrofonów były głównie sposobem na uprzejmość). Dlatego jedyną osobą, która mogła swobodnie rozmawiać, był ten człowiek, który rzeczywiście uważał, że znajomość naszych planów jest taktycznie korzystna dla policji. Wszyscy inni zaczynali wyglądać coraz bardziej niespokojnie i~nieswojo. W końcu ktoś zasugerował, że posunęliśmy się tak daleko, jak się da, i~zrobiliśmy sobie przerwę na obiad; z~Emmą i~kilkoma innymi osobami przekazującymi wiadomość, że ci, którzy naprawdę poważnie podchodzili do projektu i~mieli kogoś, kto mógłby za nich ręczyć, spotkają się później na imprezie CASA tej nocy, aby ponownie się zebrać. Tymczasem w~naszym oficjalnym raporcie napiszemy, że jest za wcześnie na jakiekolwiek konkluzje, ale chcemy zwołać radę przedstawicieli, żeby zaplanować określone działania kilka dni przed Szczytem, kiedy będziemy mieć pojęcie, jak sprawy rzeczywiście będą wyglądać.

\noindent \textbf{ 20:00, Impreza w~Scanner }

 Impreza odbyła się w~miejscu zwanym Scanner Bistro, ,,klubie multimedialnym'' z~kawiarenką internetową i~barem na dole oraz małą estradą. Na górze był kolejny bar, stół bilardowy, piłkarzyki, flipper Judge Dredd oraz porozrzucane monitory i~głośniki na ścianach, które umożliwiały oglądanie i~słuchanie występów na żywo ze sceny na dole. Kiedy weszła nasza ekipa -- około dwunastu z~grupy sesyjnej, w~tym większość nowojorczyków -- na scenie pojawiły się dwie kobiety, które wykonywały jakiś kawałek mówionego słowa w~bardzo kolokwialnym francuskim. Później był człowiek, o którym myślę, że był komikiem; Powiedziano nam, że zespół wyjdzie później, ale do tego czasu nikt z~nas nie zwracał na to większej uwagi. Skończyliśmy na górze, szukając stolika, bo w~końcu ktoś znalazł odpowiednią mapę.

 Albo prawie wszyscy. Dean podszedł prosto do stołu bilardowego, gdzie wkrótce wdał się w~długą rozmowę z~chudym, jasnowłosym facetem, z~którym ostatecznie miał mieć burzliwy sześciomiesięczny romans.

 Znaleźliśmy miejsce w~kącie, w~miejscu, gdzie wcześniej był darmowy obiad. Zsunęliśmy kilka stołów i~szybko zajęliśmy się pozostałym jedzeniem, które składało się z~ogromnej miski ryżu, dania z~fasolą i~warzywami w~sosie pomidorowym oraz kilku bochenków francuskiego chleba i~margaryny. Weganie nie tknęliby margaryny, ale wszyscy żuli chleb przez pierwszą połowę dyskusji. Duże mapy miasta zostały rozłożone na powierzchni stołu i~przyklejone taśmą. Wszyscy się stłoczyli, a rozmowa trwała godzinami, dzbanki piwa okresowo pojawiały się znikąd, zawsze do kolejnego zbiorowego toastu ,,rozwal państwo!”.

 To było idealne spotkanie, może poza tym, że byliśmy tuż pod głośnikami, a w~połączeniu z~otaczającym hałasem dziesiątek świątecznych rozmów sprawiało, że było to trochę trudno słuchać. Rzeczywiste spotkanie zawsze dotyczyło siedmiu lub ośmiu osób w~centrum w~dowolnym momencie, które rzeczywiście mogły się nawzajem słyszeć, zwykle z~kilkoma innymi wiszącymi na krawędziach, czekającymi na wejście. Nigdy nie trwało to zbyt długo. Ktoś zawsze leciał po piwo, palił jointa lub korzystał z~łazienki, a potem musiał czekać na marginesie, kiedy wracał. Mimo to utrzymywaliśmy to przez jakieś trzy godziny, małą bańkę aktywistycznej intensywności, prawie całkowicie nieświadomi coraz bardziej hałaśliwej imprezy tanecznej, która w~końcu nas ogarnęła, a później wieczorem zaczęła zanikać.

 To tutaj w~końcu zaplanowaliśmy atak na mur. Przeglądanie możliwości nie trwało długo. Nawet jeśli udało się wejść do strefy bezpieczeństwa, nie było żadnej oczywistej rzeczy do zrobienia, gdy już bylibyśmy w~środku. Zawieszenie banerów byłoby możliwe, ale prawdopodobnie wymagałoby to współpracy właścicieli domów wokół mura, bez wątpienia byliby tacy, ale sami mogliby zawiesić banery. Moglibyśmy zająć budynek, ale doprowadziłoby to do absolutnie pewnego aresztowania i~nie było jasne, po co. Było tylko jedno rozwiązanie. Musieliśmy zniszczyć Mur. Było to całkowicie uzasadnione. Wyświadczylibyśmy usługę publiczną. Przywódcy wszystkich stanów obu Ameryk przybywali do tego miasta, aby ustawić płoty w~sąsiedztwie dzielnic; my, anarchiści, przybywaliśmy, aby je obalić. Pytanie brzmiało jak, i~większość następnych trzech lub czterech godzin spędziliśmy na rozważaniu możliwości: haków, szczypiec, taktyki, narzędzi, dywersji, kątów podejścia. Normalne szczypce do cięcia drutu nie są w~rzeczywistości wystarczająco mocne, aby przeciąć ogniwa ogrodzenia ochronnego; są jednak wystarczająco mocne, aby odciąć druty łączące ogniwa łańcucha z~pionowymi słupkami. Po odcięciu była to kwestia wagi: przynajmniej jedna osoba musiała wspiąć się na szczyt ogrodzenia i~odchylić do tyłu, gdy inni ciągnęli. Ewentualnie płoty mogły być obalone przez niewielki zespół uzbrojony w~haki i~liny. Prawdopodobnie najlepszym rozwiązaniem byłoby nierozpoczynanie wszystkiego w~tym samym miejscu. Powinniśmy mieć kilka kolumn. Najlepiej trzy, każda z~własną osobliwą taktyką. Spotkanie Ya Basta! mogłaby zejść się jedną wielką aleją, Czarny Blok następną, osoby z~CLAC/CASA (z których żaden nie był obecny) ruszyli by trzecią. W ten sposób każdy zbliżyłby się do innej części ogrodzenia, ale wszyscy byliby w~zasięgu wzroku. Każdy miałby też swój własny styl: ludzie z~CLAC bardziej bojowi, Ya Basta! głupsi, Czarny Blok mobilniejszy. Członkowie Toronto i~Montrealu Ya Basta! -- dwóch grup, o których do tej pory słyszałem tylko niejasne plotki -- obiecali poprowadzić do działania każdego innego Yabbasa, ponieważ znali teren.

 W rzeczywistości odkryliśmy, że będą cztery różne kontyngenty Ya Basta! : dwa z~Kanady, jeden z~Nowego Jorku i~jeden z~Connecticut. Ten ostatni był reprezentowany przez młodą kobietę, którą wszyscy znali jako ,,Kitty z~Connecticut'', studentkę muzyki w~Connecticut College, którą znałem jako aktywistkę CGAN (Connecticut Global Action Network). Kitty właśnie dotarła do miasta i~przegapiła większość rad, ale skierowała się bezpośrednio na spotkanie w~klubie Scanner. Byłem naprawdę zadowolony, że ją widzę; była utalentowaną facylitatorką i~wszechstronnie imponującą aktywistką (CGAN odniosła już dwa duże zwycięstwa w~ciągu ostatniego roku: pierwsze, kiedy zablokowali centrum Hartford sojuszem anarchistów i~dozorców, drugi, kiedy prawie w~pojedynkę udało im się zmusić lotnisko Hartford do rozstrzygnięcia strajku z~pracownikami restauracji, proponując akcję, aby wesprzeć pikiety, co najwyraźniej sprawiło, że kierownictwo przekonało się, że czeka ich horda szalejących Czarnych Bloków). W tej chwili jednak była zainteresowana głównie znalezieniem kogoś, kto mógłby jej załatwić skręta. Zniknęła, ktoś z~Wysp Księcia Edwarda wsunął się na jej krzesło, a Sasha, świeżo z~IMC, zajął pozycję, którą osoba opuściła przy pobliskim stole.

 Konferencja była kontynuowana. Jeśli amerykańskiemu Ya Basta! nie udałoby się przedostać przez granicę, musielibyśmy zredukować plan do dwóch kolumn. Ciągle musieliśmy sobie przypominać, że prawdopodobnie nie bylibyśmy w~stanie po prostu podejść do ogrodzenia; bardziej prawdopodobne jest, że musielibyśmy walczyć o ostatnie trzy bloki, aby nawet znaleźć się w~pozycji, aby zacząć używać przecinaków do drutu. A kiedy już tam byśmy dotarli, potrzebowalibyśmy co najmniej czterech do sześciu minut, żeby obalić ogrodzenie. Tak więc plan zadziałałby tylko wtedy, gdyby przyłączyła się do nas większa liczba innych protestujących. Prawdopodobnie w~końcu połowa Żółtego Bloku zostanie zainspirowana do przyłączenia się, a druga połowa ucieknie. To, czy będzie ich wystarczająco dużo, abyśmy mogli wywalczyć sobie drogę do muru, zależało od łącznych liczb i~nikt nie miał jasnego pojęcia, jakie będą te liczby. Kolumny mogą liczyć od kilkuset do kilku tysięcy. Tak naprawdę wszystko zależało od miejscowych studentów. Z pewnością wydawali się wystarczająco wojowniczy. Ale czy się przedostaną?

 Około pierwszej w~nocy, po szesnastej rundzie ,,rozwalić państwo!'', stworzyliśmy wezwanie do działania -- nazwane, ponieważ dobrze brzmiało, ,,Porozumienia z~Scanner''. Zaczęło się tak: ,,Wzywamy wszystkich, którzy czują się osaczeni przez mury, aby przybyli do Quebec City''. Tak naprawdę to był tylko akapit, ale jakoś dopiero po jego opublikowaniu spotkanie wydawało się kompletne. Napisaliśmy pięć lub sześć zdań na kartce papieru, zredagowaliśmy je wspólnie, umieściliśmy tekst anonimowo na stronie internetowej IMC gdzieś w~Stanach Zjednoczonych, z~adnotacją, że ma być przekazywany dalej. Potem wyszliśmy na zewnątrz i~podpaliliśmy papier. Sasha zaproponował, że  sfilmuje rytuał, ale ktoś się sprzeciwił, na wypadek, gdyby można było użyć zaawansowanych technologicznie środków do zebrania odcisków palców ze zbliżeń naszych dłoni. (Prawie wszystkim wydawało się to śmieszne, ale jak się każda osoba uczy, w~kwestiach ,,kultury bezpieczeństwa'', najlepiej jest nie dyskutować). Ruszyliśmy do domu, umawiając się na spotkanie o 13 jutro, gdy rady będą się kończyć, aby rozpoznać teren, gdzie najprawdopodobniej odbyłoby się pierwsze podejście do muru.


\section{Niedziela, 25 marca}
\noindent \textbf{ DZIEŃ 2 CONSULTY}\medskip


 Nasza grupa przespała oficjalną wycieczkę CLAC/CASA po mieście, która miała odbyć się rano, ale udało nam się dotrzeć na rady około godziny 11:00, dla odmiany, kiedy się właśnie zaczynały. (Właściwie to miały się zacząć się o 10 rano, ale wydawało się, że mamy do czynienia z~poważnym przypadkiem ,,czasu aktywistów''. Osób było mniej niż dzień wcześniej, ale niewiele. Lesley, Dean, Lynn i~ja zrekonstruowaliśmy nasze małe gniazdko, teraz gdy dołączyły do  nas Sasha i~Kitty, Emma wyjechała ze swoimi nowymi przyjaciółmi z~Czarnego Bloku. Zespół CLAC też się zmienił: Jaggi nie był już tłumaczem, ale tym razem pomagał, razem ze starszą kobietą, której wcześniej nie widziałem.

\noindent \textbf{11:00, Spotkanie Plenarne}

 Spotkanie rozpoczęło się raportami z~sesji podgrupy poprzedniego wieczoru; następnie rozważylibyśmy szereg konkretnych propozycji. Myślę, że raporty zwrotne są warte udokumentowania, ponieważ dają pewne wyobrażenie o tym, jak poprzez takie otwarte i~czasami pozornie bezproduktywne dyskusje, plany działania mogą naprawdę przybrać formę. W każdym przypadku chodziło o stworzenie podsumowania pomysłów, które rzecznicy mogliby zabrać z~powrotem do swoich grup afinicji w~Ameryce Północnej, aby ustalić, które chcieliby rozwinąć i~podłączyć, oraz zapewnić środki do pozostawania ze sobą w~kontakcie (zwykle e-mail).

\noindent 1. ,,Świąteczny artystyczny rodzaj grupy”

 Postanowiliśmy zadbać o to, by w~całym mieście odbywały się imprezy. Jeden pomysł: zapewnić świąteczne występy, które wspierałyby blokady, nie będąc w~nich faktycznie częścią. Innym było przekształcenie Muru w~rodzaj sztuki (to znaczy, zanim zostanie zaatakowany). Możemy go animować, dekorować. Mówiliśmy o potrzebie tworzenia bardzo dużych obiektów, takich jak lalki, z~dużym wyprzedzeniem i~zapewnienia przestrzeni, w~której można to zrobić. Jeśli chodzi o materiały eksploatacyjne: tkaniny, odpadki, ze znalezionych przedmiotów można zrobić wiele rzeczy. Prosimy wszystkich, aby odłożyli na bok wszystko, co znajdą, co można wykorzystać do kostiumów, rekwizytów lub projektów budowlanych.

 Chcielibyśmy również poruszyć kilka drobnych kwestii: usłyszeliśmy wiele pomysłów na bębny, teatr uliczny, lalki; dużo tego oczekujemy. Niektórzy zasugerowali, aby być może również wydzielić jakiś obszar na trwające ciche lub nieruchome czuwanie, aby reprezentować głosy uciszone przez tego rodzaju szczyt.

\noindent 2. Obrońcy Centrum Konwergencji

 Zdecydowaliśmy, że obrona Centrum Konwergencji jest rzeczywiście priorytetem i~że zastosujemy trzy metody:

 a. ciągły nadzór wewnątrz i~na zewnątrz

 b. organizacja ewakuacji osób i~materiałów w~przypadku ataku

 c. organizacja czynnego oporu wobec wszelkich policyjnych prowokacji lub ataków.

\noindent 3-4. Grupy blokujące

 W końcu nie miało większego sensu mieć dwóch różnych grup roboczych ds. blokady, więc się połączyliśmy.

 Większość z~nas opowiada się za blokowaniem autostrad, ale nie jesteśmy w~stanie wysunąć konkretnych propozycji. Pojawia się też pytanie, jak przynieść sprzęt (np. narzędzia), który byłby niezbędny do utrzymania naprawdę skutecznej blokady. Granica jest dużym problemem dla ludzi z~USA, którzy w~innym przypadku mieliby dostęp do takich rzeczy; również CLAC/CASA jest zbyt zajęty, aby to zorganizować. Sugerujemy, aby grupy afinicji z~wyprzedzeniem umawiały się z~przyjaciółmi w~innych miejscach w~Kanadzie, na przykład w~Maritimes, w~celu dostarczenia rzeczy, gdyby zostały wysłane tutaj, prawdopodobnie zostałyby przechwycone. Zdecydowaliśmy, że miasto powinno zostać podzielone na strefy, aby mieć pewność, że wszystko jest pokryte.

 Jest też konkretna propozycja od GOMM dotycząca planu zorganizowania blokady w~stylu świątecznym dla trzystu lub więcej osób w~pobliżu centrum miasta.

 Dyskutowano o możliwości zablokowania lotniska, być może autostrad niektórych głów państw, ale nie dyskutowano o konkretnych propozycjach.

 Innym pomysłem było zablokowanie poszczególnych symboli kapitalizmu; jak pociągi czy centra handlowe. Ktoś zaproponował spotkanie organizacyjne w~tej sprawie o 15:30 po południu, po tym spotkaniu. To była 15:30, prawda?

\noindent [Kobieta w~koszulce Spider-Mana: 15:30, zgadza się.]

 Wreszcie pojawił się pomysł zablokowania niektórych głównych mass-mediów i~żądania odtworzenia przygotowanej taśmy, przedstawiających niektóre z~naszych głównych zastrzeżeń do traktatu.

\noindent 5. Okupacje budynków

 W Québec City są trzy uczelnie, a jedna z~nich jest już zajęta przez OQP. W przypadku pozostałych dwóch dyskutujemy, czy i~jak je okupować.

\noindent 6. Chodzenie/posuwanie się/zwiedzanie/ruch w~kierunku ściany

 Wiele grup afinicji wyraziło chęć złożenia wizyty na granicy muru. Wyraźnie widać było silną chęć podjęcia tego i~poczucie, że ma to służyć wielu celom: zakłócić granicę, zakłócić szczyt, a może nawet przeniknąć. Ale to chyba tak daleko, jak możemy się posunąć w~tym kontekście. Wiele informacji wciąż wymaga wyjaśnienia, a większość logistyki wymaga dopracowania. Musielibyśmy zadecydować o oficjalnych punktach wizyt i~środkach, jakie należy zastosować, aby wprowadzić ewentualne poprawki w~ścianie. Czy to przez masową mobilizację, czy przez oddzielne działania grupowe? Ponieważ jest tak wiele do rozważenia i~tak wiele zależy od liczb, informacje są jeszcze niedostępne, sugerujemy, aby rady delegatów zebrały się kilka dni po tym, gdy Płot zostanie ostatecznie postawiony, w~celu podjęcia decyzji.

\noindent 7. Redekoracja miejskich obiektów

 Lub, jak sądzę, powinno to być właściwie ,,miejski krajobraz''. (Były pewne problemy z~tłumaczeniem. Głównie wydają się dotyczyć rozsądnego używania farby w~sprayu i~innych materiałów artystycznych). Nie mieliśmy formalnego spotkania, naprawdę, ale po prostu się przywitaliśmy, a potem wszyscy poszli dołączyć do innych grup. Zalecamy, aby te kwestie pozostawić poszczególnym grupom afinicji. Nie ma niczego, co naprawdę wymagałoby koordynowania w~skali całego miasta.

\noindent 8. Ponowne zawłaszczenie żywności i~innych przedmiotów

 Czerwone, Żółte i~Zielone Bloki mogą użyć różnych środków, aby odzyskać rzeczy. Naszym pomysłem jest przeprowadzenie zwiadu potencjalnych miejsc dla komandosów żywnościowych (\textit{commando du boeuf}). Manifest Żywności zostanie napisany, aby wyjaśnić, dlaczego ma miejsce ten rodzaj akcji.

 Na marginesie: Montreal Food Not Bombs przygotowuje obecnie dużą ilość żywności, która zostanie zamrożona i~przywieziona na wspólną ucztę, być może odbędzie się pod autostradą w~piątek lub sobotę wieczorem.

\noindent 9. Oddziały lotne (\textit{groupes mobiles})

 Zadaniem oddziałów lotnych jest zapewnienie wsparcia gorących punktów podczas akcji; także, aby wykorzystać możliwości, które mogą się nagle otworzyć. Wszystko to oczywiście zależy od posiadania dokładnych informacji o tym, co się dzieje. System komunikacyjny jest niezbędny i~nie jesteśmy pewni, jaka infrastruktura komunikacyjna (radio? walkie-talkie?) została już skonfigurowana. Wyobrażamy sobie liczne stosunkowo małe grupy po trzy, cztery lub pięć osób, dobrze ze sobą skoordynowane. Sami zdecydują, który z~trzech bloków będą wspierać, na co odpowiadają. Koordynacja jest już organizowana.

\noindent 10. Żywa rzeka

 Zdecydowaliśmy\ldots  cóż, ta akcja jest organizowana z~poganami z~Vermont. 
 
 \noindent [\textit{Obecnych jest pięciu członków Klastra Pogańskiego: cztery kobiety, jeden mężczyzna. Starhawk} z~nimi nie ma. Wszyscy siedzą, nieco niestosownie, na krzesłach. Zauważam, że są w~większości starsi, więc to mogą być po prostu problemy z~plecami.] 
 
 Będziemy brać za sojusznika rzekę świętego Wawrzyńca, używając jej, razem z~ogólnie użyciem tematu wody, jako reprezentacji przeciwko czemu walczymy i~o co walczymy, jako formę, która pozwoli nam łatwo poruszać się tam i~z powrotem od jednej akcji lub jednego rodzaju akcji do drugiej.

 Osoby, które chciałyby wziąć udział, prosimy o przyniesienie niebieskiego materiału, wstążek, odzieży. Chodzi o to, aby stworzyć coś w~rodzaju Niebieskiego Bloku\ldots 

\noindent [To jest przetłumaczone. ,,O nie! Jeszcze jeden blok'', wzdycha jeden z~facylitatorów. Wszyscy się śmieją.]

 \ldots w~ten sposób nie będziemy ograniczać się do jednej strefy czy stylu działania. Ludzie mogą się przyłączyć, strumyki mogą się rozdzielić, strumienie znów spłyną razem. Jeśli ludzie chcą pozostać przy blokadzie, mogą to zrobić; inni mogą odprawiać ceremonie lub oferować wsparcie dla innych grup.

\noindent Poganin: Jeśli ludzie chcą dołączyć, zachęcamy ich do wcześniejszego dołączenia do grup afinicji. Niekoniecznie do przyłączenia się jako jednostki.

 O tak: naszym innym tematem jest swobodny dostęp do wody dla wszystkich ludzi, zainspirowany deklaracją Cochabamba. W konsekwencji zapewnimy wszystkim butelkowaną wodę i~zachęcamy ludzi do przynoszenia próbek wody z~waszych domów, aby wziąć udział w~jednym wielkim rytuale, który odbędzie się w~tym samym czasie, co ceremonia otwarcia Szczytu.

\noindent \textit{Facylitator}: Mamy bardzo krótki czas na pytania, tylko pięć minut, bo inaczej może to trwać w~nieskończoność.

\noindent Rzecznik Drużyny Lotnej: Och, grupa mobilna zapomniała dodać: będziemy mieć listserv, żeby porozmawiać o sprzęcie komunikacyjnym, bo to jest bardzo ważne. Możesz zarejestrować się na stronie CLAC.

\noindent Brodaty mężczyzna: Jeden z~pomysłów, który wyszedł z~naszej pierwszej sesji w~sprawie Centrum Konwergencji, polegał na tym, by zrobić coś podobnego do LA i~uzyskać nakaz prawny, który powstrzyma gliny i~strażaków przed wejściem. Chcemy się upewnić, że dział prawny jest tego świadomy.

\noindent [Następuje kilka pytań, ale wiele ogłoszeń o konfigurowaniu listservs, informacje kontaktowe i~tak dalej.]

\noindent \textit{Facylitator}: Przejdźmy więc do nowych propozycji. Przypominam ludziom, że musimy wyjść przed piątą.

 Pierwszą propozycją była po raz kolejny obrona Centrum Konwergencji. Dla żadnego z~nas nie było jasne, dlaczego ta propozycja musiała zostać powtórzona, w~rzeczywistości była to dokładnie ta sama propozycja, którą przedstawił wcześniej CLAC. Przypuszczalnie była to jakaś formalność. Teoretycznie, każda propozycja powinna być oparta o pięć osób za, pięć przeciw, ale ponieważ nikt nie był zainteresowany wypowiedzeniem się przeciwko propozycji, uznano, że została ona przyjęta i~przeszliśmy do następnej.

 Kolejna była o wiele ciekawsza, bo wprowadzała element ostrego konfliktu. Daje też przykład tego, jak faktycznie funkcjonuje konsensusowe podejmowanie decyzji (ponieważ pomimo formalnych zasad skutecznie korzystaliśmy z~systemu ,,zmodyfikowanego konsensusu'', przede wszystkim dlatego, że konflikt nigdy wprost nie wyszedł na jaw. Propozycje budzące zastrzeżenia są rzadko odrzucane. Nawet jeśli teoretycznie jedna osoba ma prawo zawetować (,,zablokować'' propozycję, prawie nigdy się to nie zdarza: zamiast tego istnieje proces, który można by prawie opisać jako zabijanie z~miłością.

 Propozycję przedstawiła młoda kobieta w~dużym białym swetrze z~warkoczami i~różowej wełnianej czapce:

\noindent Różowa czapka: Wśród grupy blokującej uznaliśmy, że bardzo przydatne byłoby utworzenie komitetu taktycznego.

 Taki komitet składałby się z~ludzi chętnych do wcześniejszego przyjścia, a także oczywiście z~CLAC/CASA. W ten sposób będzie mógł rozejrzeć się po mieście w~miarę wznoszenia się muru, dowiedzieć się, jakie hotele lub inne ważne miejsca należy uderzyć, aby jak najbardziej zakłócić Szczyt. Tak więc, kiedy różne osoby przybędą do Centrum Konwergencji w~środę i~czwartek, będziemy mieli plan, dzięki któremu będziemy mogli skierować ludzi do najlepszych miejsc, w~których mogą wywrzeć wpływ, zakłócić, a nawet zatrzymać Szczyt. Mam nadzieję, że o tym będzie spotkanie o 15:30, więc \textit{proszę}, przyjdź, jeśli masz jakieś spostrzeżenia lub po prostu chcesz pomóc w~jakikolwiek sposób.

\noindent \textit{Starszy Facylitator}: Jaka jest zatem propozycja? Stworzyć taką grupę? Czy po prostu zapraszasz ludzi na spotkanie, czy składasz oficjalną propozycję?

\noindent Różowa czapka: Czujemy, że potrzebujemy pomocy miejscowych, aby to pociągnąć. Chcemy więc wiedzieć: czy jest to pomysł przyjęty przez grupę? Bo jeśli nie, to nie możemy tego zrobić. Chodzi o to, aby wziąć pod uwagę przeszłe doświadczenia, zastanowić się, co się udało, a co nie w~Seattle, DC i~tak dalej.

\noindent \textit{Starszy Facylitator}: Czy zatem, co do wniosku, są jakieś kwestie wyjaśniające lub pytania?

\noindent Kobieta w~tęczowych dredach: Czy to wezwanie od jednej grupy, czy zdecentralizowane, otwarte dla wszystkich? Ponieważ w~CLAC staraliśmy się opracować proces, który zapewni, że żadna pojedyncza grupa nie zdominuje koordynacji. Uważamy, że to bardzo ważne.

\noindent Różowa czapka: Wyobrażamy sobie różnych ludzi, ludzi z~wielu grup afinicji, ludzi z~różnych części USA i~Kanady, którzy zbierają się z~pomysłami. To byłoby jak spin-off rady delegatów. Aby zapewnić, że gdy przyjdą tysiące ludzi, naprawdę możemy zamknąć miasto, naprawdę wywrzeć wpływ na Szczyt.

\noindent \textit{Starszy Facylitator}: Widzę jeszcze jedno wyjaśniające pytanie.

\noindent Amerykanka: To nie jest pytanie, ale: jeśli to wszystko jest scentralizowane tylko wokół blokad ulic\ldots 

\noindent \textit{Jaggi}: Um, w~tym momencie prosimy tylko o pytania wyjaśniające.

\noindent  Inna Amerykanka: Właściwie uważam, że facylitator wezwał do ,,wyjaśnienia punktów lub pytań''. Więc mam też jedną z~nich. Propozycję tę składają ludzie, którzy wywodzą się z~grupy blokującej i~chociaż wszyscy uważamy to za ważne, mamy również nadzieję, że każda taka komisja uwzględni taktykę innych grup afinicji, aby pomóc nam koordynować całą akcję jako całość, bez centralizacji. Byłoby przydatne, gdyby to było kanałem informacyjnym, aby ludzie wiedzieli, skąd wziąć informacje taktyczne, aby działania były jak najbardziej efektywne.

\noindent  \textit{Starszy Facylitator}: Proszę nie interweniować, prosimy o uzupełnienie lub wyjaśnienia. Pytanie od mężczyzny?

\noindent  Mężczyzna: W porządku, chciałbym wyjaśnić, czy komitet będzie tylko zbierać informacje, czy przedstawiać sugestie. A może będzie miał jakieś inne funkcje? To znaczy, czy miałby inne funkcje niż bank informacji?

\noindent  Różowa czapka: To byłoby jedno i~drugie. Więc kiedy ludzie przyjadą spoza miasta, będą mieli pojęcie, gdzie są ważne miejsca, ponieważ mogą nie znać miasta\ldots 

\noindent  \textit{Starszy Facylitator}: Pytanie od kobiety?

\noindent  \textit{Kitty }[która wcześniej podniosła rękę]: Nie, rezygnuję.

\noindent  \textit{Starszy Facylitator}: Zatem kobieta w~szarym kapeluszu.

\noindent  Lesley: Myślałam, że istnieje już lokalny komitet działania. Chciałabym wiedzieć, jaka będzie ich rola w~stosunku do tego nowego taktycznego.

\noindent  Różowa czapka: Jest komitet akcji? A więc gdzie jest?

\noindent  Nicole: W CLAC/CASA istnieje komitet ds. działań, stworzony do zajmowania się logistyką i~proponowania działań. Jeszcze o tym nie rozmawialiśmy, ale gdybyśmy to zrobili, prawdopodobnie bylibyśmy szczęśliwi, dzieląc się doświadczeniami, ponieważ pomogłoby nam to w~pracy.

\noindent  Kobieta w~koszulce Spider-Mana: Tak, też uważam, że to świetny pomysł.

\noindent  Różowa czapka: Naprawdę czuję, że jest to coś, w~czym możemy razem pracować, aby naprawdę wywrzeć wpływ na Szczyt.

 Ciekawą rzeczą w~tej rozmowie jest delikatność, z~jaką została przeprowadzona. Wtedy miałem tylko intuicję, co się dzieje. Z pewnością wydawało mi się to trochę dziwne, że kobieta składająca propozycję używała w~kółko tych samych zwrotów (,,naprawdę wywierając wpływ na Szczyt''), a później, że jej główna zwolenniczka, kobieta w~T-shircie Spider-Mana, używała zdumiewająco podobnych określeń. Normalnie, słowo ,,komitet'' również byłoby wskazówką. Anarchistka powiedziałaby ,,grupa robocza'', ale byliśmy w~obcym środowisku, więc nierozsądne wydawało się wczytywanie w~dobór słów. W miarę upływu czasu stawało się coraz bardziej jasne, że nadepnięto na palce u stóp, ale etos niekonfrontacji był taki, że nikt nie chciał wyrazić tego wprost. Raczej prawie wszystkie odpowiedzi były bardzo konstruktywne, przynajmniej w~tonie.

\noindent Kobieta: W grupie oddziałów mobilnych wielu z~nas zauważyło, że w~przeszłości pojawiał się problem z~nierzetelnymi informacjami: Grupy mobilne trafiają w~jakieś miejsce na podstawie plotek, które okazują się nieprawdziwe. Czy ta komisja udzieliłaby nam pomocy w~komunikacji?

\noindent Spider-Man: Tak, absolutnie

\noindent Różowa czapka: Tak.

\noindent Medyk w~niebieskiej bandanie: A czy ta komisja będzie odpowiedzialna przed radą delegatów? Jeśli tak, to jak by to działało w~rzeczywistości?

\noindent  Spider-Man: Odpowiedź na pierwsze pytanie brzmi tak, dostarczyłaby informacji każdemu, kto jest w~radzie. Po drugie: to zależy od tego, kto bierze udział, ale sądząc po poprzednich akcjach, może skończy się to podziałem miasta na sekcje. Więc jeśli pojawi się grupa afinicji i~ludzie powiedzą: ,,Chcemy przejść do Żółtego Bloku, chcemy znaleźć blokadę, ale nie będziemy opierać się aresztowaniu'', możemy powiedzieć: ,,Cóż, wiemy, że potrzebują tu następnych pięćdziesięciu osób w~tym sektorze''. Grupa może również pomóc w~ułatwieniu zbierania sprzętu.

\noindent  \textit{Starszy Facylitator}: Pozwolimy, aby rozmowa trwała przez piętnaście minut, czyli maksimum, na które zdecydowaliśmy się pozwolić na konkretne propozycje, ponieważ przy obecnym tempie nie będziemy mogli zdecydować. Przejdźmy od wyjaśniania pytań do wątpliwości.

\noindent  Eric: Właściwie nadal chciałbym coś wyjaśnić. Brzmi to podobnie do tego, o czym rozmawialiśmy w~grupie zespołów lotnych, ponieważ ludzie zdawali się nie wiedzieć, co było już na miejscu w~zakresie komunikacji lub taktyki. Musimy jakoś wymyślić, w~jaki sposób oddziały taktyczne, komunikacyjne i~mobilne CASA/CLAC mają ze sobą współpracować.

\noindent  Kobieta w~tęczowych dredach: Uważam, że pomysł utworzenia grupy strategicznej jest interesujący, ale chcę się upewnić, że nie będzie tu powielania pracy. CLAC/CASA utworzyły niedawno grupę ds. komunikacji, więc chcę zapewnić, że ten komitet będzie tylko koordynował blokady.

\noindent  Spider-Man: Zapraszamy do przyłączenia się do grupy.

\noindent  Różowa Czapka: Chcemy z~wami pracować.

\noindent  Kitty: Trochę się martwię, że ta nowa grupa podejmuje zbyt wiele działań i~może zostać przytłoczona. Może lepiej byłoby zdecentralizować, trochę podzielić obowiązki.

\noindent  \textit{Jaggi}: Być może nadszedł czas, aby przejść do sondy, aby sprawdzi poczucie w~sali. Jeśli mamy konsensus, możemy przejść do czegoś innego; w~przeciwnym razie możemy przeprowadzić pełną debatę. Pamiętajmy: to jest tylko dla delegatów, ludzi upoważnionych przez swoje kolektywy lub grupy afinicji, którzy mają małe czerwone lub niebieskie karteczki.

\noindent  Mężczyzna: Ostatnie pytanie przed głosowaniem: to jest komisja tylko do koordynowania blokad?

\noindent  Niebieska bandana: Czekaj, a to nie jest generalny komitet taktyczny koordynujący działania?

\noindent  Wiele: Nie! Nie!


 Zwolennicy komitetu nie tylko kierowali skoordynowanym wysiłkiem, ale wydawali się mieć zamiar pchać go tak daleko, jak to tylko możliwe. Oznacza to, że propozycja rozpoczęła się jako komitet do przekazywania informacji o blokadach i~wydawała się przekształcać w~coś o znacznie szerszych uprawnieniach.

 Lesley, która uważnie obserwowała, dźgnęła mnie, kiedy sięgnęłam po swój papier. 
 
 -- Nie głosuj na ,,tak''! Każdy z~ludzi nalegających na tę propozycję, wszyscy są ISO. To zamach ISO!

 Co by wiele wyjaśniało. Jeśli chodzi o głosowanie, jako jedyni zagłosowaliśmy na ,,nie'', ale jest około piętnastu osób wstrzymujących się od głosu. To samo w~sobie było niezwykłe.

 Nie było dla mnie do końca jasne, co będzie dalej, ponieważ CLAC, technicznie rzecz biorąc, nie korzystał z~procesu konsensusu. Gdyby to był DAN, bylibyśmy zablokowani i~to byłby koniec. Lub, alternatywnie, jeśli facylitator był wystarczająco umiejętny, byłoby wcześniej jasne, że niektóre osoby miały wystarczająco silne odczucie w~kwestii, że mogłyby ją zablokować, a zatem, jeśli propozycja nie zostałaby po prostu wycofana, zostałaby zmieniona: różne osoby proponowałyby ,,przyjazne poprawki'', dopóki wszystkie obawy nie zostaną omówione. CLAC używał jednak systemu zmodyfikowanego głosowania: teoretycznie mieliśmy rozpocząć debatę z~jednym mówcą za, jednym mówcą przeciw itd., a na koniec głosowanie wymagające 75\% większości. Ale w~rzeczywistości to, co się wydarzyło, jest dokładnie tym, co by się stało, gdyby był to czysty konsensus.

\noindent \textit{Jaggi}: Więc teraz, ponieważ nie mamy pełnego konsensusu, przechodzimy do debaty. Najpierw zobaczmy, czy ci, którzy głosowali przeciw, chcą wyjaśnić powody swojego sprzeciwu; wtedy wysłuchamy trzech osób za propozycję i~trzech przeciw.

\noindent Lesley: Byłem już w~komitetach taktycznych\ldots 

\noindent Ktoś: Czy mógłbyś wstać, proszę? Łatwiej byłoby słyszeć.

\noindent Lesley: Tak, przepraszam. Jestem Lesley z~NYC DAN. Byłam już wcześniej w~komitetach taktycznych i~z mojego doświadczenia wynika, że  nie działają one zbyt dobrze. Pamiętajcie, że w~Seattle nie było centralnego komitetu koordynującego. Wszystko odbywało się na zasadzie konsensusu między grupami afinicji, nawet na ulicach. Na A16 mieliśmy jednak pewne problemy, pewne luki w~blokadzie i~dlatego podczas protestów na konwencie w~Filadelfii i~LA organizatorzy postanowili stworzyć zespoły taktyczne, aby zapewnić ogólną koordynację, naprawdę bardziej w~formie eksperymentu niż czegokolwiek innego. Odkryliśmy, że w~Philly gliniarze byli w~stanie dość łatwo dobierać się do członków zespołu, co powodowało więcej zakłóceń, niż gdybyśmy w~ogóle nie mieli żadnej scentralizowanej koordynacji. Nie byłem w~LA, ale z~tego, co słyszałam, zespoły taktyczne szybko stały się strukturami władzy samej sobie, ludzie z~LA DAN byli traktowani jak bogowie i~to kompletnie zblokowało jakąkolwiek niezależną inicjatywę.

 Na koniec mam pewne obawy, że stworzenie takiego zespołu może skończyć się scentralizowaniem władzy z~dala od lokalnych organizatorów. Więc sprzeciwiam się temu, ponieważ uważam, że ważne jest, abyśmy podtrzymali bardzo wyraźne zobowiązanie do zachowania władzy w~lokalnych rękach.

\noindent \textit{Jaggi}: A inne głosy nie?

\noindent Dawid [przerwał w~trakcie pisania notatek]: Kto ja? Um, podobne obawy.

\noindent Jaggi: No to otwórzmy debatę.

\noindent Stary Punk: Chciałbym zaproponować jako przyjazną poprawkę, aby komisja została utworzona w~taki sposób, aby zapewnić reprezentację jak największej liczby grup afinicji.

\noindent Jaggi: [do Różowej Czapki]: Jeśli \textit{to }przyjazna poprawka\ldots  Czy tak?

\noindent Różowa czapka: Tak, dobrze.

\noindent Mężczyzna: Proponowałbym również, aby wyjaśnić, że komisja nie jest ciałem decyzyjnym, ale takim, które będzie zbierać informacje i~sugerować możliwości działania. Myślę, że to również należy dodać jako przyjazną poprawkę.

\noindent Inny mężczyzna: Kiedy po raz pierwszy rozmawialiśmy o tej propozycji, sformułowaliśmy ją jako komitet strategiczny specjalnie do koordynacji blokad. Od czasu sondażu, wydaje się, że mówimy o czymś, co będzie koordynowało całość akcji. Wydawałoby się więc, że jest tu trochę zamieszania, nie jest dla mnie jasne, co to jest. Stworzenie pierwszego byłoby świetne. Jeśli jest to drugie, istnieją już grupy, które to robią. Byłbym za tym, gdyby to było raczej to pierwsze.

\noindent [Krótka konsultacja między facylitatorami]

\noindent \textit{Jaggi}: Słownik, który mamy, mówi ,,komitet strategiczny do koordynacji z~innymi grupami'', pamiętając o przyjaznych poprawkach\ldots 

 Których pojawiło się więcej. Zanim to się skończyło, mieliśmy komitet strategiczny zobowiązujący się do przestrzegania zasady decentralizacji, do koordynacji z~CLAC/CASA, i~który miałby nie więcej niż jednego przedstawiciela z~określonych grup afinicji i~tak zróżnicowany zakres takich grup reprezentowanych, jak to tylko możliwe. Co ciekawe, kiedy w~końcu doszło do głosowania, było kilka głosów odmownych, ale także sporo braw -- rodzaj wzajemnego uznania za rozwiązanie problemu -- i~groźba pojawienia się jakiegokolwiek rodzaju komitetu centralnego była zdecydowanie obalona.

 Po głosowaniu Lesley i~ja poszliśmy do przodu, aby naradzić się z~ludźmi z~CLAC. Helene -- tak nazywała się kobieta z~tęczowymi dredami -- serdecznie podziękowała nam za nasz sprzeciw. 
 
 -- Oczywiście istnieje komitet strategiczny -- powiedziała nieco niepewnym angielskim. -- Ale nie chcieliśmy sprawiać wrażenia, że  ich wykluczamy. Mimo to widziałem tam ludzi z~ISO\ldots 

 Mogę zauważyć, że to, co się wydarzyło, było również doskonałym przykładem innej kluczowej zasady podejmowania decyzji w~drodze konsensusu: nigdy nie wolno kwestionować uczciwości ani dobrych intencji innej aktywistki. W rzeczywistości nawet wspomnienie o ISO w~dyskusji byłoby postrzegane jako niemal szokująco konfrontacyjne.

 Zaczerpujemy trochę powietrza; chociaż w~końcu wracam dość szybko, ponieważ na zewnątrz wciąż jest zimno, a na spotkaniu zostawiłem wszystkie swetry. Znajduję kawę i~wracam w~samą porę, by uchwycić jedyny poważny incydent, w~którym ostrożna powierzchnia wzajemnego szacunku i~hojności faktycznie zaczyna się załamywać, dość przewidywalnie, wokół kwestii niestosowania przemocy. Najwyraźniej sprawa była niemal wyłącznym tematem pierwszej rady prasowej miesiąc wcześniej. Teraz ktoś próbuje do tego wrócić. Nie jestem pewien, kim był ten mężczyzna, ale był to wielki, brodaty, anglojęzyczny gość w~koszuli drwala, z~kartką papieru w~rękach i~małym oddziałem kibiców za nim, jego agresywne gesty wydawały się określać go niemal natychmiast jako jednego z~tych klasycznych stereotypów aktywistów: wojowniczy pacyfista.

\noindent Drwal: Chciałbym porozmawiać o różnorodności taktyk.

\noindent [słyszalne jęki z~całego pokoju]

\noindent \textit{Starszy Facylitator}: Nie sądzę, żeby to był odpowiedni czas i~miejsce na dyskusję na ten temat.

\noindent Drwal: Cóż, jeśli nie mogę tego zrobić teraz, gdzie indziej mogę to zrobić? Mam oświadczenie, które chciałbym przeczytać. Niektórzy z~nas przygotowali oświadczenie\ldots 

\noindent \textit{Starszy Facylitator}: Przepraszam, staram się to wyjaśnić\ldots 

\noindent Drwal: \ldots oświadczenie, które ma przyjąć Czerwony Blok. Uznaliśmy, że byłoby to właściwe, ponieważ w~końcu wezwałaś do dyskusji na temat stosunku każdego bloku do policji. Więc jeśli pozwolisz mi zacząć: 

\noindent [\textit{zaczyna czytać}]

 ,,Celem Czerwonego Bloku jest wyrażenie demokratycznego sprzeciwu ludu wobec FTAA i~Szczytu Ameryk. W tym celu nasze działania będą polegać na zakłóceniu lub uniemożliwieniu spotkania na Szczycie. Nasze bezpośrednie działanie usuną wszelkie bariery, które będą blokować możliwość wyrażenia naszego sprzeciwu bezpośrednio uczestnikom. Nie będziemy również honorować żadnych działań ani próśb policji, które w~podobny sposób będą próbowały zablokować nasz dostęp do tych spotkań. Naszą sprawą jest sprzeciw wobec FTAA i~Szczyt; dlatego nie podejmiemy działań przeciwko robotnikom tego miasta. i~chociaż nie pozwolimy policji ani jej barykadom blokować nam dostępu do Szczytu, nie będziemy używać broni ofensywnej ani atakować policji; jeśli jednak zostaniemy zaatakowani, zareagujemy w~sposób defensywny''. 

\noindent [mowa jest nieustannie przerywana przez gwizdy i~okrzyki]

\noindent \textit{Jaggi}: Jeśli pozwolisz mi tutaj przetłumaczyć te wrzaski. Odbyły się już niekończące się dyskusje na ten temat, a to tutaj, to jest nie w~porządku. To, co mówisz, jest sprzeczne z~zasadami różnorodności taktyk, o których już (bardzo obszernie) dyskutowaliśmy i~co do których ostatecznie się zgodziliśmy.

\noindent Drwal: Cóż, dla tych z~nas, którzy nie są w~Quebec City, ale w\ldots  odległych miejscach, ciężko jest przetłumaczyć, co właściwie ma oznaczać niejasna fraza, taka jak ,,różnorodność taktyk''. Uważamy, że jeśli zostaniemy poproszeni o rozszerzenie naszej odpowiedzialności za solidarność na wszystkich w~grupie, mamy prawo poprosić grupę o wzięcie odpowiedzialności za wyjaśnienie, jakie ograniczenia, jeśli w~ogóle, narzucają. Popieramy ideę różnorodności taktyk, ale to wcale nie oznacza poparcia dla dowolnych taktyk.

\noindent \textit{Starszy Facylitator}: Jako jeden ze współfacylitatorów\footnote{Właściwie użyła słowa ,,współanimator'', od francuskiego \textit{animateur} or \textit{animatrice}.} nie sądzę, abyśmy mogli rozpocząć debatę na temat różnorodności taktyk. Wezwanie do wzięcia udziału w~tej radzie wystosowano na zasadzie różnorodności taktyk. Pamiętaj też, że nasza organizacja jest zdecentralizowana, więc nie ma nadrzędnej władzy, która mogłaby stawiać bariery lub ograniczać to, co mogą zrobić poszczególne grupy afinicji. Jesteśmy organem doradczym, nie możemy narzucać. Tak więc chciałabym przejść do prawdziwej propozycji, jeśli ktoś ją ma.

 To znaczy, chyba że w~sali mamy głębokie uczucie, że powinniśmy o tym porozmawiać. Jest?

 Nie? Czy powinniśmy zrobić badanie sali?

\noindent [W sali pozostało około 120 osób]

\noindent \textit{Jaggi}: Pozwólcie, że wyjaśnię każdemu, kto nie jest zaznajomiony z~naszym procesem, że jeśli ktoś prosi o ,,sondę'', nie jest to wiążące głosowanie, ale sposób na zrozumienie pokoju, odczuć ludzi w~sprawie pytania, naprowadzenie facylitatorów. W tym przypadku byłoby to sprawdzenie, czy ludzie chcą dyskutować nad propozycją. Kto jest za debatą?

\noindent [Na korzyść: jeden poganin, mała grupka zwolenników drwala]

\noindent [Przeciw: przytłaczająco wielka liczba]

\noindent [Wstrzymało się: około dwunastu]

\noindent \textit{Jaggi}: W porządku, mamy 75\% za przejściem do kolejnego punktu, więc to zrobimy.

 Kolejna propozycja dotyczyła punktu wyjścia marszu: czy zebrać się na Równinie Abrahama. Pojawiły się obawy, że byłoby nierozsądne, aby tysiące aktywistów chłodziło się w~dużym parku, w~zasięgu wzroku policji, przez kilka godzin przed poważną akcją. Inni uważali, że nierozsądnie jest zmieniać plany tak późno, ponieważ dla Zielonego Bloku ważne było przynajmniej wcześniejsze poznanie ostatecznej lokalizacji. Opinia wydawała się skłaniać ku temu pierwszemu.

 Kitty wyszła, wyjaśniając, że obiecała przyjaciółce z~USA, że sprawdzi drogę na lotnisko. Jej przyjaciółka słyszała, że  jest tylko jednopasmowa autostrada, bez alternatywnych tras. Dean, Sasha i~ja wybieramy się na naszą nieformalną trasę z~ludem Scanner (Emma gdzieś zniknęła). Zbieramy się, zgodnie z~obietnicą, o 13:00 i~jemy kanapki, spacerując brukowanymi uliczkami wkrótce zakazanej strefy.

\noindent \textbf{13:15, Końcowe Ustalenia}

 Równina Abrahama, ogromny obszar parku na szczycie klifów Jaggiego, wciąż jest całkowicie pokryta śniegiem. Jest prawie pusta w~mroźne niedzielne popołudnie. Kilkunastu z~nas wyruszyło na poszukiwanie podobno już zainstalowanego odcinka muru. Wyglądamy niesamowicie w~naszych czarnych bluzach z~kapturem, wojskowych spodniach i~niekończących się naszywkach (dzieciak obok mnie, w~blond dredach, ma na sobie kurtkę z~napisem ,,Vegan Death Squad''). Tylko Buffy, facylitatorka z~poprzedniej nocy, jest incognito w~brązowej zamszowej kurtce i~z aparatem fotograficznym. Udaje turystkę nie do końca nieprzekonującą (aparat ma w~rzeczywistości dokumentować informacje o możliwym wykorzystaniu taktycznym). Sasha ma ogromną kamerę wideo, która dokumentuje naszą wyprawę. Inni też mają aparaty.

 Kiedy podchodzimy do oszołomionego narciarza w~średnim wieku po wskazówki, zdaję sobie sprawę, że staliśmy się ucieleśnieniem innego klasycznego stereotypu aktywisty. Właściwie jest to idealne uzupełnienie wojowniczego pacyfisty: tłum anarchistów wyglądający jak zgraja żołnierzy z~jakiejś bezbożnej armii -- jakiego typu armii, nawet nie chcesz sobie wyobrażać -- którzy, kiedy tak naprawdę z~nimi rozmawiasz, stają się być najsłodszymi, najbardziej skromnymi ludźmi, jakich można sobie wyobrazić. Ktoś nieśmiało pyta narciarza o Mur. Najpierw myśli, że pytamy o mury starego miasta, ale tłumaczymy. 
 
 -- Och, nowy rodzaj muru -- uśmiecha się i~wskazuje nam starożytną wieżę w~dół wzgórza.

 Wieża jest ogromną wieżą armatnią z~widokiem na klify; potem sprawy stają się bardzo strome bardzo szybko. Kilku z~nas próbuje zejść; jedno z~dzieci z~Wysp Księcia Edwarda ma spontaniczne krwawienie z~nosa; tylko kilku z~nas (ja, Dean, dwóch członków Montréal Ya Basta!) faktycznie schodzi na dół. Ogrodzenie nie było tak naprawdę widoczne, ale Sasha robi piękne zdjęcia panoramiczne do przyszłego filmu dokumentalnego.

 Później zrobiliśmy obszerne zdjęcia okolicy w~pobliżu Grand Théâtre, gdzie nasz wyobrażony trójstronny atak najprawdopodobniej napotka silny opór. 
 
 -- Widzisz ten mały park, tuż obok teatru? -- zapytał Greg, jeden z~mieszkańców Montrealu. -- Właśnie tam w~zeszłym roku stoczyliśmy wielką bitwę o reformę szkolnictwa.

 Ktoś inny wyjaśnia, że  rząd zorganizował wysłuchanie publiczne na temat tego, jak przeprowadzić cięcia środków na edukację. 
 
 -- Obiecali, że dopuszczą do udziału grupy studenckie, ale wtedy zaprosili tylko te prawicowe. -- Wykluczeni ogłosili zamiar przerwania konferencji; rząd ogłosił zamiar otoczenia budynku oddziałami policji. Ostatecznie wszystko sprowadzało się do walki pozycyjnej: z~jednej strony gliniarze uzbrojeni w~gaz łzawiący i~plastikowe kule, a z~drugiej studenci uzbrojeni w~cegły, kule bilardowe i~koktajle Mołotowa.

-- Koktajle Mołotowa?

-- Mają tu zupełnie inne standardy. Trzeba pamiętać, że w~latach 70. toczyła się tu jakaś wojna partyzancka. Ludzie ginęli. Sam Quebec przez lata znajdował się w~stanie wojennym. To zupełnie inne miejsce niż reszta Kanady.

 Piętnaście minut później, skulony w~wiacie przystankowej, by negocjować taktykę, Greg, trochę zakłopotany, ponownie porusza tę sprawę. 
 
 -- Właściwie to chciałem o tym pogadać. Dużo o tym dyskutowaliśmy w~Montrealu i~myślę, że konsensus jest taki, że wszyscy myślimy, że koktajle Mołotowa zdecydowanie nie są dobrym pomysłem. -- Milton, z~tej samej grupy, energicznie kiwa głową. 
 
 -- Nie mówię tego ze względów moralnych -- zauważa wobec Amerykanów -- ponieważ nigdy nie widziałem, by mołotowy były używane przeciwko ludziom, którzy są naprawdę bezbronni. Używasz ich tylko przeciwko policji w~pełnym ognioodpornym sprzęcie do zamieszek, o której wiesz, że nie zostanie poważnie ranna bez względu na to, co zrobisz. Więc\ldots  to nie jest tak, że naprawdę próbujesz kogoś podpalić. To jest więcej\ldots  Ok, sposób, w~jaki to widzę, jest sposobem na pokazanie \textit{naprawdę }poważnego celu, pokazanie, że jesteś zdeterminowany, aby przejść. Gliniarz, który widzi nadlatującą w~jego stronę bombę zapalającą, nie może nie powstrzymać zaskoczenia, nawet jeśli wie, że to go nie zabije; to nie może pomóc, ale sprawi, że zacznie się zastanawiać, czy naprawdę chce utrzymać swoją pozycję. To sposób na odciągnięcie ludzi. I~to działa.

-- Zdecydowanie zadziałało w~zeszłym roku podczas bitwy w~parku -- mówi jeden z~dzieciaków PEI. Po chwili: -- nie żebym ja też to popierał.

-- Problem z~mołotowami\ldots  -- mówi Milton. -- No dobra, po pierwsze, jeśli coś rzucasz, musisz to zrobić \textit{z pierwszej linii}. Odnosi się to do wszystkiego, co rzucasz i~wydaje się to oczywiste, ale nie mogę uwierzyć, jak często jakiś idiota o tym zapomina. W zeszłorocznej bitwie mieliśmy ścianę tarczy i~niektórzy ludzie rzucali cegłami i~butelkami przez linię z~dala od tyłu, więc oczywiście od czasu do czasu ktoś trafiał tarczownika w~tył głowy. Gdybym nie miał kasku, całkowicie by mnie załatwili.

 Jeden z~dzieciaków PEI wtrąca się: 
 
 -- Nawet gorzej, jeśli zamierzasz używać mołotowa, musisz najpierw poćwiczyć. To niesamowite, jak wielu ludzi nie zdaje sobie z~tego sprawy. Przynajmniej musisz poćwiczyć pakowanie. Jeśli tego nie zrobisz, to w~połowie przypadków, gdy odchylisz rękę do tyłu, by rzucić, szmata wyskoczy i~benzyna wyleje się na faceta za tobą, więc teraz jego ubrania są przesiąknięte benzyną i~ludzie się bawią z~otwartymi płomieniami wokół niego.

-- Więc nie dla mołotowów.

-- Ta.

-- Jedynym naprawdę uzasadnionym zastosowaniem mołotowa -- zauważa Buffy -- może być niszczenie mienia. Na przykład: powiedzmy, że jest armatka wodna. Teraz jest to całkowicie uzasadniony cel.

-- Pamiętaj, że armatka wodna nie spowolniła zbytnio ludzi w~tym ostatnim pokazie.

-- Armatki wodne mogą być całkiem skuteczne, jeśli są właściwie używane.

-- Jednak -- mówi Greg. -- Powodem, dla którego chciałem mieć tę małą dyskusję, było uzyskanie konsensusu, że \textit{nie }chcemy mołotowa, to taktycznie nie jest dobrym pomysłem. A więc: czy ktoś rzeczywiście ma do tego zastrzeżenia? A może mamy konsensus?

 Wszędzie kiwnięcia głową. Zapewniam go, że nikt ze strony amerykańskiej nawet nie rozważał ich użycia.

 Wróciliśmy do rady w~samą porę, by zobaczyć, jak Emma i~jej nowy przyjaciel Craig wychodzą w~ogromnej irytacji. Najwyraźniej przedstawiciel GOMM rzeczywiście wszedł i~poprosił, aby niektóre strefy nazwano tylko żółtymi; jeden był prawdopodobnie obszarem autostrady, którego i~tak nie chcemy, więc w~porządku, ale drugi obszar był tuż pod płotem na jednej z~trzech ulic, którymi mieliśmy maszerować. Wzruszamy ramionami i~zastanawiamy się, oni to rozwiążą. W każdym razie, jeśli chcemy wrócić do domu przed 2 w~nocy, lepiej zacznijmy jechać.

\noindent \textbf{Droga do domu}

 Przez około godzinę Emma nadal grzmi na pacyfistów. 
 
 -- Dlaczego ludzie upierają się przy próbach narzucania innym swoich własnych kodeksów postępowania? Jak mogą nazywać siebie anarchistami? Te rzeczy powinny być pozostawione wyłącznie do decyzji każdej grupy afinicji.

-- Czyli mówisz -- pytam -- że sprzeciwiasz się pisanym kodeksom? Albo jakikolwiek zasadom?

-- Mówię dowolnym zasadom. Jakiemu możliwemu celowi służą?

 Od jakiegoś czasu się tym zajmujemy. Zwracam uwagę na możliwość pojawienia się nazistów. Emma podkreśla, że  naziści dość regularnie próbują rozbić anarchistyczne wydarzenia. Dlatego wiele grup afinicji dopuszcza tylko jeden wyjątek od ogólnej zasady niestosowania przemocy: kiedy ma się do czynienia z~nazistami.

-- W porządku, w~takim razie powiedz, że bierzesz udział w~akcji i~zauważasz, że pojawiła się inna grupa afinicji z~taktycznym urządzeniem termojądrowym.

 Emma przewróciła oczami. 
 
 -- Czemu oczywiście moglibyście z~łatwością zapobiec, gdybyście wcześniej opublikowali kodeks postępowania określający ,,żadnych taktycznych urządzeń termojądrowych''? Słuchaj, ktoś robi coś szalonego, to w~porządku, ludzie wokół niego muszą zrobić to, co muszą.

 Na szczęście Sasha zmienia temat. Spędzamy kolejne piętnaście minut, próbując ustalić różne rodzaje ka\-na\-dyj\-skiej och\-ro\-ny, które mają być skierowane przeciwko nam: od RCMP (Królewskiej Kanadyjskiej Policji Konnej) po Sûreté du Québec -- nazwy, które dają amerykańskiemu aktywiście wrażenie, że wkrótce będziemy zaatakowany przez kombinację Dudleya Do\dywiz Righta i~inspektora Clouseau. (Nieunikniona repartee: ,,Czy twój pies gryzie?'' ,,Czy masz licencję na to?''). Zaznaczam, że przynajmniej w~Vermont, z~jego socjalistyczną administracją, możemy oczekiwać, że policja będzie nas traktować w~rękawiczkach. Emma podchodzi do tego niezwykle sceptycznie. Bardziej prawdopodobne, że będą szczególnie męczący, aby się wykazać. Zresztą jaki wpływ mają lokalni politycy na policję? 
 
 W Montrealu mówimy o rodzinach. Sasha dorastała w~Hollywood. Pochodzę z~lewicowej rodziny robotniczej z~Nowego Jorku. Okazuje się jednak, że zarówno Emma,  jak i~Dean pochodzą z~katolickich rodzin robotniczych ze Środkowego Zachodu, a to przebija wszystko. Na przykład rodzice Emmy należą do jakiejś skrajnie charyzmatycznej sekty. Dean uważa, że  jego mama ma lekką schizofrenię (to jest w~rodzinie); przestraszyła się, kiedy miał szesnaście lat i~przeczytała jego pamiętnik i~odkryła, że  jest gejem (,,i to nie tak, że nie było w~tym nic wyraźnie seksualnego; właśnie przyznałem, że się w~kimś podkochiwałem''). Pokryła pamiętnik obrazami świętych i~Matki Boskiej i~do dnia dzisiejszego mu go nie oddała. Posyłała mu bieliznę potajemnie błogosławioną świętym olejem, aby kontrolować jego genitalia. Wizje i~znaki: mama Emmy myśli, że jest opętana przez diabła i~dlatego została anarchistką. Ma mnichów modlących się o uratowanie jej córki. Sasha dorastał w~Hollywood, jego mama była Żydówką, a tata Polakiem. Mama zakazała mu muzyki pop przez wiele lat. Emma i~Dean nie są pod wrażeniem. Wymieniają katolickie historie przez około dwie i~pół godziny. Gdzieś na północy stanu Nowy Jork udaje mi się zasnąć.

\chapter{Od Burlington do Akwesasne}

Następne kilka tygodni było coraz bardziej gorączkowych. Podam tylko schematyczne zestawienie.


\section{Czwartek, 29 marca 2001}
\noindent \textbf{ Spotkanie Ya Basta!, Brooklyn
}\medskip

\noindent NOWY JORK DZIENNIK CIĄG DALSZY


To spotkanie jest pierwszym pojawieniem się przyjaciela Smokeya i~Flammy, Jessego, zarozumiałego młodzieńca, który niedawno przybył z~Luizjany. Mówi nam, że jest ,,organizatorem'', potrzebuje czegoś do zorganizowania i~Ya Basta! wyraźnie potrzebuje pomocy. Właściwie jest całkiem niezłym facylitatorem i~nalega, abyśmy mieli właściwe spotkanie, ale prawie wszyscy spoza frakcji Smokey i~Flamma natychmiast odczuwają do niego niechęć.

\section{Piątek, 30 marca}
\noindent \textbf{ Niezależne Centrum Medialne IMC, Manhattan}\medskip


 Godziny w~IMC, głównie spędzone na pocieszaniu Moose w~związku z~niedawną romantyczną katastrofą. Wszyscy spieszą się z~przygotowaniami do akcji granicznej. Warcry idzie jak banknot dolarowy. Julie z~Urban Justice League pojawia się i~wychodzi, wyglądając na przemian słodko i~odświętnie. Twinkie i~Brad\footnote{Brad Will -- ten sam Brad Will zamordowany w~Oaxaca w~listopadzie 2006 roku. Nie ma teraz sensu ukrywanie jego tożsamości.} jadą na rowerach, kiedy przyjeżdżam, szukając sushi. W lodówce IMC jest ogromny zapas rzeczy, głównie ze starannie usuniętymi częściami ryb.

\section{Niedziela, 1 kwietnia}


 Pojawiają się wczesne wiadomości o akcji granicznej. Wygląda na to, że poszło całkiem nieźle -- wszyscy zostali zatrzymani, a większości zakazano wjazdu do Kanady przez pięć lat, ale tego można było oczekiwać, a przynajmniej dostaliśmy relacje w~WBAI, a nawet w~jakiejś kanadyjskiej telewizji. Mimo to wydaje się, że między aktywistami doszło do jakiejś kłótni. Akcja SalAMI w~Ottawie również poszła bardzo dobrze i~trafiła na nagłówki w~całej Kanadzie. Oczywiście amerykańskie media nawet o tym nie wspomniały, ale można było się tego spodziewać.

\noindent \textbf{Spotkanie DAN, Charas El Bohio}

 Byłem na spotkaniu DAN w~Charas o 18:00. Lesley i~ja przekazaliśmy nasz raport z~rad Quebecu, próbując wyjaśnić dynamikę trzech kolorowych bloków. Były zwykłe obawy o to, co faktycznie działo się w~Akwesasne i~o retorykę Shawna, jak również długa dyskusja na temat zbliżającego się walnego zgromadzenia PGA w~Cochabamba i~potrzeby, aby Continental DAN w~końcu wszedł na pokład i~formalnie zatwierdził zasady PGA (co robimy).

 Różne osoby w~kontakcie telefonicznym z~załogą na granicy kanadyjskiej wyjaśniają, na czym polegał problem: to znowu była Julie. Nikt nie wydaje się zaskoczony. Tym razem to niewrażliwość rasowa. Twinkie uczestniczyła w~akcji granicznej głównie po to, by zwrócić uwagę na kwestie imigracyjne: gdzie biali mogą normalnie przechodzić do woli, wszystko wygląda zupełnie inaczej dla każdego, kto wygląda, jakby pochodził z~Azji, Ameryki Łacińskiej lub Afryki; i~oczywiście, jeśli biali próbują zrobić z~tego problem polityczny, to nagle oni też nie mogą przejść. Julie, w~swoim niepowtarzalnym stylu, nie tylko kompletnie nie zwróciła na to uwagi reporterowi WBAI, ale też zignorowała samą Twinkie, gdy próbowała znaleźć miejsce przy mikrofonie, żeby to wyjaśnić. Twinkie była bardzo, bardzo wściekła.

\section{Wtorek, 3 kwietnia}

 Pojawia się ,,Pogańskie wezwanie do działania'', jedno z~kilkunastu mniejszych wezwań do różnych grup lub klastrów biorących udział w~nadchodzących akcjach. Rzeczywiście cytuje Deklarację z~Cochabamba, sformułowaną przez grupy boliwijskie, które skutecznie odparły próbę rządu sprywatyzowania lokalnej wody na rzecz Bechtel:

\noindent \textit{Deklaracja z~Cochabamby:}

\noindent 1) Woda należy do ziemi i~wszystkich gatunków i~jest święta dla życia, dlatego woda na świecie musi być chroniona, odzyskiwana i~chroniona dla wszystkich przyszłych pokoleń, a jej naturalne wzory należy szanować.

\noindent 2) Woda jest podstawowym prawem człowieka i~dobrem publicznym, którego winny strzec wszystkie szczeble władzy, dlatego nie należy jej utowarowiać, prywatyzować ani sprzedawać w~celach komercyjnych. Prawa te muszą być zagwarantowane na wszystkich szczeblach władzy. W szczególności traktat międzynarodowy musi zapewnić, że zasady te są niepodważalne.

\noindent 3) Woda jest najlepiej chroniona przez lokalne społeczności i~obywateli, których należy szanować jako równorzędnych partnerów z~rządami w~zakresie ochrony i~regulacji wody. Ludy ziemi są jedynym środkiem promowania ziemskiej demokracji i~oszczędzania wody.

 Tutaj, nad brzegiem św. Wawrzyńca/Magtogoek, z~rzeką jako naszym sprzymierzeńcem i~maszerującymi z~nami przodkami, staniemy się żywą rzeką, aby nieść tę deklarację jako wyzwanie dla rządów świata i~inspirację dla jego narodów.

\section{Środa, 4 kwietnia}

 Wyzwiska na listservs stają się niezwykle obelżywe, ponieważ wszyscy wydają się rzucać na wszystkich innych w~ramach Akcji Granicznej 1 kwietnia. Sami organizatorzy niewiele mówią, ale w~chwili, gdy ktoś podnosi kwestię rasizmu, ktoś inny wydaje się uważać ich za marksistowskiego sekciarzy. Sama Twinkie niczego nie opublikowała, ale w~końcu jeden z~jej znajomych przesyła własną wersję wydarzeń Twinkie:

\textit{ Czy ktoś może mi przypomnieć, dlaczego protestujemy przeciwko FTAA? Hmm?? Aby rekrutować więcej osób dla naszych organizacji??? Albo o tym, że korporacje ignorują granice, a ludzie są przez nie uciskani! A co z~Cornwall i~co się dzieje z~Mohawkami? Czy jedziemy tam, ponieważ jest to łatwa droga do Quebecu, czy też dlatego, że naprawdę popieramy fakt, że granica jest codziennym afrontem dla ich życia i~suwerenności?}

 \textit{ZATEM! To właśnie wydarzyło się 1 kwietnia podczas tej akcji medialnej. Nikt nie odniósł się do tych kwestii i~skupił się tylko na ich kulawych, uprzywilejowanych, białych dupach, które nie były w~stanie dostać się do Kanady JEDEN RAZ z~powodu protestu masowej mobilizacji\ldots }

 W międzyczasie zauważyła, że  gdy byliśmy grzecznie i~szybko sprawdzani, na linii ciągle czekali biedni, kolorowi ludzie, niektórzy prawdopodobnie po to, by skończyć w~areszcie imigracyjnym. 
 
 \textit{Czy ktokolwiek pomyślał o zabraniu ulotek z~informacją o swoich prawach lub jakichkolwiek innych materiałów informacyjnych? Czy ktoś w~ogóle o nich wspominał na konferencji prasowej?} 
 
 Twinkie kończy dźwięczną deklaracją:
 
\textit{  ,,KONIEC Z TEATREM ULICZNYM DLA UPRZYWILEJOWANYCH AKTYWISTÓW W MIEJSCACH OPRESJI !!!! Nazwij mnie separatystą, jeśli chcesz, ale nie będę pracowała z~ludźmi ze złymi zasadami i~publicznie będę wyzywała ludzi od rasistów''.}

\section{Czwartek, 5 kwietnia}

 \textit{Montréal Gazette }donosi, że prokuratorzy w~Quebecu twierdzą, że zostali poproszeni o opóźnienie wszystkich przesłuchań w~sprawie kaucji dla protestujących aresztowanych na zbliżającym się szczycie o trzy do pięciu dni, aby utrzymać ich z~dala od ulic (Marsden 2001). Kilku oburzonych ogłasza, że  zamierza odmówić współpracy.

\noindent \textbf{Spotkanie Ya Basta! w~Aladdin's Place w~Chelsea, godz. 18.00}

 Tymczasem Ya Basta! jest bliski rozpadu. 5 kwietnia miało być spotkaniem, na którym omówilibyśmy wspólne zasady: za czym ma ostatecznie stanąć kolektyw. Jesse grozi, że zablokuje taką dyskusję na tej podstawie, że Ya Basta! ma być ,,antyideologiczna''. Laura i~ja ledwo zdołaliśmy powstrzymać Moose'a przed wymarszem. 
 
 -- Antyideologiczne oznacza, że  nie ogłaszamy się anarchistami, komunistami ani zwolennikami żadnej konkretnej\ldots  wiesz, ideologii. To nie znaczy, że w~ogóle nic nie popieramy. Albo dlaczego jedziemy do Quebecu? Może powinniśmy utworzyć dwie drużyny, jedną protestującą przeciwko FTAA, drugą wspierającą i~walczyć ze sobą!

 Jako kompromis, wyciągam kopię zasad jedności PGA, którą nosiłem właśnie na taką okazję. Ale to również zostało zestrzelone z~powodu sprzeciwu wobec wyrażenia ,,obywatelskiego nieposłuszeństwa bez przemocy'', które, jak wskazują Target i~Jesse oraz kilku innych, może być interpretowane jako potępienie grup na Globalnym Południu, takich jak Zapatyści, którzy nie mają odwrotu i~muszą uciekać się do walki zbrojnej. Kiedy próbuję wskazać, że Zapatyści faktycznie stworzyli PGA, Smokey, który facylituje, inicjuje dyskusję: 
 
 -- Mamy szereg praktycznych kwestii, które wciąż musimy przepracować dzisiaj wieczorem i~wyraźnie będzie to długa rozmowa. Zobaczmy, czy zdążymy wrócić do tego w~przyszłym tygodniu. -- W tym momencie wychodzę i~znajduję Moose, który siedział na zewnątrz w~korytarzu obok windy, żeby mu powiedzieć, że jeżeli ciągle chce odejść, ma moje pełne poparcie.

\section{Niedziela, 8 kwietnia}
\noindent \textbf{ Spotkanie DAN, Charas El Bohio, 18:00}\medskip


 Małe spotkanie, nieco ponad dwadzieścia osób, głównie zainteresowanych tym, co zrobić z~tym, co zaczyna się nazywać ,,krwotokiem Akwesasne''. Otrzymujemy tylko same złe wieści. Wydawałoby się, że Rada Plemienia definitywnie przejrzała blef Shawna. Krążą pogłoski, że federalni rozsyłają nagrania z~walk ulicznych w~Pradze, twierdząc, że zamierzamy zrobić to samo w~ich społeczności. Krążą plotki. Wygląda na to, że niektóre z~Domów Wojowników mobilizują się przeciwko nam. Z drugiej strony Shawn zapewnia nas, że to tylko kwestia przepracowania procesu, trzeba się spodziewać sprzeciwu, zawsze są reakcjoniści. Trudno jednak nie zauważyć, że jego publiczne wypowiedzi całkowicie zmieniły ton: teraz wzywa nas do wzięcia udziału w~święcie smażenia ryb, uroczystym, ,,przyjaznym dzieciom'' wydarzeniu, aby omówić kwestie handlowe ze społecznością, po którym nastąpi całkowicie pokojowa przeprawa, w~której aktywiści i~członkowie społeczności połączą się i~przytłoczą straż graniczną naszą samą liczbą. Stwarza to dylemat: z~jednej strony koniecznie będą krążyć plotki, że akcja zakończy się katastrofą. Z drugiej strony, ponieważ wszystko zależy od liczb, jeśli wystarczająca liczba osób uważa, że  będzie to katastrofa, to wystarczy, aby to się stało.

\section{Wtorek, 10 kwietnia}

 Doniesienia z~Quebec City stają się coraz bardziej surrealistyczne. Anonimowy kanadyjski celebryta ogłosił, że jest gotów zapewnić fundusze na budowę gigantycznej średniowiecznej katapulty, za pomocą której może oblegać Szczyt. Tymczasem 1700 strażników więziennych, którzy otrzymali rozkaz usunięcia setek więźniów z~aresztów Orsainville i~Hull, aby zrobić miejsce dla protestujących, decyduje się na strajk. Policja zostaje wezwana do przejęcia więzień, a strażnicy przyjmują taktykę pokojowego nieposłuszeństwa obywatelskiego, blokując wejścia do więzienia. Atak policji i~kilkunastu strażników jest aresztowanych.

-- Przybyli w~formacji. Zmiażdżyli nas. Bili nas pałkami -- powiedział Michel Gauthier, strażnik w~Orsainville od dwudziestu trzech lat.

-- Protestujący na szczycie, którzy boją się tu przybyć, słusznie się boją. Jesteśmy dzisiaj dowodem na to, że policja tutaj jest bardzo niebezpieczna. (Król i~Van Praet 2001)

\section{Czwartek, 12 kwietnia}
\noindent \textbf{ Spotkanie Ya Basta!, Manhattan}\medskip


 Poprzedni tydzień był pełen wewnętrznych wysiłków na rzecz pojednania w~Ya Basta!: partie, wiadomości, propozycje, aby być może podzielić się na sojusznicze, ale autonomiczne grupy afinicji. W końcu, gdy przychodzi czas na kolejne spotkanie, mamy zbyt wiele praktycznych spraw do załatwienia, żeby zepsuć: treningi w~Burlington, scenariusze na granicy kanadyjskiej, kwestie prawne, komunikacyjne, taktyczne. Moose czuje się coraz bardziej winny z~powodu pomysłu, że może zachęcać ludzi do sytuacji, w~której niektórzy mogą zostać poważnie zranieni. Kończymy spotkanie otwartą dyskusją, w~której wszyscy rozmawiamy o naszych parametrach i~granicach dotyczących przemocy i~niestosowania przemocy. Co ciekawe, prawie każdy mówi dokładnie to samo. Nikt z~nas nie byłby skłonny zaatakować kogoś innego ani wykonać czynności, która może spowodować obrażenia fizyczne innej osoby; nikt z~nas nie miał najmniejszego problemu moralnego z~uszkodzeniem własności firmy; dla prawie nas wszystkich naprawdę trudnym pytaniem było, co byśmy zrobili, gdyby towarzysz lub ktoś, na kim nam zależało, został fizycznie zaatakowany -- to znaczy, czy bylibyśmy skłonni zaatakować kogoś, aby go uratować? Większość z~nas uważa, że  tak naprawdę nie bylibyśmy w~stanie przewidzieć, jak zareagujemy na taką sytuację, dopóki naprawdę się nie wydarzy.

 Być może, pomyślałem, nie byliśmy tak daleko od siebie, jak sobie wyobrażałem.

 Kolejny drobny kryzys wymagający ode mnie działania jako ministra informacji: Rada Plemienia, czyli Rada Naczelników, wydała oświadczenie, w~którym wyraziło zaniepokojenie perspektywą przemocy i~zniszczenia oraz błagało aktywistów, aby nie siali niezgody ani nie dopuszczali się nielegalnych czynów w~swojej społeczności. Poproszono mnie o napisanie odpowiedzi.

\noindent \textit{ Do Rady Mohawk, Akwesasne,}

 Piszemy w~odpowiedzi na Wasz niedawny list dotyczący naszych planów przeprawy przez Akwesasne przez Cornwall i~do Kanady 19 kwietnia.

 Chcielibyśmy przede wszystkim powiedzieć, że jesteśmy Wam głęboko wdzięczni za zrozumienie i~ducha tolerancji, które wykazujecie w~swoim liście, i~pragniemy zrobić wszystko, co możliwe, aby uspokoić wasze umysły w~zgłaszanych obawach. Zapewniamy, że przyjeżdżamy do Akwesasne tylko jako goście mieszkańców, którzy nas do tego zaprosili; nigdy nie planowaliśmy zrobić czegokolwiek, nie mówiąc już o niczym gwałtownym lub destrukcyjnym, z~własnej woli. Ostatnią rzeczą, jakiej chcielibyśmy, byłoby spowodowanie zakłóceń w~waszym życiu lub stworzenie wam trudności.

 Rozumiemy, że zostaliśmy zaproszeni na spokojne, świąteczne wydarzenie, które będzie obejmowało smażone ryby, dzieci i~sesję edukacyjną, podczas której nasi gospodarze wyjaśnią nam niektóre z~kwestii politycznych ważnych dla narodu Mohawk i~Rdzennych Narodów bardziej ogólnie. Następnie przejdziemy spokojnie przez most, pozostawiając otwarty jeden pas, aby uniknąć niedogodności dla mieszkańców i~możliwość przejazdu pojazdów uprzywilejowanych. Nigdy nawet nie rozważaliśmy angażowania się w~konfrontację z~kimkolwiek; raczej uważamy się za gości na cudzej ziemi i~pragniemy działać jako tacy, z~całym możliwym szacunkiem dla narodu Mohawk i~wszystkich jego mieszkańców. Jako działacze polityczni mamy nadzieję, że ta akcja pozwoli nam wszystkim lepiej zrozumieć problemy stojące przed Waszym Narodem, Waszymi osiągnięciami, i~wasze nadzieje na przyszłość, aby lepiej umożliwić nam działanie w~solidarności z~wami w~przyszłości, tak jak nasi gospodarze okazali nam już ogromną życzliwość, zrozumienie i~solidarność. Przyjeżdżamy jako przyjaciele i~mamy nadzieję nawiązać przyjaźń, która przetrwa długo po naszym odejściu.

 Z wyrazami szacunku, \\ Kolektyw i~Sieć Działań Bezpośrednich DAN Miasta Nowy Jork \\ Członkowie Kolektywu Ya Basta! w~Nowym Jorku \\ Kolektyw Członkowie Filadelfii Direct Action Group

\section{Sobota, 14 kwietnia}

 Policja Québec ogłasza (\textit{La Presse}, 14 kwietnia 2001), że ,,przed użyciem gazu łzawiącego zostaną zbadane wszystkie możliwości'' i~że nawet wtedy będzie to poprzedzone komunikatami w~czterech językach. Jeśli chodzi o plastikowe kule, policja powiedziała, że  będą one używane tylko w~ostateczności przed użyciem śmiertelnej siły, zawsze przeciwko jednostce, nigdy przeciwko tłumowi i~tylko wtedy, gdy ta konkretna osoba ,,stanowi poważne zagrożenie dla policji''.

 Marina, która wykonuje prawniczą pracę na rzecz mobilizacji Burlington, donosi, że jej konto w~telefonie komórkowym zostało nagle zlikwidowane, podobnie jak dwa różne konta e-mail. Jedna firma wysłała jej notatkę wyjaśniającą, że jej konto zostało zamknięte, ponieważ było wykorzystywane do ,,nielegalnych działań''.

 Krążą różne plotki o zbliżającej się katastrofie w~Akwesasne. Kilku z~nas spędza godziny na e-mailach, próbując je stłumić -- Target często sugeruje, że plotki są rozpowszechniane przez policję, a ja podkreślam, że bez frekwencji akcja nie może działać.


\section{Środa, 17 kwietnia}
\noindent \textbf{ BURLINGTON}\medskip

 Po zakończeniu ostatniego tygodnia zajęć wreszcie mogę rzucić się w~wir akcji w~pełnym wymiarze godzin.

 Dotarłem do Burlington Convergence po kilku dniach, prawie na samym końcu. Większość mojego czasu była dziwna, chaotyczna, wzburzona. Patrząc wstecz, myślę, że część z~tego miała związek z~faktem, że w~połowie miejsc, do których chodziłem -- w~samochodach, kawiarniach, miejscach publicznych -- ktoś wydawał się grać w~Ramones (,,I Want to be Sedated'', ,,Now i~Wanna Sniff Some Glue'', ,,We Want the Airwaves''). Dopiero później ktoś mi wyjaśnił, że Joey Ramone właśnie zmarł na raka wątroby. Ale głównie dlatego, że wszystko wydawało się rozpadać. Zameldowałem się przy biurku mieszkaniowym i~wpadłem na Raoula, jednego z~Yabbów -- wielkiego pluszowego faceta w~malutkim kapeluszu typu porkpie.
 
  -- David, \textit{nie }masz pojęcia, jak się cieszę, że cię tu widzę -- powiedział, mocno mnie przytulając.

-- Czemu? Co się dzieje? Jak szkolenia Ya Basta! się odbywają?

-- Mieliśmy tylko jedno. To była katastrofa. Teraz nie jest nawet jasne, czy Ya Basta! istnieje.

 Najwyraźniej napięcie między trenerami było iskrą. Trening, który odbył się na boisku piłkarskim kampusu UVM, przyciągnął całkiem spory tłum, około pięćdziesięciu, i~to było pierwszego dnia zjazdu, kiedy przybyło niewielu ludzi. Odbyło się coś w~rodzaju ogromnej imprezy piankowej, podczas której wszyscy bawili się różnymi rodzajami możliwych zbroi, a potem wszyscy ćwiczyli formacje grupowe w~nowo stworzonym sprzęcie. To tam wszystko zaczęło się rozpadać. Betty, tancerka -- i~oczywiście jedyna z~trenerek z~prawdziwymi umiejętnościami lub doświadczeniem w~nauczaniu -- była systematycznie odsuwana na bok przez triumwirat Moose, Target i~Jesse, którzy walczyli ze sobą o uwagę. Doszło do tego, że nawet Betty, zwykle najradośniejsza filozoficzna osoba, jaką można sobie wyobrazić, zaczęła narzekać. Podobnie jak wiele innych uczestniczek. Moose wybuchnął na dwóch pozostałych mężczyzn z~powodu ich niewrażliwości na płeć. 
 
 -- Skończyło się na krzykach.

-- Naprawdę krzyczeli?

-- Cóż, może nie całkiem dosłownie krzyczeli. Ale nie starając się ukryć faktu, że byli naprawdę wkurzeni.

 Jednak sam pokaz wściekłości sprawił, że wiele kobiet odeszło, zabierając ze sobą sporą porcję uczestników nie-Ya Basty!.

 Ktoś wezwał aktywistkę z~Zachodniego Wybrzeża o imieniu Laura, wielokrotnie opisywaną jako ,,zajebista trenerka wrażliwości gender'', która po krótkiej obserwacji grupy stwierdziła, że  jej dynamika jest tak głęboko problematyczna, że  prawdopodobnie nie będzie warte czasu i~wysiłku, aby spróbować ją uratować.

-- I\ldots ?

-- I~to było to. To był nasz ostatni trening. Żadne z~innych zaplanowanych szkoleń nawet się nie odbyło.

-- Gdzie jest Moose?

-- W końcu po prostu podniósł ręce i~powiedział, że z~nami rezygnuje. Dołączył do innej grupy afinicji, jacyś ludzie z~Filadelfii.

-- Och\ldots  jestem pewien, że Betty naprawdę to doceniła. A co z~Smokeyem i~Flammą? Emmie? Gdzie oni są?

-- Od tamtej pory ich nie widziałem. Nie wiem. Ktoś powiedział, że mogli wyjechać z~miasta.

 Odkryłem również, że jeśli chodzi o sympatie lokalnej socjalistycznej administracji, Emma dobrze trafiła. Dalecy od powitania nas, robili wszystko, czego można by oczekiwać od lokalnego rządu przygotowującego się do poważnej akcji -- pomimo faktu, że wielokrotnie nalegaliśmy im, że nie będzie żadnych akcji w~Burlington, po prostu spotkania. Lokalne firmy zostały ostrzeżone przed potencjalnymi włamaniami, patrole policyjne były wszędzie; aktywiści regularnie byli śledzeni przez nieoznakowane czarne SUV-y, które wydawały się służyć jedynie do stworzenia atmosfery strachu i~zastraszenia. To była kolejna rzecz, którą słyszałem wszędzie, poza muzyką Ramones: przerażające historie. Jeden samochód pełen ewidentnych federalnych podjechał do Kitty i~zapytał ją, czy chce wskoczyć na przejażdżkę. Inny SUV gonił Targeta w~uliczce. Ktoś szedł ulicą w~stroju Ya Basta!, gdy wrócił do samochodu dziesięć minut później, odkrył jakiegoś ogromnego mięśniaka w~garniturze, badającego bagażnik. Kilku lokalnych działaczy zgłosiło już tajemnicze włamania. 

 Po podrzuceniu moich toreb w~Burlington IMC i~skoordynowaniu z~ludźmi, którzy mieli dzielić moje zakwaterowanie, wyruszyłem na radę delegatów. 

\noindent \textbf{Rada Delegatów w~Burlington}

 W biurze mieszkaniowym odebrałam ulotkę, która wyjaśniała, że  rada będzie odbywać się w~miejscu zwanym ,,Billings Student Center'', na tarasie uniwersyteckim UVM, niedaleko centrum miasta. Budynek okazuje się ogromną budowlą z~wieżyczkami z~czerwonego kamienia, wyglądającą gdzieś pomiędzy kościołem a zamkiem. Podobno kiedyś była to biblioteka kampusowa. Na trawniku na zewnątrz jest już kilka czarnych flag i~transparentów. Przy drzwiach jesteśmy proszeni o potwierdzenie, że nie jesteśmy policjantami ani pracującymi dziennikarzami, a następnie przeglądamy zwykłe stoliki pełne dokumentów, wraz z~dużymi czarnymi markerami, którymi można napisać na nodze numery telefonów do spraw prawnych i~medycznych (wywieszone wszędzie). Samo spotkanie znajduje się w~dużej, okrągłej sali z~otaczającym ją okrągłym balkonem; za małym, by użyć go do prawdziwego teatru, to musi być coś w~rodzaju kampusowej przestrzeni spotkań. Najwyraźniej na balkonie znajdują się różne biura klubów studenckich, w~tym studencka rozgłośnia radiowa, w~której mały tłum technicznych typów -- głównie ludzi z~IMC -- korzysta ze sprzętu. Pośrodku dużego pokoju stoi duży okrągły drewniany stół; upoważnieni delegaci siedzą bezpośrednio wokół niego; wszyscy inni kręcą się za nimi, siedząc w~kępach na podłodze lub wchodząc i~wychodząc z~innych pokoi. Nie ma oczekiwania, że publiczność będzie milczeć podczas spotkań -- w~rzeczywistości oczekuje się, że delegaci będą stale naradzać się ze swoimi grupami afinicji, a członkowie grup afinicji ze sobą. Chociaż w~tak małym pomieszczeniu facylitatorzy muszą interweniować od czasu do czasu, aby przypomnieć wszystkim, aby ograniczyli głośność do rozsądnego poziomu. 

 Spotkanie oczywiście już się rozpoczęło (czy kiedykolwiek byłem świadkiem początku rady?), choć zaledwie dwie trzecie delegatów jest obecnych. Facylitatorzy, kobieta i~mężczyzna, zorganizowali zwyczajowe pasy papieru plakatowego na ścianie obok i~wypisują agendę kolorowymi mazakami. To jest prawdziwa rada przedstawicieli, więc wszyscy uczestniczą w~konstruowaniu agendy.

 Szukając kogoś z~Nowego Jorku, kto mógłby mnie wprowadzić w~sytuację, dostrzegam Twinkie, pochłaniającą muffina w~rogu.

-- Więc jak się sprawy mają?

 Przerywa, bierze głęboki oddech, szuka odpowiednich słów. 
 
 -- Mogłoby być lepiej. Wczoraj przyjechała delegacja Mohawków z~Akwesasne, prosząc nas, abyśmy nie przyjeżdżali.

-- To byli ludzie z~Rady?

-- Jakieś siedem lub osiem osób, wódz, parę osób, które wykonało pewien rytuał\ldots 

-- Naprawdę? W sensie jak Rytuał Dziękczynienia? Gdzie dziękują Stwórcy za stworzenie nieba, wody, truskawek i~wszystkiego? -- (Ja też doczytywałem w~tamtym czasie.) -- Wiesz, że to standardowy sposób Mohawk na rozpoczęcie każdego ważnego wydarzenia. Zawsze chciałem zobaczyć taki jeden. 

-- Tak, tak, dokładnie tak było. Zaczęli od rytuału Dziękczynienia. To było bardzo piękne. Potem powiedzieli nam, żebyśmy nie przyjeżdżali.

-- A więc byli ludzie z~Rady \textit{i }Tradycjonaliści? Więc teraz -- zaskoczony -- \ldots   tradycjonaliści są przeciwko Wojownikom?

-- Nie, byli też ludzie, którzy mówili, że są ze Stowarzyszenia Wojowników. Postępowcy, tradycjonaliści, wojownicy\ldots  To była katastrofa.

 Usiadłem w~pobliżu i~zacząłem pisać wstępne notatki: wydawało się, że jest około 150-200 osób, z~dość rozsądną równowagą płci, coś w~rodzaju mieszanki etnicznej, chociaż, jak zauważyłem, absolutnie nie było ani jednego Afroamerykanina w~pokoju. Nie, właściwie jeden. Na balkonie mężczyzna wyglądający na Karaiba. Ale tylko tyle.

 Laura, trenerka wrażliwości płciowej, działała również jako ko-facylitatorka z~jakimś facetem z~Bostonu o imieniu Mark. Kiedy zacząłem spisywać notatki ze spotkania, w~rzeczywistości mówiła wszystkim, że mieli w~tym zakresie duże problemy. Jakkolwiek niechętnie, musiałam przyznać, że była w~tym całkiem dobra. 

\noindent \textit{Laura}: Chciałam tylko powiedzieć, zanim zaczniemy, że jestem naprawdę pod wrażeniem szacunku, jaki ludzie okazywali w~odniesieniu do dynamiki rasowej tutaj. i~to pomimo tego, że jesteśmy w~naprawdę trudnej sytuacji. Jeśli chodzi o dynamikę płci, mamy pewne problemy. Więc zanim zrobimy cokolwiek innego, pozwólcie, że powiem: ludzie, proszę, pilnujcie się, zanim zaczniecie mówić. Jeśli mówiliście już dwa lub trzy razy na ten sam temat, a inni nic nie powiedzieli: wycofajcie się. Dajcie innym głosom szansę bycia wysłuchanym. Jeżeli wasze zdanie jest tak ważne, jak sądzicie, to zapewne ktoś inny też o tym wspomni. A jeżeli nie, zawsze możecie dopisać się do listy. Pamiętajcie: to gówno jest głęboko zinternalizowane w~każdym z~nas, więc \textit{proszę}, bądźcie świadomi. 

 Zaczynamy jak zwykle od rundy dookoła, każdy delegat identyfikuje siebie i~grupę afinicji, którą reprezentuje.

\noindent \textit{Laura}: Jeszcze jakieś kwestie gospodarcze? Nie? W porządku, kilka osób, które falicytowały wczorajszą noc, było również zaangażowanych we wcześniejsze negocjacje z~Mohawkami. Daliśmy im piętnastominutową przerwę, aby w~jakiś sposób przybliżyć nam część historii. Prochowiec?

\noindent [Ludzie zaczynają szukać Maca. Ktoś na balkonie mówi, że rozmawia przez telefon. Na szczycie stołu jest krótka grupka]

 W porządku, więc Lesley z~NYC-DAN zajmie jego miejsce. Nie, czekaj, oto Mac.

\noindent [Inne osoby przedstawiają się jako część zespołu: Twinkie z~AUTODAWG, Jessica z~Philadelphia Direct Action Group, Nisha, aktywistka z~Nowego Jorku, która wyjaśnia, że  nie mówi w~imieniu nikogo poza sobą]

\noindent Mac: A ja jestem Mac. Jestem z~DAN i~Kolektywem Prawa Ludowego. Cześć, jak się wszyscy mają?

 Byłem główną osobą rozmawiającą z~Klanem Boots i~Wojownikami z~Rezerwatu Akwesasne i~Tyendinaga. Głównym organizatorem, z~którym miałem do czynienia był Shawn Brant z~Tyendinaga. Spędziłem trzy lata na ulicach z~Shawnem w~Ontario; jest jednym z~najbardziej solidnych, niezawodnych działaczy, jakich znam.

 W styczniu pracowaliśmy z~grupami w~Kanadzie, aby pomóc ludziom przenieść się na drugą stronę, chodziło o zamknięcie każdego posterunku granicznego, który nie chce nas wpuścić. Shawn powiedział, że porozmawia ze społecznością Mohawk w~Akwesasne, i~w końcu niektórzy z~nas poszli się z~nim spotkać. W tamtym czasie określał to jako bardzo zdecydowane działanie, rozpoczynając od stwierdzenia, że  most jest codzienną obelgą dla ich suwerenności i~twierdząc, że zrobią wszystko, aby go przejąć.

 W rezultacie niektórzy ludzie po naszej stronie wygłosili przedwczesne stwierdzenia.

 Tymczasem Shawn udał się do Boots, którzy są ważni w~Klanie Niedźwiedzia w~Akwesasne. Z szacunku zwrócił się również do Rady Plemienia, która jest formalnym, wybieralnym organem, ale zaniepokoiła ich perspektywa ewentualnego podziału na ich ziemiach.

 Tak więc, kiedy poszliśmy na drugie spotkanie w~Akwesasne, po pierwsze wyjaśnili, że akcja nie będzie obejmować faktycznego zamykania granicy, ponieważ Rada była tym zaniepokojona. Powiedzieliśmy im, jasne. Później było więcej obaw: na trzecim spotkaniu Harriet Boots zdecydowanie poparła akcję, wraz ze swoim mężem Johnem i~ich synem Stacey. Chciała się upewnić, że podkreśliliśmy straszne warunki zdrowotne w~rezerwacie, fakt, że lokalna klinika mówi kobietom, aby nie karmiły piersią swoich dzieci, ponieważ woda jest tam tak toksyczna, a most jest obelgą dla suwerenności Mohawków. Powiedziała też, że ponieważ ich celem jest zjednoczenie narodu, chcą, abyśmy byli pokojowi i~zorganizowani. Że o samym rezerwacie nie powinno się za dużo mówić, ale zamierzali zorganizować smażalnię, potem poszliśmy spotkać się z~naszymi kanadyjskimi sojusznikami w~połowie mostu. Pomysł polegał na tym, żeby najpierw pojazdy przejeżdżały zgodnie z~prawem, a potem, po prostu samą liczbą, moglibyśmy pokojowo obezwładnić straż graniczną i~wszyscy mogliby się przedostać. Po tym Rada Plemienia powiedziała w~porządku, ale zgłosiła pewne poważne zastrzeżenia.

 Shawn powiedział mi, że wczoraj wyszło wspólne oświadczenie poparcia. Klan Wilka jest bliżej Rady Plemienia (była swego rodzaju wojna domowa między Wilkami a Niedźwiedziami, które mają tendencję do współpracy z~Wojownikami, w~związku z~planami budowy kasyna kilka lat temu), więc zdecydowanie byli podejrzliwi.

 Jednak zeszłej nocy wystosowali list poparcia, więc szkoda, że  niektórzy z~członków Rady i~Brian Skidder przyszli do naszej rady i~odradzali nam przychodzenie. Powiedzieli nam, że policja pokazała im filmy, które rzekomo pochodziły z~Seattle, ale myślę, że musiały pochodzić z~Pragi, przedstawiające ludzi rzucających koktajlami Mołotowa i~walczących z~policją, i~powiedziała, że to był ten rodzaj problemów, które sprowadzimy na ich ludzi. Po spotkaniu delegaci powiedzieli, że po tym, jak nas spotkali i~zobaczyli, jak traktujemy się nawzajem, uznali, że jesteśmy dobrymi ludźmi i~już się nas nie boją. Ale Brian Skidder wciąż mówił, żebyśmy nie przyjeżdżali.

 Rozmawiałem dziś rano ze Stacey i~Shawnem i~próbuję zdobyć informację prasową, którą wydali. Wciąż bardzo proszą nas, abyśmy przybyli, chcą naszego wsparcia, chcą spokojnego, bezpiecznego działania. Nadal mają święto smażonej ryby, zaprosili nas na nie, a także chcą nam pomóc w~przeprawie do Kanady. Zdaję sobie sprawę, że nie jest to prosta akcja, ale wierzę w~naszą akcję i~uważam, że najwyższy czas, aby ruch antyglobalistyczny zrobił coś takiego i~nawiązał więzi z~aktywistami Pierwszego Narodu po obu stronach granicy.

\noindent \textit{Mark}: Czy są jakieś pytania wyjaśniające?

\noindent \textit{Laura}: \ldots czyli dla zespołu, który współpracował z~organizatorami Mohawk?

\noindent Jessica: Powinnam również zaznaczyć, że byłam na drugim i~trzecim spotkaniu w~Akwesasne i~że to ,,oświadczenie o poparciu'' było bardziej deklaracją braku sprzeciwu.

\noindent Kobieta: Co się teraz dzieje po kanadyjskiej stronie?

\noindent Mac: Nasi sojusznicy w~Kingston mówią, że będą po drugiej stronie mostu, będą elastyczni i~chętni do pomocy w~każdy możliwy sposób. Och, powinienem dodać, że krąży plotka (a będzie wiele plotek; będzie nam trudno to wszystko rozwiązać), że policja już ustawiła autobusy i~dwie ciężarówki wzdłuż autostrady po drugiej stronie mostu. To może być prawda, ale nie musi, ale musimy założyć, że policja też tam \textit{będzie}.

\noindent Mężczyzna: Nasza grupa afinicji chce wiedzieć, czy kiedykolwiek zwrócono się bezpośrednio do Klanu Wilka?

\noindent Mac: Nie, czuliśmy, że powinniśmy pozwolić naszym sojusznikom tym się zająć. Co mogło być błędem.


\noindent  Kobieta: Jestem z~zespołu prawnego i~jesteśmy gotowi przenieść się do innego przejścia granicznego, jeśli zajdzie taka potrzeba. Również wyjaśniające pytanie: dlaczego Shawn, który tak naprawdę nie pochodzi z~Akwesasne, przemawia w~imieniu tej społeczności?

\noindent  Mac: Shawn nie mówi w~imieniu społeczności. Jest organizatorem od dziesięciu czy piętnastu lat; pochodzi z~czołowej rodziny w~Tyendenaga; przeszedł przez właściwe protokoły, aby uzyskać wsparcie rodziny Boots, ale nic więcej.

\noindent  Kobieta: Moja grupa afinicji jest zaniepokojona: czy nadal będą musieli przejść przez odprawę celną?

\noindent  Mac: Jest szansa, że  tak. Mamy nadzieję, że ich pokonamy, ale może nie. Najlepsze, co możemy powiedzieć, to to, że szanse na zrobienie tego tutaj są równie dobre lub lepsze niż gdziekolwiek indziej.

\noindent  Mężczyzna: Jeśli Shawn nie jest z~Akwesasne, to kto właściwie chce, żebyśmy tam byli?

\noindent  Mac: Nie gwarantuję liczb, ale jeden z~najpotężniejszych klanów nas tam chce. Rada Plemienia kręci się w~tę i~z powrotem, a Klan Wilka jest zdecydowanie przeciwko nam. Nasi sojusznicy twierdzą, że mamy dziewięćdziesiąt procent poparcia w~całej społeczności, ale nie wiemy, co się tam naprawdę dzieje. Nie chcę ci powiedzieć czegoś, co okaże się inaczej.

\noindent  \textit{Laura}: Dobra, pozwól, że teraz dam głos każdemu, kto chce zadawać pytania.

\noindent  Sławny: Cześć, jestem Sławny. Jestem z~medykami. Chciałbym wiedzieć, czy będziemy mieli eskortę, gdy będziemy zbliżać się do Rezerwatu?

\noindent  Mac: Nie, nasi sojusznicy będą koncentrować się na bezpieczeństwie samego rezerwatu. Kiedy już tam dotrzemy, nie będzie żadnego faktycznego sprzeciwu, ale może być wcześniej, mówiono o policyjnych blokadach dróg. Ale to od nas zależy.

\noindent  Sławny: Nie, mam na myśli na skraju rezerwatu.

\noindent  Mac: Tak, będzie.

\noindent  \textit{Mark}: Pamiętaj, że to wyjaśnienie historii, a nie scenariusze logistyczne.

\noindent  Tony: Cześć, jestem Tony, też z~medykami. To, co zrobiło na mnie wrażenie w~delegacji, która przybyła tu wczoraj wieczorem, to ich obawy o otwarcie ran po wojnie domowej. Czy ktoś byłby w~stanie się tym zająć?

\noindent  Mac: Cóż, nasi sojusznicy mówią, że to będzie jednoczące działanie, że są teraz bardziej zjednoczeni niż kiedykolwiek. Nie mogę ci powiedzieć, kto ma rację.

\noindent  \textit{Laura}: Inne pytania konkretnie od delegatów, niedotyczące logistyki, to będzie później. W tej chwili pytania dotyczące historii, które wymagają wyjaśnienia.


 Źle to wygląda. Mac jest oddanym anarchistą i~zwykle jednym z~najbardziej otwartych, przyjaznych ludzi, jakich można sobie wyobrazić. Jego zwyczajowe zachowanie jest tak niewinne i~zabawne, że niektórym trudno jest brać go całkowicie poważnie. Teraz, uwięziony między swoim przyjacielem Shawnem a amerykańskimi aktywistami, gdy odpowiada na jedno pytanie za drugim starannie sformułowanymi stwierdzeniami, zaczyna brzmieć jak polityk. Przypuszczalnie w~jego sytuacji jest prawie niemożliwe, aby tego nie robić. W końcu zastanawiam się, czy to nie jest właśnie to, co sprawia, że  politycy mówią jak łasice: bycie złapanym między okręgami wyborczymi, które chcą radykalnie różnych rzeczy, próbując uszczęśliwić wszystkich? Jednak publiczność to zauważa i~wielu nie jest zadowolonych.

\noindent Kobieta: Proponuję porozmawiać o tym bezpośrednio z~Shawnem.

\noindent Mac: Nadal z~nim rozmawiam. Szczerze mówiąc, jest bardzo sfrustrowany naszym ruchem. Czuje, jakby musiał trzymać nas za rękę.

\noindent \textit{Laura}: Pytania konkretnie o historię negocjacji? Nie? Dobra. Jako facylitatorzy nie wiemy, co będzie dalej. Powiedziano nam, że może dostaniemy telefon z~Akwesasne, może dostaniemy przefaksowany list. Więc 

\noindent [\textit{do zespołu}] czy macie coś do powiedzenia? Pomóżcie mi tutaj.]

\noindent Nisha: Jodie, mogłabyś podejść? To jest teraz twoja sekcja w~porządku obrad.

\noindent [Podchodzi kobieta o imieniu Jodie.]

\noindent Jodie: Cześć. Pochodzę z~Philly, dużo pracuję z~Western Shoshone i~innymi grupami rdzennych Amerykanów na Zachodzie. Miałam zorganizować szkolenie z~szacunku dla kultury przed akcją, ale wygląda na to, że nie będziemy mieć na to czasu. Mam ulotkę, którą zamierzam wykorzystać (ludzie mogą podzielić się nią z~sąsiadami, jeśli nie ma wystarczającej liczby kopii), ale najważniejsze jest to, że mamy tu także Russella Blacka z~Oglala Lakota. i~czułam, że może najpierw powinniśmy wysłuchać go.

 Więc Russell, czy mógłbyś wstać i~podzielić się trochę swoim zrozumieniem tej sytuacji?

 Pojawia się wysoki, chudy dzieciak, który wygląda, jakby miał około siedemnastu lat. Stoi po drugiej stronie stołu. 
 
 -- Jestem tutaj w~imieniu moich starszych -- zaczyna. Następnie wypowiada krótką modlitwę i~nieco dłuższą przemowę, podkreślając, że jego naród, Oglala (nadal błędnie zwani Siuksami) dzieli podobny frakcyjność między tradycjonalistami a tak zwaną ,,pragmatyczną'' grupą związaną z~oficjalnym zarządem rezerwatu, który jest skorumpowany i~naprawdę tylko agentami rządu federalnego. Tylko tradycjonaliści zajęli pryncypialne stanowisko przeciwko ludobójstwu i~gwałceniu ziemi\ldots  Wszyscy słuchają w~pełnym skupienia, pełnym szacunku milczeniu. Ja sam nie mogę się powstrzymać, ale refleksja nad tym byłaby nieco bardziej przekonująca, gdyby nie tradycjonaliści z~wczorajszą partią, którzy mówili nam, żebyśmy nie przyjeżdżali. Na końcu, ludzie w~sali reagują częściowo aplauzem, częściowo migocząc dłońmi.

-- Bez względu na to, co zdecydujemy dziś wieczorem -- mówi inna kobieta -- chcemy to zrobić z~szacunkiem. Zostaliśmy zaproszeni na posiłek. Z pewnością niepojawienie się będzie okazaniem braku szacunku.

 Madhava, jeden z~ludzi IMC, ogłasza, że  otrzymaliśmy telefon z~Akwesasne.

 Potem wydaje się, że znowu go straciliśmy.

-- Bardzo ciekawa informacja -- zauważa Laura, gdy Mac i~pracownicy techniczni wspinają się na piętro. -- Nasze linie faksowe zostały w~tajemniczy sposób zajęte przez cały dzień. Nie mogliśmy nic wysłać ani odebrać. Plus linie telefoniczne są niepewne. Próbujemy połączyć się przez DSL\ldots 

\noindent Technik na piętrze: Myślę, że mamy to w~systemie głośników\ldots 

\noindent Mac: Gotowy Shawn? [\textit{trzask}] Nadal tam jesteś? [\textit{trzask}]

\noindent Technik: Cholera, to nie zadziała. Będziemy musieli użyć innego telefonu.

\noindent [Dużo zawracania sobie głowy sprzętem]

\noindent \textit{Laura}: Więc dajcie mi dobre wieści, chłopaki. Możemy kontynuować?

\noindent Mac: [\textit{przez telefon z~Shawnem}] Jak czuliby się ludzie, gdybym zszedł i~powtórzył słowa Shawna?

\noindent [Mruganie]

\noindent [Mac wyjaśnia sytuację Shawnowi]

\noindent \textit{Laura}: Nie, nie schodź. Po prostu powiedz to z~balkonu.

\noindent Mac: Dobrze. [z balkonu zaczyna powtarzać to, co mówi Shawn]:

 Po pierwsze, chcę przeprosić za robienie rzeczy w~ten sposób. Byłoby o wiele bardziej odpowiednie, gdybyśmy byli tam z~wami, ale po prostu nie jest to możliwe w~tej chwili. Chcę tylko powiedzieć, że dzieje się tu dużo bzdur\ldots 

\noindent Ktoś: Um, możesz to powtórzyć, przepraszam?

\noindent [Dużo śmiechu]

\noindent Shawn [\textit{przez Maca}]: Jako aktywiści ponosimy wspólną odpowiedzialność. ,,Wolny handel'' dotyczy ludzi zmanipulowanych przez rząd. To, co wydarzyło się na waszym spotkaniu zeszłej nocy, było manipulowaniem nami przez miejscowy rząd. Ci ludzie nie reprezentują najlepszego interesu mieszkańców Akwesasne, a ludzie potwierdzili wczoraj swoje nastroje wobec Amerykanów przybywających, by zaprotestować przeciwko FTAA. Nie mamy indiańskich tytułów za naszymi nazwiskami, ale mamy honor i~uczciwość i~jesteśmy prawdziwymi przywódcami Narodu Mohawk. Ten honor i~uczciwość znajduje odzwierciedlenie w~naszym zaangażowaniu i~fakcie, że zrobiliśmy to, co obiecaliśmy. Te próby zmniejszenia naszej liczby poprzez proszenie aktywistów, aby nie przyjeżdżali, są oparte na strachu: rząd wie, że nie ma kontroli, że są częścią systemu, który pozwolił na zatrucie naszej społeczności, narodziny naszych dzieci z~wadami wrodzonymi, utratę integralności i~kultury. A teraz twierdzą, że działają w~naszym najlepszym interesie, aby uniemożliwić ludziom przyjście.

 Wczoraj potwierdziliśmy, że potwierdzimy honor, jakiego wymaga się od osób udających się do Quebec City, aby legalnie odwieść rządy od dalszych negocjacji w~sprawie wolnego handlu. Uznajemy, że ci, którzy idą walczyć z~rządami, z~którymi walczymy, winni zostać docenieni za ich zaangażowanie, ponieważ mamy tego samego wroga. Jeśli ludzie są zniechęceni do przyjścia, to z~ich wyboru, złożyliśmy przyrzeczenie i~zobowiązanie i~je podtrzymujemy. Wszyscy jesteście mile widziani w~Akwesasne i~to w~takim samym stopniu, jak mówiliśmy w~poprzednich dyskusjach.

 Mogę tylko powiedzieć, że mam nadzieję, że ludzie przyjdą, ale z~pewnością rozumiem zamieszanie, które ludzie zeszłej nocy wprowadzili do umysłów ludzi. Ale lud jest z~wami.

\noindent [Dzikie oklaski]

 Po kilku bezskutecznych próbach utrzymania linii otwartej, aby ludzie mogli zadawać Shawnowi pytania, połączenie załamuje się i~odtąd nie można uruchomić żadnego telefonu w~budynku. W końcu zwracamy uwagę na logistykę spotkań (z opóźnieniami, trening wrażliwości kulturowej Jodie, zaplanowany na 19:00, będzie musiał zostać przeniesiony do innego budynku o 20:00 lub 21:00. Potem jest problem z~kolacją\ldots ).

\noindent \textit{Laura}: Więc. Czy grupy afinicji rzeczywiście przedstawiły propozycje dotyczące tego, jak powinniśmy od tego momentu postępować?

\noindent [Wskazania z~kilku, które mają]

 Nie każdy musi, ale jeśli w~ogóle, możemy spróbować ustalić, w~jaki sposób różne propozycje nakładają się na siebie, powiązać je i~dopracować do praktycznej listy alternatyw.

\noindent Kobieta w~Żółci: Oto nasza propozycja. Proponujemy, abyśmy udali się do Akwesasne, ale miejmy umysły otwarte na możliwości, czy przejdziemy tam, czy w~innym miejscu. Powinniśmy pozostać w~kontakcie z~Mohawkami po stronie kanadyjskiej, aby mogli nam powiedzieć, co się tam dzieje. Idziemy na smażenie ryb, dokonujemy ponownej oceny, odbywamy tam radę.

\noindent \textit{Laura}: Dobrze, to jedna propozycja. Wspaniale. Inni? Aha, i~pamiętajmy, że dziś wieczorem możemy również opracować alternatywne scenariusze, możecie zaproponować coś nowego. A więc: jakieś inne? Nie?

\noindent Kobieta w~błękicie: Nasz kolektyw złożył alternatywną propozycję po wczorajszej nocy, kiedy wszystko wydawało się chwiejne. Po pierwsze, ze względu na solidarność z~Mohawkami, powinniśmy pójść na spotkanie, jeśli nadal jesteśmy zaproszeni (oczywiście jesteśmy), aby nawiązać współpracę z~tamtejszymi działaczami i~odebrać innych, którzy będą przychodzić Akwesasne w~celu przejścia. Potem powinniśmy spróbować przedostać się do Kanady w~innym miejscu.

 Inną możliwością zaproponowaną wczoraj wieczorem było dołączenie do Kanadyjczyków i~wszystkich Mohawków, którzy chcą przejść, i~próba zrobienia tego razem w~innym miejscu.

\noindent Eric z~NYC DAN: Kiedy miała zostać podjęta decyzja o innej lokalizacji?

\noindent Inna kobieta z~tego kolektywu: Nie sądzę, żeby mówienie o tym tutaj było strategicznie rozsądne. Ale na pewno są ludzie, którzy nad tym pracują.

\noindent Laura: Również mamy to. [Ktoś zaczyna rozdawać drukowaną wersję pierwszej propozycji po pokoju; wszyscy delegaci wydają się już mieć jedną]

\noindent Kobieta: Czy była propozycja rozmowy z~Radą Plemienia?

\noindent \textit{Laura}: Właściwie myślę, że to wszystko, co mamy teraz. Możesz zaproponować przyjazne poprawki, ale w~tej chwili przejdźmy najpierw do obaw\ldots 

\noindent Enos: Cześć, jestem Enos i~mówię za kolektyw Ya Basta!, wraz z~NYC DAN. Słyszałem od mieszkańców Nowego Jorku dwie obawy: po pierwsze, pierwotna propozycja, by po prostu iść i~przejść, sprzed wczorajszego przybycia delegacji, wciąż jest na stole i~nikt o tym nie dyskutuje; po drugie, że podjęcie decyzji może być zbyt trudne, gdy już tam dotrzemy.

 Enos jest radykalnym rysownikiem z~Nowego Jorku, po czterdziestce, z~długim blond kucykiem i~tylko śladem brooklińskiego akcentu. Nie jest dla mnie jasne, w~jaki sposób stał się delegatem; nie jest jasne, w~jakim stopniu kolektyw Ya Basta! w~tym momencie nawet istnieje, chociaż teraz zauważam, że w~pokoju jest może z~tuzin Yabbów. Wygląda na to, że rekonstruujemy się, przynajmniej jako grupa afinicji. Jednak na początku jestem zbyt zajęty robieniem notatek, żeby brać w~tym udział.

\noindent \textit{Laura}: Cóż, w~takim razie, czy ktoś może powtórzyć pierwotny plan?

\noindent Kobieta: [czyta ulotkę] Że karawana jedzie na imprezę smażenia ryb; spotykamy się tam około 12 w~południe; wysłuchać dwóch lub trzech prelegentów oraz wszelkich innych wydarzeń, które zaaranżowali nasi gospodarze; potem o 16:00, po jedzeniu, wracamy do naszych pojazdów, jedziemy na most (utrzymując jeden pas autostrady otwarty, zgodnie z~życzeniem Rady Plemienia, aby pojazdy uprzywilejowane i~tak dalej mogły przejeżdżać), spotykamy się z~Kanadyjczykami w~centrum, mieszamy się razem z~nimi, przechodzimy na drugą stronę i~razem podchodzimy do pograniczników.

\noindent \textit{Laura}: Czy są jakieś inne propozycje, które powinny znaleźć się tutaj?

\noindent [Najwyraźniej nie]


\noindent  \textit{Mark}: Dobra, więc pierwszą propozycją, jaką dziś usłyszeliśmy, jest pójście na imprezę, utrzymywanie kontaktu z~naszymi sojusznikami w~Akwesasne, ponowne oszacowanie naszego poparcia, ponowne zebranie delegatów i~podjęcie decyzji, jak dalej postępować. W każdym razie tak jest teraz, dziś wieczorem z~pewnością możemy dodać do tego dalsze opracowania.

\noindent  [Członkowie Rady podnoszą ręce]

\noindent  Kobieta: Mam problem: czy próbujemy teraz dojść do konsensusu w~sprawie tej propozycji?

\noindent  \textit{Mark}: Nie, nie staramy się dojść do konsensusu, ale tylko wyczucia atmosfery, jak sądzę, od którego zacząć\ldots 

\noindent  [Trzy ręce wystrzeliwują wokół stołu]

\noindent  Kobieta: Może lepiej byłoby zacząć od sondażu, aby zobaczyć, gdzie jesteśmy, na którą propozycję skłania się większość z~nas, a następnie zrobić sesję grupową, aby przedstawiciele mogły skonsultować się ze swoimi grupami afinicji, jak dalej postępować?

\noindent  \textit{Mark}: Nie, myślę, że naprawdę musimy to ustalić. Będzie coś w~rodzaju sesji później, kiedy będziemy jeść. Wtedy wszyscy możemy naradzić się bardziej szczegółowo z~naszymi grupami afinicji.

\noindent  Tony: Gdybyśmy przeprowadzili ankietę z~sali, czy dotyczyłoby to wszystkich w~pokoju, czy tylko przedstawicielek?

\noindent  \textit{Mark}: Zakładałem tylko delegatów. Chyba że ktoś chce zaproponować, że rozszerzamy?

\noindent  [Żadna taka sugestia się nie pojawia. Aż do\ldots ]

\noindent  Enos: Obawiam się, że ta sala naprawdę reprezentuje większość grupy, która faktycznie pójdzie. W takim przypadku prawdopodobnie powinniśmy po prostu wysondować wszystkich teraz, gdy mamy ich w~jednym miejscu; ponieważ im więcej czasu mija, tym więcej ludzi prawdopodobnie zacznie odpływać. Więc im szybciej będziemy mogli naradzać się z~naszymi grupami, tym lepiej.

\noindent  [dużo migotania dłońmi]

\noindent  \textit{Laura}: Dobra, widzę duże poparcie dla tej sugestii. Zrobimy to w~ten sposób.

\noindent  Kobieta: Czy mogłabyś najpierw przeczytać każdą propozycję?

\noindent  \textit{Mark}: Dobrze, sprawdzimy atmosferę, a potem zrobimy szybkie sesje w~grupach.

\noindent  \textit{Laura}: Jak długo ludzie sugerują przerwę? Widzę dwie minuty\ldots  pięć minut\ldots  dziesięć\ldots  Nie, proszę, nie sala, tylko przedstawiciele\ldots  W porządku, więc wydaje się, że dziesięć minut.

\noindent  \textit{Mark}: Propozycja punkt 1 to iść do smażalni i~wtedy zdecydować; punkt 2 to iść na smażenie, nie przechodzić, ale zaprosić innych Mohawków, aby poszli z~nami na inne przejście; numer 3 to pierwotny plan, w~którym wszyscy spotykamy się na środku mostu i~próbujemy pokonać celników.

\noindent  \textit{Laura}: Widzę trzy ręce delegackie, które chcą coś powiedzieć. [\textit{Tworzy listę}]

\noindent  Mężczyzna w~blond dredach: Chcę złożyć propozycję, żebyśmy w~ogóle nie szli na smażenie ryb i~znaleźli zupełnie inne miejsce do przebycia.

\noindent  \textit{Laura}: Czy jest jakiś powód, dla którego to nie wyszło wcześniej?

\noindent  Dredy: Właśnie zwrócono mi uwagę na ten temat.

\noindent  Kobieta w~Żółci: Rozumiem, że ma to być spotkanie tuż przed rezerwatem, abyśmy mogli spotkać się z~naszymi sojusznikami, ale także zgromadzić społeczność, Shawn mówił, że będzie to ,,przyjazne dzieciom'', coś w~rodzaju impreza z~balonami i~grami, okazja dla nich, aby posłuchali, jak rozmawiamy o wolnym handlu, a potem dla nas głównie do słuchania. Poza tym robią nam wegańskie jedzenie, tradycyjną zupę kukurydzianą, oprócz ryb. To samo w~sobie było niezwykłe. Nigdy o czymś takim nie słyszałam.

\noindent  \textit{Laura}: Widzę, że jest tu dużo energii, ale\ldots  ktoś ma tam pytanie. Tak?

\noindent  Mężczyzna: Tak, chodzi o tę ostatnią propozycję. Jeśli się nie mylę, zadzwoniliśmy do tej konkretnej rady, żeby omówić plany dla Cornwall. Teraz oczywiście nikt z~nas nie ma obowiązku jechać do Cornwall, jeśli nie chce jechać, ale jeśli ktoś chce rozmawiać o tym, żeby w~ogóle nie jechać do Cornwall, czy nie powinien wycofać z~tego przedstawicieli i~po prostu wezwać innych delegatów dla ludzi, którzy nie chcą jechać?

\noindent  \textit{Laura}: Więc. [do Dredów] Czy rzeczywiście chcesz wycofać propozycję? Albo nie?

\noindent  Dredy: Tak, w~takim razie chciałbym tak zrobić.

\noindent  \textit{Mark}: OK, potrzebne są jakieś dodatkowe wyjaśnienia dotyczące pierwszej propozycji?

\noindent  Kobieta: Gdybyśmy zorganizowali radę w~Akwesasne, czy przedstawiciele obejmowaliby Mohawków?

\noindent  [Dużo dyskusji. Nie jest jasne, czy ktokolwiek wie.]

\noindent  Enos: Ya Basta! właśnie przekazał mi informację, że przy smażalni będą jacyś członkowie Rady Plemienia.

\noindent  Neala: Pierwsza propozycja mówi, że jeśli pójdziemy, powinniśmy być ,,otwarci w~naszych umysłach'' na to, co robić dalej. Ale tak jak Enos, naprawdę wolałabym, żeby decyzja została podjęta wcześniej. Nie mamy pojęcia, jak tam będzie, czy w~ogóle będziemy w~stanie utrzymać radę.

\noindent  \textit{Mark}: OK, ale technicznie wciąż wracamy do wyjaśniania pytań dotyczących pierwszej propozycji, a nie obaw.

\noindent  \textit{Laura}: Również tłum nie powinien przemawiać bezpośrednio do delegatów. To facylitator powinien. Wiem, że to brzmi ograniczająco, ale jeśli nie zrobimy tego w~ten sposób, przedstawiciele mogą poczuć się zagrożeni.

\noindent  Kobieta w~Żółci: Chcę doprecyzować moją propozycję (teraz jest to propozycja nr 1. To, o czym mówimy, to fakt, że Mohawkowie chcą nam zrobić wegańskie jedzenie, to niesamowity, bezprecedensowy pokaz gościnności. \textit{Musimy }przyjść.

\noindent  \textit{Ocena}: Na razie nie słyszę żadnych pytań wyjaśniających, a jedynie obawy i~popierające argumenty. Czy to oznacza, że  przechodzimy do obaw?

\noindent  \textit{Laura}: Zostałam poinformowana, że odpowiedź na ,,czy Mohawkowie będą zaangażowani w~radę delegatów'' brzmi ,,jeśli chcą''.

\noindent  [Zaczyna pisać na jednej z~kartek papieru plakatowego na ścianie, rozpoczynając kolumnę zatytułowaną ,,troski'' ]

\noindent  Kobieta: Ups. Mam jeszcze jedno pytanie wyjaśniające. Czy byłoby niegrzecznie iść na imprezę i~nie przekroczyć granicy? A może już o to zapytano, a Mac powiedział, że tak nie będzie?

\noindent  Mężczyzna: Również pomysł ponownego zebrania tam przedstawicieli sprawia, że\ldots 

\noindent  \textit{Mark}: Teraz zinterpretuję to pytanie jako troskę.

\noindent  Mężczyzna: \ldots czy byłoby to brakiem szacunku?

\noindent  Kobieta: Justin właśnie mi powiedział, że będą ludzie z~całego północnego wschodu, którzy przyjadą prosto do Akwesasne, nie przejeżdżając przez Burlington. Może kilkaset.

\noindent  \textit{Mark}: To właściwe wykorzystanie informacji, ale nadal szukam tutaj pytań wyjaśniających lub wątpliwości.

\noindent  Enos: Jeśli chodzi o pierwszą propozycję: jakie będą kryteria odwołania działania?

\noindent  Kobieta w~Żółci: Jeśli ludzie ze społeczności nie przyjdą, nie odezwą się do nas\ldots  Jeśli poczujemy, że nie jesteśmy potrzebni, odejdziemy tak szybko, jak to możliwe.

\noindent  Mężczyzna: Jeśli przejdziemy granicę, czy to będzie oznaczało przejście przez odprawę celną?

\noindent  Inny mężczyzna: Czy to tak, że nie chcemy nic robić sami, to wszystko zależy od nich? Nie chcemy robić niezależnego zakłócenia?

\noindent  Fred: A jeśli zostaniemy zawróceni, czy przejdziemy do alternatywnego miejsca?

\noindent  Kobieta w~Żółci: Pytanie brzmiało, jakiego rodzaju przekroczenie granicy nastąpi z~każdą propozycją? W przypadku mojego nr 1, myślę, że odpowiedź może brzmieć tylko: jakikolwiek rodzaj zaproponowany przez Mohawków. Lucy, negocjowałaś z~mieszkańcami Akwesasne. Co myślisz?

\noindent  Lucy: Nie słyszałam jeszcze żadnych wytycznych. Innych niż niestosowanie przemocy.

\noindent  Mężczyzna: Mac powiedział mi, że uznano by to za brak szacunku, jeśli po prostu pójdziemy do smażalni i~natychmiast wyjdziemy.

\noindent  Kobieta: Jeśli pójdziemy na granicę, to wtedy wszystko lub nic? A jeśli niektórzy z~nas przejdą, a inni nie? Rozdzielamy się, czy wszyscy solidarnie zawracamy?

\noindent  Kobieta w~Żółci: To logistyczna decyzja. Myślę, że po zakończeniu tej części robimy propozycje logistyczne.

\noindent  \textit{Mark}: Czy są jakieś inne pytania wyjaśniające lub wątpliwości dotyczące propozycji nr 1? Nie? Dobra. A co powiecie na numer 2?

\noindent  [Ponownie przedstawiają propozycję ]

\noindent  Enos: Jak byłoby to spójne z~podążaniem za bezpieczeństwem Mohawk?

\noindent  \textit{Mark}: Dobra, to problem.

\noindent  [Laura to zapisuje]

\noindent  Nancy: Cześć, jestem Nancy z~Pittsburgha. Co to znaczy ,,zaprosić'' Mohawków do przekroczenia granicy z~nami gdzie indziej? Czy zamierzamy z~nimi usiąść i~opracować strategię, czy po prostu przedstawimy im już propozycję? Bo jeśli to pierwsze, nie ma dużej różnicy między tymi dwiema propozycjami.

\noindent  Kobieta w~niebieskim [\textit{która złożyła propozycję}]: Dla mnie to kwestia semantyki. Nie wiem, ale powiemy im, że mogą z~nami pojechać.

\noindent  Nancy: Ale pomysł jest taki, że mamy gotowy plan?

\noindent  Kobieta w~niebieskim: Nie widzę innego sposobu na zrobienie tego.

\noindent  \textit{Laura}: Dobra, więc przychodzimy z~wcześniej ustalonym planem.

\noindent  Kobieta w~niebieskim: Ze względu na tę sytuację wiele osób tutaj nie ma innej szansy, więc zdecydowanie powiem ,,wszystko albo nic''. Przy pierwszej osobie, która zostanie zawrócona, reszta z~nas też zawraca, solidarnie.

\noindent [Wiele migotania]


 To było kluczowe: powstającym planem było przytłoczenie posterunku granicznego czystymi liczbami, a to zadziała tylko wtedy, gdy nalegamy, aby wszyscy przeszli razem. Pojawił się więc pewien rodzaj konsensusu. Po ustaleniu tego wszyscy przerwaliśmy radę, aby przy kolacji skonsultować się z~naszymi przedstawicielami.

 Pozostałości Ya Basta! w~Nowym Jorku spotkały się w~jednym rogu pokoju, z~plastikowymi talerzami pełnymi jakiegoś wegańskiego kuskusu i~papierowymi kubkami pełnymi cydru jabłkowego. Była godzina 19. Odkryłam, że to był pierwszy raz, kiedy nasz kolektyw spotkał się, w~jakimkolwiek charakterze, od ostatniego nieudanego treningu. Moose zniknął, ale poza tym byli to głównie ludzie z~DAN -- hardkorowa frakcja, nigdy duża, do tej pory całkowicie zniknęła. Szybko przeanalizowaliśmy te trzy propozycje i~zdecydowaliśmy, że jeśli pójdziemy, co prawdopodobnie zrobimy, najlepiej będzie trzymać się pierwotnego planu i~spróbować przejść w~Akwesasne. Propozycja nr 1 wydawała się zbyt słaba. Propozycja nr 2, którą moglibyśmy zachować jako kopię zapasową, gdyby coś poszło nie tak. Mieliśmy też silne poczucie, że powinniśmy wspierać zasadę ,,wszystko albo nic''. Upoważniliśmy Enos do blokowania wszelkich propozycji, które ich nie zawierały. Enos wrócił do stołu, ja znalazłem coś do jedzenia i~próbowałem odszukać ludzi, u których miałem nocleg, żeby upewnić się co do warunków.

 Kilka minut później wpadłem na Kitty z~Connecticut, która zapytała o załamanie się w~Ya Basta!. To wszystko bardzo ją irytowało, zauważyła, jako przedstawicielka prawdopodobnie drugiej co do wielkości kolektywu Ya Basta! na Wschodnim Wybrzeżu. 
 
 -- To znaczy, zdaję sobie sprawę, że dynamika płci była popieprzona. Ale po prostu rozłożyć ręce i~uciec w~ten sposób. Gdzie to nas stawia? W każdym razie mam pomysł. Wciąż mamy cały sprzęt. Dlaczego nie spróbujemy zorganizować spotkania wszystkich, którzy zamierzali być częścią kontyngentu Ya Basta! i~nie zobaczymy, jakie zasoby wciąż mamy, jakie liczby? Spróbujmy sprawdzić, czy nadal nie możemy czegoś zrobić?

 Powiedziałem, że brzmi to dla mnie jak doskonały pomysł.

 W każdym razie w~końcu miałem projekt. Tuż obok przedsionka znajdowała się pusta sala konferencyjna, ze stołami już ustawionymi w~kwadracie. Zlokalizowaliśmy papier i~magiczny marker, umieściliśmy informację, że będzie spotkanie Ya Basta! o 22:15, a następnie poszliśmy, aby zacząć informować ewentualne zainteresowane strony.

 Zanim wróciłem do głównego pokoju i~robiłem notatki, o 19:35, sprawy stawały się coraz brzydsze. Najwyraźniej sondaż sali podzielił się dość równo między trzy propozycje, dając niewiele wskazówek, jak postępować. Laura pisała obawy, jedna po drugiej, na ścianie za nią, próbując zobaczyć, jakie są punkty sporne, czy można połączyć propozycję, która zawierałaby wszystkie z~nich. Coraz bardziej zaczynało wyglądać, jakbyśmy w~końcu poparli numer 3 -- nowy argument polegał na tym, że gdybyśmy się pojawili, a nie próbowali przejść, obrażalibyśmy Wojowników, którzy zorganizowali przejście.

\noindent  Enos: Słuchaj, nigdy nie będziemy w~stanie zrobić niczego, co by kogoś nie uraziło. i~tak, czasami osoba, którą zamierzamy obrazić, \textit{będzie }członkiem uciskanej grupy. Może powinniśmy po prostu to przeboleć.

\noindent  \textit{Laura}: Czy mógłbyś mówić jaśniej, żebym mogła zapisać?

\noindent  \textit{Mark}: Poza tym słyszeliśmy wiele tych samych punktów, które powtarzały się w~kółko, więc pozwólcie, że zapytam: jeśli jesteście na liście, ale ktoś inny wyrazi Waszą obawę przed tobą, proszę, nie powtarzajcie. Po prostu zrezygnujcie i~pozwólcie mówić następnej osobie.

\noindent  Kobieta w~Żółci: Cóż, w~odpowiedzi na pytanie, propozycja nr 2 została zaproponowana w~odpowiedzi na obawy ludzi zeszłej nocy. Może Russell może wyjaśnić, dlaczego uważam, że uczęszczanie na smażenie jest dla mnie kluczowe. Russella?

\noindent  Russell: Obawiam się, że jest dużo zamieszania w~związku z~Pierwszymi Narodami. W moim Narodzie, gdyby społeczeństwo wojowników zaprosiło was formalnie i~zaoferowało jedzenie i~cenne danie, gdybyście je odrzucili, byłby to najwyższy brak szacunku. Gorąco zachęcam was do wsparcia Stowarzyszenia Wojowników, ponieważ będą na czele walki, a ja będę również reprezentował moje społeczeństwo. W moim społeczeństwie są też ,,postępowcy'', którzy twierdzą, że mówią w~imieniu wszystkich, ale tradycjonaliści powinni zawsze mieć najsilniejszy głos.

\noindent  Mężczyzna: Czuję, że to bardzo ważne, kiedy tam dotrzemy, aby zobaczyć, jakiego rodzaju wsparcie naprawdę mamy w~społeczności, zanim się zobowiążemy.

\noindent  Inny mężczyzna: Moja grupa afinicji absolutnie nie przejdzie przez odprawę celną. Nadal czekam, aby dowiedzieć się, czy jesteśmy o to proszeni, czy nie.

\noindent  \textit{Mark}: Punkt informacyjny: czy są inne rady, inne akcje dla tych, którzy nie chcą jechać do Cornwall? Czy ktoś organizuje alternatywy? Nie?

\noindent  W porządku, czy są jakieś inne obawy?

\noindent  Kobieta: Co się stanie z~innymi ludźmi, którzy przyjdą na imprezę, \textit{gdy} odejdziemy?

\noindent  Neala: W odpowiedzi na wcześniejszą tezę Enosa: jeśli mamy kogoś urazić, nie powinni to być nasi sojusznicy.

\noindent  [Rozproszone oklaski]

\noindent  Mężczyzna: Jeśli o mnie chodzi, cały ten proces jest rasistowski. Powinniśmy rozmawiać ze wszystkimi stronami od samego początku. To niefortunne, że daliśmy się zwieść ludziom, którzy bagatelizują konflikt w~społeczności, i~wiem, że niesprawiedliwe jest mówić, że jakakolwiek osoba jest rasistką, ale wiele kwestii, o których tu słyszałem, to bzdury. Nie mówię, że powinniśmy iść do domu albo nie iść, ale naprawdę czuję się zobowiązany, by to podkreślić.

\noindent  \textit{Laura}: Dobra, czy mogę prosić, abyśmy nie identyfikowali czyjegoś punktu jako ,,bzdury'' ani nie robili podobnej emocjonalnej gry? Spójrz, wszyscy jesteśmy wykończeni. Ale musimy pamiętać, dlaczego tu jesteśmy: jesteśmy tutaj, ponieważ wszyscy staramy się znaleźć najlepszą rzecz do zrobienia w~trudnej sytuacji. Poza tym trochę się martwię, że ludzie zachowują się bardziej drażliwie, ponieważ są głodni. Więc ludzie, jeśli wasza przedstawicielka nie została jeszcze nakarmiona, zapytajcie, wciąż jest mnóstwo jedzenia.

\noindent  Enos: Słuchaj, przepraszam, jeśli powiedziałem coś, co kogoś uraziło. Rozumiem, że wszyscy jesteśmy tutaj z~właściwych powodów. Nigdy nie chciałem sugerować inaczej.

\noindent  \textit{Mark}: Musimy dać sobie nawzajem domniemanie uczciwości i~dobrych intencji. Konsensus to nie to samo, co zasada większości; to nie jest konkurencja. Wszyscy pracujemy razem, aby ustalić, co należy zrobić.

A więc, to powiedziawszy: czy istnieją inne obawy dotyczące propozycji nr 2?

\noindent  Ariel: Czy powinnam przeczytać oświadczenie, w~którym wyjaśniamy naszym sojusznikom Mohawk, dlaczego przejeżdżamy gdzie indziej i~zapraszamy ich, aby przyszli?

\noindent  \textit{Mark}: Cóż, brzmi to trafnie, ale myślę, że lepiej będzie przeczytać to później.

\noindent  Kobieta w~Żółci: Obawiam się, że wystosowanie takiego zaproszenia do Mohawków zostałoby zinterpretowane jako sprzeczne z~pierwotną ideą naszego wsparcia. Teraz zapraszamy ich do odrzucenia ich własnego działania?

\noindent  \textit{Laura}: [wciąż patrzę na listę ,,obaw'' na ścianie ] Czy to pasuje do kategorii pójścia do ich ziemi i~zignorowania ich inicjatywy? Ponieważ ta obawa została już podniesiona.

\noindent  Kobieta w~Żółci: Dla mnie byłaby to ostateczna negacja tego, dlaczego tu przyjechałam, czyli żeby \textit{ich }wesprzeć.

\noindent  Enos: Jeśli ludzie opracowali alternatywne lokalizacje, mam nadzieję, że o nich usłyszymy. Ciągle podkreślam, że zarówno numer 1, jak i~numer 2 zakładają alternatywną trasę, ale czy taka w~ogóle istnieje? Nie \textit{możemy }po prostu improwizować później. Potrzebujemy planu!

\noindent  Kobieta: Pamiętaj, powodem, dla którego na początku wydawało się, że numer 2 jest najbardziej pożądany, było to, że nie mamy jasnego wyobrażenia o tym, czego chce tamtejsza społeczność. Przybyłam tutaj, aby wspierać Mohawków, ale wyraźnie istnieje różnorodność pragnień. A niektóre obawy, które słyszałem, brzmią głęboko. Nie jestem przekonana\ldots 

 To może trwać wiecznie. Niektórzy upierają się, że są tutaj, aby wspierać naród Mohawk jako całość i~zastanawiają się, jak to zrobić. Inni są tutaj, aby wspierać naszych sojuszników, chociaż nie jest jasne, kim i~ilu ich w~rzeczywistości jest. Organizatorzy używają telefonów komórkowych, aby skontaktować się z~rodziną Shawn and the Boots, od czasu do czasu komunikują się na tyle długo, aby uzyskać dodatkowe wyjaśnienia.

 Znowu wyczuwając salę, zaczynam rozumieć, na czym polega problem. To nie jest zwykły tłum aktywistów. Albo nawet anarchistów. Pokój ma charakterystyczne poczucie czarnego bloku. Wezwanie Warcry'a i~Targeta do IMC było o wiele bardziej skuteczne, niż ktokolwiek z~nas się spodziewał: prawie każdy anarchista, który wiedział na pewno, że nie będzie w~stanie legalnie przedostać się przez granicę, niektórzy nawet z~Los Angeles, utknął tutaj w~Burlington. Z jednej strony istniał silny kontyngent -- na przykład Twinkie była jedną z~nich -- który uważał, że kiedy już zobowiązaliśmy się współpracować z~aktywistami Mohawk w~kwestiach Mohawk, naszym obowiązkiem było postępować zgodnie z~nimi, a jeśli to oznaczało, że nie pojechaliśmy do Quebec City, więc niech tak będzie. Dla nich myślenie o Mohawkach jako o środku do celu, jako o drodze do Kanady, było kolejnym przykładem arogancji, rasistowski wyzysk. Inni czuli równie mocno, że nie przyjechali aż z~Iowy czy Południowej Karoliny tylko po to, żeby zjeść lunch w~rezerwacie, gdzie większość ludzi i~tak ich nie chciała.

 Wchodzę i~wychodzę, informując ludzi o nadchodzącym spotkaniu. W głównym pomieszczeniu przedstawiciele powoli zmierzają w~kierunku zaakceptowania oryginalnej propozycji, ale nikt nie jest z~tego szczególnie zadowolony. O 21:48 Enos prawie krzyczy. 
 
 -- Jak dokładnie ten plan całkowicie się zmienił w~ciągu ostatnich dwóch godzin? Teraz proszą nas, abyśmy poddali się \textit{celnikom}! 
 
 Mac upiera się, że plan się nie zmienił, pomysł jest i~zawsze był przeciążyć przejście graniczne. Dzięki temu, że ludzie przychodzą z~obu stron mostu, mając wystarczającą liczbę, możemy stworzyć dla nich logistyczny koszmar, który w~końcu po prostu nas przepuszczą.

\noindent Mac: Może jestem tępy, ale nie widzę, jak to się różni od tego, co mówi Russell. i~nie sądzę, żeby Mohawkowie byli zmartwieni pomysłem, że wszyscy razem przejdziemy. Tak, mogą przetrzymywać ludzi na posterunkach granicznych, ale to nie jest takie powszechne, a jeśli spróbujemy przedostać się w~innym punkcie wzdłuż granicy, możemy tam też trafić do więzienia.

\noindent \textit{Mark}: W porządku, plan jest taki, że wypróbujemy oryginalny plan, ale mamy plan awaryjny. Widzę wiele kiwnięć głową, ilekroć słyszę słowa ,,wszystko albo nic'', więc uważam, że to też nasza decyzja. Jeśli ktoś zostanie zawrócony na początku, wszyscy odejdziemy i~wycofamy się z~naszej akcji awaryjnej?

\noindent [Ogromne migotanie dłońmi]

\noindent \textit{Mark}: Więc taka jest propozycja. Czy są jeszcze jakieś obawy.

\noindent [\textit{Nie}]

 W porządku, w~końcu możemy przejść do konsensusu. Ktoś staje z~boku?

\noindent [\textit{Nie}]

 Bloki?

\noindent [\textit{Nie}]

 Zatem mamy to.

\noindent [Rozbrzmiewa ogromny wiwat ]

\noindent Ktoś: Oklaski dla naszych facylitatorów. Wykonaliście niesamowitą robotę.

 Po zatwierdzeniu planu przechodzimy do kolejnego etapu, jakim jest logistyka. Są dwaj nowi facylitatorzy. Występują prelegenci z~działu prawnego, medycznego i~transportowego. Zespół prawny rozpoczyna rozdawanie formularzy. Wychodzę na spotkanie z~Kitty i~przygotowuję się na to, co wszyscy teraz nazywają ,,spotkaniem 22:15'' 

\noindent \textbf{,,Plan B''}

 Wtedy zaczyna się dziać coś ciekawego. Jakoś nie jest jasne, kiedy plan na spotkanie Ya Basta! przekształca się w~coś innego. Staje się spotkaniem, sponsorowanym przez Ya Basta!, dla każdego, kto czuje się przytłoczony strukturą Rady Delegatów i~chce porozmawiać o strategiach rzeczywistego przejścia. Kiedy po raz pierwszy wchodzę do pokoju, jestem zaskoczony: przy stole jest już co najmniej sześćdziesiąt osób, całkiem spora część aktywistów wciąż znajduje się w~budynku, a coraz więcej napływa. Myślę, że do pewnego stopnia wielu przyszło tylko po to, żeby zabrzmieć. Pierwsze dziesięć minut było niekończącą się sesją narzekań, z~naciskiem na to, jak mało oni lub niektórzy członkowie ich grupy afinicji byli przygotowani do poddania się straży granicznej (niekończące się sprawy, zaległe nakazy itp.). Była jedna dziewczyna, która miała siedemnaście lat, która rok wcześniej uciekła z~domu. Od tego czasu ona i~jej rodzina pogodzili się, ale nadal była oficjalnie wymieniona jako osoba zaginiona; przypuszczalnie, gdyby próbowała przekroczyć granicę, nie tylko zostałaby zatrzymana, ale każdy w~tym samym samochodzie mógłby zostać aresztowany jako jej porywacz. Wiele z~nich jest szczególnie rozgoryczonych po porzuceniu innych, całkowicie realnych opcji, takich jak niepatrolowane odcinki lasu lub niejasne wiejskie drogi czy szanse na przekroczenie granicy kilka tygodni wcześniej. Wszyscy to akceptują, tak, nie mamy innego wyboru, jak tylko iść na smażenie. Solidarność jest ważna. W każdym razie złożyliśmy zobowiązanie i~musimy szanować naszych sojuszników, nawet jeśli, jak niektórzy podejrzewają, nie byli z~nami do końca szczerzy. Ale jakie są nasze szanse na przytłoczenie przejścia granicznego, w~każdym razie? Kto ma prawdziwe informacje? A jeśli nie jest to możliwe, czy nie nadszedł czas, abyśmy zaczęli pracować nad jakimś Planem B?

 Uciekłem, by zlokalizować Erica, który w~tym czasie był de facto grupą roboczą mediów w~Nowym Jorku (podobnie jak ja dla Ya Basta!, z~wyjątkiem tego, że miał pewne pojęcie o tym, co robi). Eric był na bieżąco z~rozwojem techników i~dał mi krótką informację na temat tego, co rozumie teraz przez oficjalny plan. Po imprezie smażenia ryb wszyscy pomaszerujemy na most. Będzie to pokojowy marsz, w~którym 50-100 Wojowników i~ich rodziny, w~tym dzieci, połączą się z~aktywistami. Wtedy miejmy nadzieję, że ich pokonamy. Wiele osób jest sceptycznych, czy to zadziała. Ale wydaje się, że to najlepsze, co możemy wymyślić.

 Podczas gdy sesja narzekania trwa, wpadam i~wypadam, próbując znaleźć ludzi (Twinkie przechodzi obok: ,,Co to jest?'' ,, To spotkanie ludzi, którzy chcą priorytetyzować wjazd do Quebec” Chcesz wpaść?'' ,,Nie!'' Przewraca oczami z~irytacją.) 

 Wreszcie odnajduję Maca, który wygląda na wstrząśniętego odkryciem ponad osiemdziesięciu osób na spotkaniu, o którym nawet nie wiedział, że się dzieje. 
 
 -- Um, jaka jest relacja tego spotkania do rady przedstawicieli, która wciąż trwa w~sąsiednim pokoju?

 Ludzie ignorują pytanie i~zadają własne pytania. Jeden dzieciak z~Czarnego Bloku z~Zachodniego Wybrzeża z~uszkodzonymi zębami pyta, co może się stać, jeśli ktoś zostanie zatrzymany:

\noindent  Zepsute zęby: Jeśli ktoś zostanie zatrzymany, próbując przekroczyć granicę, co na pewno byłbym, jeśli poddam się odprawie celnej, co najprawdopodobniej się stanie? Co zrobią Wojownicy?

\noindent  Mac: Radziłbym trzymać się z~tyłu. Jeśli przeciążymy odprawę celną, nie będziesz potrzebować dowodu tożsamości. W przeciwnym razie ludzie z~przodu zostaną zawróceni i~wszyscy zawrócimy solidarnie.

\noindent  Zepsute zęby: Ale co zrobiliby Wojownicy Mohawk? \textit{Wiem}, że będę z~tyłu. Nie musisz mi tego mówić. To oczywiste. Chcę wiedzieć, czy Mohawkowie powiedzieli nam, co \textit{zrobiliby}?

\noindent  Mac: Przejdą razem z~nami. Oczywiście nie zaatakują posterunku granicznego ani nic w~tym stylu, ale jako kolektyw musimy się wzajemnie chronić, a jeśli zawrócą ludzi, to jebać ich. Po prostu pójdziemy gdzie indziej.

\noindent  Ktoś: Nie rozumiem. Rada Plemienia poprosiła nas, abyśmy nie blokowali mostu, aby pas był otwarty. Jeśli zamierzamy ominąć odprawę celną, oczywiście skutecznie będziemy blokować most. Więc już przeciwstawiamy się ich woli. Dlaczego pójście trochę dalej miałoby być takie inne?

\noindent  Mac: Słuchaj, nie mam magicznej odpowiedzi, wiem tylko, że jako kolektyw jesteśmy silniejsi niż jako jednostki.

\noindent  Ktoś: Tak. A także cholernie dużo wolniejsi.

\noindent  Kitty: Moim osobistym odczuciem jest to, że jesteśmy tutaj, aby wymyślić alternatywny plan, co zrobić, jeśli zostaniemy zawróceni, ponieważ jeśli powiemy ,,wszystko albo nic'', wtedy bądźmy szczerzy: prawdopodobnie będziemy dość szybko zawróceni. Czy ktoś chce z~tym rozmawiać?

\noindent  Ktoś inny: Cóż, czy ktoś ma mapę?

\noindent  Mac: Pójdę po jedną.

 Wychodzę na chwilę z~Macem, kiedy to robi, tylko po to, żeby sprawdzić. Trudno mi sobie wyobrazić, jakim koszmarem musi być to wszystko dla niego. -- Problem -- mówi -- wszyscy chcą magicznych odpowiedzi. Nie ma żadnych magicznych odpowiedzi. Zresztą, jaki dokładnie \textit{jest }związek tego spotkania z~radą delegatów?

-- Myślę, że ludzie zdali sobie sprawę, że w~tempie, w~jakim rada działa, nie ma mowy, abyśmy mieli plan do 23:00, kiedy budynek zostanie zamknięty. -- mówię. -- Postanowili więc ustanowić siebie jako autonomiczną grupę roboczą ludzi, którzy naprawdę chcą się przedostać.

-- Och. Cóż, myślę, że nie ma powodu, dla którego \textit{nie mogliby }tego zrobić.

-- Anarchia w~akcji.

-- Aha.

 Niedługo wszyscy w~środku patrzą na mapy i~dyskutują o logistyce, ale ledwo zaczynamy, gdy ktoś wsuwa głowę i~mówi nam, że jest 23 i~mamy opuścić budynek. Ludzie gromadzą się na schodach. Gdy samochody jeżdżą, odtwarzając piosenki Ramones, Eric próbuje mnie porwać, aby dołączyć do kilku innych członków nowo utworzonej Media Working Group, aby przesłać faksem jakieś oświadczenie Russella. Mówię mu, że nie mogę, obiecałem spotkać się ze współlokatorami o 23:30. Zespół medialny wyrusza na poszukiwanie otwartej kawiarni. O 22:30 ludzie wciąż wychodzą z~budynku (właściwie nikt jeszcze nie przyszedł, żeby go zamknąć) i~wreszcie odnajduję moich ludzi -- Rufusa, Warcry, Chango, a teraz także Betty Tancerka -- którzy, jak się okazuje, siedzieli od jakiegoś czasu w~niedalekim parku, pod wiązem, dzieląc się papierosami z~goździkami, czekając na naszą przejażdżkę. Kitty i~spora grupa, przeważnie ubranych na czarno aktywistów, wyruszyli w~innym kierunku, by pracować nad naszym Planem B. Wyglądają dość oczywisto ze swoimi dwiema gigantycznymi czerwono-czarnymi flagami.

 Wreszcie przyjeżdża nasz samochód, w~którym są już dwie kobiety. Wszystkim jakoś udaje się wcisnąć.

 Większość z~nas jest po prostu wyczerpana. Kierowca, Sara, dwudziestokilkuletnia kobieta, gada o kwestiach higienicznych. Wdaje się w~długą diatrybę o aktywistach, którzy nie chcą się myć.

-- O tak, ,,Cruddies'' -- zgadza się Rufus.

-- Może jestem po prostu stary, ale myślę, że to nietowarzyskie. To brak szacunku dla innych.

-- Co to cruddies? -- Nie zauważyłem nikogo, kto wydzielałby zauważalny zapach na radzie.

-- Wiesz, wszystkie te dzieciaki z~brudnymi dredami i~brudnymi ubraniami, które są zadowolone z~zapachu własnego ciała? Oni wszyscy jakby jedli fasolę, puszczali wiatrów, śmierdzieli i~odmawiali umycia? 

-- Och. 

Ponieważ nie było sensu się spierać, zauważam, że kilka grup reprezentujących osoby kolorowe, z~którymi DAN pracowała w~Filadelfii, zawsze robiło problem z~tego rodzaju sprawami. ,,Śmierdzący biali anarchiści'' stali się rodzajem zaszyfrowanego słowa -- formą rasowego przywileju wymachującego im przed twarzami.

 Ale Sara nie interesuje się zbytnio aspektami rasowymi. 
 
 -- Nie zrozum mnie źle -- kontynuowała. -- Rozumiem apel. Kiedy miałam szesnaście lat, byłam dokładnie taka sama. Byłam zakochana we własnym zapachu. To było jak\ldots  cóż, naturalne. Tak właśnie pachną istoty ludzkie. To pewien rodzaj uczciwości, rozumiem to. Ale weź! Nadchodzi moment, w~którym musisz zacząć myśleć o innych ludziach. 
 
 Okazuje się, że po kilku latach życia jako dzika lokatorka Sara w~końcu dostała prawdziwą pracę w~mieście, w~jakiejś organizacji non-profit. Z pensją, świadczeniami, wszystkim. Wciąż próbowała przyzwyczaić się do nowego życia.

-- To chyba taka faza. Chodzi mi o to, czy są jacyś nieumyci aktywiści, którzy \textit{nie są }nastolatkami?

 Jej przyjaciółka Janna, pracownica katolicka z~Denver, jest jednak bardzo zaangażowana w~kwestię rasową. -- Wciąż zastanawiam się, czy powinnam być naprawdę zła z~powodu tej całej sprawy. Myślę, że naprawdę powinnam. Cały proces był całkowicie rasistowski.

-- Rasistowski w~jaki sposób?

-- Rasistowski, ponieważ pracowaliśmy tylko z~jedną małą grupą i~nawet nie próbowaliśmy skontaktować się z~nikim innym w~społeczności. Zawsze było, mówili tak ,,Mohawkowie'', ,,Mohawkowie'' tego chcą. Jakby wszyscy byli jak jedna osoba. Tak naprawdę przez cały czas rozmawiali tylko z~dwiema lub trzema osobami. Zwróć uwagę, jak robiliśmy to nawet w~radzie. 

 -- W porządku- powiedziałem -- z~pewnością masz rację co do języka, przyjmuję, ale\ldots  -- Zastanowiłem się. -- Cóż, czego \textit{chciałabyś}, aby zrobili organizatorzy?

-- Powinni byli porozmawiać ze wszystkimi w~społeczności.

-- I~gadać za plecami naszych sojuszników? Nie wiem. Naprawdę łatwo jest zacząć rzucać słowami takimi jak ,,rasizm'', kiedy ktoś coś spieprzy. Ale co, gdybyśmy mieli do czynienia ze społecznością, och, nie wiem, Francuzów? Albo Szwedów czy coś? Czy zachowalibyśmy się inaczej? Gdziekolwiek pójdziemy, zawsze będziemy rozmawiać z~najbardziej radykalnymi elementami społeczności (a właściwie w~tym przypadku to oni się z~nami skontaktowali). Gdybyśmy zaczęli dokonywać niezależnych podchodów do polityków Mohawk za ich plecami, ludzie powiedzieliby, że jesteśmy z~\textit{tego} powodu rasistami.

-- No cóż -- podsumowała -- może oskarżenie rasistowskie jest niesprawiedliwe. Ale nadal jestem zła.

-- Sam nie jestem szczęśliwy.

\noindent \textbf{Później tej nocy}

 W końcu wysadzili nas w~domu naszego gospodarza, starszej kwakierki, która zgłosiła się na ochotnika swój dom dla aktywistów. Był to przytulny, piętrowy dom z~dywanem, z~tarasem tak pełnym roślin doniczkowych, że przypominał trochę szklarnię, i~papugą latającą bez klatki. Około ośmiu lub dziewięciu osób ułożyło śpiwory na podłodze. Współczuliśmy z~powodu śmierci Joeya Ramone. Warcry uzyskał pozwolenie na korzystanie z~komputera w~gabinecie na piętrze; chwilę później poprosiła mnie, żebym podszedł i~przyjrzał się szkicowi opowiadania, nad którym pracowała, o Timothym McVeigh. W końcu znowu zszedłem na dół i~zakończyłem dość długą rozmowę z~naszym gospodarzem o Towarzystwie Przyjaciół. Niedawno zmarł jej mąż, ale miała dzieci i~wnuki w~Burlington i~okolicach. Pochodziła ze starej rodziny kwakrów i~przez całe życie była aktywna w~Kościele i~lokalnie w~aktywizmie. Czy to prawda, zapytałem, że spotkania kwakrów działają na zasadzie konsensusu? Ponieważ anarchiści też to robią, a słyszałem, że ostatecznie wiele z~tego, co robimy, było inspirowane przez Towarzystwo Przyjaciół. Zaczęła dość szczegółowy opis działania spotkań kwakrów, przerywana tylko od czasu do czasu moimi komentarzami (,,Wow, to jest \textit{takie }podobne''). Powiedziała, że  ludzie siedzą w~kręgu. Jeśli duch skłania ich do mówienia, pojawiają się propozycje i~każda osoba może teoretycznie zablokować propozycję, jeśli czuje się wystarczająco silnie w~tej sprawie. Bloki rzadko się zdarzają, ale w~zasadzie każdy ma prawo odrzucić każdą propozycję, a sam fakt, że każdy wie, że może, jest wystarczający, aby zapewnić odpowiedzialne działanie. Tak, powiedziałem. Dokładnie tak, jak my to robimy. Dawanie każdemu prawa do blokowania jest jak mówienie ludziom: ,,Ośmielamy się działać odpowiedzialnie'' i~ogólnie rzecz biorąc, chyba że masz do czynienia z~totalnym oszołomem, to wystarczy.

 Kontynuowała: 
 
 -- Na samym spotkaniu kwakrów zawsze jest facylitator, który nie powinien wyrażać własnej opinii, ale po prostu prowadzić spotkanie, słuchać i~powtarzać, jeśli coś wymaga wyjaśnienia. -- ( Aha. To tak jak my.) -- Uczestnicy mogą rozmawiać tylko z~facylitatorem. Nie ma rozmów pomiędzy.

-- Chwila, to znaczy, że nikomu w~ogóle nie wolno ze sobą rozmawiać?

-- Nie. Rozmowa dotyczy wydarzeń świeckich. Spotkanie to święte wydarzenie, więc możesz rozmawiać tylko z~moderatorem.

-- Och, to\ldots  naprawdę inne. Na naszych spotkaniach facylitatorzy trzymają listę, to znaczy obserwują, kto chce mówić i~liczą, czyja jest kolej, ale to nie tak, że możesz mówić tylko do nich. Zwykle mówisz do każdego, ale \textit{możesz }bezpośrednio odpowiedzieć na czyjeś stanowisko. Więc nie możesz tego zrobić na spotkaniu kwakrów?

-- Nie, możesz rozmawiać tylko z~facylitatorem.

-- Dlaczego?

-- Ponieważ inaczej byłoby to wydarzenie świeckie.

 Zastanowiłem się przez chwilę -- pomiędzy rytuałem Dziękczynienia, modlitwą Russella, kwakierskim pojęciem spotkań jako wydarzeń duchowych -- czy było jakieś znaczenie dla faktu, że ,,proces'', na którym anarchiści mają taką obsesję, jest zawsze, gdzie indziej, postrzegany jako uczestnictwo w~sacrum. Tworzenie zgody to tworzenie społeczeństwa. Społeczeństwo jest bogiem. A może bóg jest naszą zdolnością do tworzenia społeczeństwa. Konsensus jest więc rytuałem ofiary, ofiary egoizmu, gdzie akt powołuje do życia tego samego boga. Ale byłem zbyt zmęczony, a mój mózg zbyt rozmyty, żeby zrobić cokolwiek prócz zanotowania to w~pamięci. W każdym razie wiedziałam tylko, że jeśli ta akcja miała być podobna do innych, w~których byłem, spałbym co najwyżej dwie lub trzy godziny przez następne kilka nocy, a może nic. Wymamrotałem kilka uprzejmości i~położyłem się do łóżka.



\section{Czwartek, 19 kwietnia}
\noindent \textbf{ AKWESASNE}\medskip

 Następnego ranka wstaliśmy, zrobiliśmy cokolwiek w~IMC w~Burlington, kupiliśmy coś w~Quaker Meetinghouse, pełnym magazynów dla aktywistów i~ulotek, i~wyruszyliśmy karawaną. Jedziemy dopiero o 10:46.

 Jest pięć furgonetek Ya Basta! z~około dwudziestu, które składają się na karawanę, za którą podąża wynajęty autobus. Warcry przywiózł z~jakiegoś wydarzenia IMC ogromne, srebrne serpentyny, aby ozdobić pojazd. Mówi, że dopóki ma to być uroczyste wydarzenie, równie dobrze możemy wyglądać.

 Przyczepa kempingowa ma system łączności, krótkofalówki rozprowadzane co kilka pojazdów, a my mamy jedną, ale ludzie z~łączności spędzają większość czasu na monitorowaniu funkcjonariuszy stanowych, którzy eskortują nas poza miasto, i~pojawiają się okresowo z~kamerami, filmując nas w~różnych punktach po drodze. Poza tym nie ma zbyt wiele do przekazania, ale okresowe wiadomości, takie jak ,,wspaniały wodospad po lewej stronie'' lub ,,dobra muzyka na 105,7''. 

 W furgonetce przeglądam niekończący się stos dokumentów pobranych z~listserv i~stron internetowych, zanim wyszedłem. Jedna dotyczy różnicy między amerykańskim i~kanadyjskim systemem prawnym i~tego, jak może to wpłynąć na protestujących. Drugi to dokument o tym, jak radzić sobie ze skutkami gazu łzawiącego i~pieprzowego, a także dwa różne dokumenty dotyczące hipotermii. Jest dokument ze wskazówkami, jak nie robić z~siebie dupka w~rezerwacie Mohawk, i~inny, który ma dać działaczom pewne informacje na temat nacjonalistycznej wrażliwości w~Quebecu. W przeciwieństwie do Montrealu nie można zakładać, że przeciętny mężczyzna lub kobieta na ulicy w~Quebec City mówi po angielsku. Nie obrażą się, jeśli twój francuski jest słaby, znacznie lepiej jest podjąć próbę, niż po prostu zaczepiać ich po angielsku. Moja ulubiona ulotka to okólnik z~,,Québec Medical Fashion Brigade'' ze szczegółowymi radami co do ubrania:

\textit{ Dzisiejszy dobrze ubrany bojownik w~Quebec City na szczycie ma na sobie długą bieliznę wykonaną z~nowych materiałów syntetycznych, takich jak miękki, ciepły poliester, który ODDYMIA pot ze skóry. Znaczna część obwodu Szczytu znajduje się na wzgórzu, a wspinaczka ulicami, aby do niego dotrzeć, lub bieganie tu i~tam, sprawi, że będziesz się pocić. A pot przy skórze może sprawić, że będzie ci zimno\ldots }

\textit{ Powinnaś mieć wiele luźnych warstw, które można zdjąć, gdy zrobi się gorąco, i~założyć z~powrotem, gdy jest zimno\ldots  Twoja zewnętrzna warstwa powinna być wodoodporna. BARDZO polecamy tani kombinezon przeciwdeszczowy -- nie tylko zapewni ci to ochronę przed deszczem lub śniegiem, ale także zapobiegnie wchłonięciu przez ubrania tych nieprzyjemnych zanieczyszczeń, takich jak gaz łzawiący i~gaz pieprzowy. Jako bonus zablokuje również wiatr. Jeśli nosisz polar, upewnij się, że znajduje się pod ubraniem przeciwdeszczowym, jeśli znajdujesz się w~strefie zagrożenia bronią chemiczną (w pobliżu policji). Gaz pieprzowy i~gaz łzawiący są wchłaniane przez włókninę, a następnie uwalniane z~czasem na twarz. Fuj! Aby uzyskać ten wyjątkowo seksowny wygląd, wypróbuj te tanie, przezroczyste poncza złożone w~małej plastikowej torbie -- będą one wyglądać jak prezerwatywa, a otrzymasz dodatkowy punkty szacunku za informację o bezpiecznym seksie!}

\textit{ Rozumiemy zastrzeżenia, jakie możesz mieć, jeśli nie możesz uzyskać sprzętu przeciwdeszczowego w~podstawowej czerni. Jednak twój plastikowy kombinezon przeciwdeszczowy jest idealnym środkiem do malowania natryskowego (czarny, tak?), magicznych markerów i~wszystkich twoich naklejek. Czarne worki na śmieci mogą również działać przeciwko wodzie i~chemikaliom\ldots }

 Pojawiają się sugestie dotyczące masek przeciwgazowych, gogli, używania bandan nasączonych octem jako ochrony przed gazem łzawiącym. Sprawdzam, czy nie mam dodatkowych skarpet, warstw itp. Przyczepa jedzie prawie niewyobrażalnie wolno, jakieś 70 kilometrów na godzinę na dwupasmowej autostradzie i~nikt nie jest do końca pewien przyczyn.

-- Nie sądzę, żebyśmy mogli przynajmniej założyć nasze stroje Ya Basta! do smażenia ryb? -- ktoś pyta. -- To prawdopodobnie będzie nasza ostatnia szansa, ponieważ nie ma mowy, abyśmy przeszli w~nich przez odprawę celną.

-- Nie sądzę. Mac mówił, że jeśli nawet pojawimy się, wyglądając, jakbyśmy byli przygotowani do działania, może to zostać odebrane jako agresja.

-- Cóż\ldots  może rzeczywiście pokonamy ich na moście.

 Wydaje się, że nikt nie uważa, że  jest to szczególnie prawdopodobne.

 Mijają godziny. Przenosimy się na małe wiejskie drogi, mijając opuszczone farmy i~sklepy z~bronią, jadąc jeszcze wolniej. Ktoś wyjaśnia, że  jego aktywizm sięga czasów, gdy z~dzieciństwa zdał sobie sprawę, że Power Rangers byli naprawdę źli. Od czasu do czasu policjanci filmują nas z~pobocza autostrady, niektórzy w~mundurach, inni po cywilnemu. Kiedy zatrzymujemy się na postój nad rzeką, większość z~nas wychodzi w~maskach, a niektórzy mężczyźni dzielnie tworzą ludzką ścianę, aby dać kobietom trochę prywatności przed glinami po drugiej stronie drogi, którzy próbują je sfilmować, kiedy sikają. Przynajmniej nie ma blokad drogowych. Wreszcie, po pozornej wieczności, może około 16, radio trzeszczy: ,,mamy obraz na Akwesasne'' 

\noindent  \textbf{Samo Akwesasne}

 Jak się okazuje, nie było nikogo, kto by nas przywitał przy głównym wejściu do rezerwatu, choć może to, jak sądzę, mieć coś wspólnego z~faktem, że jesteśmy już jakieś trzy godziny spóźnione na imprezę, która miała się zacząć o ma 13:00. W każdym razie scena jest chaotyczna. Wszystko w~Akwesasne wydaje się przypadkowe. Karawana przejeżdża rzez rezerwat, od czasu do czasu obserwowana przez ciekawskich, ale prawie nikogo nie ma nawet na gankach.

 W końcu wjeżdżamy na bardzo dużą trawiastą przestrzeń ze stolikami rozstawionymi dookoła smażalni ryb. Nie ma dzieci. Właściwie prawie nikogo nie ma. Zaledwie kilkudziesięciu aktywistów, którzy czekali od południa, kilku dziennikarzy i~coś, co wygląda na czterech prawdziwych Mohawków. Później ta liczba wzrasta do sześciu. Jedzenie, podawane na papierowych talerzach, rozdają ci, co wydają się szkieletową załogą; wszystko jest minimalne; jest tam rodzina Bootsa (Stacey rzeczywiście ma włosy obcięte na irokeza, co jest jakoś dziwnie satysfakcjonujące). Jest kilku innych Wojowników, którzy pojawiają się od czasu do czasu, aby z~nimi porozmawiać, najwyraźniej sprawdzając pozycje policji za wzgórzem, ale to wszystko. To oczywiste, że zostaliśmy całkowicie wymanewrowani. Społeczność jest nieobecna. Nawet lokalizacja okazuje się pustą działką, czyli, jak dowiadujemy się później od dziennikarzy, technicznie tuż za i~właściwie nie w~samym Rezerwacie.

 Kilku aktywistów wędruje po okolicy, próbując znaleźć miejsce, w~którym można by wysikać; nie ma tam nocników ani oczywistych wychodków i~nikt nie jest pewien, czy uznano by to za zbezczeszczenie indiańskiej ziemi. W końcu ktoś mówi im, że szefowie powiedzieli, że można iść w~zarośla, o ile odciążamy się w~kierunku przeciwnym do rezerwatu. Łapię trochę ryb z~Warcry, aż ktoś zadzwoni do nas, aby porozmawiać z~zespołem informacyjnym z~PBS \textit{Frontline}. Warcry zmienia się natychmiast, bez wysiłku, od zrzędliwego do namiętnego, przyspieszając krótkie przemówienie na temat związku między rdzennym uciskiem a FTAA. Stoję z~boku, lekko zdezorientowany, i~robię małe oświadczenie o solidarności, a potem znowu odchodzę.

 Trochę dalej znajduję Twinkie siedzącą samotnie na drewnianej ławce i~płaczącą.

 Siadam obok niej. 
 
 -- Co się dzieje, Twinkie? To znaczy, przyznaję, że sytuacja jest trochę przygnębiająca\ldots 

-- Nie, nie o to chodzi -- powiedziała, próbując się uśmiechnąć, mimo że łzy nie ustawały. -- To ryba. -- Na jej talerzu są jeszcze resztki szczupaka.

-- Ryba?

-- Jestem wegetarianką. Moja rodzina jest z~Tajlandii. Wychowywali nas bardzo surowi buddyści.

-- Więc dlaczego to zjadłaś? Jestem całkiem pewny, że mieli opcję wegańską. Kaszka kukurydziana, nie? -- Twinkie była, chociaż wahałem się, czy to podkreślić, kimś w~rodzaju eksperta w~wydobywaniu ryb i~mięsa kraba z~rolek sushi wyciągniętych ze śmietników restauracji.

-- Cóż, pomyślałam, że to będzie gest solidarności. W końcu jesteśmy na ich ziemi. i~zrobili je specjalnie dla nas. A kiedy faktycznie je jadłam, było w~porządku. Ale potem po prostu zaczęłam płakać.

 Rozważałem dokonanie jakiejś filozoficznej obserwacji na temat tego, jak wszyscy ostatnio czuli się złapani w~podwójne więzy, ale postanowiłem tego nie robić. Zamiast tego powiedziałem: 
 
 -- Naprawdę? Nie wiedziałem, że twoi rodzice pochodzą z~Tajlandii. Myślałem, że słyszałem, jak ktoś powiedział, że jesteś z~Filipin!

-- Co? Nie! Jesteśmy z~Tajlandii.

-- Urodziłaś się tam?

-- Byłam dość młoda, kiedy przyjechała moja rodzina.

 Odbywa się krótka ceremonia, która zaczyna się od przemówienia Stacey Boots z~dachu furgonetki. Opowiada o historii rdzennych Amerykanów witających i~chroniących cudzoziemców, którzy przybyli z~pokojowymi intencjami. 
 
 -- A teraz, jak sądzę, będziemy was chronić. 
 
 Działacz latynoski z~Nowego Jorku wstaje i~wygłasza przemówienie o tym, że FTAA jest tylko ostatnim przejawem pięćsetletniej kampanii podboju i~ludobójstwa, która rozpoczęła się od Krzysztofa Kolumba. Piosenkarz folklorystyczny wspina się na dach furgonetki z~gitarą i~gra coś, co nazywa się ,,Indian Wars'' Po jednym lub dwóch spontanicznych występach aktywistów, karawana zbiera się z~powrotem i~kierujemy się w~górę rampy w~kierunku ,,placu poboru opłat'', gdzie najwyraźniej zamierzamy rzeczywiście spróbować przekroczyć granicę.

\noindent \textit{Niepełna akcja na granicy}

 Nasza furgonetka stoi przed przyczepą, może pięć samochodów z~przodu: ja, Warcry, Betty, Rufus, Sasha dokumentalista i~jego dziewczyna Karen, która pomaga mu w~jego projekcie wideo, ponieważ Sasha jest w~tym momencie w~zasadzie aktywistą. Z drugiej strony Karen jest profesjonalistą medialnym, uzbrojonym w~drogi sprzęt i~będzie dokumentowała wszystko, co robi.

 Długa linia wolno posuwających się vanów ciągnęła się do stacji granicznej, która była białą, jednopiętrową konstrukcją, gdzie na zewnątrz, za licznymi barykadami, zebrała się co najmniej setka funkcjonariuszy policji. Tyle o pomyśle udania się bezpośrednio na środek mostu i~przytłoczenia ich po drugiej stronie. Wszyscy mieli być sprawdzeni po tej stronie. Obiecaliśmy, że pozostawimy otwarty pas, aby pojazdy ratunkowe mogły się przez nie przejechać, ale natychmiast wypełnił się pieszymi. Początkowo, gdy aktywiści zaczęli maszerować po podjeździe, impreza miała charakter marszu świątecznego: ktoś był na szczudłach, były bębny, porozrzucane instrumenty muzyczne, kilka prób porywających śpiewów. Ale było zupełnie niejasne, czy to rzeczywiście była akcja. Wtedy wszystko się zatrzymało. Warcry wychodzi na zwiad i~nigdy nie wraca. Wystawiam głowę, żeby zobaczyć, czy uda mi się złapać jakieś ślady naszej rzekomej eskorty Wojowników, i~poza jednym lub dwoma, którzy byli na podium, nikogo nie widziałem. Na pewno żadnych rodzin.

 Prawie nic, co nam obiecano, tak naprawdę się nie zmaterializowało.

 Betty udaje się na przerwę na papierosa, po czym wraca. W końcu długo rozmawiamy na temat płci, katastrofalnego szkolenia Ya Basta! i~wynikające z~tego rozpadu. 
 
 -- To nie tak, że naprawdę \textit{chcę }odciąć kilku małym chłopcom penisy -- mówi. -- To znaczy, rozumiem, że mali chłopcy potrzebują swoich penisów. Naprawdę nie mam nic przeciwko, jeśli czują potrzebę pomachania nimi. Wszystko, czego chciałam, to powiedzieć ostro. Ale jak tylko podniosę ten problem, wszyscy zaczynają na siebie krzyczeć, a teraz nawet ze sobą nie rozmawiają i~czuję, że to wszystko moja wina.

-- Naprawdę krzyczeli na siebie?

-- Dobra, może nie dosłownie krzyczeli\ldots 

 Po chwili. 
 
 -- Więc co o tym myślisz? Czy powinnam spróbować dowiedzieć się, co się stało z~Warcry?

-- Jasne -- mówi Betty. -- Prawdopodobnie będziemy tu siedzieć jeszcze przez godzinę w~taki czy inny sposób. Idź, rozprostuj nogi. Złap trochę powietrza. To będzie dla ciebie dobre.

 Karen zgłasza się na ochotnika, aby pojechać ze mną, aby zobaczyć, czy może uzyskać przydatne nagranie. Sasha całuje ją i~przejmuje prowadzenie.

 Wysiadam i~idę w~kierunku punktu poboru opłat. Kiedy przechodzę, Moose robi sobie przerwę na papierosa kilka furgonetek bliżej, wygląda na zakłopotanego, starając się unikać kontaktu wzrokowego. Teraz nigdzie nie widać Mohawków. Po kanadyjskiej stronie granicy nie ma też śladu robotników pocztowych, hutników, a właściwie nikogo -- choć wydaje się, że za płotem z~siatki, na czymś, co wygląda jak ogromne boisko do koszykówki, za plecami wydaje być tłum nastolatków -- a przed nimi policjanci z~Mohawk. Przed stacją graniczną gromadzi się gęsty tłum aktywistów; niektórzy są wściekli, niektórzy mają nadzieję, że uda im się wynegocjować przejście. Są flagi i~sztandary. Jedna kobieta w~czerni wspięła się do połowy słupa drogowego, bębniąc. Od czasu do czasu ktoś próbuje rozpocząć kolektywną pieśń. Chłopak z~zepsutymi zębami próbuje skakać w~górę i~w dół, zaczynając skandować ,,Dni gniewu! Dni gniewu!'', a kilka osób podejmuje to, ale tak naprawdę to się nie przyjmuje i~wraca do narzekania i~mamrotania.

 Wreszcie widzę przyczynę opóźnienia. Pierwsza furgonetka zatrzymuje się na przejściu granicznym; Kanadyjska policja wyjęła każdą pojedynczą torbę, która się w~niej znajdowała i~ułożyła je wszystkie na asfalcie, i~wydaje się, że jest zdeterminowana, aby przejrzeć każdy przedmiot w~każdej z~nich. Enos, kierowca, był jednym z~pierwszych, którzy poddali się odprawie celnej, prawdopodobnie nie był to dobry pomysł, ponieważ odmówiono mu już wjazdu do Kanady podczas akcji z~1 kwietnia dwa tygodnie wcześniej. Po kilku pytaniach jego nazwisko jest wprowadzane do komputera i~zostaje poproszony o wejście do budynku przypominającego szopę. Minutę lub dwie później widzimy go zaprowadzonego do policyjnej furgonetki, w~plastikowych kajdankach, ze znużonym, zirytowanym wyrazem twarzy, rodzajem wizualnego westchnienia.

 Warcry stoi z~Target i~małą grupą dziennikarzy IMC. 
 
 -- Widziałeś, jak zabrali Enosa?

-- Ta.

-- Więc dlaczego po prostu tu stoimy? Myślałem, że to wszystko albo nic? Wynośmy się stąd do diabła i~chodźmy gdzieś, gdzie naprawdę możemy przejść.

 Po krótkiej rozmowie stwierdzamy, że musimy zrobić coś bardziej dramatycznego. Zbierzemy cztery lub pięć osób, które na pewno nie przejdą, ale prawdopodobnie też nie zostaną aresztowane. Warcry i~Target są prawdopodobnie znane każdemu agentowi FBI w~USA, ale nie mają zaległych nakazów; Nie mam przy sobie paszportu ani oficjalnego dowodu osobistego; Przy poprzednich okazjach Madhava i~Jenka nie mieli wstępu. Postanawiamy, że podejdziemy i~formalnie poddamy się razem. Następnie, gdy zostaniemy zawróceni, możemy spróbować zawrócić linię i~udać się na kolejne przejście.

 Karen proponuje udokumentowanie wydarzenia na wideo.

 Nasza piątka łączy ramiona i~idzie naprzód. Karen, wyglądająca w~każdym calu na profesjonalną kamerzystę w~schludnej beżowej kurtce, z~blond włosami związanymi do tyłu, filmuje nas z~góry, gdy posuwamy się naprzód.

 Kiedy podchodzę, wypróbowuję kwestię, która po raz pierwszy przyszło mi do głowy w~furgonetce -- bardziej niż cokolwiek innego, ponieważ jestem ciekaw, jaka będzie reakcja. 
 
 -- Mam nadzieję, że pamiętasz -- mówię do pierwszego policjanta, patrząc mu w~oczy -- że robimy to tylko po to, by uratować twoje plany opieki zdrowotnej.

-- Nie jesteśmy tego nieświadomi -- mówi. -- To jeden z~powodów, dla których postanowiliśmy potraktować was tak łagodnie.

 Och. Nieco zdziwiony, czekam, aż wyślą nas każdego, abyśmy porozmawiali z~innym oficerem. Warcry idzie pierwsza. Potem ja. Nagle udzielam wywiadu muskularnemu kanadyjskiemu funkcjonariuszowi granicznemu w~płaskiej białej czapce.

-- Cel twojej podróży do Kanady?

-- Zamierzam zaprotestować przeciwko konferencji FTAA w~Quebec City.

-- Więc kiedy ostatnio byłeś przed sędzią?

-- Sędzia? Cóż, w~zeszłym roku byłem przez kilka tygodni na ławie przysięgłych.

 Lekka niecierpliwość. 
 
 -- Wiesz, co mam na myśli.

-- Jako oskarżony? Nigdy. Nigdy nie stanąłem przed sędzią oskarżony o przestępstwo.

 Pokiwał głową. 
 
 -- W porządku. -- Potem wskazał mi młodszego oficera, pryszczatego dzieciaka, który wyglądał, jakby właśnie skończył liceum. Dzieciak poprosił mnie o paszport.

-- Przepraszam, nie przyniosłem.

-- Prawo jazdy?

-- Nie. Jestem z~Nowego Jorku. Nie jeżdżę.

 Pauza.

-- Wielu ludzi z~Nowego Jorku nie umie prowadzić.

 Karen w~jakiś sposób przeszła na drugą stronę posterunku i~teraz filmuje wszystko z~Kanady.

 Dzieciak jest wyraźnie zdenerwowany. Wychodzi na naradę z~umięśnionym facetem, który wydaje się jego dowódcą, po czym wraca.
 
 -- Dobra, no to co \textit{masz}?

-- Mam identyfikator uniwersytecki -- powiedziałem, wyciągając go z~portfela. W rzeczywistości był to identyfikator wydziału Yale, ale nigdzie na karcie nie było napisane ,,wydział'' Pewnie założył, jak większość ludzi, że jestem studentem. On też nie wydawał się przejmować. Miał zdjęcie, a zdjęcie wyglądało jak ja.

-- Dobrze -- powiedział.

-- Co masz na myśli?

-- Mam na myśli w~porządku.

 Byłem zaskoczony.
 
  -- Masz na myśli\ldots ?

 Gest. 
 
 -- Po prostu podejdź do tamtego znaku stopu.

-- Czekaj\ldots  To znaczy, że przeszedłem?

 To był wynik, którego nigdy się nie spodziewałem. Nawet nie zastanawiałem się, co by się stało, gdybym rzeczywiście się przedostał. Pryszczaty oficer zaczyna wyglądać na zniecierpliwionego. Karen była już za granicą w~Kanadzie; za mną czekała niekończąca się kolejka ludzi, a z przodu równie niekończący się most.

-- Cóż, spójrz -- powiedziałem. -- Mam umowę z~przyjaciółmi. Obiecałem, że bez nich nie pójdę. Czy nie mogę po prostu na nich poczekać?

-- Nie możesz na nich czekać tutaj -- powiedział. -- To jest obszar przetwarzania. Będziesz musiał na nich poczekać tam, w~Kanadzie.

 Ponieważ ,,Kanada'' w~tym momencie wydaje się składać z~krótkiego kawałka asfaltu oddalonego o dwa lub trzy metry, nie wydaje się to z~natury nierozsądne. Karen idzie ze mną, robiąc kilka ujęć szerokokątnych karawany, która wciąż stoi nieruchomo po amerykańskiej stronie. Próbuję dowiedzieć się, co stało się z~Target i~Warcry, ale nigdzie ich nie widać. Myślę, że mogli zostać zabrani do środka na przesłuchanie. Jednak po minucie lub dwóch pojawia się inny funkcjonariusz straży granicznej i~mówi: 
 
 -- Nie możesz czekać na posterunku granicznym. Jeśli zamierzasz na kogoś czekać, będziesz musiał przejść tam do tego znaku stopu -- wskazując ponownie na znak na skraju drogi.

 Robimy tak. Znak stopu wydaje się jednak być bardzo dużym znakiem stopu, ponieważ kończy się na tym, że znajduje się dwa lub trzy razy dalej, niż się wydawało od stacji granicznej. Teraz jesteśmy na kawałku asfaltu z~dala od wszystkiego i~mimo tego, że zmrużyłem oczy, nie mogę dostrzec, co dzieje się po amerykańskiej stronie. Za ogrodzeniem na boisku do koszykówki wciąż jest tłum dzieciaków i~machamy im, ale jesteśmy zbyt daleko, żeby naprawdę zobaczyć, czy odmachują.

 W tym momencie pojawia się ogromny gliniarz z~Mohawk z~paralizatorem, prowadząc samochód. Można powiedzieć, że był gliną z~Mohawk, ponieważ na jego ramieniu jest naszywka z~napisem ,,Policja Akwesasne Mohawk''. W przeciwieństwie do policji na posterunku granicznym, która wahała się od rzeczowej do prawie przyjaznej, ten wygląda wyjątkowo nieprzyjemnie. Informuje nas, że jesteśmy na terenie rezerwatu i~będziemy musieli z~niego zejść i~przejść przez most. Karen kieruje na niego kamerę -- zwykle jest to dość skuteczny sposób na wywołanie lepszego zachowania policji. Nie jest pod wrażeniem.

-- Przepraszam, oficerze, widzi pan, że jesteśmy tu tylko dlatego, że ludzie z~granicy powiedzieli nam, że musimy\ldots 

-- Jesteście na ziemi indiańskiej! Nie jesteście mile widziany przez społeczność. Idźcie stąd. Natychmiast zaczniecie przechodzić przez most.

 Obecność paralizatora wydała mi się bardzo przekonującym argumentem. W każdym razie w~połowie rampy dojazdowej był kolejny mały słupek i~mogłem rozpoznać kilku aktywistów -- mogli być tylko aktywistami, sądząc po ich ubiorze -- kręcących się w~takim samym zamieszaniu jak my.

-- Cóż, chcesz przyjść? -- zapytałem Karen.

-- Wygląda na to, że nie mam dużego wyboru.

-- Cóż, przypuszczalnie by \textit{cię }wypuścili. A Sasha wciąż jest z~karawaną.

-- Racja. Ale Sasha chciałaby, żebym przynajmniej spróbował nagrać jakiś materiał z~Quebec City. Właściwie mam w~torbie kilka godzin pustych taśm i~baterii. i~to może być nasza jedyna szansa, żeby jedno z~nas tam dotarło.

 Zaczynamy wspinać się po moście i~odkrywamy, że aktywistami, których tam widzieliśmy, byli w~rzeczywistości Kitty z~Connecticut wraz z~kilkoma jej koleżankami Yabbasami -- szczupłą, nieco zniewieściałą Azjatką o imieniu Lee, kobietą o imieniu Andrea -- wszyscy wyglądali na zdziwionych, gdy przeszliśmy.

 Zaczęliśmy iść w~kierunku mostu. Po pewnym zamieszaniu, kiedy jedna grupa policji powiedziała nam, że nie możemy wejść na most, a inna powiedziała nam, że nie możemy wrócić (i negocjując dostęp do łazienki na małej stacji u podstawy mostu), podsumowaliśmy.

 Nie trzeba dodawać, że cały sprzęt Yabby był w~furgonetce. Nie wziąłem nawet torby na ramię z~notesami i~innymi niezbędnymi rzeczami -- tak przekonany, że  nie uda mi się przedostać. Miałem jeden kieszonkowy notes, który wtedy miałem w~kieszeni; telefon komórkowy z~energią może kilka godzin (bez ładowarki). Poza tym miałem na sobie w~zasadzie to, co ubrałem: czarną bluzę z~kapturem, która rzekomo miała podszewkę z~,,arktycznym wypełnieniem'', czarną bandanę w~kieszeni, spodnie wojskowe na bieliźnie termicznej, trzy różne kaszmirowe swetry nałożone na dość ładną czerwoną koszulę wizytową (uważam, że warto mieć coś reprezentacyjnego do przechodzenia przez linie policyjne). To było to.

 Moi przyjaciele byli w~tej samej sytuacji. Nikt nie miał żadnego sprzętu ani bagażu. Z wyjątkiem Andrei, która miała śpiwór.

-- Tyle dla Ya Basty! -- mówi Lee.

-- Tak, wygląda na to, że wszyscy będziemy robić Czarny Blok -- zgadza się Kitty.

-- Więc \textit{idziemy}?

 Kitty spogląda z~powrotem na plac poboru opłat. 
 
 -- Cóż, jeśli wrócimy, co to będzie oznaczać? Karawana porusza się tak wolno, że nie będziemy w~stanie nawet spróbować ponownie przejechać, aż do jutra.

-- Istnieje jednak ,,Plan B'', prawda? Mam na myśli, że wymyśliliście coś na radzie?

-- Cóż, tak, właśnie o to chodzi. Wymyśliliśmy. Pomyśleliśmy, że ważne jest, aby było bezpiecznie, więc tylko dwie osoby mają mapy i~znają wszystkie szczegóły. Problem w~tym, że jedną z~nich jestem ja.

-- Najlepiej ułożone plany wspaniałych ludzi -- uśmiecha się Lee.

-- Cóż, jaka jest szansa, że  oboje przeszliście? -- sugeruję.

 Karen poszła nakręcić panoramiczny materiał z~boku rampy.

-- W każdym razie to nie był tak niesamowity plan. Prawdopodobnie każdy, kto miałby dobrą mapę, mógł to wymyślić.

 Postanawiamy przynajmniej spróbować zameldować się w~naszych grupach afinicji, ale tylko ja mam działający telefon. Co jest dość cudowne. Komórka wszystkich pozostałych zepsuła się na kilka godzin przed naszym przybyciem do Akwesasne -- nikt nie był do końca pewien, czy to z~powodu ingerencji policji, czy też dlatego, że byliśmy po prostu zbyt daleko w~tej krainie. Poświęcam kilka minut na telefonowanie do Betty, Rufusa i~różnych osób z~zespołu prawnego. W każdym przypadku jestem wysyłany natychmiast na pocztę głosową. To samo dzieje się, gdy Kitty używa mojego telefonu, aby spróbować skontaktować się z~innymi członkami jej własnej grupy afinicji.

 Na koniec mówi: 
 
 -- Wydaje mi się, że to dość oczywiste, co zamierzamy zrobić. Możemy stać tutaj i~męczyć się nad tym przez kolejną godzinę, a potem odejść, albo możemy już iść. Co powiecie?

-- Jestem za.

 Wszyscy przytaknęli.

 I~tak zaczęliśmy iść przez most.

 Seaway International Bridge okazał się mieć prawie trzy kilometry długości i~składał się z~dwóch różnych konstrukcji, połączonych małą wyspą pośrodku. Droga była prawie pusta. Bardzo rzadko przejeżdżał jakiś pojazd, zwykle pickup. Od czasu do czasu przejeżdżały też policyjne quady z~Mohawk, najwyraźniej tylko po to, żeby nas niepokoić. Spędziliśmy dużo czasu, wpatrując się w~Wodospad Św. Wawrzyńca. Widok był niezwykle piękny. Były tam zatoczki, wysepki, maleńkie łódeczki, tu i~ówdzie wzdłuż brzegu chaty przypominające chatki. Większość dawała poczucie nieskazitelnego naturalnego piękna, kontury wybrzeża prawie nie zmieniły się od pierwszego przybycia ludzkich osadników dziesięć tysięcy lat temu. Intelektualnie wiedziałem, że to nie jest prawda: w~rzeczywistości jednym z~tematów, które organizatorzy Mohawk chcieli, abyśmy podkreślili, był rasizm środowiskowy. Na granicy rezerwatu po stronie amerykańskiej zbudowano fabrykę GM (właściwie myślałam, że uda mi się ją rozróżnić we mgle), a miejsce to było tak konsekwentnie wykorzystywane do wyrzucania toksyn, że matki w~Akwesasne były powiedziano, aby nie karmić piersią, a dzieci czasami rodziły się z~jelitami na zewnątrz. Ale z~mostu wszystko to było prawie niemożliwe do wyobrażenia. Po prostu wyglądał imponująco, pięknie.

 Słońce już zachodziło, kiedy dotarliśmy do Cornwall. 

 \noindent \textbf{Cornwall, Ontario}

 Nadal nie jestem pewien, jak wygląda Cornwall; nigdy jej nie zobaczyłem. To, co zobaczyłem, było rodzajem luźnego centrum handlowego na końcu mostu, małej, niskiej otwartej przestrzeni ze sklepami handlowymi umieszczonymi na wzniesieniach po obu stronach. U samego podnóża mostu minęliśmy dwie linie oddziałów prewencji, w~sumie może czterdziestu, uzbrojonych w~szykach i~po prostu tam stojących, naprzeciw małego tłumu około stu lub dwustu kanadyjskich aktywistów, którzy najwyraźniej byli pozostałością po znacznie większym tłumie. Niektórzy byli zamaskowani. Większość wyglądała na zmęczoną. Obie strony wydawały się nieco śmieszne, przyćmione ogromem mostu. Nigdy nie widzieliśmy obiecanej herbaty, ale minęliśmy jeden transparent Koalicji przeciwko Ubóstwu Ontario, witający nas. Karen sumiennie to sfilmowała. Wszędzie byli ludzie z~aparatami, ale bardzo niewiele wydawało się naszymi kamerami. Minęliśmy Shawna Branta, stojąc na tylnym siedzeniu pikapa i~składając jakąś wyzywającą deklarację dla kanadyjskiej telewizji. Wyglądał dokładnie tak, jak na zdjęciach.

 Wśród Kanadyjczyków rozsiani byli inni Amerykanie, którzy tak jak my próbowali pogodzić się z~faktem, że się przebili. Stopniowo ludzie zaczęli się gromadzić, znajdować miejsca na pasie wilgotnej trawy w~pobliżu autostrady, aby uformować miniradę, aby ocenić naszą sytuację.

 Nasza grupa rozdzieliła się tymczasowo, żeby zdobyć jedzenie. Kupiłem tanią kanapkę z~kurczakiem w~barze na wynos na wzgórzu, zbadałem Walmart. Nigdy wcześniej nie byłem w~Walmarcie. Był ogromny. Wziąłem małą butelkę Tylenolu z~kodeiną, którą, jak pamiętałem, można dostać bez recepty w~Kanadzie, myśląc, że może się później przydać. Wracając, odkryłem, że spotkanie odbywa się w~pełnej sesji, na której około czterdziestu amerykańskich aktywistów siedziało w~kręgu, próbując sporządzić listę nazwisk do przekazania działowi prawnemu, aby grupy afinicji ludzi w~Akwesasne wiedziały, że nic im nie jest i~móc iść dalej bez nich. Musieliśmy być zamaskowani na spotkaniu, ponieważ wkrótce otoczyły nas kamery telewizyjne. Kiedy jeden niezwykle zadufany w~sobie dziennikarz wideo CBC odmówił zaprzestania filmowania ludzkich twarzy, niektórzy z~nas są w~końcu zmuszeni mocno sugerować, że możemy być skłonni pomalować jego soczewkę sprayem. Karen filmuje konfrontację, a następnie używa swojej kamery, aby udokumentować każdy jego ruch, gdy w~końcu zaczyna pakować swój sprzęt. Wyjaśnia, że  nic bardziej nie irytuje dziennikarzy telewizyjnych niż \textit{ich }filmowanie.

 W końcu zostałem przydzielony do dzwonienia według listy na moim umierającym telefonie komórkowym. Zostawiam nagranie na maszynie kancelarii prawnej w~Nowym Jorku i~w kilku innych miejscach i~mam nadzieję, że jakoś dotrze do ludzi.

 Do tego czasu jest już ciemno. Centrum handlowe jest prawie puste. Nawet gliniarze zniknęli. Kiedy przyjechaliśmy, było kilka furgonetek, większość już odjeżdżała podczas rady, w~tym jeden lub dwóch pełnych aktywistów Mohawk, prawie wszyscy, którzy faktycznie byli przy smażalni ryb. Mac i~Lesley pojawiają się i~znikają. Obawiam się, że przez jakiś czas nie znajdziemy podwózki, ale znajdujemy wraz z~niektórymi protestującymi ze Szkoły Ameryk, jedną z~niewielu grup w~Akwesasne, której pojazd rzeczywiście przejechał. O 22.00 jesteśmy w~drodze do Quebec City.

\chapter{Szczyt Ameryki w~Quebec City}

Od tego momentu wrócę do trybu dziennika. To, co następuje, opiera się w~dużej mierze na notatkach szybko zapisanych w~tym czasie, uzupełnionych z~pamięci, a następnie porównanych z~notatkami innych uczestników oraz opublikowanych (zwykle publikowanych w~Internecie) relacji z~pierwszej ręki.

\section{Piątek, 20 kwietnia}

\noindent \texttt{ 2:30 rano}\medskip

 Zawsze z~uporem nie mogłem spać w~poruszających się pojazdach. Kitty i~załoga z~Connecticut szybko odpadają na tylnym siedzeniu furgonetki. Karen i~ja, cierpiąc na bezsenność, kończymy długą rozmowę z~Janną, katolicką robotnicą z~Denver, która jest tam z~kontyngentem SOA. Janna jest właściwie poganką, ale dla radykałów w~tej części kraju, jak wyjaśnia, nie ma zbyt wielu wyborów. 
 
 -- Dołączyłabym do Pogańskiego Robotnika, gdyby istniało coś takiego. 
 
Została zagazowana w~Seattle, a potem przez sześć miesięcy przebywała w~szpitalach. Wyjaśniła, że  trzeciego dnia protestów sprowadzono Gwardię Narodową, która zaczęła używać CS, gazu łzawiącego tak potężnego, że tylko wojsko może go używać (kiedy serbska armia użyła go przeciwko rebeliantom w~Kosowie, rząd USA nazwał to zbrodnią wojenną). Jedna ciężarna kobieta straciła dziecko; inny działacz zmarł z~powodu komplikacji kilka miesięcy później. Lekarze Janny powiedzieli jej, że jej płuca zostały poważnie uszkodzone i~że powinna za wszelką cenę unikać w~przyszłości kontaktu z~takimi toksynami.

-- Co, przyznaję, trochę mnie podkręciło, jadąc do Quebec City. Ale niektóre rzeczy są po prostu zbyt ważne.

\bigskip
\noindent \texttt{5:30, Przyjeżdżamy}\medskip


 Ludzie z~SOA wysadzają nas na Uniwersytecie Laval, na obrzeżach miasta. Zarówno Ya Basta! Nowy Jork, jak i~Connecticut już zarezerwowało dla nas miejsca do spania na piętrze głównej sali gimnastycznej. Nastolatek pracujący do późna w~nocy wskazuje nam właściwy kierunek.
 
 -- Tak -- zauważa -- uniwersytet był dość hojny, jeśli chodzi o zaplecze. Bali się, że zajmiemy kampus.

 Siłownia wygląda jak boisko do piłki nożnej. Jego lśniącą drewnianą podłogę pokrywa może dwa tysiące śpiących aktywistów, ułożonych w~geometryczne kępy oddzielone ścianami z~toreb i~plecaków. Przedzieramy się przez trybuny (również pokryte śpiącymi ciałami), w~końcu znajdujemy nasze wyznaczone miejsce, D17, oddzielone białą taśmą, i~rzucamy nasze skromne rzeczy na stos.

 Jednak dzieci z~Connecticut nigdy nie chodzą spać. Po prawie godzinnym przygotowaniu, myciu i~naradzaniu się Kitty oznajmia: 
 
 -- Wiem, że to naprawdę popieprzone, ale zdecydowaliśmy, że lepiej zacząć szukać jakiegoś sprzętu, bo inaczej będziemy zupełnie bezużyteczni na ulicy. 
 
 Cała trójka, Kitty, Lee i~Andrea, zebrała się i~przeliczała swoje pieniądze, które wynoszą około czterdziestu dolarów. Pożyczam im kartę kredytową i~znikają w~poszukiwaniu zapasów. Oznacza to przynajmniej, że Andrea, która była na tyle mądra, by nosić śpiwór, zostawia go (była dyskusja na temat używania go jako wyściółki, ale doszliśmy do wniosku, że noszenie go ze sobą byłoby zbyt denerwujące). Karen i~ja układamy go jako rodzaj długiej poduszki, zrzucamy nasze kurtki i~swetry jako materace i~łapiemy kilka godzin snu.

\bigskip 
\noindent \texttt{08:30 }\medskip


 Prawie wszyscy zaczynają wstawać. Rozespani aktywiści ziewają, przeciągają się, szukają szczoteczek do zębów, szukają łazienki. Karen i~ja postanawiamy udać się do IMC, aby zdobyć dla Karen przepustkę Indymedia. W ten sposób może filmować w~jakimś oficjalnym charakterze. Można sobie wyobrazić, że może to stanowić niewielką ochronę przed aresztowaniem. To wymaga chodzenia po korytarzach Laval -- jednego z~tych szarych modernistycznych kompleksów z~ogromnymi fluorescencyjnymi halami, w~których czujesz się jak pod ziemią, nawet jeśli prawdopodobnie nie jesteś -- z~filiżankami złej kawy z~automatu, szukając stolika z~mapami i~informacjami. W końcu znajdujemy jeden, obsadzony przez kilku uczniów o zaczerwienionych oczach, którzy próbują wyjaśnić lokalny system autobusowy.

 Na szczęście autobusy nadal jeżdżą; chociaż nigdy do końca nie rozgryźliśmy systemu biletowego, a wygląda na to, że konduktorzy autobusowi i~tak nie zadają sobie trudu, aby je sprawdzić. Podążamy za mapą w~kierunku IMC, które niejasno pamiętam z~ostatniej wizyty. Zaledwie jedną przecznicę dalej spotykamy cud. Na rogu, na widoku, znajduje się sklep Army/Navy. Jest nadal otwarty! i~tam, w~szklanym oknie, wielkim jak życie, jest maska  przeciwgazowa. Wpadam do środka i~pytam, czy mają jeszcze jakieś na stanie i~- równie cudownie -- okazuje się, że mają. Dokładnie jedną. Czterdzieści dolców kanadyjskich. i~to też jedna z~tych dobrych, kanadyjskich wojskowych masek przeciwgazowych, z~filtrem z~boku, a nie jak te gówniane izraelskie maski przeciwgazowe z~pierwszej wojny w~Zatoce, na które wszyscy narzekają, gdzie oczy zachodzą mgłą, a plastik potrafi pęknąć. Tutaj to gruby czarny plastik, z~kilkoma paska z~tyłu w~czerni z~drobnymi, żółtymi paskami, które w~moich oczach, wtedy, są dziwnie piękne.

 Każdy z~nas też kupuje torbę na aparat.

 IMC (nikt już nie nazywa go CMAQ, przynajmniej po angielsku) znajduje się na brukowanej alei na bardzo stromym wzgórzu -- tak stromym, że budynek, w~którym się znajduje, ma dwa piętra z~jednej strony i~pięć po drugiej. Wydaje się, że jest w~dużej mierze pusty; wchodzisz przez niedawno odnowioną witrynę sklepową, która wygląda, jakby była tymczasowo przyłączona do jakiejś radykalnej grupy (jest nieumeblowana, z~wyjątkiem kilku krzeseł i~plakatów na ścianie). Odwiedzający muszą przejść przez puste biura, a następnie skierować się na dół do samego IMC -- wciąż na wpół pustego, chociaż w~kącie śpi wielu aktywistów medialnych, a jeszcze kilkunastu bawi się sprzętem lub wkleja listy zadań, zbiorowe zasady. Zerkam na jeden arkusz, na którym uczestnicy mogą przypisywać sobie różne wydarzenia (operacja, jak przewidział Madhava, została skutecznie zdemokratyzowana). W recepcji siedzi niski, brodaty, podobny do gnoma facet, który wydaje się zaangażowany w~przedłużający się flirt z~dwiema młodymi kobietami przy komputerach za nim (nie robią nic poza kpieniem z~niego, a on wydaje się zachwycony ich kpiną). Pstryka każdemu z~nas cyfrowe zdjęcie, a potem radośnie zauważa, że  z~powodu jakiejś usterki w~komputerze nie można było przez cały dzień drukować nowych legitymacji prasowych. Pracuje nad tym. Po około pół godzinie w~końcu udaje nam się załatwić odznaki. Ja też dostaję: w~końcu na pewno będę relacjonować tę historię dla magazynu z~Chicago \textit{In These Times}, dla którego piszę. Karen i~ja podpisujemy uroczyste oświadczenie, w~którym mówimy, że zgadzamy się na zasady jedności IMC i~że w~pewnym momencie w~przyszłości poświęcimy przynajmniej godzinę naszego czasu na jakąś pracę dla IMC. 
 
 -- Nie martw się tym teraz -- zauważa jeden zaspany aktywista -- ale prawdopodobnie będziemy potrzebować wszelkiego rodzaju pomocy w~ciągu następnego dnia lub dwóch. Po prostu zamelduj się ponownie.

 Następnie uzbrojeni w~maskę gazową i~odznakę prasową wracamy na uniwersytet.

\bigskip
 \noindent \texttt{11:00, Konwergencja, Uniwersytet Laval}\medskip


 Całe zamieszanie związane z~obroną Centrum Konwergencji okazało się czymś w~rodzaju czerwonego śledzia. Kiedy pomysł zejścia się na Równinie Abrahama musiał zostać porzucony z~obawy przed atakiem wyprzedzającym, decyzja zapadła na wycofaniu się z~uniwersytetu. Uniwersytet jest jednak oddalony o dziesięć kilometrów od Muru. To będzie bardzo długi marsz.

 Zanim dotarliśmy, są już tam tysiące ludzi, w~większości rozrzuconych po rozległym, otwartym czworoboku w~pobliżu sali gimnastycznej, w~której spaliśmy, przygotowując się do marszu CLAC/CASA Carnival Against Capitalism. Niemal natychmiast wpadam na ludzi, których znam. Sam, aktywny w~nowojorskiej IWW i~DAN Labor, nie był w~karawanie, ale przybył do Akwesasne niezależnie i~jakoś się przedostał. Związał się z~grupą radioaktywistów i~niezależnych dziennikarzy: dwie pary, Shawn (nie mylić z~Shawnem Brantem, organizatorem Mohawk) oraz Lyn, Ben i~Heidi. Są w~większości po trzydziestce, co sprawia, że  -- podobnie jak ja -- są raczej starzy jak na standardy aktywistów. Ponieważ wszyscy jesteśmy oddzieleni od naszych zwykłych grup afinicji, postanawiamy ustanowić siebie jako nową, którą nazywam ,,Uchodźcami z~Akwesasne''. Po krótkiej naradzie szybko dochodzimy do konsensusu co do naszych parametrów i~roli. Będziemy śledzić główną akcję, działając częściowo jako reporterzy, częściowo jako uczestnicy. Nasz udział będzie miał orientację czerwono-żółtą, ale skoncentrujemy się na udzielaniu wsparcia, a nie na bezpośredniej konfrontacji. Pozostaniemy mobilni, staramy się uniknąć aresztowania, rozstajemy się, kiedy nam na tym zależy, ale jeśli to zrobimy, zawsze ustalamy godziny i~miejsca, w~których będziemy mogli się później spotkać.

 Na szczęście Shawn zapewnił sobie miejsce na pobyt: Heidi ma przyjaciela Pierre'a, który buduje sobie dom w~Dolnym Mieście. Jest niedokończony, ale doskonale nadaje się do użytku, jeśli nie mamy nic przeciwko spaniu na podłodze. Shawn ma też samochód.

 To wszystko stawia mnie w~dość dziwnej sytuacji, ponieważ faktycznie należę teraz do dwóch różnych grup afiliacyjnych: ponieważ jestem również de facto członkiem grupy Yabba, mimo że teraz składa się ona z~trojga dzieci, około dwudziestu lat, które wyjechały, by dołączyć do Czarnego Bloku. No cóż, myślę, że da mi pewną elastyczność, by móc poruszać się tam i~z powrotem.

 Wybieramy duży zielony sztandar obok kobiety na szczudłach przebranej za Statuę Wolności i~postanawiamy, że jeśli ktoś odejdzie, będzie to nasz punkt zbieżności.

 Więc odchodzę z~notatnikiem w~ręku. Karen wyciąga aparat.

 Ta część kampusu była w~całości ogromnymi czworokątnymi przestrzeniami i~betonem, z~wyraźnym brakiem zieleni. W tej chwili jednak była wypełniona ogromną różnorodnością kolorowych sztandarów, zwiniętych i~rozwiniętych, niektóre w~po prostu nietypowych kolorach -- łososiowym i~lawendy -- ale także nieskończone wariacje na temat czerwieni i~czerni. Wszędzie młodzi ludzie popijali wodę butelkowaną lub filiżanki kiepskiej kawy z~mikrofalówki, kręcili się, siedzieli w~kółko, grali w~urywki na bębnach zrobionych z~odwróconych dwulitrowych butelek po wodzie, majstrowali przy sprzęcie. Pogoda wciąż była rześka, ale dawała wszelkie oznaki chęci przekształcenia się w~prawdziwy wiosenny dzień. Brak chmur na niebie. Śnieg, który miesiąc wcześniej pokrywał większą część miasta, stopniał. Wyruszyłem na poszukiwanie ludzi Ya Basty!, bez większego szczęścia. W pewnym momencie zobaczyłem grupę mężczyzn, którzy patrzyli z~daleka, jak grupa afinicji Tute Bianche, ale okazało się, że wszyscy byli przebrani za maskotkę Québec o nazwie ,,Bonhomme'' w~śmiejących się maskach Mikołaja i~brudnych, białych kombinezonach.

 Jaggi z~megafonem chodził do każdej grupy ludzi, aby ogłosić, że parada GOMM ma rozpocząć się o 12:30 do dolnego miasta, parada CLAC/CASA Carnival Against Capitalism, która ma wyjechać o 13:00, miała iść w~dół Avenue René Lévesque, ponieważ równiny Abrahama zostały uznane za pułapkę.

 Łapię go na sekundę.

-- Hej, David -- uśmiecha się. -- No to gdzie jest Ya Basta!? Jak było w~Akwesasne?

-- Coś w~rodzaju porażki. Nie udało nam się przejść. Jeśli chodzi o New York Ya Basta!, w~tej chwili myślę, że jestem z~tego zadowolony.

-- Więc wszyscy inni zostali zawróceni?

-- Podjęliśmy decyzję, że albo wszyscy albo nikt.

-- Co? Dlaczego to zrobiłeś? Potrzebujemy wszystkich ciał tutaj!”

-- Cóż, ponieważ\ldots  -- Zastanawiając się, dlaczego to zrobiliśmy? Wzruszyłem ramionami. -- solidarność. Wtedy wydawało się to mieć sens.

 Jaggi miał czas, by podać mi tylko najkrótszy opis tego, co wyłoniło się z~wczorajszej rady. GOMM miał swoją własną paradę, w~skład której wchodzili ludzie SalAMI i~różni Trocki. Zamierzali przeprowadzić czystą Żółtą, klasyczne nieposłuszeństwo obywatelskie, z~blokadą i~tym podobnym, poniżej jednej z~bramek bezpieczeństwa. Parada CLAC, znacznie większa, byłaby żółta (ale nie ,,bezpiecznie żółta'') i~zawierałaby również kontyngent zielony. Plan zakładał, że  Zielony Bloc zboczy, zanim dotrzemy do ogrodzenia, i~zajmie obszar jeszcze niżej w~dół wzgórza od GOMM, skupiony w~strefie zwanej Ilot Fleuriot pod wiaduktem autostrady, w~sąsiedztwie osiedla Jean Baptiste. Wszyscy inni pójdą bezpośrednio do Muru.

 Potem uciekł.
 
\bigskip 
\noindent \texttt{11:40}\medskip


 Czarny Blok w~tym momencie liczy 250 osób, może mniej. Przeważnie noszą czarne bluzy z~kapturem, choć są też ubrania w~stylu wojskowym, a nawet winylowe przeciwdeszczowe. Wszyscy są oczywiście na czarno. Większość z~nich ma maski przeciwgazowe naciągnięte na czubki głowy i~czarne bandany zawiązane na szyjach. W tym momencie głównie wylegują się, paląc lub drzemiąc, ale przed kolumną, która ma być ich kolumną, znajduje się ogromny czerwony sztandar, a dookoła rozrzucone są wszelkiego rodzaju czerwono-czarne flagi. Niedaleko znajduje się kobieta przebrana za Statuę Wolności na szczudłach, a nieco dalej średniowieczny blok z~blaszanymi kapeluszami i~tarczami. Z przyjemnością odkrywam, że rzeczywiście mają katapultę: dość dużą, długą na osiem metrów. Wokół nich roi się od lotnych oddziałów, które wydają mi się w~połowie Czarnym Blokiem, z~maskami przeciwgazowymi i~bandanami, czasem w~ochraniaczach hokejowych, tylko w~radosnych kolorach, nie w~czerni.

 W rzeczywistości sporo osób jest już zamaskowanych; nie tyle ze względów bezpieczeństwa (wydaje się, że nigdzie nie ma policji), ile dlatego, że mają zdecydowanie najfajniejsze bandany w~historii: które po złożeniu na pół zakrywają dolną połowę twarzy obrazem naturalnej wielkości dolnej połowy twarzy innej osoby. Wszędzie je zauważam: są w~kolorze czerwonym, pomarańczowym i~żółtym.

 Ben już ma taką, w~kolorze pomarańczowym. Z dumą pokazuje: jedna strona to szczęśliwa strona, z~wielką uśmiechniętą twarzą; drugi ma twarz z~ustami zaklejonymi taśmą za drutem kolczastym.

-- Tak, najwyraźniej Reclaim the Streets w~Londynie wysłało co najmniej tysiąc. Rozdawali je wcześniej, ale chyba je przegapiłeś. Historia była taka, że  zostały zaprojektowane przez jakiegoś starego faceta, który pracował z~oryginalnymi francuskimi sytuacjonistami. Czy coś. Nie jestem do końca pewien.

 Na marginesie widnieją w~języku francuskim i~angielskim następujące wiersze:

 \textit{Pozostaniemy bez twarzy, ponieważ odrzucamy spektakl celebrytów, ponieważ jesteśmy wszystkimi, ponieważ karnawał kusi, ponieważ świat stoi do góry nogami, ponieważ jesteśmy wszędzie. Nosząc maski, pokazujemy, że to, kim jesteśmy, nie jest tak ważne, jak to, czego chcemy, a to, czego chcemy, jest wszystkim dla każdego.
}

 Wielką niespodzianką są nasze liczby, o których wszyscy mówią, że są znacznie wyższe niż oczekiwano. Ciągle słyszę liczby takie jak pięć tysięcy, może nawet dziesięć. Nigdzie nie widać policji, choć tu i~ówdzie są grupki legalnych obserwatorów, których łatwo rozpoznać w~jaskrawożółtych kamizelkach.

 Jaggi nieustannie pędzi w~górę i~w dół z~aktualizacjami i~ogłoszeniami; ,,W Ekwadorze w~solidarności z~nami okupują kanadyjską ambasadę!''. Najwyraźniej w~Meksyku trwają też akcje graniczne i~ludzie blokują most w~Chicago.

 Wreszcie, powoli, niezgrabnie, rozpoczyna się Karnawał Przeciw Kapitalizmowi.

\bigskip
\noindent \texttt{13:30 \\ Rozpoczyna się Marsz Karnawału Przeciwko Kapitalizmowi}\medskip


 Po dwudziestu minutach parady dochodzi do jakiejś kłótni, kiedy uniwersytecki ochroniarz wije się z~kimś na trawniku przed budynkiem przy trasie parady. Przybywam, gdy ludzie próbują złagodzić eskalację i~nigdy się nie dowiadują, co dokładnie się stało. Właściciel domu i~ośmioletni chłopiec stoją tuż obok swojego ganku. Ktoś na niego krzyczy: 
 
 -- Zabierzcie tego dzieciaka z~powrotem do domu! Z tymi wszystkimi gliniarzami nie jest bezpiecznie! 
 
  Ktoś inny powiedział mi, że strażnik przestraszył się i~wyciągnął broń (tylko po to, by natychmiast otoczyli go aktywiści z~kamerami wideo), ale nie było jasne, co spowodowało incydent.

 Niedługo potem (około 13:50) następuje kolejna drobna plątanina, gdy dziennikarze telewizyjni próbują przejechać samochodem przez tłum. Wokół niego roją się maszerujący, niektórzy walą w~niego, inni kładą się przed nim. 
 
 -- To był dupek -- mówili mi ludzie, ale nie do końca jak, domyślam się, że po prostu arogancko próbował się przebić. W końcu samochód cofa się w~boczną uliczkę i~marsz trwa.

\bigskip
\noindent \texttt{14:00}\medskip


 Na początku przejeżdżamy przez dzielnicę czysto mieszkalną, wszystkie domy jednorodzinne i~okazjonalnie mały ceglany blok mieszkalny. Nigdzie w~zasięgu wzroku nie ma placówki handlowej. Pieśni są po francusku, angielsku, a nawet hiszpańsku. Większość z~nich jest bardzo znajoma: ,,Nie ma takiej władzy jak moc ludu, bo moc ludu się nie kończy!'' ,,Czyje ulice? Nasze ulice!'' ,,El pueblo, unido, jamas sera vencido'' Inni znowu: ,,Sol! Sol! Sol! Sol-i-dar-i-te!''.

 Karen melduje się, krążyła po paradzie, zdobywając wszelkiego rodzaju przydatne materiały. Wszędzie są kamery, ale tym razem to prawie wszystko nasze kamery. Nawet osoby fotografujące nas z~pobocza drogi nie wydają się gliniarzami, ale zwykłymi obywatelami.

 Zauważam, że marsze są zawsze trochę jak akordeon. Mają tendencję do rozciągania się w~miarę upływu czasu, co oznacza, że  musimy od czasu do czasu zatrzymywać się, aby każdy mógł ponownie zebrać swoje grupy afinicji. Czarny Blok, który nigdy nie był duży, już teraz staje się coraz bardziej rozproszony. Korzystam z~tego, aby dostać się do środka i~wreszcie zlokalizować moich przyjaciół z~Connecticut: którzy są teraz częścią grupy około sześciu lub siedmiu osób, po zlokalizowaniu kilku innych byłych Yabbasów z~Nowej Anglii. Nazywają siebie La Resistance (później staje się La Resistance II, gdy odkryją, że nazwa jest już zajęta). Kitty nadała sobie pseudonim akcji Kid A, choć wszyscy ciągle zapominają o jego używaniu. Lee -- surowy weganin -- nazywa siebie Cheesebacon, a Andrea nadal jest po prostu Andreą. Jest jedyną osobą, która ma maskę przeciwgazową (została zawinięta w~jej śpiwór). Pozostali noszą świeżo nabyte zielone wojskowe hełmy i~inny ekwipunek, który kupili wcześniej w~mieście. 
 
 -- Dziękuję bardzo za kartę -- mówi Lee, oddając mi ją. -- Ratujesz życie. Obiecuję, że odeślę ci pieniądze.

-- Och, nie martw się o pieniądze.

-- Nie, naprawdę. Obiecuję, że dostanę twój adres po akcji i~zwrócę ci je.

\bigskip 
\noindent \texttt{14:10}\medskip


 Okrzyki dookoła, gdy marsz się zatrzymuje.

 Nikt w~pobliżu nie wie dlaczego.

 Czarny Blok maszeruje bezpośrednio za czymś, co wydaje się jakąś grupą marksistowską, niosącą tuzin identycznych czerwonych flag ozdobionych wizerunkami amerykańskiego więźnia politycznego Mumii Abu Jamala. Możesz wskazać grupy marksistowskie, ponieważ, podobnie jak ludzie związkowi, mają tendencję do noszenia pewnego rodzaju munduru. W Stanach istnieje grupa o nazwie Rewolucyjna Komunistyczna Młodzieżowa Brygada, która przychodzi na wielkie akcje w~strojach Czarnego Bloku, tylko że w~identycznych koszulkach pod bluzami i~wszystkich noszących dokładnie tę samą czerwoną bandanę -- wyglądającą tak doskonale jak anarchiści, których znałeś, tak naprawdę nie mogliby być anarchistami, ponieważ chociaż cała idea czarnych bloków polega na tym, że każdy jest nie do odróżnienia, żadna grupa anarchistów nigdy nie byłaby tak naprawdę ubrana identycznie. Nie widzę żadnego odpowiednika tutaj w~Quebec City (chociaż później dowiaduję się, że faktycznie tam byli, zmieszani z~Czarnym Blokiem). Istnieje jednak wiele odcinków parady, które w~oczywisty sposób reprezentują tę lub inną grupę socjalistyczną, zwykle rozpoznawalną po dopasowanych koszulkach i~tym, że noszą profesjonalnie wyglądające, wydrukowane znaki. Większe bloki socjalistyczne są prowadzone przez marszałków z~pasującymi opaskami, patrolujących obwód, łączących ramiona, gdy marsz się zatrzymuje. Nawet mniejsze grupy mają zwykle lidera z~megafonem, często idącego tyłem, zaczynającego ich pieśni. To oczywiście sprawia, że  wyróżniają się z~tłumu, podczas gdy anarchiści, z~ręcznie malowanymi znakami i~transparentami, w~większości się wtapiają -- dając niejasne wrażenie, że każdy, kto nie jest związany z~określoną, możliwą do zidentyfikowania grupą, jest najprawdopodobniej anarchistą. takiego czy innego rodzaju. W tym konkretnym marszu to też prawdopodobnie jest prawdą.

 Siadam na ulicy na chwilę, żeby obejrzeć przedstawienie. Po przejściu brygady Mumii i~Czarnego Bloku nadchodzi Blok Średniowieczny ze swoją katapultą. Za katapultą podąża drewniany wózek pełen wypchanych pand i~innych miękkich zabawek, które mogą służyć jako pociski. Następnie pojawiają się autonomiczne elementy we wszystkich ich grupach afinicji, ich znaki, flagi i~sztandary, nieskończona anarchistyczna heraldyka wszelkich możliwych odmian czerwieni i~czerni. (Moje ulubione, szkarłatne serce na sobolowym polu, które powtarzałem z~drobnymi zmianami sześć lub siedem razy, czasem samotnie, czasem z~napisem ,,ANARCHIA = MIŁOŚĆ''). Pojawiły się znaki: \textit{Autonomez Vous }(Autonomize Yourself), \textit{Betail en Revolte }(Cattle in Revolt) i~dziesiątki sztuk o akronimie FTAA/ZLEA (FTAA, Forced To Accept Aristocracy). Są Radykalne Cheerleaderki i~szalejące babcie, żonglerzy, szczudlarze i~przynajmniej jeden mężczyzna na monocyklu. W pewnym momencie rozpoznaję grupę śpiewającą ,,Ya Basta!'', dostrzegam znak Ya Basta! wśród nich, szybko się zbliżają -- ale okazują się być jakąś grupą wsparcia Zapatystów, w~T-shirtach bez żadnego sprzętu. Za nimi natychmiast podążają ludzie SOA noszący maski szkieletowe z~ogromnym zielonym sztandarem.

 Brakuje tylko gigantycznych marionetek: podobno kilka zabrano poprzedniej nocy na paradę z~pochodniami, ale teraz są ukryte, czekając na jutrzejszy marsz robotniczy.
 
\bigskip
\noindent \texttt{14:20}\medskip

 Ktoś ogłasza, że  jesteśmy dziesięć minut od Muru. Teraz zaczynamy widzieć sklepy, w~większości zamknięte.

 Działacz w~kostiumie zajączka wielkanocnego próbuje rzucać cukierki grupie dzieci oglądających paradę z~tarasu mieszkania. Natychmiast staje się celebrytą: ,,króliczkiem'', wszyscy go nazywają, na przykład ,,Hej, widziałeś królika?'' Nie należy go jednak mylić z~innym ,,króliczkiem'', uczniem, który podczas marszu niósł ze sobą swojego królika. Bunny Guyowi udaje się dorzucić kilka na jednym tarasie, a dzieci chętnie je zgarniają.

 Widzowie nadal wydają się ostrożni, choć ich liczba rośnie. Mijani aktywiści uderzają w~znaki drogowe, głośniej niż jakoś szczególnie rytmicznie. Tu i~ówdzie gromady z~prawdziwymi instrumentami muzycznymi.

 Nadal nie ma chmurki na niebie. Właściwie robi się dość gorąco. Stopniowo zdejmuję warstwy, a ci, którzy są przygotowani -- tacy jak na przykład moi przyjaciele z~Czarnego Bloku -- naprawdę zaczynają to odczuwać. Ludzie wołają o wodę. Czasami jestem z~resztą Refugees, którzy ustawili się za i~na skraju Czarnego Bloku, czasami eksploruję paradę, od czasu do czasu sprawdzając co z~Karen. La Resistance, uzbrojona aż po rękojeść, też chce wody, więc Refugees przeczesują ulice w~poszukiwaniu jakiegoś miejsca, żeby coś kupić (zdecydowaliśmy, że będziemy wspierać pracę), ale bez większego sukcesu. W końcu znajduję sklep spożywczy, który jest otwarty, ale wpuszcza ludzi tylko w~grupach po dwoje, z~jakimś facetem ustawionym przy drzwiach, aby za każdym razem je zamykać i~otwierać. Shawn i~ja czekamy przez chwilę w~kolejce, ale zdajmy sobie sprawę, że zanim wejdziemy, zgubimy paradę.

 Shawn, który od jakiegoś czasu monitoruje lokalne media, jest zdumiony całkowitym brakiem policji. 
 
 -- Od miesięcy prowadzą kampanię terroru, mówiąc wszystkim, że zamierzamy zniszczyć miasto. Popatrz? Widziałeś jednego gliniarza? W jakimkolwiek miejscu? Gdyby ktoś naprawdę chciał, moglibyśmy spalić całą okolicę. 

-- Może mają nadzieję, że ktoś to zrobi, aby dać im pretekst do ataku.

-- Może. Ale chodzi mi o to: albo wiedzieli, że kłamią, kiedy próbowali przekonać wszystkich, że stanowimy zagrożenie dla miasta, albo naprawdę mają w~dupie ludzi, których powinni chronić. 

\bigskip
\noindent \texttt{14:25}\medskip


 Mijamy plac budowy. Niewielki tłum idzie w~górę alejką utworzoną przez dwa płoty z~siatki, ale nie zamierzają, jak się domyślałem, szarpać fragmentu płotu, żeby go zabrać ze sobą. Zamiast tego mężczyźni i~kilka kobiet zakładają maski i~zaczynają łamać i~zbierać cegły oraz skały. Nad nimi stoi (głównie żeński) chór i~skanduje,,Dzisiaj coś rozjebiemy!'' w~lekko akcentowanym angielskim. 

 W rzeczywistości nie są w~strojach Czarnego Bloku, ale wydają się studentami, a może po prostu lokalnymi nastolatkami. Właściwie nie mam pojęcia, kim oni są, ale domyślam się, że to musiałby być Czerwony Blok. 

\bigskip
\noindent \texttt{14:30}\medskip


 Niektórzy z~Black Block niosą ze sobą materac jako rodzaj gigantycznej tarczy. W jakiś sposób przed nimi jest teraz ciężarówka, tuż za batalionem Mumia, grająca jakiś rodzaj francuskiej muzyki rap. Mac i~Lesley podskakują, zamaskowani, w~wojskowych strojach. Wymieniamy uprzejmości. Potem znowu znikają.

 Parada zatrzymuje się co chwila. Zaczyna iść.

\bigskip
\noindent \texttt{14:50}\medskip

 Avenue de Erables to punkt, w~którym parada ma podzielić się na dwie kolumny, zieloną i~żółtą. Grupa Zielonych pomaszeruje na północ w~górę Avenue Cartier, która jest dwie przecznice na północ, a następnie wejdzie do robotniczej dzielnicy St Jean-Baptiste, która leży na stromych ulicach opadających na północ od Muru. Heidi, która udzielała wywiadów radiowych w~górę i~w dół parady, wyjaśnia, że  sama okolica, wraz z~obszarem dalej na północ, wokół autostrady, została ogłoszona Zieloną Strefą. Lalkarze i~grupy teatrów ulicznych zajmą teren i~będą wystawiać spektakle dla grup społeczności lokalnych, które ściśle z~nami współpracują. (CASA od miesięcy chodziła od drzwi do drzwi w~Jean Baptiste z~ulotkami i~petycjami wyjaśniającymi, co się stanie). Taki był plan. W tym momencie jednak wygląda na to, że niewielu Zielonych tak naprawdę odchodzi: nawet bębniarze ważek -- grupa teatralna z~przezroczystymi lalkami ważek podskakujących nad ich głowami -- i~inne oczywiście zielone grupy na razie kontynuują naszą współpracę. W międzyczasie, gdy zatrzymujemy się, ktoś w~food trucku wykorzystuje okazję, by zapewnić szybką przekąskę. Wszyscy podają sobie talerze z~makaronem. Trochę łapiemy, ale większość przekazujemy La Resistance.


\bigskip
\noindent \texttt{15:05}\medskip


 Czekając, wracam z~Lyn do sklepu spożywczego i~z powodzeniem kupuję kilka butelek wody. Kiedy wracam, słyszymy pogłoski, że widziano trzy oddziały gliniarzy zmierzających w~naszą stronę (żaden się nie zmaterializował).

\bigskip
\noindent \texttt{15:15}\medskip


 Wreszcie ruszamy. Okazuje się, że przez cały ten czas byliśmy tylko kilka przecznic od muru. Mijając Avenue Turnbull, marsz wkracza na teren, który tak dokładnie zbadaliśmy podczas naszej ostatniej wizyty. Mijamy Grand-Théâtre de Québec, wchodząc do małego parku, który wkrótce wielu z~nas będzie znany jako ,,Ground Zero''. Park to w~większości tylko ogromny trawnik z~kilkoma pagórkami i~kilkoma małymi zagajnikami drzew tu i~tam. Na drugim końcu znajduje się Mur, metrowa betonowa podstawa i~dwumetrowa siatka na szczycie. Biegnie wzdłuż następnej ulicy z~północy na południe, Rue d'Amerique Francaise, a następnie ostro zakręca na północ. Mrużąc oczy, zauważam, że w~większości miejsc jest już pokryty wstążkami, wizerunkami i~rzeźbami wplecionymi w~niego podczas kobiecej akcji poprzedniego wieczoru. Podstawa została obficie pomalowana sprayem.

 Pośrodku parku stoi rząd gliniarzy, może czterdziestu lub pięćdziesięciu, w~pełnym rynsztunku. Nigdy nie widzieliśmy tego dnia policji, która nie byłaby w~pełni opancerzona. Ci wydają się stać tam, aby chronić dostęp do punktu kontrolnego/wejścia naprzeciwko północno-wschodniego rogu parku. Poza nimi nie ma nikogo dookoła. Nawet dwie ciężarówki z~mediami z~wystającymi z~nich antenami satelitarnymi wydają się bezobsługowe. Nad głowami terkoczą żółte helikoptery obserwacyjne. 

 \begin{figure}
\begin{center}
\includegraphics[width=\textwidth]{english-quebec-city.png}
\end{center} 
 \caption{\\
 1. Droga wspólnego marszu\\
 2. Marsz Czerwonego/Żółtego Bloku\\
 3. Marsz Zielonego Bloku\\
 4. ,,Ground Zero''\\
 5. Brama wejściowa\\
 6. Obalona pierwsza sekcja Muru\\
 7. Mur\\}
 \end{figure}

Parada zaczyna wlewać się w~otwartą przestrzeń. Wszyscy maszerują wprost na policję. Policja się waha (można sobie tylko wyobrazić, jak to jest być w~oddziale czterdziestu kilku gliniarzy z~prewencji, obserwujących kilka tysięcy anarchistów maszerujących bezpośrednio na ciebie). Potem odwracają się, maszerują z~powrotem za punkt kontrolny, a my wpadamy do parku.

 Obok mnie ktoś krzyczy ze złością po francusku i~rzuca do połowy pełną butelkę wody w~wycofujących się gliniarzy. Towarzysz bierze go delikatnie pod ramię, jakby chciał powiedzieć: ,,Wszyscy wiemy, co się wydarzy. To nie my powinniśmy zacząć''.

 Czarny Blok nie stoi na czele marszu. Awangarda jest całkowicie niejednorodna, choć zawiera niektórych z~najlepiej przygotowanych: wielu w~takiej czy innej formie ochraniaczy, niektórzy w~hełmach i~tarczach. Kiedy podchodzę do przodu, jest już jeden facet w~żółtej kurtce, który wdrapał się na szczyt jednego odcinka Muru w~pobliżu punktu kontrolnego (w rzeczywistości nie ma na nim drutu kolczastego). Kołysze się w~przód i~w tył, próbując użyć własnego ciężaru, aby rozchwiać. Tłum gromadzi się wokół niego z~hakami z~liną -- a właściwie są to haki wielkości pięści, w~kształcie orzecha, przymocowane do długich, mocnych linek. Inni zabrali się do pracy z~przecinakami do drutu. Policjanci bez twarzy, wszyscy w~maskach przeciwgazowych i~zbroi bojowej, stoją beznamiętnie, jakieś dziesięć metrów dalej, wewnątrz obwodu, gdy pierwszy panel schodzi z~betonowych podstaw i~upada na ziemię. Policja nic nie robi.

 Niedługo wszyscy znaleźli sobie pustą część ogrodzenia. Przeważnie procedura wygląda tak: małe zespoły z~linami używają haków, aby przymocować je do ogniwa łańcucha, a następnie wszyscy pomagają w~pociągnięciu. Kiedy Mur zacznie ustępować, ludzie wspinają się na górę, aby ją zepchnąć. Kiedy docieram, osiem lub dziewięć sekcji jest już obalonych i~muszę ruszyć na północny wschód od punktu kontrolnego, aby znaleźć miejsce, w~którym jestem potrzebny. W końcu ciągnę tę samą linę, co Mac i~Lesley i~jednego szalenie dużego wojownika Mohawk (później powiedziano mi, że takie osoby są określane w~rezerwacie jako FBI, ,,Fucking Big Indians''), który prawdopodobnie ma siłę trójki nas razem. W tym momencie nikt nawet nie nosi masek. Jak wiele osób mam maskę przeciwgazową na czubku głowy. Kiedy nasz odcinek ogrodzenia opada, przechodzimy do kolejnego. W pewnym momencie część spada bezpośrednio na moją głowę i~kilka osób obok mnie. Wszyscy się śmiejemy, dwoje z~nas ściska sobie ręce, potem przechodzimy do następnego miejsca.

 Wkrótce skręcone kawałki zburzonego ogrodzenia z~siatki są rozrzucone na krawędzi obwodu. Z jakiegoś dziwnego powodu gliniarze nadal nic nie robią, po prostu tam stoją. Najwyraźniej otrzymali rozkazy, aby oprzeć się wszelkim próbom wejścia do strefy bezpieczeństwa i~traktują je niezwykle dosłownie.

 W końcu mały oddział aktywistów, chyba około dwudziestu, zbiera się na natarcie. Wydaje mi się to kompletnie szalone, ale może mają jakiś plan. Jeśli tak, nigdy się nie dowiem. Ponieważ niemal w~momencie, gdy zaczynają biec w~stronę policji, wokół nich zaczynają eksplodować bomby pieprzowe. Niektórzy zaczynają się potykać, upadać; w~ciągu kilku sekund cały kontyngent wycofuje się w~nieładzie.

 Od tego momentu przez następne dwa dni trwa nieustanna wojna chemiczna. Policja zaczęła strzelać wzdłuż Muru na zespoły wciąż ciągnące sekcje w~dół (w tym momencie około pięćdziesiąt metrów zostało całkowicie oczyszczonych). Kanistry z~gazem łzawiącym zaczęły podskakiwać, wirować, eksplodować wokół nas. Założyłem maskę gazową; podobnie około połowy ludzi. (Widziałem co najmniej tuzin marek masek gazowych, izraelskich, czeskich, belgijskich, kanadyjskich, jakiś dziwny rosyjski produkt z~długą rurką spływającą do sakiewki przypiętej do paska). Inni używali szalików, bandan, cokolwiek było pod ręką. Widziałem ludzi, którzy majstrowali przy przyłbicach i~plastikowych okularach pływackich, a łzy i~śluz spływały im po twarzach. W pewnym momencie, gdy szukałem nowej pozycji na ścianie, bomba pieprzowa musiała wybuchnąć tuż obok mnie. W przeciwieństwie do gazu łzawiącego przeszła przez moją maskę gazową i~nagle straciłem wzrok i~nie mogłem oddychać. Cofnąłem się poza zasięg, na otwarte powietrze parku, oczy wciąż płonące i~niezdolne do skupienia się, z~trudem łapiąc oddech, i~przez chwilę wędrowałem w~kółko, aż znalazłem wolne miejsce, zdjąłem maskę i~usiadłem na trawie. Po może minucie byłem w~zasadzie znowu sprawny.

 Park był wtedy pełen grup ludzi, poruszających się z~różnymi prędkościami w~różnych kierunkach; zauważono również coraz liczniejsze obłoki gazu. Początkowo rozproszyły się dość szybko: wiał silny wiatr, który, ku uciesze wszystkich, wiał bezpośrednio na policję. Tu i~ówdzie małe grupki aktywistów siedziały w~kółko na trawie na skrawkach wyższego terenu, zaangażowane w~poważne konsultacje -- jak sądzę, żółte grupy afinicji, próbujące wymyślić, co robić. Dla większości decyzja wydawała się sprowadzać się do pozostania w~parku i~stworzenia jak najlepszej atmosfery karnawału, pomimo ataku chemicznego.

 Kilka minut później, gdy znów znalazłem się przy Murze, przekształciło się to w~walkę pozycyjną. Po postawieniu ściany gazu policja najwyraźniej próbowała iść naprzód, ale została odepchnięta przez deszcz kamieni. Zamaskowane postacie blisko ogrodzenia, teraz oznaczonego tylko przez poobijaną betonową podstawę dawnego płotu, z~której połowa się przewróciła, nadal przerzucały w~ich stronę kamienie i~butelki, kiedy przybyłem. Z jakiegoś powodu było tam paru pacyfistów -- przynajmniej kilka kobiet gniewnie krzyczało: ,,Przestań rzucać gównem!''. Policjanci byli już schronieni za szeregiem plastikowych osłon, strzelając bezpośrednio w~aktywistów gazem łzawiącym i~plastikowymi kulami. Pacyfiści dokonali pospiesznego odwrotu.

 Ja też. Wróciłem do parku i~zanotowałem kilka uwag.

\bigskip
\noindent \texttt{15:43}\medskip
 

\noindent [z Notatek]
\medskip

 Policja w~tym momencie wciąż ma beznadziejną przewagę liczebną. Miotacze kamieni pojawiają się za każdym razem, gdy próbują iść naprzód, ale poza tym w~dużej mierze wydają się wstrzymywać ogień. Nikt też nie próbuje posuwać się naprzód po strzaskanym ogrodzeniu.

 W tym czasie puszki z~gazem spadają dość regularnie, nie tylko w~pobliżu granicy, ale wszędzie. Spadają jak pociski moździerzowe, wzlatują łukami w~górę, zwykle od trzech do pięciu naraz, a potem spadają grupami, uderzając w~otwartą przestrzeń parku. Na początku za każdym razem, gdy lądują, wzniecają małą panikę.

 A jednak stawało się to czymś w~rodzaju karnawału. Ludzie tańczyli, bębnili i~klaskali, próbując stworzyć świąteczny okupowany teren w~chmurach gazu łzawiącego i~poza nimi. Mijam cztery kobiety tańczące w~pajęczych szalikach, wszystkie w~maskach przeciwgazowych. Inni kręcą się bez nawet bandan, tylko z~czystego buntu.

 Bunny Guy podchodzi do ściany, machając rękami, z~wielkim dramatyzmem. Zagazowany wykonuje pospieszny odwrót.

 Działacze z~kijami hokejowymi systematycznie odrzucają kanistry z~powrotem na obwód, a jeden facet w~masce przeciwgazowej zgarnia jeden, podbiega do obwodu z~kłębiącym się za nim pióropuszem gazu i~wyrzuca go z~powrotem przez Mur.

-- Nie rób tego bez rękawiczek -- ostrzega mnie medyk. -- Są rozgrzane do czerwoności. Mogą spowodować poważne oparzenia.

-- I~to nie mogą być dowolne rękawiczki -- mówi inny. -- Przepalą cienką skórę. Naprawdę potrzebujesz rękawicy hokejowej.

\bigskip
\noindent \texttt{15:50}\medskip
 

 Kiedy odnajduję Shawna i~Heidi, podekscytowany informuje, że udaremniliśmy pierwszą próbę manewru flankującego gliniarzy. Próbowali sprowadzić armatkę wodną -- w~zasadzie był to opancerzony wóz strażacki -- z~północnego zachodu, za teatrem, żeby nas odciąć. Kilka grup afinicji Czarnego Bloku wbiegło na miejsce zdarzenia i~wyłączyło go, rozbijając szyby i~atakując opony, aż kierowca, przekonany, że zostanie wyciągnięty z~kabiny, włączył wsteczny i~wycofał się pospiesznie. Nikt nie został ranny, ale krążyły plotki, że towarzyszący mu oddział policji złapał kilku przypadkowych aktywistów w~pobliżu miejsca zdarzenia (oczywiście nie dzieci z~Czarnego Bloku, to byłoby zbyt trudne) i~zabrał ich ze sobą -- prawdopodobnie pierwsze tego dnia aresztowania.

 Patrzymy z~daleka, jak kolejna linia gliniarzy maszerujących w~stronę teatru zostaje obrzucona tak mocno, że też musieli się wycofać. 
 
 -- Niebiescy -- wskazuje Heidi -- to policja prowincji. Ci na zielono to lokalni, miejscowi policjanci. To nic wielkiego. To niebiescy faceci są naprawdę przerażający, ponieważ są brutalni i~mają wszystkie najnowocześniejsze gadżety.

 Ktoś twierdzi, że właśnie zobaczył jednego z~gliniarzy w~pobliżu wycieczki do teatru, jak upadł i~z frustracji wielokrotnie uderzał pałką o ziemię. Kolejne drobne zwycięstwo. Ktoś to słyszy i~się uśmiecha.

 Wszędzie były różnego rodzaju kamery. Wielu aktywistów dokładnie dokumentuje akty karnawałowego sprzeciwu; inni filmują gliny. Karen nas znajduje. Mówi, że dławi się gazem i~nie może już filmować; idzie do Zielonej Strefy. Mówimy, że spotkamy się tam później. Prawie jak tylko poszła, wpadam na Time's Up Bill, aktywistę rowerowego z~Nowego Jorku. Bill został zdemaskowany, wyglądał ponuro obojętny na gaz, ale był uzbrojony w~ogromną kamerę wideo.

 Zauważa mnie, bo na chwilę zdjąłem maskę.

-- Hej, David, jesteś teraz zajęty? Czy zechciałbyś udzielić krótkiego wywiadu na temat Akwesasne?

-- Pewnie. Dobra, jak długo mówimy?

-- Tylko minuta lub dwie.

 Uśmiecham się. 
 
 -- Chcesz to zrobić tutaj?

-- Ta.

 Podchodzimy do miejsca o stosunkowo czystym powietrzu, około piętnastu metrów od punktu kontrolnego, i~zaczynam podawać krótki opis karawany, smażenia, przeprawy. Mniej więcej w~połowie patrzymy w~górę i~zauważamy trzy kanistry opadające wdzięcznym parabolicznym łukiem bezpośrednio na nasze głowy. Zaczynamy biec, śmiejemy się, zmieniamy pozycję nieco dalej od akcji i~kończymy wywiad.

\bigskip
\noindent \texttt{16:10}\medskip


 To zamienia się w~impas. Nikt nie rzuca kamieniami, chyba że policja próbuje iść naprzód, a na razie już nie próbuje. Zamiast tego, po prostu rzucają do parku niekończące się bomby z~gazem łzawiącym i~pieprzem, a aktywiści na obwodzie albo je odrzucają, albo rzucają czymkolwiek, co może wyglądać jak odpowiedź w~naturze. Zaczęło się głównie od wymiany gazu łzawiącego na bomby dymne, które przelatywały w~podobny sposób. Są też całkowicie nieszkodliwe -- czysto symboliczne qui pro quo, ale w~jakiś sposób bardzo satysfakcjonujące. Później ludzie wydawali się strzelać racami, a ja widziałem kolorowe światła, które moim zdaniem musiały być rzymskimi świecami, rakietami z~butelek lub czymś takim. Dalej katapulta rzucała pluszowymi misiami na pozostałe fragmenty muru. To wszystko było czysto ekspresyjne, prawie jak kwestia zasady, że możemy dać tyle, ile mamy.

 Początkowo lądowanie kanistra w~tłumie wywołało panikę, mimo że ludzie krzyczeli, żeby nie uciekać. Stało się to zwłaszcza wtedy, gdy policja zaczęłaby używać kanistrów, które stawały w~płomieniach i~zaczynały szaleńczo wirować, oczywiście niemożliwych do odrzucenia. Jednak panika szybko ustąpiła, ponieważ to głównie ludzie w~maskach gazowych lub mocni mieli potrzebne środki do pozostania. Ktoś pokazał mi sztuczkę zatrzymania się bezpośrednio przed grupą spanikowanych, uciekających ludzi z~rozłożonymi rękami; niezmiennie zwalniali, a potem się zatrzymywali. Ale niedługo spanikowane ucieczki i~tak się zatrzymywały.

\bigskip
\noindent \texttt{16:17}\medskip
 

 Na północ od parku znajduje się małe skupisko drzew, które stało się rodzajem centrum obserwacyjnego dla osób nie walczących. Obok stoi kilku wojowników Mohawk, w~tym Stacey Boots, który najwyraźniej nigdy nie podszedł do ściany, ale cofnął się jak prawdziwy dowódca wojskowy, udzielając od czasu do czasu porad taktycznych. Jest też pięciu lub sześciu pracowników-metalowców, niektórzy anglofońscy, niektórzy frankofońscy, bez masek, ale na wszelki wypadek mają bandany i~ocet. Nie są w~akcji, ale dosłownie pokazują flagę: otaczają dużą tabliczkę, którą umieścili w~pobliżu drzewa w~ich unijnych kolorach.

 Mniej więcej w~tym momencie zaczynam zauważać, gdy badam strefę w~pobliżu obwodu, że wiele zamaskowanych postaci wokół mnie to w~rzeczywistości przyjaciele. La Resistance wyłania się z~mgły, podczas ogólnej wymiany uścisków. Nieco dalej na północ jest Buffy, w~całości w~kolorze czarnym, ze wzmocnionym kaskiem rowerowym i~okrągłą pokrywą na śmieci jako tarczą. Za nią jest większość innych dzieci z~Wysp Księcia Edwardsa, podobnie ubranych. Buffy na chwilę zdejmuje maskę, żeby pomachać. Jeśli grupa PEI przyjmie rolę peltastów, lekkich i~mobilnych, Montreal Ya Basta! to hoplici. Około dwudziestu z~nich stoi w~szyku niedaleko -- za ścianą z~tarcz składającą się z~trzynastu i~pięciu lub sześciu bębniarzy: także w~większości w~czarnym, w~czarnych kaskach motocyklowych, czarnych maskach przeciwgazowych i~metrowych czarnych plastikowych tarczach, ale wszyscy pokryci dziwnym, piankowym, tęczowym obiciem, z~kolcami dinozaurów na plecach, złożonymi kształtami wyłaniającymi się z~hełmów, symbolami pentagramu na tarczach. Bębny wykonano z~plastikowych butelek po wodzie. Jest to jednak niezwykłe wizualnie, choć taktycznie, trochę bezcelowe. Na tak szerokiej, otwartej przestrzeni falanga jest mniej więcej tak skuteczna, jak pierwotna linia gliniarzy: jeśli nie masz linii setek, można dość łatwo Cię oskrzydlić i~otoczyć. Jednak tarcze są bardzo skuteczne przeciwko pojemnikom z~gazem łzawiącym i~plastikowym kulom (których policja zaczyna używać dość bezkrytycznie), jeśli nie nadają się do utrzymywania pozycji. Wygląda na to, że Yabbowie znaleźli jakiś cel, po prostu wystawiając się i~ściągając ogień. 

 To zdaje się, że powstaje podział pracy. Czarny Blok, a zwłaszcza Amerykanie, przejmują rolę pierwszej linii obrony. Sami nie rzucają pociskami, po prostu trzymają pozycję -- choć są gotowi wykorzystać każdą okazję, by zburzyć nowe sekcje ogrodzenia. Wszyscy rzucający kamieniami wydają się lokalni. Wydaje się, że wielu mogło być tymi bojowymi siedemnastolatkami, o których mówił mi Sebastien, którzy, przeciwnie niż Bloc, nigdy nie przyjęli zasad niestosowania przemocy.

\bigskip
\noindent \texttt{16:22}\medskip
 

 Wiele akcji w~tym momencie dzieje się z~boku obszaru, w~którym pierwszy raz upadł mur: tuż pod nim biegnie szeroka ulica i~kolejny pas muru jako taki. 

 Wracam do punktu obserwacyjnego, gdzie ogromny wojownik Mohawk, z~którym wcześniej dzieliłem linę, wydaje się, że właśnie wrócił z~walki, najwyraźniej po raz pierwszy. Z radością opowiada historię o tym, jak po raz pierwszy runęło ogrodzenie. Stacey, zawsze stoicki, pozwala sobie na krótki uśmiech. Zwraca się do dwóch zamaskowanych Czarnych Bloków, oferując strategiczne porady.

-- Uważajcie, aby tam strzec swojej lewej flanki i~zostawić drogę ucieczki, ponieważ jeśli ruszą tą ulicą i~cię otoczą, może ona zmienić się w~,,strefę śmierci''. W ten sposób dochodzi do masakr.

 Strategia policji, teraz, gdy wcześniejsze próby wbicia klina w~park lub odcięcia, się nie powiodły, wydaje się po prostu polegać na pompowaniu gazu łzawiącego -- i~zauważam, że to coraz bardziej paskudny gaz łzawiący -- w~strefę otaczającą mur godzinami, aż do nasze liczby zaczną się przerzedzać. Potem prawdopodobnie wyprowadzą się i~zabezpieczą teren na ceremonię otwarcia zaplanowaną na 17:30. Ostatecznie nie da się ich powstrzymać, bo otrzymują posiłki, a nasza liczba może się tylko zmniejszyć. Nigdy nie będzie nas tylu, ilu mieliśmy, kiedy po raz pierwszy uderzyliśmy na Mur. Naszym celem staje się zatem spowolnienie ich tak bardzo, jak to możliwe.

 \begin{figure}
 \begin{center}
 \includegraphics[width=\textwidth]{quebec-city.png} 
 \end{center}
 \caption{\begin{small} Szczegół Quebec City wskazujący granicę bezpieczeństwa (linia gruba) i~przybliżony obszar rozmieszczenia gazu łzawiącego (cienka linia, szara strefa)\\
 1) Miejsce akcji CLAC/CASA w~piątek 20 kwietnia. To na tym skrzyżowaniu ściana po raz pierwszy runęła. \\
 2) Miejsce wielu akcji na Rue St. Jean Wyznaczone w~piątek jako zieloną strefę, stało się czerwone, gdy w~sobotę ogrodzenie zostało zniszczone. \\
 3) Alternatywne centrum medialne. \\
 4) Miejsce akcji GOMM w~piątek i~konfrontacja kontynuowana w~następnych dniach. \\
 5) I'llôt Fleuri, punkt końcowy marszu przy świecach z~Laval, początek obchodów czwartkowej nocy, miejsce zarówno darmowej kuchni, jak i~akcji zielonej strefy. \\
 6) Miejsce zbiórki i~punkt wyjścia na sobotni marsz. \\
  \texttt{+} -- centrum medyczne. \\
  Centrum medyczne, punkty 3 i~5 zostały bezpośrednio zagazowane przez policję, pomimo oznaczeń stref zielonych i~odległości od Płotu. (Raphaël Thierrin i~Steve Daniels)\end{small} }
  
 \end{figure} 
 

 Późne popołudnie zamienia się w~rodzaj stopniowego, bojowego odwrotu.
 
\bigskip
\noindent \texttt{16:30}\medskip
 

 Duża wymiana gazu łzawiącego na bomby dymne.

 Park jest teraz pod ciągłą chmurą gazu łzawiącego. Różne grupy afinicji zajęły w~nim pozycje, oznaczone flagami: niektóre czerwone, niektóre czarne, niektóre wielokolorowe. Jest jedna bardzo kolorowa flaga rdzennych Amerykanów z~głową Wojownika w~czerwono-żółtym kolorze, którą Mac, jak mówi mi, nazywa się ,,Flagą Wielu Narodów'', wyeksponowana w~widocznym miejscu na środku placu. Ludzie używają ją jako sygnału wskazującego, gdzie policja próbuje iść. Przed chwilą pomogło to zmobilizować ludzi do odpędzenia szeregu gliniarzy poprzez rzucanie cegłami 
 
 Gliniarze, jak podkreśla Mac, byli całkowicie opancerzeni, więc nikt z~nich nie zostanie poważnie ranny. W większości pociski po prostu odbijają się od ich tarcz.  Nadal jest prawie niemożliwe manewrowanie, nie mówiąc już o rozpoczęciu aresztowań, pod ciągłym deszczem cegieł, więc tak efektywnie ich to odpiera.

 Gaz łzawiący jest ciągle wyrzucany z~powrotem w~pobliżu płotu. Medycy, którzy początkowo przebywali głównie na drugim końcu parku, przemywając oczy i~lecząc astmatyków, zaczynają poruszać się, by leczyć ofiary poparzeń -- gliniarze coraz częściej używają wyrzutni gazu łzawiącego, takich jak pistolety, strzelając nimi bezpośrednio w~klatki piersiowe i~głowy ludzi. W kółko słyszę okrzyki ,,Medyk!'' lub częściej francuska pieśń: ,,Sol! Sol! Sol! Solid-dar-i-té!''. Ilekroć ktoś upadł, trafiony kanistrem lub plastikową kulą, ludzie zbierali się i~zaczynali skandować o solidarności. Inni aktywiści przychodzili i~tworzyli ludzką ścianę, gdy zespoły medyków podbiegały -- zwykle trzy lub cztery do zespołu, zawsze w~bieli, z~ogromnymi czerwonymi krzyżami na całym ciele -- aby przenieść ofiary poza zasięg. Medycy musieli biec szybko, bo inaczej policja zaczynała do nich strzelać.

\medskip
\noindent \textit{ Wstępne uwagi fenomenologiczne dotyczące działań QC, napisane wkrótce potem:}

\begin{enumerate}

\item W dużej akcji absolutnie nie ma sposobu, aby uchwycić nawet ułamek tego, co się dzieje. Dzieje się jednocześnie setka drobnych dramatów, którym później uczestnicy nadadzą formę opowieści. W każdej chwili prawdopodobnie widzisz malutkie odłamki tuzina -- kogoś uciekającego w~czymś, co wydaje się losowym kierunkiem, kogoś stojącego czymś zajętego, grupę ludzi robiących coś, czego nie możesz do końca dostrzec w~oddali. Pięć metrów dalej -- za murem, pod skarpą -- mogą mieć miejsce ważne wydarzenia, o których absolutnie nie masz pojęcia; przynajmniej póki dużo później, gdy zaczniesz syntetyzować historii.

\item Gaz łzawiący tworzy całkowicie wrogi krajobraz miejski. To, co powinno być zaprojektowane dla naszej wygody, parki, ulice, potencjalnie własne ubranie, staje się bolesne, ale też zachęca do niekończącego się przytulania i~budowania więzi, ponieważ każdy, kogo widzisz, kto tak naprawdę do ciebie nie strzela, jest twoim przyjacielem.

 Zagazowanie jest trochę jak podpalenie; a przynajmniej to, co można sobie wyobrazić jako bycie podpalonym. Gaz pieprzowy jest taki sam, tylko bardziej.

\item Normalnie każdy może skonfrontować się z~glinami. Kiedy jeden z~nich robi coś wyraźnie niesprawiedliwego, zawstydzasz go: często dosłownie słychać śpiew ,,Wstyd! Wstyd! Wstyd!{\textquotedbl} Wstyd!{\textquotedbl}, ,,Cały świat patrzy!''. W Nowym Jorku popularnym śpiewem podczas oczywistych aktów represji jest ,,Idź walczyć z~przestępczością! Idź walczyć z~przestępczością!''. Nic z~tego nie jest tutaj możliwe. Nawet kiedy, jak na A16, policjant bije cię pałką, gdy leżysz na ziemi, masz pojęcie, kto cię bije. Możesz go porównać do łobuzów, którzy bili cię w~szkole podstawowej, albo do policji w~telewizji. Ci gliniarze to widma, duchy, mechaniczne abstrakcje. Całkowicie niemożliwe jest postrzeganie ich jako jednostek. To tylko figury na planszy i~źródła przeróżnych form przerażenia i~bólu.

\item Maski przeciwgazowe sprawiają, że człowiek czuje się trochę jak maszyna -- przytulanie i~przytulanie ma po części przypominać, że tak nie jest.
\end{enumerate}

\bigskip
\noindent \texttt{16:35}\medskip
 

 Więcej gazu -- okresowe wezwania ,,Medyk!'' -- gdy ludzie są uderzani przez kanistry lub przez plastikowe kule, które są obecnie używane mniej lub bardziej na oślep. Tyle o zasadach zaangażowania ogłoszonych z~taką fanfarą przed Szczytem. Ludzie podbiegają i~rzucają bomby dymne i~kanistry z~gazem łzawiącym prosto w~gliniarzy.

 Okrzyki pojawiają się, gdy jeden z~policjantów upada, uciekając. Bitwa ciągle jest nieokreślona. Widzę, jak ktoś jest zabierany z~krzykiem, z~poważnymi oparzeniami i~zakrwawionym ubraniem.

 Craig, wielki gość z~rady delegatów, gramoli się w~kierunku ogrodzenia, uzbrojony w~wielką sztachetę, siedem centymetrów na dziesięć, który gdzieś znalazł, niosąc go jak miecz, wyglądając na niezmiernie z~siebie zadowolonego. Ma na sobie coś, co można opisać tylko jako czarny strój bojowy, owinięty w~plastikowe torby, z~okrągłą tarczą i~maską przeciwgazową na głowie. Jakieś dwadzieścia sekund później podbiega dwóch medyków i~pyta, czy mogą użyć drągu do szyn -- ktoś nie może chodzić, trzeba go zabrać. Wzdycha, dobrodusznie wzrusza ramionami i~go oddaje.

\bigskip
\noindent \texttt{16:45}\medskip
 

 Zaczynamy mocno obrywać.

 Kitty, stojąca jakieś trzydzieści jardów od ściany, zostaje uderzona w~stopę kanistrem z~gazem łzawiącym. Podbiega ekipa medyków, zdejmuje but, potwierdza, że  nic nie jest złamane. Mimo to boli jak diabli, a potem przez jakiś czas kuleje. Kitty nie ma maski przeciwgazowej, tylko dwie lub trzy bandany nasączone octem. Nieco z~przodu Craig zostaje uderzony w~żebra i~zwija się w~rozdzierającym bólu. Podczas badania medycy proszą wszystkich w~okolicy, aby utworzyli wokół niego krąg w~celu ochrony. Początkowo myśleliśmy, że został trafiony jakimś kołkiem lub drewnianą kulą, ale okazuje się, że jest to kolejny kanister z~gazem łzawiącym, taki, który został wystrzelony w~powietrze, ale w~jego przypadku został wystrzelony bezpośrednio w~niego. Najwyraźniej złamał kilka żeber dokładnie w~tym samym miejscu w~A16 rok wcześniej -- stąd agonia. Ludzie podbiegają z~wodą, próbując pomóc. W końcu czterech ludzi go wynosi.

 
\bigskip
\noindent \texttt{17:22}\medskip
 
 Cofam się, żeby sprawdzić Refugees, którzy w~większości zatrzymują się z~powodu braku masek przeciwgazowych.

 Wielkim pytaniem w~tym momencie były kierunki odwrotu. Pamiętając uwagi Stacey o strefach śmierci, przyszło mi do głowy, że drogi ucieczki będą miały coraz większe znaczenie. Zwłaszcza że  obiecaliśmy, że będziemy starać się trzymać z~dala od St. Jean Baptiste i~nikt, z~kim rozmawiałem, nie był do końca pewien, jak moglibyśmy odejść, gdyby ponownie spróbowali odciąć René Lévesque, a my nie moglibyśmy po prostu cofnąć się tą samą drogą, którą przyszliśmy. Wszyscy zgadzamy się, że w~miarę zbliżania się ceremonii otwarcia Szczytu będzie to stawało się coraz ważniejsze. Najwyraźniej nie będą w~stanie przeprowadzić ceremonii, gdy wielka bitwa toczy się dwadzieścia metrów dalej i~wszędzie gaz łzawiący, policja zwiększa liczbę i~prawdopodobnie przygotowuje się do dużego uderzenia, aby odepchnąć nas przynajmniej, co uważają, na rozsądną odległość. Staramy się znaleźć wolną przestrzeń do oglądania map, ale mapy, które mamy, są nieczytelne, ponieważ nie pokazują wysokości, więc nie mamy pojęcia, czy to, co wygląda na przejście, jest w~istocie klifem.

 Wygląda na to, że kolektyw Barricada z~Bostonu zajął północny kraniec parku. Jest jedna zamaskowana postać, całkowicie w~czerni, stojąca na podstawie pustej fontanny w~pobliżu kilku dużych kolonialnych budynków, które wyznaczają północną krawędź, wyglądająca jak sobolowy paw, gdy obserwuje akcję poniżej. Podciągam maskę i~pytam: 
 
 -- Czy te ulice przechodzą tutaj z~tyłu?

-- Nie wiem. Czemu?

-- Po prostu martwię się, że zostaniemy odcięci, jeśli przeniosą się na tę stronę parku.

-- Dlaczego nie sprawdzisz?

 Spędzam trochę czasu na badaniu. Rzeczywiście są klify, w~niektórych miejscach, a przynajmniej bardzo urwiste odcinki z~głazami (był to też jeden z~nielicznych obszarów wciąż pokrytych brudnym śniegiem), ale także schody i~kilka ulic, które wyglądają na wystarczająco szerokie, że trudno sobie wyobrazić, aby ktoś mógł je zamknąć. Nawet klify wyglądają na takie, na które można się wspinać. Wygląda więc na to, że nie będzie problemu.

 Głośne eksplozje rozbrzmiewają, gdy używany jest nowy, jeszcze bardziej paskudny gaz łzawiący. Krążyły też uporczywe plotki, że policja zamierza przywieźć psy bojowe. Krótko mówiąc, rzeczywiście widzę jednego, owczarka niemieckiego na smyczy, na półce zajmowanej przez policję daleko w~oddali. To jedyny, którego zauważam.

-- Nic dziwnego, że nie wykorzystują psów -- zauważa ktoś. -- Gdyby wypuścili psa w~to wszystko na dłużej niż kilka minut, prawdopodobnie udusiłyby się w~gazie.

 Ktoś inny wzdycha filozoficznie. 
 
 -- Wiesz, rzuciłem palenie rok temu. Teraz jeden dzień prawdopodobnie wyrządzi szkody, jakie prawdopodobnie miałoby dziesięć lat palenia.

-- To właśnie dostajemy, próbując walczyć z~zanieczyszczeniem.

\bigskip
\noindent \texttt{17:40, Schodzę Po Kawę na Côte D'Abraham}\medskip


 Mac zmierza w~dół wzgórza, aby spotkać się z~Lesley i~kilkoma przyjaciółmi na przerwę na kawę na Côte d'Abraham na północy, na skraju Zielonej Strefy. Zapewnia mnie, że rzeczywiście są tam otwarte kawiarnie. Czy chciałbym pójść z~nimi? Odnajduję większość pozostałych Refugees, którzy uważają, że nie zaszkodzi trochę czasu na oczyszczenie płuc.

 W rzeczywistości Côte d'Abraham w~niczym nie przypomina zamkniętych przestrzeniami René Lévesque (która była przecież, jak nas ostrzegano, ,,ulicą burżuazji''). Tu wszystko jest otwarte: sklepy, restauracje, co najmniej kilkanaście ulicznych kawiarenek. Protestujący kłębią się w~grupach. Niektórzy mają maski przeciwgazowe naciągnięte jak średniowieczne hełmy, większość ma chustki owinięte wokół szyi i~brzęczący sprzęt akcji takiego czy innego rodzaju: plecaki, gogle, butelki z~wodą, liny i~chwytaki, lornetki lub głupie maski i~rekwizyty do teatru ulicznego przyklejone taśmą do pleców. Trudno było ich postrzegać jako coś innego niż przypadkowy tłum lub w~najlepszym razie meandrujące zespoły, ale pod spodem kryła się cała niewidzialna architektura organizacji -- kolektywy, klastry, bloki, grupy afinicji. Staram się wyobrazić sobie, jak by to wyglądało, gdyby jakoś wszystkie te organizacje stały się widoczne: ulice nagle podświetlające się tysiącami kolorowych linii, kół, grafów.

 U stóp stromej, brukowanej ulicy znajduje się dramatyczny, uderzająco piękny kościół. Przed nim Lesley rozmawia z~kimś z~\textit{MacLean's}, jednego z~bardziej popularnych kanadyjskich magazynów.

-- Hej, David -- pyta -- chcesz porozmawiać z~reporterem?

-- Jasne.

 Kobieta jest po trzydziestce, ma na sobie kurtkę z~frędzlami i~nosi podkładkę. Jest wesoła, entuzjastyczna, a nawet zuchwała. Czuję się, jakbym miał do czynienia z~gościem z~innego świata.

-- David Graeber? Czy twój ojciec nie jest profesorem w~Yale? Jest jakimś anarchistą, prawda? Czytałam o nim w~ostatnim numerze \textit{Montréal Gazette}.

-- Nie, to ja, jestem profesorem w~Yale.

-- Czy mogłabym zadać ci kilka pytań?

-- Yyy \ldots  nie. To znaczy tak, jasne. Nie mam nic przeciwko. Lecisz.

-- Cóż, niedawne badanie wykazało, że większość obywateli Kanady w~rzeczywistości opowiada się za wolnym handlem. Dla mnie rodzi to wiele pytań o to, jak bardzo możesz twierdzić, że reprezentujesz ,,społeczeństwo'' w~takich protestach.

 Absolutnie nie mam pojęcia, o czym mówi: jakiego rodzaju ankieta, jak sformułowano pytanie, jakie były odpowiedzi na inne pytania. Nawet myślenie o tym sprawia, że boli mnie mózg. Rozważam poruszenie kwestii tego, co, tak czy inaczej, ma oznaczać słowo ,,wolny handel'', jak jest to naładowane określenie, jak nawet bym się wahał, gdyby ktoś mnie zapytał, czy jestem przeciwny wolnemu handlowi. Ale to jest bardziej skomplikowane, niż jestem w~stanie w~tej chwili wyrazić. Zamiast tego staram się wykazać, że fakt, że rząd celowo stara się utrzymać treść traktatu w~tajemnicy, pokazuje, że nie wierzą, by opinia publiczna zaakceptowała go, gdyby miała jakiekolwiek pojęcie o tym, co tak naprawdę oznacza. Przynajmniej to chciałem powiedzieć. Odchodzę z~wyraźnym wrażeniem, że właśnie wypadłem jak bełkoczący idiota. Uderza mnie też, że przynajmniej teraz rozumiem, dlaczego protestujący przeciwko globalizacji, z~którymi przeprowadzano wywiady w~telewizji, prawie niezmiennie wyglądają jak bełkoczący idioci. Normalnie jestem całkiem dobrym dyskutantem. W istocie można powiedzieć, że jako profesor, być w~stanie brzmieć inteligentnie, nawet, aby płynnie odpowiadać na niespodziewane pytania, jest czymś, czym zarabiam na życie. Jeśli nie potrafię ułożyć spójnego zdania niewyspany, po wyjściu z~dwugodzinnej wojny chemicznej, to jak u licha mogą tego oczekiwać od kogokolwiek innego?

 Mac i~Lesley znowu zniknęli. Reszta z~nas zbiera się, popijając cappuccino w~maleńkiej restauracji, w~której nawet kelnerzy wciąż mają zawiązane bandany na szyjach. Właściciel rozdaje darmowe butelki wody każdemu, kto wygląda, jakby wrócił z~frontu, a aktywiści nieustannie wchodzą i~wychodzą z~łazienki, aby umyć chusteczki, oczy i~twarze.

-- Ostrożnie -- od czasu do czasu mówi właściciel po francusku. -- Pamiętaj, że jeśli zamoczysz ubrania, gaz łzawiący wyjdzie ponownie. Pamiętaj, że jest też we włosach\ldots 

 Wszyscy mają jedno pytanie. Ktoś musi o to zapytać.

-- Więc -- mówię -- co się stało? Jak wygraliśmy? To znaczy, tak szybko. W zeszłym miesiącu w~consulcie wszyscy zakładaliśmy, że będziemy musieli przebić się przez tysiące gliniarzy, aby w~ogóle dostać się do ściany.

 Ogólne wrażenie jest takie, że nie obliczaliśmy dobrze.
 
-- W końcu -- zauważyła Heidi -- kiedy mówią, że będą ,,trzy tysiące gliniarzy'', nie oznacza to, że wszyscy będą pełnili służbę w~tym samym czasie. Nawet jeśli mają potrójne nadgodziny, może tylko połowa z~nich będzie miała dyżur w~danym momencie. Dodatkowo muszą utrzymywać rezerwę strategiczną. Więc masz może tysiąc gliniarzy do obrony siedmiokilometrowego obwodu, a także robi wszystko, co zwykle gliniarze robią.

-- Podczas gdy nasze siły były skoncentrowane w~jednym punkcie.

 Dużą wiadomością na ulicy jest to, że Jaggi został już aresztowany, nieuchronnie. Ktoś przy sąsiednim stoliku zna wszystkie szczegóły. Najwyraźniej nigdy nie zbliżył się do muru, ale zawrócił z~marszem Zielonych. Godzinę temu kręcił się z~kilkoma innymi organizatorami na Côte d'Abraham, kiedy kilku cywilów przebranych za protestujących złapało go od tyłu. Jego przyjaciele -- w~tym najwyraźniej kilka kobiet, które były kofacylitatorami w~radzie delegatów -- próbowali interweniować i~prawie udało im się odciągnąć go z~powrotem, gdy tamci wyciągnęli pałki i~przedstawili się jako policja. Potem potrząsnęli nim i~wrzucili na tył czarnego SUV-a. Odjechał i~to był ostatni raz, kiedy ktokolwiek go widział.

-- Jakieś wieści z~marszu GOMM Green? -- pytamy naszych nowych przyjaciół przy sąsiednim stoliku.

 Ktoś się uśmiecha. 
 
 -- Historia, którą słyszałem, jest taka, że  wszyscy usiedli przed murem w~pobliżu autostrady, migając symbolami pokoju. Oczywiście policja zaczęła ich gazować jak wszystkich innych. Ktoś zaczął wyrzucać z~powrotem gaz łzawiący i~niedługo potem zburzyli też swoją część ściany. 
 
 -- Poszli na czerwono?

-- Spontanicznie.

-- Chwileczkę -- mówi przy innym stoliku kobieta w~średnim wieku w~okularach w~rogowej oprawie. -- Słyszałam o odrzucaniu gazu łzawiącego. Ale jestem prawie pewna, że nie zaatakowali muru przy autostradzie. W każdym razie przechodziłam obok niecałą godzinę temu, a ogrodzenie wciąż tam było.

-- Byłem tam, kiedy to się stało -- mówi ktoś inny. -- To, co się stało, to\ldots  tak, ktoś zaczął odrzucać gaz łzawiący. Ale prawie jak tylko zaczęli to robić, pojawił się jakiś przywódca z~megafonem i~ogłosił, że osiągnęli swój punkt i~że akcja się skończyła, i~wszyscy wycofali się do Zielonej Strefy.

\bigskip
\noindent \texttt{18:30, Powrót do Ground Zero}\medskip


 Gdy Refugees wracają do muru, cały ruch wydaje się zmierzać w~przeciwnym kierunku. Być może siedem osób dryfuje w~dół na jedną osobę wracającą pod Mur. Mijamy bębniarzy ważek w~małym kółku na środku ulicy. Próbują zmobilizować ludzi, ale niezbyt skutecznie. Kiedy docieramy na szczyt, powód staje się oczywisty: falangi policji zajmują środek parku, a mniejsze eskadry systematycznie zajmują pozycje na każdej ulicy dojazdowej, dławiąc dostęp, a potem szaleńczo gazują wszystkich w~zasięgu wzroku. Linie prewencji posuwają się systematycznie naprzód, po dziesięć czy dwadzieścia metrów na raz. W końcu zaczynają poruszać się trzema głównymi ulicami północ-południe -- Turnbull, Claire-Fontaine i~Sainte-Claire -- które prowadzą w~dół wzgórza do St. Jean Baptiste.

 Nie wygląda na to, że próbują dokonywać masowych aresztowań. Przynajmniej jeszcze nie. Po prostu próbują oczyścić teren.

 Flaga wielu narodów i~kilka czarnych flag anarchistów znajduje się już u podnóża wzgórza, wzdłuż St. Jean, a jedyną możliwą grą, jaka pozostała, to opóźnienie marszu policji. Nie wiadomo, gdzie jest Czarny Blok. To samo z~Czerwonym Blokiem: nikt w~tym tłumie nawet nie myślał o rzucaniu kamieniami. Raczej to kwestia siedzenia na ulicach, śpiewania piosenek i~czekania na atak. Proste, uparte nieposłuszeństwo obywatelskie.

\bigskip
\noindent \texttt{18:55, Avenue Turnbull}\medskip


 Na początku ulicy, na wzgórzu, znajduje się około dziesięciu do dwudziestu Darthów Vaderów, którzy wyłaniają się z~niespokojnej mgły ich własnego stworzenia, przygotowując się do zejścia na nas. Stopniowo grupa z~nas gromadzi się wzdłuż Lockwell Street i~decyduje się pomaszerować, aby się im przeciwstawić. Przedzieramy się przez mgłę -- częściowo prowadzeni przeze mnie, ponieważ jestem jednym z~nielicznych z~maską gazową -- i~siadamy na ulicy, a Shawn i~Lyn podążają za nimi z~minidyskami, aby upewnić się, że każdy dźwięk jest nagrany. Młoda kobieta z~megafonem pyta, czy ktoś ma kopię ,,Karty Praw i~Wolności'' z~kanadyjskiej konstytucji (obserwatorzy prawni rozdawali je przed akcją).

-- Wydaje mi się, że mam gdzieś jedną w~mojej torbie -- mówi Shawn. Lyn również wyciąga kopię.

 Siedzimy na bruku, około trzydziestu lub czterdziestu osób. Zdejmuję maskę gazową. Zauważyłem, że znajdujemy się w~samym środku dzielnicy czysto mieszkalnej. Kobieta z~megafonem, ubrana w~zamszową kurtkę i~bez żadnego sprzętu, rozkłada gazetę i~rozpoczyna dramatyczną recytację rozdziału dotyczącego wolności słowa i~wolności zgromadzeń. Dziennikarka radiowa IMC wyciąga swój mikrofon tuż obok kobiety, klękając z~dramatycznie wzniesioną ręką. Za nami dostrzegam kilka kamer wideo skupionych na policji.

 Oczywiście wiedzieliśmy, że nas zagazują.

 Zaledwie dwadzieścia czy trzydzieści metrów od policyjnych pozycji po raz pierwszy rzeczywiście mogliśmy spojrzeć im w~oczy i~zobaczyć ich twarze. Większość z~nich nie miała masek przeciwgazowych -- prawdopodobnie dlatego, że wiedzieli, że będą strzelać z~daleka i~w dół wzgórza. Wszyscy patrzyliśmy jak sparaliżowani, jak jedna z~policjantek o prostej, nieszkodliwej twarzy i~blond włosach mocno ściągniętych za wizjer, wyciągnęła wyrzutnię i~zaczęła celować.

 Ludzie zaczęli do niej wołać:
 
-- Nie rób tego! Proszę! Nie gazuj nas!

-- To jest pokojowe zgromadzenie!

-- Nie jesteśmy twoimi wrogami. Proszę, nie strzelaj!

 Potem strzeliła. Kanister przepłynął kilkanaście centymetrów obok trzymanego mikrofonu i~eksplodował bezpośrednio za nami.

 W ciągu kilku sekund to był ostrzał. Osiem, dziewięć, dziesięć puszek wirowało wokół nas, eksplodując w~płomieniach, rozsypując się wszędzie. My też się rozproszyliśmy. Młoda kobieta z~megafonem ruszyła powoli, wyzywająco do tyłu, a potem odwróciła się przez ramię i~po raz ostatni podniosła megafon.
 
  -- Chcę tylko zaznaczyć, że właśnie złamaliście prawo!

 Kolejny gaz łzawiący wylądował około pół metra od niej, a następnie obrócił się, wystrzeliwując płomienie. Inny uderzył w~czyjeś okno tuż nad nami, gdzie, o ile wiemy, jakaś rodzina właśnie zasiadała do kolacji. Cały obszar zamienił się w~chmurę CS.

 To było, jak zauważył Shawn, pierwsze użycie gazu łzawiącego, które widzieliśmy w~dzielnicy mieszkalnej.

 Wkrótce wracamy na Côte, gdzie powiewa flaga Wielu Narodów. Ktoś nam mówi, że podczas gdy na Turnbull policja rozpraszała się, gazując pacyfistów i~okolicznych mieszkańców, na pobliskiej bocznej uliczce -- Burton, a może Claire-Fontaine -- pojawiło się kilka grup afinicji z~Czarnego Bloku, myśląc o zrobieniu czegoś manewr oskrzydlający i~odkryło trzy puste czarne SUV-y całkowicie niestrzeżone. Były to pojazdy używane przez oddziały przechwytujące, całkiem możliwe, że te same, których wcześniej użyto do złapania Jaggi. Zatem wybili okna i~zabrali dziesiątki plastikowych tarcz i~innych materiałów, w~tym kilka dokumentów dotyczących taktyki formacji gliniarzy.

 Shawn i~Lyn, wciąż dysząc z~gazu, udają się na poszukiwanie swojego samochodu, który, jak przypuszczają, zostawili gdzieś w~odległości spaceru poprzedniej nocy. i~tak wszyscy mamy się spotkać za godzinę lub dwie, z~powrotem w~Laval.

\bigskip
\noindent \texttt{19:27, Wzdłuż St. Jean}\medskip


 Do tej chwili, istnieje silne poczucie, że wszystko się kończy. Słyszymy, że ceremonia otwarcia została opóźniona do godziny 22.00 (okazuje się to nieprawdą: faktycznie zaczęła się dziesięć minut później, o 19:30, ale mimo to opóźniła się o kilka godzin).

 Kolejny atak gazowy: ten dość blisko nas. Płonące kanistry wirują aż do miejsca, w~którym zgromadziliśmy się w~St. Jean, na małym skrzyżowaniu w~pobliżu opuszczonej parceli. Ludzie spływają tą samą ulicą. Niektórzy Refugees wychodzą i~rozkładają ręce, aby zapobiec panice, ale nie ma sensu utrzymywać pozycji. Jeden młody człowiek z~czerwoną flagą próbuje atakować, prawie sam. Wkrótce znów musi się wycofać. Inny facet z~sowiecką flagą Quebecu (!) -- w~połowie fleur de lis, w~połowie sierp i~młot -- umieszcza ją obok Flagi Wielu Narodów. Jakiś koleś z~CLAC z~megafonem próbuje zebrać wszystkich w~okolicy po francusku. Mała grupa ciągnie śmietnik na środek skrzyżowania i~wznieca w~nim ogień. Jest to płaska i~dobrze wentylowana przestrzeń, tuż nad kolejnym stromym zboczem; wydaje się równie dobrym miejscem, jak każde inne, aby spróbować się obronić. Zauważam też, że kilka osób kręcących się wokół śmietnika nie wygląda na aktywistów, ale na lokalnych mieszkańców, wkurzonych gazem łzawiącym. i~na pewno nas nie obwiniają. Jeden z~ludzi CLAC tłumaczy im, że ogień wypali resztki gazu łzawiącego.

 Po chwili inna osoba z~CLAC -- wysoki facet o długich, brązowych, potarganych włosach -- zwraca się do mnie. Pamięta mnie z~Consulty.

-- Jedziemy do Laval: jest rada przedstawicieli -- mówi. -- Czy chciałbyś przyjść?

-- O tak. Właściwie to tam mam za kilka minut spotkać się z~resztą mojej grupy afinicji. Jak się tam dostaniesz?

-- Autobusem.

 Wyjeżdżam z~zespołem CLAC, jednym mężczyzną i~dwiema kobietami, ale zanim tam dotrzemy, postanawiają najpierw zatrzymać się na piwo. Czy chciałbym pójść z~nimi? Rozważam to, dociera do mnie, że jestem kompletnie wykończony. Kierują mnie więc na przystanek autobusowy i~po miłej pogawędce z~zaprzyjaźnionym reporterem \textit{LA Times }na sąsiednim miejscu, docieram do Laval.

\bigskip
\noindent \texttt{20:07, Głupie Małe Rady}\medskip


 W sali, na której na ścianach wiszą wszelkiego rodzaju flagi, mieści się może ze dwieście osób, ale w~spotkaniu bierze udział w~najlepszym razie tylko połowa. Wkrótce rozumiem dlaczego. Rozmowa przerodziła się w~kolejny spór o różnorodność taktyk. Niektórzy gorzko narzekają na rzucanie kamieniami, inni twierdzą, że był to jedyny sposób radzenia sobie z~masowymi atakami policji. Wydaje się, że nikt nikogo nie słucha, nie mówi o planach na następny dzień (a może to ma być później? Nie widzę agendy na ścianie). Cała Rada Delegatów wydaje się po prostu okazją dla ludzi, by zabrzmieć.

 Większość Refugees jest już w~pokoju lub w~pobliżu, wyleguje się, bawi się swoimi minidyskami i~ogląda obrazy akcji z~kamer wideo innych ludzi. Zameldowałem się i~wszyscy zgadzamy się wrócić do domu za półtorej godziny.

 W tym momencie odkrywam też, że nie jestem już jedynym członkiem NYC Ya Basta! w~Quebecu. Laura, kobieta z~Włoch i~studentka CUNY, właśnie przyjechała z~ładunkiem Yabbas -- to znaczy Yabbas prawdziwej, włoskiej odmiany: Beppe, Sandry i~Roberto. Laura zaczyna się śmiać, gdy mnie widzi. Podbiega, żeby mnie długo uściskać. 
 
 -- Ha! To jest takie idealne! Tak wspaniałe! Wszyscy wielcy pragmatyczni ludzie czynu w~Ya Basta!, ani jeden się nie przedostał. Wszyscy zrezygnowali. A kto właściwie wprowadza to w~życie? Tylko ty i~ja. Dwóch intelektualistów!

 Jej przyjaciółki są ubrane we wspaniałe włoskie garnitury. 
 
 -- To był jedyny sposób, w~jaki mogliśmy się przedostać -- wyjaśnia mi radośnie Roberto.

-- Tak -- powiedziała Laura. -- Kiedy próbowaliśmy przejechać przez odprawę celną, mężczyzna zapytał, dokąd jedziemy. Powiedzieliśmy, że do Quebec City. Następnie zapytał o cel naszej podróży, a Beppe powiedział ,,turystyka''. Zaczął więc przeglądać paszport Beppe, patrząc na pieczątki. ,,Hmmm\ldots Genewa, czerwiec 1999; Seattle, listopad 1999; Praga, wrzesień 2000. Więc przypadkiem pojawiałaś się na każdym większym proteście na szczycie globalizacji w~ciągu ostatnich trzech lat? Nie sądzę''.

-- Więc co zrobiliście?

-- Wszystkie zaczęłyśmy na niego krzyczeć: ,,JESTEŚMY WŁOSKIMI OBYWATELAMI, PRÓBUJĄCYMI ODWIEDZIĆ KANADĘ! JAK ŚMIESZ? MYŚLISZ, ŻE KIM JESTEŚ? NIE BĘDĘ TAK TRAKTOWANA! NAZWISKO DOWÓDCY? JAKI JEST NUMER TWOJEJ ODZNAKI? ZADZWONIMY DO KONSULATU WŁOSKIEGO i~ZŁOŻYMY OFICJALNĄ SKARGA! ZROBIMY Z TEGO MIĘDZYNARODOWY KURWA INCYDENT!'' i~w końcu po prostu się wycofał.

-- Mówisz, że to faktycznie zadziałało?

-- Garnitury pomogły.

 Jedyną rzeczą, która naprawdę mnie martwi, jest to, że nikt nie słyszał nic od Karen. Byłam prawie pewna, że  wyjaśniliśmy jej, jak ważne jest upewnienie się, że inni członkowie twojej grupy afinicji znają twoje miejsce pobytu -- a przynajmniej informacja, zanim po prostu opuścisz miasto. W każdym razie wydawało się to podstawowym zdrowym rozsądkiem. Znajduję miejsce, aby sprawdzić pocztę. Nic. Moja komórka nie działa, wszystkie moje numery są niedostępne (na przykład Saszy), ale pożyczam telefon, z~którego mogę sprawdzić wiadomości. Nic. Oczywista implikacja jest taka, że  została aresztowana, co jest zarówno możliwe (mówiono mi, że biorą na cel niezależnych dziennikarzy), jak i~niepokojące (ponieważ nie ma pojęcia, co robi). Próbuję sobie przypomnieć: czy w~ogóle upewniliśmy się, że zapisała na ramieniu numer do prawników? Tak. Zrobiliśmy to w~IMC. Znowu pożyczam komórkę i~dzwonię do działu Prawnego. Dostaję tylko sygnał zajętości. Dzwonię do IMC. Brak informacji.

 Wreszcie się poddaję. Włoszki mają samochód i~zapraszają mnie na krótką przejażdżkę, aby obejrzeć akcję. Skończyliśmy na zwiedzaniu Górnego Miasta, mijając René Lévesque i~Pola Abrahama, oglądając od czasu do czasu nocne bitwy -- w~pewnym momencie byłem prawie pewien, że widziałem, jak ktoś rzuca w~oddali mołotowem. Jakoś, po dziesięciu minutach, wszyscy śpiewamy:

\noindent \textit{Riot riot - I wanna riot \newline Riot riot - a riot of my own \newline Riot riot - I wanna riot \newline Riot riot - a riot of my own}

 (Wszyscy, nie zauważając, porzuciliśmy ,,białą'' część.) Myślę, że właściwie to zacząłem. Co jest nietypowe, ponieważ nie umiem zaśpiewać nuty. Nie, żeby miało to duże znaczenie w~przypadku Clash.

-- Ach -- wzdycha Roberto, którego angielski nie jest płynny. -- Nawet jeśli ledwo możemy ze sobą rozmawiać, wszyscy znamy te same piosenki.

\section{Sobota, 21 kwietnia}

 Dotarliśmy do domu około północy i~odkryliśmy, że Janna już tam jest. Okazuje się, że jest przyjaciółką Lyn.

 Jednym ze skutków medycznych doświadczeń Janny jest to, że stała się kimś w~rodzaju eksperta od skutków ,,nieśmiercionośnej'' broni chemicznej. Ubrania, wyjaśniła nam, pochłaniają toksyny. Ważne jest, aby bardzo dokładnie wyprać wszystko, co mamy na sobie, zanim weźmiemy prysznic, bo gdy następnym razem zmokniemy, będzie tak samo źle, jak za pierwszym razem. Moje ubrania były wyraźnie przesiąknięte różnego rodzaju toksynami. Z drugiej strony nie miałem torebek, a więc nic, w~co mógłbym się przebrać. Skończyło się na wędrowaniu po domu nago o drugiej w~nocy, podczas gdy wszyscy inni spali, robiąc pranie w~pralce w~na wpół wykończonej piwnicy. Moje swetry nie nadawały się do prania, ale na szczęście większość z~nich mogłem zostawić, bo według wszystkich doniesień niedziela miała być jeszcze cieplejsza niż dzień wcześniej.

 Potem złapałem dobre sześć lub siedem godzin snu -- rzadki luksus na dzień pełen akcji.

 Następnego ranka na śniadanie Heidi przyniosła rogaliki, \textit{pain au chocolate} i~kopię każdej lokalnej gazety. Znalazła też kilka zagranicznych. Podawaliśmy je sobie, oglądając wiadomości telewizyjne, które w~nieskończoność odtwarzały najważniejsze momenty piątkowych marszów i~konfrontacji. Relacja była niesamowita ze względu na szczegóły. Były takie nagłówki, o jakich marzą amerykańscy aktywiści medialni, takie, jakich nigdy nie zobaczycie w~USA za milion lat: ,,MUR UPADA!'' ,,ŁZY DEMOKRACJI'' (to ostatnie odnosi się do reakcji ludzi na gaz łzawiący) i~tak dalej.

Dostępne nam informacje były mylącą mieszanką plotek, doniesień prasowych, plotek podawanych w~wiadomościach i~oficjalnych oświadczeń policyjnych -- prawie wszystkie z~nich można było założyć, że są zasadniczo nieprawdziwe. W piekarni Heidi słyszała, że  grupa osiemdziesięciu zakonnic, rozwścieczonych gazem, przygotowuje się do marszu na główny punkt kontrolny, by zburzyć mur. Telewizja poinformowała w~piątek o zaledwie pięćdziesięciu aresztowaniach, ale Ben i~Lyn, którzy rozmawiali przez telefon z~kimś z~IMC, słyszeli znacznie wyższe liczby: w~tym 126 w~obławie zaledwie kilka godzin wcześniej (obie liczby okazały się szalenie niedokładny). W piątek wieczorem policja zorganizowała konferencję prasową, ogłaszając, że specjalna operacja złapała ,,przywódcę Czarnego Bloku'' -- co oczywiście oznaczało Jaggiego. Obecne miejsce pobytu Jaggiego było nieznane. (Kilka dni później policja przyznała, że go przetrzymuje, oficjalnie został oskarżony o ,,nielegalne posiadanie katapulty'').

 Nawet prasa amerykańska była znacznie lepsza niż zwykle:
\bigskip

\noindent \textit{ Protestujący przejmują dzień w~Quebecu Wrogowie handlu zagazowani łzami na Szczycie Ameryk}

 Dana Milbank \newline \textit{Washington Post}, dziennikarka, sobota, 21 kwietnia 2001 roku.
 
\noindent QUEBEC CITY, 20 kwietnia -- prezydent Bush i~33 innych przywódców zachodniej półkuli, którzy chcą zbudować największą na świecie strefę wolnego handlu, otworzyli dziś spotkanie na szczycie, gdy chmury gazu łzawiącego i~gwałtowne demonstracje spustoszyły plany i~opóźniły spotkania.

 Bush pozostał w~swoim hotelu, ponieważ ceremonia otwarcia szczytu była opóźniona o ponad godzinę. Został zmuszony do odwołania jednego spotkania i~odroczenia lub skrócenia innych, ponieważ ruchy głów państw wokół Québec City były utrudnione przez protesty antyglobalistyczne.

-- Jeśli protestują z~powodu wolnego handlu, powiedziałbym, że się nie zgadzam -- powiedział Bush. -- Myślę, że handel jest bardzo ważny na tej półkuli. Handel nie tylko pomaga szerzyć dobrobyt, ale handel pomaga szerzyć wolność.

 W holu hotelu Loews panowało zamieszanie, gdy pomocnicy Busha starali się śledzić zmieniający się harmonogram, oglądając zamieszki w~telewizji. Prezydent Kolumbii Andres Pastrana czekał na opóźnienia w~barze koktajlowym\ldots 

 Krążyły pogłoski, że zgromadziła się już ogromna liczba: dwadzieścia pięć tysięcy w~Starym Porcie, u samego podnóża miasta, by rozpocząć marsz robotniczy i~Szczyt Ludowy; grupa studencka zbierająca się na Równinach Abrahama; ogromne liczby w~Laval. Wszyscy, łącznie z~gazetami, mówili o rozmiarze wydarzenia: po prostu nie ma mowy, żeby policja sobie z~tym poradziła. Zgadzamy się, że wielką niewiadomą będzie marsz robotniczy. Organizatorzy, jak można było przewidzieć, ustawili to tak, że każdy zaczyna się dziesięć czy piętnaście przecznic od obwodu, a potem maszeruje w~przeciwnym kierunku, by skończyć na wiecu na jakiejś odległej działce. Pytanie brzmi, czy szeregowym manifestantom to wystarczy. Wiemy, że zarówno CLAC, jak i~NEFAC (Northeastern Federation of Anarcho-Communists -- Północno-Wschodnia Federacja Anarcho-Komunistów, grupa anarchistyczna zorientowana na związki zawodowe) będą miały tam ludzi, próbując odciągnąć ludzi do muru.

\bigskip

 Porównując notatki, staramy się też ułożyć więcej obrazu tego, co musiało się wydarzyć wczoraj. O Akwesasne, postanawiamy, że nigdy nie ustalimy, dopóki nie zdobędziemy więcej informacji. Oczywiście coś spieprzyło się w~wielkim stylu. Shawn chce wiedzieć: dlaczego do cholery spóźniliście się cztery godziny? Połowa Wojowników już wyszła. Szczerze nie mogę mu powiedzieć. A czym był ten idiotyzm ,,wszystko albo nic''. Ironią piątkową było to, że kiedy wszyscy byliśmy w~Akwesasne, jedząc ryby, ludzie z~CLAC w~radzie prasowej niemal wpadli w~panikę, że Karnawał Przeciw Kapitalizmowi zakończy się fiaskiem. Parada z~pochodniami i~akcja kobiet były piękne, ale stosunkowo niewielkie. Nikt nie miał pojęcia, ile osób pojawi się w~sobotę. Dlatego w~sobotę rano było tyle radości z~cyferek.

 Janna spóźnia się, pociąga nosem, w~nocnej koszuli -- przynajmniej przeszła z~całym bagażem. Mówi, że większość piątku spędziła w~Zielonej Strefie, której centrum znajdowało się poniżej autostrady u podnóża wzgórza, i~przez chwilę dostrzegła Żywą Rzekę.

-- Och, racja, Niebieski Blok! Zastanawiałem się, czy ci ludzie w~ogóle się przebili.

-- Byli w~porządku, właściwie kilkaset z~nich. Widziałem ich na St. Jean, niedługo po tym, jak usłyszeliśmy, że mur runął. Mieli całą złożoną organizację z~czterema flagami, z~których każda przedstawiała jeden z~czterech żywiołów, zielony dla ziemi, niebieski dla wody, czerwony dla ognia i\ldots  czy to był biały dla powietrza? Nie, myślę, że był żółty. Starhawk była tam z~małym bębenkiem, a oni zatańczyli spiralny taniec i~wezwali moc rzeki, aby umieścić żywioły po naszej stronie.

 Sam wygląda podejrzanie, jakby starał się nie mamrotać do kawy czegoś cynicznego.

-- Cóż -- zauważam -- jak na to, co było warte, wczoraj mieliśmy wyjątkowo dobrą pogodę.

-- Tak, wiatr był cały czas na naszych plecach -- powiedziała Lyn. -- Widziałeś, jak cały czas wydmuchiwał gaz łzawiący z~powrotem na policję? Zwłaszcza na początku, kiedy strzelali tuż przed nimi, wszystko po prostu spływało z~powrotem na ich twarze.

-- Ziemia jest po naszej stronie -- powiedziała Janna. -- Naprawdę w~to wierzę.

-- Może powinniśmy zrobić znak, żeby zanieść do parku -- mówię -- ,,Wiemy, w~którą stronę wieje Wiatr''. 

\bigskip
\noindent \texttt{11:00, Orsainville}\medskip


 Wciąż martwiąc się o Karen, marnuję resztę poranka i~wczesne popołudnie na plan zaproponowany przez Heidi i~jej przyjaciółkę, producentkę \textit{Frontline} o imieniu Claudia, aby odwiedzić lokalne więzienie w~lesie kilka kilometrów za miastem. Claudia ma samochód. Jest już garstka aktywistów solidarności więziennych przed więzieniem, ale mają tylko ograniczoną listę osób, które są w~środku, i~nikt nie słyszał ani słowa o przetrzymywaniu tam kogokolwiek z~IMC lub innego niezależnego filmowca.

 Później ta garstka ma się rozszerzyć do istnej ,,Wioski Solidarności'', kiedy ludzie rozbijają namioty, sprowadzają żonglerów i~muzyków oraz tworzą ciągły rytm pieśni i~muzyki, aby upewnić się, że więźniowie wiedzieli, że tam są. Pojawi się oddział gliniarzy od zamieszek, a artyści z~megafonami będą opowiadać dowcipy i~próbować ich rozśmieszyć. Będzie wegetariańska kuchnia i~niekończąca się kolejka dziennikarzy. Ale nie teraz. Tylko około dwudziestu osób, które mają powody, by sądzić, że członkowie ich grup afinicji lub bliscy krewni są za kratkami, kilku przedstawicieli prawnych i~jedna dość żałosna para w~średnim wieku, martwiąca się o swoją szesnastoletnią córkę.

 Wszystko trwa dłużej niż powinno. Wreszcie, po maratonie sesji telefonicznej, Claudia mówi, że chce złapać końcówkę Szczytu Ludowego, który organizatorzy celowo umieścili daleko, z~dala od akcji, w~pobliżu portu, kilka kilometrów dalej. Parada miała wyruszyć w~południe, maszerując na szczyt; zdecydowanie to przeoczyliśmy. W każdym razie niechętnie wyjeżdżam tak daleko od miasta, nie wiedząc, jak wrócę. Proponuje, że wysadzi nas oboje w~IMC, gdzie Heidi musi zrobić audycję radiową. Umawiamy się tego wieczoru z~resztą naszej grupy na imprezie pod autostradą w~Zielonej Strefie, a ja ruszam w~stronę parku, żeby zobaczyć, czy uda mi się znaleźć La Resistance.

\bigskip
\noindent \texttt{15:20, Wreszcie W Mieście}\medskip


 Wszędzie jest graffiti: tysiąc kółek, ,,PIERDOL GLINIARZY'', ,,BEZ WYBORU'', ,,MURS BLANCS, PEUPLE MUET'', maski przeciwgazowe namalowane na twarzach każdego na wpół rozebranego modelu na reklamie na przystanku autobusowym, żaden billboard w~dowolnym miejscu nie był pozostawiony niezmieniony. Na poboczu autostrady, w~różnych miejscach:

\noindent QUI EST LE CHEF DU BLACK BLOC? \newline BRAMY NIEBA ZAJMIE SZTORM \newline Y'EN PAS EPAIS \newline WŁASNOŚĆ JEST KRADZIEŻĄ

 W kawiarni, ciągle tylko aktywiści. W ciągu pięciu minut znam większość historii dnia. Parada była ogromna: wiadomości mówią o sześćdziesięciu tysiącach ludzi, z~niekończącym się pokazem lalek, transparentów, pływaków i~przedstawień teatralnych. Zakończyła się zlotem z~przemówieniami Jose Bové, Maude Barlow i~wszelkiego rodzaju międzynarodowych celebrytów. 
 
 -- Czy ktoś się wyłamał, żeby podejść do ściany? 
 
 -- Cóż, nie w~tysiącach, nie, ale było wielu związkowców, którzy przynajmniej odwiedzili Płot. 
 
 Jedna kolumna kilkuset robotników samochodowych utworzyła grupy afinicji i~pomaszerowała do bramy gdzieś po wschodniej stronie obwodu i~została poważnie zagazowana. Wielu wciąż tam jest, myśli facet przy sąsiednim stoliku. W każdym razie będzie naprawdę ciekawie, myśli, kiedy rajd się skończy, zastanawia się, kiedy zlot się zakończy, ponieważ wielu uczestników mówi, że zamierza iść na imprezę przy autostradzie.

-- To Zielona Strefa, prawda? Ile Floriot? -- pytam, patrząc na mapę.

-- Tak, tam. Boulevard Charest Est. Widzisz, to ogromne skrzyżowanie sześciu różnych dróg? To niedaleko od IMC. 

Kilka minut później wznowiłem wspinaczkę w~kierunku Starego Miasta. Gliny gazowały cały dzień. Dosłownie unosi się gęsta chmura tego materiału, wisząca jak szkodliwa ściana nad St. Jean Baptiste i~rozciągająca się znacznie poniżej niej. Jedno jest jednak przyjemną zmianą: do tej pory lojalność otaczającej społeczności stała się całkowicie wyraźna. To było tak, jakby w~piątek nadal obserwowali, mierzyli, czekali, by zobaczyć, czy anarchiści naprawdę zniszczą miasto, jak obiecywały im władze federalne, czy gliniarze naprawdę ich zagazują, jak anarchiści powiedzieli, że to zrobią. Teraz już wiedzieli. Nikogo nie skrzywdziliśmy i~niczego nie uszkodziliśmy. Zrobiliśmy wszystko, co w~naszej mocy, żeby nie zrobić z~ich okolic pola bitwy. Policja odpowiedziała gazowaniem i~atakowaniem wszystkich bezkrytycznie, wystrzeliwując toksyny bezpośrednio na ich patio i~ogrody.

 Do sobotniego popołudnia połowa domów wywiesza jakiś baner lub znak: ,,Jesteśmy z~tobą!'', ,,Bez FTAA!'', a nawet raz ,,Wspieramy Czarny Blok'' (oczywiście po francusku). Wielu przyniosło też węże ogrodowe na swoje werandy lub wywieszało je z~okien, aby zapewnić protestującym darmową wodę. Babcie machają i~uśmiechają się z~ganków. Dzieci chichoczą i~chodzą za nami. To jak jakaś szalona anarchistyczna fantazja. Jedynym wyjątkiem, jak mijam, jest krępy mężczyzna w~średnim wieku, który wpada w~napad złości w~garstkę dzieciaków z~Czarnego Bloku przed swoim budynkiem, na samym końcu stromej ulicy prowadzącej do parku.
 
 -- Dlaczego wciąż tu jesteście? -- krzyczy: -- Rozumiem, że wczoraj zburzyliście mur, pokazaliście. To dobrze, wspieram was. Ale wystarczy! Nadal musicie walczyć z~glinami, nadal gazują, mój dom jest pełen gazu łzawiącego, od dwóch dni jest pełen gazu, musiałem odesłać małego synka do ciotki na przedmieściach, bo się nim dusił. Moja mama musiała opuścić swoje mieszkanie. Wystarczy! W tej chwili w~Dolnym Mieście odbywa się marsz robotniczy, w~telewizji mówi się, że maszeruje 60 000 ludzi. Dlaczego z~nimi nie maszerujesz? Dlaczego wciąż tu sprowadzasz na nas gaz? 
 
  Dzieciaki z~Czarnego Bloku wydają się zdenerwowane; wydaje się, że znają francuski na tyle, by go zrozumieć, ale niewystarczająco, by udzielić jakiejkolwiek elokwentnej odpowiedzi. Wreszcie trzech lub czterech sąsiadów zbiera się i~próbuje go uspokoić. 
  
-- To nie ich wina, chcą tylko upewnić się, że głowy państw usłyszą ich przesłanie.

-- Nie możesz oczekiwać, że wszyscy odejdą od miejsca, w~którym faktycznie spotykają się delegaci

-- To nie dzieci nas gazują -- nalega kobieta -- to policja.
 
\bigskip
\noindent \texttt{15:35, Ground Zero }\medskip


 Park jest znowu nasz, z~rozproszonymi grupami ludzi na placu, siedzącymi na ziemi, grającymi przedstawienia. Eksplozje gazu są okresowe, ale nie są tak intensywne jak poprzedniego dnia (lądują teraz mniej więcej raz na trzy minuty, mówi ktoś z~zegarkiem kieszonkowym). 

 Czarny Blok nie jest widoczny. Powiedziano mi, że przez większość dnia rozproszyli się w~małych grupach, atakując odsłonięte sektory obwodu. Jestem jednak rozczarowany, widząc, że fragment muru, który wczoraj rozebraliśmy, został odbudowane. Jest nowa, trochę kiepsko zbudowana brama. Tuż za nią umieścili armatkę wodną -- w~rzeczywistości całkiem sprytny ruch, ponieważ oznacza to, że nie możemy zbliżyć się na tyle, by zniszczyć tę rzecz. Wydaje się, że armatka wodna jest ustawiona na autopilota, wystrzeliwując ogromny pióropusz wody, który powoli przesuwał się tam i~z powrotem po łuku przestrzeni przed nim. To tak, jakby znaleźli zraszacz do trawnika, który pracował przy tysiąckrotnie wyższym ciśnieniu i~objętości. Jako broń defensywna była dość skuteczna. Skoordynowany atak na ten odcinek muru byłby teraz niemożliwy. Z drugiej strony, obecność pióropuszy wody -- nieważne jak intensywnej -- w~upalny dzień jest najwyraźniej zbyt wielką pokusą w~środku antykapitalistycznego karnawału. Ludzie podskakują i~robią z~siebie widowisko, pluskając się w~wodzie. Niektórzy padają z~nóg i~ślizgają się wesoło. Inni opierają się na nim i~pozostają w~górze -- wyglądając jak mimowie uliczni idący pod wiatr -- lub w~inny sposób błaznują. Wydaje się, że wszystkim podoba się przedstawienie; w~każdym razie gliniarze nie wydają się strzelać do nikogo. Pomimo wielokrotnych ostrzeżeń o zamoczeniu mojego przesiąkniętego toksynami ubrania, nie mogę się powstrzymać.

 Zanurzam się na chwilę. To dość orzeźwiające.

 W parku ludzie grają we frisbee, odbijając piłki plażowe. Przez połowę czasu mam maskę naciągniętą na czubek głowy.

 Starzy przyjaciele są wszędzie. W pewnym momencie pojawia się Janna, całkowicie owinięta w~wyszukany strój ochronny wykonany z~plastikowych worków na śmieci, gogle, poncho i~wysokie plastikowe buty, niosąc dużą torbę z~materiałami medycznymi do leczenia skutków działania gazu łzawiącego. Zakłada punkt przy drzewie na samym skraju ogromnej toksycznej chmury.

-- Jezu, Janno!  Co Ty tutaj \textit{robisz}? -- pyta jeden z~jej kolegów Uchodźców.

-- Po prostu nie mogłam usiąść i~nic nie robić, gdy ludzie są gazowani.

-- Oszalałaś? Słyszałem, że znowu używają CS. Kto wie, co by się stało, gdybyś znowu została wystawiona na niego!”

-- CS? Czy to prawda? 

 Kilku świadków potwierdza tę plotkę. Sprawa staje się spontaniczną dyskusją grupową. W końcu Janna zgadza się wrócić do St. Jean i~założyć tam punkt. Towarzyszą jej dwie osoby postronne.

 W końcu dostrzegam rozproszone klastry anarchistów z~Czarnego Bloku, które skupiają się na odległym krańcu parku. Wygląda na jakąś wcześniej zaaranżowane zgrupowanie. W tym momencie rozmawiałem z~kilkoma przyjaciółmi o wykonalności naszego manewru oskrzydlającego przeciwko południowej części muru, która wydawała się niebroniona. Przeszukując teren, natknąłem się na Deana, który leżał na długiej, płaskiej skale w~dość szykownym trenczu. Wyjaśniam mój projekt.

-- Wchodzę -- uśmiecha się, wyciągając spod płaszcza ogromną parę nożyc do drutu.

 Ale miejsce okazuje się lepiej bronione, niż się wydawało. Kanistry z~gazem łzawiącym lądują bezpośrednio pod naszymi stopami i~pojawia się pięciu robocopów z~czymś, co wygląda jak gigantyczne strzelby, albo do wystrzeliwania plastikowych kul, albo nasączonych pieprzem worków z~fasolą. Tak naprawdę nie chcemy się tego dowiedzieć i~szybko się wycofujemy.

 Jednak w~tym czasie Blok, wciąż tylko około czterdziestu osób, maskuje się i~zamierza wyruszyć. La Resistance nie ma wśród nich, ale spotykam dwóch znajomych z~wczoraj, którzy proponują, żebym poszedł. Zawsze możemy być czujką, mówią. W każdym razie najwyraźniej jest plan. Zapinam bluzę z~kapturem, ubierając się całkowicie na czarno, maskuję się i~podążam za nim.
 
\bigskip
\noindent \texttt{16:00, Kanadyjski Imperial Bank of Commerce}\medskip


 Poniżej znajduje się jeden z~zaledwie trzech głównych przypadków celowego niszczenia mienia podczas Szczytu. Celem jest lokalna siedziba jednego z~największych kanadyjskich banków, CIBC -- jednej z~głównych sił lobbujących za przejściem FTAA, wraz z~czerpaniem zysków z~rządowych programów pożyczek studenckich, jednocześnie naciskając na masowe cięcia w~finansowaniu opieki zdrowotnej i~edukacji.

 Biura banku znajdują się zaledwie kilka przecznic od parku, na skraju osiedla mieszkaniowego. Kilka przecznic dalej dochodzi do konfrontacji między pacyfistami a szeregiem gliniarzy od zamieszek, ale naprawdę nie wiem, co się tam dzieje. Sam bank znajdujemy na pierwszym piętrze jakiegoś pomniejszego biurowca, już oblężonego. Jednak sprawy są również nieco bardziej skomplikowane, niż się spodziewaliśmy. Dwóch członków grupy afinicji, która zaplanowała akcję, podniosło policyjną barykadę i~szykuje się do rozbicia szklanych szyb banku. Jednak na ich drodze staje dwóch pięćdziesięcioletnich hipisów, najwyraźniej małżeństwo w~identycznych tęczowych kurtkach i~ufarbowanych ubraniach. Oboje metodycznie próbują się wtrącić. W końcu kobieta się poddaje, ale jej mąż jest wytrwały. Spry, jak tancerz, skacze przed ich trajektorią za każdym razem, gdy się cofają, aby się rozbujać. Dwójka dzieci z~barykadą jest zdeterminowana, by go nie skrzywdzić, ale też nie zamierzają się poddać. Następuje osobliwy balet zwodów i~pchnięć, aż dzieci z~Czarnego Bloku wymyślą system: jeden go blefuje, a drugi macha mocnym zamachem w~innym kierunku. Wkrótce na chodniku jest potłuczone szkło.

 Szukamy kamer lub policji i~nie widzimy żadnej. Jest kilku przechodniów, którzy prawdopodobnie są reporterami, ale niosą tylko zeszyty; linia policyjna dwie przecznice dalej wydaje się nieświadoma, a może po prostu nie otrzymali jeszcze rozkazu marszu. Istnieją niezwykle destrukcyjni pacyfiści, którzy przeszli przez jakiś ,,trening deeskalacji'' SalAMI i~wypróbowują wszystkie swoje techniki. Gdy idziemy skrajem sceny, jeden brodaty pacyfista, przypominający raczej drwala z~Rady Delegatów (ale nie, to nie on) podąża za nami, powtarzając w~kółko dokładnie te same słowa: ,,To nie jest odpowiednia taktyka. To nie jest odpowiednia taktyka. To nie jest odpowiednia taktyka''.

 Zastanawiam się nad pytaniem go, czy uważa to za formę dyskusji. Moi towarzysze każą mi go zostawić w~spokoju. Co wydaje się rozsądne.

 Trochę dalej wygląda na to, że może dojść do popychania lub rzeczywistych bójek.

 Czas na własną deeskalację. The Bloc maszeruje, prowadzony przez wysokiego blondyna śpiewającego ,,Kumbaya''.

 Z wyjątkiem jednego małego zespołu, z~którego jeden zostaje przy malowaniu sprayem:

\noindent \emph{Banki nie krwawią. Protestujący krwawią.}

Inny przykleja przygotowany na tę okazję kartonowy napis:

\noindent \emph{Jestem ci winna za rozbite okno}

\noindent \emph{-- REWOLUCJA}

 A trzeci rozpryskuje we wnętrzu wiadro białej farby.

 Maszerujemy na zachód, z~dala od parku, ale zanim miniemy więcej niż jedną lub dwie przecznice, wita nas delegacja mieszkańców w~średnim wieku (myślę sobie: Mam ochotę nazywać ich ,,mieszczanami'', z~wyjątkiem tego, że żaden z~nich nie jest gruby. Proszą nas, żebyśmy nie wchodzili do ich sąsiedztwa. Jest mieszkalny.

 Jeden z~anarchistów z~przodu próbuje wyjaśnić, że nie mają się od nas niczego obawiać: nigdy nie atakujemy małych firm ani własności osobistej. Tylko budynki korporacji.

-- Cóż, nie ma żadnego z~tych w~tym kierunku. Tylko domy ludzi. Więc nie ma potrzeby tu wchodzić.

 Po chwili niewygodnego wahania się w~tę i~z powrotem, mężczyzna obok niego jest bardziej bezpośredni: 
 
 -- Nie niszcz miasta -- mówi, wskazując z~powrotem na park. -- Idź walczyć z~glinami!

-- Tak, walcz z~glinami -- mówi ktoś inny. -- Zdajemy sobie sprawę, że walczysz po naszej stronie. Wspieramy Was. Ale ludzie boją się o swoje sąsiedztwo.

 Dla wielu członków Czarnego Bloku musi to być moment ostatecznego moralnego zamętu. W końcu większość anarchistów uważa, że  celowe niszczenie własności jest uzasadnione, \textit{ponieważ }tak naprawdę nie jest to forma przemocy. Nie możesz być agresywny wobec przedmiotu nieożywionego. Ponieważ nikt tak naprawdę nie zostaje zraniony. Właśnie dlatego tęczowy facet mógł zachowywać się tak, jak on: wiedział, że nikt z~nas nie będzie chciał skrzywdzić drugiego człowieka -- a w~każdym razie z~pewnością nie takiego, który nie atakowałby nas bezpośrednio, a gdyby próbował wystawić swoje ciało w~ten sposób przeciwko gliniarzowi, gliniarz po prostu by go uderzył. Nagle staliśmy w~obliczu opinii publicznej, która wzywała nas do rezygnacji z~niszczenia mienia, a zamiast tego do zaangażowania się w~przemoc. Z pacyfistami mogliśmy się kłócić, a nawet krzyczeć na siebie, ale krzyczeliśmy w~tym samym języku. Tutaj mamy do czynienia z~kompletnie innym wszechświatem moralnym.

 Po krótkiej wymianie zdań odwracamy się i~maszerujemy z~powrotem w~stronę parku, przy jak zwykle głośnych wiwatach i~brawach. Ktoś krzyczy: 
 
 -- To Ludowa Policja Prewencyjna!
 
\bigskip
\noindent \texttt{16:20, Jean Baptiste}\medskip


 Park to samo święto: 
 
 -- Wygraliśmy! Szczyt zamknięty z~powodu gazu łzawiącego!

 Oczywiście nieprawda.

 Po chwili w~końcu odnajduję La Resistance, aby wymienić uściski. Opowiadam Kitty o banku. Mówi mi, że przez cały dzień toczą się bitwy wzdłuż północnej strony muru, gdzie przecina on Jean Baptiste. Linie policyjne są tak słabo rozciągnięte, że zwykle można znaleźć miejsce, które nie jest bronione. Przeważnie używali haków, lin i~nożyczek, jak w~piątek, ale czasami można wykorzystać strome zbocza, aby wtoczyć płonące śmietniki lub nawet wózki z~zakupami w~ogrodzenie.

 Refugees nigdzie nie widać, więc przez resztę popołudnia myślę, że jestem La Resistance. Wkrótce docieramy na skraj starego cmentarza kościelnego, gdzie ogrodzenie jest szczególnie mocno ozdobione znakami i~hasłami, szarpiąc hakami, używając kostki brukowej do rozwalania głównych słupów lub rzucania ponad ogrodzeniem w~pojazdy policyjne lub nawet, raz lub dwa razy, w~policjantów. Duża część muru jest już obalona w~tej okolicy. Wydaje się, że każdy kosz na śmieci ma w~sobie ogień -- aby wypalić gaz łzawiący -- co oznacza, że  podczas marszu poruszamy się przez naprzemienne strumienie dymu, toksycznej bieli i~gryzącej szarości.

 Kitty wyjaśnia, że  zwracają szczególną uwagę na cmentarz, ponieważ znajduje się on bezpośrednio za Centrum Kongresowym, gdzie odbywa się Szczyt.

\medskip

\noindent [Wpis w~Notatniku, napisany następnego dnia, 22.04.2001]
\medskip

 Czarny Blok nigdy nie był duży tego dnia, rzadko więcej niż trzydziestu czy czterdziestu ludzi, choć od czasu do czasu miał pięćdziesiąt czy sześćdziesiąt osób. Ludzie rozproszyliby się, grupy afinicji składające się zwykle z~sześciu lub ośmiu osób zostałyby zredukowane do dwóch lub trzech osób z~powodu urazów lub wyczerpania. Chociaż od czasu do czasu otrzymywaliśmy również posiłki od ludzi, którzy dopiero co przybyli do miasta: na przykład trzech Yabbów z~Connecticut, którzy pojawili się w~sobotę rano i~dołączyli do La Resistance. Prawie każdy został w~pewnym momencie uderzony -- często w~stopy lub kostki, głównie przez pojemniki z~gazem łzawiącym. Ale coraz częściej używano plastikowych kul, a także broni z~celownikami laserowymi, więc w~nocy ludzie często widzieli, że gliniarze celowo celują w~głowy lub pachwiny. 
 
 -- Uderzyli mnie w~pachwinę. Ale miałem na sobie filiżankę! -- oznajmił triumfalnie jeden z~naszych nowo przybyłych.

 Kiedy zwiadowca zauważał rozsądny cel, zbieraliśmy wszystkich dostępnych i~tworzyliśmy krąg, aby o tym porozmawiać. To zawsze wiązało się z~nakłonieniem paru ochotników do sprawdzenia kamer, krążąc przez otaczający tłum, ponieważ zawsze była jakaś, prosząc każdego, kto ma kamery wideo lub sprzęt fotograficzny, aby nie robił zdjęć ze spotkania. To pomimo tego, że wszyscy byli cały czas zamaskowani. (Brad powiedział mi, że tak samo było w~Pradze. Sztuka polega na tym, żeby podejść z~nieco przerażającym wyglądem, wszyscy w~czarnych maskach i~zazwyczaj w~hełmach; a potem być skrupulatnie uprzejmym i~wdzięcznym, kiedy rzeczywiście otwierasz usta. Połączenie okazało się niezwykle skuteczne). Dyskusja była dość swobodna, ale oparta na konsensusie. Potem ruszaliśmy do akcji -- często wiwatowani przez demonstrantów i~coraz częściej mieszkańców miasta, gdy tylko pojawialiśmy się w~nowym miejscu.

 Blok miał tylko minimalną łączność -- w~pewnym momencie myślę, że cały nasz system łączności składał się z~dwóch facetów połączonych przez Nextel, których zadaniem było koordynowanie, tak aby nie zostać odcięci i~otoczeni przez gliniarzy. Kiedy zaatakowaliśmy -- tak jak w~St. Jean -- jedna osoba również zatrzymywała się jako zwiad. Ale tylko tyle. Wydawało się to jednak typowe dla całej akcji: jeśli CLAC miał system łączności lub zwiadu, który prawdopodobnie musi mieć, nigdy nie widziałem tego śladu. Musiał być bardzo mały. Time's Up Bill, który spędził trochę czasu, okrążając Mur na rowerze, później skarżył się, że przez cały dzień widział wiele niestrzeżonych wyłomów w~ogrodzeniu ochronnym. Gdyby istniała jakakolwiek porządna organizacja, ludzie mogliby włamać się od razu. Ale oczywiście większość z~nas już dawno zdecydowała, że  nie chcemy wchodzić na teren.

 Częściowo również ataki na mur mają na celu wytrącenie policji z~równowagi, próbę powstrzymania ich przed gromadzeniem sił i~ponownym najazdem na okoliczne sąsiedztwo. 

 Niedaleko cmentarza, przy Rue St. Genevieve, znajdowała się ogromna gromada ludzi, rodzaj ogniska intensywności, tam, gdzie Blok wcześniej atakował fragment muru. Najwyraźniej podpalili śmietnik i~wrzucili go przez ogrodzenia. Przebił się i~przewrócił wewnątrz Muru. Następnie gliniarze próbowali zablokować wyłom buldożerem, ale Blokowi udało się go unieruchomić -- kiedy go zobaczyliśmy, wyglądał na kompletnie zniszczony, z~rewolucyjnymi hasłami wymalowanymi na nim w~sprayu -- i~blok uciekł, gdy oddział może trzydziestu gliniarzy prewencji maszerowało w~szyku, by zabezpieczyć teren. Kiedy przyjechaliśmy, śmietnik wciąż się tlił, traktor zepsuty i~przekrzywiony, a trzydziestu policjantów stało zupełnie nieruchomo, otoczone setkami pacyfistów. Aleja była na tyle ciasna, że  udało im się całkowicie ich odciąć. Policja miała może kilka metrów przed nimi i~za nimi, potem była nieprzenikniona ściana ludzkich istot. Ktoś powiedział nam, że pat trwa już prawie godzinę. Nieco wyżej na zboczu znajdowała się spora grupa bębniarzy i~innych muzyków, grających powolną, rytmiczną muzykę -- właściwie to była bardzo dobra, z~wszelkiego rodzaju zawiłymi synkopami -- a ludzie tańczyli w~hipnotycznym stylu. Czasami ktoś opuszczał ludzką ścianę i~dołączał do tańca lub odwrotnie. Zafascynowany, odsunąłem się na chwilę od Bloku, obiecując, że spotkam się później.

\bigskip
\noindent \texttt{17:25, Park}\medskip


 Teraz plotka jest taka, że  Szczyt jest opóźniony, ponieważ gaz łzawiący dostał się do systemu wentylacyjnego. Albo, alternatywnie, że delegacja brazylijska wykorzystała to jako pretekst do odmowy wejścia. (Wszyscy liczyli na to, że Brazylijczycy staną na czele sprzeciwu wobec traktatu).

 Policja zaczyna wchodzić na Jeana Baptiste'a, pomimo naszych najlepszych starań, by ją opóźnić. Jedna jednostka otoczyła pobliskie skrzyżowanie.

 Próbują też ponownie zająć park, obficie posługując się granatami ogłuszającymi i~gazem pieprzowym. Odpowiedzią jest niemal oszałamiająca różnorodność taktyk. Jest grupa około trzydziestu aktywistów, w~większości chyba studentów, w~dżinsach i~podkoszulkach, niektórzy nawet bez bandan, rozpoczynających blokowanie przez siedzenie. Ustawiają się dokładnie na ścieżce linii policyjnej, ci z~przodu podnoszą obie ręce w~powietrze, aby pokazać znaki pokojowe. Śpiewają:

\noindent\emph{Nie stosujemy przemocy, a ty? \newline Nie stosujemy przemocy, a ty?}

 Gdy gliniarze się zbliżają, aktywiści zmieniają na ,,cały świat patrzy!'', a dwóch policjantów zaczyna strzelać plastikowymi kulami w~sam środek tłumu. Ktoś krzyczy. Ktoś kogoś unosi, ale reszta pozostaje na swoim miejscu. Pojawia się ksiądz i~interweniuje. Rozmawia z~policją. Niektóre Radykalne Cheerleaderki, z~czarno\dywiz czerwonymi pomponami i~skandalicznymi fryzurami, podchodzą i~zaczynają w~pobliżu jedną ze swoich wyszukanych pieśni. Najwyraźniej uspokojeni gliniarze wracają do ogrodzenia.

 Niemal natychmiast potem przez płot płyną za nimi cztery koktajle Mołotowa. Widzę kilka postaci uciekających we mgle. Co dziwne, nie wyglądają zbytnio na aktywistów: ci dwaj, których widzę, najwyraźniej wydają się krępymi, czterdziestolatkami. Jeden raczej przypomina mi faceta, który kilka godzin wcześniej narzekał na gaz łzawiący na swojej werandzie (ale jestem prawie pewien, że to nie on).

 Nigdy nie widziałem w~ten weekend nikogo z~bombą zapalającą, kto nie mówił po francusku.

 W końcu kawałki zaczęły się układać: Montreal Ya Basta! wyjaśniające, jak istnieją różne standardy dotyczące przemocy w~Quebecu, myląca odmowa zakazu mołotowa przez CASA, nawet gdy apelowali o wsparcie społeczności, delegacja obywateli mówiąca nam, abyśmy walczyli z~glinami -- nawet diatryba Maca w~Little Italy o tym, jak naprawdę uciskani albo cofają się lub walczą i~nie są zainteresowani wymyślnymi kodeksami niestosowania przemocy. To społeczność o niezwykle bojowej tradycji oporu. Zarówno ksiądz, jak i~zamachowcy reprezentowali to samo zjawisko: społeczność zaczęła aktywnie interweniować w~naszym imieniu.

 Brodaty facet na szczudłach, w~wyszukanym kostiumie wyszytym zielonymi cekinami, podchodzi do ogrodzenia z~ogromnym znakiem pokoju. Gliniarze włączają działko wodne i~trafiają go prosto w~klatkę piersiową. Leci do tyłu około pięciu metrów. Medycy podbiegają, upewniają się, że nie ma złamanego kręgosłupa, potem zamieniają szczudła w~szyny i~szybko, trzymając nisko głowy, zabierają go.
 
\bigskip
\noindent \texttt{17:53}\medskip


 Nad parkiem unosi się ogromny pióropusz. Helikoptery grzechoczą nad głową.

 Kolejny pocisk z~moździerza. Wiwaty, gdy ktoś je odrzuca. Do tego dwie bomby dymne.

-- Hej, bez przemocy! -- ktoś krzyczy.

 Ktoś inny: 
 
 -- Czy jest ktoś, kto może być w~ciąży? Używają gazu CS!

 Oddział policji zaczyna łapać aktywistów na skraju parku. To chyba pierwszy raz, kiedy byłem świadkiem aresztowania. Wychodzę z~parku i~znów ruszam w~dół.


\bigskip
\noindent \texttt{18:00, Jean Baptiste}\medskip

 

 To, co następuje, jest czymś w~rodzaju rozmycia. Całkowicie zrezygnowałem z~robienia notatek. Jakoś skończyłem w~kolumnie około dwudziestu pięciu lub trzydziestu Black Bloc's, którzy próbują szarżować na odgrodzoną pozycję\ldots  Myślę, że znowu było to wzdłuż St. Jean, gdzie płonący wózek sklepowy prawie zawalił ścianę w~ciągu godziny lub dwa przed. Mniej więcej w~połowie szarży jesteśmy bombardowani pieprzem; przynajmniej to samo oślepiające uczucie, którego doświadczyłem pod ścianą, przechodząc przez moją maskę gazową. Potykam się z~powrotem. Przypadkiem na pobliskim ganku stoi sanitariusz, osiemnasto-, dwudziestoletni mężczyzna, wyglądający jakby pochodził z~Senegalu lub Kamerunu, ze szpiczastymi włosami i~mocną plastikową apteczką. Proponuje mi pełną kurację anty-pieprzową, a my znajdujemy osłoniętą przestrzeń, w~której dokładnie myje mi oczy i~twarz jakimś środkiem zobojętniającym kwas, następnie szoruje i~spłukuje olejem mineralnym. Czuję się znacznie lepiej.

 Kiedy jednak znajduję pozostałości Bloku, jest ich tylko trochę ponad tuzin, a duży blondyn mówi surowo, że nikt mnie nie rozpoznaje. Gdzie jest moja grupa afinicji? Może powinieneś spróbować ich znaleźć. Rozglądam się za Kitty i~resztą La Resistance, ale nikogo nie ma. Mógłbym przysiąc, że byłem z~nimi wcześniej. Wciąż trochę oszołomiony, nie pamiętam niczyich imion, nie mówiąc już o nazwach akcji. Buffy jest daleko od oparcia się o ścianę z~zamkniętymi oczami. Ja też zapomniałem jej imienia. Mówię mu, dobry pomysł, tak, jestem pewien, że przenieśli się do parku czy coś w~tym stylu i~idą sobie odpocząć.

 Duży blondyn nagle przybiera uprzejmy ton. 
 
 -- Hej, powodzenia! Jestem pewien, że ich odnalezienie nie potrwa długo.

 W parku robi się coraz bardziej intensywnie. Mały oddział ludzi z~bombami zapalającymi próbuje zniszczyć armatkę wodną. Za każdym razem, gdy zbliżają się do zasięgu, ostrożnie nurkując między jego mechanicznymi ruchami, policja otwiera ogień plastikowymi kulami. Obserwuję, jak dwa mołotowy robią piękne łuki i~lądują w~odległości kilku stóp od maszyny, wytwarzając spektakularne, ale chwilowe rozbłyski ognia. Wydaje się, że nie wyrządzają żadnych szkód.

 Wygląda na to, że w~tej chwili nie działają żadne oddziały chwytające, więc kładę się, by odpocząć; ale wydawało mi się, że po kilku minutach odpoczynku Buffy klepie mnie po ramieniu.
 
  -- Hej, Davidzie! Próbujemy zebrać paru ludzi, by udali się na autostradę. Jest tam główny punkt wejścia, który jest naprawdę słabo strzeżony.

-- Och okej.

-- Czy w~parku są jacyś inni członkowie twojej grupy afinicji?

-- Nie, myślę, że ich zgubiłem.

 W ciągu kilku minut wracam do Bloku, w~tym samym miejscu co poprzednio, ale tym razem wszyscy są tam: La Resistance, grupa Craiga, dzieciaki z~PEI -- brakuje tylko Montrealu. Kierujemy się w~dół St. Jean, a następnie w~dół do okazjonalnych wiwatów pieszych, schodzimy na autostradę i~badamy sytuację.

 Sytuacja okazuje się jednak trochę zbyt nerwowa jak na mój gust. Toczy się już bitwa, w~której co najmniej pięciu czy sześciu gliniarzy kuca w~ciemności za szeroką bramą z~siatki, a czerwone lasery z~celowników miotają się i~miotają wszędzie. Ogromny pusty odcinek asfaltu i~osłonięte przestrzenie, w~których ludzie -- myślę, że są studentami, zdecydowanie nie Czarnym Blokiem, ale tak naprawdę nie mam pojęcia, kim są -- mieszają koktajle Mołotowa w~pustych szklanych butelkach po coli. Co minutę ktoś wychodzi zza ich osłony i~rzuca go nad bramą.

-- Nie wygląda już na tak lekko bronioną -- mówię do Buffy.

 Marszczy brwi, gdy kolejny rozbłysk płomieni rozświetla na chwilę bramę. 
 
 -- Cóż, zobaczymy, co możemy z~tym zrobić.

 Sam Blok już dawno nie zgodził się na żadne koktajle Mołotowa, ale teraz, kiedy dżin wyszedł z~butelki, niektórzy z~nas przynajmniej chcieli pomóc w~ich przygotowaniu. 
 
 -- W końcu -- mówi ktoś -- powiedzieliśmy, że pójdziemy za przykładem miejscowej ludności. 
 
  Inni, na przykład Lee, wyglądają na niezwykle sceptycznie nastawionych. Cała scena była dla mnie niezwykle niepokojąca. CS lądował wszędzie. Gliniarze strzelali najwyraźniej na oślep. Jeden dziwny facet kręcił się powoli pośrodku wszystkiego, tańcząc w~światłach i~chmurach do muzyki, która musiała istnieć tylko w~jego głowie.

-- Jezu, co sprawiłoby, że ktoś się tak zachowywał? -- ktoś pyta.

-- Może Ecstasy.

 Myślę, że właśnie dlatego mówią ci, żebyś nigdy nie używał narkotyków na akcji.

 Ja sam nie jestem zainteresowany pomaganiem nikomu w~próbie podpalenia kogoś innego -- nawet policji w~ognioodpornej kamizelce kuloodpornej -- więc wydaje mi się, że to może nie być zły moment na skontaktowanie się z~IMC. Podejmę ostatnią próbę zlokalizowania Karen. Prawdopodobnie będą mieli jaśniejsze pojęcie o tym, co dzieje się w~mieście i~czy naprawdę zamknęliśmy Szczyt. Cofając się do pobliskiej latarni, by sprawdzić mapę, zdaję sobie sprawę, że jest dość blisko. Wszystkim życzę powodzenia (nikt tak naprawdę tego nie zauważa), patrz przez chwilę, jak schodzą na pozycje bliżej bramy.

 Kilka minut później przechodzę pod rampą autostrady, gdzie Food Not Bombs rozstawia ogromne wazy na nadchodzącą darmową kuchnię. Jest mała wioska namiotowa i~punkrockowcy ustawiający nagłośnienie z~tyłu ciężarówki. To musi być Ile Fleuriot. To trochę brudna, wilgotna przestrzeń, ale już kilkaset osób zaczyna gromadzić się na imprezie. Zapisuję: mam się później spotykać z~ludźmi na przyjęciu. Potem mijam zamknięty już sklep Army/Navy i~wreszcie schodzę do IMC.
 
\bigskip
\noindent \texttt{19:15, Wchodzę do IMC}\medskip


 W IMC wszystko jest inne. Po pierwsze, teraz jest bezpiecznie. Nikt nie może wejść do środka bez identyfikatora Indymedia. Przy drzwiach stoi duży facet, który wydaje się należeć do budynku. Na dole, gdzie kiedyś była garstka sennych, szczęśliwych aktywistów, przestrzeń jest teraz zatłoczona i~pełna ponurej wydajności. Na stołach rzędy komputerów i~kamer wideo; na podłodze są laptopy. Przewody pokrywają wszystko. Każde gniazdko elektryczne ma przedłużacz i~jest do niego podłączonych siedem lub osiem urządzeń. W pobliżu drzwi leży ogromny stos sprzętu, masek przeciwgazowych, płaszczy przeciwdeszczowych, butelek z~wodą, wszelkiego rodzaju sprzętu ochronnego. Na ścianach spisy regulaminów, zmiany pracy, zespoły, numery telefonów, wydarzenia. Obok drzwi znajduje się zaimprowizowane stanowisko ochrony, gdzie po raz drugi pokazujesz legitymację; Za biurkiem, dziewczyna o ciemnych kręconych włosach, która wygląda jak uczennica liceum. Pokazuję kartę IMC. Okazuje się, że tak naprawdę jest uczennicą liceum: częścią małej grupy z~innej prowincji, która przebywa w~mieście na jakimś stypendium dla alternatywnych mediów. Wygląda na bardziej niż tylko trochę przytłoczoną.

 Może jedna trzecia twarzy jest mi znana z~innych akcji. Dostrzegam Celię, którą poznałem w~IMC w~Filadelfii podczas Konwencji Republikańskiej. W tym czasie oboje pracowaliśmy w~zespole, współpracując z~mediami korporacyjnymi. Byłem kompletnym neofitą. Ona, po trzydziestce, była doświadczoną działaczką medialną, która ostatecznie zorganizowała większość naszych konferencji prasowych.

-- Hej, Celio!

-- Och, cześć, Davidzie. Po prostu wpadłeś do miasta?

-- Nie, udało mi się przedostać w~Akwesasne. Jeden z~kilku. Byłem tu od piątku rano. Ty?

-- Ja? Jestem w~mieście od środy. -- Przerwała. -- Więc co o tym myślisz?

-- Nigdy nie doświadczyłem czegoś takiego. -- Zacząłem opowiadać o piątku i~radości z~burzenia muru.

 Celia jednak nie jest pod wrażeniem bohaterskich macho i~zamiast tego zaczyna opowiadać mi o swoim własnym punkcie kulminacyjnym: ceremonii pierwszego dnia prowadzonej nad Żywą Rzeką. Czy to widziałem? Właśnie montowała z~niego obrazy na pobliskim komputerze: błękitny strumień zatrzymujący się wzdłuż ulic St. Jean, wszystko ucichło, a potem nagle setka ludzi jednocześnie wyrzucała w~powietrze rolki papieru toaletowego, tworząc efekt jak trzepoczące, falujące morze. Po czym Wiccanie przedstawili piękną inkantację.

 Obrazy na ekranie komputera były małe i~wątpiłem, aby naprawdę oddawały pełny sens chwili. Mimo to robiły wrażenie. Przeglądam je nawet po tym, jak Celia zostaje wezwana chwilę później, a potem grzebię, aż znajdę osobę prawną, która śledziła aresztowanych IMC. Było ich kilkoro, ale większość z~nich w~ciągu ostatnich kilku godzin i~żadna z~nich nie była Karen. Nikt się z~nią nie kontaktował.

 Znalazłem kogoś, kto zna numer Sashy, więc dzwonię do niego z~telefonu IMC. Ale idzie prosto na pocztę głosową. Sprawdzam wiadomości. Dzwonię do znajomych.

 Nic.

 Czy to możliwe, że po prostu poszła do domu i~nikomu nie powiedziała? Dla aktywisty byłoby to niewiarygodnie nieodpowiedzialne. Ale oczywiście Karen nie jest aktywistką.
 
\bigskip
\noindent \texttt{19:30, nadal w~IMC}\medskip


 Independent Media Centers, Niezależne Centra Medialne to kolejna instytucja zrodzona z~protestów WTO w~Seattle: miały być sposobem, w~jaki dziennikarze-aktywiści mogą przedstawiać własne relacje z~wydarzeń i~faktycznie przekazywać protestującym przesłanie, czego korporacyjne media prawie nigdy nie robią. Do 2001 roku w~większości głównych miast Ameryki Północnej i~coraz częściej na całym świecie istniały stałe IMC. Ogromne powstawały też tymczasowo podczas każdej większej mobilizacji. IMC działały na zasadniczo anarchistycznych zasadach. Wszystko odbywało się wspólnie: ludzie redagowali nawzajem swoje historie; nie było hierarchii redaktorów i~reporterów; wszystkie decyzje podejmowane były w~drodze konsensusu. IMC mógł prowadzić audycje radiowe na żywo, przygotowywać filmy, a w~kluczowych dniach akcji wydawać codzienną gazetę informującą o wydarzeniach. Jednak najszybciej od razu utworzył stronę internetową, gdzie można było znaleźć najświeższe informacje na temat działań, które miały miejsce. Jedna strona strony była otwarta -- każdy mógł ją publikować -- i~dlatego w~dużej mierze przypominała młyn plotek na ulicach; ale środek strony stanowiły wszystkie depesze od reporterów IMC, którzy szczycili się utrzymywaniem bardziej rygorystycznych standardów dokładności niż prasa korporacyjna. 

 Tutaj, w~centrum informacyjnym dla aktywistów, w~końcu mogłem zacząć układać w~całość obszerne, panoramiczne informacje, które były po prostu niedostępne na ulicach. Obraz był przerażający.

 Policja schodziła ze wzgórza równomiernie od 18:00, znacznie szybciej niż poprzedniego dnia. Tym razem ich strategia polegała najpierw na zajęciu kluczowych punktów i~skrzyżowań, a następnie kontynuowaniu operacji sprzątania i~aresztowaniu każdego, kto nadal znajduje się na ulicach na okupowanym terytorium. Zaczynali także być brutalni. Kilku kamerzystów Indymedia zostało już pobitych i~aresztowanych. Jednym z~pierwszych celów policji była Klinika, gdzie nasi medycy leczyli najgorsze obrażenia. Najpierw policja wrzuciła gaz łzawiący bezpośrednio przez okna, rozbijając szkło i~zmuszając medyków do ewakuacji rannych. Piętnaście minut później oddział policji pojawił się w~nowej, prowizorycznej klinice, którą stworzyli w~bocznej uliczce na zewnątrz i~wyprowadził wszystkich pod groźbą użycia broni, zrywając pacjentów z~noszy, zawłaszczając środki medyczne, zdejmując wszystkim gogle i~maski przeciwgazowe, a nawet przesiąknięte octem bandany, a następnie sprowadzając ich w~dół po długich schodach prowadzących z~Côte d'Abraham. Wielka bitwa przeniosła się teraz do serca Zielonej Strefy. Tysiące ludzi zgromadziło się na darmowym jedzeniu i~imprezie tanecznej, która miała uczcić akcję tego dnia. Wielu przybyło z~marszu i~Szczytu Ludowego; były dzieci i~starcy. Nagle policja zaatakowała. Szeroka na pół hektara,,Tymczasowa Strefa Autonomiczna'' pod autostradą została przekształcona w~ogromną chmurę gazu łzawiącego. Przyszli imprezowicze odpowiedzieli, zajmując górną część autostrady. Policja próbowała ich usunąć, ale było ich już co najmniej trzy tysiące i~stawiali twardy opór.

 Nie mieliśmy jeszcze liczb rannych i~aresztowanych. Oficjalne liczby, sumiennie powtarzane przez telewizje i~serwisy informacyjne, były czystą fantazją: od piątku gliniarze donosili o czterdziestu rannych, z~których, jak twierdzili, około połowa to policja. Nasi medycy donosili, że pierwszego dnia wyleczyli ponad tysiąc obrażeń: w~tym kilku astmatyków, którzy omal nie zmarli od gazu, dziesiątki złamanych kości i~kilka bardzo poważnych oparzeń. Władze nadal twierdziły też, że aresztowano tylko kilkadziesiąt osób, mimo przeczesywania Starego Miasta. To również nie mogło być prawdą. Otrzymywaliśmy już zwykłe przerażające raporty, znane już w~USA, o celowych nadużyciach więźniów. Autobusy pełne więźniów w~kajdankach jeździły w~kółko po mieście przez dwanaście lub trzynaście godzin, aby uniknąć biurokracji; aresztowani, którzy faktycznie zostali wpisani, byli krępowani, odmawiano im dostępu do wody lub toalet; rannym działaczom odmawiano leczenia, rozbierano, polewano lodowatą wodą i~zamrażano w~nieogrzewanych celach.

 Wszyscy obawiają się, że następnym celem będzie IMC. Nie wynika to tylko z~jego oczywistego strategicznego znaczenia w~przekazywaniu aktywistom (i wszystkim innym) pojęcia, co się dzieje. Najwyraźniej ktoś zeskanował kilka stron tekstu, który wydawał się pochodzić z~policyjnych samochodów terenowych, do których włamano się poprzedniej nocy, ze szczegółowymi raportami wywiadowczymi i~planami awaryjnymi dotyczącymi strategii policyjnej, i~przesłał go na stronę internetową IMC poprzedniego wieczoru. Redaktorzy natychmiast ją usunęli i~przekazali do IMC w~Seattle, która je opublikowała, zaznaczając, że nie ma sposobu, aby upewnić się, czy były sfałszowane, czy autentyczne. Kilka godzin później policja z~Seattle zamknęła tam IMC. Wydawało się rozsądne oczekiwać, że biorąc pod uwagę okoliczności, Quebec IMC może być następny.

 Najgorszą wiadomością jest jednak to, że wygląda na to, że jeden z~protestujących został zastrzelony. To nie jest do końca pewne. Raport przychodzi najpierw telefonicznie, od reportera IMC przy autostradzie. Stwarza to poważny kryzys, ponieważ teraz pojawia się pytanie, co opublikować. Zwoływane jest spotkanie. Zaczyna się od kilkunastu osób stłoczonych wokół biurka, a kończy się na prawie wszystkich:

\medskip
\noindent [Z notatnika, 21.04.01, 19:50, spotkanie awaryjne, Quebec IMC]
\medskip

\noindent Chuck: Cóż, pozwólcie, że przedstawię to jako formalną propozycję. Mamy raport naocznego świadka, że  protestujący został zabity po trafieniu plastikową kulą w~gardło w~pobliżu autostrady. Najwyraźniej niektórzy medycy próbowali CPR, a kiedy się nie udała, w~końcu udało im się zabrać go do karetki i~to był ostatni raz, kiedy ktoś go widział. Proponuję więc umieszczenie informacji, które posiadamy na stronie internetowej. Mając na uwadze, że w~ten sposób skutecznie udostępnimy go także korporacyjnym mediom. 

\noindent Celia: Jest też kontrpropozycja, aby mała grupa z~nas wykonała pracę reporterską, aby uzyskać pełne potwierdzenie, zanim cokolwiek wyślemy. Nie jesteśmy mediami korporacyjnymi. Dla nich wystarczyłoby jedno potwierdzenie; naszym zadaniem jest robić to lepiej. Propozycja polega więc na niepublikowaniu historii, chyba że mamy co najmniej dwa potwierdzenia.

\noindent Chuck: Cóż, zgadzam się, że zdecydowanie powinniśmy stworzyć taki zespół w~każdym przypadku.

\noindent \textit{Helen} Tworzę listę dla każdego, kto chce teraz wyrazić obawy lub komentarz. To nie jest do końca pewne.

\noindent  Bill: Cóż, ze swojej strony wolałbym, jeśli to możliwe, trzymać korporacyjne media z~dala od tego, ponieważ są skurwielami.

\noindent  Suzette: Zgadzam się z~drugą propozycją. Musimy sprawdzić lepiej.

\noindent  Andrew: Udało mi się również dodzwonić do ulicznego medyka, który potwierdził pierwszą część historii: młody człowiek został postrzelony w~gardło, upadł, nie oddychał, medycy próbowali go reanimować i~ostatecznie został zabrany do szpitala.

\noindent  \textit{Helena}: Czyli możemy to zgłosić jako potwierdzone?

\noindent  Ben: Powiedziałbym, że skoro ta część jest potwierdzona, zbierzmy mały zespół do śledztwa; sprawdźmy, czy możemy uzyskać dalsze informacje ze szpitala.

\noindent  \textit{Helena}: Więc popierasz drugą propozycję? 

\noindent  Ben: Tak.

\noindent  [Ludzie wchodzą ze schodów, zdejmują sprzęt, rozmawiają z~podnieceniem] 

\noindent  \textit{Helena}: Cisza proszę! Tutaj trwa proces konsensusu!

\noindent  Annette: Myślę, że powinniśmy poważnie pomyśleć o skutkach uwolnienia czegokolwiek potencjalnie wybuchowego bez uzyskania absolutnego potwierdzenia.

\noindent  Randy: Co do pierwszej propozycji, zgadzam się z~Annette. Mamy tu ponad dziesięć tysięcy ludzi i~kilka tysięcy gliniarzy. Tam jest już na wpół wojna. Jeśli rozgłosimy, że ktoś zginął, czy chcemy być odpowiedzialni za wynik?

\noindent  [Kilka osób krzyczy ,,tak!'']

\noindent  Annette: Słuchaj, wiemy, że media korporacyjne obserwują wszystko, co robimy. Umieścimy to, będą z~tego korzystać. Jeśli powiemy coś, co nie jest prawdą, nie chcę nawet myśleć o tym, co się stanie.

\noindent  Noah: A ludzie są już wystarczająco wkurzeni na gliny.

\noindent  Chuck: \ldots i~bardziej prawdopodobne, że informacja i~tak będzie na ulicach, z~plotek. Pójdzie słowo, że tak się stało. Możliwe, że jeśli opublikujemy historię mówiącą tylko o tym, co jest już potwierdzone, to ktoś, kto zna resztę historii, zadzwoni i~nam opowie. To może być jedyny realistyczny sposób, w~jaki możemy się dowiedzieć.

\noindent  Riley: Otrzymaliśmy już doniesienia o rzuceniu kilku mołotowów, w~kilku miejscach, w~których doszło do bitew z~glinami.

\noindent  [Wszyscy stoją teraz w~kręgu.] 

\noindent  Suzette: Pojawiło się również wiele nieprawdziwych plotek o zamknięciu Szczytu. Skąd wiemy, że którakolwiek z~tych historii jest prawdziwa?

\noindent  David: Cóż, mogę potwierdzić Mołotowy. Widziałem już sporo z~nich.

\noindent  Sheila: Przepraszam, punkt procesu. Całe to spotkanie jest prowadzone w~języku angielskim. Czy jest ktoś, kto nie mówi po angielsku i~chce, aby mu wyjaśnić, co się dzieje?

\noindent  [Jedna kobieta chce. Sheila informuje ją po francusku i~zapewnia tłumaczenie symultaniczne do końca dyskusji.]

\noindent  \textit{Helena}: Cóż, wydaje mi się, że doszliśmy do\ldots 

\noindent  Jamie [\textit{nowo przybyły}]: Słuchajcie, widziałem, jak ten facet został postrzelony! To się stało.

\noindent  Andrew: Czekaj, byłeś tam? Widziałeś to?

 Od drzwi: GLINY WCHODZĄ DO BUDYNKU!

 Spotkanie przeradza się w~zamieszanie. Policjanci muszą być w~biurach na piętrze, bo podobno nie są jeszcze na schodach. Ktoś krzyczy: ,,SZYBKO, WEŹ KLUCZ! ZABIERZ JUŻ TERAZ!''. Ktoś inny sprawdza schody; inni chwytają za telefony, wbijają numery, próbując znaleźć otwartą linię, próbując skontaktować się z~reporterami IMC na ulicy. Po chwili kryzys mija. Wygląda na to, że gliniarze nie zrobili nic więcej, jak tylko wsadzili głowy, wystrzelili kulę w~kierunku schodów tylko po to, żeby nas przestraszyć, a potem zniknęli. Powoli wszyscy próbują znowu oddychać, zmieniać nastrój, wychodzić z~trybu kryzysowego. Spotkanie ponownie się toczy, a Jamie, naoczny świadek, wciąż ubrany w~czerwoną bandanę i~zielone gogle na czubku głowy, podaje więcej szczegółów: ten jeden facet, ofiara, z~jakiegoś powodu był sam niedaleko ściany, może dwadzieścia metrów od policji. Nagle rozległy się dwa strzały i~został uderzony dwa razy z~rzędu, raz w~ramię, raz w~gardło. Po laserach widać było, że celują bezpośrednio w~jego głowę.

 Pomyślałam z~przerażeniem: czy to ten sam facet, którego widziałam tańczącego w~środku starcia przy autostradzie. To musiał być on. Kto jeszcze mógłby to być? Byłoby wspaniale, gdyby ten facet \textit{nie został }postrzelony. Albo nie\ldots  czy ktoś inny nie powiedział, że wydarzyło się to na obszarze poza akcją?

\noindent \textit{Helena}: Brzmi to tak, jakby wyłaniający się konsensus dotyczy drugiej propozycji: nie wysyłać niczego od razu, ale spróbować potwierdzić historię. Czy ktoś ma poważne obawy?

\noindent Bill: Nadal nie wiem, jak moglibyśmy to zrobić. Nie znamy nazwiska faceta. Jedyny sposób, aby potwierdzić nazwisko, to spytać się gliniarzy.

\noindent Celia: Możemy skontaktować się ze wszystkimi lokalnymi szpitalami. Zgłoszę się na ochotnika do zespołu, abyśmy w~końcu mogli to opublikować.

\noindent Joe: Naprawdę boję się, że jeśli będziemy rozpowszechniać fałszywe plotki, to poważnie się zdyskredytujemy.

\noindent Riley [\textit{przez telefon}]: Dostaję raport od reportera IMC z~ulicy; mówi, że na górze dzieje się wiele przypadków brutalności policji. Podobno sześciu gliniarzy właśnie otacza drzwi\ldots 

\noindent Ktoś inny: Niektórzy medycy mówią, że schodzą. To jest sytuacja awaryjna.

\noindent Annette: Musimy pamiętać, że cały świat nas obserwuje. Jeśli coś, co zgłosimy, okaże się niedokładne, nikt tego nigdy nie zapomni. Nie wspomniałbym nawet o tym, że w~tym momencie krążą plotki.

\noindent (Medycy wchodzą)

 Jak można było się spodziewać, medycy szukali nowej przestrzeni do założenia punktu. Czy mogliby wykorzystać biura na piętrze na tymczasową klinikę? Wydaje się, że konsensus jest taki, że prawdopodobnie nie będzie to bardzo bezpieczna lokalizacja, ponieważ prawdopodobnie sami zostaniemy zaatakowani, ale nie ma zbyt wielu realnych alternatyw. Medycy ruszają, by powiadomić swoją sieć.

 Jako przestrzeń, IMC była szczególnie wrażliwa. Przede wszystkim był tylko jeden punkt dostępu: schody. Gdyby policja się pojawiła, wszyscy bylibyśmy natychmiast uwięzieni w~piwnicy. Po drugie, budynek wydawał się nie mieć działającego systemu wentylacji. Jeden kanister z~gazem łzawiącym na dole schodów uczyniłby go niezdatnym do zamieszkania. Już ludzie wpychali szaliki i~swetry pod szczeliny drzwi, aby zapobiec przedostawaniu się złego powietrza do środka. Pytanie brzmi, czy policja próbuje wejść, czy powinniśmy bronić przestrzeni, czy powinniśmy praktykować pokojowe nieposłuszeństwo obywatelskie (wszyscy siedzą na podłodze, odmawiają wykonania rozkazów, utykają, jeśli spróbują nas wynieść), czy też powinniśmy się poddać i~podporządkować? Na wcześniejszym spotkaniu uzgodniono drugą strategię, ale w~świetle wydarzeń kluczowe było upewnienie się, że wszyscy nadal są po tej samej stronie. Także, żeby zapewnić, że jesteśmy dostatecznie wcześnie ostrzeżeniu, by każdy, kto nie chce ryzykować aresztu, mógł wcześniej wyjść.

 Dopiero jednak to spotkanie się zaczęło, gdy mamy do czynienia z~kolejnym kryzysem medycznym. Schodami sprowadzana jest młoda kobieta niosąca siedmiomiesięczne dziecko. Ona cicho szlocha. Dziecko krzyczy. Jej eskorta IMC rozpaczliwie poszukuje medyka.

-- Medycy? Myślę, że właśnie wyszli.

-- Czemu? Co się stało?

-- Czy dziecko jest chore?

-- Skurwiele je zagazowali.

-- Co? Zagazowali dziecko?

 Jej eskorta wyjaśnia, że  matka jest wolontariuszką Food Not Bombs, która była w~Zielonej Strefie, gdy zaatakowała policja. Natychmiast złapała swoje dziecko i~ruszyła na wyższy poziom, ale kanister wylądował bezpośrednio u jej stóp, gdy uciekała.

-- Och. Myślisz, że to był wypadek, czy naprawdę widzieli dziecko?

 Matka, która do tej pory milczała, spojrzała na niego. 
 
 -- Oczywiście, że widzieli dziecko! -- powiedziała, posługując się mocno akcentowanym angielskim. -- Byli trzydzieści metrów od nas!

-- Skurwysyny!

 Połowa osób w~pokoju zaniemówiła. Dwie kobiety zaproponowały, że potrzymają niemowlę, którego twarz była jasna, szkarłatna, próbowały kołysać i~uspokoić go. Dowiedzieliśmy się, że ma na imię Gabe.

-- Nie mogę uwierzyć, że zagazowali pieprzone dziecko. -- I~zastanawiają się, dlaczego ludzie rzucają kamieniami w~policję.

 Ktoś przynosi wodę; ktoś inny sugeruje, aby poczekali na medyków na samym szczycie klatki schodowej, gdzie na szóstym piętrze znajduje się podest z~otwartym oknem i~stosunkowo nieskażonym powietrzem (wcześniej był używany jako sekcja dla palących IMC). Jest po ósmej i~zaczynam myśleć, że pod autostradą będzie bezpieczniej, gdzie przynajmniej będą drogi ewakuacyjne. Kobieta, która należy do jakiegoś lokalnego kolektywu dokumentalnego, jest przy stercie sprzętu, trzymając moją maskę przeciwgazową. Czy może go użyć ,,tylko przez dziesięć minut?''. Jej ekipa chce tylko wyjść na zewnątrz, żeby zrobić kilka zdjęć policji.

-- No cóż, czy jest szansa, że  potrwa to dłużej niż dziesięć minut? Bo naprawdę muszę iść.

-- Nie, nie -- mówi. -- Zaraz wrócimy.

 Waham się, dokonuję subiektywnej oceny sytuacji. Jest zawodową kamerzystką z~aurą skuteczności, która przynajmniej mnie sugeruje ,,osobę, która kłamie w~takich sprawach, nawet o tym nie myśląc''. Z drugiej strony jesteśmy w~takiej sytuacji wspólnotowej, w~której nie można tak naprawdę odmówić bezpośredniej prośby bez wyraźnego powodu, a naprawdę nie mogę powiedzieć, że taki mam.

-- Dobra. Ale naprawdę będę potrzebować go z~powrotem za dziesięć minut.

 Pół godziny później nadal czekam. Spędzam trochę czasu w~biurze, po raz kolejny potwierdzając, że żadna z~niezliczonych ładowarek do telefonów komórkowych w~IMC w~rzeczywistości nie naładuje mojej marki telefonu. Próbuję sprawdzić, czy jest praca, którą mogę wykonać -- obiecałem poświęcić godzinę pracy, kiedy dostałem dowód osobisty. Ale wszyscy są zbyt rozkojarzeni. Nie można też znaleźć darmowego komputera, na którym mógłbym sprawdzić pocztę. Bez mojej maski praktycznie jestem tu uwięziony. W każdym razie, jeśli odejdę, na pewno nigdy jej nie odzyskam. Idę więc na górę, żeby pomóc dziecku, które wciąż jest na podeście. Nawet tam niewiele mogę zrobić poza wsparciem moralnym, ale to fascynująca przestrzeń, całkowicie betonowa i~industrialna, z~dwoma dużymi oknami w~stylu fabrycznym, lekko uchylonymi. Z jednego widać dach zajmowany przez policję. Są tylko dziesięć, piętnaście metrów dalej, niektórzy z~nich wciąż są całkowicie bezosobowi w~maskach przeciwgazowych, przyłbicach i~zbrojach. Wydaje się, że nie są nas świadomi.

 Nie mając nic lepszego do roboty, zacząłem bazgrać:

\medskip
\noindent [Znowu z~moich notatek]
\medskip

 Problem z~IMC polega na tym, że jest to bańka -- nie tylko w~sensie dosłownym (nikt nie chce otwierać drzwi ani okien i~ryzykować przedostania się gazu łzawiącego), ale także dlatego, że jest odizolowana od poczucia natychmiastowości, wspólnoty i~spontanicznej intymności, którą doznajesz na ulicach, gdzie stoisz w~obliczu ciągłego, namacalnego niebezpieczeństwa. Tutaj wszystko jest zapośredniczone. Znajdujesz się w~świetlistym pokoju pełnym ekranów i~monitorów, nie widzisz nic dla siebie, ale mimo to znasz każdą najgorszą rzecz, jaka się dzieje: każde aresztowanie, każde poważne zranienie, każdy nowy skandal policji. Wynikający z~tego nastrój nie jest dokładnie histerią; jest to raczej rodzaj maniakalnej nerwowości, która wynika z~posiadania zbyt dużej liczby informacji.

 Ale po zastanowieniu, czy nie z~tego zasadniczo składają się wiadomości? Raport krajowy w~dużej mierze składa się z~najgorszych rzeczy, które zdarzają się każdego dnia w~Ameryce. Międzynarodowy raport wymienia najgorsze rzeczy, które wydarzyły się na świecie.

 Wreszcie, około 20:45, ekipa wideo powraca, trajkocząc z~ożywieniem po francusku. Potem znowu mają odejść.

-- Przepraszam, moja maska  gazowa?

-- Och, tak.

 Na górze ochroniarz budynku wypuszcza ludzi tylko w~grupach, bojąc się wpuścić do budynku gaz. 
 
 -- Naprawdę nie polecam wychodzenia na zewnątrz w~tej chwili -- mówi do mnie. -- Wszędzie są gliniarze. To bardzo niebezpieczne.

 Mówię mu, że zaryzykuję. W końcu po około pięciu minutach ktoś puka do szklanych drzwi z~zewnątrz i~jestem z~powrotem na ulicy.

\bigskip
\noindent \texttt{20:50, Na Dworze}\medskip


 Wreszcie wolny! Przynajmniej, co dziwne, takie mam uczucie z~powrotem w~strefie wojny.

 Gliniarze zajmują drewnianą scenę na samym szczycie wielkich schodów prowadzących na Stare Miasto; wydaje się, że zajęli wszystkie najwyższe wzniesienia. Cały ten obszar miasta spowija gaz. Używają potężniejszego sprzętu wojskowego, który wszyscy nazywają ,,CS'', chociaż nie wiem, czy to naprawdę jest CS (ludzie z~IMC nie byli pewni). Samo oddychanie bez maski jest już fizycznie bolesne; przechodzenie przez nisko położone tereny pozostawia kaszel i~krztuszenie; regularnie padają nowe pociski. Na ulicy jest tylko kilka zacienionych postaci. Schodzę na pas za IMC, który wydaje się prowadzić na autostradę, i~prawie natychmiast wpadam na Kitty. Oboje zaczynamy się śmiać.

 Ściskamy się. To chyba siódmy raz, kiedy się dzisiaj przytulamy.

-- Więc co tam? Gdzie są wszyscy? Czy wszystko w~porządku?

-- Cóż, Andrea została trafiona dwa razy i~wróciła do domu. Oddała swoją maskę gazową Lee (dostaję śpiwór). Wszyscy inni są w~porządku. Wszyscy jesteśmy w~Tymczasowej Strefie Autonomicznej pod autostradą. Od co najmniej godziny jesteśmy atakowani. To niesamowite! Teraz są tam tysiące ludzi, przybywa coraz więcej. Doszło do zaciętej bitwy i~wygraliśmy.

 Następnie opisuje budowę gigantycznego ogniska w~przestrzeni TAZ, aby zneutralizować gaz łzawiący. Policja przyniosła armatkę wodną,  aby spróbować je zgasić. Ale ludzie to wytrzymali. Tymczasem dołącza do nas coraz więcej zwykłych obywateli. Na autostradzie są teraz tysiące. Nazywają ich ,,bangers'', ponieważ już od godziny rytmicznie uderzają w~metalowe barierki na poboczu autostrady, robiąc tyle hałasu, że można go łatwo usłyszeć w~Convention Center dziesięć przecznic dalej. Policja również zaczęła ostrzeliwać autostradę i~wysłała linie funkcjonariuszy, aby oczyścili teren za pomocą plastikowych kul, ale bezskutecznie. Nawet kiedy zaczęli używać armatek wodnych. Starzy ludzie, rodziny, związkowcy, wszyscy zaczęli rzucać na gliniarzy deszcz cegieł, desek i~płonących gruzu. Wreszcie, policja się wycofała.

-- Więc, co tutaj robisz? -- pytam.

-- Słyszeliśmy plotkę, że mogą ruszać na IMC. Przyszłam sprawdzić, czy ludzie nie potrzebują tu pomocy. A ty?

-- Sprawdzałem wiadomości o Karen i~zostałem uwięziony w~IMC na godzinę, kiedy ktoś pożyczył moją maskę przeciwgazową.

-- Och, słyszałam, że ktoś skontaktował się z~Saszą, która powiedziała, że  Karen została aresztowana i~zabrali ją do Montrealu.

-- Naprawdę? Od kogo to usłyszałaś?

-- Kogoś. -- Myśli przez chwilę. -- Nie, nie pamiętam. Może ktoś z~Nowego Jorku? A czy wiesz coś o tej plotce, że ktoś zginął przy autostradzie?

-- To wszystko, o czym rozmawiali w~IMC przez ostatnią godzinę lub dwie. Ale wydaje się, że nikt nie wie, czy facet naprawdę nie żyje.

 Przeszukując policyjne pozycje wokół IMC, wciąż wpadamy na starych znajomych. Simon z~Nowego Jorku wychodzi z~mgły w~hełmie, tarczy oraz ochraniaczach na ręce i~golenie, dokładnie takich, jakich używaliśmy w~Ya Basta! Wydaje się tak samo zaskoczony jak my, że udało mu się przez to przejść, i~tak zadowolony z~siebie, jak tylko ktokolwiek mógłby być. Mówi, że wielu nowojorczyków w~końcu się przebija. Dołączamy do większości Refugees, różnych elementów Czarnego Bloku oraz lokalnych mieszkańców i~tworzymy prowizoryczną obronę IMC. Gdy policyjne helikoptery brzęczą nad głowami, ludzie demontują płyty ze sklepów, które były zabite deskami, i~rozpalają ognisko. Potem wszyscy zaczynamy budować barykady, korzystając z~metalowych ogrodzeń zebranych z~małego parku u podnóża schodów.

 To nie jest za wcześnie, ponieważ autobusy i~furgonetki pełne policyjnych posiłków zaczynają koncentrować się zaledwie przecznicę lub dwie dalej. Następuje bitwa. Zostajemy wypędzeni z~naszych pozycji, rozpraszamy się, wracamy, ponownie budujemy barykady. Wykonujemy niekończące się telefony, próbując skłonić reporterów z~prasy korporacyjnej, by byli świadkami tej sceny, mając nadzieję, że ich obecność powstrzyma policję przed wtargnięciem do budynku. Nigdy nie odpowiadają. Niemniej jednak, pomimo kilku pocisków z~gazem łzawiącym wbitych w~okna na klatce schodowej, policja nigdy nie wchodzi do samego budynku.
 
\bigskip
\noindent \texttt{22:45, Côte D’Abraham}\medskip


 W końcu mamy szansę spłacić nasze zaangażowanie w~pracę na rzecz IMC. Shawn ma radio i~zgadza się robić raporty uliczne na zmianie od 23 do 4. Daje to także Refugees nowy powód i~pretekst, by mniej więcej śledzić wydarzenia w~tej części miasta.

 Samo miasto nabrało charakteru niemal powstańczego. Wkrótce okazuje się, że policja kompletnie przesadziła. Rozpraszając swoje siły tak daleko od muru, nie mieli żadnych wyraźnych stref kontroli: nawet większość Jean Baptiste jest ponownie wyzwolonym terytorium, z~barykadami i~ogniskami wzniesionymi w~kilkunastu różnych miejscach. Wędrujemy wzdłuż Côte d'Abraham, krętej ścieżki wzdłuż stromego urwiska, u samych stóp Starego Miasta, próbując znaleźć drogę powrotną. Drogą idą odosobnione grupy ludzi. Wielu wydaje się apolitycznymi, miejscowymi chłopcami, którzy dobrze się bawią. Jedna dobroduszna załoga wznosi toast na nas, gdy przechodzą: ,,To bardzo dobra noc na picie piwa!''. Inny młody mężczyzna został trafiony plastikową kulą w~pośladki i~pokazuje pręgę wszystkim, których spotyka. (,,Popatrz? Widzisz, co te świnie mi zrobiły?''). To jakby koleżeństwo z~wczoraj rozciągnęło się na całe miasto, choć, gdy wspinamy się na Stare Miasto, widzimy kilka dzieł pijanego przypadku, gdy butelki rozbijają szyby zamkniętych sklepów.

 Jednym z~nich była narożna drukarnia, która wydawała się dość wyraźnie lokalna.

-- Tsk. To nie jest uzasadniony cel, prawda? -- mówi Lyn.

-- Z drugiej strony -- zauważa Heidi -- w~porównaniu z~tym, co dzieje się po mistrzostwach hokeja w~tym mieście, to nic. Uszkodzenia zwykle wynoszą setki tysięcy dolarów. Myślę, że nawet chuligani się powstrzymują.

 Na Starym Mieście afrykańscy i~azjatyccy imigranci są wśród tłumów broniących pozycji przed policją. Dzieci i~starcy już zostali ewakuowani. Ciągle wpadamy na aktywistów z~Nowego Jorku. Brad Will, eko-aktywista mieszkający w~nowojorskim IMC, właśnie przybył do miasta; ma ogromny plecak, a jego twarz jest owinięta podartą koszulką, cuchnącą octem. 
 
 -- Problem -- mówi -- jest taki, że ludzie po prostu nie radzą sobie z~gazem. Wygnalibyśmy ich z~całej okolicy, gdyby nie gaz.

 Brad kieruje nas do szczególnie dramatycznej sceny na górze wzgórza, która może być warta opisania. Ben i~ja wspinamy się na wzgórze, żeby zbadać sprawę. Trwa wielka bitwa, gdy mieszkańcy kucają za barykadą sof, drewnianych drzwi i~bibelotów najwyraźniej wyciągniętych z~piwnic; gliniarze strzelają do nich z~pozycji za trzema lub czterema pojazdami policyjnymi dalej na ulicy. Młodzi mężczyźni wlewają benzynę i~cukier do pustych butelek z~dużego plastikowego kanistra; potem zapychają butelki szmatami i~zostawiają je przy krawędzi barykady. Od czasu do czasu ktoś bierze jeden, wbiega na małą działkę między dwoma budynkami, zapala go i~rzuca w~policję, a potem ucieka. Z kolei policja strzela zza barykady bombami pieprzowymi, aby zmusić ludzi do podniesienia się, następnie strzela im w~głowy plastikowymi kulami. Patrzę, jak mołotow wypływa w~górę, nie trafia w~cel i~ląduje na drewnianym nadprożu okna na piętrze, wzniecając mały ogień. Nikt nie wydaje się szczególnie zaniepokojony.

-- Jej, oni zamierzają spalić własną dzielnicę!

 Chwilę później byłem ślepy i~nie mogłem oddychać. Kolejna bomba pieprzowa. Mam wyraźne wspomnienie, kiedy mówiłem sobie ,,trzymaj głowę w~dole, trzymaj głowę w~dole'', a sekundę później czułem się, jakby ktoś właśnie rozbił mi butelkę na głowie. To dziwne, bo nikt nigdy nie stłukł mi butelki nad głową i~właściwie nie mam pojęcia, jak by to było, ale to była moja pierwsza reakcja. Najwyraźniej udało mi się utrzymać w~większości na dole, a kula odbiła się rykoszetem od samego czubka głowy i~zatrzymała się dziesięć metrów za nami. Na sekundę usiadłem na ziemi, a potem cofnąłem się, gdy ktoś przebiegł obok, wykrzykując coś pomocnego po francusku. Z powrotem u podnóża wzgórza Brad, wciąż tryskający gazem łzawiącym, podarował mi kulę -- a przynajmniej była to prawdopodobnie ta sama kula. 
 
 -- Jeśli to twoja pierwsza -- powiedział -- możesz ją zatrzymać.

-- Dzięki.

 Kula jest gigantyczna: zrobiona z~czegoś, co przypomina twardą, zieloną gumę, w~kształcie młotka, wystarczająco duża, by wypełnić moją dłoń. Mówię sobie: to szczęście, że godzinę temu połknąłem jedną z~tych tabletek z~kodeiną, tak na wszelki wypadek.

 Kilka godzin później lądujemy w~małym, pełnym sklepów parku na skraju dolnego miasta, na rogu Coronne i~Charest, gdzie kolejne wielkie ognisko stało się centrum spontanicznej imprezy ulicznej. Ktoś wyciągnął nagłośnienie, ludzie palą narkotyki i~tańczą w~blasku płomieni. Inni napływają z~autostrady lub zostają wezwani do bitw kilka przecznic dalej. Od czasu do czasu pojawiają się samochody, jedno spojrzenie na scenę i~desperacko zawracają. Kiedy wracamy do domu około czwartej nad ranem, mówi się o kolejnym ataku na mur, tym razem z~Równiny Abrahama. Krążą też pogłoski, że rząd wzywa wojsko.

\section{Niedziela, 1 kwietnia}

 Następnego ranka wszyscy byliśmy rozgorączkowani.

\noindent Ben: To był ogromny sukces.

\noindent Shawn: To było zdecydowanie najbardziej imponujące demo, w~jakim kiedykolwiek byłem.

I~wiem, że mieszkańcy Quebec City wkrótce będą mieli kolejną.

-- Więc kim dokładnie byli -- spytałem -- ci wszyscy ludzie, którzy całą noc hałasowali na autostradzie? Czy to naprawdę byli związkowcy ze Szczytu Ludowego?

-- To niesamowita rzecz -- powiedziała Lyn. -- Oni byli wszystkim. Ludzie z~Unii. Dzieci. Babcie. Starzy hipisi. Zwykli obywatele wszelkiego rodzaju.

-- Widziałem dzieciaki z~liceum -- ktoś się wtrąca -- matki z~dziećmi, jedna drużyna matka-córka waliła kijami w~barierki. Ludzie stworzyli coś w~rodzaju improwizowanego systemu rotacji, aby mieć pewność, że dźwięk nigdy nie będzie cichnąć.

-- Wielu członków związku przyjechało z~maskami i~butelkami octu z~nimi w~autobusie, już zorganizowanych w~grupy afinicji i~tak dalej.

-- Jeśli się nad tym zastanowić -- powiedział Shawn -- było to idealne nieposłuszeństwo obywatelskie, ponieważ mogliśmy zrobić ten ogromny hałas, który można by usłyszeć z~odległości kilometra. Na pewno mogli to usłyszeć na Szczycie i~w hotelach, w~których przebywali delegaci. Ale jednocześnie po prostu nie dało się nas usunąć. Ludzie zaczynali bić już w~południe, a ja wróciłem kilka godzin później i~nadal szło tak samo dobrze.

-- Ponadto byli tak wysoko, że wydaje mi się, że delegaci w~Convention Center rzeczywiście mogli ich zobaczyć.

 Rozmowy takie jak ta miały trwać przez wiele dni, a nawet tygodni, i~stopniowo krystalizować w~formalnych ,,raportach'' dla grup w~domu, narracjach internetowych i~opublikowanych raportach IMC, filmach i~książkach, o których wszyscy wiedzieliśmy, że w~końcu wyjdą z~tego, jeśli ograniczą się do prawie wyłącznie aktywistycznej publiczności. Wszak podczas akcji otacza nas niemal nieskończoność potencjalnych narracji, jedne bardziej bezpośrednie (,,gliniarze wkraczają do IMC!'', inne bardziej abstrakcyjne (,,Brazylijczycy szukają pretekstu do sabotowania Szczytu''), wszystkie otwarte, niepewne, o których większość wie, że okaże się nieistotna lub nieprawdziwa. Nikt, nawet w~IMC, nie jest w~stanie zacząć spekulować o tym, jak cała historia zostanie później opowiedziana, zwłaszcza kto wygrał. Jeśli gra się w~grę, zasady gry -- nawet dokładny charakter pola i~graczy -- były nieustannie negocjowane i~renegocjowane, w~akcji. Nikt zaangażowany w~to nie miał bezpośredniego kontaktu z~więcej niż maleńkim procentem (ja na przykład nigdy nie widziałem parady, Bridge CD czy Living River) i~tylko z~perspektywy czasu mogliśmy wymyślić prawdopodobną teorię, jaka była stawka bitwy. Nie chodzi o to, że istnieje kiedykolwiek jedna ostateczna historia, nawet po latach -- nigdy nie ma, z~żadnym wydarzeniem historycznym. Ale te rozmowy odegrały kluczową rolę w~zawężeniu kwestii. 

 Do południa wróciliśmy na kolejną radę przedstawicielską CLAC, gdzieś na Côte d’Abraham. Nocne bitwy dobiegły końca, ogniska nawet się nie tliły -- wszystko dobiegło końca. Wydawało się nawet, że barykady były systematycznie niszczone przez buldożery, a duża liczba aktywistów opuściła już miasto. Wiele dyskusji dotyczyło tego, czy uda się zebrać wystarczającą liczbę ludzi, aby maszerować na Ministerstwo Sprawiedliwości, aby zaprotestować przeciwko policyjnym represjom w~weekend. Miało się też odbyć demo w~Grand Théâtre, w~pobliżu armatki wodnej i~impreza gdzieś indziej, ale nic na tyle inspirującego, żebyśmy nie wycofali się z~powrotem na Uniwersytet, żeby zacząć zbierać swoje rzeczy na wyjazd.

 Najniebezpieczniejsze są zawsze końcowe dni akcji. W dużych mobilizacjach liczba aktywistów osiąga szczyt na początku, a następnie stale spada z~powodu obrażeń, aresztowań, a wkrótce ludzie po prostu wracają do swojego życia lub pracy. Z drugiej strony liczebność policji pozostaje stała. Gdy tylko równowaga sił zacznie się znacznie przechylać, zwykle zaczynają mścić się za upokorzenia z~poprzednich dni. Wszelkiego rodzaju działania stają się coraz bardziej niebezpieczne; często chodzenie ulicą, ponieważ gliniarze często rozpoczynają przypadkowe masowe aresztowania, których wcześniej nie byli w~stanie przeprowadzić. Każdy, kto chodzi samotnie w~ekwipunku, a nawet w~zielonych włosach, kolczykach lub tatuażach, może być celem; ale małe grupy też niekoniecznie są bezpieczne. W tym samym czasie dopiero w~fazie skończenia ci, którzy brali udział w~akcjach, zaczynają mieć jasny obraz tego, co się wydarzyło -- są w~stanie oddzielić dobre informacje od złych, a przede wszystkim zacząć konstruować ogólny obraz wydarzenia jako całości. Rezultatem jest połączenie rosnącej paranoi w~terenie i~ogromnego przepływu nowych i~retrospektywnych informacji. Wyglądało to tak, jakby poczucie, które miałem w~IMC -- połączenie rozległej panoramy i~klaustrofobicznego terroru -- teraz rozszerzyło się, by wypełnić całe miasto, a przynajmniej te części, które zamieszkiwali aktywiści.

\bigskip
\noindent \texttt{14:15}\medskip
 

 W Laval Mac ciężko pracował, odpowiadając na telefony i~przeglądając listy aresztowanych w~kancelarii prawnej. Shawn przeprowadził wywiad z~organizatorem CASA z~Comité Populaire du St. Jean-Baptiste, który podkreślił potrzebę odejścia od Szczytów i~pracy w~społecznościach. Plotka głosiła, że  właśnie przyjechało więcej ludzi z~Nowego Jorku. Wróciłem do sali gimnastycznej, teraz w~dużej mierze pustej, z~wyjątkiem niekończących się stosów plecaków, aby ich znaleźć. Zostało co najwyżej sto osób. Montreal Ya Basta! grała na perkusji małą improwizację. Spędziłem z~nimi trochę czasu na pogawędce, robiąc notatki na temat sprzętu i~taktyk, które chciałem przedstawić w~nowojorskiej Ya Basta!, jeśli rzeczywiście taka jeszcze istniała.

 W rzeczywistości była tam siedmioosobowa grupa, która właśnie przeszła, w~tym Eric i~Enos z~Nowego Jorku oraz słynny aktywista Bork z~DC. Spotkanie z~nimi było początkowo trochę dezorientujące. Spędziłem właśnie dwa dni na ulicach, gdzie każdy, kogo spotkałeś, kto nie strzelał do ciebie, był twoim bratem lub siostrą; po prostu szli do akcji, pełnej tajnych planów i~ponurej intensywności. Mimo to muszę się trochę dowiedzieć o tym, co przydarzyło się moim przyjaciołom. Kiedy wszyscy odwrócili się przed bramką celną w~Akwesasne, zebrali się, aby zdecydować, co dalej. Zapadła noc, nasi nieliczni klienci Mohawk przekroczyli granicę i~porzucili nas, a postacie w~ciemności zaczęły od czasu do czasu strzelać paintballem w~nasze pojazdy. Karawana wróciła do Burlington dopiero około 3 nad ranem. Niektórzy poszli do domu. Niektórzy próbowali nazajutrz poddać się odprawom celnym w~innych miejscach. Niektórzy się przedostali. Wszystkie osoby zgłaszały agresywne przesłuchania mające na celu ustalenie, czy są w~jakikolwiek sposób powiązani z~organizacją o nazwie ,,Ya Basta!''. Warcry dołączył do załogi nazwanej ,,Brygadą Rakiet Śnieżnych'', która przeszła pieszo przez las w~środku nocy. Zostali złapani, gdy jeden niedoświadczony dzieciak wpadł w~panikę i~zapytał gliniarza o drogę. Pozostały kontyngent Ya Basta! próbował skorzystać z~legalnej drogi, ale po dwukrotnym poddaniu się odprawie celnej wszyscy znaleźli się w~areszcie imigracyjnym. Z wyjątkiem, zabawnie, Moose. Po prostu został zawrócony. Sasha był z~nimi zamknięty: dlatego jego telefon nie działał. Wszystkich zabrano do Montrealu w~celu przeprocesowania i~prawdopodobnie wypuszczono za kilka dni. Karen, która, jak wychodzi na jaw, rzeczywiście wyjechała w~piątek bez mówienia komukolwiek, jest już w~Montrealu, żeby spróbować dotrzeć do Sashy.

 W końcu organizujemy małe spotkanie Nowego Jorku. Brad donosi, że ulice stają się coraz bardziej niebezpieczne, wszędzie pojawiają się czarne SUV-y, a także dyskretne vany z~osłonami, które wydają się kanadyjskim wywiadem. Zabierają każdego, kto ma sprzęt -- na pewno ochraniacze lub tarcze, ale nawet medyków lub dziennikarzy IMC z~kamerami wideo. Simon został aresztowany dziś rano. Kilku innych nowojorczyków pojawiło się w~mieście tylko po to, by zostać natychmiast przyłapanym w~obławie.

 Wymyślamy plan. Ci, którzy brali udział w~akcji, powinni chyba najlepiej wyjechać z~miasta. Wrócimy do Montrealu i~udzielimy wsparcia więziennego naszym przyjaciołom z~aresztu imigracyjnego, którzy wkrótce powinni stawić się na przesłuchania.

 Wracając do kancelarii, ze zdziwieniem odkrywam Rufusa, starego przyjaciela i~legendarnego medyka akcji z~Nowego Jorku, czekającego w~kolejce po wegetariańskie burrito w~darmowej kuchni, która została założona w~pobliskim holu. Kitty i~Lee też tam są. Rufus pracował z~zespołem medycznym od soboty i~ma wszystkie szczegóły dotyczące ofiar. Okazuje się, że rzeczywiście ktoś został postrzelony w~gardło, ale żyje. Na chwilę przestał oddychać, ponieważ miał zmiażdżoną krtań, ale medycy zdołali go zmusić do oddychania, a lekarze później uratowali mu życie, wykonując tracheotomię. Nigdy więcej już nic nie powie. To była najgorsza pojedyncza kontuzja. Innemu mężczyźnie oderwał palec, próbując zburzyć Mur, ale sanitariusz przyszył go z~powrotem. (Kitty: ,,\textit{Och, widziałam, jak to się stało! To nie było z~bliska, ale\ldots  ściskał ten sznur i~próbował zburzyć fragment ogrodzenia, kiedy ten gliniarz wspiął się na płot i~szarpnął go z~powrotem. Jego palec odpadł. Po prostu stał tam oszołomiony, a wszyscy krzyczeli ,,Medyk!'' Potem jeden podbiegł, złapał za palec i~odszedł z~nim}''). Był jeszcze jeden, który stracił ucho, gdy pojemnik z~gazem łzawiącym uderzył w~jego kolczyk. Dużo złamanych ramion i~złamanych żeber.

-- Nie zostałeś trafiony, prawda? -- zapytał Lee. -- Ponieważ na pewno celowali w~ulicznych medyków. Widziałem to. Nie tylko strzelali, żeby ich przestraszyć, ale \textit{celowali} w~nich.

 Na głównej drodze biegnącej przez kampus znajduje się długa kolejka autobusów; co godzinę, cztery lub pięć wyjeżdża, by zawieźć ludzi z~powrotem do Montrealu. Pojawia się pytanie, czy trzeba być studentem, ale wydaje się, że nikt nie sprawdza legitymacji. Wielka historia w~lokalnych gazetach jest taka, że  wszystkie duże hotele i~restauracje musiały wyrzucać tony jedzenia, ponieważ było ono skażone gazem, i~że podobno George W. Bush próbował wypić łyk ze skażonej butelki i~miał wypluć to wszystko -- choć trudno sobie wyobrazić, jak to się naprawdę stało.

 Zbieramy się w~małą grupkę: Rufus, Kitty i~Lee, Janna, jeszcze kilku.

\bigskip
\noindent \texttt{16:25}\medskip
 

 Przechodzi marsz obok Des Jardins, może dwustu ludzi, prowadzonych przez czerwono-czarne flagi. Myślę, że kierują się do Ministerstwa Sprawiedliwości. Kitty, która dołączy do nas w~więzieniu, jakoś zdobyła dla nas czarną flagę i~sztandar.

 Jakimś cudem kancelaria ma kompatybilną ładowarkę do telefonów komórkowych. Mając około piętnastu minut baterii, dzwonię do Alison Haynes, reporterki Montreal Gazette, do której miałam zamiar zadzwonić przez cały weekend. Okazuje się, że ona też była w~banku CIBC, prawdopodobnie jednym z~tych reporterów, których zauważyłem wśród gapiów. Mówi, że później przeprowadziła wywiad z~tęczową parą. Byli z~Vancouver. Po naszym wyjściu napisali notatkę do CIBC: 
 
 -- Przepraszamy, zrobiliśmy co w~naszej mocy, aby uratować twój bank.

 Nie rozmawiałem z~nią dłużej niż minutę lub dwie, kiedy Rufus przychodzi mi powiedzieć, że spóźnimy się na nasz autobus. Potem oczywiście telefon umiera. Następnego dnia w~Montrealu biorę gazetę i~znajduję artykuł z~krótkim cytatem ze mnie, wyjaśniającym, że został przerwany przez moją konieczność wywiezienia mnie z~miasta.
 
\bigskip
\noindent \texttt{18:25, Autobus do Montrealu}\medskip
 

 W autobusie wszyscy wymieniają opowieści wojenne. Kilku montrealskich Yabbów już wraca do domu. Greg wymienia trzy cele korporacyjne, które zostały trafione: CIBC, stację Shell Oil, która została zdewastowana (napastnicy namalowali sprayem słowa ,,Viva Ken Saro Wiwa!'') oraz sklep z~kanapkami Subway. Nie McDonald, jak niektórzy mówili. Wybrano Subway, ponieważ była to druga co do wielkości sieć fast foodów w~Ameryce Północnej i~była własnością Kanady. Ponadto niektórzy ludzie zniszczyli jedną z~ciężarówek telewizyjnych pozostawionych na środku parku, aby zaprotestować przeciwko mediom korporacyjnym. Greg wątpi co do ,,małych zamieszek tamtej nocy''. To było dość kiepskie. Nie widziałem tego, ale słyszałem, że grupa nacjonalistów z~Quebecu oszalała i~siała spustoszenie w~całym Starym Mieście. Słyszałem, że wybili nawet szyby w~naszej klinice!

-- Nie, nie -- powiedziałem -- to byli gliniarze.

-- Jesteś pewien?

-- Absolutnie. Byłem wtedy w~IMC. Rozmawiałem nawet z~medykami, którzy przyszli później, aby znaleźć nowe miejsce.

-- Och. Nadal nie wiem. Nie brałem udziału w~wyborze celów, ale wiem, że włożono w~to wiele namysłu. Jeden bank, jedna firma naftowa, jedna sieć fast foodów, jedna sieć telewizyjna. Po prostu nie znoszę patrzeć, jak banda pijanych chłopaków z~bractwa wychodzi i~osłabia wiadomość.

 Dwoje dzieciaków z~terytorium Tyendinaga Mohawk opowiada o tym, jak przybyli na Akwesasne, ale nie mogli przedostać się przez linię policyjną w~Cornwall.

-- Serio? -- pytam. -- Ponieważ nikt z~nas nie był naprawdę pewien, czy naprawdę mamy jakiekolwiek wsparcie społeczności.

-- Nie, nie, po prostu nie mogliśmy wejść, ponieważ policja była tam z~tarczami i~pałkami blokującymi drogę wszystkim. Pieprzone świnie! To jest nasz pieprzony dom i~wyglądało to tak, jakby był pod okupacją wojskową.

-- Tak -- mówi drugi dzieciak. -- Byliśmy gotowi do zamieszek. Byliśmy z~karawaną w~Windsor i~chcieliśmy dołączyć do was w~Akwesasne. Ale było ich po prostu za dużo.

-- Serio? -- pytam. -- Och. Szkoda tylko, że wtedy o tym nie wiedzieliśmy. Czuliśmy się tam strasznie samotni.

 Głównie jednak wszyscy są po prostu wyczerpani. Kitty przez chwilę wygląda przez okno. 
 
 -- Co za dziwny zjazd -- mówi. -- Wiesz, co to przypomina? Schodzenie z~kwasu. Wiesz, jak wtedy, gdy potykasz się od kilku dni i~zjeżdżasz i~nagle wszystko jest \textit{do bani}?

 Lee się zgadza. Nadal czuje się dziwnie z~powodu mołotowa. 
 
 -- Czuję się brudny i~zużyty.

 -- Nie wiem. -- Kitty. -- W każdym razie niezużyta. Ale problem polega na tym, że kiedy schodzisz z~akcji, nie ma sposobu, aby po prostu przyjąć kolejny cios.

\chapter{Akcje bezpośrednie, Anarchizm, Demokracja Bezpośrednia}

 Ponieważ jest to książka o akcji bezpośredniej, najlepiej zacząć od wyjaśnienia, co to jest.

\section{Co to jest akcja bezpośrednia?}

Przez lata setki anarchistów próbowały odpowiedzieć na to pytanie w~broszurach, ulotkach i~przemówieniach. Oto próbka:

\bigskip

\textit{ Akcja bezpośrednia oznacza działanie dla samego siebie, w~sposób, w~którym można bezpośrednio rozważyć problem, z~którym się mierzymy, bez potrzeby pośrednictwa polityków lub biurokratów. Jeśli widzisz buldożery, które zamierzają zniszczyć twój dom, angażujesz się w~bezpośrednie działania, aby bezpośrednio interweniować, aby spróbować je powstrzymać. Akcja bezpośrednia przeciwstawia sumienie moralne urzędowemu prawu\ldots  Jest wyrazem gotowości jednostki do walki, przejęcia kontroli nad własnym życiem, do bezpośredniego wpływania na otaczający nas świat, do wzięcia odpowiedzialności za swoje czyny.}

\begin{flushright}
\textit{Sans Titres Bulletin}, ,,Czym jest działanie bezpośrednie?”
\end{flushright}
\bigskip

\textit{ Weźmy pospolity przykład. Jeśli rzeźnik waży czyjeś mięso i~dociska kciuk na wadze, można na to się poskarżyć i~powiedzieć mu, że jest bandytą, który okrada biednych, a jeśli upiera się i~nie robi się nic więcej, to tylko gadanie; można wezwać Departament Wag i~Miar i~jest to działanie pośrednie; można też, bez rozmowy, nalegać na samodzielne zważenie mięsa, zabrać ze sobą wagę, żeby sprawdzić wagę rzeźnika, przenieść interes w~inne miejsce, pomóc otworzyć sklep spółdzielczy itp., a to są działania bezpośrednie.}

\begin{flushright}
David Wieck, ,,Habits of Direct Action”
\end{flushright}
\bigskip

\textit{ Działanie bezpośrednie ma na celu osiągnięcie naszych celów poprzez własną działalność, a nie poprzez działania innych. Chodzi o to, by ludzie wzięli władzę dla siebie. Pod tym względem różni się od większości innych form działań politycznych, takich jak głosowanie, lobbing, próby wywierania nacisku politycznego poprzez akcje protestacyjne lub za pośrednictwem mediów. Wszystkie te działania\ldots  oddają naszą władzę istniejącym instytucjom, które działają, aby uniemożliwić nam działanie na rzecz zmiany status quo. Akcja Bezpośrednia odrzuca taką akceptację istniejącego porządku i~sugeruje, że mamy zarówno prawo, jak i~moc zmieniania świata. Pokazuje to przez robienie tego. Przykłady akcji bezpośredniej obejmują blokady, pikiety, sabotaż, skłoting, zagważdżanie drzew, lokauty, okupacje, strajki, spowolnienia, rewolucyjny strajk generalny. We wspólnocie działanie bezpośrednie zawiera, między innymi zakładanie własnych organizacji, takich jak spółdzielnie żywnościowe oraz radio i~telewizja z~dostępem do społeczności\ldots  Akcja Bezpośrednia to nie tylko metoda protestu, ale także sposób na ,,budowanie przyszłości już teraz''. Każda sytuacja, w~której ludzie organizują się, aby rozszerzyć kontrolę nad własną sytuacją bez uciekania się do kapitału lub państwa, stanowi Akcję Bezpośrednią\ldots  Tam, gdzie się powiedzie, Akcja Bezpośrednia pokazuje, że ludzie mogą kontrolować swoje życie -- w~efekcie możliwe jest społeczeństwo anarchistyczne.}

\begin{flushright}
Rob Sparrow, ,,Anarchist Politics and Direct Action”
\end{flushright}
\bigskip

\textit{ Każda osoba, która kiedykolwiek myślała, że  ma prawo coś domagać, i~śmiało tego dowiodła, sama lub wspólnie z~innymi, którzy podzielali jej przekonania, działała bezpośrednio. Każda osoba, która miała plan, lub przedstawiła swój plan innym i~zdobyła ich współpracę, aby to zrobić z~nim, bez zwracania się do władz zewnętrznych, aby zrobić to za nich, była za akcją bezpośrednią\ldots  Każda osoba, która kiedykolwiek w~jego życiu miała cokolwiek z~kimkolwiek do załatwienia i~udała się prosto do tych zaangażowanych osób, aby to załatwić, albo według pokojowego planu, albo w~inny sposób, działała bezpośrednio.}

\begin{flushright}
Voltairine De Cleyre, ,,Direct Action”
\end{flushright}

\bigskip
\textit{ Człowiek ma tyle wolności, ile gotów jest jej sobie wziąć. Anarchizm opowiada się zatem za bezpośrednim działaniem, otwartym nieposłuszeństwem i~oporem wobec wszystkich praw i~ograniczeń: ekonomicznych, socjalnych i~moralnych. Jednak nieposłuszeństwo i~opór to rzeczy nielegalne. i~w tym właśnie tkwi zbawienie człowieka. Wszystko, co poza prawem, wymaga uczciwości, samowystarczalności i~odwagi. Krótko mówiąc, anarchizm domaga się duchów wolnych, niezależnych, ,,ludzi, którzy są ludźmi, i~którzy mają twardy i~sztywny kręgosłup''.}

\begin{flushright}
Emma Goldman, ,,Anarchizm: co to naprawdę oznacza”
\end{flushright}

\bigskip
\bigskip

 Teraz powinno być dość łatwo zrozumieć, dlaczego anarchistów zawsze pociągała idea akcji bezpośredniej. Anarchiści odrzucają państwa i~wszystkie te systematyczne formy nierówności, które państwo umożliwia. Nie starają się wywierać nacisku na rząd, aby wprowadził reformy. Nie starają się też przejąć dla siebie władzy państwowej. Chcą raczej zniszczyć tę moc, używając środków, które są -- na ile to możliwe -- zgodne z~ich celami, które ich ucieleśniają. Chcą ,,zbudować nowe społeczeństwo w~skorupie starego''. Akcja bezpośrednia jest z~tym doskonale zgodna, ponieważ w~swej istocie akcja bezpośrednia jest naciskiem, w~obliczu struktur niesprawiedliwej władzy, na działanie tak, jakby ktoś był już wolny. Nie zabiega się o państwo. Niekoniecznie wykonuje się nawet wielki gest sprzeciwu. O ile jest się zdolnym, postępujemy tak, jakby państwo nie istniało.

 Taka jest w~zasadzie różnica między działaniem bezpośrednim a obywatelskim nieposłuszeństwem (chociaż w~praktyce często obie te rzeczy się pokrywają). Kiedy ktoś pali kartę poboru, wycofuje zgodę lub współpracę ze struktury władzy, którą uważa się za nieuprawnioną, jest to nadal forma protestu, aktu publicznego skierowanego przynajmniej częściowo do samej władzy. Zazwyczaj osoba praktykująca nieposłuszeństwo obywatelskie jest również gotowa zaakceptować prawne konsekwencje swoich działań. Działanie bezpośrednie idzie o krok dalej. Działaczka nie tylko odmawia płacenia podatków, by wesprzeć zmilitaryzowany system szkolny, ale łączy się z~innymi, próbując stworzyć nowy system szkolny, który działa na innych zasadach. Działaczka postępuje, jakby Państwo nie istniało i~pozostawia decyzję reprezentantom państwa, czy wysłać uzbrojonych ludzi, żeby ją powstrzymali.

 Czytelnik mógłby się teraz sprzeciwić: z~pewnością akcja bezpośrednia zwykle wiąże się z~bezpośrednią konfrontacją z~przedstawicielami państwa. Nawet jeśli nie zaczyna się od takiej konfrontacji, wszyscy są świadomi, że prawdopodobnie w~końcu do niej dojdzie. To z~pewnością wydaje się sugerować uznanie ich istnienia. To prawda, ale nawet tutaj sprawy są bardziej subtelne. Kiedy dochodzi do konfrontacji, dzieje się tak zazwyczaj dlatego, że ci, którzy prowadzą akcję bezpośrednią, upierają się przy tym, by zachowywać się tak, jakby przedstawiciele państwa nie mieli większego prawa do narzucania swojego poglądu na temat praw lub błędów sytuacji niż ktokolwiek inny. Jeśli mężczyzna jedzie ciężarówką pełną toksycznych odpadów, aby wyrzucić ją do lokalnej rzeki, działaczka bezpośrednia nie zastanawia się, czy korporacja, którą reprezentuje, jest do tego prawnie dozwolona; traktuje go jak każdego, kto próbuje wrzucić kadź z~trucizną do lokalnego źródła wody. (Przy takiej interpretacji, fakt, że wspomniany działacz bezpośredni rzadko próbuje po prostu fizycznie obezwładnić sprawcę, jest niezwykłym świadectwem oddania większości aktywistów wobec niestosowania przemocy). Zwykle wniosek jest taki, że jest to uzasadnione dla każdego sumienia mężczyzny lub kobiety w~pobliżu, aby połączyć siły, aby spróbować odwieść niedoszłą wywrotkę, a jeśli to konieczne, powstrzymać ją, powiedzmy, kładąc się przed ciężarówką lub przebijając jej opony. Jeśli to zrobią, a wtedy pojawi się dwudziestu uzbrojonych mężczyzn w~niebieskich strojach i~każe im oczyścić ulice, to z~kolei nie traktują tego żądania jako porządku prawnego, ale raczej jako moralnie równorzędne z~każdym innym żądaniem grupy ludzi stojących na ulicy. W związku z~tym, jeśli policja zażąda, aby ci, którzy blokują ciężarówkę, opróżnili ulicę, ponieważ karetka pogotowia próbuje przejechać, prawie na pewno zastosują się do tego; jeśli policja wystąpi z~takimi żądaniami po prostu ze względu na swoją władzę prawną jako przedstawicieli miasta, osoby blokujące je zignorują; jeśli grożą atakiem, osoby blokujące zastanowią się, czy są w~stanie podjąć ryzyko związane z~zajęciem stanowiska\footnote{Oczywiście w~rzeczywistej sytuacji akcji bezpośredniej takie pytania są zwykle wcześniej wypracowywane przez grupy afinicji. Jednak w~każdym przypadku, jaki poziom ryzyka jesteśmy gotowi ponieść, zawsze powinien być indywidualną decyzją.}. Kluczową kwestią jest jednak to, że nadal zachowujemy się tak, jakby państwo nie istniało, przynajmniej jako byt moralny\footnote{Można powiedzieć, używając terminologii Althussera, że  akcja bezpośrednia wiąże się z~systematyczną odmową interpelacji. Jeśli ktoś przyjmuje ten moralny pogląd systematycznie, trudno postrzegać policję jako coś innego niż uzbrojony i~niezwykle niebezpieczny gang uliczny; tak właśnie często nazywają ich anarchiści.}. W każdym razie możliwe byłoby przeprowadzenie tajnej akcji bezpośredniej. Z definicji niemożliwe jest przeprowadzenie tajnego aktu nieposłuszeństwa obywatelskiego.

 Rozwijam tutaj coś, co można by nazwać klasyczną definicją akcji bezpośredniej -- rozwiniętą i~opisaną przez co najmniej półtora wieku refleksji anarchistycznej. Często w~dzisiejszych czasach termin ten jest używany w~znacznie luźniejszym znaczeniu. ,,Akcja bezpośrednia'' staje się jakąkolwiek formą politycznego oporu, która jest jawna, bojowa i~konfrontacyjna, ale nie jest to jawne powstanie militarne (por. Carter 1973). W tym sensie, jeśli ktoś robi coś więcej niż maszerowanie ze znakami, ale nie jest jeszcze gotowy, aby wyruszyć w~góry z~AK-47, to jest działaczem bezpośrednim. Boston Tea Party, podczas którego grupa kolonialnych rewolucjonistów przebranych za Indian wyrzuciła do bostońskiego portu mnóstwo mocno opodatkowanej brytyjskiej herbaty, jest często przywoływany jako klasyczny przykład tego rodzaju akcji bezpośredniej \footnote{Rzeczywiście, jedną z~popularnych gier współczesnych aktywistów jest wyobrażenie sobie, jak wydarzenie takie jak Boston Tea Party byłoby relacjonowane przez amerykańskie media, gdyby miało miejsce dzisiaj.}. Takie działania bywają jednocześnie bojowe i~symboliczne. Używany w~ten sposób termin ,,akcja bezpośrednia'' może obejmować ogromny zakres: może oznaczać wszystko, od upierania się przy swoim prawie do siedzenia przy ladzie w~barze dla białych po podpalenie jednego z~takich barów, od stawiania się buldożerom w~starym lesie do nabijania gwoździami drzew, aby drwale, którzy lekceważą ostrzeżenia, aby nie ścinać w~niektórych obszarach, ryzykowali życiem.

 Aktywiści również często mówią tak, jakby różnica między akcją bezpośrednią a obywatelskim nieposłuszeństwem polegała po prostu na bojowości. Dla niektórych oznacza gotowość do zaakceptowania aresztowania. Osoby wykonujące ,,ON'' mogą dobrowolnie oddać się policji; nawet jeśli tego nie zrobią, kiedy blokują wejście do siedziby firmy lub kładą się przed prezydenckim pasem samochodowym, działają w~pełnym oczekiwaniu, że trafią do więzienia, a gdy policja zamierza ich aresztować, nie będą uciekać i~opierać się tylko biernie lub wcale. W przeciwieństwie do tego, akcjoniści bezpośredni, niezależnie od tego, czy wybijają nocą okna, czy lutują drzwi w~fabrykach okupowanych przez robotników, robią wszystko, co w~ich mocy, aby uszło im to na sucho. Albo, alternatywnie, rozróżnienie może dotyczyć tego, jak bardzo czyjaś taktyka zbliża się do konwencjonalnych definicji ,,przemocy''. Kiedy angielskie sufrażystki odmawiały płacenia podatków, zwykle opisywane były jako praktykujące nieposłuszeństwo obywatelskie; kiedy zaczęły systematycznie wybijać witryny sklepowe, zwykle mówiono, że przeszły do akcji bezpośredniej. Oczywiście, według klasycznych anarchistycznych definicji, wybijanie okien w~celu wywarcia nacisku na rząd, aby wprowadził reformę głosowania, nie jest w~żadnym sensie akcją bezpośrednią -- jest całkowicie pośrednią -- ale użycie tego słowa pokazuje, jak bardzo termin ten stał się synonimem pewnego stopnia bojowości.

 Wszystko to sprawia, że  łatwo zrozumieć, dlaczego kwestia ,,działań bezpośrednich'' tak często znajduje się w~centrum debaty politycznej. Na przykład w~pierwszej połowie XX wieku toczyły się niekończące się spory o rolę akcji bezpośredniej w~ruchu robotniczym. Dziś łatwo zapomnieć, że kiedy pojawiły się związki zawodowe, postrzegano je jako skrajnie radykalne organizacje. W rzeczywistości reprezentowały rodzaj roszczenia do rewolucyjnej dwuwładzy. Strajk, niszczenie maszyn, okupacja fabryk, tworzenie pikiet, aby fizycznie uniemożliwić łamistrajkom wejście na miejsce pracy: wszystko to było kwestią tego, by robotnicy przejęli dla siebie prawo do stosowania przymusu, wbrew roszczeniom państwa posiadania monopolu na przemoc. Pierre-Joseph Proudhon, jeden z~pierwszych XIX-wiecznych filozofów anarchistycznych, i~blisko związany z~ówczesnym francuskim ruchem robotniczym, faktycznie sprzeciwiał się strajkom, ponieważ uważał, że ruch ten powinien ograniczać się tylko do wyraźnie pokojowych form akcji bezpośredniej. Bardzo szybko jednak państwa, które nie mogły całkowicie stłumić związków, przystąpiły do  ich kooptacji. Niektóre formy akcji protestacyjnych (takie jak pikiety) zostały zalegalizowane, ale ściśle uregulowane; inne (np. sabotaż w~miejscu pracy) są surowo zabronione. Jak można sobie wyobrazić, wszystko to wywołało ożywioną debatę w~ruchu syndykalistycznym. Georges Sorel uchwycił nieco posmak tych debat w~swoim eseju ,,Reflections on Violence'', opublikowanym we Francji w~1908 roku. Twierdzi w~nim, że nawet wtedy, gdy strajk lub akcja robotnicza rzeczywiście podważają monopol państwa na przemoc, nawet jeśli ma do czynienia z~nielegalnym, dzikim strajkiem, strajki nie są naprawdę rewolucyjne, ponieważ zwykle celem strajku jest uzyskanie ustępstw w~zakresie płac, godzin pracy lub warunków, które państwo zagwarantuje i~ostatecznie wyegzekwuje. Dlatego nie kwestionuje się przemocy państwa, ale próbuje się ją pozyskać dla własnej strony. Sorel argumentował, że z~anarchistycznego punktu widzenia jedynym prawdziwie rewolucyjnym strajkiem byłby strajk generalny, którego celem było obalenie systemu przemocy państwowej jako całości. Działania robotnicze były zatem uzasadnione tylko o tyle, o ile były próbami podążania w~tym kierunku, być może próbami generalnymi lub formami agitpropu.

 Również w~Stanach Zjednoczonych różnice filozoficzne często kończyły się walką w~dużej mierze poprzez spory o taktykę. Na początku XX wieku nastąpił głęboki rozłam między głównymi związkami, takimi jak Knights of Labor, które ostatecznie stały się kręgosłupem AFL-CIO, a związkami rewolucyjnymi, takimi jak IWW (Industrial Workers of the World -- Przemysłowi Robotnicy Świata, lub Wobblies). Ostatecznym celem tych ostatnich było ,,zniesienie systemu płac'' i~odmówili zmiany przez państwo, które postrzegali jako nielegalną instytucję. Byli w~istocie, jeśli nie oficjalnie, anarchosyndykalistami. Tam, gdzie związki głównego nurtu kładły nacisk na wyższe płace i~bezpieczeństwo pracy, Wobblies byli -- podobnie jak europejskie związki anarchistyczne -- bardziej zainteresowani redukcją godzin. Jednak najważniejszą rzeczą, którą otwarcie proponowali, było poparcie Wobbly dla ,,akcji bezpośredniej'', która w~tym kontekście oznaczała sabotaż miejsca pracy.

 Należy tutaj podkreślić, że \textit{praktyka }sabotażu w~miejscu pracy nigdy nie była uważana za szczególnie skandaliczną -- przynajmniej wśród pracowników. Niszczenie własności korporacyjnej, okupacja miejsc pracy, celowa tandeta, spowolnienia -- wszystko to od dawna stanowi część repertuaru, można powiedzieć, standardowego zestawu narzędzi zorganizowanej pracy przez stulecia. Pozostają takimi do dziś. Sam dorastałem w~budynku na Manhattanie z~wadliwą hydrauliką z~powodu sabotażu w~miejscu pracy, który miał swoje korzenie w~sporze pracowniczym z~późnych lat pięćdziesiątych. Amerykańscy strajkujący nadal regularnie przebijają opony, a nawet podpalają sprzęt firmowy. Jednak nic z~tego nie jest oficjalną polityką związkową. Urzędnicy związkowi niezmiennie potępiają takie działania lub zaprzeczają ich występowaniu. Częściowo jest to spowodowane tym, że wolno \textit{im }strajkować. Związki są paradoksalnie jedyne organizacje w~USA legalnie upoważnione do angażowania się w~bezpośrednie działania; ale mogą to zrobić tylko wtedy, gdy tego tak nie nazywają; i~tylko za cenę zaakceptowania niekończących się i~zawiłych przepisów dotyczących tego, jak i~kiedy mogą strajkować, jakie rodzaje pikiet mogą organizować i~gdzie, czy wolno im angażować się w~inne taktyki, takie jak wtórne bojkoty, a nawet kampanie reklamowe itd. Wszystko, co wykracza poza te ograniczenia, bywa definiowane jako ,,akcja bezpośrednia'' i~oficjalnie jest zabronione. To jest powód, jak zobaczymy, że przywódcy związkowi niezmiennie robią wszystko, co w~ich mocy, by szeregowi robotnicy nie uczestniczyli w~bezpośrednich akcjach, takich jak te w~Seattle i~Quebec City. Gdyby członkowie związku -- jako członkowie związku -- pomogli na przykład zburzyć mur w~Quebecu, nie tylko braliby udział w~nielegalnych działaniach, ale też narażaliby podstawy specjalnej relacji ich przywódców z~Państwem.

 Ci, którzy kontynuują pracę w~ramach tradycji syndykalistycznej, sprzeciwią się, nie zaskakując, takiemu rodzajowi utożsamiania akcji bezpośredniej ze zwykłą wojowniczością. Mają tendencję do preferowania definicji takich jak te, od których zacząłem rozdział. Niektórzy posunęli się tak daleko, że twierdzą, że akcje na dużą skalę, takie jak Seattle czy Quebec, wcale nie były akcjami bezpośrednimi, właśnie z~tego powodu. Na przykład krótko po zamknięciu Światowej Organizacji Handlu w~Seattle w~listopadzie 1999 roku norweski anarchosyndykalista Harald Beyer-Arensen napisał artykuł, który miał na celu pokazanie, że Seattle nie było tak naprawdę akcją bezpośrednią, ponieważ nie dotyczyło ludzi działających bezpośrednio, aby zmienić swoją bezpośrednią sytuację.

 Prowadząc kampanię na rzecz robotników najemnych, aby dołączyli do Przemysłowych Robotników Świata, Eugene V. Debs stwierdził w~grudniu 1905 roku: ,,Kapitaliści są właścicielami narzędzi, których nie używają, a robotnicy używają narzędzi, których nie posiadają''. Do tego można by dodać: Czasami akcja bezpośrednia może oznaczać wyłączenie narzędzi, których nie posiadamy, czasami może to oznaczać wykorzystanie ich do własnych, zdefiniowanych przez nas potrzeb i~celów. W ostatecznym przypadku może to oznaczać jedynie działanie tak, jakby wszystkie narzędzia były w~rzeczywistości naszymi własnymi (Beyer-Arensen 2000:11).

 Po raz kolejny akcja bezpośrednia oznacza naleganie na działanie tak, jakby ktoś był już wolny. Oto dlaczego, argumentuje dalej, leży ona w~sercu ,,anarchistycznego, socjalno-rewolucyjnego projektu'' jest środkiem, za pomocą którego klasy robotnicze mogą się wyemancypować dzięki własnym wysiłkom, a nie pod kierownictwem jakiegokolwiek rodzaju rewolucyjnej awangardy lub elity.

 Z tej perspektywy możemy zdefiniować akcję bezpośrednią jako akcję prowadzoną w~imieniu nikogo innego, jak tylko nas samych, gdzie środki są natychmiast również celami, a jeśli nie, to, jak w~strajku płacowym, niezapośredniczone przez żadną związkową biurokrację, gdzie środki (zmniejszenie zysków szefów przez nasz brak pracy, a tym samym zmniejszenie władzy szefów) pozostają w~bezpośrednim związku z~określonymi przez nas celami (podwyższenie płac i~rozszerzenie naszej władzy). Skutecznie przeprowadzona akcja bezpośrednia prowadzi do bezpośredniego przeorganizowania istniejących warunków życia dzięki połączonym wysiłkom osób bezpośrednio dotkniętych (tamże).

 Co wydarzyło się w~Seattle? Grupa aktywistów próbowała -- i~przez pewien czas się to udawało -- zamknąć spotkanie biurokratów handlowych, aby zakłócić negocjacje w~sprawie nowej rundy WTO i~upublicznić problem z~samym istnieniem Światowej Organizacji Handlu. To, na co zgadza się Beyer-Arensen, pod pewnymi względami przypomina działanie bezpośrednie. Z pewnością ci, którzy stworzyli ,,Sieć Akcji Bezpośrednich'' (DAN -- Direct Action Network) w~celu koordynowania postępowań, wierzyli, że to właśnie robią. Jeśli ktoś po prostu zastosuje kryterium bojowości, może ulec pokusie, aby się zgodzić, ponieważ wydarzenie to obejmowało przedłużoną (jeśli bez przemocy) konfrontację z~policją. Ale w~rzeczywistości, jak twierdzi Beyer-Arensen, nie była to tak naprawdę akcja bezpośrednia, ponieważ tak naprawdę nie była ,,bezpośrednia''. Podaje przykład. Wyobraź sobie miasto, które cierpi z~powodu braku wody. Co więcej, jakiś magnat nieruchomości jest właścicielem wszystkich okolicznych gruntów i~ma burmistrza w~kieszeni, więc mieszkańcy nie mogą po prostu budować nowych studni. Jeśli ktoś miałby zebrać grupę mieszczan do wykopania nowej studni i~tak, wbrew prawu, to byłaby to akcja bezpośrednia. Ale gdyby ktoś kazał im zablokować dom burmistrza, dopóki ten nie zmieni swojej polityki, to z~pewnością by tak nie było. Może to być o wiele bardziej bojowe niż pisanie petycji, listów czy lobbing, ale jest to po prostu inna wersja tego samego: apelu do władz, aby zmieniły swoje zachowanie. Wciąż uznaje autorytet, który prawdziwy akcjonista bezpośredni odrzuciłby. Beyer-Arensen konkluduje, że próba zamknięcia spotkań WTO w~Seattle nie była przykładem akcji bezpośredniej, ponieważ ostatecznie była to po prostu próba stworzenia widowiska medialnego, które następnie ,,wpłynęłoby na władze za pomocą jakiejś wyimaginowanej \quotedblbase  opinii publicznej\textquotedblright (200:12). Same spotkania WTO miały przecież w~zasadzie charakter ceremonialny. Większość prawdziwych decyzji podejmowana była gdzie indziej. Dlatego prawdziwym celem protestów było zapewnienie swoistej kontrceremonii, mającej na celu przyciągnięcie uwagi opinii publicznej, ponieważ jej rzekome cele (zamknięcie WTO jako instytucji) nie mogły być zrealizowane za pomocą zastosowanych środków. Był to w~istocie akt propagandy, teatru partyzanckiego, który miał wpływać na politykę rządu.

 Beyer-Arensen kończy utwór przyznaniem, że każda akcja bezpośrednia jest do pewnego stopnia aktem ,,propagandy czynem'', ponieważ ma uczyć poprzez przykład. Społeczność, która przeciwstawia się prawu, budując własną studnię, nie działa po prostu dla siebie; dają również przykład samoorganizacji innym społecznościom. Ale jest to efekt uboczny bezpośredniej akcji, a poza tym nie próbują wpłynąć na rząd.

 Nie przedstawiam tego argumentu tak szeroko, ponieważ jest szczególnie przekonujący. Reprezentuje on opinie jednego, starszego, dość zrzędliwego anarchosyndykalisty i~sądzę, że przytłaczająca większość współczesnych anarchistów z~pewnością nie zgodziłaby się z~jego wnioskami. W końcu, jak zauważył Sorel, tę samą logikę można zastosować do samych akcji robotniczych, które Beyer-Arensen aprobuje: ponieważ ostatecznie strajkujący domagają się wiążącego arbitrażu przez rządowych mediatorów, a nawet jeśli nie, wszelkie porozumienia, które zawierają z~pracodawcami, będą w~końcu wyegzekwowane przez państwo. Jeśli poprowadzi się argumentację Beyer-Arensen do logicznego wniosku, żadne działanie, które odbywa się w~ramach legalności, lub w~którym opinia publiczna jest czynnikiem, nie może zostać uznane za bezpośrednie. W sumie, jeśli ktoś postawi swoje ciało na drodze buldożera, który ma zniszczyć jego dom lub ogród, to chociaż można by pomyśleć, że to, co robi, po prostu odwołuje się do moralnego sumienia kierowcy, nie można realistycznie zaprzeczyć, że Kierowca prawdopodobnie będzie również myślał o możliwości bycia oskarżonym o niedbalstwo w~sprawie zabójstwa lub opisanym w~gazetach następnego dnia jako bezduszny zabójca. Sam Beyer-Arensen nie jest całkowicie nieświadomy tego dylematu -- przynajmniej w~przypadku strajków. Kończy swój esej sugestią, że niektóre strajki są w~rzeczywistości lepszymi przykładami akcji bezpośredniej niż inne. Jego ulubionym przykładem jest strajk robotników transportowych w~Melbourne w~latach 80., w~którym zamiast odejść z~pracy, kierowcy autobusów i~konduktorzy pociągów pozostali na miejscu, ale przestał pobierać opłaty -- skutecznie uwalniając masowy transport do czasu zakończenia akcji. Wyobraź sobie, on sugeruje, co by się stało, gdyby przez jeden dzień pracownicy każdej gałęzi przemysłu i~handlu usługowego robili to samo. Już samo to może być ważnym krokiem w~pokazaniu, w~jaki sposób gospodarka kapitalistyczna może zostać przekształcona w~gospodarkę wolności.

 To mocny obraz, ale niezwykle podobny do działań, które Beyer-Arensen bez wątpienia potępiłby jako czysty teatr. Weźmy na przykład akcję reklamową zorganizowaną przez członków społeczności skłotu Christiany, zlokalizowaną na terenie dawnej bazy wojskowej pod Kopenhagą.

 W 1974 roku społeczność ta zaangażowała się w~różne formy teatru ulicznego, aby zyskać bardziej przychylny wizerunek publiczny. Zorganizowano pierwsze Boże Narodzenie dla biednych i~samotnych, a Solvognen zorganizowała armię Świętych Mikołajów, którzy hojnie rozdawali prezenty z~miejskich domów towarowych zarówno młodym, jak i~starszym. Oczywiście zostali aresztowani, ale w~konsekwencji zdjęcia policji bijącej Świętego Mikołaja trafiły na pierwsze strony gazet na całym świecie\footnote{\url{http://www.ri.xu.org/arbalest/index.html}\textit{, dostęp 17.08.2000.}}.

 Innymi słowy, zdobyli prawie dokładnie to samo, co napastnicy z~Melbourne, ale praktycznie bez żadnej bezpośredniej akcji. Powstaje więc pytanie: gdzie narysować linię? Jak bezpośrednia akcja musi być? Jeśli dostarczanie darmowych towarów i~usług czterem lub pięciorgu przypadkowym dzieciom na ulicy nie wystarczy, aby stało się to rzeczywistością, dlaczego dziesięć tysięcy osób dojeżdżających do pracy w~ciągu jednego dnia miałoby być czymś innym?

 Powodem, dla którego przytoczyłem ten argument, jest to, że zapewnia on widok na pewien wszechświat moralny. Większość amerykańskich anarchistów, których znam, uważa rozważania na temat tego, czy Seattle rzeczywiście było akcją bezpośrednią za nieco niemądre -- w~najlepszym razie mogą stanowić nieco odwracający temat do dyskusji przy piwie, ale zbyt poważne traktowanie takich pytań wydaje się akademickie, a nawet sekciarskie. Mimo to omawiane kwestie są krytyczne. Jak zobaczymy, większość zarzutów podniesionych do idei akcji granicznych na kilka tygodni przed Québec City opierała się na przekonaniu, że takie akcje byłyby jedynie symboliczne, a nie autentyczną akcją bezpośrednią. Co więcej, istota krytyki Beyer-Arensen -- że działania takie jak Seattle są w~dużej mierze symboliczne, i~że chodzi o pracę w~prawdziwych społecznościach w~sposób, który pozwala ludziom przejąć władzę nad własnym życiem -- jest to coś, z~czym zgodziłby się każdy zaangażowany w~ruch. Jeszcze zanim Naomi Klein (2000) napisała swój słynny artykuł w~,,Nation'', ostrzegający aktywistów przed niebezpieczeństwem ,,skakania na szczyt'', ,,podążania za biurokratami handlowymi, jakby byli Grateful Dead'', wszystko to było już głównym tematem debaty. Ci, którzy bronili akcji takich jak Seattle, nie tylko upierali się, że była to bezpośrednia interwencja, ponieważ ludzie stawiali swoje ciała na linii, aby zablokować delegatom wejście do budynku, ale zrobili to w~sposób, który Beyer-Arensen podkreśla jako kluczowy : poprzez mobilizację społeczności ludzi w~formie samoorganizacji, która stanowi żywą alternatywę dla istniejącej struktury władzy.

 To rzeczywiście miało charakter edukacyjny. Z jednej strony postanowili zdemaskować niedemokratyczny charakter WTO i~podobnych instytucji, które, ich zdaniem, razem tworzyły kręgosłup nierozliczanego światowego neoliberalnego rządu, który w~imię władzy korporacji dążył do zdławienia istniejących praw demokratycznych. Z drugiej strony byli zdecydowani zorganizować całą akcję zgodnie z~bezpośrednio demokratycznymi zasadami, a tym samym dać żywy przykład tego, jak może działać autentycznie egalitarne podejmowanie decyzji. W kontaktach z~globalnymi instytucjami jest to tak bezpośrednie, jak to tylko możliwe. 

 DAN -- Sieć Działań Bezpośrednich, która jest głównym tematem tej książki, wyłoniła się bezpośrednio z~tego projektu. Miało to po części służyć do organizowania akcji przeciwko instytucjom neoliberalnym; częściowo jako model opartej na konsensusie, zdecentralizowanej demokracji bezpośredniej. Mimo wszystkich swoich wad (o których będziemy się dużo jeszcze dowiadywać) odegrała w~tym ważną rolę.

 Reasumując zatem: akcja bezpośrednia reprezentuje pewien ideał -- w~jego najczystszej postaci, prawdopodobnie nieosiągalny. Jest to forma działania, w~której środki i~cele stają się skutecznie nie do odróżnienia; sposób aktywnego zaangażowania się w~świat w~celu wywołania zmiany, w~którym forma działania -- lub przynajmniej organizacja działania -- sama w~sobie jest modelem zmiany, którą chce się wprowadzić. W swojej najbardziej podstawowej formie odzwierciedla bardzo prosty pogląd anarchistów: nie można stworzyć wolnego społeczeństwa przez dyscyplinę wojskową, społeczeństwa demokratycznego przez wydawanie rozkazów lub szczęśliwego przez pozbawione radości samopoświęcenie. W swojej najbardziej rozbudowanej strukturze własnego aktu staje się rodzajem mikroutopii, konkretnym modelem własnej wizji wolnego społeczeństwa. Jak zauważyła Emma Goldman (i inni), fakt, że władze określają takie czyny jako przestępstwa, nie stanowi w~tym zakresie problemu -- o ile służy to ciągłemu przypominaniu uczestnikom o wzięciu odpowiedzialności za swoje czyny, zachowywaniu się z~odwagą i~uczciwością, a to może być dużym atutem. Problemy pojawiają się raczej, gdy ktoś wychodzi poza konfrontację do innych form zaangażowania w~świat zorganizowany według innych linii.

 Rewolucyjna strategia oparta na akcji bezpośredniej może odnieść sukces tylko wtedy, gdy zasady akcji bezpośredniej zostaną zinstytucjonalizowane. Tymczasowe bańki autonomii muszą stopniowo przekształcać się w~trwałe, wolne społeczności. Jednak w~tym celu społeczności te nie mogą istnieć w~całkowitej izolacji; nie mogą też mieć czysto konfrontacyjnej relacji ze wszystkimi wokół nich. Muszą mieć jakiś sposób na nawiązanie kontaktu z~większymi systemami gospodarczymi, społecznymi lub politycznymi, które ich otaczają. Jest to najtrudniejsze zagadnienie, ponieważ osobom zorganizowanym według radykalnie demokratycznych linii okazało się niezwykle trudne zintegrowanie się w~jakikolwiek znaczący sposób w~większe struktury bez konieczności dokonywania niekończących się kompromisów w~ich fundamentalnych zasadach. Dla grup opartych na działaniu bezpośrednim nawet praca w~sojuszu z~radykalnymi organizacjami pozarządowymi lub związkami zawodowymi często stwarzała problemy, które wydawały się nie do pokonania. Na bardziej bezpośrednim poziomie strategia zależy od rozpowszechniania modelu: na przykład większość anarchistów nie postrzega siebie jako awangardy, której historyczną rolą jest ,,organizowanie'' innych społeczności, ale raczej jako jedna społeczność dająca przykład do naśladowania innym. Podejście -- często określane jako ,,zanieczyszczanie'' -- opiera się na założeniu, że doświadczenie wolności jest zaraźliwe, że każdy, kto bierze udział w~akcji bezpośredniej, prawdopodobnie zostanie trwale przemieniony przez to doświadczenie i~będzie chciał więcej. Dość często tak się dzieje, ale nasuwa się pytanie, jak w~pierwszej kolejności uświadomić innym ten pomysł. To, co uczestnicy postrzegają jako głębokie i~transformujące, często wygląda z~zewnątrz, w~najlepszym razie osobliwie -- w~najgorszym kultowo lub szalenie. To z~kolei podnosi kwestię mediów. Ale odnosząc się do takich strategicznych kwestii, naprawdę przechodzę od mówienia o akcji bezpośredniej do bardziej ogólnej kwestii anarchizmu.

\section{Czym jest anarchizm?}

Jednym z~powodów, dla których zacząłem ten rozdział, było to, że chciałem również przekazać coś z~anarchistycznej debaty, która zawsze różniła się od bardziej znanego, marksistowskiego stylu, poprzez skupianie się bardziej na tego rodzaju konkretnych kwestiach praktyki. Wielu narzekało, że anarchizmowi brakuje teorii. Nawet ci, których uważa się za jej założycieli -- Godwin, Proudhon, Bakunin, Kropotkin -- często wydają się bardziej pisarzami i~moralistami niż prawdziwymi filozofami, a najsławniejsi anarchiści bliższych nam czasów znacznie częściej tworzyli chwytające slogany, dzikie poematy lub powieści science fiction niż wyrafinowane ekonomiczne polityczne lub dialektyczne analizy \footnote{W rzeczywistości anarchiści od dawna przejmowali znaczną część swojej ekonomii politycznej od marksistów - tradycji, która sięga Bakunina, który, choć był politycznym rywalem Marksa, był również odpowiedzialny za pierwsze tłumaczenie Kapitału na rosyjski - zamiast czuć się zobowiązanym do założyli własną anarchistyczną szkołę ekonomii politycznej. Choć trzeba przyznać, że pierwsi anarchiści mieli tendencję do wskazywania, że  prawie wszystkie koncepcje przypisywane Marksowi (lub Proudhonowi) były tak naprawdę rozwinięte w~ruchu robotniczym tamtych czasów i~jedynie usystematyzowane i~opracowane przez teoretyków.}. Istnieją tysiące marksistowskich akademików, ale bardzo niewielu anarchistów. Nie dzieje się tak dlatego, że anarchizm jest tak bardzo antyintelektualny, ile dlatego, że nie postrzega siebie jako, fundamentalnie, projektu analizy. To bardziej projekt moralny.

 Jak pisałem w~innym miejscu (Graeber 2002, 2004), marksizm miał tendencję do bycia teoretycznym lub analitycznym dyskursem o rewolucyjnej strategii; anarchizm jako etyczny dyskurs o praktykach rewolucyjnych. Podstawowe zasady anarchizmu -- samoorganizacja, dobrowolne stowarzyszenia, pomoc wzajemna, sprzeciw wobec wszelkich form przymusu władzy -- są zasadniczo moralne i~organizacyjne.

 Trzeba przyznać, że stoi to w~sprzeczności z~popularnym wizerunkiem anarchistów jako szaleńców rzucających bomby, przeciwnych wszelkim formom organizacji -- ale jeśli ktoś przyjrzy się, jak powstała ta reputacja, wydaje się to potwierdzać mój punkt widzenia. Okres mniej więcej 1875-1925 oznaczał szczyt pewnej fazy anarchistycznego organizowania się: istniały setki anarchistycznych związków, konfederacji, lig rewolucyjnych i~tak dalej. Na początku nastąpił zryw wezwań do zabójstwa głów państw (Anderson 2006), był on dość krótki, a anarchistyczni delegaci i~zorganizowane grupy szybko wycofały poparcie dla tej strategii jako bezproduktywnej. Niemniej jednak, w~następnych dziesięcioleciach nastąpił nieustanny strumień dramatycznych zabójstw dokonywanych przez ludzi nazywających siebie anarchistami. Nie znam żadnego prawdziwego zabójcy w~tym szczególnym okresie, który faktycznie był wytworem tych organizacji anarchistycznych, a tym bardziej ich akcje były przez nie planowane lub sponsorowane; raczej prawie zawsze okazywali się odizolowanymi jednostkami, które nie miały więcej trwałych powiązań z~anarchistycznym życiem niż Unabomber, i~zwykle mniej więcej w~podobnym stopniu trzymały się zdrowego rozsądku. Wyglądało to raczej tak, jakby istnienie anarchizmu dało samotnym bandytom coś, od czego mogliby się wziąć nazwę\footnote{Malatesta przedstawił dokładnie ten argument w~tamtym czasie (1913).}. Jednak taka sytuacja stworzyła niekończące się dylematy moralne dla anarchistycznych pisarzy i~wykładowców, takich jak Peter Kropotkin czy Emma Goldman. Jakim prawem anarchista mógłby potępić osobę, która zabija tyrana, bez względu na to, jak katastrofalne są skutki dla większego ruchu? Cała sprawa była przedmiotem niekończącej się, intensywnej debaty moralnej: nie tylko o to, czy takie czyny były (lub kiedykolwiek mogą być) uzasadnione, ale czy było uzasadnione dla anarchistów, którzy nie czuli, że takie czyny są mądre, a nawet uzasadnione, aby je publicznie potępiać. Zawsze tego rodzaju praktyczne, moralne pytania budziły anarchistyczne namiętności: Czym jest akcja bezpośrednia? Jakie taktyki są poza nawiasem i~jaką solidarność zawdzięczamy tym, którzy je stosują? Lub: jaki jest najbardziej demokratyczny sposób prowadzenia spotkania? W którym momencie organizacja przestaje się wzmacniać, staje się duszna i~biurokratyczna? W przypadku analiz natury formy towarowej lub mechaniki alienacji większość zadowoliła się czerpaniem z~prac pisemnych intelektualistów marksistowskich (którzy zwykle sami wywodzą się z~idei, które pierwotnie przenikały przez szerszy ruch robotniczy, w~którym anarchiści byli bardzo zaangażowany). Co oznacza również, że pomimo wszystkich gorzkich i~często gwałtownych sporów, jakie anarchiści mieli z~marksistami na temat tego, jak przeprowadzić rewolucję, zawsze istniał tu pewien rodzaj komplementarności, przynajmniej \textit{in potentia}\footnote{Gdyby istniejący marksiści całkowicie porzucili praktyczną politykę i~wycofali się do akademii, tworząc niekończące się tomy marksistowskich analiz na każdy temat pod słońcem i~przytłaczając wszystkie inne tendencje intelektualne, to większość anarchistów uznałaby to za całkowicie pozytywny rozwój.}. 

 Dlatego uważam, że zwodnicze jest pisanie historii anarchizmu w~taki sam sposób, w~jaki pisze się historię tradycji intelektualnej, takiej jak marksizm. Nie jest tak, że nie można opowiedzieć historii w~ten sposób, jeśli się chce. Większość książek o anarchizmie tak jest napisanych. Zaczynają od pewnych założycielskich postaci intelektualnych (Godwin, Stirner, Proudhon, Bakunin), wyjaśniają radykalne idee, które rozwinęli, opowiadają historię większych ruchów, które ostatecznie zostały zainspirowane tymi ideami, a następnie dokumentują walki polityczne, wojny, rewolucje i~projekty reform społecznych, które nastąpiły. Ale jeśli spojrzy się na to, co faktycznie powiedziały te rzekome postacie założycieli, okaże się, że większość z~nich tak naprawdę nie postrzegała siebie jako tworzących jakąś wielką nową teorię. Bardziej prawdopodobne było, że postrzegali siebie jako nadających imię i~głos pewnemu rodzajowi powstańczego zdrowego rozsądku, uznali, że jest tak stary jak historia. Podczas gdy anarchizm jako ruch był bardzo silnie zakorzeniony w~masowym organizowaniu się proletariatu przemysłowego, anarchiści (w tym ci, którzy sami byli robotnikami przemysłowymi) również mieli tendencję do czerpania inspiracji z~istniejących sposobów praktyki, zwłaszcza ze strony chłopów, wykwalifikowanych rzemieślników, a nawet, do pewnego stopnia, banitów, włóczęgów, włóczęgów i~innych, którzy żyli swoim rozumem -- innymi słowy tych, którzy w~pewnym stopniu kontrolowali swoje życie i~warunki pracy, których można by uznać za przynajmniej do pewnego stopnia elementy autonomiczne. Można powiedzieć, posługując się terminologią marksistowską, że byli to ludzie z~pewnym doświadczeniem w~produkcji niewyalienowanej. Tacy ludzie mieli doświadczenie życia poza biurokracją państwową lub kapitalistyczną, płacami i~pracą najemną; zdawali sobie sprawę, że takie relacje nie są nieuniknione; dość często postrzegali je jako wewnętrznie niemoralne. Często sami byli bardziej pociągani do anarchizmu jako wyraźnej filozofii politycznej, a przynajmniej w~niektórych czasach i~miejscach (hiszpańscy chłopi, szwajcarscy zegarmistrzowie) tworzyli jego masową bazę, co więcej, te elementy przemysłowego proletariatu, które miały tendencję do znajdowania największego pokrewieństwa z~anarchizmem byli tymi, którzy byli najmniej odsunięci od innych trybów życia. Sam Marks miał tendencję do lekceważenia bazy anarchistycznej jako szczególnie niepomyślnej kombinacji ,,drobnomieszczaństwa'' i~,,lumpenproletariatu'' i~uważał za niedorzeczny pogląd, że mogą oni w~jakikolwiek sposób stać poza kapitalizmem. Kapitalizm był dla Marksa systemem totalizującym. W najbardziej intymny sposób ukształtowała świadomość wszystkich, którzy pod nim żyli. Ten rodzaj krytyki kapitalizmu, jaki widzieliśmy u autorów takich jak Proudhon czy Bakunin, argumentował Marks, był po prostu głosem drobnomieszczańskiej moralności, drobnych kupców i~producentów złorzeczących na konkurentów. Nie mogli niczego nauczyć rewolucjonistów. Tylko proletariat przemysłowy, który nie miał żadnego interesu w~istniejącym systemie, mógł być prawdziwie rewolucyjną klasą.

 Niektórzy bez wątpienia sprzeciwiliby się temu, że ten pogląd na myśl Marksa jest nieco surowy i~niedopracowany i~prawdopodobnie mieliby rację. Ale reprezentuje pogląd, który wkrótce stał się kanoniczny wśród tych, którzy twierdzili, że przemawiają w~imieniu marksizmu. Moim celem nie jest tutaj uzasadnienie meritum sprawy, ale podkreślenie stopnia, w~jakim postrzegamy cały projekt anarchistyczny, zasadniczo oczami jego rywali. Co więcej, anarchizm ma tendencję do opisywania innego stosunku teorii i~praktyki niż to, co zaczęto nazywać ,,marksizmem'' Ta ostatnia jest -- mimo wszelkich aspiracji materialistycznych -- głęboko idealistyczne. Historia marksizmu jest nam przedstawiana jako historia wielkich myślicieli -- są leniniści, maoiści, trockiści, zwolennicy Gramsciego, Althussera -- nawet brutalni dyktatorzy tacy jak Stalin czy Enver Hodża musieli udawać wielkich filozofów, ponieważ zawsze chodziło o to, że zaczyna się od głębokiej wiedzy teoretycznej jednego człowieka i~wynikającej z~tego tendencji politycznej. W przeciwieństwie do tego, tendencje anarchistyczne nigdy nie wywodzą się z~poglądów pojedynczego teoretyka -- nie mamy proudhończyków i~kropotkinistów -- ale asocjacjonalistów, indywidualistów, syndykalistów i~platformistów. W prawie każdym przypadku podziały opierają się na różnicy filozofii organizacyjnej i~rewolucyjnej praktyki. 

 Jak zatem myślimy o ruchu politycznym, w~którym praktyka jest na pierwszym miejscu, a teoria jest zasadniczo drugorzędna? 

 Uderza mnie, że może być pomocne, zamiast zaczynać od słowa ,,anarchizm'', aby zacząć od słowa ,,anarchista''. Do jakich ludzi, idei lub instytucji może się odnosić to słowo? Ogólnie rzecz biorąc, można znaleźć trzy różne sposoby wykorzystania tego terminu. Po pierwsze, można odnieść się do ludzi, którzy popierają wyraźną doktrynę znaną jako ,,anarchizm'' (lub czasami ,,anarchia'' -- a może dokładniej, pewną wizję ludzkich możliwości. To mniej więcej konwencjonalna definicja. Anarchiści stają się nosicielami tradycji intelektualnej: takiej, której historia rzeczywiście sięga do postaci założycieli w~XIX wieku, która całkiem szybko upowszechniła się na przełomie stuleci, aż literatura anarchistyczna była zachłannie czytana w~miejscach jak Chiny i~India długo, zanim marksizm lub inne odmiany zachodniej myśli odcisnęły się, ale w~ciągu wczesnego dwudziestego wieku została nim zastąpiona\footnote{Nawet tutaj sprawy są nieco bardziej subtelne - wiele postaci założycieli anarchizmu było Rosjanami, którzy tak naprawdę nie identyfikowali się z~tym, co uważali za ,,Zachód'' - ale tak zwykle opowiada się tę historię.}. Wiele wybitnych osobistości tamtych czasów, od Picassa po Mao, zaczęło swoje życie polityczne jako anarchiści, a skończyło jako komuniści. Ale można też mówić szerzej. Z pewnością nie jest niczym niezwykłym słyszeć, jak historycy odnoszą się, powiedzmy, do chłopskich buntowników we wczesnych Chinach lub religijnych radykałów w~średniowiecznej Europie jako ,,anarchistów'', co oznacza, że  odrzucali autorytet rządów i~wierzyli, że ludziom byłoby lepiej w~świecie bez hierarchii. W tym sensie anarchiści istnieli od zawsze i~nie ma wielkiej tradycji intelektualnej, która nie widziałaby rozwoju idei anarchistycznych w~takiej czy innej formie. (Oczywiście dlatego idee dziewiętnastowiecznych anarchistów europejskich mogą mieć sens na początku dla ludzi w~innych częściach świata). Wreszcie, jest jeszcze trzeci sens. Kiedy antropolog taki jak Evans-Pritchard opisuje Nuerów jako żyjących w~,,uporządkowanej anarchii'', lub Joanna Overing używa tego słowa, by opisać amazońskich Piaroa, nie odnoszą się do ani doktryny, ani nawet do antyautorytarnej buntowniczości. Odnoszą się przede wszystkim do instytucji, zwyczajów i~praktyk. Znaczy to, że istnieją pewne społeczeństwa charakteryzujące się egalitarnymi formami organizacji -- czy to systemy wymiany, formy podejmowania decyzji, czy po prostu zwyczajowe sposoby życia codziennego -- i~to ma tendencję do krzewienia i~jest wspierane przez szeroko egalitarny etos. W tym sensie anarchizm jest sposobem życia, a przynajmniej zbiorem praktyk. 

 Innymi słowy, ,,anarchizm'' można postrzegać albo jako wizję, jako postawę, albo jako zestaw praktyk. Rozróżnienie między dwoma ostatnimi jest wprawdzie nieco rozmyte. Ci, którzy prowadzą swoje codzienne życie w~sposób egalitarny, robią to, ponieważ czują, że ludzie powinni to robić; ci, którzy uważają, że wszelkie formy hierarchii są niedopuszczalne, zwykle zrobią wszystko, co w~ich mocy, aby znaleźć sposób na życie bez nich. Jednak w~pierwszym przypadku etos egalitarny może pozostać w~dużej mierze nierozwinięty. Przynajmniej teoretycznie osoba żyjąca w~społeczeństwie anarchistycznym może być całkowicie nieświadoma, że  istnieje inny sposób życia; w~każdym razie taka osoba prawdopodobnie rozwinie wyraźne postawy antyautorytarne dopiero wtedy, gdy spotka kogoś o zupełnie innych założeniach, na przykład obcego zdobywcę. Podobnie, ci, którzy są oburzeni, że są popychani przez przełożonych społecznych, często będą rozważać własne sposoby radzenia sobie z~przyjaciółmi i~sąsiadami jako dowód, że hierarchia nie jest naturalną i~nieuniknioną cechą ludzkiego życia. Równie dobrze mogliby zacząć doceniać równość tych relacji, a nawet próbować radzić sobie z~takimi ludźmi w~sposób bardziej świadomie egalitarny niż dotychczas. Dziewiętnastowieczni hiszpańscy chłopi i~szwajcarscy zegarmistrzowie, którzy uważali idee Proudhona czy Bakunina za tak podatne -- i~których Marks potępił jako drobnomieszczańskich -- wyraźnie dokładnie tak robili dokładnie. 

 Chciałbym argumentować, że o ,,anarchizmie'' najlepiej myśleć nie jako o żadnej z~tych rzeczy -- nie jako wizji, ale nie jako postawie czy zbiorze praktyk. Raczej najlepiej jest myśleć jako o tym ruchu tam i~z powrotem między tymi trzema. W końcu doświadczenie obcego podboju lub podporządkowania niekoniecznie spowoduje, że niegdyś egalitarne społeczności odrzucą samą ideę hierarchii lub staną się bardziej egalitarne w~swoich wzajemnych stosunkach: skutek może być dokładnie odwrotny. Dzieje się tak, gdy te trzy wzmacniają się nawzajem -- kiedy wstręt do ucisku sprawia, że  ludzie próbują żyć w~bardziej świadomy sposób egalitarny, kiedy czerpią z~tych doświadczeń, by tworzyć wizje bardziej sprawiedliwego społeczeństwa, kiedy te wizje, w~końcu, sprawiają, by postrzegali istniejące układy społeczne jako jeszcze bardziej bezprawne i~wstrętne -- że można zacząć mówić o anarchizmie. Dlatego anarchizm nie jest w~żadnym sensie doktryną. To ruch, związek, proces oczyszczania, inspiracji i~eksperymentu. To jest jego istota. Wszystko, co naprawdę zmieniło się w~XIX wieku, to fakt, że niektórzy ludzie zaczęli nazywać ten proces.

 Patrzenie na to w~ten sposób znacznie ułatwia zrozumienie pewnych rzeczy, które w~innym przypadku byłyby niezwykle zagadkowe. Na przykład: dlaczego to, co uchodzi za teorię anarchistyczną, często ma tak mały związek z~tym, co mówi i~robi większość anarchistów? Gdyby ktoś próbował zrozumieć północnoamerykański anarchizm po prostu czytając teoretyczne lub ideologiczne wypowiedzi w~najbardziej znanych i~szeroko rozpowszechnionych, jawnie anarchistycznych czasopismach, można by odnieść wrażenie, że większość anarchistów była albo prymitywistami sprzeciwiającymi się wszelkim formom technologii, nawet rolnictwu lub skrajnymi przeciwnikami organizacji, podejrzliwych wobec jakiejkolwiek grupy liczącej więcej niż sześć lub siedem osób -- a większość pozostałych zadeklarowała wierność dokumentowi zatytułowanemu ,,Platforma Organizacyjna Ogólnego Związku Anarchistów'' napisanemu przez rosyjskich emigrantów w~Paryżu w~1924 roku. Można również dojść do wniosku, że powszechne wrażenie anarchistów jako o dzikich oczach, niepraktycznych nihilistów oddanych buntom dla samego buntu nie było prawdopodobnie tak dalekie od prawdy; a przynajmniej, że anarchiści wydawali się podzieleni na nihilistów i~zagorzałych sekciarzy, których główną formą politycznej praktyki jest wzajemne potępianie. Badanie anarchistycznych stron dyskusyjnych w~Internecie niewiele zrobi, by wyprowadzić z~tego wrażenia\footnote{Na przykład wiele anarchistycznych grup dyskusyjnych jest zdominowanych przez prawicowych entuzjastów wolnego rynku, którzy nazywają siebie ,,anarcho-kapitalistami'', którzy wydają się istnieć tylko w~Internecie; przynajmniej jestem zaangażowany w~politykę anarchistyczną od pięciu lub sześciu lat i~jeszcze żadnego nie spotkałem.}. Kiedy po raz pierwszy zaangażowałem się w~politykę anarchistyczną, ze zdziwieniem odkryłem, że przeważająca większość aktywistów, którzy uważali się za anarchistów, nie tylko nie identyfikowała się z~żadnym z~tych stanowisk, ale wielu nie było ich nawet świadomych. Inni, którzy czytali te czasopisma, czytali je głównie dla rozrywki. W innym miejscu nazwałem tych niesekciarzy anarchistami przez ,,małe a'', aby odróżnić ich od tych, którzy identyfikują się z~jakimkolwiek szczególnym szczepem: anarchiści zieloni, indywidualiści, anarchosyndykaliści, postlewicowcy, platformiści i~tak dalej. Chociaż statystyki są niedostępne, Chuck Munson, który od czasu do czasu przeprowadza ankiety wśród osób często odwiedzanych \url{infoshop.com} -- prawdopodobnie najpopularniejsza anarchistyczna strona internetowa w~Ameryce Północnej -- informuje mnie, że około 90\% amerykańskich anarchistów wydaje się pasować do kategorii ,,małych a'', ponieważ tylko około 10\% jest skłonnych utożsamiać się z~jakimkolwiek konkretnym podzbiorem.

 Co więcej, nawet wielu z~tych, którzy identyfikują się z~jednym konkretnym szczepem, działa w~sposób, który byłby niemożliwy do zrozumienia, gdybyśmy mieli do czynienia z~ideologią polityczną w~czymś podobnym do tradycyjnego znaczenia tego terminu. Pozwolę sobie wziąć jeden przykład -- prymitywizm -- być może najbardziej oczywisty \textit{outré}. W Ameryce idee prymitywistyczne po raz pierwszy zaczęły formować się w~kręgach otaczających czasopismo o nazwie \textit{Fifth Estate }w Detroit w~latach 70. i~80. XX wieku. Kwestia rozpoczęła się jako synteza pewnego szczepu marksizmu z~ideami wyartykułowanymi przez socjalistycznych heretyków, takich jak Jacques Ellul i~Jacques Camatte, którzy zaczęli postrzegać samą naturę technologii jako leżącą u podstaw większości tego, co Marks postrzegał jako wyobcowanie i~opresja kapitału, a tym samym odrzucili ideę, że proletariat, jako zasadnicza część globalnej ,,megamaszyny'', może być agentem rewolucji (Millet 2004). Jako część szerszej krytyki, która rozwinęła się w~tym czasie, kiedy nastawienie produktywistyczne w~tradycyjnej myśli lewicowej, trudno było postrzegać to jako odbiegające od całkowicie normalnej debaty. Jednak w~latach dziewięćdziesiątych najbardziej agresywny nurt myśli prymitywistycznej zaczął skupiać się wokół postaci Johna Zerzana, niegdyś ultralewicowca, który zaczął wyrażać zupełną wrogość nie tylko do ,,lewicy'', ale do samej ,,cywilizacji''. Zerzan w~zasadzie zajął najbardziej radykalne stanowisko, jakie było możliwe, argumentując, że wszystko, od udomowienia roślin po muzykę, pismo, matematykę, sztukę, a ostatecznie nawet mowę -- w~zasadzie wszystkie formy reprezentacji symbolicznej, wszystko inne niż absolutne, bezpośrednie, niezapośredniczone doświadczenie -- były tak naprawdę formami wyobcowania, które można było przezwyciężyć jedynie poprzez zniszczenie całej cywilizacji i~powrót do epoki kamienia. Wpływ Zerzana na anarchizm został znacznie zawyżony w~mediach, ale jest znaczna liczba Zielonych Anarchistów, którzy traktują jego idee bardzo poważnie, i~ci Zieloni Anarchiści produkują wiele zinów i~czasopism, które agresywnie propagują te idee, angażując się w~ciągłe jadowite debaty z~każdym, kto chce poddawać w~wątpliwość jakikolwiek aspekt ultraprymitywistycznego stanowiska\footnote{Zerzan stał się sławny zaraz po Seattle, po części dlatego, że wszyscy dziennikarze nagle zapragnęli porozmawiać z~anarchistą, a on był jedynym, którego większość miała na swoim Rolodex, ponieważ przez jakiś czas był podejrzanym w~sprawie Unabombera.}.

 Pomysł powrotu do paleolitu -- odrzucenie udomowienia roślin, nie mówiąc już o języku -- jest oczywiście absurdalny. Wymagałby zmniejszenia populacji Ziemi o co najmniej 99,9\%. Ani prymitywiści nie są tego całkowicie nieświadomi: ludzie z~\textit{Fifth Estate} prowadzili długą debatę na temat tego problemu w~latach 70., redaktorzy doszli do wniosku, że skoro tak naprawdę nie chcieli widzieć globalnej katastrofy, takiej jak wojna nuklearna, najlepsze, na co można było liczyć, to stopniowy proces ujemnego przyrostu ludności. Większość obecnych prymitywistów wydaje się na przemian otwarcie opowiadać się za upadkiem przemysłowym i~demograficznym -- słyszałem, jak niektórzy twierdzą, że ludzkość jest wirusem, który należy w~dużej mierze wykorzenić -- aby, wbrew wszelkiej logice i~zdrowemu rozsądkowi, zaprzeczać, że masowy spadek populacji byłby nawet konieczny (Zerzan często robi to przed nieanarchistycznymi publicznościami). Jednocześnie ci sami autorzy będą regularnie potępiać każdego, kto opowiada się za klasyczną anarchistyczną strategią ,,budowania nowego społeczeństwa w~skorupie starego''. Wyśmiewają każdą rozmowę o powolnym, bolesnym tworzeniu nowych instytucji jako przestarzałej ,,lewicowości'', argumentując, że tylko całkowite zniszczenie wszystkich istniejących struktur i~instytucji, a następnie powrót do naszej instynktownej ,,dzikości'', może doprowadzić do prawdziwego wyzwolenia.

 Moim celem tutaj nie jest krytyka stanowiska prymitywistów: jest oczywiście bezcelowa. Oczywiście nie ma sensu atakować każdej strategii poza czekaniem na katastrofę, a następnie zaprzeczać, że opowiada się za katastrofą. W istocie chodzi mi o to, że gdyby to było klasyczne stanowisko ideologiczne, należałoby oczekiwać, że skutki będą całkowicie odpolityczniające. Gdyby ktoś naprawdę nie mógł się doczekać upadku przemysłowego lub podobnej apokalipsy, najbardziej oczywistym kierunkiem działania byłby ten, który poszli prawicowi survivalowcy w~latach 80.: udaj się do lasu, wykop bunkier i~zacznij gromadzić żywność w~puszkach i~broń automatyczną. Albo, na przemian, znajdź odległą wyspę i~spróbuj rozpocząć wskrzeszenie technologii z~epoki kamienia. O ile mi wiadomo, żaden zwolennik zielonego anarchizmu nigdy nie zrobił niczego takiego. Zamiast tego mają tendencję do zachowywania się jak każdy inny anarchista. Prymitywiści mogą być bardziej skłonni do angażowania się w~kampanie ekologiczne lub na rzecz praw zwierząt niż, powiedzmy, w~organizowanie związków, ale na przykład w~Nowym Jorku znam zagorzałych Zielonych Anarchistów, którzy pracowali z~Niezależnym Centrum Medialnym w~DAN, w~kolektywach wideo, oddziałach Food Not Bombs, ogrodach społecznościowych, sieciach wsparcia więźniów, grupach feministycznych, kampaniach rowerowych, squatach, spółdzielczych księgarniach, kampaniach antywojenne i~na rzecz praw imigrantów, praw mieszkaniowych, programach pilnowania policji i~prawie wszystkich innych główny przejaw organizacji anarchistycznej. Często w~rzeczywistości prymitywiści okazują się jednymi z~najbardziej wiarygodnych i~oddanych aktywistów.

 W obliczu tego rodzaju sprzeczności trudno uniknąć zadania tego samego pytania, które Evans-Pritchard zadał na temat czarów Zande: ,,jak skądinąd rozsądni ludzie mogą twierdzić, że wierzą w~takie rzeczy?''. Jeśli ktoś wskazuje niektóre z~tych sprzeczności rzeczywistym zwolennikom prymitywizmu -- na przykład prosząc ich o zastanowienie się nad tym, co by się faktycznie stało, gdyby ludność, powiedzmy, Bangladeszu pewnego dnia zdecydowała się przestać uprawiać rolnictwo -- zwykle odpowiedź będzie brzmiała: ,,ale to nie jest program! . To krytyka''. Ewentualnie mogą kwestionować logiczne, pragmatyczne pojęcia argumentacji i~twierdzą, że są poetyckim, intuicyjnym rozumieniem stanu świata, który jest zasadniczo przemieszczony i~niepoprawny. Podobnie, nawet najbardziej zagorzali fani Zerzana zazwyczaj przyznają, po naciśnięciu, że nie są za porzuceniem języka, ale zamiast tego podkreślają, w~jakim stopniu język może wprowadzać w~błąd, być narzędziem ideologii, maskować lub pochłaniać bardziej bezpośrednie formy doświadczenia.

 Myślę, że to wyjaśnia atrakcyjność i~powód, dla którego prymitywizm skłania się ku takim absolutom. Tak naprawdę jest to próba potraktowania absolutnie poważnie tych uczuć całkowitej alienacji, które w~pierwszej kolejności prowadzą tak wielu białych nastolatków z~klasy średniej do anarchizmu, i~przynajmniej próba wyobrażenia sobie świata, w~którym każdy aspekt tej alienacji byłby został całkowicie usunięty. Rezultatem może być tylko rodzaj mitu. Prymitywiści często to przyznają, twierdząc, że rozpowszechnione mity o apokalipsie i~ogrodzie Eden są intuicyjnym zrozumieniem prawdziwych prawd: że kiedyś żyliśmy w~jakimś raju, że go straciliśmy i~że przez katastrofalny upadek społeczeństwa przemysłowego, odzyskamy go ponownie. Mit apokalipsy zastępuje wiarę w~rewolucję. To w~pewnym sensie to samo, tylko bardziej absolutny: tradycyjne anarchistyczne odrzucenie reprezentacji politycznej staje się odrzuceniem reprezentacji w~jakiejkolwiek formie, nawet sztuki czy języka. Większość prymitywistów ma do czynienia głównie z~tym, z~czym mamy do czynienia: kompleksową krytyką alienujących instytucji i~rodzajem niemożliwej do zrealizowania wizji totalnego wyzwolenia, która może przynajmniej dostarczyć inspiracji i~nieustannie przypominać, dlaczego się buntuje. Dla wielu fakt, że nie ma to żadnego sensu dla osób postronnych, jest prawdopodobnie głównym elementem jego atrakcyjności.

 Pozwolę sobie wziąć bardzo inny przykład. Jedną z~głównych form rozpowszechniania idei anarchistycznych w~ostatnich latach w~Ameryce były feministyczne powieści science fiction: od \textit{Wydziedziczonych} Ursuli LeGuin (1974) po \textit{Piąty święty żywioł}Starhawk (1993). Działają w~podobny sposób. Są krystalizacjami pewnych tendencji myślowych, ekstrapolacjami z~pewnych form praktyki, eksperymentami w~utopijnym wyobrażeniu. Główna różnica polega na tym, że ponieważ wizje rozwinięte w~powieściach nie są niczym innym jak fikcją, ci, którzy lubią je czytać (lub pisać), nie mają skłonności do twierdzenia, że alternatywne wizje są błędne. W przypadku Zielonego Anarchizmu jadowity charakter tak dużej części pisarstwa wydaje się wynikać z~połączenia dwóch czynników. Z jednej strony pilność sprawy ekologicznej, poczucie, że planeta jest niszczona i~wszyscy jesteśmy skazani na zagładę, jeśli coś nie zostanie zrobione bardzo szybko, oraz pewien nawyk niezwykle kontrowersyjnej argumentacji odziedziczony po sekciarskich marksistowskich korzeniach\footnote{W rzeczywistości można by powiedzieć, że niekończące się powszechne potępienia ,,lewicowości'' Zerzana lub Boba Blacka są same w~sobie skrajną wersją jednej tendencji w~obrębie tej samej lewicy, którą potępiają.}.

 W tym prymitywiści są niezwykli. Jak wspomniałem, anarchiści od dawna mają tendencję do unikania wielkich teorii. Jak ujął to David Wieck w~1971 roku (na długo, zanim ktokolwiek pomyślał o terminie ,,postmodernizm'' :

 Anarchizm zawsze był antyideologiczny: anarchiści zawsze podkreślali, że życie i~działanie mają pierwszeństwo przed teorią i~systemem. Podporządkowanie teorii oznacza w~praktyce podporządkowanie się władzy (partii), która autorytatywnie interpretuje teorię, a podporządkowanie to fatalnie podważyłoby zamiar stworzenia społeczeństwa bez centralnej władzy politycznej. Zatem żadne pisma anarchistyczne nie są autorytatywne ani ostateczne w~tym sensie, jak pisma Marksa były postrzegane przez jego zwolenników (1971: ix).

 W rzeczywistości większość tego, co pełni rolę teorii w~anarchizmie, wykonuje jakiś gest, aby podważyć jakąkolwiek możliwość użycia jej jako autorytatywnego tekstu. Być może prymitywizm najbardziej przypomina tradycyjną ideologię sekciarską w~próbie pokonania wszystkich przeciwstawnych stanowisk, ale jego treść jest namacalnie fantastyczna i~w większości nie może być odzwierciedlona w~praktyce. Niektóre wizje przybierają formę powieści. Inni są jak komedie. Jeden z~popularniejszych anarchistycznych autorów lat 90. -- na przykład wynalazca koncepcji ,,Tymczasowej Strefy Autonomicznej'' -- pisze pod imieniem Hakima Beya, szalonego izmailickiego poety z~erotyczną obsesją na punkcie młodych chłopców, w~formie komunikatów nieistniejącego mauretańskiego Kościoła prawosławnego.

 Mistyczne pretensje Beya są typową cechą innej tendencji: identyfikowania przestrzeni, którą w~przeciwnym razie mogłaby wypełnić teoria, niejako pozycji transcendentalnej, ze sacrum, ale potem czynienia sacrum śmiesznym. Opowiem o tym zwyczaju później, kiedy będę omawiał rolę gigantycznych marionetek -- które można nazwać głównymi świętymi obiektami ruchu (ale także świadomie głupimi). Tutaj wystarczy powiedzieć, że stosunek anarchizmu do duchowości zawsze był złożony i~ambiwalentny. W XIX i~na początku XX wieku europejski anarchizm zawsze był najsilniejszy w~krajach -- Rosji, Hiszpanii, Włoszech -- z~potężnym Kościołem i~miał tendencję do przyjmowania radykalnie ateistycznego tonu, utożsamiając samo pojęcie Boga z~zasadą hierarchii i~niekwestionowanego autorytetu. (Stąd słynne zdanie Bakunina ,,gdyby Bóg naprawdę istniał, należałoby go obalić''. Były wyjątki -- chrześcijańscy anarchiści, tacy jak Tołstoj -- ale zwykle nie byli blisko z~ruchami społecznymi). Niektórzy twierdzą, że hiszpański anarchizm, zwłaszcza w~jego wiejskich przejawach, sam w~sobie nabrał cech proroczej, tysiącletniej religii (Brenan 1943; por. Borkenau 1937) -- ale jeśli tak, to była to taka religia, której główne rytuały obejmowały działania takie jak palenie kościołów lub usuwanie zmumifikowanych ciał zakonnice z~krypt kościelnych, aby ujawnić czającą się poniżej korupcję (Lincoln 1991). We współczesnym anarchizmie ta wrogość w~dużej mierze zanikła: po części dlatego, że w~wielu krajach Kościół stracił tak wiele ze swojej władzy; po części dlatego, że tak wielu anarchistycznych sojuszników (na przykład ludy rdzenne lub w~Stanach Zjednoczonych kwakrzy, radykalni księża i~duchowni) prawdopodobnie doszli do swojej polityki poprzez przekonania religijne; a częściowo także z~powodu rozwoju specyficznie anarchistycznych form duchowości, takich jak feministyczne pogaństwo. Jednocześnie konkretnie anarchistyczne formy duchowości -- oprócz tego, że są z~natury pluralistyczne i~otwarte (stąd politeizm) -- prawie zawsze są przynajmniej trochę skromne i~zdolne do zdystansowania się od siebie\footnote{Barbara Epstein już zastanawiała się nad tym fenomenem, omawiając rolę duchowości feministycznej w~ruchu akcji bezpośredniej na początku lat 80. - fakt, że ,,wielu pogan jednocześnie wierzy w~Boginię jako rzeczywistość i~Boginię jako metaforę potęgi ludzkiej zbiorowości i~więzi człowieka z~naturą. W ten sam sposób wielu uczestników ruchu akcji bezpośredniej miało jednocześnie naiwne i~wyrafinowane koncepcje magicznej polityki'' (Epstein 1991:184). Ma ona na myśli zarówno przekonanie, że blokada sama w~sobie może zamknąć elektrownię jądrową, jak i~że może podnieść świadomość i~zmienić ramy rozumienia opinii publicznej w~taki sposób, że może przyczynić się do jej zamknięcia. Właściwie, argumentowałem gdzie indziej, że ten rodzaj dwójmyślenia jest typowy dla praktyki magicznej praktycznie wszędzie, od Madagaskaru do Nepalu (Graeber 2002).}. Wielu pogan ma uderzającą zdolność postrzegania swoich poglądów jako głęboko prawdziwych, a jednocześnie jako rodzaj kapryśnej komedii. Często wydają się angażować jednocześnie w~rytuał i~parodię rytuału; moment, w~którym najprawdopodobniej pojawi się śmiech i~autoironia, to właśnie punkt, w~którym zbliżamy się do tego, co najbardziej boskiego, niepoznawalnego lub głębokiego. Ta sama kapryśna, żartobliwa jakość znajduje odzwierciedlenie w~znacznej części pogańskiej literatury feministycznej, jak w~innych gałęziach teorii anarchistycznej, i~wydaje się odzwierciedlać wrażliwość, która w~najlepszym razie postrzega ,,teorię'' jako, jeśli w~ogóle, formę twórczego pisania, zarówno głęboko prawdziwego, ponieważ uwydatnia pewne skądinąd niewidoczne aspekty rzeczywistości, ale jednocześnie głęboko głupiego, ponieważ czyni to, będąc świadomie ślepym na inne aspekty\footnote{Jak ujął to Bob Black w~,,The Abolition of Work'' ,,Możesz się zastanawiać, czy żartuję, czy mówię poważnie. Żartuję \textit{i }mówię poważnie”.}. Również taką, w~której wyobraźnia, zdolność do tworzenia nowych teorii, wizji czy czegokolwiek innego, jest sama w~sobie ostateczną, niepoznawalną, świętą rzeczą. 

 Wszystko to może trochę przesadzone: czytelnik nie powinien chyba brać moich własnych wywodów teoretycznych zbyt poważnie niż te, o których piszę. Najważniejsze jest jednak to, że -- w~przeciwieństwie do niektórych ,,klasycznych'' dzieł Proudhona, Kropotkina, Rockera, Malatesty, De Santillana i~innych, napisanych w~cieniu marksizmu -- współczesna anarchistyczna ,,teoria'', taka jak jest, najwyraźniej nie ma na celu zapewnienia wszechstronnego zrozumienia, które poinstruuje innych o prawidłowym prowadzeniu rewolucji. Nie jest ideologią, teorią historii. Skłania się raczej ku inspirującej, twórczej zabawie. Jest to przede wszystkim ekstrapolacja i~wyobrażeniowa projekcja pewnych form praktyki: doświadczenie pracy w~małej grupie afinicji staje się wzorem dla prymitywistycznych idealizacji zespołu łowców/zbieraczy, uważanych za podstawową jednostkę społeczną przez większość historii ludzkości; doświadczenie prawdziwych eksperymentów w~kontroli robotniczej staje się podstawą wyimaginowanej planety w~opowiadaniu science fiction; doświadczenie siostrzeństwa staje się wzorem dla matriarchalnej religii Bogini; przeżycie dzikiego momentu zbiorowej poetyckiej inspiracji czy nawet szczególnie dobrej imprezy staje się podstawą teorii Tymczasowej Strefy Autonomicznej. Nawet gdy współcześni anarchiści zwracają się ku marksizmowi, ich przeważającymi ulubionymi teoretykami są sytuacjoniści Raoul Vaneigem (1967) i~Guy Debord (1967), teoretycy marksistowscy najbliżsi awangardowej tradycji prób zjednoczenia teorii, sztuki i~życia. 

 Jeśli anarchizm nie jest próbą wprowadzenia w~życie pewnego rodzaju wizji teoretycznej, ale jest nieustanną wymianą między inspirującymi wizjami, antyautorytarnymi postawami i~egalitarnymi praktykami, to łatwo dostrzec, jak etnografia może stać się tak odpowiednim narzędziem dla jego analizy. To jest dokładnie to, co ma robić etnografia: wydobyć ukrytą logikę w~sposobie życia, wraz z~powiązanymi z~nim mitami i~rytuałami, aby uchwycić sens zestawu praktyk. Oczywiście innym sposobem na zrobienie tego byłoby po prostu śledzenie anarchistycznych debat, tak jak zrobiłem to na początku, ponieważ zwykle skupiały się one na kwestiach etycznych i~organizacyjnych. Obecnie debaty te skupiają się przede wszystkim na tym, jak zwalczać rasizm i~seksizm w~ruchu, o formach podejmowania decyzji oraz o pytaniach o przemoc i~niestosowanie przemocy. Ponieważ to ostatnie odnosi się najbardziej bezpośrednio do kwestii relacji między anarchizmem a akcją bezpośrednią, pozwolę sobie przejść do krótkiego rozważenia relacji między nimi, zanim przejdę do zbiorczej historii roli akcji bezpośredniej i~demokracji bezpośredniej w~ruchach społecznych w~północnej Ameryce w~drugiej połowie XX wieku -- zaczynając od lat 60., kończąc na latach 90., w~punkcie w~którym te dwa prądy zaczęły się ostatecznie łączyć.

\section{Przemoc i Nie-Przemoc}

 Kwestia przemocy, niestosowania przemocy i~niszczenia własności nawiedza anarchizm od co najmniej XIX wieku.

 Są oczywiste powody, dla których powinno to być problemem. Z jednej strony istnieje wiele powodów, dla których anarchiści mogą być podejrzliwi wobec przemocy. Po pierwsze, anarchiści wychodzą od zasady, że nasz sposób oporu powinien ucieleśniać świat, który chce się stworzyć. Prawie nikt nie chce tworzyć bardziej brutalnego świata. Anarchiści próbują organizować się na zasadach niehierarchicznych i~argumentują, że jest to nie tylko bardziej sprawiedliwe, ale i~bardziej efektywne. Przemoc -- szczególnie przemoc agresywna -- jest jedną z~niewielu form ludzkiej aktywności, która wydaje się bardziej efektywna, jeśli jest zorganizowana na podstawie odgórnego, dowodzenia. To i~towarzysząca jej potrzeba zachowania tajemnicy sprawia, że im bardziej ktoś przygotowuje się do wojny lub czegoś podobnego, tym trudniej jest zorganizować się demokratycznie.

 Z drugiej strony anarchiści chcą widzieć rewolucję społeczną i~trudno sobie wyobrazić, jak mogłaby się wydarzyć bez jakiegokolwiek gwałtownego konfliktu.

 Co więcej, nalegają również na moralną suwerenność jednostki i~często czują się bardzo niekomfortowo z~kodeksami postępowania. W zasadzie każdy, kto stawia opór, powinien decydować, co jest uzasadnionym aktem oporu wobec wewnętrznie niesłusznej władzy. Teraz ważne jest, aby nie przesadzać: w~praktyce milczące porozumienia istnieją zawsze. Zasada CLAC dotycząca ,,różnorodności taktyk'', o której tak wiele słyszeliśmy we wcześniejszych rozdziałach, mogła brzmieć dla pacyfistów takich jak SalAMI jak ,,wszystko ujdzie'', ale opierała się na wspólnym zrozumieniu, że nikt nie pojawi się z~bronią palną lub materiałami wybuchowymi. To byłoby po prostu nie do pomyślenia. Jeśli moje doświadczenie jest czymś godnym uwagi, gdyby ktokolwiek choćby zasugerował, że tak zrobi, natychmiast uznano by go za policyjnego infiltratora właśnie z~tego powodu. Niemniej jednak takie milczące porozumienia istnieją tylko wśród aktywistów. Jeśli przyłączą się osoby z~zewnątrz, nigdy nie można być całkowicie pewnym, co zamierzają zrobić. Na przykład w~Quebecu wokół Czarnego Bloku w~pewnym momencie akcji pojawiła się przerażająca historia, że  ,,francuscy gangsterzy'' mieli pojawić się pod Murem z~bronią palną (czyn, który, jak zakładali, zostałby automatycznie przypisany do nich). W Seattle, po starannie wymierzonym niszczeniu celów korporacyjnych przez Czarny Blok, w~kilku przypadkach nastąpiły epizody oportunistycznych grabieży dokonywanych przez miejscowych afroamerykańskich nastolatków. W takim przypadku jest mało prawdopodobne, by ktokolwiek z~bloku się sprzeciwiał. Widzieć, jak uciskane społeczności powstają i~przyłączają się do ciebie, jest w~pewnym sensie istotą sprawy. I, jak w~St. Jean Baptiste, standardy uciskanej społeczności akceptowalnych taktyk tej mogą być inne niż twoje własne. Jednakże większość dużych mobilizacji (w tym Québec City) także widziało przynajmniej kilka pomniejszych epizodów tego, co nazywam ,,problemem pijanego bractwa'' -- oportunistycznej przemocy, głównie dla zabawy, ze strony młodych ludzi, których polityka prawdopodobnie nie ma nic wspólnego z~aktywistami, lub nawet jawnie prawicowych. W Europie może do tego faktycznie zachęcać policja, traktując to jako pretekst do stosowania środków represyjnych. Ekstremalny przykład miał miejsce w~Genui, kiedy policja najwyraźniej dała do zrozumienia, że  przymknie oko na tego typu sprawy, a na miejsce zjechali się faszyści i~chuligani futbolowi z~całej Europy.

 Mimo to Genua była ekstremalna i~zwykle jest to raczej drobny problem. Najgorszy dylemat moralny dla anarchistów pojawia się, gdy izolowane jednostki, twierdząc, że z~anarchistycznej inspiracji, robią coś naprawdę brutalnego. i~znowu, anarchiści, którzy zamordowali głowy państw na przełomie XIX i~XX wieku, są prawdopodobnie najbardziej dramatycznym przykładem. Fascynujące w~takich przypadkach jest to, że większość takich zabójstw była dokonywana przez pojedyncze osoby, a nie osoby działające w~rzeczywistych organizacjach anarchistycznych. Wielu miało jedynie mgliste pojęcie o anarchistycznych zasadach. Jeśli jednak poważnie potraktuje się zasadę autonomii moralnej, trudno takie czyny traktować jako całkowicie nieusankcjonowane. Z anarchistycznego punktu widzenia, o ile uzasadnione jest angażowanie się w~jakikolwiek akt przemocy międzyludzkiej, głowy państw, główni kapitaliści, lub wysocy urzędnicy są jasno najbardziej uzasadnionymi celami. Przyjęcie bardziej konwencjonalnej strategii wojny partyzanckiej, tworzenia małej armii i~atakowania posterunki policji lub posterunki wojskowe -- w~ten sposób próbując zabić grupę zwykłych ludzi, którzy w~żadnym sensie nie są bezpośrednio odpowiedzialni za politykę, której się sprzeciwiamy -- byłoby zdecydowanie bardziej problematyczne. (Właściwie trudno zaprzeczyć, że według jakichkolwiek standardów moralnych zabójstwo jest o wiele lepsze od wojny). Z drugiej strony, ponieważ głowy państw mają tendencję do uznawania tego rodzaju logiki za wysoce kontrowersyjną, skutki są niezmiennie katastrofalne. Anarchistyczni pisarze, tacy jak Piotr Kropotkin czy Emma Goldman, zajmujący się głównie rozpowszechnianiem anarchistycznych idei wśród szerszej publiczności, często boleśnie zmagali się z~tym, co zrobić lub powiedzieć o takich ludziach. Czy ich potępienie jest \ uzasadnione? Jaka im się należy solidarność? Czy przynajmniej nie istnieje obowiązek wyjaśniania światu swojego punktu widzenia? Debaty na temat wybitych okien i~niszczenia mienia, czy możliwości wybuchu koktajli w~Québec City, to po prostu nowsze wersje tego samego. 

 Działacze, którzy są na scenie zaledwie od dwóch, trzech lat, narzekają na konieczność ciągłego wymyślania koła w~takich sprawach. Za każdym razem, gdy odbywa się poważna akcja, każdy musi przejść dokładnie te same debaty. Niektórzy twierdzą, że taktyki konfrontacyjne lub niszczenie mienia tylko sprawią, że aktywiści będą wyglądać źle w~oczach opinii publicznej. Inni będą argumentować, że media korporacyjne nie sprawią, że będziemy dobrze wyglądać, cokolwiek byśmy robili. Niektórzy będą argumentować, że jeśli rozbijesz okno Starbucksa, będzie to jedyna historia w~wiadomościach, skutecznie blokująca wszelkie inne problemy; inni odpowiedzą, że jeśli nie dojdzie do zniszczenia mienia, nie będzie żadnej historii. Inni będą twierdzić, że konfrontacyjne taktyki pozbawiają aktywistów moralnej wyższości; inni oskarżą tych ludzi o elitaryzm, i~twierdzą, że przemoc systemu jest tak przytłaczająca, że odmowa skutecznej konfrontacji jest sama w~sobie przyzwoleniem na przemoc. Niektórzy stwierdzą, że taktyka bojowników zagraża pokojowym demonstrantom; inni będą upierać się, że o ile nie stworzy się jakiejś policji pokojowej, która fizycznie grozi każdemu, kto maluje sprayem lub wybija okno, niektórzy prawdopodobnie to zrobią, a jeśli tak, koordynacja z~bojownikami, a nie izolowanie ich, jest o wiele bezpieczniejsze dla wszystkich zainteresowanych. W końcu prawie zawsze kończy się to samo postanowienie: że dopóki nikt nie atakuje drugiego człowieka, ważne jest zachowanie solidarności. Ostatnią rzeczą, jakiej chcesz, to znaleźć się w~sytuacji takiej jak Seattle, gdzie faktycznie pacyfiści fizycznie atakowali anarchistów próbujących wybić szyby lub próbowali oddać ich policji. Wielu zauważa, że wniosek jest tak nieunikniony, że chciałoby się po prostu przyspieszyć debatę, ale, jak wielu z~rezygnacją zauważa, wydaje się, że za każdym razem, gdy toczy się ważna akcja, nowo wprowadzeni do ruchu muszą wypracować wniosek dla siebie.

 Jednym z~rezultatów jest jednak rodzaj stałego paradoksu wewnątrz anarchizmu. Nie chodzi o to, że nie można znaleźć pacyfistycznych anarchistów. Wielu pacyfistów uważa się za anarchistów. Jednak ci współcześni anarchiści, którzy nie są pacyfistami, mają tendencję do unikania jakichkolwiek skojarzeń z~pacyfizmem i~prawdopodobnie zareagują na wzmiankę o tym słowie z~energicznym potępieniem -- pomimo faktu, że w~szerszej perspektywie ich idee i~praktyki wyłoniły się w~znacznym stopniu z~tej tradycji. Trudno byłoby znaleźć anarchistę, którego instynkt nie byłby bardziej po stronie Malcolma X niż Martina Luthera Kinga czy Gandhiego; jednak pozostaje faktem, że pod względem ogólnego podejścia ,,Stań się zmianą, którą chcesz zobaczyć'' Gandhiego wydaje się tysiąc razy bardziej zgodne z~duchem anarchistycznym niż ,,za wszelką cenę'' Malcolma X -- a sam Gandhi dostrzegł silne pokrewieństwo filozoficzne swoich własnych idei i~anarchizmu, a z~pewnością Malcolm X nie dostrzegał. ,,Za wszelką cenę'' w~rzeczywistości wydaje się strasznie podobna do logiki celu-usprawiedliwiającego-środki, którą anarchizm konsekwentnie odrzuca. Jednak praktyczne irytacje wobec pacyfistów, w~połączeniu z~nieuniknionym instynktem identyfikowania się z~najbardziej radykalną opcją, sprawiają, że prawie zawsze anarchista będzie utożsamiał się z~Malcolmem X. 

 Większość anarchistów w~dzisiejszych czasach, na przykład, lubi przytaczać argumenty, takie jak w~\textit{Pacifism as Pathology}(1998), aktywisty rdzennych Amerykanów Warda Churchilla, to jest, że sam pacyfizm jest głównie sposobem na dobre samopoczucie białych liberałów, że autentycznie uciskane grupy nie mają takich luksusów, że wyjątki -- zwycięstwa Gandhiego lub Kinga -- były możliwe dzięki lękowi przeciwników przed bardziej przemocowymi alternatywami. (Fakt, że autorzy tacy jak Churchill mają tendencję do odrzucania anarchistycznej krytyki hierarchii na rzecz przywództwa w~stylu wojskowym, zwykle pozostaje niezauważony lub spisany na straty jako nieistotny)\footnote{Przynajmniej autor pamięta, że  Ward Churchill robił to, gdy był przesłuchiwany przez anarchistów na forum w~2002 roku. Inni - zwłaszcza mój wydawca, Charles - mówią mi, że od tego czasu moderował swoje poglądy w~takich sprawach.}. Fakt, że Churchill jest rdzennym Amerykaninem, jest jednak znaczący. W rzeczywistości bardzo niewielu anarchistów w~Ameryce Północnej wyszłoby daleko poza wybicie okna; prawie wszyscy skrupulatnie unikają krzywdzenia innych w~jakikolwiek sposób. Jak od czasu do czasu zwracam uwagę dziennikarzom, trudno nie znaleźć ciągłych odniesień do anarchistów z~Czarnego Bloku jako ,,silnie'' interesujących, gdy ktoś spędza z~nimi jakikolwiek czas i~obserwuje ich, na przykład ostrożnie unikających nadepnięcia na robaki lub debaty o tym, czy naprawdę jest uzasadnione zabicie komara. Prawdziwy punkt rozłamu pojawia się właśnie w~kwestii solidarności. Aby zająć konsekwentnie stanowisko niestosowania przemocy, należałoby na przykład: powiedzieć Zapatystom w~Chiapas, że tak naprawdę nie powinni byli przeprowadzać zbrojnego powstania -- choćby krótkotrwałego -- lub Czarnym Panterom, że banda białych anarchistów z~klasy średniej miała więcej uprawnień, by im powiedzieć, jaką taktykę zastosować. Ta dychotomia -- między budowaniem społeczności (w której anarchiści mają wszystko wspólne z~pacyfistami) a solidarnością z~uciskanymi grupami -- jest stałym dylematem, który pojawi się w~tej książce.

 Warto zauważyć, że historycznie anarchizm kwitł jako ruch rewolucyjny przede wszystkim w~czasach pokoju i~w społeczeństwach w~dużej mierze zdemilitaryzowanych. Jak zauważył Eric Hobsbawm (1973:61), w~ostatnich latach XIX wieku, kiedy większość partii marksistowskich szybko stawała się reformistycznymi socjaldemokratami, to anarchizm stał w~centrum rewolucyjnej lewicy\footnote{,,W latach 1905-1914 marksistowska lewica znajdowała się w~większości krajów na marginesie ruchu rewolucyjnego, główne ciało marksistów utożsamiano z~de facto nierewolucyjną socjaldemokracją, podczas gdy większość rewolucyjnej lewicy stanowiła anarcho- syndykalistyczny, a przynajmniej znacznie bliższy ideom i~nastrojom anarchosyndykalizmu niż klasycznemu marksizmowi'' (Hobsbawm 1973:61).}. Rzeczy naprawdę zmieniły się dopiero wraz z~i~wojną światową i~oczywiście po rewolucji rosyjskiej. Konwencjonalna historiografia zakłada, że  to powstanie Związku Radzieckiego doprowadziło do upadku anarchizmu i~wszędzie postawiło komunizm na pierwszym planie. Mimo to wydaje mi się, że można by na to spojrzeć inaczej. Pod koniec XIX wieku większość ludzi szczerze wierzyła, że  wojna między uprzemysłowionymi potęgami staje się niepotrzebna. Do 1900 roku nawet używanie paszportów uważano za przestarzałe barbarzyństwo. Chociaż przygody kolonialne zawsze były czymś stałym, wojna między, powiedzmy, Anglią i~Francją wydawała się równie nie do pomyślenia jak dzisiaj. Dla kontrastu, ,,krótki wiek dwudziesty'' (który zaczął się w~1914 roku i~zakończył około 1989 lub 1991 roku) był prawdopodobnie najbardziej brutalny w~historii ludzkości. Był to wiek, w~którym główne mocarstwa nieustannie zajmowały się prowadzeniem wojen światowych lub przygotowywaniem się do nich. Nic więc dziwnego, że anarchizm mógł wydawać się nierealistyczny. Tworzenie i~utrzymywanie ogromnych zmechanizowanych maszyn do zabijania wydaje się jedyną rzeczą, w~której anarchiści z~definicji nigdy nie mogą być bardzo dobrzy. Nie dziwi też, że partie marksistowskie (już zorganizowane jako struktura dowodzenia i~dla których organizacja ogromnych zmechanizowanych maszyn do zabijania często okazywała się jedyną rzeczą, w~której były szczególnie dobre) zaczęły wydawać się w~porównaniu wybitnie praktyczne i~realistyczne. Jest więc całkowicie zrozumiałe, że w~momencie zakończenia zimnej wojny i~gwałtownego konfliktu między uprzemysłowionymi mocarstwami znów wydaje się nie do pomyślenia, anarchizm powrócił do stanu, w~jakim był pod koniec XIX wieku: ruchu międzynarodowego w~samym centrum rewolucyjnej lewicy. Zaskakujące było to, że stało się to niemal natychmiast.

 Co więcej, można by argumentować, że skuteczność bardziej bojowej taktyki anarchistycznej zależy od skutecznej demilitaryzacji społeczeństwa. Rozważmy tutaj bitwy o skłoty w~Niemczech czy Włoszech, a nawet bitwy wokół rozbudowy lotniska Narita w~Japonii, w~których anarchiści lub ich miejscowi odpowiednicy byli w~stanie toczyć zaciekłe bitwy z~policją, bronić terytorium pałkami i~kamieniami przed gazem łzawiącym i~armatkami wodnymi, i~dość często, rzeczywiście mogli wygrać. Trudno wyobrazić sobie coś podobnego w~Stanach Zjednoczonych. W Ameryce policja po prostu nie pozwoli sobie na przegraną. Jeśli zdecydują się na wejście siłowe do skłotu, skłot ten zostanie utracony; jedynym powodem, dla którego warto go bronić, jest utrudnienie policji w~taki sposób, że będą się wahać przed atakowaniem innych squatów w~przyszłości. Nie tylko dlatego, że społeczeństwo amerykańskie jest o wiele silniej nadzorowane; dzieje się tak również dlatego, że Niemcy, Włochy i~Japonia -- wszystkie, co ważne, dawne mocarstwa Osi -- zostały tak skutecznie zdemilitaryzowane. Walki pozycyjne z~policją są możliwe tylko w~społeczeństwach, w~których wszyscy, w~tym opinia publiczna, są świadomi, że prawie nikt nie posiada broni palnej, a taktyka policyjna odpowiednia dla społeczeństwa, w~którym większość przestępców, jak można zakładać, jest ciężko uzbrojonych -- na przykład drużyny SWAT -- wydają się szalenie nieodpowiednia. i~na pewno w~tych częściach Europy, gdzie broń palna i~wojskowy know-how są znacznie szerzej dostępne (myślimy o Rosji, Albanii, byłej Jugosławii, i~podobnie Irak), klasyczny anarchizm i~taktyki anarchistyczne nie znajdują tak żyznego gruntu.

 Co ciekawe, prawdziwą inspiracją dla tego rodzaju taktyk stosowanych w~obecnej fali protestów globalizacyjnych są ruchy w~częściach Globalnego Południa, które do niedawna nie były w~stanie w~ogóle zaangażować się w~pokojową akcję bezpośrednią. People's Global Action, która wystosowała wezwanie do Seattle, została założona z~inicjatywy Zapatystycznej Armii Wyzwolenia Narodowego (EZLN) w~Chiapas. Wydaje mi się, że ruch zapatystów najlepiej można postrzegać jako próbę przejęcia go przez ludzi, którym historycznie odmawiano prawa do pokojowego, obywatelskiego oporu; zasadniczo, sprawdzenie blefu neoliberalizmu i~jego pretensji do demokratyzacji i~oddania władzy ,,społeczeństwu obywatelskiemu''. Jest to, jak mówią jej dowódcy, armia, która nie aspiruje do bycia armią. Od czasu ich pierwszego, trzytygodniowego powstania w~styczniu 1994 roku, stała się również najmniej brutalnej ,,armią'', jaką można sobie wyobrazić (jest czymś w~rodzaju tajemnicy poliszynela, że  od co najmniej pięciu lat nie nosili nawet prawdziwej broni). EZLN jest rodzajem armii, która organizuje ,,inwazje'' na meksykańskie bazy wojskowe, do których setki rebeliantów wdzierają się całkowicie bez broni, by krzyczeć i~próbować zawstydzić tamtejszych żołnierzy. Pozostali dwaj kluczowi członkowie założyciele PAR to KRRS, gandhijski ruch chłopski w~Indiach i~MST, czyli Ruch Chłopów Bezrolnych, w~Brazylii. Ci ostatni zdobyli w~Brazylii ogromny autorytet moralny dzięki pokojowym akcjom masowym, mającym na celu całkowicie pokojowe ponowne zajęcie nieużytkowanych ziem. Podobnie jak w~przypadku Zapatystów, jest całkiem jasne, że gdyby ci sami ludzie spróbowali tego samego dwadzieścia lat temu, po prostu zostaliby rozwaleni. W rzeczywistości najbardziej radykalne ruchy w~Ameryce Południowej są obecnie tak pokojowe, jak im się wydaje, że może ujść im na sucho: większość, podobnie jak bojownicy w~Québec City, ograniczy się do rzucania kamieniami, a następnie zazwyczaj przeciwstawia się całkowicie opancerzonej policji od zamieszek, ale nigdy nie próbowałoby użyć broni palnej. Sytuacja jest skomplikowana, ponieważ w~wielu częściach Ameryki Łacińskiej istnieje, i~od dawna była, znacznie bogatsza tradycja pokojowych akcji bezpośrednich niż w~Europie czy Ameryce Północnej, ale bezpośrednia inspiracja ruchu globalizacyjnego wydaje się pochodzić głównie od grup, które lub trzydzieści lat temu, prawie na pewno byliby zmuszeni uciekać się do wojny partyzanckiej, ale które, widząc tak wiele wcześniejszych ruchów partyzanckich niszczących się lub degenerujących się w~nihilistycznych gangsterów, zamiast tego wybrali radykalnie inne podejście. Odchodząc od taktyki wojskowej, często kończyli również -- często raczej wbrew sobie -- w~kierunku znacznie bardziej anarchistycznych form organizacji.

\section{Niezwykle krótka historia relacji między działaniami bezpośrednimi a demokracja bezpośrednią w USA od lat 60  }

 Przed II wojną światową głównym ośrodkiem akcji bezpośredniej w~Ameryce Północnej był, jak wspomniałem, ruch robotniczy. Okres po wojnie był świadkiem stopniowego łączenia się tradycji akcji bezpośredniej i~demokracji bezpośredniej, przy czym te dwie rzeczy naprawdę połączyły się pod koniec lat 70. i~na początku lat 80. i~były gotowe do odrodzenia pod wpływem Zapatystów. Historia jest bardzo skomplikowana, ale karykaturalna wersja może wyglądać mniej więcej tak:

 Nowa Lewica z~lat 60. rozpoczęła się od wezwania do ,,demokracji uczestniczącej'' w~słynnym Oświadczeniu z~Port Huron z~1962 roku, dokumencie założycielskim Students for the Democratic Society (SDS -- Studenci na rzecz Demokratycznego Społeczeństwa). Jego główny autor, Tom Hayden, został zainspirowany przez Johna Deweya i~C. Wrighta Millsa, a dokument wyróżniał się wezwaniem do szerokiej demokratyzacji wszystkich aspektów amerykańskiego społeczeństwa, aby stworzyć sytuację, w~której ludzie sami podejmują ,,decyzje, które wpływają na ich życie''\footnote{Bardziej bezpośrednią inspiracją dla Haydena był jego były nauczyciel filozofii Arnold Kaufman z~Uniwersytetu Michigan.}. Można to postrzegać jako bardzo anarchistyczną wizję, ale SDS, na początek, miało zupełnie inną orientację. W rzeczywistości ich pierwotny program polityczny polegał na radykalizacji Partii Demokratycznej (porzucili ją tylko wtedy, gdy stanęli w~niemożliwej pozycji dzięki ciągłemu dążeniu Demokratów do wojny wietnamskiej). Co ważniejsze, ci, którzy sformułowali to oświadczenie, wydawali się mieć tylko najbardziej szkicowe wyobrażenia o tym, co ,,demokracja uczestnicząca'' może oznaczać w~praktyce. Jest to najbardziej widoczne w~sprzecznym charakterze struktury SDS. Jak zauważyła Francesca Polletta (2002), na papierze SDS była dość formalną organizacją odgórną, z~centralnym komitetem sterującym i~spotkaniami prowadzonymi zgodnie z~Regułami Porządku Roberta. W praktyce składał się z~w dużej mierze autonomicznych komórek, które działały w~rodzaju surowego, de facto procesu konsensusu. Z kolei nacisk na konsensus wydaje się zainspirowany przykładem SNCC, Student Nonviolent Coordinating Committee (Studenckiego Bezprzemocowego Komitetu Koordynacyjnego), studenckiego skrzydła ruchu praw obywatelskich. SNCC zostało pierwotnie utworzone z~inicjatywy Anity Baker i~wielu innych aktywistów zaangażowanych w~Southern Christian Leadership Conference (SCLC), którzy mieli nadzieję stworzyć alternatywę dla odgórnej struktury SCLC i~charyzmatycznego przywództwa (oczywiście w~postaci dra Martina Luthera Kinga, Jr.). Słynący z~organizowania z~siedzących blokad barów, przejażdżek wolnościowych i~innych akcji bezpośrednich, SNCC było zorganizowane na całkowicie zdecentralizowanych zasadach, z~pomysłami na nowe projekty, które miały wyłonić się z~poszczególnych oddziałów, z~których wszystkie były obsługiwane przez rodzaj szorstkiego konsensusu.

 Ten nacisk na konsensus wydaje się zaskakujący, skoro w~tamtych czasach nie było zbyt wielu modeli. Zarówno w~SNCC, jak i~SDS wydaje się, że wynikało to z~przekonania, że  skoro nikt nie powinien robić niczego wbrew jego woli, decyzje powinny być naprawdę jednomyślne. Jednak wydaje się, że nie było czegoś takiego, jak to, co obecnie nazywa się ,,procesem konsensusu'' w~formalnym znaczeniu tego terminu. Problem polegał na tym, że nie było oczywistego przykładu. Jedyne społeczności w~Ameryce Północnej z~żywą tradycją konsensusowego podejmowania decyzji (kwakrzy i~różne grupy rdzennych Amerykanów) były albo nieznane, albo niedostępne, albo niezainteresowane nawracaniem. Kwakrzy w~tamtym czasie postrzegali konsensus zasadniczo jako praktykę religijną; byli, według Polletty (2002:195), w~rzeczywistości dość oporni na ideę nauczania tego kogokolwiek innego.

 Nowa Lewica była, jak wszyscy wiemy, zasadniczo ruchem kampusowym. Paul Mattick Jr. (1970) twierdził, że fala aktywizmu lat 60. wydaje się wyłonić z~pewnego rodzaju społecznego korka. Ideałem państwa opiekuńczego w~tamtych czasach było rozładowywanie napięć klasowych poprzez oferowanie widma nieustannej mobilności społecznej (w podobny sposób jak kiedyś pogranicze). Po wojnie ze strony rządu podjęto bardzo świadome wysiłki, aby wpompować środki w~system szkolnictwa wyższego, który zaczął się gwałtownie rozwijać wraz z~większą liczbą dzieci z~klasy robotniczej uczęszczających na uniwersytety. Problem polegał oczywiście na tym, że takie krzywe wzrostu niezmiennie osiągają swoje granice i~jak każdy rząd Trzeciego Świata, który próbował zastosować tę strategię, przekonał się, że kiedy to się dzieje, wyniki są zazwyczaj wybuchowe. W latach sześćdziesiątych zaczęło się tak dziać. Miliony studentów nie miały żadnej realistycznej perspektywy znalezienia pracy, która miała jakikolwiek związek z~ich rzeczywistymi oczekiwaniami lub możliwościami -- co jest normalną perspektywą w~społeczeństwach przemysłowych, ale nagle sytuacja znacznie się pogorszyła. Byli to studenci, którzy jako pierwsi zaangażowali się w~SDS; ludzie, którzy, jak podkreśla Mattick, podobnie jak ich odpowiednicy na Globalnym Południu, zawsze postrzegali siebie jako swego rodzaju oderwany fragment elity administracyjnej. Sugeruje, że było to kluczowe dla zrozumienia ograniczeń Nowej Lewicy --  że aktywiści niezmiennie postrzegali siebie jako ,,organizatorów'', pracowników socjalnych\footnote{Badania demograficzne (np. Flacks 1971; Mankoff i~Flacks 1971) wskazywały, że we wczesnych latach SDS ruch składał się w~dużej mierze ze studentów sztuk wyzwolonych na elitarnych uniwersytetach, pochodzących z~zamożnych, lewicowych lub lewicowych rodzin zawodowych: dzieci lekarzy, prawników, nauczycieli, a nie biznesmenów; dzieci odnoszących sukcesy rodzin imigrantów, a nie członków elity starych pieniędzy. Jednak po rozszerzeniu SDS pod koniec lat 60. baza społeczna stała się znacznie szersza i~zaczęła obejmować również wielu uczniów pochodzących z~klasy robotniczej. Jak zobaczymy, ten ostatni wzorzec jest zasadniczo tym, który zawsze powtarza się w~ruchach rewolucyjnych: zbieżność wyalienowanych i~zbuntowanych dzieci z~klas zawodowych ze sfrustrowanymi, ale ambitnymi dziećmi z~klasy robotniczej z~pewnym doświadczeniem w~wykształceniu wyższym.}. 

 Tym, co jednoczyło wszystkie frakcje lewicy, była koncepcja ich związku z~rzeczywistymi lub wymyślonymi społecznościami jako organizatorów -- na wzór związkowców i~pracowników socjalnych -- a nie jako ,,kolegów studentów'' lub pracowników ze szczególnym zrozumieniem sytuacji dzielonej z~innymi i~pomysłów, co z~tym zrobić. Pomimo sporu co do głównego celu organizowania się -- bezrobotnych, robotników fizycznych, pracowników umysłowych, młodzieży porzucającej naukę -- w~każdym przypadku ,,społeczność'' była postrzegana jako potencjalna ,,grupa wyborcza'' (lub, w~język PL [Progressive Labor Party -- Postępowej Partii Pracy] ,,baza''). Radykałowie postrzegali siebie jako zawodowych rewolucjonistów, siłę, że tak powiem, poza społeczeństwem, organizującą tych wewnątrz we własnym imieniu. Działacz odgrywał więc rolę zarezerwowaną w~teorii liberalnej dla państwa, punkt, który nie może być pominięty w~próbie zrozumienia przejścia Nowej Lewicy z~orientacji liberalnych reform rządu do leninowsko-stalinowskich koncepcji socjalizmu (Mattick 1970:22)

 Sprzeczności tej sytuacji w~końcu ujawniły się w~miarę upływu dekady. Kryzys został zapoczątkowany najpierw w~grupach takich jak SNCC, kiedy żądania dotyczące praw obywatelskich zaczęły ustępować miejsca wezwaniom do Black Power. Radykałowie w~SNCC, którzy w~końcu mieli założyć Czarne Pantery, wezwali białych aktywistów do zaprzestania pracy w~sojuszu i~powrotu do własnych społeczności, w~szczególności w~celu organizowania białych społeczności przeciwko rasizmowi. Działacze SDS zawsze przyjmowali takie wezwania z~wielką ambiwalencją (Barber 2001) -- po części dlatego, że nigdy nie byli do końca pewni, jakie powinny być ich własne społeczności. Można powiedzieć, że coś podobnego próbowano na początku lat sześćdziesiątych w~ramach projektu Economic Research Areas Project (ERAP), który miał być białym odpowiednikiem oddolnego organizowania praw obywatelskich, która wprowadziła aktywistów SDS do biednych białych społeczności, i~próbowało zmobilizować społeczności wokół spraw będących przedmiotem wspólnego zainteresowania. Niektóre z~tych projektów zwyciężyły w~przeprowadzaniu lokalnych reform, ale organizatorzy nigdy nie czuli się częścią społeczności, w~których pracowali, czuli się odizolowani od innych aktywistów, a niewielu uznało wyniki za warte poświęcenia. Projekt rozpadł się w~1965 roku. Zamiast tego, jak tak przenikliwie zauważył Mattick, wielu zaczęło zdawać sobie sprawę, że jeśli istnieje sposób na przezwyciężenie wyobcowania miejsc pracy bez przyszłości, na znalezienie pracy, która faktycznie spełnia ich wyobraźnię, to właśnie sam aktywizm. Innymi aktywistami, w~rzeczywistości, \textit{były }ich społeczności.

 Kryzys zapoczątkowany przez Black Power ostatecznie poprowadził w~dwóch bardzo różnych kierunkach. Ponownie, za cenę rażącego uproszczenia: kiedy ich sojusznicy w~ruchu na rzecz praw obywatelskich porzucili ich, biali aktywiści faktycznie mieli dwie opcje. Mogliby albo próbować budować własne instytucje kontrkulturowe, albo mogli skupić się na sprzymierzaniu się ze społecznościami lub grupami rewolucyjnymi w~walce za granicą, tj. z~Viet Congu lub innymi rewolucjonistami z~Trzeciego Świata, którzy zabraliby praktycznie wszystkich sojuszników, jakich mogliby zdobyć. Gdy SDS zaczęło się rozpadać na skłócone frakcje maoistów, grupy takich jak Diggers i~Yippies (założone w~1968 roku) wybrały pierwszą opcję. Wielu z~nich było jednoznacznie anarchistami i~z pewnością zwrot pod koniec lat 60. w~kierunku tworzenia autonomicznych kolektywów i~budowania instytucji mieścił się wprost w~tradycji anarchistycznej, podczas gdy nacisk na wolną miłość, środki psychodeliczne i~tworzenie alternatywnych form przyjemności znajdował się wprost w~tradycji bohemy, z~którą europejsko-amerykański anarchizm zawsze był przynajmniej w~pewnym stopniu powiązany. Slogan Yippie, ,,rewolucja dla rewolucji'', może być postrzegany jako wyłaniający się bezpośrednio z~uświadomienia sobie, że sam aktywizm może stać się głównym środkiem przezwyciężenia alienacji. Inną opcją było postrzeganie siebie jako sojusznika przede wszystkim z~rewolucyjnymi społecznościami za granicą: stąd obsesja gloryfikowania rewolucyjnych bohaterów na Kubie, w~Wietnamie, Chinach i~gdzie indziej (ludzi, którzy, jak wskazywali krytycy sytuacjoniści i~autonomiści, byli zasadniczo ikonami tego rodzaju nowych, radykalnych elit administracyjnych, z~którymi SDS zawsze milcząco się utożsamiała) i~poczucie, żeby uderzyć w~imperium z~łona Bestii.

 Każda strategia oznaczała powrót do akcji bezpośredniej, ale jednocześnie odrzucenie całego projektu tworzenia egalitarnych struktur decyzyjnych. Hipisów i~Yippiesów można uznać za nieco ambiwalentne przykłady pod tym względem, ponieważ małe gminy i~wiele alternatywnych instytucji powstałych w~tym procesie generalnie działało na zasadach demokratycznych. Mimo to Yippies, ze swoimi dzikimi, inspirowanymi kwasem psikusami i~medialnymi wyczynami, mieli tendencję do przekształcania się w~platformę dla charyzmatycznych impresariów, takich jak Abbie Hoffman i~Jerry Rubin, w~stylu, który okazywał się notorycznie wyobcowany dla niektórych członków białej klasy robotniczej. Weathermen z~kolei podjęli próbę zamachów bombowych skierowanych na cele wojskowe i~korporacyjne, miały inspirować do spontanicznej emulacji i~popychać społeczeństwo do rewolucyjnej konfrontacji -- choć z~istotnym ograniczeniem, że nie chcą nikogo zabijać. Skończyło się głównie na wysadzaniu pustych budynków. Co ciekawe, oba podejścia miały ogromny wpływ na późniejszą politykę medialną, ponieważ dziennikarze głównego nurtu zaczęli czuć się współwinni, dochodząc do wniosku, że coraz bardziej dzikie i~destrukcyjne czyny były w~rzeczywistości inspirowane potrzebą nieustannej eskalacji, aby trafiać na pierwsze strony gazet. Słyszałem na przykład wielokrotnie plotki od weteranów z~lat 60., że kampania bomb Weathermen była znacznie bardziej rozległa i~niszczycielska niż kiedykolwiek zostały zarejestrowane, ale krajowe media podjęły świadomą decyzję, by zaprzestać o tym informowania. Nie mam pojęcia, czy to prawda. Jedno jest jednak jasne, że od tego okresu, amerykańskie media stały się, bardziej niż w~dowolnej innej demokracji przemysłowej, niesamowicie ostrożne w~informowaniu o dowolnych akcjach aktywistów, lub nawet demonstracjach. 

 Ten punkt stanie się później ważny. Na razie jednak kluczową kwestią jest to, że żadna z~tych grup nie połączyła swojego zainteresowania akcją bezpośrednią z~naciskiem na zdecentralizowane podejmowanie decyzji; wręcz przeciwnie, czy to dlatego, że z~jednej strony skupiono się na charyzmatycznych postaciach, które były przynajmniej potencjalnymi gwiazdami mediów, czy też na podobnej do komórek, militarnej strukturze zdolnej do przeprowadzania ataków w~stylu partyzanckim, impuls był kierowany w~odmienną stronę. Co więcej, obie strategie wybuchały przez kilka lat i~bardzo szybko wygasały (choć alternatywne instytucje powstałe w~tym czasie często trwały znacznie dłużej).

 Kontrastowanie poważnego aktywizmu Nowej Lewicy z~początku lat sześćdziesiątych z~rzekomym dziecięcym ekstremizmem końca lat sześćdziesiątych i~wczesnych siedemdziesiątych stało się konwencjonalnym zwyczajem w~liberalnej nauce. Nie chcę pozostawiać wrażenia, że się z~tym zgadzam. Standardowym zarzutem liberałów jest to, że kontrkultura lat sześćdziesiątych -- w~istocie pierwszą masową, przemysłową bohemę -- sama się zniszczyła w~ultraradykalizmie. Co więcej, w~ten sposób, jak głosi argument, pozostawiło to prawicowym działaczom możliwość przyjęcia wielu z~tych samych oddolnych technik organizacyjnych opracowanych przez SDS, aby dotrzeć do bardzo białych okręgów klasy robotniczej, które SDS miała taki problem zorganizować, i~zmobilizować ich przeciwko tej właśnie kontrkulturze. Z pewnością jest w~tym ironia. Ale wydaje mi się, że lepiej widzieć oba okresy jako próby przepracowania pewnych fundamentalnych dylematów, które są z~nami do dziś. Osobiście podejrzewam, że prawdziwym winowajcą wzrostu i~ostatecznej hegemonii Nowej Prawicy nie są ekscesy maoistów i~Yippies, ale raczej fakt, że Ameryka przestała wykorzystywać szkolnictwo wyższe jako środek mobilności klasowej. Ponieważ większość sfrustrowanych klas administracyjnych według Mattick'a została ponownie wchłonięta przez nowy, bardziej elastyczny kapitalizm, biała klasa robotnicza była coraz bardziej odcięta od jakiegokolwiek dostępu do środków produkcji kulturalnej -- poza, być może, ich kościołem. Rezultatem była być może przewidywalna niechęć do rzekomych kontrkulturowych ekscesów ,,liberalnej elity''\footnote{W rzeczywistości te okręgi wyborcze, które najpewniej nadal głosują na demokratów, to właśnie te, które mają nadzieję na mobilność poprzez edukację: imigranci, Afroamerykanie, a nawet kobiety, które obecnie uczęszczają na uniwersytety znacznie częściej niż mężczyźni. Z pewnością nie ma paraleli w~społecznościach kolorowych do wyraźnego antyintelektualizmu tak dużej części radykalnej prawicy.}.

 Jak to bywa, drugi okres był znacznie bardziej złożony i~twórczy niż krytycy zwykle przyznają. Wiele pomysłów, które z~niego wynikły, było niezwykle proroczych. Rozważmy na przykład pojęcie ,,międzywspólnotowości'' Hueya Newtona, które stało się oficjalnym stanowiskiem Czarnej Pantery w~1971 roku i~które głosiło, że państwo narodowe jest w~trakcie rozpadu, i~że każda skuteczna rewolucyjna polityka musiałaby zacząć się od sojuszu między lokalnymi, samoorganizującymi się społecznościami, niezależnie od granic państwowych. Prawdziwy problem polegał na tym, \textit{jak }były one samozorganizowane: Czarne Pantery, które uosabiały postaci takie jak sam Newton, w~końcu ucieleśniały epokę, w~której macho, szowinistyczne style przywództwa zaczęły wydawać się synonimem bojowości.

 To prawdopodobnie znaczące, że w~SNCC pierwszy krok w~kierunku odrzucenia zdecentralizowanego podejmowania decyzji został zainicjowany przez wschodzącą frakcję Black Power. Dokładna analiza historii organizacyjnej ruchu przeprowadzona przez Poletta (2002) wyraźnie pokazuje, że konsensus i~decentralizacja nie były kwestionowane, ponieważ były w~rzeczywistości nieefektywne. Były raczej używane jako kwestie klinowe. Przez obsesję na punkcie procesu demokratycznego biali aktywiści w~SNCC i~ich sojusznicy mogli być utożsamiani z~niekończącymi się rozmowami i~zamieszaniem; im bardziej wojownicza, frakcja Black Power mogłaby przedstawiać się jako idealny model bezwzględnej skuteczności, odpowiedni dla prawdziwie bojowej organizacji. Nie bez znaczenia jest też fakt, że Stokely Carmichael, która stała się głównym rzecznikiem stanowiska Black Power, lubiła powtarzać rzeczy jak ,,jedyna pozycja dla kobiety w~SNCC to pozycja leżąca''.

 Fakt, że nawet w~połowie lat 60. takie rzeczy można było mówić w~organizacji, która została pierwotnie założona przez kobietę jako bunt przeciwko charyzmatycznemu męskiemu autorytetowi, jest sam w~sobie zdumiewający. Ale może to dać poczucie, że polityka seksualna zawsze leży tuż pod powierzchnią starej Nowej Lewicy. Bojowe ruchy nacjonalistyczne są oczywiście znane z~zapewniania platform dla energicznego przywrócenia pewnych rodzajów męskiej władzy. Ale sentymenty podobne do Carmichael można znaleźć również w~ustach białych aktywistów tamtych czasów. W rzeczywistości ruch feministyczny powstał w~dużej mierze w~obrębie Nowej Lewicy, jako reakcja na właśnie ten rodzaj przywództwa macho, lub po prostu wśród tych zmęczonych odkrywaniem, że nawet podczas zajęć uniwersyteckich, nadal miały przygotowywać kanapki i~zapewniać bezpłatne usługi seksualne, podczas gdy aktywiści płci męskiej pozowali przed kamerami. Z kolei ożywienie zainteresowania tworzeniem praktycznych form demokracji bezpośredniej -- w~istocie rzeczywista geneza obecnego ruchu -- wywodzi się nie tyle z~tych męskich radykałów z~lat 60., ile z~ruchu kobiecego, który powstał w~dużej mierze w~reakcji na nich (na przykład, Freeman 1971, Evans 1979).

 Kiedy zaczynał się ruch feministyczny, był organizacyjnie bardzo prosty. Jego podstawowymi jednostkami były małe kręgi podnoszące świadomość; podejście było nieformalne, intymne i~antyideologiczne. Większość pierwszych grup wyłoniła się bezpośrednio z~kręgów Nowej Lewicy. O ile stawiały się w~stosunku do poprzedniej radykalnej tradycji, był to zwykle anarchizm. Podczas gdy nieformalna organizacja okazała się wyjątkowo dobrze przystosowana do podnoszenia świadomości, gdy grupy zaczęły planować działania, a zwłaszcza gdy się rozrastały, problemy miały tendencję do ujawniania. Niemal zawsze takie grupy były zdominowane przez ,,wewnętrzny krąg'' kobiet, które były lub stały się bliskimi przyjaciółkami. Charakter wewnętrznego kręgu był różny, ale jakoś zawsze się pojawiał. W rezultacie w~niektórych grupach lesbijki czuły się wykluczone, w~innych to samo działo się z~heterokobietami. Inne grupy szybko rosły, a potem większość nowoprzybyłych szybko odpadała, ponieważ nie było możliwości ich zintegrowania. Nastąpiły niekończące się debaty. Jednym z~rezultatów był esej zatytułowany ,,Tyrania bezstruktury'', napisany przez Mary Jo Freeman w~1970 roku i~opublikowany po raz pierwszy w~1972 roku -- tekst, który do dziś chętnie czytają różnego rodzaju organizatorzy. Argument Freeman jest dość prosty. Bez względu na to, jak szczere jest oddanie zasadom egalitarnym, faktem jest, że w~każdej grupie aktywistów różni członkowie będą mieli różne umiejętności, zdolności, doświadczenie, cechy osobiste i~poziomy oddania. W rezultacie nieuchronnie rozwinie się jakaś struktura elit lub przywództwa. Na wiele sposobów, mając nieuznaną strukturę przywódczą, argumentowała, może być znacznie bardziej niszczące niż posiadanie struktur formalnych: przynajmniej przy formalnej strukturze jest możliwe dokładne ustalenie, czego oczekiwać od tych, którzy wykonują najważniejsze koordynujące zadania i~ich z~tego rozliczyć.

 Jednym z~powodów niesłabnącej popularności eseju jest to, że można go wykorzystać do wsparcia szerokiej gamy stanowisk. Liberałowie i~socjaliści regularnie powołują się na ,,Tyranię braku struktury'' jako uzasadnienie, dlaczego każda organizacja anarchistyczna jest skazana na porażkę, jako kartę powrotu do starszych, odgórnych stylów organizacji, pełnych urzędów wykonawczych, komitetów sterujących, i~tym podobne. Egalitaryści sprzeciwiają się, że nawet w~takim stopniu, w~jakim jest to prawdą, o wiele gorzej jest mieć przywództwo, które czuje się w~pełni uprawnione do swojej władzy, niż takie, które musi poważnie traktować oskarżenia o hipokryzję. Dlatego anarchiści zwykle odczytywali argumenty Freeman jako wezwanie do sformalizowania procesu grupowego w~celu zapewnienia większej równości, a tak naprawdę większość jej konkretnych sugestii -- wyjaśniania, jakie zadania są przypisane jakim jednostkom, szukania sposobów do oceny wyników jednostki przez grupę, rozdzielania odpowiedzialności tak szeroko jak tylko się da (może przez rotowanie), zapewnianie wszystkim dostępu do informacji i~zasobów -- z~pewnością były pomyślane w~tym celu.

 W obrębie samego większego ruchu feministycznego większość z~tych argumentów ostatecznie została przedyskutowana, ponieważ moment anarchistyczny był krótki. Zwłaszcza po tym, jak w~sprawie \textit{Roe przeciwko Wade }wydawało się, że strategicznie mądre wydaje się poleganie na władzy rządu, ruch kobiecy zamierzał wystartować w~zdecydowanie liberalnym kierunku i~coraz bardziej polegać na formach organizacyjnych, które nie były egalitarne. Ale dla tych, którzy nadal pracują w~egalitarnych kolektywach lub próbują je stworzyć, feminizm skutecznie sformułował warunki debaty. Jeśli chcesz ograniczyć podejmowanie decyzji do możliwie najmniejszych grup, jak te grupy koordynują? W ramach tych grup, jak zapobiec przejęciu władzy przez klikę przyjaciół? Jak zapobiegać marginalizacji pewnych kategorii uczestników (heteroseksualnych kobiet, gejów, starszych kobiet, studentów -- w~mieszanych grupach wkrótce po prostu kobiet)? Co więcej, nawet jeśli mainstreamowe feministki porzuciły politykę akcji bezpośredniej, wokół było mnóstwo radykalnych feministek, nie mówiąc już o anarchafeministkach, które starały się, aby takie grupy były uczciwe.

 Początki obecnego ruchu akcji bezpośredniej sięgają właśnie do prób rozwiązania tych dylematów. Elementy naprawdę zaczęły się łączyć w~ruchu antynuklearnym późnych lat 70., najpierw wraz z~założeniem Clamshell Alliance i~okupacją elektrowni jądrowej Shoreham w~Massachusetts w~1977 roku, a następnie przez Abalone Alliance i~walkę o fabrykę Diablo Canyon w~Kalifornii kilka lat później. Główną inspiracją dla działaczy antynuklearnych -- przynajmniej w~kwestiach organizacyjnych -- były idee głoszone przez grupę o nazwie Movement for a New Society (MNS -- Ruch na rzecz Nowego Społeczeństwa ) z~siedzibą w~Filadelfii. Na czele MNS stał działacz na rzecz praw gejów George Lakey, który -- podobnie jak kilku innych członków grupy -- był również anarchistycznym kwakrem. Lakey i~jego przyjaciele zaproponowali wizję rewolucji bez przemocy. Zamiast kataklizmu przejęcia władzy, proponowali ciągłe tworzenie i~rozwijanie nowych instytucji, opartych na nowych, niezbywalnych sposobach interakcji -- instytucji, które można by uznać za ,,prefiguratywne'', o ile dawały przedsmak tego, czym naprawdę demokratyczne społeczeństwo może być. Takie prefiguratywne instytucje mogłyby stopniowo zastępować istniejący porządek społeczny (Lakey 1973). Sama wizja nie była nowa. Była to pozbawiona przemocy wersja standardowej anarchistycznej idei budowania nowego społeczeństwa w~skorupie starego. Nowością było to, że ludzie tacy jak Lakey, którzy wychowali się jako kwakrzy i~zdobyli duże doświadczenie w~kwakierskim podejmowaniu decyzji, mieli praktyczną wizję tego, jak niektóre z~tych alternatyw mogą faktycznie działać. Wiele z~tego, co teraz jest standardowymi cechami formalnego procesu konsensus -- zasada, że facylitator nigdy nie powinien działać jako zainteresowana strona, idea ,,bloku'' -- były upowszechniane na treningach MNS w~Filadelfii i~Bostonie.

 Ruch antynuklearny był również pierwszym, który uczynił ze swojej podstawowej jednostki organizacyjnej grupę afinicji -- rodzaj minimalnej jednostki organizacyjnej opracowanej po raz pierwszy przez anarchistów w~Hiszpanii i~Ameryce Łacińskiej na początku XX wieku -- oraz rady delegatów. Jak zauważyła Starhawk w~Rozdziale 1, wszystko to było w~dużej mierze procesem uczenia się, rodzajem ślepego eksperymentu, a sprawy często były niezwykle trudne. Początkowo organizatorzy byli tak konsensusowymi purystami, że upierali się, że każda osoba ma prawo do blokowania propozycji nawet na poziomie ogólnokrajowym, co okazało się całkowicie nierealne. Mimo to akcja bezpośrednia poruszenia kwestii energetyki jądrowej okazała się spektakularnym sukcesem. Jeśli już, to ruch padł ofiarą własnego sukcesu. Chociaż rzadko wygrywał bitwę -- to znaczy o blokadę uniemożliwiającą budowę jakiejkolwiek konkretnej nowej fabryki -- bardzo szybko wygrał wojnę. Plany rządu USA dotyczące budowy stu nowych generatorów zostały pokrzyżowane po kilku latach i~od tego czasu nie ogłoszono żadnych nowych planów budowy elektrowni jądrowych. Próby przejścia od elektrowni jądrowych do pocisków jądrowych, a stamtąd do rewolucji społecznej, okazały się jednak większym wyzwaniem, a sam ruch nigdy nie był w~stanie przeskoczyć od kwestii jądrowej i~stać się podstawą szerszej kampanii rewolucyjnej. Po wczesnych latach 80. w~dużej mierze zniknął.

 Nie znaczy to, że pod koniec lat 80. i~90. nic się nie działo. Radykalni aktywiści AIDS współpracujący z~ACT UP i~radykalni ekolodzy z~grupami takimi jak Earth First!utrzymywali te techniki i~rozwijali je. W latach 90. podjęto próbę stworzenia północnoamerykańskiej federacji anarchistycznej wokół gazety o nazwie \textit{Love \& Rage}, która w~szczytowym momencie angażowała setki aktywistów w~różnych miastach. Mimo to prawdopodobnie trafne jest postrzeganie tego okresu nie jako ery wielkich mobilizacji, a raczej jako ery molekularnego rozpowszechniania. Typowym przykładem jest historia Food Not Bombs, grupy pierwotnie założonej przez kilku przyjaciół z~Bostonu, którzy byli częścią grupy afinicji dostarczającej jedzenie podczas akcji w~Shoreham. Na początku lat osiemdziesiątych weterani grup afinicji założyli sklep w~skłotowanym domu w~Bostonie i~zaczęli nurkować na śmietnikach ze świeżymi produktami wyrzucanymi przez supermarkety i~restauracje oraz przygotowywać bezpłatne wegetariańskie posiłki do dystrybucji w~miejscach publicznych. Po kilku latach jeden z~założycieli przeniósł się do San Francisco i~tam rozpoczął podobną działalność. Wieści rozeszły się (po części z~powodu dramatycznych aresztowań w~telewizji) i~w połowie lat 90. autonomiczne oddziały FNB zaczęły pojawiać się w~całej Ameryce, a także w~Kanadzie. Na przełomie tysiącleci było ich dosłownie setki. Ale Food Not Bombs nie jest organizacją. Nie ma nadrzędnej struktury, członkostwa ani corocznych spotkań. To tylko pomysł -- żeby jedzenie trafiało do tych, którzy tego potrzebują, i~w taki sposób, że ci nakarmieni mogą sami stać się częścią procesu, jeśli chcą -- plus kilka podstawowych informacji instruktażowych (teraz łatwo dostępnych w~Internecie) oraz wspólne zaangażowanie w~egalitarne podejmowanie decyzji i~ducha zrób to sam. Stopniowo na całym kontynencie zaczęły powstawać spółdzielnie, anarchistyczne sklepy informacyjne, grupy obrony klinik, anarchistyczne kolektywy więzienne Czarnego Krzyża, pirackie kolektywy radiowe, squaty i~oddziały Akcji Antyrasistowskiej. Wszystko stawało się warsztatami tworzenia demokracji bezpośredniej. Ale, zwłaszcza, że  tak wiele z~tego rozwijało się nie na kampusach, ale w~środowiskach kontrkulturowych, takich jak scena punkowa, pozostawała znacznie poniżej radaru nie tylko korporacyjnych mediów, ale nawet standardowych postępowych czasopism, takich jak \textit{Mother Jones} czy \textit{The Nation}. To, znowu, wyjaśnia jak, kiedy te grupy zaczęły się jednoczyć i~koordynować w~Seattle, wydawało się, dla reszty kraju, jakby ruch nagle pojawił się znikąd.

 Jednak zanim znajdziemy się w~Seattle, nie można nawet udawać, że takie sprawy mogą być omawiane w~ramach ogólnokrajowych. To, co prasa upiera się przy nazywaniu ,,ruchem antyglobalistycznym'', od samego początku było świadomym ruchem globalnym. Działania przeciwko WTO w~Seattle zostały po raz pierwszy zaproponowane przez PGA, planetarną sieć, która powstała z~inicjatywy rebeliantów zapatystowskich w~Chiapas. Nacisk na WTO odzwierciedlał obawy grup rolników w~Indiach, a stosowana taktyka mogła być równie dobrze postrzegana jako amalgamat idei zaczerpniętych głównie z~Globalnego Południa, niż jako rozwój technik rdzennych Amerykanów. Umożliwił to przede wszystkim Internet. Internet umożliwił choćby skok jakościowy w~zakresie i~szybkości rozpowszechniania molekuł: są teraz oddziały Food Not Bombs, na przykład w~Caracas i~Bandung. Rok lub dwa bezpośrednio po Seattle pojawiły się także sieci Independent Media Centers, radykalnego dziennikarstwa internetowego, które całkowicie przekształciły możliwości przepływu informacji o akcjach i~wydarzeniach. Aktywiści, którzy przez miesiące i~lata walczyli o podjęcie działań, które wówczas były całkowicie ignorowane przez media, teraz wiedzą, że wszystko, co zrobią, zostanie natychmiast wychwycone i~ogłoszone w~zdjęciach, historiach i~filmach na całej planecie -- choćby w~formie, z~której korzystają głównie inni aktywiści. Wielkim problemem było przełożenie przepływu informacji na struktury kolektywnego podejmowania decyzji, ponieważ podejmowanie decyzji jest jedyną rzeczą, której prawie nie da się zrobić w~Internecie. A dokładniej pytanie brzmi: kiedy i~na jakim poziomie potrzebne są struktury zbiorowego podejmowania decyzji? Sieć Działań Bezpośrednich i~struktura Continental DAN, które zaczęły być tworzone w~miesiącach po Seattle, były pierwszymi próbami rozwiązania tego problemu. Ostatecznie upadły. Czyniąc to, odegrały jednak również kluczową rolę w~rozpowszechnianiu pewnych modeli demokracji bezpośredniej i~uczyniły ich praktykę nieodłączną od idei akcji bezpośredniej. To właśnie połączenie tych dwóch zjawisk, obecnie prawie nieodwracalnie utrwalone w~najbardziej radykalnych ruchach społecznych w~Ameryce i, coraz częściej, gdzie indziej, jest prawdziwym tematem tej książki. 

\chapter{Kilka uwag dotyczących ,,Kultury Aktywistów''}

 Zacząłem tę książkę od pierwszej wycieczki CLAC, która przeszła przez Nowy Jork na początku 2000 roku. Pozwólcie, że przesunę się do przodu około roku i~opowiem o drugiej trasie CLAC w~tym celu: jednej, która odbyła się przed ich akcją ,,Take the Capital'' w~Ottawie podczas 2002 Spotkania G8 w~Kananaskis.

 Publiczność takich wycieczek składała się zwykle głównie z~białych anarchistów, ale tym razem ludzie z~CLAC postanowili sprowadzić przynajmniej jednego mówcę z~grupy opartej na lokalnej społeczności z~każdego miasta, przez które przejeżdżali. W Nowym Jorku okazało się, że jest to organizator o imieniu Ranjanit z~radykalnej południowoazjatyckiej grupy Desis Rising Up and Moving (DRUM). W tamtym czasie DRUM zyskał ogromny szacunek w~kręgach nowojorskich aktywistów za swoją pracę nad kwestiami przetrzymywania imigrantów specjalnego zainteresowana bezpośrednio po 11 września, kiedy setki ludzi pochodzenia bliskowschodniego lub południowoazjatyckiego zostało złapanych i~efektywnie znikniętych.

 Prelegenci z~Kanady opisali kampanie, w~które byli zaangażowani, i~mówili o organizowaniu różnego rodzaju dylematów. Przemówienie Ranjanita było inne. Polegało ono głównie na potępieniu ,,kultury aktywistów''. On sam, podkreślał, był nie tylko indyjskiego pochodzenia, ale też robotniczym dzieciakiem z~Queens. Wiedział coś o społecznościach, z~którymi pracował. Od Seattle wszyscy anarchiści mówią o tym, jak odejść od ,,jeżdżenia na Szczyty'' do ściślejszej współpracy ze społecznościami w~walce. Problem, jak podkreślił, polegał na tym, że wypracowali własne style ubierania się, maniery, sposoby mówienia, gusta muzyczne i~kulinarne -- rodzaj hybrydowego miszmaszu hippisów, punków i~mainstreamowej białej kultury klasy średniej, z~włączonymi kawałkami bardziej egzotycznych tradycji rewolucyjnych -- a to prawie uniemożliwiało im komunikowanie się z~kimkolwiek spoza ich własnego, zaczarowanego kręgu. Niektóre elementy tej kultury aktywistów -- na przykład odrzucenie standardów higieny osobistej -- były uważane za wręcz obraźliwe przez większość tych, z~którymi chcieli zawrzeć sojusze. Inne, jak diety wegańskie, uniemożliwiały zasiadanie przy stole prawie z~każdym, kto nie był aktywistą. Kultura aktywistyczna dusiła obietnicę ruchu, a anarchiści musieli zdecydować, co naprawdę chcą zrobić: stworzyć własną (małą, stosunkowo uprzywilejowaną) społeczność, pokazywać się na spotkaniach IMF i~składać wielkie deklaracje na temat zła globalnego kapitalizmu, czy poważnie popracować z~prawdziwymi społecznościami, które rzeczywiście przyjmowały impet kapitalistycznej globalizacji.

 Nie można być anarchistą w~dużym mieście w~Ameryce bez dość regularnego słuchania jakiejś wersji tej krytyki. Po części dlatego, że jest to krytyka, którą musi się pojawić. Podobnie jak aktywiści SDS opisani w~poprzednim rozdziale, niewielu białych uczestników ruchu akcji bezpośredniej uważa się za pochodzących z~,,kultur'', większość postrzega siebie po prostu jako ogólnych (,,nieprzypisanych'') Amerykanów, których problemy i~obawy są traktowane jako uniwersalne, nawet jeśli jednocześnie czują, że jest coś w~tym ogólnym amerykańskim stylu życia, który jest głęboko nieludzki, niezrównoważony i~zły. Jako anarchiści i~rewolucjoniści stają zatem przed tym samym dylematem: czy próbować stworzyć własną alternatywną kulturę, czy skoncentrować się na pracy sojuszniczej, wsparcie walk tych, którzy najbardziej cierpią w~ramach istniejącego systemu, ale którzy są również gotowi współpracować z~nimi jako sojusznikami. Krótko mówiąc: muszą wybierać między skupieniem się na własnej alienacji, czy na ucisku innych.

 Z pewnością w~rzeczywistości prawie wszyscy robią po trochu jedno i~drugie. Ale to właśnie prowadzi do sprzeczności, na które wskazywał Ranjanit. Im bardziej tworzy się własną, alternatywną kulturę, tym dziwniej i~bardziej dziwacznie wydaje się osobom z~zewnątrz, także tym, z~którymi pozornie chce się sprzymierzyć. Wielu osób kolorowych postrzega samą kulturę anarchistyczną jako oznakę białego przywileju wymachującego im przed twarzą (jak zauważył pewien afroamerykański anarchista w~odniesieniu do punkowych stylów ubioru i~zachowania: ,,Gdybym wyszedł na ulicę z~takim wyglądem, za piętnaście minut zostanie zaciągnięty do posterunku gliniarzy''). Z drugiej strony nierozsądne wydaje się proszenie anarchistów o porzucenie wszelkich prób budowania kultury alternatywnej, o powrót do znienawidzonego przez nich stylu życia, tylko aby nie zrazić innych.

 Ale czy naprawdę można być \textit{przeciwko }kulturze?

 To jest pytanie, które chcę zbadać w~tym rozdziale. ,,Kultura'' to termin kojarzący się dziś z~tak uniwersalnie pozytywnymi skojarzeniami, że już trochę dziwnie słyszeć, że fakt, że niektórzy ludzie mają kulturę, jest traktowany jako problem. Tym bardziej że kultura, o której mowa, rodzi się ze świadomego wysiłku stworzenia mniej hierarchicznej, mniej wyalienowanej i~bardziej demokratycznej i~ekologicznie zrównoważonej formy życia -- by stworzyć rodzaj kultury, który mógłby pasować do prawdziwie wolnego społeczeństwa. Wydaje mi się, że rozwikłanie tego paradoksu doprowadzi nas do sedna fundamentalnych dylematów projektu anarchistycznego.

\section{Dylematy białego przywileju}

 Najczęściej kultura aktywistów jest postrzegana jako problematyczna -- tak jak to było w~przypadku Ranjanita -- ponieważ jest postrzegana jako forma przywilejów białych, a spory dotyczące kultury aktywistów są formułowane w~kategoriach rasy. Podziały rasowe w~Ameryce były oczywiście plagą radykalnej polityki w~Stanach Zjednoczonych od wieków. Historycznie sprawiały, że utrzymanie trwających sojuszy klasowych było niezwykle trudne. Takie dyskusje regularnie rozrywają grupy akcji bezpośrednich.

 Pozwólcie, że rozważę jeden szczególnie dobrze udokumentowany przykład. W latach 90. Love \& Rage Federation (Federacja Miłości i~Wściekłości) (Filipo 1993) rozwiązała się z~powodu kwestii przywileju białych. Love \& Rage powstało jako inicjatywa stworzenia kontynentalnej sieci anarchistycznej wokół gazety o tej samej nazwie. Pod wieloma względami była całkiem udane. Jednak po dziesięciu latach uparcie stwierdzili, że nie są w~stanie wykroczyć poza swój pierwotny rdzeń białych aktywistów z~klasy średniej ani włączyć do niej znacznej liczby osób kolorowych\footnote{Zdobyli meksykański oddział, Amor y Rabia. Ale jego członkowie pochodzili również w~dużej mierze z~klasy średniej.}. Ostatecznie wybuchły wściekłe spory o przyczyny tego stanu rzeczy: które stały się również teoretycznymi debatami na temat natury przywilejów białych i~sposobów przezwyciężenia białej supremacji.

 Niektórzy twierdzili, że problem ma charakter kulturowy. Zdecydowana większość białych anarchistów po raz pierwszy odkryła anarchizm poprzez punk rock i~jego kulturę DIY. Wejdź do typowego anarchistycznego infoshopu, jak zauważyli, i~prawie nieuchronnie zostaniesz powitany przez ludzi z~zielonymi włosami i~kolczykami na twarzy. Nie ma znaczenia, jak bardzo byli gościnni: sam ich wygląd wyraźnie ograniczał atrakcyjność takich miejsc dla członków białej klasy robotniczej, nie mówiąc już o biednych, kolorowych. Inni twierdzili, że problem leży znacznie głębiej. Argumentowali, że Stany Zjednoczone są narodem zbudowanym na białej supremacji, a bycie białym nie jest kulturą. Kiedy biali mówią o swoim dziedzictwie kulturowym, mówią o byciu Niemcem, Irlandczykiem lub Litwinem, ale nigdy o byciu białym. To dlatego, że białość jest kategorią przywileju, milczącym porozumieniem z~innymi osobami sklasyfikowanymi jako ,,białe'' -- ze stowarzyszeń udzielających pożyczek mieszkaniowych lub nadinspektorów policji -- w~celu zapewnienia pomocy i~ochrony, która nie jest zapewniana osobom niesklasyfikowanym w~ten sposób. Jedynym sposobem na zniszczenie systemu przywilejów jest obalenie kategorii białości, aby ostatecznie ją zniszczyć.

 To stanowisko rozwijało się w~kręgach związanych z~czasopismem \textit{Race Traitor}, które było wydane w~tym czasie i~zachłannie czytane w~kręgach aktywistów. Jego motto brzmiało: ,,Zdrada białych to lojalność wobec ludzkości'' To był bardzo pociągający pomysł, ale wtedy pojawiło się oczywiste pytanie: jak właściwie to zrobić? Jak zostać skutecznym zdrajcą rasy? Kto może być przykładem skutecznego wzoru do naśladowania? Wielu in Love \& Rage znalazło inspirację na przykładzie Subcomandante Marcos, słynnego zamaskowanego rzecznika meksykańskich zapatystów. Marcos był pierwotnie Meksykaninem z~klasy średniej, który przewodził grupie w~większości uprzywilejowanych miejskich rewolucjonistów w~organizowaniu rdzennych społeczności w~Chiapas, a po dziesięciu latach spędzonych w~dżungli, porzucił swoją awangardową ideologię, by stać się przedstawicielem realizującym decyzje podejmowane przez społeczności tubylcze. W swojej gotowości do wycofania się i~zaakceptowania przywództwa uciskanych społeczności można go uznać za przykład prawdziwego zdrajcy rasy. Ale Marcos ze swojej strony miał tę zaletę, że mógł sprzymierzyć się ze społecznościami tubylczymi, które już zachowywały się bardzo jak anarchiści, z~własnym stylem demokracji bezpośredniej opartej na konsensusie. Co to oznaczało dla anarchistów w~Stanach Zjednoczonych, gdzie większość rewolucyjnych grup opartych na społecznościach kolorowych była o wiele bardziej hierarchicznie zorganizowana, gdzie w~rzeczywistości wielu postrzegało podkreślanie demokracji bezpośredniej jako formę przywileju białych? Czy to wszystko oznaczałoby konieczność porzucenia jakiejkolwiek idei budowania nowego społeczeństwa w~skorupie starego? Lub przynajmniej, białych anarchistów odgrywających w~tym procesie znaczącą rolę? W ciągu roku lub dwóch Love \& Rage podzieliło się na zwaśnione frakcje o kwestie rasowe, a cały projekt ostatecznie upadł.

 Podobne debaty wybuchły we wczesnych dniach ruchu globalizacyjnego. W tym przypadku inauguracją był utwór zatytułowany ,,Where Was the Color in Seattle?'' (Martinez 2000), który wywołał nieustanne spory o naturę rasowych przywilejów, modeli pomocy w~stosunku do sojuszu, o to, jak zaakceptować przywództwo kolorowych społeczności i~o tłumiące skutki białej winy. Przytłaczająco biały pudrowanie rodzącego się ruchu uznano za ciągły kryzys. Z pewnością dotyczyło to New York City Direct Action Network, pierwotnie założonej w~celu pomocy w~koordynowaniu działań przeciwko IMF i~Bankowi Światowemu w~Waszyngtonie 16 kwietnia 2000 roku. Drugą ważną inicjatywą DAN była pomoc w~organizacji działań przeciwko Konwencji Republikańskiej w~Filadelfii tego lata. W tym celu grupa organizatorów DAN zaproponowała sojusz ze SLAM\footnote{SLAM - Ruch aktywistów na rzecz wyzwolenia studentów.}, radykalną grupą studencką z~Hunter College o znacznie bardziej zróżnicowanym członkostwie oraz kilka innych organizacji opartych na POC. W tych dniach tuż po Seattle wszyscy byli chętni do nauki taktyk i~form organizacji DAN, więc ci ostatni nie byli niechętni; ale nalegali także, aby same działania skupiały się na sprawie Mumii Abu-Jamala (czarnego aktywisty i~dziennikarza przebywającego wówczas w~celi śmierci w~Filadelfii) i~szerzej na amerykańskim więziennym kompleksie przemysłowym i~rasistowskim charakterze systemu sądownictwa karnego. Żądania te wyizolowały znaczącą frakcję w~DAN, która postrzegała protesty na konwencji jako szansę na przejście od kwestii handlu światowego do szerszego wyzwania dla istniejącego systemu politycznego jako całości; zestawienie własnego modelu demokracji bezpośredniej z~rodzajem zdominowanej przez korporacje demokracji przedstawicielskiej, którą ucieleśniają konwencje. Niektórzy uważali, że można je pogodzić: że kwestie kary więzienia i~kary śmierci można ostatecznie wykorzystać do postawienia tych samych szerszych kwestii. Inni uważali, że kompromis był wart możliwości stworzenia trwającego sojuszu. Ostatecznie wysiłek nie doprowadził w~rzeczywistości do trwającego sojuszu, a wynikające z~tego wzajemne oskarżenia spowodowały, że całkiem sporo aktywistów całkowicie zrezygnowało z~DAN. Jednakże sojusz, jakkolwiek tymczasowy, był całkiem pomocny w~rozpowszechnianiu taktyk i~stylów podejmowania decyzji podobnych do DAN, a nawet samych anarchistycznych idei, w~szerszych kręgach aktywistów. Krótko po tym, jak NYC DAN skutecznie rozwiązało się w~2003 roku, nowa sieć ,,Anarchist People of Color'' (APOC -- Anarchistyczni Ludzie Koloru) była w~trakcie nabierania kształtu, oparta na prawie identycznych zasadach organizacji.

 Jednak wczesne doświadczenia APOC już teraz stanowią doskonałą ilustrację tego, dlaczego grupy zorientowane na akcje bezpośrednie były zwykle zdominowane przez osoby sklasyfikowane jako ,,białe''. Kiedy ci, którzy nie mają białych przywilejów, zaczęli przyjmować taką politykę, odkryli, że spotykają się z~zupełnie innymi poziomami policyjnych represji. Jak ujawnił jeden szczególnie zaskakujący incydent na Brooklynie, APOC nie mogło nawet zorganizować przyjęcia dobroczynnego we własnych biurach, nie martwiąc się, że lokalna policja wpadnie, by pobić i~aresztować uczestników imprezy rozmawiających na ulicy.

 Wszystko to było być może przewidywalne. Wiadomo, że podczas akcji na dużą skalę policja wydaje się atakować osoby kolorowe za szczególną przemoc. W rezultacie wiele (nie-anarchistycznych) grup aktywistów POC postrzega akcję bezpośrednią jako formę przywileju rasowego i~stara się trzymać tych, którzy prawdopodobnie angażują się w~wojownicze taktyki z~dala od ich wydarzeń. Krótkotrwały DAN w~Los Angeles, który zorganizował protesty przeciwko konwencji demokratów w~2000 roku, potraktował potrzebę sprzymierzenia się z~grupami społecznymi tak poważnie, że w~ogóle odmówił pozwolenia na wykorzystanie ich przestrzeni na spotkania anarchistów, a nawet zatrudnił marszałków do wykluczenia anarchistów Czarnego Bloku z~ich marszów.

 Nowojorski DAN był zupełnie inny. Pod każdym względem sam był grupą anarchistyczną. Mimo to szybko znalazł się w~tarapatach za odmowę pójścia tą samą ścieżką co LA DAN. Zaraz po A16, na przykład, NYC DAN i~sojusznicza grupa -- New York Reclaim the Streets -- połączyły się z~kilkoma meksykańskimi grupami imigrantów, aby zorganizować marsz pierwszomajowy przez dolny Manhattan. Miało być całkowicie spokojnie -- w~istocie legalne -- wydarzenie, pełne zespołów muzycznych i~gigantycznych marionetek. Mimo to, gdy maszerujący po raz pierwszy zebrali się na Union Square, pojawiła się malutka grupa, być może szesnastu anarchistów z~Czarnego Bloku, którzy po prostu zamierzali pokazać flagę, jakby to było, i~ustanowić jawnie anarchistyczną obecność na tym wydarzeniu. Jeszcze przed rozpoczęciem marszu policja wkroczyła i~aresztowała kilkunastu z~nich\footnote{Przetrzymywano je na podstawie niejasnego ,,prawa dotyczącego masek'' z~początku XIX wieku, które pierwotnie uchwalono w~celu powstrzymania irlandzkich rozbójników, które czyniło nielegalnym ukrywanie twarzy przez członków grupy ponad trzech osób zgromadzonych publicznie. Blok faktycznie został o tym ostrzeżony, ale fałszywie poinformowano go, że jeśli na ich maskach są napisane hasła, nie można ich pociągać do odpowiedzialności.}. Meksykańscy organizatorzy byli oburzeni, ale mniej na policję niż na ich kolegów organizatorów DAN, oskarżając ich o narażanie swoich ludzi -- wielu z~nich nieudokumentowanych pracowników -- na ryzyko, pozwalając na początek na gromadzenie się Czarnego Bloku. Przysięgli, że nigdy więcej nie będą pracować z~DAN.

 Jest całkiem oczywiste, że kiedy policja przeprowadza takie uderzenia prewencyjne, podżeganie tego rodzaju podziałów to połowa sukcesu. Nowojorska policja okazała się naprawdę biegła w~graniu w~tego rodzaju grach i~faktycznie miała zwyczaj, podczas szczególnie wrażliwych marszów organizowanych przez grupy POC, łapać jednego lub dwóch białych anarchistów pod sfabrykowanymi zarzutami. Na przykład rok po marszu pierwszomajowym, podczas marszu apelującego o ułaskawienie dla indiańskiego aktywisty Leonarda Peltiera w~grudniu 2000 roku, oddział przechwytujący nowojorskiej policji nagle wdarł się w~sam środek marszu, by rozprawić się z~czterema (niezamaskowanymi) anarchistami. Jeden został oskarżony o posiadanie megafonu na baterie bez pozwolenia na emisję, pozostali o ,,przeciwstawianie się aresztowaniu''. To była bardzo delikatna sprawa, i~wszyscy usilnie starali się uniknąć wszystkiego, co można by interpretować jako prowokację: żaden z~anarchistów nie nosił masek, kobieta z~megafonem w~rzeczywistości go nie używała, ale po prostu przenosiła go z~jednego dozwolonego punktu zbiórki do drugiego (i w~każdym razie, jak wielu wskazywało, nie ma czegoś takiego jak pozwolenie na emisję dźwięku). Jednak fakt, że wszyscy wiedzieli, że aresztowania były pretekstem i~świadomie miały na celu siać niezgodę, nie miał większego znaczenia. Później wielu aktywistów, którzy oparli swoją strategię na budowaniu sojuszy z~grupami POC (w tym przypadku kilku byłych członków Love \& Rage, którzy teraz stali się maoistami) argumentowało, że sama obecność odzianych na czarno anarchistów może być uważana za prowokację. W rezultacie tacy działacze często kwestionowali samą zasadę działania bezpośredniego.

 Niezależnie od przyczyn leżących u podstaw, jest jedna rzecz, którą należy podkreślić. Grupy takie jak DAN były w~większości białe. Szczególnie uderzająca była nieobecność Afroamerykanów. Przez większość swojej historii NYC DAN miał jednego czarnoskórego członka w~aktywnej, liczącej około pięćdziesięciu osób grupie podstawowej. Nie oznacza to, że było to coś w~rodzaju wyłącznie białego. Latynosów zawsze było sporo (chociaż częściej pochodzili z~krajów takich jak Brazylia czy Argentyna niż, powiedzmy, z~Meksyku czy Portoryko), a jeszcze większa liczba aktywistów z~Azji Południowej lub Wschodniej (Chińczyków, Tajwanu, Korei) lub pochodzenia bliskowschodniego (tureckie, egipskie, irańskie). Mimo to ich liczba razem wzięta rzadko dochodziła do więcej niż jednej trzeciej aktywnych członków. Co do reszty, jeśli mieli jakąkolwiek samoświadomą tożsamość etniczną, najprawdopodobniej byli to Żydzi lub Irlandczycy. Choć DAN z~całą pewnością był bardziej zróżnicowany niż, powiedzmy, wczesny SDS, w~mieście tak różnorodnym jak Nowy Jork, to było postrzegane jako skandal.

\section{Dylematy przywilejów, które nie są koniecznie rasowe}

 Do kwestii specyficznie rasowych będę wracał okresowo. Są zmorą wszelkiej radykalnej polityki w~Ameryce Północnej. Chcę tutaj podkreślić, że te dylematy nie są po prostu skutkami rasizmu. Podobne dylematy pojawiają się, gdy jakiś ruch próbuje walczyć z~sytuacjami skrajnej nierówności społecznej. Zawsze ci na dole, którzy mają najwięcej powodów, by chcieć zmierzyć się z~takimi nierównościami, będą mieli również tendencję do dysponowania najbardziej ograniczonym zakresem broni, aby to zrobić. Nieuchronnie powoduje to niekończące się dylematy moralne dla tych, których przywilej faktycznie pozwala im się buntować. 

 To nie jest nowe zjawisko. Istnieje obszerna literatura na ten temat. Na przykład Eric Wolf (1969) wskazał, że w~każdej znanej nam rewolcie chłopskiej kręgosłupem armii partyzanckich jest zawsze średniozamożne chłopstwo; ponieważ najbiedniejszej warstwy brakuje środków do trwałego powstania, a najbogatszym brak motywacji. Podobnie EP Thompson (1971) i~inni wykazali, że podpory wczesnonowożytnych ,,chlebowych zamieszek'' -- w~rzeczywistości wydarzenia bardzo podobne do tego, co teraz nazwalibyśmy akcjami bezpośrednimi -- zwykle pochodziły od bardziej zamożnych klas robotniczych: ani burżuazji, ani żebraków, ale członków szanowanej klasy robotniczej. W rzeczywistości większość wczesnej literatury na temat ruchów radykalnych zdawała się twierdzić, że to niemożliwe, aby prawdziwie uciskani stali się prawdziwymi rewolucjonistami. Karl Mannheim (1929, również Norman Cohen 1957) twierdził na przykład, że naprawdę uciskani nie tylko nie angażują się w~trwałą rewoltę, ale ich sposób wyobrażania sobie alternatyw społecznych ma tendencję do bycia zazwyczaj absolutnym i~ostatecznym. Podczas gdy klasa środkowa ,,dyscyplinowała się poprzez świadomy samorozwój, która uważała etykę i~kulturę intelektualną za swoje podstawowe usprawiedliwienie'' (1929:73) i~rozwijała racjonalne utopie, prawdziwie marginalna klasa miała skłonność do faworyzowania swego rodzaju ekstatycznej wizji nagłego i~całkowitego zerwania. Mannheim nazwał to ,,chiliazmem'' -- ,,strukturą mentalną charakterystyczną dla uciskanych chłopów, czeladników i~pojawiającego się ,,Lumpenproletariatu'' [i] fanatycznie uczuciowych kaznodziejów'' (1929:204)\footnote{Co ciekawe, to właśnie te ostatnie okręgi wyborcze były tradycyjnie spisywane na straty jako ,,anarchiści''.}. Dlatego też, gdy najbiedniejsze elementy się buntowały, robiły to zwykle w~imię jakiejś wielkiej milenijnej wizji, wierząc, że świat, jaki znamy, wkrótce się zakończy i~istniejące hierarchie zostaną zmiecione. Teraz, podczas gdy niewielu w~dzisiejszych czasach uwierzyłoby w~ideę, że biedni żyją w~wiecznej teraźniejszości lub nie są zdolni do długoterminowego planowania, Mannheim ma coś do powiedzenia. Ruchy rewolucyjne zawsze miały tendencję do przejmowania znacznej części swojego temperamentu i~kierunku od tych bardzo ,,średnich warstw''. Przynajmniej zawsze istniała pod tym względem przepaść między tymi, którzy najbardziej ucierpieli w~nierównym społeczeństwie, a tymi, którzy najlepiej potrafili zorganizować skuteczną, trwałą opozycję. Innymi słowy, ci ,,najbardziej dotknięci'' -- jak to ujmuje obecne hasło aktywistyczne -- przez struktury feudalne lub kapitalistyczne rzadko, jeśli w~ogóle, organizują się otwarcie przeciwko niemu. Można argumentować, jak Jim Scott (1985, 1992), że ukryty opór pokornych jest wielką, nierozpoznaną siłą w~historii świata -- i~na pewno miałby rację. Rzadko jednak ten opór przybiera formę jawnego buntu.

 Kiedy te rozbieżności nakładają się na głębsze opisy różnic -- takich jak rasa, kultura, pochodzenie etniczne -- stają się znacznie bardziej widoczne. Ale wydaje mi się, że zawsze będą tam w~takiej czy innej formie. Są po prostu jednym z~nieuniknionych skutków ubocznych nierówności społecznych\footnote{Wszystko to brzmi trochę jak słynne politologiczne pojęcie ,,środkowej trzeciej części'' populacji, która może utożsamiać swoje interesy z~zamożną trzecią częścią nad nimi, tworząc konserwatywną większość, lub z~biedną trzecią poniżej, tworząc postępową. Myślę, że tendencja do redukowania stratyfikacji społecznej do kwestii bogactwa (lub nawet władzy) jest nieco zwodnicza i~że bardziej sensowne jest rozpoczęcie od pojęć, które zacząłem rozwijać powyżej, aby przyjrzeć się relacji między tymi, którzy się buntują głównie przeciwko uciskowi, a buntujących się głównie przeciwko alienacji. Za chwilę rozwinę ten argument.}.

 Oczywiście w~przypadku ruchu globalizacyjnego jednym z~powszechnych poglądów jest to, że nie mówimy nawet o członkach warstwy średniej, ale o członkach elity. Idea ta została tak głęboko zakorzeniona, że  stała się powszechną mądrością nie tylko wśród konserwatywnych komentatorów, ale do pewnego stopnia w~opinii publicznej. Zanim przejdę dalej, pozwólcie, że pokrótce przeanalizuję to wyobrażenie: takie, które jest oczywiście samo w~sobie zjawiskiem społecznym.

\section{Mit funduszy powierniczych}

 Stereotyp działa mniej więcej tak. Rdzeń ,,ruchu antyglobalistycznego'' składa się z~bogatych nastolatków z~wyższej klasy średniej, ,,dzieci z~funduszy powierniczych'', które mogą sobie pozwolić na spędzenie życia, podróżując ze Szczytu na Szczyt, sprawiając kłopoty. W pewnym sensie oskarżenie było dość przewidywalne. Prawicowy populizm w~Stanach Zjednoczonych w~dużej mierze opiera się na oskarżeniu, że liberałowie są częścią elity wyższej klasy średniej, której wartości są głęboko obce wartościom Amerykanów z~klasy robotniczej. Trudno się dziwić, że w~obliczu lewicowych radykałów pierwszym instynktem prawicowego gospodarza audycji radiowej byłoby założenie, że jeśli liberałów wywodzą się z~zamożnych, to rewolucjoniści będą musieli wywodzić się z~rzeczywiście bogatych. Z drugiej strony, jeśli ktoś przeanalizuje zapisy, to odkryje, że pierwszymi postaciami, które stawiały taki zarzut -- to było około okresu konwencji Republikanów i~Demokratów latem 2000 roku\footnote{R2K i~D2K w~żargonie aktywistów lub, w~ich połączonej formie, R2D2. Niestety nie udało mi się wyśledzić prawdziwych nazwisk większości osób, które wysuwały takie twierdzenia i~dlatego zmuszony jestem polegać na mojej (bez wątpienia niedoskonałej) pamięci z~tamtych czasów.} -- były autorytetami w~miastach, które spodziewały się protestów (na przykład burmistrz Los Angeles i~szef policji w~Filadelfii, John Timoney), tonem, który z~pewnością sugerował dostęp do pewnego rodzaju rzeczywistych informacji socjologicznych, których nie mogli mieć. To były w~istocie te same polityczne postacie, które natychmiast wydały rozkaz policji, by atakowała bezprzemocowy, nawet według konwencjonalnej definicji, protest. To z~pewnością daje powód do zdziwienia: zwłaszcza że  tak wielu policjantów w~Seattle początkowo wzdrygnęło się, gdy wydano podobne rozkazy. Biorąc pod uwagę fakt, że w~tym samym czasie w~tajemniczy sposób pojawiła się cała seria innych plotek o aktywistach atakujących policję kwasem i~moczem, można się tylko zastanawiać, czy było to częścią bardziej wyrachowanej kampanii mającej na celu odwołanie się do uprzedzeń klasowych samej policji. Przesłanie na konwencjach i~podobnych mobilizacjach wydawało się brzmieć: ,,Nie myśl o sobie jako o facecie z~klasy robotniczej, któremu płaci się za ochronę bandy bankierów, polityków i~biurokratów handlowych, którzy żywią wobec ciebie pogardę; Pomyśl o tym raczej jako o okazji do pobicia ich zasmarkanych dzieci'' -- porozumienie, które byłoby idealne dla celów polityków, ponieważ również nie chcieli, aby policja faktycznie okaleczała lub zabijała protestujących. Niezależnie od tego, czy tego rodzaju obrazy wyłaniają się ze źródeł wywiadu policyjnego, które zwykle w~dużym stopniu opierają się na jednostkach badawczych z~prywatnych firm ochroniarskich i~konserwatywnych think tanków, a często odtwarzają bardzo dziwne formy prawicowej propagandy, czy też policja rzeczywiście słuchała konserwatywnych radiostacji, jest w~tym momencie niemożliwe do określenia\footnote{Niektóre z~tych pytań rozważymy w~dalszej części rozdziału 9.}. Aktywiści na dużych szczytach regularnie mniej więcej te same zarzuty policji, jak przyjaciel mi je podsumował:
 
 -- Jesteście tylko bandą bogatych dzieciaków, które zakładają maski, żeby tatkowie nie widzieli waszych twarzy, gdy niszczycie rzeczy, a potem wracacie do rezydencji, oglądacie to w~telewizji i~śmiejecie się z~nas. 

 Plotki stały się niezwykle spójne.

\section{Zatem: kim naprawdę są aktywiści?}

 I) Praca i~edukacja

 To, co następuje niżej, nie jest oparte na metodologii statystycznej, ale spędziłem ponad siedem lat wśród anarchistów i~innych zaangażowanych w~działania bezpośrednie i~myślę, że jestem w~stanie dokonać pewnych wstępnych uogólnień. Po pierwsze, aktywiści z~naprawdę bogatych środowisk są niezwykle rzadcy. Jeśli chodzi o zaplecze ekonomiczne, w~rzeczywistości anarchiści są bardzo zróżnicowani. Jeśli jest coś, co ich wyróżnia od większości Amerykanów, to jest to, że jest nieproporcjonalnie duże prawdopodobieństwo, że uczęszczali na studia. Wielu z~nich to oczywiście sami studenci, ale rdzeń aktywisty wydaje się składać z~czegoś, co można by nazwać post-studentami: młodych kobiet i~mężczyzn, którzy ukończyli studia, ale nadal żyją czymś podobnym do studentów, przynajmniej w~takim zakresie, że najczęściej nie spędzają czasu w~regularnej, zorientowanej na karierę pracy przez osiem godzin lub wychowując dzieci.

 Powinienem podkreślić, mimo że to jest sedno, z~pewnością nie jest to przytłaczająca większość. Na przykład w~Nowym Jorku istnieje obecnie grupa matek anarchistycznych. Przeciętne spotkanie nowojorskiego DAN obejmowałoby zwykle uczniów szkół średnich i~emerytów, a także, powiedzmy, czterdziestolatków, z~których wielu nigdy nie uczęszczało do instytucji szkolnictwa wyższego. A NYC DAN był uważany przez wielu innych aktywistów za zdecydowanie ekskluzywny. Im bliżej sceny skłoterskiej, tym częściej spotyka się aktywistów bez formalnego wykształcenia, a dzieje się to niemal powszechnie, gdy dochodzi się do poziomu ,,podróżników'' -- głównie nastolatków, mężczyzn i~kobiet po dwudziestce, uciekinierów lub żyjących życiem dobrowolnej bezdomności, przemieszczania się z~miasta do miasta. Tak jak w~okresie rozkwitu IWW na początku XX wieku, istniała bogata kultura włóczęgów i~hobo pociągów towarowych, tak istnieje do dziś. i~wtedy, jak i~teraz, większość uważa się za anarchistów. Wielu z~nich to sieroty, zbiegowie lub uciekinierzy o bardzo skromnym pochodzeniu, z~niewielkim dostępem do instytucji edukacyjnych, choć wielu z~nich to zapaleni czytelnicy, a wielu jest zaznajomionych z~radykalną teorią, z~mojego własnego doświadczenia wynika, że  najczęściej jest to jakaś odmiana francuskiego sytuacjonizmu. Chociaż ,,podróżnicy'' mogą być liczebnie stosunkowo niewielkim elementem w~ruchu i~nieco marginalnym (większość nienawidzi spotkań), prawdopodobnie będzie ich znacznie więcej podczas jakiejkolwiek większej mobilizacji niż ktokolwiek, kto faktycznie ma fundusz powierniczy. Oni są też niezwykle ważni symbolicznie, ponieważ wyznaczają rodzaj romantycznego standardu autonomicznej egzystencji -- jedzenie z~nurkowania w~śmietniku, odmawianie pracy na etacie -- co reprezentuje jeden z~możliwych ideałów dla tych, którzy pragną żyć poza logiką kapitalizmu.

 Są też tacy, którzy dołączają do takiego świata dobrowolnie: zwykle są wykształceni w~college'u, a czasem porzucają studia, pochodząc ze wzniosłej klasy społecznej. Jest to rodzaj uniwersum celebrowanego w~popularnych anarchistycznych książkach, takich jak powieść CrimethInc Evasion (2001), pół-fantazji białych, punkowych dzieciaków z~klasy średniej, które rzucają się, by dołączyć do tego świata, żyjąc ze śmieci i~resztek społeczeństwa industrialnego\footnote{Rozważmy tutaj następujący pean na przetrząsanie śmietników: ,,Głosząc zbawienie przez śmieci, zmagałem się przez całe życie z~uwarunkowaniami wyższej klasy średniej. Umrzesz od zjedzenia tego jedzenia! mówili. Żywe trupy ,,siły roboczej'' udzielające porad zdrowotnych. Jaką logiką jest to, że jedzenie jest śmiertelne w~momencie, gdy trafiło do worka na śmieci lub przeszło przez tylne drzwi? Jedzenie, które chwilę wcześniej znajdowało się na półce. Była to naiwna wiara w~czystość kupowanego w~sklepie jedzenia i~niezłomna pewność, że śmieci są trucizną. Prawie zabawne. Cóż, nie byłem pewien, skąd się nauczyli ich przesądów o śmieciach, ale płacili za to każdego dnia od 9 do 17. To było smutne, głęboko zakorzenione uwarunkowanie. Warunkowanie korzyści tylko dla korporacji, kosztem milionów złamanych kręgosłupów i~zmarnowanego życia tych, którzy pracują, aby jeść'' (CrimethInc 2001:26).}. Takie życie może reprezentować swego rodzaju wizję moralnej czystości, całkowite odrzucenie społeczeństwa przemysłowego postrzeganego jako motor do produkcji ogromnych ilości odpadów. O ile zakłada się, że nie jest już możliwe po prostu opuszczenie systemu, aby ustanowić autonomiczną egzystencję w~lasach\footnote{Choć kilku anarchistów interesuje się tymi, którzy w~tym kierunku przeprowadzają wiejskie eksperymenty.}, najlepsze, co można zrobić, to żyć z~jego szczątków i~odrzutów. Wielu nurków śmietnikowych jest dość dumnych z~faktu, że pomimo tego, że żyją ze śmieci, udaje im się utrzymać rygorystyczną dietę wegetariańską. Wielu młodszych anarchistów, tych bardziej ,,hardkorowych'', w~różnym stopniu idzie w~ich ślady. W Nowym Jorku jest młody człowiek o imieniu Thaddeus, który twierdzi, że udaje mu się przetrwać za około pięć dolarów miesięcznie, zajmując puste budynki, dopóki policja go nie wyrzuci, nurkując w~śmietniku żywności i~przez cały czas produkując z~kilkoma przyjaciółmi miesięczny przewodnik po bezpłatnych wydarzeniach w~Nowym Jorku. Thaddeus jest stałym bywalcem sceny akcji bezpośredniej. Jest kimś w~rodzaju ekstremalnego przypadku i~w rezultacie uważany za postać bohaterską, ale wielu postrzega to jako naprawdę ,,życie prawdziwym życiem'' w~sposób, w~jaki większość nie żyje. Podczas gdy niewielu ucieka się do, powiedzmy, ulicznych oszustw lub kradzieży, ci, którzy to robią, nalegają, że jedynym uzasadnionym celem są wielkie sklepy korporacji, nigdy drobne sklepiki\footnote{Znam wielu młodych mężczyzn, którzy przyznali się, a nawet chwalili się, że byli prostytutkami w~takim czy innym czasie, ale mniej młodych kobiet, chociaż pewne potwierdzenie roli prostytutki stało się bardziej powszechne od połowy dekady.}, o ile mogą tego uniknąć. Praktyki takie jak nurkowanie w~śmietniku są uważane za całkowicie normalne w~kręgach anarchistycznych. W kuchni nowojorskich biur Independent Media Center (IMC) przez wiele lat wisiał harmonogram wskazujący, w~jakich godzinach lokalne restauracje są prawnie zobowiązane do wyrzucania sushi. Aktywiści na rowerach regularnie robili rundy, aby zbierać stosy rolek sushi, wszystkie starannie opakowane w~plastikowe tacki i~pojemniki, i~umieszczać je w~lodówce IMC. Na innym punkcie regularnie można znaleźć doskonale jadalne bułki i~bajgle. Jak często zauważają aktywiści Food Not Bombs przed głównymi mobilizacjami, nie ma absolutnie żadnego problemu ze zdobyciem darmowej żywności dla, powiedzmy dziesięciu tysięcy ludzi, w~mieście takim jak Nowy Jork,jeżeli ktoś poświęci czas, choć znalezienie sztućców może być znacznie trudniejsze.

 Istnieje również, powinienem również zauważyć, kontr-dyskurs. Większość aktywistów, którzy próbują dojść do pewnego rodzaju kompromisu z~ekonomią głównego nurtu, równie łatwo może odrzucić podróżnych, skłoterów i~nurków śmietników jako “crusties,'' “cruddies,'' “gutter-punks'' (odp. ,,chrupiących'', ,,cudaków'', ,,punków rynsztoku'') korzystających ze swoich białych przywilejów czy jako dzieciaki z~klasy średniej bawiące się w~biedę w~sposób uwłaczający prawdziwym trudom bezdomnych lub wywłaszczonych\footnote{Niektóre z~nich z~pewnością są - w~tym wszystkie postacie z~powieści CrimethInc. Często ci, którzy wysuwają takie oskarżenia, nie są świadomi istnienia autentycznie bezdomnych lub wywłaszczonych anarchistów.}. Ale często krytyka miesza się również z~ambiwalentnym szacunkiem.

 Większość aktywistów -- i~znowu używam tutaj terminu ,,aktywista'' głównie jako skrótu dla ,,anarchisty lub innych zaangażowanych w~inspirowaną przez anarchistę politykę akcji bezpośredniej'' -- czuje, że muszą iść na kompromis z~istniejącym porządkiem gospodarczym. Większość uważa, że  sposób, w~jaki się to robi, jest bardzo osobistą odpowiedzią. Z mojego doświadczenia wynika, że  raczej rzadko słyszy się podobne oskarżenia o ,,sprzedaż'', o kompromis jako o zdradę, które były tak powszechne w~latach sześćdziesiątych i~siedemdziesiątych. Oczywiście, jeśli ktoś zostanie agentem reklamowym Monsanto lub maklerem giełdowym, z~pewnością narazi na szwank swoje referencje aktywistyczne. Ale musiałoby to być coś prawie tak ekstremalnego\footnote{W czasie, gdy byłem profesorem w~Yale, zaskakująco rzadko spotykałem się z~wyzwaniami z~tego powodu: kiedy byłem, zawsze był to e-mail, przez ludzi, których nie znałem i~z którymi właściwie nie pracowałem. Oczywiście w~Yale to była nieco inna historia.}. Oczywiście i~tutaj są wyjątki. Im bardziej hardkorowe są własne wybory, tym bardziej prawdopodobne jest, że odrzuci się tych, którzy prowadzą wygodniejszy lub bardziej pojednawczy styl życia.

 Starsi aktywiści (powyżej trzydziestu, a zwłaszcza czterdziestu lat), którzy najprawdopodobniej mają pracę na pełen etat, często pracują w~branżach skupiających się na rozpowszechnianiu wiedzy i~idei. Na scenie nowojorskiej znam garstkę pisarzy i~dziennikarzy, dużą liczbę nauczycieli (zwłaszcza od szkoły podstawowej do liceum), bibliotekarzy, nawet jednego doradcę zawodowego w~szkole średniej i~wielu związanych w~ten czy inny sposób z~przemysłem drukarskim ( bardzo tradycyjny, radykalny zawód). Niektórzy są dyrektorami teatrów, dramaturgami, choreografami lub w~inny sposób mają kontakt ze sztuką. Z mojego własnego doświadczenia wynika, że  zaskakująco mała liczba pracuje na pełny etat dla organizacji pozarządowych (przynajmniej jest to prawdą w~przypadku konkretnie akcji bezpośredniej). Młodsi działacze -- większość, żyjąc tego rodzaju przedłużonej quasi-młodości, którą nazwałem ,,post-studiami'' -- skłania się do pracy w~niepełnym wymiarze godzin, która pozwala na bardzo elastyczne dni i~godziny. Dzieje się tak częściowo dlatego, że zmieniający się charakter rynku pracy w~Stanach Zjednoczonych sprawił, że praca w~pełnym wymiarze czasu pracy jest trudniejsza -- wielu z~nich kończy się w~pracach tymczasowych -- ale także dlatego, że tak ważna jest dla nich elastyczność. Niektórzy nabierają konkretnej umiejętności, którą można przetłumaczyć: uczą się barmaństwa lub projektowania stron internetowych, zostają technikiem oświetlenia lub dźwięku, zdobywają umiejętności w~gastronomii. Wszystkie są umiejętnościami, które sprawiają, że dość łatwo jest podjąć pracę na tydzień lub miesiąc, a następnie przejść dalej. (Praca jako muzyk również daje elastyczność, ale tak mało się opłaca, że  naprawdę nie można się utrzymać bez pracy na pełny etat). Niektórzy pracują w~przedsiębiorstwach przyjaznych aktywistom: najczęściej w~kuchniach wegańskich lub sklepach ze zdrową żywnością. Inni zostają inżynierami budownictwa\footnote{Istnieje interesująca tendencja do przyciągania anarchistów do urbanistyki.}. Istnieje również garstka pełnoetatowych organizatorów, którzy pracują dla grup aktywistów, takich jak Rainforest Action Network, Ruckus Society, różne grupy pokojowe, związki zawodowe lub programy wymiany, chociaż te prace są notorycznie niskopłatne, a aktywistów o skromniejszych środkach często nie stać na ich przyjęcie. Wiele takich prac nic nie kosztuje, ale aktywiści nadal będą je wykonywać na zasadzie wolontariatu w~niepełnym wymiarze godzin.

 Poniżej postaram się naszkicować typowy dla aktywisty przebieg życia, uogólniając od ludzi, których znałem w~DAN, CLAC, ACC, IMC i~podobnych grupach na Północnym Wschodzie około 2000-2003. Wykonanie takiego szkicu jest z~konieczności ćwiczeniem hipotetycznym, ponieważ zakłada, że  historia pozostanie stała (co jest mało prawdopodobne), ale rzutując aktualne wzorce, można wymyślić coś takiego:

 Nasza idealna typowa działaczka bezpośrednia prawdopodobnie albo zostanie upolityczniona w~szkole średniej, zwłaszcza na scenie punkowej, albo na studiach, działając w~organizacjach kampusowych. Po ukończeniu lub porzuceniu college'u prawdopodobnie spędzi od roku do dziesięciu lat intensywnego zaangażowania w~grupach aktywistów. W ciągu pierwszych kilku lat będą regularnie uczestniczyć w~spotkaniach, być może trzy, cztery lub pięć tygodniowo (w dniach tuż przed akcją, czasem cztery lub pięć dziennie), zwykle w~różnych grupach, utrzymując się przy tym dorywczo lub z~pracy w~niepełnym wymiarze godzin. Ta pierwsza faza jest bardzo intensywna i~prawie niemożliwa do utrzymania na zawsze. Większość przerywa ją to w~taki czy inny sposób. Na przykład można spędzić sześć miesięcy, wykonując pracę aktywistów w~swoim rodzinnym mieście, a następnie spędzić kilka miesięcy intensywnie pracując za pieniądze; potem gdy już zaoszczędzisz wystarczająco dużo na bilet lotniczy, udać się do jakiegoś odległego miejsca: aby pomóc założyć IMC w~Ameryce Południowej, pracować solidarnie na Zachodnim Brzegu lub Chiapas, wchłonąć scenę skłoterów w~Europie lub wziąć udział w~blokadach wycinek. Wiele osób na tym etapie jest w~drodze mniej więcej połowę czasu. Albo można zachować zdrowy rozsądek, od czasu do czasu pogrążając się w~zupełnie innym projekcie -- artystycznym, na przykład intensywnym romansie -- tylko po to, by pojawić się ponownie kilka miesięcy później. Można uciekać na kilka miesięcy do pracy na farmie ekologicznej -- zwyczaj tak powszechny, że w~rzeczywistości jest na to skrót: to woof (Work on an Organic Farm -- praca na farmie ekologicznej)\footnote{Zostałem poinformowany, że technicznie ten akronim to w~rzeczywistości WWOOF dla World Wide Opportunities on Organic Farms i~wywodzi się z~formalnej sieci w~Kanadzie. Ale może być również używany jako czasownik bardziej nieformalnie.}. Ci, którzy koncentrują całą swoją energię w~jednym miejscu, często mają tendencję do całkowitego wypalania się po roku lub dwóch i~rezygnują z~rozdrażnienia; albo znaleźć jakiś konkretny, międzynarodowy lub związany ze społecznością projekt, aby skoncentrować swoją energię i~wycofać się ze wszystkiego innego. W rezultacie grupy takie jak NYC DAN wkrótce zaczęły składać się z~aktywnego rdzenia i~pewnego rodzaju półcienia półemerytowanych aktywistów, których nigdy już tak naprawdę nie widywano na spotkaniach, ale często pojawiali się na akcjach lub imprezach, i~których wiedza, kontakty i~doświadczenie były dostępne dla tych, którzy nadal mieli z~nimi osobisty kontakt.

 Młodsi anarchiści, którzy nie mieszkają na skłotach -- ale większość nie mieszka -- często mieszkają w~kolektywnych domach lub mieszkaniach, często w~biednych lub artystycznych, gentryfikujących się dzielnicach. Niektórzy mieszkają w~przestrzeniach aktywistów: kilka osób mieszkało w~nowojorskim IMC w~latach 2000-2003, a inni w~Walker Space, rodzaju pomocniczego IMC, który mieścił przestrzeń do występów i~studia telewizyjne. Osoby na tyle zamożne, że było ich stać na mieszkanie w~rozsądnej wielkości, często pozwalały na wykorzystanie przynajmniej części miejsca w~mieszkaniu na większe wspólnotowe potrzeby.

 W końcu prawie każdy przechodzi na swego rodzaju pół-emeryturę. Ci, którzy zostają zawodowymi, płatnymi aktywistami, zazwyczaj trafiają do innego środowiska społecznego. Niektórzy chodzą do szkoły podyplomowej: absolwenci zazwyczaj pozostają zaangażowani przez kilka lat, a następnie, gdy są przytłoczeni pracą i~doświadczają presji profesjonalizacji, całkowicie rezygnują z~aktywności\footnote{Oczywiście są wyjątki, ale zaskakująco niewiele.}. Inni mają dzieci lub ustatkowują się -- często z~nieaktywistami -- lub w~końcu podejmują pracę zawodową na pełen etat. Z pewnością są tacy, którzy mimo to utrzymują stałą obecność, ale zwykle dzieje się tak dlatego, że znajdują jakąś karierę, która trzyma ich blisko świata aktywistów -- na przykład zostają prawnikami związkowymi i~nadal wykonują legalną pracę dla anarchistów; lub kierują radykalną księgarnią; lub dlatego, że nadal mieszkają w~kolektywnym domu, skłocie lub społeczności; albo dlatego, że uczą się, jak ostrożnie ograniczać swoje zaangażowanie do jednego, wykonalnego projektu. To drugie jest trudne, ponieważ wymagania dotyczące czasu aktywisty są potencjalnie nieskończone. Sztuczka, aby pozostać zaangażowanym przez długi czas, polega na znalezieniu sposobu na oparcie się pokusie nadmiernego zaangażowania. Stosunkowo mało, z~mojego doświadczenia, osób to się udaje.

 Późne trzydziestki, a na pewno czterdziestki i~pięćdziesiątki, są więc zazwyczaj okresem całkowitego lub prawie całkowitego wycofania. Jeśli jednak utrzymują się wzorce historyczne, dla pewnej liczby istnieje jeszcze późniejszy okres ponownego zaangażowania. Po ukończeniu przez dzieci studiów, zerwaniu z~długoletnim partnerem lub przejściu na emeryturę, może się okazać, że znów zostaną wciągnięci w~świat aktywizmu, od czasu do czasu, przynajmniej na chwilę, z~taką intensywnością, jak na początku.

 II) Tła klasowe i~trajektorie

 Wspomniałem, że jedynym sposobem, w~jakim osoby zaangażowane w~akcję bezpośrednią można uznać za część elity, jest edukacja: znaczna większość miała dostęp do szkolnictwa wyższego, mimo że większość Amerykanów (nieco ponad połowa) nie ma takiego dostępu.

 W przeciwnym razie, jeśli spojrzymy na tło i~trajektorie klasowe, napotkamy nieskończoną różnorodność. Ponownie, nie przeprowadzałem ankiet. Mimo to mogę powiedzieć z~własnego doświadczenia, że  na północnym wschodzie rzeczywistą liczbę aktywistów z~funduszami powierniczymi można policzyć na jednej stronie. W rzeczywistości jest ich znacznie mniej niż, powiedzmy, aktywistów, których rodzice są zawodowymi oficerami wojskowymi, co właściwie jest zaskakująco wysokie. Ale w~obu przypadkach mamy do czynienia ze stosunkowo małymi liczbami.

 Mówiąc szerzej, wydaje mi się, że środowiska aktywistów najlepiej postrzegać jako punkt styku, rodzaj miejsca spotkań między opadającymi, mobilnymi elementami klas profesjonalnych, a wznoszącymi się po drabinie społecznej, mobilnymi dziećmi klasy robotniczej. Pierwsza grupa składa się z~dzieci z~białych kołnierzyków, które odrzucają sposób życia swoich rodziców: córka księgowego podatkowego, która postanawia pracować jako stolarz, córka weterynarza, która zdecydowała się żyć jako grafik, syn menedżera średniego szczebla, który decyduje się zostać inżynierem budownictwa lub działaczem zawodowym. Druga składa się z~dzieci ze środowisk robotniczych, które idą na studia.

 Pod względem historycznym oba te elementy odpowiadają klasycznemu stereotypowi. Pierwszy reprezentuje klasyczną bazę rekrutacyjną dla artystycznej bohemy; jeśli nie dzieci burżuazji, jak często zakładano w~Paryżu lat 50. XIX wieku, gdzie termin ten został wymyślony po raz pierwszy, to dzieci urodzone przez członków elit administracyjnych lub zawodowych, żyjące w~dobrowolnym ubóstwie, eksperymentujące z~przyjemniejszymi, artystycznymi, mniej wyobcowanymi formy życia. Drugi reprezentuje klasyczny stereotyp rewolucjonisty, zwłaszcza na Globalnym Południu: dzieci z~klas pracujących (robotników, chłopów, nawet drobnych właścicieli sklepów), których rodzice przez całe życie starali się posłać synów lub córki na studia, a nawet własnymi siłami zdobyli burżuazyjny poziom wykształcenia, tylko po to, by odkryć, że burżuazyjne poziomy edukacji w~rzeczywistości nie pozwalają na wejście do burżuazji, a często na jakąkolwiek regularną pracę. W szeregach rewolucyjnych bohaterów ostatniego stulecia można zestawiać nieskończone przykłady: od Mao (dziecko chłopów, który został bibliotekarzem), po Fidela Castro (bezrobotnego prawnika z~Kuby) i~tak dalej. W rzeczywistości zarówno bohema, jak i~kręgi rewolucyjne historycznie miały tendencję do bycia miejscem spotkań obu typów.

 Oczywiście jest to bardzo schematyczny obraz. Przede wszystkim całkowicie pomija pewne znaczące grupy: na przykład tych, którzy przyjęli bohemy styl życia, ponieważ ich rodzice byli bohemą, czy dzieci aktywistów zawodowych. Nie należy lekceważyć stopnia samoreprodukcji w~takich podklasach. Ponadto: podczas gdy stereotyp bohemy jako bogatego dzieciaka -- potajemnie wspierającego swoje nawyki absyntu pieniędzmi z~domu, by w~końcu albo umrzeć z~rozpusty, albo wrócić do zarządu firmy tatusia -- jest uderzająco podobny do stereotypu aktywisty jako trustu-fund baby, prawdopodobnie nie jest to dokładniejsze. Z pewnością w~obu środowiskach zawsze byli potomkowie burżuazji, tym bardziej wpływowi za ich pieniądze, umiejętności społeczne i~powiązania. Ale środowiska bohemy ostatnich 150 lat nigdy tak naprawdę nie składały się głównie z~dzieci z~wyższych, a nawet zawodowych klas. Jak niedawno wykazał Pierre Bourdieu (1993), społeczne podstawy dziewiętnastowiecznej kultury bohemy w~Europie wyłoniły się po części w~wyniku dokładnie tych samych procesów, które ukształtowały społecznych rewolucjonistów na Globalnym Południu: na przykład pośród utalentowanych dzieci chłopów, które skorzystali z~nowego systemu edukacji we Francji, a potem i~tak zostali wykluczeni z~konwencjonalnej kultury elitarnej. Co więcej, te środowiska miały tendencję do nakładania się. Bohemy pełne były nie tylko intelektualistów z~klasy robotniczej i~ekscentryków samouków, ale wręcz rewolucjonistów. Przyjaźń Oscara Wilde'a i~Piotra Kropotkina nie była nietypowa; właściwie można ją uznać za symboliczną. Podobnie, kręgi rewolucyjne zawsze były wypełnione uprzywilejowanymi dziećmi, które odrzucały swoje rodzinne wartości; archetypowym przykładem jest Karol Marks (syn prawnika, który stał się dziennikarzem bez grosza przy duszy). Każdy Mao miał swojego Chou En-lai, nawet Castro miał swoje Che. Konstytucja obu środowisk jest więc bardzo podobna. Co prawdopodobnie wyjaśnia, dlaczego artystów tak konsekwentnie pociąga rewolucyjna polityka.

 Warto o tym pamiętać, zwłaszcza że są tacy, którzy konsekwentnie starają się rozdzielić te środowiska. Na przykład w~latach 90. ekolog społeczny Murray Bookchin rzucił rękawicę w~eseju zatytułowanym ,,Anarchizm społeczny lub anarchizm stylu życia: podział nie do pokonania'', w~którym twierdził, że teoria anarchistyczna zawsze miała dwa źródła: burżuazyjne postacie bohemy, takie jak Stirner, i~społeczny anarchizm, który wyłonił się z~ruchu robotniczego, z~Proudhonem, Bakuninem i~Kropotkinem.

 Prawie żaden anarchoindywidualista nie miał wpływu na wyłaniającą się klasę robotniczą. Wyrażali swój sprzeciw w~wyjątkowo osobistych formach, zwłaszcza w~ognistych traktatach, skandalicznym zachowaniu i~anormalnym stylu życia w~kulturowych gettach fin de siécle Nowego Jorku, Paryża i~Londynu. Jako credo indywidualistyczny anarchizm pozostał w~dużej mierze artystycznym stylem życia, najbardziej widocznym w~swoich żądaniach wolności seksualnej (,,wolnej miłości'' i~rozkochanym w~innowacjach w~sztuce, zachowaniu i~ubiorze (Bookchin 1997).

 Bookchin sugeruje, że nawet rzucający bomby z~lat 90. XIX wieku, zabójcy głów państw, nie byli społecznymi anarchistami (i to prawda, że  prawie nigdy nie wydawali się częścią zorganizowanych grup), ale skrajnymi indywidualistami, którzy wyrażali swój osobisty gniew. Chociaż Bookchin tak naprawdę nie kontynuuje argumentacji -- artykuł jest głównie platformą do ataku na Johna Zerzana, Boba Blacka i~Hakima Beya -- praktyczne implikacje wydają się prowadzić w~bardzo podobny sposób, jak w~przypadku Ranjanita: odrzucenie jakiegokolwiek istniejącej ,, kultury aktywistycznej'' jako produktu burżuazyjnych przywilejów, jako odróżnienie od prawdziwie uciskanych.

 Esej, jak można sobie wyobrazić, wywołał niemal niekończące się ataki i~uczynił nazwisko Bookchina klątwą dla całych sekcji ruchu anarchistycznego. W rzeczywistości wydaje mi się, że założenie jest po prostu błędne. To nie jest przepaść nie do pokonania. Nigdy nie było w~niej nic niemożliwego do przebycia. Zamiast tego, argumentowałbym, że głównym problemem dla niedoszłych koalicji rewolucyjnych jest to, że zawsze łączą tych, którzy przede wszystkim buntują się przeciwko alienacji i~tych, którzy przede wszystkim buntują się przeciwko uciskowi, i~że dylematem jest zawsze, jak zsyntetyzować te dwa bunty.

\section{Sztuka i alienacja}

 Jednym z~moich najbardziej uderzających wspomnień z~NYC Direct Action Network było bardzo wczesne spotkanie, na którym omawialiśmy potencjalną zbiórkę pieniędzy. Ktoś ogłosił, że zarezerwował miejsce na pokaz dobroczynny i~zapytał, czy ktokolwiek w~pokoju ma jakieś szczególne umiejętności lub talenty do wniesienia wkładu. Prawie każda ręka w~pokoju się uniosła. Na koniec facylitator poprosił wszystkich, aby krążyli w~kółko i~ogłaszali, co mogą zrobić: byli poeci, malarze scen, żonglerzy ognia, członkowie grup śpiewających a cappella, tancerze cieni, performerzy, gitarzyści flamenco, śpiewacy punkowi, magowie\ldots  Okazało się, że z~czterdziestu dwóch osób w~pokoju było dokładnie pięciu, którzy nie byli w~stanie wymyślić niczego, co mogłyby wnieść. Było to tym bardziej niezwykłe, że DAN -- w~przeciwieństwie do powiedzmy Reclaim the Streets, sojusznicza grupa nowojorska -- nie była nawet uważana, według standardów aktywistów, za grupę szczególnie artystyczną. Ogólnie scena akcji bezpośredniej jest w~przeważającej mierze zdominowana przez ludzi, którzy byli również zaangażowani w~pewien rodzaj kreatywnej autoekspresji. Muzycy. Lalkarze. Aktorzy. Rysownicy. Artyści. Można powiedzieć, że wiele z~tego wywodzi się w~takim samym stopniu z~etosu ,,zrób to sam'' kultury punkowej, jak z~małoskalowej, zorientowanej na rzemieślników kreatywności kultury hippisowskiej\footnote{Większość hippisów z~lat 60., którzy nie porzucili całkowicie swojego stylu życia, przeszła do drobnej produkcji rzemieślniczej, jeśli nie w~rolnictwie, to w~rękodziełach i~biżuterii; w~efekcie stali się ludźmi, którzy niegdyś byli najsilniejszymi ,,naturalnymi'' okręgami anarchizmu, niezależnymi rzemieślnikami i~drobnymi rolnikami, których Marks wyśmiewał jako ,,drobnomieszczan''.}.

 Tylko jedno wymowne studium przypadku:

 Ojciec Glassa jest policjantem, matka instruktorką aerobiku i~jogi. W liceum była punkiem, który tworzył własne ubrania, projektując wyszukane kreacje z~ubrań porzuconych i~wyciągniętych ze śmietników. Opowiada mi, że ma żywe wspomnienia z~bycia wyśmiewaną się przez ,,modowych punków'', bogatych dzieciaków, które kupowały swoje podarte ubrania w~drogich butikach, i~jak śmieszni byli, nieświadomi, że udowadniają, że są oszustami w~całym duchu tego, co robią. Poszła na studia, głównie wygrywając konkursy pisarskie. Po ukończeniu studiów przez krótki czas pracowała w~błyszczącym magazynie ekologicznym, straciła pracę, gdy pismo zbankrutowało (za większość pracy nigdy jej nie zapłacono), a teraz, po dwudziestce, przeplata pracę barmanki z~przygodami aktywistki, wszędzie mieszka w~squatach od Cleveland do Buenos Aires do Honolulu, sporadycznie publikuje artykuły w~czasopismach krajowych. Jej celem, jak mówi, jest nabycie ziemi i~spędzanie co najmniej połowy czasu na wspólnie zarządzanej, permakulturowej farmie.

 Takie postacie mogą być postrzegane, jak mówię, jako uwięzione w~pewnego rodzaju zawieszonej społecznej młodości. Przecież w~Ameryce każdy jako dziecko angażuje się w~twórcze działania (w szkole jest się do tego zmuszonym, od malowania palcami po szkolne przedstawienia). Zwykle, gdy ktoś kończy okres dojrzewania, oczekuje się, że większość z~tego zrezygnuje. Oczekuje się, że dorośli, o ile nie mają szczęścia, aby znaleźć karierę związaną z~pracą twórczą, wyrażają się głównie poprzez konsumpcjonizm, a może jakieś hobby -- to ostatnie zwłaszcza po przejściu na emeryturę. Moim zdaniem jednak pomaga to wyjaśnić jeden z~wielkich paradoksów radykalnej polityki. Można powiedzieć: wiek dojrzewania jest dla większości Amerykanów etapem, w~którym jest się jednocześnie najbardziej i~najmniej wyalienowanym. Dlatego czasami rewolucję można przedstawić jako ostateczne przezwyciężenie dojrzewania ludzkości -- zerwanie z~przeszłością, które w~końcu wybawi nas z~naszego wiecznego wyobcowania -- lub jako początek pewnego rodzaju wiecznego dojrzewania, ,,początek historii''. Dla większości z~nas, którzy nie żyją w~ramach społeczeństwa kastowego lub gildii, dorastanie jest okresem potencjału: można zrobić lub być prawie wszystkim. Dojrzałość, dorosłość społeczna, to nie tyle kwestia zaakceptowania swojej szczególnej roli (jako sekretarki, ochroniarza, menedżera funduszu, mechanika), ale nawet bardziej zaakceptowania wszystkich tych rzeczy, którymi się nigdy nie zostanie: gwiazdą rocka, olimpijskim skoczkiem narciarskim, reporterem śledczym globtroter, pierwszą kobietą-prezydent itd. Jeśli spojrzy się na jedną słynną (i notorycznie minimalną) próbę zdefiniowania komunizmu przez Marksa, jest ona prawie całkowicie zdefiniowana w~taki sposób, że nikt nie musi z~niczego rezygnować: można łowić ryby rano, po południu zaganiać owce i~krytykować przy obiedzie, a wszystko to bez zostawania rybakiem, pasterzem czy krytykiem. Osoba staje się zwykłym człowiekiem, niezdefiniowanym przez swoją obecną rolę. Według współczesnych terminów wiecznym nastolatkiem.

 Nie oznacza to, że aktywiści są niedojrzali -- chyba że zakłada się, że dojrzałość musi koniecznie być kwestią wyrzeczenia się własnej kreatywności i~poczucia możliwości oraz zaakceptowania życia w~otępiającej nudzie i~codziennej uległości. Nie uważam też za przydatne postrzeganie tego wszystkiego po prostu w~kategoriach ,,oporu'', przynajmniej w~konwencjonalnym akademickim sensie, który zakłada, że  skoro władza jest ostateczną rzeczywistością, każda forma praktyki może być postrzegana tylko jako albo reprodukowanie lub opór\footnote{Dobrą krytykę logiki oporu można znaleźć we wstępie do Fletcher 2007.}. Dlatego bardziej pożyteczny wydaje mi się powrót do alternatywnych tradycji intelektualnych, które w~dużej mierze preferują aktywiści, i~postrzeganie terminów operacyjnych jako równowagi między buntem przeciwko alienacji a buntem przeciwko uciskowi.

\section{Style bohemy}

 Hipisi lat 60., a następnie ruch punkowy późnych lat 70. i~80. postrzegano jako pierwsze ruchy masowej bohemy\footnote{Termin wydaje się wywodzić od krytyka rockowego Roberta Christgau (np. Christgau 2000).}: szeroka popularyzacja artystycznego ideału poświęcenia mieszczańskich wygód dla pogoni za spontanicznością, kreatywnością i~przyjemnością. Można też postrzegać je jako punkty, w~których same formy bohemy nabrały charakteru ruchów masowych. Toczy się oczywiście niekończąca się debata na temat znaczenia: do jakiego stopnia to wszystko jest formą oporu (np. Hebdige 1979), do jakiego stopnia te ruchy są naprawdę awangardą konsumpcjonizmu, badając dziedziny doświadczenia, które mogą być skutecznie utowarowione w~następnym pokoleniu (Campbell 1987). Dla mnie jednak jedną z~interesujących rzeczy jest stopień, w~jakim te historycznie ukonstytuowane kategorie stają się faktycznie trwałe. Są postrzegane jako sposoby bycia. Dzisiejszy sens jest taki, że zawsze będą punkowie i~hipisi:

\medskip
\noindent [Wyciąg z~zeszytów, zima 2001]
\medskip

 Krótka wycieczka na temat terminów ,,punk'' i~,,hipis”

 Nikt nigdy nie użyłby tych terminów do opisania siebie. Nigdy nie słyszałem, żeby ktoś powiedział ,,Jestem punkiem'' lub ,,Jestem hipisem''. Są to terminy, których używasz do opisania kogoś innego. W kręgach na Wschodnim Wybrzeżu nazywanie kogoś hipisem jest zawsze wyśmiewaniem się z~niego, przynajmniej w~niewielkim stopniu: pomimo faktu, że w~połowie przypadków sama mówczyni może być uznawana z~innego punktu widzenia -- np. komentarz Brooke na temat nowego oddziału DAN w~Santa Cruz, ,,prawdopodobnie banda hipisów i~szaleńców, ale i~tak ich kochamy'' Albo: ,,kiedy proponujesz, że organizujemy krąg perkusyjny, mówimy o \textit{dobrym }bębnieniu, czy po prostu złym hipisowskim bębnieniu?''. W przeciwieństwie do tego termin ,,punk'' prawie nigdy nie jest pejoratywny. Zwykle używa się go w~prosty opisowy sposób: np. ,,Mówię o Laurze. Wiesz, ta punkowa dziewczyna z~zielonymi włosami? 

 Mimo to jest bardzo niewiele osób, które mogą być łatwo i~wyraźnie skategoryzowani jako jeden lub drugi. Niektóre istnieją. Ariston ze swoim irokezem jest oczywiście bardzo punkowy; Nealę trudno uznać za hipisa (nawet jeśli jej partner jest takim Gotą, jak to tylko możliwe). Ale to są skrajne przypadki. Większość przypomina bardziej, powiedzmy, Warcry, która nosi brudne bluzy z~kapturem i~naszywki, gdy układa liście i~kwiaty na ścianach Independent Media Center -- idiosynkratyczna mieszanka obu.

 Często terminy są skontrastowane pokoleniowo, a hipisi zawsze są ociężałym starszym pokoleniem. Brad opowiada o uderzającym kontraście między staromodnymi, hipisowskimi blokadami lasów w~Oregonie i~Północnej Kalifornii w~stylu lat 60., a nową energią i~taktyką bojową, która została wprowadzona, gdy zaangażowały się punkowe dzieciaki. To wypowiedź aktywistki leśnej, która, choć walnie przyczyniła się do sprowadzenia punków do lasu, według nowojorskich standardów jest nikim innym jak hipisem. ,,Hippis'' w~rzeczywistości regularnie staje się synonimem ,,pacyfisty'', a ,,punk'' -- ,,młodszego, wojującego anarchisty''. Tak więc w~Seattle, kiedy samozwańczy ,,gliniarze pokoju'' w~niektórych przypadkach fizycznie zaatakowali anarchistów z~Czarnego Bloku, aby powstrzymać ich przed wybiciem okien (anarchiści z~Czarnego Bloku odmówili oddania ciosów, ponieważ nie stosowali przemocy), jest to zwykle opisywane jako przypadek ,,hipisów bijących punków''.

 Oczywiście nie są to jedyne przywoływane terminy (nawet nie wchodzę w~wpływ na przykład sceny rave czy radykalnego hiphopu), ale nie sądzę, by koncentrowanie się na centralnym miejscu punka było nieuzasadnione, choćby ponieważ wydaje się, że wielu najbardziej aktywnych białych anarchistów zostało wciągniętych przez wczesne doświadczenia sceny punkowej.

 Dużo napisano o punku jako subkulturze, ale chcę tutaj podkreślić rolę punka jako miejsca rozpowszechniania pewnego rodzaju popularnego sytuacjonizmu. Ta spuścizna sytuacjonistów jest prawdopodobnie najważniejszym teoretycznym wpływem na współczesny anarchizm w~Ameryce i~oznacza to, że -- chociaż wielu anarchistów zna terminologię akademicką -- używają oni bardzo odmiennego słownictwa teoretycznego.

 Sytuacjonistyczna Międzynarodówka była pierwotnie grupą radykalnych artystów, którzy w~latach 50. i~60. przekształcili się w~ruch polityczny. Można ich postrzegać jako kulminację pewnego trendu. Co najmniej od czasów dadaistów i~futurystów awangardowe ruchy artystyczne zaczynały zachowywać się jak partie awangardowe, publikując manifesty, robiąc czystki i~tym podobne. Sytuacjoniści byli pierwszymi, którzy dokonali całkowitego przejścia, ostatecznie nie tworząc w~ogóle własnej oryginalnej sztuki. Jako grupa zachowywali się jak rodzaj karykatury sekciarzy marksistowskich, nieustannie oczyszczających i~potępiających się nawzajem\footnote{W rzeczywistości usunęli jednego członka, architekta, tylko dlatego, że był powiązany z~kimś, kto faktycznie zaprojektował budynek.}. Guy Debord (1967) przedstawił skomplikowaną dialektyczną teorię ,,społeczeństwa spektaklu'', dowodząc, że w~kapitalizmie nieustępliwa logika towaru, która czyni nas pasywnymi konsumentami, stopniowo rozciąga się na każdy aspekt naszej egzystencji. W końcu stajemy się tylko widzami własnego życia. Środki masowego przekazu to tylko jedno technologiczne ucieleśnienie tego procesu. Jedynym środkiem zaradczym jest tworzenie ,,sytuacji'', improwizowanych momentów spontanicznej, niezbywalnej kreatywności, w~dużej mierze poprzez odwrócenie narzuconych znaczeń spektaklu, rozbijanie fragmentów i~składanie ich w~przewrotny sposób. (Stąd najbardziej popularnym produktem sytuacjonistycznym, o nazwie,,Can the Dialectic Break Bricks?'', często pokazywanym na zbiórkach pieniędzy, jest film kung-fu z~Hongkongu, z~ponownie dodanymi napisami). Raoul Vaneigem (1967, 1979) opracował teorię rewolucji zbudowaną wokół zniszczenia wszystkich relacji zbudowanych na zasadzie wymiany, na ,,przetrwaniu'' w~przeciwieństwie do ,,życia'', z~często dziwną, jaskrawą, ale wciąż w~jakiś sposób radosną mieszanką ultralewicowego marksizmu -- gloryfikacją spontanicznych rad robotniczych i~powstańczych dzikich strajków -- oraz pogonią za niezapośredniczonymi formami przyjemności, rozpętaniem pożądania i~rozpadem sztuki w~życie.

 Właściwie istnieje konkretny, genealogiczny związek między punkiem a sytuacjonizmem. Malcolm McLaren, angielski producent, który skutecznie wynalazł Sex Pistols, a tym samym ruch punkowy, był zaangażowany w~sytuacjonistyczną grupę odłamową, a artysta Sex Pistols, Jamie Reid, używał zasad sytuacjonizmu do projektowania okładek i~ogólnej estetyki (Savage 1991). Bez względu na to, czy McLaren mówił poważnie, czy nie (niektórzy -- np. Elliot 2001 -- twierdzą, że po prostu zmyślał), zasady sytuacjonistów mocno zakorzeniły się w~punkowej filozofii -- szczególnie wśród setek mniejszych, wyraźnie anarchistycznych zespołów punkowych, które wyłoniły się w~latach 80. i~90. (Crass, Conflict, Exploited, Dead Kennedys). Chwytliwe wersety z~Vaneigema nieustannie powracają w~tekstach piosenek, a literatura sytuacjonistyczna jest szeroko dostępna w~każdym anarchistycznym infoshopie lub księgarni, wraz z~ich współczesnym, Corneliusem Castoriadisem i~innymi członkami grupy Socialisme ou Barbarie, oraz materiałami historycznymi na temat francuskiej blisko rewolucji 1968 roku. W większości takich księgarń brakuje znaczącej przestrzeni dla większości tego, co we Francji określa się mianem ,,myśli 1968 roku'' Deleuze'a, Foucaulta czy Baudrillarda -- tych autorów postrzeganych jako reprezentujących radykalną myśl francuską w~akademii. Zasadniczo punki i~rewolucjoniści wciąż czytają francuską teorię sprzed 1968 roku, akademicy czytają głównie teorię z~okresu tuż po tym, z~czego większość składa się z~przedłużonej refleksji nad tym, co poszło nie tak, najczęściej, konkludując, że rewolucyjne marzenia są niemożliwe (Starr 1995).

 Punk, oczywiście, ma być nieco zniechęcający dla niewtajemniczonych. To sprawia, że  outsiderowi trudno jest zauważyć, że -- pomimo brutalnej, gniewnej, nadmiernie wzmocnionej estetyki -- skutecznie odgrywał tę samą kulturową rolę dla białej miejskiej młodzieży z~późnych lat 70., 80. i~90., jak muzyka ludowa w~latach 50. i~60. -- jako rodzaj uproszczonej muzyki ludu, którą każda osoba może robić. Odgrywał też podobną rolę polityczną. Najlepiej tego ducha można podsumować w~punkowym zinie z~końca lat 70., cytowanym przez Dicka Hebdige'a (1979:123), który zawierał małą tabelę układu palców z~trzema akordami i~podpisem: ,,teraz załóż zespół i~zrób to sam''. Podstawowym credo punka stało się DIY. Stwórz własną modę. Stwórz własny zespół. Odmów bycia konsumentem. Jeśli to możliwe, zostań nurkiem w~śmietniku i~niczego nie kupuj. Jeśli to możliwe, odrzuć pracę na etat. Nie poddawaj się logice wymiany. Ponownie wykorzystuj i~ponownie używaj fragmenty spektaklu i~systemu towarowego, aby stworzyć artystyczną broń, aby go obalić. 

 W rzeczywistości można powiedzieć, że istnieją dwa nurty intelektualne, które wyłoniły się z~okresu maja '68 we Francji, które wciąż są żywe w~Stanach Zjednoczonych i~świecie anglojęzycznym: nurt rewolucyjny sprzed 1968 roku, utrzymywany przy życiu w~zinach, anarchistycznych infoshopach, czy Internet oraz odmiana po 1968 roku, w~dużej mierze wątpiąca nad możliwością masowej, zorganizowanej rewolucji, utrzymywana przy życiu na seminariach magisterskich, konferencjach akademickich i~czasopismach naukowych. Pierwsza dąży do uznania kapitalizmu za wszechogarniający system symboliczny, który tworzy skrajne formy ludzkiej alienacji, ale widzi w~nim możliwość buntu przeciwko niemu w~imię przyjemności, pożądania i~potencjalnej autonomii ludzkiego podmiotu. Drugi ma tendencję do dostrzegania systemu (czy nazywa się go teraz kapitalizmem, władzą, dyskursem itp. ) jako tak wszechogarniającego, że jest istotną cechą pożądającego podmiotu, przez co jakakolwiek krytyka wyobcowania lub możliwość rewolucji przeciwko samemu systemu jest efektywnie nieprawdopodobna. Ryzykując redakcję (choć w~tym kontekście nieuczciwe byłoby udawać, że mógłbym zrobić cokolwiek innego), sytuacja jest pełna niekończącej się ironii. Sytuacjoniści argumentowali, że system czyni nas pasywnymi konsumentami, ale wezwali do aktywnego oporu. Obecna radykalna ortodoksja akademicka wydaje się albo odrzucać albo pierwszą, albo drugą część: to znaczy albo argumentuje, że nie ma systemu narzuconego konsumentom, albo opór jest niemożliwy. Ten pierwszy nurt od dawna cieszy się największą popularnością: od początku lat 80. każdy, kto wysuwa argument w~stylu sytuacjonistycznym na forum akademickim, może spodziewać się natychmiastowego potępienia jako purytańskiego i~elitarnego za sugerowanie, że konsumenci pozwalają się biernie manipulować. Konsumenci raczej twórczo reinterpretują konsumenckie style, modę i~produkty na różne wywrotowe sposoby (np. Miller 1987, 1995). Innymi słowy, zwykli ludzie już praktykują detournement.

 Wielka ironia losu polega na tym, że ta rodząca się ortodoksja, która szybko stała się ostoją kulturoznawstwa (a później antropologii), ograniczała się ściśle do akademii. Opracowania kulturoznawcze rzadko, jeśli w~ogóle, były czytane przez ,,zwykłych ludzi'', o których mowa, podczas gdy literatura sytuacjonistyczna, która według tych standardów była najbardziej elitarną pozycją z~możliwych, faktycznie ma pewną popularną publiczność. Na przykład \textit{Rewolucja życia codziennego }(Vaneigem 1967) prawie nigdy nie jest przypisywana na kursach ani cytowana w~tekstach akademickich, ale jest tak samo regularnie czytana przez radykałów na studiach, jak trzydzieści lat temu. Wszystko to raczej potwierdza, że, jak zasugerował mi kiedyś mój przyjaciel Eric Laursen, powodem, dla którego sytuacjonizmu nie można zintegrować z~akademią, jest po prostu to, że ,,nie można go czytać jako nic innego jak wezwanie do działania''. To jest oczywiście dokładnie, co sprawia, że  jest tak popularny wśród aktywistów. Sytuacjonizm, z~jego całkowitym odrzuceniem systemu, jego wezwaniem do wojujących interwencji artystycznych, jego wiarą, że mogą one ostatecznie przyczynić się do rewolucji społecznej, jest idealną filozofią dla aktywisty, którego po raz pierwszy przyciągnęło do punka poczucie głębokiego wyobcowania ze społeczeństwa masowego i~zdecydowanie, aby coś z~tym zrobić.

 Innym skutkiem tego rozłamu jest to, że akademia, poczynając od myślicieli we Francji po 1968 roku, w~dużej mierze odrzuciła ideę ,,alienacji''. Bez jednolitego podmiotu, bez jakiegokolwiek wyobrażenia o bardziej naturalnej czy autentycznej relacji tego podmiotu ze światem i~innymi ludźmi, starsze teorie wydawały się naiwne i~nie do obrony. Termin ten zniknął w~wielu teoriach społecznych. O ile został zachowany, to w~pewnych gałęziach socjologii alienacja stała się czymś, co można było sformalizować statystycznie i~zmierzyć w~kwestionariuszach: prowadząc szybko do wniosku, że najbardziej wyobcowani (izolowani, rozgniewani) członkowie społeczeństwa byli najbardziej marginalni (cudzoziemcy bez dokumentów, na przykład, lub członkowie uciskanych mniejszości). Częściowo w~rezultacie alienacja zaczęła być postrzegana jako psychologiczne doświadczenie ucisku: współczesne badania nad tym tematem mówią o ,,wyobcowaniu rasowym'', ,,alienacji płciowej'', alienacji na podstawie tożsamości seksualnej lub ubóstwa itd. (Schmidt i~Moody 1994, Geyer i~Heinz 1992, Geyer 1996). To samo w~sobie pomaga wyjaśnić nieustanną atrakcyjność teoretyków lat 60.: wszystko teraz jest opisywane w~kategoriach wykluczenia z~głównego nurtu społeczeństwa. Miarą tego wykluczenia jest alienacja. Jest to jednak zasadniczo koncepcja liberalna. Siłą starszego poglądu na alienację było upieranie się, że nie jest to tylko kwestia wykluczenia, ale że jest coś głęboko, fundamentalnie nie tak z~samym głównym nurtem. Że nawet zwycięzcy są ostatecznie nieszczęśliwi, przynajmniej w~porównaniu z~tym, czym mogliby być w~wolnym, egalitarnym społeczeństwie. Anarchiści -- przynajmniej ci, którzy nie mogą twierdzić, że pochodzą z~uciskanych grup -- są pozostawieni z~poczuciem wściekłości i~odrzuceniem systemu, który wydaje się wszechogarniający i~potworny, oraz oficjalną kulturę intelektualną, która nie może ustanowić teoretycznego wyjaśnienia powodów, dla których powinni tak się czuć.

 Niektóre pytania podjąłem gdzie indziej. Na przykład we wcześniejszym eseju o anarchizmie (Graeber 2003:337) zapytałem, dlaczego, nawet jeśli nie istnieje prawie żaden elektorat dla polityki rewolucyjnej, wciąż można znaleźć rewolucyjnych artystów, pisarzy i~muzyków. Mój wniosek: musi istnieć jakiś związek między doświadczeniem pracy niewyalienowanej, wyobrażaniem sobie rzeczy, a następnie ich realizacją, a umiejętnością wyobrażania sobie alternatyw społecznych. Doszedłem do wniosku, sugerując, że o koalicjach rewolucyjnych zawsze można mówić, że opierają się na swego rodzaju sojuszu między społeczeństwem najmniej wyalienowanym i~najbardziej uciskanym (i że rewolucje faktycznie mają miejsce, gdy te dwie kategorie w~dużej mierze się pokrywają). Pomoże to przynajmniej wyjaśnić, dlaczego prawie zawsze wydaje się, że są wieśniakami i~rzemieślnikami albo jeszcze bardziej, nowoproletariatyzowanymi byłymi chłopami i~rzemieślnikami albo nawet bardziej zagadkowy fakt, że tak wiele nastolatków może ponawiać doświadczenie pogo w~klubach punkowych, aby dojść do wniosku, że ich własna wolność jest ściśle związana z~losem zubożałych rolników posługujących się językiem Tzeltal w~Chiapas.

 Jednak to sformułowanie pozostaje bardziej niż trochę surowe. Prawdopodobnie rzeczywista opozycja powinna istnieć między tymi, którzy są doprowadzeni do radykalnej polityki w~buncie przeciwko alienacji, a tymi, którzy buntują się przeciwko uciskowi. Oczywiście nie jest tak, że jest wielu, dla których jest to po prostu jedno lub drugie. Jednak z~punktu widzenia aktywisty istnieją bardzo dobre powody, by nie rezygnować z~tego rozróżnienia całkowicie. Bez niego niemożliwe byłoby twierdzić, że rewolucyjna zmiana byłaby w~interesie wszystkich, nawet tych, o których nie można powiedzieć, że są w~jakikolwiek sposób uciskani. Z drugiej strony trudno byłoby twierdzić, że rozpacz bogatego nastolatka z~przedmieść Stanów Zjednoczonych, w~obliczu życia w~bezdusznym konsumpcjonizmie, ma prawie taką samą wagę moralną, jak, powiedzmy, rozpacz powoli umierającego, na choroby, którym można zapobiec, biednego nastolatka z~Mozambiku. To właśnie ten dylemat prowadzi do niekończących się napięć i~wzajemnych oskarżeń, które nawiedzają życie aktywistów.

\section{Przypadkowe obserwacje dotyczące kultury aktywistycznej}

\begin{flushright}

\texttt{Społeczeństwo, które odmawia nam każdej przygody, czyni z~własnego obalenia jedyną możliwą przygodę.}


-- Hasło Reclaim the Streets
\end{flushright} 

 Jeśli ktoś postrzega kapitalizm jako gigantyczny, bezsensowny silnik niekończącej się ekspansji, który sprowadza większość mieszkańców planety do beznadziejnej nędzy, który sprowadza nawet jego beneficjentów do samotnych, odizolowanych atomów, skazanych przez strach i~niepewność na życie pełne ogłupiającej pracy i~bezsensownego konsumpcjonizmu, nawet gdy grozi zniszczeniem planety, ale jeśli ten ktoś jednocześnie nie chce lub nie wierzy, że można po prostu uciec z~systemu, ale chce zostać i~walczyć, to co właściwie może zrobić? Jaki rodzaj relacji społecznych można stworzyć wśród tych, którzy chcą uczynić swoje życie odrzuceniem samej logiki kapitalizmu, nawet jeśli z~konieczności w~nim pozostają?

 Logika życia bohemy zawsze była próbą odpowiedzi na to pytanie. Zawsze skłaniała się ku kultywowaniu przygody, niebezpieczeństwa i~ekstremalnych form doświadczenia, ale jednocześnie relacji wzajemnej pomocy i~zaufania między tymi, którzy do niej dążą, a często nawet z~tymi, którzy w~przeciwnym razie mogliby być obcy. Taka właśnie wrażliwość spotyka się również w~działaniach bezpośrednich.

 Zastanów się ponownie nad ideą pogo, w~której tancerze rzucają się na siebie lub zanurzają się w~tłumie. Chodzi o tworzenie niebezpiecznych, a nawet brutalnych sytuacji, ale jednocześnie o niemal ślepą wiarę w~otaczających nieznajomych -- o pomoc i~wsparcie -- ponieważ w~końcu, jeśli cię nie złapią ani nie zabezpieczą, możesz skończyć się ze złamaną szyją. W zasadzie logika agresji w~zabawie i~ostatecznego zaufania ma wiele wspólnego z~sadomasochizmem, do którego nieustannie nawiązuje (choć rzadko praktykowany) punkowa estetyka. To rodzaj przyjemności, która rodzi się z~przygody: ekscytacja, nieprzewidywalność, wiara i~poleganie na towarzyszach, które mogą być prawdziwe tylko przy nieskończonej możliwości zdrady. Jednocześnie jednak nie jest to etos machismo. Jedna sprawa, którą szybko zauważyłem, angażując się w~kołach anarchistycznych, to akceptacja fizycznej słabości.

\medskip
\noindent [Fragmenty notatek: czerwiec 2000, z~późniejszymi dodatkami]
\medskip

\noindent \textit{Słabość:}

 Większość aktywistów nie wydaje się niesamowicie sprawna fizycznie, na pewno nie są sportowcami. Zwykle są żylaści, czasami grubi, ale prawie nigdy umięśnieni. ,,Chude weganki'', jak głosi stereotyp. (Słynny komentarz w~gazecie Los Angeles podczas protestów DNC w~2000 roku: ,,Policji było dwa razy więcej niż demonstrantów; lub jeśli liczyć na wagę, cztery razy więcej'' ) Podobnie z~drugiej strony w~opublikowanym ,,Przewodniku anarchistycznym po LA'': ,,wysportowany facet ubrany jak hollywoodzka wersja punk rocka, który namawia cię do ataku na gliny to glina''. Innymi słowy, jednym ze sposobów na wykrycie infiltratora jest czysta sprawność fizyczna. Pomimo faktu, że wielu ma, jak można się spodziewać, mnóstwo umiejętności i~doświadczenia na świeżym powietrzu, wspinania się po drzewach i~ścianach i~tego typu rzeczy. Hipisi w~butach turystycznych i~doświadczeniem szlaków wydają się bardziej wysportowani niż punkowie: są przynajmniej żylaści i~odporni. Jest to szczególnie zaskakujące na początku, gdy po raz pierwszy poznasz dzieciaki z~Czarnego Bloku, które w~prasie podobno są ,,brutalni'' i~które nawet wśród aktywistów zostali nazwanymi ,,piechotą morską naszego ruchu'' i~odkryjesz, że to głównie banda nieśmiałych, ektomorficznych nastolatków. Oczywiście najczęściej są też weganami. Podejrzewam, że jest to jedna rzecz, która musi naprawdę skomplikować stosunki z~policją, ponieważ prawdopodobnie są to dokładnie takie dzieciaki, na jakich zwykły znęcać się te dzieciaki z~podstawówki, które później miały zostać glinami\footnote{Oczywiście jest też element klasowy, którym władza lubi się bawić\ldots  tak jak w~gimnazjum.}.

 Ciekawy nacisk na słabość wydaje się odzwierciedlać wyraźną troskę o osoby niepełnosprawne i~schorzenia biorących udział w~działaniach, które ja -- jak większość nowo przybyłych, jak sądzę -- początkowo wydawały się dość niepokojące. Na szkoleniach prawniczych toczyły się niekończące się dyskusje na temat tego, czego się spodziewać w~przypadku aresztowania i~konieczności podawania insuliny, leków na AIDS lub wielu innych schorzeń. ,,Czy policja pozwoli ci zatrzymać lekarstwo? Nie. Mają dostarczyć ci lekarstwa od lekarza policyjnego, ale zwykle tego nie robią. A co z~hipoglikemią?'' (Szeroko krążyła historia o kobiecie z~hipoglikemią na A16, która miała problem z~cukrem i~została aresztowana, gdy złapała czyjś telefon, myśląc, że to jej własny). Oczywista pierwsza reakcja, którą większość neofitów musi stłumić: co w~ogóle robi cukrzyk chory na AIDS, narażając się na gaz łzawiący, pałki i~przede wszystkim aresztowanie? Ale jest to połączenie oczywistego pragnienia maksymalnej otwartości z, jak podejrzewam, ukrytym poczuciem, że jeśli ktoś jest zaangażowany w~moralną walkę z~policją, słabość może być siłą. Musimy zmusić ich do humanitaryzmu!

 W połączeniu z~niekończącymi się tabu dotyczącymi jedzenia, wszystko to tworzy swego rodzaju labirynt barier: niektórzy są wegetarianami, inni weganami, inni mają alergię na psiankowate lub cierpią na choroby środowiskowe, wielu wydaje się bardzo bliskich hipochondrii z~niekończącymi się prawdziwymi lub urojonymi dolegliwościami. Jednak ci sami ludzie często prowadzą życie pełne przygód, jakie można sobie wyobrazić.

 Wtedy możemy wejść w~fenomenologię masażu pleców jak łańcuch masaży w~przerwie od treningu facylitacyjnego. Trzymanie się za ręce lub łączenie rąk w~ludzkich łańcuchach. Ogólne wzorce dotykania: zwykli Amerykanie prawie nigdy się nie dotykają. Wydaje się, że anarchiści szczególnie lubią uściski (chociaż niektórzy, Crusty Kanadyjczycy z~CLAC znani są z~tego, że ze zdumieniem pytają nas, nowojorczyków, czy zostaliśmy skorumpowani przez kalifornijskie typy jak Starhawk z~tymi wszystkimi drażliwymi bzdurami), ludzi opierających się o siebie, trzymających się za ręce. Od samego początku, na szkoleniu prawniczym w~Waszyngtonie, zauważyłem, jak wiele z~tego: wszystkie szkolenia obejmowały kontakt fizyczny, od unoszenia ludzi bezwładnych, aż do po prostu siedzenie przyciśnięci do innych w~przepełnionych pokojach.

 Zastanawiam się, czy jednym z~powodów dla drażliwości/wybrednych pokarmów/obejmowania słabości jest znaczenie kobiet w~ruchu, choć jest to nieco mylące, ponieważ kobiety prawie nigdy nie stanowią większości na dużych spotkaniach i~często stanowią co najwyżej jedną trzecią ludzie w~pokoju. Z drugiej strony są to często najwybitniejsze organizatorki i~uczestniczki. Czy lepiej powiedzieć, że dominuje kobieca wrażliwość, czy też, że styl procesu konsensusu w~interakcji ma tendencję do zachęcania do czerpania z~wrażliwości, które w~Stanach Zjednoczonych historycznie były kojarzone ze sposobem, w~jaki kobiety wchodzą ze sobą w~interakcje, niż ze sposobem, w~jaki robią to mężczyźni? Jest w~dużej mierze, ale nie ściśle, pozbawione seksualności. Często uczucie, przynajmniej jeśli nie należy do jakiejś grupy tożsamości seksualnej, jest takie, że ktoś powinien się zachowywać (przynajmniej publicznie), jakby seks nie był szczególnie ważny, po prostu jeden z~aspektów ogólnie rozumianej fizyczności.

 Oczywiście wszystko to różni się w~zależności od subkultury. Przez wiele lat w~ABC No Rio, anarchistycznym centrum społecznym w~Lower East Side, istniał -- poza zwykłym magazynem zinów, komputerami i~tym podobnymi -- siłownia używana przez członków grupy zwanej RASH, ,,Red Anarchist Skinheads'' (Czerwoni Anarchiści -- Skinheadzi).

 Gra pożądania i~wzajemnej zależności pojawia się na różnych subtelnych poziomach. Oto fragment z~tego samego notatnika, niedługo potem:

\medskip
\noindent [Wyciąg zeszytu, lipiec 2000]

\noindent \textit{Papierosy:}
\medskip

 Wielu aktywistów pali. Wydaje się, że większość starszych osób paliła w~pewnym momencie swojego życia. Zawsze wydawało mi się to trochę niestosowne, na A16, widzieć te wszystkie idealistyczne dzieciaki blokujące ulice z~papierosami zwisającymi im z~ust; zwłaszcza nastoletnie dziewczyny siedzące wokół i~palące od siebie papierosy. Ale w~rzeczywistości jest to raczej właściwe, ponieważ powoduje ciągłą mobilizację poczucia potrzeby, dyscypliny, dzielenia się i~pragnienia (,,wspólnota uzależnień'', jak ją nazywałam, która wiąże wszystkich palaczy). Zwykle na trzech lub czterech aktywistów, którzy palą lub mogą, przypada jeden, który faktycznie ma paczkę. Kevin został obsadzony w~tej roli z~Scully et al. w~zeszłym tygodniu. Dystrybucja papierosów, odpalanie innych itd. staje się ciągłym, upragnionym upadkiem autonomii -- ja, kiedy paliłem, to była zasada, żeby \textit{nigdy }nie dać się złapać w~sytuacji, w~której skończyłbym się i~nie byłem w~stanie kupić więcej, ale tutaj jest odwrotnie. Jest się zależnym od dobrej woli wspólnoty i~dzielenia się tym, czego \textit{naprawdę }pragnie się najpilniej na świecie, przynajmniej w~tej chwili.

 Szczególnie duży odsetek wegan pali.

 Przypomina mi to raczej historię, którą słyszałem o Martinie Lutherze Kingu. W rzeczywistości był nałogowym palaczem, ale na początku był przekonany, że to, że gdyby kiedykolwiek widziano go publicznie, byłoby to niewłaściwą lekcją dla młodych ludzi. Niekończąca się dyscyplina, ale z~niekończącym się pragnieniem czającym się za publiczną fasadą. Nie trzeba dodawać, że nikt nie pali na spotkaniach ani w~ogóle w~pomieszczeniach. Tak więc po zakończeniu spotkania zwykle grupki ludzi natychmiast wybiegają na papierosa, siedząc na betonie, by skręcić tytoń, obijając się tyłkami, zaciągając się kilkoma zaciągnięciami lub rozdając papierosy.

 Inne narkotyki wydają się odgrywać mniej znaczącą rolę, ponieważ nie są tak uzależniające. Dlatego cała dynamika pożądania i~wspólnoty nie wchodzi w~grę. Moje notatki w~tym przypadku kontynuowały:

\medskip
\noindent \textit{Inne środki:}
\medskip

 To zależy od sceny. Zioło jest okazjonalne, ale zaskakująco rzadkie. Jest używane mniej więcej w~stopniu, jakiego można by oczekiwać od młodych ludzi z~tej samej klasy lub pochodzenia społeczno-ekonomicznego. Piwa jest całkiem sporo, często w~barach. Ecstasy jest popularna wśród typów raverów, z~którymi wyraźnie pokrywają się pewne części aktywistycznej sceny. Oczywiście podczas akcji ulicznych narkotyki są całkowicie złą wiadomością i~zawsze przypomina się, aby ich nie przynosić: ,,Nawet jeśli rzucisz jointa w~momencie, gdy pojawiają się gliniarze, ktoś nam to przypnie''. Tak więc sprowadzenie narkotyków na akcje byłoby aktem całkowitego braku solidarności. Pojawienie się aktywisty podczas akcji kompletnie pijanego lub naćpanego jest albo znakiem, że nikt nie chciałby być z~nimi w~grupie afinicji, albo, z~mojego doświadczenia, najczęściej na znak, że dany aktywista osobiście się rozpada i~potrzebuje pomocy. Jeśli chodzi o paranoję narkotykową, istnieją różne poziomy kontekstu i~doświadczenia historycznego: przypomina mi się czas, kiedy robiłem drinka podczas pokazu filmu z~niektórymi byłymi Czarnymi Panterami. Kiedy zaproponowałem, żebym kupił kokę, jedna zaskoczona kobieta natychmiast mnie poprawiła: -- Proszę! Mów ,,Coca Cola!'' -- Oczywiście byli to ludzie przyzwyczajeni do ciągłej obserwacji w~czasach, gdy naloty narkotykowe często umieszczały aktywistów w~więzieniach. Nigdy nie słyszałem czegoś takiego wśród anarchistów w~dzisiejszych czasach: paranoja jest skierowana na inne rzeczy. W rzeczywistości na mniejszych imprezach lub imprezach ulicznych, które i~tak są w~połowie drogi do rave, stosunek do narkotyków może być bardzo luźny. Jeden z~przyjaciół opowiedział mi długą historię o tym, jak został przeszukany i~zamknięty w~więzieniu na noc po imprezie RTS Times Square, ale po wyjściu odkrył, że zapomniał, że przez cały czas miał skręta w~bucie. Ale to są ,,tymczasowe strefy autonomiczne'' innego rodzaju.

 Jedynym tematem, który w~tym wszystkim powraca bez końca, jest ,,autonomia'', jednocześnie największa wartość anarchistyczna i~największy dylemat. Pewne formy autonomii -- wyizolowany indywidualizm głównego nurtu amerykańskiego społeczeństwa, z~jego samotnymi przyjemnościami -- są dokładnie tym, przeciwko czemu się buntuje. A może, można powiedzieć, pytanie brzmi, jak zrównoważyć autonomię, solidarność i~wolność. Na przykład Cornelius Castoriadis (1987, 1991) zdefiniował ,,autonomię'' jako zdolność społeczności do życia wyłącznie według reguł, które sami stworzyli i~posiadanie prawa do ciągłego weryfikowana. Dla wielu anarchistów wolność wydaje się oznaczać zdolność do tworzenia nowych społeczności i~więzów wzajemnych zależności, mniej lub bardziej na miejscu, oraz poruszania się między nimi tam i~z powrotem według własnego uznania. Akcja, impreza, piknik, taniec, wszystkie mogą być tymczasowymi strefami autonomicznymi, w~których pragnienia się łączą, a skok wiary związany z~zaufaniem nieznajomym staje się dużą częścią przygody, nawet wtedy, gdy policja jest nieobecna, co, jak zobaczymy, zdarza się rzadko, ponieważ policja ma zauważalną tendencję pojawiać się, gdy spotykają się anarchiści. Dylematy stają się jednak znacznie bardziej dotkliwe, gdy podejmowane są -- jak to zwykle bywa -- próby przekształcenia TAZ w~PAZ, przejścia od tymczasowych do bardziej trwałych, permanentnych stref autonomii.

 W następnej sekcji pozwólcie, że opowiem trochę o bardziej stałych przestrzeniach aktywistów. Jak zobaczymy, prawie nigdy nie są one całkowicie, całkowicie trwałe. Każda przestrzeń musi zostać do pewnego stopnia podbita, a większość jest niemal natychmiast oblegana.

\section{Krajobrazy aktywistyczne}

 W mieście takim jak Nowy Jork anarchistyczne przestrzenie często mają jakość archipelagu. W niektórych dzielnicach znajdują się stosunkowo gęste skupiska squatów, ogrodów społecznościowych, centrów społecznych lub społecznościowych, radykalnych księgarni/infoshopów i~innych mniej lub bardziej przyjaznych instytucji: spółdzielni, restauracji wegetariańskich, sklepów z~używanymi rowerami, awangardowych teatrów, przyjaznych kościołów, a nawet kawiarni i~barów, w~których można spotkać aktywistów.

 Czasami jest jakieś centrum; czasami są bardziej rozproszone. Pomiędzy początkiem 2000 a końcem 2001, w~czasach rozkwitu nowojorskiego DAN, w~nowojorskim Lower East Side znajdowało się centrum sceny aktywistycznej. Był to lokalny ośrodek kultury o nazwie Charas El Bohio, mieszczący się w~dawnym budynku szkolnym. Charas El Bohio stał w~centrum splotu instytucji, z~których prawie wszystkie zostały zdobyte w~wyniku długotrwałej walki społeczności.

 Historia Charas jest dość ciekawa. Technicznie rzecz biorąc, ,,Charas'' to nazwa grupy społecznej, a ,,El Bohio'' odnosiło się do budynku. Grupa społeczna została założona w~1965 roku przez grupę byłych członków gangu Portorykańczyków. Kiedy po raz pierwszy go stworzyli, pracowali z~Buckminsterem Fullerem nad budową kopuł geodezyjnych dla biednych, ale wkrótce zostali sponsorami festiwali kulturalnych. Z kolei El Bohio powstało, gdy w~1979 roku niektórzy z~nich, współpracując z~byłymi Panterami, zasiedlili Christadorę, piękne, ale wtedy opuszczone osiedle położone bezpośrednio na wschód od Tompkins Square Park i~górującą nad okolicą. Doprowadziło to w~końcu do kłótni z~władzami miasta, które ostatecznie były skłonne rozwiązać sprawę, oferując lokatorom opuszczoną szkołę w~dół ulicy, dawną S.P. 64. Budynek był pusty od 1975 roku, znajdował się wówczas w~stanie bliskim zawalenia się i~był zamieszkały głównie przez osoby uzależnione od heroiny. Umowa została sformalizowana za pomocą pewnego rodzaju dżentelmeńskiej umowy: Christadora została sprzedana prywatnemu deweloperowi i~ostatecznie stała się kosztownym kondominium, a Charas szybko zaczął odbudowywać nowo nazwany El Bohio, oferując wolną przestrzeń artystom i~rzemieślnikom w~zamian za prace przy renowacji okien i~dachy. Wkrótce miejsce to stało się centrum artystów, grup teatralnych i~tanecznych (którzy wynajmowali sale na próby za znikome opłaty) i~gościło wszelkiego rodzaju grupy polityczne i~wydarzenia. Charas stał się także skutecznym centrum politycznym sieci squatów i~ogrodów społecznościowych w~okolicy Tompkins Square, w~większości założonych w~podobnym czasie -- lata 70. i~wczesne 80. -- kiedy większość dzielnicy została porzucona.

 Ta historia była opowiadana wiele razy (Abu-Lughod 1994; Mele 2000; Tobocman 1999). W latach 70. XX wieku trzy czwarte zasobów mieszkaniowych w~okolicy zostało porzuconych przez właścicieli, przejętych przez miasto za niepłacenie podatków. Nowojorska scena punkowa tak naprawdę wyłoniła się właśnie z~tego czasu i~miejsca, a jej aura miejskiej apokalipsy i~rozpaczy miała wszystko wspólnego z~poczuciem miasta, które dosłownie popadało w~ruinę, zostało oddane szczurom, ćpunom, i~podpalaczom. W odpowiedzi wielu artystów, skłotersów, aktywistów i~nowych imigrantów odzyskało budynki i~tereny zielone, a te z~kolei wkrótce stały się przedmiotem intensywnych walk -- czasami niemal wojskowych -- pod koniec lat 80. i~na początku lat 90., kiedy teren znów zaczął być gentryfikowany. Najsłynniejszym incydentem oczywiście były ,,zamieszki na Tompkins Square'' z~1988 i~1989 roku, opór wobec wysiłków policji mających na celu usunięcie obozowisk bezdomnych z~samego parku. Jednak równie wyczerpujące były bitwy o okoliczne skłoty. Pojawiły się tajemnicze pożary, do których straż pożarna odmówiła przyjazdu, nagłe naloty o świcie gliniarzy wspieranych przez helikoptery i~transportery opancerzone. W niektórych przypadkach miały miejsce przedłużające się oblężenia tak gorzkie, że minęły lata, zanim policja ponownie podjęła próbę usunięcia skłotu. Ostatecznym rezultatem było to, że do 2002 roku dwadzieścia dwa squaty zostały zredukowane do jedenastu, jednak w~tym roku miasto w~końcu się poddało i~pozwoliło pozostałym skłoterom uzyskać tytuł do swoich domów.

 Historia ogrodów społecznościowych była podobna: archipelag terenów zielonych odzyskanych z~pierwotnie opuszczonych parceli pełnych szczurów i~śmieci. Zostały zasadzone i~utrzymywane przez lokalne kolektywy lub organizacje sąsiedzkie; następnie zostały oblężone, gdy obszar zaczął się gentryfikować. More Gardens!, grupa aktywistów oddana ich obronie, spotykała się regularnie w~Charas, a ogrodnicy społeczności również wykorzystywali duże pokoje na parterze w~Charas do planowania konkursów. Okolica słynęła z~pięknych wiosennych i~zimowych festiwali, z~ich wyszukanymi kostiumami, lalkami, pokazami świetlnymi, muzyką i~przedstawieniami dramatycznymi -- tak jak przed akcjami i~marszami te same pomieszczenia były używane do malowania sztandarów i~składania lalek. Sam budynek zawsze był pełen sztuki, ogromne malowane rzeźby lub pływaki w~tajemniczy sposób pojawiały się w~holu i~znikały kilka dni później, było tam audytorium, z~którego można było korzystać na przedstawienia i~sale, które można było wynająć na spotkania, zawsze za minimalne opłaty, oraz cały budynek można wynająć na imprezy, pod warunkiem, że nie potrwają zbyt późno w~noc. Dla aktywistów był to nieoceniony zasób.

 Charas stała dokładnie na wschód od parku, na Ósmej Ulicy między Alejami B i~C. Jeśli wyruszyło się z~Charas i~jechało na południe wzdłuż Alei B, minęło się szereg aktywistycznych punktów orientacyjnych: kilka popularnych restauracji wegetariańskich, jeden bardzo duży i~rozbudowany ogród społeczności oznaczony przez strzelistą rzeźbę wykonaną z~czegoś w~rodzaju zainscenizowanej piramidy ozdobionej wszelkiego rodzaju wyrzuconymi pluszami i~podobnymi bibelotami, a następnie Blackout Books, anarchistyczny infoshop znajdujący się w~sklepie obok centrum Hare Krishna, po drugiej stronie ulica od unii kredytowej i~szeregu sklepów z~używanymi rzeczami. Blackout było miejscem, do którego można było wpaść o każdej porze dnia i~ciekawie porozmawiać. Po drugiej stronie Houston Street wchodziło się do szybko gentryfikującej się okolicy pełnej modnych hipsterskich knajpek, wyłaniających się z~w dużej mierze biednej latynoskiej dzielnicy, z~kilkoma starożytnymi żydowskimi synagogami i~firmami, aby dotrzeć do ABC No Rio. ABC zostało nazwane na cześć squatu, który był miejscem słynnej bitwy, i~początkowo był zarówno squatem, jak i~przestrzenią artystyczną. Później został zalegalizowany jako centrum społeczne, to znaczy skłoterzy musieli utrzymywać budynek pod warunkiem, że już tam nie mieszkają -- choć wydawało się, że zawsze w~skomplikowanych warunkach, które sprawiały, że ich obłożenie było raczej niepewne. Na dole ABC gościło pokazy sztuki, ale przede wszystkim stało się centrum lokalnej sceny punkowej i~hardcore. Na piętrze znajdowała się biblioteka zinów, wolna przestrzeń komputerowa, ciemnie, pracownia sitodruku i~kuchnia, z~której kilka razy w~tygodniu korzystał nowojorski oddział Food Not Bombs. FNB nurkowałby w~śmietniku za jedzeniem -- głównie ze świeżą żywnością, często w~plastiku lub opakowaną, wyrzuconych przez wielkie instytucje -- i~wytworzyłby wegetariańskie posiłki, które później rozdawaliby za darmo z~tyłu parku przy Tompkins Square.

 Istniało również szereg innych przyjaznych instytucji: Księgarnia Bluestockings Womyn na Allen Street, Liga Oporu Wojny na Sullivan i~Lafayette -- centrum pacyfistyczne, które istniało od lat dwudziestych i~miało cały własny budynek (,,Pentagon Pokoju, '' to czasami nazywano), dom kultury przy Szóstej Ulicy, radykalne przestrzenie teatralne, kościół św. Marka (gdzie jednym z~księży był skłoter). Ale kluczowe instytucje, na których działacze wiedzieli, że zawsze mogą polegać, wszystkie były do  pewnego stopnia wątłe. Podobnie jak squaty, zostały one wywalczone walką, zazwyczaj bezpośrednią akcją, utrzymywaną pod wielką presją instytucji państwowych i~wszystkim groziło nieustanne niebezpieczeństwo wywiezienia. Prawie każdy z~nich wydawał się otoczony poczuciem rozpaczliwego dramatu; jeśli można na nich polegać w~tym sensie, że są wyraźnie przyjazne dla anarchistów, nie można na nich polegać w~tym sensie, że można być pewnym, że nadal będą tam za sześć miesięcy lub rok. i~rzeczywiście, do 2002 roku, zaledwie kilka lat po założeniu DAN, cała sieć w~dużej mierze się rozpadła.

 Doskonałym przykładem były ogrody gminne. Administracja Guilianiego, po objęciu urzędu w~1994 roku, niemal natychmiast rozpoczęła szeroką ofensywę przeciwko całej sieci ogrodów społecznościowych, definiując je na nowo jako puste działki i~wprowadzając plan licytacji 741 z~nich w~całym mieście w~celu rozwoju ,,niedrogich mieszkań'' (W jednym z~tygodniowych wystąpień radiowych Giuliani wyjaśnił, że jest to atak na samą zasadę wspólnej własności: ,,To jest gospodarka wolnorynkowa'', powiedział ,,Epoka komunizmu się skończyła''). Wywiązała się długa walka, osiągając szczyt w~latach 1998 i~1999 z~licznymi akcjami bezpośrednimi, w~których aktywiści More Gardens! siadali przed buldożerami, a także jedna akcja Reclaim the Streets, która zamknęła Avenue A na kilka godzin, i~druga, która doprowadziła do aresztowania sześćdziesięciu dwóch osób podczas blokady na West Side Highway. Kilka ogrodów zostało zniszczonych, ale w~końcu Giuliani poniósł jedną z~nielicznych większych porażek swojej administracji, kiedy koalicja bogatych mecenasów interweniowała, by wykupić kilka ogrodów w~celu ich zachowania, a wkrótce potem stanowy prokurator generalny pozwał miasto, aby zapobiec kolejnym licytacji, ponieważ naruszało to obowiązujące w~mieście przepisy, zgodnie z~którymi na każdy tysiąc mieszkańców przypada co najmniej hektar terenów zielonych.

 To było wielkie zwycięstwo, ale aktywista szybko uczy się, że żadne zwycięstwo nie jest nieodwracalne. Ponadto każdemu zwycięstwu towarzyszą straszliwe, tragiczne straty. Kolejnym celem Giuliani była sama Charas. W rzeczywistości, jego zniszczenie wkrótce stało się rodzajem obsesji jego administracji. Tak przynajmniej wydawało się lokalnym działaczom mieszkaniowym. Przez cały okres istnienia DAN budynek był prawnie oblężony. Ponieważ jego status opierał się na tym, co w~rzeczywistości było dżentelmeńską umową z~rządem -- budynek był dzierżawiony od rządu za dolara rocznie -- administracja Guilianiego mogła zerwać umowę i~zlicytować go, co zrobiła, na tej samej aukcji, 20 lipca 1998 roku tak jak kilka największych ogrodów wspólnotowych. Sama aukcja stała się czymś w~rodzaju legendy wśród aktywistów Lower East Side, którzy wykorzystywali wszelkie możliwe środki, aby ją zakłócić, od protestów na zewnątrz, przez fałszywych kupujących próbujących podbić cenę w~środku, po wypuszczenie na aukcję dziesięciu tysięcy świerszczy na piętrze budynku, co zdołało oczyścić biura, ale tylko tymczasowo. Ostatecznie tytuł został przekazany anonimowemu nabywcy, który -- pomimo wysiłków miasta mających na celu ochronę swojej tożsamości -- wkrótce okazał się Gregg Singerem, drobnym deweloperem z~Upper West Side. Singer był teraz technicznie właścicielem budynku (El Bohio), a Charas jedynie jego najemcą. Natychmiast ruszył do eksmisji, ale było to trudne: jego ręce były związane umową o ograniczonym użytkowaniu, która pozwalała na używanie budynku tylko do ,,użytkowania obiektów komunalnych''. W rezultacie, aby wydalić Charas, musiał wykazać, że miał nowych lokatorów, którzy również zamierzali wykorzystać budynek do celów kulturalnych lub użyteczności publicznej. Z jego punktu widzenia problemem prawnym -- przynajmniej do czasu zakończenia długiego procesu odwołań i~prawnych potyczek -- było więc znalezienie legalnej instytucji kulturalnej, która zechciałaby wydzierżawić budynek, nawet jeśli wiedzieli, że oznaczałoby to eksmisję ośrodka społeczności sąsiedzkiej. Było to prawie niemożliwe, ale oznaczało to, że cała społeczność aktywistów, która korzystała z~Charas, podlegała ciągłym ,,ostrzeżeniom Singera'': nowy właściciel był zobowiązany do ogłoszenia wizyt z~potencjalnymi najemcami z~trzygodzinnym wyprzedzeniem, więc Charas natychmiast wysyłała wiadomość na listservy aktywistów, a także telefonicznie, wzywając wszystkich do wpadnięcia na podwórko przed Charas na demo, łapiąc plakaty pozostawione w~tym celu w~lobby Charas, wyjaśniając gościom -- powiedzmy, pastorowi jakiegoś kościoła w~Harlemie szukającego miejsca na ćwiczenia chóru, lub jakieś grupie charytatywnej poszukującej biur -- co właściwie się dzieje.

 Takie podejście było z~pewnością skuteczne. Singer nigdy nie znalazł legalnego najemcy, który chciałby zastąpić Charas. Ale w~końcu udało mu się wypędzić Charas innymi sposobami. Po procesie, w~którym ława przysięgłych orzekła jednogłośnie na korzyść Charas i~przeciwko Singerowi, inny sędzia (o którym wszyscy zakładaliśmy, że musiał zostać przekupiony, choć oczywiście nie możemy tego udowodnić) unieważnił wyniki, uzasadniając, że sprawa nigdy nie powinna zostać postawiona przed ławą przysięgłych i~po prostu przekazał majątek Singerowi. Lokalni dzicy lokatorzy byli przygotowani do rozpoczęcia wielkiej okupacji i~obrony -- argumentując, że każdy budynek porzucony bez walki ośmiela miasto do przejścia na inny -- ale mieszkańcy Charas ostatecznie zawetowali ten plan, ponieważ jako organizacja społeczna, ich jedyna szansa na zdobycie innej przestrzeni polegała na utrzymywaniu jakichś relacji z~miastem, a zaciekła bitwa z~pewnością by to uniemożliwiła. Dlatego po (w dużej mierze uroczystym) zamknięciu budynek został zabity deskami i~- w~momencie pisania tego tekstu pięć lat później -- pozostaje pusty, ponieważ nowy właściciel nadal nie był w~stanie znaleźć nikogo chętnego do wynajęcia i~jeszcze nie otrzymał zezwolenia na jego zburzenie. Organizacja Charas pozostaje bezdomna.

 Na rynku nieruchomości, takim jak Nowy Jork, jedyną alternatywą dla okupacji jest uzależnienie się od kaprysu jakiegoś bogatego mecenasa -- los, który charakteryzuje historia Blackout Books. Blackout był kolektywem; wszyscy, którzy tam pracowali, byli wolontariuszami. Był administrowany demokratycznie i~był dość skuteczny w~zapewnianiu przyjaznego i~przyjaznego środowiska dla zainteresowanych anarchizmem w~sąsiedztwie. Problem polegał na tym, że sam sklep był opłacany przez zamożną starszą kobietę z~sąsiedztwa, która płaciła cały miesięczny czynsz. Pewnego dnia w~2000 roku właścicielka podwoiła czynsz, a patronka nagle ogłosiła, że  zawsze miała nieco ambiwalentny stosunek do projektu, ponieważ na swój sposób uczynił ją współwinną gentryfikacji Lower East Side i~wycofała wsparcie. Blackout miał miesiąc na stworzenie zupełnie nowej bazy finansowania. Członkowie kolektywu mówią mi, że prawdopodobnie byliby w~stanie to zrobić, mimo że sam sklep na pewno nie przynosił zysków. Ponieważ jednak wydarzyło się to w~czasie wielkiej wewnętrznej niezgody na temat relacji Blackouta i~otaczającej go społeczności, wysiłek ostatecznie się rozpadł. Po około roku Blackout pojawił się ponownie, w~osłabionej formie, jako Mayday Books, w~holu awangardowej przestrzeni teatralnej zwanej Teatrem Nowego Miasta na Pierwszej Alei: w~dużej mierze dlatego, że właścicielka teatru była skłonna oddać im miejsce za tylko nominalny czynsz. Jest to jednak instytucja, znowu bardzo zależna od kaprysu jednej osoby. Ich patron od czasu do czasu irytuje sposób, w~jaki księgarnia działa jako miejsce spotkań aktywistów -- prawie zawsze ktoś wpada, czyta, rozmawia, szuka wydarzeń lub informacji -- i~już kilkakrotnie kazał im się spakować. Jak dotąd przynajmniej takie kryzysy zwykle rozwiązywane są po tygodniu lub dwóch, a ci, którzy rozpoczęli spanikowane poszukiwania nowej lokalizacji, czują, że mogą się ponownie ustatkować\footnote{Do czasu ostatecznej edycji (2008) Blackout został w~rzeczywistości usunięty i~teraz już nie istnieje.}.

 Alternatywnie można stworzyć własną bazę finansowania. Ale to samo w~sobie pochłania ogromną ilość czasu i~energii aktywistów. Independent Media Center (IMC -- Niezależne Centrum Medialne ), które otworzyło nad importerem orientalnych dywanów na Dwudziestej Dziewiątej Ulicy między Madison i~Park w~2000 roku, było, podobnie jak Blackout, początkowo zależne od bogatego mecenasa -- w~tym przypadku wydawcy magazynu hakerskiego, który wcześniej używał tego biura jako przestrzeni hakerskiej i~nadal płacił czynsz po wprowadzeniu się IMC. W końcu, jak zwykle, właściciel podwoił czynsz, a mecenas wycofał swoje wsparcie. Kolektywowi udało się utrzymać tę przestrzeń, ale tylko za cenę spędzania około jednej trzeciej każdego spotkania na kwestie związane z~finansowaniem, a ostatecznie na zamieszczanie ogłoszeń w~bezpłatnej gazecie i~narażanie w~ten czy inny sposób wielu z~ich pierwotnych zasad. Inną szczególnie wymowną przestrzenią jest ABC No Rio, jak wspomniałem, założone jako przestrzeń artystyczna i~squat w~1980 roku, które otrzymało milczącą zgodę z~miastem, że mogą utrzymać się jako dom kultury, jeśli mieszkańcy się wyprowadzą. Niemal natychmiast po zawarciu tej umowy przybyli inspektorzy miejscy i~oświadczyli, że budynek wymaga naprawy wartej osiemdziesiąt tysięcy dolarów, aby zachować zgodność z~zasadami, i~jeśli nie zdołają zebrać pieniędzy i~przeprowadzić naprawy w~ciągu dwóch lat, budynek zostanie zamknięty, a ABC eksmitowany. Pokazy punkowe i~inne świadczenia odbywały się tak daleko, jak Polska, aby zebrać pieniądze, ale ponownie w~rezultacie kolektyw stworzony w~celu przeciwstawienia się kapitalizmowi, świadczenia bezpłatnych usług i~zapewnienia ogólnej alternatywy dla gospodarki gotówkowej, został zmuszony do spędzania bardzo dużej części swojego czasu na pozyskiwanie funduszy. Skłotersi często opisują podobne historie: nawet, gdy są zalegalizowani, inspektorzy budowlani są znacznie bardziej rygorystyczni w~żądaniach niż kiedykolwiek byli dla innych obraźliwych właścicieli w~dzielnicy.

 Utrata najpierw Blackout, a potem Charas sprawiła, że  społeczność aktywistów Loisaidy została zdecentralizowana i~bezdomna, znalezienie miejsca na duże spotkania podczas mobilizacji stało się nieustannym problemem. Mimo to Lower East Side nigdy nie była jedynym takim skupiskiem. Podobne archipelagi ośrodków i~hangoutów dla aktywistów lub aktywistów można znaleźć wokół samego biura Independent Media Center, inny, nieco inny, w~Harlemie, inny dość rozległy, z~centrami społecznymi, skłotami i~ogrodami społecznymi, w~Bronksie, inny w~Dumbo i~tak dalej. Ale także wszystkie z~nich mają to samo poczucie bycia enklawami pod ciągłym atakiem.

 Ta sama niepewność, nawiasem mówiąc, jest odczuwana również w~innych instytucjach aktywistów. Pirackie stacje radiowe to przestrzenie zdobyte od FCC; mają tendencję do zamykania się. Nawet Pacifica, najbardziej przyjazny serwis medialny, znajdował się w~ciągłym niebezpieczeństwie po ,,przewrocie bożonarodzeniowym'' pod koniec 2000 roku, kiedy to został skutecznie przejęty przez frakcję prokorporacyjną. Wielu członków zostało usuniętych i~zdelegalizowanych, a pozostali radykałowie w~większości zepchnięci na margines. Potrzeba było dwóch lat nieustannej mobilizacji, akcji bezpośrednich, lobbingu i~propagandy, aby w~końcu przywrócić jej pierwotny zarząd. Trzeba bronić całego wolnego lub nawet półwolnego terytorium. Jednym z~rezultatów jest wzmocnienie nieco nieporządnego, zaimprowizowanego odczucia wszystkich przestrzeni. Wszystko jest lekko niedokończone lub w~trakcie budowy. Jak zobaczymy, jest to częściowo estetyka; ale częściowo dzieje się tak, ponieważ wszystko w~takich przestrzeniach jest albo w~trakcie zajmowania albo zabierania.

\section{Trzy imprezy}

 Mając trochę poczucia aktywistycznych krajobrazów, pozwolę sobie zakończyć ten -- z~konieczności dość schematyczny -- rozdział, umieszczając w~nich niektórych ludzi. Poniżej przedstawiam ponownie fragmenty moich zeszytów. Zostały zapisane tego samego weekendu wiosną 2001 roku, na zakończenie przedłużającego się strajku pracowników Muzeum Sztuki Nowoczesnej w~Midtown. Robotnicza grupa robocza DAN odegrała dużą rolę we wspieraniu oficjalnej linii pikiet UAW za pomocą marionetek, teatru ulicznego, dodatkowych blokad i~propagandy, a kiedy pracodawcy w~końcu się ugięli, poczuliśmy, że jest to równie nasze zwycięstwo. Uroczystość odbyła się w~biurach innego, nieco niecodziennego lokalnego UAW, związku muzyków, w~ich biurach w~śródmieściu. Później wielu z~nas poszło na podobno rave na dachu w~Queens, a następnego dnia odbyła się Impreza Reclaim the Streets! -- czyli nie impreza uliczna, ale przyjęcie zorganizowane przez ludzi z~RTS, pewnego rodzaju zbiórka pieniędzy, która odbyła się niedaleko, na Czterdziestej Drugiej Ulicy. W tym czasie miałem zwyczaj sporządzania podsumowań prawie wszystkiego, co się wydarzyło; wyniki mogą dać trochę wyczucia tekstury i~jakości takich wydarzeń: 

\bigskip
\noindent \textit{Impreza Zwycięstwa w~MOMA}

\noindent \textit{(Kwatera Główna UAW Musician's Union HQ, 48th Street między 8 a 9) }

 Przyjeżdżam dość późno, bo o 22. Większości jedzenia już nie ma. Pokój jest zaśmiecony talerzami sałatki ziemniaczanej, sałatki coleslaw, sałatek potluck w~ogromnych drewnianych miskach, pudłami, w~którym kiedyś znajdowały się dwumetrowe lalki, pizza, piwo. Gra zespół. Obrazy szczurów są wszędzie. Wisi szczurza piniata, a na stole kolejny, plastikowy szczur. Szczury są uniwersalnym wizerunkiem akcji strajkowych w~Nowym Jorku: związki dzielą kilka gigantycznych nadmuchiwanych szczurów, z~których największy ma około dwóch pięter, które można dostarczyć na linie pikiet wokół miasta. Większość z~nich każdego dnia jest gdzieś używana (są trzymane w~nocy w~magazynie po drugiej stronie rzeki w~New Jersey). Strajkujący MOMA mieli nawet strajkowy zin o nazwie \textit{Rat Poison}, który jest tutaj wyraźnie eksponowany. Na przyjęcie, kiedy wchodzę, bierze udział może ze sto osób i~wielu tańczy; wykonuje pociąg do ,,Love Train'' z~dodatkowymi chórkami i~w ogólnej zabawie.

 Miejsce akcji: instytucjonalne. 
 
 Przypomina mi dobitnie szkołę podstawową. Te same rodzaje tanich stołów i~składanych krzeseł. Kościelne pokoje socjalne też takie mają, podobnie jak stare radykalne przestrzenie, takie jak Liga Oporu Wojny, czy też siedziba Partii Komunistycznej na Dwudziestej Trzeciej Ulicy, która zawsze oferuje miejsce na imprezy i~pokazy, i~z której czasami korzystamy, zawsze z~lekkim zakłopotaniem. Myślę, że to definiuje tanią przestrzeń grupową: wszystko minimalne, składane stoły, składane krzesła, projekty niezmienione od lat 50. czy 60.. Nie wiem, ile godzin mojego aktywistycznego życia spędziłem na układaniu i~składaniu krzeseł po spotkaniach w~kościołach i~salach związkowych.

 Celebranci -- pracownicy MOMA i~ich zwolennicy -- bardzo różnili się kształtem i~rozmiarem, wiekiem i~pochodzeniem, od jednej malutkiej pięćdziesięciolatki, która wyglądała jak bibliotekarka, po ogromne palooki i~hipsterki w~czarnych strojach z~fantazyjnymi okularami. (Związek obejmował wszystkich, od malarzy po kasjerów księgarni).

 Punktem kulminacyjnym imprezy było zniszczenie szczurzej piniaty, do której podeszli w~tradycyjny sposób, imprezowicze z~zawiązanymi oczami uderzali w~nią kijem. Dużo wiwatów. Podczas kolejnego tańca lovetrain jeden Azjata niósł ze sobą szczątki szczura, wyrzucając je w~powietrze w~geście podboju. Impreza nie trwała długo -- jak mi powiedziano, zaczęła się około 18:30. Chociaż zespół wystartował dopiero około 22, ich set trwał może godzinę. Był to jeden z~tych perfekcyjnie dobrych zespołów wykonujących covery, które grają ogromną gamę rzeczy, od Motown po reggae, jeśli trzeba. (-- To niesamowita rzecz w~Nowym Jorku -- zauważa Rufus. -- Nawet złe zespoły tutaj są dobre. -- Z wyjątkiem, zgadzamy się, dla tych Teamsters z~tymi wszystkimi gitarami elektrycznymi na paradzie podczas Święta Pracy. Byli raczej okropni.)

 Potem większość hardcorowych aktywistów wybiera się na imprezę na dachu gdzieś w~Queens: według informacji Rufusa niezupełnie rave: muzyka będzie bardziej industrialna. Grupki sześciu lub siedmiu osób wciąż kierują się do metra. Około północy kończę w~samochodzie zajmowanym głównie przez aktywistów -- także wielu różnych, od anarchistów przez ludzi pracy po zagorzałych ISO. Wysiadamy w~przemysłowej części miasta i~podążamy za kimś o imieniu Alex z~kolektywu Lower East Side, który przyniósł ściągniętą mapę.

\bigskip
\noindent \textit{Impreza na dachu w~Queens}

 Impreza odbywa się w~domu Jessiki Rockstar, znanej mi głównie jako członek jednego z~zespołów wideo I-Witness, które monitorują policję podczas akcji. (W rzeczywistości jej imię nie brzmi ,,Rockstar'', ale coś całkiem podobnego, a ktoś mi powiedział, że faktycznie zmieniła to oficjalnie na ,,Rockstar'', ponieważ uważa, że  naprawdę powinna być jedną z~nich). JR mieszka w~jeszcze jednym z~tych radykalnych, półprzemysłowych przestrzeni, w~których żyje tyle osób z~akcji bezpośredniej. Budynek jest wysoki na kilka pięter, na obszarze pełnym magazynów, parkingów dla wywrotek, ulic pełnych ciężarówek i~pojazdów użytkowych różnego rodzaju. Po drodze mijamy kilka dużych hal produkcyjnych, prawdopodobnie stolarni lub przemysłu lekkiego, z~wciąż włączonymi światłami i~pracownikami w~środku, mimo że w~piątkowy wieczór jest tuż po północy. Ulice w~tej części miasta są szerokie, często zakończone płotami. Patrząc z~góry, poza odsłanianiem kolejnego pięknego pejzażu Manhattanu, nie było nic poza ogromnymi, przypominającymi bloki, płaskimi, magazynami/przemysłowymi przestrzeniami. To dość daleko od stoczni marynarki wojennej, ale budynki są z~tej samej formy, z~ogromną masywnością pustaków, windami towarowymi, wielkimi pustymi korytarzami, okazjonalnymi drzwiami, ciężkimi metalowymi schodami. Budynek JR miał pięć pięter i~parę otwartych drzwi. Przypuszczam, że gdzieś tam mieszkała nasza gospodyni. (Reakcja Alexa: Och, czyż nie organizowaliśmy się stąd w~pewnym momencie?) Na niektórych pustych ścianach wisiały naklejki Nadera; raczej nie pasujące przy ich nacisku na zieleń, ponieważ nigdzie nic nie rosło.

 Dach był ogromny, z~pewnością wielkości jednego bloku, i~pełen ekranów, na których rzucano głównie koncepcyjne kształty i~kolory; tego rodzaju rzeczy, które widzisz w~swoich oczach, być może, jeśli bierzesz dobre narkotyki, ale same narkotyki nie były zbyt widoczne. Oficjalnie istniała opłata wstępna w~wysokości pięciu dolarów, za którą dostajesz pieczątkę (,,prosimy o pięciodolarowy datek'', powiedziała kobieta z~kolczykiem w~nosie), ale jak na każdym aktywnym wydarzeniu, nikt nie zostałby odrzucony, gdyby nie miał pieniędzy. Był też bar z~jakimś brazylijskim napojem z~trzciny, również za pięć dolarów. Muzyka, daleka od industrialu, była właściwie dość zmysłowa i~miała nawet instrumentalną wersję ,,Work for Love'' Ministry.

 O północy takie imprezy dopiero zaczynają się ruszać. Dach był w~połowie zajęty i~Rufus i~prawie wszyscy, których znałem, natychmiast udali się na najbardziej dramatyczną cechę dachu, wysoką platformę z~wyjątkowo rozklekotanymi schodami, z~których roztaczał się jeszcze bardziej panoramiczny widok na otaczające miasto. Kręciłam się z~kilkoma przyjaciółmi, knując i~planując, i~czekając, aż pojawi się sama JR, chociaż w~rzeczywistości plotki, że leżała z~grypą, okazały się prawdziwe, a nasza gospodyni nigdy się nie zmaterializowała. Impreza się rozkręciła dopiero około drugiej w~nocy, a wkrótce potem wyszedłem, mniej więcej w~tym samym czasie, kiedy zaczęli się żonglerzy ognia, połykacze ognia i~ludzie bawiący się płonącymi obręczami hooli i~tym podobne. Według Rufusa, około 3 nad ranem ,,tłum stał się znacznie młodszy''.

\bigskip
\noindent \textit{Impreza Reclaim the Streets}

\noindent \textit{(Teatr Chashama na 42. ulicy, między 6. a 7.)}

 Przyjeżdżam około północy z~kilkoma przyjaciółmi.

 Chashama to ciekawa przestrzeń, pusty teatr położony pośrodku Czterdziestej Drugiej Ulicy, w~epicentrum niegdyś seks dzielnicy miasta, obecnie rodzaj marginalnej strefy między teatrami Disneya a zabudową wokół Times Square na zachodzie, a Nowojorska Biblioteka Publiczna na wschodzie. Za Chashamą, jak zwykle, kryje się historia. Okazuje się, że jest taki bogaty deweloper, który systematycznie wykupuje każdą nieruchomość w~bloku, żeby wszystko zburzyć i~postawić wielki wieżowiec czy coś takiego. Problem w~tym, że jest jeden sklep, którego właściciel odmawia sprzedaży, więc muszą go przeczekać. Tymczasem córka dewelopera przyjaźni się z~niektórymi ludźmi RTS-a, a ponieważ i~tak nikt nie korzysta z~tej przestrzeni, przekonała swojego tatę, by pozwolił im z~niej korzystać.

 Mam wrażenie, że dla aktywistów sama pustka i~próżnia przestrzeni wydaje się przemawiać. Istnieje cała estetyka pustych przestrzeni związanych z~aktywistycznymi wydarzeniami. Podobnie jak w~wielu pokojach w~Charas -- czyli tych, które nie są pomalowane kolorowymi malowidłami ściennymi dokumentującymi kluczowe wydarzenia z~historii Ameryki Łacińskiej -- wszystko jest pustą funkcjonalnością: puste pokoje z~często czarnymi ścianami, pełne bardzo dużych przedmiotów, które są niebezpieczne do poruszania się wokół -- wysięgniki, kozły, maszyny -- lub, w~innych pokojach, białe pokoje, w~których nie ma nic. Jest radykalnie inny niż biura czy przestrzenie domowe, gdzie wszystko jest zasadniczo tworzone z~myślą o komforcie, wygodzie lub wydajności. Takie przestrzenie już sugerują Ci ich wykorzystanie. Te tutaj nie. Jeśli są do czegokolwiek przeznaczone, to wyraźnie jest to coś innego niż to, do czego są używane. To samo dotyczy większości napotykanych tu obiektów. Wszystko jest tym, co z~tego zrobisz. To glina. Po prostu zwykle bardzo duża, ciężka, nieporęczna glina.

 Wchodząc tej nocy do Chashamy, najpierw mija się jakiś kawałek wybiegu na plaży, a także improwizowanej małej skalnej kapliczki, która wydaje się używana do produkcji Living Theatre. Ściany są pomalowane na czarno. Ktoś wytyczył miejsce na estradę.

 W kolejce było kilka zespołów, ale większość z~nich skończyła. Gdy wchodzę, grają ,,Niemieckie samochody, a nie amerykańskie domy''. Nazwa mogłaby sugerować, że to naprawdę zespół stworzony na to wydarzenie, ponieważ ta impreza zbiera pieniądze na akcję antysamochodową w~Village w~przyszłym tygodniu, chociaż nie jest to tak naprawdę w~duchu raver, którego można by się spodziewać po RTS. Właściwie to prawie zespół punkowy; niektóre piosenki mogły być autorstwa Sex Pistols, inne prostszego rock'n'rolla, ale bardzo mocnego. Ludzie podskakują, pogują, szaleńczo tańczą, wiele rąk wyciągniętych w~stronę sceny. Powiedziałbym, że w~środku może ze 100 do 150 osób, ale trudno powiedzieć dokładnie, ponieważ dzień jest upalny, a deszcz przestał padać dopiero pół godziny wcześniej, wszyscy wychodzą na zewnątrz, rozmawiając z~przechodniami, którzy w~tej części śródmieścia krążą praktycznie całą noc.

 Od czasu do czasu pojawia się ludowa reklama RTS, wyglądając nieco oficjalnie. Wielebny Billy, fałszywy kaznodzieja i~performer, który w~tym czasie był oficjalnie profesorem w~New School, skakał w~kostiumie podczas muzyki, tańczył od czasu do czasu, czekając, aby wejść w~postać. Kilku miliarderów Busha i~Gore'a było w~pobliżu, w~smokingach obozowych i~wieczorowych sukniach. Kilku innych było w~kostiumach: jeden facet w, jak sądzę, masce Kiss z~wielkim językiem, inny w~fedorze z~doczepionym wielkim, błyszczącym w~dzień wielkim białym rekinem. Brooke miała na sobie maskę, jedną z~tych przerażających białych włoskich masek \textit{commute del arte}, ale przez całą noc miała ją na głowie i~nigdy jej nie założyła. Przeważnie jednak strój był niezwykle nieformalny i~bezpretensjonalny.

 Kiedy przyszedłem, w~dwóch aktywistach, Simon i~Brooke pracowali w~prowizorycznym barze z~tyłu: piwo za trzy dolary z~beczki, Rolling Rock w~butelkach. Jak zwykle nie byli wielkimi pedantami co do pieniędzy i~wydawało się, że co trzecia osoba była spłukana. Z drugiej strony, być może jeden na dziesięciu dorzucił jakiś ekstremalnie wygórowany napiwek, więc w~sumie impreza wydaje się zarabiać pieniądze. Było też miejsce, w~którym można było kupić losy na loterię po dolarze za los.

 Sama główna sala miała ciemne, puste ściany, z~wyjątkiem niebieskich i~czerwonych świątecznych lampek na górze. Hol, który prowadził do łazienki, był jednak niezwykle jasno oświetlony, jasny i~fluorescencyjny, ściany pokryte ręcznie rysowanymi kreskówkami w~formacie A4 i~hasłami, w~zakresie od pięknych dzieł, które wydawały się autorstwa profesjonalnych artystów, do rysunków sześciolatka. Przeważnie motywy propartyjne i~antypolicyjne, choć zróżnicowane (na jednym z~nich było zdjęcie uroczej syreny z~napisem ,,Co robiłam na wakacjach: poszedłem na paradę syren'').

 Mimo punkowego klimatu zespołu zauważyłem, że całe wydarzenie miało niezwykle przyjazną, otwartą atmosferę, zwłaszcza gdy muzyka się skończyła i~mogliśmy porozmawiać. Witam się z~Jessicą Rockstar, w~końcu czuję się na tyle dobrze, by się pojawić. Pokazuje nowy tatuaż na plecach, a właściwie jest to obecnie szkic, rzeczywisty tatuaż, który ma zostać wykonany w~dalszej części tygodnia, w~formie anielskich skrzydeł. Wyglądają jak pączkujące skrzydełka dziecka, które dopiero zaczynają wynurzać się z~jej pleców. Przedstawia mnie wysokiemu, naiwnemu facetowi, który wyjaśnia, że  właśnie ukończył album muzyczny o A16. Milczymy, gdy wielebny Billy wziął mikrofon (obok niego stała niska kobieta z~wiadrem kuponów), aby zareklamować akcję w~najbliższy piątek i~potem poprowadzić główne wydarzenie tego wieczoru -- loteria przedmiotów podarowanych przez członków RTS i~solidarnościowych instytucji z~sąsiedztwa Lower East Side.

 Wielebny Billy był znakomitym MC. Przedmioty loterii obejmowały wszystko, od książek dla majsterkowiczów po płyty jazzowe, trochę ,,drewna opałowego pozyskanego w~zrównoważony sposób'', bon podarunkowy do księgarni św. Marka, parę amazońskich kolczyków z~piór (wygranych przez kogoś z~IMC), ,,złą fryzurę'' (zgłoszony przez jakiś salon East Village), sesje shiatsu i~jeszcze więcej książek. Około połowa z~nich została wygrana przez kogoś o imieniu Chuck, którego nikt nie znał, a ponieważ tak naprawdę go tam nie było, skończyły na małej kupce pod sceną (nieunikniona gra słów: ,,Jak Chuck ma tyle szczęścia?'' i~,,kim jest do cholery Chuck?'').

 Zakończyliśmy kazaniem w~stylu Jimmy'ego Swaggarta. Wielebny B wykonał swój zwykły występ, tym razem przedstawiając ,,jakiego dupka z~New Jersey w~wielkim Lincoln Mercury, który może zobaczyć taką akcję, wrócić i~zobaczyć, jak ich samochód zamieniono w~dom dla niezamężnych matek, i~musi rzeczywiście chodzić po miejscach i~myśleć o jego życiu''. Prawdziwym punktem kulminacyjnym był jednak koniec, kiedy Kelvin z~Dumba Collective zaoferował, że wylicytuje swoje ubrania. Do tej pory znałem Kelvina, który wyglądał raczej jak długowłosy David Bowie, jako niezwykle rozważnego i~dobrodusznego aktywistę, który zwykle zasiadał w~barze z~absyntem na przyjęciach Complacent. 
 
 -- Nie mam nic poza koszulą na plecach -- oznajmił -- ale wszyscy możemy to dać. 
 
 Kelvin wyjaśnił, że zamierza wyrecytować dość długie opowiadanie napisane przez francuskiego surrealistę -- jedynego surrealistę, jak zauważył: który faktycznie zrobił coś z~radykalną polityką, którą wszyscy popierali i~zgłosił się na ochotnika do walki z~anarchistami w~hiszpańskiej wojnie domowej -- jako ,,słowo mówione”, jak to ujął ks. B. Wyjął książkę i~zaczął czytać -- nie jestem do końca pewien, czy rzeczywiście czytał po francusku, czy po angielsku, zresztą nikt nie słuchał, bo gdy tylko to zrobił, wielebny zaczął licytować jego ubrania. Najpierw odpadła jedna skarpetka (zauważyłam, że odłożył już buty, które ciężko wymienić), a potem biała koszula, T-shirt, spodnie, druga skarpetka, bokserki, aż w~końcu stanął z~przodu mikrofonu, czytanie, zupełnie nago. Wielebny B zakończył się pytaniem: ,,Ile dostanę, żeby przestał czytać?'', co przyniosło największe oferty ze wszystkich. W tym momencie Emily, bardzo ładna młoda rysowniczka w~śmiesznym stroju uczennicy z~majtkami na spodniach, wyszła i~wystawiła na aukcję bluzkę i~koszulę, ale zachichotała i~uciekła, gdy ktoś zaczął licytować stanik.

 Potem muzyka (DJ już grał i~skreczował fragmenty jakiejś fundamentalistycznej tyrady na temat grzesznego striptizu w~dużym mieście, kiedy wielebny Billy kończył swój występ), która była techno, ale bardzo żywiołowa i~zabawna. Przedstawienie się skończyło, wszystko zaczęło się rozpadać. Emily ponownie pojawiła się na scenie chwilę później, by ogłosić, że o 6:30 jakiś słynny fotograf prowadził masową sesję zdjęciową nagości na 125th Street (fotografował wiele głównych ulic Nowego Jorku wypełnionych aktami).  ,,Pod koniec wieczoru chcę zobaczyć większość z~nas nagich'', ale pomimo tego, że schody prowadzące do salonu również mówiły ,,salon, ubranie opcjonalne'', temat tak naprawdę nie wystartował. Pół godziny później nawet Kelvin wrócił w~czerwonej koszuli i~spodniach w~kratę, które ktoś podarował, kiedy zobaczyłem go stojącego na zewnątrz gadającego z~aktywistami, które wyszły na papierosa.

 Jedno, co wyłania się z~tego wszystkiego, to nieustanna preferencja dla miejsc budowy -- a czasem niszczenia -- gdzie zwykłe powierzchnie życia albo są łatane, albo burzone. (Czarne bloki, jak zobaczymy, uwielbiają place budowy i~znajdują improwizowane zastosowania dla ogrodzeń przemysłowych, śmietników i~tym podobnych). Środowiska przemysłowe. Wydaje się, że pomysł jest taki, by ułożyć sprawę na odpowiedni sposób sytuacjonistyczny, by zajrzeć za spektakl i~zamiast tego jak najwięcej krążyć wokół najbardziej brudnych, najbardziej nieprzyjemnych miejsc, w~których powstaje sam spektakl; być może po to, by tworzyć własne spektakle, ale zbiorowo, transparentnie, w~sposób partycypacyjny, bez podziału na kulisy i~scenę, na warsztaty i~widownię, co jest pierwotną formą wszelkiej alienacji. Pewien anarchista mieszkał w~squatowym mieszkaniu na poddaszu nad warsztatem, w~którym produkowane są figurki z~Gwiezdnych Wojen; to miejsce wyglądało w~połowie jak fabryka, w~połowie jak scenografia. Trzech weteranów DAN mieszkało na poddaszu pośród rzędu magazynów, pełnych masek i~wyszukanych kostiumów. Wszystko na ścianach lub na wystawie można było zdjąć i~nosić. Kolejny dom aktywistów znajduje się na opuszczonej, zarośniętej ulicy na Brooklynie, między składem drewna a miejskim parkingiem, gdzie stawiane są szkolne autobusy -- wszystko to są rzeczy, o których normalnie nie powinno się pamiętać, że istnieją. Większość sal w~Charas czy Chasham to teatry, w~których nie ma formalnej sceny, każde miejsce jest jednocześnie sceną i~za kulisami.

 Colin Campbell (1987) zasugerował kiedyś, że jednym z~powodów, dla których bohema zawsze nienawidziła burżuazji, jest to, że pierwsi postrzegają siebie jako ludzi, którzy porzucili wygody dla pogoni za przyjemnościami, podczas gdy burżuazja to ludzie, którzy postępowali dokładnie odwrotnie. Jakkolwiek ładnie, jest tu pewna prawda. Campbell twierdzi również, że bohema jest w~rzeczywistości awangardą konsumpcjonizmu, odkrywającą nowe formy przyjemności, które mogą być utowarowione w~następnym pokoleniu, i~tutaj myślę, że mija się z~rzeczywistością. Chodzi o to, że ta przyjemność jest w~szczególności w~punkcie tworzenia: przyjemność niszczenia granic, które tworzą takie kategorie jak produkcja i~konsumpcja. Przyjemność z~produkcji nigdy nie jest wygodna. Ale często może to być tym bardziej ekscytujące z~tego powodu.

\section{Wniosek, z uwagami dotyczącymi skutków ideologicznych regulacji rządowych}

 Globalny ruch antykapitalistyczny, który zadebiutował w~Stanach Zjednoczonych w~Seattle pod koniec 1999 roku, pojawił się w~bardzo osobliwym momencie historycznym. Był to czas ,,konsensusu waszyngtońskiego'', moment całkowitej hegemonii ideologicznej kapitalizmu. W czasie zimnej wojny tylko przeciwnicy kapitalizmu tak to nazywali; Zwolennicy kapitalizmu wolą mówić o ,,demokracji'', ,,wolności'' lub ,,prywatnej przedsiębiorczości''. Dopiero w~latach 80. kapitalizm odważył się wypowiadać swoje imię. Dziesięć lat później, po upadku Związku Radzieckiego, osiągnął taką ideologiczną potęgę, że jego zwolennicy argumentowali, że superdoładowany, wolnorynkowy kapitalizm jest jedynym możliwym modelem ekonomicznym czegokolwiek i~pozostanie nim do końca historii ludzkości. To, że następny wielki globalny ruch społeczny określi się jako antykapitalistyczny, było na swój sposób nieuniknione; jako ruch pierwszego pokolenia młodzieży wychowanej w~świecie bez alternatyw; było to dosłownie wszystko, przeciwko czemu można było się zbuntować. O ile stał się ruchem rewolucyjnym, pod względem demograficznym nie różnił się zasadniczo od ruchów rewolucyjnych z~przeszłości. W rezultacie musiał się zmierzyć z~większością tych samych dylematów. Zanim przejdziemy dalej, podsumuję je. Myślę, że takie dylematy istnieją nawet w~momentach spontanicznego powstania, ale stają się one tym bardziej wyraźne, im bardziej długofalowa staje się walka rewolucyjna: 

 W każdym ruchu rewolucyjnym będzie istniało napięcie między tymi, którzy mają najwięcej środków do przeprowadzania aktów buntu, a tymi, którzy mają najwięcej powodów do buntu.

 W rezultacie skład grup rewolucyjnych często łączy w~sobie dzieci z~klasy robotniczej, próbujące przenieść się do wyższej klasy, lub w~inny sposób pozbawione praw obywatelskich z~(często dobrowolnie przemieszczającymi się w~dół drabiny społecznej) dziećmi elit, ponieważ te dwie grupy najprawdopodobniej produkują osoby, które zarówno pragną radykalnej zmiany, jak i~mają zasoby społeczne, kulturowe i~ekonomiczne, aby móc zaangażować się w~skuteczną długoterminową walkę.

 Wszystko to ma tendencję do zaostrzania innego, bardziej koncepcyjnego napięcia w~każdym ruchu rewolucyjnym: stopnia, w~jakim jest on inspirowany nie tylko odrzuceniem struktury danego porządku społecznego, to znaczy dystrybucją tych rzeczy, których ludzie chcą lub potrzebują ( bogactwo, honor, bezpieczeństwo, żywność itd) i~to, co muszą zrobić, aby je zdobyć, ale odrzuceniem standardów, które określają, czego ludzie \textit{powinni }chcieć. Innymi słowy, napięcia wynikają ze stopnia, w~jakim ruch opiera się na szerokim odrzuceniu istniejących standardów wartości. Z kolei alienację można zdefiniować jako subiektywne doświadczenie tego: to, co się czuje, gdy czyjaś koncepcja wartości -- tego, czego należy pragnąć od życia, tego, co powinno być w~nim ważne lub warte zachodu -- jest radykalnie niezsynchronizowana z~obowiązującymi standardami społecznymi. Problemem jest tu zawsze napięcie między tego rodzaju polityką wyobcowania a bardziej bezpośrednimi problemami ucisku: radykalnym wykluczeniem z~podstawowych potrzeb, tych środków egzystencji, które muszą być w~pewnym stopniu zagwarantowane, aby móc realizować inne wartości.

 W Stanach Zjednoczonych kwestie te stają się nieskończenie bardziej skomplikowane i~często wybuchowe, o ile nieuchronnie ulegają wpływowi przez kwestie rasowe.

 Ci, którzy brali udział w~tym ruchu, zostali najpierw skreśleni jako naiwni utopiści lub całkowicie wariaci. To również jest normalne, chociaż można powiedzieć, że tym razem zwolnienie było znacznie bardziej bezwzględne i~trwałe niż zwykle -- zwłaszcza w~Stanach Zjednoczonych. Być może nie jest to zaskakujące, biorąc pod uwagę połączenie upadku ,,faktycznie istniejącego socjalizmu'' z~faktem, że tak wielu rewolucjonistów uważa się za anarchistów. Mimo to uważam, że dobrze byłoby pomyśleć o tym, co sprawia, że  anarchizm i~ogólnie rewolucyjne marzenia wydają się nieanarchistom tak nierealne. Efekt ideologiczny działa w~sposób o wiele bardziej subtelny, niż mogłoby się wydawać na pierwszy rzut oka.

 Często mówi się, że ideologia jest najskuteczniejsza, gdy sprawia, że  pewne układy społeczne -- te, które mogą być ułożone inaczej -- wydają się naturalne i~nieuniknione. Na tyle rynek, państwo czy patriarchalna rodzina wydają się tak oczywiste, że każdy, kto proponuje im alternatywę, wydaje się -- dokładnie tak jak nasi rewolucjoniści -- w~najlepszym razie nierealistycznym marzycielem, w~najgorszym szalonym, mamy do czynienia z~klasycznym efektem ideologicznym . i~z pewnością prawdą jest, że kapitalizm zawsze był w~tej grze niezwykle skuteczny. Czyni to w~dużej mierze poprzez definiowanie siebie nie w~kategoriach pracy najemnej czy jakichkolwiek stosunków produkcji, czy nawet kapitału, ale po prostu jako połączenie praw własności prywatnej i~wymiany opartej na własnym interesie. Oba te zjawiska można zatem uznać za uniwersalne, w~istocie, zjawiska naturalne: łączą one przypuszczalnie naturalne pragnienie posiadania rzeczy na własność i~to, co Adam Smith (1776) nazwał słynną ,,naturalną skłonnością ludzi do ciężarówek, handlu wymiennego i~wymiany jednej rzeczy na drugą''. Karty czasowe i~ograniczona odpowiedzialność można więc postrzegać jako bardziej złożone emanacje z~tej wszechobecnej bazy. Część siły tego poglądu można ocenić za pomocą retoryki często słyszanej po upadku Związku Radzieckiego, kiedy analitycy wydawali się odejść, w~ciągu zaledwie kilku miesięcy, od argumentowania, że  gospodarka nakazowa nigdy nie będzie w~stanie wkroczyć do wieku komputerów, do argumentowania, że  gospodarka nie oparta na motywie zysku ,,po prostu nie może działać'', ponieważ jest sprzeczna z~uniwersalnymi ludzkimi skłonnościami -- pomijając to, jak Związek Radziecki mógł w~ogóle istnieć przez siedemdziesiąt dwa lata.

 Zgodnie z~tą logiką anarchizm -- jako forma libertariańskiego komunizmu -- jest nie tylko nierealistyczny, jest sprzeczny w~kategoriach. Komunizm może przybrać tylko formę kontroli państwa. Ponieważ jakakolwiek wolna gospodarka zawsze przybierze formę rynku, każda próba stworzenia zbiorowych alternatyw albo się skończą, ponieważ są sprzeczne z~ludzką naturą, albo, alternatywnie, muszą skończyć się przymusem państwa. Zakłada się, że państwo i~rynek to przeciwstawne zasady. Tego rodzaju argumentację można prześledzić co najmniej do XIX wieku -- kiedy nazywano go ,,liberalizmem'', a jeszcze nie ,,neoliberalizmem'' -- u autorów takich jak Herbert Spencer, który twierdził, że państwo ostatecznie rozpadnie się całkowicie, ponieważ zostanie zastąpione swobodnymi stosunkami umownymi opartymi na zasadach rynkowych. Emile Durkheim (1893) dawno temu wskazał tutaj błąd: Przewidywania Spencera nie były w~żaden sposób potwierdzone dowodami empirycznymi. W rzeczywistości stwierdził, że wraz ze wzrostem liczby wolnych porozumień umownych państwa faktycznie stawały się znacznie większe: konieczne było niekończące się opracowywanie nowego ustawodawstwa i~mechanizmów administracyjnych w~celu monitorowania i~egzekwowania ich wszystkich -- co nazwał ,,pozaumownym element umowy''. Właściwie współczesne siły policyjne zostały stworzone właśnie w~okresie rozkwitu ,,wolnego kontraktu'' (Neocleous 2000) i~zajmowały się przede wszystkim ochroną własności prywatnej, tłumieniem tradycji ulicznej mobilizacji i~niesfornych form proletariackiego społeczeństwa oraz nawet regulacja rynku pracy. 

 Kiedy przyjrzymy się, co tak naprawdę stwarza praktyczne problemy, gdy anarchiści próbują zacząć tworzyć ,,nowe społeczeństwo w~skorupie starego'', to właśnie to napotkamy. Z pewnością zawsze pojawiają się skargi dotyczące ,,kwestii związanych z~rozliczalnością'', jak lubią to określać aktywiści, jak upewnić się, że wolontariusze rzeczywiście pojawiają się na swoich zmianach lub aktywiści faktycznie wykonują zadania, do których dobrowolnie się zgłaszali na spotkaniu. Ale nigdy nie słyszałem o takim projekcie jak spółdzielnia księgarni, czy o sklepie rowerowym, który w~rezultacie się zawalił. Zamiast tego jedyną rzeczą, jaką przynosi natychmiastowe, codzienne doświadczenie ludzi próbujących tworzyć alternatywy, jest stopień, w~jakim prawie wszystko w~Ameryce jest otoczone niekończącymi się i~zawiłymi regulacjami rządowymi. Siła przymusu państwa jest wszędzie. Przede wszystkim przymus przylega do wszystkiego, co duże, ciężkie, i~cenne ekonomicznie; innymi słowy, do każdego wartościowego przedmiotu, którego nie da się po prostu ukryć: w~samochodach, na łodziach, w~budynkach, w~maszynach.

 Pozwolę sobie przedstawić ilustrację.

 W pewnym momencie w~2002 roku ktoś podarował NYC Direct Action Network samochód. Był to stary samochód, z~którego darczyńca nie miał żadnego pożytku; wręczył go wraz ze wszystkimi odpowiednimi papierami w~schowku na rękawiczki. Szybko odkryliśmy, że ,,samochód DAN'' był w~zasadzie niemożliwością z~prawnego punktu widzenia. W świetle prawa samochód musi mieć właściciela. Zwykle zakłada się, że właściciel jest osobą, a nie zbiorowością. Oczywiście możliwe jest, aby samochód był własnością zbiorowości, ale ten zbiorowy podmiot musi być uznany przez państwo. Oznacza to, że o ile samochód nie jest własnością samego rządu (lub obcego rządu), zbiorowość musi być jakąś korporacją. Można sobie wyobrazić DAN jako rodzaj korporacji non-profit, ale w~rzeczywistości prawnie uznanie za organizację non-profit wymaga dużej ilości papierkowej roboty. Wymaga to również przynajmniej udawania, że  ma się określoną formę organizacji, z~dyrektorem i~różnymi odpowiedzialnymi osobami gotowymi wypełnić papierkową robotę. Rządy niemal niezmiennie upierają się, że grupy, z~którymi mają do czynienia, są zorganizowane hierarchicznie. Na przykład IMC cały czas boryka się z~tym problemem. Nie trzeba nawet mieć do czynienia bezpośrednio z~rządem; wystarczy regularnie mieć do czynienia z~organizacjami działającymi w~ramach gospodarki formalnej (które państwo monitoruje i~reguluje). W ten sposób natychmiast wchodzi się w~świat, w~którym zakłada się, że wszystkie zbiorowości mają określone stanowiska: prezesa, rady dyrektorów, redaktora naczelnego. To samo dotyczy w~rzeczywistości każdej transakcji finansowej, która nie jest dokonywana w~gotówce: aby potencjalni wpłacający mogli wypisywać czeki dla DAN, na przykład grupa musiałaby być zupełnie inaczej zorganizowana, przynajmniej na papierze. W każdym razie otwarte sieci aktywistów nie mogą legalnie posiadać samochodów.

 Oczywiście można po prostu udawać. Tym właśnie zajmuje się IMC i~zasadniczo to zrobiliśmy z~samochodem: technicznie tytuł nie został przeniesiony na DAN, ale na jednego z~jej członków, Moose, który w~ten sposób stał się osobą przewodnią dla ,,grupy roboczej DAN ds. samochodów''. Ale to znacznie utrudniło wspólne zarządzanie samochodem. Teoretycznie w~grupie roboczej były również dwie inne osoby. Mimo to wszyscy wiedzieli, że jeśli Moose nie jedzie, a samochód zostanie zatrzymany, trzeba będzie wykonać papierkową robotę; a jeśli samochód był holowany (co szybko się stało, ponieważ były właściciel miał niezapłacone mandaty), tylko Moose mógł go wyciągnąć. Oznaczało to, że musiał zapłacić pieniądze, a to z~kolei oznaczało, że reszta z~nas, mimo że sami pomogliśmy zebrać pieniądze, traktujemy samochód coraz bardziej jako jego samochód.

 Powinienem zaznaczyć, że nic z~tego by się nie wydarzyło, gdyby ktoś dał DAN, powiedzmy, palmę w~doniczce lub rower. Albo nawet drogi komputer. Bez wątpienia są w~księgach wszelkiego rodzaju podobne prawa i~przepisy dotyczące własności i~przenoszenia książek, komputerów i~palm doniczkowych, ale są one tak rzadko egzekwowane, że większość z~nas nie ma pojęcia, czym one są, a do tego istnieje bardzo prosty powód. Książki, palmy w~doniczkach i~komputery są stosunkowo niewielkie; są dość łatwe do ukrycia; w~rezultacie rząd nie ma możliwości skutecznego ich regulowania. Fakt, że samochód jest duży, ciężki i~nie da się go łatwo ukryć (przynajmniej, jeśli rzeczywiście zamierza się z~niego korzystać) oznacza, że może być stale monitorowany przez gałąź państwa, której zadaniem jest właśnie monitorowanie samochodów -- ich prędkość, lokalizacja, status rejestracji, czy ich kierowca jest licencjonowany i~tak dalej -- i~egzekwuje niezliczone, bardzo szczegółowe prawa, które regulują takie sprawy -- prawa, które, raz jeszcze podkreślam, zakładają różne rzeczy dotyczące tego, jakie grupy społeczne mogą i~nie mogą mieć statusu prawnego. Te zasady są egzekwowane przez groźbę użycia siły. Uzbrojeni przedstawiciele państwa mogą w~każdej chwili zatrzymać twój samochód i~sprawdzić dokumenty, a jeśli tak się stanie, pasażerowie lepiej, żeby nie odpowiadali impertynencko. Jeśli twój samochód zostanie odholowany i~próbujesz go po prostu odebrać bez płacenia mandatu, przedstawiciele stanu użyją siły, aby cię zatrzymać. Fakt, że samochód DAN stał się natychmiastowym problemem i~po kilku miesiącach został porzucony, nie był dowodem na to, że egalitarne kolektywy nie potrafią zarządzać własnością (historia ludzkości jest pełna przykładów egalitarnych kolektywów z~powodzeniem zarządzających własnością). W rzeczywistości jest to dowód na natychmiastoą efektywność przemocy państwowej w~zmuszaniu do określonych wizji ludzkich możliwości.

 Co jest prawdziwe wobec samochodu lub łodzi. Jeszcze bardziej prawdziwe w~przypadku budynku. Istnieją nieskończone przepisy dotyczące tego, w~jaki sposób budynki mogą i~muszą być konserwowane. Squattersi niezmiennie narzekają, że pierwszą rzeczą, jaką robią przedstawiciele miast, jeśli w~jakiś sposób zdobędą tytuł prawny do swojego budynku, jest wysłanie inspektorów, aby zażądali każdej możliwej naprawy, aby budynek był zgodny z~prawem: żądania, na które, ci sami dzicy lokatorzy, zawsze zwracają uwagę, inspektorzy prawie nigdy nie stawiają wobec nieobecnych właścicieli, bez względu na to, jak głośno błagają ich najemcy. Niektóre z~tych prac mogą i~zwykle są wykonywane w~ramach gospodarki alternatywnej: zawsze są squatterzy hydraulicy lub elektrycy, którzy są chętni do świadczenia swoich usług. Niektóre materiały można często uratować lub odzyskać. Ale nie wszystko. Efektem jest, jak wspomniałem wyżej w~przypadku ABC No Rio, że jest się zanurzonym w~formalnej gospodarce w~bardzo traumatyczny sposób i~zmuszonym do poświęcenia dużej ilości czasu i~energii na organizowanie koncertów dobroczynnych, zbiórki pieniędzy, sprzedaż T-shirtów lub inne zbieranie pieniędzy. Ale znowu nie jest to w~żaden sposób efekt imperatywu ekonomicznego. To efekt groźby przemocy. Jeśli ktoś nie spełniał wymagań, przychodzili uzbrojeni ludzie i~wyrzucali ich z~budynku. Z kolei, jeśli sprzedajesz koszulkę, sprawy muszą mieć określoną formę prawną, ponieważ trzeba nałożyć podatek obrotowy. Jeśli chcesz ubiegać się o dotacje, musisz zarejestrować się jako organizacja non-profit.

 To, co chcę tutaj podkreślić, to efekt ideologiczny. Nazwę to ,,efektem rzeczywistości''. Regulacje rządowe zasadniczo wymuszają pewien model społeczeństwa, w~którym poszczególni aktorzy lub hierarchicznie zorganizowane firmy dążą do zysków, a każdy, kto chce zorganizować się inaczej -- wokół jakiejkolwiek koncepcji dobra wspólnego -- musi albo być częścią aparatu państwowego, albo zarejestrować się jako korporacja non-profit. Teoretycznie każdy aspekt ,,społeczeństwa obywatelskiego'' jest tak uregulowany. Zasadniczo \textit{jedynymi }obszarami, które są całkowicie niedostępne dla tego rodzaju regulacji wspieranych siłą, są obszary komunikacyjne: mowa, dyskusja na spotkaniach, wymiana w~Internecie itp\footnote{Walter Benjamin (1978) był bardzo zaniepokojony przepisami dotyczącymi oszczerstw, ponieważ mowa jest jedynym obszarem, w~którym przemoc państwowa wcześniej nie wkroczyła.}. Gdy tylko wkracza się w~świat przedmiotów materialnych, pojawia się mnóstwo przepisów. A im większe, cięższe i~bardziej widoczne obiekty, tym częściej te przepisy są egzekwowane. Oczywistym rezultatem jest pozostawienie ludziom poczucia, że  radykalna polityka jest nierealna. To wszystko jest efemerycznym światem snów, który rozpływa się w~momencie, gdy uderza w~materialną rzeczywistość. Gdy tylko wkracza się do ,,rzeczywistego świata'', świata dużych, ciężkich rzeczy, takich jak budynki i~maszyny itd., wszystko zdaje się nierealistyczne. W rzeczywistości dzieje się tak dlatego, że o wiele łatwiej jest regulować ciężkie obiekty fizyczne. W rezultacie duże, ciężkie, cenne przedmioty są zwykle otoczone groźbami siły fizycznej, która wspiera pewną ideologię tego, w~jaki sposób ludzie mają wchodzić w~interakcje, a jeśli tego nie robią, są one ci zabierane. Przedmioty, które są w~oczywisty sposób realne, są w~istocie tymi, najbardziej otoczonymi siłami i~abstrakcjami.

 Aby uprzedzić argument, który przedstawię w~konkluzji: zastanów się przez chwilę nad niektórymi zastosowaniami słowa ,,realny''. Można mówić o najłatwiejszych do uregulowania formach własności -- największej, najtrudniejszej do ukrycia, a więc najskuteczniej zagrożonej przemocą -- jako ,,nieruchomości'' (real estate lub real property), w~przeciwieństwie do ruchomości. Należy zauważyć, że własność ,,realna'' nie jest w~żadnym sensie bardziej realna empirycznie niż ruchomości: w~rzeczywistości, o ile obejmuje złożone abstrakcje, takie jak prawa do powietrza, można powiedzieć, że w~porównaniu z, powiedzmy, pomidorem, jest zdecydowanie mnie\footnote{Pochodzi również etymologicznie nie od łacińskiego ,,res'', ale od hiszpańskiego ,,real'', ,,royal'', ostatecznie należący do króla, a więc pod jurysdykcją władzy państwowej.}. Ale można też mówić o ,,realpolitik'' lub politycznym ,,realizmie''. Na przykład w~stosunkach międzynarodowych bycie ,,realistą'' (w przeciwieństwie do ,,instytucjonalisty'') oznacza wychodzenie z~założenia, że  narody nie zawahają się użyć siły w~dążeniu do własnych interesów narodowych. Po raz kolejny nie ma to nic wspólnego z~rozpoznawaniem tego, o czym lubimy myśleć jako o rzeczywistości empirycznej: ,,narody'' o zbiorowych ,,interesach'' są konstruktami czysto urojonymi. Stają się ,,realne/rzeczywiste'', gdy grożą wysłaniem wojska. ,,Rzeczywistość'', którą rozpoznaje się, gdy jest się ,,realistą'', to czysta przemoc. Jednak to właśnie wpływ skutków przemocy w~pozorną solidność przedmiotu powoduje efekt rzeczywistości, o którym mówię, i~sprawia, że  alternatywy społeczne wydają się tak nierealistyczne. Abstrakcje jak prawa i~państwo dołączają się, poprzez groźbę przemocy, do największych, najcięższych przedmiotów, rzeczy, które wydają się empirycznie ,,realne''.

 Wszystko to może pozwolić nieco inaczej zrozumieć anarchistyczne zamiłowanie do industrialnych przestrzeni, placów budowy, przestrzeni za kulisami i~tym podobnych. To, co jest tam ,,odwracane'' -- by użyć nieco sfatygowanej wersji wyrażenia sytuacjonistycznego -- to właśnie ten efekt rzeczywistości, aby, jak sądzę, zaproponować inną, w~której ostateczną rzeczywistością nie jest zdolność do użycia przemocy, władza zniszczenia, ale raczej sama moc kreatywności.

 Do tego tematu powrócę na zakończenie.

\chapter{Spotkania}

 W części i~starałem się dać czytelnikowi pewne wyobrażenie o tym, jak niekończący się łańcuch mniejszych spotkań może najpierw doprowadzić do masowych zgromadzeń, a następnie do masowych akcji. Te spotkania są ważne. W pewnym sensie są one ważniejsze nawet niż same działania, ponieważ działania wiążą się z~konfrontacją z~wrogimi siłami, a spotkania są czystymi strefami społecznego eksperymentu, przestrzeniami, w~których aktywiści mogą traktować się nawzajem tak, jak uważają, że ludzie powinni traktować się nawzajem i~zacząć tworzyć coś ze świata społecznego, który chcą wydobyć.

 Ten rozdział w~dużej mierze dotyczy Sieci Działań Bezpośrednich Miasta Nowy Jork. Po krótkim wprowadzeniu do pojęcia grupy afinicji i~pewnych pokrewnych pojęć oraz pewnym tle historii NYC DAN przedstawię wewnętrzny proces DAN. Pomyślałem jednak, że zamiast naszkicować to samodzielnie, pomyślałem, że bardziej interesujące byłoby dla czytelnika uczenie się rzeczy mniej więcej tak, jak ja: więc odtworzyłem tekst pierwszego szkolenia konsensusu / facylitacji, w~którym kiedykolwiek uczestniczyłem, przeprowadzonego dla nowych członków DAN na wiosnę 2000 roku. Napisałem też refleksje na temat tego, jak definiowane są ideały zachowania leżące u podstaw konsensusu.

 To wszystko jest w~pierwszej połowie rozdziału, który pokazuje, jak konsensus powinien funkcjonować w~zasadzie. Druga część rozdziału dotyczy problemów: trudnej dynamiki rasowej i~płciowej, napięć związanych z~klasą społeczną i~innych czynników, które prawie zawsze powodują napięcia w~grupach aktywistów. Proces konsensusu opiera się na pewnego rodzaju zinstytucjonalizowanej hojności ducha. Na spotkaniu z~innymi aktywistami, obowiązkowe jest domniemywanie u innych uczciwość i~ich dobrych intencji. W większości przypadków zasada ta działa wyjątkowo dobrze w~tworzeniu rzeczywistego uczciwego i~pełnego dobrych intencji zachowania. Tam, gdzie się nie udaje, napotyka dokładnie to, co aktywiści nazwaliby głęboko zinternalizowanymi formami ucisku. Rasizm, seksizm, uprzedzenia klasowe, homofobia, wszystko to są formy przemocy, które są postrzegane zarówno jako absolutne zło, jak i~jako tak głęboko zinternalizowane, że po prostu nie można oczekiwać, że ludzie sami będą się pilnować. Co więcej, mają tendencję do wikłania się w~siebie nawzajem w~sposób, który bardzo utrudnia walkę z~nimi wszystkimi w~tym samym czasie. Centralnym punktem ostatniej części rozdziału jest zatem rozszerzone studium przypadku, zaczerpnięte z~rzeczywistego spotkania DAN, ilustrujące, jak trudne może być radzenie sobie z~takimi kwestiami w~ramach dużej, opartej na konsensusie grupy. Koncentruje się na wysiłkach kobiecego klubu DAN, aby spróbować stworzyć pewne mechanizmy kontroli seksistowskiego zachowania, ale jednocześnie na usilnym sprzeciwie wobec ich wysiłków przez Dennisa, nieco szalonego, członka DAN, pochodzącego z~klasy robotniczej. Kończę pewnymi uwagami na temat nawet większego problemu, który się pojawia, gdy grupy oparte na zasadach autonomii i~demokracji bezpośredniej muszą wejść w~relacje, na bieżąco, z~innymi, które są zorganizowane na bardziej hierarchicznych zasadach.

\section{Część I: Kontekst}

\noindent GRUPY AFINICJI

 Zacznę od grup afinicji, ponieważ można je uznać za elementarne cząstki dobrowolnego stowarzyszania. Zasadniczo są to po prostu małe grupy ludzi, którzy czują, że mają ze sobą coś wspólnego i~decydują się na wspólną pracę nad wspólnym projektem. Sam termin wywodzi się z~hiszpańskiego \textit{grupos de afinidad}, które pierwotnie odnosiło się do grup przyjaciół (powszechnym synonimem były \textit{tertulia}, grupy kumpli od kieliszka lub młodych ludzi, którzy zwykli spędzać czas w~kawiarniach), ale które w~latach 20. stało się podstawowym jednostką organizacyjną hiszpańskiej konfederacji anarchistycznej, FAI. Kiedy podczas kampanii antynuklearnych we wczesnych latach 80. zebrały się pierwsze grupy oparte na konsensusie na dużą skalę, zawsze zakładano, że podstawową jednostką są grupy afinicji.

 Zgodnie z~podręcznikiem szkoleniowym z~zakresu nieposłuszeństwa obywatelskiego ACT UP:

\textit{ Grupy afinicji to samowystarczalne systemy wsparcia liczące około 5 do 15 osób. Kilka grup afinicji może współpracować w~celu osiągnięcia wspólnego celu w~dużej akcji, lub jedna grupa afinicji może wymyślić i~przeprowadzić akcję samodzielnie. Czasami grupy afinicji pozostają razem przez długi czas, istniejąc jako grupy wsparcia politycznego i/lub grupy badawcze i~tylko okazjonalnie uczestniczą w~działaniach}\footnote{\url{http://www.actupny.org/documents/CDdocuments/Affinity.html}\textit{; udostępniono 31 lipca 2004 roku}}.

 Podczas akcji każda grupa afinicji musi rozdzielić określoną liczbę ról:

 W ramach grupy afinicji istnieje szereg różnych ról, które mogą pełnić jej członkowie. Wiele z~tych ról będzie określanych przez \textit{raison} \textit{\ d’etre} grupy afinicji, ale może obejmować rzecznika prasowego, który będzie rozmawiał z~mediami informacyjnymi lub zajmował się nimi, facylitatora szybkich decyzji, pierwszą pomoc w~opiece nad ludźmi, którzy są ranni, rzecznika, który przekazuje pomysły i~decyzje grupy afinicji innym grupom, obserwatora Prawny i~osoby wsparcia przy aresztowaniu\footnote{\url{http://www.starhawk.org/activism/affinitygroups.html}\textit{, dostęp 31 lipca 2004 roku}}.

 Minimalna wersja, której nauczyłem się na szkoleniach DAN, zakładała, że  przynajmniej powinien być: (1) facylitator do organizowania grupowego podejmowania decyzji, (2) ktoś, kto przeszedł przynajmniej jedno szkolenie medyczne, (3) ktoś, kto poszedł na szkolenie prawnicze. Osoba prawna zwykle robi wszystko, co możliwe, aby uniknąć aresztowania, aby móc śledzić wszystkich innych. Jest to osoba, która również prowadzi listę osób, które będą potrzebowały kogoś, kto nakarmi kota, jeśli zostanie aresztowany, kto potrzebuje kogoś, kto okłamie szefa i~tak dalej. Ponadto można mieć kogoś, kto zajmie się zaopatrzeniem, komunikacją lub innymi potrzebami, a na końcu przedstawiciela.

 Przedstawiciel czy delegat przemawia w~imieniu grupy na większych spotkaniach, na przykład, gdy grupy afinicji tworzą większe ,,zgrupowania'' lub, oczywiście, w~,,radach delegatów''. Te ostatnie, odbywające się przed dużymi akcjami, często mogą angażować tysiące ludzi -- o wiele za dużo, by każdy mógł mówić. Delegat nie jest jednak reprezentantem. Zwykle przedstawiciele nie mają uprawnień do podejmowania decyzji w~imieniu grupy, są kanałami informacyjnymi; stąd, na radzie delegatów, podczas gdy ,,umocowani przedstawiciele'' siedzą pośrodku w~wielkim kręgu, oczekuje się, że reszta grupy afinicji będzie pod ręką, szepcząc do siebie nawzajem i~ostatecznie przekazując instrukcje. W zasadzie przedstawiciele są dosłownie jak szprychy wielkiego koła.

 Kiedy grupy afinicji przetrwa od działania do działania, trudno je odróżnić od kolektywów, grup minimalnych, które działają na zasadach egalitarnych. Z pewnością są grupy, które odgrywają obie role, które przez większą część roku pracują jako kolektywy medialne, grupy wsparcia, drukarze broszur lub projekty feministyczne, a potem pojawiają się na akcjach jako grupy afinicji. Istnieją również grupy afinicji, które istnieją jako sieci przyjaciół przez większość roku, ale można je zmobilizować w~ważnych momentach: były dwie takie grupy w~nowojorskim DAN, na przykład Latające Wiewiórki dla Wolności i~Front Wyzwolenia Subway, prawdopodobnie trzy, jeśli wliczyć Harper's Ferry, chociaż ten ostatni w~dużej mierze miał siedzibę w~Nowym Jorku.

 W tym rozdziale nie będę mówił dużo o spotkaniach tak stosunkowo małych i~intymnych grup, często postrzeganych jako bardziej nieformalne i~,,organiczne'' niż grupy takie jak DAN. Ale nie zamierzam też zbytnio rozwodzić się nad radami delegatów -- czytelnik ma już wyobrażenie o tych w~części I. Raczej skupię się na DAN, próbie stworzenia trwalszej struktury na tych samych zasadach. Od samego początku DAN była niepewna co do tego, jak dokładnie można to zrobić. Czy DAN powinna przybrać formę stałej rady dla istniejących kolektywów i~grup afinicji (,,model konwergencji'', czy też powinna organizować spotkania otwarte dla wszystkich i~ich grup roboczych? Czy była to sieć grup, czy grupa sama w~sobie? Żadne z~tych pytań nie zostało ostatecznie rozwiązane. DAN zawsze pozostawała po trochu obu i~dlatego jego struktura zawsze była problemem.

\noindent WZROST i~UPADEK KONTYNENTALNEJ DAN -- CDAN

 Na pierwszy rzut oka próba stworzenia kontynentalnej sieci akcji bezpośrednich DAN (Direct Action Network) wydaje się oczywistą porażką. Pomysł stworzenia kontynentalnej sieci grup akcji bezpośredniej zrodził się w~przypływie entuzjazmu po zaskakującym sukcesie działań WTO w~Seattle w~listopadzie 1999 roku. W ciągu mniej więcej następnego roku sieć szybko się rozrosła. Ale DAN szybko zaczął tracić wielu swoich najbardziej entuzjastycznych wczesnych członków po frustrującej serii mniej udanych działań; i~w ciągu kilku lat skutecznie się rozpadł. To oczywisty sposób na przedstawienie tej historii. Jest jednak jeszcze inny. Kiedy po raz pierwszy związałem się z~DAN, prawie wszyscy podkreślali, że nie spodziewali się, że grupa będzie w~pobliżu na zawsze. Sam DAN nie miał zamiaru wywołać rewolucji. Raczej, większość nalegała, że  DAN istnieje po to, by rozpowszechniać pewną wizję demokracji bezpośredniej, by zapewnić model egalitarnych procesów decyzyjnych, które ostatecznie stałyby się standardową praktyką dla wszystkich zainteresowanych bezpośrednią konfrontacją z~państwem i~kapitalizmem. Kiedy już to zrobi, nie będzie powodu, by DAN istniało. W pewnym sensie tak właśnie się stało i~to znacznie szybciej, niż ktokolwiek się spodziewał. W ciągu dwóch lub trzech lat DAN, jako formalna jednostka, zniknęła, ale w~innym sensie była wszędzie, ponieważ przynajmniej wśród grup zorientowanych na bezpośrednie działanie, jakaś wersja jej modelu organizacji stała się całkiem uniwersalna.

 Pomysł na pierwotną Sieć Akcji Bezpośrednich (DAN) w~rzeczywistości pochodził z~obozu Ruckus -- obozu szkoleniowego dla aktywistów -- w~1999 roku i~został stworzony w~celu koordynowania tego, co nazywano wówczas N30, akcjami przeciwko zebraniu WTO, które odbyły się w~Seattle 30 listopada 1999 roku. Towarzystwo Ruckus, które organizuje te obozy, jest organizacją pozarządową, ale bardzo niezwykłą, która specjalizuje się w~szkoleniu młodych ludzi w~technikach pokojowego nieposłuszeństwa obywatelskiego i~akcji bezpośredniej\footnote{To niezwykle rzadkie zjawisko. Poza kilkoma spektakularnymi wyjątkami, takimi jak Ruckus czy Greenpeace, organizacje pozarządowe rzadko same prowadzą lub szkolą innych w~działaniach bezpośrednich.}. Organizuje obozy przed niemal każdą większą mobilizacją, zwykle w~jakimś nieokreślonym, pięknym, zalesionym miejscu, ze szkoleniami na temat wszystkiego, od zrzucania sztandarów po przezwyciężanie form nieświadomego rasizmu. W tym obozie pojawił się pomysł na zdecentralizowaną sieć, która koordynowałaby różne grupy afinicji, które miały wziąć udział w~N30 według demokracji bezpośredniej. Model działał tak dobrze, że nawet przed akcją niektórzy sugerowali utrzymanie go w~jakiejś formie później, ale zaraz po akcjach okazało się to trudne do omówienia, ponieważ tak wiele kluczowych postaci przebywało w~więzieniu. Ci, którzy wciąż są na zewnątrz, stworzyli nieco chaotyczne Ciało Tymczasowe, którego zadaniem było ,,spędzić kolejne trzy miesiące ze swoimi grupami lokalnymi, aby przygotować propozycję dla przyszłego kontynentalnego DAN, które działałoby na zasadach nie-hierarchii, decentralizacji, lokalnej autonomii i~demokracji bezpośredniej''\footnote{[nyc-dan] Projekt ratyfikacji CDAN, data: poniedziałek, 15 maja 2000 roku}. Składała się ona z~dwunastu regionalnych delegatów, którzy wrócili do domu, skonsultowali się ze swoimi lokalnymi grupami, zaangażowali się w~cotygodniowe telekonferencje, a następnie w~końcu wrócili do Seattle pod koniec lutego 2000 roku, aby naszkicować statut CDAN\footnote{Obejmowały one Boston, Nowy Jork, Filadelfię, DC, Południowy Wschód, Chicago, Arizona, LA, San Francisco, Północne Kalifornia, Seattle i~Vancouver; przemawia jeden zespół prawny; i~jest tylko jeden łącznik z~pracownikami.}.

 Na początku wielkie pytanie brzmiało, czy będzie to ośrodek komunikacji, czy też prawdziwy organ decyzyjny. Wielu uważało, że to ostatnie oznaczałoby naruszenie autonomii lokalnych grup. Inni nalegali, że, jak ujęła to strona internetowa CDAN, ,,powinniśmy włożyć wszystkie nasze wysiłki w~zbudowanie modelu tego, jak mogłaby wyglądać prawdziwie zdecentralizowana, skonfederowana, bezpośrednio demokratyczna organizacja'' -- i~było to szczególnie ważne, aby zademonstrować, że można to zrobić w~skali kontynentalnej. To był pogląd, który zwyciężył, z~zastrzeżeniem, że wszelkie inicjatywy miały pochodzić od miejscowych, a każda lokalna grupa mogła się wycofać w~dowolnym momencie. CDAN miał więc funkcjonować jako swego rodzaju rada delegatów. To z~kolei podniosło kwestię zasad jedności, ponieważ technicznie członkowie grupy mogli blokować propozycje na podstawie tych zasad. Tutaj przedstawiciele wymyślili listę zasadniczo wzorowaną na ,,znakach rozpoznawczych'' Ludowej Globalnej Akcji (PGA) i, podobnie jak oni, starannie przygotowaną, aby ucieleśniać anarchistyczne zasady, jednak nigdy nie odwołując się do żadnej konkretnej ideologii politycznej. Chodziło o to, aby pozostawić sprawy maksymalnie otwarte, aby móc utrzymać szeroką koalicję anarchistów (zwłaszcza tych z~tego, co nazywam ,,małym'', ekologów, organizacji pozarządowych i~działaczy związkowych, która okazała się tak skuteczne w~Seattle. Wszystko było więc celowo zwięzłe:

\noindent MISJA CONTINENTAL DAN

 Jesteśmy kontynentalną siecią zaangażowaną w~przezwyciężanie korporacyjnej globalizacji i~wszelkich form ucisku. Jesteśmy częścią rozwijającego się ruchu zjednoczonego we wspólnej trosce o sprawiedliwość, wolność, pokój i~zrównoważony rozwój wszelkiego życia oraz zobowiązanie do podjęcia bezpośrednich działań w~celu urzeczywistnienia radykalnej, wizjonerskiej zmiany.

\noindent KONTYNENTALNE ZASADY JEDNOŚCI DAN

 DAN przyjmuje następujące Zasady Jedności inspirowane i~wywodzące się z~zasad międzynarodowej Globalnej Sieci Działań Ludowych:

\begin{itemize}

\item Odrzucenie neoliberalnej polityki i~instytucji promujących niszczącą społecznie i~ekologicznie globalizację.

\item Konfrontacyjna postawa wobec niedemokratycznych instytucji, w~tym rządów i~korporacji, w~których kapitał jest jedynym prawdziwym twórcą polityki.

\item Wezwanie do pokojowych działań bezpośrednich, obywatelskiego nieposłuszeństwa i~budowania lokalnych alternatyw przez miejscową ludność.

\item Filozofia organizacji oparta na decentralizacji, demokracji bezpośredniej i~autonomii lokalnej.

\item Odrzucenie wszelkich form hierarchii, ucisku i~wyzysku.

\item Zaangażowanie w~solidarną pracę lokalną i~międzynarodową w~celu zbudowania popularnego ruchu na rzecz radykalnych zmian społecznych i~globalnej sprawiedliwości.
\end{itemize}

 Jednak PGA było -- jak zauważył Olivier de Marcellus w~Rozdziale 1 -- niezwykle luźną siecią. Musiało tak być, składać się z~grup, które były nie tylko rozproszone po całym świecie, ale wahały się od małych kolektywów skłotersów w~Barcelonie po organizacje takie jak KRRS, z~dziesięciu milionami członków. Sama PGA była niewiele więcej niż zbiorem zasad i~w dużej mierze nieformalną siecią komunikacji. DAN, składający się z~jednostek mniej więcej tej samej wielkości i~natury, mógł dążyć do tego, by stać się czymś więcej.

 Problem polegał na tym, że nigdy nie było jasne, czy na poziomie kontynentalnym istnieje coś, co naprawdę wymaga koordynacji. Rozpowszechnianie informacji o masowych mobilizacjach lub rozpowszechnianie obrazów i~literatury można dość łatwo osiągnąć za pośrednictwem Internetu; transport i~organizacja konwergencji mogłyby być zorganizowane przez nieformalne sieci, które już istniały; kilkunastu delegatów, którzy uczestniczyli w~odbywających się co dwa tygodnie telekonferencjach CDAN, szybko zorientowali się, że poza przypadkami, gdy nowe grupy chciały dołączyć do sieci, tak naprawdę nie trzeba było podejmować żadnych decyzji. CDAN był eksperymentem organizacyjnym, który istniał głównie dla własnego istnienia. Lub, mówiąc bardziej łaskawie, jego celem było umieszczenie nazwy i~tożsamości organizacyjnej w~nieformalnych sieciach, które równie dobrze działałyby bez nich. A wkrótce okazało się, że przyjęcie nazwy nie do końca jest pozytywne.

 To prawda, że  nazwa ,,DAN'' wchłonęła wiele prestiżu z~akcji w~Seattle. Ale nawet to przyniosło problemy. Seattle DAN koordynowało złożoną serię blokad i~zamknięć z~udziałem pięciu lub sześciu tysięcy aktywistów, z~których wszyscy zgodzili się na kodeks postępowania bez przemocy, który był szeroko publikowany w~radach delegatów i~centrach konwergencji. Jednak nawet w~Seattle byli tacy, którzy nie zgadzali się z~taką definicją niestosowania przemocy lub sprzeciwiali się koncepcji jakiejś grupy roszczącej sobie prawo do narzucenia kodeksu postępowania. Kilkuset anarchistów, wielu wywodzących się z~kolektywów i~grup afinicji z~Zachodniego Wybrzeża, które były zaangażowane w~działania na drzewach i~inne kampanie środowiskowe w~regionie, odmówiło udziału w~radach delegatów DAN i~stało się rdzeniem słynnego Seattle Black Bloc. Drugiego dnia akcji po tym, jak spotkania zostały skutecznie zakończone przez blokady, a policja zaczęła atakować blokujących, Czarny Blok rozpoczął kampanię ukierunkowanego niszczenia mienia. Okna Starbucksa i~Citibanku zostały rozbite, najechano Niketown. Obrazy rozbłysły na całym świecie. W mediach, gdy tylko szyby zaczęły się wybijać, blokady i~izolacje w~zasadzie zniknęły, tak że działania Czarnego Bloku, które rozpoczęły się dopiero długo po tym, jak policja zaczęła bić, gazować i~rozpylać gaz pieprzowy w~grupach afinicji DAN, stały się, z~mocą wsteczną, uzasadnieniem dla wszystkiego, co policja zrobiła wcześniej. Rzecznicy DAN skarżyli się z~oburzeniem prasie, że nie mają nic wspólnego z~Blokiem, a nawet otwarcie potępili ,,wandalizm''. Niektórzy twierdzili nawet, że wskazali policji ludzi z~Czarnych Bloków w~celu aresztowania.

 Jak można sobie wyobrazić, sytuacja ta doprowadziła do różnego rodzaju obelg i~niechęci między aktywistami.

 W tym, co stało się standardową medialną wersją tego wydarzenia, Czarny Blok stał się ,,anarchistami'', reprezentowanymi jako ,,gwałtowny'' margines z~Eugene w~stanie Oregon, oczekujący całkowitego zniszczenia cywilizacji technologicznej. DAN lub podobne grupy zasadniczo zniknęły, połączyły się w~,,protestujących bez przemocy '', prawdopodobnie maszerujących wokół z~tablicami. Wszystko to przesłania to, co naprawdę było kłótnią między anarchistami o definicję niestosowania przemocy. Zwolennicy Czarnych Bloków w~efekcie proponowali, że ,,przemoc'' należy zdefiniować jako wyrządzanie krzywdy lub cierpienia żywym stworzeniom; według tego standardu wyrzucanie kawiarni prowadzonej przez właściciela można bezsprzecznie określić jako ,,gwałtowne'', ponieważ podkopało to źródło utrzymania właściciela -- ale wyrzucenie Starbucksa nie. Wielu przedstawiło ten argument całkiem wyraźnie. Wielu po drugiej stronie, w~tym takie znane w~całym kraju postacie jak Medea Benjamin z~Global Exchange i~pogańscy anarchiści, tacy jak Starhawk, albo kwestionowali logikę, albo argumentowali, że jest to nieistotne: społeczeństwo postrzega niszczenie własności jako brutalne, daje policji uzasadnienie do atakowania wszystkich bezkrytycznie, i~to pozwala mediom bezustannie skupiać się na obrazach zniszczenia i~ignorować rzeczywiste przesłanie, które próbowali przekazać protestujący. Obrońcy taktyki Czarnego Bloku odpowiedzieli, że nie jest to gra o sumie zerowej: gdyby nie niszczenie mienia, media w~ogóle by nie donosiły o wydarzeniu. Rzecznicy Seattle DAN oskarżyli Czarny Blok o pogwałcenie solidarności poprzez odmowę udziału w~spotkaniach lub przestrzegania uzgodnionego kodeksu postępowania. Aktywiści Czarnego Bloku argumentowali, że od początku nigdy nie uzgodnili kodeksu postępowania i~oskarżyli pacyfistów o skandaliczny brak solidarności wobec wskazywania ich policji.

 Z kolei spory o niszczenie mienia zastąpiły szereg innych pytań: głównie o autonomię organizacyjną. Można to zobaczyć całkiem wyraźnie, patrząc na to, co wydarzyło się w~samej DAN. Oddziały DAN działające w~różnych miastach wkrótce zostały podzielone na dwie szerokie tendencje: anty-korporacyjne lub antykapitalistyczne. Ci pierwsi mieli raczej orientację reformistyczną, byli bardziej zorientowani na tradycję obywatelskiego nieposłuszeństwa i~podejrzliwi wobec bardziej wojowniczych stylów akcji bezpośredniej, bardziej zainteresowani odwoływaniem się do klasy średniej wokół takich koncepcji, jak sprawiedliwy handel i~zielony konsumpcjonizm. Ci ostatni byli wyraźniej anarchistyczni i~rewolucyjni. Najbardziej znanymi przykładami tej wcześniejszej tendencji były Seattle DAN i~LA DAN, które nadal były zdominowane przez działaczy organizacji pozarządowych lub, w~każdym razie, wśród nich było wielu, którzy stanęli na granicy między światem pozarządowym a anarchistycznym. W rezultacie mieli tendencję do zachowywania wrogiego stosunku do taktyk bojowych, argumentując, że zraziliby potencjalnych sojuszników, takich jak grupy robotnicze lub społeczności kolorowe. Często podejmowali aktywne działania, aby trzymać Czarne Bloki z~dala od ich działań. Szereg sprzymierzonych grup, które formalnie nie były częścią sieci DAN, takich jak Mobilization for Global Justice (MobGlob) w~Waszyngtonie, lub New England Global Action Network (NEGAN) miały mniej więcej ten sam skład i~miały podobne podejście. Jednakże ogromna większość grup, które należały do  sieci DAN, w~tym NYC DAN, Philadelphia Direct Action Group (PDAG), DAN z~San Francisco i~hrabstwa Humboldt, Chicago DAN i~wiele innych, była wyraźnie antykapitalistyczna. Mieli niewielki udział organizacji pozarządowych, ale zamiast tego składali się głównie z~niezależnych aktywistów i~członków lokalnych kolektywów anarchistycznych, w~większości byli to anarchiści przez małe ,,a''. Prawie w~każdym przypadku te antykapitalistyczne odłamy ostatecznie zaakceptowały definicję niestosowania przemocy według Seattle Czarnego Bloku i~często ściśle współpracowały z~kolektywami, które faworyzowały taktykę Czarnego Bloku.

 Zwłaszcza na Wschodnim Wybrzeżu wiele napięć wywodzących się z~Seattle zostało załagodzonych w~trakcie kolejnych trzech głównych mobilizacji. Podczas blokad spotkań IMF i~Banku Światowego w~Waszyngtonie z~16 kwietnia, zorganizowanych przez DAN, istniał ogromny Czarny Blok (,,Rewolucyjny Blok Antykapitalistyczny''), liczący może dwa tysiące ludzi. Nie uczestniczyli w~radzie DAN, ale w~swoich własnych radach zgodzili się na politykę unikania niszczenia własności i~wspierania blokad DAN\footnote{Było to częściowo spowodowane powszechną krytyką Czarnego Bloku Seattle za wybijanie szyb na ulicach zajętych przez blokady, a potem, gdy przybyła policja, uciekania, zamiast wsparcia swoich kolegów aktywistów.}. Blok spędzał większość czasu budując barykady i~konfrontując się z~policją. Podczas działań przeciwko Konwencji Republikańskiej w~Filadelfii z~1 sierpnia istniała przynajmniej cicha koordynacja: blok (tutaj nazwany ,,Rewolucyjnym Blokiem Antyautorytarnym'' zgodził się odciągnąć uwagę policji, przemieszczając się przez jedną część miasta, podczas gdy blokady, zorganizowane przez DAN i~koalicję innych grup, powstawały w~innym. Inauguracyjne protesty w~styczniu 2001 roku okazały się kolejnym kamieniem milowym. Po odkryciu, że Justice Action Network (JAN), lokalna grupa DC pospiesznie zorganizowana dla protestów, w~ogóle nie planowała żadnej bezpośredniej akcji, większość członków oddziałów DAN w~Nowym Jorku i~Filadelfii porzuciła swoich niegdysiejszych sojuszników i~przystąpienie do Rewolucyjnego Bloku Antyautorytarnego, który przebił się przez barykady otaczające trasę parady prezydenckiej i~na krótko zatrzymał konwój Busha podczas tego, co było później nazwane ,,bitwą o pomnik marynarki wojennej''.

 Podobne rzeczy działy się na Zachodnim Wybrzeżu. Podczas gdy grupy takie jak Global Exchange i~Ruckus Society, zasadniczo organizacje pozarządowe, które (jak dawniej Greenpeace) były skłonne zastosować akcję bezpośrednią, gdy uważały, że jest to taktycznie właściwe, nadal izolowały się, publicznie potępiając niszczenie własności jako ,,przemoc'' (stanowisko to, jeśli nic innego, byli zmuszeni przyjąć, aby nie zrazić swojej bazy finansowej). Starhawk i~inni związani z~Klasterem Pogańskim, autonomiczną grupą, stopniowo ustanowili milczący sojusz z~lokalnymi Czarnymi Blokami. Do czasu Quebecu, na początku 2001 roku, antykorporacyjne DAN zasadniczo zaniknęło. Mimo to, w~oczach większości anarchistów w~stylu Czarnego Bloku, nazwa ,,DAN'' była nierozerwalnie związana z~tymi potępiającymi niszczenie własności w~Seattle. Bez względu na to, jakie stanowiska zajęli działacze DAN, nazwa była postrzegana co najwyżej z~podejrzliwością, często z~wrogością, na scenie hardcore skłotersów lub tych, którzy od dawna pracowali z~infoshopami i~Food Not Bombs lub podobnymi projektami, i~którzy często postrzegali aktywistów DAN jako medialnie szczęśliwych nowicjuszy ,,ze swoimi telefonami komórkowymi i~laptopami'', jak to ujął jeden z~nich, torujący sobie drogę do ustalonej sceny\footnote{Szereg przykładów można znaleźć w~The Black Bloc Papers (David i~X 2002).}.

 Myślę, że to było prawdziwe niebezpieczeństwo związane z~próbą umieszczenia nazwy w~nieformalnych sieciach. W kręgach akcji bezpośredniej nazwane grupy mają tendencję do gromadzenia złych skojarzeń. Wielu postrzega wszelkie próby sformalizowania sieci lub koalicji jako próby stworzenia de facto struktury przywódczej, aby dać niektórym jednostkom możliwość ,,przemawiania w~imieniu grupy'' i~przypisywania sobie zasług za czyny lub osiągnięcia innych. Nawet ci, którzy nie widzą rzeczy w~ten sposób, mają tendencję do przyjmowania systemu moralnej księgowości, w~którym grupom jest niezwykle trudno akumulować kredyty, a łatwo akumulować debety. To samo wydarzyło się, gdy DAN próbowała ustanowić sojusze z~radykalnymi grupami opartymi na społecznościach kolorowych: niewrażliwe, wstrętne lub rasistowskie zachowania poszczególnych członków były zwykle pamiętane ze złością i~identyfikowane z~DAN jako grupą; dramatyczne akty solidarności i~samopoświęcenia były zwykle pamiętane jako akty poszczególnych jednostek. Z czasem nazwa stała się ciężarem. Ponieważ nie było powodów do utrzymywania sieci, pod koniec 2001 roku CDAN faktycznie upadł. Większość lokalnych DAN miała pójść w~ich ślady wkrótce potem lub, częściej, wrócić do bycia w~dużej mierze nieformalnymi sieciami, z~których powstały.

\medskip
\noindent SZCZEGÓŁY O NEW YORK DAN

 Przez większość lat osiemdziesiątych i~dziewięćdziesiątych największa energia na scenie akcji bezpośrednich w~Nowym Jorku była związana z~kryzysem AIDS. Pod koniec lat Reagana ACT UP organizowało cotygodniowe spotkania z~setkami uczestników i~angażowało się w~szereg akcji w~całym mieście: protesty, strajki okupacyjne, blokady, zrzucanie sztandarów i~tak dalej. W latach 90. w~Nowym Jorku pojawił się także jeden z~głównych krajowych oddziałów Love and Rage, projektu stworzenia ogólnokrajowej sieci rewolucyjnych anarchistów, zbudowanej głównie wokół tygodnika o tej samej nazwie. Podobnie jak wiele takich grup, Love \& Rage popadło w~wewnętrzne konflikty i~ostatecznie rozwiązało się z~powodu kwestii przywilejów białych w~1998 roku.

 Na początku 2000 roku ACT UP było cieniem swojej dawnej osobowości, Love and Rage zniknęło\footnote{Wielu z~jej byłych członków stało się maoistami - wielu zaczęło współpracować z~SLAM (Student Liberation Action Movement), radykalną grupą, która przejęła kontrolę nad samorządem studenckim w~Hunter College.}. Ludzie, którzy zebrali się, aby stać się rdzeniem DAN, pochodzili głównie z~nowszych grup, takich jak Reclaim the Streets (RTS), Kolektyw z~Lower East Side i, do pewnego stopnia, Nowojorscy Industrial Workers of the World (IWW). Pierwszym był luźny kolektyw inspirowany bardzo znaną grupą o tej samej nazwie, afiliowaną w~PGA, działającą w~Londynie. W Wielkiej Brytanii RTS wyłonił się z~konwergencji kampanii antydrogowych ze sceną rave i~zasłynął z~organizowania dzikich imprez ulicznych, blokowania autostrad i~innych działań wokół szerszego tematu ponownego zawłaszczenia przestrzeni publicznej. W Nowym Jorku RTS narodziło się z~Lower East Side Collective w~1997 roku. Miało swoją główną bazę w~artystycznych kręgach bohemy w~Williamsburgu i~zorganizowało już kilka własnych nielegalnych imprez ulicznych na mniejszą skalę. RTS z~kolei nałożył się na Times Up!, grupą organizującą comiesięczne przejażdżki rowerowe w~ramach Masy Krytycznej. Kolektyw z~Lower East Side był szczególnie aktywny w~obronie ogrodów społecznych i~innych kampaniach przeciwko gentryfikacji. Z kolei Wobblies (IWW) byli bardziej robotniczy w~tle i~orientacji, i~mieli na celu zorganizowanie miejsc pracy, chociaż w~porównaniu z~Zachodnim Wybrzeżem byli stosunkowo nowicjuszami, nadal w~tym czasie zaangażowani w~ostatecznie bezowocną kampanię na rzecz uzwiązkowienia Borders Books\footnote{Od 2004 lub 2005 roku, kiedy IWW rozpoczęła kampanię Starbucks, wszystko to się zmieniło i~związek rozrósł się dramatycznie, zdobywając hale produkcyjne i~przyciągając również wielu imigrantów zarobkowych. Wszystko to wydarzyło się jednak później.}.

 Innymi słowy, ludzie, którzy stali się rdzeniem NYC DAN, w~większości nie pochodzili ani z~organizacji pozarządowych, które do tego czasu zajmowały się głównie kwestiami globalnego neoliberalizmu, ani z~sieci wyraźnie anarchistycznych instytucji, skupionych wokół sceny skłoterskiej, Blackout Books, Food Not Bombs i~tak dalej, które istniały w~Nowym Jorku przynajmniej od czasu zamieszek na Tompkins Square pod koniec lat 80.. Nie oznacza to, że przedstawiciele obu nie byli zainteresowani projektem, zwłaszcza na początku. Pierwotny okólnik, rozesłany do wszystkich nowojorskich listserwerów aktywistów, brzmiał:

\medskip
\noindent Dołącz do NYC Direct Action Network!\newline  \newline 31 stycznia 2000 20:45 \newline [oto ogłoszenie e-mail do skopiowania i~przekazania do swoich list]

\noindent PROSZĘ PRZESYŁAJCIE DALEKO i~SZEROKO

\textit{ Opierając się na sukcesach protestów Światowej Organizacji Handlu w~Seattle, różnorodna koalicja nowojorskich grup aktywistów łączy się w~celu wzajemnej pomocy. Tworzymy sieć, aby wspierać się nawzajem i~ułatwiać masowe mobilizacje w~różnych kwestiach, zaczynając od działań w~dniach 16-17 kwietnia przeciwko Bankowi Światowemu i~Międzynarodowemu Funduszowi Walutowemu}\ldots 

\medskip

 Pierwsze spotkania były niezwykle licznie odwiedzane: prawie każda radykalna grupa lub kolektyw brała w~nim udział. Początkową wizją było, aby DAN działała jako rodzaj szerokiej grupy parasolowej, skupiającej wszystko, od Food Not Bombs po radykalne grupy wsparcia queer. W rzeczywistości jedna z~pierwszych dużych debat dotyczyła tego, czy prowadzić samą DAN na zasadzie rady delegatów. Ponieważ nie była to organizacja, ale sieć, Brooke i~inni zaproponowali, aby walne zebrania DAN składały się tylko ze zgromadzenia upoważnionych przedstawicieli z~grup roboczych i~istniejących kolektywów. Inni uważali, że uniemożliwiłoby to DAN spełnianie jednej z~jej najbardziej oczywistych funkcji: zapewnienie miejsca dla nowych ludzi, aby mogli połączyć się ze sceną aktywistów i~połączyć się z~kolektywami i~grupami roboczymi. Bob, który miał wieloletnie doświadczenie z~ACT UP, zamiast tego zaproponował podejście jako pewną wariację ,,masowego spotkania'' czy ,,ogólnej rady'', która, jak napisał w~email w~tamtym czasie:

\medskip

\textit{ działała całkiem dobrze nawet pod koniec lat 80., kiedy nasze spotkania co tydzień przyciągały 500-800 osób. Myślę, że klucze do sukcesu są trojakie:}

\textit{ 1) energiczne grupy robocze, które opracowują większość strategii i~muszą je tylko wprowadzić do grupy w~celu uzyskania szerokiego poparcia politycznego (lub odrzucenia, modyfikacji lub uzupełnienia)}

\textit{ 2) silne facylitowanie masowych spotkań oraz}

\textit{ 3) przestrzeganie ustalonych terminów na posiedzeniach.}

 \textit{i~oczywiście duch zaufania i~dobrej woli w~podejmowaniu decyzji w~drodze konsensusu.}
 
 \medskip
 
 Ostatecznie taki został wybrany model i~coś podobnego zostało wybrane w~większości innych miast, w~których pojawiły się DAN. Działało to całkiem dobrze w~przypadku dwóch rzeczy, do których najlepiej nadawała się DAN -- pomocy w~organizacji masowych mobilizacji i~rozpowszechniania pewnego modelu procesu demokratycznego -- ale zapewniło, że NYC DAN z~czasem będzie coraz mniej przypominać sieć i~coraz bardziej przypominać organizację. Mimo to zawsze łączyła elementy obu. Członkostwo było otwarte. Każdy mógł pojawić się na walnych zebraniach DAN, które odbywały się w~każdą niedzielę w~Charas; stamtąd można było kierować nowych ludzi do grup roboczych, które odzwierciedlałyby ich zainteresowania. Podczas gdy pełna lista grup roboczych zmieniała się w~czasie, podstawowa lista wyglądała mniej więcej tak:

 \textit{NUTS \& BOLTS} -- dba o wszystko, co potrzebne do spotkań

 \textit{FINANSE} -- utrzymuje skarbiec, organizuje zbiórki pieniędzy

 \textit{KOMUNIKACJA}{}- prowadzi stronę internetową i~listserv

 \textit{OUTREACH} -- przygotowywanie broszur, plakatów, propaganda

 \textit{PRAWNY}{}- konsultuje różne kwestie prawne związane z~protestami

 \textit{KONTYNENTALNY DAN}{}- zespół rotujących przedstawicieli, którzy brali udział w~odbywających się co dwa tygodnie telekonferencjach i~wykonywali bieżącą pracę nad pisaniem i~przepisywaniem statutu

 \textit{GRUPA ROBOCZA SOLIDARNOŚCI PRACY}{}- składał się z~kilkunastu aktywistów, którzy spotkali się w~ABC No Rio. Partia Pracy DAN, jak ją powszechnie nazywano, była próbą utrzymania sojuszu ,,Turtles and Teamsters'' założonego w~Seattle przy udzielaniu poparcia strajkom i~innym kampaniom związkowym. Było na to wiele okazji, ponieważ to, co związkom zawodowym wolno, a czego nie wolno robić, jest dokładnie uregulowane. Związkom zawodowym zabrania się, na przykład, grożenia wtórnym bojkotem lub ustanawiania linii pikietowania przeciwko tym, którzy zaopatrują lub zawierają kontrakty ze strajkowanymi firmami, ale nie ma sposobu, aby temu zapobiec całkowicie niezależnej grupie, takiej jak DAN Labor. Ponadto DAN zapewniała wsparcie strajkującym, na przykład dostarczając lalki i~teatry uliczne na wiece, parady i~linie pikiet. Niektórzy aktywni członkowie byli również zaangażowani w~IWW lub ISO; prawie wszyscy byli pochodzenia robotniczego.

 \textit{GRUPA ROBOCZA POLICJI i~WIĘZIEŃ} -- grupa mniej więcej podobnej wielkości, która działała głównie na rzecz wsparcia kampanii organizacji społecznych i~grup aktywistów w~całym mieście, które w~dużej mierze składały się z~osób kolorowych, i~prowadziła kampanię przeciwko brutalności policji i~tym, co określano w~kręgach aktywistów jako Więzienny Kompleks Przemysłowy. Sama grupa była całkowicie biała i~podobnie jak DAN Labor postrzegała swoją rolę jako dostarczanie zasobów do kampanii inicjowanych przez innych, w~tym przypadku próbując wykorzystać fakt, że biali aktywiści byli w~stanie stosować taktykę akcji bezpośredniej przy znacznie mniejszym ryzyku aresztowania niż ich sojuszników, ze względu na bardzo systemowy rasizm, przeciwko któremu prowadzili kampanię.

 Były inne grupy, które organizowały się wokół poszczególnych trwających kampanii -- między innymi grupa obligacji Banku Światowego, kampania Genetycznie Zmodyfikowanej Żywności -- ale były one niewielkie lub krótkotrwałe. W końcu, w~dowolnym momencie, miała miejsce jakakolwiek liczba innych kampanii w~tej chwili, począwszy od próby zamknięcia spotkań IMF/Banku Światowego w~kwietniu 2000 roku, pierwotnej racji bytu nowojorskiego DAN. A16 uznano za mieszany sukces. Nie było tak spektakularne, powalające zwycięstwo jak Seattle. Spotkania nie zostały anulowane. Ale była to ogromna i~udana mobilizacja, która sprawiła, że  wszyscy poczuli, że ruch jest skuteczny i~rozwija się, budowane są sojusze, a ponadto udało mu się uczynić narodową kwestią z~roli międzynarodowych instytucji finansowych, której większość Amerykanów do tej pory nawet nie miała świadomości.

 Sprawy stały się trudniejsze, kiedy CDAN postanowił wkrótce potem zaplanować symetryczne działania w~Los Angeles i~Filadelfii, przeciwko Narodowej Konwencji Demokratów i~Republikanów. Wiązało się to z~budowaniem sojuszu i~DAN wkrótce odkryła, że  jej wybrani sojusznicy (Ruch Akcji Wyzwolenia Studentów, z~siedzibą w~Hunter College i~Koalicja Mumia) mieli bardzo różne poglądy na temat politycznego ukierunkowania działań. CDAN pierwotnie wymyślił ideę równoczesnych działań przeciwko konwencji republikańskiej i~demokratycznej jako sposób na podkreślenie z~natury niedemokratycznej natury amerykańskiego systemu wyborczego -- zakwestionowania w~rzeczywistości samej definicji ,,demokracji'' -- jego sojusznicy mieli bardziej bezpośrednie obawy . W końcu, po wielu napięciach i~kłótniach wewnętrznych, NYC DAN zgodził się, że działania w~Filadelfii będą miały na celu skupienie uwagi opinii publicznej na amerykańskim więziennym kompleksie przemysłowym. Sojusznicy DAN również nie chcieli organizować się pod nazwą DAN, w~przeważającej mierze białej grupy, więc teoretycznie działania przeciwko RNC w~Filadelfii były prowadzone przez nowo wymyślony podmiot zwany ,,Koalicją 1 sierpnia'', od proponowanego dnia akcji. Same akcje okazały się mieszanym sukcesem, a w~kategoriach medialnych czymś w~rodzaju katastrofy. Podczas gdy w~Seattle i~Waszyngtonie było prawie niemożliwe, aby media głównego nurtu wyjaśniły, \textit{dlaczego }protestujemy, działania te spowodowały uczynienie problemu z~samego istnienia instytucji -- WTO, IMF, Banku Światowego -- które większość Amerykanów nawet o tym nie wiedziała. W efekcie wystarczyło wskazać. W Philly nawet wskazywanie nie działało. Pomimo wysiłków bardzo doświadczonego zespołu medialnego, nie udało nam się nawet zmusić prasy do wzmianki o takich sformułowaniach jak ,,więzienny kompleks przemysłowy''. Sporo aktywistów odpadło po Philly z~powodu rasizmu.

 Québec również doprowadził do problemów -- podczas gdy same działania w~Quebecu były spektakularnym sukcesem, bardzo niewielu nowojorczyków do nich dotarło, a Akwesasne było katastrofą, która pozostawiła wiele oskarżeń z~powodu tego, co poszło nie tak. W tym momencie DAN nie była już siecią, była grupą inicjującą koalicje. Potem nadszedł 11 września. To oczywiście zaszokowało społeczność aktywistów w~samym Nowym Jorku bardziej niż gdziekolwiek indziej: aktywiści musieli radzić sobie z~tym samym żalem i~paranoją, co inni nowojorczycy, z~dodatkową obawą, że ich ruchy miały być systematycznie stłumione przez nowe organy bezpieczeństwa narodowego. W czasie pospiesznie zmontowanych akcji przeciwko Światowemu Forum Ekonomicznemu, które odbyło się kilka miesięcy później w~Waldorf Astoria na Manhattanie, w~rzeczywistości istniały dwie różne koalicje zainicjowane w~dużej mierze przez członków DAN: jeden o nazwie Inny świat jest możliwy (AWIP), który zakończył się zorganizowaniem marszu, oraz Konwergencja Antykapitalistyczna (ACC), bardziej radykalna grupa, która zaplanowała w~dużej mierze nieudane akcje bezpośrednie. W tym momencie znacznie zmniejszony DAN stał się zasadniczo centrum aktywistów, którzy wiedzieli, jak tworzyć większe koalicje i~nawet nie aspirowali do bycia siecią obejmującą każdy aspekt organizowania się w~mieście. Wkrótce popadł w~rodzaj nieuleczalnego kryzysu związanego ze swoim statusem. Czy DAN była grupą? Czy była to bardziej ograniczona sieć? Czy powinien wrócić do swojej pierwotnej wizji? Były pewne grupy -- na przykład CLAC w~Montrealu -- które przetrwały kryzys, wracając do modelu rady delegatów, który DAN odrzuciła na samym początku. Wielu członków NYC DAN naciskało na coś w~tym kierunku, ale ostatecznie ich argumenty nie sprawdziły się: po części dlatego, że prawdziwym rdzeniem DAN nie były w~tym momencie grupy robocze, ale General DAN, który stał się rodzajem puli zasobów aktywistów, umiejętności prawnych, organizacyjnych, medialnych i~tak dalej. Pod koniec 2002 roku najpierw Policja i~Więzienia, potem Partia Pracy, popadły w~kryzys i~rozwiązały się, a na początku następnego roku sama DAN już formalnie nie istniała: chociaż zasadniczo byli to ci sami ludzie, którzy odgrywali kluczową rolę w~prawie wszystkie najbardziej radykalnych koalicjach pokojowych, międzynarodowych grupach solidarnościowych i~radykalnych grupach protestacyjnych w~następnych latach. DAN jako model rzeczywiście rozprzestrzenił się wszędzie.

\section{Część II: Proces}

\noindent KONSENSUS i~FACYLITACJA

 Pozwolę sobie teraz przejść do niektórych etnograficznych podstaw: dynamiki spotkań, konsensusu i~sztuki facylitacji. Jak obiecałem, zacznę od pierwszego szkolenia facylitacji, w~którym sam uczestniczyłem, wiosną 2000. To było szkolenie DAN: ponieważ DAN regularnie rotował facylitatorów, uważano, że każdy w~DAN powinien przynajmniej być zdolny do odgrywania tej roli. Ekipa składała się z~trzech trenerów, Maca i~Lesleya, dwóch mieszkańców Toronto, którzy byli z~NYC DAN od jej powstania, oraz Jima, czterdziestoletniego działacza, który wtedy pracował z~Hudson Valley DAN, wraz z~około tuzinem stażystów aktywistów. Wszyscy byli stosunkowo niedawnymi rekrutami DAN, począwszy od Chrisa, siedemnastoletniego gitarzysty punkowego, po Nat, kobietę po siedemdziesiątce, od dawna aktywną w~grupach marksistowskich, która w~ciągu ostatnich kilku lat coraz bardziej angażowała się w~grupy anarchistyczne. Wszyscy mieli przynajmniej jakieś doświadczenie w~procesie konsensusu i~byli zaznajomieni przynajmniej z~jedną teorią konsensusu.

\bigskip

\noindent \textbf{Szkolenie facylitacyjne, Charas El Bohio}

\noindent \textbf{niedziela, 21 maja 2000 roku}

\bigskip

 \textit{[}Zaczynamy od planu, w~którym każdy, trzech trenerów i~ośmiu lub dziewięciu praktykantów, podsumowuje swoje własne doświadczenia z~konsensusem, który rozciąga się od pracy w~spółdzielniach żywnościowych lub księgarniach spółdzielczych po szkolenie w~akcji bezpośredniej na A16 lub obserwowanie rad w~Seattle. Nigdy nie wiadomo, zauważył Jim, kiedy możesz znaleźć się w~sytuacji, która wymaga umiejętności facylitacji, zwłaszcza w~mniejszych grupach. Są ludzie, którzy są po prostu dobrymi facylitatorami, ale każdy może się tego nauczyć, a jeśli nie, to niesprawiedliwe jest oczekiwanie, że ci sami ludzie będą musieli to robić przez cały czas.\textit{]}

 \textit{[}Lesley wyjaśnia porządek obrad, wskazując na kartkę papieru plakatowego przyklejoną do ściany.]

\medskip
 PORZĄDEK OBRAD

 1. WPROWADZENIE i~DOŚWIADCZENIE (10 min.)

 2. CO DZIAŁA, A CO NIE (20 min.)

 3. KONSENSUS -- CO TO JEST? (20 minut.)

 4. NARZĘDZIA FACYLITACJI (30 min.)

 5. ROLE PLAY(10 min.)

 6. PROGRAM DAN (20 min.)

 7. INFORMACJE ZWROTNE (otwarte)

\bigskip

\noindent \textit{Mac}: Pomyślałem, że może powinniśmy zacząć wyjaśniać, dlaczego organizujemy szkolenie w~ten sposób. W pewnym sensie oparliśmy nasze podejście na popularnych modelach edukacyjnych, w~których chodzi o to, aby najpierw stworzyć wspólną analizę tego, jak każdy z~nas coś widzi -- w~tym przypadku konsensus -- a następnie spróbować wprowadzić tę analizę w~życie.

\noindent \textit{Lesley}: W każdym razie taki jest pomysł. Zobaczymy, czy to faktycznie działa.

\noindent \textit{Mac}: Więc model, jak prowadzić warsztaty, który dostaliśmy z~tej książki\ldots  

\noindent [\textit{rozdaje kserokopie}] 

Właściwie nigdy wcześniej tego nie robiliśmy, więc jest to rodzaj eksperymentu. Jeśli to nie zadziała, zawsze możemy spróbować czegoś innego.

\medskip
\noindent CO DZIAŁA? CO NIE DZIAŁA?

\noindent \textit{Jim}: Dobrze, więc, czy powinniśmy zacząć od poproszenia ludzi, aby porozmawiali o szczególnie skutecznych formach konsensusowego podejmowania decyzji, które zaobserwowali, momentach, podejściach lub technikach, które według nich zadziałały naprawdę dobrze, i~co im się w~nich podobało? Potem możemy przejść do rzeczy, które nie działały tak dobrze.

\noindent Neala: Bardzo lubię, kiedy na początku spotkania poświęca się trochę czasu, żeby się poznać. Na wielu spotkaniach; wszyscy po prostu przeskakują do sedna sprawy, a nie ma sposobu, aby uspokoić ludzi, stworzyć poczucie wzajemnego spokoju, zacząć rozwijać poczucie grupowego umysłu.

\noindent \textit{Jim}: Więc mówisz o jakimś lodołamaczu?

\noindent Christa: Tak, czy jest to ćwiczenie ze słuchania, gdzie wszyscy łączą się w~pary i~jeden powinien przez minutę porozmawiać o czymś, o czym często myślał tego dnia, a drugiemu nie wolno nic mówić, ale po prostu musi słuchać, a potem się zmieniają. Albo coś głupiego, jak wtedy, gdy wszyscy chodzą w~kółko i~mówią, jakim zwierzęciem chcieliby być.

\noindent Sara: Albo jeśli to ludzie, którzy się nie znają, po prostu, dlaczego zdecydowali się przyjść na spotkanie.

\noindent Neala: Albo co chcieliby z~tego wynieść.

\noindent \textit{[}Lesley ma czystą kartkę z~napisem ,,działa/nie działa'' u góry i~magiczny marker, którym kapryśnie macha w~powietrzu. Zatrzymuje się i~pisze ,,lodołamacze'' \textit{]}

\noindent  Sara: Bardzo lubię sesje burzy mózgów, jak to nazywasz? {\textquotedbl}Popcorn{\textquotedbl}. Kiedy przeznaczasz dziesięć minut, kiedy wszyscy będą mogli po prostu wykrzyczeć pomysły, cokolwiek przyjdzie im do głowy, bez względu na to, jak głupie lub śmieszne, i~nikomu nie wolno komentować ani krytykować ich, ale może zawołać tylko swój. W takich chwilach czuję, że naprawdę jestem w~obecności umysłu grupowego\ldots  cóż, zwłaszcza gdy po sesji burzy mózgów możesz zacząć łatać propozycję, która łączy wszystkie najlepsze pomysły.

\noindent  Mark: Przekształcenie propozycji. Nie potrafię powiedzieć, ile razy przesiedziałem dziesięcio, piętnastominutową kłótnię i~okazało się, że jedynym powodem, dla którego ludzie się kłócili, jest to, że nie rozumieli, o co tak naprawdę chodzi. Ludzie wciąż zgłaszają wszelkiego rodzaju obawy i~zastrzeżenia, które okazują się całkowicie nieistotne, gdy przypomni się im rzeczywiste sformułowanie rzeczy.

\noindent  \textit{Lesley}: [\textit{długopis w~dłoni, zmarszczone brwi}] To brzmi tak, jakby rzeczywiście dotyczyło kolumny ,,co nie działa''. Co myślisz?

\noindent  Christa: Dlaczego nie postawimy tego na obu: ,,zagubienie idei propozycji'' po jednej stronie, ,,przekształcanie'' po drugiej. [\textit{Ona pisze}.]

\noindent  Walter: Myślę, że to naprawdę przydatne, gdy facylitator wkracza, by przypomnieć wszystkim o ich wspólnych cechach, czy chodzi o zasady grupy, czy o powody, dla których próbujemy dojść do porozumienia na początku. Widziałem kilka razy, kiedy wydawało mi się, że wszyscy byli skłóceni o jakiś drobny punkt, albo schodziło się to w~jakiś rodzaj głupiego konkursu sikania, a jeśli masz zręcznego facylitatora, może wkroczyć, by subtelnie przypomnieć wszystkim, dlaczego są wszystko tutaj w~taki sposób, że cała sprawa po prostu rozpływa się i~wydaje się głupia.

\noindent  Megan: Lub bardziej ogólnie, gdy facylitator jest w~stanie upewnić się, że wszyscy pozostaną w~trybie rozwiązywania problemów, a nie w~trybie debaty. (Może powinieneś to napisać.)

\noindent  George: A skoro już przy tym jesteśmy: kiedy facylitator pamięta o wyjaśnieniu, kiedy przemawia \textit{jako }facylitator, a kiedy wyraża swoją opinię. Myślę, że to naprawdę ważne, aby mieć takie zdanie, jak ,,pozwólcie, że się teraz wypowiem'', aby pokazać, że teraz przemawiają jako członek grupy, a nie jako osoba prowadząca spotkanie.

\noindent  \textit{Mac}: A nawet to powinno być ograniczone do minimum. W DAN zawsze mówimy ludziom, że jeśli mają przedstawić grupie propozycję, nie mogą również facylitować tego spotkania.

\noindent  Christa: Może to też powinno być po stronie ,,nie działa'', kiedy facylitatorzy przedstawiają swoje własne opinie\ldots 

\noindent  George: \ldots albo nie wyjaśniają jasno, że nie robią tego \textit{jako} facylitator.

\noindent  \textit{Lesley}: Po prostu napiszę to ponownie po obu stronach.

\noindent  \textit{Jim}: Więc myślę, że może nie ma sensu zajmować się najpierw dobrym procesem, a potem złym -- może powinniśmy po prostu uruchomić je oba razem, skoro i~tak to robimy.

\noindent  \textit{[}Wkrótce stworzyliśmy dość obszerną dwukolumnową listę ze szczególnie długą listą potencjalnych problemów, brak ograniczeń czasowych, ludzie, którzy lubią słuchać siebie, stronnicze facylitacje, spekulacyjne dyskusje na temat tego, co robić w~oparciu o nieprzewidziane okoliczności, które mogą nie mieć wpływu na to, co się faktycznie dzieje, złe wibracje, załamanie zaufania, i~szereg dodatkowych dobrych pomysłów na proces, od utrzymania parytetu płci wśród mówców po znaczenie posiadania kogoś w~pobliżu, aby powitać i~ukierunkować nowych ludzi, którzy nie rozumieją proces.]

\medskip
\noindent  CZYM JEST KONSENSUS?

\noindent  \textit{Mac}: Cóż, to było przydatne, jednym z~powodów, dla których lubimy zaczynać w~ten sposób, jest po prostu podsuwanie nam pomysłów, jak ulepszyć nasz własny proces w~DAN.

 Więc teraz zamierzaliśmy porozmawiać trochę o konsensusie, co odróżnia go od innych form podejmowania decyzji, w~szczególności głosowania i~rządów większości. Zacznę od odrzucenia sposobu, który jest dla mnie odmienny, czyli konsensusu jako procesu. Głosowanie to \textit{tylko }sposób podejmowania decyzji. Fakt, że kończysz sprawy głosowaniem, niekoniecznie mówi ci coś o procesie, który prowadzi do głosowania: chociaż zwykle jest to jakaś formalna debata, \textit{Zasady Porządku Roberta}. Konsensus to nie tylko sposób na podjęcie decyzji, a tak naprawdę nawet nie jest to przede wszystkim sposób na podjęcie decyzji. To proces. Sposób, w~jaki ludzie radzą sobie ze sobą, kładąc nacisk na wzajemny szacunek i~kreatywność oraz starając się, aby nikt nie był w~stanie narzucać swojej woli innym i~aby wszystkie głosy były słyszane. Jako proces, niekoniecznie jest to nawet najskuteczniejszy sposób podejmowania decyzji. Myślę, myślę, że większość z~nas tak myśli, jeśli jesteśmy zaangażowani w~DAN, że jest to proces, który z~największym prawdopodobieństwem doprowadzi do najmądrzejszej decyzji, ale powiedziałbym, że nawet jeśli czasami tak nie jest, jest to ważniejsze, aby podjąć decyzję w~prawdziwie egalitarnym procesie, niż za każdym razem wymyślać absolutnie idealny sposób działania. Decyzje można zwykle później zmienić. A są chwile, kiedy powiedziałbym nawet, że lepiej w~ogóle nie podejmować decyzji.

 Teraz jest tyle stylów konsensusu, ile jest grup. Grupy takie jak DAN stosują dość formalny proces, chociaż niektóre grupy stosują znacznie bardziej formalny, inne, mniejsze grupy są w~swoim procesie znacznie bardziej nieformalne.

\noindent \textit{Jim}: Chociaż wiesz, że stopień formalności zależy nie tylko od wielkości grupy, to także kwestia zażyłości. Widziałem całkiem spore grupy, które znają się od lat i~są przyzwyczajone do tego procesu, które zazwyczaj całkowicie rezygnują z~formalności.

\noindent \textit{Lesley}: Także nie mówimy, że konsensus zawsze będzie najlepszym sposobem na zrobienie rzeczy. Czasami najważniejsza jest wydajność, powiedzmy, że gliniarze idą prosto na ciebie i~musisz zdecydować, co zrobić. Lub gdy jest dużo ludzi pracujących, którzy po prostu nie mają czasu na długie spotkania. Lub gdy pracujesz z~sojusznikami o bardzo różnych tradycjach. Wiele grup osób kolorowych jest bardzo podejrzliwych wobec konsensusu. Postrzegają to jako chrupiącą białą granola, aw takiej sytuacji byłoby naprawdę aroganckie upierać się, że to jedyna droga.

\noindent \textit{Mac}: A są sytuacje, w~których konsensus po prostu nie zadziała. Kiedy w~Toronto organizowaliśmy bezdomnych, próbowaliśmy i~próbowaliśmy. Spotkania trwały w~nieskończoność, wszyscy wstawali i~wygłaszali przemówienia, nikt nie szanował stosu, ale przerywali i~kłócili się ze sobą\ldots 

\noindent George: [\textit{Śmieje się}] Brzmi jak banda starzejących się marksistów.

\noindent Megan: A właściwie większość anarchistów, którzy mają ponad czterdzieści lub pięćdziesiąt lat.

\noindent Sara: O Boże, w~zeszłym tygodniu byłam na Forum Brechta na spotkaniu Libertariańskiego Klubu Książki i~prawie wszyscy byli anarchistami starszego pokolenia. i~nie mogłem w~to uwierzyć: czy ci faceci kiedykolwiek słyszeli o procesie? A może nawet podstawowym szacunku dla innych ludzi? Wszyscy skakali na krzesłach i~odcinali się nawzajem, aw pewnym momencie przysięgam, że dwóch z~nich dosłownie krzyczało na siebie.

\noindent David: Tak, więc teraz wiesz, dlaczego trzymałem się z~dala od anarchistycznej polityki przez pierwsze trzydzieści osiem lat mojego życia.

\noindent \textit{Mac}: W każdym razie w~Toronto w~końcu po prostu się poddaliśmy i~przyjęliśmy inny proces.

\noindent \textit{Lesley}: Więc jak mam spisać twój punkt widzenia? ,,Proces kontra decyzja''?

\noindent \textit{Mac}: Tak to jest dobre. W każdym razie przepraszam, zagarnąłem głos. Ktoś jeszcze?

\noindent Chris: Cóż, myślę, że idea konsensusu polega na tym, że jest to sposób na poszukiwanie wspólnoty. Zaczynasz od założenia, że  wszyscy w~pokoju prawdopodobnie mają nieco inną perspektywę i~nie próbujesz tego zmienić, po prostu próbujesz sprawdzić, czy możesz stworzyć jakiś wspólny grunt.

\noindent Neala: Poza tym ma to być proces, w~którym każdy ma równe szanse uczestniczenia w~kształtowaniu ostatecznej decyzji. W przeciwieństwie do głosowania większościowego, w~którym zawsze kończysz z~jakąś wyalienowaną mniejszością, która głosowała przeciwko propozycji, ale i~tak po prostu muszą z~nią żyć. Każdy ma jakiś wkład, szansę zasugerowania zmian.

\noindent \textit{[}Lesley gryzmoli]

\noindent Jessica: Chociaż myślę, że to coś więcej. Zdarzało się, że byłam na spotkaniach i~pojawiła się propozycja, która mi się nie podobała, ale w~trakcie dyskusji stało się oczywiste, że prawie wszyscy uważają, że to naprawdę dobry pomysł. Odkryłam, że jest coś przyjemnego w~tym, że mogę po prostu odpuścić to, zdając sobie sprawę, że to, co myślę, niekoniecznie jest aż tak ważne, ponieważ naprawdę szanuję tych ludzi i~ufam im. To naprawdę dobre uczucie. Ale oczywiście czuję się dobrze tylko dlatego, że wiem, że to była \textit{moja }decyzja, że  mogłam zablokować tę propozycję, gdybym naprawdę chciała. Postanowiłem nie brać siebie zbyt poważnie. 

\noindent \textit{Lesley}: Więc jak mam to zapisać?

\noindent Jessica: Może\ldots  cóż, ,,egoizm'', ,,Konsensus osłabia egoizm'' Coś w~tym stylu.

\noindent \textit{Mac}: Świetnie. Co jeszcze?

\noindent Nat: Dla mnie fajną rzeczą w~konsensusie jest to, że każdy ma włączony mózg. Nie kładę się po prostu spać, jak to robiłem na większości spotkań, na jakich kiedykolwiek byłem, ponieważ to, co myślę, może mieć jakiś wpływ na to, co się dzieje, w~dowolnym momencie.

\noindent Sara: Poza tym musisz \textit{słuchać }tego, co mówią inni.

\noindent David: Właściwie to jest jedna z~rzeczy, które bardzo lubię w~procesie konsensusu. W polityce większościowej zawsze starasz się, aby pomysł przeciwnika wyglądał na zły, więc motywacją jest zawsze sprawienie, by jego argumenty wydawały się głupsze, niż są w~rzeczywistości. W konsensusie próbujesz znaleźć kompromis lub syntezę, więc motywacją jest zawsze szukanie najlepszej lub najmądrzejszej części argumentów innych ludzi. 

\noindent Chris: Napisałbym ,,kreatywność''. Niektóre z~najpiękniejszych przykładów konsensusu, jakie kiedykolwiek widziałem, to sytuacja, w~której wszyscy wydają się skłóceni, masz dwie różne propozycje i~wydaje się, że nie ma możliwości ich pogodzenia, zaczyna wyglądać na to, że grupa jest podzielona 50/50 i~wszyscy zaczynają kopać się po kostkach, a potem nagle ktoś po prostu wyskakuje z~zupełnie nowym pomysłem i~wszyscy natychmiast mówią: ,,Och, dobrze. Zróbmy to w~takim razie tak''.

\noindent \textit{Mac}: Właściwie to naprawdę ważna kwestia, ponieważ powszechnym nieporozumieniem jest to, że konsensus dotyczy głównie kompromisu, więc wtedy krytycy powiedzą, że proces konsensusu oznacza, że  kiedy podejmujesz decyzję, zawsze ma ona tendencję do wychodzenia jako taka niepewna. To nieprawda. Czasami chodzi o kompromis. Ale chodzi też o pozostawienie rzeczy maksymalnie otwartych na zbiorową kreatywność, więc czasami, zamiast próbować znaleźć kompromis, możesz po prostu wymyślić zupełnie nową propozycję.

\noindent Megan: Poza tym możesz podejmować decyzje tak radykalne, jak grupa je podejmująca\ldots 

\noindent \textit{[}I tak dalej. Na koniec spędziliśmy minutę lub dwie rozmawiając o wyzwaniach i~pułapkach, głównie o niebezpieczeństwach ,,konsensusu przez wyniszczenie'', kiedy zdeterminowana mniejszość próbuje zmęczyć wszystkich innych, ale większość z~nich została już wpisana w~sekcję ,,nie działa'' .\textit{]}

\medskip
\noindent HISTORIA

\noindent \textit{Mac}: Zrobię to krótko. Oczywiście istnieje wiele społeczności rdzennych Amerykanów, które od tysięcy lat podejmują decyzje w~drodze konsensusu. Jednak w~Stanach Zjednoczonych proces konsensusu naprawdę cofa się do kwakrów; zaczął być przyjmowany przez grupy aktywistów w~ruchu antywojennym i~antynuklearnym w~latach 70., w~który zaangażowanych było wielu kwakrów. Były sekcje ruchu praw obywatelskich, które wykorzystywały konsensus, SNCC, ale inne, takie jak Southern Christian Leadership Council, już nie. SDS i~inni aktywni w~ruchu antywojennym lat 60. również do pewnego stopnia wykorzystywali konsensus.

 W latach 70. feministki \textit{naprawdę }zmieniły i~rozwinęły tę ideę, wiele grup feministycznych przyjęło konsensus jako rodzaj antidotum na niektóre z~bardziej nieprzyjemnych stylów przywództwa macho z~lat 60, i~tam naprawdę powstał ten rodzaj konsensusu, który używamy obecnie. Stamtąd został przyjęty w~kampaniach antynuklearnych w~latach 70. i~80. i~został szeroko zaadoptowany w~ruchu ekologicznym, szczególnie w~radykalnych grupach ekologicznych, takich jak Earth First!. To stamtąd tak naprawdę przeszedł do DAN.

 Ruch związkowy \textit{nie posługuje }się konsensusem. Korzystają z~\textit{Reguł Porządku Roberta}(Robert's Rules of Order). Nawet jeśli solidarność robotnicza jest dużą częścią DAN. To może czasem doprowadzić do konfliktu kulturowego. Właściwie w~grupie Policja i~Więzienia czasami mamy te same problemy, ponieważ wiele grup, z~którymi pracujemy, jest zorganizowanych zupełnie inaczej.

\noindent Nat: Tak, zawsze się zastanawiałem: czy robimy rzeczy w~ten sposób, ponieważ uważamy, że jest to najbardziej efektywne, czy też pomysł, że tak będzie działać w~przyszłości, że zaczynamy tworzyć nasze marzenia dzisiaj.

\noindent \textit{Jim}: No cóż, najlepiej byłoby po trochu obu.

\noindent Sara: Ale mówisz, że Związki tak naprawdę nie przyjęły konsensusu. Czy były więc jakieś próby zastosowania tych pomysłów do organizacji miejsc pracy, lub cokolwiek, jak sądzę, poza planowaniem działań lub małych spółdzielni i~tym podobnych?

\noindent \textit{Jim}: To z~pewnością nie jest nieznane. IWW na pewno przeprowadziła kilka eksperymentów z~kolektywizacją, przedsiębiorstwami prowadzonymi przez pracowników, które\ldots  jestem prawie pewien, że działały w~ramach pewnego rodzaju procesu konsensusu. i~faktycznie istnieje spora liczba organizacji non-profit, a nawet kapitalistycznych firm, które wykorzystują jakąś wersję konsensusu w~swojej codziennej działalności. Gdzieś widziałem listę: w~rzeczywistości jest zaskakująco długa. Wszystko, od amerykańskiej służby leśnej po Saturna i~Harleya-Davidsona, i~oczywiście prawie każdą dużą korporację, powiedzmy, w~Japonii, działa na zasadzie pewnego rodzaju konsensusu. Ale takie przykłady pokazują również, że jest konsensu i~jest konsensus; realizujesz nawet bardzo egalitarno-pozorny proces w~ramach tego, co wciąż jest całkowicie hierarchiczną, odgórną organizacją, a sam proces staje się formą przymusu lub ucisku, sposobem ciągłego wymuszania na tobie udawania, że zgadzasz się z~decyzjami, w~których nic nie masz do gadania.

\medskip
\noindent TERMINOLOGIA

 Podstawowe pojęcia, zgodnie z~nową kartką na ścianie, to:

\emph{ PROPOZYCJE \newline PRZYJAZNE POPRAWKI\newline ODEJŚCIE NA BOK i~BLOKI \newline ZMODYFIKOWANY KONSENSUS}


\noindent \textit{Lesley}: Zatem zakładam, że jesteście zaznajomieni z~podstawową strukturą spotkań DAN. Zwykle mamy dwóch facylitatorów, jednego mężczyznę, jedną kobietę, i~zwykle na zmianę, w~dowolnym momencie, jeden z~nich prowadzi dyskusję, a drugi zarządza \textit{listą }mówców. Utrzymywanie listy jest w~rzeczywistości ważną umiejętnością, ponieważ nie chcesz zostawiać ludzi stojących tam z~podniesionymi rękami przez dziesięć minut, dopóki nie zostaną wywołani. Chcesz móc przyciągnąć ich uwagę, kiwnąć głową lub wysłać jakiś mały sygnał, że są na liście, a następnie śledzić kolejność, nawet jeśli niekoniecznie wiesz, kim oni są, więc możesz musieć wywołać ,,kobieta w~zielonej koszuli'' lub ,,mężczyzna w~czerwieni z~przodu''. W takim przypadku ważne jest, aby być konsekwentnym. Zawsze używam koloru koszuli, jeśli nie pamiętam imienia. Oczywiście nie będziesz wołał ,,gruba laska z~przodu'' lub ,,facet z~gigantycznym kolczykiem w~nosie'', ale zdziwiłbyś się, jak prawie wszystko, czym kogoś wyróżniasz, może mieć subtelny efekt w~sprawieniu, że czują się wyobcowani lub\ldots  no cóż, wyróżnieni. Więc utrzymuj jednolite nazwy.

 W każdym razie: \textit{Propozycje}, propozycja to sugestia co do sposobu działania, którą ktoś stawia przed grupą. Propozycje mogą być przedstawiane przez osobę lub przez grupę, w~DAN zwykłą ideą jest to, że ważne propozycje są przedstawiane Generalnej DAN przez przedstawicieli jednej z~grup roboczych. Ale nie musi tak być, każdy może coś zaproponować.

\noindent Sara: Czy propozycja musi być spisana?

\noindent \textit{Lesley}: Nie. Prosimy grupy robocze o przedstawienie swoich propozycji na piśmie, ale w~połowie przypadków nie pamiętają, a propozycje poszczególnych osób prawie nigdy nie są spisywane.

\noindent Ktoś: Czy propozycja \textit{musi }odnosić się do sposobu działania?

\noindent \textit{Lesley}: Właściwie to dobre pytanie. Czy musi? Cóż, myślę, że to zależy od tego, jak zdefiniujesz ten termin. Na przykład, kiedy po raz pierwszy tworzysz grupę, musisz dojść do konsensusu wokół swoich zasad jedności. Albo możesz uzgodnić, powiedzmy, popieranie czyjegoś działania lub tekst jakiejś literatury popularyzatorskiej. Ale ogólnie rzecz biorąc, jest to coś, co chcesz \textit{zrobić.} Jedyną rzeczą, do której na pewno nie używasz konsensusu, są kwestie definicji: czy interwencja USA w~Somalii powinna być uważana za przykład imperializmu czy coś w~tym rodzaju. Nie próbujesz zdefiniować rzeczywistości. Próbujesz zdecydować, co zrobić.

\noindent David: Więc nigdy nie znajdziesz się w~takiej sytuacji jak w~ISO, albo, myślę, że to była ich brytyjska filia, SWP, gdzie słyszałem, że wszyscy antropolodzy zostali ostatnio usunięci, ponieważ nie zgadzali się z~linią partyjną, że ludzie naprawdę stali się ludźmi dopiero w~neolicie. (Nie wiem, czy to naprawdę prawda.)

\noindent \textit{Mac}: Tak, cały pomysł polega na upewnieniu się, że takie szalone gówno nigdy się nie wydarzy. O ile nawet mówimy o takich pytaniach, jak ,,czy jesteśmy organizacją antykapitalistyczną?'', ,,czy jesteśmy przeciwni wszelkim formom hierarchii?'', będzie to zawarte w~deklaracji misji lub zasadach jedności. A te są ważne, ponieważ są podstawą blokowania. Ale staramy się również ograniczać te do punktów, które będą miały jakiś wpływ na akcję.

\noindent \textit{Lesley}: Tak więc ogólnie rzecz biorąc, propozycja jest propozycją działania, przedstawioną grupie. Jako facylitator pierwszą rzeczą, którą robisz, gdy ktoś składa propozycję, jest proszenie o\textit{ pytania wyjaśniające: }upewnienie się, że wszyscy dokładnie wiedzą, co jest proponowane. Następnie pytasz, czy ktoś ma jakieś \textit{obawy}: problemy, jakie taki sposób działania może spowodować, powody, dla których może to nie być najlepszy sposób działania. (Jako facylitator przekonasz się, że czasem trudno jest odróżnić pytania wyjaśniające od obaw). Czasem w~tym momencie, staje się jasne, że propozycja wzbudza silne uczucia, a osoba, która ją przedstawia, może po prostu zdecydować ją wycofać. Alternatywnie, ludzie mogą zaproponować jedną lub więcej \textit{przyjaznych poprawek}, aby rozwiązać problemy, które, jeśli osoba składająca propozycję je zaakceptuje, stają się częścią propozycji.

\noindent \textit{Jim}: Dobrze mieć do tego skrybę, kogoś, kto wszystko spisuje, gdy trzeba przeformułować propozycję w~obecnej formie.

\noindent \textit{Lesley}: Albo ktoś może, zamiast tego, zdecydować się na przedstawienie alternatywnej propozycji. Albo możesz skończyć z~całą ich masą. Chociaż zwykle robi się \textit{naprawdę }bałagan, jeśli przejdziesz przez dwie lub trzy.

\noindent \textit{Mac}: Istnieją techniki pozbycia się irytujących propozycji, których nikt tak naprawdę nie lubi. Na przykład w~Policji i~Więzieniach czasami mówimy ,,może powinniśmy utworzyć grupę roboczą, aby to przedyskutować'' i~przekazujemy arkusz rejestracji dla grupy roboczej. A potem oczywiście nikt się nie zapisuje.

\noindent \textit{Lesley}: Ale to jest \textit{główna }rola facylitatora: przeprowadzić grupę, wyjaśnić, jakie są propozycje, jakie problemy lub problemy mogą mieć z~nimi ludzie, czy trzeba coś dodać lub zmienić. To może być naprawdę trudne, jeśli na stole jest więcej niż jedna propozycja. W takim przypadku możesz skorzystać z~szeregu narzędzi. Możesz rozpytać wszystkich i~poprosić, aby zastanowili się nad pytaniem. Możesz też wypróbować niezobowiązującą sondę: podniesienie rąk. To nie to samo, co głosowanie, ponieważ w~rzeczywistości nie jest to sposób na podjęcie decyzji, ale może dać ci poczucie pokoju i~często, jeśli odkryjesz, że jedna propozycja ma bardzo silne poparcie, a druga prawie żadnego, to naprawdę wszystko, co musisz wiedzieć. Lub możesz przejrzeć każdą po kolei ,,czy ktoś nadal ma poważne obawy związane z~propozycją nr 1?”

\noindent \textit{Jim}: Należy pamiętać, że propozycje \textit{niekoniecznie }muszą się wzajemnie wykluczać.

\noindent \textit{Lesley}: Tak, jedyną rzeczą, do której zawsze chcesz zachęcić ludzi, jest przełamanie dychotomii, wskazanie, w~jaki sposób ludzie mówią to samo. Nawet w~bardzo praktycznych sprawach zadaniem facylitatora jest próba zdefiniowania wspólnego gruntu: ,,Więc słyszę bardzo silnie, że do wtorku nie powinniśmy robić nic bardzo bojowego, a także wiele obaw, żebyśmy nie zagrozili otaczającej społeczności. Czy jest ktoś, kto uważa, że  nie powinniśmy w~ogóle podejmować działań bojowych, nawet we wtorek?''\ldots  

\noindent \textit{Jim}: Albo może pójść w~drugą stronę. Jeśli będziesz powtarzać propozycje, możesz czasami odkryć, że ludzie w~rzeczywistości inaczej interpretują słowa i~tak naprawdę są bardzo różne pomysły na to, co się dzieje.

\noindent \textit{Lesley}: Na koniec, miejmy nadzieję, sprowadziłeś rzeczy do jednej propozycji i~jesteś w~takim punkcie, że  próbujesz znaleźć konsensus. W tym momencie pytasz, czy są jakieś \textit{stanie z~boku}lub \textit{bloki}. Teraz, w~przypadku odejścia na bok, istnieją właściwie różne interpretacje tego, co to oznacza. Jednym z~nich jest to, że właściwie mówisz: ,,Sam nie będę brał udziału w~tej akcji, ale nie mam problemu z~wykonaniem jej przez resztę grupy''. Innym jest to, że jesteś przeciwny temu pomysłowi, ale nie czujesz, że jest to tak poważny problem, że opuściłbyś grupę z~tego powodu. Jest to sposób na zarejestrowanie drobnego sprzeciwu i~ważne jest, aby po osiągnięciu konsensusu dać każdemu, kto stał z~boku, szansę na wyjaśnienie, dlaczego sprzeciwili się, i~zapisanie ich w~notatkach ze spotkania, jeśli tego chcą.

 Jeśli w~grupie dwudziestu jest dużo stania z~boku, powiedzmy pięć lub sześć, to jest to poważny problem. Oznacza to, że w~pewnym momencie proces się załamał, ponieważ ci ludzie powinni mieć możliwość wyrażenia swoich zastrzeżeń, zanim do tego doszło.

 Jeśli chodzi o \textit{bloki}, to naprawdę fajnie jest wiedzieć, że \textit{możesz }zablokować propozycję, że jeśli czujesz to tak mocno, możesz zatrzymać propozycję przed startem, ale w~zasadzie jest to zabezpieczenie. Jeśli to zrobisz, sprawy mogą stać się brzydkie. Ponieważ zasadniczo mówisz: ,,to narusza podstawowe zasady organizacji i~nie mogę na to pozwolić''. 

\noindent \textit{Mac}: Oczywiście jest to również całkowicie krytyczne, ponieważ bez możliwości blokowania nie jest to konsensus. Dlatego próbowaliśmy uzyskać jakiś mechanizm blokowania w~Continental DAN, chociaż trudno jest wymyślić, jak można to zrobić w~dużej strukturze federacyjnej.

\noindent \textit{Lesley}: Nie można tego lekceważyć. Zwykle, jeśli blokujesz, ryzykujesz odizolowaniem się, ludzie często będą kuszeni, by nękać blokującego, dlatego ważne jest, aby pamiętać o tym jako facylitator i~upewnić się, że ta osoba jest szanowana.

\noindent \textit{Jim}: Jedna osoba, którą kiedyś widziałem, to było jak najgorszy koszmar facylitatora, to było na konwencji anarchistycznej, i~tam był jeden facet, który po prostu zablokował wszystko, ponieważ był zasadniczo przeciwny konsensusowi. Nigdy nie jestem do końca pewien, gdybym był facylitatorem, co byłbym w~stanie to zrobić. [\textit{patrzy na Maca}]

\noindent \textit{Mac}: Cóż, blok ma opierać się na zasadach założycielskich lub powodach bycia grupą, więc powiedziałbym, że na tej podstawie można go kwestionować. Jeśli grupa opiera się na konsensusie, trudno jest zobaczyć, jak blokowanie, bo lubisz konsensusu, może być z~tym spójne. To jest definicja bloku pozbawionego podstaw.

\noindent \textit{Jim}: Tak, ale czy nie jest to również podstawowa zasada podejmowania decyzji poprzez konsensus, że nie można kwestionować motywów innych aktywistów? Musisz dać im korzyść z~domniemania uczciwości i~dobrych intencji. Jak więc rzucić im wyzwanie?

\noindent \textit{Mac}: Cóż, kiedy mówisz ,,blok bez podstaw'', myślę, że oznacza to, że \textit{nie }jest zakorzeniony w~zasadach jedności grupy. Nie mówisz, że dana osoba jest osobiście pozbawiona zasad. Wygląda na to, że masz do czynienia z~osobą, która jest całkowicie szczera i~kieruje się swoimi motywami, to po prostu nie są zasady grupy. Co rodzi pytanie, dlaczego w~ogóle przyszła na to spotkanie. Powiedziałbym mu: dlaczego nie dołączysz do grupy, której zasady się podobają, a jeśli tak nie jest, spróbujesz założyć nową?

\noindent Christa: Ale myślałam, że idea bloku polega na tym, że mówisz ,,to jest dla mnie kwestia tak ważna, że  z~tego powodu byłabym skłonna opuścić grupę''? Niekoniecznie kwestia zasad założycielskich.

\noindent \textit{Mac}: Cóż, nie musi tak być, choć to prawda, że  niektórzy tak to interpretują.

\noindent Chris: Na A16, w~mojej grupie afinicji, mieliśmy propozycję zbudowania blokady drogowej z~materiałów z~pobliskiego placu budowy, a ktoś ją zablokował, ponieważ uważał, że nie mamy prawa zabierać rzeczy, które nie należą do nas i~nie mają nic wspólnego z~IMF czy Bankiem Światowym. Ale nasza grupa afinicji w~rzeczywistości nie miała żadnych formalnych zasad jedności. Jak więc miałoby to być uzasadnione?

\noindent \textit{Lesley}: Cóż, zwykle chodzi o to, że albo mówisz, że propozycja narusza twoje zasady założycielskie, albo, że narusza podstawowe powody istnienia lub cele grupy. Więc jest miejsce do interpretacji. Ale ogólnie rzecz biorąc, nie chcesz być superlegalistyczny w~tego typu sprawach. A może lepiej powiedzieć, że jeśli ludzie zaczynają być superlegalistyczni, to zazwyczaj jest to znak, że masz prawdziwy problem w~grupie.

\noindent Megan: W pewnym momencie, na A16, byliśmy pod więzieniem, było nas około sześćdziesięciu osób, które robiły więzienną solidarność. Spodziewaliśmy się, że nasi prawnicy będą mogli zobaczyć aresztowanych, ale gliniarze odprawili ich wszystkich. Niektórzy z~nas chcieli wnieść naprawdę głośny i~wyzywający protest. Była tam ekipa z~marionetkami i~perkusjami, której naprawdę wpadł na pomysł zorganizowania wielkiej parady wokół więzienia. Ktoś jednak zauważył, że przetrzymywano tam nie tylko aktywistów, że były rodziny innych więźniów, które też chciały się dostać. Utworzyliśmy krąg i~zastanawialiśmy się, co robić. Gdybyśmy wzniecili zamieszanie, nie mówiąc już o próbie zamknięcia, wszyscy ci inni nie mogliby dostać się do środka. Ktoś zablokował wszystko, co mogłoby narobić tyle hałasu, że sprawiłoby to problemy innym odwiedzającym. Więc niektórzy z~ludzi od marionetek ogłosili, ,,Nie mamy konsensus, więc założymy nową grupę afinicji dla osób, które nadal chcą zrobić paradę''.

\noindent \textit{Mac}: Cóż, tak, możesz mieć pewne, hm, kreatywne rozwiązania tego rodzaju impasu.

\noindent Christa: Więc mieli paradę?

\noindent Megan: Właściwie nie jestem pewna, co się stało. Mniej więcej wtedy odeszłam. Myślę, że mieli paradę, ale była ona o wiele bardziej dyskretna, niż pierwotnie zamierzali. Co więcej, myślę, że duży problem polegał na tym, że blokerka była nowicjuszką, większość ludzi jej nie znała, co komplikowało sprawę.

\noindent \textit{Lesley}: Właściwie to kolejna rzecz, z~którą facylitatorzy muszą wymyślić sposoby radzenia sobie, ponieważ jeśli pojawi się nowa osoba, często nie będą traktowani tak poważnie.

\noindent Sara: Czy można kiedykolwiek sprowadzić się do otwartego kwestionowania motywacji blokera? Na przykład, właściwie nie \textit{wiesz}, czy ta kobieta nie była gliną.

\noindent \textit{Mac}: Cóż, przypuszczam, że w~takim razie mógłbyś, ale byłbym bardzo ostrożny, by publicznie sugerować, że ktoś może być gliną.

\noindent \textit{Lesley}: Właściwie, to powiedziałbym, żeby nie. Nie możesz kwestionować motywacji osoby. To kwestia podstawowej zasady. Ale możesz kwestionować ich rozumowanie. Lub, jako facylitator, możesz spróbować przeformułować rzeczy, zapytać osobę: ,,Czego byś potrzebowała, aby czuć się komfortowo z~propozycją?''. To znaczy, jeśli masz pewność, że ktoś jest gotowy do zablokowania. A jeśli faktycznie zablokują, to czasami dobrym pomysłem jest zasugerowanie, aby blokujący dołączył do grupy roboczej, która pierwotnie przedstawiła propozycję, lub, w~każdym razie, współpracował z~kimkolwiek, aby sprawdzić, czy nie mogą wymyślić jakiegoś rodzaju alternatywy, z~którą mogliby żyć.

 Co w~rzeczywistości prowadzi do innej koncepcji, \textit{zmodyfikowanego konsensusu}. Sama DAN tak naprawdę nie zdecydowała, czy ma możliwość skorzystania z~tego, ale\ldots 

\noindent Neala: Chwileczkę: myślałam, że tak.

\noindent \textit{Mac}: No tak, technicznie myślę, że to jest w~naszych zasadach, ale tak naprawdę nigdy nie zdefiniowaliśmy, co to oznacza w~praktyce.

\noindent \textit{Lesley}: Zmodyfikowany konsensus byłby, na przykład, jeśli masz tylko jeden lub dwa bloki, ale inni uważali, że przeprowadzenie sprawy jest absolutnie niezbędne, możesz mieć opcję, aby przejść do głosowania ważonego: powiedzmy, większość dwóch trzecich lub siedemdziesiąt lub osiemdziesiąt procent. Czasami nie uda się nawet uzyskać takiej większości, ponieważ fakt, że jedna osoba czuła się na tyle mocno, by zablokować, wystarczy, aby przekonać wiele innych osób do zmiany zdania i~zagłosowania przeciwko tej propozycji. W każdym razie istnieją inne formy zmodyfikowanego konsensusu: na przykład konsensus minus jeden, w~którym jeśli ktoś blokuje, sprawdzasz, czy jest przynajmniej jeden inny członek grupy, który uważa, że  jego argument jest na tyle przekonujący, że go poprze. Niektóre grupy stosują konsensus minus dwa lub trzy i~tak dalej. W każdym razie najważniejszą rzeczą jest to, że jest to ostateczność; używasz jej tylko wtedy, gdy zrobiłeś wszystko, co możliwe, aby uzyskać konsensus, a po prostu nie możesz. Byłem zaangażowany w~wiele grup ze zmodyfikowaną opcją konsensusu, ale nie w~takiej, w~której faktycznie musieliśmy z~niej korzystać, z~czego bardzo się cieszę, że mogę to powiedzieć, ponieważ cały pomysł sprawia, że  czuję się naprawdę nieswojo. Nikt nigdy nie był w~stanie mi wyjaśnić, w~jaki sposób cała idea tak naprawdę zgadza się z~zasadą konsensusu.

\noindent \textit{Jim}: Grupy, które naprawdę najczęściej używają zmodyfikowanego konsensusu, to bardzo duże grupy, takie jak rady delegatów, w~których ludzie tak naprawdę się nie znają, a czasami po prostu nie masz czasu, aby pozwolić jednej osobie przejąć kontrolę nad procesem.

\noindent George: Czy nie miał być przypadku jednego oddziału DAN na Zachodnim Wybrzeżu, gdzie niektórzy ludzie z~ISO chcieli pokazać, że konsensus nie może tak naprawdę działać, więc po prostu wszystko zablokowali? Coś w~rodzaju tego, o czym mówił Jim na zjeździe anarchistów.

\noindent \textit{Jim}: Och. Nie słyszałem o tym.

\noindent \textit{Mac}: [\textit{wzdycha}] Tak, to był DAN z~San Francisco. To prawie zniszczyło grupę. Było tylko trzech ludzi z~ISO, ale próbowali systematycznie sabotować proces, aby zmusić ludzi do przejścia na głosowanie większościowe.

\noindent David: Co w~końcu zrobili?

\noindent \textit{Mac}: Cóż, pewnego dnia odbyło się spotkanie, na którym ludzie z~ISO się nie pojawili, więc wszyscy natychmiast przedstawili propozycję, aby grupa działała na zasadzie konsensusu minus trzy.

\medskip
\noindent NARZĘDZIA I~ZASADY

\noindent \textit{Jim}: Pomyśleliśmy więc, że zakończymy narzędziami i~zasobami dla facylitatorów, z~których możesz chcieć lub możesz nie chcieć korzystać. Pierwszym z~nich jest \textit{osoba z~zegarkiem}. Jest to ważne, ponieważ tworząc agendę, zdecydowanie chcesz, aby ludzie zgodzili się, ile czasu chcesz przeznaczyć na każdy punkt, ale nie ma sensu tego robić, chyba że ktoś zwraca na to uwagę i~jest w~stanie powiedzieć wszystkim, że czas na dyskusję jest i~ktoś będzie musiał zaproponować przedłużenie, jeśli mamy iść dalej. Lubię pilnować czasu, ale niektórzy wskazują taką osobę lub zlecają to współprowadzącemu.



 Jest też \textit{skryba}, co moim zdaniem jest naprawdę ważne. Zwłaszcza jeśli jest to przed wielką akcją i~jest mnóstwo informacji do śledzenia, a nie możesz zakładać, że wszyscy w~pokoju robią notatki. W przeszłości często zapominałam upewnić się, że jest protokolant, a czasami to naprawdę mnie prześladowało, ponieważ ludzie mieli inne wspomnienia dotyczące tego, co tak naprawdę postanowiliśmy. Na dużych oficjalnych spotkaniach DAN chcesz się upewnić, że przynajmniej jest ktoś, kto zapisuje wszystkie propozycje, dokładnie to, co zostało uzgodnione, ze wszystkimi przyjaznymi poprawkami i~tak dalej. Przydatne jest również prowadzenie stałego rejestru ważnych decyzji w~miejscu publicznie dostępnym, takim jak strona internetowa, ponieważ staje się to instytucjonalną pamięcią grupy. Jeśli nie, może to stać się podstawą struktury niejawnej władzy, ponieważ niektórzy ludzie mają natychmiastowy dostęp do tych informacji tylko dlatego, że są w~pobliżu od dłuższego czasu lub śledzą je, mogą nagle przerwać dyskusję i~powiedzieć ,,ale chwila! już tak zdecydowaliśmy rok temu'', a inni po prostu nie wiedzą. Jedną z~kluczowych rzeczy, jakie można znaleźć w~grupie egalitarnej, jest to, że dostęp do informacji staje się główną podstawą wyłaniających się struktur władzy, więc musisz zrobić wszystko, co możliwe, aby spróbować zdusić tego rodzaju rzeczy w~zarodku.

 Co jeszcze?

 \textit{Woda}. Bardzo pomocne jest posiadanie przy sobie małej butelki wody podczas spotkania. To nie tylko dla facylitatorów, każdy powinien mieć dostęp do wody. Jeśli czujesz, że masz tak suche gardło, że ciągle sięgasz po wodę, to dobry znak, że mówisz za dużo i~powinieneś się zamknąć.

\noindent \textit{Lesley}: \textit{Jedzenie }też. Nie jest złym pomysłem, aby mieć coś do jedzenia na tyłach sali, zwłaszcza jeśli spotkanie może trwać godzinami. i~powinieneś zwrócić uwagę, aby nie było innych czynników, które mogą powstrzymywać niektórych ludzi przed uczestnictwem w~twoich spotkaniach: na przykład brak opieki nad dziećmi lub tłumaczy.

\noindent \textit{Mac}: Rozmawialiśmy już o \textit{sondażach}. Jeśli masz na tapecie różne propozycje, jest to przydatna technika oceny ludzkich uczuć. Ponadto, jeśli jest to coś, co nie mogłoby się odwrócić z~zasady, na przykład, czy powinniśmy mieć następne spotkanie we wtorek czy w~środę, czasami wystarczy sonda. Um, co jeszcze?

\noindent Jessica: A co \textit{z sygnałami ręcznymi}? W kooperatywie, w~której brałem udział w~Oberlin, mieliśmy całą serię sygnałów ręką: facylitator mógł poprosić ludzi o odczucia wobec oświadczeń, a ty odpowiadałaś albo kciuk w~górę, w~dół, albo w~bok, jeśli jesteś niezdecydowany.

\noindent \textit{Mac}: Każdy używa innych. W DAN oczywiście migoczemy, wiesz, machając palcami w~powietrzu, by wyrazić silne poparcie lub aprobatę dla propozycji lub czyjegoś punktu, chociaż kilka osób uważa, że  cała idea ,,migotania'' to coś w~rodzaju kalifornijskiego zwyczaju.

\noindent Jessica: W Oberlin robiliśmy ,,pukanie'', zaciskasz pięść i~gestykulujesz w~ten sposób.

\noindent \textit{Mac}: Jest ich milion. Niektórzy używają małych diabelskich palców, wiesz, jakbyś umieścił palce za głową kogoś na zdjęciu, jeśli masz sześć lat i~myślisz, że takie rzeczy są naprawdę sprytne?

\noindent \textit{Lesley}: Ale jest kilka standardowych, które są przydatne. Wiele grup używa gestu w~celu ,,bezpośredniej odpowiedzi'', jeśli ktoś wypowiada się, a Ty masz faktyczne informacje, które odnoszą się bezpośrednio do tego, ale bardzo bezpośrednio, na przykład ,,nie, odwołali to wydarzenie'' lub jeden mówca rzeczywiście pyta Ci pytanie i~chcesz odpowiedzieć. Musisz być bardzo ostrożny z~tym. Ponieważ naprawdę nie chcesz, aby wszystko przeszło w~interakcję, co oznacza, że  możesz skończyć z~jakimś rodzajem rywalizacji ego między dwojgiem ludzi, a wszyscy inni są zirytowani i~wykluczeni. Zwykle lepiej jest trzymać się listy i~pozwolić, by rozmowa była trochę frustrująco zawiła, niż dać ludziom wymówkę, by byli zarozumiali i~zdominowali rozmowę. Jest też mały trójkąt, który tworzysz opuszkami palców, który oznacza ,,punkt procesu'' -- to kolejny sposób na wcięcie się w~listę, ale to tylko komentarze, które piszesz bezpośrednio do moderatora, na przykład ,,czy nie mamy nadal dyskutować nad drugą propozycją?'' lub ,,czy nie zdecydowaliśmy już o tym w~zeszłym tygodniu?”.

\noindent Christa: Mam pytanie dotyczące \textit{rund dookoła}. Czy uważasz je za skuteczne? i~kiedy ich używasz?

\noindent \textit{Lesley}: Musisz uważać na rundy. To dobry sposób, aby zachęcić ludzi, którzy mogą być zbyt nieśmiali lub niepewni siebie, do wyrażenia swojej opinii, ale zajmuje to dużo czasu. Zdecydowanie musisz ustalić limity czasowe. Nawet jeśli pozwolisz, powiedzmy, na minutę na mówcę, jeśli w~sali jest sześćdziesiąt osób, to jest godzina. Więc są najlepsza dla małych grup. Z drugiej strony, jeśli jest to mała grupa i~jest to bardzo ważne, możesz nawet spróbować dwóch rund, aby zobaczyć, czy pomysły ludzi ewoluują, gdy słyszą, co inni ludzie mają do powiedzenia. Jeśli jest to duża grupa, lepiej wrócić do starej sztuczki ,,po prostu posłuchajmy od ludzi, którzy do tej pory nie rozmawiali''. To ostatnie jest przydatne, ponieważ, powiedzmy, jeśli biali mężczyźni całkowicie dominują w~rozmowie i~żadna z~kobiet ani osób kolorowych nie mówi, możesz to zauważyć, choć brzmi trochę protekcjonalnie, gdy powiemy ,,dla odmiany posłuchajmy niektórych kobiet'' Lub ,,posłuchajmy niektórych Afroamerykanów'' Ale proszenie o ludzi, którzy jeszcze nie mówili, może mieć prawie ten sam efekt.

\noindent George: A co z~całym ,,słyszeniem''?

\noindent Lesley: Co? 

\noindent George: Wiesz, kiedy prowadzący mówi: ,,Słyszę dużo energii wokół idei tego a tego''.

\noindent \textit{Lesley}: Cóż, zwykle jest to sposób na uchwycenie atmosfery pomieszczenia, zasugerowanie, że pojawia się jakiś rodzaj konsensusu ze strony grupy, przynajmniej wokół pewnych aspektów propozycji. Oczywiście to tylko sugestia. Po części jest to sposób na sprawdzenie, bo czasami, jeśli powiesz ,,och, nie wiem'', ,,słyszę duże poparcie dla idei jakiejś parady'', to ktoś od razu powie ,,Nie. Właściwie to uważam, że w~ogóle nie powinno być jakiejkolwiek parady''.

\noindent \textit{Jim}: Trochę nam tu kończy się czas, więc pozwólcie, że bardzo szybko rzucę kilka innych technik. Niektóre z~nich to rzeczy, które, cóż, \ldots  na Zachodnim Wybrzeżu często mają obserwatora wibracji, którego zadaniem jest śledzenie emocjonalnej jakości pokoju. Jeśli ludzie są znudzeni, spięci lub źli, albo ktoś czuje się wyobcowany lub wykluczony, albo jeśli jest za gorąco lub za mało światła, wkraczają i~interweniują. Tutaj zazwyczaj to facylitator musi śledzić te rzeczy, co może bardzo utrudnić, ponieważ jednocześnie żonglujesz tyloma innymi obowiązkami. Najważniejszą rzeczą jest jednak możliwość interwencji, jeśli sprawy stają się zbyt napięte i~konfrontacyjne. Często, jeśli po prostu zorganizujesz przerwę, pozwól ludziom wstać, rozciągnąć się, niech ludzie wyjdą i~zapalą papierosa lub napiją się wody, jeśli chcą, kiedy wrócą, cały nastrój jest zwykle inny i~to, co wcześniej wydawało się poważnymi problemami, po prostu wygląda głupio lub nieważne. Niektórzy facylitatorzy sugerują nawet jogę lub ćwiczenia oddechowe.

\noindent \textit{Lesley}: Lub jednym wielkim faworytem jest grupowe pocieranie pleców.

\noindent \textit{Mac}: No i~jeszcze jest cała idea komitetu pojednania. Jeśli jest blokada, czasami możesz poprosić o przerwę i~skorzystać z~okazji, aby porozmawiać z~osobą składającą propozycję i~osobą, która ją zablokowała, a może jedną lub dwiema innymi osobami, o których wiesz, że oboje im ufają i~sprawdzić, czy nie możesz ich nakłonić do wspólnego wyjścia z~pokoju, aby omówili całą sprawę, a następnie wrócili na spotkanie z~nową propozycją.

\noindent Neala: Wiesz, nie miałabym teraz nic przeciwko małej przerwie na rozciąganie.

 Co też zrobiliśmy. Po tym nastąpiło odgrywanie ról, w~którym wykorzystaliśmy to, czego później się dowiedziałem, to klasyczne, proste odgrywanie ról do treningów konsensusu, w~których nie masz zbyt wiele czasu: dwanaście osób zamawiających pizzę. Jeśli masz czas, możesz dodać różnego rodzaju komplikacje: różni uczestnicy potajemnie wręczają im skrawki papieru, informując ich, że namiętnie lubią anchois, są weganami i~tak dalej. Zadanie polega na tym, aby sprawdzić, czy osoba o imieniu facylitator może przezwyciężyć te trudności w~dość krótkim czasie, w~tym przypadku dwie minuty, co było nieco śmieszne.

 W szkoleniach, w~których brałem udział -- z~których większość brałem udział w~pomaganiu w~organizacji, choć nigdy nie byłem głównym organizatorem -- odgrywanie ról zajmowało coraz więcej czasu i~stawało się coraz bardziej dopracowane. W jednym, osiem osób o różnych poglądach politycznych -- marksiści doktrynerzy, wojowniczy anarchiści, reformistyczni ekolodzy -- ale tylko jeden transparent, miało wymyślić hasło do napisania na nim (myślę, że skończyliśmy z~,,Palić banki, nie drzewa''); później mieliśmy ćwiczenie, podczas którego próbowaliśmy wkroczyć do oblężonego kościoła w~Betlejem i~ciężko uzbrojeni żołnierze izraelscy kazali nam się rozejść, co ostatnio spotkało jedneo z~organizatorów (najskuteczniejszym podejściem, jak odkryliśmy, było wysłać jedną małą grupę do negocjacji, podczas gdy reszta próbowała przejść inną drogą). Być może najciekawsze był na innym szkoleniu, gdzie dwudziestu aktywistów przyjęło rolę dziennikarzy IMC w~środku wielkiej akcji. Jedna z~nich właśnie weszła z~nagraniem wideo, które nakręciła, przedstawiając szefa policji mordującego z~zimną krwią aktywistę; był tylko jeden egzemplarz; budynek został otoczony i~policja zaczęła się wkraczać. To miało zilustrować granice konsensusu.

 Jim poświęcił trochę czasu, aby wyjaśnić, jak ważne jest, aby zachować spokój jako facylitator; uspokoić się i~skoncentrować wcześniej. 
 
 -- Zawsze przychodzę pół godziny wcześniej, kiedy o niczym nie myślę, po prostu odpoczywam. Spacer po plaży lub w~parku byłby idealny, jeśli to możliwe. 
 
 Następnie przeszliśmy do tego, jak faktycznie zorganizowane było spotkanie DAN, z~pomocą niedawno ukończonego arkusza struktury, nieusuwalnym atramentem, z~laminowanymi arkuszami na wierzchu, tak aby facylitatorzy mogli zapisać szczegóły na wierzchu magicznym markerem i~wytrzeć go przed następnym spotkaniem.

 Mac wyjaśnił, że na ostatnim spotkaniu DAN w~końcu uzgodniliśmy podstawową strukturę szkieletową spotkań, która działa mniej więcej tak:
\medskip
\noindent I) ORIENTACJA (Zwykle trwa dziesięć do piętnastu minut, ponieważ ludzie się spóźniają. Tutaj robimy lodołamacze, zwykle ćwiczenie słuchania, rozmawiamy o celach spotkania)

\noindent II) WPROWADZENIE (Wszyscy chodzą w~kółko i~mówią, kim są, w~jakich innych projektach lub grupach roboczych mogą być aktywni)

\noindent III) PRZEGLĄD AGENDY i~POPRAWKI (Gdzie ludzie mogą dodawać punkty do agendy)

\noindent IV) OGŁOSZENIA ZDARZEŃ AWARYJNYCH (nie więcej niż dziesięć minut)

\noindent V) RAPORTY GRUPY ROBOCZEJ (Dziesięć minut; w~tym miejscu przekazujesz arkusze zapisów dla swojej grupy roboczej lub projektów lub wydarzeń)

\noindent VI) BIEŻĄCE SPRAWY

 A) Propozycje przedłożone przez wewnętrzne grupy robocze (po dziesięć minut każda)

 B) Nowe sprawy (maksymalnie sześćdziesiąt minut)

 C) Edukacja grupowa (właściwie nigdy tego jeszcze nie robiliśmy, ale jeśli mamy coś takiego jak wideo do pokazania, mówca z~zewnątrz)

\noindent VII) DYSKUSJA NASTĘPNEGO SPOTKANIA (Imię nowego facylitatora itp.)

\noindent VIII) OGŁOSZENIA KOŃCOWE

 Po krótkiej dyskusji (pamiętaj, aby ludzie z~ogłoszeniami określili, ile czasu im zajmie; jeśli raporty zwrotne z~grup roboczych zaczną przekształcać się w~długie dyskusje, przerwij je i~wrzuć wszystko do Nowych Spraw itp.):

\noindent \textit{Jim}: Bycie facylitatorem może być bardzo trudne i~frustrujące. Może się wydawać, że próbujesz ciągnąć ludzi wbrew ich woli. Ale zawsze musisz pamiętać, że to nie jest relacja wrogości. Jeśli jesteś facylitatorem, grupa jest twoim sprzymierzeńcem. \textit{Chcą}, abyś odniósł sukces.

\noindent \textit{Lesley}: Lepiej, jeśli pozwolisz grupie samemu ustalić agendę, a zwłaszcza zdecydować, ile czasu przeznaczyć na różne punkty. Im częściej to robisz, tym bardziej nie będą mieli nic przeciwko, gdy powiesz ,,OK, czas minął''.

 Różne grupy domagają się różnych stylów facylitacji. Jeśli ludzie są zbyt pasywni, łagodni i~ugodowi, musisz stać się bardziej liderem. W przeciwnym razie każdy może biernie zgodzić się na decyzje, na które wszyscy będą później narzekać, i~poczuje się, jakby ktoś na nich coś nałożył. Więc w~takich przypadkach musisz upewnić się, że połowa osób w~pokoju nie ignoruje potajemnie swoich zastrzeżeń. Spróbuj ich przekonać. Ponadto w~przypadku większych grup często potrzebujesz bardziej praktycznego podejścia, silniejszego stylu facylitacji.

 Często można usłyszeć, że niektórzy aktywiści nazywani są ,,silnymi facylitatorami'' w~tym sensie; ci zdolni do agresywnej interwencji, zwłaszcza w~dużych grupach. Jest to uważane za cechę godną podziwu. Z mojego doświadczenia, co ciekawe, silnymi facylitatorami są prawie zawsze kobiety. Po części jest to prawdopodobnie spowodowane tym, że mężczyźni, którzy zachowują się w~ten sposób, bardzo szybko działają komuś na nerwy. Ale jest również powszechną mądrością, że większość najlepszych facylitatorów to kobiety.

\bigskip
\noindent KONSENSUS ZDEFINIOWANY PRZEZ JEGO PRZECIWIEŃSTWO

\medskip
\textit{ I: Demokracja Amerykańska} 

 Jak zauważyłem w~rozdziale 5, konsensusowe podejmowanie decyzji przez długi czas było utożsamiane z~pacyfizmem. Grupy, które kładą silny nacisk na niestosowanie przemocy, nadal często mają takie pretensje, widząc w~niej jedyną formę podejmowania decyzji w~pełni zgodną z~ideałami pacyfistycznymi. Dzieje się tak z~tych samych powodów, jak u wielu anarchistów: jeśli ktoś odmawia użycia siły fizycznej, aby zmusić innych do zaakceptowania decyzji grupowej, to każdy musi, przynajmniej na pewnym poziomie, wyrazić zgodę\footnote{Nie jestem pewien, czy istnieje jeden opisany przykład bezpaństwowego społeczeństwa, które podejmowało decyzje poprzez system głosowania większościowego. Niewielkie, autonomiczne społeczności prawie zawsze stosują pewną odmianę systemu konsensusu. Powody tego są dość oczywiste; Pisałem o nich szerzej w~innym miejscu (Graeber 2004). Na przykład, kiedy Mao próbował zastąpić konsensus systemem większościowym w~lokalnych zgromadzeniach wiejskich w~wiejskich Chinach, starsi prawie natychmiast sprzeciwili się, argumentując, że jeśli propozycje będą musiały być przegłosowane w~górę i~w dół, to będą zwycięzcy i~przegrani, i~niektórzy ludzie zostaliby publicznie upokorzeni i~straciliby twarz. Ogólnie rzecz biorąc, w~grupach lokalnych znacznie łatwiej jest ocenić, co większość chce zrobić, niż wymyślić, jak przekonać mniejszość, która się z~nimi nie zgadza, a~organizowanie publicznego konkursu, w~którym mniejszość ma przegrać, jest prawdopodobnie najgorszym sposobem decydowania.}.

 To dlatego Lesley uznała zmodyfikowany konsensus za pewien problem. Jeśli cała idea konsensusu polega na tym, że niczyja opinia nigdy nie jest zlekceważona, że  nikomu nigdy nie powiedziano ,,Przepraszam, możesz nienawidzić tego pomysłu, ale mieliśmy głosowanie i~przegrałeś, więc teraz nie obchodzi nas to, co ty o tym pomyślisz'', wtedy nie ma znaczenia, czy uzyskasz siedemdziesiąt, osiemdziesiąt, czy nawet dziewięćdziesiąt pięć procent większości. Niektórym ludziom nadal będzie się mówiło, żeby się zamknęli i~robili to, co im się każe. Ci, którzy wierzą, że zmodyfikowany konsensus jest czasami konieczny -- a powiedziałbym, że obecnie jest to przytłaczająca większość anarchistów -- zwykle wskazują, że, jak podkreślił Mac, konsensus, w~przeciwieństwie do głosowania, nie jest tylko sposobem podejmowania decyzji. To proces. Podjęcie decyzji to dopiero ostatni krok. Jeśli ktoś szanuje proces, ,,ducha konsensusu'', jak niektórzy lubią mawiać, dokładna forma tego ostatniego kroku nie jest najważniejsza. W każdym razie nie jest tak, że mniejszość jest naprawdę zmuszana. Nikt nigdy nie jest zmuszany do wzięcia udziału w~akcji, a facylitatorzy często przypominają wszystkim, aby w~takich przypadkach nie robili niczego, co mogłoby wydawać się wywieraniem presji moralnej. Nawet jeśli jest to decyzja mająca wpływ na strukturę grupy, nikt ich nie zmusza do pozostania.

 Podczas mojego pierwszego roku w~DAN spędziłem dużo czasu, próbując zrozumieć, o co tak naprawdę chodzi z~tym ,,duchem konsensusu''. Oczywiście nie chodziło tylko o podejmowanie decyzji. Nie chodziło nawet o zachowanie podczas spotkań. Była to raczej próba -- zainspirowana refleksją nad strukturą i~przebiegiem spotkań -- próbą ponownego wyobrażenia sobie, jak ludzie mogą żyć razem, rozpoczęcia -- choć powoli, choć boleśnie -- budowania prawdziwie demokratycznego sposobu życia. Odwieczny przykład zamawiania pizzy (nie mogę powiedzieć, ile razy widziałem użycie na szkoleniach) może być postrzegany na swój sposób jako oskarżenie skierowane w~samo serce amerykańskich roszczeń do społeczeństwa demokratycznego. Jak często przeciętny Amerykanin siada, nawet w~grupie czterech lub pięciu osób, i~próbuje podjąć wspólną decyzję, w~której wszyscy mają równy głos? Jasne, dzieci często robią to podczas zabawy. Ale dla dorosłych doświadczenie demokracji ogranicza się w~dużej mierze do decyzji dotyczących jedzenia, a może filmów. U osób w~wieku studenckim prawdopodobnie zdarza się to najczęściej przy zamawianiu pizzy; dla osób starszych, głównie przy wyborze restauracji.

 Popularne koncepcje demokracji we współczesnych Stanach Zjednoczonych\footnote{,,Popularne'' w~tym przypadku oznacza, zasadniczo, ,,te, które otrzymały pewną legitymację w~korporacyjnych mediach''.} można powiedzieć, że obracają się wokół dwóch pojęć: ,,wybór'' i~,,opinia''. Oba są słowami, które, co istotne, prawie nigdy nie są używane przy podejmowaniu decyzji w~drodze konsensusu.

 Demokracja, jak się ciągle słyszy, oznacza, że  ludzie mogą dokonywać wyborów. Wybierają między różnymi partiami lub kandydatami. Mogą nawet zdecydować się na głosowanie ,,tak lub nie'' w~referendum. Niemal zawsze jednak oni sami odgrywali niewielką lub żadną rolę w~kształtowaniu rzeczy, między którymi dokonuje się tego wyboru. To oczywiście ideologia wyboru, która umożliwia postrzeganie demokracji i~rynku jako ekwiwalentu: także wybór konsumenta oznacza wybór z~szeregu opcji zaprojektowanych przez kogoś innego.

 Wydaje mi się, że koncepcja ,,opinii'' -- osobistych opinii, opinii publicznej -- wynika również z~braku realnego doświadczenia partycypacyjnego podejmowania decyzji. W amerykańskich szkołach dzieci są zawsze proszone o wyrażenie swoich opinii. To spuścizna tradycji Deweya, dość samoświadoma próba nasycenia dzieci demokratycznym duchem. Problem w~tym, że te opinie z~reguły nie przynoszą efektu. Uczniowie mogą zostać poproszeni o decydowanie i~wyrażanie, co myślą o wszystkim, od polityki zagranicznej USA po organizację zajęć gimnastycznych, ale doskonale zdają sobie sprawę, że opinie te nie mają wpływu na osoby faktycznie podejmujące decyzje, nawet w~szkole. Trwa to przez całe życie. Myślę, że to jest to, co zwykle daje tak wiele ,,osobistych opinii'', które słyszy się w~Ameryce, wyrażają ich dziwnie swobodny charakter, ich częsty ton arbitralności, zamykania się w~sobie, nietolerancji, te właśnie cechy, które sprawiają, że wielu zakłada, że  demokracja uczestnicząca w~rzeczywistości nie byłaby możliwa. Sformułowanie ,,każdy ma prawo do własnej opinii'' jest powszechnie używane jako zmyłka. Mają prawo do swojej opinii, ponieważ opinie nie mają znaczenia. Ci u władzy nie mają opinii. Robią politykę.

 Zrozumienie tego może również pomóc w~wyjaśnieniu niektórych luk w~teorii politycznej. Jak zauważył Bernard Manin (1994), teoretycy od Rousseau do Rawlsa zawsze zakładają, że obywatele zaczynają od zbioru wcześniej istniejących interesów (zazwyczaj uważanych za zasadniczo materialne), a następnie postrzegają debatę polityczną -- co anarchista nazwałby ,,procesem'' -- jako sposób, w~jaki konkurują, szukają kompromisu, manewrują i~ogólnie starają się uzyskać jak najwięcej tego, czego już wiedzą, że chcą. Pojęcie ,,opinii'' doskonale pasuje do tej logiki. Zakłada się również, że opinie są wstępnie formułowane. W najlepszym razie można nimi manipulować lub wpływać na nie. Można je tak postrzegać tylko wtedy, gdy tak naprawdę nie ma narady, może poza rozmowami w~barach lub przy kolacji. Jeśli obserwujemy, jak faktycznie działają procesy rozważania, jest całkowicie niemożliwe, aby aktorzy po prostu przenieśli istniejące wcześniej ,,opinie'' lub ,,interesy'' na jakiś polityczny rynek. W procesie obrad -- a właściwie wszelkich obrad politycznych, chociaż proces konsensusu ma na celu maksymalizację tego -- wszyscy nieustannie zmieniają zdanie, uczą się nowych informacji, utożsamiają się z~różnymi perspektywami, zmieniają kwestie, mierzą i~ważą argumenty na różne sposoby. ( -- Cóż, ryzykując zaprzeczenie sobie, spróbuję innego podejścia -- ogłosił Alexis podczas jednej z~debat w~Ya Basta! -- Dlaczego nie? -- odpowiedział Moose -- Do diabła, zaprzeczyłem sobie już co najmniej trzy razy, po prostu na tym jednym spotkaniu.).

 Przepaść dzieląca powszechne amerykańskie koncepcje demokracji i~praktyki anarchistyczne jest tak wielka, że  niektórzy anarchiści całkowicie odrzucają termin ,,demokracja'', woląc ograniczyć go do rządu przedstawicielskiego i~rządów większości. Twierdzą, że demokracja jest formą rządu. Chcą stworzyć coś innego: anarchię. Szczególnie prawdopodobne jest, że zrobią to prymitywiści i~hardkorowe typy antyorganizacyjne, a oni, co istotne, są tymi, którzy najprawdopodobniej odrzucą eksperymenty takie jak DAN jako same w~sobie opresyjne -- chociaż, przynajmniej wśród ,,małych a'', wydają się w~zdecydowanej mniejszości. Większość anarchistów zaangażowanych w~tworzenie szerszych sojuszy przyznaje, że według słów kolektywu CrimethInc (2003) ,,większość ludzi nienawidzi rządu i~kocha demokrację. Anarchia: to po prostu demokracja bez rządu'' 


\medskip
\textit{ II: Trzy punkty kontrastu}

 Jedną rzeczą, która uderzyła mnie dosyć mocno podczas moich pierwszych miesięcy w~Sieci Działań Bezpośrednich, było to, jak podobny był ten proces decyzyjny do sposobu podejmowania decyzji grupowych w~społecznościach wiejskich na Madagaskarze, gdzie przeprowadziłem swoją pierwszą antropologiczną pracę terenową w~latach 1989-1991. Główną różnicą było to, że proces DAN był o wiele bardziej sformalizowany i~samoświadomy. W języku malaaskim nie było określenia ,,proces konsensusu'', tak właśnie podejmowano decyzje i~zawsze tak było. O ile w~ogóle można było mówić o tym, że było to ,,malgaskie'' podejście do podejmowania decyzji, kontrastujące pod tym względem z~typowymi dla instytucji uznawanych za zasadniczo obce, jak szkoły, zagraniczne przedsiębiorstwa czy urzędy państwowe.

 Takie ideały są jednak zawsze do pewnego stopnia definiowane przez kontrast. Potrzeba tego jest, jeśli już, tym bardziej dotkliwa, gdy tworzy się coś nowego, w~świadomej opozycji do panujących norm społecznych. Odkryłem, że dla większości zaangażowanych w~DAN konsensus oznaczał \textit{nie }zachowywanie się tak, jak robi się to w~pracy, \textit{nie }zachowywanie się jak członek sekciarskiej grupy marksistowskiej i~\textit{nie} angażowanie się w~tego rodzaju debatę, która dominuje w~Internecie. 

 Pozwolę sobie zilustrować:

\medskip
\noindent 1) Świat Pracy

 Ten jest dość oczywisty. Każdy aktywista, który ma jakiekolwiek doświadczenie w~pracy w~sektorze korporacyjnym -- a jest to przytłaczająca większość -- prawdopodobnie będzie w~stanie przedstawić głęboką różnicę między stylami interakcji międzyludzkich typowymi dla środowisk pracy i~projektów aktywistycznych. Nowi na scenie mają tendencję do ciągłego mówienia o nowo odkrytych uczuciach wyzwolenia, solidarności, wolności, zaufania i~tak dalej. Słyszałem, jak mówiono o objawach fizycznych, które nagle zniknęły -- astmie, przewlekłych bólach głowy i~tym podobnych -- lub o nocnym wyzdrowieniu z~przewlekłej depresji.

 Kontrast ze światem pracy nie jest zaskakujący. W końcu praca jest zarówno miejscem, w~którym większość dorosłych Amerykanów spędza większość czasu na jawie, jak i~miejscem, w~którym mają najbardziej regularne doświadczenie w~hierarchicznej organizacji, w~szczególności tam, gdzie mają do czynienia z~tymi, którzy mogą wydawać im polecenia. Poza tym dorośli Amerykanie w~dużej mierze radzą sobie z~rozkazami lub ci, którzy mogą traktować ich bezkarnie podczas interakcji z~biurokratami lub policją, domyślają się, że mogą zwyczajnie unikać tego tematu, gdy tylko jest to możliwe. Podczas gdy dla większości praca wymaga dużej współpracy i~wzajemnego wsparcia, szczególnie w~przypadku tych o równym statusie, arbitralność szefów jest dla większości anarchistów najbardziej bezpośrednią postacią wszystkiego, co nie powinno być w~anarchistycznym podejmowaniu decyzji.

 Odkryłem, że w~DAN oskarżenia o despotyczne, autorytarne zachowanie pojawiały się najczęściej, gdy aktywiści angażowali się w~role zbyt podobne do tych, do których byli przyzwyczajeni w~świecie korporacji. To była pułapka, ponieważ trudno było jej uniknąć. Jeśli ktoś ma duże doświadczenie, powiedzmy, w~technikach public relations lub montażu wideo, wniesienie własnej wiedzy do ruchu wydaje się oczywiste. Często jednak osobom przyzwyczajonym do korzystania z~takich umiejętności w~świecie korporacji niezwykle trudno jest całkowicie zerwać z~nawykiem traktowania osób z~mniejszym doświadczeniem jako podwładnych. Znam kilku, którzy świadomie unikają angażowania się w~pracę zbyt podobną do tego, co robią w~sektorze formalnym właśnie z~tego powodu.

\medskip
\noindent  2) Grupy Sekciarskie

 Ilość czasu, jaki anarchiści spędzają na narzekaniu na sekciarskie grupy marksistowskie, jest dość uderzająca; jest to tym bardziej w~przypadku anarchistów zaangażowanych w~grupy takie jak DAN, które często pracują w~dużych koalicjach i~których kontakt z~marksistami wykracza poza zwykłe interakcje na wiecach, imprezach lub dzielenie od czasu do czasu celi więziennej. Muszą na bieżąco współpracować z~członkami takich grup. Co więcej, każda duża grupa anarchistyczna może zawierać co najmniej jedną lub dwie osoby, które same były zaangażowane w~jedną lub inną sekciarskich grup marksistowskich, tylko po to, by zrezygnować lub zostać wydalonymi. Zapewnia to zarówno poziom głębokich osobistych uczuć, jak i~dość dokładną wiedzę o tym, jak faktycznie działają takie grupy. -- Byłam tam przez dwa lata -- powiedziała Marina, mówiąc o ISO. -- Nie, język kultu jest zamierzony. Tak o tym mówimy. Naprawdę tak jest.

 Oczywiście członkowie grup takich jak Międzynarodowa Organizacja Socjalistyczna (ISO -- International Socialist Organization), Rewolucyjna Partia Komunistyczna (RCP -- Revolutionary Communist Party), Światowa Partia Robotnicza (WWP -- Worker’s World Party), a nawet Liga Młodzieży Spartystycznej, nie odnoszą się do siebie jako ,,sekciarzy'', a tym bardziej do członków ich organizacji frontowych. Kiedy nazywam ich ,,sekciarzami'', po prostu przyjmuję anarchistyczny punkt widzenia. Anarchiści mają tendencję do używania tego terminu bardzo szeroko, w~zasadzie, w~odniesieniu do każdej organizacji politycznej, która ma intelektualnego przywódcę i~linię partyjną. Pozwolę sobie zatem podsumować ten stereotyp bez żadnych roszczeń w~ten czy inny sposób co do jego prawdziwości: 

 Grupy sekciarskie są niezmiennie zorganizowane jako partie polityczne, nawet jeśli składają się tylko z~piętnastu osób i~nigdy nie rozważały kandydowania na jakikolwiek urząd publiczny. Są zorganizowane hierarchicznie, z~charyzmatycznym przywódcą (niezmiennie mężczyzną), który jest jednocześnie głównym teoretykiem grupy. Ten przywódca jest uważany za założyciela własnej szkoły teorii marksistowskiej, ponieważ w~zasadzie grupa postrzega siebie jako wyłaniającą się z~pewnego teoretycznego rozumienia świata, a nie teoretycznego rozumienia wyłaniającego się z~grupy. (Grupy maoistów na przykład odróżniają się od grup trockistowskich przede wszystkim dlatego, że akceptują, że chłopi, a nie tylko proletariusze, są potencjalnie rewolucyjną klasą. Jest to prawdziwe nawet w~krajach, w~których chłopi nie istnieją.). Komitet centralny grupy wykorzystuje tę analizę, aby dostarczyć wszechstronnej analizy sytuacji na świecie, wytyczając stanowiska w~głównych kwestiach politycznych dnia, a często odpowiadając na pytania, które niektórzy mogą uznać za słabo związane z~polityką. Rzekomo na przykład RCP stoi na stanowisku, że podczas gdy związki osób tej samej płci między kobietami można uznać za uprawnioną formę oporu wobec patriarchatu, związki osób tej samej płci między mężczyznami są burżuazyjną dewiacją. Chociaż wstępowanie do partii gejów nie jest surowo zabronione, ci, którzy publicznie nie zgadzają się z~tym stanowiskiem, jak mi powiedziano, początkowo podlegają presji, a ostatecznie są usuwani.

 Partia również przyjmuje stanowiska wobec wirtualnie wszystkich konfliktów świata (np. Czy popieramy autonomię lub niezależność regionu baskijskiego w~Hiszpanii? Jakie jest nasze stanowisko w~sprawie porozumień z~Kioto?), podobnie jak oczekujący rząd i~zastanawia się nad wydarzeniami z~przeszłości, takimi jak sowiecka inwazja na Węgry lub czy rząd chiński miał rację, tłumiąc demonstrantów demokracji na placu Tiananmen. Te stanowiska są zwykle publikowane głównie w~gazetach partyjnych. Uroczystym obowiązkiem każdego członka partii jest sprzedaż jak największej liczby egzemplarzy tych gazet, zarówno w~celach zbierania funduszy, jak i~propagandy. Zaangażowanie i~lojalność członka partii często mierzy się tym, ile gazet w~danym miesiącu udaje mu się wyładować. Dla anarchisty najłatwiejszym sposobem zidentyfikowania sekciarza na pokazie jest szukanie stosu gazet pod jego pachą. Wreszcie, podczas gdy grupy sekciarskie regularnie wspierają i~bardzo ciężko pracują dla kampanii organizowanych przez innych lub same inicjują znacznie szersze koalicje, zawsze robią to albo z~myślą o przejęciu kontroli nad tą koalicją, albo przynajmniej, wykorzystanie jej do realizacji własnego programu strategicznego, jakkolwiek pomyślanego.

 Kilka cech tutaj się ujawnia. Po pierwsze: grupy te zawsze wyobrażają sobie siebie jako projekty intelektualne. Wyłaniają się z~pewnej definicji rzeczywistości. To dlatego tak trudno tolerować dysydentów, dlaczego poważny teoretyczny spór, jeśli nie rozwiązany, nieuchronnie prowadzi do rozłamu i~powstania nowej grupy. Wyjaśnia również nacisk na propagandę: sposób, w~jaki koalicje są postrzegane, to w~dużej mierze sposoby rozpowszechniania idei, znaczenie gazety. Celem jest ostatecznie doprowadzenie mas do pewnego poziomu świadomości; kiedy to zrobią, będzie można dokonać rewolucji. Wreszcie, sformułowanie grupy jako projektu intelektualnego może uzasadniać hierarchiczne struktury wewnętrzne, które w~innym przypadku byłyby bardzo trudne do obrony: jeżeli lider jest naprawdę jedyną osobą zdolną do teoretycznej analizy światowej sytuacji, trudno zakwestionować jego autorytet.

 Pod wieloma względami grupy oparte na konsensusie są doskonałym odwróceniem tego wszystkiego. Zaczynają od założenia, że  różnorodność perspektyw jest wartością samą w~sobie, że nikt nie może tak naprawdę przekonwertować innego całkowicie na ich punkt widzenia, a próbowanie jest prawdopodobnie złym pomysłem. Debata nie dotyczy kwestii definicji, ale bezpośrednich kwestii działania w~teraźniejszości, z~czego bezpośrednio wynika nacisk na utrzymanie egalitarnych struktur. Najwyraźniej widać różnicę, gdy przychodzi chwila na opracowanie deklaracji misji lub zasad jedności. Na przykład natychmiast po 11 września 2001 roku, działacze w~Nowym Jorku zebrali się, aby przedyskutować, co robić. Jedna grupa, zdominowana przez marksistów i~starszych weteranów lat 60., zaczęła spotykać się na Forum Brechta; inny, młodszy, bardziej anarchistyczny tłum zaczął spotykać się w~Charas. Obie natychmiast wdały się w~kontrowersyjną debatę na temat ich zasad jedności. Grupa Forum Brechta natychmiast zaczęła debatować nad swoimi stanowiskami na temat tego, co USA powinny, a czego nie powinny robić w~Afganistanie, kwestii legalności międzynarodowej, roli ONZ, ale także tego, czy ich grupa powinna określać się jako ,,antyimperialistyczna''. W drugiej takie sprawy nigdy się nie pojawiły. Zamiast tego niemal natychmiast pogrążył się w~debatach na temat własnego procesu demokratycznego i~mechanizmów zwalczania wewnętrznego rasizmu.

 Nie chodzi o to, że grupy sekciarskie nie mają obsesji na punkcie działania. Mają. Ale ich podejście jest zupełnie inne. Zwykle zaczynają od strategicznej wizji, a następnie myślą w~kategoriach najskuteczniejszych środków do jej realizacji. Rezultatem jest (znów mówię z~perspektywy anarchistycznej) rodzaj bezwzględnego, bezdusznego utylitaryzmu, świat racjonalnych kalkulacji środków i~celów. Często może prowadzić do niemal całkowitego poświęcenia osobistej samorealizacji lub budowania społeczności na rzecz dyscypliny w~stylu wojskowym, która często wydaje się nie do odróżnienia od kapitalistycznej racjonalności, której twierdzi, że się sprzeciwia. Obraz schludnie skrojonego dzieciaka z~ISO w~zapinanej koszuli, który pojawia się na czyimś wiecu, by sprzedawać gazety, stał się dla wielu w~DAN idealnym ucieleśnieniem tego ducha. Jak to się stało, że jednym z~kluczowych mierników lojalności wobec organizacji antykapitalistycznej jest gorliwość w~angażowaniu się w~agresywne techniki marketingowe? W przeciwieństwie do tego anarchiści mają tendencję do upierania się, że akcja rewolucyjna powinna być również formą wyrażania siebie i~że powinna opierać się na etyce radykalnie innej niż etyka panująca w~kapitalizmie.

 Wreszcie pojawia się kwestia dyscypliny partyjnej i~tego, co można by nazwać wynikającą z~niej etyką siebie. Wielu anarchistów będzie podkreślać, że nigdy nie można wiedzieć, co tak naprawdę myśli każdy członek sekciarskiej grupy. Jasne, członek partii może być doskonale zdolny do wyrażania niezgody, a nawet naśmiewania się z~niektórych aspektów linii partyjnej, zwłaszcza przy piwie. Ale nigdy nie można być pewnym, jak to zinterpretować. Nigdy nie jest się do końca pewne, kiedy rozmawia się z~jednostką, a kiedy z~członkiem partii, w~jakim stopniu nakładają się one na siebie, w~jakim stopniu mogą pojawiać się wątpliwości wewnętrzne, w~jakim stopniu przedstawienie wewnętrznych wątpliwości może być chwytem strategicznym, czy osoba flirtująca z~tobą robi to, ponieważ faktycznie uważa cię za atrakcyjną lub dlatego, że ktoś inny, ktoś, kogo nigdy nie spotkałeś, zdecydował, że jesteś potencjalnym rekrutem i~zachęcił ich do tego. Takich rzeczy po prostu nie da się poznać. To właśnie takie pytania skłaniają wielu anarchistów do myślenia o takich grupach jako niewiele lepszych niż sekty lub generują pogłoski na temat ich wewnętrznej polityki seksualnej (na przykład, że od członków oczekuje się jedynie romantycznego zaangażowania z~innymi członkami lub że są nawet kojarzeni przez ich przełożonych, że oczekuje się, że członkinie grupy będą dostępne seksualnie dla Przywódcy itd.) -- plotki, że słyszałem o wielu takich grupach i, oczywiście, nie mam absolutnie żadnego sposobu, aby je potwierdzić. W wielu przypadkach takie plotki są prawie na pewno nieprawdziwe. Mimo to wydają się niezwykle podobne do tych, które prawie zawsze otaczają grupy religijne, również określane jako ,,kulty'', co samo w~sobie jest znaczące.

 Do pewnego stopnia jest to dokładnie to, czego można by się spodziewać, łącząc intensywne osobiste zaangażowanie ze stosunkowo formalną, odgórną strukturą organizacyjną. W przeciwieństwie do tego, grupy i~sieci anarchistyczne opierają się na czymś, co w~istocie jest siecią wysoce osobistych relacji. Obejmują one bardzo niewiele czysto formalnych, bezosobowych mechanizmów, z~wyjątkiem być może roli facylitatora. W rezultacie anarchiści rozwinęli -- znowu, często w~dość świadomym przeciwieństwie do sekciarzy -- swego rodzaju ideał przejrzystości moralnej i~etos solidarności. Ideał przejrzystości jest oczywiście właśnie ideałem; nikt nie wyobraża sobie, że jest to całkowicie osiągalne. Niemniej jednak częścią celu organizacji antyautorytarnych jest stworzenie środowiska, w~którym przynajmniej można być szczerym co do własnych motywów lub imperatywów\footnote{Można by przynajmniej powiedzieć, że tworzą środowisko, w~którym jeśli flirtuje się z~kimś z~ukrytych pobudek, można być całkiem pewnym, że są to własne ukryte pobudki, a nie przełożone.}. Z kolei ,,Solidarność'' oznacza swobodnie wybraną decyzję podporządkowania się motywom lub imperatywom innych. Opisany przez Jessikę opis przyjemności odczuwanej w~swobodnym decydowaniu o traktowaniu własnej opinii jako nieważnej może być postrzegany jako intelektualny ekwiwalent. Jak podkreśliłby każdy anarchista, jeśli stosuje się jakąkolwiek presję moralną, to nie jest to tak naprawdę solidarność. To dlatego anarchistyczne związki zawodowe w~Hiszpanii na początku stulecia nalegały, aby każdy, kto głosował przeciwko strajkowi, nie był zobowiązany do poszanowania linii pikiet (zwykle w~rezultacie uzyskiwali 100\% zgody) lub dlaczego obecnie ci praktykujący więzienna solidarność, w~której aktywiści zapychają system więzienny, odmawiając podania swoich nazwisk, zawsze nalegają, aby każdy, kto ma osobisty powód, by się wycofać, to zasadą jest zapewnienie ich, że nikt nie będzie myślał o nich gorzej. Jeśli jest presja, to nie jest to prawdziwa solidarność. Ale solidarnością nie jest też postrzeganie siebie jako osoby, która przede wszystkim stara się realizować własne interesy. Malcolm, jeden z~mocarnych członków DAN Labor, zwykł wdawać się w~niekończące się kłótnie z~członkami ISO, którzy od czasu do czasu uczestniczyli w~grupie roboczej, kiedy powiedział im, że uważa, że w~ogóle nie praktykują solidarności. -- To prawda -- powiedział nam pewnego wieczoru w~jakimś lokalnym barze. -- Jeśli spojrzysz na program ISO, tak naprawdę mówi ona, że  w~przypadku konfliktu imperatywy organizacyjne partii muszą zawsze mieć pierwszeństwo przed interesami twoich sojuszników, sprawy lub koalicji. Jeśli pomagasz innym ludziom tylko po to, by realizować swój własny program, nie praktykujesz solidarności. Zawsze się irytują, jak im to wykazuję.

\medskip
\noindent  3) Internet

 Zwykle, gdy inicjuje się nową grupę, sieć lub koalicję, pierwszym krokiem jest zorganizowanie spotkania. Nowa grupa decyduje o procesie decyzyjnym i~uzgadnia zasady jedności, a także wybiera czas i~miejsce przyszłych spotkań. Czasami wymyślają też nazwę, choć często pojawia się ono później. Ale prawdziwym znakiem, że taka grupa faktycznie istnieje, jest utworzenie strony internetowej i~serwera listserv. Witryny internetowe wymagają sporo konserwacji. Jeśli jakaś grupa zanika, strona internetowa po prostu tam jest, często przez lata bez aktualizacji. Z drugiej strony, listserv wymagają bardzo niewielkiej konserwacji, więc często wydaje się, że raz uruchomione, nigdy nie znikną. Grupy mogą przestać się spotykać i~skutecznie już nie istnieć, ale listserv będzie kontynuowany jako środek do ogłaszania wydarzeń lub, czasami, ograniczonej dyskusji przez lata. Kiedy pytałem o takie grupy -- na przykład Connecticut Global Action Network, czy Texas DAN -- odpowiadano mi: 
 
 -- Cóż, to właściwie tylko listserv.

 Odkąd krótko po Seattle, Naomi Klein (2001) opisała zdecentralizowaną organizację sieciową nowych grup aktywistów jako naśladującą strukturę i~ducha Internetu, niekończący się strony zostały zapisane na opis relacji między anarchistyczną organizacją a nowymi technologiami informacyjnymi. Z pewnością takie powiązania istnieją. Wiele nowych globalnych sieci lub sojuszy, takich jak PGA czy Indymedia, byłoby nie do pomyślenia bez Internetu. Jednocześnie wpływ Internetu był znacznie bardziej skomplikowany, niż się to zwykle przedstawia. Przede wszystkim wszyscy, których znałem, przyznawali, że choć jest to niezwykłe narzędzie do rozpowszechniania informacji, to nie można używać Internetu do podejmowania decyzji. Biorąc pod uwagę wagę procesu decyzyjnego, jest to niezwykle istotne ograniczenie. Zamiast tego na listservach aktywistów można spotkać styl debaty, który przez definicję nie może prowadzić do kolektywnych decyzji i~który, dla wielu, obejmuje wszystko to, czym proces decydowania oparty na konsensuie nie powinien być.

-- Problem z~e-mailami -- powiedział mi kiedyś Enos -- jest tak bardzo podobny do mowy. Budzisz się pewnego ranka, nie piłeś kawy i~czytasz coś, co cię wkurza. Więc rzucasz odpowiedź. A może dziesięć minut później już myślisz o tym lepiej, ale jest już za późno. Nie możesz tego cofnąć. Jest tam na tysiącu maszyn i~nawet jeśli wyślesz przeprosiny minutę później lub spróbujesz usunąć je z~serwera, ktoś w~Cleveland może znaleźć je na swoim dysku twardym za pięć lat i~wściec się na ciebie. To tak, jakbyś myślał, że po prostu mruczysz do kawy, ale kończy się wypisane na egipskiej piramidzie. 

 Z punktu widzenia wielu aktywistów, listserv wydaje się łączyć najgorsze aspekty mowy i~pisania: przypadkową bezmyślność jednego, trwałość drugiego. i~bez względu na powody, debata na temat listservs zmierza w~kierunku wszystkiego, czego spotkania konsensusu starają się uniknąć: pozowania, wyniosłych twierdzeń, sarkazmu, obelg, wielkich oskarżeń o seksizm, rasizm, głupota, reformizm, hipokryzja. Po latach monitorowania listservs aktywistów mogę stwierdzić, że jest prawie niemożliwe, aby znaleźć dyskusję na gorący temat, który ostatecznie nie sprowadzi się do jakiejś wojny, zwykle opartej na stylu debaty -- rodzaju pugilistycznego, macho udawany racjonalizm -- którego po prostu nigdy nie obserwuje się w~nieformalnych spotkaniach twarzą w~twarz, przynajmniej między aktywistami, którzy zakładają, że mają ze sobą coś wspólnego\footnote{Dość często spotyka się osoby, które są znane z~tego, że osobiście są łagodnie usposobione i~nieobraźliwe, ale są wyjątkowo agresywne w~Internecie. Nigdy nie słyszałem o przypadkach, w~których było odwrotnie.}. Bez żadnego z~mechanizmów rozładowania takich konfliktów (żadnego facylitatora, bez migania, bez mowy ciała, a przede wszystkim bez możliwości usłyszenia, czy cała publiczność jęczy lub wyraźnie życzy sobie, abyś się wyłączył), konflikty mają tendencję do eskalacji. Jak bez końca podkreślają aktywistki, wynikająca z~tego debata jest zwykle prowadzona prawie wyłącznie przez mężczyzn. Wynika to częściowo z~tonu debaty, a częściowo również dlatego, że pozornie bezosobowe medium pozwala męskim uczestnikom powrócić do uderzających wzorców seksizmu, których nigdy nie wyobrażali sobie, by stosowali osobiście: na przykład systematyczne odpowiadanie na kobiece (w przeciwieństwie do do mężczyzny) plakaty nie z~kontrargumentami, ale z~protekcjonalnością lub spekulacjami na temat osobistych cech osoby. Te nieliczne aktywistki, które nawiedzają fora e-mailowe, robią to, ponieważ lubią walczyć tak dobrze, jak tylko mogą, i~są nawet bardziej kontrowersyjne niż mężczyźni. W każdym razie, debaty DAN listserv zwykle wybuchały na dzień lub dwa i~prawie zawsze kończyły się, gdy jedna lub druga centralna kobieta DAN wydawałaby się zauważać seksistowską naturę wymiany, brak kobiecych głosów i~mówić mężczyznom. aby to wycięli\footnote{Nacisk na spotkania twarzą w~twarz i~kulturę werbalną może być częścią przyczyny wspomnianego powyżej zaskakującego rozdźwięku między teorią anarchistyczną lub ogólnie anarchistycznym pisaniem a praktyką anarchistyczną. Na stronach internetowych, na przykład, zaczyna się widzieć niektóre z~tych samych radykalnych, antyorganizacyjnych, prymitywistycznych, indywidualistycznych lub sekciarskich głosów, które spotyka się na stronach ukazujących się od dawna anarchistycznych periodyków. Jedyne podobieństwo wydaje się polegać na tym, że są to fora pisane, na których nie ma bezpośrednich obecnych odbiorców.}.

 Wracamy więc do tego samego pytania postawionego w~części dotyczącej sekciarstwa: jeśli czyjś ruch nie wyłania się z~teorii lub intelektualnej debaty, z~uprzedniej analizy sytuacji na świecie -- to z~czego właściwie się wyłania?

\medskip
\noindent POROZUMIENIA WYNIKAJĄCE Z PRAKTYKI

 Pozwólcie, że zmapuję, co uważam za najistotniejsze cechy podejmowania decyzji na podstawie konsensusu.

 Przede wszystkim konsensus to sposób na podejmowanie decyzji zgodnych ze społeczeństwem, które nie stosuje systematycznej przemocy w~celu egzekwowania decyzji. Jest to próba znalezienia formuły moralnej, która jednocześnie maksymalizuje indywidualną autonomię i~zaangażowanie we wspólnotę. W pewnym sensie, pomimo wyraźnego odrzucenia tego rodzaju opartej na teorii wersji marksizmu, typowej dla wielu sekciarskich grup marksistowskich, jest to całkowicie zgodne z~innymi nurtami myśli marksistowskiej. Myślę, że to nie przypadek, że teoretykiem marksizmu najbardziej podziwianym dzisiaj przez amerykańskich anarchistów nie jest tak naprawdę Antonio Negri, ale prawie nieznany w~amerykańskiej akademii John Holloway, którego prace -- zwłaszcza \textit{{\textgreater}Change the World without Taking Power}(2002) -- można znaleźć w~niemal każdym anarchistycznym punkcie w~Ameryce Północnej.

 Holloway twierdzi, że sama idea, jakoby partia polityczna mogła posiadać prawdziwe, ,,naukowe'' rozumienie świata, jest nadużyciem wobec wszystkiego, co było naprawdę rewolucyjne w~ideach Marksa. Przyjęcie rozróżnienia, wywodzącego się zarówno z~radykalnych, feministycznych (np. Starhawk 1987), jak i~włoskich odłamów ruchu autonomicznego (np. Virno \& Hardt 1996; Negri 1999), między władzą działania lub tworzenia a władzą przymusu lub podporządkowania innych -- często sformułowane jako rozróżnienie między ,,władza wobec'' i~,,władza nad'' -- rozróżnia dwie odpowiadające sobie formy wiedzy. Jednym jest wiedza immanentna w~praktyce, w~jakimś aktywnym projekcie tworzenia lub transformacji społecznej, drugim, który nazywa ,,wiedzą o'', udaje, że unosi się ponad takimi subiektywnymi formami wiedzy i~osiąga prawdziwą obiektywność naukową. Udając mistrzostwo i~transcendencję, jego tendencja do redukowania świata procesów do stałych, identycznych ze sobą obiektów, które następnie mogą stać się przedmiotami wszechstronnej wiedzy, ,,wiedzy o'' jest doskonałym uzupełnieniem ,,władzy''. Jak Marks dobrze zademonstrował, potężni zawsze próbują zredukować złożone procesy tworzenia i~działania do stałych przedmiotów, o których mogą twierdzić, że są właścicielami. Sam kapitał jest najlepszym przykładem. Ich ,,nauka'' staje się środkiem, za pomocą którego to robią.

 Jak wszystkie takie wspaniałe sformułowania, Holloway jest bez wątpienia pewnym uproszczeniem. Mimo to myślę, że jest tu coś głęboko prawdziwego, i~coś ważnego dla obecnych celów, choćby dlatego, że jego analiza wyłania się właśnie z~tradycji intelektualno-praktycznej, z~której wyłonił się sam proces konsensusu. Można by zatem zacząć od stwierdzenia, że  konsensus jest podejściem, które zastępuje ideologię ,,wiedzę o'' formami wiedzy immanentnymi w~praktyce. Jak pisałem w~innym miejscu (2002, 2003), jego praktykowanie \textit{jest }ideologią. Wskazałem również, że w~przeciwieństwie do grup sekciarskich, grupy oparte na konsensusie mają tendencję do unikania debaty, nie mówiąc już o opieraniu swojej tożsamości na kwestiach definicji. Zamiast tego zawsze starają się sprowadzić sprawy z~powrotem do kwestii działania. Zatem moją pierwszą sugestią jest, żebyśmy spojrzeli na to, jakbyśmy zajmowali się polityczną ontologią, która zakłada, że to działania, akcje, a nie obiekty, są podstawą rzeczywistości.

 Jeśli tak, myślę, że reszta dość łatwo się układa: 

\noindent  1) \textit{Zasady założycielskie}

 Każda grupa konsensusu -- czy to mała grupa afinicji, czy rozległa sieć -- opiera się na zasadach założycielskich. Zasady te zwykle odnoszą się do tego, co grupa robi lub ma nadzieję osiągnąć (swoje ,,\textit{cele }lub powody istnienia'' i~jak organizuje się, aby to zrobić. Innymi słowy, sama grupa jest projektem działania. Na innych szkoleniach konsensusu doświadczeni facylitatorzy podkreślaliby, że zawsze, gdy wydaje się, że pojawia się problem nierozwiązywalny lub różnica zdań, najlepszym podejściem jest zawsze przypominanie członkom grupy o racji ich istnienia. -- Bez względu na to, jaki to może być wspólny projekt, nawet jeśli jest to pięciu uczniów mieszkających razem w~domu, zawsze zaczynaj od tego, że wszyscy zgadzają się na coś, co chcą wspólnie osiągnąć. Bo kiedy wszyscy zaczynacie się kłócić o zmywanie, najlepszym sposobem na powstrzymanie ludzi przed rzuceniem się sobie do gardeł będzie zawsze powrót do tego. -- Oczywiście blokady muszą być zakorzenione również w~tych zasadach: można tylko zatrzymać określony kierunek działania, twierdząc, że jest on sprzeczny z~tymi bardziej ogólnymi celami.

\noindent  2) \textit{Założenie różnorodności}

 Kiedy skupimy się na wspólnym działaniu, a nie na zgodzie co do natury jakiejś wyższej Prawdy, zestawu definicji lub prawidłowej analizy, łatwiej jest dostrzec, jak różnorodność perspektyw może wydawać się siłą, niż słabością. Fakt, że ludzie żyją w~niewspółmiernych światach, rzadko uniemożliwiał im skuteczne realizowanie wspólnych projektów. Może się to wydawać sprzeczne -- filozof mógłby argumentować, że jeśli ludzie żyją w~niewspółmiernych światach, nie mogą dążyć do tych samych celów, ponieważ nie mogliby nawet uzgodnić, jakie te cele w~ogóle są -- ale jest to rodzaj sprzeciwu, który wyłania się ze świata, który zaczyna się od form platońskich i~stara się z~nich wywodzić, aby wyjaśnić rzeczywistość empiryczną. Lub, jak ująłby to Holloway, zaczyna się od ,,wiedzy o''. Faktem jest, że wszystkie perspektywy są do pewnego stopnia niewspółmierne, a mimo to ludzie współpracują. Wracając do przykładu decydowania, do której restauracji się udać: ekonomiści mają świadomość, że nawet tak podstawowe obliczenia jak ,,która restauracja jest najlepsza za te pieniądze?'' nie dają się formalnie modelować, bo mówi się o jakościach niewspółmiernych: ilościowe kontra jakościowe, kardynalne kontra porządkowe. Matematycznie takie obliczenia są zupełnie niemożliwe. To nie powstrzymuje zwykłych ludzi przed robieniem ich przez cały czas. Wszelkie podejmowanie decyzji w~świecie rzeczywistym jest w~rzeczywistości serią formalnie niemożliwych kompromisów między niewspółmiernymi terminami. Dzieje się tak nawet wtedy, gdy decyzję podejmuje tylko jedna osoba; tym bardziej, gdy jest ich wielu. Tak więc z~jednej strony różnorodność jest nieunikniona\footnote{Jest to również pożądane: jeśli ktoś próbuje rozwiązać problem lub wykonać zadanie, prawie zawsze będzie to łatwiejsze w~grupie pięciu różnych osób niż w~grupie identycznych klonów.}.

 Obserwatorzy akademiccy, wystawieni na perspektywę, która odrzuca jakiekolwiek pojęcie prawdy transcendentnej i~celebruje różnorodność, mają tendencję do postrzegania tego zjawiska jako formy ,,postmodernizmu''. Większość aktywistów, którzy sami nie są akademikami, nie lubi takich określeń, choćby dlatego, że niemądre wydaje się przyklejanie ideologicznej etykiety do postawy antyideologicznej\footnote{Reclaim the Streets London dość ładnie podsumował panujące nastawienie na swojej stronie internetowej: ,,Ludzie wokół RTS są zdecydowanie przeciwni totalizującym ideologiom. Nie znaczy to, że jest dużo sympatii dla postmodernizmu czy czegokolwiek. Jest to głównie aberracja akademicka, którą należy kopnąć na ścieżce ku temu, co naprawdę nadejdzie po modernizmie'' (\url{http://rts.gn.apc.org/ideas.htm}, dostęp 20 czerwca 2005).}. Bardziej prawdopodobne jest, że postrzegają to, co robią, jako powrót do zasad prostej przyzwoitości i~zdrowego rozsądku -- powrotu, jeśli trzeba użyć tego określenia, do świata, w~którym wiele z~tego, co nazywamy ,,nowoczesnością'', nigdy się nie wydarzyło.

\medskip
\noindent [Z moich wczesnych notatek terenowych:]

\noindent [Notatniki lipiec 2000]
\medskip

 Przyjęcie podejścia opartego na konsensusie ma tendencję do sprawiania, że wszystko staje się kompromisem i~promuje szczególny rodzaj poglądu na prawdę, który może nie jest całkiem relatywistyczny, ale raczej świadomość, że prawda jest nieco posklejana. Proces podejmowania decyzji wyraźnie wpływa na ogólną postawę intelektualną. Wszystkie wypowiedzi są zbiorowe, co czasami może sprawić, że będą nijakie, ale nigdy ukryte. Duże obszary są zawsze dyskusyjne, tzn. czy głównym problemem jest kapitalizm, czy brak demokracji? Czy chcemy państwa? Oczywiście wszystko to ma wiele wspólnego z~różnorodnością stanowisk w~DAN, ale prawdą jest również, że DAN aktywnie poszukuje tego rodzaju różnorodności. Jeśli zapytasz jedną z~podstawowych grup DAN, jakie instytucje ekonomiczne wyobrażają sobie dla przyszłego społeczeństwa, prawdopodobnie odpowiedzą, że nie do nich należy decydowanie: próbują stworzyć demokratyczne instytucje, które pozwolą ludziom samemu odpowiadać na takie pytania. Nawet jeśli chodzi o anarchizm, podczas gdy większość uczestników wydaje się anarchistami, jest trochę marksistów, liberalnych reformistów, a nawet typów organizacji pozarządowych. Nic z~tego nie jest uważane za problem, o ile wiem, chyba że wydają się mówić nie za siebie, ale za organizację. Slogan może brzmieć: ,,Tak długo, jak jesteś gotów zachowywać się jak anarchista w~teraźniejszości, twoja długoterminowa wizja zależy w~dużej mierze od ciebie''.
\medskip

\noindent  3) \textit{Etos wzajemnej solidarności}

Jak zaobserwowano, łączy nacisk na indywidualną autonomię z~zaangażowaniem na rzecz innych. Zakłada się tutaj, że indywidualna wolność nie jest brakiem zobowiązań czy uwikłań, ale raczej, że w~dużej mierze polega na wolności decydowania o sobie, w~jakie projekty lub społeczności chce się zaangażować.

 W każdym razie w~taki sposób można podejść do pomysłu z~perspektywy jednostki. Z perspektywy grupy można argumentować, że etos solidarności wywodzi się z~centralnej zasady anarchizmu: że tak jak ci, którzy są traktowani jak dzieci, będą zachowywać się jak dzieci, najlepszym sposobem na zminimalizowanie egoizmu, złośliwości, obłudy, lub małostkowego zachowania polega na skutecznym prowokowaniu ludzi do dojrzałości. Przyznając każdemu członkowi grupy prawo do blokowania, zmusza się każdego do bycia świadomym, że w~dowolnym momencie może wprowadzić w~grupie spustoszenie. To, w~połączeniu z~odmową stosowania presji moralnej, sprawia, że  niezwykle trudno jest każdemu wcielić się w~rolę ofiary lub powiedzieć sobie, że robi tylko to, co musi, aby wygrać wcześniej ustaloną grę polityczną\footnote{Oczywiście, ludzie mogą być nieskończenie kreatywni w~takich sprawach i~jeśli ktoś jest absolutnie zdeterminowany, by ułożyć narrację, w~której jest ofiarą, zawsze jest na to sposób. Mimo to zadziwiające jest, jak rzadko się to zdarza.}. To trochę tak, jak wręczanie każdemu, kto wejdzie do pokoju, granatu, tylko po to, by pokazać, że ufasz, że go nie użyją. Jednak okazuje się to zaskakująco skuteczne.

 Tutaj zasady autonomii grupowej i~indywidualnej autonomii moralnej wzmacniają się nawzajem. Na przykład grupy anarchistyczne często czują się bardzo niekomfortowo w~kwestii finansowania z~zewnątrz. Nieraz widziałem grupy pogrążone w~drobnym kryzysie, gdy ktoś z~zewnątrz oferuje kilkaset dolarów datku, nawet jeśli prezent wydaje się nie mieć żadnych zobowiązań. Takie oferty są często odrzucane wprost, z~obawy, że jakakolwiek grupa, która zacznie akceptować finansowanie z~zewnątrz, może w~końcu się ograniczać z~obawy przed wyobcowaniem potencjalnej bazy finansowej. Ostatecznym ideałem w~takich grupach jest zawsze stworzenie sytuacji, w~której nic nie stoi między własnym poczuciem moralności a zdolnością do działania, nie ma powodu, aby \textit{nie }mówić dokładnie, w~co się wierzy, lub robić tego, co uważa za słuszne, gdzie nie ma potrzeby forteli.

 Inaczej można to ująć: celem jest stworzenie środowiska, w~którym jeśli ktoś zachowuje się hojnie lub okropnie, można być względnie pewnym, że dzieje się tak dlatego, że faktycznie jest hojny lub wstrętny. Paradoks polega na tym, że ta próba stworzenia warunków dla moralnej przejrzystości może być podtrzymana jedynie poprzez rodzaj ciągłej gry w~udawanie. Stąd nacisk, aby nigdy nie kwestionować cudzych motywów, należy zawsze dawać im korzyść z~domniemania uczciwości i~dobrych intencji (niezależnie od tego, co ktoś może o nich sądzić osobiście). Naleganie na traktowanie wszystkich jak odpowiedzialnych dorosłych może nie zawsze gwarantować dojrzałe zachowanie, ale z~mojego własnego doświadczenia wynika, że  okazuje się to zaskakująco skuteczne, i~sam ten fakt zaskakuje prawie każdego, kto pierwszy angażuje się w~grupy anarchistyczne, ponieważ w~końcu jest to wprost sprzeczne z~prawie wszystkim, czego nauczono nas myśleć o ludzkiej naturze. Ma to wiele wspólnego z~tym, dlaczego ludzie często zauważają, że takie doświadczenia zmieniają samo poczucie ludzkich możliwości.

 Jest jednak jeden bardzo problematyczny wniosek. Chociaż ta hojność ducha jest jedną z~podstawowych zasad podejmowania decyzji w~drodze konsensusu, skutkuje ona tym, że przy całej otwartości ich sieci wymaga od takich grup nakreślenia bardzo wyraźnych linii między jakimś wewnętrznym kręgiem a wszystkimi. Do tego wrócę później.

\section{Część III: Problemy.}

 Mimo to (a może właśnie z~tego powodu) bardzo często obserwuje się egzaltację, po którym następuje wypalenie. Osoby wciągnięte w~świat organizacji horyzontalnej często uznają to doświadczenie za zdumiewające, wyzwalające, transformujące; otworzy im oczy na zupełnie nowe horyzonty ludzkich możliwości. Sześć miesięcy później mogą równie dobrze zrezygnować z~obrzydzeniem. Albo grupy, z~którymi pracowali, mogą rozpłynąć się w~wyniku gorzkich oskarżeń. Jednak oskarżenia prawie nigdy nie dotyczą samego procesu. W Ameryce co najmniej w~dziewięciu przypadkach na dziesięć podnoszą argumenty na temat rasy -- a w~drugiej kolejności -- klasy i~płci -- zwłaszcza, czy obsesja na punkcie procesu konsensusu i~demokracji bezpośredniej, czy nawet akcji bezpośredniej, są same w~sobie formami białej przywilej.

 Jak widzieliśmy, takie argumenty sięgają co najmniej lat sześćdziesiątych, kiedy przedstawiciele rodzącego się ruchu Black Power zaczęli kwestionować konsensus jako sposób na izolowanie białych członków SNCC (Polletta 2002). Nawet około 2000 roku nadal istniało poczucie, że konsensus był w~jakiś sposób zjawiskiem białej klasy średniej; Trenerzy antyrasistowscy byli skłonni pouczać aktywistów w~stylu DAN o arogancji zakładania, że  ich podejście do organizacji jest w~jakikolwiek sposób moralnie lepsze od tych zatrudnionych przez inne społeczności, bez względu na to, jak hierarchiczne mogą się wydawać. Jest w~tym pewna ironia, bo w~większości krajów sytuacja jest zupełnie odwrotna\footnote{Na przykład wśród Zapatystów przywódcy wojskowi Metysów, tacy jak Marcos, mają tendencję do podejrzliwości wobec konsensusu, podczas gdy to rdzenna baza nalegała na to, aby było to jedyne podejście zgodne z~ich tradycyjnymi wartościami.}.

 Niektóre z~podstaw tych zastrzeżeń uległy erozji, gdy grupy oparte na społecznościach kolorowych zaczęły przyjmować niektóre z~tych samych technik. Ale wyraźnie był w~tym element prawdy. Styl wypowiedzi oczekiwany na spotkaniach DAN, jak sądzę, odzwierciedlał, a może lepiej powiedzieć, że był oparty na, pewnym bardzo białym rozumieniu społeczeństwa przez klasę średnią: potrzeba tłumienia niestosownych emocji, szczególnie kontrowersyjnych lub gniewnych, nacisk na zachowanie pozorów wzajemnej uprzejmości, czy pozorów szerzej, przy jednoczesnym unikaniu dramatycznych, performatywnych gestów. O ile było to faktycznie narzucone, a nawet wyraźnie wyrażone, było to zwykle odwoływanie się do feminizmu, a zwłaszcza odrzucenie męskiej postawy macho. Oczywiście był to mocny argument. Z pewnością, nikt nie chciał wysuwać poważnego argumentu na rzecz okazywania osobistej wściekłości lub pozerstwa macho. Można jednak argumentować, że istniała cienka granica między tworzeniem ,,bezpiecznego'' środowiska dla kobiet a odgrywaniem stereotypowej roli łaskawej gospodyni z~wyższej klasy średniej, od której oczekuje się wykonywania niekończącej się pracy nad niwelowaniem różnic i~utrzymywaniem stałą, przyjemną fasadę, aby utrzymać sprawnie działającą działalność społeczną. Prócz tego z~biegiem czasu zauważyłem pewne dość niepokojące wzorce. Na przykład, kolektywy anarchistów lub aktywistów czasami wydalają członków za destrukcyjne zachowanie. Z mojego doświadczenia wynika jednak, że prawie każda osoba, która została w~ten sposób zakazana, pochodziła z~klasy robotniczej lub była kolorowa. Zwykle konkretne powody wydalenia -- ekstremalne nadużywanie substancji, zachowanie agresywne lub gwałtowne -- wydaje się oczywiście uzasadnione. Ale w~Nowym Jorku nie słyszałem jeszcze o konkretnym przykładzie białego aktywisty z~wyższej klasy średniej, wyrzuconego z~anarchistycznego kolektywu, lub nawet, jeśli o to chodzi, opisywanego jako ,,wingnut'' (czubek). W szczególności ,,Wingnut'' jest zasadniczo terminem klasowym: bogaci ludzie mogą być określani mianem ,,ekscentryków'', ale prawie nigdy ,,wingnuts''\footnote{Jednym ze sposobów ujmowania tego może być to, że kiedy ludzie z~bardziej uprzywilejowanych środowisk stają się niestabilni emocjonalnie lub psychicznie, sposób, w~jaki zwykle na to postępują, nie jest uważany za z~natury destrukcyjny. Jedna z~moich przyjaciółek z~klasy robotniczej została skutecznie wyparta ze spółdzielni mieszkaniowej z~powodu konfliktu z~inną kobietą zamożnego pochodzenia, którą w~rzeczywistości uznano za cierpiącą na jakąś formę choroby psychicznej (kiedyś wędrowała dookoła kolektywnego domu w~bieliźnie w~dziwnych godzinach nocnych, przestawiała meble itp. i~wymyślał osobliwe oskarżenia przeciwko kobiecie, którą uznała za swojego wroga). Kiedy pierwsza kobieta poruszyła tę sprawę, powiedziano jej, że niesprawiedliwe jest kwestionowanie upośledzenia umysłowego drugiej; w~tym samym czasie, jej własne okazjonalne okazywanie emocji na spotkaniach było uznawane za całkowicie nieakceptowalne.}.

 NYC DAN był, jak już zauważyłem, niezwykle białą grupą, uderzająco, biorąc pod uwagę demografię miasta\footnote{Wydaje się, że większość jej kluczowych uczestników należała do tych, którzy czterdzieści lat wcześniej mogli być członkami SDS.}. Przez pierwszy rok swojego istnienia grupa znajdowała się w~ciągłym kryzysie, co zrobić z~tym faktem. Jest to wzorzec powtarzany w~nieskończoność w~moim własnym doświadczeniu: grupa w~większości białych aktywistów zbierze się i~tworzy grupę, a następnie natychmiast zacznie się męczyć nad tym, jak poradzić sobie z~własną, wypaczoną kompozycją\footnote{Oczywiście do pewnego stopnia ta przekrzywiona kompozycja jest statystycznie nieunikniona, ponieważ zakłada się, że wielu aktywistów koloru woli pracować w~grupach opartych na tożsamości, ale ci zidentyfikowani jako ,,biali'' będą działać tylko w~grupach wielorasowych.}. Zasadniczo można zastosować dwa podejścia. Można zdecydować się na model rekrutacyjny i~spróbować zachęcić kolorowych aktywistów do przyłączenia się do czyjejś sieci lub model sojuszu, który zakłada, że  większość osób kolorowych będzie wolała tworzyć organizacje oparte na ich własnych afinicjach opartych na wspólnych doświadczeniach ucisku i~że rolą bardziej uprzywilejowanych, głównie białych organizacji jest wspieranie ich walk. W końcu zwykle sprowadza się to do tego drugiego. Argumenty na temat rasy zostają następnie uwikłane w~spory o przywileje, o to, kto ma środki, aby zaangażować się w~,,jazdę na szczyty'' lub tendencję białych aktywistów z~klasy średniej do zajmowania się abstrakcyjnymi kwestiami, takimi jak WTO, zamiast pracować ze społecznościami ,,najbardziej dotkniętymi'' przez politykę neoliberalną w~kraju. Są to kwestie krytyczne, ale w~praktyce często paraliżujące, i~doprowadziły do  zniszczenia dosłownie setek radykalnych projektów. Jednak ze względu na demografię grup takich jak DAN były to bardziej strategiczne pytania niż pytania o rzeczywistą dynamikę spotkań.

 Kwestie płci były bardziej subtelne. Jak wyjaśniłem, proces konsensusu stosowany przez grupy takie jak DAN wyłania się w~dużej mierze z~feminizmu, z~buntu przeciwko standardom Nowej Lewicy lat sześćdziesiątych. Jeśli po prostu przyjrzymy się zwyczajom mówienia lub stylom osobistego zachowania, które są wspólne dla radykalizmu lat 60. -- lub tym, które wciąż często widuje się wśród weteranów lat 60. -- i~porównamy je ze sposobem, w~jaki ludzie zachowują się w~grupach takich jak DAN, zmiana jest namacalna. Wydaje się, że język ciała zmienił się całkowicie. Tam, gdzie kiedyś styl polegał na pchaniu się do przodu i~przemawianiu, gdzie kiedyś wszechobecność wydawała się wojowniczość i~coś, co można by nazwać teoretyczną męskością, obserwuje się bardzo świadomy wysiłek zacierania. Zwłaszcza mężczyźni mają tendencję do odchylania się do tyłu zamiast do przodu; robią to zwłaszcza podczas mówienia. Mają tendencję do ciągłego wykonywania małych gestów szacunku wobec większej grupy. Wszelkie oznaki postawy macho, oratorstwa lub ogólnego zarozumiałości są zwykle zauważane i~szeroko krytykowane poza sceną.

 Można pójść dalej. Klasycznym stereotypem relacji płci w~ruchu w~latach 60. było to, że mężczyźni wygłaszali przemówienia i~plany, kobiety wykonywały nudną i~niewdzięczną pracę urzędniczą i~organizacyjną, która była wymagana do realizacji tych planów. Potem gdy praca została zakończona, mężczyźni ponownie zajmowali centralne miejsce -- czy to jako uliczni bojownicy, mówcy, czy nawet strajkujący głodem -- by występować na publicznych scenach, które faktycznie zbudowały kobiety. Teraz takie wzorce bynajmniej nie zniknęły całkowicie. Być może wśród wychowanych we współczesnej Ameryce nigdy nie uda się ich całkowicie wytępić. W rzeczywistości mają tendencję do ponownego pojawiania się nawet w~tych grupach, które najbardziej świadomie sprzeciwiają się głównym stereotypom płci. Jak ujął to jeden z~weteranów ACT UP, ,,geje śniliby piękne sny. Potem lesby rzeczywiście by poszły i~je zbudowały''.

 Z drugiej strony, deprecjonowanie ideologii, przemawiania i~charyzmatycznego autorytetu oraz odpowiadający temu nacisk na działanie sprawia, że  cały proces bardziej przypomina to, co kiedyś uważano za ,,pracę kobiet''. Jednocześnie brak formalnych stanowisk kierowniczych oznacza, że  dana osoba w~organizacji jest wprost proporcjonalna do ilości pracy, jaką jest się w~stanie wykonać. Rezultat był taki, że o ile w~DAN istniała milcząca struktura przywódcza, składała się ona prawie wyłącznie z~kobiet. Jeśli należałoby wymienić jedną osobę, która najbardziej ucieleśniała projekt DAN, byłaby to Brooke, studentka Instytutu Ekologii Społecznej i~teoretyk demokracji bezpośredniej pod koniec dwudziestego roku życia, która, pochodząc z~zamożnej rodziny, była w~stanie spędzić prawie całą swoją czas wolny w~grupie. Brooke zasadniczo \textit{była }grupą roboczą Nuts \& Bolts; robiła wszystkie żmudne porządkowanie koszyków z~agendą i~formularzy zgłoszeniowych oraz pisanie i~przepisywanie propozycji dotyczących tego, jak składać propozycje. Była także rdzeniem grupy roboczej Continental DAN i~organizowała krajową rozmowę telefoniczną, chociaż technicznie rzadko była mówcą. Jeśli inni pojawili się na którymkolwiek z~tych spotkań, Brooke zawsze traktowałaby ich jak równych sobie, ale faktem był, że zawsze tam była, miała instytucjonalną pamięć i~wszyscy o tym wiedzieli, więc jeśli ktoś miałby pytania dotyczące spraw struktury DAN, to do niej zawsze się zwracali. Tak samo było z~większością strukturalnych grup roboczych. Na przykład przez pierwszy rok DAN to kobieta (Nicky) zarządzała stroną internetową, trzy kobiety (Rachel, Vicky, Marina) tworzyły rdzeń legalnej grupy roboczej, inna kobieta (Zosera), skutecznie tworzyła grupę finansową i~tak dalej. Było kilka wyjątków: adminem listserv był mężczyzną, podobnie jak dwie osoby, które tworzyły rdzeń grupy medialnej, a także zwykle osoba, która kompilowała cotygodniowy kalendarz wydarzeń. Grupa pomocowa była mniej więcej równo podzielona\footnote{Podobnie było ze stałymi grupami działania: Pracy i~Policji i~Więziennictwa, ale nie były one w~tym samym sensie strukturalne.}. Mimo to wzór był nie do pomylenia.

 Tak więc, chociaż można było od czasu do czasu słyszeć narzekania na temat milczącej struktury przywódczej, często wydawało mi się to trochę nieszczere. Jedną z~pierwszych osób, które spotkałem w~DAN, był były działacz IWW o imieniu Sam, który zaangażował się na krótko przede mną. Sam ciągle narzekał na ,,hierarchię'', jak ich nazywał -- kilku z~nich, jak zauważył, pochodziło z~całkiem zamożnych środowisk -- lub wskazywał na to, co uważał za subtelne oznaki, że ta podstawowa grupa naprawdę uważała się za część elity. To nie była tylko jego wyobraźnia. Zacząłem też zauważać, że na przykład podczas wielkiego marszu lub wiecu, gdzie wszyscy zawsze proszą innych o zwolnienie ich z~konieczności noszenia jakiegoś znaku lub sztandaru, nikomu z~nas nigdy nie przyszło do głowy, aby coś takiego zasugerować jednej z~kobiet, powiedzmy, z~Nuts \& Bolts lub grupy Legal. Po prostu miało się wrażenie, że poczuje się, że to jest poniżej nich. i~rzeczywiście, nigdy nie widziano żadnej z~nich niosącej znaki. Jednak z~biegiem czasu stawało się coraz bardziej oczywiste, że cynizm Sama był bezpośrednio związany z~faktem, że on sam nigdy nic nie \textit{zrobił }{}- poza uczestnictwem w~pisaniu wypowiedzi, próbując skierować dyskusję na jeden lub drugi drobny punkt anarchistycznej teorii, lub angażowanie się w~gorące debaty na listservach aktywistów. Bardziej niż cokolwiek innego, jego oburzenie wydawało się wywodzić z~faktu, że w~DAN takie zachowanie samo w~sobie nie było wystarczające, aby zdobyć mu szanowany status w~grupie.

 Wszystko to może wydawać się zaskakujące, że DAN znajdowała się w~niemal nieustannym kryzysie związanym z~kwestiami płci. Główny problem polegał na tym, że tak niewiele kobiet brało udział w~spotkaniach. Podczas gdy proporcje zmieniały się w~czasie, stosunek płci często wynosił dwa do jednego na korzyść mężczyzn; czasami liczby były jeszcze bardziej skrzywione. Prowadzę dość staranne zapisy i~zauważyłem, że w~tych wahaniach były wyraźne wzorce. Zasadniczo kobiety miały tendencję do rezygnacji, gdy nie odbywały się żadne większe akcje. To wtedy spotkania i~tak były najmniejsze, ale na takich spotkaniach na każdą kobietę mogło przypadać czterech mężczyzn. W miesiącach i~tygodniach poprzedzających główną akcję odsetek kobiet stale rósł, często osiągając prawie równość na tydzień lub dwa wcześniej (choć zawsze blisko parytetu: nigdy nie obserwowałem ani jednego dużego spotkania, w~którym kobiety stanowiły większość). Podczas akcji często wydawało się, że kobiety pełnią zdecydowaną większość kluczowych ról: facylitatorki, zespoły taktyczne, rzecznicy mediów i~tym podobne. Potem, po kilku tygodniach, większość kobiet zniknęłaby, pozostawiając tylko te kobiety, które stanowiły część podstawowej grupy DAN -- milczącego przywództwa -- aby kontynuować.

 Wydawało się, że powodem jest to, że zwłaszcza w~okresach ciszy kobiety zaczęły postrzegać większe spotkania jako bezsensowne fora zdominowane przez mężczyzn, którzy lubili słuchać swoich przemówień. Większa część spotkania byłaby zdominowana przez dyskusje na temat tego, czy popierać działania jakiejś innej grupy, lub na temat sformułowania niektórych proponowanych materiałów informacyjnych. Może nie wymagało to dramatycznego pozowania i~przemawiania, ale w~rzeczywistości było tak samo.

 Wkrótce rozwinęło się poczucie, że spotkania DAN nie są w~rzeczywistości całkowicie wygodną przestrzenią dla aktywistek. Wielu narzekało, a niektórzy zaczęli rezygnować z~frustracji. Kilka z~nich zorganizowało kobiecy klub DAN jako sposób na przedyskutowanie problemu i~zaproponowanie rozwiązań. Po kilku spotkaniach w~parku Tompkins Square, uczestniczki klubu kobiet postanowiły zaproponować DAN użycie ,,obserwatora wibracji''. Jest to rola dość znana w~kręgach aktywistów na Zachodnim Wybrzeżu; głównym zadaniem obserwatorów wibracji jest pomoc facylitatorom poprzez monitorowanie ogólnego emocjonalnego odczucia w~pomieszczeniu, ale w~tym przypadku prawdziwym naciskiem było posiadanie kogoś, kto byłby w~stanie monitorować dynamikę płci i~wskazywać seksistowskie zachowania. Opis, które następuje -- etnograficzny rdzeń tej sekcji -- pochodzi ze spotkania DAN 19 czerwca 2000 roku;

\bigskip
\noindent SAGA TRZECIEGO FACYLITATORA
\bigskip


 Pozwolę sobie odtworzyć kilka fragmentów, zrekonstruowanych z~moich notatek, prawdopodobnie najbardziej hałaśliwego spotkania DAN, w~jakim kiedykolwiek uczestniczyłem. Projektując ten rozdział, początkowo wahałem się, czy nie nadaję mu za dużego znaczenia, ponieważ będzie to oznaczać, że jedyne spotkanie DAN, które reprodukuję w~czymś podobnym do całości, było również wyjątkowo dzielące, pełne oskarżeń o seksizm, klasowe. stronniczość i~przynajmniej jeden uczestnik, który wydawał się totalnym wariatem. Mimo to bardzo dobrze służy to wydobyciu napięć, które opisałem, i~pokazanie czytelnikowi, jak mogą -- w~najgorszym przypadku -- rozegrać się podczas rzeczywistego spotkania.

 Spotkanie rozpoczęło się około 17:00, z~Lesley i~aktywistą medialnym Ernest jako facylitatorami. Zaczęło się od około trzydziestu osób w~pokoju, a osiągnęło szczyt około pięćdziesięciu pięciu. Oficjalnym skrybą był Mike, student w~zielonej czapce, który rozpoczął spotkanie, trzymając się ogromnej kopii \textit{The Grundrisse}. Tim, transseksualny aktywista z~grupy Church Ladies for Choice, był osobą z~zegarkiem. To szczególne spotkanie uświetniło również wielu gości, którzy, niestety, zostali umieszczeni na końcu agendy, zwłaszcza czterdziestoletni organizator związkowy nazwiskiem Nathan z~komórki nr 1199 w~czapce UNITE i~młodszy działacz o nazwisku Jack. Griffin, który przyjechał z~partnerką i~dwoma osobami z~ISO, aby poprosić nas o wsparcie dla strajku pralni na Long Island\footnote{Jack Griffin, krajowy koordynator kampanii pralniczych dla Związku Pracowników Przemysłu i~Włókiennictwa (UNITE).}. Kiedy przybyłem i~usiadłem w~kręgu, byliśmy już głęboko pogrążeni w~dyskusji na temat porządku obrad. Moje notatki zaczynają się nieco schematycznie:

\bigskip
\noindent \textit{DAN Generalne, Charas El Bohio}

\noindent \textit{Niedzielne popołudnie, 19 czerwca 2000}
 

\medskip
 Zaczynamy od przeglądu agendy. W agendzie znalazły się akcje w~Windsor (umieszczone w~kategorii Nowe Sprawy), propozycje nauczania, oraz propozycja Griffina dotycząca poparcia strajku pralniczego. Po przydzieleniu czasu na każdy przedmiot kilka osób podniosło ręce, aby ogłosić serię wydarzeń awaryjnych:

 \textit{Tim ogłasza 7 doroczny marsz drag},,celebrujemy kulturę drag, cokolwiek to jest''. Jest to, jak wyjaśnia, wydarzenie organizowane przez ludzi wykluczonych z~Parady Równości, w~tym jego grupę, Radykalne Faerie. ,,Idź w~cokolwiek chcesz, czy to korporacyjny realizm, garnitury, podwiązki i~strąki grochu, cokolwiek. Szukamy też marszałków, którzy będą nosić korale i~kokardki''.

 Chris z~Policja i~Więzienia zapowiada trzy ważne nadchodzące dema. 

 Cindy z~Wetlands zapowiada wielkie demo w~The Gap, którego właściciele są również związani z~niszczeniem starych lasów na Zachodnim Wybrzeżu.

 Ana z~IMC chce opowiedzieć o akcjach organizowanych wokół Szczytu Milenijnego w~ONZ. Dodajemy 5-10 minut na ten przedmiot pod koniec Nowych Spraw.

\noindent RAPORTY GRUPY ROBOCZEJ

 NAKRĘTKI i~ŚRUBY

\noindent Brooke: Wyjaśnię tylko jedną zmianę w~,,szaleństwie lekarskim'' \ldots 

\noindent \textit{Lesley}: dla tych nowych, Nuts \& Bolts robi wiele\ldots 

\noindent Brooke: \ldots nudnych rzeczy\ldots 

\noindent \textit{Lesley}: \ldots to jednak musi zostać zrobione. Sprawy strukturalne, takie jak kto faktycznie trzyma koszyk ze wszystkimi kartami zgłoszeniowymi, jakie są zasady spisywania ofert. Um, co dalej? Finanse?

 FINANSE

\noindent Rebecca: Kiedy ostatnio sprawdzaliśmy, mieliśmy 91 dolarów w~kapeluszu.

\noindent Jordan: Naprawdę potrzebujemy pomysłów na zbieranie funduszy. Przekazywanie składki raz w~tygodniu nie jest po prostu opłacalnym podejściem na dłuższą metę.

\noindent \textit{Ernest}: Czy jest tu ktoś z~grupy roboczej zajmującej się zbieraniem funduszy?

\noindent \textit{[}Najwyraźniej nie; Zosera się spóźnia, nikt nie jest pewien, czy ktoś jeszcze w~niej jest.]

\noindent JORDAN: W każdym razie ludzie powinni \textit{pomyśleć }o problemie.

\noindent \textit{Ernest}: A tymczasem, zastanów się nad tym, czy ktoś rzeczywiście ma kapelusz, który chciałby przekazać w~tym tygodniu? Przynajmniej chcielibyśmy móc zaoferować Charas coś za pokój. [Ktoś oferuje czapkę z~daszkiem. Zaczyna krążyć po okręgu.]

\noindent \textit{Lesley}: W porządku, Komunikacja? [\textit{Nieobecni}.] Prawnicy?

 I~tak to szło. Po słowach Mariny w~Prawnym i~Ernesta w~kwestiach medialnych (Wolfensohn, szef Banku Światowego, pojawił się na konferencji prasowej w~Amsterdamie, oskarżając Towarzystwo Ruckus o uczenie dzieci robienia koktajli Mołotowa; Ruckus może pozwać, ale nie jest jasne, czy holenderskie prawo na to pozwala.), Brooke przedstawiła aktualne informacje na temat prac nad zasadami CDAN, a Jordan z~DAN Labor ogłosił wiadomość o swoim poparciu dla strajku w~Muzeum Sztuki Nowoczesnej.

\noindent Jordan: To hałaśliwa, naprawdę dzika linia pikiet. Przywoziliśmy lalki, zawsze przywoziliśmy ludzi, odgrywaliśmy naprawdę pozytywną rolę w~radykalizacji ludzi, ale jednocześnie szanowaliśmy ich postawy. Dwa tygodnie temu udało nam się zamknąć księgarnię MoMA na czterdzieści pięć minut kawałkiem teatru partyzanckiego. Nikt nie został aresztowany. W zeszłym tygodniu Andrew zdołał przykleić naklejki strajkowe na pół tuzinie imprezowiczów na przyjęciu dla Davida Rockefellera. (śmiech) Zawsze jesteśmy zastraszani, ale nigdy aresztowani.

 Pamiętajcie ludzie: są tam ludzie, którzy na co dzień walczą z~kapitalizmem i~nazywają się związkami zawodowymi.

\noindent Bob: Pamiętaj, mamy tu kilku nowych ludzi. Może poświęcisz kilka sekund na wyjaśnienie, dlaczego pracownicy MoMA strajkują?

\noindent Jordan: Jasne. W zasadzie są trzy kwestie: brak kontraktu, próby rozbicia związku, są też kwestie zdrowotne i~płacowe. DAN Labor spotyka się w~każdy wtorek w~ABC No Rio, byłym skłocie i~centrum społeczności w~Rivington. Każdy powinien przyjść. (Czy to jest ok?)

 Kontynuujemy przez Policję i~Więzienia, koalicję 1 sierpnia przygotowującą się do protestów RNC, aż w~końcu dotrzemy do tego, o czym wszyscy wiedzą, że będzie prawdziwą kością niezgody -- ,,propozycja facylitacji''. Prezentują ją dwie kobiety z~Klubu Kobiet, siedzące w~północno-zachodnim rogu sali w~pobliżu facylitatorów:

\noindent Marina: Wiele kobiet w~DAN rozmawiało nieformalnie i~było wiele skarg na sposób, w~jaki sprawy miały się w~ciągu ostatnich kilku miesięcy. Były punkty, w~których równowaga płci na spotkaniach wynosiła dwa do jednego, trzy do jednego, a nawet cztery do jednego na korzyść mężczyzn. W ciągu ostatnich kilku tygodni zaczęło się trochę poprawiać, ale wciąż coś jest nie tak. Dlatego starałyśmy się wymyślić kilka pomysłów, jak stworzyć klimat, który kobiety uznają za bardziej zachęcający lub komfortowy.

\noindent Miriam: Jednym ze sposobów, na jaki wpadliśmy, było utworzenie Klubu Kobiet. Chodzi o to, aby była jak najbardziej różnorodna (szczególnie chcemy dotrzeć do osób transpłciowych); i~że może to być przestrzeń, w~której ludzie będą mogli rozmawiać o nowych podejściach do facylitacji, o tym, jak zapewnić większy dialog na spotkaniach na temat płci, rasy i~płci.

 Mamy kilka sugestii:

 Po pierwsze, proponujemy, aby facylitatorzy mieli nawyk do umieszczania osób z~niedostatecznie reprezentowanych grup na szczycie listy.

 Po drugie, chcemy położyć większy nacisk na powitanie i~zachęcenie nowych ludzi.

 Po trzecie, proponujemy, aby DAN pozyskała coś, co jest luźno nazywane ,,obserwatorem wibracji'', kimś, kto może stale monitorować liczby, jak spotkania rozkładają się pod względem rasy i~płci, aby ostrzec ludzi, jeśli liczby spadną i~kto będzie w~stanie używać pewnych narzędzi do interwencji w~przypadku poważnych problemów.

\noindent \textit{Lesley}: Cóż, moje wibracje mówią mi, że ten kąt pokoju (\textit{wskazujący na północ}) mówi znacznie więcej niż którykolwiek z~pozostałych.

 Dobra, może lepiej weźmy tę propozycję kawałek po kawałku. Jaka jest pierwsza część?

\noindent Miriam: Umieszczenie niedostatecznie reprezentowanych grup na szczycie kolejki.

\noindent \textit{Lesley}: Jakaś dyskusja?

\noindent Tim: Wiesz, możesz czasem zrobić to samo, wywołując ludzi, którzy wcześniej nie przemawiali.

\noindent Jakiś mężczyzna: Myślę, że to niewiele pomoże.

\noindent \textit{Lesley}: Więc zgłaszasz poważny sprzeciw?

\noindent Mężczyzna: Jestem po prostu sceptyczny.

\noindent \textit{Lesley}: Jeszcze dyskusja? [Cisza] Jeśli nie, to od razu przejdę do konsensusu. W porządku: jakieś stanięcie z~boku? [Nie]. Jakieś bloki? [Nie].

\noindent Bob: Mam punkt procesu. Czy potrafisz szybko wyjaśnić te terminy, stanie z~boku, bloki, osobom, które mogą ich nie rozumieć? W sali jest dużo nowych osób.

\noindent \textit{Lesley}: O tak, dobry pomysł. [\textit{Robi}] W porządku, to było łatwe. Jakieś dyskusje na temat numeru 2?

\noindent David: Naprawdę uważam, że posiadanie osoby witającej byłoby niezwykle ważne. Nie wiesz, jak bardzo wyobcowane może być pokazanie się na jednym z~tych spotkań. Powiedziałbym, że większość ludzi, którzy pojawiają się z~ciekawości, nie wraca, ponieważ, o ile już kogoś nie znają, prawie nie ma szans na spotkanie lub rozmowę z~osobą z~DAN.

\noindent Marina: Inną rzeczą, o której myśleliśmy, jest rozdawanie literatury orientacyjnej.

\noindent \textit{Ernest}: Ten problem brzmi jak coś, co byłoby dobrze przedyskutować na listserv.

\noindent \textit{Lesley}: Właściwie powiedziałbym, że wszystko, co ma związek z~kwestiami płci, to coś, o czym \textit{nie powinniśmy }dyskutować na listserv.

\noindent Brooke: Wiesz, wydaje mi się, że cała ta propozycja powinna dotyczyć stworzenia nowej grupy roboczej.

\noindent Miriam: Myślałyśmy o tym bardziej jako o klubie. W takim przypadku naprawdę nie musimy mieć zgody grupy na jej utworzenie.

\noindent \textit{Lesley}: Czy porozmawiamy o tym problemie po tym, jak zakończymy sprawy bieżące?

\noindent Brooke: No wiesz, jeśli spotykasz się co tydzień, zasadniczo mówisz o grupie roboczej, jakkolwiek to nazwiesz.

\noindent Miriam: Nie. To klub. Do tej pory spotykaliśmy się głównie nieformalnie.

\noindent Brooke: No dobra, chyba to nie ma znaczenia. Chociaż, jeśli zamierzasz prowadzić szkolenie dla facylitatorów, jest to technicznie domena Nakrętki i~Śruby. Może powinniśmy razem pracować.

\noindent Mac: Naprawdę nie uważam, że to dobry pomysł, abyśmy ustanawiali precedens, z~którym wszyscy powinni się zgodzić, zanim kobiety (lub na przykład queer) będą mogły tworzyć kluby.

\noindent Sam: Przez jakiś czas trzymałem rękę w~górze.

\noindent \textit{Lesley}: Dobra.

\noindent Sam: Myślę, że dla nowoprzybyłych byłoby pomocne zdefiniowanie różnicy między klubem a grupą roboczą.

\noindent \textit{Lesley}: Cóż, nie jest to do końca jasne. Kluby to nowy pomysł. Nigdy nie dyskutowaliśmy wprost, czym byłby klub. Ale myślę, że byłaby to grupa ludzi, którzy czują jakąś pokrewieństwo, którzy chcą się spotkać, aby omówić swoje problemy i~sympatie. To prawie wszystko.

\noindent \textit{Ernest}: [\textit{odpowiadając na sygnał od Tima}] Pięć minut, które przeznaczyliśmy na dyskusję, dobiegło końca. Myślę, że powinniśmy przejść do Nowych Spraw.

\noindent Miriam: Rozumiem, ale myślę, że ta dyskusja jest ważna.

\noindent Marina: Rozmawiamy o tym, ponieważ coraz więcej kobiet przestało przychodzić na spotkania. Powiedziałabym więc, że jest to bardzo ważne dla całej grupy.

\noindent \textit{Ernest:}Wydłużymy więc czas o pięć minut ?

\noindent \textit{[}\textit{Różne osoby migoczą]}

\noindent \textit{Lesley}: Cóż, technicznie wciąż jesteśmy przy drugim punkcie, witamy nowych ludzi. Nie słyszałem żadnych poważnych obaw co do tej części. Może po prostu przez to przejdziemy, aby przejść do punktu 3. Jakieś odstąpienie na bok co do osób witających? Jakieś bloki? [\textit{Nie}].


 Dobra, mamy konsensus. Miriam, może powinnaś powtórzyć propozycję dotyczącą obserwatora wibracji.

\noindent Miriam: Jasne. Zwykle rolą obserwatora wibracji jest monitorowanie emocjonalnej dynamiki spotkania, jeśli ludzie się nudzą lub są drażliwi, lub jeśli ktoś czuje się wyobcowany, zwracają na to uwagę i~są różne narzędzia, których mogą użyć, aby interweniować, począwszy od otwierania okien, aby uzyskać więcej powietrza w~pomieszczeniu lub proszenia ludzi, aby mówili głośniej, aby ktoś z~aparatem słuchowym mógł wziąć udział, aż do, w~skrajnym przypadku, przerwanie dyskusji na trzydzieści sekund lub minutę ciszy na uspokojenie. Jeśli w~grupie jest ktoś, kto zachowuje się systematycznie destrukcyjnie, może spróbować porozmawiać z~tą osobą, zwrócić jej uwagę na skutki swojego zachowania. Lub nawet, jeśli nic innego nie działa, zaproponować limit czasu specjalnie dla nich. Mamy poczucie, że trening obserwatora wibracji, aby zwracał uwagę na dynamikę rasową i~gender, może pomóc stworzyć atmosferę, która nie odrzucałaby tak wielu kobiet i~kolorowych osób.

\noindent \textit{Lesley}: Pytania wyjaśniające? [\textit{Nie}]. Obawy?

\noindent Tim: Ja wolałbym nie mieć określonej osoby na tej funkcji, bo cóż\ldots  każdy ma swoje problemy i~priorytety. Osoba heteroseksualna może nie być w~stanie wykryć wibracji homofobicznej. Albo ktoś taki jak ja, który ma trzydzieści osiem lat, może nie być najlepszą osobą do postrzegania ageizmu. Czy nie powinniśmy wszyscy monitorować tego typu rzeczy? i~wszyscy mają prawo interweniować, jeśli trzeba coś wskazać grupie?

\noindent Miriam: Jasne, teoretycznie tak to \textit{powinno }działać. Problem w~tym, że ludzie tego nie robią. Dlatego w~pierwszej kolejności musiałyśmy opracować tę propozycję.

\noindent Sara: Myślę, że to niezwykle ważne. Kilka tygodni temu niektórzy z~nas próbowali wskazać przykład rażącego seksistowskiego zachowania na spotkaniu, faceci rozmawiający ponad kobietami, wcinający się. Ale kiedy próbowałyśmy zwrócić uwagę na to, że wchodzono w~głos kobietom, w~sali pojawił się ogromny nacisk, aby rozmowa z~powrotem skoncentrowała się na kwestiach klasowych. Jakby pochodzenie z~klasy robotniczej jakoś usprawiedliwiało takie zachowanie. Zgadzam się, że nie powinniśmy rozwodzić się nad tego rodzaju sprawami, że zatrzymujemy naszą prawdziwą pracę, ale chcę mieć kogoś na czele sali, o którym wiem, że jest orędownikiem takich ludzi jak ja.

\noindent \textit{Lesley}: Widzę jeszcze trzy ręce i~prawie kończy nam się czas. Och, cztery.

\noindent Stuart: Mam pomysł na konkretną propozycję lub mam nadzieję, że może to być przyjazna poprawka. Dlaczego nie spróbujemy użyć obserwatora wibracji do, powiedzmy, czterech spotkań? A potem możemy zarezerwować trochę czasu na ocenę, czy to działa, czy powinno być kontynuowane, jak można to poprawić.


 W tym momencie Dennis wstał. Powinienem wyjaśnić coś na temat Dennisa. Dennis był człowiekiem, który wyglądał trochę i~brzmiał prawie dokładnie, jak Robert DeNiro. Tyle że był znacznie większy. Miał około czterdziestu pięciu lat, krępy, przypominający konduktora w~autobusie lub metrze, którym, jak głosiły plotki, był, zanim kilka lat wcześniej doznał niepełnosprawności umysłowej. Dennis miał bardzo donośny głos, swój osobisty megafon i~skłonność do skrajnej konfrontacji (choć nigdy do końca brutalnej) podczas dem. Wydawało się również, że jest całkowicie nieświadomy zasad konsensusu i~prawie na pewno był mężczyzną, do którego Sara odnosiła się w~swoich narzekaniach o wymyślaniu wymówek dla rzekomego zachowania klasy robotniczej.

\noindent Dennis: Ten problem jest z~mojego punktu widzenia wynikiem tego, że jeśli tak się dzieje, myślę, że to dlatego, że facylitator nie wykonał swojej pracy. Cała ta koncepcja wibracji brzmi dla mnie jak szalona. Jeśli sprawisz, że subiektywny punkt widzenia jednej osoby będzie obowiązywał wszystkich innych, cóż, czy nie jest to definicja opresji? To, co naprawdę musimy zrobić, to zastosować \textit{Reguły Porządkowe Roberta}, które zapewnią pewien poziom organizacji spotkania.

 To był typowy Dennis. Miał tendencję do zaczynania od tego, co mogło wydawać się całkowicie rozsądnym punktem, a potem odlatywał w~całkowity brak pojęcia. Ludzie starali się zachować twarz pokerzysty, dopóki nie skończy swoich rozważań na temat zalet \textit{Reguł Roberta}.

 \noindent Cindy: Zgadzam się, że każdy jest odpowiedzialny za monitorowanie tych rzeczy. Technicznie rzecz biorąc, powinno być możliwe oferowanie punktów informacji lub punktów procesu bez konieczności czekania przez listę, i~chociaż DAN nie robi tego zbyt wiele, może powinniśmy opracować to jako sposób radzenia sobie z~tego rodzaju problemami, jak dobrze. Na przykład opracować konkretny sygnał ręczny.

 \noindent  Mike: To, czego naprawdę potrzebujemy, to ciągła dyskusja. Omawialiśmy tutaj rozwiązania, nigdy nie omawiając samego problemu. Tym, co sprawia, że seksizm lub rasizm są tak podstępne, jest to, że często mogą być niezwykle subtelne. Mówimy o formach władzy i~ucisku tak głęboko zinternalizowanych, że często pozostają w~tle, informując o tym, co mówimy i~robimy, w~sposób, którego nigdy nie bylibyśmy w~stanie wykryć. Czy naprawdę moglibyśmy wyszkolić ludzi, aby niezawodnie wyłapywali takie rzeczy? Wydaje mi się, że musimy znaleźć sposób, aby dotrzeć do źródła problemu, nie żebym był pewien, co by to było, ale po prostu nie jestem pewien, czy te narzędzia naprawdę byłyby adekwatne do zadania.

\noindent Miriam: Ale to tylko pomysł, aby móc ostrzec ludzi o subtelnych formach rasistowskich lub seksistowskich zachowań, których mogą nie być świadomi. Jeśli masz lepszy sposób, chciałbym go usłyszeć. Aha, i~w odpowiedzi na punkt Stuarta: Jasne. Nie mam nic przeciwko pomysłowi trzy- lub czterotygodniowego okresu testowego.

\noindent \textit{Lesley}: Widzę jeszcze dwie ręce.

\noindent Max: Po co cała dyskusja? Dlaczego po prostu nie zaproponować, że można je anulować, jeśli to nie zadziała?

\noindent \textit{Ernest}: Cóż, w~ten sposób możemy uzyskać konsensus, ponieważ słyszeliśmy kilka obaw.

\noindent Brooke: Zdecydowanie powinniśmy zorganizować spotkanie, aby wyszkolić potencjalnych obserwatorów wibracji. W ten sposób możemy stworzyć pulę ludzi, którzy mają jakiś pomysł, jak to zrobić. Ponieważ bezlitosne piętnowanie zachowań innych może być naprawdę destrukcyjne.

\noindent Rachel: Dla mnie jednak to nie ma sensu, żeby treningi przeprowadzał sam Klub Kobiet. Powiedziałabym, że powinniśmy zarezerwować czas na walne zgromadzenie z~myślą, że doprowadzi to do szkolenia, które możemy przeprowadzić w~przyszłym tygodniu.

\noindent \textit{Lesley}: Wydaje się, że przechodzimy tutaj do innej propozycji. Ograniczmy się tylko do stworzenia obserwatora wibracji, a o treningach porozmawiajmy później. Ponieważ jesteśmy w~niedoczasie, zamierzam przejść do konsensusu. Jakieś stanięcie z~boku?

\noindent Mike: Tak. Staję z~boku. To znaczy, nie sprzeciwiam się pomysłowi nakłaniania inteligentnych ludzi do zwracania uwagi na tego typu rzeczy, ale wydaje się dziwne, by wyznaczać jedną osobę do monitorowania wszystkich innych.

\noindent Sam: Inteligencja nie ma z~tym nic wspólnego.

\noindent \textit{Lesley}: Po prostu sprawdzam, czy są stania z~boku. Nie dyskusja.

\noindent George: Stoję z~boku. Wydaje mi się, że to za bardzo przypomina policję myśli. Ale jestem gotów poczekać i~zobaczyć, jak to działa.

\noindent \textit{Lesley}: Dobra, jakieś bloki?

\noindent Mike: Może mógłbym zasugerować poprawkę, że wypracujemy konkretne wytyczne, żeby było jasne, \textit{jak }to wszystko będzie działać? Myślę, że to pomogłoby uspokoić umysły wielu ludzi.

\noindent \textit{Lesley}: Brzmi to pomocne, ale w~tej chwili staramy się tylko przejść przez tę propozycję. Jakieś bloki?


 I~faktycznie były. Trzy. Jeden był od jednego ze związkowców, Nathana w~czapce UNITE. Jeden był od Dennisa. W tym momencie wszystko zaczęło się szybko psuć.

\noindent \textit{Lesley}: Więc ta propozycja jest\ldots 

\noindent Dennis: Chciałbym stworzyć coś, co moim zdaniem byłoby konstruktywne\ldots 

\noindent Amy: Ehm. To klasyczne seksistowskie zachowanie, przerywanie kobietom w~środku zdania. Nie mówiąc już o facylitatorze.

\noindent \textit{Ernest}: Mamy dwadzieścia pięć minut na dyskusję, która miała zająć pięć. Czy możemy to przełożyć na przyszły tydzień?

\noindent Stuart: Albo kontynuować na listserv?

\noindent Amy: W przyszłym tygodniu nie spotykamy się z~powodu parady Gay Pride. Poza tym w~Klubie Kobiet panuje bardzo silne przekonanie, by \textit{nie }przenosić tego po prostu na listserv. Po pierwsze, wciąż jest kilka z~nas zacofanych dusz, których nie ma na e-mailach. Po drugie, jeśli mamy spotkania, na których niektóre kobiety nie czują się komfortowo, ostatnią rzeczą, jaką chcemy zrobić, jest przeniesienie dyskusji na e-mail, który jako środowisko jest milion razy gorszy.

 Dennis miał rękę w~górze, odkąd został wycięty, machając nią gorączkowo. Ernest chciał przejść do następnego punktu porządku obrad, dyskusji na temat wspierania nieposłuszeństwa obywatelskiego w~Szkole Ameryk. Inni nalegali, abyśmy kontynuowali dyskusję na temat obserwatora wibracji. Kilka osób zgłosiło się na ochotnika do usunięcia własnych punktów z~programu na ten tydzień, aby znaleźć czas. Ktoś zwrócił uwagę, że Zosera, jedną z~nielicznych stałych uczestniczek DAN, która jest Afroamerykanką, która przyszła na samym początku dyskusji, teraz również podniosła rękę.

\noindent Zosera: Wiecie, nie mogę nie zauważyć, ale ci, którzy najgłośniej sprzeciwiali się temu pomysłowi, cóż, nie sądzę, że to przypadek, że wszyscy byli białymi mężczyznami. Może nie widzą sensu, ponieważ \textit{oni} rzadko czują się zmarginalizowani; \textit{oni }zawsze czują się upoważnieni do mówienia. Ale kiedy zaczynasz sprzeciwiać się tej idei w~imię demokracji, mówiąc o ,,ucisku'', naprawdę muszę zacząć się zastanawiać, na jakiej jesteście planecie. Dla mnie demokracja to parytet partycypacyjny. Kiedy cała kategoria ludzi zostaje zmarginalizowana i~nie jest w~stanie uczestniczyć na równych zasadach, to \textit{jest }ucisk. Nie jakiś facet, który musi się martwić, że choć raz w~życiu, może zostać wywołany za swoje zachowanie. Miarą naszego sukcesu jest rodzaj klimatu, który tworzymy i~jeśli DAN tworzy klimat, który zaprzecza równorzędności, wtedy sama DAN staje się formą ucisku. Chcecie stworzyć rasistowską organizację? Dobra, śmiało. Chcecie stworzyć seksistowską organizację? Dobra, śmiało. Ale przynajmniej nie mówcie, że robicie to w~imię demokracji!

\noindent \textit{Lesley}: Miriam, ktoś mi mówi, że masz inną propozycję?

\noindent Miriam: Tak. Proponuję, abyśmy przedyskutowali to przez dwa tygodnie na listserv, a potem omówili to ponownie na następnym spotkaniu, kiedy wrócimy po Gay Pride.

\noindent Rachel: Chciałabym zaproponować poprawkę: ta dyskusja zaczyna się od potrzeby szkolenia na temat budowania społeczności w~DAN.

\noindent Miriam: Dobrze. Przyjmę to jako przyjazną poprawkę.

\noindent \textit{Ernest}: Jakieś stanie z~boku? Nie? Jakieś bloki?

\noindent Tim: Blokuję.

\noindent Miriam: \textit{Blokujesz}?

\noindent Tim: Blokuję tylko część dotyczącą listserv. Ponieważ, jak zaznacza Amy, nie wszyscy się tym zajmują i~nie jest to idealne forum do omawiania takich spraw. Oczywiście nie blokowałbym ponownego podjęcia dyskusji.

\noindent \textit{Ernest}: Widzę tu drugi blok. Dennis?

\noindent Dennis: Blokuję w~odniesieniu do tego, co pamiętam, gdy usłyszałem o systemie wierzeń DAN, co sprawia, że  wierzę, że ta propozycja jest sprzeczna z~tymi wartościami. Zgadzam się ze wszystkim, co ta pani powiedziała o ucisku, ale skoro już mamy do tego infrastrukturę, ten środek jest niepotrzebny. To powinno być zadanie facylitatora. Co więcej, ta grupa potrzebujesz otwartej dyskusji bez cenzury.

\noindent \textit{Lesley}: Więc mówisz, że nie możemy\ldots 

\noindent Rebecca: Punkt procesu!

\noindent \textit{Lesley}: Tak! Proszę! Pomóż mi tutaj!



 Nigdy nie widziałem spotkania z~tyloma blokami. Pamiętaj, że ogólne zrozumienie jest takie, że jeśli jest blokada, nawet jeśli jest duża liczba odstających, prawdopodobnie nastąpiła luka w~procesie. Rebecca próbowała przyjść na ratunek, z~sugestią, której Lesley prawdopodobnie nie pochwaliłaby innego dnia.

\noindent Rebecca: Wiesz, w~moim rozumieniu \textit{mamy }opcję zmodyfikowanego konsensusu. Jeśli po raz pierwszy spróbujesz zająć się obawami blokera i~jeśli okaże się to całkowicie niemożliwe, możesz przejść do głosowania większością dwóch trzecich.

\noindent \textit{Ernest}: Ważny punkt! To już drugi raz, kiedy nie udało nam się dojść do konsensusu w~tej sprawie. Sądzę więc, że mamy tutaj historyczny moment: po raz pierwszy DAN faktycznie przeszła do zmodyfikowanego konsensusu\ldots 

\noindent Miriam: Czekajcie! Ale mam propozycję, która rozwiąże te obawy. Przynajmniej mam nadzieję, że tak. Sądząc po tonie spotkania, proponuję omówić to nieformalnie, nie na liście, w~ciągu najbliższych dwóch tygodni i~odbyć szkolenie, a następnie na kolejnym spotkaniu ponownie podjąć dyskusję.

\noindent \textit{Ernest}: W porządku. Mamy propozycję. Stanie z~boku?

\noindent Dan: Czy są inne możliwości?

\noindent \textit{Lesley}: Dojdziemy do tego, jeśli ten nie przejdzie.

\noindent \textit{Ernest}: Nie widzę żadnego stania z~boku. Jakieś bloki?

\noindent Dennis: Czy mogę prosić o wyjaśnienie, więcej szczegółów na temat propozycji?

\noindent \textit{[}Ernest ponownie przedstawia propozycję ]

\noindent Dennis: Nie. Nadal blokuję.

 Teraz przyszła kolej na Dana, młodego człowieka pochodzenia południowoazjatyckiego, który od czasu do czasu pracował w~zespole prawników (siedział w~tym momencie na ziemi, nieco nietypowo w~trzyczęściowym garniturze, właśnie wrócił z~działania jako prawny obserwator podczas demonstracji na Brooklynie), aby spróbować uratować sytuację.

\noindent Dan: Dobra, mam alternatywną propozycję. Ponieważ spędziliśmy już tak dużo czasu nad tą kwestią, powiedziałbym, że zamiast czekać dwa tygodnie, porozmawiajmy teraz i~podejmijmy ostateczną decyzję. Ponieważ ludzie są teraz zdecydowanie na krawędzi. Problem nadal jest bardzo intuicyjny. Zamiast być powodem do wycofania się, może to oznacza, że \textit{powinniśmy }o tym teraz porozmawiać.

\noindent Bob: Punkt procesu. Rebecca właśnie przypomniała nam, że to nie koniec sprawy, jeśli \textit{jedna }osoba blokuje. Czy przechodzimy więc do zmodyfikowanego konsensusu, czy nie?

\noindent \textit{Ernest}: Tak. Mam wrażenie, że w~grupie panuje pewna ambiwalencja co do idei zmodyfikowanego konsensusu. Czuję pewien opór. Może powinniśmy przeprowadzić szybką sondę, aby zorientować się w~pokoju, jeśli chodzi o użycie przez DAN\ldots 

\noindent Rachel i~Miriam: Nie! Nie!

\noindent Marina: Zresztą, o czym tu mówimy? Jak możesz zablokować propozycję powrotu i~przedyskutowania czegoś? To tak, jakby ktoś powiedział ,,Chcę omówić sprawy pracownicze w~przyszłym tygodniu'', a potem ktoś to zablokował.

\noindent \textit{Lesley}: Więc czy robimy sondaż, żeby zobaczyć, czy chcemy przejść do zmodyfikowanego konsensusu, czy nie?

\noindent Miriam: Nie! My nie musimy. 

\noindent Brooke: Wiesz, jeśli chodzi o propozycję Dana, w~programie jest miejsce przeznaczone na edukację. Rzadko z~niej korzystamy, ale dlaczego nie przeprowadzimy tam po prostu dyskusji o szkoleniach?

\noindent \textit{Ernest}: Przejdźmy do dwóch trzecich pierwotnej propozycji, zanim zrobimy tę. Propozycja jest więc taka: opóźnić dyskusję, porozmawiać o tym nieformalnie, a potem wrócić do niej za dwa tygodnie. Jakieś stanięcie z~boku?

\noindent Marina: Nie! Myślę, że staraliśmy się pokazać, że nie ma potrzeby przedstawiania tego jako formalnej propozycji. Nie musimy iść na dwie trzecie. W ogóle nie musimy tego rozważać. Nie ma tam nic, co wymagałoby aprobaty całej grupy.

\noindent Cindy: Myślę, że Marina próbuje powiedzieć, że jeśli jest blokada, nie musisz natychmiast wracać do tyranii większości. Zwykle albo składasz propozycję, albo próbujesz ją zmodyfikować. To tylko przedstawienie propozycji. Nie musisz się zastanawiać, czy ją wystawić, czy nie!

\noindent Sara: Jestem \textit{tak }zdezorientowana. Więc\ldots  ok, zgadzamy się, że nie musimy głosować. Powiedziałbym, nie umieszczajmy przedmiotu w~slocie edukacyjnym, ponieważ ta rozmowa musi dotyczyć czegoś więcej niż tylko szkolenia obserwatora wibracji. Chodzi o podstawową dynamikę tej grupy. Ale jestem w~punkcie zadawania sobie pytania: jak możemy głosować na to, jeśli świadomość niektórych ludzi nie została rozwinięta na tyle, aby móc na to głosować.

\noindent Wiele na raz: Whoa!



 W rzeczywistości było to wyzwanie dla samej podstawy konsensusu -- stąd zszokowana reakcja. To był również trochę zły znak, że Sara, wciąż nieco nowa w~grupie, powraca do terminów takich jak ,,głosowanie'' -- No cóż, może po prostu nie znam dość dobrze pewnych spraw -- skrzywiła się i~zamilkła. Gdyby była obserwatorka wibracji, prawie na pewno wezwałaby na przerwę, ponieważ napięcie w~pokoju było nadzwyczajne. i~to pomimo faktu, że nikt, nawet Dennis, nie podniósł głosu. Zamiast tego konflikt wydawał się sublimować w~rodzaj swoistego legalizmu, który zwykle utożsamia się z~manewrami parlamentarnymi, ale który zwykle prawie nigdy nie występował w~grupach opartych na konsensusie.

\noindent \textit{Ernest}: Widzę pięć osób z~podniesionymi rękami. Zajmę się tymi pięcioma, a następnie zamknę dyskusję.

\noindent Jody [\textit{nowa osoba, dziewczyna punk, przyjaciółka Chrisa}]: Mam pomysł. Co powiesz na to? Dlaczego nie wyznaczyć określonej godziny na każde spotkanie, na którym wszyscy będą mogli porozmawiać o wibracjach i~zasugerować, co można z~tym zrobić. W ten sposób możemy zarówno upewnić się, że tak się stanie, jak i~zapewnić szeroki rozkład odpowiedzialności.

\noindent \textit{[}kilka migotania dłonia. Nie jest to jednak podejmowane.]

\noindent Melissa: Chciałabym, żeby Klub Kobiet mógł się spotkać przed następnym spotkaniem DAN za dwa tygodnie i~żeby czas i~miejsce były dobrze nagłośnione. Powinniśmy również przeznaczyć dodatkowy czas na nadchodzące spotkanie, aby nikt nie próbował skracać dyskusji.

\noindent Dennis: Mówię to w~najlepszych intencjach i~proszę, żeby nikt nie czuł się lekceważony, ale z~wiekiem zauważam, że ludzie często oskarżają innych o to, co sami robią. Zauważyłem to z~oskarżeniami o homofobię, co, przyznaję, jest bardzo poważnym problemem. Nie chcę być niewrażliwy na potrzeby innych, ale myślę, że tutaj musimy działać z~większą pokorą.

\noindent Miriam: Cóż, chcę tylko powiedzieć, że osobiście ta rozmowa była \textit{bardzo }irytująca.

\noindent \textit{[}Wiele migotania]

\noindent \textit{Ernest}: W porządku. Przeznaczymy czas na szkolenie na temat różnorodności i~następnym razem przywrócimy to jako dyskusję w~Bieżących Sprawach.

\noindent Rachel: Już to wszystko skoordynowaliśmy. Lesley i~ja zaproponowaliśmy to szkolenie już dwa tygodnie temu.

\noindent \textit{Ernest}: Czy możemy więc przejść do propozycji SOA?

\noindent Wiele: Tak! Tak! Proszę!


 Propozycję SOA przedstawiła stosunkowo nowa uczestniczka nazwiskiem Rebecca, sympatyczna dwudziestolatka o krępym wyglądzie, która była między innymi stolarzem związkowcem na Brooklynie. Była także (póżniej odkryliśmy) doskonałą facylitatorką. Rebecca natychmiast zaproponowała kilkuminutową przerwę na rozciąganie i~regenerację, co zostało przyjęte wieloma okrzykami entuzjazmu. Kiedy zebraliśmy się ponownie, wyjaśniła tło protestacyjnej koalicji School of the Americas Watch, która od prawie dziesięciu lat wydaje coroczne płyty CD w~Fort Benning. Koalicja została założona przez katolickiego księdza i~obejmowała wiele grup wyznaniowych, była nieco odgórna w~organizacji i~staromodna w~ich idei niestosowania przemocy, ale chętnie się od nas uczyli. Zgodziliśmy się utworzyć grupę, aby zobaczyć, jak moglibyśmy się komunikować. Następnie, Bob zaproponował szkolenie w~Więziennym Kompleksie Przemysłowym, aby zbiegło się z~naszymi zbliżającymi się akcjami na konwencji republikanów. Jason z~RTS poprosił nas o pomoc w~przygotowaniu dużego świątecznego demo, aby nagłośnić sytuację w~Charas. Osoba z~ISO zachęciła nas do wzięcia udziału w~dużym legalnym wiecu zaplanowanym na pierwszy dzień konwencji republikanów w~Filadelfii, zwanym Unity 2000, który został zaaprobowany przez główne związki zawodowe w~mieście. Następnie przeszliśmy do Nowych Spraw. Mac i~Lesley zrelacjonowali swoje doświadczenia z~ostatnich akcji w~Windsor; Sara opisała swoje plany stworzenia zespołów ,,Video I-Witness'' do monitorowania policji podczas konwencji; i~wreszcie Jack Griffin mógł przemówić do grupy. Griffin, dwudziestoparoletni mężczyzna w~związkowej marynarce i~robotniczym kapeluszu z~lat 30., był krajowym koordynatorem kampanii pracowników pralni UNITE. Wyglądał na wyraźnie zirytowanego:

\noindent Griffin: Pierwotnie zamierzałem przedstawić tej grupie dwie propozycje, ale ze względu na charakter tej \textit{jebanej }dyskusji, która odbyła się wcześniej, przedstawię tylko jedną.

\noindent Różni członkowie DAN: Hej!

\noindent Dan: Um, przepraszam za wbicie się tutaj, ale to jest \textit{właśnie }ten rodzaj problemu, o którym rozmawialiśmy. Ilekroć pojawiają się problemy kobiet, ktoś oburza się na to, że w~ogóle o nich rozmawiamy. A to tworzy represyjną atmosferę.

\noindent Griffin: Dobra, doceniam to. Nazywanie tego ,,\textit{jebaną }dyskusją'' było niewłaściwe.

 Jestem tu, by przemawiać w~imieniu czterech tysięcy związkowców, którzy pracują dla dwudziestu siedmiu różnych firm w~mieście i~na Long Island. Większość z~nich to imigranci. To są ludzie, którzy przez cały pieprzony dzień muszą radzić sobie z~gównem i~krwią. [\textit{pauza, potem delikatniej}] Nie wiem, powiem wam: wrócę tu z~prezydentem lokalnej komórki, który może wam opowiedzieć o szeregu działań, co do których mamy nadzieję ostatecznie doprowadzić do strajku generalnego wśród pracowników pralni w~listopadzie. Pracujemy nad tym z~Komórką 6 i~Komórką 100 i~planujemy przeprowadzić naszą pierwszą akcję za tydzień od jutra.

 Może nasze członkostwo powinno spotkać się z~wami wszystkimi, ponieważ być może istnieje pewna rozbieżność w~waszych doświadczeniach. Myślę, że większość dyskusji, które słyszę tego popołudnia, jest bardzo klasyczna; Nie rozumiem połowy słów, których używacie!

\noindent \textit{[}Inny facet ze związków zawodowych, Nathan z~komórki nr 1199, wybucha oklaskami. Wszyscy pozostali milczą.]

 Mimo to mają zbyt wiele do stracenia, jeśli pójdą na to sami. Kiedy wyjdą, nie będzie więcej wypłat i~na pewno nie będzie żadnych funduszy powierniczych, na których można by się oprzeć. Więc będą potrzebować wszelkiej pomocy, jaką możemy uzyskać.

\noindent Malcolm: Mówiłeś, że będzie akcja od jutra za tydzień?

\noindent Griffin: Tak, celem będzie jeszcze nieokreślony hotel na Broadwayu.

\noindent \textit{Ernest}: Czy mógłbyś opowiedzieć niektórym z~nas później szczegóły?

\noindent Griffin: Tak, tak, mamy plan. To będzie akcja bezpośrednia, która odbędzie się w~niezrzeszonym hotelu. Ale tak naprawdę nie mogę tutaj o tym mówić. Wrócę, dobrze?

 Stwierdzenie, że ktoś wyszedł, tupiąc, z~pokoju, to banał, ale w~tym przypadku wydawało się to niemal dosłownie trafne. Ernest zapytał: -- Um, czy ludzie mogą upewnić się, że on będzie tam na następnym spotkaniu DAN Labor w~ABC?

 W rzeczywistości nigdy się nie pojawił, ale na tym spotkaniu zdecydowaliśmy, że nigdy więcej nie będziemy przyprowadzać ludzi związkowych bezpośrednio do Generalnego DAN, ale zawsze zamiast tego do grupy roboczej Labor.

 Po krótkiej prezentacji Any z~IMC na temat próby pozyskania pomocy DAN w~nadchodzących protestach na Szczycie Milenijnym ONZ przerwaliśmy na noc. W drodze do domu mój przyjaciel Stuart powiedział do mnie: -- Wiesz, kiedy rzeczywiście zaczynasz pisać swoją etnografię, możesz wykorzystać to spotkanie jako studium przypadku. Wydaje się, że wszystkie główne linie błędów DAN zostały właśnie ujawnione. -- Z pewnością stanowiło piękny przykład rozdarcia obaw o płeć i~klasę społeczną.

 Wiele kłopotów wywołał Dennis. To z~pewnością nie wyszło od niego. Ale stał się symbolem. Z perspektywy większości członków DAN Dennis był ucieleśnieniem tego, co w~języku aktywistów nazywano ,,czubek'' -- rodzaju osoby, która opiera swój polityczny światopogląd na przekonaniu, że papieża kontrolują kosmici, lub że Kampania oprysku przeciw komarom prowadzona przez Departament Parków jest tak naprawdę przykrywką dla eksperymentów CIA z~bronią genetyczną. Dennis uważał, że Stany Zjednoczone są pod okupacją wojskową kilku dywizji tajnej armii ONZ pod dowództwem Michaiła Gorbaczowa: klasyczna wersja pewnego rodzaju teorii ,,czarnego helikoptera'' popularnej wśród tych samych małomiasteczkowych, białych kręgów robotniczych, które zapewniają większość rekrutów dla prawicowych milicji. Jak później zauważyła Rebecca: -- Są sposoby, w~których skrajna prawica i~skrajna lewica są zaskakująco podobne. A wszystko, czego potrzeba, aby przekroczyć tę małą granicę, która ich dzieli, to być totalnym pieprzonym wariatem.

 Problem z~Dennisem polegał jednak na tym, że po prostu nie był wystarczająco szalony. Gdyby był oczywistym wariatem, łatwo byłoby znaleźć wymówkę, by go wykluczyć. Ale w~rzeczywistości rzadko poruszał na spotkaniach swoje bardziej paranoidalne teorie. Jeden z~nielicznych przypadków, kiedy to zrobił, miał miejsce po prezentacji Any na temat Szczytu Milenijnego, podczas tego samego spotkania, kiedy wyjaśnił, że na omawianym szczycie -- spotkaniu światowych przywódców w~ONZ -- zostanie po raz pierwszy zapowiedziana Religia Świata. Dennis był, jak widzieliśmy, w~stanie przynajmniej deklarować gołosłowne deklaracje ideałów DAN, chociaż wydawał się głęboko zdezorientowany jej zasadami organizacyjnymi: ciągle blokował i~nikt nie wydawał się w~stanie wyjaśnić mu, dlaczego blok nie jest tak samo jak głosowanie na ,,nie''. Miał przyjaciół wśród hardcorowej społeczności dzikich lokatorów, wśród których było kilku najdłużej działających anarchistycznych aktywistów w~Nowym Jorku. Tam wielu szanowało go za jego poświęcenie, gotowość do podejmowania ryzyka dla sprawy, jego lojalność wobec przyjaciół. Squattersi byli przyzwyczajeni do regularnych kontaktów z~narkomanami i~uchodźcami ze schronisk dla bezdomnych; według ich standardów, Dennis nie był aż tak ekstrawagancki.

 Na szkoleniach facylitacyjnych często mówi się o ,,problemie czubka''. Co zrobić, gdy na otwartym spotkaniu przywędruje ktoś, kto jest destrukcyjnie szalony. Normalnie próbuje się taką osobę uspokoić. W końcu tak się stało. Jednak tym, co naprawdę oburzyło większość kobiet DAN (i to wyszło później w~rozmowach), była idea, że \textit{każdy }mężczyzna DAN stanąłby po stronie takiej osoby -- w~tym przypadku niestabilnego psychicznie, zagrażającego fizycznie prawicowego teoretyka spiskowego -- przeciwko nim. Bo z~pewnością niektórzy tak robili. Podczas spotkania garstka zgromadzonych na sali związkowców (Nathan z~UNITE, Griffin i~kobieta, z~którą wszedł) często kiwali głową lub wyrażali wyraźną aprobatę po jego wypowiedziach: po części to jasne, bo utożsamiali się z~jego proletariackim sposobem, głównie podejrzewam, ponieważ postrzegały ideę ,,obserwatora wibracji'' jako sposób, aby feministki z~wyższej klasy średniej mogły zamknąć takie osoby. To prawda, że  żaden z~tych ludzi nie był anarchistą ani nawet stałym bywalcem DAN\footnote{Większość faktycznie wydawała się przybyć jako członkowie lub sojusznicy ISO.}. Ale jeszcze bardziej irytująca była pozorna bierna przyzwolenie tak wielu zwykłych ludzi DAN. Wiele kobiet zauważyło, że chociaż tylko kilku mężczyzn wypowiedziało się przeciwko propozycji na tym spotkaniu -- i~miało to być kontynuowane na późniejszych -- jeszcze mniej mężczyzn opowiedziało się za nią. Nawet męski moderator, Ernest, wydawał się od samego początku dziwnie chętny do przejścia do następnego punktu porządku obrad\footnote{Chociaż sprawy stały się jeszcze bardziej skomplikowane, gdy pojawił się temat zmodyfikowanego konsensusu, ponieważ Lesley faktycznie była wobec niego podejrzliwa, jak widzieliśmy, a Ernest nie miał problemu z~tym pomysłem.}. W rezultacie Dennis zaczął wydawać się de facto rzecznikiem niewypowiedzianych uczuć reszty grupy; w~efekcie coś w~rodzaju odsublimowanej twarzy męskości DAN -- co, jak podejrzewało wiele kobiet, naprawdę kryło się za przyjemną fasadą.

 Podobny wzorzec miał się powtórzyć na kolejnych spotkaniach, gdzie Dennis znalazł silne wsparcie ze strony kilku innych znajomych twarzy ze sceny skłoterskiej i~przynajmniej pewien poziom tolerancji ze strony innych uczestników. To prawda, że  tolerancja stopniowo zanikała, gdy coraz trudniej było zaprzeczyć, że jego obecność była destrukcyjna. Niektórzy mężczyźni DAN zaczęli podejmować wysiłki, aby porozmawiać z~przyjaciółmi Dennisa lub nieformalnie podejść do samego Dennisa i~spróbować przekonać go do zmiany swojego zachowania. To nie pomogło. Kilka tygodni później, kiedy Dennis zablokował trzy propozycje z~rzędu i~z oburzeniem odmówił współpracy z~tymi, którzy wysuwają propozycję osiągnięcia pewnego rodzaju kompromisu, Brad, który miał pewne doświadczenie ze sceną skłoterów i~był szczególnie skuteczny w~radzeniu sobie z~czubkami, zgłosił się na ochotnika, by zabrać go na bok, aby rozładować sytuację.

-- Facylitatorzy -- wyjaśnił -- zasugerowali mi, że może powinniśmy odsunąć się na chwilę, tylko po to, żeby trochę się uspokoić. Może wyjaśnię ci kilka rzeczy na temat procesu, którego używamy.

-- Co? Wyjść na zewnątrz? -- zażądał Dennis. -- Chcesz wyjść na zewnątrz? Tak. Dobra. Wyjdźmy na zewnątrz!

 W tym momencie prawie wszyscy musieli się zgodzić, że jest problem, a rozmowy za kulisami dotyczyły tego, jak przekonać go, by całkowicie przestał przychodzić na walne zgromadzenia. W końcu przyniosły owoce.

 Mimo to szkody zostały wyrządzone. Dennisowi udało się przybrać bardzo brzydką twarz wobec milczącej męskiej opozycji wobec obaw kobiet, które z~pewnością wykraczały daleko poza niego. Za kulisami rozmowy panów były zdecydowanie ambiwalentne. Na przykład Sam mówił, że opuści grupę, jeśli stworzą obserwatora wibracji, ponieważ ,,była to forma faszystowskiej kontroli umysłu'' (-- Tak -- odpowiedział ze zdumieniem Eric Laursen. -- To mało znany fakt, że wszystkie jednostki SS były wyposażone w~obserwatorów wibracji). Niektórzy mężczyźni opracowywali satyryczną wersję propozycji z~użyciem gigantycznego haka i~gongu. Niektórzy członkowie Grupy Pracy sugerowali, że część trudności może leżeć w~języku. 
 
 -- Może problemem -- zasugerował jeden z~nich -- jest newageowy ton nazwy ,,obserwatora wibracji''. Gdybyśmy mogli wymyślić jakiś nowy termin w~bardziej szorstkim, brudnym, nowojorskim tonie\ldots  
 
 Wzięto to sobie do serca, przynajmniej do tego stopnia, że  zmieniono nazwę z~,,obserwator'' na ,,trzeci facylitator''. Do następnego spotkania, podczas gdy sprawy wciąż były gorące i~kontrowersyjne, a ze sceną skłotu wspierało Dennisa kilku facetów w~średnim wieku, wielu mężczyzn było zmuszonych do opowiedzenia się i~zaczęło publicznie przemawiać za propozycją. Było mnóstwo przyjaznych poprawek: ograniczenia uprawnień trzeciego facylitatora, okres próbny, formalne pisemne wytyczne i~tak dalej. Zalecono, aby po każdym spotkaniu następował ogólny przegląd dynamiki procesu, tonu i~płci. Istniało zalecenie, aby mężczyźni z~DAN stworzyli jakiś rodzaj męskiej grupy, aby zbadać ich własny seksizm, lub przynajmniej przeprowadzić szkolenie antyseksistowskie. Ostatecznie sprowadziło się to do namiętnych przemówień. Kilku z~hardcorowej frakcji czubków groziło odejściem z~grupy (w tym momencie nie było jasne, czy wszyscy czuli, że byłoby to takie złe). Argumentowali, że inni mężczyźni pójdą za nimi. Inni, w~tym ja, argumentowali, że dane demograficzne ze spotkania pokazały, że wiele kobiet już opuściło grupę, choć z~mniejszymi fanfarami. Gdyby nawet ta próba rozwiązania problemów kobiet została odrzucona, moglibyśmy oczekiwać znacznie więcej odejść. Gdybyśmy mieli kogoś stracić, może lepiej byłoby stracić kogoś z~drugiej strony, choćby dla równowagi.

 Środek został ostatecznie zaakceptowany i~DAN dostała trzeciego moderatora. Kiedy rola została stworzona, okazało się to zupełnie nijakie. O ile mi wiadomo, ten trzeci facylitator nigdy nie wezwał nikogo do przerwy i~nigdy nie został oskarżony o cenzurę. Ale takie błędów, raz wyeksponowane, są trudne do zatuszowania.

\medskip
\noindent ZINTERNALIZOWANA OPRESJA

 Konsensus działa na zasadzie zaufania. Dając każdemu członkowi moc blokowania, każdy daje mu moc wepchnięcia grupy w~kryzys w~dowolnym momencie. Chodzi o to, aby wszyscy byli świadomi zaufania, jakim obdarzyła ich grupa, aby odpowiedzialnie korzystać z~władzy, i~rzeczywiście tak się dzieje. Powodem, dla którego rasizm, seksizm i~inne formy, które aktywiści lubią nazywać ,,zinternalizowanymi formami ucisku'', są tak trudne do pokonania, jest właśnie dlatego, że nie jest się ich świadomym. Są jednocześnie absolutnym złem i~są tak fundamentalne dla natury naszego społeczeństwa, że  tworzą nieuniknione aspekty podmiotowości każdego, kto w~nim dorastał. Nie da się ich pokonać samym zaufaniem w~dobre intencje innych.

 Ujmę to w~inny sposób. W latach 80. XIX wieku Peter Kropotkin (1927: 135-36) odpowiadał na twierdzenia, że  anarchizm jest utopijny, argumentując, że tak naprawdę jest odwrotnie. To, co było naiwne i~utopijne, to wiara, że  można dać każdemu arbitralną władzę nad innymi i~zaufać, że sprawują ją odpowiedzialnie. Z pewnością z~mojego własnego doświadczenia wynika, że  klasyczny gest anarchistyczny -- powierzenie aktorom społecznym takiej władzy, że gdyby działali nieodpowiedzialnie, mogliby wyrządzić ogromne szkody, a następnie oczekiwanie, że w~pełni rozważą w~tym celu skutki swoich działań z~tego powodu -- zwykle zawodzi, gdy te uprawnienia są nierówno rozdzielone. Hierarchiczny system organizacji opiera się przecież także na przyznaniu ogromnej władzy władcom, przy założeniu, że sama wiedza, że  mają władzę nad życiem innych, wystarczy, aby zainspirować ich do odpowiedzialnego działania. To jest dokładnie ten rodzaj systemu, któremu anarchiści się sprzeciwiają. Jasne jest, że czasami przyznanie komuś takiej władzy, nawet w~systemie hierarchicznym, ma ten sam efekt. Są sędziowie i~politycy, którzy zawsze doskonale zdają sobie sprawę z~ciężaru odpowiedzialności, jaką niosą ze sobą ich urzędy. Ale to zazwyczaj są niezwykłe jednostki. W większości, sprawujący wysokie urzędy, niezależnie od ich retoryki, szybko przestają postrzegać swoją władzę jako dar lub ciężar wymagający nieustannej refleksji, ale zamiast tego zaczynają uznawać ją za część swojego istnienia. Tak jest oczywiście zawsze z~rodzajami władzy, które pochodzą z~przywilejów białych lub męskich, formami władzy, które nigdy formalnie nie są przyznawane, i~których istotą jest właśnie to, że ich posiadacze nigdy nie muszą o nich myśleć. Nawet u oddanych aktywistów wymaga to intensywnej pracy -- częstych szkoleń, technik podnoszenia świadomości, codziennych przypomnień przez przyjaciół i~facylitatorów -- aby upewnić się, że na początku pozostaną świadomi posiadania tej mocy. Jest to jedyny obszar, w~którym inni po prostu nie mogą wątpić: właśnie dlatego tak wielu aktywistów płci męskiej postrzegało instytucję obserwatora wibracji (przynajmniej tak, jak proponowano) jako atak na samą zasadę konsensusu. W pewnym sensie tak było. Ale jego przedstawiciele z~pewnością postrzegali to jako sposób na patrolowanie granic, aby zapewnić możliwość zaufania w~kręgu.

 Istnieją inne techniki obejścia tego problemu, nawet jeśli żadna z~nich nie jest całkowicie niezawodna. Jedną z~nich jest zachęcanie do ciągłej introspekcji. Dlatego naleganie na spotkaniu, że wszyscy powinniśmy sprawdzać wibracje przez cały czas, działa. Niebezpieczeństwo radzenia sobie z~głęboko zinternalizowanymi formami uprzywilejowania polega na tym, że można popaść w~niekończący się psychologizm -- ,,drażliwy dyskurs rasowy'', jak nazwaliby to niektórzy aktywiści -- że wszystko staje się głęboko spersonalizowane. W przypadku braku jakiejkolwiek autorytatywnej, nadrzędnej ideologii, kończy się jako rodzaj niekończącego się spotkania grupy osobistych narracji i~podmiotowości. Aby tego uniknąć, niektórzy anarchiści nalegali na ciągłe przywracanie spraw do praktycznych warunków działań. Niektórzy na przykład woleli w~ogóle nie używać terminów ,,rasizm'' lub ,,seksizm''. Zamiast próbować zwalczać abstrakcje, takie jak rasizm, przeformułowali problem jako ,,biała supremacja'', jako bezpośredni problem praktyczny: ,,jak możemy zapewnić, że biali ludzie nie zdominują tej grupy?''. Jak męska dominacja, biała supremacja nie była ideologią, która kształtuje świadomość, ale jest wynikiem. Założenie jest takie, że pracując w~grupach, które nie działają na zasadach białej supremacji, rasizmu można się oduczyć. Wydaje się, że jest to rozwiązanie najbardziej zgodne z~ogólnymi zasadami ruchu, ale czasami wydaje się, że przedstawia problem kurczaka i~jajka.

\section{Uwaga końcowa: angażowanie w grupy hierarchiczne}

 Podsumowując: większość amerykańskich anarchistów i~większość osób zaangażowanych w~ruch akcji bezpośredniej uważa, że  jakaś wersja podejmowania decyzji na podstawie konsensusu jest jedyną formą całkowicie zgodną ze społeczeństwem wolnym od systematycznego przymusu fizycznego. Istnieje wiele powodów, by wierzyć, że mają rację. Bardzo niewielu Amerykanów, którzy nie są anarchistami lub nie są zaangażowani w~akcje bezpośrednie, ma duże doświadczenie z~konsensusem lub partycypacyjnym podejmowaniem decyzji jakiegokolwiek rodzaju. W rezultacie wszystkiego trzeba się nauczyć; opracować nowe zwyczaje, zwyczaje i~postawy. Na przykład historię DAN można postrzegać albo jako nieudany wysiłek stworzenia kontynentalnej sieci aktywistów, albo jako niezwykle udany wysiłek rozpowszechniania tej nowej kultury demokratycznej, przynajmniej w~kręgach aktywistów.

 Doświadczenie jednostek, które wkraczają w~ten świat, jest zwykle zarówno radością z~niewyobrażalnych wcześniej możliwości społecznych, jak i~intensywnymi frustracjami związanymi z~niektórymi dylematami opisanymi w~ostatniej części. Ostatecznie jednak największym wyzwaniem stojącym przed autonomicznymi grupami eksperymentującymi z~bezpośrednią demokracją nie jest nawet walka z~uwewnętrznionymi formami hierarchii, ale tworzenie trwałych więzi z~grupami, które nie są zorganizowane według autonomicznych lub bezpośrednio demokratycznych linii. Pozwolę sobie tutaj powiedzieć kilka słów o tym problemie, zanim przejdę do następnego rozdziału, dotyczącego akcji.

 Jak zauważyłem, konsensualne podejmowanie decyzji wymaga wyjątkowej hojności ducha w~kontaktach z~tymi, których uważa się w~kręgu demokratycznym. Wydaje się, że jednym z~konsekwencji jest całkowite odrzucenie jakiejkolwiek perspektywy konstruktywnego zaangażowania z~tymi, którzy wyraźnie znajdują się poza tym kręgiem. Stąd druga pozycja zasad jedności CDAN (przejęta bezpośrednio ze znaków rozpoznawczych PGA): ,,Konfrontacyjna postawa wobec niedemokratycznych instytucji, w~tym rządów i~korporacji, w~których kapitał jest jedynym prawdziwym twórcą polityki''. W przeciwieństwie do swoich reformistycznych sojuszników -- czy to organizacji pozarządowych, związków zawodowych, czy partii politycznych -- akcjoniści bezpośredni nie mieli zamiaru rozpoczynać rozmowy z~kierownictwem WTO, IMF, Bankiem Światowym czy, jeśli o to chodzi, rządem Stanów Zjednoczonych. Nie chcieli reformy tych instytucji. Chcieli, aby zostały zniesione. Często określali ich na spotkaniach lub w~literaturze aktywistycznej jako ,,zło'' -- termin, który, rzucany tak często, jak to było, może sprawiać wrażenie, że obcy mają do czynienia z~naiwnymi fanatykami. Ale warto rozważyć zastosowanie tego terminu. Było bardzo konkretne. Z własnego doświadczenia wiem, że słowo ,,zło'' nigdy nie było stosowane do jednostek. Dotyczyło tylko organizacji. ,,Zło'' to nie ci, którzy pracują dla organizacji takich jak IMF, ani nawet nimi zarządzają, ale większa struktura instytucjonalna, w~której działają, czy to dlatego, że ,,kapitał jest jedynym prawdziwym twórcą polityki'', czy też dlatego, że takie organizacje są uwikłane w~hierarchiczne łańcuchy dowodzenia, a prawdopodobnie obu, struktury, która uniemożliwia takim ludziom działanie uczciwie i~przyzwoicie. W związku z~tym, jedynym właściwym moralnym postulatem jest zamknięcie takie organizacyjne struktury\footnote{Stąd sporadyczne debaty w~DAN na temat tego, czy, powiedzmy, Organizacja Państw Amerykańskich lub Organizacja Narodów Zjednoczonych były ,,złe'' w~tym samym sensie. Dyskutanci tylko sporadycznie używali tego rzeczywistego słowa. Ale kluczowe pytanie brzmiało: czy można było postrzegać je jako kontrolę jakiejś innej, bardziej skandalicznej formy władzy, a zatem konstruktywnie się z~nimi angażować, czy też były one z~natury zbyt skorumpowane?}.

 To dość proste. Prawdziwym problemem jest to, jak radzić sobie z~reformistycznymi sojusznikami -- tymi, którzy \textit{chcą }zaangażować się w~struktury władzy. Sojusze z~liberałami, jak nazywają je anarchiści, zawsze były napięte. Napięcia były zwykle pozornie związane z~nieporozumieniami dotyczącymi taktyki, nieuniknioną kwestią ,,przemocy''. Tutaj często pojawiał się rodzaj niewygodnego sojuszu: organizacje pozarządowe nie lubiły anarchistycznej taktyki, ale także zdawały sobie sprawę z~tego, że bardzo pożyteczne jest posiadanie wojujących radykałów, poza którymi wydawali się rozsądni, jeśli w~rzeczywistości szukali miejsca przy stole. Anarchiści nie lubili liberalnego reformizmu, ale uznali, że warto mieć kogoś w~głównym nurcie, by wzniecić zamieszanie, gdyby rząd zwrócił się ku oczywiście nielegalnym formom represji. Niemniej jednak z~biegiem czasu stało się jasne, że prawdziwy podział był organizacyjny: i~że w~rzeczywistości, nawet najbardziej pacyfistyczni członkowie grup akcji bezpośredniej czuli, że mają więcej wspólnego z~anarchistami Black Bloc niż z~najbardziej radykalnymi NGO.

 Czasami trzeba było szokującego aktu przemocy -- między aktywistami -- aby to wyjaśnić. W Waszyngtonie na przykład Mobilization for Global Justice, anty-korporacyjna koalicja organizacji pozarządowych, która teoretycznie działała na zasadach demokratycznego konsensusu, zakończyła się rozłamem w~sprawie napaści na tle seksualnym. Nie byłem wtedy osobiście zaangażowany w~scenę DC i~otrzymywałem tylko raporty z~drugiej lub trzeciej ręki, ale historia była taka. Ofiara (inny aktywista) trafiła do szpitala i~chociaż anarchiści z~grupy niechętnie przekazywali policji nawet gwałciciela, z~pewnością czuli, że konieczna jest jakaś radykalna interwencja i~byli oburzeni, że tak wielu kolegów działaczy było niechętnych cokolwiek zrobić, a nawet powiedzieć cokolwiek o człowieku, który (ze względu na swoją pozycję organizacyjną) kontrolował wszelkiego rodzaju pieniądze i~zasoby. W wynikającym rozłamie większość aktywistów zrezygnowała i~dołączyła do alternatywnej koalicji nazwanej Anti-Capitalist Convergence (ACC -- Konwergencja Antykapitalistyczna) -- tej, która była wcześniej znana jako DC Black Bloc.

 To był skrajny przypadek. Ale prawie zawsze punkty rozłamu dotyczyły podobnych kwestii zaufania, tego, o czym można, a o czym nie można mówić, oraz jak wpłynęła na to przynależność instytucjonalna. W Nowym Jorku, DAN Labor ostatecznie upadło, gdy zawodowi organizatorzy związków zaczęli brać udział w~spotkaniach; Policja i~więzienia wkrótce potem, w~kwestiach rasowych, ale zaostrzonych przez radykalne różnice organizacyjne między aktywistami DAN a tymi, których starali się wspierać. Ostateczny cios wszystkim takim sojuszom przyszedł oczywiście wraz z~11 września, po którym prawie wszystkie związki zawodowe odmówiły wiązania się z~czymkolwiek, co można by nazwać niepatriotycznym. Większość organizacji pozarządowych, przerażona swoją bazą finansowania, również się wycofała. Rozłam ujawnił się najwyraźniej podczas protestów przeciwko Światowemu Forum Ekonomicznemu w~Nowym Jorku w~lutym 2002 roku. Od samego początku dochodziło do zderzenia kultur politycznych. Urzędnik organizacji pozarządowej najbardziej zdeterminowany, by wesprzeć ruch, rozpoczął swoje zaangażowanie od pojawienia się na konferencji aktywistów, gdzie najpierw entuzjastycznie zapewnił lokalnych radykałów (nieoficjalnie), że w~Seattle doszło do ,,dokładnie właściwej ilości zniszczeń mienia'', potem niemal natychmiast stanął na podium i~wygłosił wykład publiczny, który zawierał ogólne potępienie wszelkich form niszczenia mienia. Zamiast kogokolwiek uspokajać, stał się doskonałym ucieleśnieniem tego rodzaju zachowań, których podstawę ma zniszczyć demokracja bezpośrednia. Powstały komitety łącznikowe, ludzie z~organizacji pozarządowych składali najróżniejsze obietnice, po czym w~większości znikali. Na końcu, ogłosili, że nie mogą nic zrobić bez ,,przyrzeczenia pokoju'', które ma podjąć każdy biorący udział w~protestach, przysięgając, że nie będzie żadnej przemocy ani zniszczenia. To ostatnie było tak oczywiście oburzające, z~perspektywy członków NYC DAN, nie mówiąc już o nowo założonej nowojorskiej ACC, że nikomu z~żadnej z~grup nawet nie przyszło do głowy, aby przedstawić tę propozycję na spotkaniu. Dalsza dyskusja nie była możliwa. i~to pomimo faktu, że nikt poważnie nie rozważał zdemolowania wyraźnie straumatyzowanego miasta i~każdy członek którejkolwiek z~grup, który wypowiadał się publicznie, potwierdził to.

 Wracamy do omówionego wcześniej problemu przejrzystości moralnej. Jeśli nic innego, zderzenie norm i~oczekiwań moralnych jest takie, że każdy sojusz, który przekracza granicę między autonomicznym organizowaniem a tymi, którzy są w~ciągłym zaangażowaniu w~rząd i~kulturę polityczną głównego nurtu, będzie prawdopodobnie kruchy i~niestabilny. Jednocześnie jednak często są one konieczne, a na pewno korzystne dla obu stron. Rezultatem wydaje się niekończący się cykl łączenia się i~rozpadania.

\chapter{Akcje}

\begin{flushright}

 \texttt{Społeczeństwo, które odmawia nam każdej przygody, czyni własny rozpad jedyną możliwą przygodą.}


--- Reclaim the Streets, Londyn
\end{flushright}

 Głównym celem spotkań jest planowanie wydarzeń określanych mianem ,,akcji''. Jest to prawdopodobnie skrót od ,,akcja bezpośrednia'', ale w~powszechnym użyciu termin ten jest używany w~odniesieniu do praktycznie każdego zbiorowego przedsięwzięcia, które ma zarówno zamierzenie polityczne, jak i~jest realizowane ze świadomością, że może spotkać się z~wrogą interwencją ze strony policji. Innymi słowy, podczas gdy spotkania mają być wolną przestrzenią wzajemnego szacunku i~solidarności, swoistym wyzwolonym terytorium, działania są tam, gdzie ta przestrzeń styka się -- lub przynajmniej jest zagrożona -- jej przeciwieństwem: gdzie aktywiści, którzy postrzegają istniejące struktury władzy jako nieuprawnione, stają twarzą w~twarz z~przedstawicielami tej władzy, którzy z~kolei uważają ich działania za nielegalne.

 W rezultacie termin ,,działanie'' może być używany w~odniesieniu do przedsięwzięć o bardzo zróżnicowanej skali i~bojowości: od rozdawania ulotek przed supermarketem po zamknięcie światowego szczytu. To z~kolei bardzo utrudnia generalizowanie na ich temat. Niemniej jednak postaram się w~tym rozdziale wypracować przynajmniej bardzo zgrubną typologię tego, o jakich rodzajach działań można by powiedzieć, że istnieją, ich różnych zasadach, celach i~implikacjach.

 Nie jestem pierwszym, który próbuje to zrobić. Aktywiści czasami sami opracowują prymitywne typologie. Większość z~nich skupia się jednak w~mniejszym stopniu na strukturze działania niż na zamierzonym celu. W jednym z~niedawnych esejów aktywista medialny Patrick Reinsborough wyróżnia pięć rodzajów akcji bezpośredniej:

 1. Akcja bezpośrednia w~miejscu produkcji (np. strajki)

 2. Akcja bezpośrednia w~miejscu zniszczenia (np. blokowanie buldożerów w~celu ratowania lasów lub skłotów) 

 3. Akcja bezpośrednia w~miejscu konsumpcji (np. bojkoty konsumenckie)

 4. Akcja bezpośrednia w~miejscu decydowania (np. próba zamknięcia spotkań WTO w~Seattle)

 5. Akcja bezpośrednia w~punkcie hipotezy (np. zagłuszanie kultury) (Reinsborough 2004: 183-85)

 Chociaż skuteczna kampania, jak twierdzi Reinsborough, zwykle obejmuje kombinację kilku różnych rodzajów działań, działania w~punkcie założenia są ostatecznie najważniejsze -- lub przynajmniej najgłębsze -- ponieważ skupiają się na podstawowych ramach, w~których akty są interpretowane. Próbują zmienić warunki dyskusji, lub przynajmniej stawiają pod znakiem zapytania przedmiot dyskusji.

 Są oczywiste powody, dla których aktywiści, pisząc o działaniach w~sposób abstrakcyjny, mają tendencję do przyjmowania tego rodzaju strategicznej perspektywy. Większość istniejącej literatury dotyczy właśnie tego rodzaju pytań: tego, jak działania pasują do większych kampanii, z~ich ostatecznymi skutkami bardziej niż z~ich wewnętrzną strukturą. Sprawy taktyczne pozostawia się podręcznikom, materiałom szkoleniowym i~poradnikom: o których jest dość obszerna literatura (np. Ruckus Society 1997a, 1997b, 1997c). Obszerne, ale niezbyt systematyczne. Prawdopodobnie najlepszym i~najbardziej wyczerpującym najnowszym podręcznikiem akcji tego rodzaju jest \textit{Recipes for Disaster} CrimethInc (2005), który ma służyć jako praktyczny przewodnik po akcji bezpośredniej we wszystkich jej przejawach. Rozdziały są uporządkowane następująco:

\noindent Grupy afinicji \newline Akcje antyfaszystowskie \newline Mozaiki w~asfalcie \newline Wystawianie banerów \newline Behawioralny kolaż \newline Kolektywy rowerowe \newline Rowerowe parady \newline Malowanie na rowerze \newline Jak zamienić rower w~gramofon \newline Poprawianie billboardów \newline Blocs, czarne i~inne \newline Blokady i~zapory \newline Przejęcie klasy \newline Budowanie koalicji \newline Kolektywy \newline Zwolnienia w~korporacji \newline Dystrybucja, składanie i~infoshopy \newline Nurkowanie w~śmietniku \ldots 

I~tak dalej\footnote{Podobnie, w~serii broszur szkoleniowych Ruckus Society, poruszane tematy to wspinaczka, skauting, planowanie działań, media, węzły i~tworzenie filmów.}. Powstałe w~ten sposób kompendium jest niezwykle przydatne dla każdego, kto szuka wskazówek, jak wskoczyć do pociągu towarowego lub rzucić tortem w~urzędnika publicznego. Ale jako taksonomia pozostawia trochę do życzenia. Jako etnografa interesuje mnie wydobycie ukrytej, leżącej w~tle, struktury założeń, próbując zrozumieć, jakie rodzaje działań istnieją na podstawie ich własnej wewnętrznej logiki: jako formy działania, które są w~pewnym sensie performansami, w~pewnym sensie rytuały, ale jednocześnie od razu są skuteczne w~świecie. Myślę, że aby móc myśleć o działaniach, trzeba zacząć od bardziej formalnej typologii. 

\section{Część I: Kilka przykładów szczególnych stylów działania }

 Jednak każda próba stworzenia listy elementarnych form lub jednostek akcji -- wystawianie banerów, wiec, okupacja, marsz węży, blokada i~tak dalej -- naraża się na natychmiastowe problemy. Niektóre formy zawsze kończą, obejmując lub będąc na zupełnie innej skali niż inne (inne nakładają się). Więc zamiast starać się być wyczerpującym, postanowiłem ograniczyć się do działań publicznych, które gromadzą dość dużą liczbę uczestników (częściowo po to, aby upewnić się, że mam do czynienia z~mniej lub bardziej wymiernymi przedmiotami), a następnie staram się zrozumieć strukturę każdego rodzaju działania pod kątem tego, co aktorzy próbują osiągnąć. To z~kolei będzie oznaczać ustalenie mojej typologii w~zasadzie jako różnych sposobów konfiguracji relacji czterech elementów: tych, którzy realizują działanie, przedmiotu lub celu ich działań, publiczności (prawdziwej lub wymyślonej) i~policji. Umożliwi to wykazanie, że wiele klasycznych form działania jest w~pewnym sensie wariacjami, a nawet inwersjami, różnymi sposobami przestawiania tych samych podstawowych elementów. 

 Zacznę od kilku słów o (1) marszach i~wiecach protestacyjnych, które w~większości przypadków nie są technicznie bezpośrednimi akcjami, a następnie przejdę do rozważenia (2) pikiet, (3) imprez ulicznych, (4) klasycznych nieposłuszeństwo obywatelskie (blokady i~okupacje) i~wreszcie (5) akcje Czarnego Bloku. 
 
 Każda sekcja będzie zorganizowana wokół sprawozdania z~pierwszej ręki, zaczerpniętego z~moich notatek. W rezultacie większość do pewnego stopnia przepełni swoją ramkę, jak to zwykle ma miejsce w~przypadku rzeczywistych opisów, ale lubię myśleć, że robiąc to, będą w~rzeczywistości bardziej przydatne, niż gdyby były dostosowane i~edytowane. Rozdział zakończymy kilkoma uwagami na temat relacji z~państwem: doświadczenia aresztowania oraz milczących zasad zaangażowania, które regulują stosunki między aktywistami a policją. 

\medskip
\noindent PRZYKŁAD PIERWSZY: PROTEST (MARSZE I~WIECE)
\medskip

 Marsze i~wiece to przede wszystkim liczby. Ich pozornym celem jest umieszczenie jak największej liczby ludzi na ulicach. Jak widzieliśmy w~części I, większość anarchistów postrzega ten imperatyw jako nieco bezcelowy, ,,marsz wraz ze znakami'' -- w~rzeczywistości większość postrzega go jako przeciwieństwo akcji bezpośredniej i~definiuje swoje własne formy działania w~opozycji do klasycznych marszów. Powody nie leżą daleko. Anarchiści preferują taktyki bojowe, ale odrzucają cokolwiek, co smakuje jako dyscyplina wojskowa. Konwencjonalne protesty są całkowicie pokojowe, ale prawie zawsze organizowane są w~odgórny, militarny sposób, ze szwadronami oficjalnych ,,marszałków'', aby utrzymać porządek i~poganiać skądinąd zupełnie niezorganizowane masy protestujących. Nic nie może być dalsze od anarchistycznych ideałów samoorganizacji. Z drugiej strony, gdy grupy są zorganizowane, ich wewnętrzna organizacja jest często sama w~sobie wyraźnie hierarchiczna, z~różnymi grupami ubranymi w~identyczne kapelusze lub T-shirty, noszącymi drukowane znaki, z~liderem z~megafonem wykrzykującym pieśni. Często większe marsze przypominają paradę z~okazji Dnia Św. Patryka lub Dnia Pracy, z~następującymi różnymi grupami czy blokami jeden po drugim, często w~identycznych uniformach, nasyconych zespołami, formacjami, ciężarówkami ze wzmacniaczami itd. 

 Masy i~łączenie sił, oczywiście, przywodzi na myśl obraz armii, ale zwykle armie maszerują i~zbierają się, by dotrzeć do miejsca walki. Protestujący maszerują i~gromadzą się, by zrobić czysty pokaz swojej liczebności, a następnie zwykle kończą, stojąc w~jakiejś dużej przestrzeni publicznej, słuchając inspirujących mówców, być może zawierających analizy sytuacji politycznej i~sugestie dotyczące innych sposobów działania. Ewentualnie, po zakończeniu marszu, mogą po prostu wrócić do domu. 

 Nacisk na liczebność oznacza również, że marsze i~wiece -- nawet najbardziej radykalne -- są zazwyczaj legalnymi, dozwolonymi wydarzeniami, a to z~kolei ma szereg reperkusji. Najwyraźniej oznacza to, że organizatorzy będą chcieli zapewnić, aby wszyscy biorący udział w~marszu lub wiecu -- lub ktokolwiek w~jego pobliżu -- przestrzegają prawa. Niemal zawsze oznacza to, że patrzą z~ekstremalną nieprzychylnością na każdego, kto może praktykować bardziej bojową taktykę gdziekolwiek w~pobliżu. 

 Moim celem tutaj nie jest tylko zarejestrowanie litanii typowych anarchistycznych skarg. Niektórzy anarchiści będą argumentować, że marsze i~wiece nigdy same z~siebie nie spowodowały żadnej znaczącej zmiany społecznej. To jest wyraźnie niesprawiedliwe. Równie dobrze może to być prawda w~sensie dosłownym: marsze i~wiece są w~ten sposób skuteczne tylko wtedy, gdy stanowią jeden z~elementów znacznie szerszych kampanii wykorzystujących szeroką gamę taktyk. Ale jako takie mogą odgrywać istotną, a może nawet niezbędną rolę. Marsz na Waszyngton w~1963 roku i~wiec w~Lincoln Memorial, gdzie Martin Luther King wygłosił przemówienie ,,I Have a Dream'', mogły być kulminacją lat walki, obejmującej bojkoty, strajki okupacyjne i~wszelkiego rodzaju obywatelskie nieposłuszeństwo i~działania bezpośrednie. Jednak to właśnie ten marsz i~wiec utkwił w~popularnej wyobraźni, skutecznie stać się częścią amerykańskiej mitologii. To właśnie uświadomiło nam, że wydarzyło się coś epokowego. Anarchista mógłby sprzeciwić się, że właśnie to jest problematyczne w~takich wydarzeniach: zaznaczają one moment, w~którym lata autonomicznego organizowania się przez lokalnych aktywistów są zasadniczo przypisywane osobowości tego czy innego charyzmatycznego przywódcy. To jest uzasadniona skarga. Ale można również argumentować, że aby dokonać pewnych zmian społecznych, jest to prawdopodobnie nieuniknione.

 Pomocne może tu być spojrzenie na historię. Myślę, że to znaczące, że w~Stanach Zjednoczonych wolność słowa i~zgromadzeń (prawo, w~najbardziej dosłownym tego słowa znaczeniu, do marszu i~zgromadzeń) zostały ustanowione jako prawa podstawowe dokładnie w~momencie, gdy Stany Zjednoczone również przyjęły reprezentatywną formę rządu, to znaczy moment odrzucenia tego, co w~tamtych czasach nazywano ,,demokracją'', w~sensie systemu w~stylu ateńskim, w~którym społeczności rządziły się poprzez zgromadzenia ludowe. Innymi słowy, wystąpienia publiczne i~zgromadzenia stały się niezbywalnymi prawami w~momencie, gdy zostały definitywnie odrzucone jako środek rzeczywistego podejmowania decyzji politycznych. Zamiast tego wyobrażano je sobie przede wszystkim jako środek ,,wnioskowania do rządu o zadośćuczynienie krzywd'' -- to znaczy jako formę protestu\footnote{Niektóre części kraju, takie jak Nowa Anglia, od dawna znały podejmowanie decyzji w~radach miejskich; w~niektórych - zwłaszcza w~Pensylwanii - bezpośrednio po rewolucji nastąpił rozkwit form bezpośrednio demokratycznych. Federaliści wyraźnie odrzucali ten model i~oczywiście samą ideę ,,demokracji'' w~ogóle. Dopiero na początku dziewiętnastego wieku termin ten wszedł do powszechnego użytku jako nie tylko określenie nadużycia (Graeber 2007).}. Podczas gdy marsze i~wiece były od dawna częścią repertuaru ruchu robotniczego (Davis 1985), rozwinęły swoją charakterystyczną formę w~nowej republice głównie w~kontekście polityki wyborczej, szczególnie w~obliczu poszerzenia się franczyzy i~populizmu jacksonowskiego w~latach 30. i~40. XIX wieku. Prawie każda społeczność zaczęła wspierać kluby partyjne i~,,maszerujące kompanie'', a wybory charakteryzowały się masowymi mobilizacjami obejmujące wiece, pościgi, korowody i~parady oświetlone pochodniami (Baker 1984; McGerr 1987, 1990). To pozostawało prawdą przez prawie całe XIX stuleci i~do pewnego stopnia amerykańskie partie polityczne nadal wykorzystują marsze i~wiece, choć we współczesnych, zmediatyzowanych kampaniach odgrywają one coraz mniejszą rolę. 

 Dekady po wojnie secesyjnej były jednak świadkami silnej reakcji ze strony klas wykształconych przeciwko tego rodzaju mobilizacji ludowej. Co znamienne, był to również okres, w~którym narodził się kapitalizm korporacyjny i~narodziny nowoczesnej amerykańskiej policji. Na przykład w~latach 70. XIX wieku większość amerykańskich miast zaczęła nalegać, aby obywatelom nie wolno już wygłaszać publicznych przemówień lub zgromadzeń bez uprzedniego złożenia wniosku o pozwolenie -- ograniczenia, które, ponieważ były bezpośrednio sprzeczne z~językiem Pierwszej Poprawki, podlegały poważnej debacie i~dopiero ostatecznie zostały zatwierdzone przez Sąd Najwyższy w~1897 (Baker 1983). To dzięki takim ustawom protestują ci, którzy najbardziej bezpośrednio sprzeciwiają się wzrostowi kapitalizmu korporacyjnego -- na przykład radykalne grupy robotnicze, takie jak IWW -- mogły być systematycznie tłumione (Dubofsky 1969; Preston 1994). W rezultacie dozwolone marsze i~wiece stały się swoistą domeną parapolityki -- w~istocie technik lobbingowych -- sposobów publicznego apelu do wybieranych urzędników poza formalnymi mechanizmami samych wyborów. Historia marszów na Waszyngton, ostatnio naświetlona przez Lucy Barber (2002), daje pewien sens tego, co się zaczęło dziać. Kiedy w~czasie kryzysu w~1894 roku Jacob Coxey postanowił zorganizować marsz bezrobotnych do stolicy z~propozycją projektów robót publicznych, pomysł ten został powszechnie uznany za bezprecedensowy i~oburzający. Zasadność marszu na stolicę zaczęto akceptować dopiero wraz z~marszem sufrażystek w~1913 roku; następnie, wykorzystanie przez Herberta Hoovera wojsk federalnych do rozproszenia weteranów Bonus Marchers w~1932 roku odegrało dużą rolę w~zapewnieniu jego politycznego upadku. W latach siedemdziesiątych stolica kraju stała się sceną niemal codziennych demonstracji tego czy innego rodzaju, tak jak pozostaje do dziś. Ponieważ zdecydowana większość tych protestów cieszy się niewielką lub żadną uwagą mediów, trudno je dostrzec -- lub ich odpowiedniki w~stolicach stanów i~innych miastach -- jako same w~sobie próby wywarcia presji na wybieranych urzędników. Ale prawie zawsze mają tendencję do bycia częścią szerszych, zintegrowanych kampanii, których celem jest właśnie to. 

\medskip
\noindent Wiec

 Pozwólcie, że wymienię jeden wiec, który odbył się w~2001 roku na Water Street w~dzielnicy finansowej, na dalekim południu Manhattanu, przed biurami WBAI, radykalnej stacji radiowej, zaraz po tym, co w~kręgach aktywistów nazywano ,,Bożonarodzeniowym zamachem stanu''. W tym czasie koteria konserwatywnych członków zarządu, zdecydowana zmienić swój wydźwięk polityczny, przejęła kontrolę nad jej administracją i~zwolniła wielu z~jej najwybitniejszych producentów i~osobistości. Wiec był jednym z~pierwszych momentów dwuletniej kampanii, która zakończyła się zastosowaniem szerokiej gamy taktyk, od akcji bezpośredniej po bojkoty, i~która po dwóch latach zakończyła się całkowitym sukcesem w~przywróceniu stacji do wcześniejszego radykalnego status quo.

 Wybrałem ten przykład głównie dlatego, że brałem udział w~spotkaniu planistycznym, czego nigdy wcześniej nie robiłem na legalnym wydarzeniu. Odbyła się ona w~nowojorskim IMC i~była dziwną kolekcją weteranów z~lat 60., z~garstką młodszych aktywistów, głównie ludzi z~DAN. Chociaż wszyscy korzystali z~procesu konsensusu i~starali się przynajmniej deklarować nowsze formy organizacji, było jasne, że instynkty polityczne większości uczestników ukształtowały się w~zupełnie innym czasie.

 Zaczęliśmy od wymyślenia podstawowych parametrów. W przyszłym tygodniu zorganizowaliśmy stosunkowo mały wiec, aby nabrać energii na znacznie większy marsz i~wiec w~późniejszej części miesiąca. Dyskusja dotyczyła głównie czasu. Planując jeden, zdajesz sobie sprawę, że rajd jest trochę jak pokaz różnorodności. Masz do czynienia z~serią dość ograniczonych przedziałów czasowych, o ile nie jest to doniosłe wydarzenie, musisz założyć, że ludzie zaczną wychodzić po około godzinie, więc musisz bardzo ostrożnie zaplanować czas. Potrzebujesz przynajmniej jednego MC, być może więcej, różnych głośników, ale także rozrywki i~aktów muzycznych, aby wszystko podzielić na części. Czas jest podzielony na bardzo małe segmenty -- często zaledwie minutę lub dwie -- a następnie należy go starannie rozdzielić, aby zapewnić reprezentację wszystkich możliwych okręgów wyborczych: mówca X jest zabawny, ale czy chcemy jeszcze jednego białego samca?. Mówczyni Y jest jedyną Filipinką w~programie i~jest poetycka i~inspirująca, ale nie ma mowy, abyśmy mogli utrzymać ją przez tylko trzy minuty. Ona zawsze wykracza poza czas. i~tak dalej.

 Na samej imprezie zgłosiłem się na ochotnika jako jeden z~marszałków, co jest kolejną rzeczą, której nigdy wcześniej nie robiłem. Moje notatki z~tego wydarzenia zawierają długą refleksję na temat znaczenia kojców protestacyjnych:

\medskip
\noindent \textit{Zapiski WBAI Rally, Dolny Manhattan}

\medskip
\noindent [Notatki terenowe, 12 stycznia 2002 roku]
\medskip

 Przyjechałem wcześnie, kiedy ludzie skończyli ustawiać podium. Policja toczyła swoje drewniane barykady i~konstruowała zagrody, a nasi ludzie testowali mikrofony i~wzmacniacze. Zgłosiłem się na ochotnika, aby zostać jednym z~marszałków, otrzymując czerwoną opaskę na krótkim info dla marszałków w~kawiarni kilka przecznic dalej.

 Standardową procedurą policyjną na każdym wiecu w~Nowym Jorku jest wznoszenie zagród dla protestujących -- ,,chlewni'', które często się nazywają -- z~drewnianych lub stalowych barykad, z~tylko jednym lub dwoma otworami, a następnie oczekiwanie, że protestujący zamkną się wewnątrz. Pozornym powodem jest upewnienie się, że protestujący nie blokują ruchu pieszych ani ruchu ulicznego, ale w~rzeczywistości kojce są ustawiane nawet na dużych placach publicznych, na których nie ma ruchu. Wpływ na protestujących jest głęboko demoralizujący, ponieważ sprawia, że  czują się uwięzieni i~w pułapce, a także bardzo utrudnia im kontaktowanie się z~przechodniami lub komunikowanie się z~nimi, nie mówiąc już o dołączaniu do nich zwykłym obywatelom. Większość aktywistów zakłada, że  to jest ich prawdziwy cel.

 Nie jest to technika stosowana w~większości amerykańskich miast, a kwestie prawne są niejasne. W rzeczywistości jest wielu, którzy podejrzewają, że NYPD jest świadoma, że  tak naprawdę nie ma prawa do ograniczania protestujących w~ten sposób i~że praktyka nigdy nie wytrzymałaby weryfikacji w~sądzie, przytaczając jako dowód fakt, że policja nigdy wyraźnie nie nakazuje ludziom stać w~nich (zazwyczaj zaczynają się od takich fraz, jak ,,Bylibyśmy bardzo wdzięczni, gdybyś został w~kojcach'', a potem używają zwrotów typu ,,musisz wrócić do zagrody -- blokujesz ruch'' i~że nikt nie zna nikogo, kto kiedykolwiek został aresztowany za odmowę pozostania w~środku. Mimo to policja prawie zawsze będzie próbowała zamienić marszałków na jakiekolwiek stacjonarne demonstracje w~swoich agentów w~tym zakresie.

 Zlot był klasyczną, staromodną imprezą z~nagłośnieniem i~głośnikami. Było kilku muzyków i~innych rozrywek (po długich dyskusjach ludziom z~DAN udało się uzyskać przerywnik z~Radical Cheerleaders w~programie), marszałkowie i~garstka legalnych obserwatorów z~Krajowej Gildii Prawników. Ktoś przywiózł ze sobą ogromne pudło pełne kostiumów, które znaleźli w~Charas: serię masek rdzennych Amerykanów ocalonych z~niedawnego marszu Peltiera, a my bawiliśmy się pomysłem, aby marszałkowie nosili maski wilka (naprawdę bardziej jako kapelusze) i~obserwatorzy prawni -- maski orłów, ale niewielu było chętnych do zabawy. Kiedy po raz pierwszy przybyliśmy, był tam biały oficer (dowódca) z~lokalnego komisariatu, rozmawiający z~organizatorami w~bardzo przyjazny sposób. Poparł kampanię WBAI, powiedział: ponieważ widział to jako przedłużenie ruchu robotniczego. Był tam tylko po to, by zapewnić, że wszystko przebiegnie w~uporządkowany sposób i~że żaden obywatel nie będzie sprawiał kłopotów. Po tym, jak tłum zaczął się zwiększać (w końcu osiągnął kilkaset), wyszedł. Kilku policjantów zostało umieszczonych w~pobliżu wejścia do WBAI, aby uniemożliwić ludziom wejście, a kilku innych gliniarzy było rozproszonych.

 Nie zwracałem zbytniej uwagi na mówców, ponieważ większość mojej energii wkładałem w~patrolowanie obwodu i~rozmawianie z~innymi marszałkami -- prawie wszystkimi ludźmi DAN -- o logistyce i~możliwych problemach. Coraz częściej też unikałem policji. Szybko odkryłem, że jeśli można cię zidentyfikować jako marszałka, natychmiast zaczęły się dziać dwie rzeczy. Po pierwsze, odkryłeś, że reguła kojca nie dotyczy ciebie. Możesz chodzić, gdzie chcesz, a gliny nigdy by ci nie przeszkadzały. Po drugie, gliniarze traktowaliby cię tak, jakby twoim zadaniem było upewnienie się, że nikt inny nie opuścił zagrody. To, że to była moja praca, było dla nich oczywiste -- w~pewnym momencie policjant z~oburzeniem podszedł do mnie i~warknął: ,,Hej, nie wykonujesz swojej pracy! Spójrz na tych wszystkich ludzi na ulicy!''. Wzruszyłem tylko ramionami i~odwróciłem się, ale nacisk był dość stały, a większość innych marszałków, niechętnie, zaczęła przypominać innym ludziom, że policjanci oczekują, że będą w~środku barykady.

 Jest to dość nieszkodliwy przykład, ale daje okno na kluczową dynamikę. W chwili, gdy ktoś zobowiązał się do przestrzegania rządów prawa i~ubiega się o pozwolenie, zostaje wciągnięty w~sieć władzy hierarchicznej: otrzymuje niewielkie formy immunitetu od arbitralnych reguł, które następnie powinien pomagać egzekwować wobec innych. Moje notatki dalej zawierają:

 Dla większości ludzi z~DAN rajd wydawał się trochę głupi: w~końcu mówcami byli prawie wszyscy ludzie, których setki razy wcześniej słyszeliśmy w~radiu. To prawda, że  ciekawie było stanąć fizycznie między innymi słuchaczami, aby zobaczyć, jak naprawdę wyglądają ci wszyscy niewidzialni ludzie na audytorium radiowym. Mimo to nie poświęciliśmy dużo energii na słuchanie głośników.

 Jedyną częścią, na którą czekaliśmy, były Radykalne Cheerleaderki -- nasi ludzie -- ale okazało się to wielkim rozczarowaniem. Zostały dodane jako ukłon w~stronę nowych stylów aktywizmu, i~na tę okazję skomponowali nowy wiwat, ale jak się okazało, prawie nie było ich widać na scenie, więc większość publiczności tak naprawdę nie złapała głupiego dowcipu kostiumów -- punkowo-anarchistyczna wyprawa licealnej drużyny cheerleaderek, przepełnionej czerwonymi i~czarnymi pomponami -- a cała konfiguracja opierała się na założeniu jednego mówcy, więc nie było dobrego sposobu na to, by przez mikrofony słychać było ośmioosobową linię. W każdym razie występ nie był przeznaczony na scenę, miał być wykonywany w~środku tłumu. Próbowali całkowicie pozbyć się mikrofonów, ale wtedy ludzie ledwo ich słyszeli\ldots 

\noindent co sprawiło, że ponownie zastanowiłem się nad niejawną hierarchią, która wkrada się w~takie wydarzenia, i~stopniem, w~jakim większość anarchistycznych form, które brałem za pewnik, została zaprojektowana całkiem świadomie, by je podcinać.

 Wielkie pytanie dotyczące marszów i~wieców dotyczy zawsze publiczności. Na powierzchownym poziomie, można by powiedzieć, chodzi o to, by zaimponować liczbami. Ale komu się chce zaimponować? Przypuszczalnie jednym celem są możni, zwłaszcza politycy, którzy, jak mamy nadzieję, zareagują na poczucie, że niektórzy z~ich wyborców są tak entuzjastycznie nastawieni do określonej kwestii, że może to wpłynąć na ich głosowanie. Większość wybieranych urzędników w~Ameryce używa systemu do zestawiania tych rzeczy: zakłada się, że nazwisko na petycji reprezentuje dwudziestu wyborców, list formalny pięćdziesiąt, list osobisty dwieście i~tak dalej. W szerszym sensie być może apeluje się do ,,opinii publicznej'', która w~Ameryce, przynajmniej w~tym kontekście, oznacza odbiorców mediów, których można pobudzić, wiedząc, że tak wielu ludzi z~taką pasją odnosi się do określonej kwestii lub kogoś, zanim zobaczyli wiec, mógł nie zdawać sobie sprawy, że problem w~ogóle istnieje. Ale w~innym sensie, jak organizatorzy często przyznają jako pierwsi, publicznością są w~rzeczywistości sami protestujący, którzy -- zwłaszcza jeśli nie są wieloletnimi weteranami -- prawie zawsze wracają do domu odnowieni i~zainspirowani samym doświadczeniem bycia wśród tak wielu ludzie, którzy się z~nimi zgadzają, pełni nowych pomysłów, informacji, literatury, przyjaciół i~kontaktów osobistych, odnowieni w~swoim zaangażowaniu w~mobilizację polityczną we wszystkich jej formach. Możliwość doświadczania wyobrażonej społeczności słuchaczy radia, czytelników stron internetowych lub jakiejkolwiek takiej wirtualnej społeczności ucieleśnionej jest niezmiennie potężnym przeżyciem\footnote{Przypomniało mi się to na przykład w~dniach bezpośrednio po 11 września 2001 roku, kiedy aktywiści - najpierw w~Nowym Jorku, potem w~całym kraju - zamiast zrezygnować z~zaplanowanych już marszów i~protestów, w~rzeczywistości natychmiast zebrali się razem, aby zaplanować nowe. ACC w~Waszyngtonie przez kilka tygodni planowała bardzo bojową akcję przeciwko IMF i~chociaż szczegółowe plany zostały odwołane, organizatorzy wyzywająco odrzucili również prośby o odwołanie nielegalnego marszu protestacyjnego - pomimo pewności braku w~mediach i~ekstremalnym represjom policyjnym. Wielu wyraźnie mówiło, że jeśli ktoś był, powiedzmy, anarchistycznym punkiem z~Cincinnati, to musi to być niezwykle trudny i~wyobcowany czas, a obowiązkiem większej społeczności było zapewnienie ich, że nie są sami.}.

\medskip
\noindent Gramatyka Sloganów

 Niejasności między tymi trzema widowniami wydają się tutaj absolutnie niezbędne. Rozważmy na przykład nieco dziwną gramatykę zwykle stosowaną w~hasłach protestacyjnych, rodzaj nieokreślonego imperatywu: ,,Uwolnić Mumię!'', ,,Ratujmy wieloryby!'' lub ,,Stop tej rasistowskiej wojnie!''. Do kogo konkretnie są kierowane? Oczywistą odpowiedzią są ,,do tych u władzy'', w~USA prezydent, system sądowniczy, Kongres, a może bardziej ogólnie klasa rządząca. Ale niejednoznaczność gramatyczna odzwierciedla coś z~ambiwalencji samego pojęcia protestu, które wzywając władze do zmiany postępowania, jest w~efekcie uznaniem ich autorytetu, autorytetu, który wielu, jeśli nie większość protestujących, faktycznie woleliby widzieć jako z~natury nieusankcjonowany. Tak naprawdę nie chce się, żeby George Bush uratował wieloryby. Na pewno nie chciałby, żeby mógł to sobie przypisywać. Być może ktoś chciałby zmusić George'a Busha do ratowania wielorybów. Naprawdę, wolałbym, żeby George Bush w~ogóle nie istniał. Wydaje się raczej, że imperatyw działa jednocześnie na tych samych trzech poziomach: jeden wzywa władze, jeden wzywa publiczność do przyłączenia się do walki, jeden wzywa współtowarzyszy marszu do zdwojenia wysiłków; lub, może lepiej, przywołuje się jeden szeroki nurt działania, który wywodzi się od poświęcenia maszerujących, przez przekształconą lub ożywioną świadomość społeczeństwa, do niechętnych ustępstw możnych. Przywołuje się ten nurt działania, aby zaistniał. 

 Są pewne zawołania, pieśni, który imperatyw wydaje się skierowany na władzę. Ale nawet tutaj prawie zawsze występują poziomy niejednoznaczności. Istnieje na przykład cały repertuar pieśni i~haseł, które mają zawstydzić policję, zwłaszcza gdy policja tłumi pokojowe protesty. Obejmują one od ulubionego z~lat 60. ,,The Whole World Is Watching'' (często wymawianego z~gorzką ironią przez tych, którzy wiedzą, że to nieprawda), po bardziej współczesne i~wyraźnie ironiczne chórki ,,Idź walczyć z~przestępczością!''. Kiedy inny protestujący jest aresztowany, inni prawie zawsze będą gromadzić się wokół aresztujących funkcjonariuszy i~skandować ,,Pozwól im odejść! Pozwól im odejść!{\textquotedbl}. To ostatnie z~pewnością jest na pierwszy rzut oka żądaniem skierowanym do konkretnych, dających się zidentyfikować przedstawicieli władzy. Policjanci, którzy właśnie schwytali swojego towarzysza, są proszeni o otworzenie kajdanek i~uwolnienie go. Może się to również wydawać czysto ekspresyjne -- znam tylko jeden przypadek, kiedy grupa policjantów (odizolowanych, otoczonych i~w ogromnej przewadze liczebnej) faktycznie spełniła takie żądanie. Zwykle wszyscy doskonale wiedzą, że intonowanie nie zmusi policji do wypuszczenia aresztowanego. W większości przypadków skandowanie jest skierowane tak samo do aresztowanych, aby wyrazić solidarność, pokazać im, że nie są sami. Jest to tym bardziej prawdziwe, gdy te pieśni są kontynuowane podczas akcji ,,więziennej solidarności'', kiedy grupa zwolenników ustawia się po drugiej stronie ulicy od komisariatu lub więzienia i~śpiewają bez przerwy godzinami. W tym momencie nie ma oczywiście mowy o reakcji policji, w~rzeczywistości sprawy są już prawie na pewno poza zasięgiem rąk każdego, kto znajduje się w~zasięgu słuchu, ale śpiewający zawsze doskonale zdają sobie sprawę, że podnoszenie ogłuszającej wrzawy na komisariacie jest zwykle niezwykle satysfakcjonujące dla zatrzymanych, a policja uważa je za niezwykle irytujące.

\medskip
\noindent Coda: Marsze, które Stają się Akcją Bezpośrednią 

 Bez pozwolenia marsz znów staje się akcją bezpośrednią: staje się kwestią zajmowania przestrzeni publicznej wbrew prawu. Nawet z~zezwoleniem może często przybrać wiele cech akcji bezpośredniej, jeśli nastawienie policji jest wrogie.

 Na przykład w~bezpośrednim następstwie ataków z~11 września na Nowy Jork organizatorzy Światowego Forum Ekonomicznego -- grozy światowych przywódców, która zwykle odbywa się w~Davos w~Szwajcarii -- ogłosili, że w~tym roku zamierzają zmienić miejsce pobytu na Manhattan, ,,w solidarności '' z~miastem. Było to całkiem sprytne posunięcie, ponieważ w~Davos stanęli w~obliczu coraz skuteczniejszej opozycji, a teraz byliby w~mieście, w~którym policja (NYPD była wówczas szeroko przedstawiana w~mediach jako bohaterowie 911) miałaby maksymalną przewagę. Prawie wszystkie związki i~organizacje pozarządowe, które normalnie pomagały organizować takie protesty, wycofały się, więc lokalne grupy anarchistyczne -- w~tym DAN -- zostały zmuszone do samodzielnego organizowania wszystkiego, łącznie z~dozwolonym marszem. Poza zwykłymi problemami, jakie mają anarchiści, próbując działać legalnie -- takimi jak konieczność udawania hierarchicznej organizacji (na przykład każda legalna parada musi mieć Wielkiego Marszałka) -- organizatorzy odkryli, że prawie nie ma różnicy między zorganizowaniem legalnej a nielegalnej akcji, kiedy policja czuje, że jest całkowicie bezkarna. Po długich negocjacjach dowódcy policji uzgodnili trasę marszu na tydzień przed wydarzeniem; następnie arbitralnie zmienili trasę, gdy wydarzenie się zaczęło. Marsz został wstrzymany na godzinę, zanim się zaczął; w~tym czasie policja zaatakowała, spryskała gaz pieprzem i~aresztowała kilkanaście osób za noszenie tarcz. Pojawili się detektywi filmujący wszystkich na każdym kroku, a niekończący się policjanci, motocykliści i~policja konna zatrzymywała marsz losowo, aby go rozbić, lub zmuszała grupy do czekania przez arbitralne okresy w~nadziei, że się poddadzą i~znikną. Podczas samego marszu oddziały aresztujące losowo chwytały poszczególnych maszerujących lub próbowały rozpocząć większe incydenty. W rezultacie cały marsz musiał być zorganizowany jak akcja. Wszyscy byli zorganizowani w~grupy afinicji, maszerujący często musieli łączyć ramiona, byli zwiadowcy, ludzie łączności na rowerach i~tak dalej.

 Agresywna taktyka policji nie była po prostu wynikiem wrogości -- chociaż większość z~nas uważała, że  jest tego pewien element -- ale częścią wspólnej strategii, aby utrzymać liczbę osób na niskim poziomie. Marsze są, jak wspomniałem, zawsze grą liczbową. Policja zdaje sobie również sprawę z~tego, że gazety na ogół nie obliczają całkowitej liczby uczestników marszu, którzy byli na ulicach w~ciągu dnia, lecz tylko tych, którzy w~danym momencie wyszli na ulicę\footnote{Poprzednia liczba zawsze jest większa niż na ulicach w~danym momencie, ponieważ w~przypadku marszów dowolnej wielkości niektórzy uczestnicy marszu już mają dość i~wracają do domu, podczas gdy inni wciąż czekają na rozpoczęcie.}. Co więcej, jeśli gazety są nieżyczliwe -- tak jak to było w~czasie WEF -- liczą nie tyle, który wyrusza na początek marszu, ale ile, gdy dobiega do końca. Marsz, który rozpoczął się z~co najmniej dziesięcioma tysiącami, został zatem tak poprowadzony, co zredukowało ich do około trzech tysięcy, zanim dotarli do Waldorf (gdzie odbywały się spotkania), po czym New York Times i~serwisy informacyjne mogły posłusznie ogłosić, że trzy tysiące osób protestowało przed hotelem.

 Często w~przypadku zdarzeń niedozwolonych istnieje rodzaj obliczenia sił po obu stronach. Doświadczeni aktywiści szybko rozwijają wyczucie liczb. Na przykład, aby przeprowadzić masowe aresztowanie, policja zwykle będzie musiała przewyższyć liczbę aresztowanych trzy do jednego, a nawet więcej, jeśli ich cele stosują taktyki obronne, takie jak łączenie ramion. Kilka razy wylądowałem w~roli zwiadowcy, monitorując liczbę zgromadzonej policji, licząc pojazdy i~opancerzenie, liczbę rozmieszczonych autobusów więziennych i~tak dalej, aby obliczyć szanse na masowe aresztowanie i~oddzwonić z~informacjami do organizatorów. Śmigłowce unoszące się nad miastem prawdopodobnie monitorują liczbę protestujących po drugiej stronie. Zwykle jednak w~grę wchodzą nie tylko liczby. 31 lipca 2000 roku podczas protestów na konwencji republikańskiej w~Filadelfii, Kensington Welfare Rights Union zorganizował nielegalny marsz tysięcy biednych obywateli, ignorując policyjne przysięgi aresztowania wszystkich zaangażowanych. Całe wydarzenie przekształciło się w~przedłużającą się grę w~manewry. Na czele marszu organizatorzy najpierw umieścili matki z~dziećmi, aby jak najbardziej utrudnić policji atak na marsz. Policja odpowiedziała, ogłaszając, że ponieważ matki narażają swoje dzieci na niebezpieczeństwo (prawdopodobnie niebezpieczeństwo ataku ze strony policji), każda matka aresztowana z~dzieckiem byłaby zagrożona utratą opieki. To spowodowało, że większość matek wycofała się, ale szybko zastąpili je starsi i~niepełnosprawni protestujący na wózkach inwalidzkich. Policja nie interweniowała na początku marszu, ale w~połowie trasy marszu, zwiadowcy odkryli kilkuset policjantów w~furgonetkach i~może trzydzieści autobusów więziennych. Plotki krążyły we wszystkich kierunkach i~w pewnym momencie ktoś monitorujący radio policyjne podał, że wydano rozkaz aresztowania, ale sama liczba maszerujących -- co najmniej dziesięć tysięcy -- sprawiła, że nawet wysiłek odcięcia i~zatrzymania głównych maszerujących był prawie niemożliwy. W końcu wszyscy maszerujący przeszli.

 Obecność dużej liczby osób w~takich sytuacjach jest krytyczna i~zazwyczaj radykalnie ogranicza możliwości policji. Podczas protestów wokół inauguracji Busha (J20), sześć miesięcy później, policja Waszyngtonu odcięła i~była w~trakcie masowego aresztowania około dwustu maszerujących, którzy byli częścią Czarnego Bloku, jedynej grupy, która była zdeterminowana, by angażować się w~bezpośrednie działania. Grupa została otoczona przez policję prewencji na głównym skrzyżowaniu; przywołano posiłki, zbudowano drugą linię policji, aby uniemożliwić innym aktywistom masowe poparcie. Pojawiły się autobusy więzienne, a szef policji już przyjechał nadzorować. Ci w~kręgu szybko zorganizowali się w~defensywie, łącząc ramiona i~tworząc krąg, ale sytuacja wyglądała ponuro (sam uciekłem alejką z~niektórymi z~Radical Cheerleaders i~próbowałem sprowadzić wsparcie), gdy nagle na scenie pojawiło się kilka tysięcy maszerujących ze znakami i~transparentami. Okazuje się, że byli to, co nazywaliśmy wtedy ,,wściekłymi demokratami'', masa jednostek zorganizowana przez{\textless}f1{\textgreater}Moveon.com{\textless}/f1{\textgreater}, którzy szli dozwoloną trasą marszu w~kierunku trasy inauguracyjnej parady.

 Jednak gdy tylko przybyli, sytuacja taktyczna całkowicie się zmieniła. Przywódcom marszu polecono zboczyć. Odmówili. Trzytysięczna masa naciskała teraz na cienkie zewnętrzne linie policyjne; Z zaułków wyłaniali się Black Blockers, aby poinstruować maszerujących, jak połączyć się i~powoli pokonywać barykady lub linie policji. Ostatecznie policja odwołała cały projekt i~po dwóch lub trzech symbolicznych aresztowaniach wycofała się. Przy tej okazji aktywiści powszechnie rozumieli, że przy tak dużej liczbie policji Waszyngtonu rozmieszczonej na trasie parady, jedynym sposobem na sprowadzenie wystarczającej liczby z~nich do przeprowadzenia masowego aresztowania byłoby wezwanie Gwardii Narodowej (jednostki były rozmieszczone poza miastem) i~policyjni dowódcy zdecydowali, że nie chcą zapisać się w~podręcznikach historii tym, że, aby doprowadzić do inauguracji Busha przy kwestionowanej eleksji, konieczne było wezwanie Gwardii Narodowej.

 Myślę, że te przykłady są przydatne, aby pokazać złożoną mieszankę taktycznych i~politycznych kalkulacji, które definiują równowagę sił; rachunek, który nigdy nie jest do końca zrozumiały, jak sądzę, przez żadną ze stron, ponieważ ani aktywiści, ani urzędnicy nigdy w~pełni nie rozumieją tego, co myśli druga strona. Takie rozważania staną się niezwykle ważne później, gdy spróbujemy zrozumieć, jaką skuteczność polityczną mają takie działania. Na razie zakończę po prostu podkreśleniem, że w~przeciwieństwie do samoorganizujących się grup typowych dla akcji bezpośredniej, tłumy maszerujących mają ogromną bezwładność i~są niezwykle trudne do poruszania. Wielu doświadczonych działaczy zakłada, że  zawrócenie kilka tysięcy maszerujących, powiedzmy, skłonienie tłumu maszerującego w~jednym kierunku do tyłu i~wzdłuż równoległej ulicy, aby uniknąć policyjnej blokady drogowej, jest po prostu niemożliwe. Tak nie jest (zrobiłem to sam), ale potrzeba niewielkiej garstki ludzi, którzy chcą działać jako zdecydowani liderzy: na przykład poprzez znalezienie bardzo wysokiego obiektu, takiego jak gigantyczna marionetka, najlepiej w~towarzystwie muzyków, ustawienie go w~we właściwym kierunku, przekonując kilka dużych grup ludzi, by poszli za nimi, a następnie rozbiegając się tak szybko, jak to możliwe, by powiedzieć wszystkim, że marsz zmierza teraz w~innym kierunku, jakby to był fakt dokonany. Innymi słowy, w~samej dynamice niezróżnicowanego tłumu jest coś, co w~mniejszym lub większym stopniu wymaga, aby niektóre jednostki przejęły rolę przywódczą.

 Sugerowałbym nawet, że jeśli ktoś chce zrozumieć różnicę między staromodną lewicową koncepcją ,,mas'' a nowszym pojęciem ,,wielości'', najlepiej byłoby rozważyć różnicę między niezorganizowanym tłumem -- masą niezróżnicowanych jednostek, ulegającej wszystkim plotkom, panikom i~namiętnościom tak bez końca dokumentowanym przez psychologów tłumu (Le Bon 1921; Canetti 1962; itd.) -- oraz samoorganizującym się tłumem prowadzący masową akcję. Ten ostatni składa się jednocześnie z~niekończących się grup afinicji przypominających komórki, ale poprzecinanych sieciami jednostek komunikacyjnych, medyków, performerów, obserwatorów prawnych, grup wsparcia i~łączników z~mediami, w~zależności od stopnia ryzyka, jakie są skłonni znosić i~poziomu szkolenia lub przygotowania. Grupy te zazwyczaj same są następnie zorganizowane w~,,plastry'' i~,,klastry'', a jednocześnie każda komórka podkreśla tylko jeden szczególny aspekt wielorakich tożsamości politycznych jako podstawę pokrewieństwa dla tej konkretnej akcji: queerowych aktywistów z~Cleveland, autonomicznych marksistów, pogan, Wobblies, punkowców z~LA i~obrońców praw zwierząt z~New Jersey. Jest bardzo niewiele sposobów na to, by ,,masa'' i~,,wielkość'' były takie same.

\medskip
\noindent DRUGI PRZYKŁAD: LINIA PIKIETY
\medskip

 W przeciwieństwie do większości marszów i~wieców, linie pikiet są bezpośrednio konfrontacyjne i~wszyscy zaangażowani wiedzą dokładnie, z~kim mają do czynienia. Ostatecznym celem jest oczywiście kierownictwo, właściciele lub dyrektorzy uderzonego przedsiębiorstwa. Bezpośrednim celem z~krwi i~kości jest każdy, kto grozi przekroczeniem linii pikiet: klienci, goście lub łamistrajkowie. Do przeciwników może również należeć policja, detektywi lub prywatna ochrona wynajęta \hspace{0pt}przez pracodawców i~inne osoby, które próbują zakłócić linię lub zapewnić, że ci, którzy chcą przekroczyć, mogą to zrobić, a czasami nawet kontratakerzy zgromadzeni przez drugą stronę. W tym sensie ,,publiczność'' i~,,cel'' w~dużej mierze nakładają się na siebie, w~rzeczywistości próbuje się przekonać identyfikowalnych członków społeczeństwa, aby czegoś nie robili, choć w~szerszym sensie gra się również przed szerszą ,,publicznością'', w~tym, potencjalnie, elementom środowiska prawnego i~politycznego, na które opinia publiczna może mieć wpływ, aby wstawić się w~imieniu strajkujących. (W tym drugim znaczeniu strajk rzuca cień na protest.)

 Jak zauważyłem w~rozdziale 5, sytuacja ta wyłania się z~wyjątkowej sytuacji historycznej: linie pikiet rozpoczęły się jako bardzo radykalna forma akcji bezpośredniej, o ile wiązały się z~bezpośrednim zagrożeniem siłą fizyczną -- co prawda, jako jednym z~wielu narzędzi -- by trzymać łamistrajków z~daleka. Na początku XX wieku praktyka ta została zalegalizowana, stając się być może jedyną formą akcji bezpośredniej, na jaką rząd USA zezwala, nawet jeśli tylko niektórym, prawnie określonym grupom, ale jednocześnie stała się przedmiotem szczegółowych regulacji prawnych. Zabroniono użycia siły fizycznej jako niedopuszczalnego naruszenia monopolu państwa na przemoc. Ale ten sam państwowy monopol na przemoc został -- w~bardzo ograniczony sposób -- zastąpiony, zaczął być stosowany w~imieniu samych strajkujących: jak na przykład, kiedy rząd poparł wyniki wiążącego arbitrażu z~mocą prawa. Innymi słowy, policja nie była całkowicie po drugiej stronie. W rezultacie linie pikiet stały się fascynującym połączeniem bojowości, teatru i~skrupulatnej legalności.

 Pozwólcie, że zaczerpnę swój przykład z~kampanii, w~którą DAN była blisko zaangażowana, strajku pracowników Muzeum Sztuki Nowoczesnej w~środkowym Manhattanie, który zawsze nazywaliśmy ,,strajkiem w~MoMA''. Był to strajk w~sprawie negocjacji kontraktowych przez około 175 członków Professional and Administrative Staff Association (PASTA -- Stowarzyszenia Pracowników Zawodowych i~Administracyjnych -- czyli Komórka nr 2110 United Auto Workers) -- reprezentujących kuratorów, artystów, sekretarki, pracowników księgarni i~bibliotekarzy. (Pozostałych czterystu pracowników muzeum reprezentowało pięć innych związków, które nie strajkowały.) Okazało się, że był to najdłuższy strajk w~historii MOMA, trwający 134 dni, od 28 kwietnia do 12 września 2000 roku; kiedy się skończył, większość pracowników uważała go za dramatyczne zwycięstwo. DAN Labor odegrała ważną rolę we wspieraniu rozwoju strategii związku; częściowo, ponieważ jeden z~aktywnych członków DAN Labor, Malcolm, był związany rodzinnie, będąc synem mężczyzny, który później poślubił jedną z~głównych organizatorek UAW.

 To była linia, którą Jordan z~dumą nazwał w~ostatnim rozdziale ,,najbardziej hałaśliwą linią pikiet w~Nowym Jorku''. Pikieterzy wychodzili codziennie, ale czwartki przeznaczono na ważne akcje. Poniżej znajduje się jedna z~nich, akcja zbudowana wokół zakłócającej piłki rzucanej wewnątrz muzeum na osiemdziesiąte piąte urodziny Davida Rockefellera. Ten fragment daje również ładną ilustrację roli DAN Labor, grupy aktywistów, którzy, nie będąc członkami związku, nie byli prawnie zobligowani do nieangażowania się w~akcje (na przykład wtórne bojkoty i~pikiety), co było surowo zabronione członkom związku przez National Labor Relations Board.

\medskip
\noindent \textit{Pikieta MOMA, Midtown Manhattan}
 
\medskip
\noindent [Notatki terenowe, czwartek, 14 czerwca 2000 roku]
\medskip

 Załoga DAN zwykle dostarczała gigantyczne marionetki, postacie męczenników z~Haymarket, których uratowaliśmy z~wcześniejszego marszu Mayday. Trzymano je na czyimś dachu w~Soho, ale tym razem okazało się, że facet z~kluczykiem się spóźnił, a może był to facet z~autem, nie pamiętam. W każdym razie marionetki nigdy nie dotarły i~wszyscy byli rozczarowani.

 DAN Labor przybywają stopniowo metrem. Kiedy docieram na miejsce około 20:00, teren przed muzeum jest już całkowicie zabarykadowany i~kontrolowany przez gliniarzy, którzy jednak nie byli specjalnie w~dużej liczbie. Może z~tuzin wszystkich, jak mi powiedziano. Policja ustawiła dla nas zagrody, które znajdowały się po drugiej stronie ulicy od wejścia do muzeum, wokół których ustawiono kilka umundurowanych patroli. Więc nie możemy nawet wykonać gestu w~kierunku zablokowania wejścia, po prostu spróbujmy zakłócić sytuację i~ujawnić naszą obecność.

 Wśród pikietujących w~zagrodach -- i~po tej stronie ulicy, również w~dużej mierze przepełnionych -- znajdują się liczne grupy wsparcia z~innych związków, w~tym Teamsters, SEIU, inni lokalni UAW. Ludzie SEIU mają fioletowe marynarki i~transparenty, UAW, czerwone flagi. Są tam przeważnie w~małych, trzy- lub czteroosobowych grupach. Była też niewielka delegacja z~jakiejś żydowskiej grupy robotniczej. Prawie każdy w~zagrodzie ma przyklejoną jakąś naklejkę: na obu końcach zagrody znajdują się duże kartonowe pudła pełne naklejek, ulotek i~innych materiałów, które każdy może wziąć, w~tym czarną, poświęconą specjalnie dla pracowników MOMA na strajku, okrągła czerwona naklejka z~napisem KONTRAKT TERAZ!, niebieskie z~napisem OPIEKA ZDROWOTNA JEST PRAWEM CZŁOWIEKA, a druga z~napisem SZTUKA WSPÓŁCZESNA, STAROŻYTNA PŁACA. Napastnicy są w~bardzo różnorodnych strojach, w~zakresie od związkowych kurtek do kilku w~czarnych krawatach i~innych eleganckich strojach, przypuszczalnie, żeby zrobić wrażenie na imprezowiczach, którym są rozdawane ulotki na zewnątrz.

 Kartonowe pudła tworzą swego rodzaju stanowisko dowodzenia: oprócz naklejek jest duże pudło z~napisami i~materiałami do tworzenia znaków, wiele profesjonalnie wydrukowanych komiksów i~małych napisów (NIE POWIERZAJMY NICZEGO POWIERNIKOM).

 Nasze łączne liczby wydają się mieścić się w~przedziale 200-250. Był jednak zbyt płynnie, żeby mieć pewność.

 Na ulicy spotykam starego przyjaciela ze studiów, który okazuje się kuzynem jednego z~organizatorów strajku. Następnie Malcolm, jeden z~głównych organizatorów i~twórców lalek DAN Labor, który, jak odkrywam, okazuje się również artystą: rzeczywiście jest w~klasycznym kostiumie nowojorskiego artysty, z~niezwykle krótkimi włosami, bluzą, spodniami w~stylu wojskowym, paląc papierosy bez filtra. Jego ojciec, który jest również zaangażowany w~DAN Labor, jest dawnym organizatorem pracy, brodatym z~drewnianą laską, który tłumaczy, że jego syn pracuje głównie w~olejach. Odnajdujemy resztę członków DAN (Andrew, Todd, Jordon, Nicole, Zack, Beverly, Siobhan) w~sumie około sześciu lub siedmiu osób. Zack rozdaje zatyczki do uszu, które przyniósł, co w~rzeczywistości okazują się całkiem przydatne, ponieważ częścią całego sensu pikiety jest zrobienie jak największego hałasu, a skoro byliśmy umieszczeni w~zagrodach całkiem daleko od wejścia muzeum, ludzie starają się wyjątkowo, żeby zwiedzający nas usłyszeli.

 Rytm bębnienia jest powolny, hipnotyzujący i~niezwykle monotonny: raz dwa trzy, pauza, raz dwa trzy, pauza, raz dwa trzy, w~kółko godzinami. To się nie zmienia. Czasami ktoś przychodzi i~oferuje dzwonki i~gwizdki, dosłownie krowie dzwonki i~małe blaszane gwizdki, każdy ze sznurkami do zawieszenia na szyi. Tak więc trzytaktowemu rytmowi, wybijanemu głównie na plastikowych bębnach wiaderkowych, towarzyszył gwizd, albo przypadkowy, albo dołączony do rytmu, często potężny gwizd tylko na ostatniej części. Jedna kobieta miała trąbkę powietrzną. Niektórzy strajkujący mieli kazoo, jedna kobieta z~DAN przyniosła flet.

 W przeciwieństwie do pięknych, skomplikowanych rytmów preferowanych w~akcjach masowych, ten cel nie miał być muzyką. To był hałas. A może\ldots  kiedy obserwowałem, jak wybijają w~kółko ten sam trzyczęściowy rytm, słowo, które ciągle pojawiało się w~mojej głowie, brzmiało ,,charivari'', ,,Rough music'', czy nie tak to nazywali w~XVI wieku, kiedy młodzi ludzie z~miasteczka walili w~bębny, garnki i~patelnie, wyśmiewając prominentnych obywateli, którzy z~jakiegoś powodu ich zdenerwowali, przedstawiali na ich temat satyryczne skecze, wygłaszali przemówienia? Charivari był rodzajem elementarnej -- a nawet podstawowej -- formy protestu. i~tak musiało być.

 Ktoś pyta: ,,Gdzie Szczur?''

 Kiedy byliśmy tu ostatnim razem, obok zagrody stał gigantyczny, dwupiętrowy nadmuchiwany szczur -- największy z~trzech, które pojawiały się na liniach pikiet wokół Manhattanu.

-- Na dużym zepsuł się generator, więc odesłaliśmy go z~powrotem. A inne już zostały zabrane.

\medskip
\noindent Z Tyłu

 Na początku w~pobliżu nie ma żadnych łamistrajków, od czasu do czasu słychać okrzyk, gdy ktoś zostanie zauważony, a także okrzyki ,,Wstyd! Wstyd!'' i~,,Nie wchodź!''. Jeden facet ma tabliczkę z~napisem ,,Honk!'' brnie w~ruchu uliczny za każdym razem, gdy przejeżdża jakaś liczba samochodów, a wielu faktycznie trąbi klaksonem, że w~pewnym momencie z~samochodu musi wyjść dwóch gliniarzy, aby ostrzec kierowców, by tego nie robili.

 Jordan i~kilkoro innych wpada na pomysł, żeby obejść cały blok i~zobaczyć, czy da się coś zrobić od Pięćdziesiątej Czwartej Ulicy, gdzie znajduje się wejście do ogrodu rzeźb, gdzie miało się odbyć przyjęcie. Zbieramy większość ludzi z~DAN, z~wyjątkiem jednego, który mówi, że jest umówiony i~nie chce ryzykować aresztowania. Jordan mówi, że został już rozpoznany i~wszyscy ubolewamy nad brakiem marionetek: gdybyśmy mieli taką, łatwo byłoby poprowadzić z~nami całą masę ludzi na taką wyprawę. Możesz oprowadzać całe tłumy, jeśli masz marionetkę. Więc zamiast tego wychodzimy sami, około siedmiorga z~nas, pięcioro z~DAN i~dwóch wolontariuszy z~linii pikiet oraz spory zapas ulotek.

 W pobliżu bramy znajduje się policyjna barykada, niestrzeżona -- w~okolicy jest kilku gliniarzy, ale nie za nimi -- a poza tym piesi wyraźnie są w~stanie przejść, więc rozumiemy, że barykady były przeznaczone tylko dla prawdziwych strajkujących. Przechodzimy obok nich do bramy. Ktoś konsultuje się z~gliniarzem w~białej koszuli -- najwyraźniej oficerem wyższego szczebla tutaj -- który twierdzi, że myślał, że nie wolno nam iść za barykady, ale nie jest tego pewien. Wychodzi, by sprawdzić u swoich przełożonych. Za barykadą stoi furgonetka pełna sprzętu nagłaśniającego na przyjęcie. Na całej ulicy porozrzucane są głośniki i~pudła rezonansowe, które mają zostać zabrane do ogrodu rzeźb. Rozmawiamy z~dwoma facetami z~zespołu, którzy byli całkowicie współczujący (,,Jesteście ze związku?'' -- nawet nie mieli pojęcia, że  trwa strajk). Czy moglibyśmy umieścić nasze naklejki na sprzęcie, pytamy. Tak, jasne, powiedzieli. Dlaczego nie? Andrew, w~marynarce pokrytej naklejkami i~odwróconym berecie, próbuje już hiszpańskiego na pracownikach meksykańskich restauracji przy bocznych drzwiach kuchni. Większość z~nas zapomniała zabrać ze sobą naklejki; pobiegłem i~wróciłem dziesięć minut później z~całą ich partią, ale do tego czasu załoga DAN wracała już w~grupce.

\noindent Ja: Ale właśnie dostałem te wszystkie naklejki!

\noindent Nicole: Gliniarze zmusili nas do odejścia. Prawie aresztowali Jordana.

\noindent Ja: Za co?

\noindent Jordan: Za plucie.

 Najwyraźniej biała koszula w~końcu dostała rozkazy i~mieli się nas pozbyć.

 Imprezowicze zaczynają przybywać.

 Więc przegrupowujemy się, Nicole i~ja, a później Mike, przy wschodnim wejściu do drzwi wejściowych na Pięćdziesiątej Trzeciej Ulicy. Słychać stąd hałas z~drugiej strony ulicy i~rytm jeden dwa trzy, ale nie ma sposobu, aby rozróżnić pieśni ani mieć pojęcia, jakie ma przesłanie. Związek sprytnie umieścił młodego mężczyznę w~smokingu i~dwie kobiety w~sukniach wieczorowych (jedna niosła trzy róże) tuż za linią policji, czekając na powitanie imprezowiczów, którzy przybyli od strony Piątej Alei, rozdając kartki papieru, które wyglądały, jakby były programami, ale w~rzeczywistości były ulotkami związkowymi. Za drewnianą barykadą stoi trzech gliniarzy, którzy nie przeszkadzają, ale też nie wchodzą w~interakcję. Ograniczają się do wpychania gości przez kolejkę i~pod drzwi muzeum. ( -- A w~jaki sposób wiedzą, którzy to goście? -- pyta Nicole, choć odpowiedź jest całkiem oczywista.)

 Bogaci ludzie pojawiają się głównie w~parach lub grupach po cztery lub pięć osób, mężczyźni w~czarnych krawatach. Kobiety przeważnie w~jasnych wiosennych kolorach. Kilkoro przyjeżdża limuzynami; większość przybywa na piechotę. Przychodzą w~każdym wieku, z~wyjątkiem małych dzieci. Reakcje na pikietystów znacznie się różnią. Większość mężczyzn przybiera niewzruszoną kamienną twarz i~próbuje mężnie przejść obok nas, patrząc prosto przed siebie; niektórzy są zdenerwowani lub rozgniewani; niektóre kobiety wyglądają na przestraszone lub zdenerwowane. Inni zachowują się jak podchmieleni lub chichocząc. Jeden lub dwa są wyraźnie na czymś. Próbujemy wygłupiać się z~policjantami: 
 
 -- Aresztuj tę kobietę, oficerze! Najwyraźniej jest na prochach!

 Andrew wkrótce staje się gwiazdą widowiska, z~arsenałem naklejek i~pozornym całkowitym brakiem wstydu i~zahamowań: kilka razy podchodzi prosto do jednego z~gości, udając, że go wita, klepie po plecach, a tym samym wpłaca czarną naklejkę STRAJK PRACOWNIKÓW MOMA na plecach ich smokingu. Tymczasem Nicole i~ja dołączamy do czterech lub pięciu innych związkowców, którzy próbują apelować do sumienia: ,,Proszę, nie przekraczaj naszej linii pikiet!'', ,,Wstyd, wstyd'', ,,Nie wchodź!'' itp.

 Scena sprawia, że  Nicole jest trochę przygnębiona.
 
 -- Jak ludzie mogą \textit{to }robić? -- pyta mnie, po tym, jak przeszło obok dwóch milionerów o stalowych twarzach, odmawiając wzięcia ulotek lub spojrzenia nikomu w~oczy. -- Jak mogą po prostu przejść obok linii pikiet?

-- Może dają im lekcje znieczulające w~szkole biznesu -- wzruszam ramionami.

 Jordan krzywi się. 
 
 -- Bo to szumowiny.

 Właściwie ta wymiana skłoniła mnie do zastanowienia się, w~jakim stopniu sami moglibyśmy być postrzegani jako wrodzy lub groźni, co w~rzeczywistości ułatwiło gościom przejście i~poczuliśmy się bohatersko z~tego powodu. Dlatego staram się odpowiednio dostosować moją retorykę (,,Proszę, nie rób tego'', ,,Naprawdę nie jest zbyt miło przekraczać linię pikiet'') . Nicole jest o wiele ostrzejsza (jest złym gliną, jak później zwracam uwagę wesołemu białobrodemu związkowcowi, który w~końcu wpada do nas, żeby do nas dołączyć. 
 
 -- Ojej -- zauważa -- masz tu całą małą grupę, co?'' 

 Rozdaję kilka ulotek i~oklejam limuzynę, aż podjeżdża do nas biała koszula, aby ogłosić, że następna osoba, która została przyłapana na próbie naklejenia kogokolwiek lub czegokolwiek, pójdzie do więzienia. Szybko krąży plotka, że się łamią, i~nawet Andrew niechętnie przerywa. Niemal natychmiast potem wśród przybyłych gości pojawia się para po pięćdziesiątce w~strojach wieczorowych, witana ciepłymi uściskami ze strony związkowców. Mąż, jak się okazuje, jest znanym właścicielem galerii, który zarówno podarował muzeum wiele obrazów, jak i~był znanym zwolennikiem związku. Bierze stos ulotek i~obiecuje rozprowadzić je wewnątrz budynku; jego żona bierze stos naklejek i~przyczepia kilka do paska swojej wieczorowej sukni.

 Mike, kolejna osoba z~DAN, pojawia się i~zaczyna przejmować stanowisko Nicole. (Jego podejście jest bardziej komiczne niż konfrontacyjne: ,,Przekraczanie linii pikiet prowadzi do łysienia!'' ,,Hej, postawię ci obiad za sto dolarów!''), aż w~końcu pojawia się jeden z~dwóch kapitanów związkowych, którzy rzekomo odpowiadają za demo, by nas odstraszyć.

-- Chcemy tylko członków związku przy drzwiach -- mówi.

 Więc idziemy stamtąd.

-- Jezu, co ją ugryzło? -- pyta jeden z~ludzi DAN, kiedy wracamy na drugą stronę ulicy. -- Mogła przynajmniej wyrazić uznanie.

 Mike uważa, że  istnieje napięcie między nią a innym kapitanem, który na początku sprowadził ludzi z~DAN, nad tym, kto ma kontrolę. Ktoś inny mówi, że najwyraźniej jest teraz zaangażowana w~ciężkie negocjacje z~policją i~jest pod dużą presją. Pewnie coś się działo, bo kilka minut później szwadron ośmiu policjantów przemaszerował ulicą obok nas w~szyku i~zajął pozycje pod murem, dokładnie tam, gdzie my byliśmy.

 Po ponad godzinie wszyscy imprezowicze są już w~środku. Zamieszanie trwa. Zostaję jeszcze chwilę, wystarczająco długo, by przywitać się z~dwoma dźwiękowcami z~furgonetki, którzy przyszli złożyć hołd pikietującym. Układam plany na następny raz i~wreszcie wracam metrem do domu.

 Kilka rzeczy się tutaj wyróżnia. Jednym z~nich jest to, że, jak wspomniano powyżej, pikiety pociągają za sobą bezpośrednią konfrontację. Podczas gdy w~ogólnym proteście często nie jest jasne, do kogo jest adresowany, tutaj to, co się dzieje, jest znacznie mniej dwuznaczne. Bez względu na to, jak ważne jest doświadczenie solidarności, głównym punktem jest oddziaływanie na (konkretnych, możliwych do zidentyfikowania) innych. W większości marszów i~akcji jest muzyka, która dodaje energii i~podnosi na duchu maszerujących; tutaj było słychać hałaśliwy hałas, który ma drażnić drugą stronę. A pytanie operacyjne rzeczywiście brzmi: ,,Po której jesteś stronie?''. Apeluje się do przechodniów o wyrażenie poparcia, czy to aktywnie (np. trąbiąc), albo przynajmniej biernie, nie przekraczając linii pikiet. Ci, którzy ją przekraczają, oświadczają, że w~oczach pikietujących naruszyli elementarną zasadę moralności. Stąd zamieszanie Nicole: radzimy sobie z~różnym rozumieniem moralności tak fundamentalnym, że nawet trudno sobie wyobrazić co musi się dziać w~głowie przekraczającego pikietę.

 To kolejny punkt: jest to w~dużej mierze kwestia walki moralnej. Był czas, jak wspomniałem, kiedy pikietowanie było tylko jedną z~form bezpośredniej akcji stosowanej w~ruchu robotniczym i~uważano ją za bardzo bojową, ponieważ często polegała na skutecznym obleganiu miejsca pracy i~fizycznym przetrzymywaniu innych. Kiedy pikiety stały się prawnie dopuszczalną formą protestu na początku XX wieku, wszystko to zostało wykluczone. Stały się apelem do sumienia. Ale jednocześnie zachowały swój bojowy styl. Pokaz siły nie jest już środkiem fizycznego zastraszania; staje się sposobem na wyrażenie tego, jak bardzo strajkujący czują, że przekroczyliby granice zwykłej przyzwoitości, gdyby zdecydowali się przekroczyć linię pikiet; jak nikczemnym stworzeniem by się wtedy stało.

 Dzieje się tak nawet wtedy, gdy w~ogóle nie próbuje się trzymać ludzi z~daleka. Nikt tak naprawdę nie spodziewał się, że któryś z~przyjaciół i~sympatyków Davida Rockefellera zobaczy pikietę na jego przyjęciu urodzinowym, poczuje wyrzuty sumienia i~zdecyduje się nie uczestniczyć. Kiedy pojawił się prozwiązkowy bankiet, nikt nawet nie zaproponował, żeby zawrócił. Chodziło o to, aby inni bankierzy poczuli się nieswojo: wiedzieć, po której są stronie i~jak ludzie po drugiej stronie się z~tym czuli. i~być może, aby wykorzystać ich wpływ, aby zachęcić muzeum do ustępstw i~sprawić, by ta irytacja zniknęła.

 Szczególnie interesująca jest tu rola policji. W zasadzie, oczywiście, policja działa jako bezstronni przedstawiciele państwa, państwa, które traktuje wszystkich zaangażowanych na równi jako swoich obywateli. Tak więc w~zasadzie są jedynymi aktorami, którzy nie są po żadnej ze stron, ale po prostu starają się utrzymać spokój i~ulice otwarte. Oczywiście nie są oni tak naprawdę neutralni: kiedy dochodzi do nacisku (czasem dosłownie), głównym zadaniem policji jest ochrona praw własności właścicieli i~zapewnienie, że goście Rockefellerów mogą wejść do budynku bez przeszkód, nawet do tego stopnia ostrzegając kierowców, aby nie trąbili ze współczuciem. Mimo to, jako jednostki, ich sympatie prawdopodobnie będą skłaniać się w~stronę strajkujących. To prawda, indywidualne sympatie funkcjonariusza policji są w~dużej mierze nieistotne, jeśli otrzymuje on bezpośrednie rozkazy rozprawienia się; ale nawet tutaj, jeśli ktoś zajrzy do policyjnych podręczników radzenia sobie z~,,zamieszkami obywatelskimi'', okaże się, że wytyczne dotyczące pikiet są znacznie bardziej hojne i~znacznie mniej konfrontacyjne niż w~przypadku jakiejkolwiek innej formy demonstracji. Ta rozlużnienie znacznie ułatwia napastnikom postrzeganie policji, jeśli nie jako przyjaciół, to przynajmniej nie tylko jako wrogów, nawet jeśli oznacza to również, że w~rezultacie często stają się mylącymi, a nawet frustrującymi stworzeniami\footnote{Często widziałem, jak policja na liniach pikiet jest wyjątkowo zobojętniała nawet na oskarżenia o sabotowanie ciężarówek i~podobne ataki na mienie. Z drugiej strony ta sama policja, jeśli zostanie przekupiona, może nagle stać się wyjątkowo represyjna.}.

 Interesujące jest obserwowanie tej ambiwalencji, ponieważ będzie ona pojawiać się w~kółko. Ma to zasadnicze znaczenie dla charakteru policji. Policja z~definicji to urzędnicy państwowi. Jeśli ktoś podejdzie do policjanta z~pytaniem -- zapyta go, jak dostać się do zoo, gdzie znajduje się najbliższy oddział Departamentu Pojazdów Samochodowych, a nawet o znaczenie jakiegoś insygnia lub akronimu na jego kurtce -- oczekuje się od nich dostarczenia informacji. To część ich pracy, ich odpowiedzialności wobec społeczeństwa, po to, by służyć i~chronić. Strajkujący -- i~dotyczy to oczywiście protestujących w~ogóle -- są z~definicji członkami tej opinii publicznej. W zasadzie, jeśli protestujący podchodzi do policjanta i~pyta o drogę lub pomoc medyczną, albo skarży się, że ktoś go pobił, policja ma pomóc. i~w pewnych okolicznościach rzeczywiście to zrobią. Jednocześnie strajkujący lub protestujący mogą, w~oczach policji, w~każdej chwili natychmiast zmienić się we wrogów porządku publicznego, a tym samym stać się obiektem gróźb lub faktycznej napaści fizycznej. Z punktu widzenia protestującego rezultatem jest ciągłe przełączanie rejestrów. W pewnym momencie możesz rozmawiać lub żartować z~oficerem stojącym obok ciebie, mając do czynienia z~nim indywidualnie. i~może być przyjacielski lub nieprzyjazny, w~zależności od jego osobistych uczuć. Możesz też mieć do czynienia z~nim jako z~urzędnikiem państwowym, to znaczy jako osoba zobowiązana do pomocy, do dostarczania ci informacji lub usług. Ale wtedy, często bez ostrzeżenia, mogą natychmiast zmienić się w~beznamiętną twarz przemocy państwowej, na przykład, kiedy nakazano im zaatakować ludzi, z~którymi właśnie rozmawiali, gazem pieprzowym lub gazem łzawiącym lub zastraszyć ich agresywnym użyciem pałek. Jest to tym bardziej skomplikowane, że zazwyczaj struktury dowodzenia policji są dość luźne i~sporadyczne. Często podczas strajków lub akcji, jeśli wydarzy się coś nieoczekiwanego, policja nie jest pewna zasad, których musi przestrzegać i~czeka na rozkazy. Dosłownie muszą stać z~boku, dopóki nie powiedzą im, czy osoba stojąca przed nimi i~ewentualnie próbująca z~nią porozmawiać, ma być chroniona przed przemocą, czy też może nią grozić. To nieustanne przesuwanie się i~nieuchronność nieustannej interakcji w~sytuacjach skrajnych niejasności prowadzi do żartów, prób uczłowieczenia policji, ale także do poczucia zdrady, które niemal niezmiennie nastąpi, gdy policja w~końcu otrzyma rozkazy, lub hierarchia się utwierdzi (jak wtedy, gdy rozkazy przyszły, żeby usunąć nas sprzed wejścia do muzealnego ogrodu, lub kiedy kierujący ze związku zacznie negocjować z~białymi koszulami, żeby oczyścić przejście, którym imprezowicze przechodzą).

 Mój ostatni punkt dotyczący linii pikiet dotyczy zasad zaangażowania. Jedną z~konsekwencji prawnego uznania linii pikiet jest to, że każdy jest wyraźnie świadomy tego, co może, a czego nie może zrobić. Szare obszary istnieją, ale nie aż tak dużo, jak w~dozwolonych marszach, nie mówiąc już o bezpośrednich akcjach, gdzie nie ma nic nawet przypominającego księgę zasad, z~którą obie strony mogą się zapoznać. Wielu w~DAN było przekonanych, że właśnie to sprawia, że  organizatorzy związkowi tak ciężko pracują, by trzymać swoich ludzi z~dala od akcji podczas masowych mobilizacji, takich jak Seattle czy A16. Związki zawodowe, jako jedyne organizacje w~Ameryce, którym faktycznie pozwolono angażować się w~akcje bezpośrednie, również poddały się szczegółowym regulacjom dotyczącym tego, jak mogą to robić, a pozwolenie członkom związku -- szczególnie niosącym związkowe sztandary lub insygni -- na wejście nieokreśloną i~z natury nielegalną sytuację, może narazić wszystko na szwank.

\medskip
\noindent PRZYKŁAD TRZECI: STREET PARTY
\medskip

 Akcja bezpośrednia ma na celu skonfrontowanie się z~tym, co postrzega jako niesprawiedliwą lub bezprawną formę władzy w~sposób, który w~swojej wewnętrznej strukturze sugeruje realną alternatywę dla niej. Właściwie to samo prawdopodobnie w~pewnym sensie dotyczy protestu w~ogóle. W rezultacie, jak widzieliśmy, często trudno jest określić, o ile dane działanie powinno być postrzegane przede wszystkim jako przedstawienie, które ma wywrzeć wrażenie na zewnętrznej publiczności, a o ile lepiej jest postrzegane jako zbiorowy rytuał przeznaczony do edukowania, inspirowania, bawienia i~przekształcania poczucie możliwości samych uczestników. Z pewnością zawsze jest trochę tego i~tego. Ale niektóre rodzaje działań skłaniają się znacznie bardziej ku jednemu niż drugiemu.

 Jeśli linie pikiet skłaniają się bardzo w~pierwszym kierunku -- dotyczą przede wszystkim przekazywania wiadomości o buncie konkretnym przeciwnikom i~przekonania szerszej publiczności do solidarności -- można by powiedzieć, imprezy uliczne reprezentują przeciwną skrajność. Chociaż z~pewnością mają na celu wygłoszenie oświadczenia politycznego lub osiągnięcie politycznego celu, mają również na celu zapewnienie uczestnikom wszelkich możliwych możliwości czerpania przyjemności. Ten element przyjemności -- przede wszystkim zbiorowej, społecznej przyjemności -- jest tak naprawdę głównym celem. Nawet jeśli biorący udział w~ulicznej imprezie starają się zaimponować publiczności, robią to w~taki sposób, aby zacierać granice, a nie rysować granice na piasku. Widzom proponuje się przedstawienie z~muzyką, żonglerami i~klaunami. Ideałem jest, aby przyjemność z~doświadczenia stała się zaraźliwa, żeby widownia spontanicznie wciągnęła się i~mentalnie, albo lepiej fizycznie, dołączyła do festiwalu.

 Ten element przyjemności jest uważany za kluczowy dla tego, co czyni nowe formy protestu nowymi, prawie tak samo jak zasady samoorganizacji i~autonomii. Wielkie akcje zawsze są przedstawiane jako ,,festiwale oporu'' lub ,,karnawał przeciwko kapitalizmowi'', a ich organizatorzy zawsze wyraźnie przeciwstawiają im stary, nudny styl preferowany przez liberałów i~socjalistów, polegający po prostu na maszerowaniu ze znakami.

 Oczywiście wszystko to jest ,,nowe'' tylko w~pewnym sensie. Świąteczny protest -- jeśli liczyć popularne festiwale, które rzucają wyzwanie istniejącym formom nierówności społecznej i~jednocześnie dostarczają przynajmniej sugestii innych, bardziej egalitarnych, przyjemniejszych sposobów życia -- wydaje się tak stary, jak sama nierówność społeczna. Nawet jeśli ograniczymy się do Europy, przodków współczesnych imprez ulicznych możemy znaleźć w~rzymskich saturnaliach, średniowiecznych karnawałach i~wszystkich innych rytuałach, które stawiały społeczny ,,świat do góry nogami''. Istnieje obszerna literatura na ten temat\footnote{Najnowszym w~momencie pisania tego tekstu jest \textit{Dancing in the Streets} Barbary Ehrenreich (2006).}. Jak zobaczymy, ci, którzy organizują anarchistyczne imprezy uliczne, są tego bardzo świadomi; w~rzeczywistości gigantyczne lalki z~papier-mache, które często ozdabiają te festiwale, są całkiem świadomymi odniesieniami do gigantów i~smoków z~wikliny ze średniowiecznych karnawałów.

 Istnieje jednak bardziej szczegółowa genealogia. W Nowym Jorku grupa, która w~latach 1997-2002 specjalizowała się w~urządzaniu radykalnych imprez ulicznych, nazywała się Reclaim the Streets. Została nazwana na cześć znacznie starszej i~znacznie większej sieci, która pojawiła się w~Wielkiej Brytanii w~latach 90. XX wieku. RTS, jak wszyscy go nazywali, powstał w~Wielkiej Brytanii w~bardzo szczególnej sytuacji: zasadniczo pomiędzy wyłaniającym się ruchem anarchistycznym skupiającym się na ekologicznych zmaganiach o budowę dróg i~bitwami o prywatyzację przestrzeni miejskiej, a jednoczesnym narodzinami kultury rave, która sama chodziło o użycie technik bardzo podobnych do akcji bezpośredniej, aby zawładnąć i~przekształcić nieprawdopodobne przestrzenie -- polany leśne, nieużywane magazyny lub opuszczone fabryki -- w~miejsca muzyki, ekstazy i~niekończącego się wymyślania nowych form zbiorowej przyjemności. RTS zaczęło specjalizować się w~łączeniu kultury rave z~zasadami pokojowego nieposłuszeństwa obywatelskiego. Jedną z~ich największych innowacji było na przykład zastosowanie trójnogów: w~celu odcięcia ruchu od kawałka jezdni można było skonstruować skomplikowane konstrukcje trójnożne na obu końcach, z~ochotnikiem zawieszonym od góry. Gdyby bronione, nawet zdeterminowanej policji zajęłoby wieczność, aby ich zdjąć bez robienia im krzywdy. W międzyczasie można było wprowadzić nagłośnienie, a jezdnię przekształcić w~publiczny festiwal. Niektóre z~ich najlepszych wyczynów kaskaderskich szybko zyskały status legendy. Na jednej ulicznej imprezie organizatorzy jeździli starymi samochodami, aby zablokować jezdnie, a następnie podpalili je. Na innej imprezie -- ulubionej przez wszystkich -- na zorganizowanej na nowo wybudowanej obwodnicy Londynu, dwie kobiety na szczudłach krążyły w~gargantuicznie falbaniastych sukniach przy ogłuszającej muzyce, gdy pod każdą suknią facet z~młotem systematycznie niszczył asfalt, a ochotnicy od razu sadzili rośliny\footnote{Aby zapoznać się z~historią nowojorskiego RTS, zobacz Duncombe 2002. Dla Wielkiej Brytanii zobacz Jordan 1998 i~inne eseje z~tej samej kolekcji.}.

 New York RTS nigdy nie zdołał osiągnąć czegoś na taką samą skalę. Ich scena była mniejsza, a przestrzeń publiczna w~Nowym Jorku jest znacznie bardziej i~agresywniej kontrolowana. W Nowym Jorku, inaczej niż w~prawie każdym innym mieście, policja wydaje się uważać za zasadę, że protestującym i~aktywistom nigdy, pod żadnym pozorem, nie wolno wychodzić na ulice\footnote{Nowy Jork, z~prawie czterdziestotysięczną policją, ma największą siłę ze wszystkich miast na świecie - Moskwa jest bardzo odległą drugą pozycją. Nowojorska policja jest również znacznie bardziej agresywna w~technikach kontroli tłumu niż większość amerykańskich miast, nie mówiąc już o takich jak Londyn: przykładem jest wspomniane już użycie kojców. Przede wszystkim, jak zauważyło wielu aktywistów, którzy spędzili czas w~różnych częściach świata, Nowy Jork jest prawdopodobnie jedynym miastem na świecie, w~którym policja nigdy nie opuszcza ulic. Na przykład w~większości europejskich miast policja na przemian zostawia pewne dzielnice demonstrantom lub protestującym, a następnie agresywnie - często brutalnie - umacnia się, atakując lub rozpraszając protestujących. W Nowym Jorku paradygmat polega na ciągłej kontroli. Odnoszę wrażenie, że nawet w~krajach autorytarnych, takich jak Chiny, czy, powiedzmy, Syria, model bardziej przypomina podejście europejskie, choć, oczywiście, represje, kiedy się zdarzają, zdają się znacznie bardziej dotkliwe.}. Działacze dość wcześnie odkryli, że jedynym sposobem na uniknięcie zamknięcia lub aresztowania w~Nowym Jorku jest mobilność. Być może paradygmatem była masa krytyczna, która trwała od początku lat dziewięćdziesiątych. Działania RTS w~Nowym Jorku zwykle nie skupiały się na drogach, ale na kulturze samochodowej, prywatyzacji, pracach społecznych, a przede wszystkim na przestrzeni publicznej. Zwykłe podejście nie polegało na ogłaszaniu z~wyprzedzeniem lokalizacji, ale raczej na rozesłaniu numeru telefonu lub innego źródła, które w~wyznaczonym czasie poinformuje wszystkich, gdzie mają się zbierać, a czasem po prostu, jak się tam dostać (,,wybierz pociąg B na północ od Czternastej Ulicy o 14.00, a następnie podążaj za ludźmi z~pomarańczowymi flagami''). Pierwszy eksperyment, na Times Square, zakończył się sukcesem, ale policja szybko nauczyła się, jak całkiem skutecznie demontować trójnogi. Opracowali także irytującą taktykę kierowania się prosto do systemu dźwiękowego, wyłączania muzyki tak szybko, jak to możliwe, a następnie wyciągania sprzętu, który był dość drogi i~nigdy nie został zwrócony. Do czasu akcji ,,Car Free New York City'' we wrześniu 2000 (tej, której będę używał jako przykładu), RTS zaczął budować nowe systemy dźwiękowe w~wózkach lub rowerach i~starać się zachować jak największą mobilność, grając kota i~myszy z~glinami. Mimo to był to tylko sposób na odsunięcie nieuniknionego. Przyjęcie na zbiórkę pieniędzy opisane w~rozdziale 6 miało w~rzeczywistości zebrać pieniądze na nowy system: plotka głosiła, że  zostanie on zbudowany w~dużej żelaznej klatce, aby jak najdłużej trzymać policję z~daleka. To była niedziela. W czwartek, dzień przed akcją, Jackrabbit, drobny student filozofii, znacznie częściej określany jako ,,facet od trójnoga'', ponieważ był głównym jeźdźcem trójnogów RTS, powiedział mi, że system ciągle nie był skończony, ludzie dzwonili wszędzie, mobilizując wszystkich znających się na spawaniu, żeby przyjechali do Williamsburgu i~pomogli skończyć.

 Sama akcja miała zbiegać się z~organizowaną tego samego dnia Masą Krytyczną. Ci, którzy nie jeździli na rowerach, mieli zebrać się o 20:00 na małym placu przed kościołem św. Marka, na rogu Drugiej Alei i~Dziesiątej Ulicy w~East Village -- jednym z~kilku słynnych kościołów przyjaznych działaczom w~okolicy --  a następnie otrzymać wydrukowane instrukcje dotyczące dalszego postępowania.

\medskip
\noindent \textit{Akcja Reclaim the Streets ,,Nowy Jork bez Samochodów'',\\ East Village, Nowy Jork}

\medskip
\noindent [Notatki terenowe: \\ piątek, 28 września 2000 roku]
\medskip 

 Na miejsce spotkania zjawiam się spóźniony tylko pięć minut. Około sześćdziesięciu ludzi kręci się już po otwartej przestrzeni przed bramami kościoła, wielu w~kostiumach, a pięć policyjnych furgonetek jest już na miejscu. Istnieje pewna liczba typów z~DAN, wielu stałych bywalców RTS (Emily, Kelvin i~tak dalej); kilku ludzi z~Independent Media Center, kilku legalnych obserwatorów z~czerwonymi opaskami. Policjanci kręcą się na południe i~na południowy zachód od naszej części placu; nie widać żadnych transparentów, sprzętu ani żadnego oczywistego sprzętu aktywistów, ale odkąd ogłoszono lokalizację, ich obecność jest nieunikniona.

 Ktoś mi tłumaczy, że Masa Krytyczna -- dość duża, licząca dwustu lub trzystu rowerzystów -- opuściła Union Square godzinę wcześniej. Plan jest taki, żebyśmy do nich dołączyli, ale staraliśmy się jak najdłużej chronić miejsce konwergencji przed glinami.

 To piękny, chłodny wieczór, idealny na akcję. Przyniosłem też trochę kostiumów: kilka przezroczystych masek, które kupiłem w~jakimś sklepie w~Village. Wyglądają raczej przerażająco, ponieważ tak naprawdę nie ukrywają twojej twarzy, ale przekształcają twoje rysy, aby wyglądały jak ktoś inny (raczej jak ktoś zrobiony z~wosku). Kilku znajomych bierze je i~kładzie na czubkach głowy. 
 
 -- Pytanie -- ktoś pyta -- brzmi: czy rzeczywiście naruszają one prawo przeciw maskowaniu. Bo przecież są technicznie przejrzyste.

-- To dobre pytanie.

-- Przypuszczam, że moglibyśmy zapytać gliniarzy -- sugeruję niepewnie. Nie jestem pewien, czy rzeczywiście mam czelność.

-- Cóż, to \textit{oznaczałoby }uznanie ich istnienia.

-- Słuszna uwaga.

 Stopniowo słowa docierają do wtajemniczonych. Jest zbyt wielu gliniarzy, żebyśmy stąd wyruszyli na spotkanie z~Masą Krytyczną. Jest też zbyt wielu gliniarzy na ulicy Świętego Marka, dwie przecznice na południe, która była naszym pierwotnym celem. Więc ludzie z~RTS odbyli krótkie spotkanie i~wymyślili nowy plan. Wszyscy mają się rozejść, rozejść w~różnych kierunkach, ale skończyć przed The Gap dwie przecznice dalej w~St. Mark's i~Second i~czekać tam na dalsze instrukcje.

 Zwiadowcy z~telefonami komórkowymi już jeżdżą na rowerach, oceniając sytuację.

 O 20:15 połowa z~nas już nie ma; ludzie odpływają dwójkami i~trójkami. Jedna duża grupa aktywistów maszeruje na północ w~celu dywersji. Udaję się do The Gap z~Rufusem i~kilkoma przyjaciółmi. Kiedy tam dojeżdżamy, przed nim jest już może dwadzieścia osób. Szybka konferencja i~stwierdzamy, że jesteśmy zbyt oczywiści i~lepiej rozproszyć więcej, ale nagle Emily, jedna z~organizatorek RTS-ów, sprzeciwia się. 
 
 -- Powiedziano nam, żeby odpływać zbyt daleko na zachód. Pozostańmy też po tej stronie ulicy. 
 
  Troje z~nas idzie mniej więcej jedną trzecią drogi w~górę przecznicy i~kończy, opierając się o samochód, który nagle uświadamiamy sobie, że jest zaparkowany tuż obok tajemniczego, podobnego do wózka obiektu, całkowicie przykrytego dużą niebieską plandeką z~tworzywa sztucznego.

-- Hmmm. Ciekawe, co \textit{to }może być? -- uśmiecha się Rufus.

 Staramy się wyglądać jak zwykła grupa włóczęgów i~nie patrzeć na wóz, ale w~rzeczywistości jest to dość trudne.

 Nagle z~Drugiej Alei rozlegają się okrzyki i~wiwaty. Nadjechała masa krytyczna. Cała ulica wydaje się świecić. Nawet zwykli przechodnie się interesują. Wszyscy schodzą, aby do nich dołączyć.

 Masa Krytyczna nie ogranicza się do rowerów: każdy na kołach może uczestniczyć, o ile te koła są zasilane ludzką energią. W praktyce to prawie w~całości rowerzyści, choć zauważam dwóch deskorolkarzy i~kogoś na rolkach. Garstka nosi sprzęt rowerowy i~kaski, ale większość ma na sobie jakieś kostiumy: są peruki, nosy klauna, szaty -- jeden staruszek jest pokryty jakimiś niebieskimi, świecącymi w~ciemności nitkami, które pokrywają całe jego ciało. Wiele osób, z~jakiegoś powodu, ma sportowe berła lub podobne rekwizyty. Na czele stoi wysoki rower. Naprawdę wysoko. Jak dwa piętra. Rzecz wydaje się złożona z~trzech normalnych rowerów, środkowego z~siedzeniem co najmniej dziesięciu metrów nad ziemią i~dostępnego przez rodzaj drabiny oraz skomplikowaną konstrukcję nośną. Na szczycie znajduje się Aresh, słynny działacz ogrodniczy, przebrany za zielony strąk z~kapeluszem w~kształcie zielonego liścia.

 Z radością zauważamy, że udało im się otrząsnąć z~prawie całej policyjnej eskorty. Wjeżdżają i~wyjeżdżają trzy policyjne skutery, które próbują wyglądać na tak znaczące, jak pozwala na to ich liczba, ale to wszystko. Samochody i~furgonetki, które zwykle podążają za nimi, gdzieś się gubią, bez wątpienia gorączkowo dzwoniąc do HQ, aby uzyskać informację radiową o naszej aktualnej pozycji.

 Kiedy wszyscy zbiegamy się na Drugiej Alei, nagle wydaje się, że jest nas znacznie więcej, niż ktokolwiek myślał: pojawiają się różnego rodzaju aktywiści, którzy musieli być schowani poza zasięgiem wzroku, jeszcze zanim pojawiłem się w~St. Mark's. To wszystko wydaje się nieoczekiwanym triumfem. Są okrzyki, krzyki, okrzyki radości. Nieuniknione śpiewy ,,Czyje ulice? Nasze ulice!'' -- z~wyjątkiem tego, że przez chwilę nigdzie nie widać innych gliniarzy, wydaje się to prawdą.

 Kilka chwil szczęścia i~ruszamy. Niektórzy bywalcy RTS podjechali z~systemem dźwiękowym i~popychali go, gdy kilka motocykli powoli odjeżdżało na czele. Niektórzy rowerzyści mają gwizdki i~dmuchają je wraz z~okrzykami ,,Więcej rowerów, mniej samochodów!'' i~,,Czyje ulice?''. Nikt nie wie dokładnie, dokąd idziemy. Kierujemy się na wschód, potem na południe First Avenue, w~górę Seventh Street, a następnie zatrzymujemy się w~połowie ulicy prowadzącej do Tompkins Square Park. Najwyraźniej to jest to miejsce. System dźwiękowy już w~tajemniczy sposób dotarł, a teraz jest odsłonięty. Zauważam, że rzeczywiście jest on zatopiony w~bardzo skomplikowanej okuciu. Rowerzyści znów zaczynają gwizdać: rozlega się ogromny huk wszelkiego rodzaju. Unoszące się nad nami bańki mydlane. Ludzie nadmuchują balony i~wszyscy zaczynają podskakiwać nimi w~tę i~z powrotem nad głowami innych. W końcu wszystko działa, a muzyka -- rodzaj wesołej elektroniki -- zaczyna grać w~ogromnym, dobroczynnym poczuciu triumfu i~spełnienia. Rowerzyści witają pieszych przyjaciół uściskami i~przybijają piątki. Wielu aktywistów z~kamerami skupia się na systemie: kilkudziesięciu innych już zaczyna tańczyć wokół niego. Wszędzie ludzie zaczynają podskakiwać w~górę i~w dół.

 Na okolicznych ulicach zaparkowane są rowery, wiele z~bębnami i~serpentynami. Jordan z~DAN Labor pojawia się w~całkowicie skórzanym kowbojskim stroju i~rozdaje różowe balony. Inni przebrani są za piratów lub kapłanów. Jest jeden zaskakujący facet w~garniturze z~milionem oczu przyklejonych do całej twarzy i~ubrania.

 Jednym z~powodów ciągłego poczucia triumfu jest to, że gliniarze wciąż nas nie dogonili, z~wyjątkiem trzech skuterów; chociaż wszyscy wiemy, że posiłki są nieuniknione. Rzeczywiście, kilka minut później pojawiają się dodatkowe hulajnogi. Wkrótce na obu końcach naszej imprezy zaparkują ich kolejki. Dziesięć minut później policjanci kontrolują system, wszyscy aktywiści udają, że ich nie zauważają, albo ukradkiem uśmiechają się, gdy jedna świńska biała koszula zastanawia się nad żelazną klatką i~próbuje wymyślić, jak można się dostać do środka. Wydaje się sparaliżowany, zirytowany i~starając się tego nie okazywać, jak radośnie zauważa kilku przyjaciół. Bardziej niepokojący jest stan samego systemu, ponieważ jego generator wydaje się dymić i~wydzielać zapach palonej gumy. W pewnym momencie wydaje się chwiać i~przechylać, gdy dowódca go ogląda, skłaniając wszystkich -- aktywistów i~gliniarzy -- do rzucenia się i~podparcia go, zmuszając na chwilę do bycia sojusznikami; ale potem jest bezpiecznie i~wracamy do wystudiowanej nieświadomości. W końcu wydaje się, że smród spalenizny również opada, gliniarze odchodzą, dołączając do grupy furgonetek NYPD zbierających się na skraju parku, oczekujących na rozkazy.

 Tak więc przez około dwadzieścia minut, od około 20:35 do około 20:55, około trzystu aktywistów zajmuje środek Siódmej Ulicy, nieniepokojonych. Wszyscy tańczą. Nawet ja trochę tańczę, kiedy szalona starsza kobieta z~IMC wychodzi i~mnie łapie. Brak trójnogów. Ale też nie ma samochodów, zauważamy, że policja w~pobliżu parku odpędza ich. 
 
 -- Co tak naprawdę oznacza -- zastanawia się Rufus -- że pieczemy dwie pieczenie na jednym ogniu, ponieważ to również wiąże wiele z~nich i~uniemożliwia im ruszenie na nas, dopóki nie otrzymają posiłków. 
 
 Piesi wydają się zdezorientowani (staramy się ich nakłonić do przyłączenia się do nas, ale większość pozostaje na chodniku); ciekawi mieszkańcy wychodzą i~siadają na werandzie, aby obejrzeć przedstawienie. Wielu z~nich przynosi kamery, żeby nas sfilmować. Jeden lub dwa akceptuje balony.

 \texttt{20:52}: System dźwiękowy przestaje działać. Nie, znowu się włącza. Wydaje się, że istnieje zbiorowa obsesja na punkcie tego, jak długo możemy ją utrzymać, jakby każda minuta, w~której możemy zachować muzykę, była triumfem wyrwanym z~tego, co w~przeciwieństwie do tego, co można postrzegać jedynie jako przytłaczającą codzienną rzeczywistość: brak radości wymuszony przemocą. Bańka wolności.

 Okrzyki rozlegają się, gdy Jackrabbit gramoli się na szczycie klatki z~harmonijką przypiętą do głowy. Wykonuje krótki taniec do własnej muzyki i~schodzi w~dół\ldots 

 \ldots i~tak naprawdę zeskakuje, gdy gliniarze zaczynają się poważnie poruszać. Stoję na rogu, dmuchając balon, gdy starsza kobieta w~pomarańczowej czapce bejsbolowej przechodzi, mówiąc wszystkim, żeby poszli z~ulicy. 
 
 -- Jeśli nie jesteś na chodniku -- powtarza -- można cię aresztować! 
 
 Patrzy na mnie niecierpliwie, bo właśnie wzywam przechodniów, żeby do nas dołączyli. Po prostu się uśmiecham.

-- Cóż, na wypadek, gdybyś został aresztowany, weź tę kartę.

-- Um, czy możesz dać mi sekundę? Próbuję zawiązać ten balon.

 Podnoszę wzrok i~zauważam, że na karcie jest napisane ,,National Lawyer’s Guild'' (Krajowa Gildia Prawników), grupa, która zwykle zapewnia obserwatorów prawnych. Na jej kapelusz też jest napis. Nie czeka na mnie (rozpoznała oczywistą pomyłkę) i~idzie dalej, niejasno denerwując wszystkich, którzy przeważnie starają się udawać, że jej tam nie ma. NLG są zazwyczaj dobrymi sojusznikami aktywistów, ale w~tym przypadku nie jest tak, że ktoś nie zdaje sobie sprawy, że przebywanie na ulicy powoduje, że można cię aresztować. O to właśnie chodzi. Z drugiej strony, ona, choć oczywiście ma dobre intencje, wydaje się nieświadoma, że  nie identyfikując się ani nie wyjaśniając, zapewnia, że  jej zachowanie jest nie do odróżnienia od zachowania policji.

 Najwyraźniej policja czuje teraz, że zebrała wystarczająco dużo sił, by dokonać aresztowań, na ulicy jest teraz około sześciu furgonetek, około pięćdziesięciu mundurowych. Co oznacza, że  nadszedł czas, aby druga połowa planu zaczęła obowiązywać.

\medskip
\noindent Część Druga

 Podczas imprezy ulicznej członkowie anarchistycznej orkiestry dętej zwanej Głodną Orkiestrą Marszową przeniknęli i~ustawili się dyskretnie na pobliskiej werandzie, w~pobliżu miejsca, gdzie zaparkowany był wysoki rower Aresha. Do tej pory mają, o ile wiem, jednego perkusistę, jeden puzon, dwa klarnety i~tubę. Na początku dodają tylko drobiazgi i~ozdobniki do muzyki, którą wydaje system, ale o 21:57 muzyka nagle się zatrzymuje. Kilkunastu policjantów otacza żelazną klatkę i~podejmuje poważne wysiłki, aby ją wywieźć.

 W chwili, gdy muzyka ustała, wszyscy poczuliśmy się wrzuceni na niezbadane terytorium. Nastąpiła gwałtowna zmiana zbiorowych emocji, poczucie, że wszystko może pójść wszędzie. Najpierw nadeszła chwila całkowitej niepewności, zmiana nastroju, ale nikt nie wiedział dokładnie do czego. Nagła pustka może wypełnić wszystko. Wtedy jedna osoba zaczęła skandować ,,Czyje ulice? Nasze ulice!'' i~podejmowało go coraz więcej z~nas, krzycząc niemal wprost przeciwko policji. Brzmiało to tak, jakby to, co zaczęło się jako świąteczne przyjęcie, mogło przekształcić się w~agresywny, gniewny protest, a może nawet w~bitwę uliczną. Następnie, podczas ciszy, jedna z~Radykalnych Cheerleaderek, w~różowych gorących spodniach i~spiczastych niebieskich włosach, wyszła na otwartą przestrzeń tuż obok systemu i~zaczęła śpiewać pieśń, którą jej grupa ułożyła na tę okazję:

\medskip
\noindent \textit{Nie chcemy kawałka korporacyjnego tortu,\newline  ponieważ cały ten pieprzony system oparty jest na kłamstwie\ldots }
\medskip

 Problem polega jednak na tym, że jej głos jest wysoki i~nie brzmi zbyt dobrze. Kilka innych radykalnych cheerleaderek pojawia się i~dołącza, ale inni z~jej grupy afinicji wydają się zaginieni, a ponieważ nikt z~nas nie zna słów, nikt inny nie może się tym zająć. Więc to nie działa. W międzyczasie rusza policja z~megafonami, a za nią maszeruje batalion w~szyku, z~typową dla nich dziwną mieszanką uprzejmości i~groźby użycia przemocy: ,,Proszę ruszać. Każdy, kto stoi na ulicy, zostanie aresztowany. Prosimy o \textit{przejście}''. 
 
 To tak, jakby nie byli pewni, czy zwracają się do obywateli, czy przestępców, więc zwracają się do nas jako do obu jednocześnie. W każdym razie dokładnie w~momencie, gdy policja staje się słyszalna dla większości tłumu, zespół zaczyna grać. i~wydaje się, że wybrali idealną melodię: muzyka klezmerska, głośna, ale wybitnie taneczna. Nagle wydaje się, że policja została podporządkowana, włączona. Są częścią \textit{naszego }aktu. Stali się rozrywką.

 Ludzie zaczynają skakać na rowerach. Wszyscy zaczynamy wycofywać się na zachód, z~dala od policji. Nie jest jasne, czy ktokolwiek kieruje ucieczką, ale jest w~tym porządek. Główną siłą kierującą są oczywiście muzycy (,,Mamy własną orkiestrę marszową!'' triumfalnie deklaruje jedna kobieta), choć przed nimi biega kilku organizatorów, w~tym Times Up Bill, zwykle lider Critical Mass, ponury, ciemnowłosy facet po czterdziestce mówiący bez przerwy przez krótkofalówkę i~jeszcze kilku innych: jakiś facet w~czerwonym nosie klauna, którego nie znam, inny Afroamerykanka przebrana za piratkę, którego później dowiaduję się, jest częścią chóru Reverend Billy Choir.

-- Jaka jest twoja pozycja?

-- Czy jest tam dużo gliniarzy?

-- Kierujemy się w~stronę Tompkins Square Park. Słyszysz mnie? Kierujemy się do parku.

 Gdy się wycofujemy, ludzie zaczynają śpiewać znane antypolicyjne okrzyki (,,Mamy rację, A wy nie!'' ,,Wielkie pały, małe pały!''), a zespół rozpoczyna nowy numer, tym razem naprawdę skoczna, jazzowa melodia cyrkowa. Niezwykle rytmiczna i~jednocześnie głupkowata. Wielu z~nas chodzi i~tańczy jednocześnie lub robi coś w~połowie drogi. Niektórzy ludzie skaczą. Inni się kręcą. Niezwykle trudno tego nie robić. Z kilkoma moimi przyjaciółmi robię coś w~rodzaju spacerowego klauna. Prawie nikt nie korzysta z~chodnika; większość z~nas podąża za rowerami -- prowadzonymi przez Aresha -- środkiem ulicy.

 Docieramy do skrzyżowania przy First Avenue, co jest dobre, ponieważ kilka osób narzekało, że naprawdę powinniśmy przynajmniej raz znaleźć się na prawdziwej ulicy z~prawdziwym ruchem do zablokowania. Rowery zajmują środek skrzyżowania, a my tworzymy krótką falangę, zatrzymując taksówki i~ciężarówki, po czym ruszamy dalej.

 Zaczynam zauważać, że podczas gdy Bill wydaje się koordynować wszystko, rozmawiając jednocześnie przez walkie-talkie i~telefon komórkowy, nikt tak naprawdę nie robi niczego, co mówi. Często robią dokładnie odwrotnie. W rzeczywistości wszyscy po prostu podążają za muzyką i~najwyższym widocznym obiektem, którym jest Aresh jako Groszek na wysokim rowerze. Na Second Avenue ponownie zajmujemy skrzyżowanie; taksówki trąbią; gliniarze po raz kolejny są cudownie nieobecni (jak udało nam się ich zgubić?), z~wyjątkiem tych dwóch lub trzech zawziętych skuterów zdecydowanych pokazać swoją obecność. Wiwatujemy, gdy rowery znów tworzą falangę; widzowie są głównie rozbawieni przez darmową rozrywkę. Pewien pijak w~garniturze wciąż woła Aresha: 
 
 -- Jak ty to kurde robisz?

 Ktoś pyta: -- Czy ktoś ma jakieś materiały informacyjne lub literaturę dla tych ludzi?

-- O cholera -- mówi Christine, kolejna kluczowa organizatorka RTS-ów. -- Zapomniałam. Gdzieś jest ich pudło\ldots 

-- Rozdawaliśmy trochę na samym początku -- mówi ktoś inny. -- Ale nie jestem pewien, co się z~nimi później stało.

 Policyjne skutery próbują przebić się przez rowery, ale większość gliniarzy wciąż jest w~tyle. W pewnym momencie muzyka nieco słabnie i~kilku perkusistów z~tłumu rozpoczyna rytm samby, aby dać zespołowi odpocząć. Maszerujemy ulicą pełną indyjskich restauracji, mnóstwa Sikhów w~turbanach i~innych południowoazjatyckich ludzi dostarczających nam uśmiechów i~okrzyków ,,Brawo!''. Inni przechodnie nas filmują. Bill opowiada mi, jak na Seventh gliniarze, którzy przejęli system dźwiękowy, powiedzieli mu: ,,Chcielibyśmy, abyś po prostu próbował go odzyskać'', grożąc masowymi aresztowaniami, jeśli ktokolwiek wyglądał, jakby robił krok dalej.

-- Stracony na zawsze.

-- Czy zdarzyła się kiedyś akcja, w~której nie straciłeś systemu dźwiękowego? -- pytam.

-- Jeszcze nie. Gliny zawsze idą prosto po niego. Dokonują oceny tego, co jest przedmiotem największej wartości i~przywłaszczają sobie. i~nigdy go nie odzyskamy.

 Zauważył jednak, że z~drugiej strony nie ma doniesień o aresztowaniach.

 Kolejne wiwaty, gdy po zawróceniu docieramy na Tompkins Square, znowu blokujemy aleję, furgonetki gliniarzy zniknęły, choć wszyscy zatrzymują się na chwilę, żeby zrobić miejsce dla karetki St. Vincent's. Zespół powraca do muzyki klezmerskiej, gdy dołączamy do tłumu aktywistów, którzy na nas czekali (Bill wyjaśnia: -- Park był naszym miejscem awaryjnym, jeśli miały być problemy.).

\medskip
\noindent Denouement
\medskip

 Potem jest już głównie odpoczynek. The Hungry March Band szybko rusza na kolejny koncert w~centrum miasta. Zbieramy się na otwartej przestrzeni, gdzie kiedyś była muszla koncertowa. Radykalne cheerleaderki w~końcu zaprezentowały swój nowy występ przed coraz mniejszą, ale doceniającą publicznością. Ktoś żongluje ogniem, ale iskry rozpryskują się i~zaczynają trochę płonąć wśród rozdmuchanych papierów, a kilku z~nas podbiega, by zdusić ogień. Aresh pokazuje wielu wdzięcznym widzom, jak wspinać się na wysoki rower. Wszyscy opowiadają wszystkim swoją wersję tej nocy (,,Gutter miał oczy na całej głowie, a Jackrabbit wskoczył na szczyt systemu i~tańczył jak małpa!'' słyszę radosny raport rowerzysty, gdy przyjaciel podaje mu butelkę.).

 Kobieta robi cartwheels i~część breakdance przenosi się do improwizowanej gry na perkusji. Jakiś pijak z~parku rzuca śmietnikiem i~wszyscy znów biegną posprzątać. Emily wchodzi na scenę, aby ogłosić: 
 
 -- Jutro odbędzie się wiec Nadera w~Washington Square Park i~Miliarderzy będą rozbijać imprezę. Spotkajmy się o 11:15 w~południowo-zachodnim rogu parku; przynieś filiżanki do herbaty, kostiumy, smokingi, jeśli je masz. 
 
Jest trochę niepewna w~swoim przekazie, świetnie bawi się swoją nieśmiałością, a kilka osób robi równie wielki pokaz braw i~wiwatów, aby ją zachęcić (,,Naprzód Emily!'' ,,Tak Emily!''). Krąży plotka, że  po wydarzeniu są dwie różne imprezy: impreza NYU w~East Village i~znacznie większa na promie Staten Island. The Hungry March Band zagra na promie w~obie strony: pół godziny w~każdą stronę, zaczynając o 23:15. To oczywiście miała być najlepsza impreza tego wieczoru, ale jestem zmęczony i~w końcu odpływam do domu.

 Kontrast z~liniami pikiet nie mógłby być bardziej oczywisty. W zasadzie oczywiście akcja była przekazem. Nazwano ją ,,Nowy Jork wolny od Samochodów'', próbując wyobrazić sobie miasto bez samochodów osobowych, a tym samym przywieźć do domu odpady i~zniszczenia spowodowane amerykańską kulturą samochodową. Pomysł zakazu poruszania się samochodami przynajmniej na Manhattanie pojawił się od lat 60. XX wieku. Bez wątpienia większość uczestników mogłaby, gdyby została poproszona, szczegółowo omówić ten temat: wyliczać wszelkiego rodzaju statystyki dotyczące zniszczeń ekologicznych, globalnego ocieplenia, historii wojen naftowych, zmowy korporacji w~zakresie niszczenia transportu publicznego i~związek między motoryzacją a ogólniejszymi trendami izolacji społecznej i~erozji wspólnej przestrzeni publicznej. Niewątpliwie również część z~nich była w~kopiowanej do rozdawania literaturze; ale nigdy nie udało mi się zdobyć kopii: ulotki najwyraźniej zgubiły się dość wcześnie w~akcji. W rzeczywistości, o ile te tematy zostały wygłoszone, tak naprawdę odbyło się to podczas zbiórki pieniędzy tydzień wcześniej, w~której uczestniczyli prawie wyłącznie aktywiści. To nie było takie niezwykłe. Przejażdżki masą krytyczną w~Nowym Jorku również miały zazwyczaj określone tematy polityczne, ale stwierdziłem, że można łatwo uczestniczyć w~całej akcji od początku do końca, nie mając najmniejszego pojęcia, o czym miała ona być. Aby dowiedzieć się, w~rzeczywistości, zwykle musiałbym zapoznać się z~raportem, który pojawiłby się na stronie internetowej IMC i~na listach aktywistów następnego dnia. Należy również zauważyć, że nie podjęto próby osądzania ani nawet informowania mediów -- poza mediami aktywistycznymi, takimi jak sam IMC. To może wydawać się drobnym punktem, ponieważ i~tak nie było prawie żadnych szans, aby większość mediów głównego nurtu doniosła o takiej akcji (nawet lokalne stacje kablowe, takie jak New York 1, regularnie je ignorowały), ale pokazuje stopień, w~jakim zamierzona publiczność była, przynajmniej w~najszerszym tego słowa znaczeniu samą społecznością aktywistów.

 Nie mówię tego, aby umniejszać polityczne znaczenie takich działań (choć z~pewnością są działacze, którzy by to zrobili). Pragnę jedynie podkreślić, że nie chodzi tu przede wszystkim o zasięg. Chodzi o budowanie kultury i~społeczności aktywistów. Ta społeczność zawsze nakładała się na szersze kręgi bohemy czy artystycznej. W przypadku akcji w~stylu RTS można było dostrzec ciąg działań, poczynając od ,,festiwalów oporu'' w~stylu Seattle czy Quebecu (w których brali udział ludzie RTS) -- gdzie nie było żadnej dwuznaczności co do tego, kto był konfrontacji -- na przejażdżki Masą Krytyczną lub imprezy uliczne, które nie były głównie konfrontacją, na rave’y lub imprezy w~metrze, często organizowane niemal dokładnie w~ten sam sposób, gdzie nie było żadnej próby konfrontacji ani udawania politycznego przesłania. Co więcej, członkowie tej samej grupy ludzi -- aktywiści, a także ravers i~osoby z~bohemy -- pojawiliby się na wszystkich. Choć, jak można się domyślać, nie w~równych proporcjach. Rozważmy w~tym przypadku imprezę w~metrze, na której kilkaset osób niespodziewanie zbiera się wraz z~muzykami, serpentynami i~dzikimi kostiumami w~nowojorskim metrze i~przejeżdża nim przez miasto. Czy to jest ,,akcja''? Nigdy nie słyszałem, by ktokolwiek odnosił się do takiej, ponieważ nie zawierała wyraźnej treści politycznej. Ale o ile były one zapewnieniem prawa do tworzenia nowych form społeczności poprzez znajdowanie nowych -- i~technicznie nielegalnych -- sposobów zawłaszczania wspólnych zasobów publicznych, nie różnią się one tak bardzo. 

 Jestem może tutaj nieco niesprawiedliwy. Zwykle w~praktyce imprezy uliczne i~taktyka karnawałowa są zwykle mieszane z~bardziej klasycznymi strategiami obywatelskiego nieposłuszeństwa: tak jak w~przypadku większości brytyjskich akcji RTS, które miały bardziej konkretne cele i~nie wymagały uciekania przed policją. Alternatywnie taka taktyka może stanowić jeden element w~znacznie większej akcji. Mimo to warto zobaczyć, co dzieje się z~określoną logiką, gdy jest ona posunięta tak daleko, jak to możliwe. W tym przypadku dochodzi do granic możliwości elementu utopijnego w~akcji bezpośredniej, a zwłaszcza nacisku na zbiorową przyjemność. O ile dochodzi do konfrontacji, ,,cel'' staje się abstrakcją: kultura samochodowa, kontrola przestrzeni publicznej, czy szerzej cały system społecznej alienacji i~przymusu, który odmawia nam możliwości samorealizacji, kreatywności i~zabawy. Gdy ta złowroga abstrakcja przybiera konkretną formę, jest to forma policji (nie ma tu mowy, by policja była kiedykolwiek neutralnymi mediatorami). Z kolei z~policją staramy się mieć jak najmniejszy kontakt. Kiedy nie można ich dłużej ignorować, czas uciekać. Celem jest życie jak najdłużej w~takim świecie, w~jakim chciałoby się żyć: takim, w~którym nie ma policji. Zawsze też nacisk kładzie się na obszar geograficzny: na otwieranie wyzwolonych przestrzeni, jakkolwiek efemerycznych, przekształcania ulic, placów, parków i~jezdni, które są -- w~zasadzie -- własnością publiczną (,,Czyje ulice Nasze ulice!'') poprzez ich przewracanie społeczności, uwalniając ich spod kontroli jej domniemanego przedstawiciela, państwa. i~co jest warte, grupy takie jak RTS były, w~istocie, znacznie bardziej bezpośrednio powiązane z~obecnymi organizacjami lokalnymi -- ogrodami społecznymi, na przykład, lub innymi instytucjami w~sąsiedztwie -- niż grupy jak DAN, która zawsze miały bardziej międzynarodowe relacje\footnote{W rzeczywistości, aktywiści RTS regularnie krytykowali DAN za to, co uważali za skupienie się na szczytach i~zaniedbywanie obaw lokalnej społeczności - w~szczególności za jej niechęć do organizowania akcji bezpośrednich w~samym Nowym Jorku.}.

 Za tym wszystkim kryje się dość wyrafinowana teoria. Wspomniałem już (choć krótko) o argumencie sytuacjonistycznym: że system towarowy i~wynikające z~niego zniszczenie znaczących relacji międzyludzkich czyni nas bierną publicznością naszego własnego życia -- zjawisko, które Guy Debord nazwał ,,Spektaklem''. Dla Sytuacjonistów jedynym sposobem na pokonanie Spektaklu w~życiu codziennym było podjęcie działania dla siebie i~za siebie, powrót do czystej przyjemności twórczości, co w~efekcie oznaczało obracanie elementów systemu -- przestrzeni, obrazów, przedmiotów -- przeciwko sobie. Sytuacjonistów szczególnie interesowała transformacja przestrzeni miejskiej -- praktyka, jak ją nazywali, psychogeografii -- obalania szarej, utylitarnej logiki ulic i~arterii, aby odtworzyć poczucie nie tylko wspólnoty, ale i~sacrum. Połącz z~tym ,,Immediatyzm'' autorów takich jak Hakim Bey (1991), z~jego całkowitym odrzuceniem starego modelu ponurego, poświęcającego się rewolucjonisty i~jego naciskiem na prawo do doświadczania pełnej przyjemności wolności tu i~teraz, choćby tylko w~chwilowych błyskach w~tymczasowych strefach autonomicznych, tworzących pęknięcia w~siatce totalnej kontroli, i~mniej więcej tyle każda osoba wymyśli. Jest to tym ważniejsze w~Nowym Jorku, mieście całkowicie oddanym produkcji i~ekspozycji widowiska konsumenckiego, a jednocześnie sprawującym najbardziej absolutną i~całkowitą kontrolę nad przestrzenią publiczną ze wszystkich miast na ziemi. Zagrody, barykady, stała obecność przytłaczającej liczby policji: wszystkie mają na celu demoralizowanie działaczy i~są w~tym niezwykle skuteczne. Nic więc dziwnego, że każda przestrzeń autonomii, którą można im wydrzeć, czy to przez samą wojowniczość i~wagę liczebną -- co prawie nigdy się nie zdarza -- albo przez czystą pomysłowość, wydaje się całkowicie wartościowym osiągnięciem.

\medskip
\noindent PRZYKŁAD CZWARTY: NIEPOSŁUSZEŃSTWO CYWILNE (BLOKADA)
\medskip

 W rozdziale 5 omówiłem niektóre z~teorii akcji bezpośredniej i~to, jak zasadniczo różni się ona od obywatelskiego nieposłuszeństwa. Tutaj zwracam się do praktyki. Jak zwykle w~praktyce sprawy okazują się nieco bardziej skomplikowane.

 W powszechnym użyciu aktywistów każda demonstracja, która obejmuje umyślne łamanie prawa, może być określana jako ,,ON'' -- akt Obywatelskiego Nieposłuszeństwa. Tak więc na pewnym poziomie prawie wszystkie działania można uznać za ON. Jednak ta szersza definicja jest zwykle używana zwłaszcza w~rozmowach z~osobami z~zewnątrz: obserwatorami, mediami, policją. W rozmowach z~innymi aktywistami termin ten jest najczęściej używany, gdy mówi się o działaniach, które są uroczyste, publiczne, stacjonarne, wyraźnie pokojowe i~nie wiążą się z~żadną próbą uniknięcia aresztowania. Jeśli myślimy o ,,ON'', myślimy przede wszystkim o rodzaju akcji, w~której uczestnicy zaczynają od publicznego ogłoszenia zamiaru przeciwstawienia się niesprawiedliwemu prawu lub polityce, a następnie przystępują do tego -- powiedzmy, zamykając się w~budynku, który ma zostać zburzony, lub do bram siedziby firmy, lub paląc swoje karty poboru -- nie stawiając oporu (albo alternatywnie tylko bierny lub skrupulatnie pokojowy opór), gdy wydaje się, że policja zaczyna aresztować. W wielu przypadkach bycie aresztowanym jest sednem sprawy, ponieważ aresztowanie i~proces dają możliwość wyjaśnienia światu, co się robi, lub zakwestionowania niesprawiedliwego prawa. Akcja bezpośrednia może być ukradkowa; obywatelskie nieposłuszeństwo z~definicji nie może.

 Może to sprawić, że obywatelskie nieposłuszeństwo będzie tylko przedłużeniem logiki protestu, próbą wpłynięcia na władze poprzez odwołanie się do opinii publicznej. Jeśli tak, to miałoby to niewiele wspólnego z~akcją bezpośrednią. Nie jest to do końca nieprawda. Z pewnością przy wielu akcjach oznaczonych jako ,,ON'' tak jest dość wyraźnie. Na jednym biegunie zdarzają się akty nieposłuszeństwa obywatelskiego, w~których wszystko jest wcześniej ustalane z~lokalnymi władzami: protestujący wychodzą na ulicę lub blokują biuro w~umówionym czasie i~miejscu, a następnie posłusznie wymaszerowują, aby zostać spisanymi. Często policja ustawia nawet specjalne stoliki w~pobliżu, aby szybko i~sprawnie przetwarzać aresztowania. ON tego typu są zwykle organizowane przez grupy protestacyjne jako rozszerzenie protestów. Są to wypowiedzi czysto symboliczne. Jednak na innej skrajności ON i~akcje bezpośrednie mogą stać się zupełnie nie do odróżnienia: na przykład, gdy ktoś umieszcza swoje ciało na drodze buldożera lub gdy próbuje obalić rząd przez masową odmowę płacenia podatków lub w~inny sposób przestrzegania jego praw. Co najwyżej można to zakwalifikować jako akcję bezpośrednią, w~dużej mierze w~jej negatywnym trybie: akty buntu bardziej niż tworzenie. Ale nawet to nie wytrzymuje dokładnej analizy.

 W rzeczywistości nawet klasyczni teoretycy (i praktycy) obywatelskiego nieposłuszeństwa -- Thoreau, Gandhi, King -- uważali się za akcjonistów bezpośrednich. Opisany w~tej książce ruch akcji bezpośredniej rozwinął się właśnie z~tej tradycji. Opisałem już, jak jej formy organizacyjne po raz pierwszy rozkwitły w~kampaniach antynuklearnych późnych lat siedemdziesiątych i~wczesnych osiemdziesiątych. Kampanie te składały się prawie wyłącznie z~ogłoszonych publicznie, pokojowych blokad. Działania WTO w~Seattle w~1999 roku -- z~wyjątkiem stosunkowo niewielkiego Czarnego Bloku -- również składały się z~masowych blokad. Podobnie jak akcje A16 przeciwko IMF i~Bankowi Światowemu 16 kwietnia 2000 roku oraz przeciwko konwencji republikanów w~Filadelfii tego lata. Główną innowacją taktyczną wprowadzoną w~międzyczasie był lockbox: skomplikowane urządzenia, które unieruchamiały ramiona blokerów, umożliwienie kilku zablokowaniu się w~taki sposób, że rozcięcie ich na części było niezwykle trudne i~czasochłonne dla policji. Wiele z~tych technik zostało pierwotnie opracowanych w~latach 80., według niektórych relacji, na czele z~prawicową grupą antyaborcyjną Operation Rescue. Klasyczne akcje masowe w~stylu DAN były budowane wokół blokad, nawet jeśli same blokady zwykle odgrywały w~nich dość niewielką rolę. Zazwyczaj obejmowały cztery szerokie kategorie grup afinicji: (1) zespoły blokujące, (2) ,,miękkie'' grupy blokujące, które działały bez blokowania, na przykład trzymając się za ręce, trzymając się za ramiona lub tworząc ludzki łańcuch, lub które działały we wspieraniu blokad, (3) ,,drużyny lotne'', które mają być wezwane w~celu wypełnienia dziur w~linii lub w~razie potrzeby, oraz (4) grupy teatrów lalek/ulicznych/muzyki, a także, skutecznie, drużyny latające, który tworzyły rodzaj wędrownego karnawału. Te ostatnie były najczęściej sprowadzane w~celu podniesienia na duchu lub rozładowania sytuacji napięcia. W sumie efekt nie różnił się aż tak bardzo od linii pikiet: przynajmniej o tyle, o ile celem było otoczenie i~odcięcie dostępu do niektórych budynków. Różnica polegała na tym, że blokada nie była ani legalna, ani symboliczna; miała na celu fizyczne uniemożliwienie ludziom przedostania się za pomocą wszelkich środków, które nie są stosowane przy użyciu przemocy. Była też znacznie większa, bardziej mobilna i~bardziej elastyczna. 

 Grupy afinicji były zorganizowane w~klastry. Obszar wokół Centrum Kongresowego został podzielony na trzynaście sekcji, a grupy afinicji i~klastry zobowiązały się do utrzymywania poszczególnych sekcji. Co więcej, niektóre grupy były ,,latającymi grupami'' -- swobodnie przemieszczały się tam, gdzie były najbardziej potrzebne. Wszystko to było koordynowane na spotkaniach Rady Delegatów, gdzie każda z~grup afinicji wysyłała przedstawiciela upoważnionego do przemawiania w~imieniu grupy.

 W praktyce taka forma organizacji oznaczała, że  grupy mogły poruszać się i~reagować z~dużą elastycznością podczas blokady. Jeśli wezwanie zostanie wysłane do większej liczby osób w~określonej lokalizacji, grupa afinicji może ocenić liczby trzymające linię w~miejscu, w~którym się znajdują i~zdecydować, czy się przenieść. W konfrontacji z~gazem łzawiącym, gazem pieprzowym, gumowymi kulami i~końmi, grupy i~jednostki mogły ocenić własną zdolność do przeciwstawienia się brutalności. W rezultacie linie blokad zostały utrzymane w~obliczu niesamowitej przemocy policji. Kiedy jedna grupa ludzi została w~końcu zmieciona przez gaz i~pałki, inna wkroczyła na ich miejsce. Jednak było też miejsce dla tych z~nas w~grupie afinicji w~średnim wieku, z~chorymi płucami / chorymi plecami, aby utrzymać linie w~obszarach, które były stosunkowo spokojne, aby wchodzić w~interakcje i~rozmawiać z~delegatami, których zawróciliśmy, i~wspierać marsz robotniczy, który w~południe przyprowadził dziesiątki tysięcy ludzi. Żaden scentralizowany przywódca nie mógł koordynować sceny w~środku chaosu i~żaden nie był potrzebny -- organiczna, autonomiczna organizacja, którą mieliśmy, okazała się znacznie potężniejsza i~skuteczniejsza. Żadna autorytarna postać nie mogła zmusić ludzi do utrzymywania linii blokady podczas gazowania, ale upoważnieni ludzie do podejmowania własnych decyzji zdecydowali się to zrobić (Starhawk 2002: 18).

 W tym przypadku więc połączenie liczb, demokratycznej organizacji i~nieskończonej taktycznej pomysłowości przekształciło to, co w~innym przypadku mogłoby być symbolicznym gestem, w~formę niezwykle skutecznej akcji bezpośredniej.

 Myślę jednak, że warto zastanowić się trochę uważniej nad tym, co dzieje się na blokadzie, bo na swój sposób wydaje się to wzorową formą nieposłuszeństwa obywatelskiego.

 ,,Blokada'' może odnosić się do czegoś tak prostego, jak aktywistka przykuwająca się łańcuchem do bramy (ulubioną techniką jest tutaj użycie  zamka Kryptonite w~kształcie litery U na szyi: bardzo trudnego do wyjęcia, ale także narażającego na poważne niebezpieczeństwo urazu szyi lub kręgosłupa). Lub może odnosić się do bardzo skomplikowanych urządzeń, w~których beczki wypełnione betonem zostały wcześniej przetransportowane na miejsce. Najczęściej jednak oznacza to zastosowanie lockboxów wykonanych z~rurki PCV. Dwóch aktywistów może chwycić zablokowany łańcuch przymocowany z~jednej strony, przymocować łańcuch do śruby pośrodku, a tym samym skutecznie połączyć swoje ręce. Rurki są zbyt ciasne, aby ktokolwiek mógł sięgnąć do środka i~otworzyć zamek, a zwłaszcza jeśli są wzmocnione taśmą klejącą i~warstwami innego rodzaju materiału, nie można ich rozciąć zwykłą piłą do metalu. Policja często musi sprowadzać diamentowe wiertła.

 Taka taktyka była dość szeroko stosowana w~walkach leśnych przez grupy takie jak Earth First! w~latach 90., aby blokować drogi czy okupować urzędy polityków. W większości takich akcji tylko garstka aktywistów jest zajęta. Mimo to nawet w~akcjach masowych, kiedy może to zrobić setka, uwięzieni reprezentują rodzaj heroicznej elity, bohaterskiej, ponieważ chcą znosić co najmniej ekstremalny poziom dyskomfortu i~całkiem możliwe, że ekstremalny ból fizyczny. Oto kilka porad od CrimethInc:

\textit{ Ostatecznie nie ma sposobu, aby z~całą pewnością przewidzieć, jak zareaguje policja, więc unikaj spędzania godzin na debatach w~swojej grupie. Ważne jest, aby obecny był łącznik z~policją, aby negocjować z~władzami lub przynajmniej upewnić się, że rozumieją sytuację, a reporterzy lub inni świadkowie złagodzą lub przynajmniej udokumentują ich zachowanie. Jeśli zaczną robić coś, co wydaje się niebezpieczne, spokojnie poinformuj ich, że twoja ręka jest w~tubie i~nie jesteś w~stanie jej wyjąć, a zespół prawników z~niecierpliwością czeka na możliwość pozwania ich w~niepamięć. Policja zawsze będzie próbowała cię zastraszyć; sprawdź ich blef, zachowując spokój. W najgorszym przypadku mogą użyć na tobie gazu pieprzowego lub podobnej broni, ale pamiętaj, że będzie to ich dużo kosztować w~oczach opinii publicznej, zwłaszcza jeśli odważnie zniesiesz to prześladowanie\ldots }

\textit{ Zobowiązanie do zablokowania to poważna sprawa; musisz być przygotowany na mękę interakcji z~rozwścieczonymi funkcjonariuszami policji przez dłuższy czas, nie będąc w~stanie swobodnie się poruszać; po tym nastąpi dalsza gehenna aresztowania i~spędzenia czasu w~więzieniu. Przystąp do blokady w~stanie wewnętrznego spokoju i~determinacji, odpowiednio odżywiony i~nawodniony, przygotowany na burze niebezpieczeństw i~dramatów, a jeśli myślisz, że możesz tam być przez długi czas, załóż pieluchę dla dorosłych! }(CrimethInc 2005:171)

 Jak zobaczymy, w~tym tekście pojawia się pewien niewinny optymizm co do możliwości łagodzenia lub kontrolowania zachowania policji poprzez apelowanie do mediów, a nawet odwoływanie się do prawa. To, na co naprawdę chcę zwrócić uwagę, to moralna intensywność interakcji z~policją. Unieruchamiając swoje ramiona, stajemy się całkowicie bezbronni, a tym samym całkowicie w~mocy fizycznej wrogów. Nawet w~zwykłych okolicznościach istnieje ekstremalna dysproporcja władzy między policją a protestującymi, a przynajmniej ekstremalna dysproporcja w~zdolności i~chęci użycia siły przymusu. Tutaj ta dysproporcja jest celowo zwiększana tysiąckrotnie. W rezultacie ci, którzy są zablokowani, zmuszają swoich przeciwników do traktowania ich z~pewną dozą człowieczeństwa, choć w~innym razie przeciwnicy ci byliby bardzo niechętni, by ich tak traktować. Łatwo byłoby wbić diamentowe wiertło w~ciało protestującego; w~rzeczywistości trzeba postępować z~dużą ostrożnością i~uwagą, aby tego nie robić. Jak zauważono wcześniej, interakcje aktywistów z~policją zawsze są pełne momentów niejasności, w~których wydaje się istnieć sprzeczność między rolami policji: w~szczególności ochrony i~świadczenia usług dla obywateli oraz ,,utrzymywania porządku publicznego'' poprzez wykorzystanie siły. Dla policji (i mediów głównego nurtu) normalnym sposobem radzenia sobie z~tym napięciem jest skuteczne wykluczenie każdego zaangażowanego w~nieautoryzowaną działalność polityczną z~praktycznej kategorii ,,obywateli'', jak na ironię, biorąc pod uwagę, że teoretycznie jedną z~cech definiujących demokrację jest, że chroni wolność obywateli do uczestniczenia w~dokładnie tych aktywnościach\footnote{Przez ,,nieautoryzowany'' mam tu na myśli formy działalności politycznej, które nie są ani (a) dozwolone, ani (b) prowadzone w~imieniu uznanych instytucji. Jeśli ktoś składa lub zbiera podpisy dla kandydata na burmistrza lub dla Greenpeace, jest mało prawdopodobne, aby ktoś był nękany przez policję. Jednak ci w~jakikolwiek sposób związani z~anarchistami są traktowani zupełnie inaczej. Widziałem spontaniczne występy trzech lub czterech Radykalnych Cheerleaderek na pustych chodnikach zakłócane przez policję. W rzeczywistości, przez długi czas uprzedzenie wobec ekspresji politycznej było zakorzenione w~oficjalnej praktyce policji w~Nowym Jorku: jeśli ktoś został aresztowany za wykroczenie, takie jak zablokowanie chodnika, normalną praktyką było wystawienie mandatu i~natychmiastowe zwolnienie; gdyby robili to w~ramach lub w~związku z~oświadczeniem politycznym, byliby przetrzymywani przez noc. Ta zasada w~końcu została uznana za niekonstytucyjną, po skardze ACLU i~oficjalnie anulowana w~2003 roku.}. Aktywiści niestrudzenie zwracają uwagę policji i~mediom na tę sprzeczność, ale rzadko z~dużym powodzeniem. Blokada, podobnie jak podobne formy obywatelskiego nieposłuszeństwa, może być postrzegane jako próba zaostrzenia tej sprzeczności do tego stopnia, że  nie da się jej zignorować. Prowokuje się wyzywająco politycznym aktem, ale jednocześnie zmusza aresztującego do opiekuńczej troski, jaką ma on sprawować wobec obywateli, lub, jeśli odbywa się to bardziej w~duchu akcji bezpośredniej, do typu człowieczeństwa każdy powinien pokazać jakiemukolwiek drugiemu człowiekowi.

 Pisanie o masowym obywatelskim nieposłuszeństwie stawia pewne problemy ekspozycji. Takie działania zaprzeczają narracji, ponieważ polegają przede wszystkim na czekaniu. Najpierw godzinami czekanie z~unieruchomionymi rękoma, potem krótka i~intensywna interakcja z~policją, a potem znowu godzinami czekanie, tym razem w~kajdankach, w~karetce, autobusie aresztowym lub czekaniu na rozprawę na dworcu. Duża część tego doświadczenia polega na byciu w~więzieniu, różnego rodzaju procesach prawnych, praktykowaniu więziennej solidarności, procesach itp. Jest to dla mnie również trudne, ponieważ sam nigdy nie byłem w~blokadzie (chociaż zajmowałem się zwiadem i~wykonywałem inne prace pomocnicze dla blokad). Brałem udział w~miękkich blokadach na A16 i~innych miejscach, ale nie takich, które zostały atakowane. Zostałem aresztowany i~przetworzony, ale nigdy za udział w~ON, i~nigdy nie brałem udziału w~wymyślnej więziennej solidarności. Więc zamiast poskładać fragmentaryczną narrację osobistą, w~tym przypadku pomyślałem, że bardziej przydatne będzie przedstawienie innego szkolenia, które miałem: tego, wprowadzającego do technik blokowania bez użycia przemocy i~związanych z~nimi legalności, które miały miejsce tuż przed działania antyIMF w~Waszyngtonie w~kwietniu 2000 roku. Poniżej, jak zwykle, zaczerpnięto bezpośrednio z~moich notatek:

\medskip
\noindent \textit{Szkolenie prawne/niestosowania przemocy, Waszyngton}

\medskip
\noindent [Wyciągi z~notatek terenowych, 15 kwietnia 2000 roku]
\medskip

 \textit{Scena:}Ogromna piwnica kościoła, dzień przed rozpoczęciem akcji przeciwko Bankowi Światowemu i~Międzynarodowemu Funduszowi Walutowemu. Dwie młode kobiety prowadzą ,,szkolenie prawnicze'', które w~rzeczywistości jest w~większości szkoleniem z~niestosowania przemocy: każdy aspekt postępowania z~policją w~Waszyngtonie. Na widowni jest co najmniej trzysta osób, w~większości młodzi ludzie. Trenerami są weterani z~Seattle; jeden jest nieco hippisowski, drugi to punkowy rocker z~krótko przystrzyżonymi blond włosami, nieco nienormalnie trzymający garść sitowia ożypałki.

 Pierwsze lekcje dotyczą klasycznego niestosowania przemocy, a konkretnie taktyki deeskalacji. Dowiadujemy się, że zasadniczo istnieją trzy scenariusze, w~których gliniarze najprawdopodobniej staną się agresywni. Po pierwsze, gdy mają konkretne rozkazy, by cię zaatakować; w~takim przypadku niewiele możesz zrobić, aby ich powstrzymać. (,,Wszyscy widzieliśmy, jak funkcjonariusze policji w~Seattle płakali, kiedy kazano im zaatakować pokojowych demonstrantów. Nie powstrzymało ich to przed zrobieniem tego''). Inni zrobią to ze strachu, ignorancji i~z wściekłości. Tego pierwszego można uniknąć, czyniąc wszystko, co robisz, tak przejrzystym, jak to tylko możliwe. Nasi dwaj trenerzy przyprowadzają starszego mężczyznę, który daje nam mały wykład na temat przygotowań do Seattle: tłumaczył, że przez wiele miesięcy aktywiści przygotowywali swoją taktykę, a właściwie zaprosili policję na obserwację treningów, upewnili się, że dokładnie wiedzą, czego się spodziewać. 
 
 -- A mimo to zachowywali się, jakby zostali zaskoczeni. Myślę, że po prostu nie mogli uwierzyć w~to, co im mówiliśmy. 
 
 -- Oczywiście nie chcesz wszystkiego zdradzać, ale jeśli to możliwe, nie jest złym pomysłem, aby ogłosić swoje ruchy z~wyprzedzeniem każdej policji, która stoi obok ciebie, np. niech ktoś zawoła głośno: ,,Wszyscy wstaniemy teraz i~odejdziemy'', wszystko, co w~innym przypadku mogłoby wydawać się przygotowaniem do ataku.

\noindent Trener hipisów: Jedną z~najczęstszych przyczyn aresztowania na demonstracji jest sytuacja, w~której jeden z~policjantów po prostu wpada w~panikę i~zaczyna atakować ludzi. To się zdarza cały czas. Jest cała linia policji; mają rozkazy po prostu nie ustępować; i~nie wiem, może to upał, nuda, napięcie, po prostu buduje się aż do granicy wytrzymałości. Ktoś mówi, że jednemu z~nich coś się nie podoba, albo patrzy na niego dziwnie, a facet po prostu to traci. i~nagle wpada w~szał i~zaczyna atakować wszystkich w~zasięgu wzroku. Problem polega na tym, że kiedy to się stanie, inni gliniarze nie będą próbowali go zatrzymać. W każdym przypadku, o którym słyszałem, czują, że muszą go poprzeć. Więc oni też wyjdą, machając pałkami, a potem będą musieli aresztować jedną lub dwie osoby za napaść na oficera, bo inaczej nie byłoby powodu, żeby ten pierwszy facet uderzył kogokolwiek na początku. Istnieje realne niebezpieczeństwo, że taka sytuacja może przerodzić się w~poważną awanturę i~doprowadzić do zranienia wielu osób.

\noindent Trener punk: Więc kiedy coś takiego się dzieje lub po prostu wygląda na to, że może się wydarzyć, kluczem jest deeskalacja. Utrudnij policji brutalne działanie tak jak to tylko możliwe.

\noindent Hipis: Oczywiście ideałem jest upewnienie się, że to się w~ogóle nie wydarzy. Jest to zgodne z~tym, co mówiliśmy o strachu i~ignorancji. Istnieje kilka zdroworozsądkowych zasad, o których zawsze należy pamiętać w~kontaktach z~policją. Po pierwsze, zawsze staraj się patrzeć na rzeczy z~ich punktu widzenia. Na przykład, postaraj się upewnić, że żaden policjant nigdy nie czuje się osaczony, uwięziony lub otoczony, że jego plecy nie opierają się o ścianę, nie ma drogi ucieczki. Jak uwięziony szczur zaatakuje. Po drugie, zawsze bądź świadomy broni. Nigdy, przenigdy nie rób niczego, co mogłoby być interpretowane jako sięganie po pasek lub broń. Nawet nie sięgaj w~tamtym kierunku.

 Krótki gest dotyczący gestów ramion: wszystko, co wygląda jak wymachiwanie komuś palcami lub pięściami, zostanie zinterpretowane jako groźne; najmniej groźne jest trzymanie rąk opuszczonych i~dłoni do góry (Hę, oficerze? Masz na myśli mnie?) Oczywiście, jeśli już dostali rozkaz, by cię zaatakować, zrobią to bez względu na to, co ty robisz. Ale to zminimalizuje możliwość.

\noindent Punk: Więc co zrobić, jeśli jeden oficer wpadnie w~panikę i~zacznie wpadać w~szał? Pierwszą rzeczą, którą chcesz zrobić, to zmienić ton lub otoczenie. Tworzysz aurę spokoju i~ciszy, więc akt przemocy wydaje się maksymalnie niestosowny. Jedną z~technik jest po prostu skłonienie wszystkich do ,,ohhhmmm''. Wiecie, mantra. Może się rozprzestrzenić dość szybko, jeśli wszyscy wiedzą, że w~momencie, gdy usłyszysz, że jedna osoba to robi, wszyscy inni też powinni to zrobić. (Spróbujmy wszyscy: ohhmmmmmmm\ldots )

 Po drugie, wskazujesz. Wszyscy wokół jedynego gliniarza, który coś robi, wskazują bezpośrednio na niego, więc masz całkowitą ciszę, z~wyjątkiem omów i~wszystkich wskazujących i~patrzących wprost na niego.

\noindent Hipis: I~oczywiście jest to również wskazówka dla każdego, kto ma kamery wideo, aby podbiec i~zacząć wszystko filmować.

\noindent \textit{[} następuje odgrywanie ról ]

\noindent Pytanie: A jeśli zaatakuje cię cała banda gliniarzy? Mówią, że mają rozkazy wchodzić z~pałami? 

\noindent Hippie: Właściwie mieliśmy do tego dojść. Istnieje wiele taktyk, których możesz użyć. Ale prawdopodobnie najprostszą i, jak odkryliśmy w~Seattle, najskuteczniejszą, jest po prostu usiąść. Gdziekolwiek jesteś, gdziekolwiek stoisz: usiądź. Teraz, jeśli otrzyma rozkaz, by cię zaatakować, prawdopodobnie i~tak to zrobią, ale w~ten sposób po prostu nie będą mogli uwierzyć, że to nie jest niesprowokowany atak. 

 Ćwiczymy siedzenie, z~kilkoma aktywistami z~długimi papierowymi tubami na pałki policyjne w~roli gliniarzy. Następnie ćwiczymy utykanie. Kiedy masz zostać aresztowany, ważne jest, aby trwało to jak najdłużej. Ćwiczymy unoszenie się nawzajem: stwierdzamy, że dość łatwo jest unieść drugą osobę, jeśli ta osoba współpracuje. Jeśli ta sama osoba całkowicie zwiotczeje, nawet dwóm z~nas będzie niezwykle trudno ją podnieść. Policja zdaje sobie z~tego sprawę i~opracowała środki zaradcze: w~zasadzie, jeśli włożysz kciuki w~pewien punkt nacisku na twojej szyi, powoduje to ogromny ból i~natychmiast sztywniejesz. Aktywiści jednak opracowali też środki zaradcze. Znalezienie odpowiedniego miejsca zajmuje trochę czasu. Więc w~chwili, gdy poczujesz, jak jakiś gliniarz kręci ci wokół szyi kciukiem, należy krzyczeć niemal natychmiast. W ten sposób pomyśli, że go znalazł. Gdy tylko zaczną próbować cię porwać, znowu wiotczejesz, a on będzie po prostu naciskał ponownie w~tym samym miejscu. -- Widziałem to kilkanaście razy: gliniarz jest jak, co to za facet, nadczłowiek? -- Ćwiczymy również ,,stosik szczeniąt'' -- technikę, którą można zastosować, jeśli policja wyraźnie atakuje jedną osobę w~celu brutalnego aresztowania. Jeśli sześć lub siedem osób natychmiast rzuci się na tę osobę, w~rzeczywistości możliwe jest odciągnięcie ich od spodu i~pozwolenie im na ucieczkę. Jest to jednak dość desperacki środek i~absolutnie konieczne jest upewnienie się, że dana osoba wyraziła zgodę na tę taktykę przed kontynuowaniem.

 Są trzy poziomy wykroczeń: naruszenia, wykroczenia i~przestępstwa (może je zapamiętać wygodny akronim: NWP). Technicznie blokowanie ulicy mieści się w~pierwszej rubryce; tak naprawdę to nawet nie jest przestępstwo, to pogwałcenie pewnych lokalnych przepisów, w~zasadzie forma przechodzenia na czerwonym. W normalnych warunkach policja nigdy nie aresztowałaby, nie mówiąc już o zatrzymaniu nikogo za naruszenie. Dadzą ci mandat i~sobie pojadą. Kiedy jednak zajmują się działalnością polityczną, zasady są inne. Możesz oczekiwać, że będziesz przetrzymywany co najmniej przez wiele godzin, skuty kajdankami lub może spętany, prawdopodobnie przetrzymywany przez noc. 

\noindent Punk: Upewnij się, że nigdzie na twoim ciele nie ma absolutnie nic, co wygląda lub mogłoby być zinterpretowane jako ostrze. Wszystko, co wygląda jak ostrze, które ma siedem centymetrów lub więcej, jest nożem. Możesz i~zostaniesz oskarżony o posiadanie śmiercionośnej broni, jeśli masz, powiedzmy, scyzoryk szwajcarski lub podobne kieszonkowe narzędzie.

 Zasada nie dotyczy jednak obcinaczy do paznokci. Co jest bardzo ważne. Ponieważ dobry zestaw obcinaczy do paznokci przetnie zestaw plastikowych kajdanek, co będzie bardzo przydatne, gdy założą Ci kajdanki na ręce za plecami i~zostaniesz w~autobusie na około osiem godzin. Pamiętaj: aktywiści \textit{zawsze }dbają o zadbane i~przycięte paznokcie! Zawsze mamy przy sobie obcinacz do paznokci.

\noindent Hipis: To, hm, obcinacz do paznokci u stóp. Obcinacz do paznokci u dłoni nie będzie działał.

 Po aresztowaniu nadal chodzi o wstrzymanie współpracy, a co za tym idzie o to, aby każdy etap trwał jak najdłużej. Policja ma ograniczoną zdolność do określania, ile aresztowań może jednocześnie przeprowadzić; pięciuset aresztowanych, którzy odmawiają współpracy, może w~rzeczywistości zablokować ich system tak bardzo, że nie są w~stanie aresztować nikogo innego. To daje innym aktywistom wolną rękę w~blokowaniu swoich celów. Najprostszym sposobem na to jest odmowa podania nazwiska. Mało znanym faktem jest to, że nie ma prawa, które mówiłoby, że policjantowi należy podać swoje nazwisko, nawet jeśli jest się aresztowanym. Podanie fałszywego nazwiska jest nielegalne. Odmowa podania jakiegokolwiek nazwiska nie jest w~ogóle nielegalna. Oczywiście policja powie ci inaczej.

\noindent Hipis: Co prowadzi nas do niezwykle ważnego punktu. \textit{Gliny kłamią}. To jest coś, czego nie możemy wystarczająco podkreślić, ponieważ bez względu na to, ile razy to mówisz, bez względu na to, ile razy to widzisz, nawet najbardziej zatwardziałym działaczom często będzie to trochę trudne do zaakceptowania. Okłamywanie gliniarzy jest nielegalne, przynajmniej w~pewnych okolicznościach, ale nie ma absolutnie żadnego prawa mówiącego, że gliniarz nie może cię okłamywać. Nie zakładaj, że coś jest prawdą tylko dlatego, że powiedział to gliniarz. Zwłaszcza podczas przesłuchań musisz zakładać, że wszystko, co powie lub nawet sugeruje policjant, jest prawdopodobnie nieprawdziwe.

\noindent Punk: A to prowadzi nas do innego, czasem trudnego punktu. Ma to związek z~rozmową z~policją, przynajmniej w~jakikolwiek sposób, w~jaki absolutnie nie musisz. Najlepszą zasadą jest: nie rób tego. Nie nawiązuj rozmowy z~funkcjonariuszem dokonującym aresztowania. Teraz wiem, połowa z~was pewnie siedzi i~mówi do siebie: ,,Ale gliniarze to też ludzie. Dlaczego nie miałbym próbować dotrzeć do nich tak, jak do kogokolwiek innego? Dlaczego nie miałbym chcieć nawiązać kontaktu z~ludźmi?'' Tak, wiem. Wszyscy tak się czuliśmy. Ale jest powód. Powodem jest to, że może to skończyć się poważnym pobytem w~więzieniu. Musisz pamiętać, że kiedy rozmawiasz z~gliną, nie rozmawiasz głównie z~człowiekiem; głównie rozmawiasz z~przedstawicielem struktury instytucjonalnej, a ta struktura instytucjonalna chce widzieć twój tyłek w~więzieniu. Taki jest ich cel. Jeśli gliniarz rozmawia z~tobą, próbując uzyskać informacje, prawdopodobnie jest to motyw. Wykorzystają twoje pragnienie dotarcia i~nawiązania ludzkiego połączenia i~użyją go przeciwko tobie, tak jak zrobiliby to z~czymkolwiek innym. i~są szanse, że nie masz pojęcia, co oni naprawdę zamierzają. Na przykład: powiedzmy, że siedzisz i~czekasz na pobranie odcisków palców i~wchodzi przyjazny funkcjonariusz, a on mówi: ,,Więc po prostu tego nie rozumiem. Jak myślisz, co próbujesz osiągnąć, siedząc na tym placu?''. Więc mówisz mu wszystko o tym, jak próbowałeś zwrócić uwagę na dominację korporacji, albo zaczynasz szczegółowo opisywać skutki polityki dostosowania strukturalnego w~Mozambiku, a może nawet on uważa to za interesujące, ale w~rzeczywistości chodzi o to, jedynym powodem, dla którego poprosił, było uzyskanie wyroku skazującego. Musieli ustalić, że faktycznie byłeś na placu. W połowie przypadków nawet nie pamiętają, kim był oficer dokonujący aresztowania, gdzie zostałeś aresztowany lub co miałeś zrobić. Więc teraz cię dopadli.

 Im więcej mówisz, tym łatwiej cię obciążyć.

\noindent Hipis: Dlatego bez względu na to, jak bardzo gliniarze próbują cię podrywać, powinieneś powiedzieć tylko jedno: ,,Wybieram milczenie. Domagam się prawnika''. W kółko, jeśli musisz. W Seattle wymyśliliśmy małą pieśń, którą wszyscy zaśpiewaliśmy razem w~więźniarce:

\noindent \textit{Wybieram milczenie \newline (mmmhmmmmhmm) \newline Domagam się wizyty u prawnika. \newline (och taaaak)}

\noindent Punk: Tak, to \textit{naprawdę }ich denerwowało w~Seattle.

\noindent Hipis: A jeśli nawet nie chcesz tego mówić: wydrukowaliśmy te ładne małe plakietki, które możesz przykleić bezpośrednio do ubrania, które mówią: ,,Wybieram milczenie. Domagam się prawnika''. W przypadku aresztowania: wskaż. Nazywamy to kartą ,,Przejdź bezpośrednio do więzienia''.

 \ldots 

 Na komisariacie zostaniesz sfotografowany i~zostaną pobrane odciski palców. Niektórzy aktywiści robią miny, gdy są fotografowani lub w~inny sposób próbują wszystko załatwić, ale pamiętaj, że jeśli to zrobisz, mogą stać się agresywni. Wiele osób nakłada na palce wazelinę, aby zamazać odciski palców (to nielegalne, więc oczywiście w~żaden sposób nie zachęcamy do tego). Poważnie jednak, powinieneś uważać na wazelinę, bo jeśli jest na twojej skórze i~dostaniesz gazem, to strasznie się pali. Dlatego niektórzy ludzie wkładają te rzeczy do rękawów lub klap, aby użyć ich później, ale z~tym też byłbym ostrożny.

 Druga połowa szkolenia jest w~dużej mierze poświęcona więziennej solidarności, z~kilkoma odgrywanymi rolami w~zaimprowizowanym policyjnym autobusie zrobionym z~krzeseł. Jeśli ukrywasz nazwiska, zwykle będziesz chciał mieć serię żądań dotyczących tego, czego chcesz, w~zamian za ich podanie. Idealna lista życzeń: każdy otrzymuje te same zarzuty (tj. wycofanie wszystkich zarzutów o przestępstwo, które są zazwyczaj dość losowo oceniane w~odniesieniu do konkretnych osób), dostęp do leczenia dla rannych lub tych, których warunki tego wymagają, możliwość mandatu bez stałego zapisów. Jeśli są naprawdę zakorkowani i~chcą się ciebie pozbyć, zwykle możesz dostać większość tego, o co prosisz, zwykle z~wyjątkiem zarzutów o przestępstwo. Ale będzie to wymagało negocjacji i, ogólnie rzecz biorąc, będziesz chciał zacząć od stworzenia struktury podejmowania decyzji dla aresztowanych. Na przykład zgódź się na to, aby jedna osoba działała jako facylitator w~wewnętrznej dyskusji, a druga jako rzecznik/łącznik z~glinami. Są też kryzysy, które wymagają natychmiastowej reakcji. Na przykład, policja generalnie uważa, że  osoby kolorowe mogą być traktowane bardziej brutalnie; jeśli mogą, będą próbowali dzielić ludzi według rasowych lub etnicznych linii. Możliwy scenariusz: jeśli w~autobusie jest garstka kolorowych ludzi, gliniarze mogą usunąć jednego za specjalne zarzuty, może nawet pobić ich na zewnątrz lub sprawić, że pomyślisz, że są poturbowani, żeby cię zastraszyć. Przechodzimy przez różne możliwe reakcje, z~eskalującą taktyką prowadzącą do tego, że wszyscy rzucają się w~przód i~w tył, aby potrząsnąć autobusem (jeśli robisz to wystarczająco długo, możesz faktycznie przewrócić autobus). Choć to ostatnie jest ostatecznością, lepiej nie próbować, chyba że chcesz być pewny, że wszyscy w~okolicy dowiedzą się, że dzieje się coś bardzo, bardzo złego.

 W porównaniu z~imprezą uliczną można by powiedzieć, że sprawy zatoczyły koło: od nacisku na doświadczenie aktywistów -- w~szczególności doświadczenie przyjemności i~biesiadowania -- do działań, które w~dużej mierze dotyczą wywierania wpływu na cel lub publiczność poprzez chęć poddania się bólowi . Być może nie należy posuwać się w~tym za daleko. Działania, które nastąpiły 16 kwietnia, były w~rzeczywistości obydwoma jednocześnie. Na każdym większym skrzyżowaniu znajdowały się zespoły blokujące, otoczone blokującymi łączącymi ręce, ale także muzyka, taniec, teatr uliczny i~okresowe wizyty kolorowych gigantycznych lalek. Nawet nastrój zmieniał się w~tę i~z powrotem między momentami intensywnego cichego oczekiwania, gdy rozchodziła się wiadomość o możliwym ataku policji, do festiwalu, a potem z~powrotem do cichej intensywności.

 Niemal każdy przedstawiciel obywatelskiego nieposłuszeństwa zgodzi się, że chodzi o to, by ujawnić prawdziwą naturę systemu niesprawiedliwej władzy. Ktoś ujawnia nieodłączną przemoc systemu, ujawniając dokładnie to, co jest on gotów zrobić tym, którzy go kwestionują; nawet jeśli ci, którzy go kwestionują, nigdy nie kiwną palcem, by skrzywdzić kogokolwiek innego. Ma więc wyraźnie na celu wywrzeć wpływ na odbiorców. Głównym punktem niezgody -- a amerykańscy pacyfiści są w~tej kwestii mniej więcej równo podzieleni -- jest to, jak. Niektórzy zwolennicy niestosowania przemocy twierdzą -- jak to często Gandhi miał w~zwyczaju -- że nie chodzi o wywieranie presji, ale o odwołanie się bezpośrednio do człowieczeństwa przeciwnika, aby dać przykład, który ostatecznie zademonstruje rządzącym błędy w~ich metodach. Inni twierdzą, że to naiwne, że chodzi właśnie o zastosowanie formy ,,przymusu bez przemocy'' -- formy walki moralnej, analogicznej na swój sposób do tej prowadzonej na linii pikiet. Jak już wyjaśnił przykład blokady, często trudno jest odróżnić jedno od drugiego\footnote{W każdym razie można by argumentować, że to rozróżnienie jest psychologicznie naiwne: ludzie mają notoryczną tendencję do przyjmowania nowych postaw moralnych - tych, które szczerze przyjmują - gdy tylko okoliczności ich do tego zobowiązują.}.

 Napięcie staje się najbardziej widoczne, gdy trenerzy powyżej omawiają, jak zachowywać się w~obliczu policji. Z jednej strony, aby w~ogóle zaangażować się w~taktykę pokojową, praktykujący obywatelskie nieposłuszeństwo musi nauczyć się patrzeć na sprawy z~punktu widzenia przeciwnika. Jest to podstawowa cecha każdego treningu niestosowania przemocy. Jest to oczywiście bardzo podobne do tego, które robi się również na spotkaniu, ale kontrast nie mógłby być pełniejszy. Jak wskazałem w~poprzednim rozdziale, konsensus można znaleźć tylko wśród wspólnoty równych, gdzie nie ma instytucjonalnej struktury władzy, która czyniłaby z~uczestników jedynie reprezentantów czegoś innego. Policja jest właśnie granicą, punktem, w~którym te struktury władzy instytucjonalnej -- te, z~którymi nie można negocjować jak równy z~równym -- utwierdzają się w~życiu publicznym. Policja wykonuje rozkazy i~oczekuje, że zrobisz to samo. Jeśli się nie zastosujesz, są gotowi użyć siły, aby zapewnić uległość. W rzeczywistości są murem, który władza ukazuje światu. Ci, którzy blokują IMF lub WTO, rzadko konfrontują się z~osobami kierującymi takimi instytucjami. Konfrontują się z~zewnętrzną twarzą swojej władzy, którą jest policja. Oznacza to, że nawet jeśli stosujemy czysto gandhijski paradygmat i~próbujemy odwołać się do człowieczeństwa przeciwnika, tak naprawdę nie zwracamy się do policji, ale do ich pracodawców. W rzeczywistości osobiste opinie gliniarzy ulicznych (na przykład) w~sprawach handlu międzynarodowego są prawdopodobnie znacznie bliższe protestującym niż bankom, biurokratom czy urzędnikom rządowym, których chronią. To jest nieistotne. Wielu gliniarzy z~Seattle płakało, gdy wydano rozkaz zaatakowania łagodnych, idealistycznych nastolatków. Ale, jak nasz trener podkreślił, zaatakowali.

 Z drugiej strony, jak pokazuje szkolenie, istnieje wiele powodów, dla których można chcieć przynajmniej moderować pewne bezpośrednie formy zachowań policji. To z~konieczności wiąże się z~identyfikacją wyobrażeniową. Mimo to jest to identyfikacja wyobrażeniowa w~najbardziej absolutnie minimalnym rodzaju. Jest mniej więcej na tym samym poziomie, co w~przypadku niebezpiecznego zwierzęcia: nigdy nie pozwól mu czuć się osaczonym, skup się na nim, gdy zacznie zachowywać się agresywnie, nigdy nie wyglądaj, jakbyś sięgał po broń. Każda próba dalszej identyfikacji z~wyobraźnią uderza w~ścianę. Większość doświadczonych działaczy ma szczegółową wiedzę na temat różnych rodzajów pojazdów policyjnych (każdy uczy się tego szybko, jeśli ktoś spędza dużo czasu, powiedzmy, obserwując, podczas gdy inni naklejają plakaty), sprzęt policyjny, taktykę oraz różnorodność i~skuteczność policji uzbrojenie. Ale prawie nie mają pojęcia, co myśli przeciętny gliniarz. Właściwie nie sądzę, bym kiedykolwiek uczestniczył w~większej akcji, gdzie nie słyszałem choćby jednego aktywisty zastanawiającego się na głos, co może się dziać w~głowach policji, w~taki sposób, aby było jasne, że nie mają najmniejszego pojęcia. Lub, na przemian, tylko po to, by odrzucić całe pytanie. Oto kilka przykładów tych ostatnich, które zanotowałem:

\medskip
\noindent Rozmowa dwóch blokujących, 16 kwietnia 2000

\noindent Blokujący 1: Jak myślisz, co myślą gliniarze? To znaczy, kiedy kazano im gazować lub sprayować ludzi, którzy najwyraźniej nikogo nie skrzywdzą. Albo, w~porządku, oto jakiś facet, który podchodzi do grupy piętnastolatków, które stoją tam, trzymając się za ręce, i~zaczyna uderzać jednego z~nich pałką. Co on może myśleć, kiedy to robi?

\noindent Blokujący 2: Myślę, że myślą: ,,Hej, i~tak dostanę zapłatę. Cokolwiek''.

\medskip
\noindent Rozmowa między dwoma nastoletnimi Black Bloc'erami, 20 czerwca 2001 roku

\noindent Zamaskowana Black Blocker: Wiesz, dużo o tym myślałam. O glinach i~o tym, co z~nas robią. i~myślę, że głównie się boją. Nie mają pojęcia, co się dzieje. Nie mają pojęcia, kim jesteśmy ani co możemy zrobić. To w~zasadzie tylko banda zwykłych ludzi z~klasy robotniczej, którzy są przestraszeni i~zdezorientowani jak wszyscy inni. Nie sądzę, że powinniśmy nienawidzić gliniarzy.

\noindent Zamaskowany Black Blocker: Och, daj mi spokój.

\noindent Zamaskowana Black Blocker: Nie, naprawdę. To tylko banda ludzi z~klasy robotniczej i~są przestraszeni.

\noindent Zamaskowany Black Blocker: Słuchaj, każdy gliniarz, z~którym miałem do czynienia na ulicy, był dupkiem i~każdy gliniarz, którego znałem osobiście, też był dupkiem. Kogo obchodzi, co myślą?

\medskip

 Sytuacja jest tym bardziej dotkliwa, ponieważ większość anarchistów zdaje sobie sprawę, że historycznie, kiedy anarchiści wygrywają -- kiedy taktyka akcji bezpośredniej prowadzi do obalania rządów -- prawie zawsze dzieje się tak dlatego, że policja odmawia strzelania. Z mojego doświadczenia wynika na przykład, że przed prawie każdą większą mobilizacją w~północnoamerykańskim mieście, na przykład, policja grozi strajkiem, a w~kręgach aktywistów zawsze są niekończące się spekulacje na temat możliwości zawarcia sojuszu. Nigdy do niczego nie dochodzi. Ale pomysł jest zawsze obecny. Stąd ostateczny dylemat: nie można wygrać z~drugą stroną, jeśli nie chce się z~nimi rozmawiać. Z drugiej strony, jak zauważył trener hippisów, zagadywanie gliniarzy może okazać się katastrofalne z~prawnego punktu widzenia.

 Wydaje mi się, że ten dylemat nie dotyczy tylko obywatelskiego nieposłuszeństwa. Jest to nierozerwalnie związane z~samą naturą akcji bezpośredniej. Ideałem przy prowadzeniu akcji jest zachowywanie się tak, jakby już żyło się w~wolnym społeczeństwie, gdzie każdy mógłby być traktowany po prostu jak człowiek. Ponieważ nie akceptuje się zasadności większej struktury instytucjonalnej, która przypisuje mężczyznom i~kobietom określone role -- jako dyrektorów korporacji, strażników więziennych, funkcjonariuszy ds. społecznych, negocjatorów handlowych itd. -- odmawia się uznania ich w~tych rolach, tylko po prostu jako mężczyźn i~kobiety, których czyny muszą być oceniane według tych samych standardów, co innych osób. Nieuniknionym rezultatem jest to, że są postrzegani jako angażujący się w~skandaliczne akty przemocy. Konsekwencją jest to, że należy do nich podchodzić jak do jednostek zdolnych do przekroczenia swojej roli; ale tu, sam fakt, że działają na rzecz przemocowej struktury instytucjonalnej sprawa, że jest to prawie niemożliwe.

 Ten sam dylemat pojawia się czasem w~sądzie. Ten, na przykład, pochodzi z~podręcznika szkoleniowego nieposłuszeństwa obywatelskiego ACT UP:

 Niektórzy demonstranci odmawiają częściowej lub całkowitej współpracy w~ramach procedur sądowych; odmawiają przyjęcia zarzutu, zatrzymania lub przyjęcia prawnika, wystąpienia w~sądzie, mówienia do sędziego jako symbolu władzy sądowej (ale raczej rozmawiania z~nim jako z~bliźnim), zajmowania stanowiska lub przesłuchać świadków. Mogą wygłosić przemówienie do osób zgromadzonych na sali sądowej lub po prostu leżeć na podłodze, jeśli są wnoszeni, lub próbować wyjść, jeśli nie są przymusowo skrępowani. Kary za taki brak współpracy mogą być surowe, ponieważ wielu sędziów traktuje takie działanie jako osobisty afront, a także obrazę sądu. Z drugiej strony niektórzy sędziowie ignorują takie zachowanie lub próbują komunikować się z~demonstrantami\footnote{\url{http://www.actupny.org/documents/CDdocuments/Legal.html}\textit{, dostęp 22 lipca 2005 roku}}.

 Jest to, jak można sobie wyobrazić, niezwykle ryzykowna strategia. Bardzo niewielu posuwa się tak daleko, będąc na rozprawie, chociaż logika działań bezpośrednich sugeruje, że tak naprawdę należy to zrobić. Ilustruje również niektóre niejasności o źle zdefiniowanej granicy między akcją bezpośrednią a obywatelskim nieposłuszeństwem: do jakiego stopnia warto po prostu zaakceptować wszystko, co państwo ze spokojem wymierza, aby uwidocznić swój aparat ograniczeń, a do jakiego stopnia chodzi o to, by upierać się przy swoim prawie do działania tak, jakby ten aparat nie istniał. Można śmiało powiedzieć, że w~praktyce prawie nikt nie stosuje jednego podejścia przy całkowitym wykluczeniu drugiego. To zawsze coś w~rodzaju mieszanki.

\medskip
\noindent  PRZYKŁAD PIĄTY: AKCJA CZARNEGO BLOKU
\medskip

\noindent Według strony internetowej \textit{Infoshop.org} ,,Czarne bloki dla opornych”

\textit{ Czarny blok to zbiór anarchistów i~anarchistycznych grup afinicji, które zbierają się razem na konkretnej akcji protestacyjnej. Smak czarnego bloku zmienia się z~działania na działanie, ale głównym celem jest zapewnienie solidarności w~obliczu represyjnego państwa policyjnego i~przesłanie anarchistycznej krytyki wszystkiego, co jest protestowane tego dnia}\footnote{\url{http://www.infoshop.org/blackbloc_faq.html}\textit{, dostęp 25 sierpnia 2000.}}.

 Czarny blok jest więc taktyką. Nie jest to, jak wielu się wydaje, grupa czy organizacja.

 Taktyka Czarnego Bloku powstała w~Niemczech, w~ruchu skłoterów w~latach 80.. Zasadniczo był to sposób na stworzenie anonimowości: młodzi anarchiści broniący skłotów przed atakiem policji lub biorący udział w~marszach lub wiecach, wszyscy ubierali się w~identyczne czarne maski narciarskie i~identyczne czarne skórzane kurtki. Wyrażenie ,,czarny blok'' wydaje się wymysłem niemieckich mediów, a niektórzy twierdzą, że niemieckiej policji. Przyjęło się, ponieważ miało taki wizualny sens: tysiąc ludzi, wszyscy w~czerni, w~gęstym szyku, często z~rękoma połączonymi lub otoczonymi sztandarami, które również działają jak tarcze, zwieńczone czerwono-czarnymi flagami, nie był widokiem szybko zapominanym. Nawet jeśli blok nie robi nic poza marszem, ich obecność nadaje rodzaj konkretnej rzeczywistości anarchizmowi i~jego potencjałowi do radykalizacji marszu lub eskalacji jego taktyki. W Europie taktyka czarnego bloku wkrótce upowszechniła się poza Niemcy, a na dużych demonstracjach, bloc stał się znany z~podejmowania walki z~faszystami lub policją.

 Amerykańscy anarchiści po raz pierwszy zaczęli eksperymentować z~Czarnymi Blokami podczas protestów przeciwko pierwszej wojnie w~Zatoce Perskiej w~1991 i~1992 roku (w jednym przypadku anarchiści faktycznie wybili szyby w~Banku Światowym, wyczyn nigdy nie został powtórzony). Zwodnicze byłoby jednak postrzeganie północnoamerykańskich czarnych bloków po prostu jako transpozycji modelu europejskiego. Istnieją ważne różnice. Niektóre są stylistyczne: amerykanie zastąpili kurtki skórzane czarnymi bluzami z~kapturem wywodzącymi się z~punkowej kultury skejtów z~Zachodniego Wybrzeża; zamiast masek narciarskich, głównie czarne bandany. Co ważniejsze, istnieją bardzo różne oczekiwania dotyczące przemocy. W większości dużych miast Europy działają ruchy faszystowskie. Postrzegają anarchistów, prawie tak samo jak imigrantów, jako swoich naturalnych wrogów. Być otwartym anarchistą i~żyć zgodnie z~kodeksem niestosowania przemocy, oznacza zgodę na codzienne ryzykowanie życiem, lub przynajmniej świadomość bycia całkiem regularnie pobitym\footnote{Pamiętam, jak kiedyś pokazałem darmowe centrum komputerowe w~ABC No Rio przyjacielowi z~Paryża; jego natychmiastową reakcją było: ,,ale czy faszyści nie przychodzą i~nie próbują tego wszystkiego zniszczyć?''. Innymi słowy, tak jak pacyfista, który z~zasady odrzuca wojnę, może spodziewać się regularnej konfrontacji z~hipotetycznym argumentem ,,Co byś zrobił z~nazistami?'', gdzie pokojowi anarchiści muszą stawić czoła temu samemu problemowi w~znacznie mniej hipotetycznych kategoriach. Niektórzy po prostu akceptują fakt, że będą okresowo bici; inni walczą. Interesujące jest to, że nawet w~USA główną grupą anarchistyczną, która nigdy nie akceptuje braku przemocy, jest ARA, ,,Akcja Antyrasistowska'', która regularnie konfrontuje się z~nazistami.}.W Stanach Zjednoczonych większość anarchistów ma szczęście mieszkać w~miejscach, gdzie są stosunkowo odizolowani od takich niebezpieczeństw. Tak więc tam, gdzie w~Europie mniej lub bardziej oczekiwany jest pewien stopień przemocy, w~Stanach Zjednoczonych Czarne Bloki zdołały wypracować coś, co można uznać za najbardziej agresywną możliwą wersję niestosowania przemocy. Zasadniczo słowo ,,przemoc'' jest interpretowane jako wyrządzanie krzywdy lub zadawanie bólu i~cierpienia innej żywej istocie. Czarne Bloki nie atakują żywych stworzeń. Są jednak skłonni stosować o wiele bardziej konfrontacyjne taktyki niż inni aktywiści: na przykład łączenie ramion w~celu odepchnięcia linii policyjnych, a nawet, jak na A16, noszenie płotów z~siatki, aby napierać na nich; wznoszenie barykad ze śmietników, skrzynek na gazety i~innych ulicznych szczątków; nawet praktykuje ,,de-aresztowania'', wyrywając aresztowanych z~linii policyjnych i~przecinając im kajdanki. Posługują się też oczywiście repertuarem czysto symbolicznych czynności: malowania sprayem, bębnienia na latarniach, palenia flag (lub zdejmowania oficjalnych flag i~podnoszenia czerwono-czarnych).

 Jednak dla tych, którzy brali udział w~takich akcjach, niezwykle istotne jest poczucie autonomii, jakie stwarza nacisk na solidarność i~wzajemną obronę. Kiedy dołączasz do Czarnego Bloku, stajesz się nie do odróżnienia od wszystkich innych uczestników. W rzeczywistości mówisz: ,,Każdy czyn dokonany przez każdego z~nas równie dobrze mógłby zostać dokonany przeze mnie''. Jednocześnie wiesz, że każdy z~pozostałych uczestników troszczy się o ciebie, pilnuje twoich pleców, że podczas gdy wszyscy starają się uniknąć aresztowania, jedyną sytuacją, w~której większość zechce zaryzykować aresztowanie, będzie uratowanie ciebie przed aresztowaniem. Właśnie to dla tak wielu sprawia, że  taktyka Czarnego Bloku jest tak wyzwalająca: jest to sposób na stworzenie jednej, ulotnej chwili, gdy autonomia jest realna i~natychmiastowa, przestrzeń wyzwolonego terytorium, w~którym nie obowiązują już prawa i~arbitralna władza państwa, w~której sami wyznaczamy linie sił. Robienie tego bez jednoczesnego naruszania zasad niestosowania przemocy jest oczywiście sprawą delikatną i~złożoną i~jest przedmiotem niekończącej się debaty.

 Debaty te, najbardziej znane, skupiły się wokół kwestii niszczenia własności, praktyki, w~którą od czasu do czasu angażują się Czarne Bloki. Tutaj najsłynniejszym stwierdzeniem jest prawdopodobnie ,,Komunikat Czarnego Bloku'' Kolektywu Acme wydany po działaniach WTO w~Seattle:

 \textit{Własność prywatną należy odróżnić od własności osobistej. Ta ostatnia opiera się na użyciu, podczas gdy pierwsza opiera się na handlu. Założeniem własności osobistej jest to, że każdy z~nas ma to, czego potrzebuje. Założeniem własności prywatnej jest to, że każdy z~nas ma coś, czego ktoś inny potrzebuje lub chce. W społeczeństwie opartym na prawach własności prywatnej ci, którzy są w~stanie zgromadzić więcej tego, czego potrzebują lub chcą inni, mają większą władzę. Co za tym idzie, sprawują większą kontrolę nad tym, co inni postrzegają jako potrzeby i~pragnienia, zwykle w~interesie zwiększenia własnego zysku.}

\textit{ Twierdzimy, że niszczenie mienia nie jest aktem przemocy, chyba że niszczy życie lub powoduje ból w~tym procesie. Zgodnie z~tą definicją własność prywatna -- zwłaszcza własność prywatna korporacji -- jest sama w~sobie nieskończenie bardziej brutalna niż jakiekolwiek działania podjęte przeciwko niej.}

 Wynika z~tego, że zniszczenie SUV-a, gdy jest na wyprzedaży, jest legalnym aktem, ale zniszczenie tego, który stał się czyimś osobistym środkiem transportu, nie jest; rozbicie okna Starbucks lub Niketown jest legalnym działaniem, ale rozbicie kawiarni lub sklepu obuwniczego prowadzonego przez właściciela jest całkowicie bezprawne. Ogólnie rzecz biorąc, takie ograniczenia są skrupulatnie przestrzegane. Kiedy dochodzi do niszczenia mienia, cele są badane z~wyprzedzeniem i~często oferowane jest jakieś wyjaśnienie: na przykład, gdy podczas akcji w~Quebec City jedna grupa koligacyjna zdemolowała stację benzynową należącą do Dutch Shell i~namalowała obok niej sprayem słowa ,,Pamięci Kena Saro-Wiwy''.

 Ponieważ czarne bloki stały się tak utożsamiane z~niszczeniem mienia w~świadomości społecznej, ważne jest, aby podkreślić, że nie jest to ich główny cel. W rzeczywistości większość Czarnych Bloków w~ogóle nie stosuje tej konkretnej taktyki. Prawdziwym celem jest zradykalizowanie taktyk i~przekazów oraz, w~coraz większym stopniu, zapewnienie wsparcia mniej doświadczonym i~bardziej bezbronnym protestującym. Stąd ,,Rewolucyjny Blok Antykapitalistyczny'' na A16, zaledwie cztery miesiące po Seattle, podjął wyraźną decyzję, by nie angażować się w~niszczenie własności, a jedynie wspierać blokady. ,,Rewolucyjny Blok Antyautorytarny'' (RAAB), który brał udział w~działaniach wokół konwencji republikanów w~Filadelfii 1 sierpnia 2000 roku, angażował się w~pewne ograniczone ataki na mienie -- głównie na samochody policyjne i~inne symbole władzy państwowej -- ale miały one głównie na celu odciągnięcie policji od blokad. Poniższy opis jest moim własnym doświadczeniem z~tym blokiem, nawet jeśli, jak większość przykładów, których użyłem, ma on wychodzić poza ramkę przynajmniej trochę.

 Nie miałem nic wspólnego z~planowaniem Philly RAAB i~wpadłem na nich trochę przez przypadek. W tym czasie współpracowałem z~R2K Media Collective i~zostałem wysłany na ulice, aby zdać relację z~akcji działaczom medialnym, którzy z~kolei starali się przekazać mediom korporacyjnym informacje o wydarzeniach z~tamtych dni. Krótko po przybyciu do Logan Circle wpadłem na małą kolumnę około pięćdziesięciu anarchistów z~czarnego bloku\footnote{Wyrażenie ,,w czarnym bloku'' oznacza ubrane na czarno i~w odpowiednim szyku.}, poruszających się na południe na Osiemnastej i~Franka i~postanowiłem do nich dołączyć.

\medskip
\noindent \textit{Protesty na Republikańskich Konwentach, centrum Filadelfii}

\medskip
\noindent [Notatki polowe, 1 sierpnia 2000 roku, 15:55 w~bardzo upalny dzień]
\medskip

 Szacuję, że w~kolumnie jest około pięćdziesięciu osób, przeważnie ubranych na czarno, przeważnie zamaskowanych. Równowaga płci wydaje się może 60/40. W okolicy jest zadziwiająco niewielu policjantów: tylko trzech stoi na rogu, który prowadzi do Logan Square. Jest jednak już jeden kamerzysta newsów.

 Na początku maszerują, śpiewając:

 
\noindent 2,4,6,8 \textit{Jebać państwo policyjne}

 Po chwili ktoś zaczyna bardziej wymyślną pieśń i~wszyscy ją podchwytują:

\noindent  1,2,3,4 \textit{Zjedz bogaczy, nakarm biednych \newline 5,6,7,8 \newline Zorganizuj rozbicie państwa! }

 O 16.00 skandowaniem ,,Zamknij ich! Zamknij ich! rozpoczynamy wijącą się peregrynację ulicami na północ od ratusza, ciągnąc na ulicę pudełka z~gazetami i~śmietnikami, by zablokować ruch uliczny, ciągnąc śmietniki, by zmontować prowizoryczne barykady, śpiewając, wzywając przechodniów, by się do nas dołączyli, ale zawsze zaraz po przejściu\ldots  Blokowcy wydają się w~wieku od szesnastu do dwudziestu pięciu lat, z~nielicznymi starszymi aktywistami; kilkoro ma czerwone i~czarne bandany. Właściwie nie jest to technicznie ,,blok'', ktoś mi wyjaśnia, ponieważ klasyczna taktyka bloków polega na tworzeniu gęstych kwadratów przy użyciu sztandarów (lub tarcz) do ochrony. To bardziej ,,rój''. Chodzi o to, aby pozostać jak najbardziej mobilnym.

 Na rowerze towarzyszy nam jeden facet, zdemaskowany, niosący kamerę wideo. Ludzie wciąż na niego krzyczą, zakładając, że jest gliną. Ciągle temu zaprzecza.

-- Wiesz, między fryzurą na jeża, a atletyczną budową, wyglądasz jak jeden z nich -- wskazuję.

-- Co mogę zrobić? -- on mówi. -- Jestem w~wojsku!

 Amy, dziennikarka IMC, która była już z~nimi, kiedy dołączyłem, mówi mi, że grupa zmierzała w~kierunku hotelu Four Seasons na Logan Circle i~zaczęła barykadować ulicę, gdy policja zaczęła się do nich zbliżać; szybko wystartowali. To było tuż przed moim dołączeniem. Zanim schodzimy Osiemnastą w~stronę spotkania z~resztą bloku, zostaliśmy wykryci i~wkrótce śledzeni przez szwadron składający się z~kilkunastu gliniarzy na motocyklach. Przedzieramy się przez wąskie uliczki, kiedy tylko to możliwe, skręcając w~złą stronę, w~większości pustymi jednokierunkowymi ulicami, chociaż policjanci na motocyklach również ignorują przepisy ruchu drogowego.

 Punkty kulminacyjne zaczynają się na Seventeenth i~Walnut, pierwszym punkcie, w~którym napotykamy dość gęsty ruch miejski. Trzech zamaskowanych aktywistów wyskakuje na ulicę i~próbuje wyłączyć zatrzymany autobus miejski. Właściwie jest to całkiem proste: wystarczy podnieść mały panel z~tyłu autobusu, gdzie znajduje się wyłącznik, który po prostu wyłącza silnik. Oto jak później wyjaśnia mój przyjaciel Brad: 
 
 -- Właściwie to nie jest nawet szkoda na własności. Po prostu zatrzymujesz autobus. 
 
 Zatrzymane autobusy tworzą oczywiście naturalne barykady. Jednak po kilku sekundach podjeżdża do nich dwudziestu dwóch gliniarzy na rowerach. Trójka biegnie, policjanci ruszają w~pościg. W ciągu kilku sekund pięciu aktywistów zostaje zablokowanych przy budynku na północ od skrzyżowania. Tuzin gliniarzy zeskakuje z~rowerów, zmusza ich do położenia na ziemi, szarpie ręce za plecy i~związuje plastikowymi kajdankami, podczas gdy inni tworzą z~rowerów rodzaj ogrodzenia.

 Wszystko się zatrzymuje. Dzieciaki z~Czarnego Bloku przemykają przez ulicę, bez masek, z~chustami na szyjach, oceniając sytuację. Na każdego gliniarza jest nas tylko dwoje, a szanse nie są wystarczająco dobre, by rozważać zamieszki. Wściekle wciskam guziki na pożyczonej komórce, próbując uzyskać pomoc prawną. Dostaję tylko sygnały o zajętości i~pocztę głosową.

-- Chcesz tutaj działać legalnie? -- pytam kogoś, kto wydaje się wyróżniać w~obserwowaniu sceny.

-- Tak, zdecydowanie.

-- A co z~mediami?

-- Pewnie.

 Łączę się z~IMC. Próbuję namówić ludzi z~IMC, żeby wezwali do mnie pomoc medyczną i~prawną. Amy przesłuchuje oszołomionego przechodnia w~garniturze. Starsza czarnoskóra kobieta, o której później dowiedziałem się, że jest aktywistką z~Nowego Jorku o imieniu Lucinda, podchodzi, by opisać scenę za rowerami. 
 
 -- Jeden z~nich skarżył się, że ma za ciasne kajdanki -- mówi mi. -- Więc dociągnęli je jeszcze mocniej 
 
  Okazuje się, że inny z~aresztowanych to fotograf z~US News i~World Report, ubrany w~czarną bluzę, ale bez maski; najwyraźniej nie stara się przekonać gliniarzy, że jest dziennikarzem.

 To trwa co najmniej dziesięć minut. Spędzam trochę czasu na rozmowie z~Lucindą. (Ona mówi o swoich wnukach.
 
  -- Wiesz -- mówię -- po prostu myślałam, że dzisiaj będę mogła opowiedzieć moim wnukom, podczas gdy\ldots 

  -- Tak, ale mogę im o tym powiedzieć w~tej chwili). 
  
  W końcu, przyjeżdżają medycy. Potem jakiś facet z~grupy prawnej. W tym momencie resztki Bloku zbierają się na naradę i~decydują, że nie mogą już nic więcej zrobić. Czas pomaszerować na południe na ich spotkanie. Prawie tak szybko, jak tylko zaczynamy to robić, napotykamy istną armię protestujących maszerującą na północ od demonstracji Mumii. Są czerwone transparenty z~napisem ,,Uwolnić Mumię'' i~wielu ludzi SLAM na czele i~przynajmniej jedna duża grupa w~identycznych żółtych koszulkach i~czapkach bejsbolowych.

 Nagle wszystko się zmieniło. Mamy przytłaczające liczebność. Krótka konferencja i~wszyscy zaczynamy maszerować w~kierunku twierdzy z~rowerów, gdzie aresztowani mają zostać zabrani do furgonetki, która właśnie podjechała z~północy. Policja zostaje natychmiast otoczona. Bomba z~czerwoną farbą rozpryskuje się na ścianie tuż nad nimi. Bomba dymna ląduje kilka metrów na północ, gdzie kolejna grupka gliniarzy broni furgonetki. Okazuje się, że spóźniliśmy się o chwilę. Właśnie udało im się wepchnąć aresztowanych do furgonetki, przez co prawie niemożliwe jest ich sprowadzenie z~powrotem. Więc zamiast tego bardzo wściekły tłum omiata i~blokuje pojazd. -
 
 - PIERDOL SIĘ! -- para zamaskowanych dzieciaków krzyczy na gliniarzy, jakieś dziesięć centymetrów od ich twarzy. 
 
 Gliny wyglądają na przerażonych. Blok roi się, krzyczy, wygląda tak groźnie, jak można wyglądać bez faktycznego fizycznego ataku. Trwa to jednak mniej niż minutę. Potem, jak fala, znów się cofamy. Kiedy wyjeżdżamy, obserwuję, że policja poniosła co najmniej jedną ofiarę, lub w~pewnym sensie ponieśli: jeden niezwykle gruby oficer leży na ziemi, najwyraźniej upadł z~napięcia i~upału. Dwóch innych wachluje go i~podaje sole trzeźwiące.

 W końcu maszerujemy w~dół do miejsca spotkania, wzdłuż Szesnastej i~Market, gdzie miały się połączyć trzy kolumny Czarnego Bloku. Inni już tam są, mieszają się ze zwolennikami Mumii, trzema szczudlarzami przebranymi za czerwono-żółte ptaki i~elementami Rewolucyjnego Anarchistycznego Bloku Klaunów: niektórzy w~tęczowych perukach, inni bawią się wysokimi na półtora metra rowerami, grając na prowizorycznych instrumentach, śpiewając piosenki. Jest ich tak wiele, że nie widzę ich końca. Wygląda na to, że są nas dosłownie tysiące.

\medskip
\noindent \texttt{16:55}

 Poruszamy się w~górę Szesnastej, a potem Piętnastej, między Ranstead a Market, okrążając centrum miasta. To mieszana załoga, bynajmniej nie cały Czarny Blok. Ludzie z~ON Mumia w~żółtych koszulkach są na czele, za nimi podąża masa anarchistów, kontyngent z~Rewolucyjnej Komunistycznej Młodzieżowej Brygady (ubrany w~identyczne czarne koszulki i~czerwone maski), w~towarzystwie innych z~sojuszniczej grupy maoistowskiej zwanej ,,Refuse and Resist''. Następnie zespół perkusistów. Wydaje się, że ludzie Mumii inicjują większość pieśni, które na przemian ,,Jesteśmy rozpaleni, nie możemy już tego znieść'' i~bardziej uroczyste (ale równie rytmiczne)

\noindent \textit{Cegła za cegłą \newline Mur za murem \newline Uwolnimy Mumia \newline Abu Jamala}

 W pewnym momencie zatrzymujemy się przy pomniku byłego szefa policji w~Filadelfii i~notorycznie prawicowego burmistrza Franka Rizzo. Niektórzy malują sprayem wąsy Hitlera i~dokonują innych strategicznych dodatków do posągu, który wydaje się, że już ma uniesioną rękę w~nazistowskim pozdrowieniu. Skręcamy w~Broad Street, skandując ,,Zamknij kapitalizm!'' i~przemykamy obok dużego, białego biura prokuratora okręgowego, w~starym budynku YMCA obok ratusza. Biuro prokuratora okręgowego jest z~góry wybranym celem. Jest dokładnie otynkowane balonami wodnymi wypełnionymi czerwoną farbą, a zamaskowane postacie ozdabiają otaczające ściany namalowanymi w~sprayu hasłami związanymi ze sprawą Mumii (,,Nowy proces dla sędziego Slatera'', ,,Zniszcz gubernatora Ridge''). Co dziwne, nigdzie nie ma policji.

 Teraz kierujemy się na północ w~górę Broad Street, mijając Cherry Street. Po raz kolejny te małe oddziały policji, które mijamy, wydają się beznadziejnie wymanewrowane i~mniej liczne. Około 17:20 mijamy grupę policji konnej -- najwyraźniej funkcjonariuszy stanowych -- ale znowu nie robią nic, by ingerować.

 Już połowa mijanych przez nas murów jest pokryta hasłami: na autobusach widnieje napis ,,Capitalism Kills!'', a ,,A w~Okręgu'' są wszędzie. Członkowie jednej grupy afinicji, którzy przynieśli szpule żółtej taśmy, która wygląda dokładnie tak, jak policja używa do oznaczania miejsc zbrodni, ale mają napis,,Mumia 911'', próbują użyć jej do zablokowania skrzyżowania.

 Stopniowo inne elementy odpływają i~jesteśmy po prostu w~Czarnym Bloku, plus kilku przypadkowych twardzieli, w~tym ja sam: gdzieś od siedmiuset do dziewięciuset osób. Trudno jest jasno zrozumieć liczby, ponieważ ciągle się przemieszczamy. Liczebność policjantów jest wciąż skąpa i~nie stawiają żadnego znaczącego oporu. Podczas naszej drugiej rundy za Ratuszem, około 17:30, natknęliśmy się na blokadę drogową i~obsadzający ją gliniarze po prostu zniknęli. Kilka minut później, jadąc na południe Cherry, ktoś przebija opony ogromnej limuzyny, prawie na pewno, jak komentują ludzie, przeznaczonej do transportu delegatów republikańskich. Niemal natychmiast potem znajdujemy się na szerokiej alei z~może pół tuzinem samochodów policyjnych zaparkowanych, pustych, na środku ulicy. Dwóch lub trzech gliniarzy na blokadzie drogowej znika w~chwili, gdy widzą prawie tysiąc anarchistów biegnących ulicą, a kiedy większość z~nas przestaje skandować (,,The People, United, Will Never Be Defeated'' w~języku angielskim i~hiszpańskim, ,,Ain't No Power Like the Power of the People, cause the Power of the People Don't Stop'' i~kilku zakleja pobliskie skrzyżowania żółtą taśmą Mumii, inne grupy afinicji spadają na samochody, wybijają szyby, przebijają opony, malują sprejem slogany.

 To samo dzieje się w~JFK i~Broad. Dziesiątki radiowozów są systematycznie niszczone.

\medskip
\noindent \texttt{17:45}

 Znowu ruszamy.

-- Zostańmy razem, ludzie!

-- Dociśnij.

 Dużym problemem w~każdej akcji Czarnego Bloku jest zawsze to, jak utrzymać wszystkich razem w~czasie. Gdy tylko Blok zaczyna się rozpadać, tracimy przewagę taktyczną. Z kolei strategia policji zawsze będzie polegać na czekaniu, aż skoncentrują się na wystarczająco dużo mobilnych sił, by nas wcisnąć się klinem i~rozdzielić. Ten moment najwyraźniej jeszcze nie nadszedł. Zatrzymujemy się, znowu na południe niedaleko Ratusza, próbując zebrać nasze siły. Niektórzy korzystają z~okazji, aby zerwać chorągiewkę z~flagą ustawioną wokół placu i~zrobić z~niej małe ognisko. Pojawia się Brooke trzymając się za ręce z~jakimś chłopakiem: 
 
 -- Patrzcie konie! -- zauważa, wskazując na funkcjonariuszy stanowych, którzy zaczynają gromadzić większe siły. -- Kiedy przyjdą konie, pamiętaj, aby wejść między samochody.

 Brooke znika: ogólnie rzecz biorąc, nie jest wielką fanką taktyk Czarnego Bloku.

 Kolejna plama porzuconych pojazdów policyjnych: faceci z~Czarnego Bloku skaczą po dachach, przecinają opony, wywalają ostatnie bomby z~farbą bezpośrednio przez okna, podczas gdy inni wznoszą prowizoryczne barykady. Ale plotki już się rozchodzą, że znacząca grupa gliniarzy rowerowych w~końcu się do nas zbliża. Około Szesnastej i~Arch blok się dzieli. Nie byłem pewien, jak to się stało, ale wygląda na to, że gdy zaczynaliśmy się ruszać dalej, gliniarze rzucili się na nas z~dwóch stron: kilkuset biegnących pieszo z~południa, kolejny oddział gliniarzy na rowerach pojawił się z~przodu, by odciąć początek marszu. Zrzucili rowery i~zaczęli skakać na maszerujących, mocując się z~nimi na ziemi. (Później słyszę, że jeden medyk został ciężko ranny, a trzech innych aresztowanych). Kolumna około dwustu z~nas, w~tym dziennikarze, protestujący, i~przechodniów, którzy zostali uwięzieni po jednej stronie szeregu gliniarzy na rowerach, linia frontu połączyła ramiona i~zaczęła zbliżać się do gliniarzy, aby spróbować dokonać de-aresztowania. Gliniarze zaczęli bić wszystkich w~zasięgu wzroku. Ale główna grupa, w~tym ja, już ruszyła dalej, nie mając pojęcia, co się dzieje.

\smallskip
\noindent \texttt{18:00, Osiemnasta i~Vine}

 Zatrzymujemy się, aby zastanowić się nad kolejnymi krokami. Nasze liczby spadają; wiemy, że zostaliśmy rozdzieleni, ale nikt nie jest pewien, jak to się stało. Na środku skrzyżowania zbierają się mini-rady, podczas gdy inni posłusznie zaczynają oklejać skrzyżowanie i~wyciągać śmietniki jako barykady. Członkowie jednej z~grup, które próbowały usunąć kawałek ogrodzenia z~siatki z~pobliskiego placu budowy, wracają, aby ogłosić, że kolumna rowerowej policji jest w~drodze. Inna eskadra policji -- nie wiemy ile, prawdopodobnie nie za dużo, ale wyglądamy na wściekłą -- spada na grupę, przesuwając śmietniki i~powalając kilku na ziemię, kopiąc i~okładając pałkami.

 Rozwiązuje się Rada Delegatów. Wynosimy się.

 Następuje dziki pościg, gdy blok, wciąż liczący kilkaset osób, jest ścigany przez istną armię policjantów na rowerach. Policja w~końcu zebrała siły. Okazuje się, że szef policji Timoney celowo podjął decyzję, by ignorować nas przez większość dnia, sądząc -- słusznie -- że akcja miała głównie na celu dywersję, aby odciągnąć siły z~blokad po drugiej stronie miasta. W końcu najwyraźniej usunęli blokady na głównych ulicach śródmieścia i~ruszają przeciwko nam. Ich taktyka polega na tym, by raz jeszcze nas rozbić, a przynajmniej odciąć kawałki naszej kolumny, które następnie można rozszarpać i~aresztować.

 Moja pamięć tutaj staje się czymś w~rodzaju bałaganu, ale jest pełna odosobnionych, żywych chwil, poczucie ciepłej dłoni na brzuchu, gdy zmartwiona dziewczyna z~Czarnego Bloku powstrzymywała mnie przed wejściem na niebezpieczną ulicę, przeskoczenie bariery parkingowej, bardzo wyraźnej refleksji, że nigdy nie zdawałem sobie sprawy, jak szybko potrafię biegać.

-- To ślepa uliczka, to byłoby naprawdę głupie.

-- Zostańmy razem!

-- O nie! Jesteśmy w~dupie.

-- Nie, nie, możemy tego zrobić, musimy tylko przyspieszyć.

 W niektórych momentach rzeczywiście biegaliśmy sprintem, skręcaliśmy w~boczne alejki, próbując wykorzystać opuszczone parcele i~jednokierunkowe ulice. (O pierwszej zadzwoniła moja komórka i~rzeczywiście usłyszałem ją i~odebrałem, prawdopodobnie dlatego, że już trzymałem ją w~dłoni. To Nat, starsza aktywistka z~grupy medialnej, poprosiła o raport. Powiedziałem jej, że jesteśmy ścigany przez gliniarzy na rowerze gdzieś w~pobliżu Chestnut lub Ransom. 
 
 -- Czy mógłbyś podać mi swoją dokładną pozycję? -- zapytała. -- Cóż, to może być trochę trudne, biorąc pod uwagę, że w~tej chwili biegam prawie tak szybko, jak to możliwe może 
 
 Roześmiała się i~powiedziała, żebym oddzwonił później. Ostatni odcinek, który pamiętam, to przecinanie parkingu po przekątnej, gdy gliniarze zablokowali jedną ulicę i~nadciągali siłą z~drugiej. Musiało to być na zachód od ratusza, bo zaraz potem około 18:15 lub 18:20, znaleźliśmy się na schodach Penn Square, wielkim podwyższonym placu na południe, gdzie w~końcu mogliśmy się zatrzymać i~odetchnąć, ponieważ, uważaliśmy, że Penn Square jest legalnym miejscem wiecu, i~policjanci mieli zostawić miejsce w~spokoju.

 A przynajmniej tak się nam wydawało.

\medskip
\noindent \texttt{18:25, w~pułapce}

 To, co napotkaliśmy, było w~rzeczywistości nie tyle wiecem, co pozostałościami po nim. Było podium i~niezwykle głośny mikrofon, głośnik, którego nikt nie wydaje się słuchać, porozrzucane stoły z~literaturą należącą do różnych grup marksistowskich (zwracam uwagę na jedną książkę zatytułowaną \textit{Che Guevera Talks to Youth}), pozostało najwyżej kilkadziesiąt osób. Lucinda jest tam i~daje mi butelkę wody, kiedy widzi, że jestem cały gorący i~spocony. Brad opowiada historie starszym aktywistom przy murze. Wyciągam telefon, żeby się zgłosić, i~niemal natychmiast zaczepia mnie lekko oszołomiony trzydziestoletni mężczyzna w~bejsbolówce ACLU i~podkoszulku.

-- Czy mogę skorzystać z~twojego telefonu komórkowego? -- pyta. -- Jestem prawnym obserwatorem i~muszę się zgłosić. Właśnie zostałem pobity przez kilku gliniarzy.

 Bierze telefon, żeby zadzwonić, a potem wyjaśnia swoją historię. Stacjonował na Czternastej Ulicy i~JFK, gdzie osiemnaście osób usiadło, by zablokować ulicę. Natychmiast zostali otoczeni przez gliniarzy na rowerach; funkcjonariusze spraw cywilnych zdawali się informować wszystkich, że zostaną aresztowani, jeśli się nie przeniosą. Uważnie obserwował, robił notatki, gdy kolejno zabierali blokujących, gdy nagle jeden z~gliniarzy podszedł i~uderzył go w~twarz. Facet nawet nie usunął numeru odznaki.

 Brad podszedł. 
 
 -- Masz szczęście, że miałeś tę koszulkę ACLU, inaczej aresztowaliby cię za napaść na oficera. 
 
  Wyjaśnia, że  jest to odwieczny problem: jeśli jakiś policjant wpada w~panikę w~trakcie spokojnego wydarzenia i~uderza kogoś bez powodu, to inni obecni gliniarze muszą aresztować ofiarę za napaść na funkcjonariusza, bo inaczej nie byłoby żadnego usprawiedliwienia dla tego, co się stało.

 Brad, zwykle niemal nadnaturalnie wesoły, nie jest w~najlepszym humorze. Obecnie nie ma stałego miejsca zamieszkania i~od tygodnia mieszka w~magazynie lalek. Był zwiadowcą rowerowym, kiedy gliniarze najechali to miejsce, ale teraz stracił wszystko, co posiadał. 
 
 -- Zabrali cały mój sprzęt przeciwdeszczowy, wszystko -- mówi.

-- Jakaś możliwość odzyskania?

-- Cóż, jeśli chcesz zostać w~mieście i~być naprawdę wytrwałym, czasami istnieje niewielka możliwość. Ale są szanse, że już wrzucili go gdzieś do zgniatarki śmieci.

 Na placu nie było policji; ale jak tylko przyjechałem, zauważyłem, że natychmiast zaczęli blokować wszystkie wyjścia z~placu. Do tej pory w~każdym punkcie dostępu znajdują się rzędy gliniarzy w~dwóch szeregach. Najwyraźniej ktokolwiek dowodzi, trzyma nas w~więzieniu i~czeka na rozkaz masowego aresztowania.

\medskip
\noindent \texttt{18:35}

 Pół tuzina anarchistów osiedliło się na szczycie furgonetki SEPTA (to jest władza transportu publicznego w~Filadelfii) na wschód od placu, z~czerwono-czarnymi flagami i~transparentem z~napisem ,,Zakończ rządy korporacyjne''. Również szukają przerw w~linii, ale nie znajdują zbyt wiele.

\medskip
\noindent \texttt{18:40}

 Około dwudziestu lub trzydziestu Czarnych Bloków zbiera się na południowy zachód od placu, tworząc krąg mini-rady, próbując wymyślić plan. Stopniowo dołączają do nich inni, aż będzie ich może setka. Zaczynają skandować ,,Anarchia to wolność'', a następnie maszerować, by zmierzyć się z~policją. Głównie wygląda to na próbę znalezienia słabych punktów w~linii; maszerują tam i~z powrotem między różnymi pozycjami.

 Jedna odziana na czarno grupa afinicji skupiła się w~cieniu, przeżuwając chleb pita i~jabłka, gdy przechodzą obok. Niecierpliwe spojrzenia, gdy ich mijamy. 
 
 -- Przepraszam. Jestem po prostu zbyt zmęczona na takie rzeczy -- wzrusza ramionami jeden z~nich.

\medskip
\noindent \texttt{18:55}

 Nie ma wyjścia. To staje się bardziej przygnębiające. Z pewnością nie mamy tylu, aby szarżować, co mogłoby złamać ich linie. Wiele osób zdołało już przemknąć się jako jednostki. Ale poza tym wydaje się, że nie ma alternatywy dla ewentualnego masowego aresztowania. Rozważam przedostanie się, w~końcu nie jestem z~żadną grupą afinicji i~nie widzę żadnego konkretnego powodu, by dać się aresztować. Wszystko, co musiałbym zrobić, to zapiąć moją ładną, czerwoną, zapinaną na guziki koszulę, żeby zakryć anarchistyczną koszulkę, którą mam na sobie, i~zrobię się porządnym dziennikarzem.

\medskip
\noindent \texttt{Uratowani}

 Właśnie w~tym momencie przybywa Blok Cyrkowy.

 Właściwie awangardą jest ta dziwna drużyna o nazwie Kozy z~Głosem, sześciu facetów na rowerach w~białych koszulach i~kamizelkach oraz, przynajmniej w~trzech przypadkach, ogromne głowy kóz z~papier-mache. Docierają prosto do linii policyjnych, ustawiają się w~samym środku i~niemal natychmiast wpadają w~jakąś rapową piosenkę a cappella.

-- Widzisz, co możesz zrobić z~marionetką? -- zauważa Brad z~podziwem. (Brad zaczyna się pocieszać. -- Nikt nie byłby w~stanie z~tym uciec.

 Blok natychmiast zaczyna gromadzić się po drugiej stronie linii policyjnej, naprzeciw kóz. Korzystam z~tego, żeby wydostać się, zapinam koszulę, chwytam mały notatnik reporterski, prosząc kobietę-oficera, żeby przepuściła mnie przez kolejkę, żebym mógł bliżej przyjrzeć się kozom. Przechodzę tak samo jak\ldots 

\medskip
\noindent \texttt{19:15}

 \ldots po raz pierwszy pojawia się Rewolucyjny Anarchistyczny Blok Klaunów! Z trzema wysokimi rowerami i~kilkoma monocyklami na czele, na przemian śpiewają ,,Czyj Cyrk? Nasz Cyrk!'' lub po prostu ,,Demokracja? Ha! Ha! Ha!

 W tym samym klastrze przybywają miliarderzy dla Busha lub Gore, ubrani w~smokingi i~wieczorowe suknie. Jeden znajomy RTS z~Nowego Jorku prowadzi, w~ogonie, na deskorolce, dmuchając bańki. Mieli też własne przyśpiewki: ,,Precz z~plutokracją! Precz z~demokracją!'' lub ,,Czyje apartamenty? Nasze apartamenty!”

 O 19:25 klauni stają w~szranki z~gliniarzami, a przynajmniej stanęliby, gdyby Miliarderzy nie ustawili się w~kolejce, żeby ich powstrzymać. Istnieje niekończąca się meta-pieśń klauna (,,Wezwanie! Odpowiedź! Wezwanie! Odpowiedź!'' lub po prostu ,,Pieśń w~trzech słowach! Pieśń w~trzech słowach!'' . Kilku klaunów zaczyna atakować Miliarderów piszczącymi zabawkowymi młotkami, co prowadzi do szamotaniny, gdy tarzają się z~krzykiem po ziemi. Policjanci wyglądają na coraz bardziej zdezorientowanych. Szereg konnej policji stoi w~odległości około dziesięciu metrów, nic nie robiąc, obserwując. Dziennikarze zaczynają się gromadzić.

 Klauni rozpoczynają głupi taniec, skandując ,,Anarchia dla wszystkich, jesteśmy tutaj, aby to było zabawne!''. Przywódca miliarderów, niejaki Phil T. Rich, stara się ich odpędzić: -- Dobry Boże, dlaczego nie zrobicie ze swoim życiem czegoś wartościowego? Znajdź kogoś, kto będzie dla ciebie pracował! -- Kilku Miliarderów podchodzi do policjantów i~próbuje uścisnąć im ręce; dwaj mają zwitki fałszywych pieniędzy i~próbują włożyć ich duże ilości do rąk i~kieszeni policji, głośno dziękując im za tłumienie sprzeciwu. Dwóch jest zaskoczonych przez klaunów, powodując, że kilku gliniarzy rusza, by interweniować, tylko po to, by fizycznie powstrzymać ich towarzyszy.

 W wyniku zamieszania Czarny Blok ucieka.

\medskip
\noindent \texttt{21:15 (znacznie później)}

 Resztki bloku wycofały się grupami w~całym mieście, przez obszary, gdzie dawno zlikwidowano blokady; bijąc w~latarnie, zatrzymując się od czasu do czasu na bębnienie i~taniec lub malowanie sprayem, zawsze ścigani przez oddziały policji w~samochodach i~na rowerach. Wreszcie, zmęczeni czymś, co wydaje się godzinami gry w~kota i~myszy, moja grupa kończy z~gęstym tłumem innych aktywistów przed Centrum Kwakrów. Jest prasa, ale staramy się ich ignorować.

 W końcu pojawia się nieco krępa młoda kobieta w~czerni, przepychając się obok reporterów.

-- Hej hej hej! -- (Powtarza to trzy razy, aż wszyscy się odzywają.)

-- Czy wszyscy mnie słyszą?

 Szmery wyrażające zgodę.

-- Mieliśmy cholernie cudowny dzień. Teraz jesteśmy zmęczeni. Pięćset osób zostało aresztowanych i~jest przetrzymywanych w~różnych miejscach miasta. Jedyną rzeczą, jaką możemy wymyślić, która byłaby skuteczna w~pomaganiu im, jest ponowne zebranie się w~CEC i~zorganizowanie posiedzenia rady jutro o 7:30.

-- Więc na razie: prześpijcie się. Jutro możemy zrobić więzienną solidarność, iść do Centrum Kongresowego, robić cokolwiek postanowimy. Ale teraz po prostu stoimy i~patrzymy na siebie. Weźmy prysznic. Prześpijmy się. Chodźmy na seks. Zróbmy\ldots  zdobądźmy to, czego potrzebujemy, aby jutro wstać i~wrócić.

-- Uwielbiam was ludziska. Byliśmy dzisiaj kurwa NIESAMOWICI. Ale teraz nie mamy już nic do roboty. Pamiętajcie: bądźcie bezpieczny. Bądźcie bezpieczni. Bądźcie bezpieczni. Jeśli wracacie do domu, bądźcie bezpieczni i~sprytni. Idźcie z~co najmniej jednym kumplem. Gliny jeżdżą na rowerach po całym miejscu i~wszyscy wiemy, że będą wybierać maruderów. Chcę was jutro znowu zobaczyć. Dobra?

 Miriam, która jest tam z~małym oddziałem ludzi DAN, woła: ,,Grupowa wycieraczka!'' a około trzydziestu osób ściska się, wiwatując i~chichocząc. Ogólna atmosfera radości i~podziwu dla naszych własnych osiągnięć. A potem się rozdzielamy. Wracam do IMC.

 Ta relacja oczywiście nie dotyczy tylko Czarnego Bloku, ale przekazuje coś z~poczucia bycia w~jednym: poczucie egzaltacji, wolności, poprzecinane chwilami wściekłości, radości, paniki, podniecenia i~rozpaczy. Głównie jednak, mówiąc o tym później, wszyscy mają tendencję do podkreślania tego samego: doświadczenia autonomii, możliwości choćby chwilowego zajęcia przestrzeni poza ich kontrolą, w~której jedynymi regułami są te generowane zbiorowo, przez grupę, i~w której jest po równo pewność zaufania, świadomość, że każdy, kto stanie za tobą, pilnuje tyłów. 

 Dlatego akcje w~stylu Black Bloc są postrzegane przez tak wielu, którzy w~nich uczestniczą, jako istota akcji bezpośredniej. Tworzą wyraźną równowagę między tworzeniem zbiorowego doświadczenia wolności (jak na przykład w~akcjach karnawałowych) a bezpośrednią konfrontacją z~władzą. To, co jest bagatelizowane, a nawet ignorowane, to zazwyczaj pośrednicy: ,,publiczność''\footnote{Czarne Bloki prawie nigdy nie niosą ze sobą literatury ani w~żaden inny sposób nie starają się wyjaśnić przechodniom, co robią; choć zwykle wydają anonimowe komunikaty. Podczas akcji w~Filadelfii pojedyncze zamaskowane postacie od czasu do czasu zwracały się do zdziwionych przechodniów i~wołały ,,dołączcie do nas!'' - ale ten gest zawsze był co najmniej trochę śmieszny. Większość osób postronnych, choć zafascynowanych, najwyraźniej nie miała pojęcia, kim są ani czym się zajmują.}. Ale oczywiście właśnie to sprawia, że akcja jest bezpośrednia. 

 Tam, gdzie nieposłuszeństwo obywatelskie staje się kwestią spektakularnej bezradności wobec policji i~bohaterstwa w~przeciwstawieniu się wynikającej przemocy, taktyki Czarnego Bloku kładą nacisk na wzajemną ochronę. Bloki są masą równych sobie, z~których każdy zaryzykuje aresztowanie tylko po to, aby zapobiec aresztowaniu swoich towarzyszy lub ich uratować. Wszyscy zgadzają się, że Czarne Bloki nie inicjują ataków na innych ludzi. O ile wśród uczestników trwa debata, chodzi o to, czy przemoc interpersonalna jest odpowiednia, aby uratować towarzysza, który pomimo odmowy skrzywdzenia kogokolwiek, jest jednak fizycznie atakowany przez policję. To była właściwie powszechna kwestia, którą można było usłyszeć na spotkaniach przygotowawczych, kiedy każdy został poproszony o opisanie, co zrobi lub czego nie zrobi: ,,Nigdy nie zaatakuję innej żywej istoty, ale nie jestem pewien, co bym zrobił, gdybym zobaczył, jak ktoś próbuje skrzywdzić kogoś, kogo kocham''. A kiedy się stworzyło bloc, często czuło się, że miłość rozciąga się na wszystkich swoich towarzyszy. Kiedy rozmawiamy z~ludźmi po akcjach, to poczucie absolutnego zaufania w~chaosie zawsze było kluczowe. Jeden z~weteranów aktywistów -- jego pseudonim to (nieco niestosownie) Zły -- wskazał na słynny moment, kiedy aktywista Czarnego Bloku otoczony przez policję na platformie u podstawy masztu flagowego w~Naval Memorial podczas protestów inauguracyjnych w~2000 roku, dosłownie skoczył na głowę, ponad głowami policji do zamaskowanego tłumu aktywistów, ze świadomością, że kimkolwiek byli, na pewno go złapią. i~złapali. Naprawdę, powiedział Zły, mamy do czynienia z~,,elegancką dynamiką płynów'', która ostatecznie sprowadza się do wspólnych doświadczeń z~pogo: 

 W pogo na punkowym lub hardcorowym koncercie wszystkie dzieciaki wariują, wszyscy razem, nurkując na scenie, krążąc po kręgu, surfując w~tłumie, ochroniarze są dwa razy więksi, więc rozwijasz wyczucie przestrzeni, płynnego ruchu i~akcji. Łączenie ramion w~celu przebicia się przez linie policyjne podczas akcji jest jak przepychanie się do przodu tłumu na pokazie z~powolną, stałą presją. Nie chodzi o to, że wszyscy Black Blockerzy są punkami lub odwrotnie, ale kiedy Black Blocker przeskoczył nad głowami oddziałów prewencji pod pomnikiem marynarki wojennej podczas inauguracji George'a W. Busha w~2001 roku, aby uniknąć aresztowania, to po prostu nurkował ze sceny i~bodysurfował. 

 Równość, autonomia, wzajemna pomoc -- wszystko to są oczywiście podstawowe zasady anarchizmu. 

 Wreszcie nie jest tak, że Black Bloc są całkowicie obojętni na wrażenie, jakie wywierają na szerszej publiczności. Po prostu nie są zainteresowani zdobyciem poparcia tego, co w~mediach nazywa się ,,publicznością'', w~dużej mierze wyimaginowanej społeczności białych rodzin z~klasy średniej, która w~opinii większości anarchistów jest w~dużej mierze wytworem samych mediów. Po raz kolejny chodzi o rozbicie Spektaklu: w~tym przypadku całkiem dosłownie. Podczas gdy krytycy będą bez końca podkreślać, że niszczenie mienia niszczy przedstawienie pokojowego nieposłuszeństwa obywatelskiego i~jest wykorzystywane do usprawiedliwiania wszelkiego rodzaju represji (represji, które prawie nigdy nie są skierowane głównie do tych, którzy wybijają szyby), trudno zaprzeczyć, że obraz uderza w~jakąś strunę. Z pewnością jest to jedna z~niewielu rzeczy, które prawie każdy w~Ameryce wie o anarchistach: że byli znani z~rozbijania szyb Starbucksa. Oczywiście jest to głęboko ambiwalentny akord. Ale jeśli kogoś cel jest rewolucyjny, to odwołuje się przede wszystkim do najbardziej wyobcowanych i~najbardziej pozbawionych praw. Jak Mac zauważył na samym początku tej książki, takim elementom nie trzeba pokazywać przemocy tkwiącej w~systemie. Wiedzą o niej wszystko. To, czego potrzebują, to mieć jakiś powód, by sądzić, że system jest podatny na ataki; że można mu skutecznie rzucić wyzwanie, a przynajmniej, że wyzywający system mogą uciec bez szwanku. 

 W tym momencie jednak odchodzimy od wewnętrznej struktury akcji i~zaczynamy zajmować się kwestiami reprezentacji, na których skupimy się w~następnym rozdziale. Zanim to zrobię, chciałbym zakończyć to kilkoma krótkimi praktycznymi refleksjami na temat natury państwa. 

\section{Część II: Władza Państwa}

\noindent ZATRZYMANIE

\begin{flushright}


\texttt{Chrześcijanin we mnie mówi, że to złe, ale strażnik więzienny we mnie mówi: ,,Uwielbiam sprawiać, że dorosły mężczyzna sika pod siebie''.}

-- Charles Grainer, były amerykański strażnik więzienny przydzielony do Abu Ghraib
\end{flushright}

 Pięć przykładów, które złożyły się na większą część tego rozdziału, oczywiście nie ma na celu obszernej typologii. To bardzo przybliżona lista, mająca na celu przede wszystkim uwypuklenie pewnych cech strukturalnych działań w~ogóle: w~szczególności złożonych i~ciągle zmieniających się relacji między aktywistami, docelowymi instytucjami lub innymi władzami, publicznością i~policją. (Zajmę się relacjami z~mediami w~rozdziale 9). Ze względu na ograniczony zakres przykładów, wszelkiego rodzaju ważne elementy głównych akcji zostały okrojone lub zostały całkowicie pominięte. Nie mam prawie nic do powiedzenia, na przykład o zespołach lalkowych, teatrze ulicznym, ,,różowych blokach'', taktyce tute bianchi czy pogańskich rytuałach, chociaż będzie trochę o niektórych wyraźniej performatywnych i~rytualnych aspektach w~następnym rozdziale. 

 Rozdziały o Québec City miały na celu zilustrowanie tego, w~jaki sposób wszystkie te elementy łączą się ze sobą: miesiące planowania, dreszczyk konwergencji, paranoja i~kultura bezpieczeństwa, które często przenikają grupy afinitywne, przejście od agresywnych działań do więziennej solidarności i~ogólne wsparcie dla aresztowanych. Solidarność więzienna -- oraz doświadczenie aresztowania i~procesu -- jest zwykle bardzo ważnym elementem formacji każdego aktywisty-weterana. Aresztowanie oznacza zmierzenie się z~rzeczywistością władzy państwowej w~tym, co każdy anarchista nazwałby jej najczystszą postacią: to znaczy z~odartą wszelką prezencją najwyższej życzliwości. Ci, których spotyka się, gdy są przetwarzani, przetrzymywani i~więzieni -- przedstawiciele ,,wymiaru karnego'', a zwłaszcza drobni funkcjonariusze -- z~reguły nie mają obowiązku nawet udawać, że są sprawiedliwi wobec swoich podopiecznych. Szok wynikający z~poznania faktu, że policja kłamie i~atakuje niewinnych, przechodzi w~kolejny szok, że za zamkniętymi drzwiami oczekuje się od nich zachowania jak nieprzejednani sadyści. Zadawanie bólu i~upokorzenie jest uważane za normę (przynajmniej każdy akt przyzwoitości jest uważany za szczególną przysługę), ale jednocześnie sadyzm prawie zawsze łączy się z~prawie całkowitym i~systematycznym biurokratycznym zamieszaniem i~niekompetencją\footnote{Jest faktem mało znanym większości Amerykanów (choć dobrze znanym większości aktywistów), że wiele praktyk, które wywołały międzynarodowe skandale, gdy zostały ujawnione w~prowadzonych przez Amerykanów więzieniach w~Iraku lub Guantanamo, to po prostu standardowa procedura w~amerykańskich więzieniach: na przykład oziębienie przez podlewanie więźniów i~zostawianie ich nagich przez wiele dni w~niemalże mroźnych celach lub przykuwanie ludzi w~miejscach tortur do krat.}. Będąc całkowicie pod władzą państwa, wydawałoby się, że napotykamy zarówno jego brutalność, jak i~głupotę w~nieskażonej formie. 

 Jest to regularne doświadczenie każdego, kto przeszedł przez duże miejskie więzienie w~Stanach Zjednoczonych, ale aktywistyczna praktyka więziennej solidarności -- odmowa podania nazwisk, systematyczna odmowa współpracy z~systemem w~celu zapchania systemu i~utrudnienia aresztowania innych aktywistów -- zwykle pogłębia zarówno brutalność, jak i~zamieszanie. Na przykład w~Filadelfii aktywiści odmawiali podania swoich nazwisk i~często nie współpracowali przy pobieraniu odcisków palców i~fotografowaniu. Rezultatem była systematyczna przemoc. Poniższe fragmenty relacji z~ówczesnych listserv aktywistów oddają nieco posmaku tego doświadczenia.

 Odmówiłem wszelkich informacji poza informacjami medycznymi, na które odpowiedziałem bardzo szczegółowo, ponieważ bardzo się martwiłem, że zignorują moją hipoglikemię.

-- Jeśli mój poziom cukru we krwi spadnie zbyt nisko -- powiedziałem pielęgniarce -- będę miał konwulsje.

-- Bardzo fajnie będzie to oglądać -- odpowiedziała -- ponieważ nie dostaniesz tu tyle jedzenia.

 \ldots 

 W piątek ludzie zaczęli być pozywani. W tym czasie system zintensyfikował swoją taktykę zastraszania, próbując przestraszyć ludzi, aby podali swoje nazwiska. Jedno, co wpłynęło na nas wszystkich, była klimatyzacja. Ponieważ wszyscy byliśmy zmarznięci, odkąd zabrano nas do Roundhouse, ponieważ był klimatyzowany i~wszyscy byliśmy ubrani w~letnie ubrania, rozumiem, że w~piątek wieczorem jedna kobieta rzeczywiście dostała hipotermii. Strażnicy weszli w~swetrach i~zimowych płaszczach, więc taktyka była najwyraźniej zaplanowana. Jedna kobieta, która minęła termostat, powiedziała mi, że wskazuje 7 stopni Celsjusza. Układaliśmy się jeden na drugim (dosłownie, wielcy ludzie na dole stosu, a mali na wierzchu -- lub pośrodku, jeśli zrobiło im się zimno), próbując wykorzystać ciepło naszego ciała, aby się ogrzać.

  Jeśli ludzie nie współpracowali przy fotografowaniu, ich głowy były rozbijane o ścianę. Powiedziano mi, że był znak, gdzie robili te zdjęcia, który nakazywał funkcjonariuszom wytrzeć krew przed zrobieniem zdjęć. Sam nie widziałem tego znaku, ponieważ został usunięty do czasu, gdy zostałem przetworzony cztery dni później. Jednak widziałem wystarczająco dużo krwi i~siniaków na kobietach wracających do celi, aby czuć się naprawdę przerażony tej nocy.

\medskip

\noindent  2 nad ranem, dowiadujemy się, że liderki kobiet są zabierane i~izolowane. W mojej sześcioosobowej celi trzem z~nas w~końcu udaje się oddać mocz w~bliskim towarzystwie, po trzydziestu godzinach odsiadki. Nikomu jeszcze nie udało się wypróżnić, ponieważ cała nasza szóstka musi siedzieć w~celi kolano do kolana. Nie ma prywatności. Nadal nie widzieliśmy naszego prawnika.

\noindent  3 nad ranem -- obrońca z~urzędu -- a nie jeden z~naszych prawników R2K -- zostaje w~końcu wpuszczony do Roundhouse.

\noindent  5 rano -- dociera do naszego bloku. Obrońca nie jest zaznajomiony z~solidarnością więzienną i~nie może udzielać porad. Po prostu posępnie wykłada na temat maksymalnych kar. Mamy wrażenie, że nie jest po naszej stronie.

\noindent  6:20 -- JOE HILL zostaje przykuty kajdankami za to, że nie oddał dobrowolnie swoich odcisków palców.

\noindent  6:55 -- JOE HILL w~końcu zostaje rozkuty.

\noindent  9:00 -- Jedenastu z~naszego bloku zostaje wyciągniętych z~naszych cel, skutych razem i~odmaszerowuje.

\noindent  9:15 -- Woda w~naszych celach jest wyłączona. Nawet toaleta nie działa. Oficer mówi do mojej celi: 

-- W toalecie jest woda. Pijcie to!

\noindent  9:30 -- wyprowadzają mnie z~celi i~staję pod ścianą, czekając na oskarżenie. Kiedy czekam, funkcjonariusz Cassady (odznaka 1976) przeciąga twarz WOLFMANA przez rynsztok, a następnie uderza nią w~kraty celi, ponieważ porusza się zbyt wolno. WOLF później wykazał otarcia na prawym ramieniu, które były tym spowodowane.

\noindent  [Inny aktywista] również zostaje wepchnięty do krat przez 1976.

\noindent  9:50 -- Kiedy tam stoję, woda jest w~końcu ponownie włączona po trzydziestu pięciu minutach intonowania.

\noindent  11:00 -- W~końcu zostałem wciągnięty w~proces, gdzie po raz pierwszy słyszę moje zarzuty. Wszystkie są wykroczeniami, ale zawierają zarzuty takie jak ,,Utrudnianie ruchu na autostradzie'', które określają warunki i~miejsce aresztowania, wobec których jestem oczywiście niewinny. Papierkowa robota jest wymieszana.

\medskip

  Należy tutaj pamiętać, że zwykle przytłaczająca większość aresztowanych na akcjach masowych nie jest faktycznie oskarżona o żadne przestępstwo. Jak wskazali trenerzy A16, są oni zwykle odbierani za ekwiwalent wykroczenia na wózku lub parkowania: ,,naruszenia'' lub ,,wykroczenia'' (sformułowanie zależy od jurysdykcji), które nie są sprawami karnymi i~w zwykłych okolicznościach w~najgorszym przypadku oznaczają mandat i~skromną grzywnę\footnote{Działacze Czarnego Bloku są w~rzeczywistości rzadko aresztowani, a jeśli tak, to nigdy, o ile mi wiadomo, nie są oskarżani o to, co faktycznie zrobili.}. Okazjonalne próby podniesienia stawki przez wymyślanie poważniejszych zarzutów przeciwko osobom zaangażowanym w~blokady -- jak próbowano w~Philly -- prawie zawsze kończą się niepowodzeniem w~sądzie\footnote{W Filadelfii, z~dosłownie setek oskarżonych aktywistów, w~większości z~wieloma zarzutami, tylko jeden działacz został uznany za winnego czegokolwiek --- i~był to jeden bardzo drobny zarzut spośród kilku.}. W rzeczywistości w~połowie przypadków aresztowani nie są nawet winni wykroczeń, ponieważ bardzo duża część aresztowań przy każdej dużej mobilizacji ma charakter prewencyjny. Policja często zmiata tłumy setek osób maszerujących chodnikiem lub kręcących się po ,,zielonych strefach''\footnote{Na przykład przed A16 policja zamieszek w~Waszyngtonie otoczyła kilkuset demonstrantów, którzy szli całkiem legalnie ulicą miasta, kazali im się rozejść, uniemożliwili im rozejście się, a następnie aresztowali ich wszystkich za ,,brak rozejścia''. Aktywistów załadowano następnie do autobusów, skrępowano kajdankami przypinającymi im nadgarstki do kostek i~przetrzymywano przez kilka dni. Dwa lata później, podczas ,,Strajku Ludowego'' ACC, byłem z~kilkuset aktywistami w~zielonej strefie na ,,Plaza Wolności'', kiedy cały park został otoczony i~wszyscy aresztowani. Kiedy zapytałem mojego oficera dokonującego aresztowania, za co zostaliśmy aresztowani, odpowiedział: ,,Dowiemy się o tym później'' Po sześciu czy siedmiu godzinach marnowania w~kajdankach w~autobusach aresztowych, dość zakłopotany oficer przyszedł ogłosić, że wszyscy mamy zostać oskarżeni o ,,FTB'' - to znaczy, ,,niewykonanie'' nakazu policji opuszczenia parku - nakazu, który, jak wszyscy doskonale wiedzieli (a później okazało się, że wewnętrzne dokumenty policyjne), nigdy nie został wydany. Takie oczywiście nielegalne czyny zwykle prowadzą do pozwów sądowych, chociaż ich rozpatrzenie zajmuje wiele lat i~bardzo niewielu z~aresztowanych kończy się uwzględnieniem w~jakiejkolwiek ostatecznej ugodzie.}. Ponieważ zatrzymanie prewencyjne jest w~Stanach Zjednoczonych nielegalne, aktywiści aresztowani podczas takich akcji są doskonale świadomi, że jeśli są w~więzieniu, to dlatego, że to policja, a nie oni, jest winna złamania prawa.

  Na przykład powyższe relacje zostały napisane przez aktywistów, którzy znaleźli się wśród siedemdziesięciu aresztowanych za przebywanie w~słynnym ,,magazynie kukiełkowym'' w~Philly -- budynku, w~którym produkowano rekwizyty i~sztukę polityczną na potrzeby akcji.

  Kilku przygotowywało się do blokowania później tego dnia; większość przygotowywała się do wzięcia udziału w~grupach lalek, klaunów lub występów. Nikt nie popełnił niczego przypominającego przestępstwo. Zostali zbiorowo oskarżeni o przestępstwa, począwszy od ,,posiadanie narzędzia przestępstwa'' (rurki PCV znalezione w~magazynie, które można wykorzystać do robienia lockbox) po ,,zablokowanie autostrady'' -- zarzuty, o których wszyscy wiedzieli, że nigdy nie mogą zostać postawione w~sądzie, ale były stosowane po prostu w~celu uzasadnienia wysokich kaucji. Nikt ostatecznie nie stanął przed sądem. Po tygodniowym przetrzymywaniu, po czym zwolnieni, ale zmuszani do powrotu do Filadelfii na kolejne rozprawy sądowe, prokuratorzy nagle ogłosili, że ponieważ policyjni infiltratorzy w~magazynie nie byli w~stanie zidentyfikować żadnego z~nich w~linii, wszystkie zarzuty zostały oddalone.

  Dodajmy do tego tendencję do wybierania niektórych aresztowanych za coś, co zwykle wydaje się całkowicie losowymi oskarżeniami o przestępstwo (na przykład napaść na funkcjonariusza, te również niezmiennie zawodzą w~sądzie, ale tylko po niekończących się odroczeniach, które pochłaniają ogromną ilość czasu i~energii aktywistów), i~trudno sobie wyobrazić, jak działacze mogliby postrzegać wymiar sprawiedliwości w~sprawach karnych jako coś innego niż tępy instrument głupoty i~represji.

 W Filadelfii działaczom nieustannie grożono rozmieszczeniem wśród zwykłych więźniów, którzy -- jak wyjaśniali strażnicy w~często obrazowych terminach -- będą ich terroryzować, brutalizować i~gwałcić. Kiedy w~pewnym momencie władze spełniły swoje groźby, podstęp całkowicie się nie udał. Zwykła populacja więzienia okazała się dość sympatyczna, a przede wszystkim niezwykle zainteresowana nauką taktyk aktywistycznych. Zwykli więźniowie szybko zaczęli nadawać sobie nawzajem pseudonimy na akcję, odmawiać współpracy i~koordynować zbiorowe żądania, tak szybko, że w~ciągu dwudziestu czterech godzin aktywiści zostali ponownie wyprowadzeni i~odseparowani. Jednak prawie wszyscy aresztowani ujawnili długie historie o więźniach, których spotkali wśród zwykłych, którzy zostali zabrani za drobne lub nieszkodliwe wykroczenia bez użycia przemocy (posiadanie marihuany, wtargnięcie na skróty przez opustoszałą parcelę) i, podobnie jak oni, poddawani ciągłej przemocy i~brutalności. W każdym razie w~tym momencie uznano sytuację analogiczną: fakt, że prawa działają zupełnie inaczej dla pewnych kategorii ludzi, czy są to biedni Afroamerykanie, czy (przynajmniej podczas akcji) idealiści polityczni, którzy mają odwagę podjąć na ulice.

 Biorąc pod uwagę nieustanną brutalność, podczas przeglądania tych relacji jestem zawsze nieco zaskoczony naciskiem, jaki tak wielu aktywistów kładzie na to, co w~innym przypadku wydawałoby się całkiem trywialne akty niesprawiedliwości. Relacja jednego z~aresztowanych z~Filadelfii zwracała szczególną uwagę na to, jak ona i~jej koleżanka z~celi zostali umieszczeni w~odosobnieniu na dwa dni jako kara za ,,zniszczenie farby'' na ścianie ich celi, oprócz konieczności zapłaty grzywna w~wysokości kilkuset dolarów. W rzeczywistości upierała się, nie tylko ściana została wyszczerbiona przed jej przybyciem, ale funkcjonariusz zadał sobie trud wskazania odprysków farby na podłodze jako ,,dowodu'' jej przestępstwa, odprysków, które (ponieważ nie było ich tam, gdy podłoga była szorowana dzień wcześniej) mogły zostać tam umieszczone tylko celowo, gdy więźniowie kilka godzin wcześniej brali prysznic. Oczywiście, mając dwa dni samotności w~celi, nie mając nic innego do roboty, trudno byłoby nie mieć obsesji na punkcie tego, co skłoniłoby strażnika więziennego do podłożenia odłamków farby w~celi pozornie losowo wybranej pary aktywistów, a następnie udawania, że skazuje ich za ,,przestępstwo'', o którym wszyscy wiedzieli, że nie popełnili. Wydaje się jednak, że istnieje głębszy powód, dla którego aktywiści przywiązują taką wagę do takich gestów. Wydaje się, że są to próby przebicia przekazu: że będąc w~rękach państwa, należy odłożyć na bok wszelkie wyobrażenia, że stosunki z~jej przedstawicielami będą regulowane przez jakikolwiek rozpoznawalny kodeks sprawiedliwości. ,,Nie oczekuj, że będziemy uczciwi'' ,,Rzeczywistość jest tym, czym mówimy, że jest'' ,,Jesteś w~naszej mocy i~możemy zrobić z~tobą, co chcemy”.

 To wydaje się przesłaniem.

 W tym świetle równomierny nacisk w~tych relacjach na pozorne biurokratyczne zamieszanie i~niekompetencję przybiera bardziej subtelny i~podstępny aspekt. W niektórych przypadkach ta niekompetencja jest wyraźnie zamierzona. Jak wielu zauważyło po protestach RNC w~Nowym Jorku cztery lata później, bardzo trudno uwierzyć, że ta sama policja, która wykazała się skutecznością w~zmiataniu protestujących z~ulic, naprawdę potrzebowała od szesnastu do czterdziestu ośmiu godzin w~każdym przypadku, aby zlokalizować dokumenty wymagane do wypuszczenia ich. Ale często wydawało się, że dzieje się coś bardziej subtelnego. Jeden z~przyjaciół aresztowany w~zamknięciu w~Filadelfii powiedział mi, że w~ciągu tygodnia, który spędził w~więzieniu, trzykrotnie stawał przed sądem i~za każdym razem pojawiał się inny policjant, który twierdził, że jest funkcjonariuszem dokonującym aresztowania. O ile mógł zrozumieć, żaden z~tych trzech nie był nawet w~pobliżu w~czasie jego faktycznego aresztowania. (-- Jak myślisz, jak to się stało? -- zapytałem go. -- Nie mam pojęcia.) To tak, jakby władze próbowały komunikować nie tylko, że nie muszą być uczciwe, ale że nawet nie mają zachowywać się w~sposób, który ma jakikolwiek sens. Mogli zrobić prawie wszystko, co chcieli. Mogli zachowywać się zupełnie przypadkowo i~nie można było nic zrobić.

 Myślę, że aktywiści mają rację, dostrzegając w~tym coś istotnego w~naturze państwa. Są to przejawy arbitralnej władzy, która twierdzi, że nie wymaga żadnego powodu ani wyjaśnienia. To, co sprawia, że  błyskawica jest odpowiednim symbolem boskiej mocy, to nie tylko to, że jest niszczycielska, ale że jest losowa. Z drugiej strony symbolem sprawiedliwości jest waga: sprawiedliwość jest zawsze pojmowana jako kwestia równowagi lub wzajemności. Z kolei suwerenna władza twierdzi, że jest tą, która ustanawia równowagę; to ręka trzymająca wagę; dlatego z~definicji nie może być ważona na tej samej wadze. Stąd wysiłek, aby ustalić, że nie ma tu absolutnie żadnej wzajemności. Przesłanie nie brzmi ,,jeśli będziesz grać zgodnie z~zasadami, nie zostaniesz ukarany'', ponieważ oznaczałoby to istnienie jakiegoś kontraktu. Umowa oznaczałaby, że obie strony są w~pewnym sensie równoprawnymi stronami. Przesłanie brzmiało raczej: ,,Musisz grać zgodnie z~zasadami. My nie musimy. Aby to zademonstrować, wyjaśnijmy, że nawet jeśli będziesz postępować zgodnie z~zasadami, i~tak możesz zostać ukarany''. To zdolność państwa do nakładania takiej arbitralnej kary upoważnia je do ustanawiania zasad. -- Stan medyczny? Może weźmiemy to pod uwagę. Może nie. W każdym razie nie będzie negocjacji. Przede wszystkim pod żadnym warunkiem nie będziesz mieć prawa narzekać, że nie postępujemy uczciwie.

 Ironia polega oczywiście na tym, że policja i~strażnicy w~ogóle nie są w~stanie sprawować absolutnej i~arbitralnej władzy. Nawet gdyby mieli do czynienia ze zbiorem biednych czarnych nastolatków lub nieudokumentowanych migrantów z~Bliskiego Wschodu, istniałyby \textit{pewne }ograniczenia (choć w~tym przypadku najwyraźniej niewiele). Kiedy mają do czynienia z~bezimiennym tłumem, w~większości białych aktywistów, zgromadzonym na demonstracji, policja doskonale zdaje sobie sprawę, że każdy z~zatrzymanych może być dzieckiem kogoś ważnego. Jest wysoce nieprawdopodobne, że dziewczyna z~dredami przed tobą jest córką prokuratora generalnego, ale nie wiesz tego na pewno. Gdyby zdarzyło się jej zabić, okaleczyć lub trwale oszpecić, a okazałoby się, że tak, to przynajmniej jeden z~nich stanie w~obliczu niewielkiego narodowego skandalu. Stąd preferowanie technik mających na celu dręczenie, przerażenie i~poniżanie, ale bez wyrządzania oczywistych trwałych szkód. Zaciska się kajdanki na tyle, aby ręce zmieniły kolor na niebieski, ale nie do trwałego ich uszkodzenia; uderza się głową o ścianę, ale nie łamie kończyny. Większość z~tych technik to łagodne formy tortur. Umieszczanie więźniów na długi czas w~celach w~mrozie (w niektórych przypadkach najpierw zdejmowanie odzieży lub polewanie ich wodą) jest standardową procedurą podczas przesłuchań, a także w~wielu więzieniach w~USA. Podobnie jak granie na naturalnych wstrętach, takich jak te przeciwko szkodnikom lub ekskrementom (personel organów ścigania wydaje się, jak zobaczymy, być osobliwie zafascynowany psychologiczną mocą ekskrementów, co może się zamanifestować się odmową na wizytę w~łazience przez dwanaście godzin w~policyjnym busie, lub technikami nacisku specjalnie zaprojektowanymi, żeby ofiara zesrała się w~spodnie).

 Mimo to wszystko to zasadniczo działa w~stosunkowo ograniczonym oknie prawnym. Jak wskazują socjologowie policji (np. Bittner 1990), postępowanie funkcjonariuszy organów ścigania jest w~dużej mierze nieuregulowane. Większość przepisów dotyczy użycia określonych narzędzi lub broni. Poza tym istnieje kilka prawnych wytycznych dotyczących tego, co policja może, a czego nie może robić na ulicach lub co strażnicy mogą, a czego nie mogą robić w~więzieniach. Istniejące zasady są rzadko egzekwowane. Pociągnięcie do odpowiedzialności prawnej policjanta za, powiedzmy, pobicie cię w~trakcie aresztowania jest prawie niemożliwe, aby zostać oskarżonym w~takim przypadku, policjant musi w~zasadzie zrobić coś tak szokującego (na przykład sodomię pałką), że przez kilka dni trafia na pierwsze strony gazet\footnote{Przykładami są tu oczywiście pobicie Rodney Kinga lub przypadek Amadou Diallo.}. Powodem, dla którego większość Amerykanów ma wrażenie, że policja działa pod ekstremalnymi ograniczeniami, jest to, że istnieje wiele przepisów, które wpływają na wszystko, co ma związek z~procesem. Zasadniczo sprowadza się to do tego, że jeśli policja złamie przepisy, jedyną rzeczą, jaką ryzykuje, jest możliwość uzyskania wyroku skazującego. Paradoksalny rezultat jest taki, że policja musi być o wiele bardziej skrupulatna w~kontaktach z~mordercami lub gwałcicielami niż z~aktywistami, którzy będąc w~większości niewinni nawet w~stosunku do naruszeń parkingowych, prawdopodobnie nigdy nie zostaną oskarżeni o jakiekolwiek przestępstwo. Policja w~Philly była całkiem dobrze świadoma, pomimo wykorzystywania obrońców publicznych do straszenia zatrzymanych, że nie mają prawie żadnych szans na uzyskanie wyroku skazującego. W rezultacie aktywistów nie można było długo przetrzymywać.

 To była prawdziwa ironia odpryskiwania farby: cały mały dramat pozorowanego procesu mógł wydawać się sposobem na ustanowienie totalnej arbitralnej władzy, ale był to także sposób na wykreowanie jedynego procesu, w~którym działacz zostałby faktycznie uznany za winnego. Tak jak aktywiści próbują tworzyć przestrzenie autonomii i~kreatywności w~szczelinach normalnego porządku prawnego, tak i~policja robi coś bardzo podobnego: wycina w~szczelinach stworzonych przez prawo niewielką przestrzeń czystej suwerennej władzy. Gra arbitralności, sadystyczna przemoc, kłamstwa, pogwałcenie zwykłych norm i~oczekiwań to wszystko sposoby na usiłowanie ustanowienia absolutnej nierównowagi między państwem a jego władzami, pomimo faktu, że policja jest dość ograniczona w~rodzaje mocy, które mogą faktycznie wykorzystać.

\medskip
\noindent KILKA KRÓTKICH UWAG DOTYCZĄCYCH ZASAD ZAANGAŻOWANIA
\medskip

 Myślę, że powyższe obserwacje mają również wpływ na to, co dzieje się na ulicach.

 W moich wcześniejszych dyskusjach na temat akcji bezpośredniej jednym z~głównych punktów kontrastu między różnymi rodzajami akcji była ich relacja z~policją. Czy próbuje się z~nimi pogodzić, skonfrontować się z~nimi, próbować stwarzać sytuacje, w~których są zmuszeni działać z~umiarem, czy też całkowicie ich unikać i~zachowywać się tak, jakby nie istniała? Jednak w~pierwszych dwóch przykładach, marszu/wiecu i~linii pikiet, można powiedzieć, że policja i~protestujący działają w~ramach tej samej legalistycznej siatki. Istnieje nadrzędna struktura prawa i~precedensów prawnych; szczegóły mogą zostać opracowane na tej podstawie poprzez bezpośredni kontakt między zainteresowanymi stronami. W przypadku nieposłuszeństwa obywatelskiego i~akcji bezpośredniej tak nie jest. Mamy do czynienia ze zderzeniem dwóch całkowicie odmiennych moralnie światów. Nie chodzi o to, że nie ma żadnych zasad. Zarówno protestujący, jak i~policja działają zgodnie z~wyszukanymi kodeksami postępowania. To bardziej jak gra, w~której każda ze stron gra według własnego zestawu zasad.

 Niektórzy aktywiści w~rzeczywistości upierają się, że chodzi o to, aby znaleźć sposób na wykorzystanie przepisów drugiej strony przeciwko nim:

 Cała podstawowa idea nieposłuszeństwa obywatelskiego polega na stworzeniu ,,pułapki zasad zaangażowania'' (pułapka ROE), w~której wiesz, jakich taktyk może i~nie może użyć twój wróg oraz w~jakich sytuacjach, i~odpowiednio zaplanować swoją taktykę. Załóżmy na przykład, że chcesz zablokować paradę inauguracyjną. Wiesz, że prawo nie zezwala na użycie śmiertelnej siły w~celu zmuszenia ludzi do szukania ukrycia i~wiesz, że aresztowania ,,przed przestępstwem'' nie będą zbyt często stosowane, ponieważ nie są bardziej legalne niż samo blokowanie parady. Liczysz i~obliczasz, że jeśli zdołasz skłonić 10 000 osób, aby po prostu usiadło na Penn Ave, legalne aresztowanie ich wszystkich zajmie więcej czasu niż czas przeznaczony na paradę.

 Utworzyłeś teraz pułapkę ROE. Opozycja ma wybór między robieniem tego, co chcesz (w tym przypadku odwołanie parady), a łamaniem własnych praw, zapominaniem o legalnych aresztowaniach i~uciekaniem się do niekontrolowanej przemocy. Wadą tego dla Nieprzyjaciela jest głównie polityczny wpływ bycia postrzeganym jako represyjna, bezprawna dyktatura\footnote{Luke Kuhn, wpis do listy ACC DC, 18 grudnia 2003 roku}.

 Zwróć jednak uwagę, jak szybko pytanie dotyczy zarządzania wrażeniami i~roli mediów: ,,bycie widzianym'' zależy od tego, kto przekazuje informacje. Autor jednak omija to, zauważając, że prawdziwym niebezpieczeństwem dla państwa jest eskalacja: ilu z~nich ,,przejdzie do fizycznej akcji bezpośredniej'' lub, jak już o tym mowa, do wojny partyzanckiej.

 Taka jest perspektywa oddanego rewolucjonisty. Moja perspektywa jest tutaj mniej strategiczna niż taktyczna, jak każdy etnograf, chcę wydobyć ukryte zasady działania. Jakie są zatem skuteczne zasady zaangażowania, które stanowią podstawę tej kalkulacji i~jak są one opracowywane?

 Rozważmy jeszcze raz nasze ostatnie studium przypadku: starcie między Czarnym Blokiem a policją w~Filadelfii podczas konwencji republikanów w~2000 roku. Wiele z~tego można opisać jako rodzaj walki bez przemocy, pełnej manewrów, ruchów flankujących, prób utrzymania terytorium, postępów i~odwrotów. Obie strony również starannie wypracowały własne zasady zaangażowania. Wszyscy uczestniczący w~radach delegatów zgodzili się na pewne minimalne podstawowe zasady, na przykład, że nikt nie wnosi na akcję narkotyków, alkoholu ani broni, że nie wyrządziłoby to krzywdy żywym istotom. Chociaż z~pewnością istniały różnice, powiedzmy, między kodeksami postępowania tych, którzy przyjęli zasady klasycznych reguł nieposłuszeństwa obywatelskiego bez przemocy (którzy, na przykład, przeszli szkolenie w~zakresie niestosowania przemocy) a anarchistami z~Czarnego Bloku, te ostatnie również działały w~ramach bardzo wyraźnego kodeksu etycznego, który między innymi określał, jakie rodzaje własności są uzasadnionymi celami, a jakie nie. Koledzy aktywiści wiedzieli lub mogli łatwo dowiedzieć się, co to za kodeksy.

 Przepisy policyjne pozwalały im atakować protestujących mniej więcej do woli, ale przynajmniej w~tym historycznym momencie wydawało się, że muszą to zrobić w~taki sposób, aby mieć całkowitą pewność, że nikt nie zostanie zabity lub okaleczony i~nie więcej niż garstka wymaganej hospitalizacji. Innymi słowy, sytuacja wyglądała podobnie jak w~więzieniach, z~tym wyjątkiem, że na ulicach, w~otwartej i~zmieniającej się sytuacji quasi-walki, znacznie trudniej było zapewnić taki efekt. Podobnie jak aktywiści, policja opracowała różne specjalne techniki i~technologie oraz przeprowadziła szkolenia, aby móc to osiągnąć. Fascynujące jest to, że nie tylko zasady po obu stronach nie były bezpośrednio negocjowane, ale nie jest do końca jasne, czy większość członków każdej ze stron była nawet świadoma, że  druga w~ogóle przestrzega zasad. Jeśli nic innego, zarówno aktywiści i~policja odnieśli wrażenie, że druga strona była przygotowana do bycia znacznie bardziej brutalną niż była, i~obie strony uznały swoje ograniczenia za jednostronne\footnote{Kiedy rozmawiałem z~policją, nikt nie był skłonny zaakceptować faktu, że protestujący nie zaatakują ich, chociaż nikt nie był w~stanie przedstawić przykładów - poza takimi miejscami jak Praga czy Włochy - policji, która została faktycznie zaatakowana. Podobnie działacze bardzo niechętnie pogodzili się z~myślą, że władze nie są przygotowane do ich zabicia.}. Tak dzieje się prawie zawsze podczas akcji masowych. Niemniej jednak wyraźnie, wypracowuje się pewien rodzaj milczących ustaleń, a zasady zaangażowania zmieniają się w~czasie. \textit{Istnieje }proces, w~którym zasady są negocjowane, chociaż pośrednio. Pytanie do etnografa polega na zrozumieniu, co to jest.

 Pozwólcie, że zajmę się przez chwilę tą ideą walki bez przemocy.

 Clausewitz jest znany z~tego, że zdefiniował wojnę jako nieograniczone użycie siły, moment, w~którym wszystkie zasady są skutecznie odrzucane. Jak zauważyły  pokolenia późniejszych teoretyków, po prostu tak nie jest. Wojna nie jest i~nigdy nie była czystą walką sił bez reguł. Historycznie prawie wszystkie konflikty zbrojne miały bardzo złożone i~szczegółowe zestawy wzajemnych porozumień między walczącymi stronami. (Kiedy dojdzie do wojny totalnej, jej praktycy -- Attila, Cortes -- są pamiętani tysiąc lat później właśnie z~tego powodu). Jak zauważa historyk wojskowości Martin Van Creveld (1991), zawsze będą:

\noindent --- zasady układów i~rozejmów oraz traktowania negocjatorów

\noindent --- zasady dotyczące poddawania się i~traktowania jeńców

\noindent --- zasady, jak odróżnić walczących i~niewalczących oraz co można, a czego nie można z~tymi ostatnimi zrobić

\noindent --- zasady określające poziomy i~rodzaje siły dopuszczalne między walczącymi -- która broń lub taktyka jest niehonorowa lub nielegalna (tj. nawet podczas II wojny światowej ani Hitler, ani Stalin nie próbowali się nawzajem skrycie zamordować ani nie użyli broni chemicznej przeciwko swoim siłom).

 Są też inne -- na przykład dotyczące traktowania lekarzy -- ale ta lista na razie wystarczy.

 Van Creveld wysuwa interesujący argument, że takie zasady w~żaden sposób nie stoją na przeszkodzie efektywnemu użyciu siły; raczej nie można bez nich skutecznie zastosować siły. Bez zasad nie da się utrzymać prawdziwego morale czy struktury dowodzenia. Armia bez kodeksu honoru i~dyscypliny staje się jedynie bandą maruderów, a w~obliczu prawdziwej armii bandy maruderów zawsze przegrywają. Albo są rozproszeni, albo uciekają. Ale Van Creveld sugeruje inny powód, który moim zdaniem jest jeszcze bardziej odkrywczy. Zauważa, że  w~bitwie bez reguł nie można ustalić, kto wygrał. Ostatecznie obie strony muszą się zgodzić przynajmniej w~tej kwestii. W przeciwnym razie wojna nigdy się nie skończy, chyba że jedna ze stron całkowicie wytępi wroga.

 W tym świetle rozważmy policję. Policja często lubi myśleć o sobie jako o żołnierzach. Przywiązują dużą wagę do utrzymania morale i~dyscypliny. Jednak o ile postrzegają siebie jako walczących w~wojnie -- ,,wojnie z~przestępstwami'' -- wiedzą również, że są zaangażowani w~konflikt, w~którym zwycięstwo jest z~definicji niemożliwe.

 Jak to wpływa na zasady zaangażowania? Cóż, myślę, że tutaj zauważyć można coś bardzo interesującego. Jeśli chodzi o poziomy siły, jakiego rodzaju broni lub taktyki można użyć i~w jakich okolicznościach, policja oczywiście działa według zasad znacznie bardziej restrykcyjnych niż jakikolwiek żołnierz. Zasady zaangażowania (tj. policja absolutnie nie może zastrzelić białej osoby, chyba że ta biała osoba strzeli do niej pierwsza) są bardzo ograniczające. W rzeczywistości za każdym razem, gdy policjant strzela z~pistoletu, zwykle musi być przeprowadzone śledztwo. W rezultacie zdecydowana większość amerykańskiej policji nigdy nie strzelała ze swojej broni. Jednak w~każdych okolicznościach, które nie wiążą się z~przyszłym procesem lub potencjalnie śmiertelną siłą, jak już wspomniano, prawie nie istnieją żadne skuteczne regulacje.

 Jeśli chodzi o pozostałe pozycje, to odkrywa się, że podczas akcji policja systematycznie je wszystkie narusza. Regularnie angażują się w~praktyki, które w~czasie wojny zostałyby uznane za całkowicie niehonorowe. Policja regularnie aresztuje mediatorów. Jeśli członkowie grupy afinicji zajmują budynek, a jeden członek nie wchodzi do budynku, ale działa jako łącznik policji, może się okazać, że negocjator jest jedyną osobą, która zostanie aresztowana. Jeśli ktoś wynegocjuje porozumienie z~policją, policja prawie zawsze je złamie. Policja często atakuje tych, którzy oferują bezpieczne przejście. Jeśli protestujący, przeprowadzając akcję bezpośrednią w~jednej części miasta, próbują stworzyć ,,zielone strefy'' lub bezpieczne przestrzenie w~innej -- innymi słowy, starają się stworzyć teren, w~którym nikt nie będzie łamał prawa ani nie prowokował władz, jako sposób na odróżnienie walczących i~niewalczących -- policja prawie zawsze atakuje lub zaczyna aresztować ludzi w~bezpiecznej przestrzeni. Podobnie jak w~Quebecu, często celują w~medyków.

 Czemu? Bez wątpienia jest wiele powodów. Niektóre są po prostu pragmatyczne. Nie ma potrzeby rozumieć, jak traktować więźniów, jeśli możesz aresztować protestujących, ale protestujący nie mogą cię aresztować. Jednak w~szerszym sensie odmowa przestrzegania zasad wojny jest sposobem na odrzucenie implikacji równoważności, która miałaby zastosowanie w~przypadku walki z~inną armią. Policja reprezentuje państwo. Państwo ma monopol na legalne stosowanie przemocy. Jest więc z~definicji niewspółmierna do jakiegokolwiek innego elementu społeczeństwa. Jak zauważyli socjologowie policji, tacy jak Egon Bittner, jedyną wspólną cechą tego rodzaju sytuacji, do której przyporządkowana jest policja, jest możliwość narzucenia ,,nienegocjowanych rozwiązań popartych potencjalnym użyciem siły'' (Bittner 1990). Kluczowym terminem jest tutaj ,,nienegocjowany. Policja nie negocjuje, ponieważ oznaczałoby to równoważność. Kiedy są do tego zmuszeni, prawie zawsze łamią swoje słowo\footnote{Rozważmy tutaj fakt, że ,,negocjatorzy policyjni'' są na ogół zatrudniani w~sytuacjach zakładników; innymi słowy, aby faktycznie skłonić policję do negocjacji, trzeba dosłownie przyłożyć broń do czyjejś głowy. A w~takich sytuacjach trudno oczekiwać od policji, że dotrzyma swoich obietnic; w~rzeczywistości mogliby równie dobrze argumentować, że są moralnie do tego zobowiązani.}.

 Oznacza to jednak, że policja znajduje się w~paradoksalnej sytuacji. Uosabiają monopol państwa na użycie siły przymusu, ale ich wolność użycia tej siły jest poważnie ograniczona. Odmowa traktowania drugiej strony jako honorowych przeciwników, jako równorzędnych na każdym poziomie, wydaje się jedynym sposobem na zachowanie zasady bezwzględnej niewspółmierności, którą z~definicji muszą utrzymywać przedstawiciele państwa. To, nawiasem mówiąc, wydaje się powodem, dla którego, jeśli usunie się ograniczenia dotyczące użycia siły przez policję, skutki są katastrofalne: za każdym razem, gdy widzi się wojny, które naruszają wszystkie zasady i~wiążą się ze straszliwymi okrucieństwami wobec ludności cywilnej, niezmiennie są przedstawiane jako ,,działania policyjne'' 

 Nic z~tego w~rzeczywistości nie odpowiada na pytanie, w~jaki sposób negocjowane są zasady zaangażowania, ale przynajmniej wyjaśnia, dlaczego nie można tego zrobić bezpośrednio lub otwarcie. Wydaje się to szczególnie prawdziwe w~Stanach Zjednoczonych. W innych krajach, od Madagaskaru po Włochy, warunki czasami mogą być ustalane milcząco, a nawet nie tak milcząco, między organizatorami a policją. W rezultacie protest może stać się rodzajem gry, w~której zasady są jasno rozumiane przez każdą ze stron -- np. ,,uderzaj nas tak mocno, jak chcesz, pod warunkiem, że uderzasz nas w~nasze ochraniacze; nie uderzymy cię, ale spróbujemy przedrzeć się przez barykady w~naszych watowanych kombinezonach; zobaczmy, kto wygra!''. Na przykład przed spotkaniami G8 w~Genui władze włoskie zostały zmuszone do sprowadzenia LAPD w~celu przeszkolenia włoskiej policji, jak nie wchodzić w~interakcje z~protestującymi lub pozwolić którejkolwiek ze stron na skuteczne uczłowieczenie w~oczach drugiej. Organizatorzy z~Ya Bastą! i~podobnych grup powiedzieli mi później, że wiedziały, że stanie się coś strasznego, kiedy policjanci, których numery telefonów komórkowych zebrali, nagle przestali odbierać telefony. Ale przynajmniej w~USA proces negocjacji jest prawie zawsze pośredni.

 Jednak sposób, w~jaki przebiegają negocjacje, jest krytyczny, ponieważ to jest prawdziwe miejsce władzy. Jak każdy antropolog polityczny może ci powiedzieć, najważniejszą formą władzy politycznej nie jest możliwość wygrania konkursu, ale władza określania reguł gry; nie moc do wygrania kłótni, ale moc do zdefiniowania, o co chodzi w~kłótni. Tutaj jest jasne, że w~rzeczywistości władza nie jest po jednej stronie. Ograniczenia policyjne nie są narzucane jej samej. Lata walki moralno-politycznej ze strony każdego, od National Lawyers Guild lub ACLU po prawicowych libertariańskich entuzjastów broni, w~tym setki grup o bardzo różnych stosunkach z~rządem, stworzyły sytuację, w~której policja musi zaakceptować pewne ograniczenia użycia siły. Te ograniczenia są, jak wciąż podkreślam, bardzo nierówne (znowu wszystko to jest o wiele bardziej prawdziwe, gdy mamy do czynienia z~ludźmi określanymi jako ,,biali''), niemniej jednak działają jako realna granica zdolności państwa do tłumienia sprzeciwu. Problem dla tych, którzy wyznają zasadę akcji bezpośredniej, polega na tym, że chociaż te zasady zaangażowania -- w~szczególności poziomy policji siłowej, które mogą ujść na sucho -- podlegają ciągłym renegocjacjom, oczekuje się, że proces ten będzie odbywał się w~dużej mierze na drodze formalno-prawnej i~poprzez kanały polityczne i~media głównego nurtu. Innymi słowy, poprzez instytucje wyraźnie odrzucają.

 Tutaj wracam z~pełną siłą do pytania, które w~dużej mierze pomijałem w~trakcie tego rozdziału. Protest ma na celu wywołanie zmian w~dużej mierze poprzez próbę wpłynięcia na coś, co nazywa się ,,publicznością''. Nieposłuszeństwo obywatelskie polega na próbie ,,publicznego'' ujawnienia przemocy lub niesprawiedliwości systemu. Zatem ostatecznym sędzią w~sprawach zasad zaangażowania jest coś, co nazywa się ,,publicznością''. Ale co to jest? Przynajmniej w~Stanach Zjednoczonych zasadniczo zakłada się, że opinia publiczna jest odbiorcą mediów korporacyjnych. Albo na przemian wyborcami i~konsumentami usług publicznych. Mimo wszystko to w~zasadzie to. ,,Społeczeństwo'' istnieje zatem tylko w~relacji do mediów i~klas politycznych. Z kolei ,,opinia publiczna'' może wyrażać się jedynie poprzez pewien rodzaj mediacji: na przykład sondaże, które mogą (lub nie) wpływać na politykę. Można zobaczyć, jak daleko jest to od aktywistycznego -- a zwłaszcza anarchistycznego -- ideału samoorganizacji, biorąc pod uwagę fakt, że zgodnie z~językiem zwykle używanym przez media i~klasy polityczne, w~momencie, gdy członkowie społeczeństwa dokonują samoorganizacji w~jakikolwiek sposób (powiedzmy, przystępując do związków zawodowych lub stowarzyszeń politycznych), nie są już społeczeństwem, ale ,,grupami specjalnego interesu''. W ten sposób samo pojęcie społeczeństwa staje w~sprzeczności z~tym, co próbują osiągnąć aktywiści.

 Nic więc dziwnego, że czują się głęboko ambiwalentni wobec grania w~tę konkretną grę. 

 W rezultacie negocjacje dotyczące zasad zaangażowania odbywają się w~dużej mierze poprzez wyrachowane wysiłki mające na celu wpłynięcie na zapośredniczoną ,,opinię publiczną'', którą policja, przynajmniej w~Ameryce, jest gotowa grać dość agresywnie, ale aktywiści, a zwłaszcza anarchiści, coraz bardziej niechętnie grają. Podjęto wiele prób obejścia tego problemu. Aktywiści próbowali odwoływać się bezpośrednio do społeczności -- szczególnie ubogich, imigrantów lub społeczności robotniczych. Próbowali tworzyć koalicje ze związkami i~innymi już istniejącymi organizacjami. Próbowali stworzyć własne, nowe formy mediów, a co za tym idzie, nową publiczność: na przykład poprzez Niezależne Centra Medialne (ang. Independent Media Centers -- IMC). Wyniki były nierówne, ale jak zobaczymy, biorąc pod uwagę stopień, w~jakim wszystkie karty w~mediach korporacyjnych są przeciwko nim, trudno byłoby udowodnić, że mają duży wybór.

\section{Wnioski}

 To, co badaliśmy zatem w~trakcie tego rozdziału, jest próbą stworzenia drobnych sytuacji dwuwładzy.

 Polityka protestu funkcjonuje w~określonych ramach prawnych lub instytucjonalnych; stara się zebrać powszechne poparcie dla obalenia określonych polityk; może nawet dążyć do obalenia konkretnego rządu, ale nie dąży do zmiany samych ram. Niemniej jednak nawet w~stosunkowo łagodnych formach protestu tkwią zalążki czegoś innego. O ile marszałkowie nie stają się jedynie dodatkami w~policji, o ile wiece nie istnieją tylko po to, by wspierać kandydatów, dają zapowiedź innej formy społeczeństwa i~organizacji. Jest to już przynajmniej drobny element prefiguracyjny. Kiedy przechodzi się do właściwie pojętej akcji bezpośredniej, ten element prefiguracji staje się głównym punktem: ci, którzy wykonują akcję bezpośrednią, obstają przy swoim prawie do działania tak, jakby byli już wolni. Ale jednocześnie, nawet tutaj, prawie zawsze są jakieś ślady logiki protestu. Stąd zmieniające się, niestabilne i~często bardzo niejednoznaczne relacje między społecznością, publicznością, celami i~policją, nad którymi spędziłem tak wiele w~rozdziale, próbując udokumentować. Być może akcja bezpośrednia i~protest nigdy nie mogą być całkowicie niezależne od siebie.

 Jeśli ktoś posuwa zasadę działania bezpośredniego wystarczająco daleko, jeśli ewoluuje od taktyki do strategii, logicznie zmierza w~kierunku tworzenia znacznie bardziej wyrafinowanych i~trwalszych form dwuwładzy. To kolejny powód, dla którego EZLN, Zapatystyczna Armia Wyzwolenia Narodowego w~Meksyku, okazała się taką inspiracją dla anarchistów na całym świecie: była to jedna grupa, która odniosła największy spektakularny sukces w~jej realizacji. Słynne dziesięciodniowe powstanie w~styczniu 1994 roku było przede wszystkim próbą otwarcia przestrzeni dla pozbawionej przemocy akcji bezpośredniej; EZLN natychmiast odłożył broń, ogłosił zawieszenie broni, ale dał jasno do zrozumienia, że  nadal mają środki do kontynuowania walki zbrojnej, jeśli poczują, że nie mają alternatywy. Można by to nazwać momentem negocjacji; ,,walki moralno-politycznej'', jak to określiłem wcześniej, by zdefiniować warunki zaangażowania, sztuki, w~której Zapatyści okazali się najbardziej biegli. Nastąpiła powolna i~trudna praca polegająca na utrzymaniu równowagi sił, która umożliwiła otwarcie, przy jednoczesnym wykorzystaniu okazji do powolnego budowania wspólnot autonomicznych. Kiedy nie ma się tak dramatycznego dostępu do siły zbrojnej, powszechnym podejściem jest rozpoczęcie organizowania się wokół czegoś, czemu nikt tak naprawdę nie może się poważnie sprzeciwić: na przykład bezpłatnej kliniki, nawet wspólnego ogrodu. Następnie próbuje się zbudować niezależną infrastrukturę wokół niekwestionowanej instytucji, negocjować jakieś milczące porozumienie z~władzami, aby przynajmniej pozostały na dystans, a następnie próbować rozszerzyć swoją strefę autonomii na większą społeczność i~sprzymierzyć się z~podobnymi projektami gdzie indziej. Takie wysiłki trwają ciągle. Jak zawsze podkreślają krytycy ,,przeskakiwania na szczyt'', skuteczna długoterminowa strategia będzie musiała być oparta na społeczności, chociaż, ponieważ obrońcy masowych mobilizacji często (zwykle ciszej i~niepewnie) odpowiedzą, bez okazjonalnych spektakularnych mobilizacji, jest to o wiele trudniejsze, ponieważ trudno jest zachować poczucie, że ruch się w~ogóle dzieje. W każdym razie niektórzy mogą argumentować, że koncentracja w~tym badaniu na Stanach Zjednoczonych, a zwłaszcza na Nowym Jorku, miała tendencję do wypaczania wyników: są to w~końcu epicentra imperium, a zatem najtrudniejsze miejsca na Ziemię, aby spróbować strategii podwójnego zasilania. W rezultacie grupy i~działania, którym się przyglądamy, mają zwykle pewien nieistotny aspekt, który prawdopodobnie byłby znacznie mniej zaznaczony, gdybym rozpoczął pracę w~innej części świata, ponieważ ruchy takie jak ten w~rzeczywistości zaczynają pojawiać się niemal wszędzie. Niemniej jednak ta sama niesubstancjalność jest, jak sądzę, warta przestudiowania sama w~sobie, ponieważ ułatwia obserwowanie niejako elementarnych form i~elementarnych dylematów jakiejkolwiek polityki prefiguratywnej.

\chapter{Reprezentacja}

 W typologii działań, którą zarysowałem w~rozdziale 8, podkreśliłem, że wszystkie one mogą stanowić elementy lub składniki większych akcji masowych; ale o samych akcjach masowych miałem stosunkowo niewiele do powiedzenia. Większość z~tych masowych akcji -- jak Québec City -- zawierała elementy wszystkich pięciu z~nich: marsze i~wiece, pikiety przeciwko przestępcom korporacyjnym, karnawały, blokady, akcje Czarnego Bloku i~jeszcze więcej. Nikt jednak nie miał bezpośredniego doświadczenia działania jako całości. Na przykład w~Quebec City przez trzy dni byłem w~awanturze, ale nigdy nie widziałem marszu ani Żywej Rzeki; na A16 dołączyłem do blokad, ale nigdy nie widziałem Czarnego Bloku; w~Filadelfii byłem z~Blokiem, ale nigdy nie widziałem ani jednej blokady. O ile można doświadczyć akcji masowej jako całości, można to zrobić tylko poprzez jakąś formę reprezentacji: niezależnie od tego, czy aktywiści opowiadają sobie nawzajem historie, czy raporty prasowe, czy streszczenia łatane później w~wiadomościach aktywistycznych, przez CNN lub w~czterdziestominutowej filmowej wersji akcji, która z~pewnością zostanie opublikowana kilka tygodni później przez zespoły wideo IMC. Dopiero w~tym rozdziale, który jest o takich przedstawieniach, można również mówić o akcjach masowych jako o totalnościach.

 Całość zatem nie istnieje jako przedmiot doświadczenia. Musi zostać stworzona za pomocą technik, które mogą obejmować struktury narracyjne, montaż filmów, organizację pamiętników lub esejów fotograficznych i~tak dalej. Oznacza to oczywiście, że nie ma jednej, ale jest tysiąc całości. Każda z~nich jest deklaracją polityczną: w~efekcie argumentem o ostatecznym znaczeniu wydarzenia. Jak można sobie wyobrazić, anarchiści i~aktywiści wywodzący się z~tradycji akcji bezpośredniej nie widzą tego jako walki o narzucenie ostatecznej wersji wydarzeń, jednej nadrzędnej narracji. Wielość jest częścią całego punktu. Ale to nie oznacza, że  widzą wszystkie opisy jako jednakowo ważne, lub że nie są głęboko zszokowani i~urażeni, gdy następnego dnia biorą do ręki gazetę i~czytają relacje w~mediach głównego nurtu na temat niemal każdego wydarzenia, w~którym sami brali udział.

 Ten rozdział będzie częściowo dotyczył takich konstrukcji medialnych, częściowo działań aktywistów na rzecz tworzenia alternatywnych form komunikacji i~alternatywnych odbiorców. Podniosę również argument, że dla aktywistów te nowe formy mediów, a przede wszystkim możliwość natychmiastowego odegrania roli w~opowiadaniu historii wydarzenia, są teraz krytyczne dla samego doświadczenia.

 To trochę ironiczne, że rozdział dotyczący całości musi być najbardziej częściowy i~fragmentaryczny ze wszystkich. Ale nie sądzę, żeby można było tego uniknąć. W tej grze jest po prostu zbyt wielu graczy. Od samego początku, pisząc tę  książkę, musiałem podejmować decyzje, jakie perspektywy przedstawić. Zacząłem od decyzji, mówiąc o różnicach między aktywistami, by ograniczyć się do perspektywy anarchistycznej (zwłaszcza tego, co można by nazwać perspektywą anarchistyczną ,,małego litery a''), w~przeciwieństwie do poglądów grup liberalnych lub marksistowskich. Po części dzieje się tak dlatego, że perspektywa anarchistyczna jest jedyną, z~którą czuję się doskonale zaznajomiony. Podobnie, gdy mam do czynienia z~konfrontacjami między aktywistami a policją, ograniczyłem się do perspektywy aktywisty: w~rzeczywistości odtwarzając zakłopotanie aktywisty w~obliczu konieczności wyobrażenia sobie, jak rzeczy wyglądają z~perspektywy gliniarzy. W tym rozdziale opisuję interakcje co najmniej trzech stron -- aktywistów, reporterów i~policji -- z~których dwóch perspektyw tak naprawdę nie badałem na własne oczy\footnote{I~to oczywiście jest tylko najbardziej schematyczne. W rzeczywistości wszystkie grupy dzielą się na mniejsze: perspektywy gliniarzy ulicznych są bardzo różne od dowódców, czy też FBI lub ATF, kuratorów sądowych, pracowników prywatnych firm papierów wartościowych, funkcjonariuszy wywiadu policyjnego i~tak dalej. Media korporacyjne dzielą się na reporterów telewizyjnych, dziennikarzy prasowych, pracowników serwisów informacyjnych, różnego rodzaju radiowców i~dzielą się na dokumentalistów, pracowników radykalnych lub postępowych mediów (Pacifica Radio, bezpłatne tygodniki alternatywne), które są wyraźnie przyjazne lub prawicowe miejsca takie jak Fox News, które są otwarcie wrogie; są też różnego rodzaju aktywiści lub niezależni dziennikarze o bardzo różnych problemach i~perspektywach.}. Jednak w~tym kontekście tych innych perspektyw nie można po prostu zignorować. Na przykład znaczna część dalszej części rozdziału będzie dotyczyć prób zrozumienia szczególnej wrogości policji wobec gigantycznych marionetek. Na szczęście jednak inni badacze spędzili dużo czasu rozmawiając z~gliniarzami i~dziennikarzami -- z~pewnością znacznie więcej niż prowadzili badania nad anarchistami -- i~istnieje całkiem pokaźna literatura, z~której można czerpać. Powrócę się do tej literatury, kiedy uznam to za stosowne, ale moja relacja dotyczy głównie zrozumienia aktywistycznego punktu widzenia.

\section{Część I: Media korporacyjne}

\begin{flushright}


\texttt{ O tak, w~Ameryce wszyscy mamy prawo do wolności słowa. O ile oczywiście nie zdecydujesz się z~niej skorzystać.}

-- Kasjer Howarda Johnsona do anarchistów uwięzionych w~sklepie, podczas gdy policja wyrzucała protestujących z~Times Square, 15 lutego 2003
\end{flushright}

\noindent WSTĘPNA WSKAZÓWKA DOTYCZĄCA SKUTKÓW REPREZENTACJI W MEDIACH GŁÓWNEGO NURTU
\medskip

 Anarchiści mają tendencję do brzydzenia się korporacyjnymi mediami. Większość odmawia nawet rozmowy z~profesjonalnymi reporterami. Nawet ci, którzy pracują w~mediach podczas akcji, którzy tworzą banki telefoniczne i~zespoły uliczne, aby propagować punkt widzenia organizatorów, przyjmują za pewnik, że media korporacyjne są zasadniczo miejscem propagandy, a gazety i~sieci telewizyjne są firmami kapitalistycznymi, że byłoby beznadziejnie naiwne wierzyć, że kiedykolwiek można by oczekiwać, że właściwie przekażą antykapitalistyczny punkt widzenia. Podczas dużych mobilizacji od początku zakłada się, że media będą systematycznie stronnicze na korzyść policji.

 Tutaj znowu muszę zadeklarować własne uprzedzenia. Myślę, że mają rację. Sporo pracowałem z~anarchistycznymi zespołami medialnymi i~prawie wszystko, co widziałem, zdaje się to potwierdzać. Często mówię, że najbardziej, czego naprawdę można oczekiwać od mediów korporacyjnych podczas dużej mobilizacji, to poinformowanie opinii publicznej o istnieniu budzącej zastrzeżenia instytucji. W żadnym wypadku nie można oczekiwać, że media dokładnie poinformują opinię publiczną, \textit{dlaczego }protestujący uważają to za niedopuszczalne. Działania, które odnoszą największe sukcesy medialne, to te, na które wystarczy tylko wskazać. Przed Seattle, w~listopadzie 1999 roku, bardzo niewiele osób w~USA słyszało o WTO. Podczas gdy aktywistom nie udało się przekazać głównego przesłania, że instytucje takie jak WTO są zagrożeniem dla samej zasady demokracji, do jakiejkolwiek gazety, samo wskazanie istnienia WTO miało podobny skutek. To samo dotyczyło IMF i~Banku Światowego, które pięć miesięcy później na A16 podświetlono, instytucji, które stały się jednymi z~głównych instrumentów amerykańskiej potęgi na świecie, ale o których większość Amerykanów nigdy nie słyszała. Kiedy dwa miesiące później, na konwencji republikańskiej, DAN postanowiła poruszyć kwestię ,,więziennego kompleksu przemysłowego'' -- faktu, że coraz więcej produktów w~USA wytwarzają skazani, a korporacje, które zatrudniają więźniów, prawie zawsze dostarczają ogromny wkład w~kampanię kandydatów politycznych na rzecz utrzymania surowych wytycznych dotyczących wyroków, które od lat 80. potroiły liczbę skazanych w~Ameryce, wypełniając więzienia nieprzemocowymi przestępcami narkotykowymi -- to okazało się zbyt skomplikowane przesłanie. Wymagało analizy. W związku z~tym, kiedy media nie podejmą historii, wiadomość zwyczajnie jest zgubiona.

 Jeśli chodzi o stronniczość na korzyść policji, można to dość łatwo wykazać. Wystarczy porównać typowe wrażenia zwykłych obywateli, którzy przypadkowo wkraczają na scenę akcji bez uprzedzeń lub skłonności politycznych, z~tym, co ci sami obywatele zazwyczaj myślą, oglądając wydarzenie w~wiadomościach. Jak może powiedzieć każdy, kto brał udział w~licznych akcjach ulicznych, ludzie, którzy wędrują na miejsce akcji, prawie zawsze wychodzą ze współczucia protestującym. Jednocześnie, będąc świadkami zachowania policji, najpierw są zaskoczeni, a potem oburzeni. Dzieje się tak powszechnie, że często stwarza to moralny dylemat organizatorów. Z jednej strony nie chcesz prowadzić akcji w~miejscu, w~którym wielu niewinnych przechodniów może skończyć na omijaniu pojemników z~gazem łzawiącym, uciekających w~terrorze przed uzbrojonymi w~pałkę gliniarzami lub zmieceni w~masowych aresztowaniach. Z drugiej strony każdy organizator zdaje sobie sprawę, że absolutnie nic tak nie radykalizuje zwykłych obywateli, jak zobaczenie, jak to jest być w~środku akcji. Niemal niezmiennie podejmuje się wreszcie decyzję, by nie narażać niewinnych. Ale prawie zawsze decyzja jest podejmowana z~tęskną świadomością, że zrobienie czegoś innego spowodowałoby co najmniej pół tuzina gniewnych nowych anarchistów.

 W tym samym czasie ci sami obywatele, gdyby oglądali to samo wydarzenie w~telewizji lub czytali o nim w~gazetach, prawie nigdy nie reagują w~ten sposób. Jeśli już, to relacje w~mediach z~większym prawdopodobieństwem sprawią, że publiczność będzie skłonna wspierać represje policyjne, które zawsze są przedstawiane jako chroniące, a nigdy jako zagrażające ,,publiczności''. Reprezentacje mają zatem znaczenie. Często dokładnie odwracają perspektywę, jaką miałby naoczny świadek.

 Oczywiście jest to prawdą tylko wtedy, gdy wydarzenie jest w~ogóle relacjonowane przez media. Inną sprawą, która często szokuje widzów, jest fakt, że wydarzenia, których są świadkami, nie są uważane za wiadomości krajowe, a często nawet lokalne. Wciąż mam żywe wspomnienie rozmowy z~pakistańskim sklepikarzem w~sklepie z~kanapkami, przecznicę od komisariatu policji na Dolnym Manhattanie. Była noc marszu Peltiera i~czterech anarchistów Czarnego Bloku zostało arbitralnie wyrwanych z~parady przez policję i~przetrzymywanych na posterunku; około trzydziestu lub czterdziestu aktywistów zgromadziło się na zewnątrz, aby solidaryzować się z~więźniami. W pewnym momencie trzech z~nas przyszło do jego sklepu, aby skorzystać z~łazienki i~wziąć wodę i~inne artykuły. Nawiązaliśmy rozmowę. Kiedy wyjaśnialiśmy, co się stało, był coraz bardziej oburzony. 
 
 -- Ale, powinieneś zadzwonić do CNN! -- nalegał i~nie mógł zrozumieć, dlaczego kilku z~nas zaczęło się śmiać. 
 
 -- Nie, naprawdę! Wiesz, do kogo powinieneś zadzwonić? Jednej z~tych lokalnych stacji informacyjnych jak NBC. Pozwolę ci skorzystać z~mojego telefonu. A co z~programem New York 1? Założę się, że zrobią z~tego swoją główną historię. 
 
 Wydawał się tak szczery i~miał dobre intencje, że w~końcu udaliśmy, że się zgadzamy i~ktoś wyciągnął telefon komórkowy. Wyszedł, zapewniając nas, że będzie nas obserwował w~wieczornych wiadomościach.

 W rzeczywistości, oczywiście, większości doświadczonych aktywistów nawet nie przychodzi do głowy informowanie reporterów telewizyjnych o jakichkolwiek akcjach poza najbardziej masowymi, nie mówiąc już o przypadkach niewłaściwego postępowania policji (byłoby to po prostu marnowaniem czasu i~energii), tak samo zwykłym obywatelom nigdy nie przychodzi do głowy, że główne serwisy informacyjne nie byłyby zainteresowane takimi wydarzeniami.

 Są ku temu powody. Na przykład w~Stanach Zjednoczonych zwyczaj uznawania większości protestów za niewartych opublikowania wydaje być wywołany powszechne w~branży przekonaniem, że media poświęcały zbyt wiele uwagi protestom w~latach 60. XX wieku. Zgodnie z~tym, co wydaje się mądrością ludową wśród reporterów (przynajmniej tych, z~którymi rozmawiałem na ten temat), w~szczególności relacje telewizyjne są postrzegane jako skłaniające radykalne grupy studenckie do ciągłego przewyższania się nawzajem bardziej brutalnymi lub skandalicznymi wyczynami kaskaderskimi, prowadzącymi w~końcu do zamieszek i~masakr; aż w~pewnym momencie media zdały sobie sprawę, że same stały się częścią problemu. Nową politykę może podsumować starszy redaktor New York Times, Bill Borders, który, gdy został odpytany przez FAIR, grupę nadzorującą media, aby wyjaśnić, dlaczego Times prawie nie relacjonował protestów inauguracyjnych w~2001 roku (drugi co do wielkości protest inauguracyjny w~historii Ameryki), odpowiedział, że nie uważa samych protestów za wiadomości. Borders zauważył, że o ile protestujący próbowali poruszyć sprawę nieprawidłowości w~wyborach, które doprowadziły Busha do władzy, to \textit{Times} już szczegółowo opisał tę historię. Z drugiej strony sam protest był ,,wydarzeniem zainscenizowanym'', ,,przeznaczonym do opisania'', a zatem nie był prawdziwą wiadomością\footnote{,,AKTYWIZM AKTUALIZACJA: New York Times odpowiada na krytykę inauguracyjną'' informacja prasowa (22 lutego 2001 roku), Uczciwość i~dokładność w~sprawozdawczości (FAIR).}. 

 Protesty są więc w~zasadzie sztucznymi okularami, które mają wpływać na media lub manipulować nimi. Żadna odpowiedzialna gazeta by się na to nie zgodziła. Dyrektorzy mediów cały czas wysuwają tego rodzaju argumenty. Aktywista, oczywiście, miałby tendencję do odpowiadania, że  gdyby \textit{Times }rzeczywiście ignorował sztuczne spektakle organizowane tylko po to, by je opisać, to w~ogóle nie poświęciłby pięciu całych stron ceremoniom inauguracyjnym. Kiedy mamy sytuację podobną do tej w~Filadelfii podczas konwencji republikańskiej w~2000 roku, gdzie ponad dwadzieścia tysięcy reporterów spędziło dni, próbując wymyślić, jak wydobyć kolejną wiadomość z~drobiazgów całkowicie zaplanowanych ceremonii, ignorując w~dużej mierze toczone bitwy na ulicach między policją a anarchistycznymi blokami klaunów kilkaset metrów dalej, pomysł, że media pomijają działania, ponieważ są one ,,zainscenizowanymi wydarzeniami”, staje się oczywiście nie do utrzymania.

 Tak zresztą \textit{zareagowałby }aktywista, gdyby miał dostęp do mediów. W rzeczywistości takie argumenty są po prostu tego rodzaju rzeczami, które nigdy nie zostałyby wydrukowane, gdyby zostały wysłane jako, powiedzmy, list do redakcji. W przypadku inauguracji Busha FAIR dokonał oczywistej odpowiedzi we własnym komunikacie prasowym:

 Argument \textit{New York Times}, że demonstracji anty-inauguracyjnych nie musiał w~znaczący sposób relacjonować, ponieważ relacjonował już spór wyborczy na Florydzie, jest podobny do stwierdzenia, że  nie trzeba było intensywnie relacjonować strajków na segregowanym Południu, ponieważ gazeta omówiła już prawa Jima Crowa, przeciwko którym protestowały strajki okupacyjne\ldots 

 Jeśli chodzi o zarzut, że ,,są wydarzeniami inscenizowanymi, przeznaczonymi do relacjonowania'', można to powiedzieć o prawie całym procesie inauguracji, a także o dużym odsetku wydarzeń, o których donosi \textit{New York Times} w~Waszyngtonie. Różnica polega na tym, że demonstracje są inscenizowane przez zwykłych obywateli, podczas gdy inauguracje, oficjalne konferencje prasowe itp., które New York Times woli relacjonować, są inscenizowane przez ludzi mających dostęp do władzy\footnote{Tamże.}.

 Innymi słowy, nie chodzi o to, czy jakaś grupa próbuje manipulować mediami dla celów politycznych, ale o to, czy redaktorzy lub dyrektorzy mediów uważają, że grupa ma do tego prawo. Wydarzenia organizowane przez polityków, oficjalnie uznane grupy lobbingowe lub kierownictwo korporacji są zwykle uważane za warte opublikowania; te przez grupy protestacyjne zwykle nie są (chyba że mogą być opisane jako zagrożenie dla porządku publicznego). Ostateczna lojalność mediów dotyczy więc pewnej struktury władzy.

 Jednym z~powodów, dla których media mają taką obsesję na punkcie utrzymywania legitymizacji tej instytucjonalnej struktury władzy, jest to, że postrzegają siebie jako nieodłączną jej część, tak jak są w~rzeczywistości. Myślę, że to jest prawdziwe znaczenie pierwszego komentarza pana Bordersa. Ostatecznie chodzi o to, kto ma opowiedzieć tę historię: media rezerwują sobie ten przywilej\footnote{Być może w~tym przypadku nie ma zbyt dużej różnicy między sposobem, w~jaki CNN lub Times traktuje protestujących, a tym, jak radzą sobie z~dużymi korporacjami. Jeśli na przykład Monsanto organizuje zainscenizowane wydarzenie, aby wywołać w~prasie informacje na temat nowego rodzaju genetycznie zmodyfikowanego zboża, media będą mówić o zbożu, ale nie o tym wydarzeniu.}. Prowadzący gazety i~sieci telewizyjne uważają, że jest to ich podstawowa funkcja w~demokracji przedstawicielskiej: bycie ,,odpowiedzialnym'' (to znaczy jedynym uprawnionym) kanałem debaty publicznej. Ta postawa ma pewne pozornie paradoksalne skutki. Nie jest tak, że masowe akcje nigdy nie wpłynęły skutecznie na sposób, w~jaki media głównego nurtu relacjonują wiadomości. Są wszelkie powody, by sądzić, że mają. Jednak te serwisy informacyjne rzadko, jeśli w~ogóle, uznają ten wpływ. Prawie w~każdym przypadku zachowują się tak, jakby, zmieniając treść swojego reportażu, nie odpowiadały na ruchy społeczne, ale na jakąś stopniową zmianę opinii, jaka miała miejsce wśród tych, których uważają za uprawnionych opiniotwórczych -- ekspertów, publicystów, publicznych intelektualistów -- w~samych mediach. Tutaj zamknięcie WTO w~Seattle, i~A16 (blokady IMF i~Banku Światowego w~Waszyngtonie kilka miesięcy później) są doskonałym przykładem.

 Podczas A16 zespół aktywistów medialnych podjął świadomą decyzję o podjęciu kwestii ,,dostosowania strukturalnego'' -- neoliberalnych pakietów reform narzuconych biednym krajom jako warunek umorzenia pożyczek -- co, jak argumentowali, spowodowało masowe zubożenie, głód, choroby i~śmierć wśród ubogich świata. Przesłanie: nie protestujemy przeciwko wolnemu handlowi. Nie protestujemy przeciwko globalizacji. Sprzeciwiamy się dostosowaniu strukturalnemu. Na szkoleniach medialnych wszystkich, którzy rozmawiali z~reporterami, zachęcano do jak najczęstszego używania terminu ,,dostosowanie strukturalne''. Ten rodzaj powtarzania jest klasyczną taktyką stosowaną przez specjalistów PR, i~faktycznie, wielu wolontariuszy pracujących z~zespołami aktywistów medialnych było doświadczonymi specjalistami PR, którzy pracowali w~świecie reklamy korporacyjnej. Jednak reporterzy i~redaktorzy wydaje się, że szybko zorientowali się, co się dzieje, i~podjęli świadomą decyzję, by nie grać dalej. Sformułowanie ,,polityka dostosowania strukturalnego'' nie pojawiło się w~żadnej wiadomości na temat protestów. Dziennikarze nie tylko jednolicie opisywali protesty jako wymierzone przeciwko ,,wolnemu handlowi'' i~,,globalizacji'' (o ile w~ogóle byli skłonni przypisywać protestującym spójne stanowisko), redaktorzy systematycznie odmawiali publikowania którejkolwiek z~kilkudziesięciu artykułów i~listów do redakcji wysyłane przez zespoły mediów aktywistów, wiele napisanych przez wybitnych ekonomistów aktywistów lub innych naukowców, lub nawet wspominających o istnieniu skomplikowanych konferencji intelektualnych omawiających alternatywy, które zawsze miały miejsce obok samych działań. Zamiast tego ,,New York Times'' na przykład, 16 kwietnia opublikował trzy różne artykuły twierdzące, że protestujący są głupi i~wprowadzeni w~błąd, że w~rzeczywistości, globalizacja i~wolny handel były jedyną nadzieją dla biednych na świecie\footnote{Logika, oczywiście, nigdy nie była taka, że  biedni nie rozumieją, co jest dla nich dobre, ale bogaci rozumieją -- ponieważ protestujący w~końcu jedynie powtarzali żądania tych z~Globalnego Południa dotkniętych polityką neoliberalną.}. Następnego dnia na pierwszej stronie opowiedziano historię działań niemal wyłącznie z~perspektywy porządku publicznego, wprowadzając czymś, co można nazwać tylko artykułem redakcyjnym przebranym za artykuł informacyjny przez reportera Johna Kifnera, chwalącego policję w~Waszyngtonie za zastosowanie siły w~systematyczny i~efektowny sposób niż w~Seattle\footnote{,,Liderzy finansowi spotykają się, gdy protesty zapychają Waszyngton'', John Kifner z~Davidem E. Sangerem, New York Times, 17 kwietnia 2000, sekcja A, strona 1.}. W jednym z~artykułów na pierwszej stronie z~18 kwietnia cytowano szefa policji Ramseya, który powiedział, że protesty były ,,sytuacją, w~której wszyscy wygrywają'', i~zauważył, że obie strony były w~stanie ogłosić zwycięstwo: policji, ponieważ udało im się uniemożliwić protestującym zamknięcie spotkania, podczas gdy ,,protestujący z~Mobilizacji na rzecz Sprawiedliwości Społecznej radowali się, że ich niegdyś niejasne sprzeciwy wobec międzynarodowej polityki pieniężnej znalazły się teraz na pierwszych stronach'', i~to pomimo tego, że oświadczenie było ewidentnie nieprawdziwe. W rzeczywistości nigdzie na pierwszej stronie tego czy jakiegokolwiek innego wydania \textit{Timesa} nie było żadnej wskazówki, czym w~ogóle były te ,,zarzuty''\footnote{,,W tym Waszyngtonie nie znaleziono żadnego ,,Seattle'' ani przez policję, ani przez protestujących'', John Kifner, New York Times, 18 kwietnia 2000 roku, sekcja A, strona 1.}˘.

 Fascynujące jest jednak to, co wydarzyło się później. Relacja z~A16 miała miejsce w~szczytowym okresie ,,konsensusu waszyngtońskiego'', kiedy neoliberalizm był wciąż traktowany jako oczywisty, nieunikniony kierunek historii. W kolejnych miesiącach zaczęło się to powoli zmieniać. Ponieważ ,,ruch antyglobalistyczny'', w~rzeczywistości globalny ruch przeciwko polityce neoliberalnej, wydawał się nabierać rozpędu wszędzie, rządy i~naukowcy zaczęli ponownie przemyśleć swoje stanowiska. Kilku prominentnych ekonomistów neoliberalnych, takich jak Jeffrey Sachs i~Joseph Stiglitz, wybiło się z~szeregu i~zaczęło argumentować, że polityka dostosowania strukturalnego rzeczywiście miała wszystkie katastrofalne konsekwencje, jak twierdzili protestujący. W ciągu roku lub dwóch w~\textit{Time} i~\textit{Newsweeku} pojawiły się artykuły, w~których stwierdzono, że protestujący przeciwko globalizacji przez cały czas mieli rację. W jednym lub dwóch z~tych artykułów redakcyjnych użyto nawet sformułowania ,,dostosowanie strukturalne'', wydaje się, że wielu z~nich podniosło swoje argumenty bezpośrednio z~wypowiedzi aktywistów, których te same gazety odmówiły publikacji. Najwyraźniej te wiadomości odniosły skutek. Jednak jedna rzecz pozostała niezmienna: w~żadnym momencie nie cytowano aktywistów ani aktywistów intelektualistów, ani nie pozwolono na użycie prasy głównego nurtu jako środka do samodzielnego przedstawiania tych punktów.

 Wydaje się, że to kwestia polityki. Menedżerowie mediów wydają się postrzegać swoją rolę w~takich sprawach, jak referowanie debaty publicznej wśród uprawnionych głosów. Z tej perspektywy każdy, kto próbuje zakłócić porządek publiczny jako sposób na przekazanie swojego przesłania w~gazecie, jest z~definicji nieuprawniony. Dokładnie z~tego samego powodu (by wziąć przykład, który jest mi osobiście znany), jeśli ktoś z~publiczności wtrąca się do przemówienia wiceprezydenta Cheneya, krzycząc ,,Hej, Cheney, ile do tej pory zarobiłeś na wojnie w~Iraku ?'', zanim zostanie pchnięty na ziemię przez tajne służby, gazety doniosą, że heckler przerwał mowę Cheneya ,,antywojennym hasłem'', a następnie opisze wynikłą bójkę, ale nigdy nie powielą rzeczywistych słów hecklera. Zgodnie z~panującą logiką redakcyjną, odtworzenie jego rzeczywistych słów oznaczałoby umożliwienie hecklerowi,,przejęcie'' mediów; byłoby to moralnie równoznaczne z~wydrukowaniem wiadomości wysłanej przez terrorystę i~spowodowałoby, że gazeta byłaby częściowo odpowiedzialna, gdyby ktoś działał w~podobny sposób, zakłócając autoryzowane wydarzenie, aby w~przyszłości uzyskać wiadomość w~gazecie\footnote{Miałem napisać ,,złamanie prawa'', a nie ,,zakłócanie legalnego wydarzenia'', ale jak się okazało, nie złamano żadnego prawa. W rzeczywistości nie jest nielegalne zastraszanie wiceprezydenta, nawet podczas konwencji republikanów. W końcu heckler, którego po raz pierwszy ubrano w~pomarańczowy kombinezon i~wrzucono do celi z~notorycznym członkiem Al-Kaidy, musiał zostać zwolniony bez zarzutów, chociaż powiedziano mi, że później w~Kongresie podjęto próbę stworzenia nowego prawa z~językiem zaprojektowanym specjalnie w~celu uczynienia takich zachowań nielegalnymi w~przyszłości.}.

 Można tutaj, i~działacze regularnie to robią, krytykować wszystkie podstawowe założenia: Co stanowi porządek? Czy panujący system polityczno-gospodarczy nie gwarantuje, że niewielka elita będzie żyć stosunkowo bezpiecznie, przewidywalnie, a ogromna większość ludzi będzie żyła w~niepewności i~terrorze? Co stanowi zakłócenie? Czy inwazja na Irak i~śmierć stu tysięcy Irakijczyków nie stanowią większego zakłócenia niż zablokowanie ulicy przed ośrodkiem rekrutacyjnym? Ale kiedy zaakceptuje się wrodzoną legitymizację dominujących instytucji, trudno dojść do innego wniosku. Jeśli nic innego, ta logika pomaga wyjaśnić, dlaczego aktywiści przed ważnym szczytem mogą wysyłać tysiące komunikatów prasowych do niemal każdego większego serwisu informacyjnego, zawierających starannie opracowane dokumenty przedstawiające stanowisko, i~ciągle przeczytać artykuły wstępne w~gazecie skarżące się, że jest niejasne, w~co wierzą i~o co walczą aktywiści.

 Należy jeszcze raz podkreślić, że ta struktura władzy nie ogranicza się do aparatu władzy konstytucyjnej. Reporterzy lub producenci telewizyjni często mówią tak, jakby tak było: podkreślając na przykład, że negocjatorzy handlowi są wyznaczani przez wybranych przywódców państw i~że nikt nigdy nie wybierał protestujących. Oczywiście ten sam zarzut można by podnieść w~stosunku do korporacji takich jak Monsanto -- nikt ich też nie wybrał -- i~żaden program telewizyjny nie sprzeciwiłby się co do zasady rozpowszechnianiu materiałów z~komunikatu prasowego Monsanto lub relacjonowaniu jej wyreżyserowanych wydarzeń reklamowych. Ale jest też ku temu powód. Sieci telewizyjne i~magazyny informacyjne z~definicji nie mogą postrzegać firm nastawionych na zysk jedynie jako uprawnionych głosów w~sprawach dotyczących interesu publicznego, ponieważ sieci telewizyjne i~czasopisma informacyjne same w~sobie są firmami nastawionymi na zysk. Postrzeganie rzeczy inaczej zniszczyłoby ich własną legitymację. Mamy więc do czynienia z~pewnego rodzaju kręgiem wzajemnej legitymizacji, obejmującym instytucje rządowe, świat korporacji i~główne grupy interesu (od ACLU po Heritage Foundation), które angażują się ze sobą na różne sposoby. ,,Demokracja'' dla mediów składa się z~debat politycznych między takimi legalnymi instytucjami, prowadzonych w~związku ze zmieniającymi się opiniami zmasowanej ,,publiczności'', która zasadniczo odpowiada publiczności dla samych mediów.

 Wiele z~tego, co piszę, jest odmianą standardowej krytyki aktywistów tego, co nazywają ,,mediami korporacyjnymi''. Każdy, kto wykonał dużo pracy prasowej dla grup aktywistów, doskonale zdaje sobie sprawę z~tego, jak to wszystko działa w~praktyce. Oświadczenia instytucji uznanych za uprawnione są zawsze traktowane inaczej niż te, które nie są ,,uprawnione''. W trakcie akcji oznacza to, że jeśli policja na poziomie konferencji prasowej postawiła zarzuty aktywistom, może to zostać natychmiast odtworzone przez reporterów telewizyjnych i~potraktowane przynajmniej tymczasowo jako prawdziwe; jeśli działacze na poziomie konferencji prasowej oskarżają policję, jest mało prawdopodobne, aby w~ogóle się to powtórzyło, ze względu na brak weryfikacji naocznych świadków ze strony jednego z~reporterów. Jak można sobie wyobrazić, daje to policji ogromną przewagę taktyczną, jedną z~których, jak zobaczymy,wykorzystują każdą możliwą przewagę.

 Jak zauważy każdy, kto zajmował się mediami, są ku temu bardzo dobre praktyczne powody. Reporterzy przypisani do osłony protestów to zazwyczaj reporterzy policyjni. Nie mogli dalej wykonywać swojej pracy bez dobrej woli miejscowej policji. Ich praca nie jest jednak w~żaden sposób zależna od dobrej woli społeczności aktywistów. To prawda; ale w~najlepszym razie jest to wyjaśnienie częściowe. Nie sposób myśleć o roli policji w~kulturze amerykańskiej bez wchodzenia w~obszar mitu. W momencie, gdy ktoś porusza ten temat, natychmiast zostaje wciągnięty w~niekończący się labirynt mitologicznych wyobrażeń i~ustalonych ram narracyjnych. Zwłaszcza dla dziennikarzy istnieją pewne historie, które opowiada się o policji, które są bardzo łatwe do opowiedzenia; i~bardzo trudno powiedzieć innym.

\medskip
\noindent SŁOWO O POLICJI
\medskip

 Niemal każde socjologiczne badanie policji we współczesnym świecie musi zacząć się od starannego wyprowadzenia czytelnika z~błędnej idei, że policja istnieje przede wszystkim po to, by zwalczać przestępczość. To, jak wyjaśniają, jest mit.

 Egzekwowanie prawa karnego to coś, co większość funkcjonariuszy policji robi z~częstotliwością, która znajduje się gdzieś pomiędzy praktycznie nigdy a bardzo rzadko. Przytłaczająca większość wezwań o pomoc policji ma charakter ,,służby'', a nie przestępstwa: w~przeciętnym roku tylko 15 do 20 procent wszystkich wezwań na policję dotyczy przestępstwa, a to, co jest początkowo zgłaszane przez opinię publiczną jako przestępstwo, nie jest często uznawane za przestępstwo przez funkcjonariusza policji. Badania wykazały, że mniej niż jedna trzecia czasu spędzonego na służbie to praca związana z~przestępczością; że około osiem na dziesięć incydentów, którymi zajmują się patrole różnych wydziałów policji, sama policja uważa za sprawy inne niż kryminalne; że odsetek wysiłków policji poświęconych tradycyjnym sprawom prawa karnego prawdopodobnie nie przekracza 10 procent; że zaledwie 6\% czasu funkcjonariusza patrolującego poświęca się na incydenty ostatecznie określone jako ,,przestępcze'', oraz że tylko bardzo niewielka liczba przestępstw jest wykrywana przez samą policję. Co więcej, przez większość czasu policja nie stosuje prawa karnego do przywrócenia porządku. W USA policjanci dokonują średnio jednego aresztowania co dwa tygodnie; jedno z~badań wykazało, że spośród 156 funkcjonariuszy przydzielonych do obszaru o wysokiej przestępczości w~Nowym Jorku, 40 procent nie dokonało ani jednego aresztowania za przestępstwo w~ciągu roku. W Kanadzie funkcjonariusz policji średnio odnotowuje jedno przestępstwo z~oskarżenia publicznego tygodniowo, co trzy tygodnie dokonuje jednego aresztowania z~oskarżenia publicznego i~co dziewięć miesięcy uzyskuje jeden wyrok skazujący z~oskarżenia publicznego (Neocleous 2000:93; zob. także Bittner 1990; Waddington 1999)\footnote{Większość socjologów policyjnych zaprzecza nawet, że obecność funkcjonariuszy na ulicach lub liczba patroli policyjnych ma znaczący wpływ na wskaźniki przestępczości. Wydaje mi się to ostatecznie nieco nieprawdopodobne, ale z~pewnością prawdą jest, że policja jest częściej rozmieszczona w~bogatych dzielnicach, w~których na początku jest mniej przestępczości. W każdym razie, w~neoliberalnej Ameryce, codzienne bezpieczeństwo jest coraz częściej zapewniane przez prywatne agencje ochrony, które nawet nie udają, że zapewniają wszystkim równą ochronę.}.

 Więc co właściwie robi policja? Jeśli spojrzeć tylko na to, jak policja spędza większość swojego czasu, można tylko stwierdzić, że mamy do czynienia z~grupą uzbrojonych administratorów rządowych niższego szczebla, wyszkolonych w~naukowym stosowaniu siły fizycznej lub groźbie użycia siły fizycznej pomoc w~rozwiązywaniu problemów administracyjnych. Policja to biurokraci z~bronią\footnote{Historia sił policyjnych pokazuje, jak bardzo jest to prawdą. Policja XVIII wieku zajmowała się głównie regulowaniem handlu; w~dziewiętnastym wieku, zajmując się głównie regulowaniem życia wśród biednych, radzenie sobie z~kryzysami rodzinnymi, obecnie w~dużej mierze zdegradowanymi do pracowników socjalnych i~tłumieniem alternatyw dla pracy najemnej.}. Są aktywną twarzą państwowego monopolu na użycie przemocy. Stąd przytoczona wcześniej definicja Bittnera. Nawet jeśli policja ma do czynienia z~problemami, które wydają się najbardziej odległy od sprawy kryminalnej -- na przykład włamaniem się do mieszkania w~celu sprawdzenia starszego mieszkańca, którego nikt nie widział od kilku dni, wygadywaniem pijaków z~barów, zajmowaniem się zagubionymi dziećmi -- to wciąż zajmuje się problemami, które mogą wymagać ,,nienegocjowanych rozwiązań popartych potencjalnym użyciem siły''.

 Z drugiej strony ważne są mity. Popularne założenie, że policja jest po to, by walczyć z~przestępczością, zwłaszcza z~brutalną przemocą -- założenie bez końca wzmacniane przez filmy, programy telewizyjne i~reportaże -- ma nieskończenie głębokie skutki. 

 Przypomnijmy tutaj, co powiedziałem w~rozdziale 6 o ideologicznych skutkach regulacji rządowych: w~jaki sposób przepisy dotyczące obiektów takich jak samochody i~budynki są egzekwowane przez groźbę przemocy, a przemoc staje się skutecznie niewidoczna, a zatem wywiera skutek tych przepisów, które wydają się niemal częścią materialności lub ,,rzeczywistości'' samego obiektu. Tutaj spotykamy się, jak sądzę, z~innym aspektem tego samego zjawiska. Policja oczywiście stosuje przemoc, aby aresztować, a nawet okazjonalnie walczyć z~brutalnymi przestępcami. Ale są równie zdolne do stosowania przemocy w~egzekwowaniu przepisów, które nie są technicznie sprawami karnymi w~żadnym znaczeniu tego słowa. Oczywistymi przykładami są przepisy ruchu drogowego, przepisy dotyczące otwartych kontenerów, skargi na hałas i~nielicencjonowany handel. Policja jest również dostępna, jeśli jest to wymagane, aby wesprzeć egzekwowanie innych przepisów, które zwykle nie są uważane za część ich kompetencji, takich jak, powiedzmy, kodeksy przeciwpożarowe lub przepisy dotyczące rozmiaru i~umieszczania reklam i~innych znaków poza domem. Normalnie nigdy o tym nie myślimy. Jest to, jak zauważono wcześniej, jeden z~aspektów państwa, który akcja bezpośrednia ujawnia. Technicznie rzecz biorąc, jeśli ktoś naruszy przepisy przeciwpożarowe, policja ma prawo wejść i~użyć wszelkiej niezbędnej siły fizycznej, aby ewakuować budynek, nawet wbrew woli mieszkańców; to może, a w~przypadku lokatorów często wiąże się z~wyważaniem drzwi z~wyciągniętą bronią i~biciem pasażerów pałkami po głowach. To samo technicznie dotyczy naruszeń kodeksu zdrowia, naruszeń kodeksu podatkowego, lub przepisów dotyczących rozdawania ulotek na rogach ulic lub wielkości i~rozmieszczenia znaków. Różnica polega oczywiście na tym, że mało kto jest skłonny zaryzykować, by stać się obiektem oficjalnie usankcjonowanej przemocy, otwarcie odrzucając nakaz usunięcia, powiedzmy, zbyt dużego ogłoszenia lub próbując uniemożliwić pracownikom miasta samodzielne ich zdjęcie. Dlatego łatwo zapomnieć, że jest to również ostateczna sankcja dla takich regulacji.

 Regulacje wtapiają się zatem w~prawa. W rezultacie, ponieważ zakłada się, że policja egzekwuje ,,prawo'', wszelkie łamanie jej rozkazów jest postrzegane jako zasadniczo kryminalne, a zatem w~sposób dorozumiany, brutalne, co oznacza, że  jeśli policja użyje siły, nawet przeciwko właścicielowi domu, który nie zdecydował się usunąć znak, zakłada się, że jest to uzasadniona przeciwprzemoc. 

 Wszystko to może wydawać się nieco hipotetyczne i~naciągane, ale jak widzieliśmy w~poprzednim rozdziale, prawie dokładnie tak się dzieje podczas akcji bezpośrednich. Aktywiści zwykle nie są winni niczego poważniejszego niż naruszenia pewnych kodeksów czy rozporządzeń, na przykład zakazu chodzenia czy stania na ulicy. To nie są sprawy karne. Jednak gdy odmawiają wykonania poleceń policji, są faktycznie atakowani i~często kończą z~roztrzaskanymi głowami o ściany lub zakuci w~kajdany na torturujących pozycjach: ponieważ z~tego samego powodu, o którym policja wie, aktywiści nigdy nie będą ścigani w~sądzie karnym, stąd istnieje kilka ograniczeń dla zachowania policji. Tak więc zamiast legalnego stosowania siły w~celu egzekwowania prawa, w~rzeczywistości mamy do czynienia z~w dużej mierze nieuregulowanym użyciem przemocy w~celu wsparcia przepisów, a nawet po prostu stłumienia jakikolwiek publicznego sprzeciw wobec prawa policji do brutalnego ich egzekwowania. Nie jest to jednak sposób przedstawiania spraw w~mediach.

 W rzeczywistości, nawet moje własne użycie terminu ,,przemoc'' w~odniesieniu do zachowania policji w~powyższych akapitach może wydać się wielu czytelnikom dziwnie przenikliwe. Mogą być zaskoczeni, wiedząc, że robiąc to, faktycznie używam tego słowa w~najwęższym znaczeniu: to znaczy w~tym, co filozofowie czasami nazywają ,,minimalnym'' lub ,,ograniczającym'' znaczeniem tego terminu, gdzie ,,przemoc'' odnosi się do zasadniczo do szkodliwych czynów popełnionych przez jedną osobę przeciwko drugiej. Pozwolę sobie skorzystać z~prac australijskiego filozofa Tony'ego Coady'ego (1986), który wyróżnia trzy szerokie tradycje definiowania tego terminu, z~których każda ma swoje własne implikacje polityczne. Poniżej przedstawiam moją nieco uproszczoną wersję jego typologii: 
\medskip

 \textit{Definicje restrykcyjne}: np. ,,Przemoc to celowe zadawanie bólu lub zranienia innych bez ich zgody''. Często mówi się, że jest to wersja preferowana przez politycznych liberałów, chociaż Coady twierdzi, że jest to najbliższa neutralnej definicji. 

 \textit{Szerokie definicje:}np. ,,Przemoc to celowe zadawanie bólu lub zranienia innych bez ich zgody lub grożenie, że to zrobi''. Często mówi się, że jest to wersja preferowana przez radykałów politycznych. 

 \textit{Definicje legitymistyczne}: np. ,,Przemoc to krzywda lub szkoda wyrządzona osobom lub mieniu, która nie jest autoryzowana przez prawidłowo ustanowione władze''. Często mówi się, że jest to wersja preferowana przez politycznych konserwatystów. 

\medskip

 Łatwo zrozumieć, dlaczego otrzymują oni takie polityczne atrybucje. Nr 1 jest, jak twierdzi Coady, jak najbliższa neutralnej definicji. Moja wersja szerszej definicji (nr 2) sama w~sobie nie wydaje się aż tak radykalna -- w~końcu, jeśli wymierzysz w~kogoś broń i~zażądasz wszystkich jego pieniędzy, zwykle uzna się, że popełniłeś brutalne przestępstwo, nawet jeśli nikogo właściwie nie postrzelisz. Ale ma to bardzo radykalne implikacje, ponieważ jeśli stosujesz go systematycznie, musiałbyś dojść do wniosku, że samo państwo jest zasadniczo narzędziem przemocy. Nr 3, legitymistyczna definicja, z~drugiej strony, w~rzeczywistości uniemożliwia państwu zachowanie przemocowe (chyba że  państwo, o którym mowa, jest uważane za niewłaściwie ukonstytuowane). Jest to oczywiście definicja preferowana przez konserwatystów, ale jest ona również, jak narzekają aktywiści od co najmniej lat 60., powszechnie stosowane przez amerykańskie media korporacyjne. Policji działającej na rozkaz przełożonych nie można określić mianem ,,brutalnej'', nawet jeśli łamie głowy czy otwiera ogień z~ostrą amunicją. Z drugiej strony protestujących można określić jako ,,brutalnych'', nawet jeśli dosłownie jeden na tysiąc rzuca kamieniem lub wybija okno. Funkcjonariusz policji, którego zachowanie można określić mianem ,,brutalnego'', to ten, który został już zdefiniowany jako ,,nieuczciwy gliniarz'' -- to znaczy ten, który działa poza właściwym łańcuchem dowodzenia lub porządkiem prawnym. 

 Jednak ,,definicja legitymistyczna'' jest nie tylko faworyzowana przez dziennikarzy i~konserwatystów społecznych. Jest ona preferowana również przez antropologów (np. Riches 1986). Może się to wydawać anomalne, a nawet zaskakujące, ponieważ antropologia jako dyscyplina zawsze uważa się za politycznie postępową, ale wydaje się, że to jeden z~tych dziwnych paradoksów, które często podsyca relatywizm kulturowy. W końcu trudno sobie wyobrazić, jak prawdziwy relatywista mógłby dojść do innego wniosku. Jeśli ktoś uważa, że  ,,przemoc'' (lub jakikolwiek inny termin w~tej sprawie) jest po prostu tym, czym określa ją kultura lub społeczeństwo\footnote{Zauważmy, że zakłada się, że termin ten jest w~pewnym sensie uniwersalny: każda kultura ma jakiś odpowiednik pojęcia wyrażonego w~języku angielskim terminem ,,przemoc'', ale jednocześnie, że nie istnieje żadna uniwersalna prawda leżąca za terminem. Wydawałoby się to niezgodne z~faktami.}, wtedy zakłada się, że istnieją jednolite byty, które można nazwać ,,kulturami'' lub ,,społeczeństwami'', władze, które mogą wypowiadać się w~ich imieniu w~takich sprawach, oraz jakiś dość niezawodny system, dzięki któremu zewnętrzny obserwator może je zidentyfikować. Innymi słowy, jedyną rzeczą, jaką relatywista musi uniwersalizować, są struktury władzy. Wychodząc z~takiego stanowiska, trudno byłoby nie wnioskować, że ,,przemoc'' dla danego społeczeństwa powinna być zdefiniowana jako wszelkie formy krzywdzenia lub krzywdy, które władze te uważają za nieuprawnione. 

 Relatywiści nie tylko mają tendencję do przyjmowania autorytarnej definicji przemocy; władze, przynajmniej w~tej kulturze, są zdolne do niezwykłego relatywizmu w~takich sprawach. Po raz pierwszy zacząłem to zauważać w~ciągu ośmiu godzin, które spędziłem przykuty w~autobusie aresztowym w~Waszyngtonie podczas posiedzeń IMF w~2002 roku, wraz z~czterdziestoma innymi aktywistami spośród kilkuset, którzy zostali porwani podczas masowego aresztowania w~,,zielonej strefie'' na Park Pershinga. Podczas naszego pobytu spędziliśmy sporo czasu w~nieco niechętnym dialogu z~porucznikiem policji, który szybko stał się znany wśród więźniów jako ,,oficer Mindfuck'' -- człowiek, który najwyraźniej wsiadł do autobusu po prostu dla rozrywki, debatując z~nami. Prawie w~każdym temacie przyjął to samo podejście: próbując przekonać nas, że nie zajmujemy wystarczająco relatywistycznego stanowiska. Wydawało się, że to też nie była sztuczka, przynajmniej kiedy wyszedł z~autobusu (ku naszej wielkiej zbiorowej uldze), pierwszą rzeczą, jaką zauważył przy swoich kolegom na zewnątrz, było ,,problem z~tymi facetami polega na tym, że nie rozumieją, że każde pytanie ma więcej niż jedną stronę''. W świecie, w~którym nie ma absolutnie żadnego sposobu, aby dowiedzieć się, czy polityka IMF jest korzystna, czy szkodliwa, nie ma podstaw do zajmowania stanowiska opartego na zasadach: sensowne jest, aby można było stwierdzić, że przestrzeganie zasad, bez względu na to, jakie one są, jest jedyny możliwy moralny kierunek działania. A potem zacząłem zauważać, że ilekroć policja ustanawiała prawo, traktowała obiekcje dokładnie w~ten sam sposób.

\bigskip

\noindent \texttt{Scena: kilka dni po 11 września 2001}\\ Nowy Jork. Kilkutysięczny marsz pokojowy zmierzający w~kierunku Times Square kończy się na Czterdziestej Pierwszej Ulicy. Próbuję ich dogonić.
\medskip

\noindent Policjant: Idź dalej, musisz opuścić teren.

\noindent Ja: Ale jestem marszałkiem, należę do osób, które powinny dbać o porządek w~tłumie.

\noindent Policjant: Trochę na to za późno, prawda?

\noindent Ja: Cóż, jak spodziewasz się, że ludzie będą zachowywać się w~sposób uporządkowany, jeśli nagle wejdziesz i~otoczysz ich barykadami i~nie dasz im żadnej możliwości wyjścia?

\noindent Policjant: Jasne, po prostu znasz jedyną możliwą poprawną odpowiedź na każde pytanie, prawda?

 Wielokrotnie spotykałem się z~tego rodzaju odpowiedziami: systematyczny agnostycyzm dotyczący ostatecznej moralnej prawdy sytuacji, wynikająca z~niej wiara w~możliwość jej zdefiniowania za pomocą arbitralnego dyktatu (różnica między ,,porządnym'' a ,,nieuporządkowanym'' tłumem ma zasadnicze znaczenie dla procedur policyjnych) i~wreszcie nieuniknioną sugestię, że każdy, kto się nie zgadza, jest oczywiście jakimś ideologiem lub fanatykiem, który nie jest w~stanie przyjąć wystarczająco relatywistycznego punktu widzenia.

 W tym miejscu można by zauważyć, jak bardzo różni się to od pozornego relatywizmu logiki aktywistycznej, która zamiast tego podkreśla, że  niemożność całkowitego pogodzenia różnych punktów widzenia oznacza, że  \textit{nie }należy narzucać autorytatywnych definicji. To jednak byłoby odejście dość daleko. Dla celów obecnych chciałbym tylko podkreślić, że zarówno policja, jak i~media wydają się podzielać szereg założeń dotyczących natury prawowitości, porządku i~przemocy, które nie mogą nie wywołać efektu opisanego na samym początku ten rozdział: gdzie zwykli obywatele, którzy przypadkiem wkradają się na scenę akcji bez uprzedzenia co do tego, co się dzieje (i stąd prawdopodobnie posługują się pierwszą, stosunkowo neutralną definicją przemocy), kończą z~odwrotnym wrażeniem niż ci, którzy widzieli to samo wydarzenie w~wiadomościach telewizyjnych. W rzeczywistości sprawia, że  klasyczne podejście Gandhiego do obywatelskiego nieposłuszeństwa jest prawie całkowicie nieskuteczne. Pozwolę sobie podać przykład, zanim przejdę do omówienia poetyki konstruowania informacji prasowych.

\bigskip
\noindent O NIESKUTECZNOŚCI TAKTYK GANDHIEGO WE WSPÓŁCZESNYCH STANACH ZJEDNOCZONYCH
\bigskip

\begin{flushright}
\texttt{Jedną rzeczą, która uderzyła mnie w~gliniarzach gazujących łzawiącymi i~strzelających do nas (oprócz gazu i~kul), było to, że biali ludzie w~całych Stanach Zjednoczonych oglądali to w~telewizji. Większość z~tych młodych ludzi to ich pierdolone dzieci. Muszą się przekonać, że gliniarze zrobią to KAŻDEMU (niezależnie od tego, czy będą się opierać, czy też przeciwstawić się mantrze kapitalizmu jako boga, którą karmimy od urodzenia). Muszą to dostać! Nie zniosą tego! (Starsi działacze z~lat sześćdziesiątych mówią mi, że będą; robili to wtedy i~znowu będą).}

-- Mary Margaret Fondriest (2000)

\end{flushright}

 Liberalni krytycy, którzy twierdzą, że anarchiści protestujący przeciwko WTO lub innym neoliberalnym instytucjom zrobiliby lepiej, gdyby działali w~duchu Gandhiego czy Martina Luthera Kinga, w~większości nie zdają sobie sprawy, że w~ich pierwszym odruchu tak właśnie było.

 Istotnym założeniem leżącym u podstaw takiej taktyki, przypomnijmy, jest to, że skoro każdy niesprawiedliwy porządek społeczny musi opierać się na groźbie przemocy, to powinno być możliwe niejako nazwanie blefu systemu: demaskowanie niesprawiedliwości przez wydobycie przemocy jest nieodłącznego elementu systemu, ujawniając gotowość państwa do łamania głów nawet skrupulatnie pryncypialnym, skrupulatnie pokojowo nastawionym obywatelom, którzy odmawiają posłuszeństwa niesprawiedliwemu prawu. Wróćmy do słynnego obrazu, który Lynn przywołała w~rozdziale 2, przedstawiającego tysiące indyjskich mężczyzn, którzy zamierzają robić sól wbrew brytyjskim prawom, które uczyniły z~tego monopol rządowy, spokojnie maszerując pod szeregi policjantów, którzy następnie zaczynają ich bić w~krwawe stosy, dopóki nie zostali wyniesieni, a ich miejsce zajęły nowe osoby. Jest swego rodzaju uosobieniem tego rodzaju procedury (a także klasycznym przykładem akcji bezpośredniej). Zachowanie się w~ten sposób jest apelem moralnym: najbardziej bezpośrednio do mężczyzn, którzy faktycznie dzierżyli broń, ale bardziej efektywnie do obywateli brytyjskich w~imieniu, których twierdzili, że to robią. Przesłanie: jeśli takie prawo można utrzymać tylko przez łamanie kości oczywiście przyzwoitym istotom ludzkim, to nie jest to prawo sprawiedliwe i~nie powinno być przestrzegane.

 Działania bezpośrednie przeciwko instytucjom neoliberalnym, jak już wspomniałem, prawie zawsze działają w~ramach zasadniczo Gandhiego, ponieważ mają na celu ujawnienie niesprawiedliwości szerszej publiczności, która, jak się zakłada, byłaby oburzona, gdyby naprawdę o nich wiedziała. Problem z~tym podejściem polega oczywiście na tym, że informacje te muszą zostać ujawnione opinii publicznej. To z~kolei oznacza, że  część pracy aktywistów musi być wykonywana nie przez samych aktywistów, ale przez tych dziennikarzy, którzy będą relacjonować wydarzenia. Gandhi, znacząco, kultywował swojego osobistego (brytyjskiego) dziennikarza.

 Aby zrozumieć wściekłość i~pasję rozpętaną przez Czarny Blok w~Seattle, trzeba najpierw zrozumieć, że większość aktywistów, którzy uczestniczyli, to pierwotnie działacze ekologiczni, a wielu z~nich weterani walki o zachowanie starych lasów w~północno-zachodnim Pacyfiku w~latach 90. XX wieku. Wiele z~nich zostało zorganizowanych przez Earth First!, anarchistyczny sojusz aktywistów leśnych, którzy wydają się w~dużej mierze odpowiedzialni za zaangażowanie dużej liczby nastoletnich miejskich fanów hardcore'u i~skaterów punków w~leśne walki. Krytykowani za odmowę potępienia taktyk sabotażowych, takich jak wbijanie gwoździ w~drzewa w~latach 80., w~latach 90. przeszli na stanowisko czystego Gandhijskiego obywatelskiego nieposłuszeństwa\footnote{Nie ma dowodów, że aktywiści Earth First! byli zaangażowani w~,,nabijanie'' drzew - umieszczanie metalowych kolców w~pniach drzew, które mogłyby spowodować pękanie pił i~prawdopodobnie zranienie lub zabicie ich użytkowników. Jednak w~latach 80. niektóre drzewa zostały nabite. Ci, którzy nabijali drzewa, nigdy nie zrobili tego bez wcześniejszego ostrzeżenia i~nie nastąpiła żadna śmierć, ale praktyka była potencjalnie śmiertelna i~przez kilka lat Ziemia First! odmawiała ani wyrzeczenia się, ani tolerowania tego. W latach 90. EF! formalnie ogłosili, że doszli do konsensusu, aby publicznie potępić tę praktykę.}. Ich najsłynniejszą taktyką było siedzenie na drzewie: poszczególni aktywiści zasadniczo osiedlali się na poszczególnych drzewach i~pozostawali tam czasami przez wiele miesięcy, uniemożliwiając ścinanie drzewa bez ich zabicia. Niektórzy z~tych opiekunów drzewa zdobyli sprawę rozgłosu na arenie międzynarodowej. Aktywiści na miejscu wspierali ich różnymi rodzajami blokad przed sprzętem do pozyskiwania drewna, zaczynając od prostych łańcuchów i~stopniowo przechodząc do zamków rowerowych, a następnie wyrafinowanych metalowych zamków. Wszystko to było całkowicie zgodne z~tradycją Gandhiego, polegającą na całkowitym oddaniu swojego losu w~ręce przeciwników. Na przykład nikt, kto nie złożył przysięgi niestosowania przemocy i~nie przeszedł intensywnego szkolenia w~zakresie niestosowania przemocy, nie mógł brać udziału w~blokadach.

 17 września 1998 roku w~Headwaters Forest, starym lesie sekwojowym w~hrabstwie Humboldt w~Kalifornii, drwal zatrudniony w~Pacific Lumber zdecydował się przetestować te założenia. Posłał spadające drzewo bezpośrednio na opiekuna drzewa o imieniu David Chain, zabijając go natychmiast. Początkową reakcją mediów było potraktowanie sprawy jako nieszczęśliwego wypadku. Oto co napisał ówczesny działacz leśny:

 Po raz pierwszy usłyszałem o morderstwie Davida Chaina w~Jefferson Public Radio\ldots  Po pierwsze, błędnie ogłosili, że Chain został zabity nie przez drzewo ścięte drwalem, ale przez inne drzewo przewrócone w~efekcie domina. Po drugie, po krótkim i~poruszającym oświadczeniu zapłakanego Earth First!, dziennikarz zadał trzy podstawowe pytania. Były to w~zasadzie (nie mam bezpośrednich cytatów, ponieważ prowadziłem samochód, więc nie mogłem ich zapisać): 1) czy członkowie Earth First! mają świadomość, że ich działania są niebezpieczne, prawda? i~czy nie jest prawdą, że ten działacz był zaangażowany w~szczególnie niebezpieczną formę aktywizmu? 2) Oczywiście, że drwal nie zrobił tego celowo, prawda? 3) W następstwie tej śmierci, jak Earth First! zamierza zmienić swoją taktykę, żeby to się więcej nie powtórzyło?

 Prawie zjechałem z~drogi\footnote{,,Esej o śmierci Davida Chaina'' opublikowany w~Sentient Times w~październiku 1998 roku Znaleziony pod adresem \url{http://www.derrickjensen.org/chain}, dostęp 21 lipca 2004 roku}.

 W rzeczywistości aktywiści wkrótce wyprodukowali taśmę wideo, którą zrobili godzinę przed incydentem, na której wspomniany drwal (niejaki A.E. Ammon) wykrzykiwał przekleństwa na aktywistów, w~tym Chaina, i~ostrzegał ich, aby się usunęli, bo inaczej ,,upuści drzewo'' na jednego z~nich. Jednak nawet później media potraktowały firmę zajmującą się wyrębem, twierdząc, że zdarzenie to był wypadek, jako najbardziej prawdopodobne wyjaśnienie, a lokalny Departament Szeryfa nie tylko odmówił wszczęcia dochodzenia kryminalnego, ale nawet pozwolił Pacific Lumber na kontynuowanie wyrębu na stanowisku. To zniszczyłoby wszelkie dowody, które mogłyby zostać wykorzystane w~przyszłym śledztwie.

 W odpowiedzi aktywiści Earth First! ustanowili blokadę, aby powstrzymać Pacific Lumber. W końcu szeryfowie mieli przestępstwo, z~którym postanowili się uporać: od czterdziestu do sześćdziesięciu oficerów przeprowadziło nalot o świcie w~stylu wojskowym na śpiących protestujących. Aktywiści, którzy wstali, zostali zmuszeni do położenia na ziemi. Ostrzeżeni dźwiękami poniżej, niektórzy aktywiści przy blokadach bliżej miejsca morderstwa byli w~stanie przypiąć się do sprzętu do wyrębu. Jedna z~młodych kobiet, Noel, wykrzyczała ,,gaz pieprzowy'' wkrótce po tym, jak dotarły do  niej władze. Została zablokowana wysoko na wysięgniku. Wysięgnik został obniżony, a następnie oficerowie odchylili jej głowę i~oblali jej twarz płynnym gazem pieprzowym z~kubka. W ten sam sposób oblano drugą kobietę.

 Tej nocy protestujący zreformowali swoje barykady, a następnego ranka szeryfowie wrócili. Tym razem protestujący byli gotowi i~się zamknęli. Szeryf nie marnował czasu, natychmiast wyciągając gaz pieprzowy. Funkcjonariusze rozpięli dużą plandekę, próbując powstrzymać obserwatorów przed oglądaniem ich działań, ale plandeka nie przeszkadzała ludziom słyszeć krzyków młodych kobiet oblewanych koncentratem pieprzowym. Policja przyłożyła nasączoną pieprzem gazę aktywistce Carrie ,,Liz'' McKee. Kiedy odmówiła otwarcia blokady, gaza została wykręcona bezpośrednio do jej oczu. Kiedy nadal odmawiała, policja ponownie zastosowała koncentrat. Pieprz zastosowano po raz trzeci i~zaczęła wymiotować z~bólu. Mimo to odmówiła poddania się, a policja odcięła ją z~blokady\footnote{Tamże.}.

 Na szczególną uwagę zasługuje użycie gazu pieprzowego na aktywistach w~zamknięciu. Była to innowacja wydziału szeryfa hrabstwa Humboldt rok wcześniej, kiedy jesienią 1997 roku stanęli w~obliczu serii wyrafinowanych blokad wymierzonych przeciwko Pacific Lumber i~jego politycznym sojusznikom. Po raz pierwszy został przetestowany 25 września 1997 roku na siedmiu aktywistach zamkniętych w~kręgu w~holu biur Pacific Lumber Company w~Szkocji. Policja używała patyczków higienicznych do wlewania pieprzu (znanego również jako Oleoresin Capsicum lub OC) bezpośrednio w~oczy, aby zmusić ich do uwolnienia rąk z~łańcuchów bez konieczności przecinania metalowej rurki. Wysiłek w~dużej mierze się nie powiódł (tylko dwóch z~nich odpuściło), ale powtórzyło się to tydzień później, 3 października, na dwóch aktywistkach, którzy przykuli się do buldożera w~pobliskim Bear Creek, a potem ponownie 16 października, kiedy cztery młode kobiety przypięły się do pnia drzewa w~biurze lokalnego kongresmena Franka Riggs, zagorzałego zwolennika Pacific Lumber, w~Eureka. W tym drugim przypadku cała sprawa została uchwycona na wideo przez lokalnych reporterów telewizyjnych, a w~telewizji pokazano sceny szesnastoletniej dziewczyny o imieniu Maya Portugal, z~całkowicie unieruchomionymi ramionami, błagającej o litość, a następnie krzyczącej w~agonii, gdy szeryfowie zastosowali pieprz do gałek ocznych, co wywołało coś w~rodzaju narodowego skandalu.

 Ale tylko do pewnego stopnia.

 W rzeczywistości była to nawet bardziej niż sprawa Davida Chaina, test taktyki Gandhiego, ponieważ działalność była wyraźnie autoryzowana przez właściwie ukonstytuowane władze rządowe\footnote{Została zatwierdzona przez szeryfa, a zastępca naczelny Gary Philip przeprowadził rozległe badania nad możliwymi konsekwencjami prawnymi.}. Pasuje również do każdej zwykłej definicji tortur. Zastosowanie gazu pieprzowego na najbardziej wrażliwe tkanki na ciele pacyfistki służyło tylko jednemu celowi: zadać jak najwięcej bólu w~sposób, który nie spowodowałby również poważnych obrażeń fizycznych, tak aby zmusić ją do uwolnienia łańcuchów\footnote{Nie oznacza to, że gaz pieprzowy nie może powodować długotrwałych obrażeń fizycznych. Spring Lundberg, która sama była ofiarą jednego z~tych incydentów, szacuje, że ,,spray OC jest powiązany jako możliwy czynnik w~śmierci ponad 100 osób w~całym kraju, odkąd został zatwierdzony do użytku przez organy ścigania w~1992 roku''.}. Albo należałoby powiedzieć: jak najwięcej bólu, ale w~sposób, którego nieznajomość może sprawić, że będzie to mniej oczywiste obraźliwe: powolne ściskanie jąder męskiego aktywisty prawdopodobnie spowodowałoby podobny poziom bólu, z~podobnymi małymi szansami na trwałe uszkodzenie ciała, ale byłoby to trudne do przedstawienia jako czegoś innego niż tortury. Zastosowano udawane podejście medyczne, zastępcy zwykle wykonywaliby procedurę z~klinicznie brzmiącym wyjaśnieniem tego, co się robi: ,,Teraz zacznę nakładać koncentrat pieprzowy na gałkę oczną'', gdy inny zastępca trzymał głowę i~przytrzymywał otwarte powieki, to miało sprawić, że procedura wydaje się bardziej strawna i~naukowa.

 Takie podejście okazało się całkiem skuteczne. Niektóre gazety -- zwłaszcza liberalna \textit{San Francisco Chronicle} -- szybko uznały to postępowanie za tortury. Filmy były szeroko pokazywane w~programach informacyjnych; niektórzy prominentni wybrani urzędnicy wydawali oświadczenia dystansujące się od praktyki. Frank Riggs, kongresmen, którego biuro było zajęte, jednak szybko opublikował artykuł, w~którym przekonywał, że protestujących nie należy uważać za pokojowych (ponieważ, jak twierdził, potrącali ludzi w~jego biurze, rzucali trocinami po pokoju i~że kiedy po raz pierwszy zrzucili pień drzewa, jego pracownicy myśleli, że to bomba) i~biorąc pod uwagę prowokację, reakcja była wyważona i~uzasadniona. W obliczu sporu między ukonstytuowaną władzą a protestującymi, którego nie można było zignorować, większość gazet robiła to, co zawsze robiła w~takich kontekstach: starała się wyglądać na bezstronną, zajmując stanowisko redakcyjne gdzieś pośrodku. Gdzie konkretnie należy wytyczyć granicę między uzasadnioną potrzebą egzekwowania prawa a prawami łamiących prawo\footnote{Albo w~tym przypadku znowu, głównie łamacze przepisów, ale jak zawsze potraktowano to jako kwestię ,,prawną'' To nie były sprawy kryminalne.}? Być może typowa była odpowiedź CNN: która nadała tę kwestię w~formie debaty ,,punkt-kontrapunkt'' między liberalnymi i~konserwatywnymi ekspertami: po lewej ktoś, kto twierdzi, że jest to bezprawna brutalność policji, po prawej ktoś, kto twierdzi, że biorąc pod uwagę prowokację, odpowiedź była uzasadniona.

 Ofiary z~pierwszych trzech działań złożyły federalny pozew o prawa obywatelskie przeciwko departamentowi szeryfa. Aktywiści najpierw próbowali wynegocjować ugodę oferującą rezygnację z~jakiejkolwiek ugody pieniężnej, jeśli policja zgodziłaby się zrezygnować z~używania pieprzu na gałkach ocznych, a także przejść szkolenie w~zakresie niestosowania przemocy (ciekawa próba bezpośredniego negocjowania zasad zaangażowania), ale zostało to odrzucone od ręki. Sprawa trafiła na rozprawę. 25 sierpnia 1998 roku proces zakończył się zawieszeniem ławy przysięgłych, gdy jeden z~przysięgłych odmówił zdecydowania dla powoda. Sędzia przewodniczący, mianowany przez republikanów sędzia Sądu Okręgowego Stanów Zjednoczonych Vaughn Walker, umorzył więc sprawę z~sądu, orzekając w~efekcie na korzyść policji i~zaprzeczając, jakoby stosowanie gazu pieprzowego na gałki oczne było niepotrzebną siłą. Przedstawiciele departamentu szeryfa ogłosili później, że : po usprawiedliwieniu zamierzali wykorzystać kasety wideo jako materiał instruktażowy, aby pokazać innym wydziałom policji, jak radzić sobie z~blokadami\footnote{Siedem lat później, w~2005 roku, po kilku apelacjach, aktywistom udało się cofnąć tę decyzję, w~ugodzie, w~której każdemu zaoferowano symboliczny jeden dolar odszkodowania. Było to jednak długo po Seattle.}.

 Wszystko to, po czym nastąpiła śmierć Davida Chaina, wywarło mrożący wpływ na scenę akcji bezpośrednich na całym północno-zachodnim wybrzeżu Pacyfiku. Najwyraźniej ci, którzy praktykują pokojowe nieposłuszeństwo obywatelskie, mogli być teraz torturowani, a nawet zabijani, a ani media, ani sądy nie były skłonne stanąć temu na drodze. Nic więc dziwnego, że gdy wkrótce po tym, jak nowo utworzona Sieć Akcji Bezpośrednich DAN ogłosiła plany masowego nieposłuszeństwa obywatelskiego bez przemocy w~Seattle, przepełnionego blokadami i~zamknięciami, niektórzy aktywiści ekologiczni byli sceptyczni. To prawda, że  większość aktywistów -- w~tym anarchistów -- zapisała się. Ale Czarny Blok z~Seattle wywodził się w~dużej mierze z~tych, którzy tego nie zrobili, którzy w~rzeczywistości przewidzieli, w~wyniku swoich doświadczeń, prawie dokładnie to, co faktycznie się wydarzyło: że policja zaatakuje pokojowe blokady, żeby gaz pieprzowy będzie wcierany w~oczy zamkniętym i~że media będą traktowały takie zachowanie jako usprawiedliwione. Zdecydowali się przyjąć bardziej bojowe podejście i~wielu spędziło kilka następnych miesięcy na badaniu największych korporacyjnych przestępców ekonomicznych i~ekologicznych, którzy mieli biura i~sklepy w~centrum Seattle, aby dokonać bezpośrednich ataków na ich własność\footnote{Pomijam tutaj medialny mit, że Czarny Blok składał się prawie wyłącznie z~anarchistów z~Eugene w~stanie Oregon, którzy ulegli wpływom skrajnego prymitywistycznego anarchisty nazwiskiem John Zerzan. Najwyraźniej była wśród nich spora liczba ludzi z~Eugene'a, choć nie ma to nic wspólnego z~większością, a idee prymitywistyczne od dawna są dość popularne wśród anarchistów na północnym zachodzie, ale tak naprawdę jedyną rzeczą wspólną dla Czarnego Bloku było to, że większość była powiązana w~taki czy inny sposób z~aktywizmem ekologicznym.}.

 Media oczywiście potraktowały te ataki (które rozpoczęły się wiele godzin po tym, jak policja po raz pierwszy zaczęła używać gazu łzawiącego, gazu pieprzowego, granatów ogłuszających i~pałek do pokojowych blokad) jako retroaktywne usprawiedliwienie dla policyjnej przemocy, ale anarchiści prawdopodobnie mieli rację, twierdząc, że i~tak znaleźliby sposób na ich usprawiedliwienie. Rzeczywiście, jeszcze zanim wybito jakiekolwiek szyby, na ekranie pojawili się reporterzy lokalnej telewizji Seattle, wychwalający policję za dobrą pracę, nawet gdy wcierali gaz pieprzowy w~oczy aktywistów zamkniętych przy wejściu do hotelu, w~którym odbywały się spotkania WTO.

 To ważna historia. Jednak moim głównym celem tutaj nie jest wyjaśnienie tego, co wydarzyło się w~Seattle, ale poruszenie kwestii obywatelskiego nieposłuszeństwa Gandhiego. Jeśli chodziło o obnażenie systemu dominacji, działania i~reakcja policji w~Headwaters Forest wydawały się działać dokładnie zgodnie z~planem. Wszystkie powiązania -- między korporacją, politykami i~,,siłami porządku'' -- zostały ujawnione. Każdy z~nich wyraźnie odegrał rolę w~zachęcaniu, organizowaniu i~usprawiedliwianiu tortur i~morderstw przeciwko wyraźnie pokojowym demonstrantom; dowody tej tortury i~morderstwa zostały utrwalone na taśmie wideo. Nie było wątpliwości, skąd pochodziła przemoc. Ale właśnie z~tego powodu nie mogło to być przedstawiane jako przemoc.

 Pięćdziesiąt lat temu, podczas ruchu na rzecz praw obywatelskich, w~historii Ameryki był krótki moment, w~którym zadziałała taktyka Gandhia: przemoc leżąca u podstaw segregacji rasowej została obnażona w~całej Ameryce w~przerażających obrazach rasistowskich szeryfów z~psami policyjnymi. Być może była to bardzo szczególna sytuacja: na przykład fakt, że tak wielu dziennikarzy z~północy i~tak postrzegało Południe jako kraj obcy. A może w~ciągu minionego półwiecza coś się zmieniło w~amerykańskich mediach. Bez względu na powód, ten wyczyn nie został powtórzony. Wygląda na to, że w~dużej mierze dlatego, że ci, którzy podejmują decyzje redakcyjne, czują, że ich ostateczna lojalność jest związana z~tą bardzo większą strukturą władzy, którą strategie gandhijskie mają na celu ujawnić.

\medskip
\noindent HISTORIE Z GAZET A EPICKA KOMPOZYCJA USTNA
\medskip

 Pół wieku temu Milman Parry (1971; Lord 1960, 1991) rozwinął teorię eposu homeryckiego. Argumentowali, że wiele z~pozornych osobliwości stylu homeryckiego wynikało w~rzeczywistości z~wymogów kompozycji ustnej. Ich wielką innowacją było przeanalizowanie technik komponowania ustnego, używanych jeszcze za ich czasów przez jugosłowiańskich bardów. Niektórzy z~tych bardów potrafili improwizować dziesięć tysięcy linijek heroicznej poezji na jednym posiedzeniu. Robienie tego wydawało się niemal nadludzkie; ale w~rzeczywistości byli w~stanie to zrobić, ponieważ mogli polegać na pewnych, dość niezawodnych, znormalizowanych technikach. Jednym z~nich jest użycie epitetów: niektóre imiona są kwalifikowane, za każdym razem, gdy się pojawiają, za pomocą standardowych zwrotów opisowych (,,przebiegły Odyseusz'', ,,morze niczym ciemne wino'', ,,brązowowłosi Achajowie''). Są to w~efekcie niezawodne małe fragmenty opisowego wypełnienia, które można przywołać, aby wypełnić przestrzeń bez konieczności myślenia o nich. Jednocześnie pojawiają się większe standardowe frazy, a nawet standardowe scenariusze czy toposy (czasem statyczne, jak lista bohaterów, czy rozmieszczenie uzbrojenia, częściej standardowe jednostki działania, jak pertraktacje, wyjazd czy pojedynek), pod które można wygodnie podłączyć bohaterów. Uderza mnie, że komponowanie artykułów prasowych działa na bardzo podobnych zasadach, ponieważ autorzy wiadomości działają pod podobnymi ograniczeniami. Chociaż, w~przeciwieństwie do poezji epickiej, nie muszą przestrzegać żadnych szczególnych reguł metrum, są opisami akcji pisanymi dla bezpośredniego odbiorcy bardzo szybko: większość egzemplarzy gazet jest pisana w~zasadzie ex tempore, pod presją zbliżających się terminów i, pomimo okazjonalnych interwencji redaktorów, nigdy naprawdę nie przepisywanych, zanim nie trafi na ulice.

 Nawet pobieżne spojrzenie na artykuły w~gazetach lub przewodach ujawnia odpowiedniki homeryckich epitetów: na przykład następcy polityczni są niezmiennie ,,wybierani starannie'', gospodarki socjalistyczne lub socjaldemokratyczne niezmiennie ,,powolne'', ,,skrzypiące'' lub ,,sklerotyczne'', plemienni wojownicy lub (proamerykańscy) dyktatorzy są zawsze ,,dumni'', oświadczenia polityczne, które autor uważa za nielegalne, są niezmiennie określane jako ,,chaotyczne'' i~tak dalej. Każdy, kto czyta wiele doniesień prasowych o anarchistach, szybko się o tym dowiaduje. Przeszukanie przez Lexis/Nexis amerykańskiej gazety i~reportaży o krajowych konwencjach Demokratów i~Republikanów w~2000 roku, które wymieniają na przykład słowo ,,anarchista'', ujawnia, że  tylko w~pięciu przypadkach na dwadzieścia dziewięć było dozwolone, aby słowo po raz pierwszy pojawiło się bez żadnego przymiotnika lub wyrażenia przymiotnikowego (i prawie wszystkie z~tych pięciu wyjątków były nietypowymi zwyczajami, takimi jak mowa relacjonowana). Spośród wybranych epitetów przytłaczająca większość (dwanaście) stanowiła jakąś odmianę ,,samozwańczy'' (samozwańczy, samookreślający się lub samoopisujący ), następnie ,,ubrani na czarno'' (pięć przykładów), a potem ,, zamaskowany'' (trzy). Anarchiści czasami pojawiali się w~wędrownych sforach lub zespołach. W przeciwnym razie mogą być ,,gwałtowni'' lub ,,hardcore'' (po jednym), czasami ,,młodzi'' lub ,,młodzieżowi'' (również po jednym), ale to prawie wyczerpało wachlarz możliwości. Nie pojawiły się żadne inne przymiotniki: szczególnie uderzające, biorąc pod uwagę, że podczas obu protestów zjazdowych anarchiści byli zaangażowani w~każdy aspekt postępowania, od organizowania konferencji prasowych, przez żonglowanie ogniem, po dystrybucję darmowego wegańskiego jedzenia, wielu z~nich jest ubranych w~niezwykle jaskrawe kolory lub, od czasu do czasu, w~ogóle nic. Przeglądanie zapisów wiadomości o protestach w~Ameryce oznacza, że  wciąż pojawia się ta sama garstka epitetów i, co może nawet ważniejsze, prawie żadnych innych.

 Toposy są nieco mniej oczywiste, ale wydaje mi się, że coś w~tym stylu też jest wyraźnie obecne. Aby to zilustrować, pozwolę sobie przenieść scenę do pomniejszej akcji, w~której brałem udział latem 2001 roku w~Morristown w~stanie New Jersey. Scenariusz był nieco nietypowy. W wiadomościach pojawił się wówczas skandal dotyczący profilowania rasowego przez policję w~New Jersey. Szef organizacji ,,białych nacjonalistów'' (tj. nazistowskiej) Richard Barrett wpadł na sprytny pomysł ogłoszenia, że  4 lipca na trawniku przed budynkiem sądu w~Morristown przeczyta manifest na rzecz takiego profilowania rasowego. Złożył wniosek o pozwolenie. Powodem, dla którego było to sprytne, było to, że robiąc to, zobowiązał policję w~New Jersey do ochrony go przed nieuniknioną gniewną demonstracją, a tym samym mógł złożyć oświadczenie w~otoczeniu falangi policji. Lokalni anarchiści zaangażowani w~ARA (Anti-Racist Action -- Akcję Antyrasistowską) zmobilizowali się do wzięcia udziału w~demonstracji, a ktoś wezwał NY Ya Basta!, by również pomogli. Był tam Czarny Blok, może z~trzydziestu, z~jedną gigantyczną satyryczną marionetki i~kilku perkusistów i~muzyków, a także kilkunastu Yabbów w~żółtych kombinezonach chemicznych (choć bez ochraniaczy i~hełmów).

 O ile wiem, wydarzenie to było szeroko relacjonowane tylko w~jednej lokalnej gazecie, Bergen Record, gdzie historia pojawiła się następnego dnia\footnote{Scott Falon i~Yung Kim, ,,350 Cops Guard Racist Speaker'', Bergen Record, czwartek, 5 lipca 2001 roku}. Przejdę do punktu w~artykule tuż przed pojawieniem się anarchistów:

 Policja zabarykadowała kilka przecznic wokół gmachu sądu i~przeszukała wszystkich wchodzących na teren ręcznymi wykrywaczami metalu.

-- Wolałbym być krytykowany za to, że mam zbyt dużo zabezpieczeń niż za brak wystarczającej ilości -- powiedział prokurator hrabstwa Morris John Dangler. -- Nie cieszymy się, że on tu jest, ale pozwala na to sąd i~konstytucja.

 Wśród kontrmanifestantów byli związkowcy, Ogólnopolska Organizacja Kobiet i~pojedyncze działaczki społeczne. Nie brakowało również samozwańczych anarchistów, ubranych na czarno i~noszących na twarzach bandany.

 Czytelnik zwróci uwagę na typowe użycie homeryckich epitetów: właściwie mamy tutaj wersje trzech najczęstszych (,,samozwańczy'', ,,ubrany na czarno'' i~,,zamaskowany'' wszystkie z~rzędu. Anarchiści, kiedy po raz pierwszy pojawiają się w~opowieściach, są prawie zawsze określani jako ,,tak zwani'' lub ,,samozwańczy'' nie jest jasne, czy chodzi o to, aby zrównoważyć przypuszczalnie pejoratywne implikacje tego terminu (tj. ,,To nie my nazywamy ich anarchistami; ci ludzie w~rzeczywistości nazywają siebie anarchistami!''), czy też sugerować absurdalne pretensje grupy dzieci, które chcą utożsamić się z~ruchem społecznym minionych dni\footnote{Wielu dziennikarzy w~tym czasie zauważyło, że anarchizm był ruchem politycznym, o którym większość współczesnych badaczy zakładała, że  już nie istnieje.}. Ale podejrzewam, że samo zadawanie takiego pytania jest niewłaściwe. Przypisywanie autorowi intencji w~sensie hermeneutycznym -- pytanie ,,co on tak naprawdę próbuje powiedzieć tymi słowami?'' -- jest chybieniem sedna. Autor nie myśli świadomie. Stosuje standardowe zdanie. Wydaje się nie bardziej możliwe, aby amerykański dziennikarz wspomniał o grupie anarchistów w~takim artykule, nie nazywając ich ,,samozwańczymi'', tak jak homerycki poeta wspomniał o jutrzence, nie wspominając jednocześnie o jej różowych palcach\footnote{Nie oznacza to, że hermeneutyka koniecznie zakłada indywidualną intencję autorską. Wszelkie wyrafinowane podejście do interpretacji zakłada autora skonstruowanego ze zbiorowych pomysłów i~praktyk. Ale to chyba nie jest miejsce na zagłębianie się w~teorię interpretacyjną.}.

\medskip
\noindent \textit{Dalej tekst:}

 Kilku demonstrantów, stojących zaledwie kilka kroków od policji w~strojach do zamieszek, szydziło z~funkcjonariuszy. 
 
 -- Gliniarze są tutaj, by bronić nazistów -- skandowała jedna sekcja, podczas gdy inna krzyczała: -- Gliniarze, sądy, Ku Klux Klan; wszystkie są częścią planu szefa.

 Napięcia rosły, gdy anarchiści, głównie nastolatki i~młodzi dorośli, próbowali sprowokować starcie z~policją, nieustannie wpychając się w~ich szeregi. Chociaż funkcjonariusze otoczyli kontrdemonstratorów, nie doszło do gwałtownej fizycznej konfrontacji.
\medskip

 Tutaj sprawy stają się interesujące. Pierwszy paragraf przedstawia scenę z~oklepaną marksistowską pieśnią (w rzeczywistości zapewnioną przez brodatego mężczyznę z~czegoś, co musiało być jakąś sekciarską grupą w~identycznych żółtych koszulkach) wraz z~innym, który podobnie sugeruje, że protestujący są oszukiwani podstępem Barretta polegającym na identyfikowaniu się z~policją. Wtedy wkracza element dramatu. Anarchiści ,,próbują'' -- choć ostatecznie nie udaje się -- ,,sprowokować starcie'' otacza ich policja, ale nie dochodzi do przemocy.

 Mam z~tej okazji własne notatki, wzbogacone żywymi wspomnieniami. Kontrast jest dramatyczny:

\medskip
\noindent \textit{Trawnik sądu, Morristown, New Jersey}

\medskip
\noindent [Notatki terenowe 4 lipca 2001 roku]
\medskip

 Czarny Blok jest stosunkowo mały, około trzydziestu osób z~bębnami i~flagami (marionetka, którą przywiózł Zły, została szybko wycofana na emeryturę po tym, jak policja zagroziła jej przywłaszczeniem) i, w~przeciwieństwie do reszty demonstrantów, nigdy nie wszedł na trawnik, ponieważ oznaczałoby to poddanie się do przeszukania policji w~punktach kontrolnych. Ostatecznie znaleźli miejsce na drodze tuż pod barykadami, ale niedaleko gmachu sądu, i~rozpoczęli grać na perkusji. Jedna dziewczyna podskakiwała z~czarną flagą, inne tańczyły. Większość już dawno założyła maski po tym, jak zauważyła, że  policyjni fotografowie robią zdjęcia z~górującego nad nimi wzgórza. Z wyjątkiem kilku mundurowych po drugiej stronie ogrodzenia, policji nigdzie nie było widać -- przynajmniej na drodze -- chociaż na szczytach pobliskich wzgórz było kilka linii gliniarzy.

 Nasz mały kontyngent Ya Basta! przeszedł przez punkty kontrolne i~znajdował się po drugiej stronie perymetra: my jednak zatrzymaliśmy się w~pobliżu, wraz z~małym kontyngentem IMC, na wypadek kłopotów.

 Potem pojawiły się kłopoty. Nagle policja na jednym ze wzgórz utworzyła linię i~pomaszerowała w~dół zbocza z~tarczami i~pałkami w~pogotowiu, i~bez żadnego zapowiedzi utworzyła linię całkowicie otaczającą Blok.

 My, Yabbas, szybko przemaszerowaliśmy obok punktu kontrolnego, aby ocenić sytuację, konsultując się z~jednym Black Blokerem, który przypadkiem znalazł się po drugiej stronie linii policyjnej. Wszyscy obawiali się, że gliniarze mogą szykować powtórkę z~maja 2000 roku w~Nowym Jorku, kiedy policja arbitralnie otoczyła i~aresztowała inny mały blok na podstawie niejasnego dziewiętnastowiecznego prawa dotyczącego maskowania, zanim jeszcze zaczął się marsz. Policjanci stali z~kamienną twarzą, kilka metrów od nas, nie dając żadnych wyjaśnień.

-- Wiesz, że jest taki manewr, o którym słyszałem, ale nigdy go nie widziałem -- powiedział Smokey, badając sytuację. -- Jeśli gliniarze otoczą niektórych z~was, po prostu tworzycie kolejną linię po drugiej stronie ich linii, wtedy nagle to oni są otoczeni. Może powinniśmy tego spróbować.

 Więc tak zrobiliśmy. Dwunastu z~nas wymaszerowało w~naszych żółtych kombinezonach, trochę zdenerwowanych, i~łącząc ramiona utworzyliśmy łuk po drugiej stronie gliniarzy. Zadziałało dokładnie według specyfikacji. Po minucie najwyraźniej wydano rozkaz wycofania się, a policja odwróciła się i~pomaszerowała z~powrotem na wzgórze, równie cicho.

 Jeśli wrócimy do historii gazetowej, pierwszą rzeczą, na którą należy zwrócić uwagę, jest to, że autor zestawił wydarzenia z~bardzo różnych miejsc i~odwrócił wydarzenia w~kolejności czasowej. Skandowanie i~szyderstwo policji odbywało się w~obrębie perymetru policji; anarchiści byli daleko, w~miejscu, gdzie nie było gliniarzy. Rzekoma ,,próba sprowokowania konfrontacji'' przez Czarny Blok była dla mnie, gdy pierwszy raz ją przeczytałem, kompletną tajemnicą, do tego stopnia, że  natychmiast napisałem e-mail do jednego z~autorów, pytając, czy może on dostał tę historię od jakiegoś rzecznika policji, wskazując, że policja de facto zainicjowała konflikt przez otoczenie działaczy. Co ciekawe, autor odpowiedział, że był w~środku sprawy i~doskonale zdawał sobie sprawę z~tego, że gliniarze ruszyli pierwsi. Jednak nalegał, podczas gdy otaczanie anarchistów było wprawdzie zastraszającym posunięciem, po jego okrążeniu kilku anarchistów ,,wpadło na oficerów'', co posunęło sprawę ,,o krok dalej'', więc uznał, że jego podsumowanie jest uzasadnione. Też tam byłem. Ani ja, ani żaden z~moich towarzyszy nie zauważyliśmy żadnych anarchistów wpadających na policję, nie mówiąc już o ,,nieustannym wciskaniu się w~ich szeregi'' i~jestem raczej sceptyczny, czy to się naprawdę wydarzyło. Ale załóżmy, na potrzeby dyskusji, że tak się stało. Mimo to autor nadal skutecznie przyznał się do odwrócenia kolejności wydarzeń, zmuszając policję do działania jedynie w~odpowiedzi na rzekomy akt agresji ze strony anarchistów, kiedy nawet on, zapytany, był skłonny przyznać, że stało się na odwrót. Jeśli jacyś anarchiści naciskali na linie policyjne (i zauważ, jak tutaj, jak to często bywa, dziennikarz opisał działania jednego lub dwóch protestujących jako całej grupy), nie chodziło o sprowokowanie konfrontacji, ale o uwolnienie się od tego, co mieli wszelkie powody uważać za realną groźbę nielegalnego aresztowania prewencyjnego.

 Twierdzę, że byłoby prawie niemożliwe, aby amerykański dziennikarz opisał prawdziwy przebieg wydarzeń. Wyrażenie takie jak ,,napięcia rosły, gdy policja próbowała sprowokować starcie z~dotychczas pokojowo nastawionymi anarchistami, nagle otaczając ich dziesiątkami funkcjonariuszy w~stroju do zamieszek'' po prostu nie mogło pojawić się w~głównej gazecie w~Stanach Zjednoczonych, choć w~tym przypadku z~pewnością byłoby to bardziej precyzyjne. Napisanie takiego wyroku oznaczałoby wprost stwierdzenie, że policja, przynajmniej w~tym momencie, nie starała się utrzymać porządku, ale prowokować zamęt. Wydawałoby się to naruszać podstawową zasadę takiej ochrony: rolą policji jest utrzymanie porządku i~jej zachowanie należy zawsze interpretować w~tym świetle\footnote{Przypuszczalnie, gdyby ktoś napisał takie zdanie, redaktor natychmiast zażądałby dowodu, który, ponieważ jest to przypisanie intencji, mógł mieć formę oświadczenia w~tej sprawie tylko przez samą policję.}. Oczywiście policja również postrzega swoje zadanie jako utrzymanie porządku, ale jest w~stanie przedstawić to w~znacznie dłuższej perspektywie, w~której zastraszanie, a nawet prowokowanie konfliktów z~tym, co uważają za potencjalnie brutalne elementy, samo w~sobie jest częścią strategii utrzymania porządku. Taka postawa może być celebrowana w~filmach i~programach telewizyjnych (zwłaszcza jeśli przypisuje się ją pojedynczym, ,,niezależnym'' gliniarzom), ale takich intencji nie można przypisywać oficjalnej polityce policji ani nawet rozkazom wydawanych przez dowódców.

 Oto kolejna relacja, która usuwa wszelkie wyraźne przypisania intencjonalności, zawiera twierdzenie reportera (moim zdaniem wątpliwe), że kilku anarchistów naciskało na policję, ale to porządkuje wydarzenia we właściwej kolejności.

\textit{ Około trzydziestu ubranych na czarno anarchistów, z~większością twarzy zamaskowanych bandanami, zebrało się tuż za granicą, tańcząc, bębniąc i~skandując antyrasistowskie hasła, gdy detektywi policyjni fotografowali ich z~górującego wzgórza. W pewnym momencie oddział dwudziestu policjantów zeszło ze szczytu wzgórza i~cicho ich otoczyło, odcinając wszystkie drogi ucieczki. Kilku anarchistów próbowało przeforsować swoje linie, ale nie byli w~stanie tego zrobić. Niedługo potem dziesiątki demonstrantów, którzy byli na obwodzie, wylało się na teren. Tuzin, ubranych w~jaskrawożółte kombinezony, łączył ramiona, tworząc drugą linię otaczającą policję. Ci ostatni następnie wycofali się na swoje pierwotne pozycje i~nie doszło do gwałtownej fizycznej konfrontacji.}

 Prawdę mówiąc, nie sposób sobie wyobrazić, by ta relacja pojawiła się również w~amerykańskiej gazecie (choć może, całkiem możliwe, w~liberalnej kanadyjskiej). Po pierwsze, wyjaśnia, że  anarchiści nie zrobili niczego nielegalnego, zanim zostali otoczeni, co oznacza, że  robienie tego przez policję nie może być postrzegane jako nic innego jak akt prowokacji. Tak więc nadal narusza to wspomnianą powyżej zasadę porządku publicznego. Co gorsza, ujawnia, że  to nie policja, ale druga strona skutecznie załagodziła sytuację. Po raz kolejny jest to historia, której po prostu nie da się opowiedzieć.

 Oczywiście w~opublikowanej relacji kontyngent Ya Basta! znika całkowicie: ,,anarchiści ubrani na czarno'' są częścią standardowego repertuaru, ale anarchiści ubrani na żółto wymagają wyjaśnienia. Więc po prostu odpadają. Ale wydaje się, że istnieje również standardowy repertuar sekwencji wydarzeń. W przypadku działań na małą skalę, takich jak ta, możliwości wydają się być trojakie:

\noindent 1) pokojowy protest.

\noindent 2) pokojowy protest; niektóre elementy przemocy prowokują do konfrontacji; policja utrzymuje kontrolę (reaguje powściągliwie, przywraca porządek).

\noindent 3) pokojowy protest; niektóre elementy przemocy prowokują do konfrontacji; policja traci kontrolę i~następuje chaos (prawdopodobnie w~tym chaos ze strony nieuczciwych elementów w~policji).

 I~to jest granica tego. ,,Policja prowokuje konfrontację; protestujący reagują powściągliwie i~rozładowują sytuację'' jest po prostu nie do opisania, pomimo faktu, że, jak widzieliśmy w~poprzednim rozdziale, szkolenie w~zakresie niestosowania przemocy dotyczy głównie dostarczania aktywistom technik, które mogą to zrobić. To dlatego wydarzenia musiały zostać tak uporządkowane, by sugerować narastającą serię prowokacji protestujących, najpierw ,,szyderstwo'' z~policji, a następnie próba wszczęcia bójki poprzez popychanie ich, po której nastąpiła stosunkowo powściągliwa reakcja policji. Nie trzeba dodawać, że jeśli takie ramy narracyjne mogą sprawić, że reporter całkowicie zreorganizował wydarzenia na małą skalę, których był naocznym świadkiem, są one tym potężniejsze, gdy zastosuje się je do trzydniowej akcji masowej, takiej jak Seattle, gdzie naraz dzieje się tak wiele, że konieczny jest jakiś rodzaj upraszczającej ramy narracyjnej. Stąd 1 grudnia 1999 roku CNN zwięźle podsumował wydarzenia dnia poprzedniego:

 Kiedy dziesiątki tysięcy maszerowały przez centrum Seattle, mała grupa samozwańczych anarchistów rozbijała okna i~niszczyła sklepy. Policja odpowiedziała gumowymi kulami i~gazem pieprzowym [w Ackerman 2000].

 Jak zauważyli krytycy (np. Ackerman 2000; Boski 2002), stało się to ostateczną wersją tego, co wydarzyło się w~Seattle, powtarzaną w~nieskończoność później w~artykułach i~komentarzach telewizyjnych, mimo że w~żaden sposób nie opisuje rzeczywistej kolejności wydarzeń\footnote{Typowy przykład sposobu, w~jaki incydent został zrelacjonowany z~perspektywy czasu: ,,5 000 radykalnych ekologów i~anarchistów, którzy przybyli do Seattle - wspomagani przez 15 000 demonstrantów w~wiecu wspieranym przez związki - rozbili szyby i~pomalowali sprayem graffiti w~najlepszych sklepach w~centrum miasta . Sytuacja przekształciła się w~stan wyjątkowy, który trwał kilka dni. Policja użyła gazu łzawiącego, gazu pieprzowego i~gumowych granulek, aby stłumić kłopoty'' (,,Anarchist Onslaught on Chicago'', Chicago Sun-Times, 27 września 2002 roku, Michael Sneed). Tutaj masa blokad łączy się w~trzy- lub czterystu silnym Czarnym Bloku; w~innych podsumowaniach są one połączone z~,,pokojowymi protestującymi'' W żadnym wypadku nie wolno im stać samotnie, nie mówiąc już o uznaniu, że są grupą, która faktycznie zamyka spotkania.}. W rzeczywistości zawiera dokładnie te same operacje wymazywania i~odwracania, które obserwowaliśmy w~miniaturze w~opowieści o Morristown. Tak jak mylący aktywiści w~żółtych kombinezonach znikają z~konta Morristown, tak i~tutaj przeważająca większość zaangażowanych w~akcję bezpośrednią w~Seattle, którzy w~rzeczywistości nie maszerowali ani nie rozbijali okien, ale angażowali się w~blokady i~wstrzymania. W obu przypadkach efekt był ten sam: zredukować obraz do opozycji między dobrymi ,,pokojowymi demonstrantami'' maszerującymi i~niosącymi znaki a odzianymi na czarno ,,agresywnymi'' anarchistami. Tak jak kolejność wydarzeń została odwrócona w~historii Morristown, tak tutaj policja musi być przedstawiona jako reagująca na anarchistyczną prowokację, pomimo faktu, że -- nawet według własnych relacji CNN w~tym czasie -- policja zaczęła używać gazu pieprzowego o 10 rano tego ranka, na długo, zanim pierwsze okno zostało wybite i~prawie wszyscy świadkowie informowali, że nawet gdy Black Bloc wszedł do akcji, policja nie zwracała na nich większej uwagi, ale koncentrowała się niemal wyłącznie na atakowaniu tych blokujących dostęp do hotelu.

 W rzeczywistości ataki te były również wynikiem wyraźnych rozkazów z~góry. Dzień wcześniej sfilmowano dowódców policji, którzy uspokajali aktywistów, że policja z~Seattle nigdy nie zaatakowała pokojowo nastawionych demonstrantów i~,,nie ma zamiaru zaczynać teraz''. Dopiero po zakończeniu spotkań rankiem 30 listopada wydaje się, że urzędnicy federalni wydali rozkaz opróżnienia wejścia do hotelu wszelkimi niezbędnymi środkami. O 13:00 tego dnia ówczesna sekretarz stanu Madeleine Albright dzwoniła do gubernatora z~wnętrza hotelu i~(najwyraźniej) nalegała na zastosowanie silniejszych środków. W każdym razie mniej więcej w~tym momencie policja zaczęła systematycznie używać gumowych kul i~gazu pieprzowego, znowu nie przeciwko Czarnemu Blokowi, ale prawie wyłącznie w~celu oczyszczenia pacyfistów z~blokowania dostępu do hotelu. Następnego dnia, wydaje się, że prezydent Clinton zatwierdził decyzję o wprowadzeniu Gwardii Narodowej i~eskalacji do użycia wojskowych odmian gazu CS. Wszystkie te informacje są łatwo dostępne w~publikowanych wówczas reportażach, jeśli się je uważnie czyta. Ale zasady narracji zapewniają, że, zwłaszcza gdy historia jest skrócona, wszystko znika.

 Można zatem zapytać, co się dzieje, gdy reporter zostaje skonfrontowany z~narracją, która nie mieści się w~przyjętych ramach, a mimo to jest zdeterminowany, by ją zgłosić? Odpowiedź: ten reporter musi całkowicie wyjść z~trybu epickiego i~wejść w~inny gatunek, w~którym sami reporterzy stają się bohaterami.

 W amerykańskich mediach zdarza się to tak rzadko, że aby znaleźć przykład, będę musiał chwilowo przełączyć się na Europę.

\medskip
\noindent PROBLEM AGENTÓW PROWOKATORÓW
\medskip

 Jeśli poprzednia analiza jest poprawna, naczelną zasadą narracyjną relacji prasowych dotyczących demonstracji jest to, że policja musi być zawsze reprezentowana jako starająca się utrzymać porządek. Policja jest po to, by zachować spokój. Ich celem jest zapobieganie przemocy (domyślnie zdefiniowanej jako nieuprawnione użycie siły). Poszczególne nieuczciwe elementy mogą stracić kontrolę i~zachowywać się inaczej, ale należy założyć, że policja jako struktura instytucjonalna zawsze stara się utrzymać porządek. Najlepszym sposobem sprawdzenia tej hipotezy jest przyjrzenie się, co robią dziennikarze, gdy policja wyraźnie otrzymała rozkaz robienia rzeczy mających na celu wywołanie nieporządku -- zachęcania do przemocy tam, gdzie w~przeciwnym razie nie doszłoby do niej.

 Rzeczywiście, okazuje się, że takie incydenty po prostu nie są zgłaszane. Jeśli reporterzy uliczni rzeczywiście spróbują, redaktorzy wiadomości prawie zawsze będą interweniować, aby ich powstrzymać. Na przykład podczas marszu przeciwko Światowemu Forum Ekonomicznemu w~Nowym Jorku w~lutym 2002 roku policja zastosowała taktykę znaną wielu doświadczonym działaczom: funkcjonariusze w~cywilu dokonują aresztowania lub w~inny sposób wywołują jakąś bójkę, a kiedy inni interweniują, zostają aresztowani pod zarzutem napaści na oficera. Zdarzyło mi się stać obok reportera AP, kiedy właśnie zdarzył się taki incydent: funkcjonariusz w~cywilu, nie przedstawiając się, złapał nastolatkę w~środku marszu, najwyraźniej przypadkową, i~rzucił ją na ziemię. Kiedy kilku innych maszerujących odważnie próbowało interweniować, jeden położył rękę na ramieniu mężczyzny, mundurowi natychmiast wkroczyli, by rzucić ich na ziemię, skuli ich kajdankami i~zaczęli wywozić do czekających furgonetek. Reporter był sympatycznie wyglądającym facetem w~średnim wieku, z~wąsami i~kurtką fotografa, który wydawał się całkowicie zdziwiony tym, co się stało. Wyjaśniłem mu, że to znana policyjna technika.

-- Och, \textit{to }miłe -- skrzywił się i~zaczął pisać w~swoim zeszycie.

 Dwa dni później ponownie natknąłem się na tego samego reportera na konferencji prasowej, który z~dumą poinformował mnie, że umieścił moje informacje w~zgłoszonej przez siebie historii.

-- Tak -- odpowiedziałem (przeczytałem jego artykuł) -- ale nie wspomniałeś, że policjant obalający tę kobietę był \textit{tajniakiem}. O to właśnie chodzi.

-- Tak, napisałem! -- zaprotestował.

-- Nie było tego w~wersji, którą czytałem.

-- Och. Chyba jakiś redaktor musiał to wyciąć\footnote{Z drugiej strony, ta szczególna taktyka policji może być wykorzystana do generowania historii oczerniających protestujących. Podczas protestów na konwencji republikańskiej w~Nowym Jorku w~2004 roku wiadomości telewizyjne bez końca podbijały historię funkcjonariusza policji, podobno ciężko pobitego przez ,,demonicznych'' demonstrantów. Później okazało się, że był tajnym oficerem przebranym za motocyklistę, który wjechał na motocyklu prosto w~tłum rodzin robotniczych.}.

 Takie incydenty oczywiście nadal można przedstawiać jako anomalie. Jeszcze trudniej mówić o tym z~perspektywy mediów głównego nurtu, jest użycie przez policję agentów prowokatorów: funkcjonariuszy policji wyznaczonych do przebrania się za aktywistów, a następnie nakłaniania innych do aktów przemocy, aby dać policji pretekst do ich zaatakowania lub aresztowania. Każdy doświadczony aktywista prawdopodobnie zna co najmniej pół tuzina historii na ten temat, począwszy od policyjnych infiltratorów, którzy dołączają do grup afinicji, a następnie zaczynają nakłaniać ich do rozważenia użycia materiałów wybuchowych, po niezwykle grubych lub muskularnych ,,anarchistów'' w~czarnych maskach, którzy zaczynają rzucać butelkami w~policję podczas akcji ulicznych, a potem w~tajemniczy sposób znikają. Zwykle zakłada się, że każdy, kto na spotkaniu proponuje rzeczywistą przemoc, może być tylko gliną.

 Nie istnieją absolutnie żadne wątpliwości, że policja jest znana z~zatrudniania prowokatorów. Praktyka wydaje się najbardziej powszechna w~krajach śródziemnomorskich, takich jak Włochy, Francja czy Hiszpania; bardziej sporadyczne w~Europie Północnej i~Ameryce Północnej. Ale na pewno tak się dzieje. Jest to również nie do opisania. Przecież prowokatorzy nie mogą być przedstawieni jako ,,nieuczciwi gliniarze'', są to policjanci, którzy zostali wyznaczeni przez swoich przełożonych, by przebrali się za protestujących w~celu zachęcania do przemocy. Takie działanie nie miałoby sensu, gdyby przyświecało nam założenie, że policja jest przede wszystkim zainteresowana utrzymaniem pokoju. Aby to wyjaśnić, należałoby przejść do zupełnie innych ram, w~których policja postrzega siebie jako zaangażowaną w~polityczną rywalizację z~protestującymi, że działają w~imieniu reżimu politycznego, który ich zatrudnia, aby uniemożliwić protestującym osiągnięcie ich celów i~są całkowicie chętni do siania spustoszenia, a nawet narażania zwykłych obywateli, aby to zrobić. W rezultacie nie znam ani jednej historii medialnej głównego nurtu o proteście w~Ameryce w~ciągu ostatnich pięciu lat, która zawierałaby tyle zarzutów dotyczących prowokatorów, chociaż dość łatwo jest znaleźć takie zarzuty w~raportach IMC lub innych mediach przyjazne aktywistom. Jedynym znanym mi przykładem, w~którym amerykańskie media zauważyły  użycie prowokatorów podczas protestu globalizacyjnego, był artykuł Associated Press z~Barcelony\footnote{24 czerwca 2001 roku}. Warto przytoczyć w~całości:
\medskip

\textit{ Policja prewencji dokonała w~niedzielę niesprowokowanego ataku na protestujących przeciwko globalizacji, zgromadzonych w~parku miejskim po południowym marszu głównym bulwarem. Co najmniej 32 osoby zostały lekko ranne, a 19 aresztowano. }

\textit{ Tysiące krzyczących i~krzyczących demonstrantów, niektórzy z~małymi dziećmi, uciekło w~panice, gdy policja wepchnęła się w~tłum za tarczami, dzierżąc pałkami i~strzelając ślepymi strzałami.}

\textit{-- Podnieśliśmy ręce i~krzyczeliśmy: ,,Pokój, pokój'', ale oni po prostu nadchodzili -- powiedziała kobieta, która przedstawiła się jako Yolanda.}

\textit{Marsz wzdłuż Passeig de Gracia i~wiec na Plaza de Cataluna -- wraz z~innymi wydarzeniami weekendowymi -- zbiegały się ze spotkaniem Banku Światowego pierwotnie zaplanowanym na ten tydzień. Urzędnicy odwołali spotkanie w~zeszłym tygodniu, aby uniknąć gwałtownych protestów, które w~ciągu ostatnich dwóch lat zepsuły spotkania instytucji globalnych i~regionalnych.}

\textit{ Marsz był w~dużej mierze spokojny, ale po drodze wybito niektóre witryny sklepowe, w~tym restauracji McDonald's i~sklepu Swatch. Małe grupy mężczyzn i~kobiet wyśmiewały policję.}

\textit{ Tysiące innych demonstrantów dołączyło do maszerujących w~parku po marszu. Spokojnie słuchali mówców i~skandowali hasła, gdy policja przeszła przez plac.}

\textit{ Policja zaatakowała tłum po tym, jak niewielka grupa zamaskowanych mężczyzn i~kobiet, którzy wyglądali na agentów policji, zorganizowała bójkę na skraju parku, na oczach policji stojącej przed policyjnymi furgonetkami. Do przemocy wciągnięto kilkudziesięciu demonstrantów.}

\textit{-- Policja sprowokowała walkę. Byli tego częścią -- powiedziała Ada Colau, rzeczniczka Kampanii Przeciw Bankowi Światowemu, jednej z~organizacji protestacyjnych.}

\textit{ Reporterzy obserwowali, jak policja wykorzystała zainscenizowaną bójkę jako przynętę, aby wciągnąć w~nią protestujących, a następnie użyć jej jako pretekstu do wtargnięcia do parku. Druga szarża opróżniła park w~ciągu kilku minut. Zamaskowani napastnicy, niektórzy z~nich najwyraźniej noszący słuchawki douszne, zebrali się grupami na obrzeżach marszu protestacyjnego, który przybył do parku po przejściu kilkunastu przecznic bulwaru.}

\textit{ Mieli na sobie plecaki i~kije, ale byli w~stanie swobodnie przejść obok policji, założyć maski i~ustawić się między tłumem w~parku a liniami policyjnymi 25 metrów dalej.}

\textit{ Walka zaczęła się, gdy jeden mężczyzna złapał drugiego i~pociągnął go na ziemię. Inni z~tej samej grupy zaczęli się kopać i~okładać. Kiedy demonstranci zobaczyli, co się dzieje i~włączyli się do walki, policja wtargnęła do parku. Mężczyźni i~kobiety biorący udział w~bójce przeszli przez linię policji i~wsiedli do furgonetek.}

\textit{ Reporter zapytał jednego z~nich, czy są z~policji. Najpierw powiedział tak, a potem powiedział nie, zanim niezrażony przez policję poszedł do furgonetek. Telewizja państwowa podała, że  aresztowano 19 osób, a agencja informacyjna Efe podała, że  32 osoby zostały lekko ranne z~powodu stłuczeń i~siniaków}\footnote{,,Hiszpania: policja prewencyjna pozornie niesprowokowana w~ataku na protestujących'' Associated Press, 24 czerwca 2001 roku}.

\medskip
 Tylko dzięki wyjątkowym okolicznościom można było w~ogóle opowiedzieć tę historię. Po pierwsze, ponieważ faktyczne spotkania zostały odwołane, obecnych było niewielu korespondentów zagranicznych. W rezultacie AP polegał, nieco nietypowo, na lokalnych reporterach biegle posługujących się językiem i~znających lokalne oczekiwania (żaden amerykański reporter nie nazwałby wydarzenia ,,w dużej mierze spokojnym'', gdyby w~jego trakcie wybito dwie szyby). Do tego dochodzi fakt, że policyjna prowokacja była tak niesłychanie niezdarna i~że prawie nie próbowano jej zatuszować: w~końcu jak często można spotkać tajniaka tak głupiego, że najpierw przyzna się do bycia policjantem, a potem zaraz zaprzeczy? Niemniej jednak nawet twierdzenie, że to ,,wydaje się'' być policyjną prowokacją, wymaga, aby duża część artykułu -- większość ostatnich sześciu akapitów -- składała się głównie z~dowodów i~że te dowody nie pochodzą od protestujących ani nawet naocznych świadków, ale od samych reporterów. Innymi słowy, aby zrelacjonować takie wydarzenia, gatunek zasadniczo przechodzi od reportażu do dziennikarstwa śledczego, a bohaterką staje się sama reporterka. Nie trzeba dodawać, że wymaga to znacznie więcej miejsca niż zwykłe nakreślenie standardowej narracji, co zwykle można zrobić w~dwudziestu lub trzydziestu słowach, a zatem wymaga niezwykle współczującego redaktora.

 Tak się składa, że tym razem dostrzeżono użycie prowokatorów. \textit{New York Times }zamieścił nawet krótki artykuł wstępny następnego dnia, krytykując hiszpańską policję za niezdarność. Co godne uwagi, potwierdzenie to nie miało żadnego wpływu na przyszły zasięg. Niecały miesiąc później, podczas spotkań G8 w~Genui, włoska policja zastosowała niemal dokładnie tę samą taktykę na znacznie większą skalę, a żadne amerykańskie źródło informacyjne -- nawet Associated Press, które prowadziło reportaż z~Barcelony -- nie chciało przyznać nawet faktu, że protestujący \textit{oskarżali }policję.

 Genua była ogromną i~skomplikowaną akcją, w~której wzięło udział około trzystu tysięcy protestujących, podzielonych na szereg różnych bloków, od dużego kontyngentu pacyfistów po różowo-srebrny karnawałowy blok, Tute Bianche (pod przysięgą ścisłego kodeksu niestosowania przemocy), oraz Czarny Blok (którego najbardziej bojowym działaniem było podpalenie pustego budynku, zwykle używanego jako biura administracyjne lokalnego więzienia). Każdy z~nich miał swój własny marsz na ,,perymetrze'', który, jak w~Quebecu, był otoczony misternie skonstruowanym ogrodzeniem. Z każdym marszem policja obchodziła się mniej więcej w~ten sam sposób. Po pierwsze, grupa około dwudziestu anarchistów z~,,Czarnego Bloku'' pojawiała się znikąd, przemieszczała się między policją a demonstrantami, popełniała jakiś przypadkowy akt przemocy (przewracała śmietnik, rzucała w~policję kilkoma kamieniami lub butelkami) i~ponownie znikała. Następnie policja oskarżała rzeczywistych protestujących, strzelając niezwykle silnym gazem łzawiącym, który powodował wymioty i~utratę przytomności, a także zwykle łamała kości i~zadawała inne poważne obrażenia pałkami. W niektórych rejonach, zwłaszcza wokół marszu Tute Bianche, doprowadziło to do zażartych bitew, zwłaszcza gdy policja zaczęła masowo gazować w~dzielnicach robotniczych, a zirytowani mieszkańcy (którzy absolutnie nie byli zaangażowani w~niestosowanie przemocy) przyłączyli się do walki. W jednej z~takich bitew protestujący Carlo Giuliani został zastrzelony przez policję. Następnego dnia klasyczne taktyki pokojowe były mniej lub bardziej wykluczone. Niektórzy podpalali banki; policja dokonała nalotu na bezpieczne przestrzenie aktywistów, w~tym Independent Media Center, miejsce słynnej ,,masakry'', kiedy policja włamała się do pokoju pełnego śpiących protestujących w~pobliżu IMC i~pobiła prawie wszystkich na krwawą miazgę, ostatecznie wyciągając ich i~pozostawiając miejsce puste, prócz krwi i~roztrzaskanych zębów.

 Różnica między opisem lokalnych, europejskich i~amerykańskich źródeł wiadomości była uderzająca. Wśród działaczy na scenie głównym pytaniem było, czy ci ,,anarchiści'' byli faktycznie policją, czy lokalnymi faszystami pracującymi z~nimi w~parze. Obie możliwości były omawiane w~lokalnych mediach. W rzeczywistości doszło do wielkiego dramatu, gdy policja usłyszała pogłoski o możliwym istnieniu CD-ROM-u zawierającego cyfrowe wideo dwudziestu rzekomych anarchistów wychodzących z~komisariatu. Wkrótce na celowniku znaleźli się aktywiści z~kamerami; kamery zostały przywłaszczone i~zniszczone. Następnego dnia policja dokonała nalotu na Independent Media Center, systematycznie przywłaszczając lub niszcząc każdy fragment filmu lub aparatu cyfrowego, jaki tylko mógł wpaść w~jego ręce. Jednak CD-ROM, który istniał, nigdy nie został odnaleziony i~ewentualnie został przemycony do studia TV, gdzie spowodował drobny skandal po transmisji we włoskiej telewizji.

 Żadnej wzmianki o tym nie było jednak w~żadnej z~amerykańskich wiadomości. Pierwszego dnia główna historia dotyczyła Carlo Giulianiego, który zginął w~zaciętej bitwie w~pobliżu miejsca, w~którym marsz Tute Bianche został zatrzymany przez policję. Na przykład Associated Press rozpoczęła swoją historię 20 lipca o Genui w~następujący sposób:
\medskip

\textit{ GENUA, Włochy (AP) -- Jeden protestujący zginął, a prawie 100 policjantów i~demonstrantów zostało rannych w~toczących się bitwach, które wybuchły na brukowanych uliczkach i~szerokich placach tego starożytnego miasta portowego.}

\textit{ Minister spraw wewnętrznych powiedział, że policja zastrzeliła protestującego najwyraźniej w~samoobronie.}

\textit{ Podczas całodniowego starcia policji z~brutalną awangardą masowego marszu protestacyjnego demonstranci rzucali cegłami, butelkami i~bombami zapalającymi, podczas gdy policja strzelała gazem łzawiącym i~potężnymi strzałami z~armatek wodnych}\ldots 

\medskip
 Historia kontynuowała cytowanie włoskich źródeł rządowych i~policyjnych oraz głów państw biorących udział w~szczycie, ubolewających, ale usprawiedliwiających śmierć protestującego, a zakończyła się wypowiedzią protestującego o przepaści między bogatymi a biednymi na świecie. Każda historia, która pojawiła się w~amerykańskiej prasie, odnosiła się, przynajmniej pośrednio, do scenariusza ,,policja reaguje na przemoc protestujących'', pozostawiając nieco niejasne, czy policja naprawdę ,,utraciła kontrolę'' Reuters donosił na przykład:

\medskip
\textit{ GENUA, Włochy (Reuters) -- Protestujący podpalili samochody i~rozbili witryny sklepowe, a policja zamieszek wystrzeliła gaz łzawiący i~armatki wodne podczas godzin zamieszek, które wybuchły w~dniu otwarcia szczytu\ldots }

\textit{ Wcześniej zamaskowani protestujący rzucali flarami w~policję, rozbijali witryny sklepowe, podpalali dziesiątki śmietników i~przewrócone samochody i~ciężarówki, wysyłając gęsty dym kłębiący się nad miastem przez wiele godzin.}

\textit{ Policja użyła gazu łzawiącego i~armatki wodnej w~ciągu starć z~niektórymi z~dziesiątek tysięcy protestujących wokół ,,czerwonej strefy'' o wysokim poziomie bezpieczeństwa, chronionej przez 20 000 sił bezpieczeństwa.}

\medskip

 Skończyłem na chwilę zaszyty w~Independent Media Center, gdzie, jak można sobie wyobrazić, było wiele dyskusji na temat charakteru relacji. W końcu dziennikarze indymedia byli rozrzuceni po całym mieście, nieustannie opowiadając historie i, jak zwykle, mieli dość wszechstronny obraz wydarzeń, przynajmniej z~punktu widzenia aktywisty. Nieustannie monitorowali również sprawozdania telewizyjne, kablowe i~tym podobne. Najczęstszym wyjaśnieniem zachowania prasy międzynarodowej, jakie słyszałem, było to, że dziennikarze zostali poinformowani przez ich redaktorów, że żadne oświadczenie o faktach złożone przez protestującego nie może zostać wydrukowane bez potwierdzenia z~co najmniej jednego innego rodzaju źródła (standard, który nie, nawiasem mówiąc, dotyczył zwykłych obywateli, nie mówiąc już o urzędnikach). Dotyczyło to nawet samego IMC. Zostało to zinterpretowane tak dosłownie, w~rzeczywistości po tym, jak policja otoczyła budynek, a reporterzy IMC w~środku zaoferowali przesyłanie do BBC i~CNN materiału wideo na żywo z~inwazji, przedstawiciele obu sieci odmówili, tłumacząc, że nie wolno im używać materiału, ponieważ nie mieli nieaktywistycznego źródła, które potwierdziłoby, że te wydarzenia miały miejsce.

 Rozmawiałem przez telefon z~amerykańskim reporterem AP, kiedy to wszystko się działo. Kiedy zacząłem jej opowiadać o wcześniejszym użyciu prowokatorów, odpowiedziała z~niedowierzaniem. 
 
 -- Cóż, musiałbyś dostarczyć nam absolutny dowód takiego oskarżenia. 
 
 Kiedy zwróciłem uwagę, że jej własny serwis informacyjny opisał identyczną taktykę zaledwie kilka tygodni wcześniej w~Barcelonie, najpierw wydawała się nieświadoma, a potem odrzuciła to jako nieistotne\footnote{Gazety i~serwisy informacyjne często celowo wysyłają reporterów z~niewielką wiedzą lokalną, aby relacjonowali wydarzenia międzynarodowe, ponieważ prawdopodobnie przedstawiają perspektywę, z~której czytelnicy czują się komfortowo.}. Następnie próbowałem wskazać, że włoska policja była wtedy pod ostateczną kontrolą otwartego faszysty -- ówczesnego wicepremiera Gianfranco Finiego (,,I to nie jest tylko jakiś facet, którego nazywamy faszystą. On sam siebie nazywa faszystą!''), to również został odsunięty na bok. Ogólne stanowisko reportera było jasne. Nawet w~kraju, w~którym policja była pod dowództwem politycznych spadkobierców Mussoliniego, pomysł, że chcieliby zainicjować przemoc lub chaos, musi być traktowany jako z~natury nieprawdopodobny. Postawienie takiego oskarżenia było oburzające i~jeśli nie towarzyszyło im absolutne, niezaprzeczalne, jednoznaczne dowody, takie oskarżenia można było od razu odrzucić. Co więcej, wyraźny dowód musiał być dla każdego przypadku; wyraźny dowód w~Barcelonie nie ma wpływu na prawdopodobieństwo, że to samo wydarzy się dwa tygodnie później w~Genui. Tymczasem, oskarżenia wysunięte przez rzeczników policji (tj. rzucanie rac i~bomb zapalających, o których mowa w~powyższych historiach, co wydaje się nie mieć miejsca i~zostało po prostu zacytowane na policyjnych konferencjach prasowych) można traktować jako proste fakty, chyba że zostaną przedstawione wyraźne dowody, że się \textit{nie }wydarzyły. Wreszcie, jeśli taki dowód można udowodnić, albo że bomby zapalające nie zostały faktycznie rzucone na policję, albo że policja faktycznie użyła prowokatorów, należy go przedstawić natychmiast. Jeśli zostanie przedstawiony, powiedzmy, trzy dni później, zostanie zignorowany, ponieważ protesty nie są już przełomową historią, a fakt, że policja kłamała lub używała prowokatorów, nigdy nie jest uważany za historię samą w~sobie. To właśnie wydarzyło się w~Genui. Zanim kilka dni później w~stacjach telewizyjnych we Włoszech pojawiły się obrazy fałszywego Czarnego Bloku wychodzącego z~komisariatu, amerykańskie media przestały być zainteresowane. We Włoszech w~końcu nastąpiły dochodzenia parlamentarne, które otrzymały pewną wzmiankę w~kilku amerykańskich gazetach; ale nawet w~tych opowieściach rewelacje dotyczące prowokatorów nigdy nie były uważane za warte opublikowania.

 Innymi słowy, nie tylko istnieją ustalone ramy narracyjne, ale historie takie jak Barcelona,  gdzie reporterzy przeszli na tryb śledczy, same w~sobie nie mogą przyczynić się do zmiany tych ram. Nie sądzę, że jest to całkowicie spowodowane uprzedzeniami. A może należałoby powiedzieć, o ile istnieje uprzedzenie, to nie jest ono tak osobiste, jak strukturalne. Reporterzy, jako jednostki, bardzo się różnią w~swojej polityce. Wielu może sympatyzować z~aktywistami w~tej kwestii. Ale to samo można powiedzieć o poszczególnych policjantach. W obu przypadkach chodzi o to, że indywidualne opinie nie są tak naprawdę ważne; zarówno policja, jak i~reporterzy działają w~ramach struktury instytucjonalnej, która sprawia, że  ich opinie są nieistotne. Jeśli nie siedzą w~samotnych boksach, opisując schematyczne raporty przed deadline, jeśli na przykład piszą o ważnym wydarzeniu, ich historie prawdopodobnie będą opracowywane zespołowo, ich historie przepisane przez redaktorów; Spoty telewizyjne to jeszcze bardziej kolektywne produkty; ramy narracyjne są jedyną rzeczą znaną i~akceptowaną przez wszystkich. W każdym przypadku oznacza to, że historia akcji została pod każdym względem napisana przed wydarzeniami. Opowiedzenie innej historii wymaga mozolnych wysiłków i~przypadkowych okoliczności, a gdy tylko te okoliczności się kończą, wszystko wraca do poprzedniego stanu.

\section{Część II: Odpowiedzi anarchistów}

 Wielu aktywistów twierdziłoby, że przeprowadzona analiza jest zbyt hojna dla dziennikarzy. Ograniczenia strukturalne z~pewnością odgrywają pewną rolę; ale prawdą jest też, że reporterzy czasami kłamią. To jest uzasadniony punkt. Większość z~nas, zwłaszcza socjologów, niechętnie uznaje znaczenie samoświadomego oszustwa w~ludzkich sprawach.

 Staje się to szczególnie widoczne, gdy przyjrzymy się zachowaniu serwisów informacyjnych, których polityka jest najbardziej wrogo nastawiona do przekazu protestów. Ograniczę się do jednego przykładu. W przeddzień konwencji republikanów w~Nowym Jorku w~2004 roku zagorzały republikański \textit{New York Post }opublikował artykuł ostrzegający przed niebezpiecznymi anarchistami przygotowującymi się do ataku na miasto. Lista składała się głównie z~moich przyjaciół i~zawierała wiele stwierdzeń, które można określić jedynie jako oburzające kłamstwa. Wyróżniał się Jaggi Singh, rzecznik CLAC, którego czytelnik zapewne pamięta ze wstępu, wraz ze zdjęciem kogoś, kto trochę przypominał go, ćwiczącego na strzelnicy. Tekst nawiązywał do jego aresztowania w~Quebec City, zauważając, że został oskarżony o posiadanie gigantycznej katapulty, której używano do strzelania wypchanych zwierząt na policjantów. Cały sens wypchanych zwierząt polegał oczywiście na tym, aby tego rodzaju przerażające historie wyglądały śmiesznie. Nikt nie mógł usłyszeć opowieści o anarchistach, których najbardziej wymyślną bronią był pluszowy miś, i~dojść do wniosku, że stanowili realne zagrożenie dla bezpieczeństwa publicznego. Reporter \textit{Post }znalazł prosty sposób na obejście tego dylematu: po prostu zmienił ,,używany do rzucania wypchanych zwierząt na policję'' na ,,używany do  rzucania płonących wypchanych zwierząt na policję'' 

 Nikt, a już na pewno nie kanadyjska policja, nigdy nie sugerował, że wypchane zwierzęta wystrzelone z~katapulty w~Quebec City były podpalone. Nie ma też powodu, by sądzić, że autorowi udało się w~jakiś sposób przekonać samego siebie, że tak było. Po prostu to zmyślił. Skłamał. W końcu nie było żadnego praktycznego powodu, żeby tego nie robić. Anarchiści, jako okręg wyborczy, nie mają prawie żadnego wpływu politycznego, jeśli chodzi o media. Nie mają polityków chętnych do zajęcia się ich sprawą, nie mają instytucjonalnych zwolenników, nie ma potrzeby utrzymywania z~nimi dobrych relacji jako źródeł informacji, nie mają wpływu na reklamodawców i~bez względu na to, jak bardzo ich oczerniasz, mogą być prawie gwarantowane, że nie pozwie. Autor prawdopodobnie wymyślił, że ktokolwiek pragnący nazywać się ,,anarchistą'' w~rzeczywistości prosił się o takie potraktowanie.

 To oczywiście kolejna wersja znanego już problemu tworzenia przestrzeni autonomicznych. Jak już wspomniałem, nie można tego zrobić bez odmowy szukania wsparcia instytucji głównego nurtu. Dlatego głównym sposobem, w~jaki można nawiązać kontakt z~tymi instytucjami, jest policja. Kiedy policja spotyka ludzi, którzy systematycznie odmawiają uznania ich autorytetu, mają tendencję do atakowania ich. Przemoc jest z~natury warta opublikowania. Nie ma jednak praktycznie żadnych ograniczeń instytucjonalnych wobec członka mediów, który chce twierdzić, że anarchiści są agresywni, i~całkiem sporo ograniczeń instytucjonalnych wobec każdego, kto chciałby twierdzić to samo o policji. 

 Oczywiście nie wszyscy aktywiści są anarchistami, a szczególnie masowe mobilizacje łączą uczestników o szerokim spektrum postaw i~filozofii. Nawet wśród anarchistów stosunek do mediów korporacyjnych jest różny. Prawdopodobnie większość zdecydowanie odmawia z~nimi rozmowy. Wielu bierze udział w~Indymediach, których jednym z~najbardziej znanych haseł jest właśnie ,,Przestań nienawidzić media. Zostań mediami!'' -- tak naprawdę jedyne rozwiązanie zgodne z~zasadami akcji bezpośredniej. Media aktywistyczne, jak zobaczymy, zdołały całkowicie odmienić doświadczenie udziału w~akcjach. Przynajmniej dla uczestników są one dogłębnie demokratyzujące i~odalienowane: tam, gdzie kiedyś ktoś był zignorowany lub oczerniony, nagle każdy aktywista nie tylko ma dostęp do natychmiastowych, sympatycznych relacji w~Internecie, ale wie w~każdej chwili, że może, jeśli chce, stać się pełnoprawnym uczestnikiem opowiadania o własnych czynach. Jako narzędzie praktycznego radzenia sobie i~pomocy w~samoorganizacji, Indymedia zmieniły wszystko. 

 Problemem był zasięg. Podczas gdy w~szczytowym momencie -- na przykład podczas akcji w~Genui -- strony Indymedia były częściej oglądane niż CNN, poza Internetem po prostu nie ma porównania. Ostatecznie wiadomości sieciowe, wiadomości kablowe i~gazety docierają do większości Amerykanów, a Indymedia zasadniczo działają w~ramach większej społeczności aktywistów. Aby dotrzeć do większych kręgów, bardzo trudno było uniknąć kontaktu z~korporacyjnymi mediami, bez względu na to, jakie stosowałyby karty przeciwko tobie. W związku z~tym w~każdej większej akcji prawdopodobnie istniała grupa aktywistów medialnych, z~bankiem telefonicznym i~biegaczami na ulicy, organizująca konferencje prasowe i~robiąca wszystko, co w~jej mocy, aby stoczyć bitwę spinową z~policją i~politykami. Brałem udział w~kilku takich zespołach, m. in. na konwentach republikanów w~Philly w~2000 roku i~Nowym Jorku w~2004 roku, a moje doświadczenie zawsze było takie, że inni aktywiści traktują projekt z~wielką ambiwalencją. Wielu uważa, że  projekt jest zdradą podstawowych zasad; inni unikają aktywistów, którzy kontaktują się z~reporterem tak systematycznie, jak robią to sami reporterzy. Jeszcze inni potępiają każdego, kto ośmieliłby się przemawiać w~imieniu ,,ruchu'' lub innych anarchistów. Niemal niezmiennie toczą się też debaty na temat tego, czy rzeczywiście można prowadzić bezpośrednie działania \textit{przeciwko }mediom. Takie debaty zwykle nie prowadzą donikąd, problemem staje się wtedy, jaki byłby sens, ponieważ działania medialne są (co nie dziwi) jedynym rodzajem działania, który gwarantuje, że nigdy nie będą relacjonowane, i~tak dalej.

 Zamiast samemu opisywać te debaty, może lepiej byłoby pozwolić czytelnikowi na śledzenie jednej grupy aktywistów, którzy wymyślali takie kwestie. Pozwólcie, że wrócę do nowojorskiej Ya Basta! i~do rozmowy podsumowanej już w~rozdziale 1, na spotkaniu w~tygodniach poprzedzających Quebec. Kolektyw NYC Ya Basta! był grupą złapaną w~niezwykle głęboki dylemat w~zakresie strategii medialnej. Retoryka włoskiego Ya Basty! dotyczyła przede wszystkim widoczności, zapewnieniu publicznego (choć anonimowego) spektaklu jako twarzy wszystkich tych, których media i~maszyny polityczne giną, zwłaszcza na światowych szczytach: biednych, potajemnych imigrantów, ludności Globalnego Południa. Nie twierdzili, że przemawiają w~imieniu wykluczonych, ale chcieli przypomnieć wszystkim, że istnieją. Dzięki temu stali się mistrzami przyciągania uwagi prasy, a wielu europejskich anarchistów odrzuciło całą grupę jako wyczyn medialny. Grupa nowojorska składała się z~wielu aktywistów, którzy zdecydowanie zgadzali się z~krytyką, dla których głównym argumentem taktyki Ya Basta! było po prostu to, że dopełnienie i~wyrafinowana ochrona zapewniały bardziej ,,aktywną'' i~mobilną alternatywę dla blokad. Nawet dla tych, którzy nie mają nic przeciwko próbom grania w~media, pojawiło się pytanie, w~jaki sposób. Nie było praktycznie żadnej dyskusji na temat negocjacji w~sprawie Ustawy o Wolnym Handlu Ameryk (FTAA) w~USA. Wyglądało na to, że rząd świadomie prowadził politykę unikania debaty publicznej. Wszyscy rozmawiali o tym, co zrobić z~,,blackoutem prasy''. Duża akcja wydawała się mało prawdopodobna, aby to zmienić. W porównaniu do swoich europejskich odpowiedników amerykańskie media znacznie częściej opisywały protesty po prostu jako kwestię bezpieczeństwa, a znacznie rzadziej dawały protestującym jakąkolwiek platformę do opisywania tego, co ich zdaniem robią. We Włoszech rzecznicy Ya Basta! regularnie pojawiali się w~telewizji. W USA coś takiego byłoby nie do pomyślenia. Pytanie zatem brzmiało, jak w~warunkach ogólnego zaciemnienia grać na polityce widzialności.

 Aby dać czytelnikom krótkie przypomnienie postaci dramatu: Smokey, Emma,  Tim i~Flamma byli częścią kolektywu znanego jako Batalion Słoni Babar, zwykle podejrzliwego wobec jakiejkolwiek formy masowej organizacji; Moose, Laura i~Betty podchodziły do  rzeczy głównie z~perspektywy włoskiej Ya Basta!; Jackrabbit był aktywistą w~nowojorskim Reclaim the Streets; byłem wtedy z~Laurą ,,Ministrem Propagandy'', Sasha, zawodowy filmowiec, po raz pierwszy przyjechała do Ya Basta!, by nakręcić o tym film dokumentalny i~ostatecznie została aktywnym członkiem\footnote{Normalnie nie przytaczam bezpośrednich cytatów z~mniejszych spotkań, takich jak to, nie otwartych dla publiczności. Jednak to konkretne spotkanie zostało sfilmowane przez Alexisa do swojego filmu dokumentalnego i~wszyscy zaangażowani się na to zgodzili.}. Na początku rozmowy omawialiśmy scenariusze akcji: nikt nie był zbytnio zainteresowany akcją czysto defensywną (jak pomysł SalAMI na obronę odległej przestrzeni autonomicznej). Między innymi Smokey miał bardzo mieszane uczucia, jeśli chodzi o skoncentrowanie się na zwykłym rozwalaniu ściany; ale gdybyśmy mieli przejść na drugą stronę, to pojawiło się pytanie: w~jakim celu? Niektórzy z~nas mieli kontakt z~działaczem Québéco, który prowadził schronisko dla bezdomnych za Murem, który narzekał, jak trudno jest mu prowadzić jego placówkę, teraz gdy nie wpuszczono wolontariuszy. Jednym z~pomysłów była próba przemarszu, aby tam pojechać i~pomóc; w~ten sposób moglibyśmy wyjaśnić, że hełmy i~wyściółki były po prostu tym, czego potrzeba, aby móc zgłosić się do pracy w~schronisku dla bezdomnych w~tym mieście. Ale wtedy pytanie brzmiało: wytłumaczyć komu? Przypuszczalnie Indymedia omówiłyby tę historię i~prawdopodobnie jedno lub dwa przyjazne aktywistom miejsca, takie jak \textit{Frontline }lub \textit{Democracy Now!}. W przeciwnym razie, nawet gdybyśmy tylko próbowali przejść przez dziurę w~zbroi, bylibyśmy przedstawiani jako przerażający bojownicy ,,atakujący'' policję. Innym pomysłem było zrobienie czegoś z~tarczami: przekształcenie ich w~gigantyczne plakaty, każdy z~wizerunkami wykluczonych, na każdym wygrawerowanym przesłaniem, które ma być dostarczone zgromadzonym głowom państw.

\medskip
\noindent \textbf{ Czwartek, 8 marca 2001}
\noindent \textbf{  Spotkanie Ya Basta! Manhattan (in medias res)}\medskip



\noindent Smokey: \ldots lub, alternatywnie, zamiast tarczy, jednym pomysłem może być zrobienie jakiejś retoryki o głosie bezdźwięcznych. Moglibyśmy przeprowadzić wywiad z~niektórymi z~tych najbardziej skłonnych do zniknięcia ludzi i~zapytać ich: ,,Co mielibyście do powiedzenia FTAA''.

\noindent Flamma: Och, masz na myśli nagranie ich oświadczeń, a potem jakoś ich wrzucenie?

\noindent Ktoś: Ale znowu problem polega na tym, kto będzie wiedział? Kto w~ogóle będzie mógł to usłyszeć?

\noindent Betty: Moglibyśmy to transmitować przez Indymedia.

\noindent Smokey: Albo moglibyśmy nawet próbować porwać korporacyjną prasę. Powiedzmy, że mamy ich zablokować, na przykład powiedzieć im: ,,Kiedy to zagrasz, przejdziemy dalej''. Nie mówię, że to jedyna rzecz, którą zrobilibyśmy, albo że zbudowalibyśmy na tym naszą strategię, ale może to działać jako jedno potencjalne działanie, jeden element naszej strategii.

\noindent Ktoś: To jest hardcore.

\noindent Jackrabbit: Nie rozumiem, dlaczego to wszystko jest konieczne. Można dostać prasę, nawet dobrą prasę. RTS robi to cały czas w~Nowym Jorku.

\noindent Ktoś inny: Tak, ale RTS wprowadza małe, konkretne działania z~tylko jednym przesłaniem, łatwym do wyjaśnienia i~którego nikt nie może określić jako brutalny.

\noindent Jackrabbit: Hej, FBI umieściło RTS na liście krajowych terrorystów!

\noindent Ktoś: Tak, ale to dlatego, że są idiotami. Z drugiej strony to nie jest tak, że naprawdę mówimy o \textit{porwaniu }mediów, prawda? Zasadniczo próbujemy z~nimi negocjować\ldots 


 Rozmowa zeszła na chwilę na subtelności słów, takich jak ,,porwanie'' i~,,negocjowanie''. Ludzie mediów z~pewnością pomyśleliby o tym jako o ,,porwaniu'' -- prawdopodobnie zrobiliby wszystko, co w~ich mocy, aby z~zasady zapobiec tego rodzaju rzeczom. i~jaki przekaz moglibyśmy wymyślić, który wydawałby się uzasadniony w~tym kontekście? Jedyne, co mogłoby zadziałać, to coś, z~czym nikt nie mógł się nie zgodzić, a co było istotne, ale prasa odmówiła relacjonowania: na przykład zmusić ich do przeczytania pełnego tekstu projektu umowy FTAA lub przynajmniej wybranych fragmentów. To byłaby miła ironia losu, ale oznaczałoby odrzucenie całego pomysłu przekazywania głosu niemych.

\noindent Moose: Cóż, zastanówmy się nad pomysłem porwania mediów, lub w~każdym razie podjęcia jakiejś akcji bezpośredniej przeciwko mediom. Podoba mi się pomysł nie tylko lizania tyłków mediów, jak zwykle, ale faktycznego ścigania ich, w~jakiś sposób pociągania ich do odpowiedzialności. Ale jak to miałoby zadziałać? Powiedzmy, że się przebijamy. Jesteśmy za murem. Na jaki scenariusz patrzymy? Czym właściwie jest blokada?

\noindent \textit{[}wszystkie oczy zwracają się do Saszy]

\noindent Sasha: Cóż, zobaczmy. Prawdopodobnie nie będą mieli namiotu prasowego. Prawdopodobnie centrum medialne będzie znajdowało się w~hotelu konferencyjnym, choć zazwyczaj umieszcza się je w~piwnicy lub w~jakimś aneksie. Wielkie ryby będą miały reporterów z~protestującymi, prawdopodobnie bardziej foto niż wideo. Większość ekip wideo będzie w~hotelu i~będzie relacjonować szczyt. Strefa prasowa będzie pełna ludzi od technologii, wielu potężnych, krzepkich związkowców, którzy mogą, ale nie muszą, być sympatyczni. Wszędzie będą biegły kable, biegnące do łączy satelitarnych. Wyciągnij wtyczki i~są całkiem odcięci. Jeśli centrum faktycznie jest namiotem, kable będą naprawdę łatwe do zauważenia. Ale, tak czy inaczej, musielibyśmy przejść przez mur, zanim w~ogóle zaczniemy o tym myśleć.

\noindent Emma: Ale mówisz, że większość prasy będzie za Murem.

\noindent Smokey: Nie wiem o tym. W Filadelfii, na ulicach obok nas, byli prezenterzy, ludzie CNN. Podczas J20 byli o rzut kamieniem, nawet podczas bitwy pod Pomnikiem Marynarki Wojennej: w~pewnym momencie zobaczyłem Marię Shriver praktycznie metr ode mnie. Media się nas nie boją. Pytanie brzmi, jak to wykorzystać na naszą korzyść. To, o czym tutaj mówimy, jest trochę poza schematem. Czujemy się bardziej komfortowo w~starciu z~policją (którzy też są krzepkimi, nawiasem mówiąc, związkowcami), ale nie sądzę, żeby to było aż tak inne.

\noindent Jackrabbit: Ale można powiedzieć, że policja to \textit{tylko }nasz wróg. Z prasą nie wiemy, wszystko zależy od tego, na jaki spin się zdecydują. Tak naprawdę będziemy mieli o tym pojęcie dopiero wtedy, gdy zaczniemy oglądać relację tego wieczoru, a bardziej realistycznie, drugiego dnia.

\noindent David: Nie wiem, myślę, że można by argumentować, że jako instytucja media są tak samo częścią struktury władzy, jak policja. Przynajmniej tak jest w~przypadku ludzi, którzy w~obu przypadkach decydują. Spójrz na wybory: każdy ekspert od razu zaczął mówić to samo, nie ma znaczenia, kto naprawdę wygrał wybory, nie ma znaczenia, czy Sąd Najwyższy był rażąco stronniczy, naszym zadaniem jest dopilnowanie, aby ludzie nie stracili wiary w~instytucje amerykańskie. Musimy podtrzymać autorytet. Potem, trzy tygodnie później, słyszę w~CNN, że Bush idzie na swoją pierwszą dużą imprezę w~waszyngtońskiej elitarnej grupie społecznej i~wszyscy jego ludzie są zdenerwowani, czy ludzie będą traktować go jak prawowitego prezydenta, a czyja to partia? Ta kobieta (jak ma na imię?), która jest właścicielem \textit{Washington Post}.

\noindent Ktoś: Katherine Graham?

\noindent David: Tak, ona. Wszyscy są ze sobą w~łóżku. Przynajmniej na górze. To w~zasadzie to samo.

\noindent Moose: Porozmawiajmy więcej o mediach. Rozmawialiśmy o tym trochę, kiedy był tutaj Anton [Tute Bianche z~Finlandii], a ja czytałem niektóre materiały w~Internecie, refleksje po Pradze na temat relacji w~mediach. Mam wrażenie, że przynajmniej Włosi są bardzo skrępowani tym, co próbowali zrobić. Próbują stworzyć mit. Rodzaj nowego bohaterskiego podmiotu, który jednocześnie jest satyrą na siebie i~całkowicie obala konwencjonalne rozróżnienia na przemoc i~niestosowanie przemocy. Nie wiem, czy mamy co do tego konsensus, prawdopodobnie nie może być, ale: okej, wyobraź sobie to. Mamy tych dużych, bardzo kolorowych ludzi w~głupich kostiumach, z~balonami, z~drabinami, kazoo\ldots  To tak, jakby nagle w~akcję wkraczała banda postaci z~kreskówek. Myślę, że możemy zagwarantować, że media będą na nas lecieć jak muchy. Jak nas wrobią, to zupełnie inna sprawa. Wydaje mi się, że wszystko zależy od tego, co dzieje się wokół nas. Ale da nam to małe okienko, okazję do wyeksponowania naszego przesłania.

\noindent Betty: Czekaj, jestem zdezorientowana. Więc nie próbujemy już podejmować bezpośrednich działań przeciwko mediom?

\noindent David: Po prostu nie sądzę, że zrelacjonują wiadomość bez względu na to, co zrobimy. Po prostu przedstawią kostiumy. Chociaż jednym z~pomysłów, nad którymi myślałem, jest zrobienie gównianej listy. Mam na myśli, rób to, co robi rząd, wielcy gracze, kiedy mają do czynienia z~mediami. Posługują się marchewkami i~kijami: jeśli reporter da im życzliwy reportaż, dostarczą więcej przecieków lub informacji; jeśli reporter obraża ich w~jakiś sposób, ich źródła wysychają. Nie możemy tego zrobić, ponieważ ci nieliczni anarchiści, którzy będą rozmawiać z~reporterami, będą rozmawiać z~nimi wszystkimi jednakowo. Ale co, gdybyśmy to przetestowali: jeśli jakieś media rzeczywiście przekażą nasze przesłanie lub po prostu przekażą nam życzliwą uwagę, nagradzamy ich wyłącznością. Jeśli powiedzą jakieś rażące złośliwe kłamstwo, to albo odmówimy z~nimi rozmowy, albo wyjaśnimy, że poniosą konsekwencje.

\noindent Laura: Jakie dokładnie?

\noindent Moose: Rzuć tortem!

\noindent Flamma: Tak, rzuć tortem w~kamerę!

\noindent Smokey: Ale jakie jest w~tym przesłanie?

\noindent Flamma: To zabawne?

\noindent Smokey: Nie, poważnie. Chcemy przekazać nasze przesłanie, żyjemy w~stanie blackoutu prasowego. Jak to pomoże nam się z~tego wyrwać?

\noindent David: Tak, to prawda. Prawdopodobnie nie. Zakłada, że przynajmniej nas opiszą.

\noindent Jackrabbit: W każdym razie, już jesteśmy reprezentowani jako banda chuliganów. Zamierzamy im to \textit{ułatwić}?

\noindent David: Najdziwniejsze w~tym jest to, że nie zawsze tak jest. Pamiętacie Philly? Widziałem ankietę zrealizowaną kilka dni później, w~której pytano telewidzów, jak się czuli, kiedy widzieli protestujących w~telewizji i\ldots  nie pamiętam dokładnych liczb, ale większość z~nich była pełna sympatii, a u największego pojedynczego fragmentu, na przykład trzydzieści procent, sprawiliśmy, że poczuli się ,,dumni'' -- pomimo faktu, że w~Filadelfii mieliśmy \textit{tylko }wrogie relacje telewizyjne. Rozmawiali tylko o przemocy, ale ludzie i~tak byli z~nas dumni.

\noindent Moose: Widzisz, też o tym mówię. Czy nie byłoby idealnie, gdyby wszyscy oglądający telewizję widzieli nas na ekranie, a potem napisali na nim ,,gwałtowni protestujący'', a ludzie spojrzą na to i~powiedzą: ,,Czekaj! To się nie składa''.

\noindent Laura: Tak, ludzie nie będą wiedzieć, jak nas kategoryzować. Wiąże się to z~tym, co powiedział nam ten fiński facet o początkach Tute Bianche; pamiętacie, jak przy pierwszej wielkiej akcji nie używali tarczy, bo jak media to zareklamują? Cały pomysł polegał na tym, aby przekazać przesłanie, że chronimy siebie, nie chcemy zostać zranieni, ale nie zamierzamy też skrzywdzić nikogo innego. Oczywiście, jeśli gliniarze zobaczą tarczę, nie zinterpretują jej w~ten sposób. Będą myśleć, że patrzą na jakiegoś wojownika. Media też. Pojawiło się więc pytanie: jak możemy temu zapobiec? Jak możemy być proaktywni? Dlatego zaczęli robić takie rzeczy jak trzymanie dętek zamiast osłon lub owijanie się w~gumowe kaczuszki.

\noindent Moose: Tak, dętki wyglądają naprawdę dobrze, bo są takie miękkie i~puszyste. Przypomnij sobie te zdjęcia z~[akcji WTO w] Cancun: dziesięciu facetów schodzących na plażę z~dętkami owiniętymi w~papier-mâché; w~żaden sposób nie mógłbyś nikogo skrzywdzić, bez względu na to, jak bardzo się starasz. Wszystko wygląda bardzo uroczo i~nieagresywnie.

\noindent Laura: W Europie sprawy nie wyglądają tak samo, jak tutaj. Jedynym powodem, dla którego Tute Bianche mogli maszerować do Pragi, wyglądając jak legioniści z~tarczami i~wielkimi kijami, było to, że do 2001 roku ludzie przyzwyczaili się do tego pomysłu, zrozumieli trochę z~filozofii, wiedzieli, że ci ludzie naprawdę nikogo nie skrzywdzą. Lata pracy w~mediach zajęły nam osiągnięcie punktu, w~którym mogliśmy nosić kije.

\noindent Moose: Co to było, pięć lat?

\noindent Laura: i~wymagało to ciągłej pracy: przed każdym artykułem były akcje, chwyty medialne i~komunikaty prasowe, wywiady\ldots 

\noindent Moose: Podczas gdy my mamy tu tylko cztery miesiące. Myślę, że dobrze byłoby to zrobić wolno. W porządku, że nie używamy tarcz. W tym momencie myślę, że budowanie ruchu jest równie ważne, jak przekazywanie wiadomości.

\noindent Emma: Poczekaj chwilę. Myślę, że nie zgadzam się z~całym założeniem tutaj. Wydaje mi się, że jeśli nie używamy tarcz, to musi to być decyzja taktyczna, oparta na sytuacji taktycznej, a nie na tym, jak naszym zdaniem będzie to wyglądać w~telewizji. Robimy akcję bezpośrednią. Nie popis medialny. To, jak naszym zdaniem wygląda w~mediach korporacyjnych, \textit{nigdy }nie powinno być głównym powodem podejmowania decyzji taktycznej.

\noindent Laura: Cóż, ale chcemy być opisanie w~mediach? Mamy blackout. To jeden z~naszych podstawowych problemów.

\noindent Moose: Powiem jasno: zgadzam się z~analizą, która mówi, że powinniśmy traktować media jak wroga, jak gliny. Ale to nie znaczy, że musimy je ignorować. Policjantów też nie ignorujemy. Przynajmniej musimy na nich zareagować. Dlaczego więc nie możemy myśleć o tym jako w~zasadzie o tym samym?

\noindent Jackrabbit: Jasne: organizujemy wokół tego, co naszym zdaniem zrobią gliny. Dlatego na początku nosimy tarcze. Bo myślimy, że gliniarze prawdopodobnie będą próbowali nas uderzyć. Tak więc, jeśli media tam są, zgodnie z~tą samą logiką mamy pełne prawo przewidywać, co media prawdopodobnie pomyślą o tarczach, co, nawiasem mówiąc, nie jest przesądzonym wnioskiem, który będzie negatywny, czyż nie? Z tego, co wiemy, tarcze mogą nawet wyglądać dobrze dla niektórych z~nich.

\noindent Mark: Nie wiem, dlaczego w~ogóle się o to kłócimy. Czy dętki nie mogą być tak samo dobre?

\noindent Jackrabbit: Cóż, myślę, że to bardziej kwestia koncepcyjna. Kwestia bierność kontra agresywność. Nie wiem, dlaczego koniecznie musimy szukać. Co jest złego w~byciu proaktywnym?

\noindent Laura: Właściwie problem z~dętkami polega na tym, że ciężko jest je trzymać bez odsłaniania palców. Potem gliniarze uderzają prosto po palcach. To samo dotyczy banerów. Na początku mieliśmy dużo złamanych palców. Dętki są problemem, chyba że masz na sobie jakieś wymyślne wyściełane rękawiczki, a wtedy trudno je uchwycić.

\noindent Moose: Pamiętaj, będziemy mieć tylko około pięćdziesięciu osób. Maksymalnie. To nie jest do końca armia. Nie będziemy w~stanie zrobić tego wszystkiego na podstawie samej wagi liczb, ale mamy pewne zalety. Jesteśmy dziwni. Uderzamy wizualnie. Nikt po tej stronie oceanu nie widział wcześniej czegoś takiego jak my. Więc jesteśmy potencjalną wiadomością. Bylibyśmy idiotami, żeby to wyrzucić.

\noindent Sasha: Mówiąc jako ktoś, kto pracował z~mediami przez większość mojego życia zawodowego, myślę, że jesteście tutaj lekko naiwni. Media zrobią, co chcą, nawet jeśli wszyscy wyjdziemy w~puszystych, niedźwiedzich garniturach. Wszyscy wiemy, jak czytają się te historie. Dostajesz swój typowy tekst składający się z~sześciu akapitów; zaczną od przemocy, jeśli będzie, jeśli nie, powiedzą coś o ,,karnawałowej atmosferze'', ale nic o przekazie. Pojawi się trochę trywializującego humoru, a potem przerywnik do jakiegoś oficjalnego rzecznika mówiącego, jak wszyscy jesteśmy zdezorientowani i~jeśli mamy jakiekolwiek zastrzeżenia do traktatu, to już się nim zajmują. Jestem tutaj bardziej po stronie Emmy. Skierujmy nasze przesłanie bezpośrednio do osób podejmujących decyzje i~wymuśmy je na nich. Zapomnij o mediach. Pieprzyć ich.

\noindent Smokey: Nie akceptuję tej dychotomii. Można odwołać się poprzez absurd. Większe, bardziej zorganizowane grupy robią to cały czas. Albo spójrz na klaunów w~Philly, byli naprawdę skuteczni na ulicach, nawet jeśli nie byli publikowani w~mediach. Plus pamiętajcie: to wszystko będzie działo się w~Kanadzie. Media są tam o wiele mniej monolityczne. Mają swój proces, tak jak my mamy nasz, ale możemy przerwać ich proces i~sprawić, by się zmienili.

\noindent Albo możemy pomyśleć o tym jako o jednej z~możliwych opcji. Są inne opcje. Możemy spróbować je obejść za pośrednictwem Indymedia. Lub trzecia opcja: spróbuj uzyskać nierozcieńczony obraz i~wiadomość, bez edycji. Spróbuj aktywnie przejąć kontrolę; zdominować go, przejąć na jeden dzień, przerwać ich proces. Możemy szturmować stację, a jeśli chcemy być mniej agresywni, może zastąpić gazety, wykorzystać hakerów do zastąpienia naszego przekazu na ich stronach internetowych. Będziemy nazywani bandytami, tak, jeśli to zrobimy, ale i~tak to powiedzą, powiedzą to samo, jeśli zburzymy mur. Może miękkim podbrzuszem bestii w~tym dniu będą media, a nie FTAA.

\noindent Flamma: Albo gliny.

\noindent Smokey: Prawdopodobieństwo dołączenia do bloku i~zakłócenia spotkań FTAA jest dość małe. Przygotowywali się na nas od sześciu miesięcy. Możliwe, że moglibyśmy to zrobić, ale \textit{to }, na to nigdy się nie przygotowywali.

\noindent Laura: Całkowicie się zgadzam, ale nie wiem, czy należy kłaść nacisk na media. Chodzi mi o to, że to powinno być centrum naszego działania.

\noindent Jackrabbit: Zgadzam się. Wszystkie te rzeczy o mediach, czy nasza strategia medialna nie musi zależeć od tego, co próbujemy zrobić? A może po prostu zaatakujemy media za to piekło? Ponieważ poważnie wątpię, czy cokolwiek poza przyłożeniem broni do głowy sprawi, że wygłoszą jakieś oświadczenie, którego nie chcą wyemitować. To się po prostu nie wydarzy.

\noindent David: Myślę, że media są całkowicie uzasadnionym celem strategicznym. W końcu, kiedy dochodzi do zamachu stanu, jaka jest pierwsza rzecz, którą zawsze robią? Próbują przejąć stację radiową lub telewizyjną. Oczywiście jest ku temu powód.

\noindent Flamma: i~myślę, że nie są na to przygotowani.

\noindent David: Ale Jackrabbit ma rację: pytanie brzmi, jak i~w jakim celu? Myślę, że niektóre to pytania techniczne. Na przykład, czy istnieje sposób na zablokowanie ich kanałów?

\noindent Emma: Po prostu nie sądzę, żeby koncentrowanie się na mediach było tym, co Ya Basta! powinna robić. Jadę do Quebecu, ponieważ wierzę w~akcję bezpośrednią. Oznacza to, że chcę mieć możliwość bezpośredniej konfrontacji z~naszymi władcami; nie apelować do mediów o przesłanie wiadomości, a nawet zmuszać do tego. W końcu, o czym jest Ya Basta! ? Dołączyłam, ponieważ wydawało się to bardziej proaktywne niż blokady. Ale to strategia akcji bezpośredniej, a nie symboliczny gest, nie sposób na wywieranie wpływu na prasę.

\noindent Flamma: Zgadzam się, ale myślę, że brakuje ci kluczowego punktu. Mamy ustawienie. Są gracze; media, gliniarze, delegaci. Mają proces. Chcemy zablokować to kluczem francuskim. Więc musimy zadać sobie pytanie, co jest najsłabszym punktem? Wydaje mi się, że spowodowanie blackoutu mediów, po ich stronie, byłoby tak samo skuteczne, jak umieszczenie na nim naszego przekazu (z czym zgadzam się z~Jackrabbitem, to wydaje się mało prawdopodobne).

\noindent Sasha: Zaprzeczę sobie, co jest w~porządku.

\noindent Moose: Do diabła, zaprzeczyłem sobie już co najmniej dwa razy.

\noindent Sasha: W porządku, powiedzmy, że media są narzędziem władzy państwowej. To kolejne ramię bestii lub, jak mówisz, miękkie podbrzusze\ldots 

\noindent Jackrabbit: Um, zdajesz sobie sprawę, że to logiczna sprzeczność?

\noindent David: Magister filozofii! Powiedzmy, ,,macka''. No dalej, Sasza.

\noindent Sasha: \ldots w~każdym razie przeszkadza nam to tak jak gliny. Możemy więc potraktować to jako analogiczne. Jestem sceptyczny, czy rzeczywiście możemy to zrobić, ale warto to rozważyć. A jeśli jakiś sprzeciw, cóż, może powinniśmy porozmawiać o różnorodności taktyk \textit{w obrębie }Ya Basta!.

\noindent Jackrabbit: Jeśli chodzi o sprzeciw Emmy, to przejmowanie mediów nie jest akcją bezpośrednią, myślę, że gdybyśmy mogli przejąć media, byłoby to całkowicie w~duchu akcji bezpośredniej. Musimy tylko dowiedzieć się, jak. Spin to kolejna kwestia: jak uniemożliwić im powiedzenie, że zrobiliśmy im coś brutalnego i~strasznego. Porwanie stacji medialnej może zadziałać: w~sensie odnalezienia jednej z~ich porzuconych ciężarówek i~skorzystania z~niej samemu, obrony jej, przy użyciu taktyki bez przemocy, przed gliniarzami, którzy spróbują ją odzyskać. Użyć przeciwko nim ich własnych narzędzi. Mogę sobie wyobrazić, że coś takiego się wydarzy, jeśli naprawdę mamy ludzi z~wiedzą techniczną do obsługi sprzętu.

\noindent Ktoś: Och, to proste. Znam pół tuzina maniaków Indymedia, którzy potrafią obsłużyć łącze satelitarne, nie ma problemu.

\noindent Jackrabbit: Następnie pojawia się pytanie, jak sformułować oświadczenie, którego nie można powiedzieć, że jest chaotyczne, szalone i~nielegalne. To bardzo trudne. Ale może nie niemożliwe. Powiedzmy na przykład, że Mohawkowie chcą coś oświadczyć światu.

\noindent David: Lubię to. Słuchajcie, kłamstwo to rodzaj przemocy. Zwłaszcza gdy służy do tego, by gliniarze mogli bić ludzi, a IMF dosłownie zabierało jedzenie z~ust dzieci. Musimy tylko wymyślić sposób, żeby to zrobić, żeby nie wyglądać jak kupa czubków.

\noindent Tim: Myślę, że to dobra rzecz, jak mówi David; ale myślę, że najpierw powinniśmy iść za mocną częścią, murem. Jeśli nam się uda, media pójdą za nami. Nie dlatego, że powinniśmy pominąć akcję medialną.

\noindent Moose: Byłbym całkowicie przygnębiony\ldots  cóż, gdyby hotel, w~którym są media, byłbym za zamknięciem go i~tak dalej. Albo przejęciem stacji medialnej. Ale to powoduje wiele problemów, jeśli wszyscy zostaniemy aresztowani, pochłonie to mnóstwo energii. Załóżmy, że docieramy do Quebecu: myślę, że bardziej interesujące jest pierwszego dnia, aby zgłosić żądania na spotkanie, petycję lub coś, a może dwudziestego pierwszego możemy zrobić akcję medialną, jeśli nie jesteśmy w~więzieniu.


 Później okazało się, że podobne rozmowy toczyły się wówczas w~radach delegatów w~Québec City: nawet w~wielu szczegółach, takich jak możliwość blokowania mediów w~celu zmuszenia ich do nadawania przygotowanych taśm. Nic z~nich nigdy nie wyszło. Zdając sobie sprawę, że w~końcu musiałyby same się pokonać, poszły drogą najbardziej wadliwych projektów aktywistów: zniknęły z~braku nikogo, kto chciałby im poświęcić czas. Bardziej prawdopodobne scenariusze były zbyt trudne technicznie. W końcu prawie najbliżej, jak ktokolwiek zbliżył się do akcji antymedialnej, ktoś rozbił szyby kilku furgonetek telewizyjnych zaparkowanych w~pobliżu głównego wejścia przez ścianę; czyn, który prawdopodobnie zakładali właściciele furgonetki, był niczym więcej niż przypadkowym apolitycznym wandalizmem.

\medskip
\noindent REPREZENTACJA ZBIOROWA (STAJĄC SIĘ MEDIAMI)
\medskip

 Typową reakcją anarchistów, jak powiedziałem, jest całkowite unikanie takich pułapek poprzez poleganie na alternatywnych mediach. Nie ma tu miejsca na zagłębianie się w~szczegóły tworzenia sieci Indymedia, która z~pewnością zasługuje na bardzo długą własną etnografię, ale uważam, że kluczowe jest podkreślenie różnicy, jaką media aktywistyczne uczyniły w~doświadczeniu udziału w~protestach. oraz bezpośrednie akcje dla samych aktywistów. To naprawdę nieocenione. Już sam fakt, że każdej akcji, nieważne jak małej, nieuchronnie będzie towarzyszył przynajmniej jeden zaprzyjaźniony reporter prasowy, jeden fotograf, jeden kamerzysta, już teraz ma ogromny wpływ na jakość doświadczenia.

 Weterani protestów i~akcji w~latach 80. i~90. często mówią, że najbardziej frustrujące było w~nich poczucie, że bez względu na to, jak wiele uwagi, planowania i~energii włożysz w~ich przeprowadzenie, później -- chyba że akcja była bardzo duża lub mieć szczęście -- to tak, jakby całe wydarzenie nigdy się nie wydarzyło. Nie pozostawiło śladów. Albo, żeby być dokładniejszym, pozostawiło niewiele. Kilka zdjęć wklejonych na tablicę ogłoszeń lub rozdanych wśród znajomych. Ulotki skserowane lub inne podobne efemerydy. Być może wycinek z~sąsiedniej gazety, zwykle nieco lekceważący. Bardziej prawdopodobne, że w~ogóle nie będzie drukowanej relacji. Nagłe pojawienie się internetowych Indymediów, obok listserwerów aktywistów, po Seattle oznaczało przede wszystkim, że nawet najmniejsza akcja zostanie przez \textit{kogoś }życzliwa relacjonowana. Można było po południu uczestniczyć w~wiecu lub przejażdżce Masy Krytycznej i~zalogować się tego samego wieczoru, przeczytać relację i~obejrzeć zdjęcia z~imprezy, lub też obudzić się następnego ranka i~znaleźć gazetę lub reportaż w~stylu kablowym oczekujące w~skrzynce odbiorczej. Zamiast doświadczenia odosobnionych wydarzeń zapadających w~niepamięć, człowiek miał poczucie, że na swój mały sposób przyczynił się do tworzenia historii. Był to nieusuwalny i~przypuszczalnie trwały zapis; będzie dostępny dla przyszłych historyków i~pewnego dnia zostanie omówiony w~książkach i~seminariach (wielu działaczy jest tego całkiem pewnych i~gotowych do spostrzeżenia na temat możliwości). Jeśli akcja była w~jakikolwiek sposób ważna, można też było być pewnym, że te historie i~obrazy pojawiały się w~tym momencie również na ekranach komputerów aktywistów w~Boliwii i~Danii. Nagle pojawiła się międzynarodowa społeczność podobnie myślących radykałów -- tych samych ludzi, na których opinii najbardziej nam zależało -- którzy byli pewni, że wiedzą, co się udało. W ten sposób natychmiast przekształciło się poczucie czasu i~przestrzeni aktywistów.

 To poczucie bezpośredniego tworzenia historii stało się dużą częścią tego, co sprawia, że  takie wydarzenia są z~natury mniej alienującymi doświadczeniami niż kiedyś. Istotny jest tutaj fakt, że reporterzy Indymedia opowiadają tę historię przynajmniej w~takiej formie, w~jakiej doświadczyli tego działacze. Czytelnik może sobie wyobrazić, jak może poczuć się uczestnik opisanej wcześniej akcji Morristown, która wzięła udział w~sprytnym manewrze, który rozproszyła policję i~wróciła do domu, czując się dość skuteczna i~dumna z~siebie, gdyby nie tylko musiała przeczytać artykuł w~\textit{Bergen Record }następnego dnia, ale wiedzcie, że był to jedyny zapis, jaki reszta świata kiedykolwiek miała z~tego wydarzenia. Dziesięć lat wcześniej prawie na pewno tak by było. Teraz prawdopodobnie pojawił się nie tylko artykuł w~Indymediach, ale w~formacie internetowym, który pozwalałby wskazać wszelkie dostrzeżone błędy, dodać własne anegdoty i~w inny sposób przedstawić własną perspektywę i~doświadczenia, które, o ile dyskusja internetowa nie przerodziła się (jak to się czasem zdarzało) w~kłótnie i~obelgi, zapewniały, że można nie tylko tworzyć historię, ale także odgrywać rolę w~ustanawianiu narracji historycznej, która się następnie pojawi.

 Ten ostatni punkt jest kluczowy. Indymedia postrzega siebie jako uczestniczące źródło informacji. Historie mają być napisane przez uczestników. W zasadzie ma to na celu całkowite przełamanie podziału na dziennikarzy i~publiczność. Podczas gdy w~praktyce jest to często bardziej ideał niż rzeczywistość (niektórzy aktywiści są wyłącznie dziennikarzami indymedialnymi i~pojawiają się na akcjach wprost jako reporterzy i~prawie nigdy w~żadnym innym charakterze; działacze społeczni znacznie częściej wzywają reportera indymedialnego niż wysyłają własne kawałki), jeśli nic więcej, to znaczy, że aktywiści są świadomi, że mogliby przedstawić swój własny raport, gdyby naprawdę chcieli. Albo nawet, jeśli o to chodzi, pojawić się na spotkaniu redakcyjnym, na którym omawiano relację z~większej akcji. W rzeczywistości mogliby, a niektórzy biorą udział w~dowolnym aspekcie procesu produkcyjnego: od uruchomienia akcji, na przykład, po jej sfilmowanie, montaż filmu i~produkcję ostatecznego wideo. Normalnie, po większej akcji, przestrzenie Indymedia szybko zapełniają się aktywistami, czasami przychodzącymi z~pomocą, czasami po informacje, czasami po prostu po to, by móc obejrzeć kadry lub surowy materiał, który się pojawia, z~tego, co robili na ulicach zaledwie kilka godzin wcześniej. Produkcja wydarzenia i~produkcja opowieści o tym, co się wydarzyło, stają się częścią jednego, zbiorowego projektu, wspólnego dla wszystkich zainteresowanych, w~którym udział jest ograniczony jedynie tym, ile czasu i~energii jest się w~stanie zainwestować. 

 Ten sens kolektywnej produkcji powraca na każdym poziomie reprezentacji i~komunikacji w~procesie składania i~realizacji działania. Rekwizyty i~kostiumy powstają wspólnie. Nawet zbiorowe dyskusje czy rozpowszechnianie informacji prowadzone w~trakcie akcji ulicznej odbywają się w~taki sposób, że wszyscy są zaangażowani. Na przykład: podczas spotkania otwartego w~środku akcji -- aby na przykład przedyskutować, czy porzucić stanowisko lub barykadę -- każdy mówca przemawiał do tłumu, starannie literując każde zdanie lub zdanie tego, co ma do powiedzenia, pauza na końcu każdego z~nich, aby każdy mógł powtórzyć mu dokładnie jego słowa, jako rodzaj ogromnego, zbiorowego, odbijającego się echem chóru. Kiedy pierwszy raz byłem świadkiem tego rodzaju wezwania i~powtórki na A16, założyłem, że było to spowodowane faktem, że przy całym wietrze i~hałasie otoczenia był to po prostu sposób na zapewnienie, że wszyscy usłyszą. W końcu zrozumiałem, że to w~najlepszym razie kwestia drugorzędna (a zresztą nie działało to zbyt dobrze). Rzeczywiście, wydawało się, że chodzi o upewnienie się, że każdy mówca doskonale zdaje sobie sprawę z~powagi tego, co mówi, często, mimo że dyskusja wydawała się znacznie bardziej czasochłonna, miało to odwrotny skutek, ponieważ każdy ważył swoje słowa tak ostrożnie i~wycinając wszystko, co obce, i, co może nawet ważniejsze, sprawiając, że cały proces jest w~jakiś sposób zbiorowy. W rzeczywistości przypominało to doświadczenie śpiewania sloganów, przynajmniej w~nowszych, bardziej anarchistycznych formach śpiewania, w~których każdy mógł rozpocząć nowy śpiew w~dowolnym momencie i~nagle zobaczyć ich słowa powtarzane przez setki, a nawet tysiące ludzi, tylko po to, by zniknąć i~zostać zastąpionym przez nową pieśń wygłoszoną przez innego aktywistę, gdzieś dalej, co z~kolei uczyniłoby ich ponownie częścią bezosobowego chóru. Tego rodzaju rzeczy bardzo kontrastują z~bardziej znanym, tradycyjnym stylem lidera śpiewu z~mikrofonem. W marszach anarchistów każdy może rozpocząć nową pieśń, a nawet ją wymyślić, a przypływy i~odpływy są nieustanne. Uważam, że efekty są dość głębokie. Durkheim (1912) pisał o wpływie rytualnych śpiewów i~pieśni jako zacierania się indywidualności. Sens tworzenia słów, których nie jest się autorem, jednocześnie z~setkami innych intonujących dokładnie to samo, dostarcza najbardziej bezpośredniego i~silnego doświadczenia społecznego: to ten moment, w~którym społeczeństwo, zwykle abstrakcja, jest rzeczywiście obecny dla swoich członków jako bezpośrednia konkretna rzeczywistość, której częścią jest ich ciało. Pieśni polityczne mają podobną jakość. Większość z~nich nie ma znanego pochodzenia -- jak żarty czy przysłowia -- po prostu jakoś istnieją, wywodząc się z~jakiegoś zbiorowego autorstwa. Dosłownie przemawiają przez ciebie. Sami Durkheim i~Mauss postrzegali ten rodzaj ewokacji ,,mechanicznej solidarności'' jako właściwy dla stosunkowo prymitywnych form społeczeństwa: Mauss był przerażony, gdy zobaczył podobne techniki stosowane w~celu politycznym w~ruchach faszystowskich swoich czasów (Gane 1992; zob. Bloch 1974).

 Z perspektywy Durkheima doświadczenie uczestniczenia w~autonomicznym marszu, w~którym każdy może spróbować rozpocząć pieśń lub zaimprowizować nową pieśń, a następnie, jeśli się przyjmie, doświadczyć własnej indywidualnej inicjatywy nagle staje się chwilą zbiorowego rozpadu indywidualności, jak wszyscy mówią zbiorowym głosem, a następnie podążają za tym, gdy inni zaczynają nowe, jest rodzajem demokratyzacji wrzenia.

 To prawda, że  wciąż zdarzają się wydarzenia z~prowadzącymi pieśni i~naśladowcami; ale przynajmniej w~nowszych stylach protestu prawie nigdy nie przybiera formy przywódcy z~megafonem. Na przykład noc przed wiecem przeciwko warsztatowi lub bankowi prawdopodobnie odbędzie się spotkanie, aby spróbować burzy mózgów odpowiednich do okazji, ale wyniki są zwykle rozdawane wszystkim na arkuszach. Taka postawa towarzyszy ogólnej niechęci do pieśni, które wydają się istnieć od zawsze, i~które wykazują męczący brak wyobraźni: zwłaszcza osławiony ,,hej ho, hej ho, [cokolwiek to jest] musi odejść'', które służył jako ogólny standard dla marszów protestacyjnych od co najmniej lat sześćdziesiątych. Bardzo stare piosenki, szczególnie te z~lat 20. i~30., są często popularne: nastoletni anarchiści pełniący czuwanie przy aresztowanych towarzyszach bez problemu śpiewają ,,Solidarność na zawsze''. Ale oprócz standardów, które wszyscy znają do stosowania w~określonych okolicznościach (,,Cały świat patrzy!'' lub ,,Czyje ulice? Nasze ulice!''), przyśpiewki są cenione za ich kreatywność. O ile istnieją specjaliści, każdy, kto pełni funkcję analogiczną do lidera z~mikrofonem, prawie zawsze robi się trochę śmieszny. Rewolucyjny Anarchistyczny Blok Klaunów w~Filadelfii, który wydawał się specjalizować w~tworzeniu głupich meta-pieśni (,,Pieśń w~trzech słowach! Pieśń w~trzech słowach!'') był tylko jednym szczególnie kapryśnym przykładem. Bardziej typowi są Radical Cheerleaders, eksperci od improwizacji wymyślnych pieśni lub jeszcze bardziej niedorzeczni Milionerzy.

 Wracając na chwilę do jednego z~wydarzeń, od którego rozpoczynam rozdział:

\medskip
\noindent \textit{ Niedziela, 10 grudnia 2000 roku}

\medskip
\noindent  Marsz Peltier Więziennej Solidarności, Manhattan
\medskip


 Trzydziestu lub czterdziestu z~nas w~końcu znajduje się po drugiej stronie ulicy od komisariatu, w~którym przetrzymywani są nasi przyjaciele, i~zaczyna śpiewać tradycyjne więzienne pieśni solidarnościowe, głównie po prostu ,,Pozwól im odejść, pozwól im odejść\ldots '' w~kółko. Wkrótce są one przeplatane okazjonalnymi pieśniami z~marszu, aby przełamać monotonię (,,Uwolnij Leonarda Peltiera'' i~tak dalej). Przedstawiciel prawny przychodzi odwiedzić więźniów, wchodzi na komisariat, ale szybko zostaje odprawiony. Donosi nam, że w~komisariacie wyraźnie słychać pieśń i~to naprawdę zaczyna denerwować gliniarzy. Uważa, że  są one również słyszalne dla więźniów, gdziekolwiek się znajdują, przypuszczalnie w~głębi stacji. Zachęcone, niektórzy proponują, abyśmy wytrzymali, dopóki więźniowie nie zostaną wypuszczeni lub przeniesieni do procesowania.

 Rezultat jest taki, że dwudziesto-, dwudziestopięcioosobowy tłum pozostaje, śpiewając, przez kilka godzin. Maratonowe śpiewanie tego rodzaju nie jest łatwe do wykonania. Musimy pracować na zmiany. W każdej chwili może dwie trzecie z~nas odpoczywa na werandach; ktoś kupuje wodę, ktoś inny pojawia się z~torbami z~bajglami wyciągniętymi ze śmietnika, które przyniesiono do IMC. Aktualni śpiewacy ustawili się w~linii nad krawężnikiem. Kilka zbiera materiały na instrumenty perkusyjne i~kończy, kucając w~kącie, z~założonymi maskami i~kapturami naciągniętymi dla maksymalnego efektu, tworząc improwizowane koło perkusyjne. Przynajmniej wtedy w~śpiewie pojawia się i~znika rytm.

 Pół godziny później, kiedy zaczyna mżyć, ,,Pozwól im odejść'' wyraźnie doprowadza gliny do szału, ale nas też trochę doprowadza do szału. Za linią śpiewaków tworzy się mały okrąg, skupiający się wokół trzech Radykalnych Cheerleaderek, aby improwizować nowe pieśni. Ktoś rzuci linę, ktoś inny wymyśli rymowankę, inni dodadzą więcej, pomogą edytować lub w~inny sposób wniosą wkład, wtedy ktoś przekaże to chórzystom lub po prostu dołączą do linijki. Pieśni stają się coraz bardziej specyficzne dla wydarzenia:

\noindent \textit{Jesteśmy mokrzy, zmęczeni i~chcemy sikać \newline Dosyć już dość! \newline Uwolnij naszych ludzi!}

 Wkrótce ludzie rzucają nazwy piosenek, a my zamieniamy je w~pieśni. Twinkie, która ma niezwykle donośny głos, okazuje się geniuszem w~tej grze. Potrafi nakręcić tekst prawie na wszystko.

-- A co powiesz na Ulicę Sezamkową?

 Zaczyna natychmiast:

\noindent \textit{Deszczowy dzień \newline Gliniarze zabrali \ldots  \newline naszych Przyjaciół \newline Czy możesz mi powiedzieć, gdzie mam się udać, \newline żebyśmy mogli \ldots  uwolnić? \newline Czy możesz mi powiedzieć, \newline  jak dostać się, jak dostać się na komisariat trzynasty? }

 Najbardziej rozbudowana była wersja ,,Dwunastu dni Bożego Narodzenia'' (,,W pierwszy dzień Bożego Narodzenia, państwo policyjne dało mi\ldots ''), która pojawiła się około piątego dnia przez kilka Radical Cheerleaders, które niedawno wróciły z~łazienki w~pobliskim sklepie, zauważyły, że nie sądzą, aby ten konkretny pomysł był w~najlepszym guście.

 To powinno wystarczyć, aby dać czytelnikom trochę posmaku takich rzeczy. Takie kręgi stają się małymi wersjami tego, co niektórzy brytyjscy aktywiści nazwali później ,,laboratoriami powstańczej wyobraźni''. Niemal niezmiennie są one zarówno punktami oparcia kreatywności, jak i~miejscami skrajnej komicznej autoironii. W rzeczywistości powiedziałbym, że jest to konsekwentna tendencja za każdym razem, gdy aktywiści zaczynają zbliżać się do źródeł kreatywności, wyobraźni, a nawet świętości, trzech rzeczy, które dla wielu anarchistów, jak podejrzewam, są w~dużej mierze nie do odróżnienia. Rebelianci w~Paryżu w~1968 roku, jak wie każdy anarchista, zażądali ,,władza w~ręce wyobraźni''. Powiedziałbym, że immanentną praktyką aktywistów jest teoria, że ostateczną formą władzy jest właśnie siła wyobraźni. To właśnie ta siła tworzy społeczność i~formę społeczną; doświadczenie wymyślania pieśni i~obserwowania, jak staje się kolektywnym projektem, staje się natychmiastowym doświadczeniem takiej mocy. Ale ta moc jest świętą siłą, która może być jedynie reprezentowana przez śmieszną autokpinę. Myślę, że będzie to jaśniejsze w~następnym rozdziale, w~którym omawiam rolę lalkarstwa i~aktywistycznego teatru ulicznego.

\section{Część III: Walka mitologiczna}

 W pierwszej części rozdziału zacząłem mówić o mitach, ale tak naprawdę nie rozwijałem tematu. W tej sekcji chciałbym dotrzymać obietnicy, mówiąc trochę o wojnie obrazów. Jak sugerowałem pod koniec ostatniego rozdziału, ta wojna obrazów -- zwłaszcza, gdy działa w~telewizji -- jest niezwykle ważna, ponieważ wydaje się głównym środkiem, o której milczą reguły zaangażowania, a zwłaszcza poziomy siły, które każdego Strona czuje, że mogą korzystać, są faktycznie zdeterminowane\footnote{Wiele argumentów zawartych w~tej części przyjąłem później w~eseju zatytułowanym ,,O fenomenologii gigantycznych lalek'', który pojawił się już w~zbiorze Możliwości w~2007 roku.}.

 To, co chcę wtedy zrobić, to spróbować zgłębić rodzaj symbolicznej lub, można nawet powiedzieć, mitologicznej wojny, która toczy się między aktywistami a policją, zwłaszcza, ale nie wyłącznie, za pośrednictwem korporacyjnych mediów\footnote{Zapożyczyłem zwrot ,,wojna mitologiczna'' od Micka Taussiga (2006).}. Jest to skomplikowana gra, ponieważ większość aktywistów, jak już wspomniałem, ma skrajnie ambiwalentne podejście do grania w~nią, a wielu całkowicie odmawia grania. Niemniej jednak ci, którzy grali, nie byli całkowicie nieskuteczni.

 W rzeczywistości pod pewnymi względami okazali się niezwykle skuteczni. Kampanie przeciwko IMF, WTO i~ogólnie przeciwko projektowi neoliberalnemu, jak zauważyłem wcześniej, były w~stanie zmienić warunki politycznego sporu z~niezwykłą szybkością. W przeddzień Seattle, w~1999 roku, decydenci w~Stanach Zjednoczonych niemal jednogłośnie zgodzili się, że coraz więcej ,,reform wolnorynkowych'' jest jedynym możliwym kierunkiem dla każdej gospodarki; na arenie międzynarodowej ,,konsensus waszyngtoński'', jak go nazywano, pozostał prawie całkowicie niekwestionowany, a politykę neoliberalną traktowano jako nieuniknione oblicze globalizacji. Mówiąc jako ktoś, kto zaangażował się w~ruch zaraz po Seattle, mogę zaświadczyć, że prawie nikt z~zaangażowanych nie wyobrażał sobie, że za zaledwie półtora roku ten ideologiczny aparat zostanie skutecznie roztrzaskany i~że nawet magazyny takie jak \textit{Time }i \textit{Newsweek }opublikują artykuły redakcyjne, w~których mówi się, że mamy rację. Przeważnie wyobrażaliśmy sobie, że zajmie to dekadę. (Oczywiście myśleliśmy również, że może to doprowadzić do głębokiej, rewolucyjnej zmiany społecznej: tak się nie stało). Oczywiście była to bardziej praca aktywistów z~Globalnego Południa niż tych w~Europie i~Ameryce Północnej, ale było to sam fakt, że ruch był w~rzeczywistości globalny, co czyniło go tak skutecznym.

 Mimo to, pomimo całej skuteczności ruchu w~przekazywaniu swojego negatywnego przesłania -- że neoliberalna polityka jest masowo destrukcyjna -- okazał się prawie całkowicie niezdolny do przekazania swojego pozytywnego przesłania, zwłaszcza jego wezwania do nowych form demokracji bezpośredniej. Były, jak już zauważyłem, problemy praktyczne: fakt, że reporterom i~kamerzystom nie wolno było uczestniczyć w~spotkaniach, na których faktycznie wypracowywano te nowe formy demokracji. Wiele z~tego miało też związek ze standardowymi konwencjami dziennikarskimi: standardowy podział na ,,spokojnych demonstrantów'' maszerujących ze znakami i~,,gwałtowną awangardę'' wybijającą szyby lub prowokującą policję, nie pozostawiał miejsca na nieskończenie skomplikowaną organizację prawdziwych akcji bezpośrednich, z~ich blokadami, grupami afinicji, blokadami, klastrami i~radami delegatów, a nawet, jeśli o to chodzi, banery i~teatr uliczny, to wszystko, co uczestnicy uważają za najbardziej ekscytujące i~inspirujące. Faktem jest, że nawet jeśli można sfotografować lub sfilmować spotkanie, to nie jest to zbyt ciekawa wizualna oprawa. Ze strony redaktorów i~wielu reporterów istnieje niechęć do wykorzystywania ich jako kanału dla idei ,,agresywnych'' (lub w~każdym razie nieautoryzowanych) grup. Faktem jest, że anarchiści i~aktywiści zorientowani na akcję bezpośrednią generalnie nie angażują się w~wymyślną autopromocję własnego procesu demokratycznego, ale są bardziej skłonni do pisania krytyki problemów wewnętrznych (rasizmu, seksizmu, klik, elitaryzmu) dla innych działaczy. Istnieje nieskończona liczba innych powodów. Ta książka, jak zauważyłem na początku, jest po części napisana jako próba zrekompensowania tego wszystkiego. Niezależnie od powodów, standardowa linia medialna -- że aktywiści globalizacji reprezentowali niespójny bełkot przyczyn, bez spójnej analizy; że anarchiści mają tendencję do bycia nihilistami, sprzeciwiającymi się prawie wszystkiemu bez wizji alternatywnego społeczeństwa -- ma tendencję do trzymania się. Nawet podczas działań wokół konwencji Demokratów i~Republikanów w~2000 roku, pierwotnie pomyślanych w~celu podważenia samej idei, że Stany Zjednoczone są społeczeństwem demokratycznym, żadne znane mi główne źródło medialne nie chciało ani nie było w~stanie poinformować swoich odbiorców, że to jest właśnie to, o czym miały być protesty.

 Co zatem przechodzi? Jakie obrazy lub pomysły zdołały przedostać się przez media i~w pewnym sensie uderzyły w~popularną wyobraźnię, a raczej wyobraźnię amerykańskiej widowni telewizyjnej? Chociaż nie przeprowadziłem żadnych rzeczywistych ankiet, nie trzeba poświęcać dużo czasu na monitorowanie doniesień, oglądanie filmów lub po prostu rozmawianie ze zwykłymi obywatelami, aby uzyskać sens odpowiedzi. Nawet jeśli ludzie nie wiedzą nic więcej o masowych mobilizacjach, takich jak Seattle, prawie na pewno wiedzą dwie rzeczy:

\noindent 1. Dotyczą kolorowych wielkich lalek.

\noindent 2. Wciągają protestujących w~czarne wybijanie szyb.

 W pewnym stopniu jest to tylko efekt telewizji: są to najskuteczniejsze wizualizacje, jakie zapewnia większość działań. Mimo to pozostawiają jedną z~pewnego rodzaju zgrabnej strukturalnej opozycji: z~jednej strony kolorowe gigantyczne ptaki z~papier-mâché, świnie i~politycy w~wizerunkach; z~drugiej anonimowi, zamaskowani anarchiści w~czerni wybijający okna. Jedna obejmuje spektakularne pokazy kapryśnej kreatywności, druga jest anonimowa, destrukcyjna i~śmiertelnie poważna.

 Można to postrzegać jako po prostu wizualną wersję pokojowej protestującej/agresywnej opozycji anarchistycznej, i~jest to z~pewnością do pewnego stopnia prawda. Mimo to myślę, że w~ciekawy sposób opozycja w~końcu przekazuje coś o tym, co takie działania starają się osiągnąć. Idea niszczenia własności jest często pojmowana, jak to ujęło wielu anarchistów z~Seattle, jako kwestia ,,złamania czaru'', przełamania transowego poczucia nieuchronności stworzonego przez kulturę konsumpcyjną. Słowami słynnego komunikatu N30 Black Bloc:

 Kiedy rozbijamy okno, staramy się zniszczyć cienką okleinę prawowitości, która otacza prawa własności prywatnej. Jednocześnie egzorcyzmujemy ten zestaw gwałtownych i~destrukcyjnych relacji społecznych, który jest przesiąknięty prawie wszystkim wokół nas. ,,Niszcząc'' własność prywatną, przekształcamy jej ograniczoną wartość wymienną w~rozszerzoną wartość użytkową. Witryna sklepu staje się otworem wentylacyjnym, wpuszczającym trochę świeżego powietrza w~przytłaczającą atmosferę sklepu (przynajmniej do czasu, gdy policja zdecyduje się zagazować pobliską blokadę drogową). Skrzynka na gazety staje się narzędziem do tworzenia takich otworów wentylacyjnych lub małą blokadą rekultywacji przestrzeni publicznej lub obiektem poprawiającym punkt obserwacyjny poprzez stanie na nim. Śmietnik staje się przeszkodą dla falangi buntujących się gliniarzy oraz źródłem ciepła i~światła. Fasada budynku staje się tablicą wiadomości, która zapisuje burzę idei na nowy świat.

 Po N30 wiele osób już nigdy nie zobaczy witryny sklepowej ani młotka w~ten sam sposób. Potencjalne zastosowania całego pejzażu miejskiego wzrosły tysiąckrotnie. Liczba wybitych okien blednie w~porównaniu z~liczbą złamanych zaklęć, zaklęć rzucanych przez korporacyjną hegemonię, by uśpić nas w~zapomnienie o wszelkiej przemocy popełnianej w~imię praw własności prywatnej i~całym potencjale społeczeństwa bez nich. Rozbite okna można zasłonić deskami (z jeszcze większą ilością odpadów z~naszych lasów) i~ewentualnie wymienić, ale miejmy nadzieję, że rozbicie założeń utrzyma się jeszcze przez jakiś czas (ACME Collective 1999).

 Te akty miały celowo stworzyć przekaz również w~mediach. W scenie w~anarchistycznym filmie, odpowiednio zatytułowanym ,,Breaking the Spell'', jeden z~anarchistów z~Seattle komentuje po obejrzeniu filmu \textit{60 Minutes}, w~którym przeprowadzono z~nim wywiad: ,,Spodziewaliśmy się, że \textit{60 Minutes}wywoła sensację z~powodu niszczeniu mienia. i~tego właśnie chcieliśmy, ponieważ uważamy, że niszczenie mienia jest całkiem sensacyjne''.

 Pytanie dotyczy zamierzonych odbiorców.

 Wybijanie okien i~malowanie sprayem to kwestia uchwycenia miejskiego krajobrazu pełnego niekończących się korporacyjnych fasad i~migających obrazów, które wydają się niezmienne, trwałe, monumentalne i~demonstrują, jak bardzo jest kruchy. Miało to być dosłowne rozbicie złudzeń. To zbezczeszczenie tego, co wydaje się monumentalne i~trwałe, rozbicie powierzchni Spektaklu. Z drugiej strony gigantyczne lalki są przeciwieństwem. Sprowadzają się do wzięcia najbardziej ulotnych materiałów -- pomysłów, papieru, drucianej siatki -- i~przekształcenia ich w~coś bardzo przypominającego pomnik, nawet jeśli są to jednocześnie śmieszne wizerunki. Można nawet powiedzieć, że lalki to kpina z~samej idei pomnika\footnote{Nie wymyśliłem tego wyrażenia, chociaż bardzo bym chciał. Jestem winny Ilanie Gershon.}, a ze wszystkiego, co reprezentują państwowe pomniki: niedostępność, monochromatyczna powaga, przede wszystkim trwałość, państwowa (ostatecznie nieco śmieszna) próba przekształcenia swoich zasad i~historii w~wieczne prawdy.

 Oczywiście dla samych aktywistów najważniejszy jest proces ich produkcji, który jest jednocześnie wspólnotowy, egalitarny i~ekspresyjny. Tworzenie lalek to duży wspólny projekt w~dniach lub tygodniach poprzedzających ważną akcję, a nawet paradę: zadania są zorganizowane tak, aby jak najwięcej osób mogło wziąć w~nich udział. Same obiekty mają być tymczasowe, nikt się nie spodziewa, że przetrwają do następnej wielkiej akcji.

\medskip
\noindent \textit{31 lipca 2000 roku, Filadelfia}

\medskip
\noindent [Z notatki terenowej po wizycie w~magazynie lalek]
\medskip

 Pytanie, które sobie zadaję, brzmi: dlaczego te rzeczy są nawet nazywane ,,marionetkami''? Zwykle myśli się o ,,lalkach'' jako postaciach, które poruszają się w~odpowiedzi na ruchy jakiegoś lalkarza. Większość z~nich ma niewiele, jeśli w~ogóle, ruchomych części. Przypominają raczej ruchome posągi, czasami noszone, czasami przewożone. W jakim więc sensie są ,,lalkami''?

 Lalki są niezwykle wizualne, duże, ale też delikatne i~efemeryczne. Zwykle rozpadają się po jednej akcji. To połączenie ogromnego rozmiaru i~lekkości, jak mi się wydaje, czyni je pomostem między słowami a rzeczywistością. Są punktem przejścia; reprezentują zdolność do urzeczywistniania idei i~przybierania solidnej formy, do przekształcania naszego spojrzenia na świat w~coś o równej fizycznej masie i~większej spektakularnej mocy, nawet niż machiny przemocy państwowej, które przeciwstawiają się temu. Pomysł, że są one przedłużeniem naszych umysłów, słowami, może pomóc wyjaśnić użycie terminu ,,marionetki''. Nie mogą się przemieszczać jako przedłużenie woli jakiejś osoby. Ale gdyby tak było, byłoby to w~pewnym stopniu sprzeczne z~naciskiem na kolektywną kreatywność. O ile są postaciami w~dramacie, jest to dramat z~autorem zbiorowym; o ile są manipulowane, to w~pewnym sensie wszyscy, w~procesjach, często przekazywanych od jednego działacza do drugiego. Przede wszystkim mają być emanacją zbiorowej wyobraźni. W związku z~tym, gdyby stały się one w~pełni trwałe lub w~pełni manipulowane przez pojedynczą osobę, byłoby sprzeczne z~tym celem.

 Poczucie, że jednocześnie uczestniczą i~obalają ideę pomnika, staje się szczególnie widoczne w~pewnych momentach: na przykład podczas wizyty Busha w~Wielkiej Brytanii w~2003 roku, kiedy brytyjscy aktywiści w~każdym mieście stawiali gigantyczne marionetki Busha, a potem rytualnie ściągali je w~dół; lub podczas konwencji republikanów w~Nowym Jorku w~2004 roku, kiedy gigantyczny smok-lalkowy został przywieziony bezpośrednio przed stadion, na którym odbywał się konwent i~został podpalony.

 Obrazy są próbą ogarnięcia pewnego rodzaju uniwersum, zarówno tego, za czym opowiadają się aktywiści, jak i~przeciwko czemu się sprzeciwiają. Z jednej strony mamy Wielką Świnię, która reprezentuje Bank Światowy, z~drugiej Wielką Kukiełkę Wyzwolenia (której ramiona mogą zablokować całą autostradę), zagrożone gatunki (słynne żółwie), męczenników z~Haymarket, Statuę Wolności, różnych pogańskich bogów. Z drugiej strony, prawdopodobnie będziesz miał tyle samo szyderczych podobizn: jak marionetka kontroli korporacji podczas protestów na konwencie demokratów w~Los Angeles w~2000 roku, marionetka, która z~kolei operowała mniejszymi marionetkami Busha i~Gore'a, gigantyczna marionetka policjanta, która strzelała imitacją gazu pieprzowego i~tak dalej. Podczas akcji bogowie wtapiają się w~kostiumy. Większość głównych akcji ma określony kostium tematyczny lub zwierzę totemiczne: żółwie w~Seattle, rekiny i~ptaki podczas wiosennych protestów IMF, szkielety w~Filadelfii (albo byłyby; w~rzeczywistości marionetki zostały zniszczone, zanim wyszły na ulice), karibu podczas inauguracji w~2001 roku -- kostiumy, które zwykle rozdawano masowo każdemu, kto chciał je wziąć. Zwykle skupiano się w~szerszym bloku karnawałowym, zwykle z~jakimś szerokim motywem cyrkowym, pewnym półcieniem monocyklistów, akordeonistów, klaunów i~szczudlarzy, którzy zdają się towarzyszyć każdej akcji, i~byli otoczeni marionetkami, często obok anarchistycznych orkiestr marszowych (Hungry March Band, Infernal Noize Brigade) lub Radykalnych Cheerleaderek. Kiedy Tony Blair oświadczył podczas szczytu w~2002 roku, że nie będzie miał wpływu na ,,jakiś anarchistyczny cyrk wędrowny'', wielu anarchistów uznało to zdanie za całkiem odpowiednie. Rzeczywiście jest coś w~idei cyrku, co bezpośrednio odwołuje się do anarchistycznej wrażliwości (a w~rzeczywistości istnieje wiele rzeczywistych anarchistycznych cyrków objazdowych w~Ameryce): nie tylko dziwactwo i~wyzwanie wszystkich akceptowanych pojęć możliwości, ale także, myślę, że to, że cyrk to zbiór skrajnych indywidualistów, którzy mimo wszystko angażują się w~przedsięwzięcie na wskroś kooperatywne. W każdym razie takie zespoły lalek, cyrków i~teatrów ulicznych często rzucają się w~tę i~z powrotem podczas ważnej akcji, aby ożywić słabnące duchy lub ogólnie zabawiać żołnierzy. Co więcej, być może specjalizują się w~rozbrajaniu i~deeskalacji sytuacji, które wyglądają, jakby mogły stać się brutalne. Pod nieobecność marszałków często to drużyny marionetek kończą jako de facto żołnierze sił pokojowych:jak organizatorzy starali się podkreślać prasie, na przykład, kiedy wszyscy ,,lalkarze'' zostali aresztowani jeszcze przed rozpoczęciem akcji w~Filadelfii.

 Oto opis typowej ,,interwencji marionetkowej'' podczas akcji w~Seattle:

-- Ludzie mieli połączone ramiona -- mówi Zimmerman. -- Policja już ich pobiła i~spryskała pieprzem i~zagrozili, że wrócą za pięć minut, aby ponownie ich zaatakować. 

Ale protestujący trzymali się linii, łącząc ramiona i~płacząc, oślepieni gazem pieprzowym. Burger, Zimmerman i~ich przyjaciele pojawili się na szczudłach, z~klaunami, 40-metrową lalką i~tancerką brzucha. Szli wzdłuż linii, prowadząc protestujących w~pieśni. Kiedy furgonetka ochrony wróciła, postawili gigantyczną marionetkę na jej drodze. W jakiś sposób ten pstrokaty cyrk rozproszył sytuację. 

-- Nie mogli zmusić się do zaatakowania tej grupy ludzi, którzy teraz śpiewali piosenki -- mówi Zimmerman. 

Wstrzykiwanie humoru i~celebracji w~ponurą sytuację, jak mówi, jest esencją interwencji marionetek\footnote{Z ,,Puppet Masters: Paper Hand Puppet Intervention przenosi swój teatr polityczny z~powrotem do Chapel Hill'' (Independent Online, 8 sierpnia 2001),\url{http://indyweek.com/durham/2001-08-08/ae.html}), dostęp 13 lutego 2003 roku}. 

 W poprzednim rozdziale opisałem już łączenie technik clowningu i~akcji bezpośrednich; lalki mogą być postrzegane jako część tego samego zjawiska, w~gruncie rzeczy próba uczynienia boskiego komizmu, ponieważ są one również, paradoksalnie, jednocześnie śmieszne, ale także, w~pewnym sensie, głębokie.

-- Lalki nie są urocze, jak muppety -- pisze Peter Schumann z~Bread and Puppet Theatre (jest to grupa, która jako pierwsza wprowadziła gigantyczne lalki do amerykańskiej polityki w~latach 60.). -- Lalki to kukły, bogowie i~znaczące stworzenia. 

Jednocześnie są oczywiście głupimi, niemądrymi, niedorzecznymi bogami, sposobami na jednoczesne przejęcie władzy do tworzenia bogów i~wyśmiewanie się z~tej mocy. Podobny impuls można znaleźć prawie zawsze, gdy w~tak radykalnych ruchach zbliżamy się do mitycznego lub głęboko znaczącego: rodzaj śmiesznej autoironii, która jednak nie ma na celu całkowitego podważenia powagi i~znaczenia tego, co się twierdzi, ale raczej sugerować ostateczne uznanie, że tylko dlatego, że bogowie są ludzkimi wytworami, wciąż są realni. Widzisz to w~pismach prymitywistów, którzy świadomie tworzą nowe mity o Ogrodzie Eden, upadku (to wszystko wina rolnictwa) i~nieuchronności upadku przemysłu, ale jednocześnie wydają się twierdzić, że chcą śmierci większości ludzi na ziemi, hamują na sugestię, że naprawdę to robią. Tworzą świadomie absurdalne, ultraradykalne propozycje, jednocześnie traktując je jako ostateczne prawdy o świecie wyobcowania. Widać to u pogan, najbardziej aktywnych świadomie religijnych elementów ruchu akcji bezpośredniej, którzy są całkiem zdolni do wykonywania ekstrawaganckich satyr na pogańskie rytuały, które mimo to postrzegają jako prawdziwe, skuteczne rytuały odzwierciedlające najgłębsze możliwe duchowe prawdy o świat. 

 Jednak dla obecnych celów liczy się nie tyle niejawna teoria kreatywności (choć jest to interesujące), lecz to, jak to wszystko działa jako alternatywa dla bardziej konwencjonalnych podejść, takich jak ustalanie ujednoliconych dokumentów stanowiska lub projektowanie wydarzeń wokół uprzedzeń prasy. Najwyraźniej ci, którzy rozbijają okno Starbucksa lub tworzą gigantyczną lalkę, chcą, aby te obrazy były rozpowszechniane przez media, zarówno korporacyjne, jak i~alternatywne. Zakładają też, że jest w~nich coś, co uderzy w~ludową wyobraźnię, aby stworzyć mit, jak włoscy teoretycy Ya Basta! lubili to ująć (Bui 2005) i~ponieść znaczenia poza wszelką sceptyczną lub wrogą glosę, jaką mogłyby na nią nałożyć korporacyjne media: mit o bezbronności Spektaklu, o możliwości kolektywnego tworzenia nowych form znaczeń. W tym byli dość skuteczni. Powodem, dla którego trudno to dostrzec, jest to, że ,,społeczeństwo'' w~Ameryce jest raczej niejasnym i~obciążonym terminem. W kontekście politycznym wydaje się przywoływać obraz zbioru białych par z~klasy średniej, w~dużej mierze z~przedmieść, po czterdziestce w~domu oglądających telewizję. Są one również postrzegane jako z~grubsza odpowiadające najważniejszym ,,niezdecydowanym'' lub wahającym się wyborcom w~większości kampanii wyborczych. Nie buduje się jednak radykalnego ruchu, usiłując uspokoić centrum. Robi się to przede wszystkim, próbując odwoływać się do tych, którzy już są źli, wyobcowani, uciskani lub zmarginalizowani, którzy tak naprawdę nie muszą być przekonani, że jest coś głęboko nie tak ze sposobem, w~jaki rządzi świat, ale raczej trzeba pokazać jakiś znak, że system jest podatny, że mogą zrobić coś skutecznego, a nie samobójczego. W tym sensie można powiedzieć, że anarchiści naprawdę ,,porywają'' lub ,,przejmują'' media, aby przekazać wiadomości nieoczekiwanemu okręgowi wyborczemu. Jednocześnie jednak policjanci grają w~bardzo podobną grę obrazów, z~nieskończenie większymi zasobami i, jak zobaczymy, z~dużą dozą bezwzględnego cynizmu, aby wywołać alarm wśród tej wyobrażonej ,,publiczności'', którą anarchiści w~dużej mierze postanowili zignorować.

\medskip
\noindent WOJNA MITOLOGICZNA Z PERSPEKTYWY POLICJI
\medskip

 Miesiące bezpośrednio po akcjach w~Seattle były świadkami skoordynowanych i~trwałych wysiłków rządu, a zwłaszcza funkcjonariuszy organów ścigania, mających na celu znalezienie sposobu usprawiedliwienia użycia przemocy i~środków zapobiegawczych przeciwko temu, co wskazywało na to, że staje się pączkującym ruchem społecznym. Towarzyszyło temu eskalacja stosowania agresywnej taktyki policyjnej, która początkowo koncentrowała się znacznie bardziej na pacyfistach i~taktykach oczywiście pokojowych, takich jak blokady i~okupacje, niż na kimkolwiek zaangażowanym w~działania, które można by określić jako przestępcze. Chociaż nie jestem wtajemniczony w~rozumowanie przyjęte przez policję -- lub jakiekolwiek władze, które przyczyniły się do planowania bezpieczeństwa na międzynarodowych szczytach w~tym okresie -- mogę powiedzieć, że nie jest całkowicie zaskakujące, że tak zrobili\footnote{Najlepsze dostępne dowody wskazują na rodzaj zbiegu policyjnych grup wywiadowczych różnego rodzaju, Secret Service i~innych organów federalnych oraz różnych prywatnych firm konsultingowych i~prawicowych think tanków. Jednak nawet spekulowanie jest niebezpieczne, ponieważ pomimo faktu, że wszyscy są świadomi, że władze opracowują uzgodnioną politykę w~celu radzenia sobie z~tymi, których uważają za zagrożenia dla bezpieczeństwa, każdy, kto wyraźnie spekuluje na temat takich spraw, ryzykuje, że zostanie natychmiast nazwany ,, teoretykiem spiskowym''.}. Próbując przerwać ważne negocjacje handlowe, spotkania IMF, szczyty G8, Światowe Forum Ekonomiczne i~tym podobne, anarchiści celowo próbują spowodować skrajne niedogodności i~irytację dla niektórych z~najbogatszych i~najpotężniejszych ludzi na świecie. W ramach strategicznego planu mającego na celu udaremnienie niektórych z~najbardziej cenionych projektów i~planów tej elity systematycznie psują najważniejsze imprezy, spotkania towarzyskie i~autocelebracyjne rytuały międzynarodowej elity. Być może robią to w~taki sposób, aby nie narażać ich fizycznie, a właściwie nikogo nie skrzywdzić (a od Seattle może nie udało im się faktycznie zamknąć żadnych spotkań), ale udało im się zmienić te spotkania w~koszmary, tak zdominowane przez wymyślne środki bezpieczeństwa, że nie są one w~żadnym sensie uroczystościami. Rzeczywiście naiwnym byłoby sądzić, że w~takim przypadku to właśnie legalność sprawy okazałaby się najważniejszym czynnikiem. Odkąd Madeleine Albright zadzwoniła do gubernatora podczas działań WTO w~Seattle, problem nie polegał na tym, czy użyć przemocy wobec pokojowych aktywistów, ale jak to usprawiedliwić. 

 Być może nie wiadomo, jakie polityki zostały opracowane, a nawet przez kogo, ale łatwo jest zaobserwować, co się wydarzyło. Jeśli przyjrzeć się głównym działaniom, które nastąpiły bezpośrednio po Seattle, można zauważyć, że prawie w~każdym przypadku policja przyjęła niezwykle podobne podejście. Zawsze widzimy uderzenia wyprzedzające, uzasadnione twierdzeniami o groźbach użycia przemocy przez protestujących, groźby, która nigdy się nie zmaterializowała. Oto trzy typowe incydenty:

\medskip
\noindent\textit{Kwiecień 2000, Waszyngton, DC}
\medskip

 Na kilka godzin przed rozpoczęciem protestów przeciwko IMF i~Bankowi Światowemu policja przejmuje centrum konwergencji aktywistów, jedno z~głównych miejsc wytwarzania i~przechowywania marionetek i~transparentów, które będą wykorzystywane w~proteście. Szef Ramsey głośno twierdzi, że odkrył wewnątrz warsztaty do produkcji koktajli Mołotowa i~domowej roboty gazów pieprzowych. Policja DC przyznała później, że taki warsztat nie istniał (naprawdę znaleźli rozcieńczalnik do farb używany w~projektach artystycznych i~paprykę do produkcji gazpacho); jednak centrum konwergencji pozostaje zamknięte, a duża część sztuki i~lalek znajdujących się w~środku jest zawłaszczona.

\medskip
\noindent \textit{Lipiec 2000, Minneapolis}
\medskip

 Na kilka dni przed zaplanowanym protestem przeciwko Międzynarodowemu Towarzystwu Genetyków Zwierząt lokalna policja twierdzi, że aktywiści zdetonowali bombę cyjankową w~lokalnym McDonald's i~mogli mieć w~rękach skradzione materiały wybuchowe. Następnego dnia DEA napada na dom używany przez organizatorów, wyciąga ze środka pobitych i~zakrwawionych aktywistów, zawłaszcza ich komputery i~stosy materiałów informacyjnych. Rzecznicy policji przyznają później, że nigdy tak naprawdę nie było bomby cyjankowej (w rzeczywistości była to bomba dymna) i~nigdy nie mieli żadnych dowodów ani powodów, by sądzić, że aktywiści rzeczywiście posiadali materiały wybuchowe.

\medskip
\noindent \textit{Sierpień 2000, Filadelfia}
\medskip

 Na godziny przed rozpoczęciem protestów przeciwko konwencji republikańskiej policja, twierdząc, że działa na podstawie cynku, otacza i~najeżdża magazyn, w~którym przygotowywana jest sztuka, transparenty i~marionetki użyte do akcji, ostatecznie aresztując wszystkich siedemdziesięciu pięciu aktywistów w~środku. Komendant Timoney twierdzi na konferencji prasowej, że jego ludzie odkryli w~magazynie materiały wybuchowe C4 i~balony wodne pełne kwasu solnego. Rzecznicy policji przyznają później, że tak naprawdę nie znaleziono materiałów wybuchowych ani kwasu; aresztowani jednak nie zostają zwolnieni. Wszystkie marionetki, transparenty, sztuka i~literatura, które mają być użyte w~proteście, są systematycznie niszczone.

 Chociaż możliwe jest, że mamy do czynienia z~niezwykłą serią uczciwych błędów, wygląda to bardziej jak seria ataków na materiały, które aktywiści zamierzali wykorzystać, aby dotrzeć ze swoim przesłaniem do opinii publicznej, próby przywłaszczenia lub zniszczenia środków, przez którą protestujący zamierzali umieścić obrazy w~mediach, a nawet więcej próbować zastąpić obrazy sztuki i~marionetek obrazami bomb i~cyjanku.

 Z pewnością tak to zinterpretowali aktywiści. W czasach Quebecu jedna z~największych dyskusji przed każdą nową mobilizacją stała się miejscem ukrycia gigantycznych marionetek. Po prostu zakładano, że policja ich zaatakuje. Pełna kulminacja tego modelu represji nastąpiła dopiero podczas posiedzeń ustawy o strefie wolnego handlu obu Ameryk w~Miami w~2003 roku, kiedy to rada miasta Miami faktycznie uchwaliła ustawę, która delegalizowała wystawianie marionetek podczas szczytu (rzekomo dlatego, że mogły one zostać wykorzystywane do ukrywania broni), a strategia policji składała się prawie wyłącznie z~wyprzedzających uderzeń przeciwko działaczom, z~których setki zostało aresztowanych i~oskarżonych o planowanie -- ale nigdy w~rzeczywistości o wykonywanie -- niewyobrażalnych działań. W rezultacie Czarny Blok w~Miami faktycznie poświęcił większość czasu i~energii na ochronę marionetek, kiedy w~końcu pojawiły się na ulicach. Według relacji jednego z~naocznych świadków, po tym, jak policja rozgromiła protestujących z~Seaside Plaza, zmuszając ich do porzucenia swoich marionetek, funkcjonariusze spędzili około pół godziny systematycznie atakując je i~niszcząc: strzelając, kopiąc, tnąc i~rozrywając szczątki z~papier-mâché; jeden nawet umieścił gigantyczną marionetkę w~swoim radiowozie z~głową wystającą i~jechał tak, aby rozbić ją o każdy znak i~słup uliczny w~zasięgu wzroku.

\medskip
\noindent DLACZEGO POLICJANCI NIENAWIDZĄ LALKI?
\medskip

 Wydaje się, że ze strony policji amerykańskiej istnieje szczególna niechęć do gigantycznych marionetek. W rzeczywistości policja często wydaje się nienawidzić marionetek o wiele bardziej niż Czarnego Bloku. Wielu aktywistów zastanawiało się, dlaczego tak się dzieje. Nie pomaga pytanie samych policjantów. Niemal zawsze będą mówić to samo: że takie przedmioty są niebezpieczne, nie ma sposobu, aby dowiedzieć się, co tak naprawdę jest w~nich, skąd wiemy, że nie są używane do ukrywania bomb lub broni? Albo, że drewniane ramy mogą służyć jako pałki, a nawet tarany? Wydaje się, że trudno jest przypisywać takie roszczenia w~przypadku takim jak Miami, kiedy nawet po tym, jak Rada Miejska próbowała zakazać wystawiania lalek na tej podstawie, policja w~końcu miała okazję dosłownie je rozebrać na części. Nie strzelasz plastikowymi kulami w~obiekt, o którym sądzisz, że może być bombą.

 W pewnym momencie zapytałem kilku przyjaciół-aktywistów o ich opinie na ten temat i~odkryłem, że prawie wszyscy też o tym myśleli:

\noindent \textit{David Corston-Knowles}: Musisz pamiętać, że są to ludzie wyszkoleni w~paranoi. Muszą zadać sobie pytanie, czy coś tak wielkiego i~nieodgadnionego może zawierać materiały wybuchowe, nawet jeśli wydaje się to absurdalne z~perspektywy pokojowego demonstranta. Policja postrzega swoją pracę nie tylko jako egzekwowanie prawa, ale także jako utrzymanie porządku. i~traktują to bardzo osobiście. Gigantyczne pokazy i~gigantyczne lalki nie są uporządkowane. Chodzi o stworzenie czegoś, innego społeczeństwa, innego sposobu patrzenia na rzeczy, a kreatywność jest zasadniczo sprzeczna ze status quo.

\noindent \textit{Daniel Lang}: Cóż, jedna z~teorii głosi, że gliniarze po prostu nie lubią, gdy ktoś robi większe show. Przecież zwykle \textit{oni są }widowiskiem: mają niebieskie mundury, mają helikoptery, konie i~rzędy błyszczących motocykli. Więc może po prostu nie lubią, gdy ktoś kradnie show, wymyślając coś jeszcze większego i~jeszcze bardziej efektownego wizualnie. Chcą wyeliminować konkurencję.

\noindent \textit{Yvonne Liu}: To dlatego, że są tak duże. Policjanci nie lubią rzeczy, które nad nimi górują. Dlatego lubią być na koniach. Plus kukiełki są głupie, okrągłe i~zniekształcone. Zauważ, ile gliniarze zawsze muszą utrzymywać proste linie? Stoją w~prostych liniach, zawsze starają się, abyś stał w~prostych liniach. Myślę, że niekształtne rzeczy ich obrażają.

\noindent \textit{Max Uhlenbeck}: Oczywiście nie znoszą, gdy im przypomina się, że sami są marionetkami.

 Tego rodzaju spekulacje można by mnożyć w~nieskończoność. Myślę, że to pytanie jest właściwie bardzo ważne i~wrócę do niego za chwilę, ale na razie pozwólcie, że będę kontynuował o miesiącach, które nastąpiły po Seattle.

 W tym okresie zaczęły pojawiać się coraz bardziej dziwaczne relacje o tym, co wydarzyło się w~Seattle. Podczas samych protestów WTO nikt, łącznie z~policją, nie twierdził, że aktywiści zrobili coś bardziej bojowego niż rozbicie szyby. W prasie nie było żadnych twierdzeń, że protestujący atakują policję, a nawet kogokolwiek innego; w~rzeczywistości jedyna interpersonalna przemoc, o której wiem, ze strony aktywistów, miała miejsce, gdy wielu samozwańczych ,,policjantów pokojowych'' próbowało powstrzymać niektórych anarchistów z~Czarnego Bloku przed wybiciem okien, i~w niektórych przypadkach kończyło się fizycznym napaścią na nich ( anarchiści, którzy w~większości byli dość wybredni w~swoim oddaniu niestosowaniu przemocy, odmówili oddania ciosu). Jednak nie więcej niż trzy miesiące później, \textit{Boston Herald} napisał, że oficerowie z~Seattle przyjechali, żeby przeszkolić lokalną policję, jak radzić sobie z~takykami Seattle, takimi jak atakowanie policji ,,kawałkami betonu, broń BB, rakiety na ręku i~sikawki wypełnione wybielaczem i~uryną''\footnote{Jose Martinez, ,,Policja przygotowuje się do protestów przeciwko konferencji biotechnologicznej w~Hynes'', Boston Herald, sobota, 4 marca 2000 roku}. Kilka miesięcy później, kiedy reporterka \textit{New York Times}, Nichole Christian, najwyraźniej opierając się na źródłach policyjnych w~Detroit, stwierdziła, że  demonstranci z~Seattle ,,rzucali koktajlami Mołotowa, kamieniami i~ekskrementami w~delegatów i~funkcjonariuszy policji'', nowojorski DAN zorganizował protest przed ich biurami, domagając się wyjaśnień, a gazeta została faktycznie zmuszona do wycofania się, przyznając, że władze Seattle potwierdziły, że nie rzucano w~ludzi żadnymi przedmiotami\footnote{\textit{New York Times}, 6 czerwca, Corrections, A2. Oryginalna historia została znacząco zatytułowana ,,Detroit Defends Get-Tough Stance'' autorstwa Nichole Christian, 4 czerwca 2000 roku, A6. Korekta brzmi: ,,Artykuł w~niedzielę o planach protestów w~Detroit i~Windsor w~Ontario przeciwko międzyamerykańskiemu spotkaniu odbywającemu się w~Windsor do dziś odnosił się błędnie do protestów w~listopadzie ubiegłego roku na spotkaniu Światowej Organizacji Handlu w~Seattle. Protesty w~Seattle miały przede wszystkim charakter pokojowy. Tamtejsze władze stwierdziły, że wszelkie rzucane przedmioty były wymierzone w~mienie, a nie w~ludzi. Żaden protestujący nie został oskarżony o rzucanie przedmiotami, w~tym kamieniami i~koktajlami Mołotowa, w~delegatów lub policję''. }. Mimo to relacja Christiana wydaje się być kanoniczna. Za każdym razem, gdy dochodzi do nowej mobilizacji, w~lokalnych gazetach niezmiennie pojawiają się historie z~tą samą listą ,,taktyk Seattle'' -- lista, która wydaje się również zapisana w~podręcznikach szkoleniowych rozprowadzanych wśród gliniarzy ulicznych. Na przykład przed Szczytem Ameryk w~Miami w~2003 roku okólniki rozsyłane do lokalnych biznesmenów i~grup obywatelskich, oparte na informacjach od ,,konsultantów ds. bezpieczeństwa'', wymieniały każdą z~tych ,,taktyk Seattle'' jako to, czego powinni się spodziewać na ulicach przybyli anarchiści:

\noindent \textit{Rakiety na nadgarstek}- większe procy typu myśliwego, których używają do strzelania stalowymi kulkami łożyskowymi lub dużymi bełtami. Bardzo niebezpieczna i~zabójcza broń.

\noindent \textit{Koktajl Mołotowa} -- wiele z~nich zostało rzuconych w~Seattle i~Quebecu i~spowodowało rozległe zniszczenia.

\noindent \textit{Łomy} -- by rozbijać szyby, samochody itp. Podważają również krawężniki, a następnie rozbijają beton na kawałki, którymi mogą rzucać w~policjantów. Było tak robione w~Seattle.

\noindent \textit{Pistolety na wodę} -- wypełniony kwasem lub moczem\footnote{Dokument ten został wówczas przepisany i~szeroko rozpowszechniony wśród listserwerów aktywistów. Według jednego z~artykułów opublikowanych w~Miami Herald (,,Protestujący w~handlu podchodzą do interesów, ostrzega analityk'', Joan Fleischman, 1 października 2003 roku), pochodzi on od ,,emerytowanego agenta DEA Toma Casha, 63 lata, obecnie starszego dyrektora zarządzającego Kroll Inc. międzynarodowa firma zajmująca się bezpieczeństwem i~doradztwem biznesowym''. Z kolei Cash twierdził, że czerpie swoje informacje ze źródeł ,,wywiadu policyjnego''.}.

Upewniając się, że kiedy protesty się rozpoczęły, większość śródmieścia Miami była zamknięta i~opuszczona. Do dziś większość dziennikarzy reaguje z~niedowierzaniem, gdy ktoś zwraca uwagę, że przynajmniej w~Stanach Zjednoczonych nikt podczas protestu globalizacyjnego nigdy nie wyrzucił koktajlu Mołotowa\footnote{Kanada jest, jak widzieliśmy, dość inna pod tym względem; a przynajmniej jest Quebec.}.

 Niektórzy urzędnicy policji stali się znani wśród aktywistów ze względu na ich gotycką wyobraźnię. John Timoney, szef policji w~Filadelfii podczas konwencji republikanów w~2000 roku oraz w~Miami podczas szczytu Strefy Wolnego Handlu obu Ameryk w~2003 roku, lubił szczególnie ponure zarzuty. Na przykład podczas Filadelfii wydawało się, że obowiązywała polityka ogłaszania jednego szczególnie oburzającego roszczenia każdego dnia. Początkowo policja twierdziła, że  złapała furgonetkę używaną do transportu jadowitych węży i~gadów, które aktywiści planowali uwolnić wśród obywateli; potem, że oficer był hospitalizowany po ochlapaniu twarzy kwasem; następnie, że policja odkryła ,,bomby suchego lodu'' podłożone wokół miasta. Przez większość tego czasu pracowałem z~aktywistycznymi zespołami medialnymi i~sam przekonałem się, ile czasu musieli poświęcić osoby pracujące przy telefonach, próbując dowiedzieć się, o czym u licha mówią reporterzy, ponieważ uważano je za znacznie ważniejsze historie niż: powiedzmy, masowe aresztowanie w~magazynie lalek i~zniszczenie lalek. W każdym przypadku policja albo była później zmuszona do wycofania swoich roszczeń, albo przynajmniej do zaprzestania powtarzania. W pierwszym przypadku pojazd okazał się furgonetką ze sklepem zoologicznym; w~drugim okazało się, że oficer został zbryzgany czerwoną farbą; w~trzecim wydaje się, że bomby z~suchym lodem były czymś, co policja wychwyciła, przeglądając osławioną \textit{Anarchist Cookbook}, wydaną w~latach 70. i~nie miały żadnej podstawy. Podczas FTAA w~Miami krążyły podobne, przerażające doniesienia o rannych policjantach, różnego rodzaju pociskach, a przede wszystkim o aktywistach atakujących oddziały Timoneya przy użyciu wszelkiego rodzaju płynów ustrojowych. Takie zarzuty niezmiennie trafiają na chwytliwe nagłówki po pierwszym ogłoszeniu lub sensacyjne doniesienia w~wieczornych wiadomościach. Kiedy okazują się fałszywe, to samo w~sobie prawie nigdy nie zasługuje na opowiedzenie lub poprawienie. Innymi słowy, rzecznicy policji wydają się świadomi i~w pełni wykorzystując fakt, że amerykańscy dziennikarze zgłaszają prawie wszystko, co mówią na briefingach, jako zwykły fakt i~że rzadko są zainteresowani publikowaniem historii wyjaśniających, że coś nie było prawdą. W przypadku Timoneya wzorzec (przynajmniej tak jak ja to zaobserwowałem) polegał na tym, że dziennikarze najpierw witali aktywistów z~niedowierzaniem, a nawet drwinami, gdy sugerowali, że policja może kłamać; potem, widząc, że takie twierdzenia wielokrotnie okazywały się fałszywe, po prostu przestać je zgłaszać, nigdy nie przyznając, że zostali oszukani.

 Przykład z~Bostonu jest szczególnie uderzający, ponieważ sugeruje, że w~niektórych przypadkach przełożeni policji mogą nawet nie interesować się przede wszystkim wywieraniem wpływu na media: przynajmniej w~niektórych przypadkach główną grupą docelową jest sama policja. Trudno wyobrazić sobie inny powód, dla którego można by nakazywać glinom ulicznym, aby brali udział w~szkoleniach, w~których uczy się ich oczekiwać skrajnie brutalnej taktyki, o której osoby dowodzące muszą mieć świadomość, że nigdy nie zostały zastosowane. Oczywiście tylko do tej pory można uogólniać z~jednego przypadku. Można się tylko zastanawiać, czy to, co wydarzyło się w~Bostonie, było odosobnionym wydarzeniem, czy przykładem znacznie bardziej powszechnej praktyki, która zwykle nie jest zgłaszana. Podobnie bardzo trudno jest wiedzieć, kim właściwie są te tak często cytowane źródła ,,policyjnego wywiadu''. Tutaj wydaje się, że wkraczamy w~mroczną strefę, w~której informacje są gromadzone, konstruowane, rozpowszechniane i~przekazywane między różnymi siłami zadaniowymi policji federalnej, prywatnymi agencjami bezpieczeństwa i~prawicowymi think tankami, z~których wielu może być przekonanych, że przynajmniej niektóre z~tych historii są prawdziwe, ponieważ większość informacji czerpią od innych. Jednak tutaj również podejrzewam, że prawdopodobnie pierwszym zmartwieniem tych, którzy opowiadają ponure historie o wybielaczu i~moczu, jest po prostu zmobilizowanie funkcjonariuszy. Jak odkryli dowódcy w~Seattle, policjanci, którzy są przyzwyczajeni do uważania się za obrońców społeczeństwa, często będą się wahać, a przynajmniej wahać, gdy otrzymają rozkaz szarży przeciwko grupie szesnastolatków, które w~oczywisty sposób nie dopuszczają się przemocy. Wydaje się to trudne, na przykład zrozumieć osobliwą obsesję na punkcie płynów ustrojowych w~wielu z~tych raportów, z~wyjątkiem części świadomej kampanii mającej na celu odwołanie się do wrażliwości policji. Z pewnością nie ma to nic wspólnego z~aktywistami.

 Wobec niekończących się oskarżeń o rzucanie i~strzelanie moczem i~kałem, działacze, których znam, są w~większości zdumieni. Niektórzy wstępnie sugerują, że być może historie sięgają czasów, gdy policja oblegała skłotowane budynki, a wiadra pełne ludzkich odchodów były jednymi z~niewielu dostępnych pocisków nieśmiercionośnych. Większość jednak nie ma pojęcia. Z pewnością nigdy nie słyszałem, żeby ktoś faktycznie wnosił takie przedmioty na akcję. Jednak oskarżenie raz po raz są powtarzane. Jeśli na przykład policja aresztuje aktywistów w~atakach prewencyjnych, można być prawie pewnym, że podłożyli im dowody, jeśli ogłoszą, że podejrzani zostali wykryci z~,,łomami i~fiolkami pełnymi moczu''. Wiadomo, że policja na konferencjach prasowych wystawiała takie worki z~ekskrementami lub słoiki z~moczem, które według nich miały być w~nią rzucane (powodując, że aktywiści zastanawiają się, skąd właściwie dostali te rzeczy). Twierdzenia zdają się odzwierciedlać powtarzane w~nieskończoność twierdzenie, że podczas wojny w~Wietnamie protestujący zwykli ,,pluć na każdego, kto nosi mundur'', oraz, oczywiście, szersze pojęcie, że najlepszym usprawiedliwieniem przemocy ze strony policji są umyślne napaści na ich honor. To tak, jakby ktoś próbował sobie wyobrazić najbardziej haniebną rzecz, jaką można zrobić oficerowi, a potem upierał się, że to jest dokładnie to, co anarchiści zawsze będą próbować robić. Że prawdopodobnie była jakaś koordynacja w~tym wysiłku, może być również zebrana, z~faktu, że dokładnie w~tym czasie burmistrzowie i~szefowie policji w~całej Ameryce zaczęli regularnie deklarować, niemal identycznymi słowami (i oczywiście bez żadnych dowodów), że anarchiści byli w~rzeczywistości bandą zepsutych bogatych dzieciaków, którzy ukrywali swoje twarze, żeby ich rodzice nie rozpoznali ich w~telewizji. Oskarżenie to szybko zyskało popularność wśród specjalistów organów ścigania w~całej Ameryce\footnote{Oto przykład z~listy AbolishTheBank, ,,DC Police and Posters in the City'', środa, 22 sierpnia 2001 roku ,,Wczoraj jechałem rowerem po NW i~widziałem gliniarza z~Waszyngtonu, który zdzierał plakaty na 2nd-3rd i~Mass Ave. Zapytałem go, dlaczego je burzył, a on powiedział mi, żebym odszedł i~nazwał krwawiącym sercem. Potem powiedział, że ,,ci anarchiści'' to tylko banda rozpieszczonych białych dzieciaków z~wyższej klasy średniej, którym nie wystarczają miłość od swoich rodziców''. Latynosi, chcą po prostu niszczyć rzeczy i~zachowywać się jak męczennicy. Nieco zszokowany i~zdezorientowany rozmową, którą właśnie powiedziałem, a co z~pierwszą poprawką? i~odjechałem. Czy ktoś jeszcze widział go wczoraj po południu?}. Jak wspomniałem w~rozdziale 6, tego rodzaju twierdzenie wydaje się starannie zaprojektowane, aby zarówno zmobilizować ludzi, jak i~przekazać coś z~pożądanych zasad zaangażowania: ,,Nie bądź delikatny dla tych ludzi, wyładuj na nich swoje urazy, ale nie okaleczaj ich ani nie zabijaj, bo nigdy nie wiesz, kim mogą okazać się ich rodzice''. 

\medskip
\noindent ZASADA FILMU HOLLYWOOD
\medskip

 Wydaje się, że prawdziwym problemem policji w~miesiącach po Seattle był kryzys w~odbiorze publicznym. Ku frustracji urzędników, dyrektorów generalnych i~biurokratów handlowych, amerykańska opinia publiczna odmówiła postrzegania globalnego ruchu sprawiedliwości jako zagrożenia wymagającego przymusowego stłumienia. Jak wspomniałem podczas debaty Ya Basta! cytowanej powyżej, jeden z~nielicznych sondaży opinii publicznej na ten temat, przeprowadzony podczas konwencji republikanów w~2000 roku, wykazał, że zaskakująco duża liczba telewidzów odczuwała sympatię, a nawet dumę, widząc zdjęcia protestujących, i~to pomimo faktu, że relacja była jednolicie negatywna, traktując protesty wyłącznie jako kwestię bezpieczeństwa\footnote{Poniedziałek, 21 sierpnia, ,,Protesty na konwencjach wywołują mieszane reakcje'' (Reuters/Zogby). ,,W ankiecie przeprowadzonej przez Zogby America wśród 1004 dorosłych 32,9\% stwierdziło, że jest dumnych z~protestujących, a kolejne 31,2\% stwierdziło, że są ostrożni. Kolejne 13,2\% stwierdziło, że jest współczujące, a 15,7\% zirytowane, a 6,9\% stwierdziło, że nie ma pewności'' Biorąc pod uwagę niemal jednolitą wrogość w~relacji, fakt, że jedna trzecia publiczności była jednak ,,dumna'', a mniej niż jedna szósta była pewna, że  ich reakcja była negatywna, jest dość niezwykły.}. Myślę, że jest na to prosty powód. Proponowałbym nazwać to ,,zasadą hollywoodzkiego filmu'' Większość Amerykanów, oglądając dramatyczną konfrontację w~telewizji, zadaje sobie pytanie: ,,Gdyby to był hollywoodzki film, kto byłby dobrymi facetami?''. W obliczu rywalizacji między tłumem idealistycznych młodych ludzi, którzy w~rzeczywistości nie wydają się nikogo krzywdzić, a tłumem ciężko uzbrojonych gliniarzy chroniących biurokratów handlowych, prezesów korporacji lub polityków, odpowiedź jest oczywista. W filmach logika ,,nieuczciwego gliniarza'' jest całkowicie odwrócona. Indywidualni policjanci mogą być bohaterami filmu. Gliniarze prewencji nigdy nie mogą być. W rzeczywistości w~hollywoodzkich filmach gliniarze prawie nigdy się nie pojawiają; najbliżej tego rodzaju wyobrażeń można znaleźć Imperialnych Żołnierzy Szturmowych w~Gwiezdnych Wojnach, którzy wraz ze swoim przywódcą Darthem Vaderem to dla większości Amerykanów jedna z~ikon zmechanizowanego zła. Ten punkt nie umknie anarchistom, którzy, przynajmniej od A16, regularnie przywozili nagrania muzyki Imperialnych Szturmowców z~\textit{Gwiezdnych Wojen}, aby nadawać z~ich szeregów, gdy tylko zacznie się posuwać do przodu linia gliniarzy.

 Pojawiło się więc pytanie: czego potrzeba, aby obsadzić protestujących w~roli złoczyńcy?

 Bezpośrednio po Seattle media i~urzędnicy zrobili wszystko, co w~ich mocy, aby wywołać histerię z~powodu rozbitych okien. Wygląda na to, że obrazy z~pewnością uderzyły w~strunę, jak zauważyłem, niewielu Amerykanów nie zdaje sobie sprawy, że okna zostały wybite. Ostatecznie jednak wysiłki te przyniosły zaskakująco niewielki efekt. Ale to też ma sens: w~kategoriach hollywoodzkich niszczenie mienia jest bardzo drobnym peccadillo. W rzeczywistości, jeśli popularność różnych Terminatorów, Zabójczej broni czy Szklanych pułapek cokolwiek ujawnia, to tylko to, że Amerykanom podoba się pomysł niszczenia własności. Jeśli większość nie żywiła pewnej ukrytej radości na myśl, że ktoś rozwali oddział lokalnego banku lub McDonalda (nie wspominając o samochodach policyjnych, centrach handlowych i~skomplikowanych maszynach budowlanych), dlaczego mieliby tak regularnie płacić pieniądze, aby oglądać idealistycznych dobroczyńców, którzy godzinami ich rozbijają i~wysadzają w~powietrze, jeśli zawsze w~taki sposób, że dzięki magii filmów, ale także praktykom Czarnego Bloku, odchodzą niewinni przechodnie całkowicie bez szwanku? Z pewnością jest mało prawdopodobne, że istnieje znaczna liczba Amerykanów, którzy w~takim czy innym czasie nie mieli fantazji o rozbiciu swojego banku. Można powiedzieć, że anarchiści z~Czarnego Bloku żyją ukrytym aspektem amerykańskiego snu w~krainie derby niszczenia i~monster trucków.

 Oczywiście to tylko fantazje. Jestem z~pewnością świadomy, że większość Amerykanów z~klasy robotniczej nie aprobuje otwarcie, a tym bardziej nie opowiada się za niszczeniem fasad Starbucksa. Ale w~przeciwieństwie do gadających klas, również około 2000 roku nie postrzegali takiej działalności jako zagrożenia dla narodu, nie mówiąc już o wszystkim, co wymaga represji w~stylu militarnym. 

 Można nawet powiedzieć, że w~pewnym sensie Czarny Blok wydaje się najnowszymi awatarami artystycznej/rewolucyjnej tradycji, która przewija się przez dadaistów i~sytuacjonistów: takiej, która próbuje rozgrywać sprzeczności kapitalizmu, obracając własną destrukcyjność, wyrównującą siły przeciwko niemu. Społeczeństwa kapitalistyczne -- a w~szczególności Ameryka -- są w~istocie społeczeństwami potlatch. Oznacza to, że zbudowane są wokół spektakularnego niszczenia dóbr konsumpcyjnych. Są to społeczeństwa, które wyobrażają sobie, że są zbudowane na powiązaniu ,,produkcji'' i~,,konsumpcji'', bez końca wypluwając produkty, a następnie ponownie je niszcząc. Ponieważ wszystko opiera się na zasadzie nieskończonej ekspansji produkcji przemysłowej, tej samej zasadzie, którą anarchiści Czarnego Bloku, w~większości będąc wysoce świadomymi ekologicznie antykapitalistami, są szczególnie przeciwni, wszystko to musi być bez końca niszczone, aby zrobić miejsce dla nowych produktów. Ale to z~kolei oznacza wpajanie pewnej pasji lub zachwytu w~rozbijaniu i~niszczeniu własności, która bardzo łatwo może przerodzić się w~zachwyt rozbijania tych struktur relacji, które umożliwiają kapitalizm; jest to system, który może się odnowić tylko poprzez kultywowanie ukrytej przyjemności w~perspektywie własnego zniszczenia\footnote{Może być tutaj istotne, że głównym towarem eksportowym Stanów Zjednoczonych do reszty świata, po broni, są (a) hollywoodzkie filmy akcji i~(b) komputery osobiste. Jeśli się nad tym zastanowić, tworzą one rodzaj komplementarnej pary do zestawu cegła-przez okno/olbrzyma lalka, który opisywałem - lub raczej zestaw cegieł/laleczka może być uważany za rodzaj wywrotowego, odsublimowanego odzwierciedlenia ich - pierwszy dotyczył peanów na rzecz niszczenia własności, drugi - nieskończonej zdolności do tworzenia nowych, ale ulotnych, niematerialnych obrazów w~miejsce starszych, bardziej trwałych form.}.

 Tutaj następuje historia. Radykalni lalkarze zdają sobie również sprawę, że ich sztuka nawiązuje do olbrzymów i~smoków z~wikliny, Gargantui i~Pantagruela, typowych dla średniowiecznych festiwali. Nawet ci, którzy sami nie czytali Rabelaisa czy Bachtina, z~pewnością znają pojęcie karnawału. Masowe konwergencje są prawie zawsze przedstawiane jako ,,karnawał przeciwko kapitalizmowi'' lub ,,festiwal oporu''. Punktem odniesienia wydaje się świat późnego średniowiecza bezpośrednio przed pojawieniem się kapitalizmu, szczególnie okres po czarnej śmierci, kiedy nagły spadek liczby ludności spowodował oddanie bezprecedensowych ilości pieniędzy w~ręce klas pracujących (zob. np. Federici 2004). Większość z~nich trafiła na popularne festiwale tego czy innego rodzaju, które same zaczęły się rozmnażać, aż zajęły dużą część roku kalendarzowego. Były to wydarzenia, które dziś można by nazwać wydarzeniami ,,zbiorowej konsumpcji'', celebracją cielesności i~hałaśliwych przyjemności oraz -- jeśli wierzyć Bachtinowi (1984) -- cichymi atakami na samą zasadę hierarchii\footnote{Niektóre z~tych pomysłów rozwinąłem we własnym eseju: zob. Graeber 1997.}. Można powiedzieć, że pierwsza fala kapitalizmu, moment purytański, jak go czasem nazywa się, musiała zacząć się od skoordynowanego ataku na ten świat, który został potępiony przez rozwijających się właścicieli ziemskich i~rodzących się kapitalistów jako pogański, niemoralny i~całkowicie niesprzyjający utrzymaniu dyscyplina pracy. Oczywiście ruch na rzecz zakazu wszelkich świąt publicznych nie mógł trwać wiecznie; Panowanie Cromwella w~Anglii jest do dziś piętnowane, ponieważ zakazał Bożego Narodzenia; co ważniejsze, po wyeliminowaniu momentów świątecznej, kolektywnej konsumpcji, rodzący się kapitalizm pozostanie z~oczywistym problemem, jak sprzedawać swoje produkty, zwłaszcza w~świetle konieczności ciągłego zwiększania produkcji. Rezultatem było coś, co można by nazwać procesem prywatyzacji pożądania: tworzenie nieskończonej, indywidualnej, rodzinnej, lub na wpół ukradkowe formy konsumpcji, z~których żadna, jak nam tak często przypomina, nie byłaby w~pełni satysfakcjonująca, w~przeciwnym razie nie zadziałałaby cała logika niekończącej się ekspansji. Choć trudno sobie wyobrazić, że stratedzy policyjni są w~pełni świadomi tego wszystkiego, samo istnienie policji jest związane z~kosmologią polityczną, która postrzega takie formy zbiorowej konsumpcji jako z~natury nieuporządkowane i~(podobnie jak średniowieczny karnawał) zawsze pełne możliwości gwałtownego powstania. Porządek oznacza, że  obywatele powinni wracać do domu i~oglądać telewizję\footnote{Tam, gdzie zwykle będą wystawiać programy, które przyjmują perspektywę tej samej policji, która na początku jest odpowiedzialna za usunięcie ich z~ulic. Więcej o tym później.}. 

 Ponieważ jednak poczucie zagrożenia festiwalu nie wydaje się rezonować z~dużymi segmentami widowni telewizyjnej, władze zostały niejako zmuszone do zmiany scenariusza. To, co zobaczyliśmy, to bardzo wykalkulowana kampania symbolicznej wojny, próba wyeliminowania obrazów kolorowych pływaków i~marionetek, a zastąpienia obrazami bomb i~kwasu solnego. O ile odnieśli sukces, to dlatego, że członkom tej samej widowni telewizyjnej rzadko przychodzi do głowy, że w~sprawach bezpieczeństwa publicznego ich przedstawiciele rządowi po prostu zmyślali takie rzeczy.

\section{Wnioski}

 Więc dlaczego gliniarze nienawidzą lalek?

 Najwyraźniej nie chodzi tylko o to, że myślą, że w~środku mogą być bomby, chociaż ta kukiełka pełna materiałów wybuchowych jest bez wątpienia podobna do pistoletu na wodę wypełnionego wybielaczem i~moczem, sam w~sobie wymowny symbol. Myślę, że w~tym miejscu musimy wrócić do wcześniejszej kwestii zasad zaangażowania.

 W eseju napisanym krótko po aferze Rodneya Kinga Marc Cooper zwrócił fascynującą uwagę, że w~większości przypadków, w~których Amerykanie są dotkliwie bici przez policję, ofiara jest niewinna jakiegokolwiek przestępstwa. Niewinni obywatele są bardziej narażeni na pobicie niż przestępcy, ponieważ są bardziej skłonni do pyskowania. A jeśli chcesz, aby policjant był agresywny, jest to najpewniejszy sposób. Przytacza refleksje byłego policjanta o nazwisku Jim Fyfe:

-- Włamywacze i~gwałciciele niekoniecznie są ,,dupkami'' w~oczach LAPD -- mówi Fyfe. -- Dupek to osoba, która nie akceptuje żadnej definicji policjanta dotyczącej jakiejkolwiek sytuacji. Gliniarze oczekują, że wszyscy, łącznie z~zatrzymanym kierowcą, będą się podporządkowywać. Jakiekolwiek wyzwanie, albo śmiertelny grzech odmawiania odpowiedzi, i~stajesz się ,,dupkiem''. A ,,dupki'' mają być reedukowane, żeby nie pyskowały. Prawdziwe przypadki brutalności mają miejsce w~przypadku ,,dupków''. Gliny nie biją włamywaczy. (Cooper 1991:30).

 Krytyczne zdanie brzmi tutaj: ,,nie akceptuje jakiejkolwiek definicji policjanta dotyczącej jakiejkolwiek sytuacji''. To jest siła, której policja strzeże najbardziej zazdrośnie, ta, której najchętniej broni przemocą: władza definiowania sytuacji. Jest tu coś bardzo głębokiego; być może kluczem do samej natury przemocy. Wyjaśnię, dlaczego tak myślę w~ostatnim rozdziale; na razie jednak chciałbym tylko podkreślić, że właśnie to radykalni lalkarze próbują obalić.

 Wyobraźmy sobie przez chwilę konflikt między dwiema zasadami działania politycznego. Można nawet powiedzieć, że między dwiema różnymi koncepcjami rzeczywistości. Pierwsza to polityka, która zakłada, że  ostateczną rzeczywistością jest siła, ale w~której ,,siła'' tak naprawdę działa jako eufemizm dla różnych technologii przemocy. Przecież bycie ,,realistą'' w~polityce nie ma nic wspólnego z~rozpoznawaniem realiów materialnych, chodzi tylko o chęć zaakceptowania realiów przemocy. Przemoc definiuje ostateczną prawdę sytuacji. Przypuszczalnie dlatego policja może pozwolić sobie na tak relatywistyczne podejście do kwestii ostatecznej prawdy lub moralności. Jeśli jedyną niekwestionowaną rzeczywistością jest możliwość krzywdzenia innych, prawdopodobnie najlepszym sposobem działania jest po prostu zapewnienie, że wszyscy przestrzegają przynajmniej jakiegoś jasnego zestawu zasad. Z drugiej strony możemy sobie wyobrazić politykę wyobraźni. Mam tu na myśli nie tyle polityczny projekt nadania ,,władzy wyobraźni'', ile uznanie, że wyobraźnia i~kreatywność są zawsze ostatecznym źródłem władzy\footnote{Oprócz sytuacjonistów francuskim teoretykiem najczęściej spotykanym w~anarchistycznych księgarniach jest Cornelius Castoriadis, wielki teoretyk rewolucyjnej imaginacji.}. Stąd ich szczególna cecha bycia jednocześnie świętymi i~śmiesznymi. To, czego anarchiści regularnie próbują -- i~co ucieleśniają w~marionetkach -- jest systematycznym i~ciągłym wyzwaniem dla prawa policji lub jakiejkolwiek innej władzy, by zdefiniować sytuację. Robią to, proponując niekończące się alternatywne ramy. Lub robią to, nalegając na możliwość zmiany ram, kiedy tylko chcą. To jest oczywiście cel ,,interwencji marionetkowej'', podobnie jak był to efekt nagłego pojawienia się klaunów i~miliarderów w~Filadelfii. Agresywnie przesuwają ramy. Zrobili to również w~sposób, który był bardzo korzystny dla taktycznej przewagi aktywistów i~całkowicie wyrzucił policję z~gry.

 Wróćmy zatem do pojęcia zasad zaangażowania.

 Pod koniec ostatniego rozdziału argumentowałem, że w~Stanach Zjednoczonych zwykle przyjmuje się, że zasady zaangażowania podczas akcji masowych będą negocjowane pośrednio: w~pewnym stopniu przez sądy, ale w~dużej mierze za pośrednictwem korporacyjnych mediów. Rezultatem jest zdecydowanie nierówne pole gry. Nie jest to całkowicie krzywe, ponieważ chociaż korporacyjne media tak układają swoje historie, że sprawy są bardzo skrzywione na korzyść policji i~łatwo przez nią manipulowane, ich publiczność w~żadnym wypadku nie jest zbiorem biernych naiwniaków i~skłania, przy innych rzeczy równe, się do sympatyzowania ze słabszymi. Tak naprawdę dopiero po 11 września rząd słusznie wyczuł, że grunt przesunął się do punktu, w~którym mogli uciec z~czystymi represjami, które przystąpili do wprowadzenia w~życie podczas działań FTAA w~Miami w~2003 roku, używając nie tylko bezprecedensowej przemocy (paralizatory, plastikowe i~drewniane kule, wzmożone tortury i~znęcanie się nad więźniami), ale także wielu tych samych technik medialnych, takich jak specjalni reporterzy, przeszkoleni do użytku z~jednostkami wojskowymi za granicą. Gdyby nie niemal bezprecedensowa katastrofa z~11 września, sprawy prawie na pewno potoczyłyby się inaczej.

 Anarchiści i~tak mają tendencję do odrzucania tej logiki pośrednich negocjacji. Nie chodzi o to, że są oni jednolicie przeciwni próbom wpływania na sądy lub media (w rzeczywistości byłoby to bardzo głupie, gdyby całkowicie pozostawili to terytorium wrogowi), ale żaden anarchista, myślę, że można śmiało powiedzieć, nie byłby skłonny zaakceptować układ, w~którym akcje uliczne sprowadzają się do czegoś w~rodzaju meczów piłki nożnej, z~wszystkimi zasadami ustalonymi z~góry. Cała logika akcji bezpośredniej przemawia przeciwko temu. Zamiast tego, tak jak w~procesie konsensusu, rozpadają razem dwie rzeczy, które zwykle uważa się za oddzielne poziomy -- proces podejmowania decyzji i~środki ich egzekwowania -- tak więc tutaj, w~akcjach ulicznych, dążą, o ile to możliwe, do upadku politycznego, negocjacyjnego proces w~strukturę samego działania. Próbują wygrać konkurs niejako poprzez ciągłe zmienianie definicji tego, czym jest pole, jakie są zasady, jaka jest stawka, i~robią to na samym boisku. Podczas akcji ulicznych sytuacja, która przypomina wojnę bez użycia przemocy, staje się sytuacją, która przypomina cyrk, przedstawienie teatralne lub uroczysty rytuał i~równie dobrze może ponownie powrócić do walki bez przemocy. Oczywiście z~punktu widzenia policji jest to po prostu oszustwo. Działacze nie walczą uczciwie. Z ich perspektywy lalki mogą równie dobrze nosić bomby, ponieważ wszystko jest potencjalnie oszukańcze. Ale, jak widzieliśmy, policja też nie walczy uczciwie. Nie mogą, co do zasady, traktować drugiej strony jak honorowych przeciwników, ponieważ oznaczałoby to, że są w~pewnym sensie równi. Stąd, o ile wydaje się, że istnieją milczące zasady, regularnie je łamią, a w~kontaktach z~mediami ich rzecznicy kłamią.

 Można tutaj potraktować problem jako analogiczny do znanego paradoksu władzy konstytutywnej: skoro żaden system nie może sam się stworzyć (tj. żaden Bóg zdolny do ustanowienia praw fizycznych nie może być związany tymi prawami, żaden święty król zdolny do ustanowienia porządku prawnego nie może być związany jej nakazami), każdy porządek prawny/polityczny może być stworzony tylko przez jakąś siłę, do której ta legalność nie ma zastosowania. Sama konstytucja nie może zostać stworzona za pomocą środków konstytucyjnych; i~rzeczywiście, osiemnastowieczni rewolucjoniści amerykańscy i~francuscy byli całkiem wyraźnie winni zdrady zgodnie z~prawami, w~których dorastali. We współczesnej historii europejsko-amerykańskiej oznaczało to, że prawomocność konstytucji ostatecznie nawiązuje do pewnego rodzaju rewolucji ludowej: rewolucje są właśnie sednem, moim zdaniem,

 Teraz, oczywiście, rewolucja jest dokładnie tym, co ludzie z~marionetkami ostatecznie próbują osiągnąć; nawet jeśli starają się to osiągnąć przy absolutnym minimum przemocy. Wydaje mi się jednak, że to, co naprawdę wywołuje najbardziej gwałtowne reakcje ze strony policji, to właśnie ta próba uobecnienia władzy konstytucyjnej, siły wyobraźni ludu do tworzenia nowych form instytucjonalnych; i~to nie tylko w~krótkich przebłyskach, ale w~sposób ciągły. Ruch oparty na zasadach akcji bezpośredniej to ruch, którego celem jest nieustanne kwestionowanie ich zdolności do definiowania sytuacji. Naleganie, że zasady zaangażowania można niejako renegocjować na polu bitwy, że można ciągle zmieniać narrację w~środku opowieści, jest w~tym świetle tylko jednym aspektem znacznie większego sprzeciwu władzy.

 Dzięki temu łatwiej zrozumieć, dlaczego gigantyczne lalki, które są tak niezwykle kreatywne, a jednocześnie tak celowo efemeryczne, kpiące z~samej idei odwiecznych prawd, jakie mają przedstawiać pomniki, mogą tak łatwo stać się samym symbolem tej próby zdobycia potęgi społecznej kreatywności\footnote{T-shirt kolektywu Arts in Action, który faktycznie tworzy wiele z~tych lalek, zawiera cytat Brechta: ,,Postrzegamy sztukę nie jako lustro, które trzyma się rzeczywistości, ale jak młot, którym można ją kształtować''.}. Ta władza jest władzą odtwarzania i~redefiniowania instytucji, w~zasadzie wszystko to, co przedstawiają standardowe media (które nigdy nie mówią o tym, że aktywiści próbują odtworzyć proces demokratyczny, wyobrazić sobie nowe formy organizacji itp.) znika. Z perspektywy ,,sił porządku'' lalki są naprawdę demoniczne właśnie z~tego powodu -- w~rzeczywistości istnieje długa tradycja, szczególnie żywa w~Ameryce, postrzegania kreatywności jako czegoś demonicznego -- ponieważ doskonale ucieleśniają zasadę rewolucji. Słynny aforyzm Bakunina -- ,,pragnienie zniszczenia jest także twórczym impulsem'' -- wydaje się tutaj rozpoznawany w~dużej mierze w~odwrotnej kolejności.

 Być może dlatego tak wielu Amerykanów uważa klaunów za coś przerażającego. Tak czy inaczej, do takiego wniosku doszło wielu z~tych, którzy brali udział w~,,Rewolucyjnym Anarchistycznym Bloku Klaunów'' w~Filadelfii, prawie wszyscy z~nich zostali pobici lub aresztowani, nawet gdy miliarderzy zdołali opuścić pole bez szwanku.
 
 -- Problem z~koncepcją klauna -- wyjaśnił później jeden z~nich na spotkaniu DAN -- okazał się taki, że większość Amerykanów nie uważa klaunów za zabawnych. Uważają ich za przerażających. Z drugiej strony większość Amerykanów uważa, że  pomysł pobicia klauna jest bardzo zabawny. 
 
 Mimo całej przyjemności, jaką widzowie filmowi znajdują, obserwując niszczenie centrów handlowych, takie wątki kończą się ostatecznie tym, że jakiś bohaterski gliniarz zabija złoczyńcę, a tym samym siłą przywraca poczucie normalności. Rozbijanie Spektaklu to przyjemność. Ale najwyraźniej stosowanie przemocy w~celu zdefiniowania sytuacji jest też przyjemność.

\chapter{Wyobraźnia}

\begin{flushright}

\texttt{Wszystkie narody: halucynacje \newline  Czy chcesz umrzeć za przyszłość iluzji? \newline  Jeśli nalegamy na utrzymanie państwa, któremu mamy być posłuszni, to musimy hodować dzieci, \newline  które za to umrą. Więc przestań marudzić.}

--- Anonimowy plakat anarchistyczny
\end{flushright} 

 Ten rozdział nie jest tak naprawdę zakończeniem. W pewnym sensie pisanie wniosków do dzieła etnograficznego jest zawsze przedsięwzięciem wątpliwym. Cel etnografii jest zasadniczo opisowy. Dobry opis z~pewnością wymaga odwołania się do teorii, ale w~etnografii teoria jest właściwie wykorzystywana w~służbie opisu, a nie na odwrót. Jeśli celem opisu etnograficznego jest próba udostępnienia czytelnikowi środków do wyobrażeniowego przejścia przez uniwersum moralne i~społeczne, wówczas wydaje się wyzyskujące, niemal obraźliwe sugerowanie, że inni ludzie żyją swoim życiem lub realizują swoje projekty, aby umożliwić jakiemuś uczonemu zdobycie punktu w~jakiejś tajemniczej debacie teoretycznej. W każdym razie to nieprawda.

 Zamiast tego chciałbym zakończyć kilkoma dość krótkimi refleksjami teoretycznymi, inspirowanymi moim własnym udziałem w~DAN i~podobnych grupach. Ponieważ nie trzeba dodawać, że moje zaangażowanie w~takie grupy nieustannie inspirowało mnie nowymi pomysłami, potraktujcie ten ostatni rozdział jako chwilę dialogu.

 Zacznę od poszerzenia jednej z~idei, którymi zakończyłem ostatni rozdział: podziału na polityczne ontologie przemocy i~polityczne ontologie wyobraźni. Umożliwi to połączenie kilku podobnych pomysłów, które pojawiły się we wcześniejszych częściach książki.

 Omówiłem już pierwszy rodzaj ontologii politycznej pod koniec rozdziału 6 w~części zatytułowanej ,,O ideologicznych skutkach regulacji rządowych''. Dlaczego, pytałem, projekty radykalnych przemian społecznych są zawsze postrzegane jako głęboko ,,nierealistyczne'', jako bezsensowne marzenia, które zdają się rozpływać w~momencie, gdy napotykają twardą materialną rzeczywistość? Nie jest to po prostu, jak sugerowałem, efekt siły przyzwyczajenia, czy nawet faktu, że nigdy nie można być pewnym, czy jakikolwiek eksperyment społeczny rzeczywiście zadziała. Przynajmniej nie w~sposób, w~jaki te rzeczy są natychmiast doświadczane. To fakt, że duże, ciężkie, cenne przedmioty -- domy, samochody, łodzie, nie mówiąc już o fabrykach -- są w~społeczeństwach przemysłowych niezmiennie otoczone niekończącymi się regulacjami rządowymi. Te przepisy są egzekwowane przemocą. Policja rzadko pojawia się machając pałkami w~klubach, aby egzekwować przepisy przeciwpożarowe (chyba że mają do czynienia z~anarchistami); ale to tylko pomaga uczynić przemoc niewidoczną i~sprawić, że skutki wszystkich tych regulacji -- regulacji, które prawie zawsze zakładają, że normalne relacje między jednostkami są zapośredniczone przez rynek i~że normalne grupy są zorganizowane hierarchicznie -- wydają się pochodzić nie z~monopolu rządu na użycie siły, ale z~wielkości, solidności i~ciężkości samych obiektów.

 Kiedy ktoś jest proszony o bycie ,,realistą'', to rzeczywistość, którą ma rozpoznać, nie jest rzeczywistością naturalnych, materialnych faktów; nie jest to też jakaś rzekomo brzydka prawda o ludzkiej naturze. Zwykle jest to rozpoznanie skutków systematycznego używania lub groźby krzywdy fizycznej. Mamy do czynienia z~cieniem państwa. Myślę, że to jest kluczowe; tak istotne, że warto zatrzymać się na chwilę nad niektórymi z~wymienionych wcześniej przykładów. W stosunkach międzynarodowych za ,,realistę'' politycznego uważa się takiego, który akceptuje, że państwa wykorzystają wszelkie dostępne im zdolności, w~tym siłę uzbrojenia, do realizacji swoich interesów narodowych. Jak wówczas zauważyłem, jest to idea głęboko metafizyczna. Przekonanie, że państwa -- abstrakcyjne byty, takie jak ,,Francja'' czy ,,Indie'' -- są bytami tej samej natury, co poszczególne istoty ludzkie, z~własnymi interesami i~celami, nie ma nic wspólnego z~rozpoznaniem materialnej rzeczywistości. Królowie Francji i~cesarze Indii mieli interesy i~cele. ,,Francja'' i~,,Indie'' nie mają. Dopiero dzięki skomplikowanej metafizyce ,,suwerenności'', która przenosi atrybuty królów i~cesarzy na całe populacje (za pośrednictwem jakiegoś aparatu politycznego), możemy sobie nawet wyobrazić, że mogą. To, co sprawia, że  ,,realistyczne'' wydaje się stwierdzenie, że narody mają ,,interesy'', to po prostu to, że, podobnie jak królowie, ci, którzy obecnie kontrolują państwa, mają moc powoływania armii, przeprowadzania inwazji, oblegania miast i~w inny sposób grożenia użyciem zorganizowanej przemocy w~celu realizacji tych interesów, i~że byłoby głupotą ignorowanie tej możliwości. Te rzeczy są prawdziwe, ponieważ mogą cię zabić.

 Suwerenność oznacza oczywiście zarówno prawo do prowadzenia wojny poza własnymi granicami, jak i~prawo do utrzymania monopolu na użycie siły przymusu wewnątrz. Twierdziłem, że to tworzy podobny efekt rzeczywistości w~sprawach własności. Gdyby ten argument wydawał się w~jakikolwiek sposób naciągany, można by tu rozważyć, że pochodzenie słowa ,,rzeczywisty'' (ang. real), podobnie jak ,,nieruchomość'' (ang. real property), samo w~sobie. W przeciwieństwie do innych zastosowań słowa ,,real'' nie pochodzi od łacińskiego res, które oznacza ,,rzecz''. Wywodzi się z~hiszpańskiego słowa \textit{real}, co oznacza ,,królewski'' lub ,,monarszy'' i~pierwotnie oznaczało ,,należący do króla'' Cała ziemia na suwerennym terytorium należy ostatecznie do suwerena i~nadal tak jest prawnie (dlatego państwo ma prawo przejąć ziemię za pośrednictwem domeny dostojnej). Suwerenność jest również podstawą prawną do narzucania regulacji przez państwo. Tak jak Giorgio Agamben (1998), argumentował słynnie, że z~perspektywy suwerennej władzy coś żyje, ponieważ można to zabić, więc własność jest ,,rzeczywista'', ponieważ można ją przejąć lub zniszczyć.

 Ostatecznie ten rodzaj ontologii politycznej przechodzi w~taką, w~której moc niszczenia, zadawania innym bólu, łamania, uszkadzania lub niszczenia ich ciał jest traktowana jako społeczny ekwiwalent tej samej energii, która napędza kosmos. Ponownie, może się to wydawać dziwnym stwierdzeniem, ale niektóre takie wizje wydają się ukryte w~większości języka używanego do opisu funkcjonowania państw. Weźmy słowo ,,siła''. Kiedy zmuszasz kogoś do zrobienia czegoś wbrew jego woli -- powiedzmy, grożąc, że złamiesz mu nogi, jeśli odmówi -- mówi się, że ,,zmuszasz'' go do tego. ,,Siła'' to władza oparta na systematycznym groźbie lub stosowaniu przemocy. Mówi się również, że państwo ma monopol na legalne użycie ,,siły'' przymusu. Jeśli ktoś posługuje się legitymistyczną definicją przemocy, takie, które uniemożliwia stwierdzenie, że agenci państwa zachowywali się agresywnie, jeśli robili coś, do czego zostali odpowiednio upoważnieni, to jest to słowo, którego używasz w~zamian: protestujący byli agresywni (wybito okno), policja odpowiedziała siłą (zaczęli strzelać w~tłum plastikowymi kulami). W rzeczywistości jest to bardzo subtelne użycie. Rozważ następujące sześć zdań: 

\noindent 1. Policja przybyła na plac i~otworzyła ogień do protestujących

\noindent 2. Kilka z~nich upadło na ziemię, gdy uderzyła w~nie \textit{siła }plastikowych kul. 

\noindent 3. Inni zostali \textit{zmuszeni} do położenia się na ziemi i~skuci kajdankami.

\noindent 4. Następnie policja wepchnęła ich \textit{siłą }do furgonetek aresztowych. 

\noindent 5. W rezultacie pozostali protestujący zostali zmuszeni do opuszczenia placu.

 Istnieje zatem kontinuum zwyczajów. Policja \textit{siłą }zabezpieczyła teren.

 W zdaniu nr 2 ,,siła'' odnosi się do prostej fizyki: można powiedzieć, że obiekt o określonej masie poruszający się z~określoną prędkością uderza w~inny obiekt z~określonym stopniem siły. Użycie w~zdaniu nr 3 jest zbliżone, ponieważ protestujący zostali prawdopodobnie popchnięci na ziemię przez nacisk pałek i~ludzkich mięśni, ale wtapia się w~bardziej niejednoznaczne użycie w~zdaniu nr 4, gdzie prawdopodobnie, a nie zwykła presja fizyczna (pchanie aresztowanych, szturchanie, ciągnięcie, a nawet noszenie ich) zostało uzupełnione wydawaniem rozkazów popartych dorozumianymi lub wyraźnymi groźbami. W zdaniu 5, ,,siła'' odnosi się tylko do efektów strachu przed dalszymi atakami fizycznymi. Wreszcie, ze względu na ich zdolność do stosowania przemocy i~groźby przemocy, w~najbardziej efektywny sposób, aby robić rzeczy, takie jak oczyszczanie placu, że policję można nazwać ,,siłą'' (jak w~zdaniu nr 6), tak jak generał mógłby powiedzieć, że dowodzi ,,siłą'' stu tysięcy ludzi, lub wojsko jako całość można nazwać jako ,,siły zbrojne''. Istnieje zatem kontinuum zwyczajów. Ale ogólny efekt polega na połączeniu najbardziej podstawowych zasad fizyki z~psychologicznymi skutkami zastraszania innych cierpieniem i~bólem. 

 Ktoś mógłby sprzeciwić się, że metafora jest nieunikniona, ponieważ w~kategoriach czysto fizycznych przemoc zwykle wymaga dużej siły fizycznej. To może być prawda. Ale każda forma ludzkiego działania wiąże się z~pewnym poziomem siły (na przykład śpiewanie polega na wypychaniu powietrza z~płuc), a wiele (jazda samochodem) wymaga użycia znacznie większej siły, niż potrzeba, aby zmusić tysiąc protestujących do przewrócenia się na ziemię. Wydaje się, że znajdujemy się w~obecności klasycznej formy ideologicznej naturalizacji. To, co w~innym przypadku mogłoby wydawać się raczej tandetną ludzką praktyką -- ustanawianie zbioru zasad, a następnie grożenie zranieniem każdemu, kto ich nie przestrzega -- jest traktowane jako ekwiwalent jednego z~elementarnych składników fizycznego wszechświata. 

 W rzeczywistości ciągle zapożyczamy terminy z~jednej dziedziny, aby opisać drugą: ,,prawo'', ,,siła'', ,,władza''.

 Porównaj następujące zdania: Naukowcy badają naturę praw fizycznych, aby zrozumieć siły rządzące wszechświatem. 

 Policja jest ekspertem w~naukowym stosowaniu siły fizycznej w~celu egzekwowania praw rządzących społeczeństwem. 

 To jest jedna ontologia. Aktywiści, sugeruję, zdają się pracować z~zupełnie innym: innym zestawem założeń dotyczących tego, co jest naprawdę realne, o samych podstawach istnienia. Nazwałem to ,,polityczną ontologią wyobraźni'' z~powodów, które, mam nadzieję, wkrótce staną się jasne, ale równie dobrze mógłbym nazwać to ontologią kreatywności, tworzenia, wynalazczości lub dowolnej liczby rzeczy. 

 Nie twierdzę, że jest to powszechnie obowiązujące rozróżnienie. Podejrzewam, że powód, dla którego istnieje, można wywieść z~pewnych szczególnych cech zachodnich teorii wiedzy, a zwłaszcza tendencji do traktowania wszechświata jako zbioru obiektów fizycznych, które można zrozumieć, nadając im nazwy. Widać to już w~teoriach języka autorów takich jak Platon czy Augustyn, gdzie język traktowany jest po prostu jako zbiór rzeczowników. Teoria języka wywodząca się od czasowników wyglądałaby zupełnie inaczej. Jeśli tak, pojawia się problem: jeśli świat jest zbiorem identycznych obiektów (rzeczy, którym nadaliśmy nazwy), jak cokolwiek może się poruszyć lub zmienić, a to prawie nieuchronnie oznacza, że  trzeba opracować jakąś teorię niewidzialnych sił i~mocy czających się za powierzchnią. Na przykład: najpierw wyobrażasz sobie przedmioty, i~wyobrażasz sobie je jako istniejące poza czasem i~ruchem; wtedy musisz wprowadzić ,,siłę grawitacji'', aby je poruszyć, zamiast postrzegać tendencje ruchu obiektu i~jego relacje z~innymi obiektami jako nieodłączną część samej rzeczy. W ten sposób siły zaczynają być postrzegane jako ukryta rzeczywistość. 

 Nie jest to jedyny możliwy sposób na wyobrażenie sobie świata. Inne tradycje intelektualne zaczynają się zupełnie inaczej. Ale kiedy już zajdzie się tak daleko, to raczej wydaje się, że w~końcu zobaczy się te siły albo jako siły stworzenia, albo jako siły zniszczenia. Nawet jeśli wyobrażasz sobie, że człowiek jest stałym, samoidentyfikującym się obiektem (a nie procesem w~ciągłej transformacji, definiowanym w~dużej mierze przez jego relacje z~innymi), nadal musisz przyznać, że ta istota ludzka kiedyś się urodziła i~pewnego dnia nieuchronnie umrze. Albo, jeśli spojrzysz na świat jak na zbiór towarów, żyrandoli, batoników i~tak dalej, to przynajmniej musisz przyznać, że ktoś je zrobił i~że pewnego dnia zostaną zjedzone, wyrzucone, stopione, sprasowane, spalone w~spalarni, wyrzucane na wysypiska śmieci, lub w~inny sposób zniszczone. W naszym społeczeństwie lubimy trzymać narodziny, śmierć, produkcję i~usuwanie odpadów w~dużej mierze poza zasięgiem wzroku, ale oczywiście służy to tylko wzmocnieniu poczucia, w~jakim wydają się ukrytą rzeczywistością za rzeczami. Oczywiście nawet tutaj ten wybór między mocami twórczymi a mocami niszczącymi ma większy sens, jeśli mówimy o ludziach i~wytworzonych towarach, niż gdybyśmy zaczęli od skał i~drzew, które zwykle nie mają tak oczywistych początków i~zakończeń, ale od zarania przynajmniej w~epoce przemysłowej, to są przykłady, które zwykle faworyzujemy. 

 W rzeczywistości dychotomia ta pojawiła się w~najbardziej oczywisty sposób w~okresie rewolucji przemysłowej, odpowiadając -- choć bardzo z~grubsza -- dychotomii między prawicowymi i~lewicowymi stanowiskami politycznymi, które pojawiły się w~tym czasie. Obecnie większość z~nas zna stanowisko lewicy w~dużej mierze dzięki pracom Marksa i~Engelsa, choć wiele z~tego, co mówili o znaczeniu pracy produkcyjnej, odbijało się echem argumentów niezwykle rozpowszechnionych w~radykalnych kręgach ich czasów, a być może z~różnych form romantyzmu. W rzeczywistości marksistowskie teorie o wartości pracy, siłach wytwórczych i~tym podobnych są po prostu najbardziej wyrafinowanym wypracowaniem o wiele bardziej powszechnego tematu, troski o twórcze moce i~twórcze energie, które zawsze znajdowały się w~centrum tego, co się stało znane jako lewica, orientacja polityczna, która przecież była poświęcona twierdzeniu, że skoro ludzie codziennie tworzą i~odtwarzają świat, nie ma powodu, dla którego nie mogliby stworzyć takiego, który naprawdę nam się podoba. Sam Marks, mimo całej swojej pogardy dla utopijnych socjalistów swoich czasów\footnote{Być może nieco bardziej zniuansowane, niż się czasem sugeruje: zob. Geoghegan 1987. Aby poznać niektóre niedawne celebracje radykalnej wyobraźni, zob. Kelley 2002, Duncombe 2007.}, nigdy nie przestawał upierać się, że tym, co odróżnia ludzi od zwierząt, jest to, że mogą najpierw coś sobie wyobrazić, a potem spróbować wprowadzić to w~życie. Ten akt urzeczywistniania wyimaginowanych wizji był właśnie tym, co nazwał ,,produkcją''. Utopijni socjaliści, tacy jak Simonianie w~tym samym czasie twierdzili, że artyści powinni być elitą polityczną -- \textit{avant garde} lub strażą przednią -- nowego rewolucyjnego porządku społecznego, zapewniającego wielką wizję, którą społeczeństwo przemysłowe miało moc do zbudowania. To, co wydawało się wówczas dziwną propozycją ekscentrycznej pisarki, wkrótce stało się kartą faktycznego sojuszu, który przetrwał do dziś. Jeśli artystyczna awangarda i~społeczni rewolucjoniści od tamtej pory nadal czują do siebie pokrewieństwo, jak argumentowałem w~rozdziale 6, może opierać się jedynie na przywiązaniu do idei, że ostateczną, ukrytą prawdą o świecie jest to, że jest on czymś, co sami tworzymy i~równie dobrze moglibyśmy zrobić inaczej.

 Na ten nacisk na siły wytwórcze prawica ma oczywiście tendencję do odpowiadania, że  rewolucjoniści systematycznie lekceważą społeczne i~historyczne znaczenie ,,środków zniszczenia'' państw, armii, katów, najazdów barbarzyńców, przestępców, niszczycielskich motłochów i~tak dalej. Twierdzą, że udawanie, że takich rzeczy nie ma lub można je po prostu wyrzucić, skutkuje zapewnieniem, że lewicowe reżimy w~rzeczywistości spowodują znacznie więcej śmierci i~zniszczenia niż te, które mają mądrość, by przyjąć bardziej realistyczne podejście. 

 Jak każdy schemat teoretyczny jest to oczywiście prymitywne uproszczenie. Sprawy są zawsze bardziej skomplikowane. Idea, że siły wytwórcze były motorem historii, nie była tylko ideą klasy robotniczej. Była to, jeśli w~ogóle, jeszcze w~większym stopniu ideologia rodzącej się burżuazji europejskiej. To jeden z~powodów, dla których Marks upierał się, że burżuazja sama jest siłą rewolucyjną. Lewica nigdy nie była w~stanie do końca zrozumieć, jak pogodzić pojęcie ludzkiej kreatywności z~równie atrakcyjną ideą, że rozwój wiedzy naukowej lub jakaś inna siła ewolucyjna prowadzi nas wszystkich do wyzwolenia. Jednocześnie polityka klasowa doprowadziła większość marksistów do wniosku, że ci, którzy pracują na liniach produkcyjnych, byli zaangażowani w~wytwarzającą wartość ,,pracę'', podczas gdy większość tych, którzy faktycznie wyobrażali sobie i~projektowali produkty, nie była. Elementy prawicy również parały się ideałem artystycznym, zwłaszcza ideą, że twórcze jednostki mogą, dzięki swoim natchnionym siłom, zmieniać historię. Z kolei współczesna teoria społeczna powstała w~dużej mierze w~reakcji na konserwatywną krytykę myśli rewolucyjnej, koncentrując się na zrozumieniu władzy wszystkich tych ,,rzeczywistości'' -- władzy, wspólnoty, hierarchii -- o których można by powiedzieć, że są prawdziwe właśnie dlatego, że opierały się próbom narzucenia pewnych rodzaj rewolucyjnej wizji (Nesbitt 1966, Graeber 2003). 

 Niemniej jednak uważam, że te terminy są przydatne: nie tylko w~zrozumieniu natury rewolucyjnych sojuszy, ale także w~zrozumieniu samej natury władzy społecznej. Właśnie dlatego skupiłem się na terminach ,,wyobraźnia'' i~,,przemoc'', ponieważ wydaje mi się, że te dwa elementy zawsze wchodzą w~interakcję w~przewidywalny i~znaczący sposób. 

\section{O przemocy i przemieszczeniu wyobraźni}

 Kiedy antropolodzy i~inni teoretycy kultury piszą o przemocy, często podkreślają, że przemoc działa w~dużej mierze za pomocą wyobraźni. Na przykład, nawet najbardziej brutalne reżimy polityczne zastraszają potencjalnych przeciwników znacznie bardziej poprzez przerażanie niż faktyczne ich zabijanie; większość z~nas słyszała o tysiącu brutalnych incydentów na każdy, którego byliśmy świadkami. Dlatego przemoc jest formą komunikacji i~to, jak niezmiennie wnioskują, jest w~niej ostatecznie ważne. 

 To mniej więcej to, co każdy, kto poważnie traktuje kulturę i~znaczenie, naprawdę musiałby powiedzieć, a ja prawie nie miałbym z~tym problemu. Z pewnością nie chciałbym twierdzić, że przemoc, ogólnie rzecz biorąc, nie funkcjonuje jako forma komunikacji. Nie zgodziłbym się jednak z~ostatnią częścią: ,,i to jest w~tym ostatecznie ważne''. To oczywiście prawda, że  przemoc może być komunikatywna. Ale można to powiedzieć o każdej formie ludzkiego działania. Uderza mnie, że to, co jest naprawdę ważne w~przypadku przemocy, to fakt, że jest to być może jedyna forma ludzkiego działania, która nawet daje możliwość działania na innych \textit{bez }bycia komunikatywnymi. Lub, pozwól, że przedstawię to ostrożniej. Przemoc może być jedyną formą ludzkiego działania, dzięki której możliwe jest wywieranie stosunkowo przewidywalnego wpływu na działania innej osoby, której nie rozumiesz. Prawie w~każdy inny sposób, w~jaki ktoś mógłby próbować wpłynąć na działania innych, trzeba mieć pewne pojęcie, za kogo się uważa, za kogo ciebie uważa, czego może chcieć od tej sytuacji i~wiele podobnych rozważań. Uderz ich wystarczająco mocno w~głowę i~nic z~tego nie ma znaczenia. Prawdą jest, że efekty, jakie można uzyskać po prostu przez uderzenie w~nie, są bardzo ograniczone. Są one w~zasadzie ograniczone do uniemożliwienia im działania poprzez wyłączenie lub zabicie ich. Mimo to jest to coś, a żadna alternatywna forma działania nie może, bez pewnego rodzaju odwołania się do wspólnych znaczeń lub zrozumienia, mieć w~ogóle żadnych skutków. Co więcej, nawet próby wpłynięcia na drugą osobę za pomocą groźby przemocy, która oczywiście wymaga pewnego poziomu wspólnego zrozumienia (przynajmniej druga strona musi zrozumieć, że jest zagrożona i~dlaczego), wymaga znacznie mniej niż jakakolwiek alternatywa. Większość relacji międzyludzkich -- szczególnie tych trwających, takich jak te między wieloletnimi przyjaciółmi lub od dawna wrogami -- jest niezwykle skomplikowana, nieskończenie gęsta od doświadczeń i~znaczeń. Wymagają stałej i~często subtelnej pracy interpretacyjnej; każdy musi nieustannie wyobrażać sobie punkt widzenia drugiego. Grożenie innym krzywdą fizyczną daje możliwość przebicia się przez to wszystko. Umożliwia to relacje o znacznie bardziej schematycznym charakterze: np. ,,przekrocz tę linię, a cię zastrzelę, a poza tym naprawdę nie obchodzi mnie, kim jesteś i~czego chcesz''. Jest to, na przykład, powód, dlaczego przemoc jest preferowaną bronią głupich, można by wręcz powiedzieć, że jest to atut głupoty, ponieważ jest to forma głupoty, na którą najtrudniej znaleźć jakąkolwiek inteligentną odpowiedź.

 W tym wszystkim jest jedno bardzo ważne zastrzeżenie. Im bardziej wyrównane są dwie strony pod względem zdolności do stosowania przemocy, tym mniej to wszystko wydaje się prawdą. Jeśli ktoś bierze udział w~stosunkowo równym konkursie, to rzeczywiście bardzo dobrym pomysłem jest zrozumienie jak najwięcej o drugiej stronie. Na przykład dowódca wojskowy będzie oczywiście próbował dostać się do umysłu przeciwnika. Tak naprawdę dzieje się tak tylko wtedy, gdy jedna ze stron ma przytłaczającą przewagę w~swojej zdolności do wyrządzenia krzywdy fizycznej. Oczywiście, gdy jedna ze stron ma przytłaczającą przewagę, rzadko musi uciekać się do fizycznych ataków: zwykle wystarczy zagrożenie. Ale to oznacza, paradoksalnie, że najbardziej charakterystyczna cecha przemocy -- jej zdolność do narzucania bardzo prostych relacji społecznych, które wymagają niewielkiej lub żadnej wyobrażeniowej identyfikacji -- staje się najbardziej widoczna w~sytuacjach, w~których prawdopodobieństwo rzeczywistej przemocy fizycznej jest często najmniejsze.

 Pozwolę sobie odwołać się w~tym miejscu do pojęcia ,,przemocy strukturalnej'', systematycznych nierówności wspartych groźbą użycia siły. Systemy przemocy strukturalnej niezmiennie wytwarzają skrajnie krzywe struktury identyfikacji wyobrażeniowej. Nie chodzi o to, że nie wykonuje się prac interpretacyjnych. Społeczeństwo w~jakiejkolwiek rozpoznawalnej formie nie mogłoby bez niej funkcjonować. Raczej przytłaczający ciężar pracy spada na ofiary. 

 Zacznę od przykładu z~najbardziej intymnego miejsca, z~domu. Stałym elementem komedii sytuacyjnych lat pięćdziesiątych w~Ameryce były żarty o niemożności zrozumienia kobiet. Dowcipy oczywiście zawsze opowiadali mężczyźni. Logika kobiet zawsze była traktowana jako obca i~niezrozumiała. Z drugiej strony nigdy nie odniosło się wrażenia, że  kobiety mają duże problemy ze zrozumieniem mężczyzn. To dlatego, że kobiety nie miały innego wyboru, jak zrozumieć mężczyzn. Był to okres rozkwitu patriarchalnej rodziny, a kobiety niemające dostępu do własnych dochodów ani środków nie miały innego wyboru, jak tylko poświęcić sporo czasu i~energii na zrozumienie, co według nich się dzieje. Wynikająca z~tego dysproporcja została po prostu odtworzona w~wyidealizowanych wersjach rodziny prezentowanych w~telewizji. Właściwie, tego rodzaju retoryka o tajemnicach kobiecości jest odwieczną cechą rodzin patriarchalnych: struktury, które rzeczywiście można uznać za formy przemocy strukturalnej, o ile władza mężczyzn nad kobietami w~nich jest, jak przypominają nam pokolenia feministek, ostatecznie wspierana w~sposób czasami ukryty, czasami nie tak bardzo ukryty, pod groźbą przemocy. Jednocześnie pokolenia powieściopisarek -- od razu przychodzi mi na myśl Virginia Woolf -- udokumentowały również drugą stronę tego: nieustanną pracę, jaką kobiety wykonują w~zarządzaniu, utrzymywaniu i~dostosowywaniu do ego pozornie nieświadomych mężczyzn, co wymagało niekończącej się pracy identyfikacji wyobrażeniowej i~tego, co nazwałem pracą interpretacyjną. To ma miejsce na każdym poziomie. Kobiety zawsze wyobrażają sobie, jak wyglądają rzeczy z~męskiego punktu widzenia. Mężczyźni prawie nigdy nie robią tego samego dla kobiet. Jest to prawdopodobnie powód, dla którego w~tak wielu społeczeństwach z~wyraźnym podziałem pracy ze względu na płeć kobiety wiedzą bardzo dużo o tym, co mężczyźni robią na co dzień, a mężczyźni nie mają pojęcia, co robią kobiety. W rzeczywistości wielu mężczyzn reaguje na sugestię takiego wyobrażeniowego utożsamienia prawie tak, jakby był to akt przemocy. Być może najbardziej wymowne w~tym względzie jest ćwiczenie popularne wśród nauczycieli kreatywnego pisania w~szkołach średnich, którzy od czasu do czasu proszą uczniów o napisanie eseju, w~którym wyobrażają sobie, że zmienili płeć, i~opisanie, jak by to było żyć przez jeden dzień jako członek o odmienna płci. Ci, którzy przeprowadzili eksperyment, niezmiennie zgłaszają dokładnie te same wyniki: wszystkie dziewczęta w~klasie piszą długie i~szczegółowe eseje, pokazujące, że poświęciły dużo czasu na rozmyślanie nad takimi pytaniami; mniej więcej połowa chłopców całkowicie odmawia napisania eseju. 

 Mnożenie przykładów powinno być łatwe. To samo dzieje się na poziomie mikro: powiedzmy w~miejscach pracy. Kiedy coś pójdzie nie tak w~kuchni restauracji, a szef wydaje się oceniać sytuację, raczej nie będzie zwracał uwagi na grupę pracowników usiłujących wyjaśnić, co się stało. Prawdopodobnie powie im wszystkim, żeby się zamknęli i~po prostu arbitralnie zadecydowali, co według niego mogło się wydarzyć: ,,Jesteś nowym facetem, musiałeś schrzanić. Jeśli zrobisz to ponownie, zostaniesz zwolniony''. To ci, którzy nie mają mocy zwalniania, muszą wykonać pracę polegającą na ustaleniu, co tak naprawdę poszło nie tak, aby upewnić się, że nie wydarzy się to następnym razem. Podobne rzeczy zdarzają się również systematycznie w~całym społeczeństwie. Co dziwne, był to Adam Smith w~swojej Teorii uczuć moralnych (1761), który jako pierwszy zwrócił uwagę na to, co obecnie nazywa się ,,zmęczeniem współczuciem''. Zauważył, że istoty ludzkie wydają się mieć naturalną tendencję nie tylko do wyobrażeniowego utożsamiania się ze swoimi bliźnimi, ale także, w~rezultacie, do rzeczywistego odczuwania nawzajem radości i~bólu. Jednak biedni są po prostu zbyt konsekwentnie nieszczęśliwi. W obliczu takiej perspektywy obserwatorzy mają tendencję do ich po prostu wymazywania. W rezultacie ci na dole spędzają dużo czasu wyobrażając sobie perspektywy i~- istoty ludzkie są istotami sympatycznymi -- w~rzeczywistości troszczą się o tych na górze, ale prawie nigdy nie dzieje się to odwrotnie\footnote{Powinienem się zakwalifikować: dzieje się to znacznie rzadziej, w~znacznie mniej zindywidualizowany i~realistyczny sposób. Osoby posiadające przywileje często wykazują troskę o uciskanych, ale ta troska jest często ogólna i~oparta na prawie całkowitej nieznajomości ich rzeczywistej sytuacji. Oczywiście często dzieje się tak również w~drugą stronę, przynajmniej wtedy, gdy nie mówimy o intymnych związkach, ale o chłopach zajmujących się sprawami na dworze lub ludziach czytających gazety o skandalach dla celebrytów.}. Bez względu na mechanizmy, wydaje się to zawsze mieć miejsce: niezależnie od tego, czy mamy do czynienia z~panami i~sługami, mężczyznami i~kobietami, szefami i~robotnikami, bogatymi i~biednymi. Nierówność strukturalna -- przemoc strukturalna -- niezmiennie tworzy wysoce koślawe struktury wyobraźni. A ponieważ uważam, że Smith miał rację, zauważając, że wyobraźnia zwykle niesie ze sobą sympatię, ofiary przemocy strukturalnej rzeczywiście troszczą się o swoich beneficjentów o wiele bardziej niż ci beneficjenci troszczą się o nich. W rzeczywistości może to być najpotężniejsza siła utrzymująca takie relacje, poza samą przemocą\footnote{Chociaż odwołuję się tutaj do szerokiego zakresu teorii feministycznych, najważniejsza jest ,,Teoria stanowiska'' Kluczowymi pracami, z~którymi należy się konsultować, są Patricia Hill Collins, Donna Haraway, Sandra Harding i~Nancy Harstock. Zobacz Harstock 2004 dodatkowo.}.

 Wiele z~tego może wydawać się tak oczywistych, że można by się zastanawiać, dlaczego teoretycy społeczni nie pisali o tym więcej, zwłaszcza biorąc pod uwagę ich niekończące się zainteresowanie zrozumieniem systemów władzy i~nierówności. Podejrzewam, że jednym z~powodów jest to, że po prostu niewiele można powiedzieć o ignorancji i~głupocie. Uczeni wyszkoleni w~interpretacji subtelnych systemów znaczeń mają tendencję do zapętlania się w~sytuacjach, które po prostu nie są zbyt znaczące, a nawet charakteryzują się radykalną negacją znaczenia: na przykład rzeź wojenna, gdzie nieustanny strumień skutecznie przypadkowej (a więc bezsensownej) śmierci, traumy, można by powiedzieć, że tworzy taką próżnię w~tym względzie, że wzbudza desperacką potrzebę u wszystkich zainteresowanych, aby nadać całej sprawie jakieś wyższe znaczenie\footnote{Można by nazwać to znaczenie bez znaczenia. Oderwanie ręki jest oczywiście niezwykle ważne; przynajmniej dla osoby, która do końca życia będzie musiała żyć bez ręki. Jednak oderwanie ręki prawie na pewno nie było znaczącym działaniem: pocisk nie był skierowany konkretnie przeciwko niemu, po prostu przypadkowo znalazł się w~niewłaściwym miejscu w~niewłaściwym czasie. Nie było żadnego powodu, dla którego uderzyło go raczej niż tuzin innych ludzi.}. Innym powodem jest oczywiście akademicka fascynacja relacją władzy i~wiedzy. Z pewnością zrozumienie stopnia, w~jakim systemy wiedzy przyczyniają się do systemów dominacji, ma bezpośrednie znaczenie dla radykalnych uczonych, którzy chcą zastanowić się nad etyką własnej praktyki. Można by argumentować, że byłoby to nieodpowiedzialne. Jednocześnie jednak podejrzewam, że zachęca to naukowców do przekonania, że  mają znacznie większą władzę niż w~rzeczywistości. To chyba nie przypadek, że w~Ameryce fascynacja związkiem władzy i~wiedzy zaczęła się pod koniec lat 70. i~na początku lat 80., dokładnie w~momencie, gdy wielu dawnych aktywistów uczonych było coraz bardziej odciętych od ruchów społecznych i~godzących się z~nauczaniem do końca życia dzieci burżuazji w~akademii. Nadal jest to zjawisko ciągłe. Rozważ badanie biurokracji. Dlaczego prawie wszyscy główni teoretycy społeczni, którzy pisali o biurokratycznych formach organizacji, od Maxa Webera do Michela Foucaulta, zdają się zakładać, że biurokracja faktycznie działa, pomimo faktu, że prawie wszyscy na świecie, w~tym większość biurokratów, odnosi wrażenie, że najistotniejszymi cechami takich form organizacji jest jej idiotyzm i~niekompetencja? Właściwie, moim zdaniem, może nie być nawet uczciwe stwierdzenie, że biurokracja to formy głupoty i~ignorancji. Chodzi raczej o to, że biurokracje wydają się sposobem radzenia sobie z~sytuacjami, które już są głupie lub w~każdym razie naznaczone systematyczną ignorancją, ponieważ są produktami (zazwyczaj) ogromnych nierówności strukturalnych i~(prawie zawsze) państwowego monopolu na ,,siłę''.

 Wiedza biurokratyczna dotyczy oczywiście schematyzacji. W praktyce biurokratyczna procedura niezmiennie oznacza ignorowanie wszelkich subtelności rzeczywistej ludzkiej egzystencji i~sprowadzanie wszystkiego do prostych, z~góry ustalonych formuł mechanicznych lub statystycznych. Czy to kwestia formy, reguł, statystyk, kwestionariuszy, to zawsze kwestia upraszczania. Zwykle nie różni się to tak bardzo od szefa, który wchodzi, by podjąć arbitralną, szybką decyzję dotyczącą tego, co poszło nie tak: jest to kwestia zastosowania bardzo prostych szablonów do złożonych, niejednoznacznych sytuacji. Pod tym względem przypomina to trochę samą teorię społeczną. Opis etnograficzny, nawet bardzo dobry, oddaje co najwyżej dwa procent tego, co faktycznie dzieje się w~każdym konkretnym wendecie Nuerów lub balijskiej walce kogutów. Refleksja teoretyczna zwykle skupia się tylko na niewielkiej części \textit{tego}, wyrywając jeden lub dwa wątki z~nieskończenie złożonej tkanki ludzkiej sytuacji i~używając tego jako podstawy do uogólnień, powiedzmy, na temat natury wojny lub natury rytuału. Nie twierdzę, że jestem przeciwny tego typu rozważaniom teoretycznym (właściwie teraz to robię): z~pewnością wierzę, że dzięki takiemu uproszczeniu można dowiedzieć się o świecie rzeczy, których inaczej by się nie dowiedziało. Mimo to, kiedy przechodzi się od opisu do polityki i~ponownie stosuje tego rodzaju uproszczenia w~prawdziwym świecie, wyniki prawdopodobnie pozostawią tych, którzy są zmuszeni do zajmowania się biurokratyczną administracją, z~wrażeniem, że mają do czynienia z~ludźmi, którzy z~jakiegoś arbitralnego powodu postanowili założyć dziwny zestaw okularów, który pozwala im zobaczyć tylko dwa procent tego, co mają przed sobą.

 W tym momencie możemy wrócić do policji, którą już opisałem jako uzbrojonych administratorów niskiego szczebla: jako biurokratów z~bronią. Biorąc pod uwagę to, co powiedziałem, trudno się dziwić, że najczęściej uciekają się do przemocy, gdy kwestionuje się ich prawo do określania sytuacji. Policja jest punktem, w~którym urzeczywistnia się państwowy monopol na legalne użycie siły; gdzie dowolna liczba form przemocy strukturalnej zamienia się w~rzeczywistość. Jeśli przemoc jest siłą zdolną do radykalnego upraszczania złożonych sytuacji społecznych, jeśli biurokracja jest w~dużej mierze metodą systematycznego narzucania takich uproszczonych rubryk, to przemoc biurokratyczna, logicznie rzecz biorąc, powinna polegać przede wszystkim na atakach na tych, którzy nalegają na alternatywne interpretacje. Jednocześnie takie egzekwowanie może być postrzegane jako ,,głupota'' w~najbardziej dosłownym tego słowa znaczeniu. Na przykład teoria rozwoju inteligencji u dzieci Jeana Piageta definiuje inteligencję jako zdolność do koordynowania różnych punktów widzenia. Małym dzieciom trudno jest nawet zrozumieć, że dom wyglądałby inaczej, gdyby był widziany z~innej perspektywy, albo że jeśli ja mam brata o imieniu George, to George ma też brata, którym jestem ja. Rozwój intelektualny staje się więc kwestią możliwości uwzględnienia każdej możliwej perspektywy na sytuację. Inteligencja moralna to oczywiście tylko kolejna wersja tego samego. W tym sensie narzucanie najpierw jednej autorytatywnej perspektywy, a następnie grożenie uderzeniem dużym kijem każdego, kto proponuje alternatywną perspektywę, jest samą definicją wojowniczej głupoty.

 Praktyka anarchistyczna -- zwłaszcza praktyka podejmowania decyzji w~drodze konsensusu -- skłania się do tworzenia prawdziwego moralnego imperatywu potrzeby zintegrowania niewspółmiernych perspektyw. Tam, gdzie dziewiętnastowieczni anarchiści, tacy jak Kropotkin (1909, 1924) proponowali, że wyobraźnia -- przez co miał na myśli identyfikację wyobrażeniową -- jest podstawą moralności, można powiedzieć, że jest to próba faktycznego przekształcenia jej w~jakąś rzeczywistą strukturę instytucjonalną. Nie oznacza to, że przeciętne spotkanie anarchistów obejmuje skomplikowane ćwiczenia w~patrzeniu na sprawy z~punktu widzenia innych, w~rzeczywistości nacisk na wspólne projekty działania pozwala w~dużej mierze pominąć takie ćwiczenia, które są pozostawione do ,,szkolenia'' i~inne wydarzenia edukacyjne, ale zakłada szacunek dla niewspółmiernych perspektyw. Dlatego też aktywiści uważają kontakty z~przedstawicielami państwa policyjnego (jak to nazywają) za tak przerażające. Linia policjantów prewencji jest nie tylko punktem, w~którym przemoc strukturalna przybiera namacalny kształt, ale także właśnie z~tego powodu tworzy rodzaj wyobrażeniowej ściany, bariery, której umysł nie jest w~stanie przeniknąć. Może się to wydawać sprzeczne z~moim wcześniejszym stwierdzeniem, że to beneficjenci przemocy strukturalnej mają tendencję do stawania się obiektem identyfikacji, ale nie sądzę, żeby tak było. W końcu policja sama nie jest beneficjentem przemocy strukturalnej. W przypadku, powiedzmy, szczytu handlowego beneficjentami są politycy i~kadra kierownicza. Policja zostaje złapana dokładnie pośrodku; są dosłownie murem między bankierami a ofiarami. Stąd dziwna ambiwalencja ich stanowiska. Faktycznie, w~tysiącach programów telewizyjnych i~filmów publiczność jest nieustannie proszona o wyobrażanie sobie świata z~perspektywy policjanta, ale zawsze jest to punkt widzenia wyimaginowanych policjantów, niezależnych gliniarzy, którzy spędzają czas na walce z~przestępczością, a nie na rozwiązywaniu problemów administracyjnych lub obsadzaniu barykad. Jak często z~goryczą zauważają policjanci, nic nie wiedzą o prawdziwych gliniarzach i~przeważnie nie chcą mieć z~nimi nic wspólnego.

\section{Dygresja o transcendentnej kontraimmanentnej wyobraźnii}

 Czytelnik może się w~tym momencie zastanawiać, czy nie bawię się trochę pospiesznie i~luźno z~moją terminologią, poruszając się w~rzeczywistości w~tę i~z powrotem, między dwoma zupełnie różnymi znaczeniami słowa ,,wyobraźnia''. W pierwszej części mówiłem o roli wyobraźni w~powstawaniu nowych rzeczy, czy to nowych obiektów materialnych, nowych układów społecznych, czy też rewolucyjnych wizji głęboko nowego społeczeństwa. W drugim dyskutowałem o identyfikacji sympatycznej; wyobrażanie sobie, jak rzeczy wyglądają z~punktu widzenia innej osoby. Jaki jest powód, by przypuszczać, że te dwie rzeczy mają ze sobą coś wspólnego?

 Myślę, że tak, ale aby zrozumieć, dlaczego tak uważam, warto przyjrzeć się historii słowa ,,wyobraźnia''. Jak zauważył m.in. Agamben (1993), w~powszechnej koncepcji starożytnej i~średniowiecznej to, co nazywamy ,,wyobraźnią'', było uważane za strefę przejścia między rzeczywistością a rozumem. Percepcje ze świata materialnego musiały przejść przez wyobraźnię, stając się przy tym naładowane emocjonalnie i~mieszając się z~wszelkiego rodzaju fantazmatami, zanim racjonalny umysł potrafił pojąć ich znaczenie. Intencje i~pragnienia działały w~przeciwnym kierunku. Dopiero po Kartezjuszu słowo ,,wyobrażony'' zaczęło oznaczać w~szczególności wszystko, co nie jest rzeczywiste: wyimaginowane stworzenia, wyimaginowane miejsca (Śródziemie, Narnia, planety w~odległych galaktykach, Królestwo Prester Johna), wyimaginowani przyjaciele. Zgodnie z~tą definicją, oczywiście ,,polityczna ontologia wyobraźni'' jest w~rzeczywistości sprzecznością pojęciową. Wyobraźnia nie może być podstawą rzeczywistości. Jest to z~definicji to, o czym możemy myśleć, ale nie jest prawdziwe.

 To ostatnie znaczenie będę nazywał ,,transcendentnym pojęciem wyobraźni'', ponieważ wydaje się, że wywodzi się z~opowiadań i~innych fikcyjnych tekstów, które tworzą wyimaginowane światy, które przypuszczalnie pozostają takie same bez względu na to, ile razy się je czyta. Rzeczywisty świat nie ma wpływu na wyimaginowane stworzenia -- elfy czy jednorożce. Nie może mieć, ponieważ nie istnieją. Jednak żaden z~dwóch sposobów użycia wyobraźni, których używałem do tej pory, nie jest podobny do tego. Pod wieloma względami wydają się pozostałościami starszej, immanentnej koncepcji. Przede wszystkim nie są one w~żaden sposób statyczne i~swobodne, ale są całkowicie uwikłane w~projekty działania, które mają na celu wywrzeć realny wpływ na świat materialny. Pierwszy to przede wszystkim moment w~procesie tworzenia lub kształtowania obiektów fizycznych. Drugi to bardziej moment w~procesie tworzenia i~utrzymywania relacji społecznych. Niemniej jednak każde adekwatne pojęcie produkcji -- lub troski, pracy, ludzkiej kreatywności, czy jakkolwiek zechcesz to nazwać -- musi koniecznie dążyć do zrozumienia obu: jeśli nic innego, ponieważ większość rzeczywistych form produkcji, opieki lub pracy nie ogranicza się tylko do jednego lub drugiego.

 Podejrzewam, że głównym problemem jest tutaj termin ,,praca'', który zawsze wydaje się odnosić, jako główny przykład, do pracy w~fabryce. Dla Marksa, podobnie jak dla większości innych myślicieli dziewiętnastowiecznego ruchu robotniczego, ,,praca'' i~,,produkcja'' były kluczowymi kategoriami przede wszystkim dlatego, że przyniosły do  domu paradoks, że ludzie żyli w~świecie, który wspólnie stworzyli i~kontynuowali tworzenie, pomimo faktu, że prawie żadne z~nich nie czuło, że ma dużą kontrolę nad procesem, a gdyby tak było, nie wybraliby stworzenia świata, który wyglądałby podobnie. Instytucja pracy najemnej, która zmuszała robotników do sprzedawania swoich sił twórczych, wydawała się najbardziej dramatyczną i~najgłębszą formą tego ogólnego stanu wyobcowania. Jednakże wydaje mi się, że wynikłe z~tego skupienie się na pracy fabrycznej jako modelu dla wszystkich innych stworzyło bardzo wypaczoną koncepcję tego, jak w~rzeczywistości wygląda praca dla większości ludzi. Na przykład ekonomia polityczna wychodzi od skrajnej dychotomii między miejscem pracy a domem. Pierwsza to miejsce produkcji; drugi to konsumpcja. To już zakłada, że to, co naprawdę ważne, to wyprodukowane towary. Ale nawet jeśli ktoś następnie zwróci uwagę na znaczenie pracy domowej, prowadzi to do bardzo uproszczonej dychotomii między miejscem pracy, jako miejscem, w~którym dobra materialne są wytwarzane przez (głównie mężczyzn) robotników najemnych, a gospodarstwem domowym jako miejscem, w~którym dobra te są utrzymywane (polerowane, zamiatane, czyszczone) przez nieopłacane kobiety, a przede wszystkim jako miejsce opieki, wychowania i~edukacji człowieka. Jest to powszechne w~dzisiejszych czasach (np. Negri 1984), aby powiedzieć, że od tego czasu sprawy stały się bardziej skomplikowane, a zatem staromodne wiktoriańskie teorie wartości pracy nie mają już zastosowania. To bzdura. Sprawy zawsze były bardziej skomplikowane. Z pewnością nigdy w~historii ludzkości nie było społeczeństwa, w~którym większość ludności składałaby się z~robotników przemysłowych i~gospodyń domowych. Aby otworzyć opis życia klasy robotniczej w~europejskim mieście z~czasów Marksa -- powiedzmy, powieści Charlesa Dickensa lub Victora Hugo -- trzeba natychmiast skonfrontować się z~niekończącą się serią postaci, których praca oczywiście nie pasuje do żadnej z~kategorii: kominiarze, mamki, dorożkarze, prostytutki, marynarze, guwernantki, fryzjerzy, zbieracze szmat, skrybowie, policja. Większość z~nas doskonale zdaje sobie z~tego sprawę. Niemniej jednak, mówiąc abstrakcyjnie, mamy tendencję do powracania do tych samych uproszczonych kategorii. Mamy tendencję do zachowywania się tak, jakby ,,praca'' oznaczała albo operacje na świecie fizycznym (tworzenie lub utrzymywanie rzeczy), albo, jeśli ekspansywnie podchodzimy do tego terminu, operacje na świecie społecznym (tworzenie lub utrzymywanie relacji z~innymi ludźmi, lub praca bezpośrednio na ich ciałach lub umysłach), a nie na oba jednocześnie. Niektórzy mogą dalej celebrować ,,produkcję'' jako istotę tego, co czyni nas ludźmi, i~spychać formy pracy skierowane do innych ludzi (prace domowe, opieka nad dziećmi, edukacja) do niższej sfery ,,reprodukcji''. Inni mogą podążać za Hannah Arendt (1958) i~postrzegać politykę -- próby wywierania wpływu lub wywierania wpływu na innych ludzi -- jako istotę tego, co czyni nas ludźmi, a tym samym spychać ,,produkcję'' na drugi poziom. Ale zawsze zakłada się, że są to odrębne dziedziny działalności człowieka.

 W rzeczywistości praca, jak każda inna forma ludzkiej aktywności, zwykle obejmuje po trosze jedno i~drugie, jak zauważył sam Marks, wskazując w~\textit{Ideologii Niemieckiej}, że ,,produkcja'' potrzeb materialnych jest zawsze jednocześnie produkcja ludzi i~relacji społecznych. To prawda, że  kapitalizm przemysłowy jest niezwykły w~wyznaczaniu odrębnej sfery dla kształtowania towarów materialnych. Prawdą jest również, że takie postępowanie prowadzi do zaostrzenia pewnych sprzecznych tendencji dotyczących relacji między pracą a wyobraźnią: to znaczy, że kiedy nierówności pojawiają się w~sferze produkcji materialnej, to zwykle ci na górze zrzucają na siebie bardziej wymyślne zadania (tj. projektują produkty i~organizują halę sklepową)\footnote{Nie jest dla mnie do końca jasne, na ile jest to ogólny wzór, a na ile osobliwa cecha kapitalizmu.}, podczas gdy to samo dzieje się w~sferze produkcji społecznej, to od tych na dole oczekuje się, że wykonają główną pracę wyobraźni (na przykład większość tego, co nazwałem ,,pracą nad interpretacją'', która podtrzymuje życie). Bez wątpienia wszystko to ułatwia postrzeganie tych dwóch jako fundamentalnie różnych rodzajów działalności. Ale w~każdym szerszym spojrzeniu na społeczeństwo jest oczywiste, że tak nie jest, i~o ile można tu dokonać rozróżnienia, to opieka, energia i~praca skierowana na istoty ludzkie muszą być uważane za pierwszorzędne. Rzeczy, na których nam najbardziej zależy -- nasze miłości, pasje, rywalizacje, obsesje -- to zawsze inni ludzie; w~większości społeczeństw przyjmuje się za pewnik, że produkcja dóbr materialnych jest momentem podrzędnym w~szerszym procesie kształtowania (właściwych rodzajów) istot ludzkich.

\section{O alienacji}

 Być może problem polega na tym, że słowo ,,praca'' jest tak skrzywione przez skojarzenia z~pracą w~fabryce, że nie ma terminu, który mógłby łatwo połączyć fundamentalne spostrzeżenie, że świat jest w~dużej mierze czymś, co stworzyliśmy i~co stworzyliśmy, w~dużej mierze, w~procesie projektów mających na celu kształtowanie innych ludzi. Prawdopodobnie musimy zacząć rozwijać nowy język, w~którym to, co zwykle uważa się za ,,pracę kobiet'', jest postrzegane jako podstawowa forma pracy, a inne formy są jedynie wariantami. Na razie naprawdę chcę poruszyć trzy kwestie. Po pierwsze, gdy przestaniemy myśleć o wyobraźni jako w~dużej mierze o wytwarzaniu swobodnie unoszących się światów fantasy, ale raczej o wyobraźni związanej z~procesami, dzięki którym tworzymy i~utrzymujemy rzeczywistość, wtedy ma sens postrzeganie jej jako materialnej siły na świecie, a przynajmniej tak samo jak przemoc. Kreatywność i~pożądanie (to, co często sprowadzamy, w~kategoriach ekonomii politycznej, do ,,produkcji'' i~,,konsumpcji'') są w~istocie wehikułami wyobraźni. Po drugie, struktury nierówności i~dominacji wypaczają ten proces na wiele różnych sposobów. Mogą stwarzać sytuacje, w~których większość pracowników zostaje zdegradowana do nudnych, ogłupiających, mechanicznych prac, a tylko niewielkiej elicie wolno oddawać się pracy z~wyobraźnią. Mogą tworzyć sytuacje społeczne, w~których królowie, politycy lub celebryci paradują, nieświadomi prawie wszystkiego, co ich otacza, podczas gdy ich żony, służba, personel i~opiekunowie spędzają cały swój czas na twórczej pracy, aby utrzymać ich w~swoich fantazjach. Większość sytuacji nierówności łączy elementy obu. Trzecią jest to, że subiektywne doświadczenie życia w~takich wypaczonych strukturach wyobraźni jest dokładnie tym, do czego odnosimy się, gdy mówimy o ,,alienacji''.

 To może pomóc wyjaśnić zarówno, dlaczego polityka alienacji wciąż tak trzyma się mocno młodych aktywistów, długo po tym, jak większość teoretyków społecznych porzuciła tę koncepcję, oraz dlaczego polityka zakorzeniona w~wyobraźni wydaje się oczywistym antidotum. Zauważyłem już osobliwy paradoks, że podczas gdy naukowcy pozostają zafascynowani francuską teorią z~lat po maju '68, większość z~nich bezpośrednio lub pośrednio zmaga się z~pytaniem, dlaczego rewolucyjne marzenia okazały się niemożliwe, anarchiści i~inni aktywiści wciąż czytają i~rozwijają teorii francuskich bezpośrednio przed nimi, takich jak Guy Debord (1967), Raoul Vaneigem (1967) czy Cornelius Castoriadis (1967). Sytuacjoniści w~szczególności byli wielkimi teoretykami mocy alienacji w~życiu codziennym (Castoriadis z~kolei był wielkim teoretykiem wyobraźni rewolucyjnej). Jeśli \textit{Rewolucja życia codziennego }Raoula Vaneigema, książka napisana w~Paryżu w~1967 roku, nadal wydaje się wyrażać frustrację nastolatka w~Nebrasce, to tylko dlatego, że jest to książka, która rozpoczyna się od odczuwania wściekłości, nudy, a wstręt, jaki odczuwa prawie każdy nastolatek w~pewnym momencie w~konfrontacji z~egzystencją klasy średniej, i~zamienia to w~Teorię. Poczucie życia rozbitego na fragmenty, bez ostatecznego sensu i~integralności; o cynicznym systemie rynkowym sprzedającym swoim ofiarom towary i~spektakle, które same w~sobie stanowią maleńkie fałszywe obrazy samego poczucia totalności, przyjemności i~wspólnoty, które rynek zniszczył; skłonność do przekształcania każdej relacji w~formę wymiany, poświęcania życia za ,,przetrwanie'', przyjemności za wyrzeczenie, kreatywności dla pustych, jednorodnych jednostek mocy lub ,,martwego czasu” -- na pewnym poziomie wszystko to wyraźnie nadal brzmi prawdziwie.

 Z drugiej strony, naukowcy zwykle odpowiedzą, że powiedzenie, że w~kapitalizmie jest coś nienaturalnego lub nieludzkiego, oznacza założenie, że istnieje jakaś naturalna, ludzka istota, z~którą można go porównać. Każda teoria alienacji zakłada, że  istnieje jakaś podstawowa natura ludzka, jakiś rodzaj podstawowej jaźni, która jest sfrustrowana, niezrealizowana lub odrzucona. Prawie cała teoria poststrukturalistyczna odrzuciłaby tę logikę z~góry; jest zdecydowanie antyhumanistyczna. To potężny argument. Na jakiej możliwej podstawie możemy argumentować, że niektóre społeczności ludzkie są bardziej ludzkie niż inne? Ale wynik jest dziwnie odpolityczniający. A może lepiej byłoby powiedzieć, że rezultat wydaje się prowadzić do liberalnej polityki, w~której, jeśli w~ogóle można mówić o ,,alienacji'', to tylko jako subiektywne doświadczenie marginalizacji lub wykluczenia. Jest to w~dużej mierze kierunek, który obrała to, co obecnie często określa się mianem ,,postmodernistycznej'' teorii alienacji (np. Geyer 1992, Geyer i~Heinz 1996; Schmidt i~Moody 1994), w~której mówi się, że alienacja występuje głównie wtedy, gdy ktoś zderza się z~samookreśleniem. ze sposobem, w~jaki jest się definiowanym lub kategoryzowanym w~ramach większego społeczeństwa. Alienacja staje się zatem subiektywnym sposobem, w~jaki różne formy ucisku (rasizm, seksizm, ageizm itp.) są faktycznie doświadczane i~internalizowane przez ich ofiary. Może się to wydawać użyteczną korektą dla sytuacjonistycznej czy wręcz klasycznej literatury marksistowskiej, która nie ma prawie nic do powiedzenia na temat struktur wykluczenia. Jak studenci zawsze będą podkreślać, czytając literaturę sytuacjonistyczną w~dzisiejszych czasach, autorzy nie mieli prawie nic do powiedzenia na temat rasizmu, seksizmu czy homofobii. Mimo to wydaje mi się, że może to zabrzmieć przewrotnie, to właśnie jest tak potężne w~ich pracy. Jeśli wyobrazimy sobie kapitalizm jako grę, to jedną rzeczą jest opłakiwanie losu przegranych lub wskazywanie, że większość graczy przegra, a nawet, że zasady są napisane tak niesprawiedliwie, że pewne kategorie graczy skazane są na przegraną. Co innego powiedzieć, że gra niszczy dusze nawet tych, którzy wygrali. Powiedzieć to drugie, to powiedzieć, że gra jest po prostu bezcelowa. Nawet nagroda jest zła. Pierwsza linia argumentacji może prowadzić do reformistycznej polityki wzywającej do większej integracji, lub może prowadzić do rewolucyjnej polityki wojny klasowej (lub przypuszczalnie wojny rasowej, etnicznej, a nawet płci), ale tylko druga linia pozwala na możliwość prowadzenia polityki, która jest zarówno rewolucyjna, jak i~prowadząca do powszechnego wyzwolenia. Wydaje mi się, że nie jest to rodzaj wizji, z~której rewolucjoniści powinni chcieć zrezygnować bez niezwykle ważnego powodu, i~trudno się dziwić, że tak wielu -- zwłaszcza tych, którzy nie pochodzą z~żadnej wyraźnie uciskanej grupy -- nie uważa poststrukturalistycznej krytyki tematu za wystarczająco przekonującą\footnote{Oczywiście prowadzi to z~powrotem do problemu, który zauważyłem w~rozdziale 6: mimo to nikt nie będzie kłócił się, że rozpacz bogatego białego człowieka z~powodu uświadomienia sobie, że jego życie jest bezsensowne, jest całkiem równoznaczna z~rozpaczą matki dziecka z~Mozambiku, które umiera na chorobę, której można zapobiec. Alienacja i~ucisk nie są ekwiwalentami. Sugerują jednak, że lepszy świat w~jakimś stopniu odniósłby korzyści.}.

 Jak uratować taką wizję? Myślę, że niektóre z~pomysłów, które starałem się rozwinąć w~trakcie tego rozdziału, mogą być przydatne. Jeśli alienacja jest po prostu subiektywnym doświadczeniem życia w~wypaczonych strukturach wyobraźni, które zawsze wydają się wywoływać formy nierówności społecznej, to większość konwencjonalnych zastrzeżeń wobec tej koncepcji znika. W końcu, co to znaczy powiedzieć, że nie ma czegoś takiego jak istotna natura ludzka lub ,,transcendentalny podmiot''. Ktoś powie, że te rzeczy są urojone. Dokładnie. A co w~tym złego\footnote{W rzeczywistości, jak to się dzieje, większość teorii poststrukturalnych ma tendencję do patrzenia okiem nawet na wyimaginowane totalności - tendencja, która wywodzi się z~pojęcia ,,specularity'' Lacana i~stadium lustrzanego, pierwszego stworzenia przez niemowlę jednolitego poczucia siebie, które jest zawsze budowany wokół jakiegoś zewnętrznego obiektu. Lacan nazywa ten rodzaj logiki ,,wyobrażeniowym'' i~zestawia ją z~bardziej dojrzałym ,,symbolicznym'' etapem, który pojawia się wraz z~językiem. Tendencja do dyskontowania wyobrażeniowego i~zwierciadlanego jako zasadniczo infantylnego powraca w~całej pracy Lacana. Sytuacjoniści, widząc spektakl i~formę towarową jako wyimaginowane całości, pragnęli zrekompensować brak jakiegokolwiek poczucia całości czy wspólnoty w~życiu codziennym, wyraźnie odwołują się do tego samego rodzaju logiki, ale są wystarczająco przywiązani do tradycji dialektycznej, że jedność, całość są postrzegane jako dobre rzeczy. Począwszy przynajmniej od Deleuze'a i~Guattariego dialektyka i~totalność są definitywnie odrzucane jako wartości; zamiast tego model jest jaźnią jako splotem ,,przepływów'' W każdym razie z~rozwiniętej tu perspektywy łatwo zauważyć, że cała ta literatura zakłada wyobrażenia transcendentalne, a nie immanentne: te totalności są postrzegane jako swobodne, całkowicie oderwane od świata, a nie jako momenty w~procesie działania lub kreacja.}? Można by argumentować, że jeśli istnieje jakakolwiek ludzka istota, to właśnie nasza zdolność do wyobrażenia sobie, że ją posiadamy. To z~kolei nie odbiegałoby zbytnio od punktu wyjścia Marksa: że jeśli istnieje coś zasadniczo ludzkiego, to jest nim zdolność wyobrażania sobie rzeczy i~wprowadzania ich w~życie (to, co nazywam wyobraźnią immanentną) i~ta alienacja występuje, gdy tracimy kontrolę nad procesem.

 Jeśli wyobraźnia jest rzeczywiście elementem składowym w~procesie wytwarzania naszej społecznej i~materialnej rzeczywistości, to istnieją wszelkie powody, by sądzić, że działa ona poprzez tworzenie obrazów całości\footnote{Przedstawiłem ten przypadek w~książce \textit{Toward Anthropological Theory of Value: The False Coin of Our Own Dreams }(Graeber 2001), rozdział 3.}. Tak po prostu działa wyobraźnia. Trzeba umieć wyobrażać sobie siebie i~innych jako zintegrowane podmioty, aby móc tworzyć istoty, które są w~rzeczywistości nieskończenie wielokrotne; wyobraź sobie jakieś spójne, ograniczone ,,społeczeństwo'' w~celu wytworzenia tej chaotycznej, otwartej sieci relacji społecznych, która rzeczywiście istnieje. Być może istnieje w~tym sprzeczność, ale wydaje się, że większość ludzi w~historii ludzkości wymyśliła sposób, aby z~nią żyć. Zwykle to nie wywołuje uczucia wściekłości i~rozpaczy, przekonanie, że świat społeczny jest pustą parodią lub złośliwym żartem. Jeśli w~społeczeństwach kapitalistycznych często tak się dzieje, może to być spowodowane jedynie szczególną intensywnością form przemocy strukturalnej, które tworzy, oraz wypaczeniem i~rozbiciem wyobraźni, które są ich nieuniknionym skutkiem.

 Co może pozwolić nastolatkowi z~Nebraski, wychowanemu w~społeczeństwie na wskroś kapitalistycznym, postrzegać kapitalistyczne stosunki społeczne jako w~jakiś sposób nienaturalne, nieludzkie, nieprzyjazne życiu? Czy to dlatego, że społeczeństwo kapitalistyczne z~konieczności generuje struktury wyobraźni, które sugerują coś poza nim, nawet jeśli odmawia im jakiegokolwiek sensownego życia? A może dlatego, że ,,społeczeństwo kapitalistyczne'' w~rzeczywistości nie istnieje, samo w~sobie nie jest całością, ponieważ kapitalizm po prostu pasożytuje na ogromnej pracy wyobraźni, która tworzy rodziny, przyjaźnie, wynalazki, zobowiązania, idee i~formy współpracy; pracy, która w~swoim działaniu zawsze generuje utopijne obrazy, wobec których kapitalizm musi z~konieczności wydawać się ponury, brutalny, opresyjny, okrutny? Do pewnego stopnia bez wątpienia są to obie te rzeczy. Wszystkie są ostatecznie zakorzenione w~tym samym napięciu: fakt, że aby się reprodukować, kapitalizm, jakkolwiek by go nie definiowano, musi tworzyć nie tylko obrazy wolności, których nigdy nie może faktycznie zapewnić, ale obszary prawdziwej autonomii.

\section{O rewolucji}

 Sytuacjonistycznym rozwiązaniem problemu alienacji było działanie rewolucyjne: tworzenie ,,sytuacji'', w~których można podważyć logikę Spektaklu i~odzyskać własną wyobraźnię. Było to wezwanie do ponownego wymyślenia codziennego życia na zasadzie akcji bezpośredniej, które ostatecznie miało doprowadzić do ogólnego powstania przeciwko wszelkim formom władzy instytucjonalnej, od kapitalistów po biurokratów związkowych. Łatwo zrozumieć, dlaczego ta wizja tak przemawia do pokoleń kolejnych aktywistów, a zwłaszcza anarchistów. Największa różnica między ich perspektywami a perspektywami współczesnych awatarów, takich jak kolektyw CrimethInc, polega na tym, że ci ostatni w~dużej mierze porzucili wiarę w~to, że ten ostatni moment powstania może nastąpić w~najbliższym czasie. Jeśli wydarzenia maja 1968 roku cokolwiek ujawniły, to właśnie to, jeśli nie próbuje się przejąć władzy państwowej, wówczas momenty powstania będą miały inne znaczenie i~inne skutki, nie można już sobie wyobrazić, że reprezentują one fundamentalną, trwałą przerwę, która zapoczątkuje zupełnie nowe społeczeństwo. (W pewnym sensie, oczywiście, jest to tylko uświadomienie sobie czegoś, co zawsze było prawdą). Zamiast tego pozostaje nam niekończąca się walka, świadomość, że faktycznie jesteśmy już w~sytuacji permanentnej rewolucji. Wolność staje się samą walką.

 Może się to wydawać otrzeźwiającą perspektywą w~porównaniu z~upojnymi dniami 1968 roku, kiedy niebo wydawało się gotowe otworzyć się w~każdej chwili, ale niesie ze sobą jedno duże pocieszenie. Oznacza to, że można zacząć doświadczać prawdziwej wolności, a nawet tworzyć wyzwolone terytoria, tu i~teraz. Jak Vaneigem zawsze chętnie nam przypominał, sama idea, że  obowiązkiem rewolucjonisty jest poświęcenie wszelkiej przyjemności i~spełnienia w~bezinteresownie efektywnej pogoni za ,,rewolucją'', jest sama w~sobie tylko lustrzanym odbiciem logiki kapitalizmu. Teraz można zacząć eksperymentować z~innymi sposobami bycia. Z pewnością projekt nie jest pozbawiony sprzeczności i~dylematów. Wiele z~tej książki poświęcono ich zbadaniu. Mimo to można argumentować, że ostatecznie reprezentuje postawę bardziej dojrzałą (w sensie chęci wzięcia odpowiedzialności za własne czyny) niż postawa tych, którzy czuli się sprawcami nieuchronnego rozwoju Historii.

 Nadal, rozważ następujące oświadczenie kolektywu CrimethInc:

 Musimy uczynić naszą wolność wycinając dziury w~tkance tej rzeczywistości, wykuwając nowe rzeczywistości, które z~kolei nas ukształtują. Nieustanne stawianie się w~nowych sytuacjach jest jedynym sposobem, aby upewnić się, że podejmujesz decyzje nieobciążone bezwładnością przyzwyczajeń, zwyczajów, prawa lub uprzedzeń, a to od Ciebie zależy, czy stworzysz takie sytuacje.

 Wolność istnieje tylko w~momencie rewolucji. A te chwile nie są tak rzadkie, jak myślisz. Zmiana, rewolucyjna zmiana, zachodzi nieustannie i~wszędzie, i~każdy odgrywa w~niej rolę, świadomie lub nie (CrimethInc 2003).

 Czymże jest to, jeśli nie eleganckim stwierdzeniem logiki działania bezpośredniego: wyzywającym naleganiem na działanie tak, jakby ktoś był już wolny. Oczywistym pytaniem jest, jak może przyczynić się do ogólnej strategii, która powinna prowadzić do kumulatywnego ruchu w~kierunku świata bez państw i~kapitalizmu. Nikt nie jest całkowicie pewien. Większość zakłada, że może to być tylko kwestia niekończącej się improwizacji. Z pewnością będą momenty powstańcze. Prawdopodobnie nie, całkiem sporo z~nich.

 Z perspektywy czasu uderzająco naiwne wydaje się stare założenie, że pojedyncze powstanie lub udana wojna domowa może niejako zneutralizować cały aparat przemocy strukturalnej -- przynajmniej na określonym terytorium narodowym -- że prawicowa rzeczywistość może zostać po prostu zmieciona z~dala, aby pozostawić pole otwarte na nieskrępowany wylew rewolucyjnej kreatywności. Ale naprawdę zastanawiające jest to, że w~pewnych momentach historii ludzkości wydawało się, że to właśnie się działo. Wydaje mi się, że jeśli mamy mieć jakiekolwiek szanse na zrozumienie nowej, rodzącej się koncepcji rewolucji, trzeba zacząć od ponownego przemyślenia jakości tych powstańczych momentów, kiedy ludzie czuli, że są w~obecności Rewolucji, właściwie.

 Jedną z~najbardziej niezwykłych rzeczy w~momentach powstania jest to, że wydają się pojawiać znikąd, a potem często równie szybko znikają. Jak to się dzieje, że ta sama ,,publiczność'', która dwa miesiące przed, powiedzmy, Komuną Paryską lub hiszpańską wojną domową, głosowała w~dość umiarkowanym reżimie socjaldemokratycznym, nagle okaże się gotowa ryzykować życie dla tych samych ultraradykałów, którzy otrzymali ułamek rzeczywistych głosów? Albo, wracając do maja 1968 roku, jak to jest, że ta sama opinia publiczna, która wydawała się popierać lub przynajmniej silnie sympatyzować z~powstaniem studencko-robotniczym, mogła niemal natychmiast po tym wrócić do urn i~wybrać prawicowy rząd? Najczęstsze wyjaśnienia historyczne -- że rewolucjoniści tak naprawdę nie reprezentowali społeczeństwa ani jego interesów, ale elementy opinii publicznej być może zostały uwikłane w~jakiś rodzaj irracjonalnego wrzenia -- wydają się oczywiście nieadekwatne. Przede wszystkim zakładają, że ,,społeczeństwo'' jest podmiotem, którego opinie, interesy i~lojalność uważa się za względnie spójne w~czasie. W rzeczywistości to, co nazywamy ,,publicznością'', jest tworzone, produkowane przez określone instytucje, które pozwalają na określone formy działania -- przeprowadzanie sondaży, oglądanie telewizji, głosowanie, podpisywanie petycji lub pisanie listów do wybranych urzędników lub uczestnictwo w~przesłuchaniach publicznych -- a nie inne. Te ramy działania implikują określone sposoby mówienia, myślenia, argumentowania, rozważania. Ta sama ,,publiczność'', która może szeroko oddawać się używaniu rekreacyjnych chemikaliów, może również konsekwentnie głosować za delegalizacją takich pobłażań; ten sam zbiór obywateli może podejmować zupełnie inne decyzje w~kwestiach dotyczących ich społeczności, jeśli zostanie zorganizowany w~system parlamentarny, system plebiscytów komputerowych lub system demokracji bezpośredniej. W rzeczywistości cały anarchistyczny projekt ponownego wynalezienia demokracji bezpośredniej opiera się na założeniu, że tak właśnie jest.

 Aby zilustrować, co mam na myśli, rozważmy, że nawet w~Ameryce dokładnie ten sam zbiór ludzi można określić w~jednym kontekście jako ,,społeczeństwo'', a w~innym jako ,,siłę roboczą''. Oczywiście staje się ,,siłą roboczą'', gdy angażuje się w~różne rodzaje działalności. ,,Publiczność'' nie działa, przynajmniej takie zdanie jak ,,większość amerykańskiej pracy publicznej w~przemyśle usługowym'' nie pojawiłoby się w~czasopiśmie lub gazecie. Jest to szczególnie dziwne, ponieważ społeczeństwo w~rzeczywistości musi iść do pracy: dlatego, jak często skarżą się lewicowi krytycy, media zawsze będą mówić o tym, jak, powiedzmy, strajk transportowy może spowodować niedogodności dla społeczeństwa, dla osób dojeżdżających do pracy, ale nigdy nie przyjdą im do głowy, że strajkujący sami są częścią społeczeństwa lub że jeśli uda im się podnieść poziom płac, będzie to z~korzyścią dla społeczeństwa. A na pewno ,,publiczność'' nie wychodzi na ulice. Poza wspomnianą wcześniej rolą publiczności jakiegoś publicznego Spektaklu, publiczność pojawia się głównie jako konsumenci usług publicznych. Kupując lub korzystając z~prywatnych (zamiast publicznie) dóbr i~usług, ten sam zbiór jednostek staje się ,,konsumentami'', tak jak w~innych kontekstach działania jest określany jako ,,naród'', ,,elektorat'', ,,populacja''. 

 Wszystkie te podmioty są wytworem instytucji i~praktyk instytucjonalnych, które z~kolei definiują pewne horyzonty możliwości. Dlatego też, głosując w~wyborach parlamentarnych, można czuć się zobowiązanym do dokonania ,,realistycznego'' wyboru; z~drugiej strony w~sytuacji powstańczej nagle wszystko wydaje się możliwe.

 Duża część współczesnej myśli rewolucyjnej zasadniczo pyta: czym zatem staje się ta masa ludowa w~takich momentach powstania? Przez ostatnie kilka stuleci konwencjonalną odpowiedzią był ,,lud'', istota, która twierdziła, że  dzierży władzę niegdyś posiadaną przez królów, nawet jeśli wydaje się, że w~pełni wykorzystuje tę władzę w~momentach powstania. Często zauważa się, że legitymizacja nowoczesnych porządków konstytucyjnych, które zawsze twierdziły, że opierają się na woli ,,ludu'', w~rzeczywistości wywodzi się z~czasów, kiedy ludzie powstali z~bronią w~ręku, by obalić istniejący wcześniej porządek prawny. Mimo to, jak zaczęło wskazywać wielu radykalnych myślicieli, ,,lud'' jako paradygmat zawsze stanowił problem właśnie z~tego powodu. Jest wyobrażany jako ograniczona, jednorodna masa jednostek, w~zasadzie surowiec dla państwa narodowego. Co więcej, jak zauważył Toni Negri (1992), wydaje się mieć nieuniknioną tendencję do biurokratyzacji. Wybuchom ludowej kreatywności zawsze towarzyszy proces instytucjonalizacji, tworzenie pewnego rodzaju aparatu -- pisanie konstytucji, zwoływanie parlamentów, opracowywanie zasad i~procesów formalnych -- który zawsze zdaje się kończyć wymazywaniem tego, co umożliwiła stworzyć ta popularna kreatywność, nawet jeśli twierdzi, że jest jej ostatecznym źródłem legitymizacji. Czy można było wyobrazić sobie fundamentalnie inny rodzaj Podmiotu Rewolucyjnego? Mając to na uwadze, Negri, a za nim wielu innych myślicieli francuskich i~włoskich (Negri 1991 itd., Virno 2004, także Montag, Moulier, Balibar) powrócił do literatury politycznej siedemnastowiecznej Europy i~uchwycił pojęcie ,,Wielości''. W tym nowym odczytaniu ,,Wielość'' staje się wszystkim, czym ,,ludzie'' nie są. Jest to otwarta sieć ,,osobliwości'', zmieniająca się mieszanka podobieństw, sojuszy, a przede wszystkim form współpracy, zjednoczonych wspólnym sprzeciwem wobec państwa i~kapitału. Szczególnie w~Europie aktywiści uznali ten pomysł za bardzo pociągający. Jest z~pewnością znacznie bardziej podatna na milczącą filozofię akcji bezpośredniej: na przykład w~jej założeniu, że takie siły ludowe nigdy nie mogą zostać zredukowane do jednej perspektywy, jednej logiki, jednej świadomości, ani też nie powinny; lub jego gotowość do spojrzenia na formy radykalnej wolności, które już wyłaniają się w~fałdach kapitalizmu. Nadal, w~terminach, które próbowałem rozwinąć w~tym rozdziale, wszystko to raczej mija się z~celem. Nie chodzi o zdefiniowanie rzeczy i~nadanie jej nazwy. To przede wszystkim zrozumienie odpowiednich struktur i~ram działania.

 W terminach, które rozwinąłem, to, co ,,społeczeństwo'', ,,siła robocza'', ,,konsumenci'', ,,populacja'' mają wspólnego, to to, że są tworzone przez zinstytucjonalizowane ramy działania, które są z~natury biurokratyczne, a zatem głęboko alienujące. Kabiny do głosowania, ekrany telewizyjne, boksy biurowe, szpitale, otaczający je rytuał, można by powiedzieć, że są to maszyny wyobcowania. Są instrumentami, dzięki którym rozbija się i~rozbija ludzka wyobraźnia. Momenty powstańcze następują, gdy ten biurokratyczny aparat zostaje zneutralizowany. Takie postępowanie zawsze wydaje się skutkować otwarciem horyzontów możliwości. Można się tego spodziewać tylko wtedy, gdy jedną z~głównych rzeczy, które aparat zwykle robi, jest egzekwowanie skrajnie ograniczonych możliwości. (Prawdopodobnie dlatego, jak zauważyła Rebecca Solnit (2005), ludzie często doświadczają czegoś bardzo podobnego podczas klęsk żywiołowych). Takie chwile również wydają się wyzwalać ludzką wyobraźnię: przynajmniej po rewolucyjnych momentach zawsze następuje wylew kreatywności, społecznej, artystycznej i~intelektualnej. Ponownie, nie jest to aż tak zaskakujące, jeśli ich głównym efektem jest zniszczenie istniejących ram; można by się spodziewać, że po takim zniszczeniu nastąpi ogromny wybuch improwizacji ze wszystkich stron. Zwykle nierówne struktury identyfikacji wyobrażeniowej są zaburzone; każdy eksperymentuje z~wyobrażeniem sobie nieznanych punktów widzenia. Zwykle nierówne struktury kreatywności zostają zakłócone; każdy czuje nie tylko rację, ale (ponieważ rewolucje przynoszą wszelkiego rodzaju sytuacje kryzysowe) natychmiastową praktyczną potrzebę odtworzenia i~ponownego wyobrażenia sobie wszystkiego, co go otacza. Zrozumiałe jest, że ktoś, kto przeżyje taki moment, może odczuć to jako zniszczenie sztucznych granic i~powrót do bardziej naturalnego stanu istnienia, takiego, w~którym każdy jest w~stanie odzyskać własną wyobraźnię. Po części dlatego, że nie jest to całkowicie nieprawdziwe.

 Jeśli sprawy są bardziej skomplikowane, to oczywiście dlatego, że to, co się dzieje, nie przytrafia się jednostkom. To proces społeczny. W rzeczywistości w~dużej mierze jest to zerwanie tych społecznych ograniczeń, które paradoksalnie definiują nas jako jednostki izolowane. Przecież dla autorów, od Kierkegaarda po Durkheima, alienacja, która jest warunkiem współczesnego życia, nie jest wcale doświadczeniem ograniczeń, lecz jej przeciwieństwem. ,,Wyobcowanie'' to niepokój i~rozpacz, z~którymi mamy do czynienia, gdy mamy do czynienia z~niemal nieskończonym zakresem wyborów, przy braku jakichkolwiek większych struktur moralnych, dzięki którym nadałyby one sens. Jednak z~perspektywy aktywisty jest to po prostu kolejny efekt zinstytucjonalizowanych ram: przede wszystkim dzieje się tak, gdy jesteśmy przyzwyczajeni do wyobrażania sobie siebie przede wszystkim jako konsumentów. W przypadku braku rynku, niemożliwe byłoby wyobrażenie sobie ,,wolności'' jako serii wyborów dokonywanych w~izolacji; zamiast tego wolność może oznaczać jedynie wolność wyboru rodzaju zobowiązań, jakie chce się podjąć wobec innych, oraz, oczywiście, doświadczenie życia tylko pod tymi ograniczeniami, które się wybrały z~własnej woli. W tym miejscu chciałbym odnieść się do uwagi Jessiki podczas treningu konsensusu w~rozdziale 7, że dziwnie przyjemne uczucie decydowania o własnej opinii nie jest aż tak ważne. Poddanie się osądowi większej grupy może być w~rzeczywistości doświadczane jako wolność, o ile wiadomo, że się nie musi poddać, że w~każdej chwili można wycofać swoją zgodę. W każdym razie tak jak w~momentach rewolucji zinstytucjonalizowane struktury państwowe rozpływają się w~zgromadzenia publiczne, a zinstytucjonalizowane struktury kontroli pracy rozpływają się w~samorządność, tak samo rynki konsumenckie ustępują miejsca towarzyskim i~zbiorowym świętowaniom. Spontaniczne powstania są prawie zawsze doświadczane przez uczestników karnawału; doświadczenie, które osoby planujące akcje masowe -- jak widzieliśmy -- często całkiem świadomie próbują odtworzyć. Oczywiście nic z~tego nie jest prostym powrotem do stanu natury. Nowe relacje społeczne muszą być improwizowane i~tworzone, a robienie tego zwykle wiąże się z~własnym rodzajem (znowu, często na wpół świadomego) fetyszyzmu\footnote{Przynajmniej tak twierdziłem w~przeszłości (Graeber 2006).}, których marionetki i~zaklęcia są tylko najbardziej oczywistymi przykładami. Nie jest przypadkiem, że tak wielu prymitywistów, którzy najbardziej rygorystycznie odrzucają jakąkolwiek możliwą formę wyobcowania i~którzy naprawdę wierzą, że można powrócić do stanu natury, nie ukrywa swojej pogardy dla marionetek jako bezużytecznych drobiazgów. Bardziej interesuje ich symboliczna siła rozbijania ram niż ich ponowne budowanie. (,,Pragnienie destrukcji jest także pragnieniem twórczym''). Jednak, jak powiedziałem, jest to moment, w~którym stajemy twarzą w~twarz z~siłą kolektywnej kreatywności, która dla aktywistów jest podstawą Prawdziwego, i~które, jak każda najwyższa moc w~anarchistycznej kosmologii, musi być jednocześnie niesamowita i~śmieszna.

 Wszystko to sprawia, że  łatwiej zrozumieć, dlaczego niektórzy mogą rozważać cały projekt nadania nazwy rewolucyjnej ,,Wielości'', a następnie zacząć szukać dynamicznych sił, które za nimi leżą, jako pierwszy krok tego samego procesu instytucjonalizacji, który musi w~końcu zabić tę istotę, w~imieniu której miałaby przemówić. Podmioty (publiczność, ludność, siła robocza) są tworzone przez specyficzne struktury instytucjonalne, które zasadniczo stanowią ramy działania. Są tym, co robią. To, co robią rewolucjoniści, to przełamywanie tych ram, aby stworzyć nowe horyzonty możliwości, akt, który następnie pozwala na radykalną restrukturyzację wyobraźni społecznej. Jest to być może jedyna forma działania, której z~definicji nie można zinstytucjonalizować. Jak zasugerował Colectivo Situaciones w~Argentynie: może lepiej byłoby mówić tutaj nie o władzy konstytuującej, ale o władzy pozbawionej.

 Jeśli istnieje sposób na zinstytucjonalizowanie tego doświadczenia, tego zawrotnego przestawienia sił wyobraźni, to właśnie poprzez doświadczenie bezpośredniego działania. To jest przecież to, co ci, którzy organizują festiwale oporu, celowo starają się osiągnąć: wszystko, co w~doświadczeniu spontanicznego powstania jest najpotężniejsze. Efekt jest jednak taki, jakby wszystko działo się odwrotnie. W przypadku powstania zaczyna się od bitew na ulicach, wylewów ludowego wrzawy i~święta. Następnie przechodzi się do trzeźwego biznesu tworzenia nowych instytucji, rad, procesów decyzyjnych, a ostatecznie ponownego wymyślenia codziennego życia. Przynajmniej taki jest ideał, w~historii ludzkości było wiele momentów, w~których coś takiego zaczęło się dziać, chociaż takie spontaniczne kreacje zawsze wydają się kończyć jako podporządkowane jakiejś nowej formie biurokratycznego państwa. Można powiedzieć, że ruch akcji bezpośredniej przebiega właśnie w~innym kierunku. Często uczestnicy angażują się poprzez subkultury, które polegają na odkrywaniu na nowo codziennego życia. Nawet jeśli nie, zaczyna się od opracowania nowych form podejmowania decyzji -- rad, zgromadzeń, ,,procesu'' -- i~wykorzystuje je do planowania akcji ulicznych i~uroczystości ludowych. Część i~tej książki zawiera szczegółowy opis takiego wysiłku, który w~rzeczywistości zakończył się sytuacją bliską powstania, rodzajem skromnego powstania ludowego w~robotniczych dzielnicach miasta Quebec. Nadal, nikt zaangażowany nie rozważał poważnie możliwości, że mogą wywołać rewolucję w~tradycyjnym, mesjanistycznym sensie. Nawet ci, którzy pracują nad stworzeniem warunków do powstania, nie widzą w~nich zasadniczych przerw w~rzeczywistości, ale raczej jako coś na wzór chwilowych reklam, lub lepiej, przedsmaków, doświadczeń wizjonerskich inspiracji, dla znacznie wolniejszego, żmudnego wysiłku tworzenia alternatywy.

 To tutaj chciałbym podkreślić przede wszystkim wpływ feminizmu. Historycznie rzecz biorąc, współczesny anarchistyczny nacisk na proces wyłonił się -- jak zauważyłem w~Rozdziale 5 -- przede wszystkim z~kryzysów organizacyjnych w~kolektywach feministycznych pod koniec lat 60. i~na początku lat 70. XX wieku. To właśnie w~końcu skłoniło organizatorów do poważnego spojrzenia na praktykę kwakrów i~ostatecznie do opracowania całego aparatu grup afinicji, rad, konsensusu i~facylitacji. Co więcej, widać nacisk feminizmu w~całym kierunku ruchu. ,,Sytuacje'' nie tworzą się same. Wymagają ogromnego nakładu pracy. Oczywiście przez większą część historii ludzkości to, co było uważane za politykę, składało się z~serii teatralnych scen i~wystawianych na nich dramatów. Jednym z~wielkich darów feminizmu dla myśli politycznej jest nieustanne przypominanie nam o ludziach tworzących, przygotowujących i~oczyszczających te etapy, a co więcej, utrzymujących niewidzialne struktury, które je umożliwiają, które w~przeważającej mierze były kobietami. Normalny proces polityki polega na tym, by wszyscy ci ludzie zniknęli. Można powiedzieć, że jednym z~wpływów feminizmu na kręgi działań bezpośrednich było wspieranie nowego ideału politycznego, który ma na celu zatarcie różnicy. Innymi słowy, ten nowy ideał podkreśla, że akcja jest prawdziwie rewolucyjna tylko wtedy, gdy proces tworzenia sytuacji jest tak samo wyzwalający, jak same sytuacje. Cały proces staje się eksperymentem, można by rzec, przestrajaniem wyobraźni, tworzeniem prawdziwie niewyalienowanych form doświadczenia.

 Oczywistym problemem jest to, że podczas gdy ci, którzy działają w~następstwie zwycięskiego powstania, działają przy chwilowym braku władzy państwowej, ci, którzy stosują strategie działań bezpośrednich, nie działają. Jak widzieliśmy, sprawia to, że sprawy są nieskończenie skomplikowane. Zakorzenione struktury ucisku -- rasa, klasa, płeć -- również zbierają znaczne żniwo. Wiąże się z~tym powszechny wzorzec egzaltacji i~wypalenia, którego tak często doświadczają aktywiści. Ci, którzy zostali wciągnięci do ruchu, reagują najpierw z~podziwem, o niemal nieskończonych horyzontach, na odkrycie, że możliwe są radykalnie egalitarne formy organizacji, a następnie na rosnące wyczerpanie w~obliczu państwowych represji i~narastające poczucie rozdrażnienia, gdy odkrywają niekończące się drobne kłopoty, subtelne formy dominacji i~dylematy przywilejów, które wciąż trwają. Nadal, istnieją sposoby, dzięki którym istnienie większych struktur dominacji jest faktycznie zaletą. Nie pozwalają zapomnieć, dlaczego się w~to wciągnęło. Jak podkreśliłem w~rozdziale 7, praktyka aktywistów jest w~dużej mierze zdefiniowana i~stale udoskonalana w~odniesieniu do doświadczenia hierarchicznych alternatyw. Podczas przygotowywania tej książki, na przykład, rozmawiałam z~jedną z~aktywistek, która była najbardziej żarliwie zaangażowana w~Kobiecy Klub DAN i~byłam zdziwiona odkryciem, że większość teraz czuła, że  ich sprzeciwy zostały przesadzone. -- Nie sądzę, żeby DAN był naprawdę seksistowski -- zauważyła Marina. -- Prawdopodobnie zrobiliśmy z~tego więcej problemu, niż naprawdę powinniśmy. -- Biorąc pod uwagę pasję ówczesnych debat, wydało mi się to nieco zaskakujące. Ale potem większość z~nich przeszła do pracy związkowej lub ukończyła szkołę, i~DAN prawdopodobnie wyglądał znacznie lepiej we wspomnieniach.

 Jedyny sposób, w~jaki struktury dominacji naprawdę utrudniają projekt, leży właśnie w~sferze wyobraźni: przede wszystkim w~dławieniu mediów głównego nurtu. Polityczne ontologie przemocy jak nigdy dotąd dominują w~popularnym dyskursie. Co gorsza, większość Amerykanów nie zdaje sobie sprawy, że istnieje ruch, którego celem jest przekształcanie świata poprzez akcje bezpośrednie. O ile są, niemożliwe byłoby rozwinięcie jakiegokolwiek szerokiego pojmowania tego, o co im chodzi, chyba że byliby gotowi dosłownie spędzać dni na surfowaniu w~Internecie. Należy doświadczyć działań, zanim dzieło ,,zanieczyszczania'' może rzeczywiście przynieść efekt, i~chociaż istnieją powody, by sądzić, że dzieje się to powoli, to ruchy społeczne tego rodzaju przekształcają doświadczenie codziennego życia na całym świecie w~niezliczonych, bardzo ważnych sposobach (tutaj feminizm jest z~pewnością najbardziej dramatycznym przykładem), gdzie struktury instytucjonalne okazują się zaskakująco trudne do zbudowania w~jakikolwiek trwały sposób. Infrastruktura pozostaje w~zalążku.

\section{O terrorze}

 Pomyślałem więc, że zakończę kilkoma słowami o polityce globalnej. Być może niektóre z~tych terminów mogą zapewnić świeże spojrzenie na ostatnie wydarzenia historyczne. Jak dotąd w~dużej mierze unikałem omawiania znaczenia 11 września i~następującej po nim ,,wojny z~terrorem'', poza stwierdzeniem, że jakakolwiek próba powrotu do tego rodzaju mobilizacji wojennej na pełną skalę, typowego dla okresu między 1914 a 1989 rokiem, była mało prawdopodobna. Spróbuję sprecyzować nieco bardziej, jak sądzę, szerszy kontekst historyczny. Jest to trudne, ponieważ historia toczy się szybko, a teoretycy społeczni notorycznie nie potrafią przewidywać przyszłości, z~tego, co wiem, sytuacja na świecie ulegnie drastycznej zmianie w~ciągu jednego lub dwóch lat, zanim ta książka faktycznie trafi do druku. Jednak niektóre rzeczy są wystarczająco jasne. Po pierwsze, ruch opisany w~tej książce jest tylko jednym bardzo małym elementem tego, co można by nazwać ogromnym globalnym powstaniem przeciwko neoliberalizmowi; takie, które można wywieść z~powstania zapatystowskiego w~Chiapas w~1994 roku, które po raz pierwszy uderzyło w~amerykańskie wybrzeża w~znaczący sposób w~Seattle w~listopadzie 1999 roku. Chociaż tylko rozproszona mniejszość uczestników kiedykolwiek nazywała siebie ,,anarchistami'', zasady rozwinięte w~tradycji anarchistycznej -- odrzucenie strategii opartych na upadku władzy państwowej, rozwój nowych form demokracji bezpośredniej, zasad horyzontalności, asocjacjonizmu, autonomii, samoorganizacji i~wzajemnej pomocy -- uczyniły to największym świadomym rozkwitem idei anarchistycznych w~historii. Szybko rozwinęły się w~wyraźnie antykapitalistycznym kierunku. Były też zaskakująco skuteczna. Jeśli zaczęły napotykać problemy, to, podobnie jak w~przypadku podobnie zorganizowanych kampanii antynuklearnych w~latach 70., ponieważ osiągnęli swoje bezpośrednie cele znacznie szybciej, niż ktokolwiek zaangażowany w~to naprawdę przewidywał. To hołd dla sukcesu ruchu, o którym większość z~nas zapomniała już o rodzaju retoryki, którą rzucano pod koniec lat 90.: że supernaładowany wolnorynkowy kapitalizm okazał się teraz jedynym możliwym sposobem na zrobienie czegokolwiek; że ,,wolny handel'' i~,,wolne rynki'' były nieubłaganymi, ale i~rewolucyjnymi siłami; że każdy, kto nie zgadzał się z~częścią tego programu, mógł być traktowany jako niemal dosłownie szalony. Trudno też przypomnieć sobie, jak kiedyś politycy i~eksperci medialni osiągnęli tak absolutny konsensus. Mówiąc jako osoba, która po raz pierwszy zaangażowała się w~ruch bezpośrednio po Seattle, mogę zapewnić czytelnika, że  na początku 2000 roku prawie nikt, kogo znałem w~DAN, nie wyobrażał sobie, że będzie w~stanie zniszczyć ten konsensus waszyngtoński w~ciągu jednego lub dwóch lat. Większość z~nas zakładała, że zajmie to prawdopodobnie dekadę. Zamiast tego cały aparat natychmiast się zawalił.

 Wydaje się, że wydarzyło się to, że globalne elity wpadły w~panikę, a kiedy wpadają w~panikę globalne elity, ich zwykłym instynktem jest rozpoczęcie wojny. Tak naprawdę nie ma znaczenia, przeciwko komu jest skierowana wojna. Rzecz w~tym, że wojna zmienia zasady starć z~krajowymi przeciwnikami. Radykałowie przekonują się, że ich umiarkowani sojusznicy są przerażeni tym, że wydają się niepatriotyczni; większość porzuca koalicje; opozycyjne partie polityczne czują się zmuszone do przyjęcia programu wojny; ludność jest o wiele bardziej skłonna tolerować brutalne tłumienie sprzeciwu. Wszystko to wydarzyło się w~1914 roku, a formuła działała tak dobrze, że została utrzymana w~takiej czy innej formie (wojna światowa, zimna wojna) do 1989 lub 1991 roku. Od tego czasu historia toczy się na nowo w~bardzo szybkim cyklu. Anarchiści pojawili się ponownie; powstały w~ten sposób ruch globalny był niezwykle skuteczny w~przerażaniu kapitalistycznych elit; w~ciągu zaledwie kilku lat te elity zagrały swoją kartę atutową, po czym nastąpiły wszystkie zwykłe efekty. Podczas akcji przeciwko Światowemu Forum Ekonomicznemu w~lutym 2002 roku, bezpośrednio po 11 września, grupy akcji bezpośredniej zostały skutecznie porzucone przez wszystkich ich dotychczasowych sojuszników, od związków zawodowych po organizacje pozarządowe. i~tak przeprowadzili akcję, choć tak naprawdę nie byli w~stanie odbyć więcej niż marsz, w~obliczu masowych represji. Z kolei represje stawały się coraz gorsze. Do czasu spotkań FTAA w~Miami rząd uznał, że może wyzwolić nakaz brutalności -- szerokie użycie paralizatorów, nielegalnie wymiatając aktywistów z~ulic i~poddając ich systematycznym torturom -- z~czym wyraźnie nie czuli, że ujdzie im na sucho przed wojną. Jednocześnie podobne rzeczy dzieją się za granicą, nawet w~krajach w~dużej mierze poza granicami ,,wojny z~terroryzmem''.

 Mimo to istnieją wszelkie powody, by zakładać, że ten projekt nie okaże się opłacalny. Historia wciąż wydaje się działać w~szybkim tempie. Publiczne poparcie dla konfliktu w~Iraku słabło znacznie szybciej niż poparcie dla jakiejkolwiek porównywalnej wojny lądowej w~ostatnim stuleciu; Stany Zjednoczone po prostu nie mają zasobów ekonomicznych, aby utrzymać nowy imperialny projekt; już teraz widzimy coraz większą radykalizację Ameryki Łacińskiej, ponieważ Stany Zjednoczone były zbyt związane, by znacząco interweniować.

 Myślę, że niektóre z~podniesionych wcześniej kwestii dotyczących zasad zaangażowania i~różnicy między armią a policją mogą być przydatne w~próbie zrozumienia tego -- być może teraz, łaskawie przemijającego -- osobliwego historycznego momentu tak zwanej ,,wojny z~terror''. To, co Stany Zjednoczone usiłują narzucić światu w~swoim imieniu, tak naprawdę wcale nie jest wojną. Jest to oczywiście truizm, że w~miarę rozprzestrzeniania się broni nuklearnej wypowiadane wojny między państwami już się nie zdarzają, a wszystkie konflikty są określane jako ,,działania policyjne'' tego czy innego rodzaju. Ale ważne jest również, aby pamiętać, że policja zawsze postrzega siebie jako zaangażowaną w~wojnę w~dużej mierze bez reguł, z~przeciwnikiem bez honoru, wobec którego nie jest się w~związku z~tym zobowiązany do honorowego postępowania i~którego ostatecznie nie można wygrać. Państwa zawsze mają tendencję do definiowania swojego stosunku do swoich obywateli w~kategoriach wojny nie do wygrania, a państwo amerykańskie jest jednym z~najbardziej rażących przykładów. W ostatnich dziesięcioleciach widzieliśmy, jak Wojna z~Ubóstwem przerodziła się w~Wojnę ze Zbrodnią, następnie Wojnę z~Narkotykami (pierwszą rozszerzoną na arenę międzynarodową), a wreszcie teraz Wojnę z~Terroryzmem. Jak pokazuje ta sekwencja, ta ostatnia wcale nie jest wojną, ale próbą rozciągnięcia tej samej wewnętrznej logiki na cały glob. Jest to próba ogłoszenia rozproszonego, globalnego państwa policyjnego. Z pewnością nie na wzór państwa narodowego, ale też, jak podejrzewam, na wzór bezcentrycznego humanitarnego Imperium Hardta i~Negriego (2000) (w tej chwili jest to bardziej projekt europejski, choć oczywiście otwarty na odrodzenie). W końcu państwa narodowe zawsze były czymś w~rodzaju historycznej anomalii, jeśli nie niemożliwym ideałem. Niemal z~chwilą, gdy podjęto próbę rozszerzenia logiki państwa narodowego na cały świat, aby pokryć całą planetę siatką niezależnych, suwerennych państw narodowych, cały projekt zaczął się rozpadać. Projekt Busha wyglądał bardziej jak państwo imperialne w~znacznie starszym sensie, coś jak Rzym w~jego ostatnich dniach: uniwersalne, drapieżne, sporadyczne, ale przytłaczające w~użyciu przemocy, państwo, któremu każdy, nawet Goci i~Hunowie, musieli deklarować wierność, w~tym samym czasie, gdy planowali je zniszczyć. Jakkolwiek to działa, w~ostatecznym rozrachunku ostatecznie zostało stworzone w~znacznie większym stopniu w~odpowiedzi na sukces naszego rozproszonego globalnego powstania niż na groźbę Osamy bin Ladina, nawet jeśli to ostatnie z~pewnością stanowiło ostateczną wygodną wymówkę. Po prostu, również w~skali globalnej, walka moralno-polityczna stworzyła zasady zaangażowania, które bardzo utrudniają Stanom Zjednoczonym bezpośrednie uderzenie w~tych, przeciwko którym najbardziej chcieliby uderzyć\footnote{Fakt, że prawie wszystkie główne postacie zaangażowane w~tłumienie protestów w~Ameryce skończyły jako ,,konsultanci ds. bezpieczeństwa'' w~Bagdadzie po amerykańskim podboju Iraku, wydaje się tutaj dość wymowny. Oczywiście szybko odkryli, że ich zwykła taktyka nie jest szczególnie skuteczna wobec przeciwników, którzy naprawdę \textit{byli }agresywni - zdolni, na przykład, do radzenia sobie z~urzędnikami IMF i~Banku Światowego poprzez wysadzenie ich w~powietrze.}.

 Gdyby ująć to w~terminy sugerowane w~poprzednim punkcie, można by powiedzieć (bez wątpienia), że tak jak strukturą przemocy najwłaściwszą dla ontologii politycznej opartej na wyobraźni jest rewolucja, tak struktura wyobraźni najwłaściwsza dla ontologii polityczna opartej na przemocy to właśnie terror. Można by dodać, że Bushowie i~Bin Laden pracują w~tym zakresie w~tandemie. (Myślę, że to znaczące, że jeśli Al-Kaida ma jakąś gigantyczną utopijną wizję -- odtworzenie starej islamskiej diaspory na Oceanie Indyjskim? przywrócenie kalifatu? masowe nawrócenie? -- to jeszcze nam o tym nie mówiono. ) Jest to jednak nieco uproszczone. Aby zrozumieć amerykański reżim jako strukturę globalną, a jednocześnie zrozumieć jego sprzeczności, podejrzewam, że należałoby wrócić do kosmologicznej roli policji w~kulturze amerykańskiej. Specyficzną cechą życia w~Stanach Zjednoczonych jest to, że większość obywateli amerykańskich, którzy w~ciągu dnia starają się unikać jakiejkolwiek możliwości zajmowania się policją lub sprawami policyjnymi, zwykle można oczekiwać, że powróci do domu i~spędzi godziny na oglądaniu dramatów, które zapraszają ich do spojrzenia na świat z~punktu widzenia policjanta. W latach sześćdziesiątych policja nagle zajęła miejsce zajmowane niegdyś przez kowbojów w~amerykańskiej rozrywce\footnote{Bardzo nagle: prawie nie sposób znaleźć ani jednego amerykańskiego filmu sprzed lat 60., w~którym bohaterem był policjant. Moment zmiany wydaje się istotny.}. Do tej pory te zdjęcia amerykańskiej policji są nieustannie eksportowane do każdego zakątka świata, wraz z~ich odpowiednikami z~krwi i~kości. Chciałbym tu jednak podkreślić, że obie charakteryzują się pozaprawną bezkarnością, która paradoksalnie sprawia, że są w~stanie ucieleśniać rodzaj władzy konstytucyjnej zwróconej przeciwko sobie. Gliniarz z~Hollywood, podobnie jak kowboj, jest samotnym indywidualistą, który łamie wszelkie zasady (co jest dopuszczalne, a nawet konieczne, ponieważ zawsze ma do czynienia z~niehonorowymi przeciwnikami). W rzeczywistości, jak już wspomniałem, to gliniarz angażuje się w~niekończące się niszczenie mienia, co zapewnia tak wiele przyjemności z~hollywoodzkich filmów akcji. Innymi słowy, policja jest bohaterem po części dlatego, że są jedynymi postaciami, które mogą systematycznie ignorować prawo. Jest mocą składową, która sama się włączyła, ponieważ gliniarze, na ekranie lub w~rzeczywistości, nigdy nie próbują niczego tworzyć. Po prostu utrzymują status quo. W pewnym sensie jest to najmądrzejsze przemieszczenie ideologiczne ze wszystkich, doskonałe uzupełnienie wspomnianej prywatyzacji (konsumenckiego) pożądania, przeciwko któremu odświętnie protestują marionetki. Dopóki popularne święto trwa, stało się ono czystym Spektaklem, jak powiedzieliby sytuacjoniści, z~rolą Mistrza Potlatch przyznaną tym samym postaciom, które w~prawdziwym życiu są odpowiedzialne za zapewnienie, że wszelkie wybuchy ludowego święta zachowania są brutalnie tłumione.

 Jednak jak każda formuła ideologiczna, ta jest niezwykle niestabilna, pełna sprzeczności, o czym tak żywo świadczą początkowe trudności amerykańskiej policji w~tłumieniu ruchu globalizacyjnego. Wydaje mi się bardziej sposobem radzenia sobie w~sytuacji skrajnego wyobcowania i~niepewności, który sam w~sobie może być podtrzymywany jedynie przez systematyczny przymus. W obliczu wszystkiego, co choć trochę przypomina kreatywne, niezbywalne doświadczenie, wydaje się to równie śmieszne jak reklama dezodorantu w~czasach narodowej katastrofy. Anarchistycznym problemem pozostaje, jak wprowadzić tego rodzaju doświadczenie i~kryjącą się za nim moc wyobraźni do codziennego życia tych, którzy znajdują się poza małymi autonomicznymi bańkami, które już potrafili stworzyć. To ciągły problem. Nie ma sposobu, aby być pewnym, że to w~ogóle możliwe. Jednak wydaje się, że istnieją wszelkie powody, by sądzić, że gdyby to było możliwe, siła policyjnej kosmologii, a wraz z~nią siła samej policji, po prostu by się rozpłynęła.

\chapter*{Bibliografia}
 
\begin{itemize}


\item Abu-Lughod, Janet, et al \textit{1994 }From Urban Village to East Village: The Battle for New York’s Lower East Side\textit{. Cambridge: Blackwell.}  
 
\item Ackerman, Seth 2000 “Prattle in Seattle: How Media Coverage Misrepresented the Protest,” in \textit{Globalize This! The Battle Against the World Trade Organization and Corporate Rule (}edited by Kevin Danaher and Roger Burbach), 59--66. Monore, ME.: Common Courage Press.  

\item Agamben, Giorgio 1993 \textit{Stanzas: Word and Phantasm in Western Culture.} Minneapolis: University of Minnesota Press. Homo Sacer: Sovereign Power and Bare Life\textit{. Stanford: Stanford University Press.}  

\item ACME Collective 1999 “N30 Black Bloc Communique” by the ACME Collective, 10:48AM Saturday, Dec 4, 1999.  

\item Anon 1979 \textit{You Can’t Blow Up a Social Relationship: the Anarchist Case Against Terrorism}. Tucson, AZ: See Sharp Press 1998 (originally published in Australia in 1979).  

\item Anon 1999 “Give Up Activism.” \textit{Do or Die} 9: 160--66.  

\item Arendt, Hannah 1958 \textit{The Human Condition}. Garden City, NY: Doubleday Anchor Books.  

\item Baker, C. E. 1983 ,,Unreasoned Reasonableness: Mandatory Parade Limits and Time, Place and Manner Regulations.'' \textit{Northeastern Law Review}, December 1983, 223--45.  

\item Baker, Paula 1984 “The Domestication of Politics: Women and American Political Society, 1780-1920.” \textit{American Historical Review}, Vol. 89, no. 3, 620--47.  

\item Bakhtin, Mikhail 1984 \textit{Rabelais and His World}. Bloomington: Indiana University Press.  

\item Barakat, Halim 1969 “Alienation: A Process of Encounter between Utopia and Reality.” \textit{British Journal of Sociology}, Vol. 20, no. 1 (March 1969), 1--10.  

\item Barber, David 2001 “''A Fucking White Revolutionary Mass Movement’ and Other Fables of Whiteness, with Afterward by Noel Ignatiev.” \textit{Race Traitor}, no. 12 (Spring 2001), 4--93.  

\item Benjamin, Walter 1978 “Critique of Violence.” In \textit{Reflections: Essays, Aphorisms, and Autobiographical Writings}. New York: Harcourt Brace Jovanovich.  

\item Berber, Lucy \textit{2002 }Marching on Washington: The Forging of an American Political Tradition\textit{. Berkeley: University of California Press.}  

\item Bey, Hakim \textit{1991 }T.A.Z: The Temporary Autonomous Zone, Ontological Anarchy, Poetic Terrorism\textit{. New York: Autonomedia. 1996 }Millennium\textit{. Brooklyn, NY: Autonomedia.}  

\item Beyer-Arnesen, Harald  
2000 “Direct Action: towards an understanding of a concept.” \textit{Anarcho-Syndicalist Review,} no. 29 (Summer 2000), 11--14.  

\item Bittner, Egon 
1990 \textit{Aspects of Police Work}. Boston: Northeastern University Press. 

\item Black, Bob 
1986 “The Abolition of Work.” In \textit{The Abolition of Work and Other Essays,} 17--33. San Francisco: Loompanics. 

\item Bloch, Maurice 
1974 “Symbols, Song, Dance, and Features of Articulation: Is Religion an Extreme Form of Traditional Authority?” \textit{Archives Européens de Sociologie} 15: 55--81. 

\item Bookchin, Murray 
1997 “Social Anarchism or Lifestyle Anarchism: an Unbridgeable Divide\ldots ” \url{http://dwardmac.pitzer.edu/Anarchist_Archive/bookchin/soclife.html}. (Accessed 2/27/07). 

\item Bonanno, Alfredo 
\textit{Armed Joy}. Translated by Jean Weir. London: Elephant Editions (original Italian version: 1977) 

\item Boski, Joseph 
2000 “The Costs of Global Governance: Security and International Meetings Since WTO-Seattle.” Paper presented at the CIBER Conference, Globalization: Governance and Inequality, May 31--June 1, 2002. 

\item Bourdieu, Pierre 
\textit{1993 }The Field of Cultural Production: Essays on Art and Literature\textit{. Cambridge: Polity Press.} 

\item Brenan, Gerald 
\textit{1967 }Spanish Labyrinth: An Account of the Social and Political Background of the Civil War\textit{. Cambridge: Cambridge University Press.} 

\item Bui, Roberto 
2005 “Tute Bianche: The Practical Side of Myth Making (In Catastrophic Times)”, originally prepared for the “Semi(o)reistance” panel at the make-world festival, Munich, October 20, 2001. Giap/digest \#11. 

\item Campbell, Colin 
\textit{1987 }The Romantic Ethic and the Spirit of Modern Consumerism\textit{. Oxford: Blackwell.} 

\item Canetti, Elias 
1962 \textit{Crowds and Power}. Translated by Carol Stewart. New York: Farrar, Straus and Giroux. 

\item Carter, April 
1973 \textit{Direct Action and Liberal Democracy}. London: Routledge and Kegan Paul. 

\item Castoriadis, Cornelius 
1987 \textit{The Imaginary Institution of Society}. Translated by Kathleen Blamey. Cambridge: Polity Press. 

\item Castoriadis, Cornelius 
1991 \textit{Philosophy, Politics, Autonomy: Essays in Political Philosophy,} edited by David Ames Curtis. New York: Oxford University Press. 

\item Chang, Jen and Bethany Or, Eloginy Tharmendran, Emmie Tsumara, Steve Daniels and Darryl Leroux, editors 
\textit{2001 }Resist! A Grassroots Collection of Stories, Poetry, Photos and Analyses from the Quebec City FTAA Protests and Beyond\textit{. Halifax: Fernwood Publishing.} 

\item Christgau, Robert 
\textit{2000 }Any Old Way You Choose It: Rock and Other Pop Music, 1967\textit{--}1973\textit{. New York: Cooper Square Press.} 

\item Churchill, Ward 
\textit{1998 }Pacifism as Pathology: Reflections on the Role of Armed Struggle in North America\textit{ (with Mike Ryan.) Winnipeg: Arbeiter Ring.} 

\item Coady, C. A. J. 
1986 “The Idea of Violence,” \textit{Journal of Applied Philosophy}, Vol. 3, no. 1, 3--19. 

\item Cohen, Norman 
1957 \textit{The Pursuit of the Millennium}. Fair Lawn: Essential Books. 

\item Coletta, Neil L. (ed.) 
\textit{2001 }Writing From the Red Zone: Voices of the Anti-Globalization Movement\textit{. [n.p.]: Del Suego Publications.} 

\item Conquest, Mary 
\textit{“Je Me Souviens.” }\url{http://www.forgetmagazine.com/042501.htm}\textit{.} 

\item Conway, Janet 
2003 “Civil Resistance and the ''Diversity of Tactics’ in the Anti-Globalization Movement: Problems of Violence, Silence, and Solidarity in Activist Politics.” \textit{Osgoode Hall Law Journal}, Vol. 41, nos.2 \&3, 505--30. 

\item Cooper, Marc 
1991 “Dum Da Dum-Dum.” \textit{Village Voice}, April 16, 1991, 28--33. 

\item Creveld, Martin Van 
1991 \textit{The Transformation of War}. New York: Free Press. 

\item CrimethInc Ex-Workers Collective 
\textit{2000 }Days of War, Nights of Love: CrimethInc for Beginners\textit{. CrimethInc.} 

\item CrimethInc Ex-Workers Collective 
2001 \textit{Evasion}. CrimethInc. 

\item CrimethInc Ex-Workers Collective 
2003 \textit{Fighting for Our Lives}. CrimethInc. [accessible at \url{http://www.crimethinc.com/a/fighting}]. 

\item CrimethInc Ex-Workers Collective 
\textit{2005 }Recipes for Disaster: An Anarchist Cookbook\textit{. CrimethInc.} 

\item David and X (editors) 
The Black Bloc Papers: An Anthology of Primary Texts from the North American Anarchist Black Bloc, 1991\textit{--}2002, The Battle of Seattle (N30) through Quebec City (A20),\textit{ compiled by David and X of the Green Mountain Anarchist Collective. Baltimore: Black Clover Press.} 

\item Davis, Susan G. 
“Strike Parades and the Politics of Representing Class in Antebellum Philadelphia.” \textit{The Drama Review}, Vol. 29, no. 3, 106--116. Parades and Power: Street Theatre in Nineteenth-Century Philadelphia\textit{. Philadelphia: Temple University Press.} 

\item Day, Richard \textit{2005 }Gramsci Is Dead: Anarchist Currents in the Newest Social Movements\textit{. London: Pluto Press.} 

\item De Angelis, Massimo 
2004 “Strange Common Places.” Review of Paolo Virno’s \textit{A Grammar of the Multitude} (New York: Semiotexte). \textit{Mute Magazine}, May 2004. 

\item De Angelis, Massimo 
\textit{2007 }The Beginning of History: Value Struggles and Global Capitalism\textit{. London: Pluto Press.} 

\item Debord, Guy 
1967 \textit{La Société du spectacle}. Paris: Buchet/Chastel. 

\item Dirlik, Arif 
1991 \textit{Anarchism in the Chinese Revolution}. Berkeley: University of California Press. 

\item Dubofsky, Melvyn 
\textit{1969 }We Shall Be All: A History of the Industrial Workers of the World\textit{. Chicago: Quadrangle Books.} 

\item Duncombe, Stephen 
2002 “Stepping off the Sidewalk: Reclaim the Streets/NYC.” In \textit{From ACT UP to the WTO: Urban Protest And Community-Building in the Era of Globalization} (Edited by Benjamin Shepard and Ronald Hayduk). London: Verso. 

\item Duncombe, Stephen 
\textit{2007 }Dream: Re-imagining Progressive Politics in an Age of Fantasy\textit{. New York: New Press.} 

\item Dupuis-Deri, Francis 
2004 “Penser l’action directe des Black Blocs.” \textit{Publix}, Vol. 17, no. 68: 79--109. 

\item Dupuis-Deri, Francis 
\textit{2005a }Black Blocs: la liberté et l’égalité se manifestent\textit{. Lyon: Atelier de Creation Libertaire.} 

\item Dupuis-Deri, Francis 
2005b “L’Altermondialisme à l’ombre du Drapeau Noir. L’Anarchie en heritage.” In \textit{L’Altermondialisme en France: la longue histoire d’une nouvelle cause} (under the direction of Éric Agrikoliansky, Oliver Fillieule, Nonna Mayer), 199--231. Paris: Flammarion. 

\item Dupuis-Deri, Francis 
2005c “Anarchy in Political Philosophy.” \textit{Anarchist Studies}, Vol. 13, no. 1: 8--22. 

\item Dupuis-Deri, Francis 
2005d “''Un autre monde et possible.’ Il existe déja!” \textit{Horizons Philosophiques} 15 (2): 63--85. 

\item Durkheim, Émile 
\textit{1893 }De la division du travail social. Étude sur l’organisation des sociétés supérieures\textit{. Paris, F. Alcan.} 

\item Durkheim, Émile 
\textit{1912 }Formes élémentaire de la vie religieuse, le système totémique en Australie\textit{. Paris, F. Alcan.} 

\item Ehrenreich, Barbara 
\textit{2006 }Dancing in the Streets: a History of Collective Joy\textit{. New York: Metropolitan Books.} 

\item Elliot, Karen 
2001 “Situationism in a Nutshell.” \textit{Barbelith Webzine}. \url{http://www.barbelith.com/cgibin/acticles/00000011.shtml}. (Accessed 02/25/07). 

\item Epstein, Barbara 
\textit{1991 }Political Protest and Cultural Revolution: Nonviolent Direct Action in the 1970s and 

\item Epstein, Barbara 
\textit{1980s}. Berkeley: University of California Press. 

\item Epstein, Barbara 
2001 “Anarchism and the Anti-Globalization Movement.” \textit{Monthly Review}, Vol. 53, no. 4, September 2001: 1--14. 

\item Evans, Sara 
\textit{1979 }Personal Politics: the Roots of Women’s Liberation in the Civil Rights Movement and the New Left\textit{. New York: Knopf.} 

\item Evans-Pritchard, E. E. 
\textit{1936 }Witchcraft, Oracles and Magic among the Azande\textit{. Oxford: Oxford University Press.} 

\item Evans-Pritchard, E. E. 
\textit{1940 }The Nuer: A Description of the Modes of Livelihood and Political Institutions of a Nilotic People\textit{. Oxford: Oxford University Press.} 

\item Federici, Silvia 
\textit{2004 }Caliban and the Witch: Women, the Body, and Primitive Accumulation\textit{. New York: Autonomedia.} 

\item FAIR (Fairness and Accuracy In Reporting) 
2001 “ACTIVISM UPDATE: \textit{New York Times} Responds on Inauguration Criticism”: news release, (February 22, 2001). 

\item Filipo, Roy San (editor) 
\textit{2003 }A New World in Our Hearts: Eight Years of Writings from the Love and Rage Revolutionary Anarchist Federation\textit{. Oakland: AK Press.} 

\item Flacks, Richard 
\textit{Youth and Social Change}. Chicago: Rand McNally. 

\item Fletcher, Robert, editor 
\textit{2007 }Beyond Resistance? The Future of Freedom\textit{. New York: Nova Science Press.} 

\item Flynn, Elizabeth Gurley, Walker C. Smith \& William E. Trautman 
\textit{1997 }Direct Action \& Sabotage: Three Classic IWW Pamphlets from the 1910s\textit{. Chicago: Charles Kerr.} 

\item Freeman, Jo 
1971 “The Women’s Liberation Movement: Its Origins, Structures, and Ideas.” In \textit{Recent Sociology No. 4: Family, Marriage, and the Struggle of the Sexes} (edited by Hans Peter Dreitzel), 201--216. New York: The Macmillan Co. 

\item Freeman, Jo 
1972 “The Tyranny of Structurelessness.” First officially published in \textit{The Second Wave} (Vol. 2, no 1). Reprinted in \textit{Quiet Rumours: An Anarcha-Feminist Reader} (Dark Star Collective, 2002), 54--61. Edinburgh: AK Press. 

\item Gane, Mike 
\textit{1992 }The Radical Sociology of Durkheim and Mauss\textit{. New York, Routledge.} 

\item Geoghegan, Vincent 
1987 \textit{Utopianism and Marxism}. London: Methuen. 

\item Geyer, R. Felix 
\textit{1996 }Alienation, Ethnicity, and Postmodernism\textit{. Westport, CT: Greenwood Press.} 

\item Geyer, R. Felix \& Walter R. Heinz, eds. 
1992 \textit{Alienation, Society and the Individual: Continuity and Change in Theory and Research} (edited by Felix Geyer, Walter R. Heinz). New Brunswick: Transaction Publishers. 

\item Graeber, David 
1997 “Manners, Deference and Private Property: The Generalization of Avoidance in Early Modern Europe.” \textit{Comparative Studies in Society and History}, Vol. 39, no.4: 694--728. 

\item Graeber, David 
\textit{2001 }Toward An Anthropological Theory of Value: The False Coin of Our Own Dreams\textit{. New York: Palgrave.} 

\item Graeber, David 
2002 “The New Anarchists,” \textit{New Left Review} 13, January/February 2002, 61--74. 

\item Graeber, David 
2003 “The Globalization Movement and the New New Left.” In \textit{Implicating Empire: Globalization and Resistance in the 21st Century} (Stanley Aronowitz and Heather Gautney, eds.), 325--338. New York: Basic Books. 

\item Graeber, David 
2004a “The Twilight of Vanguardism.” In \textit{The World Social Forum: Challenging Empires} (Jai Sen, Anita Anand, Arturo Escobar, Peter Waterman eds.), 329--35. New Delhi: Viveka Foundation. 

\item Graeber, David 
2004b “La sociologie comme science et comme utopie.” In \textit{Revue du MAUSS Semestrielle No. 24}, “Une théorie sociologique générale est-elle pensable?” Second Semestre, 205--217. 

\item Graeber, David 
2004c \textit{Fragments of an Anarchist Anthropology}. Prickly Paradigm Series (Chicago, University of Chicago Press). 

\item Graeber, David 
2005 “Azione Diretta e Anarchismo da Seattle in Poi.” In \textit{Affinità sovversive: i~movimenti sociali americani nella guerra globale}” (Franco Barchiesi, editor). Rome: DeriveApprodi, 65--144. 

\item Graeber, David 
2007 “There Never Was a West: or, Democracy Emerges from the Spaces in Between.” In \textit{Possibilities: Essays on Hierarchy, Rebellion, and Desire}. Oakland: AK Press. 

\item Grubacic, Andrej 
2004 “Towards Another Anarchism.” In \textit{World Social Forum: Challenging Empires}. (edited by Jai Sen, Anita Anand, Arturo Escobar, and Peter Waterman), 35--43. New Delhi: Vivika Foundation. 

\item Guilloud, Stephanie, ed. 
\textit{2000 }Voices from the WTO: An Anthology of Writings from the People Who Shut Down the World Trade Organization\textit{. Voices from the WTO Project, Olympia, Washington.} 

\item Hardt, Michael, and Antonio Negri 
\textit{1994 }The Labor of Dionysus: a critique of the state-form\textit{. Minneapolis: University of Minnesota Press.} 

\item Hardt, Michael, and Antonio Negri 
2000 \textit{Empire}. Cambridge, MA: Harvard University Press. 

\item Hardt, Michael, and Antonio Negri 
\textit{2004 }Multitude: War and Democracy in the Age of Empire\textit{. New York: Penguin Press.} 

\item Harstock, Nancy 
2004 “The Feminist Standpoint: Developing the Ground for a Specifically Feminist Historical Materialism.”In \textit{The Feminist Standpoint Theory Reader: Intellectual and Political Controversies} (Harding, Sandra, editor), 35--53. London: Routledge. 

\item Hobsbawm, E. J. 1973 \textit{Revolutionaries: Contemporary Essays}. New York: Pantheon Books. 

\item Holloway, John 
\textit{2000 }Change the World without Taking Power: The Meaning of Revolution Today\textit{. London: Pluto Press.} 

\item \href{http://Infoshop.org/}{Infoshop.org} 
n.d. “Black Blocs for Dummies,” \url{http://www.infoshop.org/blackbloc.html} (accessed 8/25/00). 

\item Jordan, John 
1998 “The Art of Necessity: The Subversive Imagination of Anti-Road Protest and Reclaim the Streets.” In \textit{DiY Culture: Party \& Protest in Nineties Britain} (edited by George McKay), 129--151. London: Verso. 

\item Katsiaficas, George, 
\textit{1997 }The Subversion of Politics: European Autonomous Social Movements and the Decolonization of Everyday Life\textit{. New Jersey: Humanities Press.} 

\item Kelley, Robin D. G. 
\textit{2002 }Freedom Dreams: The Black Radical Imagination\textit{. Boston: Beacon Press.} 

\item Klein, Naomi 
2000 “The Vision Thing.” \textit{The Nation}, July 10, 2000, 347. 

\item Klein, Naomi 
2001 “Reclaiming the Commons.” Talk at Centre for Social Theory \& Comparative History, UCLA (April 2001). 

\item Kropotkin, Peter 
1898 \textit{Anarchist Morality}. London: Freedom Office. 

\item Kropotkin, Peter 
1924 \textit{Ethics, Origin and Development}. (Authorized translation from the Russian, by Louis S. Friedland and Joseph R. Piroshnikoff.) New York: The Dial Press. 1927 \textit{Kropotkin’s Revolutionary Pamphlets}. New York: Dover. 

\item Lakey, George 
1973 \textit{Strategy for a Living Revolution}. Philadelphia: Grossman Publishers. 

\item Le Bon, Gustave 
\textit{1921 }The Crowd, a Study of the Popular Mind\textit{. New York: Macmillan.} 

\item Le Guin, Ursula K. 
\textit{1974 }The Dispossessed: An Ambiguous Utopia\textit{. New York: Harper \& Row.} 

\item Lincoln, Bruce 
\textit{1989 }Discourse and the Construction of Society: Comparative Studies of Myth, Ritual, and Classification\textit{. New York: Oxford University Press.} 

\item McGerr, Michael E. 
\textit{1988 }The Decline of Popular Politics: The American North, 1865\textit{--}1928\textit{. New York: Oxford University Press.} 

\item McGerr, Michael E. 
1990 “Political Style and Women’s Power, 1830-1930.” \textit{Journal of American History}, Vol. 77, no. 3: 864--85. 

\item Malatesta, Errico 
1913 “The Tragic Bandits.” \textit{La Société Nouvelle}, 19th year, no. 2, (August 1913). 

\item Manheim, Karl 
\textit{1936 }Ideology and Utopia; An Introduction to the Sociology of Knowledge\textit{. New York: Harcourt, Brace and Company.} 

\item Manin, Bernard 
1994 “On Legitimacy and Political Deliberation.” In \textit{New French Thought: Political Philosophy} (edited by Mark Lilla), 186--200. Princeton: Princeton University Press. 

\item Mankoff, Milton and Richard Flacks 
1971 “The Changing Social Base of the American Student Movement.” \textit{Annals of the American Academy of Political and Social Science} 395, 55--67. 

\item Martinez, Elizabeth ''Betita’ 
2000 “Where Was the Color in Seattle?” \textit{ColorLines}, Vol. 3, no. 1 (Spring 2000): 5--9. 

\item Mattick, Paul, Jr. 
1970 “Old Left, New Left, What’s Left?” \textit{Root \& Branch}, no. 1, 15--24. 

\item Mele, Christopher 
\textit{2000 }Selling the Lower East Side: Culture, Real Estate, and Resistance in New York City\textit{. Minneapolis: University of Minnesota Press.} 

\item Miller, Daniel 
\textit{1987 }Material Culture and Mass Consumption\textit{. London: Basil Blackwell.} 

\item Miller, Daniel 
\textit{1995 }Acknowledging Consumption: A Review of New Studies\textit{ (D. Miller, ed.). London: Routledge.} 

\item Millet, Steve 
2004 “Technology is Capital: \textit{Fifth Estate}’s critique of the megamachine.” In \textit{Changing Anarchism: Anarchist Theory and Practice in a Global Age} (Jonathan Purkis and James Bowen, ed.) Manchester: Manchester University Press, 46--70. 

\item Makhno, Nestor, Ida Mett, Piotr Arshinov, Valevsky, Linsky 
\textit{1926 }The Organizational Platform of the Libertarian Communists\textit{. Paris.} 

\item Mueller, Tadzio 
2003 “Empowering Anarchy: Power, Hegemony and Anarchist Strategy.” \textit{Anarchist Studies,} Vol. 11, no. 2, 122--49. 

\item Negri, Antonio 
\textit{1984 }Marx beyond Marx: Lessons on the Grundrisse\textit{. South Hadley, MA: Bergin \& Garvey.} 

\item Negri, Antonio 
\textit{1991 }The Savage Anomaly: The Power of Spinoza’s Metaphysics and Politics\textit{. Minneapolis: University of Minnesota Press. } 

\item Negri, Antonio 
\textit{1999 }Insurgencies: Constituent Power and the Modern State\textit{. Minneapolis: University of Minnesota Press.} 

\item Neocleous, Mark 
\textit{2000 }The Fabrication of Social Order: A Critical Theory of Police Power\textit{. London: Pluto Press.} 

\item Nesbitt, Robert 
1966 \textit{The Sociological Tradition}. New York: Basic Books. 

\item Notes from Nowhere, editor 
\textit{2003 }We Are Everywhere: The Irresistible Rise of Global Anti-Capitalism\textit{. London: Verso.} 

\item Overing, Joanna 
1986 “Images of Cannibalism, Death and Domination in a ''Nonviolent’ Society.” In \textit{The Anthropology of Violence}, edited by David Riches, 86--101. Oxford: Blackwell. 

\item Overing, Joanna 
1988 “Personal Autonomy and the Domestication of the Self in Piaroa Society.” In \textit{Acquiring Culture: Cross Cultural Studies in Child Development}, edited by Gustav Jahoda and I. M. Lewis, 169--92. London: Croom Helm. 

\item Overing, Joanna 
1989 “Styles of Manhood: An Amazonian Contrast in Tranquility and Violence.” In \textit{Societies at Peace: Anthropological Perspectives}, edited by Signe Howell and Roy Willis, 79--99. London: Routledge. 

\item Polletta, Francesca 
\textit{2002 }Freedom is an Endless Meeting: Democracy in American Social Movements\textit{. Chicago: University of Chicago Press.} 

\item Preston, William 
\textit{1994 }Aliens and Dissenters: Federal Suppression of Radicals, 1903--1933\textit{. Urbana: University of Illinois Press.} 

\item Randle, Michael 
1994 \textit{Civil Resistance}. London: Fontana Press. 

\item Reinsborough, Patrick 
2004 “Decolonizing the Revolutionary Imagination: Values Crisis, the Politics of Reality, and Why There’s Going to be a Common-Sense Revolution in this Generation.” In \textit{Globalize Liberation: How to Uproot the System and Build a Better World}, edited by David Solnit, 161--212. San Francisco: City Lights Press. 

\item Riches, David 
\textit{1986 }The Anthropology of Violence\textit{. Oxford: Blackwell.} 

\item Ruckus Society 
\textit{1997a }Action Preparation and Coordination\textit{. Ruckus Society. Photocopy.} 

\item Ruckus Society 
1997b \textit{Banner Making Manual}. Ruckus Society. Photocopy. 

\item Ruckus Society 
1997c \textit{Scouting Manual}. Ruckus Society. Photocopy. 

\item  
Savage, Jon 
\textit{1991 }England’s Dreaming: The Sex Pistols and punk rock\textit{. London: Faber and Faber.} 

\item Schmitt, Richard and Thomas E. Moody, eds. 
1994 \textit{Alienation and Social Criticism}. New Jersey: Humanities Press. 

\item Scott, James C. 
\textit{1985 }Weapons of the Weak: Everyday Forms of Peasant Resistance\textit{. New Haven: Yale University Press.} 

\item Scott, James C. 
1992 \textit{Domination and the Arts of Resistance}. New Haven: Yale University Press. 

\item Shukaitis, Stevphen 
2005 “Space. Imagination/Rupture, the Cognitive Architecture of Utopian Thought in the Global Justice Movement.” \textit{University of Sussex Journal of Contemporary History} Vol. 8: 1--14. 

\item Shukaitis, Stevphen and David Graeber 
2007 “Introduction” to \textit{Constituent Imagination: Militant Investigation and Collective Research} (edited by Stevphen Shukaitis, David Graeber, and Erika Biddle), 11--36. Oakland: AK Press. 

\item Smith, Adam 
1761 \textit{Theory of Moral Sentiments}. Cambridge: Cambridge University Press (2002 edition). 

\item Smith, Adam 
\textit{1776 }An Inquiry into the Nature and Causes of the Wealth of Nations\textit{. Oxford: Clarendon Press (1976 edition).} 

\item Solnit, Rebecca 
\textit{2005 }Field Guide to Getting Lost\textit{. New York: Viking.} 

\item Sorel, George 
1912 \textit{Réflexions sur la violence}. Paris: M. Riveère et cie. 

\item Starhawk 
\textit{1987 }Truth or Dare: Encounters with Power, Authority, and Mystery\textit{. San Francisco: Harper \& Row.} 
Starhawk 
1993 \textit{The Fifth Sacred Thing}. New York: Bantam. 

\item Starhawk 
\textit{2002 }Webs of Power: Notes from the Global Uprising\textit{. Gabriola, BC: New Society Publishers.} 

\item Starr, Peter 
\textit{1995 }Logics of Failed Revolt: French Theory After May ’68\textit{. Stanford: Stanford University Press.} 

\item Taussig, Michael 
2006 \textit{Walter Benjamin’s Grave}. Chicago: University of Chicago Press. 

\item Thomas, Janet 
\textit{2000 }The Battle in Seattle: The Story Behind and Beyond the WTO Demonstrations\textit{. Golden, CO: Fulcrum Publishing.} 

\item Thompson, E. P. 
1971 ''The moral economy of the English crowd in the eighteenth century,” In \textit{Customs in Common: Studies in Traditional Popular Culture}, 185--258. London: Merlin. 

\item Tobocman, Seth 
\textit{1999 }War in the Neighborhood: A Graphic Novel\textit{. Brooklyn, NY: Autonomedia.} 

\item Vaneigem, Raoul 
1979 Traité de savoir-vivre à l’usage des jeunes générations\textit{. Paris: Gallimard.}  \textit{Livre des plaisirs}. Paris: Encre. 

\item Virno, Paolo 
2004 \textit{A Grammar of the Multitude}. New York: Semiotext(e). 

\item Waddington, P. A. J. 
1999 \textit{Policing Citizens: Authority and Rights}. London: University College London Press. 

\item Wieck, David T. 
1971 “Preface” to \textit{Social Anarchism} (by Giovanni Baldelli). Chicago: Aldine. 

\item Williams, Kristian 
2004 \textit{Our Enemies in Blue}. Brooklyn: Soft Skull Press. 

\item Wolf, Eric 
\textit{1969 }Peasant Wars of the Twentieth Century\textit{. New York: Harper \& Row.} 

\item Zerzan, John 
1994 \textit{Future Primitive and Other Essays}. Brooklyn, NY: Autonomedia. 

\item Zerzan, John 
1999 \textit{Elements of Refusal}. Columbia, MO.: Columbia Alternative Library. 

\item Zerzan, John 
\textit{2002 }Running on Emptiness: The Pathology of Civilization\textit{. Los Angeles: Feral House.} 


\end{itemize}

\chapter*{O Autorze}

\begin{center}
\includegraphics[width=0.2\textwidth]{d-graeber.jpg}
\end{center}


 David Graeber jest antropologiem, wykształconym na Uniwersytecie w~Chicago. Spędził dwa lata prowadząc badania etnograficzne na górskim Madagaskarze i~był adiunktem w~Yale, kiedy po raz pierwszy zaangażował się w~globalny ruch sprawiedliwości w~2000 roku (z nieco niefortunnymi konsekwencjami dla kariery) -- i~pozostał aktywny w~różnych projektach aktywistycznych, obejmujących m.in. od Globalnych Akcji Ludowych do Pracowników Przemysłowych Świata. 
 
 Jest autorem Lost People: Magic and the Legacy of Slavery in Madagascar, Towards an Anthropological Theory of Value, Fragments of an Anarchist Anthropology oraz Mossibilities. Obecnie jest wykładowcą antropologii społecznej na Goldsmiths, University of London, gdzie pracuje nad projektem dotyczącym historii zadłużenia.


%EPUB
\newpage
\printendnotes
%EPUB

\newpage

\begin{center}
 WSPIERAJ WYDAWNICTWO AK !
\end{center}

 AK Press to kolektyw prowadzony przez pracowników, który publikuje i~rozpowszechnia radykalne książki, media wizualne/audio i~inne materiały. Jesteśmy mali: kilkanaście osób, które pracują przez długie godziny za małe pieniądze, bo wierzymy w~to, co robimy. Jesteśmy anarchistami, co znajduje odzwierciedlenie zarówno w~publikowanych przez nas książkach, jak i~w sposobie, w~jaki organizujemy nasz biznes: bez szefów.

 Obecnie wydajemy około dwudziestu nowych tytułów rocznie. Chcielibyśmy opublikować jeszcze więcej. Ilekroć nasz kolektyw spotyka się, aby omówić przyszłe plany wydawnicze, zmagamy się z~listą setek projektów. Niestety pieniędzy jest mało, a zapotrzebowanie na nasze książki jest większe niż kiedykolwiek.

 Przyjaciele AK Press to bezpośredni sposób, w~jaki możesz pomóc. Przyjaciele płacą minimum 25 dolarów miesięcznie (oczywiście nie mamy zastrzeżeń do większych sum), przez okres minimum trzech miesięcy. Pieniądze trafiają bezpośrednio na nasze fundusze wydawnicze. W zamian Przyjaciele automatycznie otrzymują (na czas trwania członkostwa) jedną darmową kopię każdego nowego tytułu AK Press, gdy się pojawi. Przyjaciele otrzymują również 20\% zniżki na wszystko, co znajduje się w~katalogu Dystrybucja prasy AK i~na naszej stronie internetowej -- tysiące tytułów od setek wydawców, z~którymi współpracujemy. Mamy również program, w~którym grupy lub osoby indywidualne mogą sponsorować całą książkę. Prosimy o kontakt w~celu uzyskania szczegółowych informacji. Aby zostać przyjacielem, przejdź do \url{www.akpress.org}



\chapter*{Seria ,,Czarny Kot''}



\begin{center}
\includegraphics[width=0.2\textwidth]{Anarchist_black_cat.png}

\begin{large}
W serii \textit{Czarny Kot} opublikowano online:
\end{large} 
\end{center}

\begin{enumerate}
\item \href{https://archive.org/details/errico-malatesta-w-kawiarni}{W Kawiarni}, Errico Malatesta
\item Akcja Bezpośrednia:Etnografia, David Graeber
\end{enumerate}

\begin{center}

\begin{large}W planach:\end{large}\end{center}

\begin{enumerate}
\item Dla Wygranej, Cory Doctorow
\item Tryton, Samuel R. Delany  
\item Dhalgren, Samuel R. Delany
\end{enumerate}

\newpage

Projekt serii został przygotowywany dzięki Wolnemu Oprogramowaniu. Zestaw narzędzi składa się z:
\begin{itemize}
\item \href{https://ubuntu.com/}{Ubuntu 23.04 Lunar Lobster} -- system operacyjny
\item \href{https://omegat.org/}{OmegaT} -- narzędzie wspomagające tłumaczenie (CAT)
\item \href{https://github.com/soimort/translate-shell}{translate-shell} -- narzędzie do tłumaczenia w~\href{https://translate.google.pl}{Google Translate} przez terminal 
\item \href{https://glosbe.com/en/pl}{Glosbe} -- największy słownik online
\item \href{https://www.wikipedia.org/}{Wikipedia} -- podstawowe źródło tłumaczeń pojęć technicznych, politycznych i~ekonomicznych czy not biograficznych
\item \href{https://www.libreoffice.org/}{LibreOffice} -- przetwarzanie dokumentów 
\item \href{http://pandoc.org}{pandoc} -- uniwersalny konwerter dokumentów 
\item \href{https://www.latex-project.org/}{LaTeX} -- redakcja, skład i~łamanie dokumentu
\item \href{https://sigil-ebook.com/}{sigil} -- przetwarzanie plików ebook
\item \href{https://calibre-ebook.com/}{calibre} -- konwersja plików ebook
\end{itemize}

\newpage

 \end{document}
