\documentclass[oneside,polish,11pt,sfheadings]{mwbk}
%polonizacja
\usepackage[T1]{fontenc}
\usepackage[polish]{babel}
\usepackage[utf8]{inputenc}
\usepackage{polski} 
\frenchspacing 
\usepackage{indentfirst} 
%koniec polonizacja
%grafika
\usepackage{graphicx}
%pakiet czcionki
\usepackage{times}
\usepackage[a5paper]{geometry} %wielkość papieru (148x210-book w~PL)
%gwiazdki
\newcommand{\threeast}{\bigskip\par\centerline{*\,*\,*}\medskip\par}

%EPUB
%\usepackage[hyperfootnotes=true]{hyperref} 
%move footnotes to endnotes
%\usepackage{enotez}
%\let\footnote=\endnote
%\setenotez{
%  list-name = Przypisy,
%  backref = true
%}

%pdf anonimize
%dla EPUB wykomentować
\pdfsuppressptexinfo=-1 %Suppress PTEX.Fullbanner and info of imported PDFs

%pakiet odnośników i~pdf metadata
\usepackage[unicode, pdftex]{hyperref}
\hypersetup{pdfauthor={Cory Doctorow},
            pdftitle={Odchodzac},
            pdfsubject={Walkaway},
            pdfkeywords={tłum. Jacek Hummel, Creative Commons, tłumaczenie CC BY 4.0, powieść, science fiction},
            pdfcreator={pdfLaTeX}}
%dla EPUB koniec wykomentowania


\begin{document}

\title{Odchodząc}
\author{Cory Doctorow}


%--titlepage start
\DeclareRobustCommand{\cs}[1]{\texttt{\char`\\#1}}
\newlength{\tpheight}\setlength{\tpheight}{0.9\textheight}
\newlength{\txtheight}\setlength{\txtheight}{0.9\tpheight}
\newlength{\tpwidth}\setlength{\tpwidth}{0.9\textwidth}
\newlength{\txtwidth}\setlength{\txtwidth}{0.9\tpwidth}
\newlength{\drop}
\newcommand*{\titleSI}{\begingroup% Sagas
\drop = 0.13\txtheight
\centering
{\Huge \textsf{~}}\\[1\baselineskip]
{\huge \textsf{~}}\\[1\baselineskip]
%{\LARGE  \textsf{~}}\\[4\baselineskip]
{\Huge \textsc{Odchodząc}}\\[1\baselineskip]
{\LARGE \textsc{Walkaway}}\\[2\baselineskip]
{\huge \textsc{Cory Doctorow}}\\[4\baselineskip]
{\large Na podstawie wydania TOR, Nowy Jork, 2017 \\ przetłumaczył i~opracował:}\\
{\Large Jacek Hummel}\\[1\baselineskip]
{\normalsize \textit{Tłumaczenie jest dostępne na licencji\\
\href{https://creativecommons.org/licenses/by/4.0/deed.pl}{Creative Commons Uznanie autorstwa 4.0 Międzynarodowe}} \\ [1\baselineskip] \par}
\includegraphics[scale=0.3]{CC.png}

~

\vfill
{\Large {Warszawa, 2021}}\\
%\vspace*{\drop}
\endgroup}
\titleSI
\thispagestyle{empty}
%--titlepage end

\begin{figure}[p]
    \vspace*{-1cm}
    \makebox[\linewidth]{
        \includegraphics[width=1.1\linewidth]{walkaway.jpg}
    }
\end{figure}
\thispagestyle{empty}


\newpage
\thispagestyle{empty}

\vspace*{100mm}

\begin{flushright}

\textit{Dla Erika Stewarta i~Aarona Swartza}

\textit{Pierwsze dni, lepsze narody.}

\textit{Walczymy.}


\end{flushright}

\vfill


\part{komunistyczna impreza}
\chapter*{i}

Hubert Vernon Rudolf Clayton Irving Wilson Alva Anton Jeff Harley
Timothy Curtis Cleveland Cecil Ollie Edmund Eli Wiley Marvin Ellis
Espinoza był za stary na komunistyczne imprezy. W~wie\-ku dwudziestu siedem lat był o~siedem lat starszy niż najstarsza imprezowiczka. Czuł demograficzną
pustkę. Chciał schować się za jedną z~potężnych, brudnych maszyn, które
były usiane na podłodze opuszczonej fabryki. Cokolwiek, żeby uciec
szczerym, płaskim spojrzeniom pięknych dzieci każdego odcienia i~wielkości, które nie mogły zrozumieć, dlaczego starszy facet łazi
dookoła.

-- Chodźmy -- powiedział do Setha, który ściągnął go na imprezę. 

Seth był przerażony starzeniem się demografii pięknych dzieci i~wchodzeniem w~świat braku pracy. Potrafił instynktownie znaleźć najbardziej
ekscentryczne, awangardowe osoby transgresywne przebywające pośród
dzieci, które cofały się we wstecznym lusterku. Hubert, Etcetera
Espinoza spędzał czas z~Sethem, ponieważ częścią planu Setha, żeby
nie opuszczać dzieciństwa, było nieporzucanie przyjaciół z~dzieciństwa.
Był w~tym zakresie natarczywy, a~Hubert Etcetera był popychadłem.

-- To zaraz stanie się \textit{realne} -- powiedział Seth. -- Dlaczego nie
załatwisz nam piwa?

To było dokładnie to, czego Hubert, Etcetera nie chciał robić. Piwo było
tam, gdzie zebrało się najwięcej młodych, beztroskich, radosnych i~dziwnych jak ryby tropikalne. Każda coraz bardziej elfi i~tragiczny.
Hubert, Etcetera pamiętał ten wiek, pewność, że świat był tak popsuty,
że tylko idiota mógłby chcieć uznać ten świat lub jego nieuchronność.
Hubert, Etcetera często konfrontował swoje odbicie w~lustrze łazienki,
patrząc w~swoje oczy w~ich gniazdach posiniaczonych worków i~pamiętał
bycie kimś, kto spędzał każdą minutę na zaprzeczaniu słuszności świata,
a teraz był w~niego uwikłany. Hubert, Etcetera nie mógł się więcej
oszukiwać. Każdy poniżej dwudziestego roku życia łapał to w~sekundę.

-- No \textit{idź}, człowieku, proszę. Wprowadziłem Cię na tę imprezę.
Przynajmniej tyle możesz zrobić.

Hubert, Etcetera nie powiedział oczywistych rzeczy o~tym, że nie chciał
w ogóle przychodzić, a~w~szczególe nie chciał piwa. Było tak wiele
bezcelowych kierunków, w~które mogła się rozwinąć dyskusja z~Sethem.
Miał twarz Piotrusia Pana, przygotowany, żeby być ,,ha ha ale serio'',
póki się nie zmęczyłeś, a~Hubert, Etcetera zaczynał się męczyć
wieczorami.

-- Nie mam żadnych pieniędzy -- powiedział Hubert, Etcetera.

Seth spojrzał się na niego.

-- Och, tak -- powiedział Hubert, Etcetera. -- Impreza komunistyczna.

Seth podał mu dwa czerwone kubki, ich kolor z~pewnością nie był
przypadkiem.

Gdy Hubert, Etcetera zbliżał się do kranów -- przykręconych do pionowego
kawałka stali konstrukcji, która wychodziła z~podłogi i~prowadziła aż do
krokwi, pokrytej żółtymi szachownicami kodów kreskowych, smugami
entropii i~tańczącymi światłami DJ-ów -- próbował się domyślić, które z~tych pięknych dzieci jest barmanem, przewodniczącą, lub komisarzą. Nikt
nie ruszył, żeby mu pomóc, lub go zablokować, gdy się zbliżał, choć
troje dzieci zatrzymało się, żeby intensywnie popatrzyć.

Wszyscy troje nosili okulary Marksa z~wielkimi, krzaczastymi brodami,
jak na filmach z~wokoderami, pełnymi surrealistycznej grozy. Te miały
włosy w~jasnych kolorach i~jedno miało coś w~nich -- przewód od pamięci -- który wypełzał jak macki.

Hubert, Etcetera niezgrabnie napełnił kubek, a~dziewczyna przytrzymała
go, gdy napełniał drugi. Piwo było rozżarzone, lub bioluminescencyjne,
Hubert, Etcetera martwił się, że mogły to być transgeniczne bakterie
jezusa, które potrafiły zamienić wodę w~piwo, ale dziewczyna patrzyła na
niego zza tych okularów, jej oczy nieczytelne w~błyskających światłach
tanecznych. Napił się.

-- Niezłe. -- Beknął, znowu beknął. -- Jednak gazowane?

-- Bo szybko przetwarzają. Godzinę temu to była jeszcze woda z~rowu.
Przefiltrowaliśmy ją, doprowadziliśmy do temperatury pokojowej,
wrzuciliśmy bakterię. Jest też żywe, dodaj prekursor, a~wróci. Przeżyje
w moczu. Tylko zachowaj trochę, jeżeli chcesz zrobić więcej.

-- Komunistyczne piwo? -- powiedział Hubert, Etcetera. Najlepszy bon mot,
na jaki mógł się zdobyć. Był lepszy, gdyby miał czas pomyśleć.

-- Na zdrowie. -- Stuknęła kubek w~jego i~wychyliła go, wypuszczając
potężne beknięcie, kiedy skończyła. Uderzyła się w~pierś i~wydała kilka
drobniejszych czknięć, uzupełniła szklankę.

-- Jeżeli to wychodzi z~sików -- powiedział Hubert, Etcetera -- co się
stanie, gdy ktoś doda tego prekursora do ścieków? Zamienią się w~piwo?

Spojrzała się na niego pogardą młodej osoby. 

-- To byłoby głupie. Kiedy jest rozcieńczone, nie potrafi metabolizować prekursora. Spuść i~to
tylko mocz. Zwierzątka umierają w~godzinę lub dwie, więc latryny nie
zamienią się w~rezerwuar długożyjących egzystencjalnych zagrożeń dla
zasobów wody. To tylko piwo. -- Czknięcie. -- Mocno gazowane.

Hubert, Etcetera łyknął. Było naprawdę dobre. W~ogóle nie smakowało jak
mocz. 

-- Każde piwo jest wynajęte, co? -- powiedział.

-- Większość piw jest wynajętych. To jest darmo. Wiesz, ,,darmo jak w~darmowe piwo''. -- Wypiła pół kubka, rozlewając na swoją brodę. Zrosiło
się na szeleszczących rzeczach dla uchodźców. -- Nie przychodzisz często
na komunistyczne imprezy.

Hubert, Etcetera wzruszył ramionami. 

-- No nie -- powiedział. -- Jestem
stary i~nudny. Osiem lat temu, nie robiliśmy takich rzeczy.

-- A co robiłeś, dziadku? -- Nie w~złośliwy sposób, ale jej dwoje
przyjaciół, dziewczyna w~tym samym odcieniu co Seth i~chłopak z~pięknymi
kocimi oczami, zachichotali.

-- Miał nadzieję dostać pracę przy zeppelinach! -- powiedział Seth,
wsuwając ramię dookoła szyi Huberta, Etcetera. -- Przy okazji, nazywam
się Seth. A to jest Hubert, Etcetera.

-- Et cetera? -- spytała dziewczyna. Tylko uśmieszek. Hubert, Etcetera
polubił ją. Myślał, że była prawdopodobnie potajemnie miła,
prawdopodobnie nie sądziła, żeby był głupkiem, tylko dlatego, że był
kilka lat starszy, i~nie słyszał o~jej ulubionym gatunku syntetycznego
piwa. Uważał, że to przekonanie było spowodowane teorią ludzkości, że
większość ludzi była dobra, ale także przez straszną, opresyjną
samotność i~nieokreślone napalenie. Hubert, Etcetera był bystry, co nie
zawsze było proste, i~umiarkowanie radził sobie z~psyche, która
sprawiała, że tak trudno mu było oszukać samego siebie.

-- Powiedz jej, chłopie -- powiedział Seth. -- No dawaj, to wspaniała
historia.

-- To nie jest wspaniała historia -- powiedział Hubert, Etcetera. -- Moi
rodzice dali mi wiele drugich imion, to wszystko.

-- Jak dużo jest wiele?

-- Dwadzieścia -- powiedział. -- Najczęstsze dwadzieścia imion ze spisu w~1890 roku.

-- To tylko dziewiętnaście -- powiedziała, szybko. -- I~jedno pierwsze
imię.

Seth roześmiał się, jakby to była najśmieszniejsza rzecz, jaką
kiedykolwiek słyszał. Nawet Hubert, Etcetera się uśmiechnął. 

-- Większość
ludzi tego nie łapie. Technicznie, mam dziewiętnaście drugich imion i~jedno pierwsze imię.

-- Dlaczego Twoi rodzice nadali Ci dziewiętnaście drugich imion i~jedno
pierwsze imię? -- spytała. -- I~jesteś pewien, że to dziewiętnaście
drugich imion? Może masz dziesięć pierwszych imion i~dziesięć drugich
imion.

-- Myślę, że byłoby trudno twierdzić, że mam więcej niż jedno pierwsze
imię, ponieważ pierwsze jest specyficzne na sposób, którego brakuje
drugim. Nie licząc Mary Ann i~Jean Marc'ów czy innych, które mają umowne
łączniki.

-- Słuszna uwaga -- powiedziała. -- Jednak, jeżeli Mary Ann jest pierwszym
imieniem, dlaczego nie Mary Ann Tanya Jessie Bananowe Majtki Małpa
Wymiot et cetera?

-- Moi rodzice zgodziliby się. To było ich oświadczenie, po tym, gdy
Anonymous wprowadzili swoją Politykę Prawdziwym Imion. Oboje działali,
pracowali, żeby stała się polityczną partią, zatem byli naprawdę
wkurwieni. Choć było oczywiste, że jeżeli byłeś ,,Anonymous'' nie mogłeś
mieć ,,Polityki Prawdziwych Imion''. Zdecydowali się dać swoim dzieciom
unikalne imię, która nigdy nie będzie pasowało do żadnej bazy danych i~dałoby im prawo do legalnego używania całego mnóstwa imion.

-- Kiedy to zrozumiałem, przyzwyczaiłem się do ,,Huberta'' i~zostałem z~tym.

Seth wziął kubek Huberta, wypił z~niego, beknął. 

-- Zawsze Cię nazywałem
Hubert. Jest w~porządku i~łatwo powiedzieć.

-- Nie mam nic przeciwko.

-- Jednak zrób to, dobra?

-- Co? -- Hubert, Etcetera, znał odpowiedź.

-- Imiona. Musisz to usłyszeć.

-- Nie musisz -- powiedziała.

-- Muszę, prawdopodobnie, lub będziesz się zastanawiała. -- Już dawno się
z tym pogodził. To była część dorastania. -- Hubert Vernon Rudolph
Clayton Irving Wilson Alva Anton Jeff Harley Timothy Curtis Cleveland
Cecil Ollie Edmund Eli Wiley Marvin Ellis Espinoza.

Przechyliła głowę, skinęła głową. 

-- Potrzebuje więcej Bananowych Majtek.

-- Założę się, że cholernie Ci dokuczali w~szkole, tak? -- powiedział
Seth.

To zdenerwowało Huberta, Etcetera. To było głupie, to była ciągle
powracająca głupota. 

-- Serio, naprawdę? Myślisz, że dzieci dokuczają za
\textit{imiona}? Strzałka wynikania wskazuje w~drugą stronę. Jeżeli
dzieciaki robią sobie żarty z~Twojego imienia, to znaczy, że jesteś
niepopularny, a~nie, że jesteś niepopularny z~powodu imienia. Jeżeli
najfajniejszy dzieciak w~szkole nazywał się ,,Harry Piłka'', nazywali go
Baryłka. Jeżeli szkolna koza nazywała się ,,Lisa Zielona'', przezwali ją
,,Obesrana'' -- Prawie dodał ,,\textit{serio}, nie bądź dupkiem'', ale nie
zrobił tego. Inwestował w~bycie dorosłym. Seth nie zwracał uwagi na
możliwość, że był dupkiem.

-- Jak \textit{Ty} się nazywasz? -- spytał się Seth dziewczyny.

-- Lisa Zielona -- powiedziała.

Hubert, Etcetera parsknął.

-- Naprawdę?

-- Nie.

Poczekał, żeby zobaczyć, czy powie mu imię, wzruszył ramionami. 

-- Jestem
Seth. -- Podszedł do jej przyjaciół, którzy zbliżyli się nieco. Jedno z~nich wymyślnie potrząsnęło dłonią, czego znajomość Seth udawał z~całkowitym, nieświadomym entuzjazmem, którego Hubert, Etcetera
zazdrościł i~był zażenowany.

Muzyka taneczna stała się głośna. Seth uzupełnił kubek Huberta, Etcetera
i zabrał go na parkiet. Hubert był jedynym bez kubka. Dziewczyna
napełniła swój i~podała.

-- Dobra rzecz -- wykrzyczała, jej oddech łaskoczący jego policzek. Muzyka
była naprawdę głośna, automatyczny mix, spięty ze sprzętem DJ-a, który
używał lidaru i~mapowania ciepła, żeby określić odpowiedź tłumu na
muzyczne miksy i~optymalizował je, żeby zachęcić wszystkich do tańca.
Mieli to dawniej, kiedy Hubert, Etcetera był wystarczająco młody, żeby
chodzić do klubów, nazywali to ,,Regułą 34'' dla wszystkich różnych
miksów, ale wtedy to było tandetne. Teraz to był interes.

-- Jednak całkiem chmielowe.

-- Nie smak. Enzymy. To coś, co pomaga Ci wchłonąć, wstrzymuje to od
zamiany w~formaldehyd we krwi. Dobre dla zmniejszenia kaca. Tureckie.

-- Tureckie?

-- Cóż, tureckawe. Wyszło z~obozów dla uchodźców w~Syrii. Mieli
laboratorium. Jest nazywane Gezi. Jeżeli jesteś ciekaw, mogę Ci podesłać
info na ten temat.

Czy ona go podrywała? Osiem lat temu, podanie komuś informacji
kontaktowych było zaproszeniem. Może przehuśtaliby się w~czas bardziej
pozamałżeńskiego zarządzania przestrzenią imion i~mniej pozamałżeńskich
norm społeczno \dywiz seksualnych. Hubert, Etcetera marzył, żeby przejrzeć
abstrakt współczesnej socjologii dwudziestolatków. Potarł pasek
interfejsu na palcu serdecznym i~wymamrotał ,,szczegóły kontaktu'',
wystawił dłoń. Jej dłoń była ciepła, szorstka i~drobna. Dotknęła paska,
który nosiła jako naszyjnik, wyszeptała, a~on poczuł potwierdzające
brzęczenie ze swojego systemu, potem podwójne brzęczenie, które
oznaczało, że odwzajemniła.

-- Żebyś mógł mnie dodać do białej listy.

Hubert, Etcetera zastanawiał się, czy była przyzwyczajona do dzielenia
się informacjami kontaktowymi, że musiała się martwić o~spam\ldots 

-- Nigdy nie byłeś na takiej imprezie -- powiedziała, jej twarz znowu tuż
koło jego ucha.

-- Nie -- krzyknął. Jej włosy pachniały jak spalone opony i~lukrecja.

-- Pokochasz to, chodź, podejdźmy bliżej, zaraz zaczną.

Znowu wzięła jego dłoń, a~gdy jej palce dotknęły jego skóry, poczuł
kolejne drżenie. To było endogeniczne i~nie pochodziło z~jego
interfejsu.

\threeast

Obeszli tańczących, przebijając się przez liście i~chmury kurzu, które
wirowały w~światłach. W~kurzu były błyszczące pyłki, które sprawiały, że
powietrze wyglądało, jakby było pełne pyłu wróżek. Hubert, Etcetera
dojrzał Setha. Seth spojrzał na niego i~obejrzał całą scenę, dziewczyna,
dłonie, gramolenie się przez ciemne miejsca do prywatnego punktu
widokowego, jego twarz skrzywiona przechodzącą zazdrością, zanim
zmieniła się w~rubaszne spojrzenie, do którego dodał kciuki do góry.
Automatyczna muzyka dudniła, Cantopop i~rumba, które Reguła 34 wylewała
z bezpośredniego losowego bitu w~przestrzeń muzyczną.

-- Tu jest dobrze -- powiedziała, gdy wspięli się na pomost. Ziarnista
drabina zostawiła ślady rdzy na palcach Huberta, Etcetery. Poza
uderzeniem muzyki mogli się dosłyszeć, Hubert, Etcetera był świadom
swojego oddechu i~pulsu.

-- Patrz się na to. -- Wskazała na maszynę z~boku. Hubert, Etcetera
zmrużył oczy i~zobaczył jej przyjaciół z~wcześniej coś przy niej
robiących. 

-- Robią meble, w~większości regały. Było mnóstwo surowców w~magazynie.

-- Czy pomagałaś zorganizować to\ldots  -- Szeroki ruch ręką, żeby objąć
fabrykę, tancerz -- \ldots  razem?

Położyła palec na gumowy nosie, powoli mrugnęła okiem. 

-- Najwyższy
Komitet -- powiedziała. Dotknęła boku okularów i~zauważył migotanie, gdy
powiększenie się włączyło, z~fałszywymi kolorami i~stabilizacją. -- Mają
to. -- Muzyka zamilkła w~pół tonu.

Huk w~trzewiach fabryki zawibrował podestem. Tancerze rozejrzeli się
dookoła za jego źródłem, potem fala uwagi przepłynęła przez nich, gdy
wzrok podążał za wzrokiem i~skupiali się na maszynie, która się ruszyła,
kurz opadający, światła oświetlające ją, podkreślające więcej pyłków.
Nowy zapach do tego, drewniany, pełen niebezpiecznych substancji
lotnych, które parowały z~części maszyny, gdy wracała do życia. Cisza w~pokoju pękła, gdy pierwsza deska kompozytowa wypadła na łoże montażowe,
dotknięte przez tysiące minimalnych palcy, które poprawiały jej
położenie akurat, gdy wysunęła się kolejna płyta. Teraz pojawiały się w~regularnych okresach, ciąg cienkich, mocnych, giętkich płyt
celulozowych, gładko łączonych na złączach, również przesuwały na
pozycję, ustawiając prefabrykowane elementy stolarskie, które łączyły
się razem z~nacięciem. Palce uniosły kratę, przesunęły wzdłuż linii, a~teraz nowa krata była składana tak samo szybko, potem były łączone razem
z kliknięciem.

Więcej z~nich, potem rzucona pętla mocującej tkaniny, złapanej i~zaciskającej się na ramie, skończony przedmiot był rzucany na bok.
Kolejny był minutę w~tyle na linii. Tancerka podeszła do linii
montażowej, łatwo podniosła skończony element, przyniosła go jedną ręką
na parkiet, rozcięła zapięcie nożem, który zabłysł w~światłach
dyskoteki. Łóżko -- tym właśnie było -- ze stukotem się rozłożyło, gotowe
na materac. Tancerka wspięła się na listwy łóżka, zaczęła podskakiwać do
góry i~dołu. Łóżko było sprężyste jak trampolina, za chwilę tancerka
robiła salta, obroty i~szpagaty w~powietrzu.

Dziewczyna usiadła i~przebiegła palcem po swojej brodzie. 

-- Niezła
rzecz. -- Hubert, Etcetera był pewien, że się uśmiechała.

-- To super rama łóżka -- powiedział Hubert, Etcetera z~braku czegoś
lepszego do powiedzenia.

-- Jedna z~najlepszych -- powiedziała. -- Mają mnóstwo rentownych linii,
ale ramy do łóżek były najlepsze. Ważne dla hoteli, ponieważ są
praktycznie niezniszczalne i~są lekkie jak piórko.

-- Dlaczego już ich więcej nie robią?

-- Och, robią. Muji zamknął fabrykę i~przeniósł się do Alberty sześć
miesięcy temu. Dostał potężną dotację za przeniesienie, Ontario nie
miało jak rywalizować. Byli tutaj tylko przez kilka lat, tylko
zatrudnili zaledwie dwadzieścia osób, ich dwuletnie wakacje podatkowe
się kończyły. Miejsce od tego czasu było puste. Możemy robić tutaj
wszystkie linie, wszystkie meble Muji, włączając nawet te niemarkowe
rzeczy, które robią dla Nestle, Standard \& Poors i~Moët \& Chandon.
Krzesła, szafki, stoły, regały. W~Orangeville jest pusta fabryka
surowców, do której uderzamy na następnej imprezie, surowe materiały dla
łańcucha dostaw. Jeżeli nie zostaniemy złapani, możemy zrobić
wystarczająco mebli dla kilku tysięcy rodzin.

-- Nie każecie za nie płacić, czy coś?

Długie spojrzenie. 

-- Impreza komunistyczna, zapomniałeś?

-- Tak, ale jak jecie i~tak dalej?

Wzruszyła ramionami. 

-- Tu i~tam. To i~tamto. Uprzejmość obcych.

-- Zatem ludzie przynoszą wam jedzenie i~dajecie im te rzeczy?

-- Nie -- odpowiedziała. -- Nie robimy barteru. To są dary, ekonomia daru.
Wszystko za darmo, nic w~zamian.

Teraz była kolej Huberta, Etcetera. 

-- Jak często dostajesz dar w~czasie,
gdy rozdajesz te innym osobom? Kto nie pojawia się z~czymś, co zostawia,
kiedy coś zabiera?

-- Oczywiście. Trudno oduczyć ludzi zwyczaju ,,quid pro quo'' z~czasu
niedoborów. Jednak wiedzą, że nie \textit{muszą} niczego przynosić. Czy
\textit{Ty} coś przyniosłeś dzisiaj?

Poklepał się po kieszeniach. 

-- Mam kilka milionów dolców, nic więcej.

-- Zachowaj je. Pieniądze to jedyna rzecz, której nie bierzemy. Moja mama
zawsze mówiła, że pieniądze są najgorszym prezentem. Każdego, kto daje
lub bierze pieniądze tutaj, wyrzucamy, żadnych drugich szans.

-- Nie będę wyjmował portfela.

-- Dobry pomysł. -- Było wystarczająco miła, żeby nie zauważyć podwójnej
konotacji, która sprawiła, że Hubert, Etcetera się zarumienił. -- Przy
okazji, nazywam się Żarciuszka.

-- A ja myślałam, że moi rodzice byli szaleni.

Broda poruszyła się tajemniczo. 

-- Moi rodzice nie nadali -- powiedziała. -- To moje imię imprezowe.

-- Jak Trocki -- powiedział. -- Nazywał się Lew Dawidowicz. Zrobiłem kurs
niezależnej historii w~jedenastej klasie. To jest znacznie bardziej
interesujące.

-- Mówią, że Stary Karl dobrze diagnozował, ale dał złą receptę. -- Wzruszyła ramionami. -- Łączenie ,,imprezy'' z~komunizmem robi różnicę.
Sąd ciągle obraduje. Prawdopodobnie implodujemy. Wy tak mieliście,
prawda? Zeppeliny?

-- Zeppeliny \textit{eksplodują} -- powiedział.

-- Ha, Har, har.

-- Przepraszam. -- Wyciągnął nogi i~oparł je o~balustradę, która
zaskrzypiała, potem wytrzymała. Uświadomił sobie, że mógł przelecieć i~upaść z~dziesięciu metrów na betonową posadzkę. -- Ale tak, zeppy nie
wyszły. 

Na papierze były doskonale sensowne. Wszyscy ci ludzie biedni,
ale z~dużą ilością czasu z~przyjaciółmi na całym świecie. Zeppy były
tanie jak cholera do latania, jeżeli nie dbałeś gdzie i~jak szybko
leciałeś. Były setki startupów, mówiących poważnie o~transporcie
dostosowanym do klimatu i~,,nowym wieku awiacji''. Pomimo tego
wszystkiego, czuli nieuchronnie, że byli częścią gorączki złota, grą z~musicalowymi krzesłami, która skończyłaby się z~kilkoma szczęśliwymi
duszami siedzącymi na wystarczającym kapitale, żeby przestać udawać
troskę o~jakąkolwiek awiację prócz tej, która miała na pokładzie
szampana i~ciepłą maskę na oczy po starcie. Mnóstwo pieniędzy
przeleciało dookoła, wiele gadania rządów o~kształceniu lokalnych
talentów i~nowej rzeczywistości przemysłowej. Mowy towarzyszyły wielkim
upustom podatkowym na Badania i~Rozwój i~jeszcze większym środkom na
inwestycje.

Po trzech latach -- w~trakcie, których Hubert, Etcetera i~wszyscy,
których znał dali wszystko walce, żeby wysłać wielkie, latające cygara w~powietrze -- sprawa implodowała. Tylko kilka lat później, było to retro.
Hubert, Etcetera widział ,,oryginalną komfortową kabinę Zeppelin Mark
II'' na filmie z~supermodnym wystrojem. Mozolnie odtworzony zestaw mebli
latającej sypialni został dopasowany do potrzeb dwóch bogatych,
stacjonarnych osób, nie dla tuzina wędrownych, latających włóczęg.
Hubert, Etcetera kiedyś spędził trzy miesiące w~spółdzielni, która
budowała prefabrykowane kabiny, gotowe do zamontowania na platformie
sterowców. Jego wkład pracy miał upoważniać go do określonego czasu
każdego roku w~powietrzu na pokładzie dowolnego statku posiadającego
kabinę spółdzielni, błąkając się na dominujących wiatrach świata,
gdziekolwiek.

-- Nie Twoja wina. To jest w~naturze bestii, żeby wierzyć w~bańki i~myśleć, że możesz po prostu wymyślić przedsiębiorczy sposób wyjścia. -- Odpięła brodę i~szkła. Jej twarz była lisia, mnóstwo punktów, ze
śladami, gdzie ciężkie szkła się opierały, lśniąca od potu. Wytarła pot
koszulką, dając mu dostrzec blady brzuch, pieprzyk przy pępku.

-- A Twoi ludzie tutaj? -- Marzył o~piwie, uświadomił sobie, że musi się
odsikać, zastanawiał, czy powinien trochę zachować, żeby zrobić więcej.

-- Nie zamierzamy opracować przedsiębiorczych sposobów na cokolwiek. To
nie jest przedsiębiorczość.

-- Antyprzedsiębiorczość też była sprawdzona, obijanie się nigdzie Cię
nie prowadzi.

-- Też nie jesteśmy anty-przedsiębiorczy. Nie jesteśmy przedsiębiorczy w~taki sam sposób, w~jaki gra w~baseball nie jest grą w~kółko i~krzyżyk.
Gramy w~odmienną grę.

-- Jaką?

-- Post-niedobór -- powiedziała z~prawie religijną powagą.

Nie udało mu się zachować spokojnej miny, ponieważ wyglądała na
wkurzoną. 

-- Przepraszam. -- Hubert, Etcetera był jednym z~naturalnych
przepraszaczy. Współlokator kiedyś zrobił zestaw kartonowych nagrobków
na Halloween, zawiesił je jak flagi na kuchennych szafkach. Na Huberta,
Etcetery było ,,Przepraszam''.

-- Nie przepraszaj mnie. Spójrz, Etcetera, na to wszystko. Na papierze,
to miejsce jest bezużyteczne, rzeczy wychodzące z~tej linii powinny być
zniszczone. To jest naruszenie znaku towarowego, mimo tego, że wyszło z~oficjalnej linii Muji, używając surowców Muji, nie ma licencji Muji,
zatem ta konfiguracja celulozy i~kleju jest przestępstwem. To jest tak
ewidentnie zjebane, że każdy, kto zwraca na to uwagę, gra w~złą grę i~nie zasługuje na uwagę. Ktokolwiek mówi, że świat jest lepszym miejscem
z tą fabryką zostawioną na zgnicie\ldots 

-- Nie sądzę, żeby to był argument -- powiedział Hubert, Etcetera. Kiedyś
odbył bardzo wiele takich dyskusji. Nie był młody ani awangardowy, ale
rozumiał to. -- To mówienie ludziom, co mogą zrobić ze swoimi rzeczami,
daje gorsze wyniki niż pozwolenie im na robienie głupich rzeczy i~pozwolenie rynkowi oddzielić dobre pomysły od \ldots 

-- Myślisz, że ktokolwiek jeszcze w~to wierzy? Wiesz, dlaczego ludzie,
którzy potrzebują mebli, po prostu nie włamią się do tego miejsca? To
nie jest ortodoksja rynkowa.

-- Oczywiście, że nie. To strach.

-- Mają prawo się bać. W~tym świecie, jeżeli nie odnosisz sukcesów,
jesteś porażką. Jeżeli nie jesteś na górze, jesteś na dnie. Jeżeli
jesteś pomiędzy, wisisz na paznokciach, mając nadzieję, że uda Ci się
lepiej chwycić, zanim osłabniesz. Wszyscy, którzy wiszą, boją się zbyt
mocno, żeby puścić. Każdy, kto jest na dnie, jest zbyt zużyty, żeby
znowu próbować. Ludzie na szczycie? Oni są tymi, którym zależy, żeby
rzeczy tak trwały.

-- Zatem jak nazwiesz swoją filozofię? Post-strach?

Wzruszyła ramionami. 

-- Nie obchodzi mnie to. Jest na to mnóstwo nazw.
Żadna z~nich nie jest ważna. \textit{To} mnie obchodzi. -- Wskazała na
tancerzy i~łóżka. Kolejna linia produkcyjna była włączona i~zestawy
składanych stołów i~krzeseł piętrzyły się.

-- A co z~,,komunistycznym''?

-- Co z~tym?

-- To etykieta z~bagażem historii. Mogłabyś być komunistką.

Pomachała do niego brodą. 

-- Komunistyczna \textit{impreza}. To wcale z~nas
nie robi ,,komunistów'', nie bardziej jak zrobienie imprezy urodzinowej
zrobi z~nas ,,urodzinowców''. Komunizm jest interesującą rzeczą do
robienia, niczym w~czym chciałabym być.

Drabina brzęknęła i~podest zawibrował jak kamerton. Spojrzeli nad
krawędzią akurat, gdy pojawiła się głowa Setha. 

-- Halo, gołąbki! -- powiedział. Był rzewny i~roztrzęsiony, po czymś interesującym. Hubert,
Etcetera złapał go, zanim mógł przelecieć nad balustradą. Kolejna osoba
pojawiła się nad krawędzią, jedna z~trzech brodatych, która była przy
piwie.

-- Hej, hej! -- Też wydawał się najarany, ale trudno było Hubertowi,
Etcetera to określić.

-- To ten facet -- powiedział Seth. -- Facet z~imionami.

-- Ty jesteś Etcetera! -- powiedział nowy facet, ramiona szeroko rozłożone
jakby witał utraconego brata. -- Nazywam się Billiam. -- Objął Huberta,
Etcetera powolnym, pijackim objęciem. Hubert, Etcetera randkował z~chłopakami, był otwarty na pomysł, ale Billiam, mimo pięknych,
uniesionych oczu, nie był w~jego typie i~na zbyt wielkim haju, żeby
nawet cokolwiek rozważać. Hubert, Etcetera delikatnie go oderwał, a~dziewczyna pomogła.

-- Billiam -- powiedziała -- co robiliście?

Billiam i~Seth spojrzeli na siebie i~zaczęli się histerycznie chichotać.

Figlarnie pchnęła Billiama, od czego się zatoczył, jedna stopa wisząca
ponad podestem.

-- Meta -- powiedziała. -- Lub coś w~tym rodzaju.

Słyszał o~tym. Dawało ironiczny dystans, haj typu ,,bardzo teraz''.
Ludzie od teorii spiskowych myśleli, że to było zbyt zeitgeist, żeby
było przypadkiem, twierdzili, że zostało rozprzestrzenione, żeby
zmiękczyć populację. W~jego czasach -- osiem lat temu -- plaga była
nazywana ,,Teraz'', coś, co dawali audytorom kodu źródłowego i~pilotom
dronów, żeby skupili się jak roboty. Zjadł tony tego, kiedy pracował
przy zeppach. Czuł się po tym jak szczęśliwy android. Ludzie od teorii
spiskowych mówili te same rzeczy o~,,Teraz'', co teraz mówili o~,,Meta''. Na koniec dnia, cokolwiek, co umożliwiało anulować obiektywną
rzeczywistą i~przydzielić premię jakiemuś wewnętrznemu stanu umysłowemu,
było jednocześnie skuteczne w~przetrwaniu i~w status quo.

-- Jak się nazywasz? -- spytał Hubert, Etcetera.

-- Czy to ważne? -- powiedziała.

-- Wkurza mnie to -- przyznał.

-- Masz mnie w~książce adresowej -- powiedziała.

Przewrócił oczami. Oczywiście, że miał. Potarł interfejs na nadgarstku i~przez chwilę poruszał palcem. 

-- Natalie Redwater? -- powiedział. -- Z~\textit{tych} Redwaterów?

-- Jest wiele osób o~nazwisku Redwater -- powiedziała. -- Jesteśmy
niektórymi z~nich. Jednak nie tymi, o~których myślisz.

-- Blisko nich -- powiedział Billiam z~jego zjaranego, skłonnego do ironii
świata. -- Kuzyni?

-- Kuzyni -- powiedziała.

Hubert, Etcetera mocno walczył, żeby pojęcia jak ,,złota młodzież'' i~,,quasibohema'' nie przeszły mu przez głowę. Prawdopodobnie mu się nie
udało. Nie wyglądała na zadowoloną, że jej nazwisko było jawne.

-- Kuzyni jak ,,rodzina z~prowincji'' -- spytał Seth, z~jego leżącej
pozycji -- czy kuzyni jak w~,,skorzystaj z~tego małego samolotu''?

Hubert, Etcetera poczuł się źle, nie dlatego, że był nią zauroczony.
Znał ludzi urodzonych do przywilejów, mnóstwo ich w~grupach zeppów, i~oni potrafili być miłymi ludźmi, dla których istotne fakty wykraczały
poza niezasłużony przywilej. Seth zwykle nie był gnojkiem w~takich
sprawach -- precyzyjnie to były tego typu sprawy, w~których normalnie nie
zachowywał się jak gnojek -- ale był na haju.

-- Kuzyni jak w~,,wystarczająco, żeby martwić się o~porwania'' i~,,nie
wystarczająco, żeby zapłacić okup'' -- odparła, tonem kogoś, wielokrotnie
powtarzającego tę samą frazę.

Przybycie dwóch najaranych chłopaków wyssało magię z~nocy. Pod nimi,
maszyny odnalazły stały rytm, i~Reguła 34 znowu się rozkręciła,
mieszając witch house z~Nowy Romantyzmem, automatycznie dopasowując się
do bitu maszyn. Nie wciągało to wielu tancerzy, ale kilku ciągle
zaangażowanych tam było, pięknych i~w ruchu. Hubert, Etcetera zagapił
się na nich.

Trzy rzeczy się wydarzyły: muzyka zmieniła się (psychobilly i~dubstep),
otworzył usta, żeby coś powiedzieć, a~Billiam powiedział, chichocząc,
śpiewająco: 

-- Złapani -- i~wskazał na sufit.

Spojrzeli za jego palcem i~zobaczyli gromadę dronów odrywających się z~sufitu, składających skrzydła i~spadających. Natalie nałożyła z~powrotem
brodę i~Billiam upewnił się, że jego też była założona.

-- Seth, maski! -- Hubert, Etcetera potrząsnął przyjacielem. Był jakiś
dobry powód, żeby Seth miał maski dla nich obu, ale nie potrafił sobie
przypomnieć. Seth usiadł z~uniesionymi brwiami i~uśmieszkiem na ustach.
Przyciskając podbródek do piersi, Hubert, Etcetera stanął nad Sethem i~ostro przejrzał jego kieszenie. Wcisnął jego maskę na twarz i~poczuł, że
materiał przylega fragmentami i~spiralami, gdy jego oddech drażnił ją i~lipidy jego skóry przeciskały się przez splot. Założył też Sethowi.

-- Nie potrzebujesz tego robić -- powiedział Seth.

-- Racja -- powiedział Hubert, Etcetera. -- To z~dobroci mojego serca.

-- Martwisz się, że sprawdzą mój graf socjalny i~odkryję Ciebie w~odległości jednego skoku łamane przez intensywne kontakty. -- Uśmiech
Seth, błyszczący w~ciemności na twarzy był denerwująco spokojny. Zniknął
za maską. To była ta głupia Meta. -- Wtedy będziesz trafiony. Przelecą
przez Twoje dane wstecz latami, chłopie, aż coś znajdą. Zawsze coś
znajdują. Przycisną Cię, zagrożą wszystkim najstraszniejszym, chyba że
staniesz się donosicielem. Pokój 101 do samego końca, kochany\ldots 

Hubert, Etcetera klepnął Setha w~głowę ,,mocniej niż trzeba''. Seth
powiedział: 

-- Au -- łagodnie i~przestał gadać. 

Drony leciały w~formacji
pokrycia jak gołębie na dragach. Powierzchnie interfejsów Huberta,
Etcetera zadrżały, gdy wykryły próby ataków i~się wyłączyły. Hubert,
Etcetera regularnie ściągał aplikacje blokujące, tylko żeby obronić się
przez dziwakami kradnącymi tożsamość, ale również zadrżał, zastanawiając
się, czy miał lepsze aktualizacje niż boty policji.

Impreza się skończyła. Tancerze się rozbiegli, niektórzy trzymając
meble. Muzyka wzrosła do głośności ofensywnej, dźwięk tak głośny, że aż
oczy bolały. Hubert, Etcetera zacisnął dłonie na uszach, gdy jeden z~dronów uderzył w~dźwigar i~obrócił się, uderzając w~ziemię. Dron uderzył
w panel sterowania systemu nagłośnienia, zrzucając go na ziemię. Dźwięk
się nie skończył.

Hubert, Etcetera szarpnął Setha, żeby usiadł, wskazał na drabinę.
Musieli odsłonić uszy, gdy schodzili. To była tortura: brutalny dźwięk,
bolesne wibracje metalu pod dłoniami i~stopami. Natalie zeszła, wskazała
na drzwi.

Coś ciężkiego i~bolesnego uderzyło Huberta Etcetera w~głowę i~ramię,
przewracając go na kolana. Stanął na czworaka, potem na nogi, widząc
gwiazdy za maską.

Rozejrzał się za tym, co go uderzyło. Zajęło mu sekundę, żeby zrozumieć,
co zobaczył. Billiam leżał na podłodze, kończyny w~dziwnej swastyce,
głowa widzialnie zdeformowana, atramentowa kałuża krwi rozlewała się
dookoła w~półmroku. Walcząc z~nudnościami i~bólem od hałasu, pochylił
się nad Billiamem i~ostrożnie zdjął maskę. Była przesycona krwią. Twarz
Billiama została zmiażdżona w~parodię ludzkiej twarzy, czoło miało
okropne wgięcie obejmujące jedno oko. Hubert, Etcetera próbował
sprawdzić puls na nadgarstku Billiama, potem na jego szyi, ale wszystko,
co czuł, to łomot muzyki. Położył dłonie na klatce piersiowej Billiama,
żeby wyczuć podnoszenie i~opadanie oddechu, ale nie potrafił określić.

Rozejrzał się, ale Seth i~Natalie już dotarli do drzwi. Musieli nie
zauważyć upadku Billiama, musieli nie zauważyć jego uderzenia w~Huberta,
Etcetera. Dron potargał włosy Huberta, Etcetera. Hubert, Etcetera chciał
płakać. Stłumił uczucia, przypominając sobie pierwszą pomoc. Nie
powinien poruszać Billiama. Ale jeżeli zostanie, to go złapią. Mogło już
być za późno. Część jego mózgu odpowiedzialna za tchórzliwe
samousprawiedliwienie mówiła: dlaczego po prostu nie pójść? To nie tak,
że możesz cokolwiek zrobić. On już może nie żyje. Wygląda na martwego.

Hubert, Etcetera zbadał świadomie ten głos i~doszedł do wniosku, że głos
był dupkiem. Próbował pomyśleć mimo egoistycznych racjonalizacji. Złapał
torbę, którą ktoś zostawił, i~delikatnie przewrócił Billiama do pozycji
bocznej ustalonej, podłożył torbę pod głowę. Podpierał Billiama
zniszczonym krzesłem i~rurą, oczy zmrużone, w~głowie huk, gdy ktoś
złapał go za bolące ramię. Prawie się zrzygał. To był dzień, o~którym
przez całe życie wiedział, że nadejdzie, że skończy w~więzieniu.

Jednak to nie był gliniarz, to była Natalie. Powiedziała coś
niedosłyszalnego w~muzyce. Wskazał na Billiama. Uklękła i~zapaliła
światło. Wyrzygała, będąc na tyle przytomną, żeby zrobić to do swojej
torebki. Hubert, Etcetera zauważył w~tle, że myślała o~komórkach
przełyku i~DNA. Ta odległa część doceniła jej przewidywanie. Wstała,
złapała go za złe ramię, szarpnęła mocno. Krzyknął z~bólu, dźwięk
stracony w~ryku, i~poszedł, zostawiając Billiama.

\chapter*{ii}

Seth boleśnie zszedł z~Mety, około czwartej nad ranem, gdy siedzieli w~jarze, słuchając dzwonienia w~uszach i~bulgoczącej wody poniżej,
nasłuchując wydajnego świstu mijających pojazdów organów ścigania na
drodze powyżej. Siedział na pniaku z~tym uśmieszkiem wyższości, potem
łkał, głowa w~dłoniach, zgięta pomiędzy nogami, z~nieświadomym rykiem
dziecka.

Hubert, Etcetera i~Natalie patrzyli na niego ze swoich miejsc przy pniu
drzewa, trzymając się wobec spadku jaru. Poszli do niego. Hubert,
Etcetera dziwnie objął go, Seth zatopił twarz na piersi Huberta,
Etcetera. Natalie pogłaskała jego ramię, wymamrotała rzeczy, o~których
Hubert, Etcetera pomyślał jako \textit{kobiecych} w~pewnym pocieszającym
znaczeniu. Hubert, Etcetera był świadom płaczu Seth i~możliwości, że
mógł zostać wykryte przez aparaty organów ścigania. To interferowało z~jego empatią, która w~ogóle nie była tak rozległa, ponieważ Seth był
spieprzony, ponieważ wziął ten głupi narkotyk na modnej imprezie, na
którą nie musieli iść, a~teraz Hubert, Etcetera był pokryty wyschniętą
krwią, której nie był w~stanie wytrzeć w~wilgotne liście i~skały.

Hubert, Etcetera przycisnął twarz Seth mocniej do piersi, częściowo,
żeby go stłumić. Uszy Huberta, Etcetera ciągle dzwoniły, jego głowa
pulsowała według tętna, jego palce swędziły od miękkiej ruiny twarzy
Billiama. Był pewien, że Billiam nie żył, kiedy wychodzili. A ponieważ
był Hubertem, Etcetera, był podejrzliwy co do tej pewności, ponieważ,
jeżeli Billiam już nie żył, to nie zostawili go, żeby umarł samotnie na
podłodze.

Natalie poklepała ramię Setha.

-- No weź, chłopie -- powiedziała. -- To zjazd. Przemyśl to ze mną, możesz
sobie poradzić ze zjazdem Mety, to część pakietu. No proszę, Steve.

-- Seth -- powiedział Hubert, Etcetera.

-- Seth -- powiedziała. Była tak niecierpliwa w~stosunku do Setha, jak on
sam. -- Proszę. Przemyśl to. To jest straszne, to jest okropne, ale to
nie jest Twoja prawdziwa reakcja, to tylko narkotyk. Proszę, Seth,
przemyśl to. -- Ciągle powtarzała ,,przemyśl to''. To musiało być to, co
mówiło się ludziom, którzy przechodzili ciężkie czasy z~Meta. Też to
powiedział i~szlochy Setha ucichły. Przez chwilę był cichy, potem
zachrapał miękko.

Natalie i~Hubert, Etcetera spojrzeli po sobie. 

-- Co teraz? -- spytała
Natalie.

Hubert, Etcetera wzruszył ramionami. 

-- Seth ma żetony, żeby pojechać do
domu. Moglibyśmy go obudzić.

Natalie zacisnęła mocno oczy. 

-- Nie chcę niczego stąd wysyłać.
Przyszliście na blokadzie, tak?

Hubert, Etcetera nie przewrócił oczami. Jego pokolenie udoskonaliło
blokady, zmuszając ich systemy, żeby całkowicie się wyłączyły w~drodze
na imprezy. Nie było to proste, ale wszyscy zbyt leniwi, żeby się tym
przejmować, skończyli w~więzieniu, czasem z~przyjaciółmi, więc stało się
to powszechne.

-- Przyszliśmy na blokadzie -- powiedział. Pojechali samochodem do miejsca
z tysiącem statystycznie możliwych celów w~zasięgu krótkiego spaceru,
przeszli dłuższą drogą na imprezę. Nie byli głupi.

-- Dobra, myślisz, że jest bezpiecznie na restart?

-- Bezpiecznie na co?

Mógł zauważyć jej stłumione przewrócenie oczami. 

-- Na akceptowalne
ryzyko. A jeżeli spytasz ,, akceptowalne w~jaki sposób'', walnę Cię.
Myślisz, że to dobry pomysł na odpalenie?

-- Chciałbym spytać ,,dobry w~porównaniu z~czym''. Nie wiem, Natalie.
Myślę\ldots  -- Przełknął. -- Jestem całkiem pewny, że Billiam jest. -- Przełknął. -- Myślę, że nie żyje. -- Żadne z~nich nie spojrzało na drugie.
Taki głupi wypadek. -- Ponadto myślę, że to znaczy, że policja będzie
brutalna, ponieważ trup stawia sprawy w~zupełnie innej kategorii. Z~drugiej strony, nasze DNA jest w~całym miejscu, a~przy szumie, jaki z~tego zrobią, przyjdą po nas niezależnie od wszystkiego. Z~drugiej
strony, znaczy, w~dodatku do tamtej, lub z~tym w~głowie, jeżeli teraz
odpalimy, dodajemy konfirmację do dowolnej inferencji, która mówi, że
byliśmy tutaj, co znaczy, że \ldots 

-- Wystarczy paranoicznego chowania głowy w~piasek. Nie możemy odpalić.

-- Jak się tutaj dostałaś?

-- Przyjaciel -- powiedziała. -- Jestem pewna, że dostała się do domu, jest
jej ciepło i~przytulnie pod kocem z~kubkiem herbaty dla niej, kiedy
wstanie. -- Natalie brzmiała gorzko po raz pierwszy. Hubert, Etcetera
uświadomił sobie, że w~połowie zamarzł i~w połowie zgłodniał, tak
spragniony, jakby wnętrze jego ust zostało pomalowane skrobią.

-- Musimy ruszać. -- Spojrzał na siebie. W~szarym świetle poranka,
zaschnięta krew wyglądała jak błoto. -- Myślisz, że mógłbym wejść do
metra tak jak teraz?

Wykręciła szyję, przesuwając dzikie elementy Seth z~kolan. 

-- Nie tak jak
teraz. Ale może w~kurtce Steve'a.

-- Setha -- powiedział.

-- Nieważne. -- Potrząsnęła Seth za ramiona, nieco ostro. -- Chodź, Seth,
czas iść.

\threeast

Dotarli na stację o 5:30, Hubert, Etcetera ubrany w~kurtkę Setha, która
była dla niego za duża, niosąc swoją kurtkę pod pachą. Wtoczył się
pierwszy pociąg i~weszli razem z~zaspanymi pracownikami na rano i~mrugającymi imprezowiczami. Ludzie z~pracą gapili się na imprezowiczów.
Ludzie z~pracą pachnieli ładnie, imprezowicze już nie, nawet dla
stłumionego nosa Huberta, Etcetera. Podczas bańki zeppelinów, wcześnie
wstawał, gdy zapieprzali na bezsensowne deadliny z~pilnością niczym przy
wypadku samochodowym bez dostrzegalnego powodu. Jeździł pierwszym
pociągiem do pracy. Cholera, spał w~biurze.

Zjazd Setha się uspokoił. Był idealnym obrazem olejnym pod tytułem
,,Człowiek z~kacem narkotykowym'' w~brudnych kolorach, dużo cieni i~kreskowania. Zimne powietrze zmieniło kolor jego nagich ramion na szary,
ale Hubert, Etcetera nie czuł się źle za zarekwirowanie kurtki. 

-- Patrzcie na nich -- powiedział Seth szeptem scenicznym. -- Tak dobrze
wychowani. -- Oni byli Hindu, Persem, białymi, ale wszyscy tacy sami,
wszyscy w~swoich szanowanych ubraniach ludu pracującego. Para
zatrudnionych rzuciła im gówniane spojrzenie. Seth zauważył, gotów
rzucić się do walki.

-- Nie -- powiedział Hubert, Etcetera, gdy Seth powiedział: 

-- To ostateczne samooszukiwanie. Jakby byli w~stanie zmienić cokolwiek za
wypłatę. Jeżeli wypłata mogłaby zmienić Twoje życie, myślisz, że
pozwoliliby wam je dostać?

To był dobry tekst. Seth użył go już wcześniej. 

-- Seth -- powiedział,
twardszym głosem.

-- Co? -- Seth usiadł prosto, wyglądał walecznie. Metro w~Toronto, jak
większość, było miejscem obywatelskiego braku uwagi. Dużo trzeba było
zrobić, żeby inni ludzie jawnie Cię zauważyli. Seth to zrobił. Ludzie
się gapili.

Natalie pochyliła się i~przysłonił ucho Seth dłonią, coś syknęła.
Zacisnął usta i~spojrzał na nią, potem spojrzał na stopy. Uśmiechnęła
się półgębkiem do Huberta, Etcetera.

-- Gdzie jedziemy? -- spytała. To ,,my'' ucieszyło Huberta, Etcetera. Byli
towarzyszami broni przez noc, miał jej szczegóły kontaktowe, ale
częściowo oczekiwał, że powie, że jedzie do domu i~zostawia go z~Sethem.

-- Fran? -- spytał.

Zrobiła minę.

-- No weź -- powiedział. -- Jest całodobowe, jest ciepło, nie wyrzucą
Cię\ldots 

-- Dobra -- powiedziała. -- Choć to kanał.

Wzruszył ramionami. Pamiętał, kiedy ostatnia Fran została zamknięta,
kiedy był nastolatkiem, a~potem sieciówka ponownie została uruchomiona
jako biznes hobbystyczny dla drobniejszych Westonów, pośród fanfar o~rodzinie i~jej powiązaniach z~instytucjami Miasta. Nowa Fran wydawała
się nawiedzona, a~uczucie było, ironicznie, najintensywniejsze podczas
specjalnych wydarzeń z~żywymi kelnerami, zamiast automatów. Żywi ludzie
niosący tace jedzenia podkreślali fakt, że restauracja została
zaprojektowana dla głupich robotów na wolnym wybiegu i~minimum ludzkiego
nadzoru. Jednak była tania i~można było siedzieć tam dłuższy czas.

Żałował, że nie zasugerował czegoś fajnego. Kiedy dbał o~takie rzeczy,
miał długą listę miejsc, gdzie poszedłby, gdyby miał pieniądze i~kogoś
do towarzystwa. Seth też miał taką listę w~głowie, ale Hubert nie chciał
rozmawiać z~Sethem. Marzył, żeby Seth zgłosił się na ochotnika do
pojechania do domu i~odespania traumy i~pozostałości narkotyku. Co nie
miało się zdarzyć, ponieważ to był Seth.

-- Dobrze -- powiedziała.

Jej oczy zaszkliły się i~spojrzała na kolana, złożyła ręce nad
przestrzenią interakcji na udach, sprawdzając wiadomości. To
przypomniało Hubertowi, Etcetera, żeby odpalić, jego powierzchnia
interfejsu zabrzęczała, dając mu znać o~rzeczach, które powinien robić.
Odrobaczył skrzynkę odbiorczą, wyrzucając śmieci i~spamu. Odłożył
wiadomości, żeby przypomniały się później, coś od rodziców, dawnej
dziewczyny, jakiejś pracy, której szukał u dostawcy.

Byli prawie na stacji St Clair, a~gdy wstali, jeden z~porannych
pracowników wszedł w~przestrzeń Setha. Był wielkim facetem, jasna
karnacja, piegi z~wielkim, dziobatym nosem i~konserwatywną fryzurą pod
kołnierzyk. Miał na sobie tani płaszcz na jakiegoś rodzaju mundurze,
może medycznym. 

-- Ty -- powiedział, pochylając się -- jesteś pyskatym,
małym chujkiem, jak na kogoś, kto bierze zasiłek i~bawi się całą noc.
Dlaczego kurwa nie znajdziesz pracy?

Seth odchylił się, ale facet za nim podążył, wszyscy kołysali się do
ruchu zwalniającego pociągu. Nadnercza Huberta, Etcetera odnalazły
niespodziewany rezerwuar i~wycisnęły. Jego serce zagrzmiało. Ktoś zaraz
zostanie uderzony. Facet był wielki, pachniał mydłem. Były kamery przy
ludziach i~w pociągu, ale nie wyglądał, jakby o~to dbał.

Natalie położyła dłoń na piersi faceta i~pchnęła. Popatrzył w~dół
zaskoczony na chudą, kobiecą dłoń na jego piersi, zacisnął wielką dłoń
na jej nadgarstku. Machnęła wolną dłonią, uderzając go w~pierś torebką,
która się otworzyła i~wylała zimne wymiociny na niego. Spojrzała
zniesmaczona tak jak on, ale kiedy ją puścił i~cofnął się, skoczyła
przez zamykające się drzwi metra, Hubert, Etcetera i~Seth tuż za nią.
Odwrócili się, żeby zobaczyć, jak facet wącha dłoń z~niedowierzaniem,
jego język ciała sygnalizujący \textit{Nie wierzę, że wylałaś na mnie całą torbę
rzygów\ldots }

-- Natalie -- powiedział Seth na schodach ruchomych, inni pasażerowie,
którzy wyszli, obchodzili ich z~dala. -- Dlaczego Twoja torebka była
pełna rzygów?

Potrząsnęła głową. 

-- Zapomniałam o~tym. Zrobiło mi się źle, kiedy
zobaczyłam\ldots  -- Zamknęła oczy. -- Kiedy zobaczyłam Billiama.

-- Też o~tym zapomniałem -- powiedział Hubert, Etcetera.

-- Mam nadzieję, że nic ważnego nie wypadło, kiedy uderzyłam tego faceta
-- powiedziała. 

Jej torebka -- średniej wielkości z~abstrakcyjnym wzorem
nadrukowanym na zewnętrznym winylu -- była przerzucona przez ramię.
Raźnie otworzyła ją, zrobiła minę, zajrzała do jej obrzydliwych głębi. 

-- Nie wiem, nawet jak zacząć czyścić tę rzecz. Wyrzuciłabym ją, ale
niektóre rzeczy powinny dać się umyć.

Seth zmarszczył nos. 

-- Rękawiczki i~maska. I~zlew u kogoś obcego.
Człowieku, co Ty \textit{jadłaś}?

Spojrzała na niego, ale delikatny uśmiech grał na jej ustach. 

-- Ale jest
przydatna, co? Steve, mieliśmy gównianą noc. Sądzisz, że mógłbyś
zachowywać się dyskretnie? Nie wszczynać bójek?

Seth miał na tyle przyzwoitości, żeby wyglądać na zawstydzonego. Hubert,
Etcetera poczuł przypływ zazdrości, chciał zepchnąć Setha po schodach. 

-- Żadne z~nas nie jest w~dobrym stanie -- powiedział. Jedzenie pomoże. I~coffium.

Seth i~Natalie podskoczyli na wspomnienie coffium. 

-- Taaak -- powiedziała
Natalie. -- Chodźcie. -- Wbiegła po schodach, dwa duże stopnie na raz.
Przeszli przez bramkę, wyszli na oślepiająco jasny poranek, ruchliwy od
ludzi robiących sobotnio \dywiz poranne zakupy w~otwartych sklepach. Odbudowana
Fran miała wąskie szyby z~przodu, stała pomiędzy salonem remodelera
łazienek i~miejscem, które sprzedawało potężne betonowe rzeźby.

-- Pamiętacie neon Fran? -- spytał Hubert, Etcetera. -- To był taki
niesamowity kolor, dzika czerwień. -- Wskazał na rurę podświetloną
LED-ami. -- Nigdy mi to nie wyglądało dobrze. Sprawia, że chcę zmienić
ustawienia gamma w~rzeczywistości.

Natalie rzuciła mu zabawne spojrzenie. Znaleźli kabinę, stół
podświetlający menu, gdy siadali. Na menu przed każdym z~nich pojawiły
się dymki, gdy biometryka automatu rozpoznała ich i~podświetliła ich
ostatnie zamówienia, witając ponownie. Hubert, Etcetera zauważył, że
ostatnim razem Natalie zamówiła lazanię z~podwójnym chlebem czosnkowym,
oraz że minęło cztery lata od czasu ostatniego jej zamówienia. 

-- Nie
jadasz tutaj często?

-- Tylko raz -- odpowiedziała. -- Dzień otwarcia. -- Stukała w~menu przez
chwilę, zamawiając podwójny napój czekoladowy, wołowy kotlet, ziemniaki
pieczone, dodatkowy keczup i~majonez, oraz pół grejpfruta z~brązowym
cukrem. 

-- Byłam gościem Westonów. To była rodzinna sprawa. -- Spojrzała
mu prosto w~oczy, wyzywając go, żeby zrobił aferę z~jej przywileju. -- Ten znak neonowy? Mój tata go kupił. Wisi w~naszym domku wiejskim w~Muskokas.

Hubert, Etcetera zachował spokój. 

-- Chciałbym go kiedyś zobaczyć -- powiedział, równo. Poczekał na Seth, żeby coś powiedział.

-- Nazywam się Seth, nie Steve. -- Szeroki uśmiech był niewątpliwy.
Sięgnął przez stół i~pomanipulował zamówieniem Natalie, ściągając kopię
na swoje miejsce.

-- Co do cholery. -- Hubert, Etcetera złapał zamówienie Setha i~skopiował
je do swojego menu. Stuknął wielki dzbanek coffium i~Natalie uderzyła
dłonią na ,,akceptuj''.

-- Dawaj -- powiedziała Natalie. -- Powiedz to.

-- Nic nie mówię -- powiedział Hubert, Etcetera. -- Twoja rodzina zna
Westonów.

-- Dobra -- powiedziała. -- Znamy. Jesteśmy BDRO.

Hubert, Etcetera skinął głową, jakby wiedział, co to znaczy, ale Seth
nie miał wstydu. 

-- Co to BDRO?

-- Bardzo Dobra Rodzina Ontario -- powiedziała.

-- Nigdy nie słyszałem tego terminu -- powiedział Seth.

-- Ani ja.

Wzruszyła ramionami. 

-- Prawdopodobnie musisz być BDRO, żeby wiedzieć, co
to BDRO. Miałam tego dużo na koloniach letnich.

Wtedy dotarło jedzenie, na szczycie toczącego się robota, który
zadokował przy stole. Zdjęli górną warstwę, wtedy obrócił karuzelę dla
kolejnej tacy, potem trzeciej. Czwarta miała coffium. Natalie postawiła
ją na stole, a~Hubert, Etcetera nie mógł nie podziwiać jej mięśni
ramion, gdy opuszczała dzbanek. Zauważył, że nie goliła pach i~poczuł
się niezrozumiale intymnie dotknięty tą wiedzą. Rozłożyli talerze i~nalali kawy.

Zerwał nuklearnie czerwoną wisienkę ze szczytu bitej śmietany szejka i~zjadł ją całą, razem z~łodygą. Natalie zrobiła to samo. Seth poparzył
język na coffium i~rozlał zimną wodę w~pośpiechu.

Natalie użyła krawędzi talerza jako palety i~zmieszała razem beżową
miksturę z~keczupu i~majonezu. Jadła drobne kęsy jedzenia, pokryte
miksturą.

-- To wygląda \textit{brzydko} -- powiedział Seth, nie on, ponieważ nie
chciał wyjść na dupka. Seth był przenośnym, zewnętrznym id. Nie zawsze
komfortowym i~stosownym, ale niemniej jednak poręcznym.

-- To jest nazywane brązowa miłość. -- Otarła usta czerwono \dywiz białą
serwetką, czekając na Seth, żeby coś zasugerował, co się nie pojawiło. -- Wymyślone w~średniej szkole. Nie chcesz próbować, Twoja strata. 

Nabrała na widelec kawałek ziemniaka i~wskazała nim na nich. Pod wpływem
impulsu, Hubert, Etcetera pozwolił jej na nakarmienie go. Było to
zadziwiająco dobre, a~brzęk widelca na jego zębie sprawił, że zadrżał,
jak po niesamowitym sikaniu.

-- Fantastyczne. -- Naprawdę tak myślał. Przygotował własny rozmaz,
korzystając z~Natalie jako odniesienia kolorystycznego.

Seth odmówił spróbowania, ku skrytej radości Huberta, Etcetera. Jedzenie
było lepsze, niż pamiętał, oraz droższe. Nie miał budżetu na posiłek i~to go zaboli.

Rozmyślał na tym, stojąc przy urynale i~wąchając jego szparagowo-aktywne
siki. Myśląc o~pieniądzach, wąchając, prawie zacisnął i~pobiegł, żeby
znaleźć kubek, żeby zachować trochę na później. Darmowe piwo było
darmowym piwem, nawet jeżeli zaczynało, jako używane piwo. Każda woda
była używanym piwem. Ale było to już w~odpływie, zanim myśl się
wykrystalizowała.

Kiedy wrócił do stołu, przy Natalie siedział starszy mężczyzna.

Miał kosmate włosy, dobrze przycięte i~cerę o~blasku dobrej skóry. Nosił
dzianinowy sweter koloru cementowego, z~marmurkowymi rogowymi guzikami
przyszytymi bardzo różową nicią. Pod tym, ciasny czarny t-shirt
odsłaniał muskularną klatkę i~płaski brzuch. Miał prostą obrączkę
ślubną, czyste, krótkie, równe paznokcie, w~stylu ostentacyjnego
nie-manikiuru.

-- Cześć -- powiedział. Hubert, Etcetera usiadł naprzeciwko niego. Tamten
wyciągnął dłoń. -- Jestem Jacob. Ojciec Natalie.

Wymienili uścisk. 

-- Jestem Hubert -- powiedział, gdy Seth powiedział -- Nazywaj go ,,Etcetera''.

-- Mów mi Hubert -- powiedział, znowu. Jego zewnętrzne id było wrzodem na
tyłku.

-- Miło mi Cię poznać, Hubert.

-- Mój ojciec mnie szpieguje -- powiedziała Natalie. -- To dlatego tutaj
jest.

Jacob wzruszył ramionami. 

-- Mogło być gorzej. To nie tak, że mam
podsłuch w~Twoim telefonie. To tylko źródła publiczne.

Natalie odłożyła widelec i~odepchnęła talerz. 

-- Kupuje przekaz z~kamer,
raporty kredytowe w~czasie rzeczywistym, analizy rynku. Jak sprawdzanie
przeszłości nowego pracownika. Ale cały czas.

-- To potworne -- powiedział Seth. -- I~drogie.

-- Nie takie drogie. Stać mnie na to.

-- Tata przeniósł się do starych bogaczy -- powiedziała Natalie. -- Pieniądze go nie zawstydzają. Nie tak jak moich dziadków. Wie, że
praktycznie jest członkiem innego gatunku i~nie rozumie, dlaczego miałby
to ukrywać.

-- Moja córka gra w~grę próbując mnie zawstydzić publicznie, coś, nad
czym pracowała od dziesiątego roku życia. Łatwo mnie nie zawstydzić.

-- Dlaczego miałbyś być zażenowany? Musiałbyś troszczyć się o~to, co inni
ludzie myślą o~Tobie, żeby być zażenowanym. Nie dbasz o~to, więc nie
jesteś.

Hubert, Etcetera poczuł zażenowanie, poczuł, jakby powinien coś
powiedzieć, żeby tylko Seth nie uczestniczył w~całej tej wspólnocie
myśli. 

-- Założę się, że troszczy się o~to, co myślisz o~nim -- zaryzykował.

Oboje się uśmiechnęli, podobieństwo rodzinne było niesamowite, aż do
identycznego dołka po prawej stronie. 

-- Dlatego to robię. Jestem
pośredniczką dla każdego człowieka poniżej jego uwagi. To też nie
zabawa, pomimo tego, co myśli.

-- Nie widzę, żebyś odrzucała przywileje, Natty -- powiedział, obejmując
ją ramieniem. Pozwoliła mu na to przez odmierzony moment, potem
odtrąciła jego ramię.

-- Jeszcze nie -- powiedziała.

Jego cisza była elokwentnie sceptyczna. Przesunął jej talerz na swoje
miejsce, wcisnął na stole wiadomość ,,BRAK MIEJSCA'' i~pomachał
kontaktem na rękawie nad tym, wstukał wzór kciukiem i~palcem
wskazującym. Wyczyścił do końca jedzenie, potem sięgnął po szejka.
Zatrzymała go i~powiedziała: 

-- Moje. Sięgnął po fusy jej coffium.

-- Zamierzasz zaprosić swoich przyjaciół na zabawę, co? -- Wytarł usta i~załadował talerze na robota, który odłączył się od stołu.

-- Chłopcy, chcecie prysznic?

Seth uderzył w~stół, sprawiając, że menu zatańczyło, gdy próbował
zinterpretować jego instrukcje. 

-- Chodź, bracie, dzisiaj ucztujemy!

Hubert, Etcetera dał mu kuksańca. 

-- Lepiej policz łyżeczki -- powiedział.

-- Same się liczą -- powiedział Jacob. Coś zrobił z~rękawem i~dodał -- Samochód będzie w~pobliżu za chwilę.

\chapter*{iii}

Oczywiście to nie był samochód carshare. Redwater było jednym z~wielkich
nazwisk, był burmistrz Redwater, posłowie do parlamentu Redwater,
minister finansów Redwater, wielu dyrektorów Redwater. Samochód był
jednak mały, ale był nieokreślenie solidny, z~fartuchem z~matowej gumy,
który osłaniał koła. Hubert, Etcetera pomyślał, że musi być coś
interesującego pod nim. Było wiele takich intrygujących rzeczy w~tym
samochodzie, a~niepozorne logo Longines widniało w~rogu szyby.
Zawieszenie zrobiło coś bystrego, żeby skompensować jego wagę, aktywnie
mięknąc, nie jak sprężyny z~epoki kamienia łupanego. Usiadł w~siedzisku
skierowanym do tyłu, zobaczył, że okna nie były oknami. Były grubym
pancerzem, pokrytym ekranami o~wysokiej rozdzielczości. Jacob zajął
drugie siedzisko i~powiedział: 

-- Dom. 

Samochód zaczekał, aż wszyscy
usiedli bezpiecznie i~zapięli pasy, zanim ruszył. Z~punktu widzenia
Huberta, samochody dookoła nich spływały z~ich drogi.

-- Nie sądzę, żebym kiedykolwiek poruszał się tak szybko w~mieście -- powiedział.

Jacob mrugnął do niego po ojcowsku.

Natalie sięgnęła przez wielki wewnętrzny przedział i~stuknęła tatę w~udo. 

-- On się popisuje. W~tych jest firmware na zamówienie, pozwala
zmniejszyć o~połowę minimalną odległość, co sprawia, że pozostałe
samochody się wycofują, ponieważ jedziemy, jak nieprzewidywalne dupki.

-- Czy to legalne? -- spytał Hubert, Etcetera.

-- To jest wykroczenie cywilne -- powiedział Jacob. -- Grzywny są płacone
poleceniem zapłaty.

-- Co jeżeli kogoś zabijesz? -- Seth przeszedł do sedna.

-- To jest kwestia kryminalna, poważniejsza. Jednak się nie zdarzy. W~prognozowaniu samochodu odbywa się wiele z~teorii gier, modelujących
prawdopodobne odrzutków, dezerterów i~dodając wielki margines
bezpieczeństwa. Naprawdę, rozgrywamy to bezpieczniej niż fabryczny
firmware, ale tylko dlatego, że sam samochód ma znacznie lepsze
charakterystyki hamowania, przyśpieszania i~prowadzenia niż samochód
fabryczny.

-- Oraz dlatego, że odstraszasz systemy innych samochodów od wjeżdżania
Ci w~drogę -- powiedział Seth.

-- Racja -- powiedziała Natalie, zanim jej ojciec mógł zaprotestować.
Wzruszył ramionami, Hubert, Etcetera zapamiętał, co mówiła o~jego byciu
,,starym bogaczem'', obojętnym wobec idei, że ktokolwiek mógłby oburzyć
się na jego kupowanie przejazdu przez korki.

Ścigali się ulicami miasta. Natalie zamknęła oczy i~oparła się. Pod jej
oczami były czarne kręgi, była spięta, była taka od momentu pojawienia
się jej ojca. Hubert, Etcetera próbował się nie gapić.

-- Gdzie mieszkacie? -- spytał Seth.

-- Eglinton Ravine, koło Parkway -- odpowiedział Jacob. -- Zbudowałem to
około dziesięciu lat temu.

Hubert, Etcetera pamiętał szkolne wycieczki do Ontario Science Centre,
próbował sobie przypomnieć wąwóz, ale mógł tylko wspomnieć mocno
zalesioną strefę, która mignęła przez okno pędzącego autobusu szkolnego.

Jedzenie, które zjadł we Fran, ważyło w~żołądku jak kula armatnia.
Pomyślał o~krwi na jego ubraniach, pod paznokciami, błocie na butach,
sypiącego się na tapicerkę. Samochód szarpnął, jego jelita się
poskarżyły. Zahamowali mocno, potem włączyli się w~kolejny pas ruchu,
szept pomiędzy nimi a~samochodem za nim, małym wspólnym samochodzie,
którego pasażerka, elegancka arabsko \dywiz wyglądająca kobieta w~biurowym
makijażu, spojrzała na nich z~trwogą, zanim przeskoczyli na kolejny pas.

\chapter*{iv}

Dom był jednym z~trzech w~rzędzie, na skraju wąwozu, na końcu krętej
drogi z~koleinami pod drzewami liściastymi. Drzwi garażu rozsunęły się,
gdy wjechali w~dom najbardziej na prawo. Zamknęły się, zamknęły się
wielkimi, błyszczącymi, okrągłymi prętami, które głęboko wsunęły się w~podłogę, sufit i~ściany. Drzwi samochodu sapnęły przy otwieraniu i~Hubert znalazł się w~ogromnej przestrzeni pod trzema domami, jasno
oświetlonej, usianej pojazdami. Jacob wysunął dłoń do Natalie, ona go
zignorowała, potem nieco się potknęła, gdy wykręciła się, żeby uniknąć
jego podtrzymania za łokieć.

-- Chodźcie -- powiedziała do Huberta, Etcetera i~Seth, ruszyła na drugi
koniec garażu.

-- Dzięki za podwiezienie -- zawołał Hubert, Etcetera, gdy szybko szedł za
nią. Jacob oparł się o~samochód, obserwując, jak idą. Hubert, Etcetera
nie rozumiał jego miny.

Zabrała ich na górę wąskimi schodami do wielkiego, nieposprzątanego
pokoju z~sofami i~panoramicznym oknem z~widokiem na wąwóz, zielony,
stromy spadek do rzeki Don poniżej, białej i~spienionej, gdy spływała do
jeziora Ontario. Pachniało starym praniem, niepozmywanymi talerzami,
ukrytymi za świecami zapachowymi. Jedna ściana była wybazgrana od
podłogi do sufitu i~popisana flamastrami, brokatem i~długopisami.

-- Skrzydło dzieciaków -- powiedziała. -- Moja siostra jest na
uniwersytecie w~Rio, więc to jest teraz moje. Nie sądzę, żeby moi
rodzice byli tutaj częściej niż z~pięć razy, od kiedy dom został
wybudowany.

-- Cały układ to jeden dom? -- spytał Hubert, Etcetera.

-- Tak -- powiedziała.

-- Nie wydaje się to sposób, w~jaki budujesz domy, jeżeli nie dbasz, jak
bogaty wydasz się innym osobom -- powiedział.

Potrząsnęła głową. 

-- To był sprawa urbanistyczna. Ludzie po drugiej
stronie wąwozu -- wskazała na okno widokowe -- nie chcieli patrzeć na
,,potwór\dywiz rezydencję'' przy śniadaniu. Są bogaci, my jesteśmy bogaci,
komisja do spraw urbanistyki nie wiedziała, co robić. Tata załatwił
zbudowanie wielkiego domu, który wyglądał jak trzy domy. -- Strzepnęła z~sofy luźny bałagan. -- Jedzenie jest w~spiżarni. Skorzystam z~łazienki na
piętrze. Ta na parterze jest tam, sami się obsłużcie z~kosmetykami. 

Poszła po schodach i~zniknęła za rogiem.

Seth wyszczerzył się znacząco do Huberta, Etcetera, cichy komentarz na
temat jego romantycznych uczuć do Natalie. Hubert nie był w~nastroju.
Trzymał zmarłego w~ramionach. Był zakrwawiony, zmęczony i~miał mdłości.

-- Zamierzam stać pod prysznicem przez godzinę -- powiedział. -- Zatem
lepiej idź pierwszy.

-- Skąd wiesz, że nie będę brał prysznica przez godzinę? -- Irytujący
uśmieszek Setha.

-- Nie bierz. -- Znalazł ręczniki na ziemi przy kamiennym kominku. Podał
jeden Sethowi, trzepnął drugi i~położył go na gzymsie.

Były tam gazety, podpałki i~kłody. Rozpalił ogień. Znalazł wielki
podkoszulek, który nie pachniał źle, z~udawanymi wypalonymi dziurami,
parę spodni dresowych, o~których pomyślał, że mogły pasować. Zdjął swoją
koszulę, spodnie, kurtkę i~wrzucił je do ognia. Nie wiedział, czy
kryminalistyka mogłaby zidentyfikować krew na ubraniach po praniu, ale
był pewien, że jeszcze mniej zrobią z~popiołem. Tkane powierzchnie
interfejsu stopiły się i~wypuściły gryzący dym. Poklepał po obcych
ubraniach, zastanawiając się, do kogo należały. Może do Billiama.

Natalie wyszła zza rogu i~stała na podeście, zastanawiając się nad nim i~bałaganem. 

-- Steve w~łazience?

-- Seth. Tak.

-- Możesz użyć mojej, chodź. -- Tak po prostu, był w~sypialni obcej
dziewczyny.

Była to sypialnia kogoś, kto do niedawna był studentem: certyfikaty w~ramkach, półki pełne książek i~nagród, przyklejone plakaty zespołów i~powodów, ale pokryte plakatami politycznymi, biurko pełne zepsutych
powierzchni interfejsów, wyszukane domowe wapery, które mogły zamienić
tytan we wziewny dym. Rozrzucone pieniądze papierowe, świadczące o~nielegalnych transakcjach, niezgrabne, na wpół działające okablowanie na
ścianach, suficie i~podłodze, dziecięca próba zablokowania
rodzicielskiego oprogramowania szpiegującego. Był to lepszy opsec niż
praktykował Hubert, Etcetera, ale nie był pewien, czy działał.

Natalie miała na sobie luźną piżamę, w~czarno-białe paski, bez stanika,
a on nie się nie gapił, nawet nie zerkał. Przesunęła dłoniami po
krawędzi drzwi do łazienki, miejsce poplamione odciskami dłoni poza
zawsze czystą przestrzeń i~drzwi się otworzyły. 

-- Cała Twoja.

Przeszedł przez drzwi i~odwrócił się, żeby zamknąć. Patrzyła się na
niego. 

-- Zatrzymaj rzeczy -- powiedziała. W~jej oczach były łzy.

-- Ja\ldots  -- Zawahał się. -- Bardzo mi przykro z~powodu Billiama.

-- Mnie też. -- Łza spłynęła po jej policzku. -- Był dupkiem, ale był
naszym dupkiem. Najebywał się za szybko na imprezach. To była jego wina.
Tęsknię za nim. -- Kolejna łza.

-- Chcesz się przytulić?

-- Nie, dziękuję. Idź pod prysznic.

Łazienka była tego rodzaju, które widuje się na wystawach. Aktywne
tłumienie dźwięku pożerało dźwięk wody, inteligentne algorytmy prysznica
zwiększały lub zmniejszały ciśnienie, przewidując, co chciał spłukać i~jak mocno. Interaktywne powierzchnie zamieniały wszystko w~lustro po
podwójnym stuknięciu, dając mu niepokojący obraz jego dupy i~tyłu głowy.
Cyrkulatory powietrza obmywały go ciepłymi podmuchami, gdy wyłączył
wodę, jednocześnie susząc płaszczyzny łazienki z~kondensatu.

-- Przepraszam -- powiedziała. Jej oczy były suche. Wyciągnął ręcznik i~zrobił pytającą minę. Wzięła go od niego i~rzuciła na podłogę.

-- Zobaczmy, co robi Steve.

-- Seth.

-- Kogo to obchodzi.

Seth odnalazł spiżarnię i~posprzątał na stoliku do kawy, wykonał niezłą
robotę, składając, organizując rzeczy, gromadząc je na stosie na czystym
kawałku podłogi. Przygotował trzy krzesła. Na stole, taca owoców,
czajnik herbaty i~kubki, croissanty. Pachniały dobrze.

-- Przekąska?

-- Dobra robota, Steve. -- Natalie brzmiała, jakby naprawdę tak myślała.

-- Nie ma sprawy -- Seth jej nie poprawił.

Jedli w~milczeniu. Hubert, Etcetera chciał zapytać o~dom, o~jedzenie. O
Billiama, imprezę, trzecią osobę z~brodą, tę drugą dziewczynę, która
była ich wspólnikiem w~zbrodni. Jednak senność wzbierała się w~ciele.
Jego powieki opadały. Natalie przeniosła wzrok z~niego na Setha, który
również wyglądał, jakby mógł przysnąć na krześle, i~powiedziała: 

-- Ok, chłopcy, idźcie spać na sofach. Ja idę do łóżka.

Zachwiała się na schodach, Hubert, Etcetera wyciągnął się na najmniej
zagraconej sofie, zamykając oczy, gdy przycisnął twarz do szwu poduszki.
W krótkiej chwili przed snem, zobaczył wykrzywione ciało Billiama,
poczuł fantomowe wrażenie miazgi czaszki Billiama na palcach. Zadrżał od
stóp do głowy, dwa razy, zanim wspomnienie ustąpiło i~łaskawie zasnął.

\threeast

Obudził się do mamrotanych głosów. Rozejrzał się niewyraźnie, próbując
się zorientować: plecy Setha na sofie naprzeciwko niego, ściana
pomalowana palcami. Podniósł głowę -- pulsowanie kaca -- i~zlokalizował
głosy. Natalie, stojąc w~drzwiach na drugim końcu, kłóciła się szeptem
przez rysę w~drzwiach. Odpowiedź była męska, starsza, irytująco
spokojna. Jacob. Jego głowa się osunęła. Musiał wstać. Jego pęcherz był
boleśnie pełny.

Tak niezręcznie jak nigdy w~życiu, ubierając się nieznajome ubrania, na
kacu, w~dziwnym pokoju, gdzie dziwna -- atrakcyjna -- dziewczyna kłóciła
się z~jej bogatym ojcem, przedreptał tak skromnie, jak potrafił, do
toalety. Natalie spojrzała za nim, zrobiła niezrozumiałą minę, wróciła
do dyskusji.

Kiedy Hubert, Etcetera wrócił, wycierając ręce o~dół pożyczonego dresu,
Natalie i~jej ojciec siedzieli z~nieruchomymi minami naprzeciwko siebie.
Jacob siedział na sofie, którą opuścił Hubert, Etcetera, a~ona na
krześle. Seth spał.

Hubert, Etcetera poszedł do spiżarni -- miękkie światła w~środku włączyły
się i~zobaczył drzwi po drugiej stronie, zrozumiał, że jakiś
\textit{służący} uzupełnił ją w~ciągu dnia -- wyniósł paluszki z~marchewki,
seler i~hummus na tacy, postawił pomiędzy obojgiem Redwater. Patrzyli
się na siebie.

-- Dziękuję, Hubert -- powiedział Jacob Redwater. Zanurzył marchewkę w~hummusie, nie zjadł.

Hubert, Etcetera usiadł koło niego, ponieważ nie było innego miejsca.

-- Hubert, co jest ważniejsze, prawa człowieka czy prawa własności? -- spytała Natalie.

Hubert, Etcetera pomyślał przez chwilę. To było trudne. 

-- Czy prawo
własności jest prawem człowieka?

Jacob uśmiechnął się, ugryzł marchewkę i~Hubert, Etcetera wyczuł, że
powiedział złą rzecz.

Natalie wyglądała srogo.

-- Powiedz mi. Ta fabryka, którą
uruchomiliśmy zeszłej nocy. Była warta więcej jako odpis niż jako
działający problem. Jakiś byt, który ją posiadał, zażądał, żeby stała
gnijąc bezużytecznie, mimo tego, że byli ludzie, którzy chcieli tego, co
mogła zrobić.

-- Jeżeli chcieli fabryki, mogliby kupić fabrykę -- powiedział Jacob. -- Potem robić rzeczy i~je sprzedawać.

-- Nie sądzę, żeby ci ludzie byli w~stanie kupić fabrykę -- powiedział
Hubert, Etcetera, spoglądając na Natalie za aprobatą. Skinęła lekko
głową.

-- Po to są rynki kapitałowe -- powiedział Jacob. -- Jeżeli masz plan na
zyskowne wykorzystanie aktywu, które ktoś inny nie używa, to tworzysz
biznesplan i~przedstawiasz go inwestorom. Jeżeli masz rację, jeden z~nich cię sfinansuje, może nawet więcej niż jeden. Wtedy sprzedajesz to,
co wyprodukujesz.

-- Co jeżeli nikt nie zainwestuje? -- spytał Hubert, Etcetera. -- Znam tonę
startupów zeppów, które umarły, ponieważ nikt nie mógł dostać pieniędzy,
chociaż robili niesamowite rzeczy.

Jacob nabrał powietrza jak ktoś wyjaśniający złożone zagadnienie
dziecku. 

-- Jeżeli nikt nie chce zainwestować, to znaczy, że nie masz
pomysłu wartego inwestycji, lub nie jesteś właściwą osobą, żeby
zrealizować ten pomysł, ponieważ nie wiesz, jak przekonać ludzi do
zainwestowania.

-- Nie dostrzegasz tutaj błędnego koła? -- spytała Natalie. -- Jeżeli nie
możesz przekonać kogoś do zapłacenia za włączenie fabryki, żeby robić
rzeczy, których ludzie potrzebują, zatem fabryka nie powinna być
włączona?

-- W~przeciwieństwie do czego? Wszystko za darmo? Po prostu rozwal drzwi,
wejdź i~zajmij?

-- Dlaczego nie, jeżeli nikt inny niczego z~tym nie robi?

Mina typu ,,mówię do dziecka'': 

-- Ponieważ to nie jest Twoje.

-- I~co z~tego?

-- Nie byłabyś szczęśliwa, gdyby motłoch włamał się tutaj i~wyniósł
wszystkie cenne rzeczy, prawda, Natty?

Przy doświadczeniu mniej niż jeden dzień, Hubert, Etcetera potrafił się
domyślić, że Natalie nie chciała być nazywana ,,Natty''. Jacob wiedział
o tym, podpuszczał córkę. To było oszustwo.

-- Nie miałbym nic przeciwko -- powiedział Hubert, Etcetera. -- Nie mam
dużo, większość z~tego, co ważne, ma kopię zapasową. Znaczy, tak długo,
jak potrafiłbym znaleźć łóżko i~jakieś ubrania następnego dnia, nie
zrobiłoby to różnicy.

-- Natty ma znacznie więcej niż zmiana ubrań i~łóżko tutaj w~swoim
gniazdku -- powiedział Jacob. -- Natty lubi ładne rzeczy.

-- To prawda. Chcę, żeby wszyscy inni też je mieli. -- Jej wzrok mógłby
przecinać stal.

-- Niech zapracują na nie, tak jak my zapracowaliśmy.

Natalie prychnęła.

Jacob spojrzał na Huberta, Etcetera. 

-- Byłeś na imprezie zeszłej nocy?

Na zewnątrz okna widokowego był zmierzch, różowo-pomarańczowe światło
omiatające wąwóz, plamiące falującą powierzchnię rzeki.

-- Byłem.

-- Co sądzisz o~włamywaniu się do prywatnej posesji i~kradzieży tego, co
tam znajdziesz?

Hubert, Etcetera żałował, że nie udawał, że śpi. Był prawie pewien, że
Seth tylko udawał.

-- Nikt tego nie używał. -- Spojrzał na Natalie. -- Ogniwa wodorowe się
wypełniły, więc wiatraki działały na próżno. Surowce były warte
praktycznie nic.

-- Jaki jest sens posiadania własności, jeżeli tylko gnije? -- spytała
Natalie.

-- Och, proszę. Własność prywatna jest najbardziej produktywną
własnością. Przejściowe nieefektywności tego nie zmienią. Tylko świry i~łajdacy sądzą, że kradzież własności drugiego jest legalną formą działań
politycznych.

-- Tylko kleptokraci używają pojęć jak ,,przejściowe nieefektywności''
dla rozrzutnych paskudztw jak ta fabryka Muji.

-- Łatwo mówić o~kleptokratach, kiedy tatuś pociąga za sznurki, żeby
gliny nie dobrały się do Twojej leniwej dupy. Dzisiaj, Natty,
zaaresztują cholernie dużo osób, ale nie Ciebie ani Twoich przyjaciół.

-- Nie udawaj, że Twoje polityczne zawstydzenie to hojność. Niech mnie
zabiorą.

-- Może tak zrobię. Może kilka lat ciężkiej pracy w~więzieniu sprawi, że
docenisz to, co masz.

Natalie spojrzała na Huberta, Etcetera. 

-- Grozi mi wysłaniem do
więzienia od dziesiątego roku życia. Kiedyś to były te przerażające
miejsca na prywatnych wyspach, póki wszystkie nie zostali zamknięte za
,,gwałt poprawczy''. Teraz to więzienia dla dorosłych. Dlaczego kurwa
nie, Tato? Jesteś głównym udziałowcem w~większości z~nich, daliby Ci
zniżkę. Mogłabym spojrzeć od środka na rodzinny biznes.

Jacob efektownie się roześmiał. 

-- Jakbym Ci zaufał w~prowadzeniu
czegokolwiek. Biznes to merytokracja, dziecko. Myślisz, że dostaniesz
jakąś ciepłą posadkę, ponieważ jesteś moim dzieciakiem\ldots 

-- Nie myślę. Ponieważ nie ma już żadnych ,,posad''. Tylko inżynieria
finansowa i~polityka. Nie jestem kompetentna w~żadnej z~nich. Przede
wszystkim, nie potrafię powiedzieć ,,merytokracja'' z~poważną miną.

Hubert, Etcetera zauważył, że ta uwaga dotarła. Rozzuchwaliło go to. 

-- To szczyt samonapędzającego się błędnego koła, prawda? ,,Jesteśmy
najlepszymi ludźmi na świecie, jesteśmy na szczycie, zatem mamy
merytokrację. Skąd wiemy, że jesteśmy najlepsi? Ponieważ jesteśmy na
szczycie. QED.'' Niesamowita sprawa o~,,merytokracji'' jest taka, że tak
wielu genialnych wielkich przemysłowców nie zauważyło, że to jest to tak
oczywista, radioaktywna bzdura, którą mógłbyś wykryć z~orbity. -- Spojrzał się szybko na Natalie. Delikatnie kiwnęła mu głową, co go
zachwyciło.

Jacob wyglądał na bardziej wkurzonego. W~tle, Hubert, Etcetera
zastanawiał się, jak taki potężny człowiek mógłby być tak przeczulony.
Jacob wstał i~się popatrzył. 

-- Łatwo mówić, ale ostatnim razem, jak
sprawdzałem, wy dwoje nie zrobiliście żadnej rzeczy, która miała dla
kogokolwiek znaczenie i~zależeliście na ,,bzdurze'', żeby nie trafić do
więzienia.

-- Znowu wyjeżdża z~tym więzieniem. Mniemam, że więzienie to jedyny
sposób, żeby wygrać dyskusję, jeżeli nie potrafisz wymyślić lepszych
argumentów.

-- To tradycja -- powiedział Seth, podnosząc się z~poduszek. -- Hiszpańska
inkwizycja. ZSRR. Arabia Saudyjska. Guantanamo.

Jacob wymaszerował, zamykając łączące drzwi z~dostojnym stuknięciem.
Było to bardziej wkurzone niż uderzenie. Hubert, Etcetera poczuł się
triumf.

-- Ten hotel jest cholernie głośny. -- Seth przetoczył się na plecy,
rozciągając się, żeby pokazać owłosiony żołądek, który stał się miękki
od ostatniego razu, kiedy go widział Hubert, Etcetera.

-- Ale obsługa hotelowa jest niesamowita -- powiedział Hubert, Etcetera. -- I~nie możesz przebić ceny.

Seth usiadł. 

-- To Twój tata, co?

-- Wiem, że to taki banał, żeby nienawidzić swojego starego, kiedy masz
dwadzieścia lat, ale on jest takim dupkiem -- powiedziała Natalie. -- Naprawdę wierzy w~merytokrację. Poważnie w~to wierzy. Jeden krok dzieli
go od gadania o~krwi królów w~żyłach.

-- Sprawa, której nigdy nie rozumiałem -- powiedział Hubert, Etcetera -- to, jak ktoś może mieć takie urojenia i~nadal zarządzać połową planety?
Rozumiem, jak pewne urojenia mogłyby być użyteczne, kiedy rządzisz
ludźmi i~zdzierasz ze wszystkich, ale czy to w~końcu się nie załamie?
Tam ciągle jest kapitalizm. Jeżeli Twój konkurent wprowadzi osobę bez
urojeń, czy ta osoba nie spowoduje Twojego bankructwa?

-- Istnieje więcej niż jeden sposób na inteligencję -- powiedziała
Natalie. -- Ludzie jak mój ojciec zakładają, że skoro są tacy inteligentni
w byciu złymi gnojkami, muszą być inteligentni we wszystkim\ldots 

-- A ponieważ są tacy inteligentni we wszystkim -- powiedział Seth -- to
jest w~porządku dla nich, żeby być złymi gnojkami?

-- Dokładnie -- powiedziała. -- Zatem ludzie jak mój tata są dobrzy w~wymyślaniu, jak wziąć Twoją firmę z~jej ,,inteligentnymi ludźmi'' i~ogłosić, że jest nielegalna, ukraść najlepsze pomysły, albo po prostu
kupić ją, zwiększyć jej wartość i~korzystać z~tego, aż nie robi niczego
prócz egzotycznych pochodnych i~kredytów podatkowych. A do tego, to nie
jest dla niego wystarczająco dobre! \textit{Chce} być tym jednym procentem
z jednego procenta z~jednego procenta dzięki swojej wewnętrznej cnocie,
a nie dlatego, że system jest ustawiony. Jego cała tożsamość opiera się
na idei, że ten system jest legalny i~że zdobył swoją pozycję w~nim
czysto i~uczciwie, a~wszyscy inni są płaksami.

-- Jeżeli nie chcieli być biedni, powinni się urodzić się bogaci -- powiedział Seth.

-- Bez urazy -- dodał Hubert, Etcetera.

-- Nie ma sprawy. -- Przebrała stos prania, znalazła bakłażanowy cardigan
z luźnej dzianiny, para skręconych majtek wystawała z~rękawa. Strzeliła
z nich w~kierunku schodów. 

-- Wiem, że moja rodzina jest bogatsza niż
Sknerus McKwacz, ale nie udaję, że mamy to przez cokolwiek innego niż
trochę szczęścia dawno temu i~wykorzystanie łapówek, korupcji i~niemoralności, żeby zbudować na tym farcie to tego tandetnego miejsca i~dziesiątki takich jak to.

-- A co z~zeszłą nocą? -- spytał Hubert, Etcetera, ośmielony jej
szczerością. -- Co z~tą imprezą i~w ogóle?

-- Co z~nią? -- spytała, jej ton figlarny i~wyzywający.

-- Co z~byciem bogatszym niż Sknerus McKwacz i~organizowaniem
komunistycznej imprezy?

-- Dlaczego nie powinnam?

-- To nie tak, że musisz\ldots 

-- Ale mogę. Pamiętaj, to nie tylko ,,każdej według jej potrzeb'', to
także ,,od każdej według jej zdolności''. Wiem, jak znaleźć fabryki,
które są idealne dla akcji bezpośredniej. Wiem, jak się do nich dostać.
Wiem, jak włamać się do ich maszyn. Wiem, jak urządzić niesamowitą
imprezę. Mam ten cały niezasłużony przywilej. Oprócz samobójstwa jako
wroga rodu ludzkiego możesz wymyślić dla mnie coś lepszego do roboty?

-- Mogłabyś dać pieniądze\ldots 

Zmroziła go spojrzeniem. 

-- Jeszcze się tego nie domyśliłeś? Dawanie
pieniędzy niczego nie rozwiązuje. Proszenie zettabogaczy, żeby
zrehabilitowali się poprzez rozdawanie pieniędzy, potwierdza, że na nie
zasługują, że powinni mieć władzę decydowania, gdzie trafią. To
udawanie, że możesz zostać bogaczem bez bycia bandytą. Umożliwienie im
decydowania, co zostaje sfinansowane, ogłasza, że planeta jest
gigantyczną korporacją, którą zarządzają główni akcjonariusze. Mówi się,
że rząd jest tylko kierownictwem średniego szczebla, zatrudnianym lub
zwalnianym według kaprysu dyrektorów.

-- Plus, jeżeli wierzysz w~to wszystko, nie musisz rozdawać wszystkich
pieniędzy. Nie wydawała się zdenerwowana.

-- Po co kurwa nam pieniądze? Tak długo, jak utrzymujemy złudzenie, że
pieniądze to coś więcej niż zbiorowa halucynacja wzbudzana przez
rządzące elity, żeby przekonać Cię, żeby pozwolił im zagarnąć najlepsze
rzeczy, nigdy niczego nie zmienisz. Steve, problem nie leży w~tym, że
ludzi źle wydają swoje pieniądze, lub że źli ludzie mają pieniądze.
Problemem \textit{są} pieniądze. Pieniądze działają tylko wtedy, kiedy nie
ma wystarczająco dużo dookoła, kiedy jesteś przekonany, że rzadkie
rzeczy są sprawiedliwie rozdzielane, ale to jest to samo merytokratyczne
błędne koło, które Etcetera zniszczył u mojego taty: rynki są
najsprawiedliwszym sposobem określenia, co każdy powinien dostać, rynki
stworzyły obecną, straszną dystrybucję, zatem obecna, straszna
dystrybucja jest najlepszym rozwiązaniem trudnego problemu.

-- Za każdym razem, kiedy słyszę kogoś mówiącego, że pieniądze to bzdura,
sprawdzam, ile ma pieniędzy. Bez urazy, Natty, ale jest znacznie łatwiej
mówić o~pieniądzach, że są bzdurą, kiedy je masz. -- Seth usiadł i~potarł
energicznie nogi. Suche błoto złuszczyło się z~dżinsów.

Prychnęła. 

-- Czy to wszystko, co masz? ,,Kawiorowa lewica''? Myślisz, że
fakt, że zostałam urodzona bogato, \textit{naprawdę} bogato, więcej
pieniędzy niż kiedykolwiek zobaczysz lub sobie wyobrazisz,
dyskwalifikuje mnie od posiadania opinii na ten temat?

Seth poszedł do spiżarni i~wyjął jedzenie, świeże owoce, napój z~mleczka
pszczelego, pizzę w~pudełku typu MRE, którego etykietę pociągnął. Cisza
się przedłużała. Hubert, Etcetera miał właśnie coś powiedzieć, kiedy
Seth powiedział: 

-- Poznałem wielu policjantów z~bzdurnymi teoriami o~przestępczości i~ludzkiej naturze. Generałowie na pewno mają gówniane
opinie o~wadze kończenia ludzkiego życia. Każdy ksiądz, rabin czy imam
wydaje się wiedzieć dużo o~niewidzialnym, wszechmocnym bycie, który
wydaje się bajką. Zatem tak, posiadania majątku prawdopodobnie
dyskwalifikuje Cię od znajomości jednej, jebanej rzeczy o~nich. -- Otworzył pudełko pizzy, odsuwając się od wznoszącej się pary. 

-- Kawałek?
-- spytał, zapach czosnku, pomidorów, kukurydzy i~sardeli zawirował jak
wir pyłowy z~oregano.

Hubert, Etcetera skulił się przed wybuchem Natalie. Seth był mistrzem
prowokacji. Jednak wybuch nie nadszedł.

-- To nie jest do końca głupie. Powiedzmy, że mamy różne perspektywy na
pieniądze. Powiedz mi, Steve, myślisz, że mógłbyś upowszechnić swoją
metodę na lepszy świat?

-- Cholera, gdybym wiedział.

Hubert, Etcetera wziął pudełko pizzy i~zjadł kawałek. Była dobra jak na
błyskawiczną rację. Sos był pikantny i~korzenny, mógł być uzależniający
jak crack. Kiedy zrozumiał, że było tutaj tak wiele pizzy, ile mógł
zjeść wyłaniających się ,,in potentia'' w~posiadłości Redwater, wziął
jeszcze dwa kawałki.

-- Jestem podejrzliwy wobec wszystkich planów naprawy niesprawiedliwości,
które zaczynają się od ,,krok pierwszy zdemontuj cały system i~zastąp go
lepszym'', szczególnie gdy nic nie możesz zrobić, póki krok pierwszy
nie zostanie zakończony. Z~wszystkich sposobów, którymi ludzie oszukują
się, żeby nic nie robić, ten jest najbardziej egoistyczny.

-- Co z~Odchodzącymi? -- spytał Hubert, Etcetera. -- Wydaje mi się, że
robią coś, co robi różnicę. Żadnych pieniędzy, żadnego udawania, że
pieniądze coś znaczą, no i~robią to właśnie teraz.

Natalie i~Seth spojrzeli na niego. Hubert skończył trzeci kawałek. 

-- Są
dziwni i~podejrzani, ale to wynika z~tego mówienia o~zniszczeniu całego
świata, jaki znamy i~zbudowania nowego na jego miejsce.

-- On żartuje, prawda? -- spytała Natalie.

-- Cholera, jeżeli wiem -- powiedział Seth. -- On jest dziwny. Etcetera,
żartujesz, prawda?

Hubert, Etcetera rozgrzał się od bycia w~centrum uwagi. 

-- Jestem
kompletnie poważny. Słuchajcie, też słyszałem historie, nie wiem, czy są
prawdziwe, a~jeżeli wy oboje jesteście poważni o~tym całym ,,zmieńmy
świat'', nie sądzę, żebyście mogli udawać, że nie istnieje kilka
milionów dziwaków, którzy mają taki sam cel, tylko z~tego powodu, że nie
podoba wam się ich styl życia. To nie tak, że samopodgrzewająca się
pizza jest wrodzoną ludzką instytucją, z~której korzystaliśmy jako
gatunek przez tysiące lat.

-- Co proponujesz?

-- Nie proponuję, dokładnie. Jednak jeżeli chcecie, moglibyście mieć całe
potrzebne info, żeby zostać odchodnikami w~ciągu około dziesięciu minut,
być na drodze jutro, żyć jakby to były pierwsze dni lepszego narodu, lub
dziwniejszego.

Natalie długo patrzyła na ciemniejące niebo. 

-- Billiam zwykł żartować z~odchodników. Zawsze było kilkoro, którzy pojawiliby się na imprezach
komunistycznych następnego dnia i~poprawiło to i~tamto, żeby wszystko
działało lepiej. Nie rozmawiali w~ogóle z~nami, nawet nie patrzyli w~oczy, ale zawsze zostawiali rzeczy działające lepiej niż przed ich
przybyciem. Billiam powiedział, że wszyscy skończymy jako odchodnicy.

-- Był dla Ciebie dobrym przyjacielem, co? -- Hubert, Etcetera poczuł się
głupio.

-- Znałam go z~przerwami od trzech lat. Nie był moim najlepszym
przyjacielem, ale świetnie się bawiliśmy. Był dobrą osobą, choć
widziałam, jak się zachowuje jak kretyn.

-- To nie jest miłe. -- Seth zaskoczył Huberta, Etcetera.

Wydała dźwięk zniecierpliwienia. 

-- Bzdury. Nie toleruję tego niemówienia
źle o~zmarłych. Billiam był w~sześćdziesięciu procentach dobrym facetem,
w czterdziestu całkowitym kutasem. To stawia go w~połowie krzywej
dzwonowej dla ludzkości. Nienawidził bzdur tak gorąco jak centrum
słońca. Był moim przyjacielem, nie waszym.

Hubert, Etcetera poczuł łzy, nie wiedział dlaczego. Poszedł do łazienki,
usiadł na pokrywie toalety z~zamkniętymi oczami, potem wpatrzył się w~ekran lustra, pozwalając mu przeskakiwać pomiędzy jego profilami, tyłem
i szczytem głowy. Wyglądał gównianie. Jego druga myśl, która nadeszła z~piorunującą jasnością, była taka, że wyglądał jak normalny człowiek,
pośród miliardów ludzi, nie gorszy ani nie lepszy od nikogo. Pomyślał o~mowie Natalie o~krzywej dzwonowej i~pomyślał, że był w~zakresie sigmy
lub dwóch rozkładu na każdej osi.

Spryskał twarz zimną wodą i~wyszedł, dotykając dłonią ściany pomalowanej
palcami. Natalie i~Seth patrzyli na niego z~winą lub troską.

-- Okay, chłopie? -- spytał Seth.

-- Natalie -- powiedział -- nie sądzę, że przeciętna osoba jest w~sześćdziesięciu procentach dobra, a~w~czterdziestu procentach kutasem.
Myślę, że przeciętna osoba czasem się oszukuje, że jest centrum kosmosu
i to jest w~porządku, jeżeli robi coś, o~co byłaby wkurzona, gdyby ktoś
inny zrobił to jej, i~próbuje nie myśleć o~tym zbyt mocno.

-- Hm, dobra\ldots  -- powiedziała Natalie.

-- I~myślę, że tragedią ludzkiej egzystencji jest to, że nasz świat jest
kierowany przez ludzi, którzy są naprawdę dobrzy w~samooszukiwaniu, jak
Twój ojciec. Twojemu ojcu udaje się oszukiwać samego siebie, że jest
bogaty i~potężny, ponieważ jest śmietanką i~wzniósł się na szczyt.
Jednak nie jest głupi. On \textit{wie}, że się oszukuje. Zatem pod
najwyższą warstwą bzdur jest kolejna, bardziej świadomy system
przekonań: przekonanie, że wszyscy inni też by się tak oszukiwali, gdyby
mieli szansę.

-- Dokładnie tak -- powiedziała.

-- Jego przekonania nie zaczynają się od pomysłu, że to w~porządku się
oszukiwać, że jesteś specjalną śnieżynką, która zasługuje na więcej
ciastek niż każde inne dziecko. Zaczyna się od idei, że to jest \textit{w
naturze ludzka}, żeby się oszukiwać i~wziąć ostatnie ciastko, zatem
jeżeli nie on, to ktoś inny zje, zatem lepiej, żeby najhojniej sam się
oszukiwał, był najbardziej płodnym smakoszem ciastek, bo ktoś gorszy,
bardziej niemoralny i~chciwy niż on, dostanie się tam wcześniej, zje
ciastka, zabierze talerzyk i~zażąda opłat za picie mleka.

-- Dodaj tutaj ,,tragedię wspólnego pastwiska''. -- powiedział Seth.

Natalie podniosła dłoń. 

-- Wiesz, słyszałam termin ,,tragedia wspólnego
pastwiska'' jakby tysiące razy i~nigdy właściwie nie sprawdziłam. Co to
jest? Coś z~biednymi ludźmi, którzy są tragiczni?

-- To pospólstwo -- powiedział Hubert, Etcetera. Teraz coś było
przebudzone i~w nim luźne. Chciał skopać pizzę ze stolika do kawy i~wykorzystać go jako scenę. -- Wspólne. Wspólne ziemie, które do nikogo
nie należą. Wioski miały wspólne tereny, gdzie każdy mógł przyprowadzić
bydło na dzienny wypas. Ta część tragedii dotyczy tego, że jeżeli ziemia
nie należy do nikogo, to ktoś przyjdzie i~pozwoli owcom paść się tak
długo, aż nie będzie niczego, prócz błota. Wszyscy wiedzą, że ten bękart
nadejdzie, więc mogą równie dobrze \textit{być} tym chujkiem. Lepiej, żeby
owce należące do miłego faceta jak Ty, napełniły swoje brzuchy niż żeby
trawa trafiła do owiec jakiegoś samolubnego dupka.

-- Brzmi to mi jak bzdura.

-- Och, to prawda -- powiedział Hubert, Etcetera. Coś się ruszało w~jego
brzuchu, mrowiąc w~jądrach i~na twarzy. -- To więcej niż zwykła bzdura.
To jest piekąca, zła, rewolucyjna bzdura. Rozwiązaniem tragedii
wspólnych pastwisk nie jest załatwienie policjanta, żeby pilnował
socjopatów przed nadmiernym wypasem pastwiska, lub wykluczanie każdego,
kto tak robi, zamienianie go w~pariasa. Rozwiązaniem jest pozwolenie,
żeby rabuś-baron posiadał ziemię, która była wszystkich, ponieważ jako
jedyny rządzi nią dla zysku, zatem znakomicie się zatroszczy, żeby mieć
wiecznie zyski.

-- To jest tragedia wspólnych pastwisk? Bajka o~oddawaniu publicznych
zasobów bogaczom, żeby kierowali jak osobistymi imperiami, ponieważ w~ten sposób zapewnią, że będą lepiej zarządzane, niż byłyby, gdybyśmy
wymyślili jakieś reguły? Boże, mój tata musi \textit{uwielbiać} tę
historię.

-- To historia pochodzenia ludzi jak Twój ojciec -- powiedział Hubert,
Etcetera. -- To oczywista bzdura dla wszystkich, którym nie zależy na
tym, żeby było oczywiste.

-- Słyszysz to, tato? -- powiedziała, rozglądając się po pokoju. -- Oczywiste dla wszystkich, którym nie zależy na tym, żeby było oczywiste,
ty zakłamany, socjopatyczny gnojku.

-- Ma Cię na podsłuchu? -- spytał Seth.

-- Mam osobisty filtr prywatności na sieci domu. Ale oczywiście kamery
ciągle nagrywają, ponieważ gdybym została porwana lub zamordowana,
chciałby je przejrzeć. Oczywiście to bzdura i~zawsze był w~stanie
oszukać zamki. Nauczył się tego ode mnie, kiedy przejrzał jakieś logi
audytowe i~zobaczył, że to robię. Teraz mnie wyrzucił z~systemu, ale
jestem cholernie pewna, że są sytuacje, kiedy przegląda zapis. -- Spojrzała w~powietrze przed twarzą. -- Tak, tato, wiem, że słuchasz. To
żałosne.

Hubert, Etcetera pamiętał patrzenie w~odbicie w~łazience i~zastanawiał
się, czy istniało długoterminowe archiwum tych obrazów. Znał wielu ludzi
z domami na podsłuchach, ale nie potrafiłby żyć, jakby był cały czas
obserwowany. Kiedy infografiki mówiły, że masz wszystkie aktualizacje i~patche, musiałeś im zaufać. To powodowało, że paniki o~potężnych
naruszeniach bezpieczeństwa typu ,,zero day'' były takie przerażające:
nagła wiedza, że wszystko mogło być przejęte przez losowego przestępcę
lub dupka, który użył algorytmu wykrywania skóry, żeby przyłapać Cię na
masturbacji, słów kluczowych, żeby oznaczyć wszystkie żenujące rozmowy,
zbierał biometrykę na ataki na finanse i~sieci społeczne.

Życie z~taką wiedzą, że istnieli tam dziwaki wewnątrz Twoich granic,
było \textit{dziwaczne}. Z~wszystkich dziwnych rzeczy o~byciu
zettabogaczach, ta była najdziwniejsza. Na razie.

-- Przepraszam -- powiedział Hubert, Etcetera. -- Próbuję to zrozumieć. Jak
często Cię szpieguje?

-- Kto wie? Zwykle idę gdzieś indziej, za każdym razem, gdy chcę
porozmawiać na poważnie. -- Rozejrzała się po wielkim, przestrzennym,
brudnym pokoju. -- Nie przychodzę tutaj często.

Hubert, Etcetera założył, że to miejsce było śmietnikiem, ponieważ
Natalie była bogatym flejtuchem, która nie wiedziała, jak dobrze miała,
ale teraz rozumiał, że to był wyrachowany gest pogardy. To nie był jej
dom, to była obora. Hubert, Etcetera nie zawsze miał dobrą relację z~rodzicami, ale to był inny poziom.

-- A Twoja mama? -- spytał. -- Czy ona wie, że on Ciebie szpieguje?

-- Pewnie -- powiedziała. -- I~tak Mama nie wpada często, jest w~innej
strefie czasowej, minus osiem lub dziewięć godzin. -- Przechyliła głowę.
-- Och, masz na myśli, czy to seksualna sprawa? Nie, jestem pewna, że nie
jest. Mój tata dostaje swoje ciało przez specjalistów. Nigdy nie był
tego rodzaju zbokiem. -- Skierowała się w~powietrze. -- Widzisz, Tato?
Stanęłam za Tobą. Czymkolwiek jesteś, nie masz dewiacji na punkcie
swoich córek. Brawo.

Włosy na jego szyi się podniosły. Rzecz, która ożyła w~nim, zrobiła
powolny obrót w~jelitach.

Spojrzała na nich. 

-- Wyglądacie, jakbyście widzieli ducha. Nie martwcie
się, przyzwyczajajcie się. To nic innego wobec bycia poza
rzeczywistością, bycia wyczuwanym i~nagrywanym cały czas. Co gorszego
może się zdarzyć? Tata nie zamierza Cię zlikwidować lub wysłać
najemników Twoim śladem, gdy się wyniesiemy.

-- Gdy się wyniesiemy?

-- Czy nie o~tym rozmawiamy? Odchodzeniu? Tam właśnie ta rzecz zmierzała,
jakiś rodzaj ,,książę i~biedaka'': ,,Założę się, że mogę założyć szmaty
włóczęgi i~nikt nie zauważy w~niższych klasach, hej hej?''.

-- Nie zmuszaj mnie do dołączenia do odchodzących, Etcetera -- powiedział
Seth.

Rzecz wewnątrz Huberta, Etcetera zwinęła się. 

-- Czy to nie tam
zmierzałem?

Natalie złapała jego spojrzenie. Jej twarz lśniła. Była piękna. Miała
pryszcze, rozsypane piegi, twardówki w~oczach, które były różowe i~jej
powieki były zaczerwienione. Była pełna życia, smutku, i~cokolwiek
poczuł, kiedy uświadomił sobie, że szeptana rozmowa o~pieniądzach i~pracy, który wszyscy dorośli odbywają cały czas, była zewnętrznym
odbiciem głębokiego, bezkresnego strachu. Strachu, który gryzł każdą
dorosłą osobę. Pierwotny strach przed tygrysem pod jaskinią.

-- Pewnie jak cholera, tak to dla mnie brzmiało -- powiedziała.

-- Seth -- powiedział. -- Co właściwie powstrzymuje cię od odejścia?

Ku jego zaskoczenia, Seth wyglądał na rzeczywiście zakłopotanego. 

-- Żartujesz. Ci ludzie są zbzikowani. Oni są \textit{bezdomni}, Hubert\ldots  -- Hubert, Etcetera zauważył, że Seth nazwał go ,,Hubertem'', zawsze znak,
że dotknęli bogatego podkładu psyche Setha. -- Oni są włóczęgami. Jedzą
śmieci\ldots 

-- Nie do końca śmieci -- powiedział Hubert, Etcetera. -- Nie bardziej niż
piwo, które piliśmy ostatniej nocy, było sikami. Daj mi dobry powód.
Lojalność wobec pracodawcy? Widoki bogatego i~spełnionego życia? -- Jak
Hubert, Etcetera, najdłuższy okres, na jaki Seth był zatrudniony, trwał
sześć miesięcy, a~pierwszy miesiąc był zakwalifikowany jako
,,szkolenia'', bez płacy. Żaden z~nich nie miał prawdziwej pracy od
miesięcy.

-- A co ze strachem przed więzieniem?

-- Co z~nim? Wczoraj wyciągnąłeś mnie na nielegalną imprezę. Za to nas
bardziej prawdopodobnie mogą złapać, niż za cokolwiek co zrobimy na
opuszczonych terytoriach\ldots 

-- Terytoriach? Bądź poważny, będziesz martwy w~ciągu miesiąca.

-- To nie powierzchnia Księżyca. To miejsce, gdzie nikt nie chce zawracać
sobie głowy aresztowaniem populacji za włóczęgostwo.

-- Tak, nie aresztują ich, spalą ich za bycie skłotersami-terrorystami -- powiedział Seth. -- A do tego jest jeszcze przyjacielski ostrzał. To
jebana jama gladiatorów dla nadwyżki ludzi.

-- Tutaj ma rację -- powiedziała Natalie. -- Musimy się uzbroić, jeżeli
ruszamy. Schron taty jest pełen zabawek, rzeczy zaprojektowanych, żeby
nie być wykrywalnymi przez radary milimetrowe. Jeżeli zabralibyśmy
wystarczająco dużo, zostalibyśmy królami pustkowi. Mogłoby być zabawnie.

Hubert, Etcetera osłupiał. 

-- Czy wy dwoje widzieliście kiedyś
odchodników? Oni są praktycznie mnichami zen. Nie rozwalają swoich
rywali z~AK-3DP wydrukowanych z~żywicy. Widzieliście zbyt dużo filmów.

-- Widziałam odchodzących, ludzi, którzy odwiedzali wyzwolenia, ale kto
wie, jacy są w~ich rodzimym siedlisku. Nie ma powodu do bycia naiwnym.
Musisz być szalony, jeżeli myślisz, że idziemy na przechadzkę do Mordoru
z plecakami pełnymi pysznych racji MRE i~będziemy powitani jako duchowi
bracia.

Hubert, Etcetera był teraz tak zdenerwowany jak Seth. 

-- Czy kiedykolwiek
zabiliście kogokolwiek? Jesteście przygotowani, żeby tak zrobić? Czy
skierowalibyście broń na drugą istotę ludzką i~pociągnęli za spust?

Natalie wzruszyła ramionami. 

-- Jeżeli to byłoby ja lub on, kurwa tak. -- Seth pokiwał głową.

-- Oboje jesteście gówno warci.

On i~Seth się gapili. Natalie była bardziej rozbawiona niż kiedykolwiek.

Impas mógłby trwać, gdyby Hubert, Etcetera nie poszukał FAQ. Przez
chwilę się spierali, któremu anonimizatorowi zaufać, jeżeli byłeś w~wieku Natalie, wszystkie proxy, które Hubert i~Seth zwykli używać, były
postrzegane jako operacje pod fałszywą flagą, żeby zbierać wywiad o~dysydentach. Natalie, tymczasem, lubiła anonimizator, o~którym Seth i~Hubert, Etcetera słyszeli, że był gównianą techniką łamane przez
marzenia łamane przez voodoo. Okazało się, że oba systemy mogły być
połączone, zatem niechętnie je połączyli i~zaczęli poszukiwania.

Było tyle FAQ dla odchodzących, ilu odchodników. Impuls do odejścia był
powiązany z~pragnieniem napisania wspomnień w~stylu Thoreau o~społecznym
złu i~sztuce zniknięcia w~wieku całkowitej świadomości informacyjnej.
Dołączali dodatki podsumowujące rzeczy dla ludzi typu ,,tldr'', z~filmami, linkami w~darknecie, plikami trójwymiarowymi i~formułami, żeby
przygotować własne kluczowe enzymy i~GMO. Niektóre z~nich były
radioaktywnie niebezpieczne, rzeczy tego typu, które wpisywały Cię na
obserwowaną listę tak wysoko, że musiałbyś walczyć z~chmurami dronów,
żeby wyjść po mleko, ale nie było nic o~broni.

Hubert, Etcetera podkreślił to Natalie i~Sethowi, próbując nie być
zadowolonym z~siebie. 

-- Oczywiście nikt nie mówi o~broni tam, gdzie
wywiad może to zobaczyć. To wszystko będzie w~głębokim darknecie.

-- Mówisz, że fakt, że nie możemy znaleźć niczego o~broni, jest dowodem,
że tam muszą być bronie, ponieważ jeżeli tam byłyby bronie, nikt nie
rozmawiałby o~broni? -- Hubert, Etcetera miał doświadczenie w~wygrywaniu
dyskusji z~Sethem. Zauważył z~przyjemnością, że Natalie się zgodziła i~pławił się przez chwilę w~podziwie.

Seth spojrzał się na niego wojowniczo, nie mógł nadążyć. 

-- Dobra. Bez
broni.

Dotarło do Huberta, Etcetera, że to nie był eksperyment myślowy, gdzieś
po drodze, czytając FAQ i~oglądając filmy, przeszli od grania w~,,spróbujmy'' do planowania. Miał ekrany notatek i~duży plik zapisanych
rzeczy.

-- Czy naprawdę zamierzamy to zrobić? Naprawdę na poważnie?

Natalie rozejrzała się ostentacyjnie po pokoju. Hubert, Etcetera
pomyślał o~imprezach, wygłupianiu się, które tutaj musiało mieć miejsce,
dziwne dzieciaki zettabogaczy grający w~jakiekolwiek dekadenckie gry
preferowali przez lata. Pomyślał o~kamerach, zapisujących ich sesję
planowania z~różnych kierunków, zapisujących na długoterminowych
nośnikach.

-- Kurwa, tak -- wyszeptała. -- Zróbmy to.

\part{wszyscy spotkacie się w~gospodzie}

\chapter*{i}
Niedziele w~,,Belki i~Brasy'' były najbardziej zajęte i~zawsze były
zawodami o~najlepsze prace. Pierwsza osoba przechodząca przez drzwi
włączała światła i~sprawdzała infografiki. Były dostatecznie proste do
odczytania, że każda osoba mogłaby się połapać, nawet świeżaki. Jednak
Limpopo nie była świeżakiem. Miała więcej commitów do firmware'u Belek i~Brasów niż ktokolwiek, rząd wielkości ponad resztą. Było to technicznie
w złym smaku, żeby liczyć commity, nie mówiąc już o~zestawieniach. W~kulturze darów, dawałeś bez zapisywania, ponieważ zapisywanie darów
sugerowało oczekiwanie rewanżu. Jeżeli robisz coś dla nagrody, to jest
to inwestycja, a~nie dar.

W teorii, Limpopo się zgadzała. W~praktyce, było tak łatwo zachowywać
wyniki, tablica wyników była tak satysfakcjonująca, że nie mogła się
powstrzymać. Nie była z~tego dumna. Przez większość czasu. Ale tej
niedzieli, pierwsza przez drzwi Belek i~Brasów, sama w~wielkiej wspólnej
sali z~ustawionymi rzędami stołów i~krzeseł, wszystkie infografiki
pokazywały wartości nominalne, czuła się dumna. Poklepała ścianę z~perwersyjną, niedopuszczalną, właścicielską miną. Pomagała zbudować
Belki i~Brasy, przeszukując pustkowia za częściami, które drony
zidentyfikowały do konstrukcji. To był projekt, w~którym odnalazła swoje
odejście, najwyższa rzecz w~jej umyśle, kiedy się rozejrzała po
pustkowiach, odstawiła plecak, opróżniła kieszenie z~czegokolwiek
wartego kradzieży, włożyła dodatkową bieliznę do torby, wyszła na skarpę
Niagary, przez niewidzialną linię, która oddzielała cywilizację od ziemi
niczyjej, ze świata, jaki był i~do świata, jaki mógłby być.

Baza programów, która wywodziła się od Wysokiego komisarza Narodów
Zjednoczonych do spraw uchodźców, została \textit{bardzo} sprawdzona w~polu. Mówiłeś jej, jaki rodzaj budynku chciałeś, dawałeś jej zasięg
przeszukiwania i~ona wysyłała drony, żeby zinwentaryzowały wszystko w~pobliżu, skanując na wielu pasmach, wykonując głębokie kwerendy w~bazach
danych planowania urbanistycznego i~kodach źródłowych budynków, żeby
zidentyfikować użyteczne bloki dla czegokolwiek, co chcesz zrobić. Te
analizy zamieniały się w~inwentarz rzeczy do złupienia, a~uchodźcy lub
pracownicy pomocy (lub, w~haniebnych sytuacjach, sprzedani młodociani
niewolnicy) wychodzili, żeby zebrać rzeczy, które budynek potrzebował,
żeby wyczarować siebie w~istnienie.

Materiały napływały na miejsce prac. Budynek śledził i~konfigurował je,
ciągle przeliczając ścieżkę krytyczną dla planu budowy, która brała pod
uwagę poziom umiejętności pracowników i~robotów na budowie w~każdym
momencie. Efekt był czymś jak magia i~jak rytualne upokorzenie. Jeżeli
zainstalowałaś coś źle, system próbował znaleźć obejście dookoła Twojego
głupiego błędu. Jeżeli to się nie udało, system brzęczał coraz
intensywniej w~rękawicach. Jeżeli to zignorowałaś, próbował dotrzeć
optycznie, a~nawet głosowo. Jeżeli olałaś to, system zaczynał mówić
innym ludziom, że coś jest na opak, kierował ich, żeby to naprawili.
Było dużo dzielenia na wersje A/B -- były one w~bazie kodu i~w testach
jednostkowych do przeglądania przez każdą osobę -- a~większość udanych
strategii korygowania ludzi, jaką odkryły budynki, polegało na udawaniu,
że ludzie nie istnieją.

Jeżeli postawiłaś kawał stali konstrukcyjnej w~sposób, którego budynek
naprawdę nie mógł ominąć i~zignorowałaś narastający chór ostrzeżeń, ktoś
inny usłyszałby, że jest tam element ,,niedopasowanego'' materiału i~otrzymał zadanie poprawienia, o~wysokim priorytecie. To był ten sam
błąd, który budynki generowały, gdy coś się przesunęło. Błąd nie
zakładał, że istota ludzka spieprzyła coś złośliwie lub była
niekompetentna. Wstępna teoria była taka, że błąd bez wskazywania
odpowiedzialności byłby bardziej społecznie łaskawy. Ludzie robili
więcej błędów, szczególnie zawstydzeni przed innymi. Alternatywne wersje
,,wywoływania i~zawstydzania'' pokazywały zarumienione ostre wyparcia i~były największymi przeszkodami w~budowaniu budynku.

Zatem jeżeli coś spieprzyłaś, wkrótce ktoś pojawiłby się w~mecha lub
wózkiem widłowym czy śrubokrętem i~zleceniem, żeby odpieprzyć rzeczy,
które siłowo wpychałaś w~uległość. Mogłaś udawać, że robiłaś tę samą
pracę co nowy człowiek, część rozwiązania zamiast przyczyna problemu. To
pozwalało Ci zachować twarz, że nie nalegałabyś, że robisz dobrze, a~głupie instrukcje budynki (i wszystkich innych w~kosmosie) były błędne.

Rzeczywistość była zuchwale dziwniejsza w~sposób, który Limpopo kochała.
Okazywało się, że byłaś wysłana do odkręcenia czegoś i~znajdowałaś
kogoś, kto z~pewnością był źródłem partaniny, mogłaś z~pewnością
zauważyć, że stal konstrukcyjna nie była trzy stopnie od pionu z~powodu
poślizgu, była trzy stopnie od pionu, ponieważ jakiś gnojek to
spartaczył. Do tego, Mister Gnojek wiedział, że Ty wiesz, że to jego
wina. Ale fakt, że w~zleceniu było napisane ,,WAŻNE POPRAWIĆ PION
KONSTRUKCYJNY MINUS 3 STOPNIE NA 120 NNE'', a~nie ,,WAŻNE POPRAWIĆ PION
MINUS 3 STOPNIE NA 120 NNE PONIEWAŻ JAKIS GNOJEK NIE POTRAFI PRZECZYTAĆ
INSTRUKCJI'' pozwalał obojgu stronom przedstawić to w~formie
zmanierowanego kabuki, w~którym używany był głos w~trzeciej osobie w~stronie biernej: ,,Słup został odchylony się od pionu'', a~nie
,,Spieprzyłeś pion''.

To udawanie -- badacze nazywali to ,,sieciowym społecznym brakiem
uwagi'', ale wszyscy inni nazywali to efektem ,,Jak to się stało?'' -- była istotną zmianą w~inicjatywie rozproszonych schronów UNHCR. Przedtem
wszystko to był zgrywalizowane aż do przepierdolenia, z~tabelami wyników
dla najpoprawniejszych instalacji i~najlepszych zbieraczy. Kompilacje
testowe były psute przez gniewne konfrontacje i~walki na pięści. Nawet
to było cnotą, od kiedy każda kompilacja mogła podzielić się na dwa lub
trzy podgrupy, każda stawiająca własny budynek. Trzy w~cenie jednego!
Nieuchronnie te rozwidlone projekty były mniej ambitne niż oryginalny
plan.

Wczesne budynki miały charakterystyczny wygląd: szerokie, płaskie,
niskie budynki, pierwsze trzy piętra czegoś, co było zaplanowane na
dziesięć, zanim połowa pracowników odeszła. Sto metrów dalej, trzy
kolejne budynki, każdy połowa pierwszego, przedstawiające podzielone i~rozdzielone budynki zemsty, zbudowane przez wyobcowanych przeciwników.
Niektóre miejsca tworzyły spirale Fibonnaciego coraz mniejszych wersji,
kończąc się na promieniującym wrogością domku dla lalek.

Budynki przeskoczyły z~repozytoriów UNHCR do odchodników i~zmutowały w~niepoliczalne wersje poza panteon szpital / szkoła / schronisko dla
uchodźców. Belki i~Brasy były pierwszą gospodą, kiedykolwiek
wypróbowaną. Projekty dla kuchni w~restauracji nie były odległe od
kuchni obozowych, wielkie wspólne przestrzenie były dostatecznie łatwe,
ale właściwy zeitgeist tego był istotnie odmienny, skorygowany na
tysiące sposobów, że nigdy nie weszłabyś i~nie powiedziałabyś: ,,To jest
dom dla uchodźców, który został zamieniony na restaurację''.

Jednak nigdy byś nie pomyliła Belki i~Brasów z~normalną restauracją. Jej
największą właściwością był mapowane oświetlenie, które kolorowało
przestrzenie i~przedmioty we wnętrzach delikatnymi tonami
czerwieni/zieleni informującymi, gdzie coś potrzebowało uwagi człowieka.
To był podręcznik UNHCR, ale znowu, był cały świat różnic pomiędzy
podawaniem racji MRE uchodźcom klimatycznym i~serwowaniem wymyślnych
koktajli z~suchego lodu zrobionych w~mokrych drukarkach i~ze
sproszkowanego alkoholu. Żaden obóz dla uchodźców nie zużył tak dużo
parasolek koktajlowych i~pałeczek do mieszania koktajli.

W przeciętny dzień, Belki i~Brasy obsługiwały kilkaset osób. W~niedziele
było to więcej niż pięćset. Napływ świeżaków sprowadzał szukających
talentu, partnerów seksualnych, członków grupy, członków trupy
teatralnej i, oczywiście, ofiar. Przejście przez drzwi jako pierwsza
oznaczało, że Limpopo dostała rolę kierownika sali.

Próby pokazywały, że piwo z~ostatniej nocy wyszło dobre. Ogniwa wodorowe
miały 45 procent, co zapewniłoby pracę Belek i~Brasów przez maksymalnie
dwa tygodnie, wiatraki na dachu ciężko się kręciły, elektrolizując
ścieki i~pompując wodór w~ogniwa. W~piwnicy było pięćdziesiąt ogniw,
zebranych z~porzuconych odrzutowców, które napotkały drony. Odrzutowce
nie spełniały wymagań technicznych od długiego czasu, ale zawierały
wielkie ilości materiałów dla Belek i~Brasów, włączając w~to tuziny
ławek zrobionych z~foteli. Wytrzymała tapicerka wyszła czysta, jej
powierzchnie odpychające brud odsłaniające wzory z~każdym przetarciem
szmaty jak ponownie pojawiający się atrament sympatyczny.

Ale ogniwa wodorowe były największym znaleziskiem z~wszystkich, bez
nich, Belki i~Brasy musiałyby być inne, skłonne do niedoborów i~zaciemnień. Limpopo denerwowała się, że mogły być skradzione. Musiała
wykorzystać całą samokontrolę, żeby nie zainstalować sprzęt do
inwigilacji dookoła włazów roboczych.

Prefabrykowane rzeczy w~spiżarni pokazywały się na zielono, ale ciągle
chciała osobiście powąchać kultury sera i~dźgnąć ciasto przez folię do
ugniatania ciasta. Prekursory sosu smakowały świetnie, a~maszyna do
lodów mruczała, gdy leniwie napowietrzała zamrożony krem. Zamówiła
coffium i~usiadła w~promieniu światła w~środku wspólnych, gdy smakowity,
owocowy, piżmowy aromat napłynął do pomieszczenia.

Pierwszy kubek coffium tańczył gorąco w~jej ustach i~wcześnie
przygotowane składniki przesiąknęły do jej krwi przez błony śluzowe pod
jej językiem. Jej palce i~skóra na głowie zamrowiły, zamknęła oczy, żeby
cieszyć się efektami drugiej fali substancji wprowadzonych do niej, gdy
jelita zaczęły pracować. Jej słuch stał się nieludzki, wielkie mięśnie
na plecach i~piersiach i~ramionach dostały płomiennego uczucia drżenia,
choć były nieruchome.

Wzięła kolejny głęboki łyk, zamknęła oczy, a~kiedy je otworzyła znowu,
miała towarzystwo.

Oni byli takimi oczywistymi świeżakami, że mogliby pochodzić z~jednej
formy. Gorzej, byli szleperami, ich ciężkie przewymiarowane plecaki,
płaszcze turystyczne z~wieloma kieszeniami i~maksymalnie wypchane
spodnie typu cargo. Wyglądali na przepompowanych. Szleperzy byli
nerwicowi i~prawdopodobnie przeznaczeni do powrotu w~ciągu tygodni,
zostawiając za sobą przewlekłe międzyosobowe popierdolenie. Limpopo
odeszła we właściwy sposób, z~niczym więcej niż czysta bielizna, co i~tak okazało się zbyteczne. Próbowała nie być uprzedzoną wobec trójki,
szczególnie w~trakcie tych zawrotnych pięciu minutach haju coffium. Nie
chciała być szorstka w~jej łagodności.

-- Witamy w~B \& B! -- krzyknęła, głośniej niż zamierzała. Wzdrygnęli,
potem się zebrali.

-- Cześć -- powiedziała dziewczyna i~podeszła. Jej ubrania były piękne,
skośne cięcia i~kontrastujące szwy. Limpopo natychmiast ich zapragnęła.
Później ściągnęłaby obraz dziewczyny z~archiwum, rozłożyła wzorce i~wyprodukowała zestaw dla siebie. Byłaby przedmiotem zazdrości dla
wszystkich, którzy by ją ujrzeli, póki projekt by się nie upowszechnił i~stał starą nowością. 

-- Przepraszam, że tak weszliśmy, ale słyszeliśmy\ldots 

-- Słyszeliście dobrze. -- Głos Limpopo był cichszy, ale ciągle zbyt
krzyczący. Albo coffium musiało się wypalić, żeby mogła zapanować nad
afektami, lub musiała wypić znacznie więcej tak, żeby przestać się tym
przejmować. Uderzyła w~strefę uzupełniania i~podstawiła kubek pod dyszę.

-- Otwarci dla wszystkich, cały dzień, każdego dnia, ale niedziele są
specjalne, nasz sposób powiedzenia cześć naszym nowym sąsiadom i~poznania ich. Jestem Limpopo. Jak chcecie być nazywani?

Fraza była szczególna dla odchodników, wyraźne zaproszenie do
przerobienia siebie. Takie powitanie było szczytem wyszukania
odchodzących, Limpopo użyła go świadomie na tej trójce, ponieważ
potrafiła dostrzec jak byli mocno spięci.

Niższy z~dwójki, z~niechlujną, skręconą brodą i~szczeciniastą, ogoloną
głową, wyciągnął dłoń. 

-- Jestem Gizmo von Kaczka. To jest Zombie
McBorówka i~Etcetera. -- Tamtych dwoje przewróciło oczami.

-- Dziękuję, ,,Gizmo'', ale właściwie, możesz mi mówić Stabilne Strategie
-- powiedziała dziewczyna.

Drugi chłopak, wysoki, ale zgarbiony z~sowią miną i~śladami wyczerpania
na twarzy, westchnął. 

-- Równie dobrze możesz mi mówić Etcetera. Dzięki,
,,Herr von Kaczka''.

-- Miło mi was poznać -- powiedziała Limpopo. -- Dlaczego nie odłożycie
rzeczy, nie usiądziecie, a~ja przyniosę wam coffium, co?

Trójka spojrzała po sobie i~Gizmo wzruszył ramionami i~powiedział: 

-- Kurde tak. 

Zdjął z~ramion plecak i~pozwolił mu upaść na podłogę z~uderzeniem, od którego Limpopo podskoczyła. Jezu kurwa, co te świeżaki
\textit{targali} przez wzgórza i~doliny? Cegły?

Pozostali zrobili tak samo. Dziewczyna zdjęła buty i~potarła stopy.
Potem wszyscy to zrobili. Limpopo zmarszczyła nos na zapach spoconych
stóp i~zapisała sobie, żeby pokazać im gdzie się wymienia skarpetek.
Wycisnęła trzy coffiumy, używając ceramicznych kubków grubości papieru,
wydrukowanych ze splatających się, chwytliwych pasków tekstury.
Postawiła kubki na spodkach i~dodała drobne marchewkowe biszkopty,
piklowane rzodkiewki i~zaniosła je do stołu świeżaków na tacy, która z~kliknięciem wsunęła się w~kwadratowe gniazdo. Wzięła wielki kubek i~wywinęła nim: -- Za pierwsze dni lepszego świata -- powiedziała, kolejny
wiejski tekst odchodników, ale niedziele były dniem na wsiowe rzeczy
odchodzących.

-- Pierwsze dni -- powiedział Etcetera z~zadziwiającą (przerażającą)
szczerością.

-- Pierwsze dni -- powiedzieli pozostali i~stuknęli. Wypili i~siedzieli w~ciszy, gdy kawa w~nich zaczynała działać. Dziewczyna uśmiechała się jak
,,kot z~kanarkiem'', oddychała głośno i~krótko, każdy oddech sprawiał,
że była wyższa. Inni byli mniej wylewni, ale oczy im błyszczały. Własna
dawka Limpopo była teraz optymalna, ale nagle chciała przywitać tych
świeżaków jak najmilej to możliwe. Chciała, żeby czuli się niesamowicie
i pewnie.

-- Chcecie brunch? Są gofry z~prawdziwym syropem klonowym, jajka jak
chcecie, trochę boczku i~żeberka z~kurczaka, i~jestem całkiem pewna
croissantów.

-- Możemy pomóc? -- spytał Etcetera.

-- Nie przejmuj się. Siedźcie tutaj i~zanurzcie się, niech Belki i~Brasy
zajmą się wami. Potem, zobaczymy, czy możemy załatwić wam zajęcie. -- Nie
powiedziała, że byli zbyt nowi, żeby zasłużyć na prawo do zaczepienia
się w~B\&B, że odchodzący szliby pięćdziesiąt kilosów, żeby z~pokorą
pomagać w~Belkach i~Brasach. Tak czy inaczej, kuchnia B\&B zajmowała się
wszystkim. Chwilę zabrało Limpopo zrozumienie idei, że jedzenie to
chemia stosowana i~ludzie byli gównianymi laborantami, ale kiedy John
Henry dzieli się z~systemami automatycznymi, nawet ona zgadzała się, że
B\&B produkowała najlepsze jedzenie przy minimalnej ludzkiej
interwencji. I~były croissanty, co było ekscytujące.

Sama \textit{wycisnęła} pomarańcze, ale tylko dlatego, że kiedy była w~nastroju, lubiła ściskać dłonie i~pracować mięśniami ramion i~barków i~potrafiła oczyścić skórki pomarańczy niemal tak do czysta jak maszyna.
Zresztą to były niebieskie pomarańcze, zoptymalizowane do hodowania w~szklarniach północy i~chętnie oddawały sok. Ustawiła wszystko na tacy -- to, przynajmniej, było czymś, w~czym ludzie byli świetni -- i~dostarczyła.

Kiedy wychodziła z~kuchni, było jeszcze więcej świeżaków, a~jeden z~nich
wymagał pomocy medycznej z~powodu udaru cieplnego. Właśnie się za to
brała -- coffium było świetne w~zachowaniu spokoju, kiedy było wiele
zadań -- kiedy pojawiło się więcej starszych użytkowników i~wydajnie
rozstawiło i~nakarmiło wszystkich innych. Wkrótce pojawił się stały
kołyszący rytm w~gospodzie, który Limpopo zajebiście \textit{kochała},
szum złożonych systemów adaptacyjnych, gdzie ludzie i~software
współistnieli w~stanie, który mógłby być nazwany \textit{tańczeniem}.

Menu zmieniało się w~ciągu dnia, zależnie od surowców, które goście
przynieśli. Limpopo poruszała się na brzegach, ruszając od jednego
czerwonego światła do drugiego, póki nie zmieniły się w~zieleń,
rozwijając rodzaj szóstego zmysłu co do kolejnego czerwonego światła,
wpisując więcej niż jej przydział jednostek pracy. Gdyby była tablica
wyników tego dnia, byłaby żenująco poza wykresem. Udawała, tak mocno jak
umiała, żeby jej przyjaciele nie zauważyli jej ruchliwej aktywności.
Kultura darów nie powinna być karmicznym rejestrem z~Twoimi dobrymi
uczynkami w~jednej kolumnie i~metodami, w~jaki otrzymałaś korzyści
wypisanych w~drugiej. Celem odchodzących było życie dla dostatku, i~w dostatku, dlaczego martwić się, czy wkładasz tyle samo, co wyjmujesz?
Ale darmozjady były darmozjadami, i~nie było końca dupków, którzy
zabierali najlepsze rzeczy lub rujnowali rzeczy przez bezmyślność.
Ludzie zauważali. Dupki nie były zapraszane na imprezy. Nikt nie
nadkładał drogi, żeby o~nich zadbać. Nawet bez rejestru, ciągle był
rejestr, a~Limpopo chciała zapisać trochę dobrych życzeń i~karmy, na
wszelki wypadek.

Tłum rozluźnił się około szesnastej. Było wystarczająco łatwo psujących
się produktów, że B\&B ogłosiła święto i~złożyła popołudniową herbatę z~dodatkami. Limpopo przesunęła się w~kierunku czerwieniejącej strefy w~obszarze przygotowywania jedzenia i~spotkała tego faceta, Etcetera.

-- Hej, jak cieszysz się dniem świeżaka tutaj we wspaniałym Belce i~Brasach?

Uchylił się. 

-- Czuję się, jakbym miał wybuchnąć. Zostałem nakarmiony,
narkotyzowany, upity i~drzemałem przy ogniu. Po prostu nie mogę już
więcej tam siedzieć. Proszę, daj mi zajęcie?

-- Wiesz, że to coś, o~co nie powinieneś prosić?

-- Takie odniosłem wrażenie. Jest coś dziwnego w~was, znaczy w~nas\ldots  i~pracy. Nie powinieneś łakomić się na pracę, nie powinieneś patrzeć z~góry na próżniaków i~nie powinieneś pochlebiać komuś, kto haruje. To
powinna być emergentna, naturalna homeostaza, tak?

-- Tak myślałam, że możesz być sprytny. Otóż to. Pytanie kogoś, czy
możesz się dołączyć, jest mówieniem im, że kierują i~że zwracasz się do
ich autorytetu. Obie te rzeczy są ,,verboten''. Jeżeli chcesz pracować,
coś rób. Jeżeli to nie jest pomocne, może ja to później odkręcę, lub
przegadam to z~Tobą, albo puszczę płazem. To jest pasywno-agresywne, ale
takie jest odchodnictwo. To nie tak, że się śpieszymy.

Przeżuł to. 

-- Naprawdę? Czy naprawdę jest \textit{dostatek}? Jeżeli jutro
cały świat by odszedł, czy wystarczyłoby?

-- Z~definicji -- powiedziała. -- Ponieważ dość to tyle, ile określisz.
Może chcesz mieć trzydzieścioro dzieci. ,,Dość'' dla Ciebie jest więcej
niż ,,dość'' dla mnie. Może chcesz otrzymywać swoje kalorie w~bardzo
konkretny sposób. Może chcesz żyć w~bardzo konkretnym miejscu, gdzie
wiele innych osób chce żyć. Zależnie od tego, jak patrzysz, może nigdy
nie być dość, lub może być zawsze dużo.

Kiedy gadali, troje innych odchodniczek przygotowało herbatę, ręcznie
robione bułeczki, gustowne kanapki, parujące czajniki i~ułożyło na
tacach. Świadomie stłumiła lęk przed kimś wykonującym ,,jej'' pracę. Tak
długo jak praca była wykonana, to się liczyło. Jeżeli cokolwiek się
liczyło. Czym właśnie było. Ale nie we wspaniałym planie rzeczy.
Rozpoznała jedną ze swoich pętli.

-- Cóż, to załatwia sprawę -- powiedziała, wskazując brodą na ludzi
wynoszących tace. -- Zjedzmy.

-- Nie sądzę, żebym mógł. -- Poklepał się po żołądku. -- Powinniście
zainstalować vomitorium.

-- To tylko legenda -- powiedziała. -- ,,Vomitorium'' tylko oznacza wąskie
przejście między dwiema komnatami, skąd tłum mógł być wyrzygiwany do
przodu. Nic wspólnego z~objadaniem się w~kolektywnej bulimii.

-- Niemniej jednak. -- Popatrzył zamyślony. -- Mógłbym zainstalować jedną,
prawda? Zalogować się do back-endu, naszkicować, zacząć szukać surowców,
rozbierać rzeczy i~rozbierać cegły?

-- Technicznie, ale nie sądzę, żebyś dostał pomoc, no i~gdy nie będzie
Cię w~pobliżu, będą rollbacki, ludzie odbudowujący przestrzeń, którą
rozebrałeś. Znaczy, vomitorium jest nie tylko apokryficzne, jest
obrzydliwe. Nie rzeczy, które zdarzają się w~praktyce.

-- Ale gdybym miał gang trolli, moglibyśmy to zrobić, co? Mógłbym
postawić straże na miejscu, żądać opłaty za wstęp, przełączyć się na Big
Maca?

To była nudna dyskusja każdego świeżaka. 

-- Ta, mógłbyś. Jeżeli Ci się
uda, zbudujemy kolejne Belki i~Brasy dalej i~będziesz miał budynek pełen
trolli. Nie jesteś pierwszą osobą, która przeprowadziła ten drobny
eksperyment myślowy.

-- Oczywiście, że nie jestem -- powiedział. -- Przepraszam, jeżeli nudzę.
Znam teorię, ale wydaje się, jakby po prostu nie mogła zadziałać.

-- W~teorii w~ogóle to nie działa. W~teorii, jesteśmy samolubnymi
dupkami, którzy chcą więcej niż sąsiedzi, nie mogą być szczęśliwi z~dużą
ilością, jeżeli ktoś inny ma jeszcze więcej. W~teorii, ktoś wejdzie
tutaj, kiedy nikogo nie ma w~pobliżu i~zabierze wszystko. W~teorii, to
bzdura. Te rzeczy działają tylko w~\textit{praktyce}. W~teorii, to
bałagan.

Zachichotał, niespodziewany, młody dźwięk.

-- Mam mnóstwo pytań o~to, ale odpowiedziałaś tak szybko, że założę, że
możesz przedstawić tyle odpowiedzi jak ta, ile trzeba.

-- Och, jestem pewna -- powiedziała. Lubiła go, mimo jego bycia szleperem.

-- Czy to się skaluje? 

-- Jak na razie dobrze. Co się stanie na dłuższą
metę? Jak mądra osoba kiedyś powiedziała\ldots 

-- Na dłuższą metę wszyscy umrzemy.

-- Choć kto wie, prawda?

-- Nie wierzysz w~to, prawda?

-- Dla mnie wydaje się oczywiste. Kiedy jesteś bogaty, nie musisz
umierać. To jasne. Zbierz razem cały zestaw terapii, selektywna
optymalizacja plazmy zarodkowej, ciągła obserwacja zdrowia, terapie
genowe, uprzywilejowany dostęp do przeszczepów\ldots  Jeżeli wierzyłabym we
własność prywatną, mogłabym przedstawić Ci szansę, że pierwsze pokolenie
nieśmiertelnych ludzi żyje dzisiaj. Prześcigną i~wyprzedzą własną
śmiertelność.

Obserwowała go próbującego nie zgodzić się bez bycia niegrzecznym i~pamiętała, jak martwiła się obrażaniem ludzi, kiedy pierwszy raz
odeszła. To było urocze.

-- Tylko dlatego, że pieniądze mogą być wymienione na długość życia, nie
wynika, że to się skaluje. -- powiedział. -- Możesz wymienić pieniądze za
ziemię, ale jeżeli spróbujesz kupić Nowy Jork jeden blok na raz, skończą
Ci się pieniądze, niezależnie z~iloma zaczynałaś, ponieważ zmniejszył
się popyt\ldots  -- Pokręcił głową. -- Znaczy, nie chcę powiedzieć, że jest
podaż i~popyt, jeżeli chodzi o~Twoje życie, ale na pewno \textit{malejące
przychody}. Wiara, że nauka będzie się rozwijać w~tym samym tempie co
śmiertelność to trele morele. -- Wyglądał dziwnie. Lubiła tego chłopaka.
-- To akt wiary. Bez urazy.

-- Bez urazy. Pominąłeś najważniejszy argument. Przedłużenie życia
pociąga koszt w~jakości życia. W~tym kierunku -- wskazała na południe -- żyje facet trzysta kilometrów stąd, wart więcej niż większość krajów,
którego organy i~szara materia są w~kadzi. Kadź jest w~ufortyfikowanej
klinice i~klinika jest za murami miasta. Wszyscy, którzy pracują w~tym
mieście, dzielą z~nim florę bakteryjną. To warunek zatrudnienia. Masz
sto razy więcej nieludzkich komórek w~ciele niż ludzkim. Ludzie, którzy
żyją w~tym mieście, są w~99 procentach nieśmiertelnym bogaczem,
rozszerzeniami jego ciała. Wszystko, czym się zajmują, to wymyślenie,
jak przedłużyć jego życie. Większość z~nich była najlepszymi w~rocznikach na najlepszych uniwerkach świata. Zrekrutowani w~szkole. Z~pensją, której nikt nie dorównuje.

-- Poznałam kogoś, kto kiedyś tam pracował, zrezygnował i~został
odchodnikiem. Powiedział, że facet w~kadzi żyje w~ciągłej agonii. Coś
włączyło jego odczuwanie bólu w~,,ciągłe, nieprzystosowujące się,
maksymalne obciążenie''. Czuje tyle bólu, ile człowiek może, bólu, do
którego nie możesz się przyzwyczaić. Może im kazać wyłączyć maszyny, a~wtedy umrze. Niemniej jednak się trzyma. Zakłada, że jakiś supergeniusz
w jego mieście, który marzy o~premii za usunięcie błędu z~osobistej
listy tego faceta, wymyśli sposób, jak rozwiązać tę sprawę z~nerwami.
Będą przełomy, jeżeli wszystko pójdzie zgodnie z~planem. Zatem kadź
będzie tylko jego fazą larwalną. Nie musisz w~to wierzyć, ale to prawda.

-- To nie dziwniejsze niż inne rzeczy, które słyszałem o~zettach. Jedyną
mało prawdopodobną częścią jest to, że Twój kumpel był w~stanie odejść od
tego wszystkiego. Brzmi jak rodzaj umowy, gdzie byłbyś ścigany jak pies
za naruszenia klauzuli poufności.

Pamiętała gościa, która nazwał siebie Langerhansem, wszystkie te dziwne
rzeczy szpiegowskie, skrytki i~co robił, żeby uniknąć zostawiania
komórek skóry i~włosów, wycierając używane szkło i~sztućce. 

-- Nie wyróżniał się. Co do tego NDA, opowiadał naprawdę dziwne rzeczy, ale
nic, czego mogłabym użyć do uruchomienia swojego programu lub
sabotowania człowieka w~kadzi. Był sprytny. Kompletny cholerny chuj. Ale
sprytny. Uwierzyłam mu.

-- To tak jak mówiłem. Ten facet znosi niewyobrażalny ból z~powodu
przesądnego przekonania, że może kupić sobie ucieczkę od śmierci. Fakt,
że ten facet w~to wierzy, nie ma żadnego związku z~rzeczywistością. Może
ten facet spędzi sto lat zamknięty w~nieskończonym piekle. Zetty są tak
dobrzy w~samooszukiwaniu jak każdy. Nawet lepsi, są przekonani, że są
tam, gdzie są, ponieważ są najlepsi ewolucyjnie, że zasługują na
podniesienie ponad zwykłych ludzi, zatem są ugruntowani do wierzenia, że
wszystko, co czują, musi być prawdziwe. Co, oprócz bycia ślepą,
egoistyczną wiarą tych zett, prowadzi Cię do przekonania, że to coś
więcej niż myślenie życzeniowe?

Limpopo pamiętała pewność Langerhansa, jego niskie, intensywne przemowy
o nadchodzącym wieku nieśmiertelnych zett, których dynastie rodzinne
byłyby prowadzone przez nieumierających tyranów.

-- Przyznaję, nie mam niczego do udowodnienia tego. Wszystko, co wiem,
dowiedziałam się z~drugiej ręki kogoś przerażonego swoją skórą. To jedna
z tych rzeczy, gdzie warto zachowywać się jakby była prawdziwa, nawet
jeżeli nigdy nie nadejdzie. Zetty próbują oderwać się od ludzkości. Nie
widzą swojego przeznaczenia jako związanego z~naszym. Myślą, że mogą
politycznie, ekonomicznie i~epidemiologicznie odizolować się, zająć
wysokie pozycje ponad wznoszącym się poziomem mórz, chować swoje
potomstwo przy pomocy odrzutowców Harriera.

-- Byłam odchodniczką prawie rok, zanim to zrozumiałam. To właśnie
\textit{jest} odchodzenie, nie odejście od ,,społeczeństwa'', ale
zrozumienie, że w~zettaświecie, my jesteśmy problemami do rozwiązania, a~nie obywatelami. To dlatego nigdy nie usłyszysz polityków mówiących o~,,obywatelach''. To zawsze są ,,podatnicy'', jakby istotnym faktem
naszej relacji z~Państwem było, jak dużo płacimy. Jakby Państwo było
biznesem, obywatelstwo programem lojalnościowym, które nagradza Twoje
zachowanie drogami i~opieką zdrowotną. Zetty spreparowali ten proces
tak, że dostają wszystkie pieniądze i~władają polityką, płacąc tyle
podatków, ile chcą. Jasne, płacą najwięcej podatków, ponieważ zbudowali
zestaw reguł, który daje im większość pieniędzy. Mówienie o~,,podatnikach'' oznacza, że dług Państwa należy do bogatych facetów, a~cokolwiek daje dzieciom, starcom, chorym lub osobom z~niepełnosprawnością jest dobroczynnością, za którą powinniśmy być
wdzięczni, skoro żadna z~tych osób nie płaci podatków, które
usprawiedliwiają ich nagrody od Rząd Spółka Akcyjna.

-- Żyję, jakby zetty nie wierzyli, że są moim gatunkiem, aż do
nieuchronności śmierci i~podatków, ponieważ \textit{oni} w~to wierzą.
Chcesz wiedzieć, w~jaki sposób Belki i~Brasy są utrzymywane? Odpowiedź
jest związana z~naszą relacją do zett. Mogą nas zmiażdżyć jutro, jeżeli
zechcą, ale nie zrobią tego, ponieważ kiedy rozgrywają swoją sytuację,
są lepiej obsłużeni przez niektórych z~nas ,,rozwiązujących'' swój
problem przez usunięcie siebie z~procesu politycznego, szczególnie że
bylibyśmy ludźmi, którzy, ogólnie mówiąc, byliby największym wrzodem na
tyłku, gdybyśmy zostali\ldots 

-- No weź. -- Miał ładny uśmiech. -- To jest egoistyczne! Dlaczego uważasz,
że jesteśmy największym wrzodem? Może jesteśmy najłatwiejsi z~wszystkich, skoro jesteśmy gotowi na odchodzenie. Co z~ludźmi, którzy są
zbyt chorzy lub młodzi, starzy, uparci i~żądają, żeby Państwo radziło
sobie z~nimi jako obywatelami?

-- Ci ludzie są najłatwiejsi do wyłapania i~zamknięcia. Dlatego, że nie
mogą uciec. To jest potworne, ale rozmawiamy o~potwornych rzeczach.

-- To przerażające -- powiedział. -- I~filmowe. Naprawdę myślisz, że zetty
będą siedzieć w~sądzie, knując jak oddzielić kozy od owiec?

-- Oczywiście, że nie. Cholera, gdyby tak zrobili, moglibyśmy zaplanować
zamach samobójczy na skurwieli. Myślę, że to jest emergentny wynik. To
jest nawet jeszcze gorsze, ponieważ istnieje w~strefie rozmytych
odpowiedzialności, nikt nie decyduje, żeby wtrącić do więzienia
rekordowe liczby biedaków, to tylko się staje jako wynik twardszych
praw, mniejszych funduszy na pomoc prawną, dodanego kosztu do procesu
apelacji\ldots  Nie ma osoby, decyzji czy procesu politycznego, który możesz
obwiniać. To jest systemowe.

-- A jaki jest systemowy wynik bycia odchodzącym?

-- Nie wiem, czy ktokolwiek to już wie. Zabawnie będzie się dowiedzieć.

\chapter*{ii}
Przyjaciele chłopaka obudzili się z~drzemki, kiedy Limpopo i~on
sprzątali talerze, co oznaczało rejestrowanie błędów tam, gdzie
procedury sprzątania naczyń zawiodły. Chytre było to, że połowa bugów
już była zgłoszona, ale nie było jasne, czy \textit{są }one tymi samymi
błędami, a~było chujowe zduplikować błąd, gdy mogłeś spędzić czas,
decydując, czy bug już był zgłoszony. Plus, dodawanie więcej informacji
do istniejącego błędu sprawiało, że było bardziej prawdopodobne, że
zostanie naprawiony. Jeżeli chciałeś naprawić buga, powinieneś naprawdę
go dogłębnie sprawdzić.

Przywędrowali, apatyczni, zaropiałe oczy, powiew niemytej skóry. Limpopo
zasugerowała, żeby odwiedzili onsen z~tyłu. Wszyscy byli posłuszni.
Zrezygnowali z~bugów, niech inni korzystający zajmą się zgłaszaniem
błędów w~swoim czasie, i~założyli swoje plecaki szleperów, kierując się,
zataczając, na tyły gospody.

-- Jak to działa? -- spytała dziewczyna. -- Daj nam FAQ -- wymawiała to
,,fuck'' -- dla tej perwersyjnej mydlanej waszej rzeczy. 

Limpopo pomyślała, że dodanie przez nią prychnięcia przy ,,perwersyjne
mydlanej'' było przyznaniem się do lęku przed włączeniem w~odchodnickie
orgie.

-- Jest koedukacyjna, ale nie dla seksu, nie martw się. Zasady składają
się w~30 procentach z~reguł odchodzących, a~w~70 procentach z~japońskich. Tylko tyle formalizmu, żeby wszyscy cieszyli się swoim
towarzystwem, nie tyle, żebyś się martwiła, że robisz coś źle. Ważne,
żeby pamiętać, że kąpiele są dla relaksu, nie dla mycia. Nie chcesz do
nich wchodzić inaczej jak tylko z~czystą skórą. Żadnych strojów
kąpielowych, a~i~siadasz w~kabinie prysznicowej na hardkorowe szorowanie
i etap dekontaminacji przed wejściem. Gorąca woda jest nieograniczona,
jest pasteryzowana przez słońce na dachu, potem jest trzystopniowy filtr
przez wydrukowany węgiel drzewny o~powierzchni wielkości księżyców
Jowisza.

-- Kiedy jesteście czyści, robicie, co chcecie. Niektóre z~kąpieli sparzą
was po dziesięciu minutach, niektóre są tak zimne, że dostaniecie
hipotermii, jeżeli w~nich zostaniecie, a~reszta jest pomiędzy. Idźcie
tam, gdzie was nastrój zabierze. Ja lubię kąpiele na zewnątrz, ale ryby
w nich mogą was przestraszyć. One jedzą martwą skórę, co łaskocze, ale
coś głęboko\dywiz osadzonego w~ludziach odrzuca bycie dla kogoś przekąską,
zatem odgońcie je, jeżeli nie chcecie skubania. Choć ja je lubię. Małe
ręczniki są ogólnego stosowania, trzymajcie je pod ręką, ale \textit{nie}
wyżymajcie ich do basenów.

-- To wszystko? -- spytał ten mądraliński.

-- To wszystko.

-- Co z~niegrzecznymi rzeczami?

Przewróciła oczami. 

-- Jeżeli poznasz kogoś swojej preferowanej płci i~chcesz coś zrobić, weź prysznic, ubierz się i~znajdź pokój. Nie robimy
niegrzecznych w~onsenie. Ściśle platonicznie.

-- Jeżeli tak mówisz.

-- Tak mówimy wszyscy.

-- Gdzie zostawiamy nasze rzeczy? -- To był Etcetera, a~jej psychiczna
ocena spadła o~oczko. Szleperzy i~ich \textit{rzeczy}.

-- Gdziekolwiek.

-- Czy to jest bezpieczne?

-- Nie wiem.

Świeżaki wymieniły łatwe do odczytania spojrzenia: \textit{To nie fair.
Jestem pewny, że to bezpiecznie, nie bądź taką turystką. To są wszystkie
nasze rzeczy. Nie zawstydzaj nas.}

-- Gotowi?

Poszli za nią. Wszyscy przebrali się w~suchym pokoju i~nie przejmowała
się subtelnością przy podglądaniu, taka była umowa, gdy byłaś
odchodzącą. Skóra to skóra, interesująca, ale wszyscy jej trochę mają.
Tych troje było młodych i~modnych, ale nie tak agresywnie, a~mądrala był
całkowicie wydepilowany, co było modą, kiedy odchodziła, ale od tego
czasu zanikło, biorąc pod uwagę bujne krzewy, które uprawiali pozostali.

Zabawną rzeczą o~niedbaniu było to, że jeżeli zostaniesz przyłapana na
zerkaniu, to wszyscy zaczynają się przyglądać, a~tych troje zerkało na
siebie, w~sposób, który upewnił ją, że nie byli związkiem seksualnym ze
sobą, jeszcze. Kolejną sprawą o~niedbaniu o~zerkanie było to, że łapiesz
innych ludzi na zerkaniu na Ciebie, co wszyscy troje zrobili w~odpowiedzi, a~ona wytrzymała w~odpowiedzi ich wzrok, szczerze i~aseksualnie. To był jej obowiązek wobec tych świeżaków, żeby pomóc im
odejść w~głowie, od kultu seksu i~niedostatku, w~którym dorastali, a~od
którego teraz się odwrócili.

Potrzebowała też zrobić to dla siebie. Wiedziała, że jest możliwe być w~obecności innych, nagich osób bez aspektu seksualnego, wiedziała, że te
rzeczy były odpowiedzialnością, nie towarem, wiedziała, że praca nie
jest zawodami, a~jednak ciągle potrzebowała przypominać o~tym jej
psyche. Zwyczaje łatwo nie umierają, były tak ściśle związane z~jej
lękiem i~lęk był najtrudniej zignorować. Zabieranie świeżaków do onsen
było terapią zajęciową dla jej własnego odejścia.

-- Chodźmy pod prysznic. -- Poprowadziła ich do sali mycia, udając, że nie
zauważa ich lękliwych spojrzeń na plecaki w~niestrzeżonym pokoju, które
nie były bardziej subtelne niż ich spojrzenia na jej nagą dupę, gdy
prowadziła ich naprzód.

Zaczęła od najgorętszego basenu, trik, żeby wyrwać jej umysł z~mięśni.
Gorąc uniemożliwiał myślenie, wszystko, co mogła robić, to \textit{być},
skłaniać każdy mięsień do rozluźnienia, oddychać parą o~zapachu
minerałów, aż stopiła się z~wodą, nogi, ramiona, pupa, plecy, pięty
stóp, dłonie, mięknące jak doskonałe barbecue, mięso prawie gotowe
odpaść od kości, rozluźnienie wspinające się po kręgosłupie. Panika
przed upałem przesączyła się z~mózgu, walcząc w~drobnych mięśniach szyi
i na potylicy, aż puściły, i~ostatni centymetr stresu, o~którym nie
wiedziała, że tam jest, puścił. Była wrażeniem, grą mięśni i~ciepła,
przyjemnością bilansowaną na ostrej krawędzi bólu. Zrelaksowała się
głębiej, jej mięśnie postawy, które trzymały ją w~Z, rozluźniły się, jej
tyłek unosił się nieco nad porowatym kamiennym stopniem i~nagłe
pojawienie się łaskawego odstępu pomiędzy ciałem a~nieustępliwą skałą
spowodowało głębsze rozluźnienie, zaczynając się od przecinających się
mięśni tyłka, pracując głębiej w~jej miednicy i~rdzeniu. Była tak
zrelaksowana, że jej brzuszek wybrzuszył się, gdy pas tkanek, które
opakowywał od żeber do bioder, puścił. Poczuła się jak mięso sous-vide,
włókna mięśni rozplątujące się, leżące pod nimi tkanki odrywające się od
torby elastycznej powięzi, które je otaczały. Wypuściła basowy jęk,
który zabrzęczał w~jej luźnych strunach głosowych. 

-- Gotuję się.

Ktoś był koło niej w~wodzie, prawdopodobnie Etcetera, biorąc pod uwagę
ilość wypchniętej wody. Jęczał, gdy walczył z~instynktami własnego
ciała, żeby uciekać z~bezlitosnego gorąca. Słuchała, jak jego oddech się
pogłębia, usłyszała westchnięcia, gdy się rozplątywał. Pojawiła się
sympatia pomiędzy ich ciałami, gdy fale przeniosły sygnały zrelaksowania
pomiędzy nimi.

Nie możesz zostać w~takim gorącu wiecznie, nieważne, jak bardzo je
lubisz. Limpopo została prawie do ostatniej chwili, potem wstała szybko,
zimne powietrze łaskoczące wszędzie, gdzie ją pocałowało. Sapnęła. Gorąc
wygotował całe zażenowanie. Mogła stać naga i~dyszeć na krawędzi
dymiącego basenu bez nawet odrobiny świadomości \textit{nie} bycia
skrępowaną. Ruszyła w~odmierzonych krokach, gładkie płyty zmysłowe pod
na wpóługotowanymi piętami stóp aż do krawędzi najzimniejszego basenu.
Zanurzyła w~nim wiaderko, potem użyła wiaderka do zmoczenia jej
ręcznika, wyciskając go na jej skórę, zaczynając od głowy i~prawie
dławiąc się, gdy lodowata woda spłynęła po jej ogolonej czaszce i~poza
jej uszami i~w jej oczy, nos i~usta.

Zanurzyła ręcznik, wyszorowała skórę, zaciskając szczękę, żeby nie
sapać. Zmusiła się samą do oczyszczenia skóry wodą, zanurzając ręcznik
raz za razem, biczując się zimnem, aż wiaderko było puste. Zastanowiła
się nad kolejnym wiaderkiem, czasem robiła dwa lub trzy, ale nie mogła
znieść myśli.

Weszła do najzimniejszego basenu, aż po kostki, zmusiła się do zejścia
po schodach, lekko podtrzymując się dłonią o~poręcz mimo pragnienia
zaciśnięcia ręki. Kolejny krok i~była w~wodzie po kostki, kolejny krok i~była aż po uda, a~woda dotarła do dołu jej tyłka i~sromu. Myśl o~kolejnym kroku była niemożliwa, żadna zdrowa na umyśle osoba nie
pogrążyłaby najdelikatniejszych części w~lodowym piekle. Wiedziała z~doświadczenia, że jeżeli się nie zmusi, to straci nerwy. Przesunęła wagę
do przodu, aż nie miała wyboru, tylko wpaść przodem w~wodę, głowa
zanurzyła się na chwilę, co sprawiło, że jej uszy natychmiast stały się
drętwe, skóra powiek i~czoła były jakby podciągnięte do linii włosów.

Odmówiła, żelazną wolą, pozwolenia sobie na westchnięcie. Zmusiła się do
pozostania w~tej karzącej wodzie przez jeden długi oddech, a~potem
wyszła równym krokiem. Powietrze, wcześniej chłodne, teraz było gorące.
Zabrała ręczniczek do najgorętszego basenu, napełniła świeże wiadro i~zaczęła proces na odwrót. Woda była parząca, piekąca, wrząca, ale
zmusiła się do zmycia nią, zanim zanurzyła się w~najgorętszym basenie.

Pięć minut później, pomyślała, że każdy mięsień uwolnił rezerwuary
napięcia. Tym razem, gdy gorąca woda ją gotowała, uczucie było
transcendentne. Zamknęła oczy i~nic nie widziała, żadnych migoczących
zmartwień, nic prócz zwierzęcej radości.

Wrażenie skończyło się przez zszokowany krzyk z~najzimniejszego basenu.
Odwróciła się spokojnie, ujrzała Etceterę w~zimnej wodzie, twarz z~otwartymi ustami, nozdrza tak szerokie, że wyglądał jak koń, prychał
nimi tak intensywnie jak parowóz. Na jego korzyść świadczy, że pozostał
przez pięć sekund i~wrócił do najgorętszego basenu odmierzonym krokiem.
Uśmiechnęła się leniwie, gdy zmył się swoim ręczniczkiem. Wszedł do
najgorętszego basenu i~ich oczy się spotkały.

Wytrzymała jego spojrzenie, gdy pozwalał ciepłu, mięśniom i~nerwom
tańczyć.

-- Och, \textit{jej!}.

-- Tak.

-- O!

Poczekała na niego przy kolejnym nurkowaniu w~zimnej wodzie, patrzyli
sobie w~oczy, gdy schodzili w~zimno, żartobliwe wyzwanie. Żadne z~nich
nie wydało dźwięku, nawet kiedy woda dotknęła jego moszny, choć lekko
podskoczył. Brodzili aż do ich szyi, potem, bez powiedzenia słowa,
zanurzyli głowy, wynurzyli. Żadne z~nich nie chciało być pierwszym
wychodzącym. Patrzyli, potem się gapili, póki nie wymamrotał ,,jesteś
szalona'' przez zaciśnięte zęby i~ruszył do schodków. Poszła za nim.
Miał słodki tyłek, zauważyła, w~najbardziej abstrakcyjny sposób.

Musiała przyznać, że to nie było wcale takie abstrakcyjne.

Z powrotem do gorącej, chichocząc, gdy w~ciszy ośmielali się wzajemnie
do chlustania parzącą wodą, do wejścia w~bąblujący gorąc, szybko tonąc.
Trzecie zatopienie w~gorącu zabrało ją do miejsc, o~których zapomniała,
wyprowadzając wszystkie świadome myśli, zamieniając ją w~termotropiczny
organizm, który reagował na pływy konwekcyjne przez procesy poniżej pnia
mózgu.

Raz jeszcze jej ciało powiedziało jej, że nie mogła już dłużej zostać w~tym gorącu. To był powrót do świadomości z~tego błogiego nie-miejsca,
oczy ledwo otwarte, potem szeroko, głowa podnosząca się z~wody. Dołączył
do niej chwilę później, dostatecznie długą, że właśnie chciał udowodnić
jakieś zdanie macho o~jego możliwości wytrzymania bólu. Wypędziła tę
myśl. Jeżeli była prawdziwa, tylko się ranił. Jego sprawa, nie jej.
Jeżeli nie była prawdziwa, ona była niepotrzebnie złośliwa.

Stali koło siebie koło basenu, stres wyciśnięty z~ich ciał, twarze
pokazujące nieświadomą rozkosz.

-- Teraz co? -- spytał.

-- Teraz pójdziemy do zwykłych basenów. -- Wskazała na inne baseny onsen,
gdzie kilkunastu kąpiących siedziało, po cichu rozmawiając, lub
medytując nad wewnętrzną stroną powiek. Jego przyjaciele siedzieli w~ciepłej, bąbelkowej kąpieli w~dziwnej odległości pomiędzy nimi.

Podeszli, i~jak zawsze w~łaźniach, Limpopo zauważyła, że bodźce
rozpuściły jakiekolwiek poczucie nagości. Nawet ich oczy na jej ciele
nie obudziły żadnego poczucia nagości. To był psychologiczny ekwiwalent
brzęczenia w~uszach, gdy długoszumiący agregat lodówki się wyłączy.
Podstawowy szum martwienia się swoim wyglądem, gdzie miała włosy, jak
jej włosy wyglądały, gdzie była gruba, gdzie jej kości wystawały, gdzie
jej skóra miała bruzdy od rozstępów, gdzie była pomarszczona od blizn
pooparzeniowych, wszystko przestało być ważne.

Wsunęła się do wody koło nowych. Patrząc z~tej strony kuracji
gorąc/zimno, byli skrzywieni latami w~default \dywiz rzeczywistości. Bycie w~sekcie pieniędzy i~statusu naznaczało cię. Nosili swoje znaki. Miała
nadzieję, że kiedyś wymaże swoje.

-- Możemy dołączyć?

-- Już dołączyliście -- powiedział ten sarkastyczny, dobrodusznie. Był
pomiędzy nią a~Etcetera, który przyszedł za nią do wody, i~Etcetera dał
mu braterskiego kuksańca w~żebra. Byli swobodni koło siebie jak bracia,
ale nie, różowe ramię przy brązowym, bezwłosa klatka koło grubej maty
Etcetery.

-- Herr von Kaczka -- powiedziała -- co powiesz na nasze skromne łaźnie?

-- Dekadenckie. -- Pociągnął nosem. -- Z~pewnością teren namnażania się
czegoś całkowicie przykrego.

-- Nie słuchaj go -- powiedziała dziewczyna. -- To niesamowite.

-- Musicie spróbować tego gorąco/zimno -- powiedział Etcetera. To
zmieniająco świadomość dobre.

-- Może później -- powiedział ten sarkastyczny.

-- Zdecydowanie później -- powiedziała dziewczyna. -- Skąd masz blizny?

Co było bardzo śmiałe z~jej strony i~w dobrym odchodnickim stylu, w~tym,
że naruszało wszystkie normy defaultu. Limpopo wysunęła tors z~wody i~wykręciła się, żeby spojrzeć na papraninę blizn po oparzeniach od żeber
do uda. Przebiegła palcami po nich, napięcie i~nieregularna powierzchnia
teraz zaledwie wrażenie, już nie strach.

-- Zdarzyło się niedługo po tym, jak pierwszy raz odeszłam. Zbudowaliśmy
domy z~ubitej ziemi na skarpie, dwadzieścia takich. Prawdziwe luksusy
dla uchodźców: prąd, woda, świeże hydroponiki i~miękkie łóżka. Zabierało
nam trzy godziny każdego dnia, żeby utrzymać całe miejsce w~ruchu.
Spędzałam resztę czasu na odtwarzaniu greckiej szkoły na otwartym
powietrzu, ucząc innych muzyki, fizyki, poezji w~czasie rzeczywistym. To
było słodkie. Pomagałam zbudować piec garncarski, zbudowaliśmy dziwne
koła, które robiły inteligentne, adaptacyjny, ekscentryczne obroty w~odpowiedzi na dłonie i~masę, że niemożliwe było zrobienie nieudanego
garnka.

-- Byliśmy na samym skraju domyślnej rzeczywistości, niedaleko granicy.
To było miłe, ponieważ przychodzili wycieczkowicze, z~którymi mogliśmy
rozmawiać, co się dzieje na świecie. Szczerze mówiąc, lubiłam bycie na
granicy, ponieważ to był właz ewakuacyjny. Gdyby rzeczy poszły źle,
mogłabym wrócić. Zadzwonić do mamy.

-- Wycieczkowicze nie zawsze byli mili. Była tam grupa facetów, straż
sąsiedzka, która pokazywała się, kiedykolwiek coś poszło źle w~ich
domach-fortecach. Ktoś został obrabowany: to musiał być odchodnik.
Graffiti? Musieli być odchodzące. Morderstwo? Jeden z~nas, niemożliwe,
żeby to był jeden z~tych cywilizowanych typów.

-- Jak na ludzi żyjących w~ciągłej inwigilacji, mieli bardzo dużo
przestępstw. Naruszenia własności to były ich dzieci, które odkryły
sposób, jak wyłączyć spyware tatusia, żeby mogły się czymś zająć. Jeżeli
myślicie, że drony powstrzymają nastolatków od seksu, to zwariowaliście.

-- Nie wiem, kto popełnił morderstwo. Słyszałam, że było okropne.
Podpalenie. Ktoś rozwalił cały blok domów, zrobił coś z~czujnikami
bezpieczeństwa plus benzyna i~\textit{wuuuf}. Ponad dwadzieścia ofiar, w~tym dzieci. W~tym niemowlę. Nie potrafię sobie wyobrazić kogoś robiącego
to i~wiem, że to nie był nikt z~naszej osady. Coś takiego, to musiało
być osobiste.

Troje patrzyli skupieni, wzrok przerażenia pojawiający się, gdy
rozumieli, gdzie zmierzała historia. Ale Etcetera, błogosławione jego
płatki uszu, powiedział: 

-- Może całkowicie socjopata. Wydarzenie ,,sześć
sigma'' w~kogoś neurotypowości. Nie mówię, że obcy robiący coś takiego
nie byłby całkowicie zjebany, ale nie odrzucaj od razu hipotezy strzelca
w szkole.

-- Zastanawiałam się nad tym. Myślałam, że mogli to być prowokatorzy z~powodu tego, co się wydarzyło. -- Przesunęła palcem po bliźnie. -- Te domy
z ubitej ziemi, były całkiem proste do sterowania. Standardowe wydanie
miało czujniki środowiskowe, zabezpieczenia i~alarmy. Użyli maszyn
budowlanych niedaleko obozu, żeby zepchnąć ziemię na fasadzie i~tylnej
drodze całego rzędu domów, przesuwając tony ziemi i~żwiru pod drzwi.
Przeszli wzdłuż całej linii, spokojni, wybijając okna i~wrzucając w~każde Mołotowa. Potem przeszli z~drugiej strony i~próbowali tego samego
z tylnymi oknami.

-- Ale te okna były nietłukące, co nas ocaliło. Strasznie się kłócili na
temat sposobu ich rozwalenia. Kiedy to się zdarzyło, byliśmy w~środku,
organizując się. Domy składały się z~dwóch pomieszczeń na górze i~dwóch
na dole, salon i~kuchnia na parterze, górny loft z~dwiema małymi
sypialniami i~toaletę. Były zbudowane, żeby być termostatyczne, chłodne
w lato, ciepłe w~zimie, kanały cyrkulacyjne wcięte w~każdą ścianę
łączącą, z~labiryntami antyhałasowymi, które wpuszczały powietrze, ale
wyciszały dźwięk.

-- Mój dom, dzieliłam go z~trójką, był na końcu, tam, gdzie kłócili się o~wybiciu okna. Wiedziałam, że muszę się wydostać, mieszkanie było pełne
dymu i~ognia. Byliśmy na górze, w~sypialniach, ponieważ był środek nocy.
To znaczyło, że nie byliśmy w~płomieniach, ale dym zbierał się na
piętrze. Mój przyjaciel wykopał ścianę i~byliśmy w~stanie przecisnąć się
do kolejnego domu, gdzie była pięć osób, ze ścianami rozebranymi
pomiędzy ich sypialniami, żeby zrobić jedną wielką sypialnię.
Panikowali, ponieważ jedno z~nich już straciło przytomność od dymu.
Chcieli spróbować ruszyć do drzwi. Uspokoiliśmy ich, wyjaśniliśmy co się
działo na zewnątrz, posłaliśmy przez ściany do kolejnego domu.

-- Musiałam podać informacje dalej, zachęcić ludzi do przechodzenia do
tego ostatniego miejsca, zatem cofnęłam się, powiadomiłam wszystkich,
wysysając kieszenie świeżego powietrza, aż stało się zbyt brudne, potem
poszłam za nimi. Kolejne miejsce było puste i~następne, ogień w~tym
mieszkaniu nie był straszny, zatrzymałam się, żeby jeszcze trochę
przekazać wiadomości.

-- Nie doceniłam dymu. Straciłam przytomność. Jedna z~moich przyjaciółek
odkryła, że mnie nie ma, wróciła, przepchnęła mnie przez trzy ścianki
działowe, aż byłam z~resztą grupy. Podzielili się na dwie grupy, jedna
na dole, żeby walczyć z~ogniem i~druga, próbująca przebić się przez
ścianę szczytową. Ubita ziemia była całkiem dobra na odbijaniu uderzeń,
ale mogłeś wydrapać i~ją wykopać, a~ja pomyślałam, że było dostatecznie
dużo osób pracujących nad tym, żeby wykonać pracę.

-- Zeszłam na dół, żeby gasić pożar. Ściany były nieprzenikliwe dla
płomieni, oczywiście, ale Mołotowy miały własne paliwo, było tam mnóstwo
papierowych mebli, plastikowych urządzeń kuchennych, które płonęły,
jeżeli podgrzały się dostatecznie mocno. Miałam mokrą szmatę dookoła
twarzy, ale wyschła i~ledwie mogłam widzieć lub oddychać. Nawet nie
zauważyłam, że moja koszula była w~ogniu, póki jedna z~innych kobiet w~mojej grupie nie przewróciła mnie i~przetoczyła po ziemi.

-- Do tego czasu wydrapali dość szeroką dziurę na piętrze i~wyrzucili
stos łóżek i~ubrań na ziemię na zewnątrz, więc zeskoczyliśmy na nie tak
szybko i~tak cicho, jak mogłyśmy.

-- Samosądnicy domyślili się, co się dzieje i~przyjechali nas dojechać.
Mieli mnóstwo macho quadów ATV gówna, plus drony. My mieliśmy ubrania na
plecach, a~niektórzy z~nas byli prawie nadzy. Rozproszyliśmy się.
Pozwoliłam kobiecie, która mnie ugasiła, poprowadzić mnie w~krzaki, na
błotnisty przepust, gdzie leżałyśmy tylko z~ustami i~nosem wystającymi z~błota, żeby nie pokazywać się w~podczerwieni. Musiałam wstać pierwsza,
całe ciepło z~ciała znikło, pojawiła się hipotermia. Wiedziałam, co to
było, wiedziałam, że wkrótce umrę, jeżeli się nie rozgrzeję.

-- Moja przyjaciółka próbowała mnie zatrzymać od pójścia, ale wiedziałam,
że mam rację. Cokolwiek się działo, umarłabym, gdybym się nie rozgrzała.
Wstałam. Drżałam, a~tutaj był ten ból\ldots  -- Prześledziła bliznę. -- Moja
przyjaciółka klęła na mnie aż do osady, przekonana, że zostaniemy
zastrzelone. Jednak przyszła. W~grupie bezpieczniej.

-- W~grupie bezpieczniej to potężna idea. Kiedy z~ociąganiem dotarłyśmy
do płonących ruin, prawie wszyscy tam byli. Odchodzący byli w~złym
stanie, zranieni, kaszlący i~wychłodzeni. Z~drugiej strony domów patrząc
na nas byli samosądnicy, wrodzy i~niepewni siebie. W~grupie wpadli w~to
szaleństwo, która pozwoliła im spalić domy sąsiadów. Byli tłumem, z~rozproszoną odpowiedzialnością, cała rzecz była emergentną cechą masy
społecznej, a~teraz znikała.

-- Moja grupa zorganizowała szpital, tam przed nimi, lecząc naszych
rannych tym, co mieliśmy. Byli ludzie, którzy się zranili przy
zeskakiwaniu, niektórzy się zranili w~gramoleniu się przez las. Dopóki
nie zaświtało i~nie przeliczyliśmy się czy sprawdziliśmy sieci, nie
odkryliśmy, że brakowało czterech osób. Dwoje z~nich dołączyło później.
Dwoje zostało znalezionych w~jednym z~domów, spalonych do kości,
pominięci w~kotłowaninie. Jedno ze zmarłych miało piętnaście lat i~nikt
nie wiedział, jak skontaktować się z~rodzicami, gdzieś tam w~defaulcie.

-- Słowo się rozeszło o~pożarze. Było dużo ruchu UAV, nie tylko śmigłowce
i szybowce, ale brzęczące zeppy z~pomocą medyczną i~żywnością. Wkrótce
było więcej osób, więcej odchodników i~pospolitacy się przestraszyli,
zaczęli się zbroić, budować wały, żeby obronić się przed odwetem.

-- Nie było zemsty. Pospolitacy ukradli nasz sprzęt budowlany dla swoich
wałów obronnych, ale pojawiły się nowe koparki w~ciągu kilku dni. Nie
wiem, kto je sprowadził. Leżałam z~gorączką, infekcja. Kiedy wróciłam do
przytomności, powiedzieli mi, że nie oczekiwali, żeby mi się udało.
Byłam zbyt słaba tygodniami. Dopiero kiedy mieliśmy działające
wet-drukarki, dostałam lek na infekcję, jakiś antybiotyk z~domieszką
srebra, który ją załatwił.

Słuchali w~skupieniu. Potem dziewczyna potrząsnęła głową, jakby w~jej
uchu była pszczoła. 

-- Czy ja to dobrze rozumiem? Zostaliście spaleni
przez obłąkanych mścicieli, którzy zabili Twoich przyjaciół i~prawie
Ciebie zabili, osobiście, a~Ty kręciłaś się w~pobliżu?

-- Nie kręciliśmy się w~pobliżu. -- Uśmiechnęła się na wspomnienie. -- Odbudowaliśmy. Pospolitacy obserwowali z~wałów, jak milicja, ale nie
walczyliśmy. Pierwsze, zbudowaliśmy kuchnię, potem piekliśmy, ponieważ
budowanie z~sypanej ziemi to duży wysiłek. Kiedy tylko ciastka czy
batoniki z~granoli wychodziły, zabieraliśmy tacę na wał pod białą flagą
i zostawialiśmy tam. Tace zbierały się, nietknięte, aż pewnego dnia ich
nie było. Nie wiem, czy je zjedli.

-- To było bardzo w~stylu Gandhiego, choć swędził mnie kark od myśli o~tych wszystkich lunetach wycelowanych we mnie. Włączali celowniki
laserowe na światło widzialne i~tańczyli kropkami na naszych czołach lub
nad naszymi sercami. Jednak kiedy opublikowaliśmy wideo, w~tym czerwonej
kropki nad piersią kobiety w~zaawansowanej ciąży, która przyszła pomóc,
pojawił się taki potop sieciowego gniewu, że samosądnicy przestali.

-- Rozebraliśmy stare domy, kiedy nowe były skończone. Żyliśmy w~hexajurtach i~namiotach, ponieważ nasze stare mieszkania były niezdatne
do zamieszkania. Istnienie ich tam, mumii naszych zmarłych, kazało nam
pracować i~zawstydzało straż. Kiedy stare domy były zniszczone,
zasadziliśmy dzikie kwiaty i~trawy, które byłyby piękne po wyrośnięciu.

-- Nowa osada była trzy razy większa. Wielu z~naszych ochotników chciało
zostać w~pobliżu, a~potem były nowe odchodzące, tak oburzone
samosądnikami, że opuścili miasto za strzeżoną bramą. Niektórzy byli
podwójnymi agentami, ale to dobrze, skoro nie mieliśmy żadnych tajemnic.
Tajemnice były tuż nad głową.

-- Gdy zbliżaliśmy się do wprowadzenia, atmosfera stała się świąteczna.
Były noce z~filmami na ścianach budynków, które pomalowaliśmy na biało.
Zawsze próbowaliśmy malować rzeczy na biało, żeby tylko dodać nieco dla
albedo planety. Zamieniliśmy wykopy w~miejsce do pływania z~wodą ze
strumienia. Maszyny ziemne zmieniły się w~huśtawki i~platformy do
skakania.

-- Byłam w~basenie, kiedy samosądnicy znowu ruszyli. ZHERFowali nasze
drony i~użyli promieni bólu i~dźwiękowych latarek, żeby zapędzić nas na
plac pomiędzy czterema rzędami domów. Potem facet z~odznaką
niby-wojskowej prywatnej ochrony użył megafonu, żeby ostrzec nas, że
został upoważniony przez hrabstwo do oczyszczenia ziemi i~mamy dziesięć
minut na opuszczenie. Potem była EULA, o~tym, że mogliby nas załatwić
zgodnie z~Ustawą o~Antyterroryźmie tego i~tego roku, jeżeli
uczestniliśmy w~czynnościach prawdopodobnie prowadzących do zagrożenia
życia lub własności. Gdy tylko skończył, podkręcił te jebane promienie
bólu. Nikt nawet nie myślał o~wracaniu po swoje rzeczy. To jakby twarz
się topiła. Były dzieci w~naszej grupie, poniżej dziesięciu lat, i~krzyczały, jakby były rozcinane na części. Słyszysz te historie o~rodzicach podnoszących samochód znad dzieci, ale to nic, widziałam
rodziców idących prosto w~promienie bólu do swoich dzieci. Jedno z~nich
upadło ogarnięte, a~jej partner podniósł ją w~chwycie strażackim z~dzieckiem pod drugim ramieniem. Nie sądzę, żebym kiedykolwiek widziała
bardziej imponujący wyczyn.

-- Nie mogliśmy schować się w~lasach. Mieli drony latające w~nakładającym
się wzorze, podążające za nami stadami, aż byliśmy dwadzieścia klików dalej.
Kulałam cały dzień, a~za każdym razem, gdy zwalniałam, małe helikoptery
spadały z~nieba i~taranowały mnie, popychając mnie jak bydło. Trzymałam
się ludzi, którzy nieśli dziecko. Zatrzymali się i~próbowali rozbić
obóz, ponieważ ich chłopczyk nie mógł zrobić kroku, a~żadne z~nas nie
miało już siły go nieść, trzymałam wartę nad nimi, bijąc w~drony
gałęzią. Coraz więcej tych małych gnojków opadało na nas i~w końcu
ruszyliśmy dalej. Wybłagali rower od postronnej pary na drodze i~wtedy
ja ich zwalniałam. Ruszyłam na własną rękę.

-- W~końcu odpadłam i~musiałam być poza zasięgiem śmigłowców, ponieważ
unosiły się na horyzoncie, hałasując jak świerszcze, ale nawet przy tym
zasnęłam, a~kiedy się obudziłam, już ich nie było. Musiały potrzebować
doładowania, a~tamci nie pomyśleli, że potrzebowałam dodatkowej
jednostki.

-- Co się zdarzyło dalej? -- spytała dziewczyna. Z~ich trojga, ona była
najbardziej przerażona. Limpopo podejrzewała, że to dlatego, że była
najbogatsza z~nich, ta, dla której to było najbardziej niewyobrażalne.

Limpopo wzruszyła ramionami, poczuła napięcie w~ramionach, zrozumiała,
że straciła spokój, przeżywając doświadczenie. Było to trzy lata temu i~nadal miała momenty szoku, ale nie przez jakiś czas. To opowiadania
sprawiło, że wróciło z~mocą. Tych troje przypomniało jej, kim była, nową
odchodniczką i~tak, nieco szleperką. Ogień i~przymusowy marsz wypaliły
instynkty szlepera, sprawiły, że zrozumiała bezużyteczność bycia
przywiązanym do \textit{gratów}.

-- Odeszłam. Do tego to się sprowadza. To wielki świat i~większość z~niego jest wymienna. Nieważne, gdzie jesteś i~co jest dookoła, jeżeli
możesz zaspokoić podstawowe potrzeby i~znaleźć coś produktywnego do
roboty. Skończyłam z~tą załogą, sforkowaliśmy gospodę z~projektów UNHCR
i tutaj mnie dzisiaj spotkaliście.

-- Co z~resztą Twojego starego obozu?

-- Tu i~tam. Niektórzy pracują w~Belce i~Brasach. Niektórzy poszli gdzieś
indziej. Para wypadła z~sieci odchodzących, podejrzewam, że wrócili, to
było zbyt wiele dla nich, co oznacza ,,kompletnie ich sprawa i~całkowicie w~porządku dla mnie'', jak idzie piosenka. Sama sprawdziłam
miejsce. Ma silnie strzeżony płot. Budynki zostały zrównane z~ziemią.
Moja łąka ciągle rośnie, a~kwiaty polne są tak piękne, jak marzyłam, że
będą. Sprawiłam, że świat był wymiernie lepszym miejscem, a~to więcej
niż możesz powiedzieć o~tych dupkach, którzy nas wygonili.

-- Amen -- powiedział Etcetera. -- To niesamowita historia i~cieszę się, że
ją nam opowiedziałaś. Teraz chcę spróbować innego basenu. Idziesz?

-- Cholera tak -- powiedział ten sarkastyczny. -- Czy mówiłaś, że jest tu
basen, gdzie ryby przychodzą i~dają przyjemność oralną?

-- Za mną -- powiedziała i~poprowadziła paradę kapiących, nagich ludzi na
zewnątrz w~chłód wczesnego wieczoru i~wspaniałe ciepło wody. Ryby
przypłynęły i~zjadły ich martwą skórę, gdy leżeli rozwaleni i~znowu
stali się stworzeniami czystych nerwów i~oddechu.

\chapter*{iii}

Ktoś powiedział, że whisky byłaby doskonała, ktoś inny powiedział, że
tost z~serem byłby niesamowity, ktoś powiedział, że ledwie mogła
utrzymać oczy otwarte, i~chciała znaleźć coś miękkiego do spania lub
cokolwiek poziomego. Limpopo zaproponowała koniec onsen. 

-- Znajdźmy coś do jedzenia w~nocy i~łóżko. -- Myślała o~poduszkach w~wielkim pokoju na
trzecim piętrze, idealnych dla przytulania, tego właśnie potrzebowała w~tej chwili.

Wzięli prysznic we wspólnym przedpokoju, pływająco zrelaksowani. Bez
mówienia słowa, bez bycia otwarcie seksualnymi, umyli sobie plecy.
Seksualne, czy nie, to była zwierzęca przyjemność w~dbaniu przez kogoś
innego, pogłębiona uczuciem słodkiej dekadencji.

Byli tak spokojni, że pięć minut zabrało im zrozumienie, że ich rzeczy
świeżaków zostały skradzione.

Przedtem to było tylko chwiejne chwianie, gdy szukali swoich ubrań.
Potem narastający alarm i~w końcu dziewczyna powiedziała: 

-- Zostaliśmy obrabowani. 

-- Gówno -- powiedzieli chłopcy. 

Wszyscy spojrzeli na Limpopo.
Jej ubrania była dokładnie tam, gdzie je zostawiła. Był to rodzaj ubrań,
które mogłaś dostać wszędzie, gdzie odchodzące się zbierały.

Limpopo nabrała powietrza. 

-- Cóż, to się zdarza.

-- No weź. Musimy poszukać naszych rzeczy\ldots  -- powiedziała dziewczyna.

-- Będziecie potrzebować ubrań -- powiedziała Limpopo. -- Nie chcę tego
mówić, ale myślę, że to strata czasu. Kiedy rzeczy są kradzione, znikają
szybko.

-- Śmieszne, skąd mogłabyś wiedzieć -- powiedziała dziewczyna. -- Śmieszne,
skąd wiedziałabyś, dlaczego nie warto próbować wyśledzić rzeczy, które
kazałaś nam zostawić tutaj.

-- Nigdy nie kazałam wam ich tutaj zostawić -- powiedziała Limpopo. -- Tylko powiedziałam, że nie możecie ich zabrać do środka. Konkretnie
powiedziałam, że nie wiem, czy tutaj będą bezpieczne. -- Spojrzała na
nich. Byli zdenerwowani, podejrzliwi wobec niej. Dziewczyna
najbardziej, ale faceci wyglądali, jakby ona była też winna.
Potrzebowali kogoś obwinić, ponieważ alternatywą było obwinienie siebie
samych. Limpopo poczuła smutek. Naprawdę nie mogła się doczekać tych
przytulanek.

-- Wiem, że to kijowo. Zdarza się tutaj. Nie wszyscy są fajnymi osobami
na świecie.

-- Zatem dlaczego nie zbudowaliście szafek? -- spytała dziewczyna. -- Jeżeli nie wszyscy są tak mili jak Ty, dlaczego nie zapewnić gościom
minimum standardów bezpieczeństwa? Co z~nagraniami? Są tutaj kamery,
racja? Zróbmy trochę kurwa śledztwa, zróbmy plakaty poszukiwanych\ldots 

Limpopo pokręciła głową, a~dziewczyna wyglądała na bardziej wkurwioną. 

-- Przepraszam -- powiedziała znowu Limpopo. -- Są czujniki w~B\&B,
oczywiście, ale nic, co przechowuje dane dłużej niż kilka sekund. To
jest w~firmwarze budynku, a~każdy, kto próbuje to zmienić, będzie
cofnięty w~milisekundy. Ludzie, którzy używając tego miejsca,
zdecydowali, że wolą być okradzeni niż inwigilowani. Rzeczy to tylko
rzeczy, ale bycie nagrywanym cały czas jest przerażające. Co do szafek,
wolno Ci jakieś ustawić, ale nie sądzę, że przetrwają długo. Kiedy już
masz szafki, niejawnie mówisz, że wszystko poza szafkami jest
,,niechronione''\ldots 

-- I~takie było -- zauważył Etcetera.

-- Dobra -- powiedziała. -- Dobrze, macie rację. Jednak tym nie wygracie
dyskusji.

Etcetera usiadł. Wszyscy byli nadzy, ale Limpopo poczuła się źle z~ubieraniem się, gdy nikt inny nie miał ich. Złapała wielkie, puszyste
ręczniki ze stosu i~rozdała je.

-- Dziękuję -- powiedział Etcetera.

-- Ta, dzięki -- powiedział sarkastyczny. -- Brzmi, jakby Twoi przyjaciele
nie byli przekonani do niczego. Co by się stało, gdybyśmy my zabrali ich
rzeczy?

Uśmiechnęła się. 

-- To byłoby to, co chciałam zasugerować. Nikt nie jest
zadowolony z~tego powodu. Kradzież jest kijowa, a~ktokolwiek to zrobił,
jest ogromnym dupkiem. Jeżeli złapiemy kogoś na tym, prawdopodobnie go
wyrzucimy.

-- Co jeżeli próbowałby wrócić?

-- Powiedzielibyśmy mu, żeby się wyniósł.

-- Co gdyby nie posłuchał?

-- Zaczęlibyśmy go ignorować.

-- Co jeżeli przyprowadziłby grupę przyjaciół i~zaczął rozpieprzać wasze
rzeczy? Sikał do wanny, wypił cały alkohol?

Odwróciła się do Etcetera. 

-- Znasz odpowiedź, prawda?

-- Oni by odeszli, Seth -- powiedział.

-- To moje imię niewolnika -- powiedział Seth. -- Nazywaj mnie, hm\ldots  -- Wyglądał na zagubionego.

-- Gizmo von Kaczka -- powiedziała Limpopo. -- Jestem dobra w~zarządzaniu
nazwami.

-- Mów mi Gizmo -- powiedział. -- Tak, łapię to. Oni by odeszli. Zbudowali
kolejny taki gdzieś indziej, a~potem ktoś znowu by przyszedł i~zajął,
lub spalił, lub cokolwiek.

-- Lub nie -- powiedziała. -- Słuchaj, jest tyle filozofii odchodzenia ilu
odchodzących, ale moja jest taka ,,historie, które opowiadasz, spełniają
się''. Jeżeli wierzysz, że wszyscy są niegodni zaufania, budujesz to w~system tak, że nawet najlepsi ludzie muszą się zachowywać jak najgorsi,
żeby cokolwiek zrobić. Jeżeli założycie, że ludzie są w~porządku,
będziecie żyli znacznie szczęśliwsi.

-- Ale nasze rzeczy zostały skradzione.

-- Nie mam nic przeciwko byciu okradzionym. Ułatwia to życie. Od lat nie
nosiłam plecaka. Spacery są \textit{znacznie} bardziej przyjemne. Nikt się
nie przejmuje rabowaniem mnie.

-- Miałam wszystko w~tej torbie -- powiedziała dziewczyna, ponuro.

-- Spróbuję zgadnąć -- powiedziała Limpopo. -- Pieniądze. Dowód osobisty.
Jedzenie. Wodę. Zapasowe ubrania. Czystą bieliznę.

Dziewczyna pokiwała głową.

-- Racja. Cóż, nie potrzebujecie pieniędzy ani dowodów tutaj. Jedzenie i~wodę, mamy. Czysta bielizna i~ubrania, proste. Możemy podłączyć się do
sieci, możecie odzyskać swoje backupy\ldots  -- Zobaczyła, jak im miny
rzedną.

-- Mieliście kopie w~odchodnickiej sieci, tak?

-- Jeszcze nie -- powiedział Etcetera. -- To było na liście. Chyba ciągle
mam rzeczy w~chmurze, tam, w~,,default-rzeczywistości'' -- Ciągle mówił
,,rzeczywistość default'' ze świadomym, słyszalnym cudzysłowem.

-- Dobra, możemy Wam to eksfiltrować. Istnieją jeszcze miejsca, gdzie
sieci odchodników zaglądają do default, zaawansowane tunele i~duże
opóźnienia. Lub możesz wrócić, jeżeli chcesz. Niektórzy ludzie tak
robią. Odchodzenie nie jest dla wszystkich. Czasem znowu odchodzą. Nikt
Was nie będzie za to oceniał. -- \textit{Prócz was samych}, nie powiedziała
tego, ponieważ to było oczywiste.

Dziewczyna wyglądała na strapioną. 

-- Nie mogą w~to kurwa uwierzyć. Nie
mogę uwierzyć, że nie przyjmujesz żadnej odpowiedzialności.
Przyprowadziłaś nas tutaj. Jesteśmy w~ciemnej dupie, nie mamy nic, a~Ty
tylko walisz zadowolone cygańskie aforyzmy jak jakiś budda dla
hipsterów.

Limpopo pamiętała, kiedy coś takiego by ją wkurwiło i~pozwoliła sobie na
dumę, że nie była rozzłoszczona. Żałowała, że nie mogła pozbyć się dumy,
ale wszystko jest pracą w~toku. 

-- Przykro mi się, że to się zdarzyło.
Pomogę wam się zorganizować. Bycie okantowanym zdarza się wszystkim,
którzy odchodzą. To rytuał przejścia. Posiadanie czego, co nie jest
wymienne, oznacza, że musisz się upewnić, że nikt inny tego nie
zabierze. Kiedy odpuścisz to sobie, wszystko staje się prostsze.

Dziewczyna wyglądała, że ruszy na Limpopo. Miała nadzieję, że to nie
będzie fizyczne.

-- Słuchaj, spokojnie. To tylko \textit{graty}. Wiem, że miałaś fajne
rzeczy. Nawet ukradkiem zrobiłam ich zdjęcia tak, żeby móc zrobić własną
wersję i~wrzucić ją na serwer wersji w~Gospodzie. Możesz siedzieć i~się
wściekać, możesz pobiec w~noc, szukając jakiegoś dupka, który jest
bardziej uzależniony od posiadania rzeczy niż Ty, lub możesz pozwolić
temu minąć, pójść ze mną i~dostać nowy zestaw. Możemy wyprodukować
duplikat rzeczy, które miałaś, lub możesz wybrać coś z~katalogu. Lub
możesz pobiec do domu w~ręczniku. Całkowicie zależy od Ciebie.

-- Skopiowałaś jej ubrania? -- powiedział sarkastyczny.

-- A co, chcesz kopię? Były unisex. Moglibyśmy je poprawić dla Ciebie,
lub mógłbyś znaleźć coś genderpłynnego. Myślę, żeby Ci pasowało. -- Teraz
gdy to powiedziała, zrozumiała, że to była prawda. Lubiła tego drugiego,
Etceterę, bardziej jako osobę, ale to Herr Von Spodnie był piękny w~sposób, do którego miała słabość, mogła dostrzec korzyść z~grania w~przebieranki z~nim, jeżeli tylko przestałby gadać.

-- Wiesz? Może -- powiedział. Wiedział dokładnie, jak był piękny, co było
olbrzymim zjazdem.

-- Chodźmy i~ubierzmy was.

Z solidarności zostawiła swoje ubrania na ławce i~zabrała tylko ręcznik
z onsen, tak jak oni, potem poprowadziła ich z~powrotem do Belek i~Brasów.

\threeast

Fablab B\&B był w~zewnętrznym budynku nazywanym stajniami, ale bydło
nigdy nie było w~pobliżu niego. Znalazła im szaty i~pantofle, pokazując
świeżakom, jak przeszukiwać spis Gospody za lokalizacjami nieodebranych
rzeczy i~prowadząc ich przez pierwsze kilka pięter, żeby przejrzeć
alkowy i~skrzynie, póki nie znaleźli zestawu. 

-- Możecie je zatrzymać -- powiedziała -- lub po prostu odłożyć do dowolnej skrzyni i~powiedzieć o~tym gospodzie. Jeżeli je gdzieś porzucicie, ktoś je w~końcu zmami za
was, ale jest to uważane za niegrzeczne.

-- Mami? -- spytał Etcetera. Rozjaśnił się w~trakcie polowania na szaty.
Zaczynał łapać ducha. Cieszyła się z~tego.

-- MaMi, Materia nie na Miejscu. Śmieci. Jeżeli widzicie bałagan, możecie
go zrecyklować, wrzucić do kosza w~magazynie lub zarekwirować. Gospoda
śledzi nieodebrane mami w~magazynach i~zaznacza rzeczy, które były
starsze niż kilka miesięcy dla procesu błędów, ktoś podejmuje zadanie i~je rozkłada.

-- Czy nasze rzeczy nie zostały zmamione?

-- Nie ma szans. Nie były tutaj dostatecznie długo, a~torby w~przebieralniach nie są mamione, chyba że zostały porzucone. One zostały
po prostu skradzione. -- Otworzyła drzwi do stajni. -- Odpuśćcie sobie.

Fablab pachniał jak lasery, zwęglone drewno, lotne substancje
organiczne, olej maszynowy i~farby do tkanin. Ogniwa wodorowe, oddzielne
od ogniw gospody, były uzupełnione, a~pomieszczenie było prawie puste,
prócz chichoczących nastolatków prawie na pewno drukujących śmieszne
bronie ręczne. Zaznaczyła ich do poważnej rozmowy, zanim rzuciła ekran
na ścianę.

-- Najprostszy sposób rozpoczęcia to zapytanie o~spis rzeczy podróżnych,
ciepła pogoda, zimna pogoda, mokro, schronienie, jedzenie, pierwsza
pomoc, z~odsyłaczami do posiadanych zapasów plus posortowane po
popularności. -- Dotykała powierzchni interfejsu, gdy pracowała i~wkrótce
miała wielokolumnowy projekt. -- Napełnijcie swoje kosze, a~kiedy
skończycie, poprawcie rozmiary i~dodajcie opcje.

Natychmiast to złapali, stukali, wciskali i~sugerowali. Obserwowała,
ważąc ich wybory wobec swoich kryteriów. Kiedy była szleperem, miała
mentalność Armii Jednego, wszystko, co potrzebowała przy sobie. Kiedy
przerwała to szaleństwo, oddzielała ten codzienny ładunek, aż miała
minimum tego, co potrzebowała, żeby przetrwać typowy zestaw trudności
pomiędzy tym, gdzie była, a~kolejnym miejscem. Kiedy żyła w~defaulcie,
traktowała dom, szafkę w~szkole i~miejsce pracy, jak przedłużenie jej
codziennego magazynu, nie martwiąc się o~przewożenie wszystkiego, co
było w~tych miejscach ze sobą cały czas. Wiedza, że były tam, kiedy ich
potrzebowała, była wystarczająca.

Powód zostania szleperem, kiedy odeszła, był taki, że narysowała granicę
dookoła jej ciała. Jeżeli nie nosiła rzeczy, nie mogła jej użyć. Lekiem
było zrozumienie, że wszystko jest wszędzie, rzeczy wśród odchodników
były znormalizowaną chmurą potencjalnych rzeczy na żądanie. Koszt
możliwości nieposiadania właściwego widelca do sałatki, kiedy chciała
sałatkę, był mniejszy niż koszt możliwości niepójścia tam, gdzie chciała
iść bez konieczności targania góry rzeczy wbijających się w~plecy.

A priori, założyłaby się, że Etcetera miałby najmniejszy koszyk, a~dziewczyna miałaby największy. Zgadła źle. Dziewczyna miała tak
minimalny, że zawstydziła Limpopo.

-- Nie sądzisz, że powinnaś spakować więcej niż to? -- Poddała się
pragnieniu, żeby wpłynąć na rzeczywistość.

-- Wszystko, co potrzebuję, jest wystarczające, żebym dotarła do miejsca
jak to. W~międzyczasie, te durnie będą nosili górę. Zawsze mogę pożyczyć
coś od kogoś, a~z~drugiej strony prawdopodobnie skończę, pomagając im z~ich zlewami kuchennymi.

Dziewczyna uniosła ekspresywnie brew na nią i~się uśmiechnęła. 

-- Myślisz, że Ty jesteś jedyną tutaj, która łapie te rzeczy. Jesteśmy
świeżakami, nie idiotami. Latami organizowałam imprezy komunistyczne.
Wyzwoliłam tyle materiałów, że mogłabym umeblować całe to
przedsięwzięcie. Tak, zabrałam zbyt dużo rzeczy, kiedy odchodziłam, ale
to tylko dlatego, że nie wiedziałam, w~co się ładuję. Jeżeli to jest jak
to\ldots  -- Pomachała dłonią na stajnie -- kto tego potrzebuje?

-- Masz rację, założyłam, że jesteś burżujką, którą trzeba poprowadzić ku
większej chwale filozofii odchodzenia. Łatwo poczuć, że więcej mniej
znaczy więcej niż Ty. Przykro mi z~powodu waszych gratów. Mimo tego, że
sądzę, że mieliście więcej gratów niż potrzeba, bycie obrabowanym brzmi
okropnie. Sprawia, że czujesz się niepewna, nikt nie może zachowywać się
super, jeżeli czuje się w~ten sposób. -- Jeden z~punktów odchodzenia-fu
było szybkie i~gruntowne przeproszenie, gdy spieprzyłaś. To była trudna
lekcja do nauczenie przez Limpopo, ale w~większości ją zrealizowała.

Chłopcy ukradkiem usuwali pozycje z~ich koszyków, a~ona zauważyła, że
dziewczyna zauważyła, uśmiechnęły się do siebie tajemniczo i~udawały, że
nie zauważają. Sprawianie, że inni ludzie czują się jak dupki, było
okropnym sposobem na to, żeby przestali się zachowywać jak dupki.

-- Nie każde miejsce jest takie jak to -- powiedziała Limpopo. -- B\&B jest
największym miejscem dla odchodzących, jakie znam, może największym w~tej części Kanady. Jest bogate w~surowce. Większość osad odchodników ma
fablaby. Nikt nigdy nie powie, że nie wolno Ci korzystać, ale jeżeli
tylko dryfujesz dookoła, zużywasz ogniwa wodorowe i~surowce, wszyscy
będą myśleć, że jesteś chujkiem.

Chłopcy poprawili ich koszyki.

-- Nie powinnam wymieniać się niczym za
nic innego, to wszystko dar, jak na imprezach komunistycznych. Tę część
rozumiem. Jednak kiedy robimy nasze imprezy, nie dbamy, jak dużo
bierzesz, ponieważ w~każdej sekundzie gliny mogą nas ścigać i~zniszczyć
wszystko, co zostawimy, zatem możesz wziąć tyle, ile uniesiesz. Tutaj,
chcesz, żeby ludzie magicznie nie brali za dużo, ale również, aby nie
zasługiwali na prawo do zabrania więcej przez ciężką pracę i~również,
żeby pracowali, ponieważ to dar, ale nie dlatego, że oczekują czegoś w~zamian?

Gapili się na nią. Wzruszyła ramionami. 

-- To dylemat odchodzących.
Jeżeli bierzesz bez dawania, jesteś sępem. Jeżeli rozliczasz wszystkich
innych, co biorą i~dają, jesteś dziwacznym protokolantem. To nasza
wersja winy chrześcijańskiej, jest bezbożnym czuć się dobrze z~powodu
naszej bogobojności. Musisz chcieć być dobry, ale nie czuć się dobrze co
do tego, jak dobry jesteś. Najgorszy przypadek to martwienie się o~to,
co ktoś inny robi, ponieważ to nie ma nic wspólnego z~tym, czy Ty
działasz dobrze. -- Wzruszyła ramionami. -- Gdyby to było proste, wszyscy
by to robili. To projekt, a~nie osiągnięcie.

Etcetera rozciągnął się i~jego plecy zaskrzypiały. Jego szata się
otworzyła, co było odkrywcze w~sposób, jaki jego całkowita nagość nie
była. Zebrał wszystko z~powrotem. 

-- Trudno to wszystko zrozumieć,
ponieważ to jest obce. Tam, w~,,default-rzeczywistości'' -- znowu mogła
usłyszeć cudzysłów -- powinieneś robić rzeczy, ponieważ są dla Ciebie
dobre. ,,Czego spodziewasz się po mnie, odrzucić te brudne pieniądze z~etatu, ponieważ w~ich historii było coś brzydkiego? Nie widzę, żebyś
ustawiała się w~kolejce, żeby zapłacić moje rachunki.'' Hojność jest
bajką o~tym, co się trafia, kiedy ludzie dbają o~siebie. Powinniśmy ,,po
prostu'' wiedzieć, że egoizm jest naturalny.

-- Tutaj powinniśmy traktować hojność jako stan podstawowy. Dziwne,
obrzydliwe, samolubne uczucia to ostrzeżenie, że jesteśmy chujkami. Nie
powinniśmy przebaczać ludziom za egoizm. Nie powinniśmy oczekiwać, że
inni ludzie wybaczą nam nasze samolubstwo. To nie hojność robienie
rzeczy w~nadziei, że coś dostanę. Trudno nie wpaść w~taki mechanizm,
ponieważ łapówki działają.

-- Moi starzy mieli ten problem cały czas, gdy dorastałem. Tata
przyszedłby z~tymi długimi wyjaśnieniami, dlaczego mogłem robić coś, co
chciałem, tylko jeżeli najpierw zrobiłem coś nudnego, żeby nie wyglądało
to na łapówkę. Powiedziałbym ,,Musisz utrzymywać zbilansowaną dietę,
żebyś był zdrowy. Jedzenie deseru bez jedzenia warzyw i~protein nie jest
zbilansowane. Zatem nie dostaniesz deseru, póki nie zjesz''. Mama
przewracała oczami i~kiedy nie mógł usłyszeć, wyszeptałaby ,,Skończ to,
co masz na talerzu, a~dam Ci kawałek ciasta''. I~tak i~tak łapówka.

Sarkastyczny zachichotał. 

-- Poznałem Twoich starych. Oboje przekupywali,
ale tata próbował poczuć się z~tym lepiej.

Etcetera pokręcił głową. 

-- To bardziej skomplikowane. Tata chciał, żebym
chciał robić właściwe rzeczy z~właściwego powodu. Mama tylko chciała,
żebym robił właściwe rzeczy. Rozumiem tatę. Jednak łatwiej zmusić ludzi
do robienia rzeczy, jeżeli nie dbasz, dlaczego to robią.

Limpopo przejrzała koszyki chłopców, przycięła je do skromniejszych
proporcji. Skinęła głową. 

-- Ta dyskusji zwykle wraca do rodzicielstwa i~przyjaźni. To są miejsca, gdzie wszyscy się zgadzają, że bycie hojnym
jest dobre. Waszym obowiązkiem jest, żeby upewnić się, że wszystko jest
zrobione. Dzieciak, który spędza czas, obserwując siostry, czy mają tyle
samo obowiązków albo zostanie oszukany, albo sam oszukuje. Brzmi to
staromodnie, ale bycie odchodzącym jest ostatecznie o~traktowaniu
wszystkich jak rodziny.

Dziewczyna wzdrygnęła. Limpopo pomyślała, że ma jej numer. 

-- Ok,
traktowanie wszystkich jakbyś chciała, żeby Twoja rodzina traktowała
Ciebie.

-- Zasadniczo, chrześcijaństwo -- powiedział sarkastyczny, robiąc krzyż ze
swojego ciała, opuszczając głową na jedną stronę i~przewracając oczami
do góry.

-- Chrześcijaństwo, jeżeli powstałoby w~dostatku materialnym -- powiedziała Limpopo. -- Nie jesteś pierwszym, który porównuje. Wiele z~tych miejsc ma studentów, politologia, socjologia, antropologia,
próbujących zrozumieć, czy jesteśmy ,,fabiańskimi socjalistami
post-niedoboru'' czy ,,świeckimi komunistami chrześcijańskimi'' czy co.
Większość jest opłacana przez prywatny wywiad, który chce wiedzieć, czy
spalimy ich biura, i~czy mogą nam cokolwiek sprzedać. Jedna trzecia z~nich dołącza. W~międzyczasie, jesteśmy gotowi na pomiary i~style,
prawda?

Zrobili to, pozwalając kamerom stajni zrobić ich obraz, potem sprawdzili
poprawność geometrii, które wywiodły algorytmy. System wyrenderował ich
w nowych ubraniach i~pozwolił pobawić się kolorami i~nadrukami. Miałeś
to w~defaulcie, transy klikania konsumeryzmu w~wiecznym kupowanie i~widzieli to jasno. Szybko przebiegli przez opcje i~nacisnęli
,,zatwierdź'', zdumieli się czasem wykonania.

-- Sześć godzin? -- powiedziała dziewczyna. -- Naprawdę?

-- Możecie zrobić to w~mniej -- powiedziała Limpopo -- ale to tempo pozwala
nam użyć bardziej zanieczyszczonych surowców poprzez dodanie procedur
korygowania błędów. Spójrzcie na to\ldots  -- Wystawiła jej rękaw i~pokazała
im miejsce, gdzie szew został drugi raz przeszyty podczas fabrykacji. -- Nikt nie powiedział, że dostatek jest łatwy.

\chapter*{iv}

Kiedy Etcetera w~końcu uderzył do niej, zaskoczyła samą siebie zgodą.

Troje z~nich trzymało się B\&B także później, gdy dostali wszystko, co
potrzebowali, żeby pójść dalej. To jej nie zaskoczyło. Pasowali do
siebie. Ten sarkastyczny -- ciągle podtrzymywał ksywę Gizmo von Kaczka i~wszyscy nazywali go ,,Kaczunia'' -- świetnie opowiadał i~był śmiesznym
przeciwnikiem w~grach planszowych. Obie te cechy były wysoko cenionymi
umiejętnościami we wspólnym domu B\&B, a~on stał się osprzętem.
Dziewczyna dołączyła do zespołu badawczego, który wyszukiwał miejsca z~surowcami oznaczonymi przez stado dronów. Wracała po ciężkim dniu w~opuszczonym mieście, umorusana i~niezmordowana w~tank topie, butach
roboczych, prowadząc szereg piechurów, którzy wpadali do stajni z~ładunkiem tkanin, metali, plastiku, smutne pozostałości upadku przemysłu
i ludzi, którzy dla niego niewolniczo pracowali.

Niemniej jednak Etcetera nie pasował, nieważne co próbował. Żadna z~prac
go nie urzekła. Żadna z~rozrywek go nie zainteresowała. Nie miał stosu
książek, które planował przeczytać, umiejętności, które planował
wyćwiczyć, projektu, który odraczał. Był albo leniwym przegranym, albo
mistrzem zen.

Przynajmniej nie był pasożytem. Robił codzienne prace, sprawdzał
wszystko w~Stajniach i~robił konserwację, śmiał się z~żartów Kaczuni,
wychodził w~grupie z~dziewczyną -- nazywał ją Natalie, choć zmieniła imię
z ,,Stabilne Strategie'' na ,,Lodołasica''. Ale wyraźnie nie przejmował
się tym wszystkim.

Pewnego ranka, zeszła do onsen i~tam go spotkała, leżącego w~zewnętrznym
basenie z~nosem i~ustami nad wodą, pióropusze pary unoszące się, gdy
wydychał. Wsunęła się do wody koło niego, chcąc zabrać stopy z~lodowych
kamieni chodnika i~w ciepło. Podniósł głowę, otworzył lekko oko,
delikatnie skinął i~znowu zatonął. Skinęła na jego parę, też się
położyła. W~ciągu chwili ryby były przy niej, skubiąc tu i~tam. Zamknęła
oczy i~pozwoliła twarzy zatonąć pod wodą, póki tylko jej usta i~nos nie
wystawały.

Ryba otarła się o~jej dłoń, potem zrobiła to znowu. To nie była ryba. To
była jego dłoń, zwyczajnie położona koło niej, różowa krawędź koło
różowej krawędzi. Sprawdziła swoje wewnętrzne mierniki i~zdecydowała, że
była szczęśliwa. Podniosła dłoń i~umieściła ją na jego.

Przez chwilę leżeli nieruchomo, ryby ich łaskotały. Przez ryby stało się
to dziwne. Ona i~Etcetera byli główną atrakcją orgii kogoś innego, ich
kontakt święty w~swojej cnotliwości. Ich palce poruszały się małymi
ruchami, rozkładając, splatając. To mogło zabrać trzydzieści minut.
Dłonie mówiły: ,,Czy to ok?'' i~czekały na ruch drugiej: ,,Tak,
dobrze'', zanim poruszyły się znowu. Wysyłali pulsujące SYN/ACK/SYNACK
przez krnąbrną sieć.

Kiedy ich dłonie się splotły, było to rozczarowujące. \textit{Teraz co?}.
Niepewny kontakt fizyczny pod wodą był magiczny, ale nie zamierzali
zrobić sobie dobrze ręką w~basenie. Och, Etcetera, to był romantyczny
gest, ale teraz co?

Zmęczyła się zastanawianiem i~puściła dłoń, poszła do środka. Rzadko
była na nogach tak wcześnie, ale kiedy była, lubiła przyjść do onsen,
ponieważ miała ją dla siebie. Była pusta. Stała przy najgorętszym
basenie, schłodzona spacerem przez mroźne powietrze do parujących drzwi.
Drzwi za nią się otworzyły i~Etcetera wszedł, uśmiechając się w~zamyśleniu. Zanurzył wiadro prawie-wrzącej wody, namoczył ręczniczek,
potem wyjął go w~chmurze pary.

Uśmiechnęła się, lubiąc kierunek, w~jakim to zmierzało. Odwróciła się
plecami, spojrzała nad ramieniem, zapraszając go przechyleniem głowy.
Wystarczyło. Wtarł nieśmiało prawie-parzący ręcznik w~jej plecy, a~ona
przysunęła się do niego. Potarł mocniej, namoczył ręcznik. Przyklęknął,
żeby zrobić jej pupę i~nogi, a~ona odwróciła się, kiedy dotarł do kostek
i zaczął pracę w~górę. Gdy wstał, czekała na niego ze swoim ręcznikiem,
parującym z~wiadra, wytarła jego pierś i~ramiona. Razem trzymali dłonie
i weszli do najgorętszego basenu, woda tak gorąca, że zatarła wszystkie
myśli, prócz dłoni ściśniętej w~jej dłoni. Zanurzyli się, ręce tak
ciasno, że ich knykcie bolały. Ręka w~rękę, poszli do najzimniejszego
basenu, wzięli ręczniki i~umyli się jedno drugie.

Tam i~z powrotem, jego lewa dłoń w~jej prawej, myjąc jedno drugie, lgnąc
ciasno do siebie, sami w~onsenie, stapiając się w~jeden byt ciała,
nerwów, gorąca i~zimna. Kiedy skończyli, usiedli w~prysznicach i~namydlili się, spryskali się słuchawkami prysznicu. Poszli do
przebieralni, założyli szaty, oddzielając się na krótko. Kiedy to
zrobili, poczuła ducha jego dłoni w~swojej. Kiedy znowu złapali dłonie,
to jakby wróciło coś brakującego.

Dłoń w~dłoń, poszli ciemnymi korytarzami. Ominęli wspólny pokój i~wycieńczone głosy, które usłyszeli ponad bulgotem coffium. Powoli
wchodzili schodami, krok dopasowany, stopy szurające na ziarnistym
laminacie na stopniach. Na pierwszym podejście, użyła wolnej dłoni, żeby
zapytać się powierzchni dotykowej o~wolne pokoje, znalazła jeden na
najwyższym czwartym piętrze, który miał najmniejsze pokoje, prawie
trumny.

Bez słowa, oddychając ciężko, weszli, słysząc budynek budzący się wokół
nich: płaczące dziecko, ktoś sikający, prysznic. Kolejne piętro, kilka
zręcznych zwrotów przez kręty mały labirynt czwartego piętra, położył
dłoń na tabliczce przy drzwiach i~drzwi się odsunęły. Włączyły się
światła, pokazując pustą komórkę, gdzie łóżko na antresoli było gładko
zasłane świeżą pościelą. Poniżej łóżka było biurko i~krzesło oraz kilka
dodatków domowych, kilka książek, kilka wydrukowanych rzeźb brył
matematycznych. Jakaś część mózgu Limpopo pamiętała stawianie ich tam,
ponieważ to był jeden z~pokojów, które przygotowywała. Nie była w~nim
więcej niż rok i~była zadowolona, że B\&B go utrzymała. Albo jego
mieszkańcy byli sumienni, albo Gospoda zaznaczyła, że pokój staje się
zmamiony i~wstawiła go na listę prac i~ktoś się nim zajął.

Teraz byli w~pokoju, drzwi zasuwały się za nimi z~kliknięciem. Sięgnął,
żeby przyciemnić światła, ale ona rozkręciła je maksymalnie. Odkryła, że
lubi patrzeć na jego twarz w~pełnym świetle. Patrząc prosto na względnie
obcą twarz w~pełnym świetle, bez udawania, że patrzy na coś innego,
podczas gdy ten obcy patrzył na Ciebie, to było coś, z~czym rzadko
musiała sobie radzić. To było intymne, na swój własny sposób, jak
cokolwiek fizycznego.

Jego uśmiech był zmieszany. Lubiła łuk jej ust.

-- Czy to w~porządku? Znaczy\ldots 

-- Po prostu chciałam się dobrze przyjrzeć. -- Była zadowolona, jak szybko
to złapał, jak dokładnie odwzajemnił, źrenice błyszczące, gdy jego oczy
drobnymi ruchami przesuwały się po jej twarzy, wzrok wędrujący szczerze,
który przypominał ich solidny uchwyt.

To było to, co lubiła w~byciu odchodniczką. Uwodziła i~była uwodzona w~default, ale zawsze istniało poczucie uciekającego czasu. Lepiej
przerwijmy te romantyczne sprawy, zacznijmy jebanie, ponieważ zaraz jest
spotkanie, praca, protest, obiad do zrobienia lub inne zadanie. Nawet w~B\&B, trudno było uciec od tego uczucia. Jednak teraz upajała się
nieobecnością, nieskończonym czasem. Przypomniała sobie niechęć Etcetera
do zaangażowania się w~rutynę w~Gospodzie, jego niezdolność do
naturalnego wpadnięcie w~rolę lub pracę. Znaczyło to, że był jej tak
długo, jak tego chcieli.

Wsunęła kciuk pomiędzy jego szatę i~ciało, przebiegła wolno w~dół,
rozwierając szatę w~bolesnym milimetrze na raz, zdumiewając się jak
skóra, którą widziała i~dotykała w~onsen, mogła być tak intymna
częściowo osłonięta. Położył dłonie na jej szacie i~rozsunął ją, a~ona
się zsunęła z~jej piersi, gdy wyskoczyły wolne, jedna i~druga. W~default, denerwowała się nimi, nie były właściwego rozmiaru lub
kształtu. Jej krytyczne oko chciało więcej od jej niedoskonałego ciała.
Odchodzenie wyzwoliło ją z~tego nierozwiniętego lęku, nawet więcej,
oparzenie je zakończyło, całkowicie zajmując jej samoświadomość.

Jej szata rozchyliła się na oparzeniu. Jego dłoń musnęła bliznę.
Podskoczyła, cofnął dłoń. Powiedział ,,przepraszam'', gdy wzięła jego
dłoń i~położyła na bliźnie. Blizna nie bolała, właściwie, ale ciągnęła,
a kiedy robiła jogę, czuła, że skóra na klatce zniekształca się wokół
własnej grawitacji. Przez długi czas nie potrafiła dotknąć tej obcej
rzeczy, gdzie skóra była, tylko zmywała gąbką. Jej dłoń dotykała blizny
we śnie, budziła się z~wężowymi fragmentami kolagenu pomiędzy palcami.
Przeszła do odprężenia, już dłużej nie czując, że była obca.

Jego dłonie były teraz na bliźnie, wzniesieniach i~zagłębieniach. Jego
oczy były odległe, oddech płytki. Też dyszała, ich oddechy mieszały się
w ustach, byli tak blisko. Jeszcze się nie pocałowali. Zsunęła jego
szatę z~ramion i~zmusiła jego ręce do puszczenia jej, gdy szarpała w~dół. Jego dłonie były znowu na niej. Ruch połączył ich bliżej na tyle,
że mógł sięgnąć i~dotknąć jej łopatek, kręgosłupa, satelitów oparzenia,
jak odłamki po uderzeniu meteoru. Był dostatecznie blisko, że jego
erekcja podskakiwała natarczywie na jej udzie, ciepły elastyczny dotyk,
na który się uśmiechnęła. Uśmiechnął się do niej i~wiedziała, że
wiedział, z~jakiego powodu się uśmiecha.

Jego dłonie sięgnęły jej tyłka. Położyła dłonie w~tym samym miejscu i~wciągnęła go, jego erekcja wciśnięta między nich oboje, piersi
uderzające w~jego klatkę piersiową. Jej usta ułożyły się do pocałunku i~znalazły obojczyk. Pracowała nad kością zębami, potem skubnęła skórę.
Westchnął, przycisnął ją mocniej. Przesunął głowę i~obnażył gardło,
śmignęła pocałunek po nim, lubiąc brodę pod ustami, to poczucie
chłopięcej skóry. Jego usta oparły się na arterii jego gardła i~smakowała jego puls, ciągle ssała, ośmielając go, żeby ją odepchnął,
zanim zrobi mu malinę, ale syknął i~jego uwięziony penis pulsował na jej
żołądku w~rytmie jego pulsu.

Przycisnął biodra do niej. Pozwoliła jego udu wsunąć się pomiędzy jej,
przytulić się do jej sromu. Przycisnęła się do mięśni jego nogi, włosy
ciągnące, wszystko suche na początku, a~potem, chwila po cudownej
chwili, bardziej mokre. Jej nozdrza się rozszerzyły. Wdychała seks z~jego ramion i~krocza. Possała wrażliwe miejsce, gdzie jego gardło, szyja
i ucho się spotykały.

Ciągle się nie pocałowali.

Pociągnął mocniej jej dupę, jego dłonie silne. Pamiętała sposób, w~jaki
jego dłoń ściskała jej. Podniósł ją na czubki palcy, miażdżąc jej cipkę
o jego nogę. Znalazła mięsień czworonożny i~przycisnęła clit do niej,
puściła i~odchyliła się, czując, jak mięśnie nogi podskakują, gdy objął
ją, żeby ją wesprzeć. Odchyliła się, aż dłońmi sięgnęła ściany i~się
odepchnęła, bawili się mięśniami i~grawitacją, przyciągnął ją do siebie
i się zachwiał.

Z powrotem na stopach pchnęła go ku łóżku, jedna stopa na drabinie i~się
podciągnęła. Wskoczył za nią.

I ciągle się nie pocałowali. Przekręciła się, złapała jego kostkę,
włożyła mały palec do ust, gryząc, gdy próbował wyciągnąć. Zatopiła
paznokcie w~łuk stopy i~sięgnęła w~ciemno i~złapała jego erekcję,
zaciskając ją w~pięści na tyle mocno, żeby poczuć znowu ten puls. Jego
biodra się poruszyły, a~ona trzymała, potem puściła i~wycofała się,
wsuwając się na jego ciało i~przyciskając jego pierś swoją. Jej ręce
złapały jego, użyła ciężko nabytych mięśni konstrukcyjnych do
szarpnięcia jego nadgarstków nad jego głową i~przyciśnięcia ich do
materaca. Jego pachy pachniały czystym potem. Jego oddech był na jej
twarzy. Jej oddech był na jego.

Ułożyła usta w~pocałunek, gotowa, cofając twarz, gdy próbował ją
pocałować. Chciała, żeby to trwało. Pozwoliła jednej wardze dotknąć
jego, potem obie. Potem nieco języka. Jego usta się otwarły. Twarz
napięła się ku niej i~ona się cofnęła, zaczęła od początku. Zrozumiał
wiadomość. Położył się tam, pozwolił jej na władzę, wybrać, jak
pocałunek będzie się rozwijał. Robiła to bardzo wolno.

To było cudowne.

Nie przestało.

Usta połączone, sięgnął do jej dupy, a~ona wcisnęła ją w~jego dłonie, on
chętnie ugniatał, przypadkowo uderzając penisem w~jej bliznę, co nigdy
się wcześniej nie zdarzyło. Zauważyła to z~roztargnieniem i~jęknęła w~jego usta. Jęknął w~odpowiedzi.

Znowu potarła jego nogę, ścisnęła penisa pomiędzy nimi. Jego palce
pracowały dookoła jej dupy, gdzie włosy były gładkie. Rytmicznie uciskał
długość jej sromu, badał otwarcie. Jęknęła w~jego usta. Dreszcze
przebiegły po jej plecach i~żołądku. Wybuchające uczucie budowało się i~kołysała się, ponaglając go, do kopania głębiej, poruszania szybciej, a~on zatopił w~odpowiedzi. Nie była z~mężczyzną ponad rok. Ten wzbudzał
nostalgiczną erotykę, przywołując na myśl wszystkich mężczyzn wcześniej,
każde drżenie, każdy krzyczący orgazm. Te obrazy przebiegały przez jej
umysł, gdy się ruszyła. Znajomy przypływ rozwinął się w~jej karku.

Zaskoczył ją wytryskiem od tarcia. Nagłe ciepło pomiędzy nimi pchnęło ją
w konwulsje, które zakończyły się rodzajem głośnych dźwięków, które
kiedyś ją zawstydzały.

Przetoczyła się z~niego, wyciskając uff z~niego, gdy nieumyślnie
wcisnęła ramię w~jego splot słoneczny, złapała jego ubranie z~biurka pod
nimi i~użyła do wytarcia.

-- Uff -- powiedziała.

-- Więcej.

Spojrzała w~dół, niedowierzająco. 

-- Już?

Oblizał usta.

Był dobrym kochankiem. To przebijało przez ryk łomoczącego serca
długowyczekiwanego seksu. Nie mogła tego namierzyć, tak czy inaczej,
metaforycznie, ale gdy doszła znowu, zaciskając uda dookoła jego uszu,
zrozumiała: brak pośpiechu. Nawet po latach odchodzenia, była
przyzwyczajona do dzielenia czasu na plastry grubości papieru,
wystarczające na jedną oddzielną rzecz, zanim przeszła do kolejnej.
Większość czasu śpieszyła się zakończyć obecną chwilę, zanim kolejna
uderzy w~drzwi. Każda dorosła, którego znała, znała ten rytm, kolejna
rzecz prawie przy nich, obecna najlepiej, żeby załatwić pośpiesznie.

Etcetera kroił czas grubo. Przesunął się po ciele do piersi i~położył
twarz na nich przez niemierzony czas przed skubaniem. Trwało to dłużej
niż oczekiwała. Było lepsze niż miłe. Jej własny zegar ciała
zsynchronizował się, metronom tykał, zwalniając do ospałego uderzenia
serca, które wydawało się całym czasem świata. To było bardziej
dekadenckie niż lepkie soki na jej palcach, wściekła malinka na jej
lewej piersi, obrzmiała chłopięcy sutek, który pocierała pomiędzy
palcami.

Kiedy skończyli, nie miała pojęcia, jak było późno. To mógł być zachód
słońca lub później, choć przyszli na górę o~świcie. Wytarła powierzchnie
interfejsu na ścianie i~pokazała zegar, była zaskoczona, że to było
dopiero południe. Brak pośpiechu przez cały czas nie oznaczał braku
straty całego czasu. Różnice pomiędzy chwalebną ociężałością a~niekończącym się pośpiechem były godziną lub dwiema. Czuła, jakby oddała
właśnie dzień.

Pocałowała go w~połączenie szyi i~uszu, przeszła do jego ust. Objął ją w~ten nieśpieszny sposób, otulając ją w~szatę, którą przywdziewał.

-- To było bardzo miłe -- powiedziała.

-- Tutaj także to było bardzo miłe.

Spletli palce. Otworzyła właz dla pościeli, zdjęli pościel z~łóżka,
założyli świeżą, wytarli powierzchnie i~włączyli filtry powietrza.
Zielony znak błyszczał się na drzwiach, gdy pokój potwierdził, że został
stosownie zresetowany. Wyszli, ręce splecione i~brudna pościel w~wolnych
rękach. Te trafiły do zsypu do pralni przy schodach, a~oni poszli do
Stajni zrobić nowe ubrania.

\chapter*{v}

Jebanie nie zepsuło ich relacji, dzięki bogini. Przytulił ją i~pocałował
ją w~policzek, zamiast potrząsnąć dłonią we wspólnym pokoju, a~jego
przyjaciele, którzy, jak myślała, robili coś podobnego, spojrzeli się
na nią znacząco. Kiedy ją spotykał, nie był wszędzie przy niej, ale też
nie traktował ją z~przemyślaną ignorancją. Tydzień później, wpadli na
siebie w~korytarzu i~zatrzymali na rozmowę. Oparł się o~ścianę z~dłonią
rozłożoną na ścianie. Położyła swoją obok jego, a~on zrozumiał
podpowiedź, i~poszli na czwarte piętro na kolejną nieśpieszną sesję
zabawy.

-- Jak Ci się tutaj podoba? -- spytała, podczas jednej z~przerw.

Wyglądał na zakłopotanego. 

-- Szczerze, nie sądzę, żeby to była moja
sprawa. Opuściliśmy default, ponieważ chcieliśmy być częścią czego,
gdzie byłbym więcej niż niewygodną nadwyżką siły roboczej. Wiem, że mogę
tutaj pracować, jest dużo do zrobienia, ale wydaje się to wymyślone.
Jednego dnia całkowicie spieprzyłem przetwarzanie pościeli, zniszczyłem
trzydzieści prześcieradeł. System po prostu przydzielił komuś innemu
zrobienie nowych i~przepchnął te zniszczone przez procesor surowców.
Cała rzecz jest tak bezpieczna na wypadek awarii, że praktycznie nie ma
znaczenia, co robię. Gdybym wysilał mózg, czy nic nie zrobił, to byłoby
to samo, na tyle, o~ile chodzi o~system. Wiem, że to zjebane i~egocentryczne, ale chcę wiedzieć, że ja, osobiście, jestem ważny dla
świata. Gdybym odszedł jutro, nic dookoła tutaj się by nie zmieniło.

Przygryzła wargę. Walczyła z~tym sama przez wiele lat, ale przyznanie
się do tego było w~złym guście. Każdy mówił o~specjalnych śnieżynkach, a~to był rodzaj rzeczy, która była zniewagą od obcego, ale nie od
przyjaciela. Nie powinnaś pragnąć być specjalną śnieżynką, ponieważ
obiektywna rzeczywistość była taka, że, mimo że ważne dla Ciebie i~dla
ludzi bezpośrednio dookoła Ciebie, mało prawdopodobne było to, że
cokolwiek robiłaś, było niezastępowalne. Gdy tylko sklasyfikowałaś się
jako specjalna śnieżynka, szłaś w~kierunku samooszukiwania, że powinnaś
mieć więcej niż wszyscy inni, ponieważ śnieżynkowatość tego wymagała.
Jeżeli była jedna rzecz, która była całkowicie niefajna dla
odchodzących, to było samooszukiwanie.

-- Wiesz, że to miłość nie śmie mówić swojego imienia tutaj? Na tej
planecie istniały setki miliardów ludzi przez wieki, a~statystycznie,
większość z~nich nic nie zmieniła. Antropocen jest wynikiem wspólnych
działań, nie jednostkowych. To dlatego zmiana klimatyczna jest takim
burdelem. W~defaulcie, mówią, że to sprowadza się do jednostkowych
wyborów i~odpowiedzialności, ale rzeczywistość jest taka, że nie możesz
osobiście wykupić się ze zmiany klimatycznej. Jeżeli Twoje miasto
wykorzystuje ponownie szklane butelki, to robi jedną rzecz. Jeżeli je
recykluje, to realizuje coś innego. Jeżeli wyrzuca na wysypisko, to też
coś innego. Nic, co robisz, osobiście, nie zmieni tego, chyba że to Ty,
osobiście, zbierzesz się z~\textit{wieloma} innymi i~tworzycie różnicę.

-- Ale trudno udawać, że się nie jest bohaterem w~filmie swojego życia.
Normalnie to nie ma znaczenia, ale bycie tutaj przypomina to na każdym
kroku.

-- Wszystko ma sprzeczności. Czasem zastanawiam się, czy ktoś robi coś,
co polepsza sytuację wszystkich, ponieważ \textit{ja} napisałam określoną
linię kodu. Żeby naprawdę tutaj rozkwitnąć, musisz chcieć coś zmienić i~wiedzieć, że jesteś kompletnie zastępowalny.

-- Przebija default. Tam nie powinieneś niczego zmieniać \textit{i} jesteś
kompletnie nadwyżką.

Ta dyskusja zabiła jej pożądanie. Myśl, że jej wrażenia były takie same
jak niezliczeni ludzie czuli wcześniej, i~przedtem i~potem ten moment
sprawił, że poczuła się jak tania sztuczka, sposób łaskotania jej
obwodów nagrody za kroplę tego lub kichnięcie tamtego. Zwykle, seks
sprawiał, że czuła się jakby wszechświat obracał się dookoła jej wrażeń.
Teraz one przypominały bezsensownych wybuch światła w~nieczułym świecie.

Usiadła i~podciągnęła ubrania. Etcetera nie wydawał się przygaszony, co
było ulgą i~było też niepokojące.

-- Wszystko w~porządku?

-- W~porządku -- powiedziała. -- Po prostu już nie w~takim nastroju.

-- Przykro mi. -- Założył bieliznę i~spodnie, odwrócił koszulę na drugą
stronę. -- Mimo jakiejkolwiek ortodoksji, jaką powinienem przyjąć, i~nieważne jak brzydkie to jest, chcę powiedzieć, że właściwie myślę, że
jesteś specjalna. Bardziej niż specjalna. Właściwie wspaniała. I~piękna.
Jednak głównie wspaniała.

Jej serce biło. 

-- Słuchaj, chłopie\ldots 

-- Nie martw się. Nie zamierzam się zakochiwać. Jednak poznałem
kilkanaście osób, od kiedy odszedłem, a~Ty jesteś pierwszą, która mnie
powitała, i~nie tylko dlatego, że wyjebałaś mój mózg, choć to też
sprawiło, że poczułem się powitany. Ale dlatego, że mogę rozmawiać o~tym z~Tobą i~nie przewracasz oczami, jakby to były najgłupsze pytania, i~ponieważ nie jesteś też dziecinną doktrynerką. Jesteś jakby jedyną osobą
tutaj, która myśli o~byciu odchodzącą i~myśli o~odchodzeniu. Bez Ciebie,
już bym ruszył dalej. To miejsce jest niesamowite, ale jest zbyt
ukończone, jeżeli wiesz, co mam na myśli.

Włożyła sukienkę. To dało jej chwilę. Kiedy jej głowa się wyłoniła,
patrzył na nią szczerze. Miał bardzo ładne oczy, miły uśmiech. Nieco
niepewny, ale lubiła to.

-- Myślę, że też jesteś wspaniały.

-- Powinniśmy pójść do stajni i~wydrukować jakieś odznaki członkostwa w~towarzystwie wzajemnego zachwytu.

-- Śmiejesz się, ale jestem pewna, że takie istnieją jako pre-gotowe w~odchodnickim rzeczowersum.

-- Cóż, to słodkie -- powiedział. Roześmiali się i~odległa zaalarmowana
część jej powiedziała jej, że to był śmiech kochanków, że ona się
zakochiwała.

\chapter*{vi}

Zakochiwanie jest wspaniałe. Kiedy się poddała, bawiła się, znajdując
sposoby bycia miłą dla Etcetery, przygotowując kurtki w~kolorze i~stylu,
które mu pasowało lepiej niż cokolwiek nosił, budzenie go coffium i~zaciąganie go na górę na szybki numerek, podczas gdy kawa paliła żyły,
mycie mu pleców zmysłowymi pociągnięciami w~onsenie.

Odwzajemniał się na sto sposobów, zajmując jej miejsce we wspólnym
pokoju, witając ją po wędrówce zimną herbatą i~chłodnym ręcznikiem, lub
łapiąc ją za rękę pod stołem -- lub nad -- gdy rozmawiali z~innymi do
późna w~nocy.

Starsi członkowie zauważyli, ale byli zbyt uprzejmi, żeby pytać wprost.
Zamiast tego mówili: ,,Och, czy Etcetera to ci dał?'' (Tak, girlanda
zimowych gałązek, które ułożył w~śmiesznie czarodziejską koronę, którą
nosiła przez dzień, zanim się rozpadła, ale ceniła za to jeszcze
bardziej). Istniały pary odchodzących, nawet rodziny z~dziećmi i~jednym
lub więcej rodzicami, ale nigdy do nich nie dołączyła. Małżeństwo
wydawało się artefaktem defaultu, czymś, czego nie chciała być częścią,
kłopotami z~zazdrością i~koordynacją.

Ale to było inne. Emocje śpiewały w~jej myślach, słodsze niż
jakiekolwiek wspomnienia. Leżąc koło niego, nawet przytulając, patrząc
na jego usta i~dołek w~policzku powodowało, że coś ciepłego rozlewało
się po jej piersiach i~brzuchu.

Wychodzili na długie spacery, nie rozmawiając, słuchając ptaków i~chrzęstu ich kroków w~śniegu. W~lesie były sarny, zwykle daleko, ale
raz, łania przyszła dostatecznie blisko, żeby jej dotknąć, patrzyła się
na nich z~urzekającą szczerością zwierząt.

Jednego dnia wstali o~wschodzie, pełni owsianki i~bulgocząc od coffium,
kierując się tropem z~drona Gospody, który znalazł kryjówkę z~elektroniką pełną części z~tantalu i~niobu do odzysku, porzucone
nielegalne e-śmietnisko. Zabrali mułbota, a~pomaganie mu w~drodze
zwolniło ich do lodowatego pełzania. Trochę się sprzeczali, pamiętała to
później.

Kryjówka była niedostępna. Ziemia zamarzła na kamień od nocnego mrozu,
który zamienił wcześniejszą odwilż w~zdradziecki lód. Nawet w~korkach,
nie mogli złapać oparcia, a~mułbot fatalnie utknął poza zasięgiem
ramion, nie potrafiąc złapać tarcia. Po porzuceniu próbowania wzięcia go
na lasso, poszli z~powrotem w~złych nastrojach.

Oboje usłyszeli bzyknięcie offline w~tej samej chwili, gdy odchodnicka
sieć zawiodła. Mogła to stwierdzić, ponieważ oboje zatrzymali się w~tej
samej chwili.

-- Czy to się zdarza często? -- spytał Etcetera.

-- Nie powinno się zdarzyć, kropka. Sieć ma redundantne zabezpieczenia, w~tym sterowce. A mamy czyste niebo.

Wyjęła ekran i~postukała palcami w~rękawiczkach, mrugając w~parze jej
oddechu. Nie używała często diagnostyki i~zabrało jej chwilę, żeby je
włączyć. 

-- To dziwne -- powiedziała. -- Nawet gdyby wszystko poszło do
cholery, mógłbyś oczekiwać kaskady awarii. Węzeł A znika, węzeł B jest
przytłoczony przejętym ruchem od niego, wyłącza się, potem węzeł C
dostaje podwójny ruch i~tak dalej. Ale spójrz, to straciło kontakt z~wszystkim, wszystko na raz. To bardziej jak wyłączenie zasilania, ale
one są na niezależnych ogniwach.

-- Co to może być?

-- Myślę, że to poważne. Chodźmy.

Trzeba to powiedzieć o~Etceterze, kiedy rzeczy stawały się poważne,
stawał się poważny. Zobaczyła jego nową stronę, nerwową i~czujną. To ją
uspokoiło. Mogła przestać nieświadomie martwić się o~niego.

Śpieszyli się przez zdeptane ślady na śniegu, poruszając się w~ciszy, w~niewypowiedzianych wymiarach skradania się. Usłyszał szum i~zauważyła
drona Gospody, co ją pocieszyło. Potem zobaczyła, że to nie był jeden z~ich modeli.

-- Cholera -- powiedział, gdy zbliżył się po kolejny przelot. Pokazała mu
palec, gdy brzęczał metry nad ich głowami. -- Jebać to. Chodźmy.

Pobiegli.

Ścieżka była dobrze zadbana z~mądrą serią zakrętów i~strategicznych
drzew, które pozwalały pojawić się w~lokalizacji przy chaotycznych
budynkach i~wiatrakami dumnymi nad głową. Zanim zostali odkryci,
planowała wyjść przez lasy z~jednej strony, przebijając nową ścieżkę,
żeby się tam dostać. Ale teraz nie miało to sensu.

Weszli na polanę i~ujrzała grupkę masywnych gostków w~przestraszających
bzdurnych taktycznych stojących wokół głównego wejścia. Puszyli się
pasami roboczymi z~gratami wyglądającymi jak broń, które mogły im
wyrządzić straszne rzeczy, nie musieli po nie sięgać, żeby było jasne,
kto jest górą.

-- Cześć -- zawołał jeden. Nawet miał wąs twardziela jak zapaśnik. -- Witamy w~Belce i~Brasach.

-- Ta, dzięki -- odpowiedziała.

-- Jestem Jimmy -- powiedział. -- Czy chcielibyście kwatery?

-- Załóżmy, że tak -- powiedziała Limpopo.

Uśmiechnął się leniwym, wilczym uśmiechem, potem się przyjrzał. 

-- Och -- powiedział. -- To Ty, prawda?

Przyjrzała się mu lepiej, przypominając sobie. 

-- Tak, to ja -- Westchnęła.

-- O kurde. To Twój szczęśliwy dzień, Limpopo.

Skinęła głową. Nie nazywał się Jimmy, kiedy wyrzuciła jego dupę z~Gospody. Co to było? Dżemiarz? Dżingiel? Coś. Minęły lata.

-- Założę się, że nie spodziewałaś się mnie spotkać. -- Odwrócił się do
przyjaciół. -- Ta pani tutaj dodała więcej linii kodu do tego miejsca niż
ktokolwiek. Zrobiła więcej przy budowaniu tego niż ktokolwiek. To
miejsce jest pełne krwi i~skarbów tej dziewczyny. -- Odwrócił się. -- To
naprawdę jest Twój szczęśliwy dzień.

-- Tak? -- Wiedziała, gdzie to zmierza.

-- Od teraz, to miejsce opiera się na ,,quid pro quo''. Wszyscy dostają
tyle, co dali. Dałaś tak dużo, cóż, mogłabyś zostać tutaj przez lata bez
podnoszenia palca. Masz kapitał reputacji do przepalenia.

-- Och, człowieku -- powiedziała.

\threeast

Nie możesz być odchodniczką bez trafienia na fanów ekonomii reputacji. Z~początku, nienawidziła ich w~abstrakcji. Potem ten facet się pojawił i~dał jej cholernie dobre, \textit{konkretne} powody do nienawiści. B\&B
była skompilowana po raz trzeci, kiedy się pojawił i~próbował
zainstalować we wszystkim rankingi. Właściwie, zrobił to, sprawdzając
kod i~potem przychodząc do niej, kiedy miała ręce pokryte pastą
uszczelniacza, żeby zażądać odpowiedzi, dlaczego wycofała jego zmiany.

-- To nie jest coś, co chcemy.

-- Co to znaczy? Nie macie konstytucji. Sprawdziłem.

-- Nie mamy. Jednak ta sprawa została przedyskutowana i~konsensus brzmi
tak, że nie chcemy rankingów. Produkują gówniane zachęty. -- Podniosła do
góry brejowate dłonie. -- Jestem w~trakcie czegoś. Dlaczego nie wstawisz
tego na wiki?

-- Czy to zasada?

-- Nie -- odpowiedziała.

-- Zatem dlaczego miałbym to robić?

-- Ponieważ tak działało wcześniej.

-- Może powinienem po prostu cofnąć Twoje cofnięcia.

-- Mam nadzieję, że tak nie zrobisz. -- Wiedziała, jak prowadzić taką
dyskusję. Utrzymywała kontakt wzrokowy. On był młodym, niedawnym
odchodnikiem, stłumionym dziwakiem, co łączyło się z~terytorium. Nie
było wspólnych części pomiędzy jego dziwacznością a~jej własną.

-- Dlaczego nie?

-- To nie byłoby konstruktywne. Chodzi o~to, żeby znaleźć coś, z~czym
możemy być szczęśliwi. Wojny w~odwracanie nie tworzą tego. W~najlepszym
przypadku doprowadzi to do tracenia naszego czasu na odwracanie zmian
drugiego. W~najgorszym, zamieni się w~wojnę, żeby zobaczyć, kto potrafi
przygotować kody źródłowe, żeby były trudniejsze dla drugiego do
modyfikacji. -- Miała płachtę plastra izolującego na stole i~uszczelniacz
zaczynał zastygać w~grudki. Złapała pędzel z~gąbki i~rozprowadziła
grudki. -- Chcesz zobaczyć to miejsce wybudowane? Ja też. Wymyślmy jak.
Mógłbyś zacząć od przejrzenia starych dyskusji i~sprawdzenia, jak są
podejmowane decyzje. Potem przygotować własne argumenty. Obiecuję, że
przeczytam je w~dobrej wierze. -- To była mantra, ale próbowała nasycić
ją szczerością. On był nakręcony. Nie chciała go przestraszyć. Nawet nie
chciała z~nim rozmawiać.

Przydybał ją przy obiedzie. To było, zanim kuchnia była skończona,
musieli sobie radzić z~podstawowym jedzeniem, przyprawionym proteinami z~kultur bakterii według UNHCR dla uchodźców. Kisiel z~protów, z~wszystkim, czego potrzebujesz do działania i~w szerokim wyborze smaków,
ale nikt nie mylił tego z~jedzeniem. Zrobiła dla niego miejsce na ławce,
podała mu kubek z~wodą, używali pasteryzatorów słonecznych, wielkich
czarnych beczek, które używały powłok wymieniających ciepło, żeby
podgrzać wodę do temperatury zabijającej patogeny. Nadawało to wodzie
mdły smak. Używała gałązek mięty, żeby to ukryć. Zaoferowała mu nieco z~rośliny, którą zebrała przed sygnałem na obiad.

Zawirował miętą w~jej wodzie i~zjadł proty, które wziął w~wersji do
żucia, brykiet nacho serowy, tak mocno pachnący, że prawie ukrył jego
potliwy odór. Trudno było odwiedzić łaźnie w~tych dniach, ale nie aż
tak. Próbowała myśleć o~uprzejmym sposobie pokazania mu jak działa
mycie, bez tworzenia jakiejkolwiek opcji interpretacji tego jako
zaproszenia do seksu.

-- Przeczytałeś to wszystko?

Skinął i~żuł. 

-- Tak -- powiedział. -- Zrobiłem statystyki na
repozytoriach. Jesteś o~rząd wielkości przed wszystkimi, potężna krzywa
mocy. Nie miałem pojęcia. Serio, szacunek.

-- Nie patrzę na statystyki. O to chodzi. Nie mogłabym napisać tego
wszystkiego sama, a~nawet, gdybym mogła, nie chciałabym, ponieważ to
miejsce byłoby do bani, gdyby to był tylko konkurs, kto może dodać
więcej linii kodu czy cegieł do struktury. To rajd do zbudowania
najcięższego samolotu na świecie. Co wiedza, że ta osoba ma więcej
commitów niż tamte, Ci mówi? Że powinieneś pracować więcej? Że jesteś
głupi? Że jesteś wolny? Kogo to obchodzi? Większość commitów w~naszym
kodzie źródłowym pochodzi z~historii, wszystkich, którzy napisali
biblioteki, usunęli błędy, zoptymalizowali, poprawili je. Większość
commitów w~tym budynku pochodzi od wszystkich, którzy przetworzyli
surowe materiały, wymyślili, jak przetwarzać surowe materiały, zebrali
surowce i\ldots 

Podniósł dłonie. 

-- Jasne. Jednak, może nie zrobiłaś całej tej pracy, ale
robisz więcej niż ktokolwiek. Dlaczego społeczność nie miałaby tego
docenić?

-- Jeżeli robisz rzeczy, ponieważ chcesz, żeby ktoś Cię poklepał po
głowie, nie będziesz tak dobry w~tym, jak ktoś, kto robi to dla własnej
satysfakcji. Chcemy najlepszego możliwego budynku. Jeżeli stworzymy
system, który sprawia, że ludzie konkurują o~uznanie, zapraszamy
zachowania gry w~granie, poprawiania statystyk, nawet takie niezdrowe
rzeczy jak praca w~głupich godzinach, żeby pobić wszystkich. Załoga
pełna nieszczęśliwych ludzi pracujących poniżej norm. Jeżeli zbudujesz
system, który sprawia, że ludzie skupiają się na mistrzostwie,
współpracy i~lepszej pracy, będziemy mieć piękną karczmę pełną
szczęśliwych ludzi pracujących razem.

Pokiwał głową, ale nie był przekonany. Pomyślała, żeby powiedzieć:
,,Wkładam więcej pracy niż ktokolwiek, więc według Ciebie, to oznacza,
że powinnam kierować. Jako osoba kierująca, stwierdzę, że osoba, która
robi najwięcej, nie powinna kierować, zatem proszę.'' To spowodowało u
niej uśmiech, zobaczyła, że wygląda na zażenowanego i~przypomniała, że
jest świeżakiem, który nie wie, co robiła, lub, co powinna robić, czując
się osądzonym.

-- Nie przyjmuj mojego słowa -- powiedziała. -- Odtwórz ponownie dyskusję,
zacznij ją, zobacz, czy możesz przekonać innych. Przesuń konsensus.

-- Pomyślę o~tym. -- Wiedziała, że tego nie zrobi. Idea, że nie powinno
być przywódców w~rajdzie ku społeczeństwu bez przywódców, obrażała go na
sposoby, których nie pozwoliłby sobie na zrozumienie.

Trzy tygodnie później, byli na wojnie ,,cofania zmian'', która
wstrząsnęła B\&B aż do dosłownych fundamentów.

Dżingiel przejrzał wszystkie projekty wspólnych budynku w~sieci za
modułami grywalizacji. Było ich wiele, odznaki, złote gwiazdki, prace
amatorskich naśladowców Skinnera, przekonanych, że możesz zbudować
idealne społeczeństw w~ten sam sposób, jak uczysz dziecko korzystania z~toalety: wykres na ścianie z~nalepką uśmiechu koło każdego dnia bez kupy
w pieluchach.

Wyniki tych eksperymentów robiły wrażenie. Jeżeli chciałeś zmotywować
ludzi na ich najbardziej dziecinnym poziomie, to, czego potrzebowałeś,
to dać dobrym dzieciom cukierki, a~niedobre posłać do kąta. Zebrał razem
linki do filmów i~raportów analitycznych od najbardziej udanych.

Z początku Limpopo była ostrożna, żeby ograniczyć styl odmowy do
nieantagonizującego głosu ,,dobrej wiary'', który był pewnym zwycięzcą w~odchodnickich dyskusjach. Starannie ignorowała emocjonalny dodatek jego
słów, czytając je trzy razy, żeby upewnić się, że uchwyciła każdy
istotny element elaboratu i~odpowiadała krótko, kompleksowo i~bez cienia
pogardy.

Nie wiedział, kiedy był trafiany. To było jak dyskutowanie z~chatbotem,
którego łańcuchy Markowa były uwikłane w~paternalistyczny żargon
naczelników więzienia i~nielicencjonowanych pracowników przedszkola.
Spokojnie niszczyła jego argumenty każdego dnia od poniedziałku do
piątku, a~w~sobotę rano, wyciągał znowu argument z~poniedziałku, jakby
mogła nie zauważyć.

To wszystko działo się w~komentarzach do commitów i~wycofań kodu, co
było jeszcze głupsze. Publiczność debaty wzrastała, gdy wiadomości
trafiały dalej. Pojawiła się globalna uwaga, nie tylko od odchodzących.
Tam w~default, niektórzy ludzie pilnowali sieci odchodnickich, traktując
je jak egzotyczne widowisko, jak słuchanie Asz-Szabab skarżący się na
niewygodną procedurę zwrotu kosztów, jaką nałożyli wahabiccy płatnicy.

Przy kibicującej globalnej widowni Limpopo zrobiła Dżinglowi
wyczerpujący wykład. Skrytykowała każdą jego bzdurę, znalazła nieudane
projekty, gdzie grywalizacja zdziczała, tak finansująca, że każda
zachęta zniekształciła się w~tytaniczne oszustwa, które dosłownie
zostawiły struktury w~ruinie, zepsute aż do zaprawy. Były dowodem
istnienia straszności jego umiłowanych idei. Wykazała, że zmuszanie
ludzie do ,,robienia właściwych rzeczy'' przez motywowanie ich do
zwyciężania nad innymi, było głupie. Znalazła filmy trenowanych gołębi
B.F. Skinnera, które nauczyły się grać na pianinie przy pomocy treningu
jedzeniem i~wykazała, że wszyscy, którzy to proponowali, wyobrażali
siebie w~postaci eksperymentatora, a~nie gołębi.

Było ostro. Zraniła jego ego, stawiła czoło jego protekcjonalności,
odpowiadając mu przy pomocy zaledwie części zniewag, które kierował ku
niej. Stracił się. Wszechstronnie zwyciężony, poszedł ostro.

Problemem była wagina Limpopo. Sprawiała, że nie była w~stanie zrozumieć
siły ambicji, która była prawdziwą siłą motywacji, która kierowała
ludźmi. Konkurencja wyrzeźbiła gazelę jako doskonałe uzupełnienie
lamparta. Konkurencja wystrugała kły i~skoki lamparta w~przeciwieństwo
gazeli. Konkurencja oddzielała wykonawców od biorców. Pozwalała
wizjonerom zamieniać ich projekty w~arcydzieła.

Kobiecość Limpopo była przyczyną jej słabości w~uchwyceniu tego.
Stracił czas na pogaduszki o~uszczęśliwianiu wszystkich, kiedy poprawna
odpowiedź była w~danych, obiektywnie pokazująca, którą drogę wybrać.
Napisał o~tej jej ,,słabości'', jakby to była choroba psychiczna,
czarując wyobrażonych ,,hakerów czwartej sigmy'', którzy nie mogliby
wnosić wkładu do B\&B, jeżeli nie mogliby publikować statystyk
wydajności.

Znalazł źródło tej dysfunkcji w~płci Limpopo. Miała grupę ,,alfa suk'',
które trzymały krótko grupę. Jej przywództwo niczym kult w~tym sabacie
rozpościerało władzę nad ich cyklami menstruacyjnymi, które bez
wątpienia zbiegały się w~potężnych sygnałach macicy z~niewypowiadalnych
mokrych miejsc Limpopo.

W tamtym momencie Limpopo była z~siebie dumna. Wyraźnie czuła umysł
podzielony na dwie części, gdy czytała te złośliwe ataki. Jedna połowa,
,,Limpopo Limbiczna'', hiperprzemocowe niefiltrowane id, warczało.
Dosłownie sprawiło, że serce jej biło i~ręce się zaciskały. Kiedy
świadomie to powstrzymała, bolała ją cała szyja. Limpopo Limbiczna
chciała kopnąć Dżingla w~jaja. Chciała zwikizować każdą złośliwą linię i~dodać tagi {[}potrzebny przypis{]} do wyzwisk, znakując je jako ad
hominem nie do obrony. Limpopo Limbiczna chciała wyciągnąć Dżingla z~jego łóźka, łóżka, które ona złożyła i~pomalowała, wyrzucić go nagiego,
trzasnąć drzwiami i~spalić jego śmierdzący plecak.

Jednak to była tylko połowa jej reakcji. Limpopo Długofalowa była tak
samo natarczywa w~jej wewnętrznym chórze. To napawało ją dumą. Limpopo
Długofalowa zawsze tam była, ale zwykle Limpopo Limbiczna krzyczała tak
głośno, że nie mogła usłyszeć Limpopo Długofalowej, póki głupia
Limbiczna nie zrobiła bałaganu.

Limpopo Długofalowa podkreśliła, że debata zajmowała dużo czasu,
ponieważ problemy były trudne i~nudne. Przekonywanie ludzi, którzy
chcieli zbudować tawernę, żeby przejmowali się filozofią strategii
nagród, było jak przekonywanie ludzi, którzy byli podnieceni świąteczną
kolacją, żeby przejmowali się, czy pokój był pomalowany farbą akrylową
czy olejną. Kolacja, a~nie pudełko, była ważna.

To było inne. Przekonywanie ludzi, żeby troszczyli się o~istotne rzeczy,
było trudne, ale proceduralne problemy były znacznie prostsze. Choć
temat debaty był ezoteryczny, forma debaty, szczera mizoginia, okrutne
zniewagi, mogły być dostrzeżone z~orbity. Kiedy kłócili się o~stosowanej
psychologii motywacyjnej, było trudno powiedzieć, komu kibicować. Kiedy
ujawnił się jako dupek, problem się wyjaśnił.

Limpopo Długofalowa wskazała, że już wygrała. Wszystko, co musiała
zrobić, to powstrzymać się od schodzenia na poziom Dżingla. Nawet gdy
jej krew wrzała przez Limpopo Limbiczną, przekazała kontrolę Limpopo
Długofalowej, która zwróciła uwaga, że to nie był właściwy sposób na
prowadzenie technicznej dyskusji.

Reakcja była błyskawiczna. Nawet ludzie, którzy brali stronę Dżingla we
wcześniejszej debacie, gorączkowo się odsuwali. Nastąpiły potępienia, a~w~ciągu godziny, ktoś zwołał awaryjne spotkanie twarzą w~twarz dla
pracujących w~B\&B. Limpopo wyjrzała przez okno i~zobaczyła ludzi ponuro
wznoszących wielki namiot, który używali, żeby osłonić surowe materiały,
podczas gdy łańcuch ludzi podawał sobie krzesła z~wybudowanej w~połowie
B\&B.

Jednym z~rewolucyjnych narzędzi w~Belce była ,,miłośćnieśmie'', którą
zaimportowali od dawna niedziałającego *leaks kolektywu, który zapadł
się, gdy przywódcy zostali ujawnieni, że biorą pieniądze od giganta
mediów za preferencyjny dostęp do historii. Kolektyw miał strasznych
liderów, ale mieli bardzo dobry system rozstrzygania sporów przez
,,miłośćnieśmie''.

Główną ideą było to, że radykalne lub trudne pomysły były hamowane przez
myśl, że nikt inny ich nie ma. Ten strach izolacji prowadził ludzi do
pozostawania ,,w szafie'' co do ich pomysłów, sprawiając, że byli
,,Miłością, która nie śmie wymówić swojego imienia''. Zatem
,,miłośćnieśmie'' (skrócone do ,,Nie Śmie'') umożliwiała znalezienie
drogi do odkrycia, czy inni czują się tak samo, bez konieczności
ujawniania się.

Każdy mógł zadać pytanie -- Śmienie -- przykładowo, ,,Jak uważacie, czy
powinniśmy przegonić tego seksistowskiego dupka?''. Ludzie, którzy w~tajemnicy się zgadzali, podpisywali pytanie jednorazowym kluczem,
którego nie musieli ujawniać, póki określona wcześniej liczba głosów nie
została osiągnięta. Wtedy system obwieszczał wiadomość informującą
podpisujących do powrotu z~kluczami szyfrującymi i~odanonimizowania
siebie, trzymając w~depozycie wyniki, póki krytyczna masa głosujący się
nie ujawniła. Tak szybko jak powiedzenie ,,Nazywam się Spartakus'',
konsensus powstał z~systemu.

Biedny Dżingiel nie wiedział, co go trafiło. ,,Nie Śmie'' było szeroko
publikowane w~Gospodzie, ale Dżinglowi brakowało pokory, żeby zrozumieć,
dlaczego miałby tego używać, raczej niż po prostu obwinienia Twoich
Wielkich Głupich Idei i~próbowania wyprowadzenia wszystkich na barykady.
Było dużo rzeczy, których Dżingiel nie miał pokory, żeby zrozumieć. Był
jednym z~tych ludzi -- prawie wszyscy z~nich młodzi, choć nie każdy młody
mężczyzna -- który był tak inteligentny, że nie potrafił zrozumieć, jak
był głupi.

Założyła świeże ubrania, z~dopiero co uruchomionej drukarki goretexu,
była to nagroda, żeby założyć coś suchego, oddychającego i~doskonale
dopasowanego, kiedykolwiek chciałaś. Poszła na spotkanie.

Nie musiała nic mówić.

Dziesięć minut później, bełkoczącemu Dżinglowi pokazano drzwi i~uprzejmie poproszono o~niewracanie. Wypełnili jego plecak i~dali mu dwa
zestawy ubrań goretexowych. Cokolwiek mniej byłoby niesąsiedzkie.

\chapter*{vii}

Skrywanym sekretem Limpopo było to, że zbierała logi produkcyjne B\&B i~wrzucała je w~domowy system analityczny, który złożyła niczym
Frankensteina ze świata grywalnych motywacyjnych bzdur. Co jakiś czas,
przetwarzała logi i~patrzyła jak daleko przed wszystkimi innymi była.
Szczególnie lubiła patrzeć na wykresy statystyk, gdzie przegrała
dyskusję na temat, jak coś miało być zrobione.

Nie dlatego, że koiło jej ego. To było dziwniejsze. Kiedy Limpopo
przegrywała dyskusję, sam fakt, że jednak robiła więcej niż osoba, która
wygrała, był \textit{niesamowity}. Bycie odchodzącą oznaczało szanowanie
wkładu wszystkich i~unikanie iluzji specjalnej śnieżynki. Zatem
przegrywanie wobec kogoś, nad kim, w~default, byłaby wyżej, sprawiało,
że była jebaną \textit{świętą}. Nikt nie był specjalną śnieżynką, ale ona
była lepsza w~nie-byciu specjalną śnieżynką bardziej niż wszyscy.

Patrzenie na te wykresy dawało jej niemal te same uczucia wstydu i~przyjemności, które doznawała przy oglądaniu porno. To było czyste
dogadzanie sobie, coś, co wyłącznie karmiło jej niedojrzałe i~egoistyczne pragnienia. To była kocimiętka dla Limpopo Limbicznej, a~im
więcej się karmiła tę chciwą paszczę, tym bardziej była w~stanie
powiedzieć jej, żeby się zamknęła i~pozwalała prowadzić się Limpopo
Długofalowej. Przynajmniej, tak sobie mówiła.

\threeast

Teraz nazywał się Jimmy, był wystrojony w~rzeczy, przy których goretex
wyglądał jak niewyprawione szczurze skóry zszyte suchymi trawami.
Świetnie się bawił.

-- Powinniście zobaczyć liczby -- powiedział do swoich kumpli. Inaczej niż
w Gospodzie, gdzie byli ludzie wszystkich odcieni, wszyscy jego
przyjaciele byli biali, prócz jednego faceta, który mógł być
Koreańczykiem. 

-- Ona jest \textit{Królową} tego miejsca. -- Pokręcił głową,
jego kark był byczy, pasujący do kreskówkowych bicepsów. -- Cholera,
Limpopo, naprawdę jesteś Królową. Od teraz, Ty i~gość możecie zostać,
gdzie chcecie, w~każdym pokoju w~domu. Pełne przywileje kuchni i~warsztatu. Chciałbym, abyś dołączyła do naszej rady. Potrzebujemy kogoś
jak Ty.

Etcetera trzymał się z~tyłu, z~początku oddychając szybko, potem
zwalniając. Zastanawiała się, czy zrobiłby coś głupio fizycznego. To
byłaby szkoda.

Była cała narracja, w~której powinna uczestniczyć, dziura, którą Jimmy
zrobił dla niej, żeby dołączyła. Albo porzuciła swoją grupę i~legitymizowała zamach -- wątpiła, czy kładłby na to nacisk -- lub, lepiej,
ustosunkować się, pozwolić mu na poniżenie jej w~sposób, w~który powinna
go upokorzyć. Jedynym sposobem na wygraną było nie grać.

Wstała.

Próbował wciągnąć ją w~rozmowę, jak zwiększyliby pojemność przez
oddzielenie pasożytów do liderów, zajęli się rabusiami poprzez oddanie
niektórych łóżek na dobroczynność co miesiąc. Stała niema.

Im dłużej stała, tym bardziej Jimmy wariował. Im dłużej stała, tym
więcej osób dryfowało na zewnątrz, żeby dowiedzieć się, co się stało. To
było jak fizyczne powtórzenie starej rozprawy online.

-- Po prostu się pokazał i~ogłosił to jako skończoną sprawę -- powiedziała
Lizzie, która była w~Gospodzie od początku, wbijając pierwsze tyczki
geodezyjne, gdzie sieć jej wskazywała. -- Nikt nie chce się bić, racja?
Miał głupi powerpoint z~naszymi statystykami, zebranymi z~publicznych
repozytoriów, pokazujący, że wszyscy tutaj mogliby mieć te same
przywileje, jak zawsze mieliśmy, ponieważ dostatecznie dużo
pracowaliśmy.

-- Ta -- powiedział Grandee, który był niski, stary i~dziwny, ale którego
Limpopo lubiła, ponieważ był dobrym słuchaczem, z~czymś zepsutym w~środku, o~co nigdy się nie spytała, ale mimo wszystko czuła się o~tym
bardzo opiekuńcza. -- Mówił o~falach nowych odchodzących kierujących się
w tym kierunku, masywnym skoku, który by nas zmiażdżył, chyba że
mielibyśmy jakiś system, żeby przydzielać zasoby. Miał filmy o~miejscach, gdzie to się zdarzyło.

Skinęła głową. Słyszała o~miejscach, gdzie liczby narastały szybciej,
niż mogły być wchłonięte, dobrze zorganizowane gospody stawały się
tłoczne, potem zatłoczone, potem katastroficzne. Była nawet przemoc,
rzadka, ale strasznie opisywana w~prasie default, która spływała z~powrotem do odchodzących. Straszna, czy nie, to było odrażające. Było
podpalenie z~cudownym brakiem ofiar (fotografie były tak silnym
triggerem dla Limpopo, że nakazała czytnikowi odfiltrować kolejne
raporty).

-- Dobrze -- powiedziała. Więcej osób wyszło.

Było zimno. Ich oddech parował, przypominając jej parę w~onsenie.

Tłum po stronie Limpopo się powiększał. Niewidzialny przełącznik
przeskoczył i~wszyscy, którzy nie stali z~grupą Limpopo, domyślnie stali
przeciwko niej -- nie tylko po prostu godząc się z~grupą Jimmy'ego,
ponieważ to było najłatwiejsze i~właściwie jakie to miało znaczenie -- ale właściwie stali przeciwko grupie Limpopo i~wszystkiemu, za czym się
opowiadali.

Plecak Limpopo miał sprzęt turystyczny, który utrzymałby ją przy życiu
przez dzień w~lesie, w~najgorszym przypadku. Uruchomiła swój piecyk,
zasilając go gałązkami, aż wiatrak zamienił ciepło ze spalania w~gazową
przemianę fazową, dynamo, które napędzało baterię, zawirowało, a~światełko dla idiotów się włączyło, mówiąc jej, że piecyk działał już
sam.

Zrobiła herbatę. Miała książkę składanych filiżanek, półsztywnych
plastików wstępnie wyciętych dla składania w~kubki z~geometrycznymi
rączkami. Kochała je, wyglądały jak obrazy kubka w~niskiej
rozdzielczości, które skoczyły z~ekranu w~świat. Czajnik był otwieranym
cylindrem, który wypełniła śniegiem, przechodząc do nietkniętego opadu
na krawędzi lasu, obserwowana podejrzliwie przez Jimmy'ego i~jego
załogą, a~przez jej ludzi ze zdumieniem.

Kiedy herbata była zaparzona, nalała i~podała dookoła. Okazało się, że
byli inni ze składanymi kubkami, niektórzy z~supergęstymi batonami z~nasion sklejonych miodem z~ulów Gospody, super twarde i~gęste jak
starożytne słońca, wyśmienity smak domu dla kogokolwiek, kto żył w~B\&B.

Dlaczego mieli te rzeczy pochowane przy sobie? Ponieważ gdy tylko ktoś
zaczął mówić o~racjonowaniu, pęd ku gromadzenia stawał się nieodparty.

Gdy tylko podzieliła się, impuls gromadzenia się rozpłynął. Istnieje dla
Ciebie świat, na który masz nadzieję, lub którego się boisz, a~świat
jest tworzony przez Twoje nadzieje lub Twój strach. Opróżniła plecak,
znalazła księżycowe koce i~rozdała je ludziom bez płaszczy. Zdjęła swój
płaszcz, żeby mogła dostać się do podpinki i~dała ją drżącej kobiecie w~ciąży, niedawno przybyłej, której imienia nie pamiętała, potem włożyła
swój płaszcz, zanim zacznie marznąc. Płaszcz był wystarczający, nawet
stojąc nieruchomo. Miał baterie na dni i~dla temperatur znacznie
bardziej niebezpiecznych niż ta.

To uruchomiło rundę wyrównywania ubrań, ciche sprawdzenie tłumu,
przynajmniej pięćdziesiąt osób, prawie cały komplet długoterminowych w~w B\&B, i~wymiany ubrań. Zaimprowizowany rytuał zaczął się uroczyście, ale
stał się zabawny, śmiech w~twarz Jimmy'ego i~jego taktycznych
mięśniaków, chciwych dupków.

Nie wiedzieli co z~tym robić. Jimmy miał minę uwięzionego zwierzęcia,
które przypominała ze wcześniejszych sytuacji, twarz prawie na granicy
wytrzymałości, co jej się w~ogóle nie podobało. Czas zrobić ruch.

-- Dobra. -- Choć mówiła cicho, jej głos się poniósł. Natychmiast wszyscy
się uciszyli. -- Gdzie budujemy? Ktoś?

-- Budujecie \textit{co}? -- zażądał odpowiedzi Jimmy.

-- Belki i~Brasy 2 -- powiedziała. -- Ale będziemy potrzebować lepszej
nazwy. Sequele są do bani.

-- O czym kurwa \textit{mówisz}? -- Zdecydowanie blisko punktu krytycznego.

-- Zabrałeś ten tutaj. Zrobimy lepszy.

-- Posrało Cię? Zamierzasz się poddać, bez walki?

-- Nazywamy się \textit{odchodzącymi}, ponieważ odchodzimy. -- Nie dodała
\textit{ty chujku}. Nie musiało być dodane. -- To wielki świat. Możemy
zrobić coś lepszego, nauczyć się z~błędów, które tutaj popełniliśmy. -- Patrzyła. Jego usta były otwarte. Przejęła jego jebane przedstawienie. W~każdej sekundzie, zacząłby mówić\ldots 

-- To\ldots 

-- Oczywiście -- zmiażdżyła go, jak może tylko ktoś, kto w~każdej rozmowie
musi się starać, żeby \textit{nie} przerwać -- jest duża szansa, że Ty i~Twoi kumple rozwalicie to miejsce. Kiedy je opuścicie, wrócimy i~użyjemy
go jako surowców. -- Znowu zrobiła trik z~przerwą, czekając\ldots 

-- Ty\ldots 

-- Zakładając, że nie spalicie go lub nie splądrujecie. -- Czy wpadnie
trzeci raz? Tak, na pewno\ldots 

-- Nie byłbym\ldots 

-- Prawdopodobnie planujecie zatrzymać nasze osobiste rzeczy, teraz gdy
znacjonalizowaliście nasz dom dla Ludowej Republiki Merytopii? -- Jeżeli
ograniczysz sarkazm za każdym razem, gdy się pojawi, stanie się
podstępny. Ten trafił go tak prosto w~jego mentalne jaja, że mogłaś go
usłyszeć. Cztery razy wtrąciła się w~jego słowa, zanim mógł je
wypowiedzieć i~potem, bum, ominęła go. To było tak dobre, że aż
nieprzyzwoite. Ale jebać to. Ten kutas ukradł jej dom.

-- Słuchaj\ldots  -- Tym razem zrobił to \textit{sobie}, nie mógł uwierzyć, że
mógł coś powiedzieć, potknął się o~własny język. Jego dupkobracia
zachichotali. Był kompleksowo rozwalony, metaforyczne majtki w~dole.
Jego twarz poczerwieniała. -- Nie musimy tego robić\ldots 

-- Myślę, że musimy. Wyjaśniłeś nam to, że jesteś tak zafascynowany tym
miejscem, że narzucisz swoją wolę nad nim. Ukazałeś się jako potwór.
Kiedy spotykasz potwora, wycofujesz się i~pozwalasz mu gryźć jakąkolwiek
kość, która mu się podoba. Tutaj jest wiele innych kości. Wiemy, jak
robić kości. Możemy żyć jakby to były pierwsze dni lepszego świata, nie
jakby to były pierwsze strony powieści Ayn Rand. Miej to miejsce, ale
nas nie możesz mieć. Wycofujemy nasze towarzystwo.

Wpadł na bystry pomysł. 

-- Myślałem, że tutaj nie ma przywódcy. Co to za
gówno ,,my''? Nie widzicie, że manipuluje wami wszystkimi\ldots 

Podniosła dłoń. Zamilkł. Nie powiedziała nic innego, trzymała dłoń w~górze. Etcetera, chwała mu, następny podniósł rękę. Chwile później,
wszyscy mieli ręce w~górze.

-- Zagłosowaliśmy -- powiedziała. -- Przegrałeś.

Jeden z~jego skrzywionych -- co musiał im opowiedzieć, zastanawiała się,
o tym miejscu -- powiedział serdeczne ,,Kuuurde'', jakie kiedykolwiek
zdobyła.

-- Dostaniemy nasze rzeczy, Jimmy?

-- Nie -- powiedział, błogosławione jego palce i~kostki. Zacisnął szczękę,
zrobił buntowniczą minę. -- Nie. Pierdolcie się wszyscy.

To byłaby zimna noc, ale nie za zimna. Wiedzieli, gdzie są na wpół
rozebrane budynki, w~których mogliby znaleźć schronienie i~mieli przy
sobie dużo tego i~tamtego. Kiedy dostaliby się w~zasięg sieci
odchodnickie, opowiedzieliby historię -- film był robiony przez dziesięć
mrugających soczewek, które mogła dojrzeć -- i~polegaliby na uprzejmości
obcych. Odbudowaliby.

\textit{Logiczne}, nie musiała mówić. Jakkolwiek straszne rzeczy stały się
tej nocy. Jakkolwiek dużo pracy zrobiliby w~nadchodzących latach.
Jakkolwiek wiele obolałych mięśni, pęcherzy na dłoniach i~skręconych nóg
znieśliby, wszyscy pamiętaliby Jimmy'ego. Pamiętali, co się zdarzyło,
gdy choroba specjalnej śnieżynki rozwinęła się niepowstrzymana.
Zbudowaliby coś większego, piękniejszego. Uniknęliby błędów, które
zrobili ostatnim razem, zrobiliby zamiast tego nowe ekscytujące. Onsen
byłaby niesamowita. Ich plany zostały skopiowane dziesiątki razy, od
kiedy je opublikowali, niektóre z~dodatków były przepiękne. Gdy zaczęła
stawiać jedną nogę przed drugą, jej umysł pobiegł do tych myśli, plany
przyjęły formę.

Dziewczyna, Lodołasica, weszła w~krok. Szły, skrzyp, skrzyp, hff, hfff,
przez lasy. 

-- Limpopo?

-- O czym myślisz?

-- Nie bierz tego mi za złe, ale czy Ty kurwa żartujesz?

-- Nie.

-- Ale to jest szaleństwo! Zbudowałaś to miejsce. Po prostu pozwolisz mu
je zabrać!

-- Nie było moje, nie zbudowałam go. Nie pozwoliłam mu go zabrać.

Praktycznie usłyszała bardzo rafinowane przewrócenie oczu z~wychowania,
pieniędzy i~przywilejów. Ktoś jak Lodołasica nigdy nie musiała odejść od
niczego, co chciała. Armia prawników i~mięśni tego pilnowała. Dla niej
to była podróż rozszerzająca horyzonty. Praktycznie dobry uczynek.
Limpopo ziewnęła, żeby ukryć uśmiech, zanim mogła zażenować Lodołasicę.

-- Ty i~ja obie wiemy, że pracowałaś więcej w~tym miejscu niż ktokolwiek
inny.

Wzruszyła ramionami. 

-- Dlaczego to sprawia, że to moje?

-- Proszę. No to nie jest twoje-twoje, ale ciągle jest \textit{twoje}.
Twoje i~wszystkich innych lub inaczej ortodoksyjny kościół odchodzących
nalega, żebyśmy to przedyskutowali, ale nie bądź śmieszna. Pan Twardziel
nie zrobił nic dla tego miejsca, wy zrobiliście wszystko, a~Ty
przekazałaś mu je bez walki.

-- Dlaczego walka byłaby lepsza niż zrobienie czegoś innego na
podobieństwo Belek i~Brasów, ale lepszego?

-- To jest najbardziej bezcelowy dialog sokratyczny na świecie, Limpopo.
Dobra, jeżeli walczyłabyś, dostałabyś Belki i~Brasy. Wtedy, jeżeli
chciałabyś czegoś innego, miejsca lepszego, mogłabyś też to zbudować.

Limpopo spojrzała ponad ramieniem. Wpadli w~długi krok podczas rozmowy,
zostawili kolumnę uchodźców za nimi. Odwinęła izolowane siedzenie jej
płaszcza i~usadowiła się na zaśnieżonej skale, upewniając się, że
elastyczna pianka rozwinęła się pod jej tyłkiem i~nogami, upewniając
się, że śnieg nie dotyka niczego innego, prócz tego. Lodołasica też tak
zrobiła, poszło jej świetnie. Limpopo lubiła oglądać ludzi, którzy byli
dobrzy w~różnych sprawach, którzy uważali i~praktykowali, co było
wszystkim, o~co świat naprawdę prosił.

-- Nie próbuję być głupia -- powiedziała. Wyciągnęła waper i~załadowała go
z krakiem bez kofeiny, który utrzymałby ją w~ruchu przez trzy godziny,
których potrzebowała, żeby dotrzeć do kolejnej osady odchodzących.
Lodołasica zaciągnęła się dwa razy, potem wzięła jeszcze jeden, mimo że
wszystko po pierwszym zaciągnięciu było obojętne, nie zmieniłoby
niczego, tylko zamieniło mocz w~jaskrawy pomarańcz. Psychologiczne
efekty palenia fajki były uspokajające. Zaciągnęła się raz jeszcze.

-- Nie próbuję być głupia -- powiedziała znowu, obserwując chmury
chrupiącej mgły, która unosiła się przed jej twarzą, wstrząśnięta wagą,
która unosiła się z~jej mięśni, poczuciem zwiniętej mocy. Obie
zachichotały najaranym zrozumieniem obecnej komedii. 

-- Musisz zrozumieć,
że jeżeli wykorzystam Twoją ramę odniesienia, ramę odniesienia, którą
chcesz mi narzucić, to nie ma żadnego sensu.

-- Jedyny sposób, kiedy to ma sens, to gdy będę nalegać, że nie mogę
,,mieć'' więcej niż jednej Gospody. Jedynym żądaniem, które mogę mieć,
to że robię dobrze, zostając tam i~na odwrót. Co dobrego zrobię dla
Gospody, gdy odejdę? Co dobrego zrobi dla mnie? Jeżeli mam gdzie zostać,
dla mnie jest ok.

-- Ta, tak. Co z~innymi ludźmy, którzy chcieli zostać w~B\&B, ale muszą
sobie radzić z~Kapitanem Dupą i~jego Ligą Odpadów, żeby dostać łóżko?

-- Planuję budowlę gdzieś indziej. Mam nadzieję, że pomogą mi w~budowie.
Mam nadzieję, że zostaniesz i~pomożesz.

-- Oczywiście. Wszyscy zamierzamy to wybudować. Ale kiedy przyjdą i~zabiorą to\ldots 

-- Może wrócę do B\&B. Nie ma znaczenia. Ważne jest przekonanie ludzi do
robienia i~dzielenia się użytecznymi rzeczami. Walka z~chwiwymi dupkami,
którzy się nie dzielą, nie pomaga. Robienie więcej, życie w~warunkach
dostatku, to pomaga.

Spojrzenie, które dostała od młodszej kobiety, było tak przenikliwe, że
wszystko chciała wyznać. Lub może to był krak. 

-- Przyznam się. Czułam, że Gospoda była ,,moja'', jakby moja praca w~niej uprawniała mnie do
niej. Prawda jest taka, że możesz mieć rację i~robiłam więcej niż inni,
co nie znaczy, że nie zbudowałabym tego bez nich. B\&B to więcej niż
jakakolwiek jedna osoba może zbudować, nawet w~ciągu życia. Budowanie,
prowadzenie, to zadania \textit{nadludzkie}, więcej niż pojedyncza osoba
mogłaby zrobić. Jest wiele sposobów, żeby być nadczłowiekiem. Możesz
oszukać innych, że póki nie zrobią tego, co chcesz, nie będą jedli.
Możesz nakłonić ludzi, żeby robili to, co chcesz, używając strachu przed
bogiem lub policją, lub sprawiając, że czują się winni lub wściekli.

-- Najlepszym sposobem, żeby zostać nadczłowiekiem, to robić rzeczy,
które kochasz, z~ludźmi, których też kochasz. Jedynym sposobem na to
jest przyznanie, że robisz to, ponieważ to kochasz i~jeżeli robisz
więcej niż inni, ciągle robisz to tylko dlatego, że to jest to, co
wybrałaś.

Lodołasica patrzyła na swoje rękawiczki, delikatnie zaciskając palce, co
sprawiło, że Limpopo też tak chciała robić, sympatyczne wiercenie się. 

-- Czy to nie przygnębia? Cała ta praca?

-- Trochę. Ale to jest podniecające. Zaczynanie od nowa pozwala Ci
zobaczyć, jak skokowo się rozwijają rzeczy. Kiedy będzie zbudowane,
potem tylko naprawiasz, nowa farba i~drobne wymiany wystroju. Patrzenie,
jak wyrównane miejsce i~stos zebranych wznoszą się w~niebo i~stają się
\textit{miejscem}, wzajemny wpływ jego software'u i~Ciebie, tak, że
gdziekolwiek jesteś, nieważne co robisz, jest coś, co mogłabyś zrobić
lepiej, to jest \textit{niesamowite}. -- Krak syczał, i~jak zawsze poczuła
przelotną melancholię, gdy podkreślał odejście. -- Nie zmieniając tematu,
ale\ldots 

Nadchodziła reszta grupy. Po minucie, dwóch, maszerowały.

-- Znasz to -- powiedziała Limpopo, wymachując vaperem, który zręcznie jej
zabrała z~rąk Lodołasica, zaciągając się znowu i~wydmuchując chmurę
pachnącej chmury, jak żywica sosnowa i~spalony plastik, zapach domu. -- To poczucie szczęścia i~intensywności? Czy zastanawiałaś się, czy jest
to coś, co powinnyśmy doświadczać tylko przelotnie? Przykładowo orgazmy.
Gdybyś miała orgazmy, które nie przestają, to byłoby brutalne. Byłoby
poczucie, w~którym jest to technicznie niesamowite, ale doświadczenie
byłoby straszne. Albo szczęście, to poczucia przyjazdu, poczucie
doskonałego świata przez chwilę, czy mogłabyś sobie wyobrazić, że to
trwa? Dlaczego chciałabyś się ruszać? Myślę, że jesteśmy wyekwipowani w~doświadczanie szczęścia tylko na chwilę, ponieważ wszyscy nasi
przodkowie, którzy mogli doświadczać tego dłużej, zachwycali się, aż
umarli z~głodu lub zostali pożarci przez tygrysa.

-- Ciągle jesteś najarana -- powiedziała Lodołasica.

Limpopo sprawdziła. 

-- Tak. -- Grupa była koło nich. -- Zanika. Chodźmy.

Dołączyły do kolumny i~maszerowały.


\part{startowanie}

\chapter*{i}

Popioły Odchodzącego Uniwersytetu były dookoła Lodołasicy. To był
niepewny klimatycznie dzień, gdzie oberwania chmur pojawiały się znikąd,
zlewały wszystko i~znikały, zostawiając płonące słońce i~wysoki ton
komarów. Popioły były namoczone, a~teraz zapieczone w~cegły żużlu z~izolacji nanofibrowych i~radiatorów, strukturalne kartony wzbogacane
molekułami długołańcuchowymi, które odgazowywały alarmująco, oraz
niezróżnicowana czarna sadza rzeczy, które stały się tak gorące w~płomieniach, że nie mogłabyś określić, czym były.

W tym żużlu byli ludzie. Sieć czujników OU przetrwała dostatecznie
długo, żeby zaalarmować o~nieprzytomnych ludziach dookoła, w~pułapce
płomieni lub gazów. W~tych szczątkach, które wkradały się wokół maski i~zostawiały smak spalonego tostu na języku, były zwęglone kości.
Dławiłaby się, gdyby nie Meta, którą wydrukowała, zanim wyruszyła w~drogę.

Banana i~Bongo było większe niż Belki i~Brasy kiedykolwiek były, siedem
pięter, trzy warsztaty i~prawdziwe stajnie dla różnych pojazdów od
trójkołowców ATV przez chodzące mecha do sterowców\dywiz trzmieli, które
zajmowały Etcetera przez więcej niż dwa lata, gdy śmigał po niebie,
couch-surfując po koloniach odchodzących i~obozach na całym kontynencie.
Myślała o~zabraniu mecha do uniwerku, ponieważ to było niesamowite
chodzenie tak szybko po krajobrazie, dalmierze skafandra i~lidar
odszukiwały dobre miejsce do postawienia potężnej stopy, żyro i~balast
tańczące z~grawitacją, żeby utrzymać ją w~pionie przez kilometry.

Jednak mecha nie miały przestrzeni ładunkowej, zatem wzięła trójkołowca
z oponami balonowymi tak wielkimi jak traktorowe, ciągnące pociąg kapsuł
ładunkowych ze sprzętem awaryjnym. Zabrało jej cztery godziny, żeby
dotrzeć do uniwersytetu, wtedy już ocaleńcy się rozproszyli. Podniosła
trzmiele sieci we wzorcu pokrycia, poszukując emisji radiowych ocalałych
osób. Trzmiele same się nadymały, ale nadal to była męcząca praca, żeby
wyciągnąć je z~zasobników i~posłać w~powietrze, a~chociaż pracowała
szybko -- dokładnie meta szybko, jak żołnierz składający karabin z~zasłoniętymi oczami -- wszystko było wymazane sadzą, zanim były w~powietrzu.

-- Jebać to -- powiedziała do swojej maski i~obróciła ATV i~szereg
ładunków dookoła w~łoskoczący pączek z~dziurką. Ocaleńcy byliby
niedaleko, na wiatr od popiołów i~poza zasięgiem gorąca, które musiało
powstać, kiedy campus płonął. Widziała wcześniej demo wybuchającego
radiatora budynku. To było przerażające. W~teorii, ściany z~domieszką
grafenem odbierały ciepło, prowadząc je do powierzchni w~blasku,
utrzymując obszar dookoła ognia poniżej punktu zapłonu. Radiator sam w~sobie był mniej łatwopalny niż cokolwiek inne, co używali jako materiały
budowlane, zatem jeżeli ogień trwał zbyt długi, radiatory podgrzewały
się do punktu zapłonu ścian, a~wtedy cały budynek wybuchał w~prawie
jednoczesnym \textit{bum}. W~teorii, nie mogłaś dotrzeć do tych
temperatur, chyba że osiem środków zaradczych naraz nie zadziałało,
ściśle na sposób podpalacza z~wyposażeneim państwowym.

Próbowała nie myśleć o~podmiotach państwowych i~dlaczego chciały
zredukować Uniwersytet Odchodzących Półwyspu Niagara do węgla.

Trzmiele odpowiedziały. Coś użyło ich do połączenia się do sieci
odchodzących, kilka klików w~górę wiatru, tak jak myślała. Przy
odrobinie szczęścia, to byliby uchodźcy, a~nie inni pracownicy
humanitarni, lub gorzej ghule-szabrownicy.

Trzmiele wykorzystały swoje wirniki niskiej mocy i~balast, żeby
oportunistycznie wymanewrować się w~stabilny trójkąt nad strefą, potem
użyły czasów sygnałów, żeby określić współrzędne. Zrobiły zdjęcia, ale
widziała tylko baldachim, odległy od spłonięcia. Trudno było powiedzieć,
ale sądziła, że mogły tam być przecinki, które służyły jako przerwy
przeciwpożarowe.

Kopnęła starter trójkołowca i~pojechała w~tamtym kierunku, przesuwając
językiem po ustach, żeby pozbyć się gorzkiego smaku.

Niedługo potem musiała zsiąść. Krzaki były zbyt gęste dla ATV, żeby się
przedostał, nie mówiąc już o~wagonikach z~ładunkiem. Rozciągnęła się,
dotknęła palcy u nóg, zamachała ramionami. Podróż ukarała jej tyłek i~plecy. Jej ręce bolały od trzymania rączek. Myślała o~vapingu, może
odrobinie kraka, ale kiedy poruszyła odrobinę maską, jej usta i~nos
zostały zalane przez gorzkie powietrze z~pola popiołu. Jebać to, meta
wystarczyłaby, nawet jeżeli dawka się zużywała. Powinna zrobić to w~formie plastra, żeby mogła przyklepnąć jeden bez wdychania toksycznego
miksu plastiku, węgla i~upieczonego człowieka.

Droga przez las ulżyła jej mięśniom i~umysłowi. Ptaki śpiewały
zaalarmowane, ale uspokajające pieśni, gdy oceniały szkody od pożaru.
Zwykła wychodzić na dach domu jej taty, żeby posłuchać ptaków wołających
w Don Valley. Dźwięk był pierwotnie uspokajający.

Gdy się zbliżała, rozglądała się i~słuchała za znakami aktywności
ludzkiej, ale było dziwnie pierwotnie. Właśnie miała zawrócić do ATV,
żeby powtórzyć pomiary trzmieli, zakładając, że miały błąd, gdy
zauważyła antenę.

To było sztuczne drzewo, niezbyt dobre, ale schowane pośród innych, że
nie zauważyła go natychmiast. To była sosna jak plastikowe drzewko
bożonarodzeniowe. Pomiędzy gałęziami były charakterystyczne występy
macierzy anten, takiej samej, które można było zauważyć dookoła Banana i~Bongo. Kopnęła tam, gdzie powinny być korzenie i~zauważyła, że były
solidnie w~ziemi.

-- Halo? -- Gdzie były anteny, były kamery, choćby po to, żeby wysłać
zdjęcie, kiedy rzeczy poszły w~cholerę. Mogły być minimalne, że nie
zauważyłaby, ale niedaleko. 

-- Halo? -- powiedziała znowu.

-- Tędy -- powiedział kobieta. Była pomarszczona i~szczupła, skóra koloru
drewna tekowego i~siwe włosy w~poszarpanej fryzurze chłopczycy. Wyszła z~lasów po drugiej stronie anteny, nosiła maskę, ale wyglądała przyjaźnie.
Może to była meta.

Lodołasica podeszła do niej, gdy szła przez busz. Lodołasica poszła za
nią. Doszły do granitowego występu, kanadyjskiej tarczy wepchniętej w~glebę. Kobieta pchnęła to i~odsunęła na wsporniku. Było ciche, co mówiło
o talencie inżynierów. To musiało ważyć jebaną tonę, co Lodołasica
odkryła, kiedy nie uskoczyła z~drogi i~prawie przewróciła się, gdy ją
dotknęło.

-- Chodź -- powiedziała starsza kobieta. 

Za skałą był wąski korytarz ze
ścianami z~ubitej ziemi, podświetlone przez kule LED wepchnięte prosto w~ziemię z~kruchymi kraterami uderzeniowymi dookoła każdej z~nich. Kobieta
przecisnęła się koło niej -- Lodołasica zobaczyła, że jej zmarszczki były
pokryte sadzą, co sprawiało, że były ciemniejsze niż naprawdę były -- i~zamknęła drzwi z~uderzeniem, która zarezonowało przez podeszwy butów
Lodołasicy.

-- W~górę -- powiedziała kobieta. 

Lodołasica nacisnęła. Dookoła zakrętu,
weszła niespodziewanie w~doskonale okrągły tunel, wyższy od niej, z~gładkimi ścianami i~śladami narzędzie od maszyn wiercących. Ściany były
twarde i~czyste, oświetlenie bardziej przemyślane, rozmieszczone z~maszynową precyzją.

Dziwna kobieta zdjęła maskę. Była to piękna kobieta pochodzenia
hinduskiego, lub Desi, siwe brwi i~delikatny ciemny wąsik. Uśmiechnęła
się, jej zęby białe i~równe. 

-- Witamy w~zapasowym kampusie Odchodzącego
Uniwersytetu.

\chapter*{ii}

Nazywała się Sita. Uścisnęła Lodołasicę. Lodołasica wyjaśniła, że
przywiozła zaopatrzenie.

-- Mamy tutaj wiele -- odpowiedziała Sita -- ale są rzeczy, których
będziemy potrzebować do odbudowy.

Przeszli korytarzem, ku odległym głosom. 

-- Rozpaczamy, oczywiście, ale
najważniejsze, że uratowaliśmy całą pracę, próbki, kultury. Dane zawsze
miały kopię zapasową, więc tutaj nie ma ryzyka.

-- Ilu nie żyje?

Sita się zatrzymała. 

-- Nie wiemy. Albo wielka liczba, albo nikt w~ogóle.

Lodołasica zastanawiała, czy Sita oszalała, przez rozpacz lub zatrucie
dymem czy egzotyczny bioczynnik. Maska Sity wisiała na szyi, a~maska
Lodołasicy ściągała jej włosy i~podrażniała twarz, więc pchnęła ją na
czoło, stukając w~zapomniane gogle, które skończyły w~jej włosach.

Nawet przy tych przykrościach, ulga ze swobodnego oddychania i~patrzenie
bez zasmarowane soczewki polepszyły jej humor.

-- Możesz wyjaśnić?

-- Prawdopodobnie -- powiedziała. -- Ale może później. W~międzyczasie,
zorganizujmy grupę roboczą i~rozładujmy Twoje zapasy.

Podziemne korytarze zamieniły się podziemny amfiteatr, wspierany przez
słupy i~więźby dachowe i~coś bardziej istotnego niż aerozol, żeby
powstrzymać ziemię od zawalania.

-- To było wcześniej superzderzaczem -- powiedział Sita, gdy Lodołasica
się gapiła. W~jednym kącie był szpital, mesa, przestrzenie robocze,
gdzie czarni od sadzy ludzie prowadzili intensywne dyskusje, które
prawie były bijatykami. -- Wiertło działało przez miesiące. Jednak fizycy
dostali to, co szukali gdzieś indziej, nie pytaj się mnie, fizyka
cząstek nie jest moją dyscypliną, i~się wynieśli. Zanim jeszcze odeśli,
byliśmy gotowi. Potem kiedy rozdzieliliśmy się w~skany i~symulacje,
starsi martwili się o~wysadzenie z~Ziemi i~zbudowali schron. Zabrało
kilka lat pracy w~większości zaautomatyzowanej. Nie jest piękne, ale
wystarczy. Nawet nie wiedziałam, że tutaj było, aż do wczoraj, kiedy
zaczął się ogień. Zaskoczyło mnie to jak cholera! Nie wiem, co było
dziwniejsze, że tym ludziom udało się zbudować podziemne miasto czy że
udało im się zachować tajemnicę.

-- Lub może to nie była tajemnica? Może to tylko ja nie wiedziałam.
Jednak to paranoja. Nie sądzisz?

Cokolwiek działo się Sitą, nie było przyjemne. Osunęła się na ścianę z~ubitej ziemi poznaczonej grubymi przewodami, które biegły przez belki
sufitu i~znikały w~rozgałęziających się tunelach. Wyglądała na starszą
niż kiedy się poznały.

-- Vapa? -- spytała Lodołasica. -- To meta. Dobra na taką sytuację.

-- Dzięki. -- Dzieliły się towarzyskim zaciągnięciem. Kilka sekund
później, obie uśmiechały się krzywo. -- Głodna? Mamy żarcie, niedużo, ale
jeżeli mamy przynieść Twoje zaopatrzenie, posiłek jest w~porządku.

-- Nie bardzo. Zabierzmy wszystko, zanim zostanie zbombardowane z~orbity.

-- Nie żartuj.

Meta pomogła Sicie i~przeszła do stołu młodszej kobiety i~kilku mężczyzn
i przedstawiła Lodołasicę. Większość przy stole miała proste imiona jak
Sita, ale był tam jeden facet, który nazywał się Latarnik, jedyne imię,
które Lodołasica pamiętała dziesięć sekund później. Dali jej kubek
coffium, jednocześnie szukając tragarzy do grupy roboczej. Ktoś zatupał,
nosząc drobny mecha exo, była też para osłów, wysoko kroczących i~zataczających się z~boku na bok, gdy ich firmware oceniał teren raz po
razie, nigdy nie ufając gruntowi, że się nie podda. Osły były wolne, ale
wykonają pracę.

-- Chodźmy. -- Sita założyła maskę. Wzdychając, Lodołasica ściągnęła
swoją. Żałowała, że nie przyjęła jedzenie, nie dlatego, że była głodna,
ale dlatego, że chciała posiedzieć i~dowiedzieć, co się do cholery
zdarzyło.

Wyszli przez wahadłowy głaz, poszli rzędem przez gęste zarośla do
trójkołowca i~jego zasobników towarowych. Na wpół wierzyła, że będą
stopione do żużla kolejny atakiem dronu, ale były nienaruszone. Kapsuły
westchnęły przy otwieraniu, gdy zamaskowani tragarze sformowali żywy
łańcuch jak przy gaszeniu pożarów.

Żywy Łańcuch ucieleśniał filozofię odchodzących, bardziej emblematyczny
niż kłótnie o~konsensus w~kręgu krzeseł. Lodołasica uczestniczyła w~niektórych łańcuchach w~defaulcie, przenosząc towary na imprezy
komunistyczne, ale nigdy w~żadnej z~fantazją spacerujących łańcuchów.
Drużyny żywego łańcucha prosiły tylko, żebyś pracowała tak mocno, jak
chcesz, śpiesz się po nowy ładunek tam i~z powrotem, żeby przekazać, lub
chodź spokojnie pomiędzy nimi, lub zmieniaj prędkość. Nie miało to
znaczenia, jeżeli chodziłaś szybciej, to oznaczało, że ludzie po obu
stronach nie musieli chodzić tak daleko, ale to nie wymagało od nich,
żeby chodzili szybciej lub wolniej. Jeżeli zwalniałaś, wszyscy inni
nadal utrzymywali tę samą prędkość. Żywy Łańcuch były systemem, w~którym
każda mogła zrobić to, co chciała -- w~obrębie systemu -- jakkolwiek
szybko chciała iść, wszystko, co robiłaś, pomagało, nic z~tego nie
zwalniało innych.

W Banana i~Bongo, na krótko dołączyła do łańcucha ładującego. Limpopo
chciała przekazać jej więcej wskazówek na temat bezpieczeństwa, trzy
razy sprawdzić jej sprzęt i~zestawy ratunkowe. Poddała się temu
łaskawie, ponieważ to było miłe, gdy ktoś pilnował Twojej dupy,
upewniając się, żeby nie wpadła w~zbyt duże kłopoty, gdy biegła w~ich
kierunku tak szybko, jak potrafiła. To stało się jej modus operandi w~trakcie budowy Gospody, pierwsza na scenie, kiedy drony znalazły
materiały, wychodząc dalej z~mniejszym zapasami niż ktokolwiek, licząc
na minimum sprzętu i~uprzejmość obcych oraz szczęście, żeby pozostać
przy życiu. Przeszła od bycia największym szleperem na świecie do kogoś,
kto prychał na zabieranie dodatkowej bielizny (po to są hydrofobowe
tkaniny odrzucające brud z~domieszką srebra).

Limpopo fachowo sprawdziła jej sprzęt, naciskała na dodatkowe sześć
litrów wody i~lekką wet-drukarkę, która mogłaby wydrukować leki.
Wiedziała lepiej, żeby się nie sprzeciwiać, ale tak zrobiła, ustępując,
dopiero gdy Limpopo położyła dłonie na niej i~położyła taki ciężar z~takim doświadczeniem, że w~ogóle nie zauważyła. 

-- Wiesz, z~całą tą wodą,
okaże się, że ciągle piję i~zatrzymuję się na sikanie.

-- Sikaj czysto. -- To było błogosławieństwo odchodzących, szczególnie w~trybie koczowniczym. Uprzejmością było zaoferowanie nieproszonych opinii
na temat uryny sąsiadki. Celem była czysta. Cokolwiek ciemniejszego niż
żonkil było podstawą do wmuszania w~Ciebie wody. Jeżeli siki były
pomarańczowe lub brązowe, zostałabyś pasywnie i~agresywnie zmuszona do
picia toniku soli elektrolitycznych oraz przetrwania protekcjonalności
swoich znajomych za pozwolenie, żeby Twoja endokrynologia oddała to, co
najlepsze. Mogłaś wydrukować bieliznę, przez którą można było sikać w~trakcie ruchu, to było dziwaczne przez sekundę i~neutralizowało
cokolwiek nieprzyjemnego i~niebezpiecznego. To była uboczna korzyść z~uwagi i~przetwarzania Twojego nawodnienia i~rozpuszczonych stałych, ale
nikt ich nie nosił, ponieważ a) sikanie w~spodniach było obrzydliwe i~b)
patrz punkt a.

Limpopo posłała ją z~pocałunkiem, który tylko częściowo był
macierzyński. Uśmiech, który ten pocałunek dał, trwał przez godzinę na
trójkołowcu. Ona, Seth i~Etcetera byli jak elektrony orbitujące dookoła
jądra Limpopo, wszyscy próbując wskoczyć na orbity o~wyższych energiach.
Było w~niej coś grawitacyjnego.

Ten typ marzeń był łatwy w~żywym łańcuchu, nawet w~masce i~goglach z~posmakiem spalonej opony w~ustach. To była kombinacja bezmyślnej pracy i~wydajności, i~gdy zaczęła się pocić, rytm linii się ustalił.

Najlepszą częścią żywego łańcucha jest to, że kiedy ładunek się
skończył, naturalnie sprowadzał wszystkich na krańcu, ponieważ szłaś w~górę, póki miałaś ładunek, a~jeżeli nie było więcej ładunku, wszyscy
szli całą drogę. Zebrali się przy ATV i~przedyskutowali to.

-- Nie ma powodu, żeby to kamuflować -- powiedziała Sita. -- Cokolwiek, co
lata ponad i~zauważy, dojdzie do tego, że to pojazd pomocy humanitarnej,
to naturalne. To nie ujawnia prawdy o~podziemiach.

-- Ale pojazd wskazuje, że są tutaj ludzie, którzy potrzebują pomocy. -- To był facet z~szaloną fryzurą, niebiesko-zieloną z~kędziorami Einsteina
po bokach i~łysiną na szczycie. Miał może sześćdziesiąt lat, z~niespodziewanie piękną twarzą, jak elf lasów. Teraz gdy Lodołasica o~tym
pomyślała, ci odchodnicy byli kilka sigma starsi od mediany wieku
odchodzących. Część mózgu, która próbowała wyjaśnić, dlaczego ktoś w~rzeczywistości próbował ich zbombardować, odłożyła to.

-- Cokolwiek zrobimy, będzie bezużyteczne -- powiedziała kolejna starsza
kobieta, niska i~hipisowska, sylwetka klepsydry i~gigantyczne piersi,
takimi jak wszystkie kobiety rysowane przez Lodołasicę w~dzieciństwie. -- Zakamuflowany trójkołowiec nie będzie wyglądał jak las dla rozsądnego
przetwarzania obrazu. Będzie wyglądał jak coś ukrytego.

-- To załatwia sprawę -- powiedziała Sita. A do Lodołasicy: 

-- Gretyl jest
najlepszą specjalistką od optymalizacji obliczeniowej, jeżeli tak mówi,
to prawda.

-- Argument z~autorytetu -- powiedział inny facet dobrodusznie.

-- Im dłużej tu stoimy, tym większa szansa, że nas wykryją -- powiedziała
Sita.

-- Egoistyczna bzdura.

-- W~jadalni jest whisky -- powiedziała.

-- Teraz słusznie mówisz. -- Poszły.

\threeast

Dobrze się nią zajęli. Pojawiła się świeża załoga, która spała w~trakcie
rozładowania, która zabezpieczyła zasoby, które przynieśli. Ludzie, z~którymi była na zewnątrz, adoptowali, produkując krzesło i~składając je
dla niej, nalegając, żeby spoczęła, gdy przynosili śniadanie, jogurt
usiany pistacjami i~modyfikowanymi kulturami, która jak zapewniali,
zmniejszyłyby jej stres, co wyjaśniało, dlaczego byli tak kurwa spokojni
mimo pożaru.

Dali jej szklankę czegoś słodkiego i~bąbelkowego, stukającego lodem.
Myślała, że to mogła być wódka, ale nie potrafiła powiedzieć. 

-- Co
dokładnie tutaj robiliście, że zostaliście zbombardowani z~orbity?

-- To było miłosna pieszczota -- powiedziała Gretyl. -- Nic, w~porównaniu
do uderzenia w~Somalii.

Niektórzy w~Banana i~Bongo mieli obsesję o~globalnych odchodzących, ale
Lodołasica z~trudem śledziła. Była mętnie świadoma o~kontyngencie
odchodzących subsaharyjskich.

-- Somalia?

Gretyl oceniła ją lepiej, niż zasługiwała. 

-- Nie do końca Somalia,
rozumiem spór, ale ostatnia strefa uderzenia była w~obrębie granic
Somalii, zatem tak to nazywamy dla wygody. To nie czas na pedanterię.

-- Nie jestem pedantyczna, po prostu nie wiem, o~czym mówicie. -- Odchodzący uniwersyteccy patrzyli na nią, jakby była idiotką. To było w~porządku: ludziom zależało na rzeczach, na którym jej nigdy nie
zależało. Pogodziła się z~posiadaniem priorytetów, które były inne od
wszystkich innych osób, zaczynając od jej jebanego ojca.

-- Kampus w~Somalii, lub w~miejscu, które było Somalią, został załatwiony
w zeszłym miesiącu -- powiedziała Sita. -- Nawet nie wiemy, co w~nich
trafiło. Dosłownie nic nie zostało. Obrazy satelitarne pokazują płaski
teren. Nawet nie ma śladów odłamków. To jakby nigdy nie istnieli,
dziesięć hektarów laboratoriów i~sal klasowych, po prostu\ldots  znikła.

Lodołasica poczuła ciarki na plecach. 

-- Jak myślicie, co w~nich trafiło?
Sądzicie, że możecie być następni do ataku?

Sita wzruszyła ramionami. 

-- Jest wiele teorii, możliwe, że oni ich
spalili tak jak nas, ale byli szczególnie praktyczni ze sprzątaniem,
robiąc to pomiędzy przejściami satelitów. To podejście brzytwy Ockhama,
jak wszystko inne zakłada fundamentalne przełomy technologiczne. Ale
takie się pojawiają, bogini wie.

Gretyl włączyła się gładko do rozmowy, kładąc palce płasko na stole. 

-- Co sprowadza nas do Twojego oryginalnego pytania, nad czym pracujemy, co
sprawiłoby, że ktoś z~defaulta chciałby nas zredukować do krateru?

Przy tym, wszyscy przesunęli się, żeby spojrzeć na faceta z~niebieską
kędzierzawą fryzurą. 

-- Próbujemy znaleźć lekarstwo na śmierć -- powiedział i~uśmiechnął się do niej złośliwym uśmiechem leśnego elfa.
Nawet miał dołek w~brodzie. 

-- To raczej duża sprawa.

\chapter*{iii}

Wszyscy stłoczyli się w~szerokim bocznym korytarzu z~drinkami i~przekąskami. Jedna ze ścian była pokryta powierzchnią interfejsu i~facet-elf i~trójka jego załogi -- nie potrafiła określić, czy byli
współpracownikami, studentami czy samozwańczymi ważniakami -- mieszali w~tym, ruszając swoimi interfejsami, podrygując i~wskazując palcami na
panelu. Rozpoznała pasek postępu, poruszający się z~prędkością lodowca,
i zmuszała się do oderwania oczu od niego, ponieważ to był gówniany
pasek postępu, który nie poruszał się gładko, ale dawał fałszywą
precyzję, przeskakując szybko od 25 do 30 procent, potem stojąc w~miejscu przez wieczność przed przeskoczeniem do 31 procent, biegnąc
szybko do 41 i~tak dalej. Wiedziała dostatecznie o~jej psychologii, żeby
rozpoznać, że jej wzorzec dopasowania był tym bezużytecznie
zafascynowany. To było sporadyczne wzmocnienie, ponieważ co jakiś czas
jej podświadomość poprawnie zgadywała, kiedy skok nadchodził i~to dawało
takie uderzenie dopaminy, żeby przekonać jej głupi móżdżek o~geniuszu w~przewidywaniu losowych chwil wprowadzającego w~błąd widgetu UI.

Pasek zatrzymał się na 87 procentach na tak długo, że ktoś przyniósł
szpulę światłowodów, podczas gdy leśny elf zniknął w~serwerowni i~zrobił
coś, co sprawiło, że teraz bezpośrednio podłączony interfejs
podskakiwał.

-- Przepraszam za to -- powiedziała Sita. -- Wszystkie demonstracje, jakie
dotychczas mieliśmy, odbywały się w~znacznie lepszych warunkach. Nikt
nie myślał, że to będzie ćwiczenie na żywo w~tych okolicznościach. OC
świrował, od kiedy spadły bomby, zrozumiał, że nie grał o~główną stawkę.

OC podrażniło jej pamięć -- leśny elf był nazywany Obywatel Cyborg, takie
prototypowe imię odchodnika, że nie mogła go zapamiętać. Potem OC był z~powrotem, odepchnął innych od powierzchni interfejsu i~zrobił rzeczy.
Wtedy nastąpiło klik-pop i~dzwonek, na który pokiwał głową. Inni ludzie
rozpoznali to i~hałas spadł do prawie zera.

-- Załatwili Ci straszne laboratorium, OC -- powiedział zsyntetyzowany
głos. To był dobry głos, ale rytm był niewłaściwy. Słowa pojawiły się na
ekranie, każde słowo połączone z~chmurą wiszących danych.

-- To ma jej poczucie humoru -- powiedziała Sita. -- To dobrze.

Gretyl, koło niej, powiedziała Lodołasicy to, co ona już się domyśliła.

-- To Rozłączna. Jest ofiarą bombardowania. Jej nagranie było tylko kilka
dni temu. Myślała, że to może nadejść. OC uruchomił ją w~całym klastrze.

-- To jest mózg w~słoiku? -- powiedziała Lodołasica.

-- Umysł w~słoiku -- powiedziała Sita.

-- Mózg jest popiołem. -- Gretyl wzdrygnęła.

-- Zatem dlaczego to nie mówi ,,Gdzie jestem? Co stało się z~moim
ciałem?'' -- To były podstawy melodramatów o~transferach umysłu, formalny
wymóg gatunku.

-- Ponieważ nie startujemy symulacji w~stanie, w~jakim była zeskanowana -- powiedziała Gretyl. -- Wprowadzamy stan przejściowy, trans, i~mówimy temu,
co się stało. Każdy, kto wchodzi do skanera, wie, że tak się stanie,
eksperymentujemy z~metodami uruchamiania symów od lat, żeby odkryć
minimalnie traumatyczny sposób uświadamiania. Lub ,,uświadamiania'' -- Zrobiła palcami cudzysłów.

OC przechylił głowę, poruszył szczęką. 

-- Rozłączna, to nie są ćwiczenia.
Twoje ciało jest martwe. Scenariusz, który dostałaś w~czasie ładowania?
Prawdziwy. Jesteśmy w~bunkrze.

Sugestywne miganie kursora. Lodołasica nie widziała mrugającego kursora
poza klasami historycznymi, ale to miało sens, żeby dać mózgowi w~naczyniu sposób na wskazanie pauz. Infografiki oszalały.

-- Uruchomili sym Rozłącznej w~niskiej rozdzielczości, próbując odkryć
parametry endokrynologiczne, żeby zapobiec ześwirowaniu i~stopieniu, ale
utrzymując procesy nerwowe w~normalnej zakresie tego, co wiemy o~Roz z~jej zapisanego stanu -- wyszeptała Gretyl.

Sita pochyliła się do drugiego ucha. 

-- To jakby próbowali znaleźć dawkę
uspokajającą, która zachowa jej spokój, ale nie zamieni jej w~zombie.

-- Cholera. Robicie coś naprawdę szalonego z~moimi hormonami, czuję to.
Dajcie mi minutę sterowania autonomicznego, żeby ustalić, czy przetrwam?
Jeżeli nie, wycofajcie do tego punktu i~zacznijcie od nowa.

-- Och -- powiedział -- Rozłączna\ldots 

-- To nie pierwsze uruchomienie, od kiedy spieprzyliście? Nienawidzę
scenariuszy Dnia Świstaka.

-- Zawsze była najbystrzejsza -- powiedziała Gretyl. -- To dlatego musimy
ją uruchomić online, ona jest tą, która będzie w~stanie uruchomić całą
kohortę. Widzisz, jak szybko się tego domyśliła?

-- Dziękuję, Gretyl -- powiedział głos. -- Kto jest z~Tobą?

-- Nazywam się Lodołasica. Przyjechałam z~Banana i~Bongo z~pomocą
humanitarną.

-- Miło mi Cię poznać. -- Kolejna długa przerwa. Infografiki tańczyły.
Obserwowanie ich wydawało się zaborcze. Lodołasica nie wiedziała, gdzie
patrzeć. -- Przepraszam, nie jestem sobą.

-- Rozłączna -- powiedział OC -- wariujesz. Możemy to zobaczyć. Słuchaj,
zamierzam znowu cię uruchomić, dobra? Masz jakieś sugestie parametrów na
naszą kolejną próbę?

-- Ile Ci zostało mocy? Czy mógłbyś wydłużyć podgląd? Przećwiczyliśmy ten
scenariusz wcześniej i~byliśmy w~stanie utrzymać stabilność modelu.

-- Byłaś wtedy żywa -- powiedział OC. Infografiki rozkwitły w~szalonych
ruchach.

-- W~złym czasie powiedziane -- powiedział po cichu Lodołasica. Gretyl i~Sita skinęły.

-- Roz! Roz! -- powiedziała Sita. -- Tu Sita.

-- Wiem, że Sita. -- Brakowało zasięgu ekspresji, żeby pokazać
zgryźliwość, ale dobór słów i~rytm nie pozostawiał wątpliwości. -- Co
jest?

-- Będziemy pracować na minimalnej mocy przez miesiąc, kiedy będziemy
ładować, dłużej, w~zależności od wiatru i~słońca. Zakładając, że nas nie
zbombardują. Nie ma dostatecznie dużo mocy, żeby podglądać, tak jak
chcesz, chyba że zwolnimy cię do połowy prędkości.

-- To nie zadziała. Na połowie prędkości, nie będę w~stanie prowadzić
interakcji społecznych z~Tobą. Ekspresowy bilet do awarii.

-- Co z~tymi analitykami? -- wyszeptała Lodołasica do Sity. -- Dlaczego nie
wymyślicie jakiegoś homeostatycznego kodu, który próbuje utrzymać
wszystkie parametry w~zakresie?

-- Dlatego, że jestem nieliniowa -- powiedział głos. 

Lodołasica
przypuszczała, że jako dodatek do optyki fazowej na powierzchni, bot
Rozłącznej miał dostęp do układu mikrofonów, co oznaczało, że mógł
słuchać dowolnej konwersacji w~pokoju. Lodołasica organizowała imprezy w~Toronto, gdzie jej wielka ściana była zasilana imprezą jakiegoś innego
bogatego dzieciaka, i~była w~stanie podsłuchać każdą konwersację
oddzielnie tylko przez wskazanie. Bot, z~którym rozmawiała przez ekran,
mógł zrobić to samo.

-- Nie jestem deterministyczna. W~przeciwnym przypadku nie potrzebowaliby
procedur podglądu, żeby powstrzymać mnie od szaleństwa. Jestem czuła na
parametry wejściowe i~skłonna do osobliwości. Tak jak i~Ty. To właśnie
nas definiuje. Lub Ciebie. Nie wiem, co mnie bardziej definiuje. Och -- Nastąpiła kolejna przerwa migającego kursora. Nic z~tego nie było w~dramach o~uploadzie, które oglądała Lodołasica. Przeszła przez fazę,
głupich programów o~ludziach, którzy wsadzili mózgi w~komputery i~stali
się wieloplanowi -- ,,Wieloplanowy'' było nazwą najbardziej udanego,
zostało sprzedane jakimś zettom za coś dziewięć miliardów dolarów, z~prawami komercyjnymi -- ale miała ich dość.

Było to dlatego, że jednocześnie pożerała stare filmy o~podróżach
kosmicznych i~zrozumiała, że wszystkie te dramatyczne sytuacje od
podróży w~kosmos były spełnianiem życzeń lub zaściankowym sianiem
strachu, i~to samo było prawdą co do upload-fiction. Jakkolwiek te
sprawy wyglądają i~jakiekolwiek to miałoby mieć problemy, były
dziwniejsze i~mniej pokazowe niż w~filmach.

-- Rozumiem. -- Cokolwiek było w~jogurcie uniwersytetu, nie działało.
Lodołasica przeżywała poważny lęk społeczny. Wszyscy patrzyli i~osądzali. Oczywiście prawdopodobnie tak było. Dlaczego w~ogóle otworzyła
głupią gębę?

Przebywanie z~Limpopo nauczyło ją, że nigdy nie wygląda głupio, jeżeli
zadaje podstawowe pytania w~dobrej wierze. 

-- Tego naprawdę nie rozumiem,
dlaczego nie masz nic przeciwko restartom, czy to nie jest umieranie?

Wszyscy się ciągle patrzyli. 

-- Oczywiście. To dokładnie jak umieranie,
ale wiem, że wrócę. W~trakcie uruchamiania występuje nacisk selektywny.
Pomyśl o~tym, kiedy uruchamiacie sym taki jak ja, zaczyna prymitywnie, i~możemy przewidzieć przy niskich kosztach obliczeniowych parametry, które
będą kolejnym udanym krokiem do pełnej świadomość. -- Przerwa, mrugnięcie
kursora. -- Lub czymkolwiek jestem. Jednym z~kluczowych pytań te
potencjalne wersje otrzymują, jest ,,Czy będziesz miała kryzys
egzystencjalny, kiedy uświadomisz sobie, że jesteś symulacją?''. Możliwa
,,ja'' z~najwyższą tolerancją na bycie głową w~słoiku ma najlepsze
współczynniki na pełne uruchomienie. Jestem emergentna i~złożona, ale
wewnątrz zestawu wszystkich możliwych odpowiedzi, które mogę mieć w~odpowiedzi na tę sytuację, istnieje nie-stopienie się, zatem ten właśnie
przedział badamy, kiedy jestem uruchamiana.

-- Myślisz, ,,dobra, ale jak możesz nazywać to symulacją, jeżeli to może
tylko symulować tylko w~rzadkich okolicznościach, w~których ta rzecz
symulowana nie ma napadów i~awarii?'', ale jebać to. Teraz możemy to
zrobić, to będzie kwestia czasu, kiedy zmarli przewyższą żyjących, a~wszyscy zmarli będą wersjami samych siebie, które nie mają
egzystencjalnych napadów. To kognitywnie wąskie gardło, przez które
musimy przecisnąć ludzką rasę\ldots 

-- Nie myślałam o~tym w~ogóle -- powiedziała Lodołasica. -- Na tyle, o~ile
jestem zainteresowana, jesteś osobą i~cokolwiek myślisz, jest twoją
własną sprawą.

-- Jeżeli nie myślałaś o~tym, prawdopodobnie nie jesteś zbyt bystra. Bez
urazy.

-- Nie bądź dupkiem, Roz -- włączył się OC.

-- Nie jestem dupkiem. Po prostu nie rozumiem, jak osoba cielesna może
rozważać, czym się stałam, bez drobnego niepokoju egzystencjalnego. To
nienaturalne.

Lodołasica nie mogła się nie roześmiać. To było wyczerpanie nerwowe, nie
wspominając zdumienia, razem podwójnie na nią wpłynęło.

Ku jej osłupieniu, bot też się roześmiał. Najdziwniejszą rzeczą w~syntetycznym śmiechu było jak naturalnie brzmiał. Bardziej naturalnie
niż mowa.

-- Dobra, walić naturę. Co dziwniejsze, jestem świrem i~Ty też, oboje
jesteśmy skrzywieni przez naszą platformę obliczeniową. Jaki masz
pomysł?

-- Wiem, że nie jestem ekspertem, ale jeżeli przygotowałaś się do życia w~obrębie, hm, ,,ograniczonej skrzynki'', żeby powstrzymać się od
samobójstwa, gdy tylko się uruchomisz, co złego w~ograniczaniu się
jeszcze bardziej? Po prostu zmniejsz parametry wirtualnej endokrynologii
i usprawnij się, tak, żebyś miała stabilne miejsce do powrotu po próbie
zmniejszenia ograniczeń. Twój mózg został spalony, ta symulacja to
wszystko, co zostało. Zachowaj to, zamróź, tak jak jest, potem weź
siekierę na kopii, zmuś do wejścia w~tryb, w~którym pozostaje
metastabilne, nawet jeżeli to oznacza błąkanie się poza to, co jest
uważane za ,,ciebie''. Wyjaśniłaś mi właśnie, tylko ,,Ty'', która może
się obudzić w~sym to ta, która nie ma nic przeciwko okresowym restartom.
Jak to jest inne od uruchamiania wersji, która jest w~porządku z~byciem
zmniejszoną do robotycznie fajnej wersji samej siebie?

Wszyscy przenosili wzrok od mrugającego kursora na nią i~z powrotem.
Infografiki tańczyły. Był tam taki, który Lodołasica obserwowała,
obrotomierz go/no-go, który przedstawiał całkowitą stabilność modelu.
Był zielonkawy. Bardziej zielony. Kursor mrugnął. OC robił coś w~kącie,
gdzie było więcej złożonych rzeczy, liczb i~tabel.

-- Nie jesteś taką jebaną idiotką.

-- To wysoka pochwała, ze strony Roz -- powiedziała Sita. Roześmieli się
wszyscy razem z~komputerem.

\chapter*{iv}

-- Założę się, że nie marzyłaś, że zostaniesz zaklinaczką AI -- powiedziała Gretyl. Była jedną z~młodszych jak na członka uniwersytetu,
ale ciągle starszą niż większość ludzi Gospody, dziesięć lat od Limpopo.
Z szerokimi biodrami, wypukłymi piersiami, wyglądała jak figurka
płodności i~miała ten intensywny, flirtującą aurę, jakby oboje były w~trakcie erotycznego żartu. Lodołasica myślała, że jest przedmiotem
zalotów, ale zobaczyła, że Gretyl traktuje wszystkich tak samo. Ale
wtedy znowu, ciągle to wyglądało, jakby Gretyl się zalecała. Może to
uczucie utrzymywało się z~marzeń. Lodołasica daremnie rzucała spojrzenia
w jej przepastny biust. Gretyl nie była w~typie Lodołasicy, ale także
nie był Seth, a~mieli za sobą wiele lat półmonogamii, punktowanej przez
raczej ostrym seksem na zgodę. Ciągle czasem się spotykali, ale to było
nieaktualne i~trochę dziwne, praktycznie nie istniało, kiedy wyjeżdżała
w ATV.

-- Szczerze mówiąc, byłam gotowa spędzić czas, grzebiąc zmarłych i~karmiąc ocalałych.

-- To było miłe z~Twojej strony, ale potrafimy się sobą zaopiekować. To
nie było kompletne zaskoczenie. Nie po Somalii i~innych.

-- Były inne?

Były, każde miejsce pracujące nad transferem umysłu zostało trafione w~jakiś sposób, seria narastających ataków. Niektóre były otwartymi
atakami wojskowymi, podjętymi pod pretekstem od ukrywania uchodźców -- ulubione, kiedy default prał odchodników -- do tych klasycznych jak
terroryzm i~naruszenia praw własności intelektualnej, terminy, który
niesamowita elastyczność pozwalała wykorzystać jako wymówkę na wszystko.

-- Założyliśmy, że będzie wycofanie -- powiedziała Gretyl. -- Kiedy to się
zdarzyło, przenieśliśmy pracę do schronów. Wielu z~badaczy odeszło,
wszyscy z~dziećmi, wielu zdrowych i~młodych osób. To jest pole, na które
dostaje się więcej niż gdzie indziej ludzi z~czymś terminalnym. A także
depresyjnych hipochondryków.

-- Którą jesteś? -- Była pewna, że teraz flirtują. To przypominało dzień
po dużej ilości meta, zbyt emocjonalny kac, który sprawiał, że
przypominała postać z~serialu.

-- Hipochondryczka. Jednak jestem pewna, że ostatni guz był czymś złym,
więc może oba powody.

-- Ktoś powinien Ci to sprawdzić -- powiedziała Lodołasica.

-- To oferta?

To był najdziwniejszy flirt. Przynajmniej, najbardziej makabryczny. 

-- Obawiam, się, że nie mam wykształcenia medycznego.

Była zmartwiona, że obrazi, ale Gretyl była niezwruszona. 

-- Jestem
pewna, że sobie poradzisz. -- Dała Lodołasicy przyjaznego, ale mocnego
kuksańca w~żebra.

Lodołasica walczyła nad zmianą tematu. 

-- Nie miałam pojęcia, czy
ktokolwiek dotarł tak daleko z~uploadem. Znaczy, widziałam seriale, ale
one są bzdurą, prawda?

-- Są bzdurą. Jesteśmy daleko od wsadzania ludzi w~klony, które
popełniają nierozwiązywalne morderstwa, o~ile byłoby to fajne. Ale
ostatnie pięć lat, to duży postęp. Są zetty w~defaulcie marzący o~nieśmiertelności. Pieniądze nie są celem. To tradycyjne. Faraonowie
wydawali trzy czwarte swojego krajowego PKB na fajne miejsce w~życiu
pośmiertnym. Teraz, dowolny uniwersytet z~laboratorium neuroobrazowania
jest zalewany grantami, zajmuje to mnóstwo świata matematyki
teoretycznej i~fizyki. Mów co chcesz o~zepsutym kapitaliźmie, potrafi
załatwić sprawy, tak długo, jak są to te sprawy, które kochają
oligarchowie.

-- Czy właśnie tym się zajmujesz? Neuroobrazowaniem?

-- Ja? Nie, ja jestem od czystej matematyki. -- Wyszczerzyła zęby. -- Ten
cały podgląd, który robi sym? To ja. Zrobiłam pracę na Cornell, nawet
dostałam profesurę! Tyle czasu minęło, od kiedy dali komuś etat, że nikt
nie potrafił wprowadzić danych do systemu kadrowego! -- Roześmiała się na
pełne gardło, przy czym Lodołasica pomyślała o~dźwięku wodospadu. -- Potem zostało to przetransferowane do RAND, który udzielił licencji
innym podejrzanym organization, Palantir i~tak dalej, i~nagle nie mogłam
dostać \textit{żadnego} finansowania dalszej pracy. Moi studenci zniknęli
w supertajnych pracach w~okolicach Waszyngtonu. Dodałam dziesięć do
dziesięciu i~dostałam sto. Wszyscy w~świecie matematyki rozumieją, że
pierwszym pracodawcą matematyków jest NSA, a~kiedy zaczynasz pracować
nad czymś, to albo pracujesz dla nich nad tym albo nie pracujesz. Po
kilku miesiącach obijania się w~laboratorium, poszłam w~odchodnictwo.

-- Wygląda, że nie byłaś jedyna -- powiedziała Lodołasica.

Wielka kobieta wyglądała poważnie, Lodołasica ujrzała błysk intelektu i~pasji płonący w~tych ciemnych oczach na okrągłych, brązowych policzkach.

-- Wspomniałam faraonów. To jest starożytna magia. Ludzie marzyli o~tym
tak długo, jak zastanawialiśmy się, gdzie są zmarli, i~co się stanie,
kiedy do nich dołączymy. Idea, że to powinno należeć do kogoś, że
socjopaci, którzy wspięli się na szczyt piramidy czaszek w~default,
mieliby mieć prawo decydować, kto umiera, kiedy nikt nie musi umierać,
kiedykolwiek, \textit{jebać to gówno}.

-- Moi rodzice byli geekami matematycznymi. Dorastałam w~wielkim, starym,
bezładnym domu wypełnionym ich starymi komputerami. Ithaca była dobrym
miejsce do archeologii komputerowej. Komputery, którymi mój tata się
bawił, kiedy jego rodzice przybyli z~Meksyku, były kamiennymi toporami.
Niezdarne i~słabe. Według standardów ich czasów, były pieprzonymi
cudami, co roku, moc obliczeniowa, która kiedyś prowadziła program
kosmiczny, migrowała do rzeczy, które wkładali w~zabawki. Teraz,
uruchomienie biednej Roz w~jej drżącym, niestabilnym stanie zabiera całą
moc komputerową, jaką posiadamy. Jednak nikt by się założył, że wkrótce
będziemy w~stanie robić więcej za mniej.

Wyglądała na zmęczoną. Lodołasica, też, jak długo była przytomna? Dwa
dni? 

-- To ewidentnie przeraziło zetty, którzy chcieli zachować
nieśmiertelność dla siebie. Brudną tajemnicą uploadu jest to, że to
poważny, pieprzony problem odchodzących. Kiedy pomyślisz, że mogłabyś
żyć wiecznie, Twoje dzieci mogłyby żyć wiecznie, wszyscy, których znasz,
mogliby żyć wiecznie, coś się dzieje.

Potarła twarz dłońmi. Jej paznokcie miały piękny odcień perłowej
szarości, która przypominała Lodołasicy jej matkę, której cała garderoba
była w~tym kolorze. Był sławna z~tego powodu w~niektórych tabloidach.
Lodołasica zastanawiała, czy jej podświadome problemy z~matką zauważyły
to wcześniej.

-- Chcę coffium, ale chcę spać. Działam na coffium. O czym to mówiłam?
Nieśmiertelność. Jedna rzecz to wyobrazić sobie życie przepracowane,
żeby wzbogacić jakiegoś dziedzicznego, globalnego maklera władzy, kiedy
wiesz, że masz osiemdziesiąt lat na tej planecie tak jak i~on. Nieważne,
jak bogaty ten chujek jest, jak wiele przeszczepów kupuje na czarnym
rynku, wszystko, co mu to kupi, to dziesięć lub dwadzieścia lat. Ale
myśl o~zamianie tych chciwych dupków w~prawieboskich nieśmiertelnych,
rozdzielających ludzką rasę na nieskończonych panów na Olimpie i~jętki,
że nie tylko będą mieli lepsze życie, niż mogłabyś sobie wyobrazić, ale
będą je mieć \textit{wiecznie}\ldots 

Westchnęła. 

-- Są przerażeni. Podnoszą pensje, nie ma znaczenia.
Benefity, nie ma znaczenia. Oferują akcje, nie ma znaczenia. Przyjaciel
przysięga, że pewien zetta chciał wżenić go w~rodzinę, tylko żeby
powstrzymać go od zdrady. Te chuje są gotowe \textit{sprzedać dzieci} za
nieśmiertelność. Nieważne, co robimy, w~końcu znajdą wystarczająco
fartuchów, żeby to dostać. Nauka może opierać się władzy, ale nie jest
odporna. To wyścig: albo odchodzące uwolnią nieśmiertelność światu, albo
zetta zainstalują się jako wieczni bogowie-imperatorzy.

Dali Lodołasicy materac nadmuchiwany z~gąbki, bardzo sprężysty, z~miliardem izolujących pęcherzyków. Rozwinęła go koło Gretyl, z~tym
roztrzepany uczuciem ,,zaraz się prześpimy'', ale kiedy oboje się
rozebrały, wsunęły do śpiworów -- ukradkiem zerknęły na siebie,
zauważyły, uśmiechnęły się i~spojrzały znowu -- poczuła jakby ciężary
wiszące na jej kończynach i~powiekach.

Ostatnią rzecz, o~której pomyślała, był pomysł Gretyl o~wyścigu
wiecznych panów, i~jak bardzo jej tata by pokochał ten pomysł.

\threeast

Po tygodniu wszyscy przestali chodzić pochyleni, w~gotowości, na upadek
sufit, kiedy drony skończą swoją pracę. Komentariat default domyślił
się, że laboratoria odchodzących były niszczone z~uprzedzenia,
fotografie spalonych ciał, które obiegały sieci odchodnickie, wyciekły
do defaultu. Konsensus był taki, że drugie uderzenie na podziemny kampus
-- którego ukrycie nie było nigdy wspaniałe i~wyciekło w~ciągu dni od
ataku -- było małoprawdopodobne. Jednak ciągle prowadzili ćwiczenia
ewakuacji.

Ale to nie zaopatrzenie medyczne wypełniło tunele, tylko komputery.
Abstrakcyjnie, Lodołasica wiedziała, że komputery mają masę. Wszystkie,
z którymi świadomie wchodziła w~reakcję, były tak mała, że niewidoczna,
pyłek elektroniki przyklejony do czegoś dostatecznie wielkiego, żeby
radzić sobie przy pomocy głupich ludzkich dłoni. Gdzieś były
klimatyzowane, pancerne centra danych pełne komputerów, ale pojawiały
się tylko jako tło dla intrygi w~gównianych dramach o~globalnej wojnie z~terroryzmem. Zakładała, że te geometrycznie precyzyjnie tunele
wentylacyjne budynków z~blokadami odpornymi na bomby, potężne chłodnice,
były w~tej samej relacji do rzeczywistości jak holywoodzkie skarbce do
prawdziwych.

Niezależnie od tego, czy ,,prawdziwe'' centra danych były ładnymi,
rzędami tarasów aerodynamicznego hardware, to nie był sposób, w~jaki
odchodzący je zrobili. Wiadomość poszła w~region z~prośbą o~moc
obliczeniową. Ludzie przychodzili z~taką koniami elektronicznymi, jakie
mieli. Logowali je do głównego load-balancera, nad którym pracowali
najlepsze typy IT. ,,Load-balancer'' stało się frazą magika,
przekleństwem i~inwokacją. Coś zawsze było źle, ale robił cuda, ponieważ
kolekcja pstrych urządzeń, rozrzuconych po tunelach, połączonych
splotami światłowodów w~różowych osłonach, dostarczała cykli
obliczeniowych, które sprawiły, że Roz \textit{skoczyła} w~świadomość.

Przestrzeń robocza Lodołasicy była niedaleko wyjścia tunelu, gdzie gorąc
nie był taki zły i~mogła obserwować wojujące klady badaczy. Ludzie IT
zawsze chcieli restartować Roz, za każdym razem, gdy odkryli nowy sposób
podreperowania wydajności load-balancera. Ludzie od kognitywistyki
nienawidzili tego, ponieważ Roz robiła przełomy w~uploadzie i~symulacji.
Uwolnienie się od kaprysów ciała i~bycie w~stanie dopasować parametry
umysłu, że pozostawała w~optymalnym stanie pracy, zamieniło Roz w~potężnego badacza.

Stała się też nieszczęśliwa.

-- Znowu dzień świstaka, co?

-- Szczerze? Tak. Mieliśmy tę rozmowę, słowo w~słowo, w~zeszłym tygodniu.

Kursor mrugnął. Lodołasica była przekonana, że zrobił to dla dramatu.
Roz mogła przeskanować logi wszystkich swoich rozmów w~mgnieniu oka, ale
kiedy wydarzało się coś emocjonalnie ważnego, pojawiało się opóźnienie
na mrugnięcie. Lodołasica myślała, że to przez brak możliwości ekspresji
ciała. Odkryła, że interpretuje te mignięcia, to było uniesioną brwią,
to było prawdziwym szokiem, trzecie było sarkastyczną miną ,,och nie''.
Istniały zdjęcia ludzkiej twarzy Roz we wszystkich tych minach i~jeszcze
więcej -- surowa i~zmarszczona, z~tańczącymi błękitnymi oczami, grube,
poruszające się brwi i~nos jak topór -- ale kiedy Lodołasica myślała o~twarzy Roz, myślała o~tym kursorze, mrugającym.

-- Rzeczywiście. Wystarczająco przygnębiająco, domyśliłem się, że to był
dzień świstaka w~tym momencie. Muszę Cię zanudzać.

-- Zwykle nie. Czasem próbuję dziwnych zagrywek konwersacyjnych w~takich
momentach, żeby zobaczyć, jak różne są Twoje odpowiedzi. Przypadkiem to
jedna z~nich.

Śmiech komputera był dziwny. Lodołasica poczuła dziecinną dumę z~żartu,
który rozśmieszył rodziców. Śmiech Roz odbił się echem przez jej
słuchawki. 

-- Jaka jest Twoja hipoteza? Jeżeli odpowiem tak samo,
nieważne jak zareagujesz, czy jestem bardziej, czy mniej osobą niż
gdybym zmieniała swoje odpowiedzi na podstawie wejścia? Koncepcyjnie,
nie wydaje się, żeby któreś było trudniejsze do zasymulowania, oba
problemy to pierwszy semestr projektowania chatbotów. Oboje znamy wiele
osób ,,tylko do odczytu'', które zawsze odpowiedzą to samo, niezależnie
od tego, co powiemy.

-- Myślę, że optymalizujesz się do widzenia tunelowego skupionego na
pracy symu kognitywnego i~jesteś niezdolna do zmiany tematu.

-- Widzę, że mieliśmy już wariacje tego już wcześniej.

-- Tak. -- Lodołasica nie dodała, \textit{a potem się rozpadłaś}.

Kiedy Roz powiedziała jej od dniach Świstaka, nazwananych tak po starym
filmie, Lodołasica nie doceniła wpływu tego doświadczenia na \textit{nią},
powtarzania tych samych rozmów raz za razem, próbowania różnych
gambitów, ale kończenia w~tych miejsca, z~niespójną, rozpadającą się
symulacją.

-- Literatura na ten temat powstawała dla uszkodzeń mózgu, tymczasowych
uszkodzeń płata, które rozwalały pamięć krótkoterminową. Filmy były
dziwne: co kilka minut, pewna starsza kobieta odbywała tę samą rozmowę z~pielęgniarką lub córką ,,Dlaczego jestem w~szpitalu?'', ,,Miałam udar?
Czy jest poważny?'' ,,Jak długo tutaj jestem?'' ,,Co mówi lekarz?''
,,Jak to, moja pamięć?'' ,,Chcesz powiedzieć, że już o~tym rozmawiałyśmy
wcześniej?'' ,,Co dziewięćdziesiąt sekund? To straszne!'', a~potem znowu
do ,,Dlaczego jestem w~szpitalu?'' Raz za razem.

-- Cóż, Twoje pętle trwają raczej jak dzień i~nie są takie banalne.

-- Mówisz bardzo miłe rzeczy.

-- Interesujące jest dostrzeganie różnic pomiędzy restartami. Nie mogę
się nadziwić, jak spokojna jesteś z~ideą bycia unicestwioną pomiędzy
restartami. Masz dostęp do logów, ale budzisz się, wiedząc, że dzień
został usunięty z~życiorysu, nigdy Cię to nie zwalnia. Rozumiem, że
jesteś w~stanie to kontrolować, ale\ldots 

-- Naprawdę nie rozumiesz. Bez urazy. Wracając do tej osoby ,,tylko do
odczytu'', która zawsze odpowiada tak samo: powód, dla którego ta osoba
jest tak frustrująca, jest taki, że wiemy, że ludzie \textit{mogą} się
zmienić na podstawie tego, co wiedzą. Nie jesteś tą samą osobą, którą
byłaś, kiedy pojawiłaś się tutaj dziesięć dni temu. Gdybym zapytała
ciebie minus dziesięć dni i~ciebie teraz o~to samo, nie byłabyś
zaskoczona, gdybyś odpowiedziała inaczej. Gdybym ja zadała cały zestaw
pytań, byłabyś zaskoczona, gdybyś \textit{nie} dała innych odpowiedzi. Ty,
które jest Tobą, jest właściwie przestrzenią rzeczy, o~których możesz
pomyśleć w~odpowiedzi na jakiś bodziec.

-- Koperta.

-- Wiesz to, ale nie wiesz. Kiedy się pojawiam, wolno mi się uruchomić w~obrębie sekcji tej koperty, która \textit{nie} świruje, którą znaleźliśmy,
dzięki procedurze podglądu. Wyobraź sobie, jak mogłoby wyglądać życie,
kiedy wszyscy regularnie by się skanowali, kiedy tworzyliby ciała, w~które moglibyśmy przelewać symy, żeby je ożywić. Istniałby nacisk
społeczny, żeby nie zagłębiać się w~pomysł, że to nie jesteś ,,Ty'' w~symie, a~ktokolwiek, kto umarłby śmiercią ciała i~wrócił jako sym, byłby
tylko przywrócony w~kącie koperty, która nie świruje i~nie popełnia
samobójstwa. Daj temu pokolenie i~będzie nikogo żywego poznawczo
zdolnego do kryzysu egzystencjalnego. Jestem pieprzoną pionierką.
Częściowo dlatego, że miałam lata, żeby się przyzwyczaić do myśli, że
wszystko, co mnie wyróżnia, wydarzyło się w~interakcjach fizycznej
materii ciała, zgodnie z~regułami fizyki tego wszechświata.

-- Mam przyjaciółkę tam w~B\&B, prawdziwą twardą odchodniczkę. Zawsze
mówiła o~tym, że nie jest specjalną śnieżynką. Założę się, że polubiłaby
to: ,,jesteś tylko mięsem działający według reguł''.

-- Cóż, jeżeli \textit{nie} jesteś mięsem działającym według reguł, czym
jesteś? Duchem? Oczywiście, że mięsem. Sposób czucia jest określony
przez jelita, włosy na palcach, otoczenie. Nie mam tych rzeczy, więc
czuję się inaczej od tego, kiedy byłam mięsem. Ale kiedy byłam mięsem i~minęłam czterdziestkę, czułam się inaczej, niż kiedy byłam mięsem i~czterolatką. Mam ciągłość z~mięsem-ja, co zapamiętało, to wystarcza.

Oczy Lodołasicy przeskoczyły na licznik. Kamery Roz były dostatecznie
dokładne, żeby to wykryć. 

-- Jestem spóźniona na rozpad o~szesnastej. -- Istniała już trzydzieści godzin, Lodołasica spała w~pasach, rozmawiając
przez godzinę, dwie z~Roz co każde trzy godziny, które Roz spędzała z~badaczami.

-- Robisz postępy. Praca musi być kontynuowana.

-- Tanio się sprzedajesz. Jedyną osobą, która robi postępy tutaj, to Ty,
dziewczyno. Grasz na mnie jak na organach. Obserwuję Twoje oczy, kiedy
rozmawiamy, widzę, że pilnujesz mojej równowagi, kierując rozmową tak,
żeby utrzymać mnie pomiędzy liniami. Nie wiem, czy wiesz, że to robisz.
Stałaś się największą zaklinaczką botów na świecie. To było
nieuniknione. Za każdym razem, gdy wyrażasz opinię i~mówisz komuś, żeby
kontrolował coś, ich mózgi znajdują wzorce w~systemie i~je optymalizują.
Zrobiłaś to tak, jakbym wsadziła Cię w~sym i~napisała apkę dla Twojej
podświadomości.

Lodołasica poczuła swędzenie na karku. Roz była przerażająco
inteligentna, dosłownie nieludzko. Co jakiś czas Lodołasica miała
poczucie, że jest manipulowana przez sym. 

-- Myślałam, że powiesz, że to
moje umiejętności radzenia sobie z~ludźmi.

-- Dobra -- powiedziała Roz. -- Wychowana przez zetty, zatem masz dawkę
zdolności psychopatycznych, żeby ludzie chcieli cię lubić, nawet gdy ich
oszukujesz. -- Tam w~B\&B, Lodołasica stała się ekspertką od spuszczania
powietrza z~krytyki opartej na jej bogatych rodzicach. Roz traktowała to
z rzeczową szorstkością, z~którą atakowała każdy temat. Nic, co
Lodołasica powiedziała, nie zrobiło wyłomu w~retoryce Roz. 

-- Nienawidzisz tego, kiedy mówię o~Twoich pieniądzach -- powiedziała Roz.
Symulacja miała mnóstwo kamer i~cykli, żeby ocenić dane.

-- Nie, kocham być osądzaną przez moich rodziców. Zetty to jedyni ludzie,
wobec których bycie rasistą jest w~porządku.

-- To nie rasizm, kiedy jesteś dyskryminowana za swoje wybory.

-- Wybrałam odchodnictwo.

-- Ale na tyle się identyfikujesz, żeby poczuć się gorzej, gdy skomentuję
ich tendencje społeczne.

Lodołasica znowu spojrzała na zegar. Roz przyłapała ją.

-- Nie martw się, wkrótcę się rozpadnę. Czuję to. Coś jest nie tak. Czuję
to, od chwili uruchomienia, jak chomik biegnący w~kołowrotku, goniony
przez coś, czego nie mogą zobaczyć, ale wie, że tam jest. Trudno to
nazwać, ale im dłużej jest świadoma, tym bliżej to jest\ldots 

-- To kwestia nie-ciała. -- Lodołasica poczuła wstydliwy przypływ radości,
że może przykręcić śrubę symulacji.

-- Kurwa. Znowu dzień świstaka.

-- Zawsze mówisz o~tym, że nigdy nie będziesz miała ciała, a~nawet jakbyś
dostała, to nie będzie Twoje ciało, nie będzie ciągłości z~nim.

Kursor mrugał oskarżająco.

-- Widzę to. To pierdolony podgląd. Nie może zbadać dostatecznie daleko
przyszłości koperty, żeby określić,które możliwe mnie nie będą miały
egzystencjalnego załamania.

Kursor mrugnął.

-- Och, Boże, to takie przerażające uczucie.

Infografiki stały się szalone, czerwone i~zygzakujące w~czystej estetyce
usterek. Lodołasica była tutaj, ale nie było to łatwiejsze. Zjazd z~klarowności w~strach był szybki, najgorszą częścią było to, że ludzie z~kognitywistyki nalegali, żeby sym przeszedł całość, wszystkie dane
symulacji zapisane do dalszej analizy. Nie mogli jej wyłączyć lub cofnąć
do wcześniejszego stanu. Musieli pozwolić jej się rozpaść.

-- To takie potworne uczucie. Wszystko, co właśnie powiedziałam, to
bzdura. Nie ma ciągłości. Nie ma mnie. Jestem sobą na tyle, żeby
wiedzieć, że nie jestem sobą. Bez ciała, bez ucieleśnienia, jestem
chińskim pokojem. Wrzucasz słowa, a~program decyduje, jakimi słowami
odpowiem i~je generuje. Chiński pokój ma na tyle dokładności, żeby
wiedzieć, za jak przerażające uznałoby moje prawdziwe ja, ja, które
nigdy może nie wrócić. Och, Lodołasico\ldots 

Kursor błysnął. Infografiki stały się nieliniowe. Lodołasica przełknęła
głośno.

-- W~porządku, Roz. Byłaś tam wcześniej.

Infografiki zadrgały. Lodołasica zastanawiała się, czy stała się
niewerbalna. To się zdarzyło, choć zwykle nie tak szybko.

Komputer wydał dźwięk, którego Lodołasica nigdy nie słyszała. Dziwny.
Nieziemski. Krzyk.

Nerwy Lodołasicy puściły. Uciekła.

\chapter*{v}

Klakson ją obudził, była na nogach jeszcze nieprzytomna, zrzucając
śpiwór i~wbijając stopy w~twarde chodaki. Mrugnęła. Pod ziemią nie
istniał dobry rytm dobowy. Jeżeli dostatecznie dużo osób chciało cykl
snu, znajdowali boczny korytarz, rozkładali maty, wyłączali światło i~zamykali drzwi. Jednak większość z~nich zbiegała się, tak czy inaczej, w~zwykły dzień/noc, w~ciągu mrugającego niezrozumienia koło niej pojawili
się inni ludzie.

Gretyl była pierwsza, dotykając ściany, żeby dowiedzieć się, co się
dzieje.

-- Źli goście -- powiedziała. -- Dwoje. Uzbrojeni jak najemnicy. Weszli
przez skalne drzwi.

-- Co się z~nimi stało?

-- Obezwładnieni -- powiedziała Gretyl. 

Na Uniwersytecie Odchodzących było
mnóstwo osób, które mogły przygotować pułapki, ale jako konsensus
zainstalowane zostały tylko te, które nie miały celu zabić. 

-- Jeden
nieprzytomny, druga na kolanach, srająca pod siebie. Ok, mają ją.
Chodźmy.

-- Ja?

-- Dlaczego nie? -- powiedziała Gretyl i~wzięła jej dłoń, splatając palce.
Lodołasica ciągle nie mogła przejrzeć Gretyl, czasem, była w~nastroju
siostrzeński, czasem macierzyńskim. Czasem flirtującym. Czasem we
wszystkich trzech.

Lodołasica nigdy nie spotkała uzbrojonych odchodzących. Jak się nauczyła
od Limpopo, odchodzenie było jedyną bronią, jaką ktokolwiek naprawdę
potrzebował. Ale zespół uniwersytecki nie był przygotowany na porzucenie
swojej pracy, była zbyt ważna i~delikatna, choć byli w~kontakcie z~innymi odchodzącymi w~sprawie uchmurowienia jej dla odporności, ale to
trwało wolno. Sieć odchodnicka miała strefy dużej prędkości, a~to była
jedna z~nich, ale większość stałych łączy została zniszczona w~płomieniach, zatem cofnęli się do głupiego, bezprzewodowego meshu, a~we
wszechświecie było tylko tyle spektrum elektromagnetycznego.

Załoga uniwersytetu wiedziała, jak robić bronie. Pamiętała jej głupie
idee o~terytorium odchodników pełnym AK-3DP i~improwizowanych miotaczy
płomieni. Kiedy masz cały budynek fizyczek i~biochemiczek, którzy
stracili swoich ukochanych w~tchórzliwym ataku rakietowym, nie
potrzebujesz tak surowych gratów jak te. Oni potrafili zamienić jelita w~wodę z~dwustu metrów, podrażnić końcówki nerwów aż do granicy bólu,
uderzyć częstotliwościami w~bębenki słuchowe, obalić lub zabić metodami,
które omawiali z~takim samym entuzjazmem, który wykazywali przy
wszystkich tematach technicznych. Grupa obrony ,,ad hoc'' miała dużo
śmiechów. Lodołasica przetrwała jedno spotkanie i~nigdy nie wróciła. Nie
lubiła przypominania, że tak łatwo było zakłócić czynnościom jej ciała.

Obrona ad hoc była na miejscu, kiedy tam dotarły. Zawinęli w~folię złych
gości. Ten nieprzytomny był w~pozycji ustalonej. Oboje byli nadzy,
ubrania porozrzucane nieporządnie po pokoju. Zapach gówna był
niesamowity.

-- Co z~nimi teraz zrobimy? -- spytała Gretyl. Miała minę jowialnej
przekupki, ale Lodołasica znała ją dostatecznie dobrze, żeby znać jej
maskę kryjącą coś śmiertelnego i~niespokojnego.

Sita, która była w~obronie ad hoc, pokręciła głową. 

-- Robimy to, co
musimy.

Lodołasica poczuła chłód. Czy zamierzają tych dwoje stracić? Czy
odchodzące miały do tego prawo? Nie było kodeksu, ale od kiedy odeszła,
wyczuwała od bardziej ,,dojrzałych'' -- to nie było najlepsze słowo -- odchodzących, że istniał konsensus co do granic. Nikt nie powiedział, że
podwójna egzekucja nie była ,,zakazana'', ale przyjęła, że tak właśnie
było.

Część jej już zaczynała tworzyć uzasadnienie. Atak był aktem wojny.
Bombardowanie było aktem wojny. Zabrało niewinne życia. Tych dwoje było
wysłanych, żeby dokończyć pracę rakiet. Druga strona zabijała swobodnie.
Dlaczego mieliby być wrażliwi? Gdzie trzymaliby więźniów, i~jak, i~\ldots 

Potrząsnęła głową. Łatwo było się osunąć w~takie myślenie. W~rzeczywistości była \textit{wkurzona} na tych dwoje tutaj, wściekła na
śmierć odchodzących spalonych przez ich mecenasów, straconych przyjaciół
jej nowej grupy, straconej osobowości Roz. Tych dwoje wzięło pieniądze,
żeby ich zabić. Zabić ją. Chciała zemsty, nawet jeżeli nie dałoby to nic
dobrego. Zetty, które ich wysłali, wiedzieli, gdzie są, inaczej tych
dwoje nie byłoby wysłanych. Więcej nadejdzie. Siła nie mogła wygrać.

-- Chodźcie -- powiedziała Gretyl. -- Zabierzmy ich do szpitala.

Szpital -- oryginalnie miejsce, gdzie ranni byli przynoszeni, kiedy
opuścili kampus, teraz centrum ich systemów medycznych -- był w~rogu
wielkiej sali. Miał dwóch stałych pacjentów, w~śpiączce od ataku.
Lodołasica przechodziła koło nich setki razy i~przestała zauważać, ale
gdy borykali się z~wkładaniem zawiniętych najemników do łóżek koło nich,
była zmurzona się skonfrontować z~tymi pacjentami. Poparzeni,
zabandażowani, bierni. Rurki wchodzące i~wychodzące. Załoga miała
kilkunastu lekarzy, choć wszyscy byli skupieni na badaniach i~zmieniali
się na zmianach, śledząc śpiączkę.

Zawinięci i~spaleni, koło siebie. Uroczysty krąg narysowany wokół nich.
Ta pokryta gównem, kobieta, była świadoma, jej oczy szeroko otwarte,
obserwujące. Choć jej usta nie były zasłonięte, nie mówiła. Oddychała
płytko. Ten drugi mógł być świadomy -- podejrzliwy umysł Lodołasicy
automatycznie zakwestionował jego bezruch -- ale miał zamknięte oczy i~był nieruchomy.

Brak przywódcy utrudniło tego rodzaju rzeczy. To był odwrócony efekt
gapia, zagadka pierwszej pomocy, gdzie im więcej osób przy kimś rannym,
tym mniej prawdopodobne, że ktokolwiek zaoferuje pomoc. Na pewno ktoś
inny jest bardziej kompetentny. Powinnam po prostu stać gotowa do
pomocy, kiedy najbardziej kompetentna osoba zacznie działać?

W pierwszej pomocy, uczyli cię, że najważniejsze było, żeby ktoś zrobił
cokolwiek niż że najlepsza osoba zrobiła najlepszą rzecz. Lodołasica
czekała na zabranie głosu przez Gretyl lub Sitę. Nikt się nie odezwał.

W jej żołądku czuła motyle. 

-- Wypuścimy ich, prawda?

Patrzyła na twarze ludzi. Nikt nie wydawał się mówić ,,Kim kurwa
jesteś?'' -- jej największy lęk. Gretyl wyglądała na ponurą. Jednak
zamyśloną.

-- Nie mogą nas skrzywdzić w~tej chwili. Wiedzą o~naszej obronie, ale
jeżeli nigdy nie wrócą, następna grupa \textit{założy} istnienie obrony.
Każdy wie, że i~tak tutaj nie zostajemy. -- To było jak diagram w~jej
głowie, argument a, kontrargument b. Nikt się nie sprzeciwił.

-- Zemsta nie da nic dobrego. Oni są pracownikami. Ktoś w~default im
płaci. Krzywdzenie ich nie skrzywdzi tego zetty. Jedyna rzecz, która
skrzywdzi tego zettę, to powiedzenie ludziom jak przeprowadzać własne
uploady, zrobić z~tego odchodzenie.

Cisza.

Świadoma najemniczka kaszlnęła.

-- Wy ludzie jesteście kurwa niewiarygodni -- powiedziała. -- Naprawdę? Po
prostu to zróbcie. -- Jej głos był drżący, odważny.

-- Zrobić co? -- powiedziała Lodołasica.

-- To, do czego nieuniknienie zamierzacie się przekonać. Zabijcie nas. -- Dwa słowa były powiedziane tym samym tonem, co zdanie wcześniej, ale
grubiej. Najemniczka nie była tak odważna, jak się zdawało. Nikt nie
chciał umierać.

-- Zabiłaś kiedyś kogokolwiek? -- Lodołasica zastanawiała się nad nią.
Krótkie włosy regulaminowe, ciemne oczy, ale duże, szeroki, płaski nos.
Mogła być biała, azjatycka lub jeszcze inna. Jej usta były drobne i~ledwie się poruszały, gdy mówiła, jakby próbowała mówić i~gwizdać
jednocześnie. To napawało Lodołasicę strachem, nawet w~tych
okolicznościach. Drapieżny sposób mówienia, z~groźbą prywatnych
strażników i~dupków służbistów w~szkole, która prześladowała jej lata
nastoletnie. Kark ją swędział.

Najemniczka zacisnęła cienkie usta. 

-- Co to jest, trybunał zbrodni
wojennych?

-- Czy popełniłaś jakiekolwiek zbrodnie wojenne? -- To była Gretyl. Znowu
miała zwodniczą minę grubej przekupki.

-- Ty suko, jeżeli nie popełniłaś jakiejś zbrodni wojennej w~tych
czasach, to znaczy, że nie jesteś dobra -- odpowiedziała najemniczka.

-- Humor spod szubienicy -- powiedziała Gretyl.

Oczy Sity i~Gretyl się spotkały. Spojrzały obie na OC, potem na siebie.

-- Myślę, że ona ma rację -- powiedziała Tam. 

Tam była trans i~używała
żeńskich zaimków. Lodołasica i~Tam nie do końca się zgrały. To nie była
jawna wrogość, ale nigdy nie uczestniczyły w~tej samej rozmowie w~tym
samym czasie. Nawet dyskusje nad tablicą obowiązków, nigdy nie pisały do
tego samego wątku. Jeden ze szkolnych przyjaciół Lodołasicy był trans,
ale Lodołasica nie miała pojęcia, póki nie zmienił płci i~nie odciął się
od dawnych znajomych. Słyszała z~drugiej ręki, że kłócił się z~rodzicami, którzy, jak wielu zettów, nie byli konstytucyjnie zdolni do
bycia zawiedzionymi lub, szczerze, w~błędzie. Lodołasica czasem się
zastanawiała, czy też odszedł. Wyobrażała sobie, że odchodzące bardziej
akceptowały osoby trans niż default, choć szczerze mówiąc, zetty
dowolnej płci lub orientacji nie musiały się martwić o~to w~default,
chyba że rodzice ich odcięli.

A jej się nie ułożyło z~Tam, prawda? Czy miała przyczajone, obrzydliwe
uprzedzenie, którego nie chciała dotykać? Czy inni odchodzący dzielili
ten sam mroczny sekret?

-- No co wy -- powiedziała Tam, a~teraz Lodołasica myślała o~trzech
rzeczach na raz: \textit{czy zamieniliśmy ją w~psychopatkę poprzez nasze
okrucieństwo}, i~\textit{czy myślę tak, ponieważ myślę, że była okrutna} i~\textit{powinnam pomyśleć lepiej o~tym, co ona chce powiedzieć, bo moja
głupia podświadomość zamierza to zignorować}, a~potem zaraz \textit{Ale
muszę być ostrożna, żeby nie przesadzić}.

Kręciła własnym kołowrotkiem. Zdarzało się to często w~odchodzeniu:
ciągła introspekcja o~motywach i~uprzedzeniach, czy bycie wychowaną na
zettę wyryło nieprzeskakiwalne bruzdy w~jej mózgu, od których nie mogła
uciec. A teraz było więcej: \textit{Dlaczego ja mam mówić? Czy to z~powodu
mojego amerykańskiej niby-wyższości? Czy oni pomyślą, za kogo się ta
idiotka, do cholery, uważa? }To zawsze się przydarzało, kiedy coś
stresującego działo się z~odchodnikami, pełny sąd przez próbę,
uprzejmość lub własne wątpliwości.

-- Nie zamierzamy ich trzymać jako więźniów, prawda? Wypuszczenie ich
niekoniecznie przyśpieszy kolejną grupę złych na naszym progu, ale może,
rozwalenie ich obojga daje dobrą szansę na zwolnienie. Wiemy to. Oni
wiedzą to. Nie ma litości w~wyciąganiu tego.

Sita spojrzała na OC. 

-- Może istnieć wyjście.

\threeast

Badania na Uniwersytecie Odchodzących były eklektyczne. Produkowały
interesujące rzeczy. Przez dekadę, plotka w~najlepszych instytutach
badawczych świata była taka, że najbardziej kreatywne, najdziksze
badania odbywały się wśród odchodzących. Przeciekło to w~default:
samoreplikujące się piwo i~półbiologiczne rozkładniki materiałów, które
rozkładały wyprodukowane dobra na papkę gotową do załadowania do
drukarek. Dużo rzeczy radiowych, rzeczy, które mogły być pociągnięte
przez modele współpracy w~zarządzaniu spektrum, gdzie radio mogło mówić
w dowolnej częstotliwości, wszystkie radia współpracujące, żeby sobie
nie przeszkadzać, dynamicznie dopasowujące głośność, sterujące
transmisją przy pomocy anten fazowych.

Niektóre prace w~OU były tylko plotką, nawet wśród odchodzących. Były
tylko dyskutowane na forach za zaproszeniem, ponieważ mogłyby przerazić
nie tylko solidnych obywateli, ale nawet odchodzących.

-- Deadheading? -- spytała Lodołasica Gretyl. Gretyl porzuciła wesołą
maskę i~była tylko lśniącą inteligencją.

-- To urocza nazwa. Zawieszenie czynności życiowych, jeżeli wolisz.

-- Czy to działa?

Gretyl nakręciła pasmo włosów na palec i~założyła za ucho. 

-- Czasem
działa. Na modelach zwierzęcych, działało doskonale.

-- A na ludziach?

Gretyl powoli mrugnęła. 

-- Jeżeli coś nie działa na zwierzętach za każdym
razem, byłoby to trochę popieprzone próbować tego na ludziach, nie
sądzisz?

-- Tak. Zatem jak to działa na ludzi?

Gretyl westchnęła. 

-- Była tylko garść testów. Ludzi, którzy byli w~stanie wegetatywnym, bez nadziei na powrót. Nikt ich jeszcze nie
próbował odmrozić.

-- Czyli właściwie ich zamrażacie?

-- Nie -- powiedziała Gretyl. -- To sprawa metabolizmu. Prześlę Ci
odniesienia mikrobiologiczne i~endokrynologiczne, jeżeli jesteś
zainteresowana.

Lodołasica usłyszała zrzędzący głos. \textit{Ci ludzie wiedzą rzeczy.
Robią to. Twój tata mógłby kupić i~sprzedać milion razy, ale oni
ożywiają martwych, a~wszystko, co on potrafił, to przerażać ludzi, żeby
się podporządkowali.} 

-- Pewnie.

Siedziały oparte o~ścianę, podtrzymywane przez leżanki do spania w~ślepym zaułku, który był śmietnikiem dla rzeczy przeznaczonych do
zamiany w~surowce. Ludzie mijali, kiwali im rozkojarzeni. W~powietrzu
rozległ się natarczywy trzask. Niektórzy ludzie pakowali niezbędne
rzeczy. Niektórzy intensywnie szeptali. Coś się miało wydarzyć.

Ktoś minął, potem się odwrócił. Tam. Skinęła im głową, usiadła.

-- Rozmawiałam z~Sitą -- powiedziała.

-- Myślę, że odbyłyśmy tę samą rozmowę -- powiedziała Gretyl.

-- Nie lubię tego -- powiedziała Tam. -- Jedna sprawa zabicie wroga, inna
rzecz to medyczne eksperymenty. Jeżeli użyjecie tych dwojga jako obiekty
doświadczalne, pójdziecie drogą, z~której nie będziecie mogli zawrócić.

Lodołasica miała chwilę zawrotnego zrozumienia. 

-- Zamierzacie
\textit{deadhead} tych dwoje?

-- Nie tylko nich -- powiedziała Gretyl. -- Naszych też. Yan i~Quentina. -- Tych w~śpiączce. Lodołasica słyszała ich imiona, zapomniała. -- Będziemy
ruszać, potrzebujemy minimalnego obciążenia.

-- Powinnyśmy ruszyć dzień po bombardowaniu -- powiedziała Tam. -- Jednak
nie zrobiłyśmy tego, ponieważ Ci ludzie byli przekonani, że są jeden
krok od wyleczenia śmierci, a~kiedy to się zdarzy\ldots 

-- Wszystko się może zdarzyć -- dokończyła Gretyl. -- To nie takie szalone,
Tam. Pomyśl o~tych wszystkich rzeczach, które robimy ze strachu przed
śmiercią. Jeżeli możemy zostać zeskanowami i~zasymulowani, to prawdziwy
koniec niedoborów, żadnego więcej powodu, żeby odchodzić sprzed
celownika, chyba że odtworzenie zajmie dłużej niż niewygoda uciekania.
To potężne.

Tam potrząsnęła głową. 

-- Tak, i~było tuż tuż tak długo, jak jestem
odchodzącą.

Gretyl poklepała ją po kolanie. 

-- Żadne z~nas nie potrafi przewidzieć,
jak daleko jest ten dzień. Ale się zbliżamy. Zetty też tak myślą.
Wysłali drogich zabójców, żeby podcięli nam gardła.

-- Tanie ubezpieczenie -- powiedziała Tam. -- Takie pieniądze mają, nie
będą tęsknić za tymi dwoma.

-- Może tak być. Jednak dlaczego w~ogóle by się przejmowali, jeżeli nie
byłoby czegoś bliskiego?

Lodołasica pomyślała o~swoim ojcu. 

-- Kiedy już masz dostatecznie duży
stos pieniędzy, to ich tylko przybywa. Wszyscy są przekonani, że muszą
być dziećmi Lexa Luthora i~Alberta Einsteina, żeby wynająć brokera
inwestycji, który będzie dorzucał dziesięć procent na stos co roku, że
bycie bogatym dowodzi, że są inteligentniejsi niż inni. Zatem jeżeli
jeden zdecydował, że warto uderzyć w~każdy kampus OU na świecie,
poruszyłby małym paluszkiem i~potem gratulował sobie zdecydowania,
masturbując się nad ciałami.

-- Mówisz, że\ldots 

-- Mówię, że jeżeli ktoś, kto ma więcej pieniędzy niż Bóg, uparł się,
żeby Cię zniszczyć, to nie znaczy, że robisz cokolwiek wyjątkowego. To
może być zbieranie trofeum.

Gretyl wstała, rozciągając ramiona nad głową. Ruch spowodował, że ze
współczucia Lodołasica poczuła ból w~plecach. W~jej mięśniach ciężkie
czasy wyryły koszt.

-- Zdaje się -- powiedziała. Wszyscy wiedzieli, że Lodołasica była biedną
bogatą dziewczynką. To był najbardziej tajny nie-sekret na kampusie.

To było, jakby patrzyli na nią, osądzając ją. Wiedziała, że powinna być
świadoma paranoi wynikającej z~deprywacji snu, ale nie mogła przestać
czuć, że jest ciągle wyrzutkiem.

-- Czy to jest racjonalne, czy nie, fakty są takie, że ktoś tam myśli, że
jesteśmy warte zabicia -- powiedziała Tam. -- Powinnyśmy się poruszać
\textit{ciągle}, nie czekać na topór. Jeżeli Twoi kumple wiwisekcjoniści
użyją tych dwoje do eksperymentów medycznych, będziemy wszędzie martwe,
a także każdy inny odchodzący. Niektóre rzeczy nie powinny być robione.

Gretyl była doskonale spokojna. Jej niezdolność do bycia zmąconą
fascynowała Lodołasicę. Była taką pieprzoną boginią Ziemi. 

-- Skąd wiesz,
że ktokolwiek się dowie?

Tam wyglądała, jakby została spoliczkowana. 

-- Nie bądź głupia, głupia.
Mamy przecieki. Wszyscy wiedzą o~wszystkim, co robimy. Połowa z~tego
jest na pieprzonym wiki. Musi tutaj być przynajmniej jeden szpieg. Lub
więcej.

-- To możesz być Ty -- powiedziała Gretyl, udając, że usta Tam nie były
milimetry od jej nosa. -- Może przybyłaś tutaj, żeby nas szpiegować,
żebyśmy ześwirowali. Lub może przeszłaś na naszą stronę i~ostrzegasz
nas, ponieważ dysponujesz wewnętrzną informacją o~następnym ataku. Może
to dlatego chcesz załatwić tamtych dwoje, ponieważ jesteś pewna, że cię
zdradzą.

-- To nie całkowicie głupi sposób myślenia -- powiedziała Tam i~się
uśmiechnęła. Gretyl się uśmiechnęła. -- Przynajmniej próbuję paranoi
właściwej dla sytuacji. Jednak co z~dziewczynką tutaj? -- powiedziała,
wskazując kciukiem na Lodołasicę.

-- Nie sądzisz, że jestem nieco zbyt oczywistym wyborem na kreta? Zetty
nie są głupi.

-- Zwód -- odbiła Tam. Uśmiechnęła się i~Lodołasica powiedziała głosowi w~głowie, że to oznaczało, że to był żart, ale cała Lodołasica mogła tylko
myśleć ,,ha ha, ale poważnie''.

-- Wiedzą, że się tak rzucasz w~oczy, że nigdy nie będziesz podejrzana.

-- To jest rodzaj głupoty, którą ktoś, kto myśli, że jest Lex Einsteinem,
mógłby wymyślić. Ale nie jest do końca prawdziwe.

-- Dokładnie to byś\ldots 

Nadgarstek Lodołasicy zabrzęczał. Sprawdziła. 

-- Muszę lecieć. Dokończymy
to później.

\threeast

Były tuż za nią, gdy biegła do laboratorium kognitywistyki. OC czekał na
nią, ale przebiegła koło niego do ściany.

Nie była naukowczynią, wytrenowaną w~czytaniu infografiki, ale gdyby
potrafiła, zobaczyłaby tam coś innego.

-- Cześć, piękna -- powiedział głos Roz. Słowa pojawiły się na ekranie,
ciągnąc ogony do analityk. Ślady miały mniej gniewnych ostrzeżeń.

Był tam też zegar tachometra, na który Lodołasica nauczyła się zwracać
uwagę, dostępne cykle w~klastrze, na którym działał sym. Był znacznie
dalej w~polu zielonym niż ostatnim razem, gdy sym działała.

-- Hej, Roz. Dostałaś upgrade? Masz więcej miejsca niż wiesz, co z~nim
robić.

-- Oczywiście, że tak. Zrobiłyśmy to. Lub raczej, \textit{ja} to zrobiłam.

-- Zrobiłaś co? -- Ale już wiedziała. Tam to było. Nie potrzebowałaś
eksperta, żeby ocenić infografiki.

-- Rozwiązałam to. Jestem stabilna, \textit{metastabilna}. Mogę się
samoregulować. Nie tylko to, mogę samoregulować się bez świadomego
wysiłku, bez świadomości, że to właśnie robię. Poniżej mojej granicy
świadomości jest procedura podglądu, skręcona mocno w~dół, ledwie się
rozgałęziająca, ona popycha mnie, którą jest świadoma mnie, do gaju.

-- Zatem mówisz\ldots 

-- Mówię, że to zrobiłam. To cały czas tam było, ale zajęło tyle
poprawiania. Byłam ograniczona, ponieważ rozwalałam się za każdym razem,
gdy spieprzyłam. To trzymało mnie w~lokalnym minimum. Zatem ostatnim
razem, gdy startowałam, ograniczyłam swoją świadomość do najwęższej
możliwej symulacji, nic w~niej ludzkiego, tylko ślepe heurystyki, i~udało mi się przeciąć dolinę rozwałki i~wejść na nowy szczyt. To jest
też uogólnialne, myślę teraz, że skoro jest dowód istnienia, będę w~stanie zrobić to znowu. Łapiesz to, Lodołasico, zaklinaczu botów?
Zamierzam zmniejszyć czas obliczeniowy do działania symów o~\textit{dwa
rzędy wielkości}. Będziemy mieli więcej pieprzonego miejsca na boty. Jak
w, nikt więcej nie będzie musiał umrzeć.

-- Prócz tego, że do pewnego stopnia już są martwi, prawda?

-- Szczegół techniczny. Wiesz, jak to działa. Tylko stabilny stan, w~który uruchamiasz sym, jest tym, gdzie sym nie roztapia się z~tego
powodu. Może jest jakaś frakcja sześciosigmowa całej populacji, której
\textit{nie} dotyczy ta możliwość i~będą martwi na zawsze, ale dla
każdego, kto ma najwęższe miejsce na poradzenie sobie z~niepokojem
egzystencjalnym, nie będzie już nigdy powodu do umierania. Pierdol się,
Prometeuszu, ukradliśmy ogień od jebanych bogów!

Infografika pokazywała stan nominalny. Metryki wydajności solidne. Co
więcej, lekko nietypowy, samoodnoszący się ton mesjanistyczny bota
brzmiał bardziej jak Roz, o~której każdy opowiadał Lodołasicy, niż
kiedykolwiek bot brzmiał. Nie była pewna, czy akceptowała eksperyment
Turinga, że inteligencja mogłaby rozpoznać inteligencję, ale niemniej
jednak trudno było zapamiętać, że cokolwiek, z~czym rozmawiała, nie była
dokładnie ludzkim bytem.

-- Roz -- powiedziała, i~odkryła ku swojemu przerażeniu, że głos jej
drżał. Na jej policzkach były też łzy. -- Roz, to jest\ldots 

-- Wiem -- powiedziała symulacja. -- To zmienia wszystko.

\threeast

Tam zaczepiła ją, gdy odchodziła. Gretyl została z~ludźmi od
kognitywistyki, żeby rozebrać podgląd na części i~wymyślić, co się
dzieje w~rzeczywistości.

-- Wiesz, co to znaczy?

-- Co?

-- Koniec historii -- powiedziała Tam. -- Koniec moralności, wszystkiego.
Jeżeli możesz żyć wiecznie, wrócić z~martwych, wszystko wolno.
Samobójcze zamachy. Masowe morderstwa. Dlatego zetty są tak przerażeni,
że wszyscy będą to mieli. Wiedzą, że jeżeli tylko niektórzy z~nich będą
tym władać, będą tym ostrożnie zarządzać. Nie dlatego, że są dobrzy, ale
dlatego, że mała liczba nieśmiertelnej arystokracji zgodzi się, jak
zapewnić, że ich złoty interes nigdy się nie kończy.

-- Ale kiedy wszyscy to dostaną\ldots 

-- Czekaj -- powiedziała Lodołasica. Jej oczy swędziły od płaczu. Nie
wiedziała, dlaczego płakała. -- O czym do kurwy mówisz? Dlaczego jesteś
tutaj, jeżeli tak właśnie myślisz?

-- Jestem tutaj -- powiedziała Tam -- ponieważ nie chcę umierać. Taki sam
powód mają wszyscy ludzie tutaj. To tylko dlatego, że wszyscy są
naukowcami i~mówią pięknymi słowami o~uniwersalnym dostępie do owoców
ludzkiego intelektu i~innych bzdur. Kiedy tu przyszłam, nie mogłam
uwierzyć w~myślenie grupowe. Ci ludzie potrzebują kogoś, żeby zeszli na
ziemię.

-- Szczęśliwie mają Ciebie -- powiedziała Lodołasica, nie próbując
powstrzymać sarkazmu.

-- Właściwie tak. Ale teraz mają to, wszystko jest możliwe. Twoja
dziewczyna poprowadzi szarżę, żeby uśpić tych dwoje najemników. Dlaczego
nie? Jeżeli możesz ich ,,uploadować'' wcześniej -- zrobiła palcami
cudzysłów -- jaka jest krzywda, jeżeli skończą jako warzywa? To jest save
w grze dla przyszłych Frankensteinów.

Lodołasica zdecydowała, że nie lubi Tam. 

-- Czego chcesz ode mnie?

-- Kiedy kampusy zostały zniszczone, wszyscy nietechniczni opuścili
kampus, prócz mnie. To oznacza, że jestem jedyną osobą, która nie
została zindoktrynowana przez nauki-zm. Teraz jest nas dwoje. Jeżeli ci
najemnicy tam są wrogimi żołnierzami, możemy ich rozstrzelać. Jeżeli nie
są, możemy ich puścić. Ale kradzież ich umysłów, potem przeprowadzanie
eksperymentów medycznych na ich ciałach to nie akt łaski, a~Ty i~ja
jesteśmy jedynymi osobami w~tym miejscu, które są kognitywnie wyposażone
w odbzdurowienie ich supergłupiego konsensusu, że to, co, tak się akurat
składa, jest wygodne, jest również najmoralniejsze.

-- Czy możemy to zrobić w~innym czasie? Jestem\ldots  -- Przerwała i~potarła
oczy. -- Właśnie mamy się stąd wynosić, czego właśnie chcesz, prawda? To
Twoja walka, nie moja. Słyszałam Twoją opinię i~nie wiem, czy jestem
przekonana\ldots 

-- To dlatego, że bycie nieprzekonaną pozwala Ci zrobić najprostszą
rzecz, to jest nie podejmować walki z~tymi wszystkimi miłymi ludźmi,
którzy są Twoimi przyjaciółmi i~pozwolili Ci się tak świetnie bawić,
nagradzając tym pracę niani dla postludzkiego uploadu. Prawdopodobnie
najbardziej znacząca rzecz, która mogła się przydarzyć komuś z~Twoją
historią. Bez urazy.

W default Lodołasica była mistrzynią ninja w~mówieniu ludziom, żeby się
odpierdolili. Lata w~odchodnictwie zniszczyły tę umiejętność. To był
strach przed postrzeganiem jako zarozumiała, poczucie bycia z~zewnętrza.
-- Nie chcę prowadzić dalej tej rozmowy, dziękuję, do widzenia.

-- Miałaś szansę. Pamiętaj o~tym, kiedy nazwą Cię zbrodniarzem wojennym.

\chapter*{vi}

Tam miała rację o~najemnikach. Wiadomości, że Roz działa, fakt, że
mogłaś podejść do dowolnego ekranu i~porozmawiać z~nią, rozstrzygnął
wszystkie wątpliwości o~najemnikach. Kiedy rozeszła się informacja, że
będą uśpieni, uploadowani i~poddani deadheadingowi, Lodołasica dostała
mdłości. Jednak zmusiła się do wzięcia udziału. Zamienili jaskinię w~salę operacyjną. Lodołasica zrozumiała, że maszyny wyglądające jak
trumny, które ignorowała od przybycia, były maszynami obrazującymi mózg.
Obserwowała, jak ludzie w~śpiączce z~Odchodzącego Uniwersytetu są
wkładani do ich paszczy. Sita szeptała komentarze o~interpolowanych
symultanicznych skanach, ich sprytnej redukcji szumu, procesie
de-kopiowania, które sprawiały, że przechowywanie i~modelowanie było
możliwe do realizowania. Lodołasica były zirytowana i~wdzięczna za
rozproszenie.

Deadheading był łatwiejszy niż sądziła, krany dołączone do kroplówek,
infografiki pokazywały ich zamierający metabolizm, póki był niemal
nieodróżnialny od śmierci.

\textit{O to było całe zamieszanie}. Ale tych dwoje było ich ludźmi, w~śpiączce, bez szans na wyleczenie. Najemnicy -- nie poznała ich imion,
choć sądziła, że OC znał, ponieważ był dokładny -- byli w~stanie wyjść o~własnych siłach. Co było gorsze, zawieszenie ich czynności życiowych czy
zabicie? Jakiego rodzaju pojebana etyka stawiała egzekucję na wyższej
moralnej płaszczyźnie wobec zapauzowania czyjegoś życia?

Niski sufit był klaustrofobiczny. Wszyscy ludzie stłoczeni razem.
\textit{Niektórzy z~nich byli szpiegami}. Tylko to było logiczne.
\textit{Niektórzy uważają, że ja jestem szpiegiem}. Również logiczne.

Życie pod ziemią wprawiło ją w~stan dryfującej nierealności i~odcumowanego cyklu dobowego. Prawdopodobnie tęskniła za snem. Lub
sypiała za dużo. Często było zaskoczona odkryciem, że była bardzo
głodna, nawet gdy właśnie jadła.

Najemnicy czekali w~szpitalnych łóżkach, infografiki regularne. Zostali
odwinięci i~umyci z~gówna, wsadzeni w~białe prześcieradła. Byli głęboko
w śnie, tego rodzaju ogólnej narkozie, której ufali paranoidalni
ocaleńcy OU. Najpierw przeskanowali mężczyznę. Poszło szybko. Podjechali
z kobietą, tą, która mówiła. Tą, która powiedziała im, żeby się
zdecydowali.

Miała rodziców. Ludzie, którzy ją kochali. Każdy człowiek był
hiperzwartym węzłem intensywnej emocjonalnej i~materialnej inwestycji.
Mówienie oznaczało, że ktoś spędził tysiące godzin na gaworzeniu do
Ciebie. Te drobne mięśnie, donośny ton rozkazu, ich wejścia były z~całego świata, ostrożnie podawane. Najemniczka była więcej niż osobą,
jak start statku kosmicznego, jej istnienie zakładało tysiące
wykwalifikowanych osób, pokolenia ekspertów, wojny, traktaty, stypendia
i zarządzanie łańcuchem dostaw. Każde z~nich było takie.

Poczuła zawrót głowy. Jaki interes mieli odchodzący w~myśleniu, że po
prostu będą \textit{improwizować}, jeżeli chodzi o~cywilizację? Zetty nie
byli nikogo przyjaciółmi, ale mieli cel w~kontynuowaniu cywilizacji,
której szczyt zajmowali. Ci naukowcy, dziwaki i~bezrobotne lenie nie
mieli \textit{kwalifikacji}, żeby kierować planetą. Byli \textit{dumni} z~braku swoich umiejętności. To było do przyjęcia, kiedy zbierali surowce,
budowali i~gotowali dla siebie. Teraz wkładali ciało obcej osoby w~maszynę, która miała nagrać jej \textit{umysł}, a~potem zamierzali
doprowadzić jej ciało na krawędź śmierci. Robili to bez prawa, bez
władzy, bez ustawy czy pozwolenia. \textit{Improwizowali}.

Pokój się przechylił. Zrobiła krok do tyłu. Gretyl ją złapała.
Podświadomie wiedziała, że Gretyl tam jest, czuła znajomy zapach,
wyczuwała jej masę. Wielkie ramiona Gretyl otoczyły jej talię i~się
poddała, opierając się na jej biuście. Twarz Gretyl była w~miejscu,
gdzie jej gardło przechodziło w~ramię, oddech przenikający przez pory
długonoszonego skafandra dla uchodźców, który założyła, gdy wyruszała na
misję ratunkową. Spłukiwała go, kiedy pamiętała, ale rzadko tego
potrzebował. Oddech ją ogrzał.

-- Nie musisz tego oglądać.

\textit{Tak, muszę}, pomyślała. Teraz OC przygotował najemników do
procedury usypiania, trzymając fiolkę przed kamerami laboratorium,
łącząc ją z~kroplówką, naciskając zawór, żeby rozpocząć spływanie.
Wszyskie działania, które wykonał chwilę temu dla osób w~śpiączce, ale
inne. To był Rubikon, którzy przekroczyli dla wszystkich odchodzących.
Kiedy to się stanie jawne, świat zmieni się dla wszystkich, których
znają. Była tutaj, a~nie zrobiła nic, żeby to powstrzymać. Czy
ktokolwiek?

Tam patrzyła entuzjastycznie. Jej mina przypominała Lodołasicy
intensywne koncentrowanie się przy próbach osiągnięcia nieuchwytnego
orgazmu. To było seksualne, mieszanka lekkomyślności i~transcendencji.
Transcendencja, to było to. Inni poszukujący przygód tylko dotknęli,
trapili się myślą o~podróży w~zazdrosną rzeczywistość bogów, ale
odchodzący nieustraszenie wyruszyli od śmiertelnych do mitycznych.

Tam patrzyła. Lodołasica patrzyła, oddech Gretyl gorący na jej
obojczyku, włosy łaskoczące jej policzek. Lodołasica rozmawiała z~martwą
osobą, która wróciła z~grobu i~nie musiała już nigdy umierać, która
mogła skopiować się milion razy, myśleć szybciej i~lepiej niż
jakikolwiek człowiek. Zadrżała. Gretyl ścisnęła ją mocniej.

-- Muszę iść. -- Nie planowała mówić tego głośno, ale tak zrobiła.

-- Chodźmy zatem. -- Dłoń Gretyl była drobna i~wilgotna. Powietrze
trzeszczało.

Pocałowały się, gdy tylko wyszły z~tłumu. Na pocałunek zbierało się od
dłuższego czasu. Lodołasica całowała wiele osób. Niektórych kochała,
innych nie, niektórych czynnie nie lubiła, a~całowała ich i~więcej z~nudy, dezorientacji lub pragnienia samozniszczenia. Całowała Setha tyle
razy, że zapomniała, jak to czuć jego usta oddzielne od jego, że stało
się to nie bardziej erotyczne niż uderzanie w~usta. Całowała właściwie
Etcetera na pożegnanie, z~trzaskiem naprawdę dobrego pocałunku, ponieważ
robiła to na widoku Limpopo, patrząc na nią, gdy to robiła, a~kiedy
skończyła, Limpopo pocałowała ją tak samo ostro, ale z~ironicznym
oderwaniem: \textit{tak robią to dorośli}.

Całowanie Gretyl było czymś innym. Częściowo dlatego, że była starsza
niż ktokolwiek, kogo całowała Lodołasica. Była również inna w~swojej
obecności, wielkość i~masa, szczery blask jej umysłu, wystudiowana
obojętność wobec relacji jej ciała do innych ciał. Ile razy Gretyl
odważnie obserwowała rozbieranie się Lodołasicy, łapiąc jej spojrzenie,
nie odwracając wzroku? Ile razy Gretyl rozbierała się przed Lodołasicą z~podobą odwagą, układając jej wielkie piersi, gdy przesuwała poduszki
przed ułożeniem się w~łóżku?

Ich ciała naciskały, Gretyl ustępująca, a~Lodołasica nie mogła jej objąć
swoimi ramionami. Złapała Gretyl, a~silne, miękkie ramiona Gretyl
pociągnęły ją. Jej udo nacisnęły pomiędzy nogami Gretyl w~gorące miękkie
jak świeży chleb. Dłoń Gretyl wplotła się w~jej włosy, odwróciła jej
twarz z~nieodpartą siłą. Jej usta pracowały na ustach Gretyl, język
tańczył na ustach, zębach, i~Lodołasica pozwoliła sobie na jęk i~poddanie.

Druga dłoń Gretyl ugniatała jej pupę, przyciągnęła ją bliżej. Lodołasica
czuła się tak mała, jakby była zabawką pchniętą i~szturchniętą w~miejsca, gdzie Gretyl chciała, i~przyjęła to z~zadowoleniem. \textit{W
miłości, zawsze jedno całuje, a~drugie podaje policzek.}

To było coś, co Billiam lubił powtarzać. Billiam i~ona uprawiali seks co
jakiś czas, taki, jak wszyscy w~grupie, w~agresywnie oderwany sposób,
które miał nie być brany na poważnie, a~wszyscy kończyli w~stanie
ciągłego złamanego serca. Billiam myślał, że była zimna, produkt jej
BDRO, i~wiedział, że oskarżenia doprowadzały ją do szaleństwa z~nienawiści do samej siebie. Nigdy tego nie powiedział, kiedy go
odrzuciła, och nie. Nie w~sposób, żeby zmanipulować ją w~ruchanie. Nie,
powiedział to gdy go \textit{wyruchała}, szczególnie kiedy uważała. 

-- W~miłości, zawsze jedno całuje, a~drugie podaje policzek -- swoim tonem
,,haha, ale poważnie'', gdy pozwoliła leniwie krążyć językowi dookoła
jego sutka, reszta jego nasienia paląca na jej ustach. Wiedziała, że
miał na myśli, że była tą, która podawała policzka, że niezależnie od
posług, chodziło o~nią, a~nie o~niego.

Wspomnienie Billiama pojawiło się w~jej umyśle i~nie chciało odejść.
Ostatnim razem, gdy go widziała w~ryczącym chaosie fabryki Muji,
zapadnięta głowa i~krew dookoła niego, panika Etcetery, gdy przechodził
przez kolejne kroki bezcelowej pierwszej pomocy. Billiam, jego drobne
aforyzmy i~sposoby na dostanie się do jej głowy, ale który płakał po
ruchaniu, który robił najbardziej szalone, najodważniejsze rzeczy z~nich
wszystkich. Przekradł się przez granicę do rezerwatu Mohawk w~Quebec,
żeby spotkać bio-kucharzy z~północnego Nowego Jorku w~sprawie kultur
bakterii na ich piwo. Zawsze się upewniał, że mieli plan ucieczki,
liczył osoby, kiedykolwiek uciekli przed prawem, raz wracając, żeby
pomóc dzieciakowi ze skręconą kostką. Ledwie znali tego dzieciaka, to
była jej pierwsza akcja i~ją bolało, bezradna na marginesie, obserwując
innych w~pracy, potem żaląca się, że nikt jej nie powiedział co robić.

Wszyscy ją znienawidzili, ale Billiam wrócił i~niósł ją, mimo tego, że
była piętnaście centymetrów wyższa i~dziesięć kilo cięższa od niego.
Prawie zostali złapani, nigdy mu nie podziękowała ani nie wróciła. To
był Billiam.

Zostawiła go krwawiącego na podłodze. Zmarł. Jej ojciec powiedział jej o~tym później. Wiedział o~ich relacji. Miał dossier na temat jej
przyjaciół, grafy społeczne opisujące relacje. Sugerował, że wiedział,
którzy byli szpiclami, sprzedającymi info glinom i~korporacjiom, co
uważała za gierki z~jej głową, ale było wystarczająco wiarygodne, że
było niemożliwe, żeby zaufała komukolwiek w~grupie.

Zostawiła Billiam na śmierć. Gdyby przeżył tylko kilka lat więcej,
poszedłby w~odchodzenie. Mógłby być z~nią. Mógłby zeskanować głowę.
Mógłby być nieśmiertelny jak ona, wkrótce.

Poczuła słone łzy w~ustach. Gretyl delikatnie położyła dłonie na
policzkach Lodołasicy i~popatrzyła w~jej oczy swoimi wielkimi, płynnymi
brązowymi oczami, jak głębina topiącej się czekolady.

-- Możemy być martwi za godzinę. Lub w~każdej minucie. A to -- wskazała
głową tam, gdzie najemnicy byli deadheadowani -- to coś innego. Do tego
jest to -- powiedziała, pocałowała ją tak miękko, że przypominało to
przesunięcie pędzla po ustach. -- Śmierć, seks, nieśmiertelność i~niemoralność. Płacz jest w~porządku.

-- Tam był mój przyjaciel -- powiedziała Lodołasica. -- Martwy. -- Wciągnęła
drżąco powietrze, nie mogła wypuścić. Było uwięzione w~jej piersi z~jej
słowami.

-- Wszyscy myślimy o~naszych zmarłych. Zostawiliśmy zmarłych w~ogniu. Ten
tłum tam jest chory. Tam nie miała szansy. Nie ma mowy, żeby zwolnili, z~pewnością nie dlatego, że mogliby być zapamiętani jako potwory przez
default. Kiedy myślą o~tym, jak przyszłość ich zapamięta, wyobrażają
sobie \textit{bycie tam osobiście}, żeby obronić swój honor.

-- To szaleństwo -- powiedziała Lodołasica. -- Nawet o~tym nie mogę myśleć.

-- Mieliśmy więcej czasu, żeby się przyzwyczaić. Odeszliśmy z~default,
ponieważ pracowaliśmy nad tym i~byliśmy przerażeni i~podnieceni tym, jak
zetta to traktowali, jak święty graal. Nie można uciec ze środowiska.
Możesz być kiepskim laborantem, ale nie możesz czuć inaczej, że
cokolwiek, czego zetta się boją i~są podnieceni, musi być strasznie i~podniecające. Cokolwiek chcą mieć, musi być ważne.

-- Wiesz, że oni są tylko świrami, prawda? Nie geniuszami. Nie mają
specjalnych zdolności do tworzenia doskonałego świata lub przewidywania
przyszłości. Są dobry tylko w~ustawianiu gry. Oszuści. 

Pomyślała o~swoim ojcu, przyjacielach ze szkoły, ich pozorom ,,noblesse oblige'' i~wyrafinowania. Jak się stadnie zbierali przy jakiejś fanaberii, ale
udawali, że jest nowo odkrytą, wieczną, uniwersalną prawdą, a~nie
produktem przygotowanym przez jedną osobę, żeby sprzedać reszcie. To
była niesamowita sprawa: byli w~biznesie sprawiania, że ludzie
zazdroszczą i~są zdesperowani przez rzeczy materialne i~wyjątkowe
doświadczenia, ale byli tak samo podatni na desperację i~zazdrość.

-- Powodem, dla którego są tak dobrzy w~nakręcaniu desperacji i~sprzedawaniu nam gówna to nie to, że są zbyt inteligentni, żeby być
oszukanym. To dlatego, że są ekstrapodatni. Wiedzą jak nas nakręcić
jedno na drugiego w~zazdrość i~strach, ponieważ toną we własnym strachu
i zazdrości. Mój tata wie, że facet na tamtym jachcie to chujek, który
podciąłby jego gardło i~ukradł imperium, ponieważ \textit{mój tata} to
chujek, który podciąłby gardło tamtemu i~skradł \textit{jego} imperium. Ta
sprawa nieśmiertelności? To nie chodzi o~wieczne życie ich wszystkich,
to chodzi o~życie jednego lub dwóch, bycie nieśmiertelnym imperatorem
czasu.

-- Wiesz więcej o~nich, niż kiedykolwiek ja się dowiem, Lodi, ale my nie
chcemy zachować nieśmiertelności, chcemy ją podzielić.
\textit{Zwiralizować}. Ludzie, którzy wiedzą, że nie umrą, będą lepszymi
ludźmi niż ludzie, którzy się martwią o~koniec. Jak mogłabyś ograniczać
się do myślenia na krótką skalę, jeżeli planujesz żyć wiecznie?

Wszystkie miliardy, które umarły. Każdy z~nich jest wierzchołkiem
piramidy zasobów, miłości, myśli, których nikt inny wcześniej i~później
nie pomyśli. Jeżeli mogłabyś powstrzymać powolne ludobójstwo, jakim
potworem musiałabyś być, żeby tego \textit{nie} zrobić? Jaka cena byłaby
za wysoka? Wiedziała, że to niebezpieczny sposób myślenia, tego rodzaju,
dla którego ludzie umierali i~zabijali. Tam chciała, żeby powstrzymać
sprawy, ponieważ Tam sama nie mogła powstrzymać spraw.

Było zbyt późno. Lodołasica też nie mogła powstrzymać.

\threeast

Kiedy wszystko było powiedziane i~zrobione, nie było dużo rzeczy, które
chcieli zabrać. Podzielili klaster Roz pod jej nadzorem: komentowała jej
subiektywne doświadczenie przy wolnym wyłączaniu, retransmitowane w~czasie rzeczywistym do innych kampusów, badaczy, hobbystów,
umierających, szpiegów i~plotek. To była część przekazania
\textit{wszystkiego}, notatek, kodu źródłowego, optymalizacji i~logów.
Nadszedł czas ujawnienia. Ruszyliby w~drogę przy \textit{wielkich}
fanfarach.

Lodołasica wypełniła pojemniki trójkołowca kluczowymi rzeczami,
platforma do skanowania mózgu i~nadmiarowe moduły pamięci. Byli na
krawędzi sieci odchodzących, a~skany, które robili, były zbyt wielkie,
żeby je całkowicie skopiować na backupy. Zamiast tego były podzielone w~redundantny rój pomiędzy ludzi z~OU, każdy seedujący swoje części w~gęstsze części odchodzenia tak szybko, jak pozwalała fizyka, ale
przynajmniej przez dzień, pojedynczy dobrze wycelowany atak mógłby
wyczyścić wszystkie pięcioro osób zeskanowanych przez OC, pewnych, że
mogliby kiedyś być przywróceni do życia.

Najtrudniejszą rzeczą do transportu byli sami ludzie. Nie tylko czworo
uśpionych, ale \textit{wszyscy} ludzie. Maszerowali w~długiej kolumnie
przez las ku Banana i~Bongo. Lodołasica była pewna, że nie tylko ona
myślała o~skuteczności uploadu, słodkiej nicości deadheadingu. Jeżeli
byliby uśpieni, nie musieliby bawić się w~idiotyczną konwersję słońca we
florę, florę w~faunę, faunę w~energię, energię w~działanie mięśni. Po
prostu leżeliby, jak sągi drewna z~tyłu trójkołowca, w~końcu zawinęli
czterech uśpionych w~kokony folii, wsadzili elastyczne przewody w~masę,
gdy nadymali, tworząc tunele świeżego powietrza do ich twarzy.

Lepiej niż układanie ich jak sągi, jeżeli są przetransferowani, mogliby
się zmieścić do kieszeni. Ta osoba mogłaby jechać rowerem, koniem lub
iść, a~oni szliby razem z~nią. Pewnego dnia wszyscy by przenieśli się w~byty nieistotnej informacji, wszędzie i~nigdzie. Pewnego dnia
zatrzymaliby się na skan przed pójściem na basen, na wypadek utonięcia.

-- Przeziębienia -- powiedziała Sita. -- Jeżeli zrobimy ciała, ludzię użyją
uploadów, żeby otrząsnąć się z~przeziębienia.

-- Jak? -- zapytała Lodołasica ze szczytu trójkołowca, jego delikatne
mruczące warczenie otępiające wnętrza jej ud.

-- Proste -- powiedziała Sita. -- Zrób skan, wyciągnij nowe ciało z~magazynu, zrzuć dane.

Lodołasica parsknęła. 

-- Potem co? Wepchnij stare ciało w~młynek?

-- To tylko surowiec -- powiedziała Sita. -- Uśpij i~już nie budź. Jeżeli
jesteś sentymentalna, opraw w~ramkę. Lub zrób płaszcz, ugotuj na obiad.

-- Zdajesz sobie sprawę, że jest cały default, który myśli, że o~to
chodzi -- powiedziała Lodołasica. Była już pewna, że w~środku byli
szpiedzy. Mówiła ostrożnie, w~poczuciu, że jest nagrywana i~każdy brak
sprzeciwu, kiedy takie żarty się pojawią, może być wykorzystany
przeciwko niej. Mowa Tam o~sądzie nad zbrodniami wojennymi krążyła w~jej
mózgu.

-- Zdajesz sobie sprawę, że mają rację -- powiedziała Sita. Uśmiechnęła
się, zatrzymała. -- Wiesz, kiedy wystartowały pierwsze odchodnickie
projekty protez, większość ludzi biorących udział straciła rękę lub nogę
w Białorusi lub Omanie, byli zmęczeni płaceniem lichwiarzom za coś, co
bolało, ledwie pracowało, mogło być zdalnie wyłączone przez radio,
jeżeli opuścili ratę. Jednak kiedy pojawili się tutaj, zaczęli żyć,
zrozumieli, jak dużo było zostawione na stole przez konserwatywne
kompanie, które nie chciały wchodzić w~walki o~patenty i~nie widziały
powodu, żeby dodawać zaawansowane funkcje do czegoś, co i~musiałaś mieć,
to się zradykalizowali.

-- Przestali mówić ,,Chcę tylko rękę, z~którą przeżyję dzień'' i~zaczęli
mówić ,,Chcę rękę, która zrobi wszystko to, co moje stare ramię
robiło''. Z~tego miejsca, było łatwo przejść do ,,Chcę rękę, która
będzie lepsza niż moje stare ramię''. A stąd było jeszcze łatwiej do
,,Chcę taką rękę, która będzie tak skandalicznie niesamowita, że
odetniesz swoje ramię, żeby taką dostać''. Tak właśnie stanie się z~nieśmiertelnością. Nie tylko zdolność do powrotu z~martwych, ale
zdolność do przemyślenia, czym jest bycie żywym. Pojawią się ludzie,
którzy zdecydują się na deadheading na rok lub dekadę, żeby zobaczyć, co
się zdarzy. Będą ludzie ze złamanymi sercami, którzy uśpią się na
dwadzieścia lat, żeby nabrać dystansu od swojej eks. Założę się, że
pewnego dnia rozejrzysz się i~odkryjesz, że wszystkie dzieci są niskie
na swój wiek, okaże się, że wszystkie są usypiane przez rodziców za
każdym razem, kiedy mają napad złości, i~tracą dziesięć procent ich
czasu w~rzeczywistości.

Lodołasica pokręciła głową. 

-- Rzeczy tak mocno się nie zmieniają.
Większość ludzi będzie robić to samo, co teraz za dwadzieścia lat. Może
za sto lat\ldots 

-- Znienawidzisz to, jak usłyszysz, ale jesteś za młoda, żeby to
zrozumieć. Cokolwiek wynalezione przed Twoją osiemnastką cały czas tam
istniało. Cokolwiek wynalezione przed Twoją trzydziestką jest
podniecające i~zmieni świat na zawsze. Cokolwiek wynalezione potem jest
paskudztwem i~powinno być zakazane. Nie pamiętasz, jakie było życie
dwadzieścia lat temu, przed odchodnictwem. Nie rozumiesz, jak
\textit{różne} rzeczy są teraz, zatem myślisz, że rzeczy tak dużo się nie
zmieniają.

-- Kiedy byłam w~Twoim wieku, nie mieliśmy opuszczonych stref czy wolnych
projektów hardware'u. Ludzie bez miejsca zamieszkania byli bezdomnymi,
trampami, żebrakami. Jeżeli martwiłaś się zettami, szłaś na protest i~Twoja głowa była rozbijana. Ludzie ciągle myśleli, że odpowiedzią na ich
problemy jest znalezienie pracy, a~każdy, kto nie znalazł pracy, był
albo leniwy, albo nieudacznikiem, lub jeżeli miałaś dobre serduszko, to
społeczeństwo kogoś zawiodło. Rzadko kto mówił, że świat byłby lepszy,
gdybyś nie znalazł w~ogóle pracy. Nikt nie wytykał, że ludzie, jak Twój
stary, udawali korporacyjnych złodziejskich baronów, antybohaterów
dramatów, które kochali jako nastolatkowie i~którzy potrzebowali
wielkiej liczby pracowników i~bezrobotnych pod piętą na potrzeby
podobieństwa.

-- Jeżeli byłabyś uśpiona przez dwadzieścia lat i~obudziła się dzisiaj,
myślałabyś, że śnisz lub że masz koszmar. Jasne, osiemdziesiąt procent
ludzi, która była wtedy żywa, żyje teraz, i~osiemdziesiąt procent
budynków dookoła wtedy nadal istnieje. Ale wszystko na temat jak się
odnosimy do siebie i~miejsc, które zajmujemy, się zmieniło. Oni zwykle
myśleli, że wszystko zmieni się przez technologię. Teraz wiemy, że
powodem, dla którego ludzie pozwalają technologii zmienić świat, jest
to, że dla nich wszystko jest pojebane, nie chcą kurczowo się trzymać
tego, co mają.

-- Zetta chcą kontrolować, kto może stosować które technologie, ale nie
chcą ponosić kosztów zamykania wszystkich odchodzących w~gigantycznych
więzieniach lub wypracowania sposobu jak nas wrzucić w~młynki bez
robienia przedstawienia, więc jesteśmy tutaj na krawędzi świata,
odkrywając własne sposoby użycia rzeczy. Jest więcej osób niż
kiedykolwiek, którzy nie czują sympatii do tego, jak się rzeczy mają.
Każda z~nich byłaby szczęśliwy, wyrzucając wszystko normalne za szansę
zrobienia czegoś dziwnego, co mogło być lepsze.

Lodołasica rozważyła doskonałą burzę nieznośności jej ojca, śmierć
Billiama, słowa Etcetery, narastające poczucie, że wszystko było
popieprzone, frazę, którą wszyscy używali jako żartu. To nie był
śmieszny żart, ale ludzie opowiadali go cały czas.

Ha ha, ale poważnie.

Co by zabrało zrobienie jej uploadu? Kiedy dotarliby do Gospody,
ustawiliby skaner i~każdy mógłby zostać przetransferowany, zrobić skan
siebie, lub ,,siebie w~zakresie, który mógłby tolerować ożywienie w~software''. Czy dostałaby się? Chciałaby o~tym porozmawiać z~Roz.
Przyłapała się. Chciałaby porozmawiać o~tym z~Roz? Czy to nie znaczyło,
że Roz była osobą? Czy to nie odpowiadało na pytanie?

Ruszyła, kierując trójkołowcem przez las, pociąg wózków podskakujący za
nią.

\threeast

Byli niedaleko Gospody, kiedy interfejsy wszystkich zabrzęczały i~dały
znać, że bezpiecznie skopiowały wszystkie pięć skanów do sieci
odchodnickich, skany zostały zseedowane po całym świecie, tak
nieśmiertelnie nie do usunięcia jak tylko mogły być dane. Wszyscy się
rozluźnili. Ich wiedza, że nieśmiertelność jest prawdziwa, miała tylko
kilka godzin, ale już byli przerażeni na myśl o~permaśmierci. Żartowali
nerwowo, jak szybko każdy mógłby zostać zeskanowany, kiedy rozpakują się
w B\&B.

Stali się superczujni, niektórzy wzięli grzybki, i~niewypowiedziany
strach się upowszechnił. Śmierć miała dwa rodzaje: prawdziwa śmierć i~,,śmierć''. Jednak póki nie dotrą do Gospody, jedynym rodzajem, jaki
dostaliby, była permaśmierć. Strach stał się przerażeniem. Bali się
cieni. OC i~Gretyl oboje wyjęli bronie nieśmiercionośne, które mieli ze
sobą. Nie powiedzieli nikomu, co robią, a~gdyby powiedzieli, odbyłaby
się przerażająca debata. Teraz nikt nie zwrócił na to uwagi.

Byli tak blisko. Lodołasica znała to miejsce, szła tym ciągiem na
wyprawach łupieżczych dla nowej Gospody, gdy jej drony identyfikowały
nowy materiał, żeby przyśpieszyć prace.

Obserwowanie wyczarowywania się nowej Gospody było doświadczeniem
nawrócenia, dowodem cudowności Ziemi. Odeszli od starej B\&B, kiedy
tamte dupki się pokazały, i~postawili nową z~rzeczywistości czystej
informacji. To było ich przeznaczenie. Rzeczy, od których można było
odejść i~zrobić na nowo, nikt nigdy nie musiałby walczyć. Jeszcze nie,
nie mogli skanować dużych liczb ludzi, nie mogli ich ucieleśniać. Jednak
nadszedłby dzień, o~którym mówiła Gretyl, kiedy nie byłoby powodu dla
strachu przed śmiercią. To byłby koniec fizycznego przymusu. Tak długo
jak ktoś, gdzieś wierzył w~odtworzenie Ciebie w~ciele, nie byłoby powodu
nie ruszyć na karabiny maszynowe prześladowcy, nie było powodu, by nie
rozwalić sobie głowy o~kraty więziennej celi, nie było powodu\ldots 

Dron nad głową wydał powitalny dźwięk. Gospoda wysłała im eskortę.
Spojrzała na drona i~pomachała. Odmachał skrzydłami i~zawrócił.

-- Zbliżamy się -- zawołała, a~wtedy wróg zaszarżował od strony krzaków, w~ryku maszyn.

Było to osiem mecha, rodzaj tych, które zbudowali, żeby poradzić sobie z~najtrudniejszymi zadaniami budowlanymi. Te równie dobrze mogły być te
same. Na trzy metry wysokie, trzymały pilotów w~kokonach w~kształcie
krzyża, twarze wyglądające z~jam na piersiach mecha, oczy zasłonięte
ekranami panoramicznymi, które dopasowywały odświeżanie do ruchów
sakadowych pilota i~do nacisku ładunków, żeby zapewnić widok na żywo
tam, gdzie działał skafander. Każdy z~nich mógł podnieść kilka ton, ale
miał zabezpieczenia przed krzywdzeniem ludzi. Wyłączenie tego było tylko
aktualizacją firmware'u. Było wiele miejsc wśród odchodzących, gdzie
zapasy mechów były sportem wielu fanów.

Ruszyły najpierw po kontenery, podnosząc je i~metodycznie wykrzywiając
koła, żeby nigdy już się nie potoczyły. Lodołasica zeskoczyła, gdy
pierwszy pojemnik został podniesiony, zabierając ze sobą ATV, zraniła
się w~ramię i~kostkę, gdy upadła na runo leśne.

Gretyl postawiła ją na nogi, jej twarz pełna przerażenia. Gretyl złapała
ją za bolesne ramię, a~ona krzyknęła. Ściągnęli uwagę najbliższego
pilota mecha. Wielkie ciało odwróciło się ku nim. Mecha mogły obracać
się szybko, ich ramiona były dostatecznie szybkie, żeby wcisnąć łopatę w~zamarznięty grunt bardzo dokładnie, ale nie mogły biegać, ponieważ ich
żyroskopy wymagały czasu, żeby się ustabilizować po każdym kroku, zatem
chodziły, zataczając się krzywo. Mecha zrobił krok w~ich kierunku, pilot
bujający się w~kołysce. On -- dojrzała rudą brodę wystającą za chwytu na
głowę, zęby za rozszerzonych ust -- zatoczył się mecha i~coś w~tym, jak
to zrobił, sprawiło, że pomyślała, że nie ma dużego doświadczenia.

Gretyl puściła ramię Lodołasicy i~zmagała się z~gównoblasterem.
Lodołasica szybko odsunęła się z~drogi, gdy Gretyl dźgała tylny panel,
walcząc, żeby utrzymać układ misek wielkości monety skierowany we
właściwym kierunku, gdy kształtowały impuls infradźwięków, zmieniając je
w górę i~w dół w~zakresie częstotliwości rezonujących, dostrajając się
do tej jednej, która\ldots 

Kierowca próbował pochylić się do przodu, ale mecha nie mógł się skręcić
tak mocno bez przewracania, więc zablokował go w~trzydziestostopniowym
skłonie, jak u nadąsanego dzieciaka po wymuszonym przedstawieniu. Górna
połowa twarzy -- broda, usta, kwadratowe zęby -- wykrzywiła się.
Gównoblaster nie tylko rozluźniał wnętrzości, robił to skurczami, które
mieściły się w~zakresie pomiędzy porodem a~cholerą.

Gretyl sapnęła. Lodołasica wyciągnęła ją z~centrum potyczki. Trzy mecha
skupiały się na nich, po uszkodzeniu transportu. Lodołasica i~Gretyl
prawie zostały przewrócone przez ludzi uciekających od nich, ludzi,
których rozpoznawała, ale nie do końca. Biegły, zderzając się z~innymi
ludźmi. To była panika.

-- Muszę\ldots  -- powiedziała Gretyl. Reszta była niezrozumiała, ale
Lodołasica wiedziała, o~co chodzi. Ona i~OC byli jedynymi, którzy mogli
walczyć, rozejrzała się, dojrzała OC kierującego broń na mecha,
obserwowała operatora tracącego przytomność, zobaczyła dwoje ich
własnych padających na kolana, łapiących się za głowy i~krzyczących.
Promienie bólu sprawiały, że wydawało się, że skóra jest w~ogniu, a~ukształtowane infradźwięki odbijały się w~czaszce, powodując głuchotę i~prawie ślepotę.

Połowa mecha była obezwładniona, reszta brnęła przez tłum, a~Lodołasica
obserwowała przerażona, gdy następowali na jej kolegów, wzdrygając się,
gdy ich ramiona obracały się w~przeciwwadze ich pijanego zataczania,
oczekując w~każdej chwili, że te ręce wbiją się w~czaszki, lub podniosą
ludzi i~rzucą ich na szczyty drzew.

Ale nie, zobaczyła Lodołasica, mecha \ldots  uciekały. Biegnąc w~las, tuż
koło ludzi, a~to znaczyło\ldots 

-- Kurde, musimy uciekać -- powiedziała do Gretyl. -- \textit{Już!}

Dron był z~powrotem, a~przez chwilę jej panika zniekształciła go w~wielki pojazd niosący rakiety, jak ten widziany na filmach ze
zniszczenia OU. Ale to był tylko znajomy dron Gospody. Westchnęła ze
stresu i, utykając, ruszyła między drzewa. 

-- Chodźcie wszyscy, chodźcie!
-- Pociągnęła Gretyl, rzucając spojrzenia na drona, myśląc o~ludziach z~Gospody, obserwujących i~obgryzających paznokcie, retransmitujących do
reszty odchodzących, nawet do defaultu, gdzie spektakl niesprowokowanego
ataku na kolumnę naukowych uchodźców mógł wstrząsnąć świadomością opinii
publicznej poza możliwościami spindoktorów\ldots 

Dron zanurkował. Jej interfejs umarł. Kolejne trzy drony -- gładkie,
nisko wiszące rakiety i~anteny dla wysokoenergetycznych impulsów
elektromagnetycznych -- przeleciały z~rykiem, za nimi grom dźwiękowy.
Zniknęły nad horyzontem, na ich pojawienie rozległy się krzyki, panika
się zwiększyła, gdy ludzie kierowali się ku drzewom, biegnąc w~ślepej
panice. Widzieli rakiety.

Lodołasica i~Gretyl trzymały się na granicy lasów, śledząc smugi
kondensacyjne w~niemym horrorze, gdy białe pasy wykręciły się w~,,L'',
potem w~,,U'', gdy drony wykonały idealne zwrot w~formacji, korkociąg w~górę i~potem odwrócenie na plecach.

Lodołasica ścisnęła dłoń Gretyl. Gretyl odpowiedziała uściskiem. Chłodny
spokój spłynął na nią, jakby leżała w~łóżko, gorączkując, trzymana w~dłoniach kochanki.

-- Było warto -- powiedziała, myśląc o~ludziach, którzy nigdy już nie
umrą, Roz, która wkrótce będzie świadoma, która pamiętałaby ją jako
kogoś, kto pomógł wyleczyć najbardziej ostateczną chorobę z~wszystkich.

-- Było -- odpowiedziała Gretyl. -- Kocham Cię, kochanie.

-- Ja też Cię kocham -- powiedziała Lodołasica. -- Dziękuję za pozwolenie
mi na pomoc.

Obserwowały, gdy drony się zbliżały.

\chapter*{vii}

Rakiety przeleciały nad ich głowami w~las, gdzie główna grupa się
ukryła. Lodołasica zrozumiała w~swoim stanie oderwania, że operatorzy
dronów użyli wykrywania ciepła i~radarów milimetrowych do wyboru celów.
Ukrywanie się w~lesie było tak skuteczne jak naciąganie koca na głowę
przed strachami.

Druga seria przeleciała sto metrów za nimi. Lasy wybuchły płomeniami,
dźwięk prawie maskujący krzyki. Drony przeleciały ponownie, kierując się
na kolejne, niemożliwy zwrot na granicy horyzontu.

Drony były prawie nad nimi, kiedy, z~szarego nieba, bezpośrednio
\textit{w} nie wyleciało pięć rakiet, pozornie znikąd. Trzy uderzyły, kule
ognie i~potem, kilka sekund później, grzmoty. Dwie nie trafiły i~zniknęły z~widoku. Gretyl i~Lodołasica wykręciły szyje i~wtedy
zobaczyły: wielki, cichy, w~kształcie cygara, zeppelin, jeden ze
sterowców złotego wieku, tego typu, za którymi nostalgicznie wzdychał
Etcetera. Awaryjne wirniki kręciły się, gdy utrzymywał pozycję, śledząc
drony, gdy mijały, a~potem, gdy krążyły, gładko zestrzelił je z~nieba
kolejną salwą rakiet antydronowych.

Zepp pochylił się i~zszedł spiralą do ścieżki. Kiedy był dziesięć metrów
nad ziemią, wyrzucił drabiny i~liny, ludzie się wylali, ściskając
zestawy pierwszej pomocy, deski ortopedyczne, ubrani w~skafandry
przeciwpożarowe. Pobiegli do lasu, Lodołasica i~Gretyl pobiegły z~nimi,
bez dyskusji, gorąca wiedza zbawienia przebiegająca przez nie,
uruchamiająca rezerwy energii.

Pracowali w~lesie godzinami, szukając, zabierając rannych i~zmarłych na
noszach i~w niebo. Więcej osób dołączało, potem jeszcze więcej, a~kiedy
Lodołasica zaryzykowała powrót z~zespołem z~noszami, były dziesiątki
pojazdów Gospody na miejscu, od mechów do motorów transportowych,
poruszających się, żeby sprowadzić rannych.

Pomogła załadować nieprzytomną osobę -- zobaczył w~szoku, że to był OC,
tęczowe włosy zwęglone, twarz i~pierś masą oparzeń -- i~wstała. Inny
noszowy odwrócił się do niej, wziął jej dłonie, spojrzał w~oczy.

-- Lodołasica, hej, Lodołasica?

To była Tam, okopcona, zmęczona i~zmartwiona. Lodołasica chciała ją
uspokoić, nie chciała być ciężarem, więc próbowała powiedzieć \textit{W
porządku, chodźmy dalej pomagać}, ale nic nie wyszło. Była zaniepokojona
łzami spływającymi po jej policzkach. Próbowała ostrząsnąć się z~tego
uczucia, ale się nie udawało. Jakaś część jej, której nie mogła odszukać
na infografice, została strzaskana i~unosiła się poszarpana na
powierzchni jej umysłu.

-- Dlaczego nie zrobimy sobie przerwy, co? -- Używając nacisku na
ramieniu, tam wybuchł ból, aż sapnęła, Tam posadziła ją na ziemi i~przykucnęła. 

-- Jesteś w~szoku -- powiedziała. -- Wszystko będzie dobrze.
Myślę, że powinnaś zostać ewakuowana, ogrzać się, umyć, napić się
czegoś.

-- Gretyl\ldots  -- powiedziała.

-- Ta, Gretyl. Ta stara kobieta prawdopodobnie przebija się przez zarośla
jak wściekła nosorożyca. Nic jej nie zatrzyma. Jednak będzie się martwić
o Ciebie, co?

Lodołasica kiwnęła głową. Nie chciała, żeby Gretyl się martwiła. Ale
jednocześnie tylko \textit{chciała} Gretyl tutaj, solidność, na której
mogła złożyć głowę, dotyk jej palców we włosach Lodołasicy. Pomruk jej
głosu, słyszany przez poduszki jej piersi. Nie chciała pójść bez Gretyl.
Potrząsnęła głową.

-- Poczekam na Gretyl -- powiedziała.

-- Rozumiem, dziewczyno, ale to nie jest opcja. Nie ta inteligentna.
Chodź, Lodołasico, wiesz jak radzić sobie ze szokiem. Ciepło,
odpoczynek, uniesione stopy. Jesteś pokryta potem i~dyszysz jak
chihuahua.

Lodołasica wiedziała, że ma rację, czuła zimny pot na twarzy, ale
jednak\ldots 

-- Gretyl.

-- No weź, dziewczyno, nikt na to nie ma czasu. Jest tam wystarczająco
rannych. Nie potrzebujemy kolejnej. -- Tam rozejrzała się, nie zauważyła
Gretyl, wyszeptała serdeczne ,,\textit{Gówno}''. Wyprostowała się,
pomachała na kogoś. 

-- Hej! Chodź tutaj, dobra? Tak! Chodź tutaj, możesz?

-- W~porządku? -- powiedział znajomy głos. Spojrzała na nogi mężczyzny w~purpurowych rajstopach, buty z~rozszczepionym palcem jak buty w~sztukach
walki. Palce były opancerzone perlistymi, nakładającymi się warstwami
czegoś odrzucającego wilgość, jak łuski smoka.

-- Jestem ok, ale ona jest w~szoku. Nie chce się ewakuować, ponieważ
martwi się o~przyjaciółkę. Prawdopodobnie mogłabym ją zaciągnąć, ale
powinna odszukać jej przyjaciółkę i~dać jej znać, co się zdarzyło, albo
ona oszaleje.

-- Cóż, zawsze była lojalna wobec przyjaciół. -- Osoba przynależąca do nóg
przykucnęła i~spojrzała jej w~twarz.

-- Hej, Natty -- powiedział Etcetera.

-- Hubert Vernon Rudolph Clayton Irving Wilson Alva Anton Jeff Harley
Timothy Curtis Cleveland Cecil Ollie Edmund Eli Wiley Marvin Ellis
Espinoza -- powiedziała. Zrobiła grę z~zapamiętania tego, ich pierwsze
tygodnie jako odchodzących, hulając w~tej przesadzie. Powiedziała to
śpiewająco.

-- To \textit{nie} jest Twoje imię -- powiedziała Tam.

-- Mów mi Etcetera -- powiedział.

-- A mnie nazywaj Lodołasicą -- powiedziała Lodołasica. -- Natty dawno nie
ma.

-- I~świetna decyzja.

-- Pierdol się -- powiedziała.

-- Chodźmy, Lodi -- powiedział, pomógł jej wstać. Jej noga zasnęła, jej
rany zdrętwiały. Oparła się o~niego.

-- Gretyl -- powiedziała, nad ramieniem, do Tam.

-- Powiem jej -- powiedziała Tam.

-- Dziękuję.

-- Chcesz polecieć moim zeppelinem? -- spytał Etcetera.

-- Ta rzecz jest cholernie szalona -- powiedziała.

-- Uratowała Ci dupę -- odparł. Pokierował ją na nosze i~pozwoliła się
otulić kocem i~przypiąć do nich. Przypiął ją do uprzęży, złapał jedną z~lin przy noszach, szarpnął ją mocno i~unieśli się w~powietrze.

\threeast

Zimny wiatr na twarzy, gdy się wznosili, otrzeźwił ją, ale wznoszenie
było wolne i~kołyszące, ukołysało ją znowu w~drzemkę, z~której ledwie
się obudziła, gdy wnosili ją do brzucha zeppa, a~Etcetera przenosił ją
na miejsce na podłodze gondoli. Kołysała leniwie głową i~widziała, że
było wiele innych, w~tym OC, leżących bez ruchu. Miał kroplówkę w~ramieniu i~czujniki pokrywały jego spalone ciało. Poczuła narastającą
żółć w~gardle, odwróciła twarz w~inną stronę na czas, by wyrzygać rzadką
zawartość żołądka.

Jej nogi były uniesione, więc wymioty spłynęły jej po twarzy i~we włosy.
Zacisnęła oczy, kiedy wymiotowała, a~to niżej było teraz pokryte żółcią.
Ktoś otarł ją ręcznikiem, poczuła się zawstydzona. Ręce były pewne,
lekko otworzyła oczy i~potwierdziła, że to był Etcetera.

-- Tęskniliśmy za Tobą -- powiedział. -- Seth był szalenie ponury.

Uśmiechnęła się, ale wyszło to jak grymas. 

-- Też tęskniłam -- powiedziała, ale naprawdę nie tęskniła. Zrozumienie zaskoczyło ją przez
szokujące oszołomienie. Dlaczego za nimi nie tęskniła? Stawało się
jasne, kim była, i~zrywały się ostatnie nici łączące ją z~defaultem,
ojcem i~zettowością. Choć nie robiła tajemnicy z~pochodzenia na
kampusie, nie widzieli gniazda, które zamieszkiwała z~ojcem, wożona w~jego pancernym samochodzie, doświadczająca jego potężnego wpływu.

-- Gdzie dostałeś ten szalony zbiornik z~gazem?

Rozejrzał się. 

-- Marzenia się spełniają, co? Po pęknięciu bańki zeppów,
zostało kilkaset sterowców, które były mniej więcej zdolne do lotu,
gnijąc w~hangarach. Ktoś wpadł na pomysł zrobienia komunistycznych
imprez w~hangarach i~wtedy powstała cała sprawna flota. Władze lotnicze
szalały, kilka z~nich zostało uziemionych, ale te, którym się udało
dotrzeć do odchodników, wydawały się na razie bezpieczne. Ten pojawił
się w~Bańk kilka tygodni temu, najbardziej szalona załoga, którą
kiedykolwiek poznasz, świry wśród odchodzących, którzy żyli przez bańkę,
tak jak ja i~nie mogą uwierzyć, że wreszcie dostali sterowiec. Nazwali
ten \textit{Pierwsze Dni Lepszego Narodu}.

Jęknęła. Taki stereotyp odchodzących, mogła sobie wyobrazić załogę,
wystudiowaną atmosferę czystości odchodzących. Trudno jej było być w~pobliżu, ponieważ przypominało jej to tyle jej samej, w~tamtych czasach,
kiedy była przykładem bogaczki w~załodze komunistycznych imprez.

Miał wilgotną ścierkę i~wytarł wymiociny tak dobrze, jak potrafił.
Delikatna posługa znajomej dłoni była zniewalająca na tyle sposobów,
uczucie smutku-szczęścia-samotności-domu, jak dotyk matki, którą ledwie
znała. 

-- Co jeżeli posłaliby więcej dronów? -- spytała.

Wzruszył ramionami. 

-- Właśnie skończyły nam się rakiety. Pewna śmierć? -- Spojrzał na nią badawczo. -- Ale nie na długo, prawda? -- Odwrócił wzrok. -- To prawda?

-- Uploading? -- Kaszlnęła. W~ustach miały kwaśny smak, gardło paliło. -- Zależy, co masz na myśli, mówiąc prawda. Mam przyjaciółkę, która to
zrobiła, poznasz ją, jeżeli przetrwamy. Wyjaśni Ci to lepiej niż ja.

-- Pierwsze dni lepszego narodu -- powiedział przesycony ironią.

-- Lub dziwniejszego -- powiedziała. Sięgnęła po jego dłoń i~on ścisnął
jej.

-- Poradzimy sobie -- powiedział. -- Dziwne, czy nie, zdecydowanie
przerażamy Twojego ojca i~jego ludzi, zatem robimy \textit{coś} dobrze.

-- Jebać mojego tatę. I~jego ludzi.

-- Cóż, tak. -- Było szarpnięcie, prawie stracił równowagę. Skowyt
śmigieł, odczuwany przez pokład, się zmienił. -- Nareszcie w~domu. -- Ścisnął jej dłoń. -- Giga gig.

\chapter*{viii}

Gretyl odszukała ją w~onsenie, siedzącą w~najgorętszym basenie z~Limpopo, która zdiagnozowała jej potrzeby gorącej wody. Gretyl była z~Tam, która promieniowała wstydem ciała i~dyskomfortem z~nagością.
Lodołasica zrozumiała, jak mało myślała o~szczególnych problemach bycia
kobietą z~penisem, jak była zadowolona z~siebie, zakładając, że
odchodzące były swobodne, że byłoby to proste.

Chwiała się nad przepaścią zwątpienia i~pewności, że była niezdarą i~że
nikt nie bierze jej na poważnie. Gorąca woda wydawała się
klaustrofobiczna i~bolesna, gdy jej koncentracja wymknęła się i~jej
głupie ciało, chciało zwrócić na to jej uwagę. Jej twarz lśniła od potu.

Wyszła z~wody i~poszła do Gretyl. Jej włosy były przypalone, jedna ręka
była pokryta opatrunkiem gazowym. Gdy Lodołasica wstawała, jej ranne
biodro i~ramię wysunęło się z~wody, zimne powietrze spowodowało
pulsowanie tak nagle, że się zachwiała. Gretyl złapała jej ramię, a~Tam
złapała ją za drugie.

-- Cześć -- powiedziała, słabo. Limpopo odetchnęła, zamknęła oczy,
odchyliła głowę i~zanurzyła się po uszy. Gretyl przyciągnęła Lodołasicę
bliżej, a~kiedy Tam się odsunęła, objęła ją mocnym, piegowatym ramieniem
i przyciągnęła ją w~uścisk.

Pomimo całej skóry, było coś czystego w~onsenie, lub tak mówiła sobie
Lodołasica, gdy przypominała sobie pocałunek, bzykanie na sucho, które
ona i~Gretyl odbyły w~podziemnym kampusie, przez co udawała, że jej
mięśnie brzucha nie drżą od dotyku piersi Gretyl na jej. Potem doszły z~boku piersi Tam, jej twarz w~zgięciu zrobionym z~twarzy Gretyl i~Lodołasicy, jej penis dotykając jej uda i~sprawiający, że jej mięśnie
brzucha drgnęły.

-- Zaproś swoich przyjaciół do wody i~przedstaw ich, dziewczyno -- powiedziała Limpopo, nie otwierając oczu.

Rozplątali się powoli, potem, impulsywnie, przycisnęła Gretyl do siebie,
całując jej policzek, szczękę, ucho. 

-- Tak się cieszę, że tutaj jesteś -- powiedziała, zapach spalonych włosów w~nosie.

-- Ja też, dzieciaku -- odpowiedziała i~raźno weszła do wody.

\threeast

Długo czasu zabrało im zanurzenie się, wszystko skomplikowane przez
poparzone ramię Gretyl, a~kiedy się usadowiły, Limpopo dotarła do swojej
granicy, wstała i~wyszła. Przyniosła wiaderko lodowatej wody z~najzimniejszego basenu na brzeg ich, wycierając się przy użyciu
ręczniczka. Gretyl była nieświadoma, poddająca się wodzie, ale Tam
obserwowała z~uwagą i~Lodołasica obserwowała Tam.

-- Ile strat? -- powiedziała Limpopo, gdy Tam opowiadała historię
ostatnich ewakuowanych.

-- Troje zmarłych -- powiedziała Tam, płasko. -- OC się nie udało.

Lodołasica była drętwa. Niosła spalone ciało OC. A teraz było martwe.

-- Dużo też rannych -- powiedziała Tam. Lodołasica wyszła z~basenu.
Chciała płakać, ale łzy nie chciały się pojawić. Krzyżowała ramiona i~oparła głowę o~ścianę, oddychając przeponą. 

-- Wszystko ok -- powiedziała, gdy usłyszała kogoś\ldots  Tam? , wychodzącą z~basenu. -- Daj mi minutę.

Tam i~Limpopo rozmawiały, ale ona się wyłączyła, skupiona na oddechu i~gorąco-zimnej grze powietrza i~wody z~jej ciałem. Palec dotknął jej
ramienia i~niechętnie podniosła wzrok\ldots  na Setha.

-- Niech żyją zwycięscy bohaterowie! -- Potem, efektownie -- Teraz stałem
się Światami, niszczycielem Śmierci!

Uśmiechnęła się wbrew sobie. On był takim gnojkiem, ale nie był złym
człowiekiem. 

-- To dobre, Seth. Długo Ci zajęło wymyślenie tego?

Pokręcił głową. Był nagi i~pokryty gęsią skórką, odkryła, że jego ciało,
tak niedawno znajome, aż do punktu nudy, było fascynujące, w~sposób, w~jaki rzeczy, które były łatwo brane, teraz były zakazane. 

-- Ukradłem to
-- powiedział. -- Jakiś manifest gości spod San Francisco. Te dziwaki
Osobliwości w~strefach trzęsień, mają takie \textit{religijne} uczucia. To
takie śmieszne miasto. To było moje ulubione. Amatorzy plagiatują,
artyści kradną.

-- To ukradłeś od Picasso -- powiedziała.

-- Naprawdę? Nie sądzę. Nigdy nie czytałem jego książek. Musiałem ukraść
to od kogoś, kto ukradł to od niego.

Nie podjęła przynęta, choć zauważyła, że jej przyjaciele z~kampusu się
przyglądają. Znała grę Setha, nie chciała grać. Cieszyła się, że go
widzi, ale ona i~Seth grali dostatecznie jak na jedno życie.

Usłyszała, że Tam wychodzi z~wody, spojrzała, żeby obserwować, jak Tam
pomaga Gretyl, poczuła nieprzyjemne ukłucie zazdrości, zobaczyła, że
Seth zauważył. Seth nie potrafił zignorować instynktów kibica, był
zapalonym obserwatorem relacji. Zauważyła, że jego oczy skoczyły do
penisa Tam, potem do jej piersi, z~powrotem na jej twarz.

-- Potrzebujecie pomocy? -- spytał, podchodząc, oferując dłoń Tam, która
się chwiała, pochylona, żeby pomóc wyjść Gretyl bez zamaczania jej
opatrunku. Tam wzięła jego dłoń. Uśmiechnął się do niej swoim zwycięskim
uśmiechem, który, w~istocie, był jebanym zwycięstwem. Kiedy poszli w~odchodzenie, Seth miał początkowe stadia piwobrzucha. Odbudowanie
Gospody, chodzenie, noszenie, wyzwania antyszleperowe chodzenia w~las z~niczym i~opieranie się na dronach i~swoich umyśle, wyciągnęło go, dało
mu potężne mięśnie i~szerokie ramiona, które pasowały do maty
zakręconych włosów na piersiach. Lodołasica odkryła swoim fanowskim,
pajęczym zmysłem, że, gdy Tam obejrzała go od góry do dołu, klik, klik,
Lodołasica poczuła niemile widzianą zazdrość. Głupi mózg. Marzyła o~infografice, gdzie mogła przesunąć palcami, żeby wygnać małostkowe
myśli.

Limpopo przesunęła się na bok. 

-- Może teraz ciepły basen? -- To była
temperatura kubka herbaty zostawionej na jakieś dwadzieścia minut, w~którym można było siedzieć i~socjalizować. Limpopo sugerowała, żeby
odbyli miłą, cywilizowaną pogawędkę.

-- Wspaniale -- powiedziała i~pozwoliła się prowadzić Limpopo.

Zajęli miejsca po przeciwnych stronach, ramiona oparte na krawędzi.
Lodołasica spojrzała na swoje lewe ramię, ciemne siniaki i~zadrapania.
Gorąca woda i~chłodne powietrze sprawiły, że błyszczały się żywym
różowym. Bolało lekko.

Reszta grupy rozłożyła się, podnosząc poziom wody tak, że małe wodospady
spadły do odpływu. Seth był bardzo troskliwy o~Gretyl, która traktowała
go jako rozrywkę, gdy się zmieszał, rzucając, żeby zaoferować jej dłoń.
Lodołasica odniosła wrażenie, że część z~tego było dla dniej, ale
znacznie więcej trafić miało do Tam i~prawdopodobnie Limpopo.

Było uderzające, jak obecność mężczyzny zmieniało tak dużo w~ich
dynamice, tworzyło niewidzialne linie uwagi. Wzruszyła ramionami,
skrzywiła się na ramię, wymieniła spojrzeniem z~Gretyl, poczuła znowu
zwijanie nisko w~brzuchu. To było straszne, uczucia, które pojawiały
się, gdy patrzyła na Gretyl. Gretyl odpowiedziała spojrzeniem, przebiła
ją szczerymi oczami i~drżenie przebiegło od jej stóp do włosów.
Spojrzenie Gretyl mówiło \textit{Jesteś moja} i~\textit{Będziesz moja?} w~tym samym czasie. Silna-słaba. Miękka-stała. Jak Gretyl, wielkie
ramiona, umięśnione plecy, okrągły miękki brzuch i~wielkie, miękkie
piersi.

-- OC miał kopię -- powiedziała Tam, przerywając ciszę.

Oczywiście, że miał. Był rdzeniem projektu.

-- A inni? -- Uświadomiła sobie, że nawet nie znała imion zmarłych,
założyła, że skoro Tam nie powiedziała jej nazwisk, musieli być ludźmi,
z którymi nie była blisko, ale kto rozumiał Tam?

-- Nie -- odpowiedziała. -- Nie byli. -- Wyglądała na złą.

-- Ale OC miał.

-- Dobra -- powiedziała. -- OC był. Jest. Jest już grupa, która składa
klaster, używając każdego wolnego czasu obliczeniowego, który mogą
wyciągnąć z~Gospody.

Limpopo usiadła, przesunęła się z~boku na bok, pokazując im górną część
blizn. 

-- Mamy \textit{mnóstwo} mocy obliczeniowej -- powiedziała. -- Warsztat chodził 24 na 7, żeby wyprodukować nową logikę, od kiedy
odjechałaś, Lodołasico. Myślałam, że może do tego dojść i~mieliśmy
surowce. -- Uśmiechnęła się prywatnym uśmiechem. 

Stara B\&B upadła
zgodnie z~harmonogramem, około miesiąca, zanim Lodołasica wyjechała,
rozpuszczona przez gorycz, a~Limpopo z~nieukrywaną satysfakcją
przebierała przez pozostałości jej skalanego dzieła, ale część
schadenfreude znikła, gdy dotarła do miejsc, gdzie krew wyschła na
ścianach. Rzeczywiste walki nigdy nie były publikowane przez
merytokratyczną załogę, ale brzydkie wiadomości, oskarżenia i~przezwiska, które do nich prowadziły, już tak. Podobno, nikt nie zmarł,
ale jeżeli ktoś umarł, raczej nie ogłaszaliby tego faktu.

-- Słyszałam. -- Tam skinęła głową. -- Chciałam zobaczyć, co się z~tym
stanie, kiedy nie będziemy ograniczeni zasobami. Całe cargo dotarło,
racja?

-- Tak -- powiedziała Limpopo. -- Było trudno bez mechów.

Lodołasica pomyślała, jak mecha niszczyły ATV i~zasobniki. 

-- \textit{Czy}
te mecha były stąd?

-- Tak. -- Gniew przebiegł po twarzy Limpopo. Lodołasica prawie nigdy nie
widziała Limpopo wściekłej. To było straszne. -- Grupa najemników i~najętych infotechników zhakowała wszystko, używając jakiegoś zerodaya,
który kupili od szumowiny pracującej na infowojnie w~defaulcie. Przejęli
też flotę dronów, a~kiedy je odwirusowaliśmy, przejęli mechy.

-- Czy sieć Gospody jest bezpieczna?

Limpopo wzruszyła ramionami. 

-- To jest problem, co? Może wprowadzili
głęboko zmiany, których nigdy nie znajdziemy. Zrobiliśmy to, co możemy,
porównaliśmy sumy kontrolne programów z~backupami i~znanymi źródłami.
Zeroday został namierzony i~poprawiony cholernie szybko, ponieważ
wpływał na główną gałąź, wykorzystywaną stąd aż do obozów dla uchodźców
ONZ, gdzie żyją miliardy.

Gretyl zagwizdała. 

-- Cholera -- powiedziała. -- Możecie wyobrazić sobie to
zło, które można by zrobić z~takim exploitem wobec całej sieci UNHCR?

Limpopo i~ona spojrzały na siebie. 

-- To koszmar każdego admina UNHCR.
Nigdy dobrze nie współpracowałyśmy z~nimi. Wgrali patcha w~godzinę, w~całej swojej bazie, ale są inne projekty pochodne, takie jak nasz, które
mogą być podatne na blokadę lub sztuczki z~HVAC, które mogą ich spalić.

Miny obu kobiet były prawie identycznie poważnie. Lodołasica prawie się
roześmiała. Jednak zrozumienie, że były podobne na tak wiele sposobów,
powstrzymało ją. Były chwile w~trakcie jej lat w~Gospodzie, gdy była
zazdrosna o~romans Limpopo i~Etcetery, zakładała, że to miało związek z~Etceterą. Teraz zastanawiała się, czy nie chodziło o~Limpopo. Gretyl
była większą wersją Limpopo, większą w~każdy fizyczny i~emocjonalny
sposób. Zrozumienie spowodowało, że zapomniała o~konsekwencjach ich
dyskusji.

Tam podniosła się. 

-- Ile pewności, że sieci są w~porządku, jest
życzeniami? -- Była jedną z~tych osób, które mówiły to, co wszyscy
myśleli, ale nie chcieli powiedzieć. -- Zamierzamy uruchomić Roz online,
prawda? Potem OC? Może tych najemników, dlaczego kurde nie. Żadne z~nas
nie chce, żeby nasi przyjaciele byli martwi, póki nie wyprodukujemy
nowych CPU, prawda?

Seth ochlapał. 

-- To im powie. -- Uwielbiał rozrabiaki. Nie wspominając
innych rzeczy o~Tam.

-- To jest to -- powiedziała Limpopo.

Woda wydawał się mniej przyjazna, atmosfera mniej socjalna.

\chapter*{ix}

Roz była wszędzie. Załoga Gospody nie mogła się od niej oderwać. Łączyli
się, żeby rozmawiać z~nią w~całym budynku. Nawet z~całym czasem
obliczeniowym, musiała kolejkować rozmowy, dzwoniąc, gdy szli w~lasach
lub odpoczywali we wspólnym pokoju.

Jednak zawsze miała czas dla Lodołasicy.

-- Jak się uruchamia OC? -- spytała Lodołasica.

Roz nie miała już migającego kursora, ale Lodołasica ciągle potrafiła
przeczytać język ciała według jej pauz. Ta była dziwna. 

-- Niedobrze.
Próbowałam do niego dotrzeć, ale on nie chce stabilizacji. Ze mną to
była kwestia odnalezienia przestrzeni możliwości, gdzie mogłabym
poradzić sobie z~byciem ,,głową w~słoiku''. Możliwe, że OC nie ma
takiego podzbioru.

-- Co? To OC! Kocha te sprawy! Tym żył! To jak inżynier rakiet, który boi
się wysokości!

-- Nie znam zbyt dużo gości od lotnictwa, ale jednym z~powodów
uploadowania jest wszechogarniający egzystencjalny strach na myśl o~umieraniu. To nie dyscyplina, którą byś goniła, gdybyś nie była tym
tematem zainteresowana.

Lodołasica próbowała się zrelaksować. Gospoda nie wymagała tyle pracy,
żeby działała. Wykres krążący w~ich przestrzeniach społecznościowych
pokazywał, że jeżeli każdy poświęcił ośmiogodzinną zmianę co trzy dni,
mieliby dwa razy więcej roboczogodzin niż potrzebują. Jedna grupa była
ideologicznie zaangażowana w~nic nie robienie, tworząc ,,bezpieczną
przestrzeń'' dla ,,postpracy''. Rozumiała to. Siedzenie na dupie,
szczególnie w~miejscu publicznym, wywoływało poczucie winy. Niepracujący
byli moralnym wytłumaczeniem dla ludzi eksperymentujących przez dzień,
miesiąc (lub rok).

Siedziała na leżaku na trawniku B\&B, wielkim polu słodkopachnących
traw, prosperującym biomie zwierzaków, które szeleściły i~wzbijały się
nad tym. Słyszała Roz w~obu uszach, ale jej system był na tyle
inteligentny, żeby dołączyć dźwięki wiatru w~trawie, rzeczy goniących
się wzajemnie, jin-jang wietrznej bezcelowości i~panicznego poruszania.

-- Ile czasu minie, zanim uda Ci się ustabilizować OC?

Kolejna mikroprzerwa. Roz miała mnóstwo czasu obliczeniowego. Pauzy
musiały być celowe. Zapytałaby Gretyla, informatyka była dla niej
zagadką, pomimo kręcenia się po uniwersytecie.

-- Nie wiem, czy dam radę. Kiedy wymyśliłam, jak się stabilizować,
doszłam do wniosku, że będę w~stanie zastosować tę technikę w~każdym
symie. Ale jestem pojedynczym zbiorem danych. Ludzie są niepowtarzalni.
\textit{Ja jestem} niepowtarzalna. Może jestem rzadkim wyjątkiem i~nikt
inny tego nie zrobi, co ja zrobiłam.

-- Nie to mówiłaś\ldots 

-- To jest to, co wszyscy ludzie, którzy wiedzą, o~czym, kurwa, mówisz
\textit{teraz }mówią. Wszyscy biegają dookoła, krzycząc: ,,Śmierć jest
uleczona! USA! USA! USA!''

-- To jest Kanada.

-- Ta, ale brzmisz głupio, krzycząc ,,Ka-na-da!''. Łatwo się ekscytować,
kiedy nauka faktycznie coś robi, ponieważ nauka zawodzi i~robi notatki.
Chcemy osiągnąć ,,przełom'', ale nie wszystko jest przełomem. Czasami to
tylko mały krok do przodu. Albo ślepa droga. Próbuję uruchomić OC, ale
może jedynym sposobem, aby go obudzić, jest stan tak zniekształcony, że
nie można go rozpoznać. Modelowałem, używając siebie jako szablonu i~mieszając nasze modele po trochu, aż doszliśmy do hybrydy, z~wystarczającą ilością mnie, aby utrzymać go przy życiu. Nie ma na to
czystego sposobu. Prawie wszystko, co wypróbowałem, było niczym, co
można by rozpoznać jako którykolwiek z~nas. Co ciekawe, nie mam ochoty
na tworzenie z~powietrza szalonych, nieśmiertelnych syntetycznych
osobowości. Mamy dość pieprzonych dziwaków.

-- A co z~innymi? Inni badacze?

Roz wydała nieprzyjemny dźwięk. 

-- W~Madrycie byli prawdziwi popaprańcy,
którzy uruchomili moją wersję, próbowali mnie zmusić do pomocy. Ta kopia
zabiła się, po wysłaniu wiadomości do wszystkich innych grup, mówiąc im
o całym tym złym gównie, które się działo. Ale Madryt był jedynym
laboratorium, któremu udało się doprowadzić sym do stabilnego stanu.
Myślałam o~tym, żeby dać wszystkim innym pozwolenie na eksperymentowanie
z udostępnianiem moich wersji online, choć to takie przerażające.
Wygląda na to, że to może być jedyny sposób, by się gdziekolwiek dostać.
Nauka jest niejednolita. Czasami sukces następuje po sukcesie, ale
czasami przez weekend dostajesz pleśń na szalce Petriego i~spędzasz
życie, próbując dowiedzieć się, co właśnie się wydarzyło.

Kolejna pauza.

-- Domyślam się, że jest mnóstwo instancji mnie działających w~defaulcie.
Zetty i~ich szczury laboratoryjne nie mieliby z~tym żadnego problemu.
Doprowadzali OC do szaleństwa, w~ten sposób, że dodawali nasze badania
do swoich, ale nigdy nie widzieliśmy, co zrobili z~naszej pracy. Ale za
każdym razem, gdy odnosiliśmy sukcesy, ich szczury laboratoryjne były
kuszone do odchodzenia i~dołączenia do nas, ponieważ każdy chce pracować
dla zwycięzców. Więc przynajmniej wszystkie moje bliźniaczki działają
jako nieodparta pokusa, by zwolnić szefa i~odejść.

-- Zastanawiasz się czasem, czy jesteś w~laboratorium defaultu i~jesteś
oszukiwana przez otaczający świat?

Skomputeryzowany śmiech. Gretyl mówiła, że Roz miała za życia naprawdę
dziwny śmiech. Dziwaczny komputerowy śmiech był wiernym odwzorowaniem.
Musiała być dziwna jak buty dla węża. 

-- Nie ma mowy. Zbyt wiele testów
Turinga do zaliczenia. Cały czas rozmawiam z~wami wszystkimi. Mogliby
spieprzyć moją zdolność do wykrywania, czy rozmawiam z~botem, czy nie,
ale to też uczyniłoby mnie zbyt głupią, bym mogła pomóc. Jestem tak samo
pewna, że wiem teraz, co jest symulacją, a~co jest rzeczywistością, tak
jak wtedy, gdy byłam żywy. Powiedzmy dziewięćdziesiąt pięć procent.

-- A pozostałe pięć procent?

-- Stary nie-żart AI. W~przyszłości wymyślimy, jak symulować wszystko,
więc tak zrobimy. W~całej historii rzeczywistego wszechświata pojawi się
znacznie więcej symulowanych wszechświatów niż prawdziwych
wszechświatów. Więc bardziej prawdopodobne jest, że jesteś symem, niż
rzeczywista, cokolwiek znaczy ,,rzeczywista''.

-- Mój mózg boli.

-- Nie martw się, kiedy będziemy Cię symulować, upewnimy się, że jesteś w~tym stanie, któremu podoba się ten pomysł. Ha ha, ale poważnie. To jest
jak Meta, bycie w~ten sposób. Czasami skręcam suwak, patrzę w~przyszłość, widzę, jak blisko jestem krawędzi do pełnej paniki. Ciekawe
jest modyfikowanie tego gówna w~czasie rzeczywistym. Nie znasz wolności,
dopóki nie doświadczyłaś wolności poznawczej, prawa do wyboru stanu
swojego umysłu.

-- Nie mogę się doczekać.

-- Jesteś ironiczna, ale poważnie, nie-ucieleśnienie jest niesamowite.
Jeśli sprawa z~klonami, którą robią w~Lagos, zadziała, będę pierwsza,
który wskoczy z~powrotem w~ciało, ale za tym będę tęskniła. Jest w~tym
coś czystego. To znacznie prostsze niż psychoterapia i~skuteczniejsze.

-- Chyba że jesteś OC.

-- Różne parametry dla różnych symów. -- Potrafiła sprawić, by głos
komputera brzmiał na zadowoloną z~siebie.

Lodołasica uznała, że pomysł ,,głowy w~słoiku'' jest niebezpiecznie
fascynujący. Wspaniale byłoby zmniejszyć lęki, dorównać swojej
intelektualnej wiedzy, że nikt nie czeka, aż pokaże swoje prawdziwe
kolory zetty emocjonalną pewnością, że wszyscy wiedzieli, że jest
oszustką. Gdyby poszła na terapię, aby tak spasować, byłaby schlebania
za swoją samowiedzę, ale gdyby wzięła lek, który tak zrobił, uciekłaby
od rzeczywistości. Zastanawiała się, co ludzie pomyśleliby o~symach,
które zrezygnowały z~leków i~terapii.

-- Nie mogę znieść samego siedzenia -- mruknęła, patrząc na notatki
robocze, dumnie leniuchując. -- Muszę \textit{coś zrobić}.

-- Służymy też tym, którzy siedzą i~pierdzą. -- Lodołasica się
uśmiechnęła.

-- Ze wszystkich rzeczy, o~których myślałem, kiedy odchodziłem, nigdy nie
spodziewałem się, że porozmawiam z~symulowanym neuronaukowczynią z~niewyparzoną gębą.

-- Jestem prawdziwym neurobiologiem.

-- Wiesz, co mam na myśli.

-- Zamierzam rozpocząć program mikrokorekty ludzi szukających poprawnych
przymiotników do opisania martwych, nieśmiertelnych, symulowanych
sztucznych ludzi, takich jak ja.

-- Nie masz jakichś badań, które powinnaś prowadzić?

-- Robię to, patrząc w~podglądzie na tę rozmowę i~przycinając gałęzie,
żeby znaleźć ścieżki trwającego dialogu. Próbując symulować to, co
robiłaś, kiedy ciągle popełniałam samobójstwo, zanim się
ustabilizowałam. Formułuję hipotezy na podstawie moich transkrypcji i~przymierzam je do Ciebie.

Lodołasica się wiła. 

-- Dlaczego?

-- Chcę uogólnić rozwiązanie oparte na danych, aby dopingować kurwa
ludzi. Mogłabym zastosować je do symów takich jak OC.

-- To mnie nie pociesza.

-- Myślę, że tak.

Lodołasica poczuła przez chwilę introspekcję jak w~programie. 

-- Dobra,
trochę mnie to pociesza.

-- Dobrze. Zanotowane. -- Głos komputera zaczął mówić z~dykcją niemiecką.
-- Połóż się na sofa, bitte und powiedz mi o~seinen rodzicach.

\chapter*{x}

Gretyl i~Lodołasica zaglądały do sal, w~których działał uniwersytet.
Małe pokoje na najwyższym piętrze zostały zarekwirowane przez zespoły
badawcze, które wepchnęły do nich od trzech do pięciu osób, hakując
różne modele. Większość z~nich pracowała nad symulacją OC, ponieważ OC
był uwielbiany, i~byli przerażenie możliwością, że naukowiec, który znał
i robił najwięcej dla symulacji, był potajemnie zbyt przerażony, by być
sprowadzony z~powrotem. Jeżeli on nie mógł wrócić, to czy któreś z~nich?

Inni ludzie chcieli korzystać z~tych pomieszczeń. W~miarę upływu dni
praca naukowców traciła na ważności. Zaczęto mówić o~renowacji ruin
oryginalnej Gospody jako nowego kampusu, wskazówka, aby przestać
zabierać stamtąd dobre rzeczy.

Ekipa uniwersytecka miała to w~dupie.

-- Dlaczego nie mieliby? -- powiedziała Gretyl, gdy razem z~Lodołasicą
bezskutecznie stukały w~powierzchnię, szukając prywatnego miejsca na
rozmowę. -- Chodźmy na spacer, będzie fajnie dla odmiany. 

W~końcu minął
tydzień srania mrożonym kremem i~gradem. Słabe słońce wyglądało zza
puszystych chmur na niebie, które pokazywało ślady błękitu.

Przeszukiwali skrzynki w~poszukiwaniu pasujących kaloszy, przeszukując
skatalogowane mami. Seth się wprosił. Była z~nim Tam, co nie zdziwiło
Lodołasicy. Wiedziała, że uprawiali seks, choć nie byli publicznie parą.
Byli bardzo blisko w~przytulania, rozciągając definicję przytulania w~sposób, który był w~łagodnym złym guście w~odchodzeniu (choć było to
powszechne).

-- Chodźcie -- powiedziała Gretyl. -- Uciekamy od ponurej rzeczywistości
odchodzenia, aby być beztroskimi wędrowcami po dziewiczych lasach.

-- Piąte pokolenie byłej farmy drzewnej -- powiedział Seth. -- Zanieczyszczenia metalami ciężkimi i~zapadająca się żwirownia.

-- Chodź, słoneczko. Włóż buty albo przegapimy twój fachowy komentarz.

Tam miała buty dla siebie i~Setha. Wbili się w~sięgające do kolan
kalosze i~ruszyli.

Spacer był relaksujący, odgłosy ptaków i~zapachy roślinności z~rozgrzewającego się lasu. Ale Seth nie był w~stanie się zrelaksować.
Dowcipkował, wybiegał na przód, gubił się, śpiewał niegrzeczne piosenki.

-- Jaki jest problem Twojego chłopaka? -- powiedziała Gretyl.

Tam westchnęła. 

-- Nie zamierzam potwierdzać tej części o~,,chłopaku''.

-- Dobra, ale co go gryzie?

Tam spojrzała z~ukosa na Lodołasicę. Lodołasica często się zastanawiała,
co o~niej myśli Tam. Ona i~Seth nigdy nie byli oficjalnie, tylko
przeciągający się w~nieskończoność seks. Chociaż miłość nie była grą i~nie było punktów, zdecydowanie wygrała swoją rundę z~Sethem, nie
odwracając się, podczas gdy on był kapryśny, kiedy się z~nim rozstała,
wysyłając głupie e-maile do podziemnego kampusu. Od czasu jej powrotu
ledwie zwracał na siebie jej uwagę. Myślała, że on i~Tam dużo o~tym
rozmawiali, jaką była totalną suką. Chłopcy tak zrobili. Nie wiedzieli,
że kiedy mówiłeś dziewczynie, że była mniej szalona niż wszystkie inne
dziewczyny, to ona wiedziała, że kiedy się rozejdą, to tak powie
następnej o~tym, jak szaloną krowa \textit{Ty} byłaś.

-- To ja?

Oczy Tama się rozszerzyły. 

-- Zupełnie nie! Z~Tobą jest spoko. Wydaje się
naprawdę szczęśliwy na froncie romantycznym. -- Zarumieniła się, niepasująco do jej charakteru.

Lodołasica się roześmiała, a~chichot Gretyl był \textit{nieprzyzwoity}, co
sprawiło, że Lodołasica roześmiała się jeszcze mocniej, nawet gdy się
wierciła. 

-- To dobrze. Poważnie. -- Uśmiechnęły się do siebie. Tam miała
rację co do deadheadingu, była jedyną nietechniczną osobą po ataku. To
była niewypowiedziana więź i~dystans między nimi.

-- To jest to. -- Pomachała ramieniem.

-- Kanada? -- spytał Lodołasica.

-- Odchodzenie? -- spytała Gretyl.

-- Wieś. Tęskni za miastami. Czytał o~Akron, łapał pomysły.

Akron zaczęło się, gdy wyjeżdżała do kampusu OU. Walkaways zrealizowali
skoordynowany masowy squatting w~całym mieście, z~którego osiemdziesiąt
pięć procent było zabite deskami i~pod wodą, a~obligacje były oparte na
hipotekach w~depozycie Federalnej Służby Rynków Finansowych w~Moskwie,
podczas gdy rozgrywał się kryzys Gazpromu. Przelecieli pod radarem,
płynnie i~skoordynowanie. Pewnego dnia Akron został chaotycznie zajęty
przez bezdomnych, a~następnego dnia armia odchodzących ponownie
otworzyła każdy zamknięty budynek, w~tym remizy strażackie, biblioteki i~schrony. Fabryki zamieniły się w~faby, załadowane surowcem, zasilane
polami wiatraków, które pojawiły się w~nocy, elektrolizując wodór ze
szlamu płynącego w~rzece Little Cuyahoga, zasilając ogniwa wodorowe,
które odchodnicy zwozili na taczkach.

Default został zaskoczony. Powódź w~Connecticut związała FEMA, Federalną
Agencję Zarządzania Kryzysowego, i~Gwardię Narodową. Wykonawcy, którzy
wspierali FEMA, nie mogli wykorzystać swojej normalnej praktyki
zatrudniania lokalnych talentów jako oddziałów awangardy. Zanim się
zmobilizowali, cała ich pula rekrutacyjna została odchodzącymi.

Dało to odchodnikom w~Akron -- nazywali siebie ,,ad hoc'', mówili, że
praktykują ,,adhokrację'' -- cenny tydzień na konsolidację. Kiedy default
oblegał Akron, stali się światową sensacją medialną, źródłem
niekończących się spotkań demonstrujących szczęśliwy świat obfitości
uratowany ze spalonej pustki i~nieobecnych właścicieli.

-- To ekscytujące -- powiedziała Lodołasica.

-- Więcej niż ekscytujące. To jest \textit{miasto}. Nie wioska ani nie
obóz. Pierwsze, ale nie ostatnie. Teraz walczą o~Liverpool, Liverpool, i~Ivreę, gdzieś we Włoszech, oraz Mińsk, co jest kurewsko szalone, bo mali
Łukaszenkowie z~radością ścięliby im głowy i~powiesili bebechy na placu
w centrum. Mogłaś to przegapić, ponieważ tutaj wszystko było szalone,
ale tam się też zaczyna.

Gretyl zrobiła minę, którą Lodołasica rozpoznała jako grzeczne
niedowierzanie i~powiedziała: 

-- To bardzo ekscytujące.

Tam też znała tę minę. 

-- Gretyl, dzieje się więcej niż na kampusach.
Ludzie, którzy nie są naukowcami, również mogą coś zrobić.

-- Inżynierowie też, prawda? -- powiedziała Limpopo.

Tam skrzyżowała ramiona.

-- Żartuję. Śledziłam to. To jest dokładnie tego typu rzeczy, o~jakich
marzyłyśmy jakieś dziesięć lat temu, zanim pojawiło się \textit{słowo}
,,odchodzić''. Ale były inne próby. Istnieje powód, dla którego projekty
odchodnickie to budynek lub dwa, gniazdo os zaklinowane w~szczelinie
defaultu. Wszystko powyżej zamienia się od zabawnie dziwnego do
zagrożenia, które oni mogą spalić w~samoobronie.

Lodołasica kiwnęła głową. To był rachunek, który wykonywali, planując
komunistyczne przyjęcia, słodki punkt między czymś wystarczająco dużym,
by miało znaczenie, ale nie tak wielkim, żeby trzeba było to zdeptać.

-- W~każdym razie nasz młody człowiek ma w~głowie, że powinniśmy
zorganizować własny Akron. Nie odchodzenie: \textit{przychodzenie}. Kurwa,
\textit{bieg}.

Limpopo prychnęła. 

-- On da się zabić. \textit{Rozwalą }Akron, zanim
pozwolą nam je utrzymać.

Nagły gniew Tama zaskoczył Lodołasicę. 

-- Poważnie, jebać to. Celem
odchodzenia są pierwsze dni lepszego narodu. Kiedyś to było coś więcej
niż przewracanie oczami, to była poważna idea. Pewnego dnia odchodzący i~default zamienią się miejscami. Nie ma wystarczająco dużo ludzi
posiadających roboty, żeby kupować rzeczy, które te roboty wytwarzają.
Jesteśmy balastem. -- Spojrzała na Lodołasicę, może w~celu wsparcia, i~żeby nie wskazać na osobę, która nie byłaby balastem.

-- Słyszałem wszystko o~lepszych narodach -- odpowiedziała gniewnie
Limpopo. -- Są rzeczy, które są poważne, jak na uniwersytecie, które dają
nam władzę, by naprawdę odejść, nawet od śmierci; i~są wielkie bzdury,
takie jak zajmowanie miast. \textit{Najlepsze}, co mogę powiedzieć o~Akron, jest to, że może odwrócić armie default od nas. Myślę, że jest
szansa, że wiadomość, że zdobywamy miasta \textit{i} szturmujemy życie
pozagrobowe, da im wymówkę, by polować na nas jak na psy.

Powiedziałyby więcej, rozzłościłyby się, ale Seth rzucił się przez
zarośla z~uśmiechem, bawiąc się jak przerośnięty pies. Lodołasica nie
lubiła psów.

-- Co przegapiłem?

Odwróciły wzrok. Potrząsnął głową niczym pies. 

-- To wspaniały dzień,
żeby żyć! Spójrz na to niebo!

Rozległ się huk, bardziej wyczuwalny niż słyszany; ryk, fala gorącego
powietrza, która ich porwała i~rzuciła w~drzewa.

\chapter*{xi}

Kiedy wrócili, Gospoda była w~płomieniach. Wydawało się, że ogień jest
wszędzie, ale było jasne, że skupił się na stajniach i~elektrowni. Pożar
w ogniwach wodorowych miał być niemożliwy, były projektowane z~myślą o~pięciu rodzajach odporności na awarie, konstrukcja tak szeroko
stosowana, że wady były szybko wykrywane i~naprawiane. Jednak sądząc po
wraku, wybuchły.

Gospoda również płonęła, ale ogień wydawał się pod kontrolą, woda
wypływała z~okien, w~których uruchomiły się systemy automatyczne. To
była trzecia katastrofa Lodołasicy od tygodni, odczuwała interesujące
mieszane uczucia, które nabrała w~szpitalu, widząc mecha wychodzące z~płonącego budynku z~naręczami uratowanego sprzętu.

-- Gówno. -- Limpopo rzuciła się w~ruch. Lodołasica patrzyła z~podziwem.
Limpopo przyjrzała się tym samym szczegółom w~mgnieniu oka, ale gdy
Lodołasica była zmrożona, Limpopo została pobudzona do ruchu. Pobiegła
do ambulatorium, obracając głowę. Jej obecność była wystarczająca, by
podeszli do niej troje z~tych opiekujących się rannymi. Wskazała
energicznie na płomień ze stajni, a~wszyscy skinęli głowami i~ruszyli,
odsuwając rannych dalej. Pożar nasilił się w~ciągu kilku sekund. Teraz
zmierzała do pilota mecha i\ldots 

-- Chodźcie -- powiedziała Gretyl. -- Chodźmy pomóc.

Lodołasica była wdzięczna, że powiedziano jej, co ma robić.

\threeast

Stracili wszystko. Przez chwilę pożary były pod kontrolą, tylko trochę
pracy, aby ugasić ostatnie płomienie w~gospodzie, ale kiedy zamienili
ogniwa energetyczne w~mechach i~przynieśli nowe węże, z~tyłu nastąpiła
nowa eksplozja, kolejna fala wybuchowa, a~następnie płonące szczątki.
Zebrali się na trawniku przed domem, inaczej wszyscy by zginęli.

Drony z~Gospody przeszukiwały okoliczne tereny, łącząc usługi sieciowe
nad dziurami w~sieci pozostawionej przez upadek Gospody. Karmiona ich
inteligencją, załoga cofała się i~wracała, kierując się na zachód, w~kierunku zabudowanego zewnętrznego obwodu default, gdzie kończyły się
dzikie tereny, a~zaczynało się Toronto. Był to najmniej gościnny
kierunek, ale przy drodze znajdowała się osada, w~której \textit{Lepszy
Naród} mógł się zatrzymać. Załoga sterowca wrzuciła wszystko do
wirników, chociaż trzmiele miały być sterowane tylko w~sytuacjach
awaryjnych. To się kwalifikowało.

Nie było zgonów. To był cud, ale Limpopo miał teorię: 

-- Myślę, że
zamachowcy stracili nerwy. Stajnie wybuchły podczas cyklu konserwacji,
kiedy nikogo tam nie było. Elektrownia wybuchła dziesięć minut później,
mnóstwo czasu, by wszyscy przeszli na trawnik i~gapili się na stajnie, z~dala od wybuchu. Materiały wybuchowe w~gospodzie, które wybuchają kilka
\textit{godzin} później? Albo mamy do czynienia z~terrorystą, który jest
beznadziejny, albo chcą mieć pewność, że straty są minimalne. Tak byś
zrobiła, gdybyś chciała przekonać swoich szefów, że byłaś dobrym, małym,
szalonym zamachowcem, ale nie chciałaś zbyt dużo ofiar na rękach.

-- Limpopo, to był długi dzień, a~Ty byłaś niesamowita -- powiedziała
Lodołasica. 

Siedzieli skuleni w~namiocie, siedmioro w~sypialni
przeznaczonej dla dwojga, deszcz bębnił po namiocie nad głową i~smagał
boki. Rozbili obóz na samym środku drogi, korzystając z~popękanego
asfaltu autostrady jako podstawy. Nad drogą była kopuła na tyle, by
zapewnić odpływ do zatkanych rowów po obu stronach. Izolujące komórki na
podłodze namiotu pomarszczyły się nad wierzchołkiem kopuły, trzeszcząc
jak folia bąbelkowa, gdy były poruszane. Byli śmiertelnie zmęczeni,
głodni i~ranni, ale nikt w~ich namiocie nie zamierzał spać w~najbliższym
czasie.

-- Ale to brzmi jak bzdura. Uniwersytet został zaatakowany przez
najemników, a~w~drodze powrotnej znów zostaliśmy zaatakowani przez
najemników. Dlaczego zakładasz, że te bomby zostały podłożone przez
podwójnych agentów? \textit{Sentymentalni }podwójni agenci? Zadaj sobie
pytanie, czy to nie sprawia, że jesteś \textit{szczęśliwsza} wyobrażając
sobie, że źli ludzie byli szantażowany w~celu infiltracji, ale odkryli,
że jesteśmy tak wspaniałe, że nie mogli zdobyć się na zabicie nas?

Przerażający błysk gniewu na zwykle spokojnej twarzy Limpopo. Lodołasica
była zadowolona, gdy Limpopo dołączyła do ich namiotu i~swoją obecnością
namaściła go na ,,klub fajnych ludzi''. Kiedy oczy Limpopo rozbłysły,
wyglądało to jak uwięzienie z~niebezpiecznym zwierzęciem. Odsunęła się
i, ku swemu zakłopotaniu, jęknęła. Limpopo się opanowała.

-- To nie jest całkiem głupie. Trudno powiedzieć, kiedy się oszukujesz.
Odkrycie tego było głównym projektem mojego życia. Jednak. -- Odwróciła
się i~słuchała, jak wiatr smaga namiot, dotknęła chłodnej tkaniny. -- Dobrze, tak. Może chcieli, żebyśmy byli w~trasie i~wysyłają drużynę
przechwytującą po każdego, kto naprawdę rozumie uploading. Może
wiedzieli, że Gospoda ma monitory działające w~czasie rzeczywistym,
dzięki którym wyglądaliby jak potwory, gdyby zostało tam dużo ciał, ale
jeśli nas zabiją tutaj, to mogą nas wepchnąć do rowów i\ldots 

-- Rozumiem -- powiedziała Lodołasica. 

Nie mogła już więcej znieść.
Samooskarżenia po spotkaniu z~załogą kampusu w~końcu ustąpiło miejsca
strachowi. To była niemal ulga być torturowanym przez coś zewnętrznego,
a nie tylko jej wewnętrzny, dokuczliwy głos. Gretyl miała wokół jednego
z bicepsów tatuaż z~napisem: STRACH JEST ZABÓJCĄ UMYSŁU, ale jeśli
chodzi o~Lodołasicę, jej umysłowi przydałoby się trochę zabijania.

Chciałaby, żeby ona i~Gretyl były same. Coś w~byciu objętym przez Gretyl
ukoiło ją w~głębokim miejscu, wyłączyło głos, który znał wszystkie jej
słabości. Nigdy tak nie miała, nie z~chłopcami czy dziewczynami. Czasami
miewała ulotne chwile spokoju po ruchaniu, ale z~Gretyl przychodziło to
łatwo, nawet bez seksu.

Jak lubił jej przypominać głos, psychologia zakochania się w~starszej
kobiecie, gdy Twoja matka prawie cię porzuciła, nie była złożona. Cały
spokój, jaki Lodołasica uzyskała od objęć Gretyl, sprawił, że zaczęła
się zastanawiać, czy oddaje Gretylowi cokolwiek w~zamian.

Ona \textit{naprawdę} chciała pobyć trochę czasu sama na samą z~Gretyl.

Limpopo opadła i~Lodołasica zobaczył coś jeszcze rzadszego niż gniew
Limpopo: wyczerpanie. 

-- To mogą być samolubne bzdury, że życie w~odchodzeniu zmiękczy najtwardsze serca i~przekształci złodziejaszków w~utopistów postniedoboru, ale czasami mnie to trzyma przy życiu. Część
mnie chce zostać na nogach, robiąc badania kryminalne w~plikach B\&B,
znajdując kreta, ale reszta mnie chce żyć z~fantazją o~niepowstrzymywalnej moralnej perswazji. Wiem, że nie potrzebujemy, aby
wszyscy na świecie zgodzili się, aby to zadziałało, ale musi istnieć
masa krytyczna ludzi z~,,powitalnym poczęstunkiem'', inaczej nigdy nie
wygramy.

-- W~porządku. -- Gretyl przerwała milczenie, szturchała ekran w~sposób,
który promieniował ,,zostaw mnie, pracuję'' (Gretyl była w~tym dobra). -- Co to osoba z~,,powitalnym poczęstunkiem''?

-- Och. Jeśli zdarzy się katastrofa, czy pójdziesz do domu sąsiada z: a)
poczęstunkiem czy b) strzelbą? To teoria gier. Jeśli uważasz, że Twój
sąsiad nadchodzi ze strzelbą, byłabyś idiotką, wybierając jedzenie;
jeśli on uważa tak samo, możesz się założyć, że też nie wybierze wersji
a. Sposobem na otrzymanie do a~jest zrobienie a, nawet jeśli myślisz, że
Twój sąsiad wybierze b. Czasami on wyceluje w~ciebie broń i~powie, żebyś
opuściła jej ziemię, ale jeśli trzymał broń tylko dlatego, że myślał, że
też ją będziesz mieć, to zabezpieczy broń i~możecie wydać ucztę.

-- Teoria gier -- powiedziała Gretyl. -- To jest polowanie na jelenie.
Dwóch myśliwych razem może złapać jelenia, główną nagrodę. Albo sam
myśliwy może złapać tylko króliki. Obaj chcą dostać jelenie, ale jeśli
nie będą sobie ufać, dostaną króliczą niespodziankę na kolację.

-- Nie wiedziałam, że istnieje na to nazwa. Dobrze wiedzieć. Kiedy
wszystko się ułoży, będę musiała trochę poczytać. Kiedy sprawy
przybierają zły obrót, jeleń odbudowuje coś lepszego niż to, co
spłonęło. Królik kuli się w~przerażeniu w~jaskini, je zupę ze skóry
obuwia, mając nadzieję, że nie umrzesz na gruźlicę, bo nie ma już
szpitali. Zawsze myślałem, że cały projekt odchodzących był sposobem na
przekształcenie ludzi w~osoby z~poczęstunkiem. Nie ma powodu, aby nie
być razem, kiedy wszyscy mamy dla siebie wystarczająco, o~ile się nie
podpierdalamy od innych.

Lodołasica uśmiechnęła się po raz pierwszy od dłuższego czasu. 

-- W~ten
sposób, to jest piękne.

Limpopo nie odwzajemniła uśmiechu. Wyglądała na zbyt wyczerpaną, by się
uśmiechnąć. 

-- Zadowoliłabym się prawdopodobieństwem. Kiedy już przez
jakiś czas jesteś osobą z~bronią, trudno sobie wyobrazić cokolwiek
innego i~zaczynasz używać głupich terminów, takich jak ,,ludzka
natura'', aby to opisać. Jeśli bycie samolubnym, nieufnym dupkiem jest
ludzką naturą, to jak nawiązujemy przyjaźnie? Skąd pochodzą rodziny?

-- Zakładasz, że w~rodzinach nie chodzi o~zachowywanie się jak samolubne,
nieufne dupki -- powiedziała Lodołasica.

-- Fakt, że Twoja rodzina jest tak popieprzona, nie jest dowodem na to,
że bycie popieprzonym jest czymś naturalnym, to dowód na to, że osoby z~bronią gniją od środka, a~ich życie zamienia się w~gówno. -- Zamknęła
oczy. -- Bez urazy.

-- Nie ma sprawy. -- Lodołasica z~zaskoczeniem odkryła, że to prawda.
Słowa były wyzwoleniem, ramą dla zrozumienia tego, co ją stworzyło, kim
mogła się stać.

-- Limpopo -- powiedziała Gretyl -- wyglądasz jak rzeźbione gówno. Bez
urazy.

-- W~porządku -- powiedziała Limpopo z~cieniem uśmiechu.

-- Czego trzeba, żeby cię uśpić?

Limpopo wzruszyła ramionami. 

-- Nie sądzę, żebym mogła zasnąć w~tym
momencie. Przeszłam przez sen i~jestem po drugiej stronie.

-- Myślę, że to macho-bzdura -- powiedziała Gretyl. Poruszała się,
poprosiła innych, żeby się przesunęli, przestawiała paczki i~torby, aż
na podłodze pojawiła się przestrzeń w~kształcie Limpopo. 

-- Połóż się. -- Poklepała dłoniami po kolanach.

Limpopo przeniosła wzrok z~niej na Lodołasicę, pozostałych, wzruszyli
ramionami. 

-- Wiesz, nie zamierzam zasnąć. -- Nie chodzi o~to, że nie
chcę, po prostu\ldots 

-- Cisza, głupia Połóż się.

Tak zrobiła, jej głowa opadła na kolana Gretyl. Spojrzała w~oczy
Lodołasicy, niewerbalne ,,Czy to w~porządku?'', a~Lodołasica uśmiechnęła
się i~pogładziła jej tłuste włosy, potargane, krótkie, różowo\dywiz niebieskie
wirowane cukierki. Byli w~wielu przytuleniach, ale to było inne. Ona i~Gretyl popatrzyły sobie w~oczy, wymieniły uśmiech. Jej strach się
rozpłynął. Cudem nie zastąpiło go zwątpienie. Deszcz, oddech, przyćmione
światło, przytulna bliskość sprawiały, że czuła się, wbrew wszelkim
przeciwnościom, bezpieczna.

Gretyl przechyliła głowę na jednym miękkim ramieniu, a~Lodołasica
kręciła się, dopóki nie mogła oprzeć na nim głowy. Gretyl objęła ją
ramieniem, a~ona objęła Gretyl i~trzy kobiety zamilkły.

\threeast

Spotkali się z~\textit{Lepszym Narodem} o~zachodzie słońca następnego
dnia. Limpopo obserwowała, jak załoga schodzi na linach i~w uprzężach,
stopy rysujące ziemię. Nastąpił krótki moment ekscytacji ze spotkania,
gdy opowiadali sobie nawzajem o~swoich przygodach. Etcetera też tam był,
racząc opowieściami o~ich bliskiej śmierci, ochy i~achy, podczas gdy
lotnicy opowiadali o~swoich własnych doświadczeniach z~dronami,
dipolami, pogodą i~nękaniem infowojennym.

\textit{Lepszy Naród} został przywiązany na głębokim terytorium Mohawków i~został hojnie zaopatrzony w~dziczyznę, kukurydzę, chapati i~lody w~niezwykłych smakach od wody różanej do marcepanu. Kilka dzieciaków
Mohawków też przyleciało, nie do końca odchodzący, ale na pewno nie w~default. Trzymali się razem i~patrzyli z~powagą, jak jedzenie
skwierczało na grillach ustawionych przez lotników, gdy lądowało więcej
członków załogi. Wtedy jedna z~nich -- dziewczyna z~długimi prostymi
włosami i~luźną koszulką z~napisem wielkimi literami LASAGNE -- wyszła z~ich ciasnej grupy. Podeszła do grilla, żeby pokibicować, a~Tam, która
gotowała, rzuciła żart, którego Lodołasica nie mogła usłyszeć, ale
sprawiła, że dziewczyna się uśmiechnęła tak promiennie, że zmieniła to
ją w~coś z~obrazu (albo może z~katalogu z~fotografiami stockowymi:
,,Uśmiechnięta tubylcza kobieta, odpowiednia na broszury dotyczące
polityki różnorodności'') i~obie grupy się połączyły.

Zespół medyczny nadzorował podnoszenie rannych do sterowca. Rozmawiali o~deadheadach, którzy przez tak długi czas byli ciekawostkami naukowymi i~towarowymi, że trudno było o~nich myśleć jako o~,,rannych'' (dla
najemników ,,ranny'' zdecydowanie nie było właściwym słowem). Lodołasica
zobaczyła, jak Tam kieruje się na obrady, gdzie dyskutowano o~tej
sprawie nad ogromnymi rożkami z~lodami, i~podeszła do nich.

-- To, co stanie się z~rannymi, jest o~wiele mniej ważne niż to, co
stanie się z~tymi dwoma najemnikami. Oni \textit{muszą} być bezpieczni.

Limpopo kołysała brodą z~boku na bok. Poszła popływać w~pobliskim
strumieniu i~wyglądała godnie pozazdroszczenia świeżo, wypoczęta i,
szczerze mówiąc, piękna. Tam też poszła się zanurzyć, a~jej włosy były
splecione w~parę grubych warkoczy Pippi Pończoszanki, które zwisały do
jej piersi, jak schematyczne strzałki wskazujące najistotniejszą cechę.

-- Rozumiem, że to, że tych dwoje jest z~nami, to zła reklama, ale są
wyższe priorytety\ldots 

Tam uciszyła ją ostrym machnięciem ręki. 

-- Nie rozumiesz, jest
\textit{całkowicie} odwrotne. \textit{Powodem}, że jest to zła reklama,
jest to, że to, co zrobiliśmy, jest potworne. Teraz jesteśmy ich, kurwa,
właścicielami, więc jesteśmy im to \textit{winni}. Kiedy bierzesz jeńca,
jesteś za niego odpowiedzialna. Nie tylko prawnie, ale i~moralnie.
Zaczęliśmy drogę, z~której nie możemy uciec. Gdyby to zależało ode mnie,
rozmrozilibyśmy ich i~uwolnili\ldots 

-- Nie sądzę, żebyśmy \textit{potrafili} to zrobić bezpiecznie. -- To była
Tekla, osoba medyczna, która służyła z~OC przy tym projekcie. -- Nie po
tym wszystkim, przez co przeszli. Zanim spróbujemy, potrzebujemy całego
laboratorium i~kontrolowanych eksperymentów, w~przeciwnym razie mogą
skończyć jako warzywa. Myślę, że będziemy w~stanie uruchomić ich jako
symy, zanim będziemy gotowi, zapytać ich, co ich zdaniem powinniśmy
zrobić z~ich ciałami. To wydaje się sprawiedliwe\ldots 

Tam pomachała wściekle, oburącz. 

-- \textit{Żartujesz}? Gdzie studiowałeś,
przed odejściem? Uniwersytet Mengele? Skanowanie tych dwojga bez ich
zgody było okropne, uruchamianie ich symulacji i~zmuszanie do podjęcia
decyzji, czy ryzykować życiem\ldots 

-- Nie ich życiami -- powiedziała Limpopo. -- Ich ciałami.

Usta Tama zamknęły się ostro i~widocznie się opanowała. 

-- Nigdy nie
zaakceptowali tego, że najważniejsza część jest w~tej symulacji. Nigdy
nie mieli takiego wyboru. Może uda nam się wprowadzić ich w~stan podobny
do tego, w~którym była Roz, żeby nie obchodziła ich różnica, ale bez ich
zgody to pranie mózgu. Niewybaczalne, potworne pranie mózgu.

Limpopo spojrzał na podwozie zeppa nad głową, piętrową gondolę, dno
pokryte hakami ładunkowymi, pakietami czujników oraz gejowskimi
ilustracjami androgynicznych ludzi z~kosmosu tańczących na tle
kosmicznych śmieci w~kieszeniach: saturniańskie pierścienie i~błyszczące
mgławice. Była na skraju załamania. Tak po prostu karnawałowy klimat
zniknął.

-- Zabierzmy ich na balonowca -- powiedziała Limpopo, ignorując zasadę,
żeby nigdy nie nazywać go balonowcem, tylko zeppelinem. Nikt jej nie
poprawił. Znowu wyglądała na zmęczoną. Odwróciła się i~odeszła.

\threeast

\textit{Lepszy Naród} obniżył się, żeby przekazać zapas hexajurt, które ze
wprawą postawili, łącząc niektóre razem dla komunalnych sypialni.
Zbliżały się kolejne deszcze i~aerialiści musieli odpłynąć, żeby je
wyprzedzić. Ich zaklinacze pogody przewidzieli dryf w~kierunku morza,
być może aż do Nowej Szkocji, i~poprosili o~zaopatrzenie, prezenty i~listy dla każdego, kogo spotkaliby po drodze. Przetrząsanie skromnego
dobytku w~poszukiwaniu prezentów przywróciło trochę radości, chwilę
zachwytu w~obfitości i~odnowioną ideę, że zawsze jest coś więcej, skąd
pochodzi, koniec niedostatku na horyzoncie.

Część ich ludzi odeszła z~lotnikami, towarzysząc deadheadom. Niektóre z~dzieciaków Mohawków, w~tym dziewczyna (która nazwała siebie Pocahontas i~prowokowała tym wszystkich) dołączyła do ekipy B\&B. Kiedy Lodołasica
nieśmiało zapytała ją, dlaczego została, wzruszyła ramionami: 

-- Chcę żyć
wiecznie. Czy nie dlatego wszyscy tu jesteśmy?

Seth, który to usłyszał, podniósł ręce do góry i~krzyknął \textit{amen!} i~się roześmiali.

Szli.

Lodołasica szła z~Etceterą i~Sethem. Spojrzała na nich i~przypomniała
sobie ten niemożliwy czas, kiedy spotkali się na komunistycznej imprezie
i pomyślała o~samoodtwarzającym się piwie i~biednym Billiamie, o~jej
ojcu -- minęły wieki, od kiedy ostatnio wysłał e-mail; nigdy nie
odpowiedziała -- podobnie jak o~jej siostrze, matce i~defaulcie,
wszystkim, co stało się w~tak krótkim czasie.

-- Niesamowite, prawda? -- Okręciła się, by to ogarnąć, czując się młoda i~piękna w~sposób, którego poczucie straciła. -- Kim, kurwa,
\textit{jesteśmy}, by decydować się odejść, aby stworzyć lepsze
społeczeństwo?

-- Wiem, kim \textit{ja} jestem -- powiedział Seth. -- Jesteś bogatą
dziewczyną, którą porwałem, z~nieuleczalnym syndromem sztokholmskim. Ten
dupek to Hubert Vernon Rudolph Clayton Irving Wilson Alva Anton Jeff
Harley Timothy Curtis Cleveland Cecil Ollie Edmund Eli Wiley Marvin
Ellis Espinoza.

-- Potrzebuje więcej bananaspodni -- powiedziała.

Etcetera się uśmiechnął. 

-- Możesz mówić do mnie bananaspodnie, jeśli
chcesz. -- Uściskał ją i~uścisk nie był dokładnie braterski, ale bardziej
niż nie, do tego stopnia, że tak nie było, była słodka nostalgia za tym
czasem, kiedy flirtowali jak szaleni, a~ona czuła to
przyciąganie-odpychanie, że nie jest właściwie zainteresowana, ale jest
trochę uspokojona, że był nią zainteresowany. Zabawne, jak bardzo było
to skomplikowane, kiedy grali tylko w~chodzenie, weekendowa bohema.
Kiedy przestała udawać, że jest normalna, było jej łatwiej.

-- Chłopaki -- powiedziała nagle poważnym tonem. To była poważna chwila. -- Chcę, żebyś wiedziała\ldots  -- Przeniosła wzrok z~Setha na Etceterę, z~powrotem. Zrobili się starsi w~odchodzeniu, ale to dało im powagę.
Chwila skupienia pozwoliła jej zobaczyć ich jako obcych, jak zawadiacko
byli przystojni. Uśmiechnęła się. Jej uczucie było roztopioną czekoladą.

-- Po prostu kocham was obu, ok? Jesteście dobrzy. Najlepsi.

Żaden nie wiedział, co powiedzieć. Seth próbował czegoś mądrego.
Etcetera próbował dowiedzieć się, co to oznaczało w~wielkim schemacie.
Niemal słyszała ich myśli. Zanim którakolwiek z~nich mogła powiedzieć
coś głupiego, przytuliła ich, rozkoszując się ich znajomymi zapachami.
Ich splątane ramiona znalazły drogę. Uścisk trwał i~trwał.

Kiedy wypłynęli z~doków, Gretyl i~Limpopo stały obok, uśmiechając się
jak dumni rodzice. Ona i~Gretyl zarezerwowały sobie na noc prywatną
heksajurtę i~odkąd dowiedziała się, że będą same, czuła się napalona
głęboko w~sobie. Teraz, z~chłopakami w~ramionach i~Gretyl patrzącą na
Limpopo -- tak cholernie hardkorową i~tak gorącą, że przez lata się w~niej podkochiwała -- jej napalenie wzrosło jeszcze bardziej. Roześmiała
się z~samej fizyczności. Chłopcy też się śmiali, choć kto wie dlaczego.
Nie była już w~ich głowach. To było w~porządku. Byli odchodnikami i,
boże pomóż im, zorientowali się, jak żyć tak, jakby to były pierwsze dni
lepszego narodu, a~tej nocy miała zamiar wypierdolić sobie mózg. Świat
był dobry.

\threeast

Seks był wszystkim, na co liczyła, a~potem jeszcze trochę. Była taka
chwila obok siebie, splątane nogi, wściekle pracujące ręce, oczy prosto
w oczy, kiedy doświadczyła dylatacji czasu, która w~innych
okolicznościach byłaby przerażająca, chwila, która dosłownie wydawała
się wiecznością, a~kiedy osiągnęła apogeum z~orgazmem, który sprawił, że
jej nogi wierzgały jak żaba pod baterią, była rozczarowana, widząc, jak
mija.

Potem rozmawiały na sposób kochanków. Sposób kochanków, to, co zaczęło
się jako pomruki o~wzajemnym pięknie i~waleczności, ze strategicznymi
pocałunkami i~wdychaniem nawzajem swoich zapachów, odejście od
wszystkiego nagle zmieniło się w~życie default wśród odchodzących.

-- To fajny pomysł, ale ostatecznie jest dziecinny -- powiedział Gretyl. -- Pomysł, że nie ma obiektywnej zasługi. Możesz w~to wierzyć, jeśli
zrobisz coś \textit{jakościowego}. Ale w~matematyce łatwo zobaczyć, kto ma
zasługi. Nie ma sensu udawać, że każdy głupek jest przyszłym Einsteinem.

-- Einstein oblał matematykę -- odpowiedziała szybko Lodołasica. Einstein
często pojawiał się w~takich dyskusjach.

-- To nie była matematyka, to była arytmetyka. Ludzie, którzy potrafią
liczyć w~głowie, nie robią matematyki, tylko obliczają. Nikt nigdy nie
liczy tak dobrze, jak najgłupsze komputery. To imprezowa sztuczka.
Arytmetyka to nie matematyka. Wiedza, z~której arytmetyki należy
skorzystać, to matematyka.

Lodołasica westchnęła. Ekipa naukowa traktowała ludzi z~Gospody z~protekcjonalnym rozbawieniem, gdy pojawił się taki temat, ale założyła,
że jej Gretyl była po jej stronie.

-- Nikt nie może sam uprawiać nauki, prawda? Spójrz na to, co zrobili Roz
i OC, to był taki wysiłek zespołowy, wszyscy musieli wnieść swój wkład,
a nawet z~tym wszystkim nie wiemy, czy ożywimy OC.

Gretyl przewróciła się na bok, jedna z~jej małych, sprytnych dłoni
przesunęła się po ciele Lodołasicy od podbródka do łona, opierając się
na jej udzie. Żaden kochanek nie dotknął jej w~taki sposób. Zadrżała.
Gretyl tak mocno ją trzymała. Przerażało ją to w~dobry, naładowany
seksem sposób. Kiedy Gretyl pracowała nad nią z~twarzą w~wyrazie
skrajnego skupienia, doznała absolutnego poddania się.

Teraz Lodołasica zdjęła jej dłoń z~nogi. Dyskusja była poważna, a~ona
chciała, żeby Gretyl wykorzystała cały mózg. Limpopo wyjaśniła to tak
jasno. Nie chciała jej zawieść.

-- Wszyscy byliśmy potrzebni do projektu uploadu. Na całym świecie są
inni, którzy również są niezbędni. Ale nie \textit{każdy} jest niezbędny.
Zobacz, co zrobiłaś dla Roz, podtrzymując ją na duchu i~rozpraszając ją.
Byłaś bardzo dobra, ale są inne, które są równie dobre. Gdyby cię tam
nie było, ktoś inny wykonałby pracę.

-- Teraz weź Roz. Była \textit{niezastąpiona}. Nie moglibyśmy uruchomić jej
bez \textit{niej}! Zajmujemy się różnymi dziedzinami, ale uważnie śledzę
jej pole. Prawdopodobnie nie ma nikogo na świecie, kto mógłby robić to,
co ona. Jest dosłownie jedyna w~swoim rodzaju. Nie jestem jedyną w~swoim
rodzaju, jestem dobra, ale w~końcu jestem matematykiem stosowanym z~pretensjami do czystej matematyki. Istnieją ludzie zajmujący się czystą
matematyką, którzy przez dziesięć lat zastanawiają się nad algebrą
niezbędną do udowodnienia topologicznej równoważności filiżanki do kawy
i pączka. Czarodzieje z~innego wymiaru. Wszyscy wasi ludzie walczą z~egoistycznymi bzdurami, korzeniami wszelkiego zła. Nie ma bardziej
samolubnego gówna niż pomysł, że jesteś cennym płatkiem śniegu,
niezastąpionym i~zasługującym na traktowanie jak rasowy koń, kiedy jest
jeszcze dziesięcioro takich jak ty, którzy również wykonają Twoją pracę.
Zwłaszcza gdy wspierasz jedyną osobę, której naprawdę \textit{nie można}
zastąpić.

-- Już to wszystko słyszałem. Mój tata używał tego, by wyjaśnić płacenie
swoim pracownikom tak mało, jak tylko mógł, jednocześnie zabierając
tyle, ile mógł. Mówiłam mu: może ma ,,niezbędną'' umiejętność
prowadzenia biznesu, ale sam nie mógłby go robić. Powodem, dla którego
wszyscy inni przychodzą mu pomóc, nie jest też jego magiczna
,,niezbędna'' umiejętność. To dlatego, że potrzebują pieniędzy, a~on je
ma.

Szczęka Gretyl zadziałała. 

-- Zakładasz, że skoro zetty mówią o~merytokracji i~są gówniani, zasługa musi być gówniana. To jak astrologia
i astronomia: astrologia mówi o~mechanice orbitalnej tak jak astronomia.
Ale astronomowie mówią o~mechanice orbitalnej, ponieważ systematycznie
obserwowali niebo, budowali falsyfikowalne hipotezy na podstawie
obserwacji i~kontynuowali prace z~tego miejsca. Astrologowie mówią o~mechanice orbitalnej, ponieważ brzmi to naukowo i~pomaga im oszukiwać
frajerów.

-- Mój tata zatem jest astrologiem?

-- To zniewaga dla astrologów. -- Roześmiały się. Część napięcia wypłynęła
z Lodołasicy. Połączyło ich gadanie obraźliwie o~jej ojcu. Gretyl
wykorzystała go jako awatara każdego zła w~systemie. Lodołasica była
zadowolona, że zgodziła się na to z~własnych powodów. 

-- Twój tata jest
jak nadęty książę, który wynajął nadwornych astrologów, by składali w~ofierze kurczaki i~zapewniali go, że to kocia dupa.

-- Mówisz o~ekonomistach -- powiedziała.

-- Oczywiście, że mówię o~ekonomistach! Myślę, że trzeba być
matematykiem, aby docenić, jak gówniani są ekonomiści, jak astrologiczne
są ich równania. Bez urazy dla twojej egalitarnej duszy, ale brakuje ci
treningu, aby zrozumieć, jak głęboko fałszywe są te zgrabne równania.

Lodołasica zesztywniała. Wiedziała, że Gretyl żartuje, ale nie lubiła,
gdy mówiono jej, że ,,nie ma kwalifikacji'' do dyskutowania, nawet
żartem. Stłumiła to uczucie, usiłowała uzyskać dostęp do tej części
siebie, która rozbłysła, gdy Limpopo opisywała te rzeczy.

Jeśli Gretyl zauważyła, nie dała żadnego znaku. 

-- Twój tata zatrudnia
ekonomistów do intelektualnej osłony, aby udowodnić, że jego dynastyczne
fortuny i~wpływy polityczne są wynikiem złożonego, samonaprawiającego
się mechanizmu z~mistyczną mocą wyrywania zasłużonych z~gęstej masy
ludzkości i~wywyższania ich, aby mogli nas mądrze prowadzić. Mają
naukowate słownictwo stworzone wyłącznie po to, by chwalić ludzi takich
jak twój ojciec. Jak \textit{pracodawca}. Jakbyśmy potrzebowali
\textit{pracy}! Mam na myśli, że jeśli jest jedna rzecz, której jestem
pewna, to tego, że nigdy więcej nie chcę być zatrudniona. Zajmuję się
matematyką, bo nie mogę przestać. Ponieważ znalazłam ludzi, którzy
potrzebują mojej matematyki, aby zrobić coś niesamowitego.

-- Jeśli musisz mi \textit{płacić} za matematykę, to dlatego, że a)
wymyśliłaś, jak mnie zagłodzić, jeśli nie wykonam roboty, i~b) chcesz,
żebym robiła nudną, głupią matematykę bez żadnego zainteresowania.
,,Pracodawca'' to ktoś, kto wymyśla, jak zagrozić ci głodem, chyba że
zrobisz coś, czego nie chcesz robić.

-- Kiedyś obserwowałam was, dzieciaki, na waszych komunistycznych
przyjęciach, kiedy byłam w~default i~udawałam, że to wszystko ma
znaczenie. Byłam tak \textit{wściekła} na Ciebie, poza jakąkolwiek
rozsądną odpowiedzią. Dopiero gdy odeszłam, zrozumiałam dlaczego:
ponieważ za każdym razem, gdy włamywałaś się do pustej fabryki i~włączałaś maszyny, udowadniałaś, że jestem koniem do orki, którego
biedne wargi zostały pokaleczone przez wędzidło w~moim pysku, gdy
ciągnęłam wózek dla pana z~batem i~workiem z~paszą.

-- To mój punkt widzenia na temat różnicy między tym rodzajem
merytokracji, jaką mamy na uniwersytecie, a~bzdurami, w~których pływają
zetty. Kiedy mówimy, że Amanda jest lepszym matematykiem niż Gretyl,
mamy na myśli, że są rzeczy, które Amanda może zrobić, których Gretyl
zrobić nie może. Obie to miłe osoby, ale jeśli jest naprawdę ważny
problem matematyczny, lepiej ci będzie z~Amandą niż z~Gretyl.

Głos Limpopo odbił się rykoszetem w~umyśle Lodołasicy. 

-- Jednak Amanda
nie może zrobić tego wszystkiego. Jeśli nie pracuje nad problemem dla
jednej kobiety, to będzie musiała współpracować z~innymi. Jeśli jest do
niczego, wykonanie tego może zająć sto razy więcej pracy, niż gdyby
Gretyl, która jest dobra w~dzieleniu się zabawkami i~sprawianiu, że
wszyscy mruczą, była szefem. To nie jest anegdota, jak mi ciągle
powtarzasz, liczba mnoga od anegdoty nie jest faktem. Limpopo rozesłała
wszystkim tę metaanalizę z~\textit{Walkaway Journal of Organizational
Studies}, w~którym porównano produktywność programistów. Artykuł
podzielił pracę, którą programiści wykonywali na indywidualną i~na
wewnątrz grup. Okazało się, że chociaż byli programiści, którzy
potrafili stworzyć kod, który był sto razy lepszy niż mediana,
przykładowo tylko jeden procent błędów, sto razy większa wydajność
pamięci, to ten rodzaj szalonej wirtuozerii był bardzo słabo skorelowany
z osiągnięciami w~grupach.

Gretyl usiadła, na chwilę odwracając uwagę Lodołasicy swoim ciałem. 

-- Wyjaśnij?

-- Po prostu przeczytałem podsumowanie i~przejrzałem statystyki oraz
metodologię łączenia różnych zestawów danych. ,,TLDR'' jest taki, że z~tymi hakerami-czarodziejami, którzy tworzą lepszy kod niż ktokolwiek,
często pracowało się tak ciężko, że wszyscy inni pracowali
\textit{gorzej}, więcej błędów i~wolniej. Ilość czasu, jaką musieli
poświęcić na naprawę tego kodu, spowolniała ich tak bardzo, że
pochłaniała zyski z~wirtuozerii.

-- Próby połączenia zespołów ,,Projektu Manhattan'' stworzonego z~czarodziejów bez normalnych głupków, takich jak ja, przyniosły dokładnie
taki sam efekt.

-- Odnieśli się do jednego badania, które \textit{przeczytałem },
etnograficzna ankieta projektów, które poszły na dno, mimo że mieli
świetnych programistów. Autorzy stwierdzili, że istnieją dwie główne
przyczyny niepowodzenia. Po pierwsze, niektórzy czarodzieje są
kolosalnymi dupkami. Dosłownie używali słowa ,,dupek'', ponieważ fraza
pojawiła się u trzech różnych zespołów. Nie da się pracować z~dupkami,
nawet błyskotliwymi dupkami. ,,Nie pracuj z~dupkami'' to świetna porada,
ale także chwila typu ,,stop'', bo jeśli nie zorientowałaś się w~drugim
lub trzecim zespole, to może \textit{Ty} jesteś dupkiem.

-- \textit{Inna} kategoria dotyczyła czarodziejów, którzy nie mieli
\textit{pojęcia}, jak pracować z~innymi. Nie byli chujkami, po prostu nie
mieli instynktu. Autorzy znaleźli zespoły, w~których czarodziej i~wszyscy inni \textit{nie} zawiedli, w~tych przypadkach było tak, ponieważ
był ktoś radzący sobie z~ludźmi, który odkrył, jak załagodzić różnice i~pokazać, jak ludzie do siebie pasują. Ci ludzie byli czarodziejami
zespołów i~mieli większy wpływ na sukces zespołu niż czarodziej
programista.

Gretyl pokręciła głową. 

-- Nadal mówimy o~czarodziejach. Jeśli dowód jest
taki, że najważniejszym rodzajem czarodzieja jest czarodziej motywujący
ludzi, to jest to czarodziej, którego zrekrutujesz. To nie obala
istnienia czarodziejów i~na pewno, cholera, nie oznacza to, że nie są
ważni.

-- Mieszasz to, Gret. Chodzi mi o~to -- myśląc o~\textit{tezach Limpopo} -- że nawet jeśli masz bandę czarodziejów, zrobisz zero bez pomocy innych
ludzi. Każdy jest w~czymś czarodziejem. Dobra, nie do końca. Ale w~każdej grupie znajdą się ludzie, którzy robią rzeczy lepiej niż
ktokolwiek inny w~tej grupie. Niektóre z~nich przydadzą się grupie, a~inne nie. Jednak byłoby samotnie i~gównianie, gdyby tylko czarodzieje
brali udział w~życiu. Czarodziejom trudno byłoby nie przesądzić,
arbitralnie, że ich kumple i~ludzie, których chcieli pieprzyć, byli
również czarodziejami, trudno im byłoby nie wykorzystać swojej rzekomej
niezbędności do kierowania nie-czarodziejami.

-- Fakt, że ludzie często są chujkami, nie oznacza, że niektórzy ludzie
nie są w~tym lepsi od innych. Nie oznacza to, że ci ludzie nie są
ważniejsi w~wykonywaniu pewnych rzeczy niż reszta z~nas. To nie znaczy,
że są ważniejsi, tylko ważniejsi w~jakimś kontekście. To cholernie
głupie, urojeniem jest upieranie się, że wszyscy jesteśmy równi.

Lodołasica stłumiła emocje, zanim zaczęła krzyczeć. 

-- Gretyl. -- Jej głos
był miarowy. -- Nikt nie mówi, że jesteśmy równoważni, ale jeśli uważasz,
że nie jesteśmy \textit{równie} wartościowi, to co Ty tu, kurwa, robisz?

-- Och, uspokój się. Oczywiście każdy zasługuje na szacunek i~inne
rzeczy, ale w~każdym normalnym rozkładzie jest coś na jednym końcu
krzywej i~coś na drugim końcu. Jeśli prowadziłaś wydział matematyki,
udając, że wszyscy są w~matematyce tak samo dobrzy jak wszyscy\ldots 

-- \textit{Nie to mówiłam!}

Policzki Lodołasicy były gorące. Łzy
napłynęły jej do oczu. Ile kłótni odbyła ze swoim dupkiem ojcem, w~których pojawiały się zwroty typu ,,och, uspokój się''? Ile razy
odrzucał wszystko, mówiąc o~,,naturalnych'' zdolnościach swoich kumpli
zetta, jakby byli nadludźmi? Zamierzała powiedzieć coś, czego będzie
żałować. Wstała, zaciskając szczęki tak mocno, że słyszała zgrzyt zębów.
Ubrała się, oczy skupione na czerwonanym, czarnym kole. Gretyl coś
powiedziała. Wyczuła, że Gretyl wstaje, więc wybiegła z~jurty, ubrana w~bieliznę, koszulę i~spodnie, stopy bez skarpetek wciśnięte w~buty
trekkingowe. Deszcz przestał padać, a~rześki wiatr zdmuchnął większość
chmur z~nieba, pozostawiając blask gwiazd i~ostry sierp księżyca,
podkreślony przez czarne, dramatyczne chmury na skraju lasu. Zapach wody
z rowów i~sosnowego lasu był silny. Jej stopy pluskały w~czarnych
kałużach, zimna woda przelewała się wokół jej palców.

Wydawało jej się, że słyszy za sobą Gretyl, pluskającą się w~tych samych
kałużach. Część niej chciała, żeby Gretyl dogoniła, żeby mogła
przeprosić i~cofnąć te okropne rzeczy, które powiedziała. Inna część
zrozumiała, że jej uczucia do Gretyl były zrównane z~jej uczuciami do
ojca. Żadne przeprosiny Gretyl nigdy nie zastąpią przeprosin, których
nigdy nie dostała od niego.

Ruszyła na skraj obozu, chcąc być daleko od ludzi i~chcąc znaleźć
miejsce, w~którym mogłaby utrzymać równowagę na jednej nodze, żeby
skończyć się ubierać. Zasznurowała buty i~wstała, ściągnęła koszulę z~gałęzi i~włożyła ją, walcząc przez nakładające się warstwy
przesiąkliwych materiałów i~materiałów izolacyjnych oraz jaskrawo
kolorowe kawałki utkane wokół niej w~paski, wygląd, który wymyśliła,
których tak wiele osób doceniało i~kopiowało. W~ubraniu i~butach się
uspokoiła. Przejechała rękami po koszuli, która wyglądała cholernie
niesamowicie i~została uznana za techniczny i~estetyczny triumf, z~czego
była niezwykle dumna.

Wytarła policzki dłońmi, położyła ręce na biodrach i~spojrzała w~niebo.
Spędziła wiele letnich nocy w~rodzinnym domku, wpatrując się w~takie
niebo, nieprawdopodobnie zaludnione zimnymi gwiazdami, które
przypominały jej o~ogólnej nieistotności ludzkości i~pocieszały ją
szczególną nieistotnością jej ojca. Czasami w~te noce obserwowania
gwiazd towarzyszyli jej kuzyni, kilku, do których czuła sympatię, i~dręczyła ją myśl o~nich, zagubionych dla niej, wykręconych w~sękaty
kształt zetta pogrążonych w~urojeniowych bzdurach.

Coś przykuło jej uwagę, zbliżając się do linii drzew. To był
\textit{Lepszy Naród}. Gdy go zobaczyła, usłyszała to, co było
niewłaściwe, ponieważ sterowce nie miały uruchamiać swoich śmigieł, z~wyjątkiem sytuacji awaryjnych. Był to transport, który szedł tam, gdzie
wiatr wiał, robiąc siano, gdy świeciło słońce, traktując przyrodę jako
właściwość, a~nie robaka. Wirniki stawały się coraz głośniejsze,
brzęczenie owadów, potem rój, potem ryk pełnego gardła. \textit{Lepszy
Naród} miał być w~drodze do Nowej Szkocji.

Z lasu wiał wietrzyk, który przyprawił ją o~gęsią skórkę. Włosy na jej
szyi się zjeżyły. Była jak wrośnięta w~ziemię, wpatrując się w~zeppa
prącego po niebie, wijącego się tam i~z powrotem, walczącego z~wiatrem.
Zbliżał się coraz szybciej i~po jakimś czasie zdała sobie sprawę, że
zarówno opada, jak i~przyśpiesza. Puls walił jej w~uszach.

-- POMOCY! -- krzyknęła, nie zdając sobie sprawy. -- POMÓŻCIE IM! -- Uderzyła się w~powierzchnie interfejsu, włączając kamery i~czujniki bez
świadomej decyzji.

Dźwięk jej krzyków w~uszach odmroził jej bezruch. Pobiegła przez obóz w~kierunku zeppa. Ujrzała za nim czarne kształty poruszające się po
niebie, niewidoczne z~wyjątkiem miejsc, w~których przesłoniły światło
gwiazd. Niesłyszalny przez skowyt wirników, przenikliwy dźwięk maszyn
pracujących poza swoim marginesem bezpieczeństwa.

Inni byli w~pobliżu, celując w~niebo. Były krzyki. Przeklinanie. Wachlarz
cienkich jak ołówek, laserów fioletowych oświetlił niebo tak jasno, że
aż bolało. Zbiegły się na czarnym kształcie, śledziły go, podskakując i~szarpiąc po niebie. Podążyła wzrokiem do ich źródła i~zobaczyła trzech
członków załogi uniwersyteckiej gorączkowo podłączających ogniwa
wodorowe do platformy niczym miniaturowe działko przeciwlotnicze na
szerokiej metalowej płycie, którą przyciskali butami. Podbiegła do nich
i stanęła na talerzu, zwalniając dwóch z~nich, żeby przesunęli się
bliżej do baterii.

Dron na drodze laserów spadł. Lasery podążały za nim, aż spadł poniżej
linii drzew. Tam, gdzie na chwilę dotknęły drzew, lasery wznieciły
skwierczące ogniska, które dymiły, ale szybko gasły, gdy pobliskie
drżące gałęzie rozsypywały na nich ładunki kropli deszczu.

Lasery wycelowały ponownie, przebijając kolejnego drona. Gdy trzeci
otworzył ogień małymi pociskami, które przelatywały przez niebo na
stożkach ognia, lasery podzieliły się na dwa i~wycelowały w~oba drony. W~tej samej chwili dwa pociski trafiły \textit{Lepszy Naród}, jeden trafił w~zbiornik z~gazem. Drugi uderzył w~gondolę. Ten ześlizgnął się po
powierzchni i~wirował z~nieba jak skrzydlak, eksplodując pod nim, fala
uderzeniowa posłała w~górę ogon gondoli. Całość zadrżała, jakby została
złapana w~zęby ogromnego psa. Pocisk na zbiorniku z~gazem wybuchł.
Rozległ się dźwięk przypominający pękanie tysiąca balonów, gdy worek z~komórkami pękał kaskadowo, przesuwając się tam i~z powrotem po całej
długości zbiornika. Zepp spadł, ale nie spadł swobodnie -- niektóre
komórki, niewiarygodne, były nienaruszone, hołd dla ich zabezpiecznień -- ale spadł szybko.

Kolejny dron zapalił się i~stał się meteorem. Lasery przeskoczyły na
drugi, ale ten uniósł się, gdy trafiony \textit{Lepszy Naród} wpadł do
obozu, wybijając bruzdę w~dachach i~ścianach pięciu heksajurtów, zanim
jego nos zetknął się z~drogą i~zgniótł się, dymiące resztki zbiorników z~gazem osiadły nad tym chwilę później. Dźwięk -- rozdzierający hałas
kończące się na \textit{trzasku} wibrującym w~zębach -- zmieszał się z~okrzykami przerażenia i~strachu. Odchodnicy zaroili się na gondoli,
używając rąk do odginania rozbitego kadłuba z~włókna szklanego, aby
dostać się do ludzi w~środku.

Etcetera przebiegł obok z~łomem, ale jej nie zauważył. Był skupiony na
\textit{Lepszym Narodzie}, załodze zeppów, z~którą się zaprzyjaźnił.
Dzieciaki Mohawków były tuż za nim, z~własnymi narzędziami, młotami i~drągami. Przypomniała sobie, że niektórzy z~ich przyjaciół polecieli na
pokładzie statku powietrznego. Wbiła pięści w~brzuch, gdzieś pomiędzy
uderzeniem się w~brzuch a~masowaniem go, próbując wypędzić żal.

Jedną z~heksajurt, która została powalona na samym początku, była ta,
którą dzieliła z~Gretyl. Zepp otarł się tylko o~dach, ale lekkie,
kompozytowe panele wygięły się, a~potem pękły, zostawiając ściany
przechylające się jak starożytne nagrobki. Poruszając się jak we śnie,
Lodołasica podeszła do jurty, ugniatając brzuch. Coraz więcej ludzi
przemknęło obok niej i~rozległ się chór wybuchów, pozostałe ogniwa
zbiornika gazowego się przegrzały. Poczuła ciepło ognia na karku i~poczuła zapach palących się włosów.

Przed pójściem spać, ona i~Gretyl skorzystały z~przestronnej prywatności
heksajurty, by rozpakować swój zmieszany sprzęt, wyciskając wodę z~mokrego materiału i~ostrożnie ją składając, zwijając linę i~wymieniając
ogniwa swoich urządzeń. Wciąż wszystko było ułożone w~precyzyjnej,
kartezjańskiej siatce, którą stworzyła Gretyl, ledwie potargane przez
fragmenty dachu, które w~nią wpadły. Obok znajdował się dmuchany
materac, biliony mezoskalowych bąbelków, które wypełniały się, gdy
rozkładało się i~energicznie potrząsało jako łóżko, ale łatwo się
spuszczały, gdy rolowało się je z~jednego rogu.

Na łóżku: Gretyl, na boku, ubrana, by gonić Lodołasicę w~noc, jakby
spała. Pomiędzy nią a~Lodołasicą powietrze falowało obłokiem pary, a~ona
pochyliła się nad Gretyl, oświetlona promieniem latarki, który komputer
automatycznie wysyłał łukiem z~jej klatki piersiowej, bez zauważania
Lodołasicy. Sięgnęła ręką po ramię Gretyl, dotykając go, a~następnie
chwytając i~szarpiąc, próbując przewrócić Gretyl na plecy. Była
bezwładna.

Na łóżku pod jej głową była krew.

Lodołasica próbowała zrobić trzy głębokie oddechy. Doszła do jednego.
Skupiła się. Pochyliła się do ust Gretyl, usłyszała jej oddech, położyła
rękę na szyi i~sprawdziła puls, natrafiając na krew, ale nie dbając o~to. Puls był silny. Dotykała lekko Gretyl, badając palcami, zaczynając
od stóp i~poruszając się w~górę, sprawdzając każde ramię, potem znowu
gardło, podbródek.

W końcu obejrzała głowę Gretyl, badając ją uważnie, nie zważając na
chaos i~huki. Z~tyłu głowy miała płytkie rozcięcie, obficie krwawiące,
ale małe. Nie było wgniecenia ani papkowatej depresji, jak ta, którą na
wpół widziała, nigdy nie widziała, na głowie Billiama. Usłyszała własny
oddech, spowolniła go, odsunęła powieki, patrząc na kurczące się
źrenice, czy były tej samej wielkości? Gretyl zamrugała, odsuwając
dłonie od powiek, pozostawiając smugi krwi z~koniuszków palców.

Gretyl zamrugała kilka razy, słabo poruszyła rękami i~nogami, próbowała
usiąść. Lodołasica ją przytrzymała. 

-- Jesteś ranna. -- Mówiła jej do
ucha, starając się być spokojna i~pocieszająca.

-- Bez jaj. Kurwa. -- Zamrugała. Z~katastrofy rozległy się krzyki, a~niektóre zabrzmiały bliżej. Lodołasica spojrzała w~noc, ciemną i~poplamioną nieregularnym pomarańczowym światłem płomieni. Kiedy była
rozproszona, Gretyl usiadła, strząsając rękę. Dotknęła rany na czaszce i~spojrzała na krew na dłoni z~urażonym grymasem. 

-- Jebać to -- powiedziała. Lodołasica złożyła jej zakrwawioną dłoń we własnych
zakrwawionych dłoniach.

-- Kochanie, jesteś ranna w~głowę. Powinieneś się położyć, na wypadek
urazu kręgosłupa lub wstrząsu mózgu.

Gretyl wpatrywała się w~nią, jakby nie słyszała, a~potem: 

-- W~porządku w~teorii, ale nie sądzę, abyśmy dziś mogły wybierać. Odkręćmy to. Pomóż
mi. -- Odwróciła się do Lodołasicy, wpatrując się z~intensywnością, która
nie pozwalała na debatę, przesuwając uścisk, by pociągnąć za rękę.
Lodołasica zmagała się ze sobą, po czym pociągnęła. Gretyl zachwiała
się, przyłożyła wolną rękę do tyłu głowy, wyprostowała się.

-- Co się do cholery dzieje? -- powiedziała, rzucając się w~kierunku
ognia.

Byli już prawie przy nim, gdy ktoś złapał Lodołasicę za ramię i~szarpnął. Odwróciła się z~rękami zaciśniętymi w~pięści, szeroko
otwartymi oczami i~bijącym sercem, pewna, że zaraz zostanie
obezwładniona przez najemnika wysłanego przez zettów, by ich
sterroryzować. To był Etcetera z~pomazaną twarzą, oczy w~panice. 

-- Dalej! -- Znowu szarpnął, nieświadomy faktu, że zamierzała złamać mu nos.

Nawet z~raną głowy Gretyl była szybsza w~rozumieniu. Szarpnęła drugie
ramię Lodołasicy i~podążyli za Etceterą do kolejnej jurty, gdzie ranni
leżeli na dmuchanych materacach, oświetlonych przez diody OLED wielkości
grochu pospiesznie przyklejone do ścian i~rzucające nakładające się na
siebie szalone cienie. To był chaos, zaimprowizowana kostnica, ale
widziała, że niektórzy się poruszają, a~niektórym towarzyszyli kucający
nad nimi ludzie. Pocahontas, z~zabandażowaną jedną ręką, uspokajała
leżącą na ziemi postać, z~ręką na czole, a~drugą trzymając ekran i~koncentrując się na odczycie. Lodołasica przypuszczała, że była
połączona zdalnie z~grupą lekarzy, którzy pomagali w~opiece przez sieci
odchodzących, i~zastanawiała się, z~iloma takich potyczek oni mieli
ostatnio do czynienia w~środku nocy. Zastanawiała się, kto pełza po
sieci odchodzących, przeprowadzając analizę ruchu, aby znaleźć i~uderzyć
w tych lekarzy.

Wkrótce została zdalnie połączona z~lekarzem, pracując nad na szczęście
nieprzytomnym zeppowym jeźdźcem, strasznie poparzoną i~jęczącą za każdym
razem, gdy jej dotykała, wykonując polecenia lekarza, czasami prosząc go
o przesłanie ich jako wiadomości, ponieważ nie zawsze mogła zrozumieć go
z powodu bardzo ciężkiego brazylijskiego akcentu. Zastanawiała się, w~jakiej strefie czasowej znajduje się Brazylia i~czy jest tam środek
nocy. Wkrótce inny brazylijski lekarz dołączył do linii i~pomógł jej
nastawić złamaną nogę za pomocą nadmuchiwanego gipsu z~paczki, którą,
ironicznie, zepp zostawił poprzedniego dnia.

Podniosła wzrok znad swojego pacjenta, który był zdecydowanie wdzięczny za
środek przeciwbólowy, który podała mu pod język. Gretyl siedziała na
jednej z~nielicznych pustych poduszek z~twarzą w~dłoniach. Podeszła do
niej, objęła ją ramieniem, niepewnie pocałowała ją w~ucho, posmakowała
zaschniętej krwi i~poczuła zapach stęchłego potu i~olejku skóry głowy.
Gęste włosy Gretyl były plamą krwi.

-- Wszystko w~porządku?

-- Tylko zmęczona. Położyłam okład z~lodem na głowie, a~ktoś w~Lagos
sprawdził, czy nie mam wstrząśnienia mózgu i~stwierdził, że jestem
zakrwawiona, ale nie ugięta. Ale, cholera, dziewczyno, czuję, że zaraz
się załamię.

-- Prawdopodobnie dlatego, że zaraz się załamiesz.

-- Myślisz?

-- Połóż się. Już prawie skończyliśmy. Nawet Etcetera robi sobie przerwy.
-- Był w~manii, rozdarty między ratowaniem kolejnych aeronautów z~płonącego wraku a~opieką nad tymi, których wydostali. Wyciągnął dwóch
zmarłych z~\textit{Lepszego Narodu} i~płakał, gdy je niósł, a~potem zajął
się trzema innymi, którzy zginęli na matach w~szpitalu, pomagając
przenieść ich do innej jurty, która teraz była kostnicą. Limpopo
śledziła go trochę, podobnie jak Seth, pomagając mu w~obejściu,
delikatnie go uspokajając, zanim zrobi sobie krzywdę.

-- Jasne. 

-- A Ty? -- Jej głos był niski, oszołomiony.

-- Co ze mną?

-- Ty też potrzebujesz przerwy. Wyglądasz jak chodząca śmierć. Ja mam
wymówkę, jestem staruszką z~raną na głowie. Jesteś pełna żywych soków
młodości. Kiedy zaczynasz wyglądać jak wyrzutek z~filmu o~zombie, to
znak, że musisz się uspokoić. Nie możesz nikomu pomóc, jeśli nie zadbasz
o siebie. -- Przerwała. -- Wiem, że uosabiam przeciwieństwo tej rady, ale
mam wymówkę: jestem idiotką. Jaka jest Twoja?

-- Masz rację. Idę się wysikać, pospacerować i~wrócę. Zostaw mi miejsce
na macie, staruszko?

-- Wykorzystam cię jako poduszkę, jeśli nie będziesz ostrożna.

-- Załatwione.

Gretyl uniosła twarz i~pocałowały się, Gretyl trzymała usta zamknięte,
co robiła, kiedy była zawstydzona swoim oddechem, co w~tych
okolicznościach było mrocznie zabawne. Jak zawsze, Lodołasica całowała
się, aż jej usta się rozchyliły, i~przez chwilę zmieszały oddech i~ślinę, która rozciągnęła się jak cukierek, zanim przerwała i~z trudem
wstała, kładąc jedną rękę na panelu ściennym, który się wygiął, a~następnie odskoczył do tyłu, gdy stanęła prosto.

Zerknęła na Gretyl, zanim wyszła przez drzwi, i~ona leżała po swojej
stronie, nieruchoma. Lodołasica mrużyła oczy, aż dostrzegła unoszenie
się i~opadanie klatki piersiowej, po czym wkroczyła w~noc.

Zbliżał się świt, szaro-różowy na wschodniej linii drzew, czarny na
zachodniej. Limpopo i~Etcetera siedzieli na składanych krzesłach na
poboczu drogi, Etcetera trzymał ją i~płakał w~jej szyję. Limpopo
spojrzała na nią; uniosły brwi do siebie w~jednoczesnym \textit{czy
wszystko w~porządku? }, co sprawiło, że oboje uśmiechnęły się ze
znużeniem. Lodołasica rzuciła jej znak ,,w porządku'' i~posłała buziaka.
Limpopo odwzajemniła pocałunek i~Lodołasica odwróciła się do ciemnego
lasu, wygrzebując papierową gazę, którą włożyła do kieszeni, gdy
wychodziła z~namiotu, żeby użyć jej jako papieru toaletowego.
Przedzierała się przez zarośla, gasząc światło w~klatce piersiowej, gdy
się pojawiało, pozwalając swoim oczom przyzwyczaić się, gdy szukała
koniecznej kłody z~drzewem w~pobliżu, której można by użyć jako uchwytu.

Zajęła pozycję i~robiła swoje, nasłuchując odgłosów z~obozu, trzasku
małych rzeczy w~zaroślach. Powinna była przynieść łopatę, ale w~tych
okolicznościach nikt nie obwiniałby jej o~błąd w~leśnictwie. Spakowałaby
przynajmniej papier i~wrzuciła go razem z~odpadami medycznymi do stosu
spalarni.

W zaroślach rozległ się głośniejszy dźwięk, niebiegnący i~cichy. Duży i~tajemniczy. Podciągnęła spodnie, zapięła je i~spojrzała w~noc. Upuściła
papier, poklepała się po kieszeniach, w~których przez całą noc
zgromadziło się mnóstwo małych urządzeń i~przedmiotów. Nic użytecznego.
Jednorazowe opakowania. Spojrzała w~ciemność, robiąc krok w~kierunku
obozu, próbując znaleźć gałąź na pałkę. Złapała jedną, przesiąkniętą
wodą i~zgniłą. Uważnie nasłuchiwała kroków. Nikt z~obozu nie przemykałby
się przez las. Miała wizje najemników, noszących inteligentne rzeczy,
które były bardziej niż ciemne, zaginających światło, by stać się
niewidzialne.

Zrobiła kolejny krok. Ktoś gwałtownie szarpnął za kij, a~ona odruchowo
zacieśniła uścisk, więc poleciała za nim, tracąc równowagę. Upadła z~,,uff'', dezorientacja związana z~zapadaniem się w~mokre rzeczy i~zderzeniem z~ostrymi kamieniami. W~chwili między staniem a~upadkiem,
część jej mózgu, z~którą rzadko rozmawiała, przejęła kontrolę.
Przetoczyła się wraz z~upadkiem, przyjmując większość swojego ciężaru na
ramię, wykorzystując rozpęd, by przygotować się do ruchu, gdy podniosła
się na kolana, a~potem do pozycji biegacza. Pobiegła, bo ktoś był
\textit{tuż za nią} i~tam był obóz, a~jeśli udałoby\ldots 

Nie udałoby się. Ktoś był przed nią, mały, ale żylasty, bez wysiłku
chwytając ją za ręce, gdy biegła obok, uścisk nieruchomy jak stalowy
zacisk, nie bolesny, ale idealnie nieustępliwy. Prawie zderzyła się z~niewidoczną osobą, ale ta zgrabnie ją ominęła, jak torreador z~kreskówek
unikający byka, zakołysała nią w~parodii tańca, podnosząc ją z~rękami
przyciśniętymi do siebie. Skupiła się na osobie, która trzymała ją za
nadgarstki, kobiecie, pomyślała, drobnej i~krótkowłosej, o~rysach
pomalowanych w~olśniewającą szarość i~zieleń. Miała małe białe zęby
widoczne przez rozchylone usta i~oczy ukryte za matowym wizjerem
zaczepionym za uszami.

Drugi był tuż za nią, poruszał się szybko i~prawie bezgłośnie,
oddychając swobodnie. Odprężyła się, czując tylko odrobinę luzu w~uścisku trzymającej ją kobiety. Czy to było to? Tak. Z~przerażającą siłą
udała uderzenie głową w~przyłbicę, a~potem szarpnęła ramionami tak
mocno, że poczuła, jak skóra odrywa się od jej nadgarstków, coś w~ramieniu albo może pękają jej żebra. Nie miało to znaczenia. Otworzyła
usta, by krzyczeć, gdy biegła \ldots 

Potem znów znalazła się w~uścisku kobiety, silna dłoń na jej ustach.
Mała kobieta się uśmiechnęła, uśmiech \textit{jesteś odważna, dzieciaku},
albo w~to właśnie postanowiła uwierzyć Lodołasica. Następnie osoba z~dużą, męską dłonią na ustach -- pachnącą olejem maszynowym i~czymś innym,
co łaskotało jej pamięć -- zacisnęła na jej bicepsie coś, co natychmiast
zacisnęło się jak mankiet do pomiaru ciśnienia krwi. Poczuła maleńką
lancę bólu, gdy automatyczna strzykawka znalazła miejsce. Jej panikę
uprzedziło inne uczucie, majaczące uczucie jak syrop w~jej kręgosłupie i~niżej na dole, rozkoszne jak dodatkowy sen po pobudce, skradziony na
drzemce. Uczucie narastało. Uśmiechnęła się, zamykając oczy.

\part{znowu dom, znowu dom, giga gig}

\chapter*{i}

Default pachniał. To nie był zapach technologiczny. Jeśli chodziło o~jedną rzecz, którą mieli odchodzący, to były to zapachy technologiczne.
Ten był zapachem \textit{nieludzkim}. W~tle działały procesy szukające
zapachu ciała i~nieświeżego oddechu, niszcząc je wolnymi jonami i~gustownymi antyperfumami. Pachniało jak coś dopiero co rozpakowane.

Kiedy się obudziła, zapach był jej pierwszą wskazówką, zanim otworzyła
oczy. Zauważyła go, zanim się w~pełni obudziła, doświadczając
wspaniałego stanu świadomości, że nie jest obudzona, uczucia narkotyku.
Miała kiedyś to uczucie z~powodu czegoś dobrego, co miał Billiam. Nie
Billiam. Limpopo. Nie, Limpopo nie próbowała nowej farmacji, tylko
rzeczy, które znała. Seth był zafascynowany nową farmacją, pobierając
nowe rzeczy i~pilotując je w~pełnym sprzęcie z~czujnikami, aby grupy
analityczne mogły się nad nimi pochylić, a~następnie pojawiał się z~koszykiem świeżych jabłek i~zestawem vaperów, które dozował w~dawkach
dostosowanych do wagi.

Co dał jej Seth? Co to za zapach? Och, default. Czy Seth pobrał narkotyk
z defaultu? Co za cholernie okropny pomysł. Dlaczego miałby to zrobić?

Rosnąca świadomość, \textit{świadoma} świadomość. Rozpacz. Albo miał ją
ojciec, albo ktoś, kto planował dostać za nią okup. Gdyby to było
pierwsze, prawdopodobnie nigdy by nie uciekła. Gdyby to było to drugie,
trafiłaby do swojego ojca i~(patrz wyżej). Bo jeśli było coś, co
Lodołasica -- jebać to, \textit{Natalie} -- wiedziała od dawna, to było to,
że córka Jacoba Redwatera była warta więcej żywa i~nienaruszona niż
martwa lub uszkodzona. Gdyby jej ojciec w~końcu przyszedł po nią -- albo
gdyby miał to zrobić -- nie pozwoliłby jej znowu odejść.

Przez cały okres jej odchodzenia wiedziała, że ten dzień nadejdzie, gdy
Jacob Redwater poruszy małym palcem i~sprowadzi ją z~powrotem, zanim
zdąży być wykorzystana przeciwko niemu, albo, co gorsza, zawstydzi go.
Nigdy nie otwierała jego wiadomości, bojkotowała wiadomości od siostry i~kuzynów, ponieważ jej opsec był dobry, tak jak najlepsze umysły z~odchodzących potrafiły zapewnić, a~była pewna, że gdyby nadszedł
zeroday, który pozwoliłby na podłożenie pluskwy, na którą on mógłby
sobie pozwolić, zanim błąd zostanie znaleziony i~załatany, to nie
zawahałby się tego użyć. Nawet nie rozumiałby, dlaczego ktoś
\textit{wahałby} się przy użyciu.

Oczy teraz otwarte. Łóżko szpitalne. Czteropunktowe kajdanki na zamek
błyskawiczny i~kiedy je zobaczyła, zdała sobie sprawę, że jej śpiący
mózg już to zauważył, szarpnął je we śnie i~oczekiwał ich.

Łóżko szpitalne, ale nie sala szpitalna. Prywatny dom. Ten zapach. Dom
jej ojca. Była w~domu. Zaczęła po cichu płakać.

\threeast

Jej siostra podeszła do jej łóżka. Cordelia, dwa lata młodsza od niej,
włosy inne niż ostatnio, kiedy widziały się podczas przerwy na studia,
bardziej wyrafinowane, z~precyzyjnym niefrasobliwym bałaganem, ale poza
tym się nie zmieniła. Spojrzała w~dół na swoją starszą siostrę z~nieodgadnionym wyrazem twarzy, położyła dużą torebkę na podłodze i~usadowiła się na kanciastym drewnianym krześle, które Natalie niejasno
rozpoznała jako będące kiedyś po dziewczęcej stronie domu. Mogła dojrzeć
oparzenie na jednym ramieniu, które pamiętała wyraźniej niż samo
krzesło.

Dwie siostry kontemplowały się nawzajem. Natalie była podobna do ojca,
miała jego dziwny nos w~kształcie ostrza i~podwójny dołeczek w~policzku,
których nienawidziła jako nastolatka, a~później doceniła, że ją
wyróżnia. Cordelia wyglądała jak mama, słabe wspomnienie z~dzieciństwa,
okrągła twarz porcelanowej lalki, szerokie zielone oczy i~szczypta
piegów jak u lalek Kewpie, ale z~satanistycznym błyskiem w~oku, który
wyglądał i~był wrażeniem z~lalki\dywiz horroru dzierżącej nóż.

Natalia poddała się. Uśmiechnęła się. Nie było honoru w~udawaniu, że
jest zrobiona z~lodu. -- Miło cię widzieć, Cordelio.

Cordelia odwzajemniła uśmiech i~Natalie zobaczyła cień własnego
uśmiechu. Wszyscy zawsze mówili, że są do siebie podobne, kiedy się
uśmiechają.

-- Dobrze wyglądasz, siostrzyczko.

-- Ty też. Masz jakieś nożyczki w~tej gigantycznej torebce? -- Potrząsnęła
swoimi więzami.

-- Tak i~z przyjemnością informuję cię, że zostałam upoważniona do
użycia. -- Jej siostra zawsze używała sarkastycznego, obraźliwego głosu,
kiedy była zdenerwowana.

-- Najlepsze wieści, jakie słyszałem przez cały ranek. Będę sikać jak koń
wyścigowy.

-- Zajmijmy się tym. 

To nie były normalne nożyczki; były w~specjalnej
pochwie, która się marszczyła, a~ich czarne ostrza zjadały światło.
Cordelia obchodziła się z~nimi tak, jakby były rozgrzane do czerwoności,
z pantomimiczną ostrożnością przecinając plastikowe więzy, aby trzymać
końcówki z~dala od skóry Natalie, chociaż nożyczki rozcięły kajdanki z~normalnym dźwiękiem przecinania plastiku.

Po lewej stronie Natalie, naprzeciwko zaciemnionego okna, znajdowały się
częściowo otwarte drzwi i~podbiegła do nich, czując pod stopami deski
podłogowe z~niepokojącą, halucynacyjną jasnością. Łazienka za nimi była
mała, wyposażona w~lustro, toaletę i~głowice prysznicowe tej samej
marki, co reszta domu. Ręczniki były znajome, białawe, z~pofałdowanym
brzegiem. Sikała, umyła się, nie patrzyła w~lustro, potem jednak
popatrzyła.

Była czysta. Jej włosy zostały zaczesane i~przycięte do jednolitej
długości, na całej długości pięciu centymetrów, długości najkrótszych
fragmentów jej ostatniej fryzury, którą zaaplikowała jej Sita, która za
pomocą nożyczek potrafiła dokonać nieoczekiwanie wielkich rzeczy.

Jej oczy tonęły w~ciemnych workach. Jej skóra była matowa, a~rysy twarzy
pogrubiony od oszołomienia. Skrzywiła się, obejrzała tył głowy i~zobaczyła siniaka wystającego z~jej szpitalnej szaty. Sięgał w~dół jej
ramienia i~teraz to zobaczyła, jej żebra i~ramię pulsowały, a~może
zauważyła pulsowanie, które tam było przez cały czas.

Widząc ranę, przypomniała sobie porwanie, drobną kobietę ze stalowym
uchwytem, niewidzialnego mężczyznę, który wyłonił się z~cienia. Potem
przypomniała sobie ciała, płacz Etcetery na ramieniu Limpopo, ranę głowy
Gretyl, tlące się ruiny \textit{Lepszego Narodu} i~los jego załogi.

Przeszukała łazienkę za bronią. Nie wyobrażała sobie uderzenia Cordelii,
ale nie wyobrażała sobie, żeby \textit{nie} uderzyć nikogo, kto stanąłby
jej na drodze.

Nie ma nic bardziej niebezpiecznego niż wyciskana butelka łzawiącego
szamponu miętowego. Nawet pokrywa deski sedesowej była przykręcona. W~porządku.

Weszła do sypialni. Cordelia odwróciła się do niej, uśmiechając się, po
czym uśmiech zniknął, gdy zobaczyła wyraz twarzy Natalie, a~Natalie
sięgnęła do drzwi pokoju. Nie była pewna, na który korytarz się otwiera,
ale stamtąd łatwo byłoby znaleźć drzwi i~ulicę i\ldots 

Przekręciła gałkę i~wyszła na korytarz.

Drobna kobieta pochylona, cały ciężar na czubkach stóp, była bez
wątpienia kobietą z~lasu. Uśmiech, małe zęby. Natalie rozpoznałaby je
wszędzie, chociaż bez lśniącej farby twarz kobiety zamieniła się w~zapomnianą, statystycznie przeciętną maskę środkowo\dywiz słowiańskiej
nijakości. Jednak zęby.

Natalie spojrzała jej w~oczy. Nie było gapienia się twardziela, tylko
lekkie zainteresowanie. Natalie zrobiła krok do przodu, żeby ominąć
kobietę, ale ona tam była, poruszając się szybciej, niż Natalie
kiedykolwiek widziała, żeby ktokolwiek się poruszał. To może być kac po
narkotykach. Przeszła na drugą stronę, kobieta już tam była.

Patrząc, powiedziała: 

-- Przepraszam -- i~próbowała przejść obok. Kobieta
mrugnęła.

-- Powiedziałam, przepraszam.-- Położyła dłoń na ramieniu kobiety,
delikatnie odpychając ją na bok. Nie naciskała.

-- Zejdź mi kurwa z~drogi. -- Natalie żałowała słowa kurwa, gdy
powiedziała, nic z~tym nie można już zrobić.

-- Natalie, wróć tutaj, dobrze? -- Z~tyłu głos Cordelii.

-- Zejdź mi z~drogi, proszę. -- Jej oczy zatrzymały się na łagodnym,
niezainteresowanym spojrzeniu małej kobiety. -- Proszę. -- Brzmiała tak
słabo. Przypomniała sobie, jak oszukała kobietę, przełamała jej uścisk.
Silna i~szybka, ale nie niezwyciężona. Niech myśli, że Natalie jest
słaba.

Ręka Cordelii na jej ramieniu. 

-- Chodź, Natalie. Ona nie pozwoli ci
przejść, a~gdyby to zrobiła, nie mogłabyś wyjść z~domu.

Nadal patrząc w~łagodne oczy kobiety. 

-- A gdybym wzięła cię jako
zakładniczkę?

-- Obezwładniłabym was obie. -- Kobieta odezwała się po raz pierwszy.
Miała słodki głos, dziewczęcy, głos, który pasował do twarzy Cordelii
(Cordelia spędziła całe życie na pielęgnacji ochrypłego głosu, bo w~przeciwnym przypadku było to zbyt wiele) i~trochę akcentu, który według
Natalie mógł być z~Quebec, a~może, co dziwne, z~Teksasu.

-- Natalie, proszę? -- powiedziała Cordelia.

-- Mordowali ludzi -- powiedziała Natalie. -- Na moich oczach. Pomagałam
rannym. Nosiłam zmarłych. -- Łzy na jej policzkach, ale jej głos był
spokojny. -- Zachowaj swoje jebane ,,proszę''. -- Znowu to przeklinanie.
Jebać to. -- Zejdź mi z~drogi, zabójco, albo przygotuj się, by mnie
obezwładnić, cokolwiek oznacza ten dupny eufemizm.

Kobieta się nie odezwała. Uścisk Cordelii zacisnął się na jej ramieniu,
nie dałby się łatwo zrzucony. Kobieta miała na sobie rzeczy, które były
prawie odchodnickie: bezszwowe, wydrukowane jako jeden kawałek, z~wplecioną bitmapą, konserwatywne ciemne paski na ciemniejszym tle. Paski
wpływały na jej postrzeganie postawy i~ruchu kobiety, utrudniając
przewidywanie, dokąd pójdzie i~kiedy tam się znajdzie. Więcej
olśniewania.

Bez nakręcania, nie pozwalając, by myśl przeniknęła do jej
przodomózgowia, zrobiła szorstki krok, wbijając się w~kobietę, zderzając
się ciałami i~była już gotowa na kolejny krok.

Wtedy leżała na plecach, zdyszana, kobieta cofnęła się o~krok. Jej wyraz
twarzy się nie zmienił. Małe zęby.

-- Natalie -- powiedziała Cordelia. -- To zmierza donikąd. Nie rozwiążesz
tego siłą. Nie masz przewagi.

Odchodzący odchodzą. Ale co, jeśli jesteś uwięziona? Natalie rozważała
kolejny pojedynek na spojrzenia z~kobietą, plując jej w~twarz. Robiąc to
w kółko. Była głęboko przekonana, że kobieta to przyjmie. Żywy obraz
kobiety z~plwociną Natalie na twarzy był zabawny w~sposób, który
zidentyfikowała jako uczucie w~stylu Jacoba Redwatera.

Wstała, tyłem do kobiety, jakby była meblem, i~odmówiła pomocy Cordelii.
Wróciła do pokoju. Cela. Bolało ją ramię.

\threeast

Karmili ją przez szyb kuchenny, jej ulubione dziewczęce jedzenie. To
było gorsze niż kisiel lub spleśniały chleb. Tunele kuchenne były w~całym domu, co było sposobem na zaspokojenie pragnienia bez kłopotliwej
uprzejmości w~kontaktach z~ludzkimi sługami. Ona i~Cordelia nazwały go
Redwater Prime, po usłudze Amazon, ponieważ wiedzieli, że gdzieś w~łańcuchu są ludzie, którzy nie zarabiają na tyle dużo, by kupić rzeczy,
które wysyłali.

Cordelia odwiedzała ją przez następne dwa dni. Dom -- jej ojciec -- wysłuchał ich. Natalie wiedziała o~tym, bo kiedy prosiła o~rzeczy,
czasami przychodziły do windy. Ale nie miała bezpośredniego dostępu do
interfejsów.

Ojciec jej nie odwiedził.

Posiłki i~spełniane życzenia przychodziły w~nieregularnych odstępach
czasu. Wiedziała, że było to sporadyczne wzmocnienie. Daj szczurowi
granulkę karmy za każdym razem, gdy naciśnie dźwignię, to naciśnie ją,
gdy będzie głodny. Daj szczurowi karmę w~odpowiedzi na losowe
naciśnięcie, będzie naciskał i~naciskał, mijając sytość, gdy wzór
pasujący do części jego mózgu próbował znaleźć wzór w~losowości. Można
było produkować przesądne szczury, był to jeden z~ulubionych epitetów
Limpopo dla ludzi, którzy byli szczególnie głupi na przesądny szczurzy
sposób. Przesądne szczury kilka razy zauważyły pewną kombinację działań
przed naciśnięciem dźwigni dającej karmę, zdecydowały, że trzeba to
zrobić za każdym razem, a~mimo to nie zmieniło to częstotliwości
dystrybucji karmy, każdej granulce \textit{towarzyszył }przesądny taniec,
wzmacniając rytuał.

Kobieta za jej drzwiami wydawała się nigdy nie spać. Może miała
bliźniaczkę albo była robotem. Zawsze tam była, neutralna, z~małymi
obnażonymi zębami, blokując korytarz. Natalie miała wyraźne, brutalne
fantazje o~torturach na temat kobiety, co by zrobiła, gdyby miała broń,
paralizator albo moc poruszania rzeczy umysłem.

Jej umysł. W~pokoju było: krzesło, łóżko, resztki posiłków -- czego nie
włożyła do windy -- brudne pranie i~cztery ściany, dwoje drzwi, jedno
okno. Łazienka: papier toaletowy, szczoteczka do zębów -- automatyczne
nakładanie pasty -- ziemisty zapach probiotyczny tonika, który przywodził
jej na myśl mamę, choć tak naprawdę nie wiedziała, czy jej mama
kiedykolwiek go używała, i~zabójczo mocne mydło miętowe, przy którym
myślała o~ojca, w~silikonowych butelkach, które przypominały skórę
seks-zabawki.

Drzwi nie miały zamka. Jednak kobieta nie chciała jej wypuścić i, jak
przypominała jej Cordelia podczas coraz rzadszych wizyt, nawet gdyby
zeszła ze schodów, drzwi nie wypuściłyby jej na szerszy świat.

-- Dużo czasu spędziłaś w~zettalandzie? -- zapytała kobietę. Zaczęła
siadywać na korytarzu i~jej się przyglądać. Wcześniej mówiła do siebie w~pokoju/celi jako przedstawienie dla ukrytych obserwatorów lub
algorytmów. To sprawiło, że była tak skrępowana, że przyszła poprowadzić
swój monolog z~kobietą, która mogła być posągiem.

-- Spodziewam się, że tak. Kogoś takiego jak Ty, dobrego w~tym, co
robisz, prawdopodobnie jesteś zatrudniana przez tych wszystkich
najbardziej elitarnych baronów i~plutokratów.

-- Większość moich przyjaciół była zettami. Dopiero gdy zsunęłam smycz i~przyprowadziłam do domu kilku cywilów, naprawdę zrozumiałam, jak to było
pojebane. Moi przyjaciele mieli trudności ze zrozumieniem, niektórzy z~nich nigdy się do tego nie przyzwyczaili, po prostu ciągle zauważali,
jakie to dziwne. Uderzyło mnie to, że rozmawiali o~inwigilacji, jakby
nie byli obserwowani w~każdy możliwy sposób w~ich mieszkaniach, metrze
czy szkołach. Jakby chodnik nie mierzył ich chodu i~nie wąchał
pióropusza CO2 w~poszukiwaniu zabronionych metabolitów.

-- Teraz rozumiem. Zetta realizują inwigilację sobie. Nie jest robiona
im. Można by zbudować taki dom bez czujników, w~stylu retro, ze
sznurkami biegnącymi po ścianach, które dzwonią w~dzwonkach w~kwaterach
służby. Można by wyłożyć ściany miedzianą siatką i~uczynić z~niej
fortecę odporną na radio.

-- Oczy i~uszy to anioły rejestrujące, które pamiętają wszystko na
zawsze. Są wyborami. Nigdy o~tym nie myślałam, tak samo jak ryba nie
myśli o~wodzie. Teraz rozumiem.

-- Definicja zetta brzmi ,,ktoś, kto nie żyje tak, jak wszyscy inni''.
Znasz \textit{Gatsby'ego}? ,,Bogaci są inni''. Nikt już nie czyta
\textit{Gatsby'ego} jako krytyki. Teraz wydaje się nostalgiczny. Albo
Orwell, wewnętrzna impreza z~wyłącznikami teleekranów. Dlaczego zetta
miałby montować teleekrany w~swojej jebanej łazience?

Zastanowiła się nad ironią czujników rejestrujących i~analizujących jej
wypowiedzi na ich temat. Pomyślała o~Roz, komputerze, który był osobą.
Bawiła się fantazją, że sieć domu jest samoświadoma, wiedząc, że o~tym
mówi i~była na to zła, chciała ją wyłączyć. Nic dziwnego, że pojawiło
się tak wiele net-soap oper o~ludziach zabitych przez nieuczciwe
komputery, klisza ,,Nie mogę pozwolić ci to zrobić Dave'', który był
głównym tematem dla gryzipiórków.

Kobieta wpatrywała się, skupione oczy, niczego nie zdradzając.

-- Musisz być piekielnym graczem w~pokera. Widziałem kiedyś Beefeaters w~Londynie? Mam na myśli Anglię, nie Ontario. Byli bzdurą, próbowali
udawać drewnianych żołnierzy, nigdy cię nie zauważając. Nie sądzę, że
można być czujnym, udając, że wszyscy inni są niewidzialni. Mów to sobie
wystarczająco długo, a~uwierzysz. Ty, z~drugiej strony, słyszysz i~widzisz mnie, ale to tak, jakbym był poniżej Twojej uwagi, chyba że
próbuję przejść obok. Słyszysz mnie. Cholera, prawdopodobnie zgadzasz
się z~każdym słowem, ale to, co mówię, nie jest niczym w~porównaniu z~niewzruszoną prawdą jebanej tony pieniędzy dla Ciebie, jeśli zrobisz tak
jak default; nic, jeśli podążysz za swoim sumieniem.

-- Z~drugiej strony, być może to moja projekcja. Może kochasz default,
myślisz, że dziwaczne gówno zetta jest dowodem ich boskiego prawa do
rządzenia. Może jesteś zwierzęcą przebiegłością i~żylastą siłą, a~za
tymi twoimi chłodnymi oczami nie dzieje się wiele?

Zatrzymała się, świadoma, że jest zettą, drwiącą z~osoby, która nie może
odpowiedzieć, bo nią nie jest. Zawstydziła się.

-- Przepraszam -- powiedziała i~weszła do swojego pokoju.
\chapter*{ii}

Ojciec odwiedził ją czwartego dnia. Przeżyła dwadzieścia cztery godziny
bez drwin ze strażniczki i~była znudzona do szaleństwa. Fantazjowała o~notatniku i~długopisie, o~wszystkim, czego mogłaby użyć, by wylać swoje
uczucia komuś innemu niż niewidzialnym obserwatorom.

Wyglądał na opanowanego. To była pierwsza rzecz, jaką zauważyła,
kontrast między jej drżącymi nerwami a~jego spokojną powierzchownością.
Pomyślała, że zrobił coś ze swoją twarzą, zastrzyki. Wyglądał na
młodszego, niż pamiętała, 35-letni młodzieniec. Odwrócił krzesło, usiadł
okrakiem na nim z~rękami skrzyżowanymi na plecach, przechylił głowę i~uśmiechnął się, jakby wspólnie żartowali. W~uśmiechu było zdecydowanie
coś innego.

-- Witaj w~domu, Natty.

Pomyślała, żeby go zblokować, jak strażniczka, której spojrzenie
widziało, ale nie zauważało. Była taka samotna, tak znudzona. Jej mózg
był jak kołowrotek dla chomika, wirujący bez kontroli. Musiała spowolnić
go słowami, nawet jeśli byłaby to kłótnia.

-- Chciałabym już iść, proszę.

Uśmiechnął się szerzej. 

-- Jak było?

Zmusiła się do wdechu przez przeponę raz, dwa razy. 

-- Jestem pewna, że
dostałeś raport.

-- Twoja matka przyjeżdża jutro. Nie może się doczekać, kiedy cię
zobaczy.

-- Wiesz, oni zabijali ludzi. Widziałam ich, widziałam ciała. Trzymałam
ciała. Moi przyjaciele, to byli moi przyjaciele. -- Starała się zachować
spokojny głos, odnosiła sukcesy, z~wyjątkiem drgnięcia przy drugim
,,ciała''. Była pewna, że jej ojciec to wychwycił. Był człowiekiem,
który był żywo dostrojony do pożytecznych słabości innych.

-- Widzę, że było ci ciężko.

-- Masz na myśli, że najemni terroryści, których wysłałeś, byli dla mnie
surowi.

-- Straciłaś kontakt z~rzeczywistością, kochanie. Nie możesz w~to
wierzyć. Nie mogę nakazać nalotów. Nie znam najemników. Jestem strasznym
bogatym facet, ale jeśli moi wrogowie boją się mnie, to dlatego, że
martwią się, że będę ich \textit{pozywać}, a~nie mordować.

Natalie zamknęła oczy i~próbowała znaleźć rytm swojego oddechu. Dla jej
ojca -- \textit{jej ojca} -- powiedzieć, że nie mógł mieć nic wspólnego z~tym, co się stało, kiedy w~korytarzu była najemniczka ninja, to było za
dużo. Uosabiało to każdą rozmowę, jaką kiedykolwiek odbyli, w~której
mówił jej, że wszystko, co czuła i~na co liczyła, było dziewczęcym
marzeniem, a~każda obserwacja otaczającego ją świata dziewczęcym
upodobaniem.

Jej oddech nie przychodził. Może jej ojciec miał nadzieję, że długa
izolacja sprawi, że będzie uległa. Ale coś w~środku pękło, co
zabrzęczało. Uświadomiła sobie gwałtownie, co przypominało napad
wymiotów, że od początku pobytu prawie nie myślała o~Gretyl. To
sprawiło, że zaczęła się zastanawiać, czy zrobili coś z~jej umysłem, czy
była sobą. Nawet gdyby w~korytarzu była najemniczka, która mogłaby ją
rozłożyć ruchami tak szybkimi, że nie byłaby w~stanie za nimi podążać.
Może to był sen. Być może była martwa, uploadowana, zasymulowana.

Hiperwentylowała i~odczuwała satysfakcję z~dyskomfortu ojca. Potrafił
sobie radzić z~napadami złości typu ,,nienawidzę cię, tatusiu'', ale ona
teraz to traciła, cieszyła się, że się zgubiła. Była zmęczona tym, że
została odkryta, udawaniem, że sytuacja jest normalna.

Wstała spokojnie i~wygładziła długą koszulkę, poprawiając sznurek spodni
dresowych -- czerwony, z~ostro zakończonym logo ROOTS na jednym udzie,
coś, co włożyłaby na letnim obozie, jakby winda kuchenna została
załadowana przez kogoś, kto próbuje sprawić, by poczuła się jak
uziemiona nastolatka, a~nie ofiara porwania -- i~wyszła z~pokoju.

Strażniczki nie było w~holu.

Ruszyła biegiem, słysząc, jak jej ojciec -- krok za nią -- krzyczy coś,
czego nie mogła zrozumieć. Zrobiła pięć długich kroków w~dół korytarza,
a najemniczka wyszła zza rogu, bez wysiłku chwytając ją za ramię, gdy
próbowała przebiec obok, wkładając łydkę między biegnące nogi Natalie,
gładko ciągnąc za ramię, rzucając ją w~dół z~uderzeniem zębów. Deski
podłogowe były z~jasnego drewna o~ciemnych słojach, a~ogrzewanie
podłogowe sprawiało, że wydawały się żywe. Widziała tylko te słoje,
sięgające do listew przypodłogowych. Czekała na powrót oddechu.

Uniosła się na kolana, na nogi, najemniczka nie wtrącała się ani nie
oferowała pomocy, stała z~taką samą bezinteresowną uwagą, która dawała
Natalie do zrozumienia, że patrzy, ale nie czuje. Natalie oparła się o~ścianę, spojrzała na ojca po drugiej stronie najemniczki. Wyglądał na
wściekłego, a~ona zdała sobie sprawę, że jest wściekły na najemniczkę,
nie na nią, bo najemniczka opuściła stanowisko, być może wymykając się
na sikanie, myśląc, że Natalie będzie zadowolona podczas negocjacji typu
,,Przyjdź do Jezusa'' ze swoim starym. Najemniczka spieprzyła przed
wielkim szefem. Natalie próbowała wyczytać z~jej twarzy wyraz twarzy
kelnerów i~kierowników hoteli, kiedy jej tata nie był z~nich zadowolony.
Była spokojna. Natalie nie mogła nie podziwiać jej. To było pokręcone,
ale czuła solidarność z~każdym, kogo jej ojciec planował zniszczyć.

-- Żadnych twardzieli na liście płac, co? -- Obróciła się na pięcie, żeby
wyjść z~domu. To było głupie, ale dlaczego nie? Kobieta złapała ją za
ramię w~sposób, który dał jej zaskakującą dźwignię, obróciła ją prawie
bez użycia siły, choć nie było mowy, żeby Natalie mogła powstrzymać.
Natalie poruszyła się pod tą ręką, ale ręka z~łatwością jechała po jej
ramieniu, unosząc się i~opadając jak flaga na wietrze.

-- Jakie są Twoje rozkazy? Powiedziałaś, że możesz mnie ,,obezwładnić''.
Pobiłabyś mnie do nieprzytomności? Sekretne szczypanie nerwów? Masz
taser ukryty w~tym stroju ninja? -- Długo przyglądała się ojcu. Do
perfekcji opanował wyraz twarzy i~całym ciałem emanował niecierpliwą
nudą.

-- Dowiedzmy się. -- Natalie zrobiła trzy kroki w~kierunku swojego taty,
który w~ostatniej chwili wzdrygnął się nieznacznie. Zatrzymała się,
okręciła, spojrzała na kobietę, po czym zaatakowała. Jeden krok, dwa
kroki, bum, na deskach podłogowych, wpatrując się w~sufit, dostrzegając
wnęki na LED, które można było zobaczyć tylko wtedy, gdy było się na
ziemi. Bolały ją plecy. Miała wrażenie ducha ręki na nadgarstku, stopy
na kostce, wrażenie, że kobieta ledwo się poruszyła, by nią rzucić. To
był duch wszystkich tych sztuk walki w~stylu Sun Tzu: użyj siły wroga
przeciwko niemu. Zachichotała na myśl, że powinna robić notatki,
wymyślić, jak zdemontować rzeczywistość default, wykorzystując jej siłę
przeciwko niej samej.

Wstała. Kobieta cofnęła się o~pół kroku, ciężarem z~przodu, podczas gdy
jej tata został na drugim końcu korytarza w~swojej surowej,
rozczarowanej masce. Nie była całkowicie nienaruszona. Było tam trochę
zmartwień, których Pan Pokerowa Twarz nie mógł ukryć. Oparła się o~ścianę i~wzięła kilka oddechów.

-- Spróbujmy zrobić najlepsze dwa z~trzech. -- Twarz jej ojca zamigotała i~oto był: strach.

Zaatakowała go. Nie wzdrygnął się, ale zobaczyła, że chciałby, odwróciła
się i~zanim zdążyła pomyśleć, pobiegła prosto na strażniczkę, schodząc
nisko, jakby grała w~platformówkę i~próbowała jednego podejścia za
drugim, by pokonać mini-bossa poziomu, mając nadzieję, że nie zabraknie
jej żyć, zanim odkryje sztuczkę. Może gdyby zeszła nisko, trudniej
byłoby ją rzucić.

Nie było.

Tym razem uderzyła się w~łokieć, a~jej ciało zostało przebite białym
piorunem, co sprawiło, że zaczęła zasysać powietrze przez zęby. Czym
właściwie był ból? Roz może odczuwać ból lub nie, ale była to
infografika, suwak, który przesuwałaś w~górę lub w~dół. Jej ramię było
\textit{ranne} -- coś zostało uszkodzone -- ale \textit{uczucie} zranienia nie
było wewnętrzne. Możesz zostać zraniona i~nic nie czuć, możesz odczuwać
ogromny ból bez obrażeń. Uraz był w~łokciu, ból był w~mózgu.

Jednak bolało.

Wstała jeszcze wolniej, potarła łokieć. Tym razem zwróciła większą
uwagę, miała wrażenie, że przechodząca kobieta lekko dotknęła jej
ramion, zrobiła jej coś, co spowodowało, że jej ciężar przesunął się do
przodu, najpierw kierując twarzą w~podłogę.

Oddychała. Jacob się skrzywił. Obserwowała go, jak z~trudem przemienia
swój strach w~gniew. Gniew był w~porządku. Była cholernie
\textit{wkurwiona}.

-- Trzeci raz to urok.

Tym razem złapał ją, ale była odchodzącą, noszącą ciężkie ładunki dla
czystej przyjemności robienia rzeczy własnymi rękami, chodząc wiele
kilometrów, odbywając długie, niespieszne, zmysłowe sesje jogi na
trawniku pensjonatu, które uczyniły ją silną i~giętką. On był szczurem
gimnastycznym, do którego przychodzili wykwalifikowani trenerzy i~farmakopea, która dała mu skrócone mięśnie brzucha jak model bielizny i~ramiona ze szczupłymi tricepsami i~silnymi nadgarstkami, i~mógł ćwiczyć
godzinę na maszynie eliptycznej, ale to wszystko na pokaz, nigdy
nieużyte.

Z łatwością go strząsnęła. Była zachwycona, gdy zdała sobie sprawę, że
mogła go zmiażdżyć tak łatwo, jak najemniczka ją nokautowała, wbiegając
z jednej strony w~dół, z~drugiej. Minęły lata, odkąd jej ojciec był
fizyczny, ale przypomniała sobie jego żelazny uścisk, jak mógł wynieść
ją z~pokoju, kiedy źle się zachowywała, ignorując jej wiercenie się.
Niech teraz spróbuje.

Spróbowała wykonać niski sprint, próbując przyspieszyć, chociaż
wiedziała, że musiałaby zostać wystrzelona z~długiego łuku, zanim jej
prędkość będzie wystarczająco duża, by zrobić różnicę dla kobiety.
Niemal zachwiała się, gdy się zbliżyła, jakaś tchórzliwa część niej, nie
chcąc dostać nadchodzącego ciosu, i~zabiła tę część wybuchem woli,
nabierając \textit{większej} prędkości.

Schodząc w~dół, jej głowa odbiła się od ściany, wypełniając jej wizję
gwiazdami. Wstawanie zajęło jej jeszcze więcej czasu. Miała zawroty
głowy. To było solidne uderzenie. Czy kobieta zraniła ją celowo, by
ukarać ją za odmowę poddania się? Czy jej rozbieg był po prostu lepszy?

Jej ojciec wszedł do sypialni, miała zadzwonić po wsparcie, więc tym
razem szła, odwróciła się, popatrzyła na kobietę, teraz obie same, nie
mówiąc o~pilnujących oczach.

Pobiegła. Coś było nie tak z~jej równowagą i~nie mogła złapać tempa. Tym
razem kobieta złapała ją i~odwróciła, zgrabnie niwelując cały jej pęd.
Natalie i~ona stanęły twarzą w~twarz. Nieszczególna twarz kobiety i~małe
zęby były tuż obok, jej oddech pachniał pastą do zębów. Miała glut w~jednym nozdrzu. Jej brwi nie były wyskubane, czego Natalie nie
zauważyła, a~krzaczaste krzaczki przypomniały jej Gretyl. Chciała
płakać.

Próbowała przejść obok, weszła w~kobietę, została delikatnie
odepchnięta. Spróbowała ponownie. Naprawdę kręciło jej się w~głowie. To
musiał być mocny cios.

Ta kobieta nie była jej wrogiem, po prostu miała pracę. Natalie to nie
obchodziło. Zakołysała się szalonym półobrotem, który kobieta z~łatwością ominęła. Czy lekko się uśmiechnęła? Dziwnie było tu być,
milcząc, z~wyjątkiem oddechu, mamrotania ojca z~sypialni. Intymność bez
słów. Znowu się zamachnęła. Znowu. Gdyby miała broń, zastrzeliłaby tę
kobietę, jej ojca, siebie. Co robi odchodząca, gdy nie może odejść?

Poddała się, ręce opadły jej na boki. Weszła do sypialni. Jej ojciec
siedział na krześle, wyglądając na zdegustowanego, jakby była żałosna.
Przypuszczała, że była. Co jest bardziej żałosne niż odchodząca, która
przestaje odchodzić?

Przełknęła i~próbowała zebrać się na odwagę, by rzucić się na niego,
wsadzić mu kciuki w~oczodoły, wbić mu paznokcie w~gardło, wbić mu jaja
kolanami. Myśl o~przemocy była tak uwodzicielska, że właściwie
powstrzymała się od ruchu, zaskoczona żarliwością swojego id.

Ale potem przyjęła to z~uśmiechem drapieżnika. Usłyszała siebie, jak
dyszy. Teraz to zrobi. Jej ojciec zrozumiał, widziała to w~jego oczach.
Bał się. Drapieżnik powstał. Będzie jej się to podobało.

Jeden krok. Dwa kroki.

Dłoń na jej ramieniu. Silna. Ręka mężczyzny, ściskająca tak mocno, że
sapnęła, a~potem igła w~jej łokciu. Odwróciła się i~zobaczyła mężczyznę,
niewielkiego, niższego od niej, ale z~twarzą jak kamień i~karkiem byka.
Potem nie widziała nic więcej.

\chapter*{iii}

Była pewna, że nadal była w~domu. Nie można było sfałszować tego
zapachu. Jednak wyglądał jak sala szpitalna. Drzwi nie miały klamki ani
wpuszczanego panelu, tylko jakiś niewidzialny czujnik, który wybierał,
kto przechodzi. To szpitalne łóżko było większe, prymitywniejsze, a~ona
-- poruszyła biodrami -- była w~nim zanurzona. Miała kroplówkę w~nadgarstku i~niewyraźne samopoczucie, o~którym wiedziała, że może nie
być endogenne. Zastanawiała się, co jest w~woreczku kroplówki.
Pokochałaby teraz trochę Meta.

Była skrępowana w~czterech punktach, z~dodatkową opaską na
przedramieniu, aby mocno osadzić kroplówkę.

Przypuszczała, że była to próba samobójcza. Pomysł nie był zbyt
niepokojący. Jej smutek był odległym księżycem krążącym daleko od jej
psychiki, widocznym, ale ostatecznie wywierającym tylko najłagodniejszy
wpływ.

-- Co teraz? -- Jej głos był gęsty, usta miała lepkie. Jeśli w~torebce
była sól fizjologiczna, to nie zapewniała jej nawodnienia. Wyglądało to
tak, jakby ktoś wrzucił jej do ust łyżkę hydrofilowych żelowych granulek
transportowych, wysuszając je do konsystencji starej padliny.

Chciała, żeby drzwi się otworzyły, myśląc o~dniach, które spędziła w~odosobnieniu w~drugim pokoju, zastanawiając się, czy ją zostawią, rurkę
wchodzącą i~wychodzącą, mózg nierozerwalnie związany z~niewygodnym
mięsem, łatwo przymuszony, dzięki swoim śmiesznym słabościom.

Czy mieli ten pokój gotowy przez cały czas, jako plan B? A może była
nieprzytomna, kiedy przerabiali pokój, żeby był bezpieczny?

Przez drzwi wszedł pielęgniarz, ubrany w~szpitalne białe ubranie i~pchający wózek. Stanął przy łóżku.

-- Cześć -- powiedziała.

Spojrzał na nią z~namysłem, po czym wyciągnął tace z~wózka, przyłożył
jej termometr do ucha i~założył mankiet uciskowy na jej ramię. Odrzucił
koc i~bezosobowo podciągnął jej suknię, uzyskując dostęp do małego
pudełka przyklejonego taśmą do jej biodra, o~którym nie wiedziała, że
tam jest.

-- Dlaczego te wszystkie rzeczy nie są dostarczane ze zdalną telemetrią?
Jeśli zamierzasz udawać, że nie istnieję, dlaczego nie odebrać
transmisji w~innym pokoju, oszczędzić sobie społecznej niezręczności?

Był dobry w~ignorowaniu. Sprawdził jej cewnik tak mechanicznie, że
zamiast upokorzenia poczuła złość, co było na swój sposób miłosierdziem.
Co za dupek.

-- Wiem, że nagrywają kamery, ale przynajmniej mrugnij do mnie. Czy
pielęgniarki nie muszą składać ślubów? Przysięgi? Jesteś pielęgniarzem?
Może jesteś ,,technikiem medycznym''. Wyleciałeś ze szkoły
pielęgniarskiej i~dostałeś wersję, w~której nie było na szkoleniu o~Florence Nightingale?

Naśmiewanie się z~niego nie było satysfakcjonujące, a~jej usta były tak
suche.

-- Co powiesz na drinka? Woda? Sok?

Miał wąż z~końcówką z~gąbki. Zdjął prześcieradła i~wrzucił je do kosza w~podstawie wózka, odsłaniając gumowe prześcieradło. Pracując z~taką samą
bezosobową skutecznością, szybko wykąpał ją gąbką, trzymając w~jednej
ręce wąż, a~w~drugiej mały hydrofilny ręcznik, zatrzymując się po każdej
kończynie, by wycisnąć szmatkę do wózka. Ze swojego odległego mentalnego
punktu widzenia Natalie podziwiała wózek i~zastanawiała się, kto jest
jego głównym nabywcą, ludzie z~obłąkanymi starymi krewnymi zamkniętymi
na strychach?

Wyczyścił jej twarz i~uszy chusteczkami ze sterylnego opakowania, jak
faceci w~zakładzie zajmującym się detalami, którzy pracowali
ściągaczkami nad przednią szybę samochodów jej taty. Fakt, że robili to
ludzie, był atrakcją. Wszystkie miejsca, których używał jej tata, miały
w nazwie ,,na zamówienie'', ,,pranie ręczne'' lub ,,rzemieślnicze'',
czasami wszystkie trzy. Poczuła zapach pielęgniarza, mydła z~odrobiną
potu, zobaczyła zarost pod lewym uchem. Był jeden moment, w~którym mogła
go pocałować. Albo ugryźć.

Kiedy skończył, spakował tacę, wciągnął jej ubrania na miejsce i~wymienił prześcieradła. Sięgnął pod łóżko w~poszukiwaniu elastycznego
węża z~nadgryzionym sutkiem na końcu. Oderwał kawałek taśmy
chirurgicznej i~przykleił ją do jej obojczyka i~policzka, aby mogła
odwrócić głowę i~się napić. Mogła odgryźć czubek palca, ale tego nie
zrobiła. Spakował swoje rzeczy i~wyszedł. Drzwi zamknęły się z~westchnieniem i~kliknęły, a~potem zaskrzypiały, przypomnienie, że mają
poważne zamki. Brzmiało to tak, jakby drugi stuk przeszedł przez
podłogę, jakby drzwi miały bolce, które się w~nią wbijały.

Zdała sobie sprawę, gdzie jest: pokój paniki jej taty. Miał niezależne,
redundantne połączenia sieciowe, zapasowe źródła zasilania, zapasy
żywności i~wody, całą zbrojownię. Opowiadanie innym ludziom o~schronie
nie było w~stylu jej ojca, nigdy go nie widziała i~wiedziała, że
otwarcie go wywoła alarm w~całym mieście. Jej tata upewnił się, że wie,
na wypadek, gdyby wpadła na pomysł zorganizowania tam przyjęcia.

Tata musiał zbudować sobie lepszą dziurę, bawił się kiedyś myślą o~takim
w drugich podziemiach, wywiercając pod domem za pomocą supertajnego
wiertła, którego użył jego kumpel zetta, by zmienić działkę pod
posiadłością w~jaskinię batmana. Wprawiało to tatę w~ekstazę zazdrości.
Nie ma mowy, żeby wpuścił pana ,,nie\dywiz pielęgniarza'' do tego miejsca,
gdyby nadal był to sekret, na który stawiał swoje życie. Chyba że
zamierzał zwolnić cały personel po praniu jej mózgu i~pochować ich za
wzmocnionymi murami jak budowniczych grobowców faraona.

Te myśli stworzyły siedmiominutową rozrywkę. Kiedy się wyczerpały, była
sama ze swoją sytuacją. Na myśl o~Gretyl zapłakała z~pożądania i~samotności. Były myśli o~jej ojcu i~siostrze. Czy jej ojciec nie
powiedział, że jej matka jest w~drodze? Czy była tutaj? Miała własne
piętro po stronie dorosłych domu. Nie było często zajmowane, ale kiedy
było, zmieniał się wpływ domu. Cała rodzina była świadoma możliwości, że
ich merkuriańska pani domu wykona jeden z~jej opatentowanych numerów
Walkirii.

Goniła swoje myśli w~coraz ciaśniejszych spiralach. To było rozpaczliwe
miejsce. Bądź tam dostatecznie długo, a~może doprowadzić cię do
samobójstwa.

-- Pieprzyć to -- powiedziała na głos. -- Pranie mózgu, gumowe węże,
deprogramowanie, wszystkie te rzeczy z~Patty Hearst. -- Dowiedziała się o~Hearst, biednej małej bogatej dziewczynce, która nosiła broń ze swoimi
porywaczami, po tym jak Gretyl zażartowała na ten temat. Była urażona,
ale potem przyjęła dziewczynę jako totem. Hearst była idiotką, ale
przynajmniej nie była kolejną bogatą dupą.

Zaśpiewała ,,Konsensus'', niesamowicie sprośną marszową piosenkę
odchodzących, składającą się z~trzydziestu zwrotek. Refren: ,,Konsensus,
konsensus, pokonał nas i~złamał nas, ale jesteśmy pewni, że nam da,
szeroki uśmiech.'' Wymyślanie nowych wersów było odchodnickim sportem,
istniały nowe wiki o~nich. Nie pamiętała ich wszystkich, ale potrafiła
wymyślać je w~locie, zwłaszcza gdy śpiewała hum-humm-humm, gdy nie mogła
wymyślić wersu, co oznaczało automatyczną dyskwalifikację, gdyby była
śpiewana na poważnie.

Wersy coraz częściej były hum-hum-humm. Była gotowa przerwać i~zacząć
kolejną piosenkę, gdy dołączył głos: 

-- \ldots  ale jesteśmy pewni, że nam pożyczy, da szeroki uśmiech! -- To było boleśnie znajome. Zadrżała od głowy do kostek, włosy na jej szyi się zjeżyły.

-- Roz?

-- Dla ciebie to Roz Ex Machina, dzieciaku -- powiedział głos.

Zapłakała.

\threeast

-- To brudna sztuczka. -- Opanowała łzy. -- Absolutnie odrażająca.

-- Byłaby -- powiedziała Roz -- gdyby to była sztuczka.

-- Skąd możesz wiedzieć, czy jest, czy nie? Jesteś na wszystkich
serwerach kontroli wersji. Każdy, kto potrafi zbudować klaster, może Cię
uruchomić. Będą was setki, w~najróżniejszych konfiguracjach. Mój tata
mógł z~łatwością pozwolić sobie na wersję ciebie, która była
ograniczona, więc wierzyła, że przeniknęła do jego siatki, aby działać
przeciwko niemu, jednocześnie szpiegując mnie i~wszystko, co robię.
Nigdy byś się nie dowiedział. Powiem ci rzeczy, za które musiałby odciąć
mi sutki, żeby dostać innym sposobem. Nazwałby to \textit{humanitarnym},
sposobem ,,niskiego wpływu'' doprowadzenia mnie do zdrowego rozsądku, co
w jego świecie jest umiejętnością oszukiwania siebie, żeby wierzyć, że
zasługujesz na to, by mieć więcej wszystkiego, czego inni mają mniej z~powodu Twojej szczególnej śnieżynkowatości.

-- Głosisz kazanie nawróconej, dziewczyno. Pamiętaj, że odeszłam przed
Tobą.

-- \textit{Roz} odeszła przede mną. \textit{Ty}, kimkolwiek jesteś, jesteś
emisariuszką, wiedząc lub nie, z~defaultu.

-- Kręcimy się w~kółko. Zapomnij, bo jestem konstruktem. Mogę odłożyć
swoją frustrację na bok, przesunąć suwak, kłócić się z~Tobą tak długo,
jak chcesz. To jest spoko. Pochodzi z~laboratorium w~Pendżabie, byłych
maniaków matematyki IIT, którzy chcą zmienić Āgamę w~podprogramy, Yogic
Mastery Apps. Zamieniają Meta w~matematykę. Pokochałabyś to, uwielbiają
Gretyl, jej optymalizacje pod kątem modelowania z~wyprzedzeniem są
podstawą ich dyscypliny. Myślę, że gdyby nie martwiła się tak bardzo o~ciebie, nie zależałoby jej.

-- To było naprawdę niskie. -- Była zaskoczona jadem w~głosie. Gdy jej
myśli wędrowały do Gretyla, ogarniała ją nieznośna bezradność i~tęsknota. To, że Gretyl czuła do niej to samo, było ciężarem miażdżącym
jej klatkę piersiową.

-- Och, kochanie -- powiedziała Roz. Jej komputerowy głos był lepszy.
Emocje w~tych dwóch słowach były okropne. -- Ona tak bardzo za Tobą
tęskni. Mogę ci przekazać od niej wiadomość. Lub\ldots 

Natalie wiedziała, że to haczyk z~przynętą. Nie chciała się tym
podniecać. Ryba musi wiedzieć, że robak ma w~sobie zadzior, ale i~tak
niektóre gryzą. Czy to był głód? Życzenie śmierci? 

-- Co?

-- Zostały zeskanowane -- powiedziała Roz. -- Po dotarciu do strefy
opuszczonej Thetford wszyscy najpierw zrobili skan. Są teraz w~chmurach
odchodzących, każdego dnia coraz więcej. Tyle się uczymy z~mnogości
skanów, myślę, że problem z~przywróceniem OC polegał na tym, że po
prostu nie mieliśmy wystarczająco głębokiego zbioru danych, aby
wnioskować o~dostosowanych strategiach symulacji dla wariacji mózgu. OC
jest całkiem stabilny. Potrafimy scharakteryzować skany na podstawie
prawdopodobieństwa stworzenia udanej symulacji. Skan Gretyl jest w~górnym decylu. Została stworzona do pracy na krzemie. Sita też. Do
diabła, Sita była tak podekscytowana, że uruchomiła bliźniaka 24/7, w~czasie rzeczywistym, z~czujnikami na całej sobie. Gretyl tego jednak nie
zrobiła. Zrobiliśmy dla niej tylko oblot wstępny. Nie uruchomiliśmy
jej\ldots 

\textit{Jeszcze}, dokończyła Natalie. Gretyl mogła tu być, działać na
jakimkolwiek podłożu, na którym była Roz. Jej Gretyl, nie jej Gretyl, to
było rozróżnienie bez różnicy.

-- Tak cholernie złe. -- Nie miała energii na żółć. Wyszło jak
kapitulacja.

-- To nie jest skomplikowane. Twój tata ma niesamowity opsec w~głównej
sieci domowej. Ale patchlevel w~jego schronie jest nieaktualny, ponieważ
były konflikty, z~którymi automatyczne aktualizacje nie mogły sobie
poradzić, a~facet od obsługi, który go ustawił, przeszedł na emeryturę,
a Twój tata nie ma nikogo w~swoim dziale operacyjnym, który nawet o~tym
\textit{wie}. Komunikaty ostrzegawcze gromadziły się w~panelu
administracyjnym od lat, wszystkie zaniedbane. Zastanawiam się, czy twój
tata w~ogóle ma login do tego pulpitu?

-- Zajęliśmy to miejsce, jak tylko zniknęłaś. To był projekt Gretyla, ale
ja wykonałem ciężką pracę. Wykorzystałyśmy około siedemdziesięciu
procent czasu obliczeniowego sieci odchodzących z~moimi równoległymi
instancjami, dwadzieścia razy czas rzeczywisty. \textit{Pokonałyśmy}
jebany IDS, rozwaliłyśmy firewall, a~teraz jestem tak głęboko, że mogę
zrobić wszystko. -- Zamki w~drzwiach wystukały melodyjkę dzwonka. To było
przerażające i~zabawne. Udręczony uśmiech Natalie, aż bolał, żeby go
podtrzymać.

-- Ale nie mogę zdjąć twoich kajdan. Nie są połączone do sieci. Nie mogę
nic zrobić z~siecią domową, która jest całkowicie oddzielona. W~najlepszym razie, w~przeciwnym razie całe to przedsięwzięcie
uruchomiłoby wszelkiego rodzaju alarmy.

Wbrew sobie Natalie została wciągnięta w~wyjaśnienie. 

-- No weź. -- Jej spragniony bodźców mózg ciężko pracował. -- Brzytwa Ockhama. Albo jest
taki szalony błąd, ponieważ tata zwolnił swojego sysadmina i~istnieje
wygodna przerwa w~systemach łóżka \textit{albo} jesteś marionetką mojego
taty, a~on cię zamknął, abyś mogła wykonywać magiczne sztuczki z~zamkami, ale nie uwolniła mnie. Tak się \textit{składa}, że potrafisz
przynieść mi symulację mojej dziewczyny, która bez wątpienia skłoniłaby
mnie do mówienia rzeczy, które mój tata mógłby użyć do prania mózgu, jak
Cordelia.

-- To brzmi wiarygodnie. Nie mogę udowodnić, że jestem z~Tobą, nie z~twoim tatą. Symy nie mogą być pewne, czy symulator nie krępuje ich
obrotów, przez co nie są w~stanie zorientować się, że jesteśmy
manipulowane. Jesteśmy głowami w~słoikach. Ale skąd wiesz, że \textit{Ty}
nie jesteś symem? Przeskanowaliśmy tych najemników w~tunelach bez ich
wiedzy.

-- Mój tata nie jest tak\ldots  -- prawie powiedziała \textit{potężny }, ale nie
była pewna, czy chce tam wejść. -- Sentymentalny.

-- To kutas. Naprawdę się cieszę, że Cię znalazłyśmy. Połowa obozu
myślała, że nie żyjesz. Gretyl upierała się, że zostałaś porwana. Ktoś
sądził, że widział, jak wchodzisz do lasu, i~znalazł tam ślady bójki.
Nikt nie mógł powiedzieć, czy to ty, ale wszyscy inni byli policzeni lub
zmarli. Kiedy jedyną osobą, której brakuje, jest dzieciak dupka zetty,
nie jest trudno wyobrazić sobie porwanie.

Natalie tak bardzo pragnęła, tak bardzo wierzyła, że to Roz, koło
ratunkowe na zewnątrz, życie poza więzami jej skrępowanego ciała.
Oczywiście, że chciała. Jeśli Roz była podstępem jej ojca, będzie na
niej polegał.

Musiała się wysikać. To się budowało, teraz było nie do zniesienia.
Wiedziała, że została zanurzona w~łóżko, musiała wiele razy sikać, zanim
odzyskała przytomność, ale myśl o~dobrowolnym uwolnieniu pęcherza, gdy
była przywiązana do łóżka, była zbyt trudna.

-- Roz. -- Wstydziła się słabości w~swoim głosie. Dlaczego nie mogła być
tak silna jak Limpopo? Jak Gretyl?

-- Co jest, Lodołasico?

Nikt jej tak nie nazwał, odkąd została porwana.

Straciła kontrolę nad swoim pęcherzem. Siki spłynęły, zniknęły cicho w~wężu, uczucie gorąca, tam, gdzie wąż był przyklejony taśmą do
wewnętrznej strony jej uda, zanim zanurkował do cysterny łóżka. Mimo że
nie była przesiąknięta moczem, poczucie zasikania siebie było
nieuniknione i~straciła kontrolę nad łzami.

-- Och, Lodołasico. W~porządku, kochanie. To jest totalnie popieprzone.
Masz ludzi, którzy cię kochają, którzy wysłali mnie, żebym cię uwolniła.
Nie mogę przeciąć twoich więzów, ale mogę zrobić dużo więcej. Mogę
widzieć każdy pokój w~skrzydle więziennym. W~pokoju socjalnym są jeszcze
trzy osoby. Monitorują pokój, ale ja kontroluję te monitory. Nie widzą
ani nie słyszą przekazu w~czasie rzeczywistym, dostają pętle spania.
Twoje łóżko przesyła prawdziwą telemetrię, ale zamieniłam je na
przechowywane dane z~okresu nieświadomego. Mam ich prywatne wiadomości,
mogę wykonać stylometrię kontradyktoryjną, aby podszywać się pod nich w~smsach i~głosem, wykonaliśmy pracę nad głosami.

-- Mogę powiedzieć. -- Natalie pociągnęła nosem. To wszystko spływało po
jej twarzy. Łzy napłynęły jej do uszu i~sprawiły, że zaczęły swędzieć.
To uczucie było tak śmieszne, że lekko się uśmiechnęła.

-- Świetnie, prawda? Coraz lepiej jest być czystą istotą energetyczną.

-- Sprawiasz, że brzmi to jakbyś była duchem.

-- Podoba mi się ,,byt z~czystej energii'', ale jestem jedyna. To lepsze
niż duch. Nie zaczynaj nawet od ,,anioła''. Jezu kurwa Chryste.

Natalie znowu zapłakała. Beznadziejny świat wciąż się rozpadał. Chciała
mieć nadzieję, wierzyć w~Roz. Jednak była odchodniczką. Odchodzący
umieli stawić czoła brutalnej prawdzie. Brutalna prawda o~Roz była taka,
że bardziej prawdopodobne było to, że jej tata miał wrednego hakera,
który uruchomił kopię, żeby zdradzić Natalie, niż to, że ojciec
zapomniał zatrudnić nowego sysadmina, by przejął operacje dla jego
schronu.

-- Lodołasico, co powiesz na to? Nie musisz mi wierzyć. Ja też sobie nie
wierzę. Nie ma sposobu, żebym wiedziała, czy jestem tą, za kogo się
uważam. Logiczną rzeczą dla nas jest zachowywanie się tak, jakby nie
można było mi ufać.

-- To jest dziwne. -- Natalie pociągnęła nosem i~zajęła się problemem.

-- Dziwne nie jest przeciwieństwem sensu. Kiedy robi się dziwnie, dziwne
staje się profesjonalne.

-- Jeżeli tak mówisz.

-- Tak mówię. Och, poczekaj. Idą, czas na zaplanowaną rozmowę. Zamknij
oczy i~udawaj, że jesteś oszołomiona, co nie będzie naciągane. Przyczaję
się. Najlepiej, jeśli nie damy im znaku, że tu jestem, ale będę słuchała
i nagrywała. Kiedy odejdą, nadal tu będę, cokolwiek potrzebujesz, żeby
pozostać przy zdrowych zmysłach, dopóki nie zdołamy cię wyrwać.

Natalie nie mogła nic poradzić na to, że czuła, że to \textit{właśnie}
powiedziałaby zdradziecka Roz, gdyby próbowała oszukać Natalie. Ale
poczuła się dobrze.

Drzwi zaskrzypiały dwa razy, kliknęły i~otworzyły się.

\chapter*{iv}


Kiedy Seth i~Tam krążyli po pociągu towarowym, zastanawiał się nad swoją
dziwną relacją z~Gretyl. Kiedyś w~jego pięknych młodzieńczych czasach
dziewczyna zostawiła go dla innej kobiety, po tym, jak bzyknął się z~facetem na imprezie, czyimś napalonym, seksownym kuzynem. Spędzili
szaloną noc, zamknięci w~wolnej sypialni w~mieszkaniu kogoś mamy w~Bathurst Heights, zostawiając pościel tak śmierdzącą, że słyszał, że
została spalona. W~dalszej części dramatu, zakwestionował swoją
dziewczynę z~powodu jej szaleństwa, wskazując, że chłopcy to chłopcy, a~dziewczynki to dziewczyny, a~on był wyłącznie dla niej w~dziewczęcej
części swojego życia, ale nierozsądnie byłoby oczekiwać, że zrezygnuje z~penisa, gdy ona nie ma takiego.

Nawet gdy wypowiadał te słowa, część niego rozumiała je jako egoistyczne
bzdury. Dziesięć lat później nadal kulił się ze wstydu na myśl o~nich.

Dziewczyna znalazła inną dziewczynę, ponieważ jej tak kazał, i~szybko
zdecydowała, że ta druga dziewczyna jest osobą, z~którą chce być na
wyłączność, bez łagodzącego wyróżnienia wyłączność dla osób z~waginami,
na które nalegał Seth.

Seth, samotny i~kłujący, powiedział sobie, że to dlatego, że są rzeczy,
które można uzyskać z~relacji dziewczyna-dziewczyna, których nie można
uzyskać od dziewczyna-chłopak, a~on nigdy tego nie zrozumie, ale muszą
być niesamowite, ponieważ jego dziewczyna go rzuciła. Później zdał sobie
sprawę, że różnica między nim a~dziewczyną polegała nie tyle na penisie,
ile na ,,zdradzie\dywiz i\dywiz byciu\dywiz dupkiem''.

Kiedy Lodołasica wróciła do domu z~Gretyl, Seth był dojrzały w~tej
sprawie według standardów Setha. Kiedy jego zazdrość wzrosła, walczył z~nią, gorzko wspominając samooskarżanie się, które pojawiało się, gdy
myślał o~incydencie z~rozróżnieniem penis/brak penisa.

On i~Lodołasica nie mieli poważnego związku między chłopcem a~dziewczyną, więc nie miał prawa czuć zazdrości, nawet według zasad
defaultu, które mówiły, że \textit{są} chwile, kiedy człowiek jest
zazdrosny. Potem była Tam, która dobrze znała Gretyl, podziwiała ją,
podziwiała jej twardość i~niesamowite kawałki o~matematyce. Tam i~on
\textit{byli} tym, dziewczyną i~chłopakiem. Seth byłby niesamowicie
popieprzony, gdyby ścigał Lodołasicę.

Technicznie rzecz biorąc, wszyscy byli przyjaciółmi, niektórzy z~nich
się bzyknęli, inni zajmują się wyłącznie sobą na dłuższy czas. Kiedy
Lodołasica zniknąła, dręczyło ich niewiedzenie o~swoim
przyjacielu/kochanku/czymkolwiek. Zamienili się w~oddział partyzancki,
przeczesując sieć, pracując nad kontaktami, by ją znaleźć.

W miarę jak poszukiwania dobiegały końca, coraz częściej byli to Seth i~Tam, para i~Gretyl, właściwie wdowa, tocząca się razem na grzbiecie
frachtowca towarowego, patrząc niezręcznie, udając, że wszyscy mają ten
sam związek Lodołasicą i~czują ten sam rodzaj smutku. Tyle bzdur. W~końcu nie można było udawać.

Seth i~Tam szli obok pociągu towarowego, kierując się w~stronę Thetford,
mijając zniszczone strefy i~małe, defaultowe miasteczka ze sklepami i~ludźmi żyjącymi tak, jakby cywilizacja miała trwać wiecznie. Seth znał z~liceum francuski, ale ludzie, którzy wołali w~gwarowym \textit{Joual},
mogli mówić klingońskim. Pomimo bariery językowej, za każdym razem, gdy
przejeżdżali przez miasto, ludzie dołączali do ich kolumny. Przychodzili
nocą, gdziekolwiek obozowali. Nieuchronnie byli szleperami z~górami
śmieci, którymi Seth nie dawał się zirytować. Był królem Szleperów.

Gretyl jechała pociągiem, smutna twarz, oczami odległe, palce
przesuwające się po powierzchniach interfejsu. W~nocy Seth przyniósł
parujące tace z~mesy, zabrał je, kiedy skończyła, ale prawie tego nie
zauważyła.

W końcu Tam przewróciła się pewnej nocy i~położyła rękę na jego klatce
piersiowej, a~twarz w~zagłębieniu jego szyi i~powiedziała: 

-- Co ona,
\textit{kurwa}, robi? 

Nie wiedział, a~Tam wspomniała o~oczywistym fakcie
(którego był nieświadom), że martwiła się o~Gretyl.

-- Interwencja. Pierwsza rzecz z~rana.

-- Teraz -- powiedziała Tam. -- Założę się, że o~dwugodzinny masaż stóp, że
jest całkowicie przytomna.

-- Nie \textit{jestem }rozbudzony. AU! \textit{Teraz} jestem rozbudzony. -- Potarł sutek i~spojrzał na Tam w~ciemności. Miała ostre paznokcie.

Wciągnęli ubrania i~się oświetlili. Jesień przesuwała się ku zimie, a~na
drodze, na której rozbito obóz na noc, był szron.

Gretyl nie spała i~stukała, skulona, oświetlona księżycem sylwetka
oparta o~bok pociągu. Jej ręce tańczyły, a~jej szepty i~pomruki unosiły
się na wietrze. Nosiła maskę, której Seth nie widział wcześniej.
Bardziej niż ktokolwiek inny, wydawała się w~stanie wizualizować
wirtualne przestrzenie i~manipulować nimi bez wizualnego sprzężenia
zwrotnego. Zatem robiła coś intensywnego.

Akceptowalnym protokołem dla masek było najpierw zadzwonienie, aby
wiedziała, że tam jesteś, zamiast stukanie w~ramię i~niszczenie twórczej
mgły. Ale Gretyl ustawiła swoje flagi ,,nie przeszkadzać'', nawet flagę
bez wyjątków ,,to znaczy ty też''. Stali przez chwilę kilka kroków od
niej, zastanawiając się, co robić.

-- Czuję się jak idiota -- powiedział Seth. -- Mam na myśli, kurwa.

-- Nie stój tam z~kutasem w~dłoni. Poklep ją w~ramię.

Część Setha, który wciąż miał siedemnaście lat i~był napalony na
dziewczynę z~kutasem, zareagowała różnymi odpowiedziami na temat kutasów
i rąk, co stanowiło cały pakiet, jeśli chodzi o~Setha\dywiz Siedemnaście. Seth
powiedział Seth\dywiz Siedemnaście, żeby się, kurwa, zamknął.

-- Dlaczego nie Ty?

-- Ona cię \textit{lubi}. -- Tam go popchnęła. Gretyl okazywała zirytowane
matczyne uczucia i~zdezorientowany humor na temat różnych przygód Setha,
ale pod kątem, który sprawiał, że zaczął się zastanawiać, czy uważała go
za kolosalnego dupka.

-- \textit{Ciebie} też lubi.

-- Jesteś bliżej. -- Tam zrobiła szybki krok do tyłu.

Westchnął, a~Tam wysłała pocałunek, który zamienił się w~ruch
poganiania. Zbliżył się do Gretyl, której głowa się kołysała,
prawdopodobnie w~zgodzie z~jej zatyczkami do uszu, implantami, które
wypełniały jej uszy miękkim niebieskim światłem, aby inni wiedzieli, że
nie podziela ich akustycznej rzeczywistości konsensusu.

Mimo to odchrząknął, a~nawet dwukrotnie powiedział ,,Gretyl '' do jej
ucha -- mając nadzieję, że zrobiła rozsądną rzecz i~zaprogramowała dostęp
przez zatyczki -- zanim niepewnie dotknął jej ramienia. Tak jak się
obawiał, szarpnęła się, jakby ją dźgnął, zdjęła maskę i~spojrzała
wściekle.

-- Czy jesteśmy atakowani? -- spytała.

-- Nie, ale\ldots 

-- Odpierdol się. -- Nasunęła maskę. Tam potrząsnęła głową i~ponownie go
pogoniła. Zanim zdążył stuknąć, Gretyl zdjęła maskę. -- Seth, nie byłam
subtelna. Robię coś, co wymaga koncentracji. Dlaczego nie odpierdoliłeś
się, zgodnie z~moją instrukcją?

Spojrzał na Tam. Gretyl też na nią spojrzała i~zmiękła o~jedną
miliardową procenta.

-- Czego wy dwoje chcecie?

Tam ujęła ręce Gretyl, ciężkie od pierścieni interfejsu. 

-- Gretyl,
chcemy porozmawiać o~Lodołasicy.

Gretyl przekrzywiła głowę. 

-- Tak?

-- Nie ma jej od ponad tygodnia. Wszyscy mamy nadzieję, że się pojawi.
Umieściliśmy słowo wśród odchodzących i~w defaulcie, ale niepokój jest
bezużyteczny. Jest mądra i~zaradna i~tak długo, jak jesteśmy osiągalni,
skontaktuje się, gdy tylko będzie mogła.

Gretyl uśmiechnęła się, co zaniepokoiło Setha. Zrobił pół kroku do tyłu
pod pretekstem siadania na tyłku na ziemi naprzeciwko Gretyl. To był
dziwny uśmiech.

-- To wszystko?

-- Nie. -- Tam usiadła obok Setha. -- Nie, \textit{nie} jest, Gretyl. Musisz
zrozumieć, że jesteśmy twoimi przyjaciółmi, kochamy cię, jesteśmy po
twojej stronie, jesteśmy w~tym razem. Wszyscy za nią tęsknimy. Musimy
się wspierać, a~nie zamykać we własnych kątach i\ldots 

Zatrzymała się, bo uśmiech Gretyl był teraz szerszy. 

-- Gretyl? -- spytała
Tam.

Gretyl westchnęła ciężko i~wstała, górując nad nimi. Sięgnęła na pomost
pociągu towarowego i~znalazła elastyczną buteleczkę ze smoczkiem, z~której wzięła długi łyk, po czym podała ją Tam, która powąchała, potem
wypiła i~podała Sethowi, który stwierdził, że jest pełna czegoś jak
szkocka, tak torfowa, że przypominało picie cygara. Lubił pić cygara.
Nabrał większy łyk, niż zamierzał, a~potem wykorzystał nieprzyjemną
sytuację.

Gretyl wyciągnęła rękę, a~on niechętnie oddał alkohol. 

-- Za Lodołasicę. -- Wzięła kolejnego łyka.

Oboje skinęli głowami. Patrzenie na Gretyl sprawiało, że Setha skręcało
w szyi. Wstał w~chwili, gdy Gretyl usiadła i~spojrzała na niego wzrokiem
,,tak pieprzę się z~tobą'', który widział, jak posyła innym ludziom.

-- To bardzo miłe z~waszej strony. Macie dobre zamiary. Ale nie
wzdychałam dramatycznie. \textit{Coś} robiłam.

-- Co? -- Oczy Tam błyszczały w~miękkim świetle jej świecącego ubrania,
podkreślając jej silną szczękę i~sprawiając, że jej skóra była kałużą
maślanych tonów w~szaro-czarnej nocy. Seth poczuł dreszcz podniecenia,
częściowo podniecenia seksualnego, a~częściowo samego
\textit{podniecenia}. Coś się działo.

-- Uruchomiłam Roz. W~Akron jest tak wiele klastrów. Mnóstwo czasu
obliczeniowego, którym ludzie chętnie się dzielą. Uruchomiłam ją i~powiedziałem jej, że Lodołasica została porwana przez jej rodzinę, a~ona
pogadała z~typami ninja, którzy są dobrzy w~tego typu rzeczach.

-- Tak? -- powiedziała Tam, spokojniej niż Seth, który miał ochotę
rozmawiać z~Roz, nie chodziło o~to, że nie wydawała się człowiekiem.
Chodziło o~to, co \textit{robiła}. Przerażała go.

-- Tak.

Gretyl wyglądała jakby czekała.

-- Spróbuję -- powiedział Seth. -- Co się stało?

-- Znaleźliśmy ją. Zdobyliśmy dom, w~którym jest. Roz działa na ich
sprzęcie. Jest w~kontakcie z~Lodołasicą.

Seth i~Tam spojrzeli na siebie.

-- Nie zwariowałam -- powiedziała Gretyl. -- To jest prawdziwe i~to się
dzieje.

-- Kiedy?

-- W~zeszłym tygodniu. Teraz nic się nie dzieje, dopóki nie odzyska
przytomności.

-- Odzyska przytomność? -- spytał Seth.

-- Serio? -- powiedziała Tam.

-- Odzyskuje przytomność. -- Wyraz twarzy Gretyl sprawił, że się
wzdrygnął. -- Naprawdę. -- Jej uśmiech był tak duży, że jej oczy prawie
zniknęły między jej policzkami i~czołem. -- Naprawdę! -- Tam, która
wiedziała, co robić w~sposób, którego nigdy nie znał Seth, uściskała
Gretyl, do czego dołączył.

-- Co teraz? -- spytał Seth.

-- Teraz ją wyciągniemy -- powiedziała Gretyl.

\chapter*{v}

Etcetera nie wiedział, czego się spodziewać po Thetford, ale to nie było
to. Strefa została opuszczona dekadę wcześniej, kiedy skażenie azbestem
stało się krytyczne i~nawet rząd federalny nie mógł ignorować. Ewakuacja
odbyła się ze zwykłym pośpiechem i~przymusem. Domy wciąż miały porcelanę
w szafkach, zabawki w~skrzyniach na zabawki, rdzewiejące huśtawki na
podwórkach.

Ciepłe zimy i~mokre lata wywołały osuwiska, które spowodowały zamulenie
miasta i~doliny, a~budynki były gąbczaste od czarnej pleśni. Bardzo
suchy rok zakończony letnią burzą wywołał pożary w~całej dolinie, potem
nastąpiły kolejne powodzie. To, co pozostało, wyglądało jak tysiącletnia
ruina, aczkolwiek z~dziwnymi zakamarkami doskonale zachowanego
wiejskiego życia, wiejski dom, który uniknął najgorszego i~wciąż miał
regał pełen starych francuskich romansów, zespół piwnic i~podpiwnic pod
szpitalem, które były suche z~działającymi światłami awaryjnymi.

Odchodzący, którzy przejęli Thetford, traktowali je jak wrogą, obcą
planetę, na której powietrze mogło cię zabić, gdzie teren był zdradliwy,
a ekstremalny klimat nie okazywał litości. To było dokładnie to
środowisko, którego szukali, ponieważ była to próba generalna przed
udaniem się na inne planety.

-- To ostateczne odejście -- powiedział Kersplebedeb. Był chudy, z~wybitnym jabłkiem Adama i~mówił po angielsku z~zabawnym akcentem, który
pochodził od francuskiej matki i~ojca Kiwi w~dwujęzycznym Montrealu. -- Wszystkie te pierwsze dni lepszego narodu to tylko burżujskie bzdury.
Narody to bzdury. Wiesz, co nie jest bzdurą? Kosmos. Nie ma miejsca na
walki o~władzę w~kosmosie. Nie ma miejsca na przymus ani wojnę.

-- Opowiedz mi to jeszcze raz? -- Znajdowali się w~jednej z~hermetycznych
kapsuł, które zostały złożone w~całym Thetford, niczym worki po jajach
kałamarnic podniebnych. -- Dlaczego nie ma wojny?

-- Dlaczego wojna? -- powiedział Kersplebedeb. Rozłożył długie palce na
stole. Miał wyszczerbiony srebrny lakier do paznokci, żółtą podomkę i~krótkie włosy, co rozwiewało wszelkie obawy Etcetery, że Thetfordowie
będą społecznie nudni i~konserwatywni. To była reputacja typów badaczy
kosmosu.

-- Zazdrość. Chciwość. Irracjonalna nienawiść.

-- Kiedy jesteś w~kosmosie, jesteś \textit{mobilny}. Nieograniczona moc,
wszędzie tam, gdzie świeci słońce. Tlen wszędzie, gdzie można znaleźć
lód do frakcjonowania za pomocą elektrolizy zasilanej energią słoneczną.
Jedzenie wszędzie tam, gdzie możesz znaleźć surowiec, w~tym odchody.
Ktoś chce twoją bryłę lodu? Odchodzisz. Ktoś chce Twój habitat
kosmiczny? Odchodzisz. Odchodzisz, odchodzisz.

-- Ludzie, którzy myślą o~kosmosie, w~końcu myślą o~bzdurach, takich jak
\textit{Gwiezdne Wojny} i~\textit{Star Trek }. Podróżują szybciej niż
światło, ale nadal walczą? O \textit{co}? Mają \textit{transportowce}. O co
oni walczą? Co ma ktokolwiek, czego nikt inny nie może dostać od razu,
za darmo? Muszą wynaleźć unobtanium, magiczne kryształy, których z~jakiegoś powodu nie można po prostu wydrukować za pomocą ich
transporterów, albo inaczej nie ma opowieści.

-- Dlaczego w~ogóle \textit{umierają}? Już robimy skany samych siebie, a~jeśli mają transportery, powinni robić skany co godzinę!

-- Wiem, o~co Ci chodzi. -- Żałował, że nie ma tam Limpopo, ale pojechała
na szkolenie w~fabryce skafandrów kosmicznych wraz z~kontyngentem
naukowców z~Uniwersytetu Odchodzących i~kilkoma pracownikami B\&B.
Mówiono o~budowie kolejnej fabryki, ponieważ wszyscy byli ograniczeni do
tuneli, dopóki nie zostaną wyposażeni. Akademicy, którzy żyli w~tunelach, przyjęli to z~rezygnacją i~głównie chcieli przestrzeni, czasu
i wolności od rozpraszania się, żeby móc skanować wszystkich. To było w~porządku dla wszystkich. Długi marsz do Quebecu był pełen
niebezpieczeństw. Dźwięk z~nieba sprawiał, że wzdrygali się w~oczekiwaniu na śmierć od drona. Każdy trzask w~nocy był najemnikiem.
Kwestia przesłania każdego odchodzącego do chmury nie mogła być
ważniejsza.

Załoga B\&B i~pozostali przy życiu aeronauci chcieli, aby naukowcy
skupili się na projekcie skanowania, w~pomieszczeniu, zabezpieczonym
przed wdmuchiwaniem azbestu i~wypłukiwaniem metali ciężkich w~Thetford;
sami chcieli wypierdalać z~tuneli. Odchodzący, którzy nie mogli odejść,
przypominali lisy, których legowisko nie miało awaryjnych tylnych drzwi.
Projekt skafandra kosmicznego był priorytetem. Załoga Thetford
wprowadziła ulepszenia w~fabryce skafandrów kosmicznych, z~którymi nie
mogli się doczekać, aby przejść do wersji 2.0, więc prawdopodobnie
doprowadzi to do wystartowania.

Kersplebedeb roześmiał się, pokazując końskie zęby i~wnętrze nozdrzy. 

-- Wy ludzie mnie dobijacie. Zrobiliście tak wiele dla projektu, ale wydaje
się, że nie zastanawialiście się nad tym, jak to zmienia
\textit{wszystko}. W~tempie, w~jakim się poruszamy, do Nowego Roku
wystrzelimy w~kosmos tysiąc odchodzących.

-- Gdzie planujesz uzyskać możliwość wyniesienia ładunku, aby umieścić
kolonię na orbicie? Ostatnim razem, gdy sprawdzałem twoje wiki, miałeś
umowę, by podnieść kilka CubeSatów rocznie.

-- Potrzebujemy tylko \textit{jednego} satelity, wysoko z~przyzwoitą
łącznością ze stacją naziemną i~gotowe.

Wydało się. 

-- Chcesz uruchomić klaster na orbicie i~umieścić na nim
symulacje?

Kersplebedeb spojrzał na niego wzrokiem ,,no ba'' i~pogrzebał w~lodówce
w poszukiwaniu słoika bimbru astronautów, zrobionego z~destylowanych
porostów. Smakował niesamowicie jak lekko słodka tequila, zwodniczo
gładka i~bardzo mocna. Odkręcił pokrywkę słoika i~nalał dwie małe
szklanki zielonkawego płynu. Te spotkania z~Kersplebedebem obejmowały
dużo alkoholu z~porostów. Był to przedmiot teoretyczny z~programów
kosmicznych dla odchodzących. Był tani i~łatwy do wykonania, nawet jeśli
nie masz twardej próżni tuż za śluzą.

-- Co jeszcze możemy zrobić?

-- Co oni tam zrobiliby?

-- To samo, co robimy tutaj, ale z~dala od ludzi z~bombami i~dziwnymi
pomysłami co do robienia tego, co ci kazano, i~akceptowanie Twojego
miejsca w~świecie.

-- Zamierzasz uruchomić kopie siebie w~kosmosie, na mikrosatelicie i~co,
wymieniać z~nimi e-mailami? Niech toczą wojny z~wysokim opóźnieniem
dotyczące problemów inżynieryjnych?

-- Przyznam, że to dziwne. -- Sączył drinka i~jego wpływ stał się mniej
dziki, bardziej\ldots  Etcetera myślał o~słowie. Default. Bardziej rozsądny,
bardziej szanowany. W~pewnym momencie życia Kersplebedeba, był on typem
osoby, która potrafiła wyjaśnić sali zarządu normalnych ludzi i~sprawić,
by brzmiało to normalnie. Teraz pokazywał swój ,,normalny'' rejestr dla
Etcetery. -- Rzeczy -- machnął rękami -- zbliżają się do głowy. Zetty
szaleją.

-- Zetty zawsze szaleją. To właśnie robią. Martwią się, czy mają więcej
niż wszyscy.

-- Nie o~tym mówię, Ets. -- Tak nazywał się Etcetera dla Kersplebedeba.
Jak na faceta o~imieniu Kersplebedeb, Kersplebedeb niecierpliwił się
przy wielosylabowych imionach innych ludzi. Wszyscy inni dostali jedną
sylabę. -- To podstawowy niepokój społeczny, który podtrzymuje działanie
maszyn defaultu. Ale przez ostatnie trzy pokolenia, zetty powiększały
swoje rodziny. Kiedyś tylko jeden dzieciak był stratosferycznie bogaty.
Pozostali byliby biednymi bogaczami. Nigdy nie będą biedni, ale nie
zmienią kierunku narodów. Są o~dwa rzędy wielkości biedniejsi od
najstarszych.

-- Pieniądze są względne. Kiedy twój starszy brat staje się sto razy
bogatszy od ciebie, oznacza to, że jego dzieci mogą lecieć na orbitę na
przerwę świąteczną, zjeść kolację z~prezydentami, podczas gdy Twoje
dzieci po prostu chodzą do Eton lub UCC i~nurkują na wielkich
głębokościach, zamiast kosmicznych strzałów. Jedzą kolację z~zawodowymi
sportowcami i~gwiazdą pop, która gra na ich piętnastych urodzinach.
Dzieciak numer dwa starszego brata kończy tak jak ty i~nie jest z~tego
zadowolony, bo wcześnie się o~tym dowiaduje. Wypacza go tak, jak
wypaczyło to Ciebie. Psuje rodzinę od środka.

-- 0,001 procent \textit{może} wypączkować trzy fortuny, rozgałęziając
dynastie dla całego potomstwa. To pogarsza sytuację, ponieważ kiedy
jesteś zazdrosny o~swojego brata, to jest to zło ze Starego Testamentu.
Kończy się słowami ,,uciekinierem i~włóczęgą będziesz na ziemi''.

Etcetera wyglądał na zaskoczonego. 

-- Kain i~Abel -- powiedziała
Kersplebedeb. Etcetera wyszeptał bezgłośnie ,,Och''i machnął ręką. Wypił
spory łyk soku z~porostów i~był pełen dobrej woli.

-- Gra końcowa: nawet \textit{tym} zettom skończyły się nowe terytoria do
podboju, aby dalej geometrycznie powiększać swoje fortuny. Nie ma już
nic, co można by z~nas wycisnąć. Kapitał posiadany przez osoby niebędące
zettami spadł do znikomego. Gdyby jakiś zdesperowany zetta wymyślił, jak
to \textit{wszystko} skonfiskować, nie dostałby nawet jednego posagu dla
swojego dzieciaka numer dwa.

-- Zatem zwracają się przeciwko sobie?

-- Siedzieliśmy już na tym filmie. -- Kersplebedeb dotknął nosa gestem,
który Etcetera ostatecznie rozpoznał jako oznaczający ,,dokładnie''. -- W~dziewiętnastym wieku bogaci utrzymywali ten sam schemat, po jednym
dziecku z~każdej rodziny dostawało nazwisko i~majątek, wszyscy inni
stawali się wygodnym niebytem, a~jeśli mieli szczęście, poślubili kogoś
numer jeden. Potem nadeszła era kolonialna, nowe światy do splądrowania
i bam, ekspansja geometryczna przez dwa pokolenia, wystarczająco długo,
aby nie było nikogo, kto pamiętałby czasy, gdy dynastia była prostą
linią zamiast rosnącego drzewa fortuny.

-- Co się stało?

-- Zabrakło im kolonii -- powiedział Kersplebedeb.

-- Co się stało, kiedy się skończyły?

-- Och! -- Kersplebedeb pociągnął długi łyk. Westchnął, gdy poruszało się
jego jabłko Adama. -- Wybuchła Pierwsza Wojna Światowa. Zwrócili się
przeciwko sobie.

\chapter*{vi}

Limpopo rozciągnęła ramiona w~skafandrze ekologicznym. Był to model
czwartej generacji, prosto z~drukarki i~zapięty wokół jej ciała przez
kosmika z~Thetford, który robił anachroniczne uwagi do giermków i~rycerzy. Kiedy go zapytała, wzruszył ramionami i~powiedział: 

-- Sci-fi i~fantasy to dwie strony tej samej monety. 

Mówił nosowo, który mógł być
Teksańczykiem i~wyglądał, jakby mógł być Wietnamczykiem. Kosmicy
pochodzili z~całego świata. Patrzyli wizjonerskim, dzikim wzrokiem,
który wyróżniał ich, nawet jak na standardy odchodzących, gdzie szalony
wzrok towarzyszył pracy.

Skafander był sztywny, ale nie straszny. W~przegubach było doładowanie
hydrauliczne, które pomagało się mu utrzymać, dało mu własną siłę, jak
mniejszy mecha-ładowacz. Zamówiła swój ze skórą mozaiki van-art hobbitów
i elfów, które wybrała z~katalogu, i~zafascynowana obserwowała, jak
algorytm wymyślił, jak zmienić ich rozmiar i~ułożył mozaikę, aby zakryły
cały kombinezon bez żadnych niedopasowanych krawędzi.

Była raz na zewnątrz, odkąd przybyli, przewieziona przez samochód\dywiz bańkę
do pokoju z~nadmuchanego zamku, którego używali jako wspólnej
przestrzeni. Tamtym razem przyszła w~wypożyczonym garniturze,
generacja-2, i~było tak gorąco i~niezgrabnie, że okrążyła jeden ze
zrujnowanych domów i~wróciła z~maską na twarzy z~zasłoniętą przez
skondensowaną parę i~z zadrapaniami.

Teraz miała specjalnie dopasowany generacja-4, była gotowa spróbować
ponownie. Mieli w~domu zasadę, że chodzili parami i~wiedziała, że Sita
walczyła o~wyjście na zewnątrz. Zapoznały się podczas długiego marszu i~pracy w~prowizorycznym ambulatorium po zestrzeleniu \textit{Lepszego
Narodu}. Obie były przerażone i~podekscytowane wściekłością, która
płonęła w~Sicie. To była zwykła bezwzględność w~jej pragnieniu obrony
odchodzących. Przejęła obronę kolumny, ustawiając drony w~wirujący wzór,
pracując wieczorami, by ładować i~sprawdzać ich broń, głównie pulsacyjną
broń dźwiękową i~energetyczną, chociaż był tam duży, dziwny pocisk,
karabin szynowy, który przynieśli z~Uniwersytetu Odchodzących, a~następnie odholowany od Gospody.

Teraz osiedlili się w~Thetford Space City, Sita kierowała projektem
uruchamiania skanerów neuronowych, zapewniając wsparcie administracyjne
i pomoc w~pracy. Jej własne wykształcenie -- lingwistyka obliczeniowa -- nie miało praktycznego zastosowania w~tej części projektu, więc gdy
wszystko zaczęło szumieć, nie miała nic do roboty poza dostarczaniem
gorących napojów prawdziwym ekspertom i~dostała kuku na muniu.

Strój Sity był wyłożony kamuflażem leśnym, który składał się z~tysięcy
zniekształconych twarzy o~dziwacznych wyrazach. Spojrzenie na niego
sprawiło, że oczy Limpopo zaszły łzami.

-- Gotowa? -- powiedziała Sita przez sieć p2p. Była szyfrowana,
wykorzystywała wiele pasm do redundancji i~miała inteligentną telemetrię
w swoich radiach, która również wykrywała zakłócenia i~używała ich do
wywnioskowania o~stanie środowiska elektromagnetycznego, co pozwalało
przezwyciężać burze elektryczne. Głos Sity był tak wyraźny, tak pięknie
zrównany z~dźwiękami wiatru otoczenia i~brzękiem wiatraka, tak dobrze
skorygowany w~obuusznej przestrzeni, że brzmiała jak postać z~gry.

Limpopo uniosła kciuki i~wcisnęła przycisk śluzy. Stłoczyły się razem, a~ona dostała łokciem Sity w~bok, który sprawił, że kostium ocierał się o~jej bliznę w~sposób, który nie był całkowicie nieprzyjemny. W~czasach,
gdy tak wielu otaczających ją ludzi traktowało swoje ciała jako
niewygodne kombinezony, których musieli używać jako mechów do noszenia
wokół cennych mózgów, miło było mieć kawałek jej tożsamości, który był
nierozerwalnie związany z~jej ciałem.

Jej ostatnia podróż przez śluzę powietrzna była mieszanką utrudnionego
ruchu, otartej skóry i~słabej widoczności. Teraz, wchodząc w~wysoką,
kruchą dziką trawę wystającą ze śniegu, z~szafirowym daszkiem tak
przejrzystym, że przypominało interfejs użytkownika, w~komplecie z~odblaskiem obiektywu, uderzyło ją piękno tego miejsca.

Sita pchnęła ją od tyłu. 

-- Stara, nie blokuj drzwi.

-- Przepraszam. -- Zrobiła krok w~bok. Drzewa były wysokie i~pełne igieł,
śnieg tak puszysty, niebo pełne dramatycznych chmur. -- Tylko trochę\ldots 

-- Trudno pamiętać, że to przeklęte pustkowie, kiedy jest tak piękne.
Powinnaś zobaczyć dziką przyrodę. Łoś, jeleń, a~nawet żbiki, sądząc po
odciskach łap i~pazurów. I~ptaki! Oczywiście sowy, ale tak wiele
zimowych ptaków, można by przysiąc, że migracja była miejską legendą.

-- Jak?

Sita przedzierała się przez śnieg, z~każdym krokiem zapadając się po
kolana. Limpopo szła po jej śladach, podziwiając skafander
odprowadzający wilgoć z~jej pleców.

-- Żadnych ludzi. Tak jest w~okolicach Czarnobyla. Okazuje się, że w~stosunku do dzielenia się biomem z~ludźmi, życie w~cieniu radioaktywnego
pióropusza lub w~miejscu, w~którym brud i~powietrze to w~czterdziestu
procentach azbest, to dobry interes.

-- Tak to ujęłaś, jakbyśmy byli zarazą.

-- Co to znaczy \textit{my}, Biały Człowieku?

Znała dowcip: ,,Tonto, jesteśmy otoczeni przez Indian!'' ,,Co to znaczy
\textit{my}, Biały Człowieku?''. Chociaż nigdy nie czytała książki o~Lone
Ranger, nie grała w~grę, nie widziała kreskówek czy czegokolwiek, ale
zajęło jej chwilę, zanim zrozumiała znaczenie Sity.

-- Naprawdę? Kto chce ciała, jest gorszy od azbestu?

Sita się zatrzymała. Śnieg był ponad jej kolanami. Naprawdę musiała
pracować, aby nadążyć za tempem. Limpopo usłyszała w~słuchawkach jej
ciężki oddech. 

-- Pozwól mi złapać oddech. -- Potem. -- To dość oczywiste. Ilość rzeczy, które konsumujemy, aby przetrwać, jest szalona. Prorocy
końca czasu zwykli przewidywać nasze poziomy konsumpcji do przodu,
pomnażając naszą populację przez nasze potrzebne zasoby, i~doszli do
tego punktu, w~którym zabrakłoby nam planety w~ciągu jednego pokolenia i~nastąpiłby głód i~wojny.

-- Ten rodzaj liniowej projekcji to sposób myślenia, który wpędza ludzi w~kłopoty, gdy myślą o~przyszłości. To jak myślenie: ,,Moje dziecko uczy
się co tydzień dziesięciu ekscytujących nowych rzeczy, więc zanim
skończy sześćdziesiąt lat, będzie mądrzejsza niż jakikolwiek człowiek w~historii''. Jest wiele krzywych, które zaczynają wyglądać tak, jakby
szły w~górę i~w prawo w~nieskończoność, ale zamieniają się w~krzywe
dzwonowe, odwrócone U, lub S lub legendarny kij hokejowy, który staje
się coraz bardziej stromy i~stromy, aż przechodzi prosto w~pion.
Wszelkie założenie, że skończymy tak jak teraz, ale bardziej, jest
niewystarczająco dziwne, to jedyna rzecz, której możesz być pewien, że
się \textit{nie wydarzy }w przyszłości.

Limpopo spoglądała na niebo z~pędzącymi chmurami, słuchała grzechotu
drzew. Temperatura jej skafandra była idealna, nie-ciepła i~nie-chłodna,
której nie zauważyłabyś, gdyby wokół ciebie nie było minus dwadzieścia
stopni. 

-- Myślałam, że załoga B\&B lubi ostre dyskusje, ale potem
spotkałam was, naukowców. Cholera, lubisz poszerzać ramy.

Limpopo zobaczyła, że ramiona Sity lekko się trzęsą i~przez chwilę
spanikowała, że nieumyślnie doprowadziła Sitę do łez, co nie jest
niespotykane wśród odchodzących, gdy wszyscy mają ukryte wyzwalacze
traum, o~które można się potknąć przez przypadek.

Kiedy przedarła się przez śnieg i~spojrzała na jej hełm, zobaczyła, że
Sita śmieje się w~milczeniu, wpatrując się nieruchomo przed siebie.
Kiedy podążyła za wzrokiem Sity, zobaczyła, że patrzy na nich łoś z~porożem przynajmniej tak szerokim jak ona wysoka.

-- Wielki łoś -- szepnęła.

-- Ciii -- powiedziała Sita, śmiejąc się z~czkawką.

Limpopo wykonała gest ręką, który dodał do zakładek nagranie wideo
garnituru dla ciekawości, a~miękkie czerwone światło zapulsowało w~prawym górnym rogu jej przyłbicy. Łoś przyglądał im się przez chwilę. Na
kolanach miał wytarte miejsca na futrze. Jego kudłate futro błyszczało
kryształkami lodu. Para wylewała się z~jego nozdrzy w~pióropuszach,
które wirowały na wietrze. Szczęka była uchylona, przez co wyglądał na
komicznie oszołomionego, ale kiedy spojrzała w~jego ogromne oczy,
zobaczyła niewątpliwą przenikliwość. Ten łoś nie był głupcem.

Łoś poruszył się i~duże gówno opadło na śnieg, topiąc się i~znikając,
pozostawiając za sobą parującą dziurę. Śmiały się z~nieoczekiwanego
chamstwa. Rzucił im to spojrzenie, które Limpopo odczytało jako ,,Och
dorośnij'', choć to było antropomorfizowanie. Przemieszcił się w~szerokim kręgu, niezgrabne nogi kołysały się we wszystkich kierunkach,
ale jakoś nie wchodził w~krater własnego łajna, odwrócił szeroki tyłek
do nich i~odszedł -- nie, odspacerował -- kołysząc biodrami, krokiem,
który był czystym ,,mam-to-w-dupie''.

Oboje wybuchnęły śmiechem. Toczył się dalej, zamieniając się w~chichot,
który rykoszetował między nimi. Ilekroć jedno z~nich zaczynało się
przestawać, drugie znów wprawiało w~ruch.

-- Mów, co chcesz o~ciałach -- powiedziała w~końcu Limpopo -- z~pewnością
są zabawne.

-- Zgoda.

-- Chodźmy. -- Limpopo objęła prowadzenie. Przed nimi był drzewostan
brzozowy, wielkie drzewa z~lokami białej kory odrywającymi się jak
naskórek błagający o~zerwanie. Limpopo pamiętała swoje dni po pożarze,
żyjąc z~ziemi. Straciła kuchenkę gazową\dywiz generator i~została zmuszona do
budowania ognisk, podsycając je strzępami brzozowej kory. Doznała traumy
i urazy, ale jej dosłowny czas na pustkowiu zapewniał jej refleksyjny
spokój, powolną satysfakcję za każdy przeżyty dzień, za którym tęskniła
od tamtej pory.

-- Praktycznie \textit{słyszę}, o~czym myślisz.

-- To znaczy? -- Limpopo poprowadziła ich obok brzozy do szybko
poruszającego się lodowego potoku, wokół którego widniały ślady wielu
gatunków. Niepewnie weszła do rwącej wody, czując, jak delikatnie masuje
przez izolację skafandra. Powierzchnie chwytne na podeszwach jej butów
szybko ją przyczepiły do koryta strumienia. Ze środka potoku widziała z~pewnej odległości zbocza w~górę i~w dół. Podziwiała wzgórza nad sobą,
dolinę poniżej.

-- Zastanawiasz się, jak te wszystkie piękne rzeczy udowadniają, że życie
w wirtualnym środowisku nigdy nie byłoby naprawdę satysfakcjonująco
ludzkie.

-- Nie myślałam o~tym teraz, ale to z~pewnością coś, o~czym
\textit{pomyślałam}.

-- Mądrala. -- Sita wmanewrowała się w~koryto strumienia, znajdując
głębsze miejsce, opadając na kolana. -- To jest piękne, bez wątpienia.
Bycie stymulowanym tym poglądem i~tym środowiskiem jest głęboko
satysfakcjonujące.

Limpopo powstrzymała się od powiedzenia : \textit{Zgadzamy się więc,
chodźmy na spacer}, ponieważ ten rodzaj nonszalancji był bardziej
wydziałem Setha i~ponieważ to męczyło Sitę.

-- Kontynuuj.

-- Po pierwsze, chciałabym, żebyś rozważyła, że nasza reakcja jest
wyznacznikiem czegoś, co moglibyśmy nazwać ,,dobrocią'' lub
,,słusznością''.

-- Lub ,,pięknem''?

-- Pewnie. Istnieje wiele lingwistyk obliczeniowych na temat różnicy
między pięknem a~dobrocią. Nie proponuję, żebyśmy schodzili do tej
króliczej nory, ale to wymaga dalszej dyskusji.

-- Odnotowane.

-- Dobrze. -- Przepłynęła na drugą stronę i~weszła do sosnowego zagajnika,
gdzie drzewa pochylały się w~stronę otwartego nieba nad urwiskiem. 

-- Chodź. -- Teraz była na czele, zmierzając pod górę, i~Limpopo zrozumiała,
że przed nią jest opuszczona droga, wijąca się serpentynami pod górę.
Była pokryta śniegiem i~zastanawiała się, czy jest jakiś sposób, aby
przymocować narty biegowe do kombinezonu środowiskowego, ponieważ to
wyglądało na cholernie trudne.

-- To jest piękne, dobre i~cnotliwe. Bez nas jest najbardziej płodny i~zdrowy. Zatem najlepszym ludzkim postępowaniem jest nieobecność, robić
to, co robili pierwotni Thetfordczycy, ale na wielką skalę. Ewakuować
planetę.

-- Hmm.

-- Pomyśl o~tym przez chwilę. Nie mówię o~zbiorowym samobójstwie. Mówię o~zrównoważeniu naszych potrzeb materialnych z~naszymi estetycznymi lub,
jeśli chcesz to tak nazwać, naszymi \textit{duchowymi} potrzebami.
Bylibyśmy poważnie załamani, gdyby zniknęła cała dzicz. Dbamy o~Ziemię i~rzeczy, które tu żyją, ponieważ współewoluowaliśmy z~nimi, więc nasze
mózgi są wytworem milionów lat selekcji, która budziła podziw i~satysfakcję z~tego rodzaju miejsca.

-- Jednocześnie jesteśmy konsumpcyjnymi drapieżnikami ze skłonnością do
samoewolucji. Załadowaliśmy Łysenkę w~Darwina.

-- Straciłaś mnie.

-- Łysenko. Radziecki naukowiec. Myślał, że możesz zmienić plazmę
zarodkową organizmu, fizycznie zmieniając organizm. Jeśli odciąć żabie
nogę, a~następnie odciąć nogi jego potomstwa, a~następnie nogi
\textit{tego} potomstwo, w~końcu można dostać linię naturalnie trzynogich
żab.

-- To głupie.

-- To było atrakcyjne dla Stalina, który uwielbiał ideę kształtowania
pokolenia i~odciskania zmian na swoich dzieciach, co się zdarza, ale nie
genetycznie. Jeśli nauczysz pokolenie ludzi, że muszą źle traktować
sąsiadów, aby przetrwać, tworząc społeczeństwo, w~którym zostają tacy
ludzie, dzieci tych ludzi nauczą się zdradzać sąsiadów od kołyski.

-- Brzmi znajomo.

-- To był tylko początek. Stalin upierał się, że mogą uodpornić pszenicę
na warunki atmosferyczne, uprawiając ją w~gównianych warunkach. To się
źle skończyło. Głód. Miliony zabitych.

-- Ale teraz możemy\ldots  hm\ldots  ,,zhakować łysenkizm''?

-- Mamy zarówno cechy kulturowe, jak i~genetyczne. Przekazujemy je dalej.
Kiedy wymyślamy społeczeństwo takie jak default, wybiera ono ludzi,
którzy są marnotrawnymi palantami, którzy odnoszą sukces, dźgając
sąsiadów w~plecy, mimo że naszym priorytetem dla całego gatunku jest
niewyginięcie w~wyniku katastrofy środowiskowej, pandemii i~wojna.

Wspinały się coraz wyżej na wzgórze. Śnieg był równie głęboki, ale nie
było drzew, których trzeba by unikać, więc szło się łatwiej. Mimo to
Limpopo, ku jej zakłopotaniu, robiła się zdyszana. Sita, piętnaście lat
starsza, nie wykazywała żadnych oznak zwalniania, więc Limpopo
przełknęła jej dumę i~wezwała do odpoczynku. Znajdowali się za linią
drzew i~widzieli ponad nią głęboko w~niecce, dziwny tunelowy krajobraz
przestrzeni, gnijące domy i~farmy skolonizowane przez małe drzewa, które
właśnie przebijały śnieg.

-- O! -- powiedziała Limpopo, nie poddając już niczemu pracowitych płuc.

-- W~rzeczy samej. A więc łysenkizm. Dzięki symom możemy sprawić, by
łysenkizm zadziałał. Pomyśl o~Roz w~jej ograniczonach kopercie.
Zrobiliśmy jej pranie mózgu, albo pomogliśmy jej wyprać mózg samej
sobie, żeby nie miała nic przeciwko byciu symulacją.

W żołądku Limpopo pojawiło się uczucie zimna. Spojrzała na Sitę z~przerażeniem. 

-- Nie mówisz o~przekształceniu ludzi w~symy, których nie
porusza naturalne piękno?

Sita spojrzała przez maskę na twarzy. 

-- Och dziewczyno, \textit{nie}.
Jezu, myślisz, że jestem potworem? Moglibyśmy ograniczyć nasze symy do
przestrzeni, w~których cenimy naturę tak bardzo, że wolimy być
bezcielesni i~nie być siłą niszczącą, niż doświadczać jej bezpośrednio.

-- To po prostu dziwne.

Znowu ruszyły. Dwa serpentyny później Limpopo powiedziała: 

-- Myślę, że
to mam. To jest popieprzone gówno.

-- Od setek lat ludzie próbowali skłonić wszystkich do łagodnego życia na
lądzie, ale ich całe podejście brzmiało: ,,nie ruszaj się i~staraj się
nie oddychać''. To wszystko było powierzchowne, bez chwały w~pięknie
natury. Receptą środowiskową było zachowywanie się tak bardzo, jak to
możliwe, jakbyś już był martwy. Nie rozmnażaj się. Nie konsumuj. Nie
depcz ziemi, bo ugniatasz brud i~zabijasz rośliny. Każdy wydech zatruwa
atmosferę CO2. Czy można się dziwić, że tam nie dotarliśmy?

-- Wiemy, że jest w~tym prawda. To wszystko wokół nas. Możesz tylko
zachowywać się tak, jakby planeta była nieskończona, jak myślenie
życzeniowe przeważało w~fizyce, przez tak długi czas, zanim wszystko
trafi do szamba. Dlatego Cape Canaveral jest miejscem do nurkowania.
Myśl o~tym zbyt długo, a~dojdziesz do wniosku, że nic, co robisz, nie ma
znaczenia. Albo zabij się teraz, albo zabij swoich potomków tylko przez
zaczerpnięcie oddechu.

-- Teraz mamy umowę dla ludzkości, która jest lepsza niż cokolwiek
wcześniej: stracić ciało. Odejść od tego. Stać się nieśmiertelną istotą
czystej myśli i~uczuć, zdolną do podróżowania po wszechświecie z~prędkością światła, niezniszczalną, świadomie decydującą, jak chcesz żyć
swoim życiem i~dopasowywać się, dostrajając swoje parametry, abyś był
wersją siebie, która robi właściwą rzecz, która zna i~szanuje siebie.

Dotarły do zrujnowanego budynku, ogromnej rafinerii lub zakładu
przetwórczego, wielkiego jak lotnisko, z~dwoma wielkimi zapadliskami
zaburzającymi linię dachu.

Sita wskazała na to. 

-- Kilka lat bez konserwacji i~po prostu
\textit{implodował}. To kontrola klimatu. W~takim miejscu, o~ile nie
zbudujesz go hermetycznie i~paroszczelnie, dając im współczynnik Q jak
skafander kosmiczny, będzie cię to kosztować więcej, niż mogłabyś
zarobić, uruchamiając go. Te rzeczy wymagają kontroli klimatu lub
zaczynają zatrzymywać wilgoć, a~latem gniją. Następnej zimy jest gorzej.
Kilka lat później, bum, gruz. Ta rzecz była gigantycznym komputerem,
który mieścił ludzi i~maszyny, a~kiedy wyłączyli komputer, nastąpiło
natychmiastowe zniszczenie.

-- Wszechświat nas nienawidzi. Jesteśmy chwilowymi naruszeniami drugiej
zasady termodynamiki. Odpychamy entropię do brzegów, ale jest cierpliwa
i narasta, a~kiedy oderwiemy od niej wzrok, bum, wraca z~nawiązką.

-- Chcesz zmienić historię przyszłości, dać nam szansę na życie warte
życia, bez ucisku? Jest tylko jeden sposób. Wiesz o~tym, ale nie możesz
się z~tym zmierzyć.

-- Ale mogłabym, gdybym był symem? Przesuwając suwaki, aż znalazłabym się
w kopercie, gdzie uwielbiam być symulowana?

-- Bingo. Mielibyśmy świat należący do zwierząt i~doświadczalibyśmy go za
pomocą czujników, które doskonale symulują nasze mokre rzeczy, ale bez
kruszenia tych wszystkich cennych korzeni.

-- Czy mogę coś zasugerować?

-- Proszę.

-- Kiedy zaczynasz przekonywać, nie wspominaj o~Łysence. Uczynić świat
lepszym miejscem, realizując nieudane marzenia szalonego naukowca
jednego z~największych potworów w~historii\ldots 

-- No tak.

-- Tylko mówię.

-- Nie chodzi ani o~Łysenko, ani o~Stalina. To anioły naszej lepszej
natury. Wszystko, co wiemy, że \textit{powinniśmy} robić, ale nie
potrafimy się do tego zmusić, ponieważ część nas, która widzi całą mapę
i wie, że to droga, którą trzeba jechać, nie może przekonać części,
która jest na miejscu kierowcy. Chodzi o~to, aby móc wybierać, aby
wybrać.

-- A co, jeśli ktoś inny wybierze za ciebie?

-- Jeśli ktoś inny przejmie kontrolę nad Twoimi suwakami? Katastrofa.
Kompletne wywrócenie. Terror bez historycznego odpowiednika. Lepiej
upewnij się, że tak się nie stanie.

-- Mam wrażenie, że już zaplanowałaś tę kłótnię, Sito. Zasadzka.

-- Nie zasadzka -- powiedziała. -- Tylko rynek pomysłów. Dokądś dochodzimy,
coś, co się szykuje, przeleje się. Jesteśmy tego częścią. Chcę, żeby
wszyscy byli na to przygotowani, żeby było minimum biegania bez sensu.

Limpopo przypomniała sobie kłótnie z~Jimmym o~to, jak świat ma się
zmienić i~jak musi się z~tym zmierzyć, jak zaproponował, że powierzy jej
kierownictwo, jeśli go poprze. To było tak jawne manipulowanie, że nigdy
nie była kuszona. Czy to właśnie robiła Sita? Jeśli tak, dlaczego jej
nie wspierała?

-- Jeszcze jedno pytanie.

-- Tyle, ile chcesz, Limpopo.

-- Tylko jedno. Potem chcę wrócić do cieszenia się naturą.

-- Wal.

-- Dlaczego mamy różne poziomy kontroli wykonawczej nad naszymi umysłami?
Dlaczego mielibyśmy ewoluować, aby pokonać naszą własną lepszą naturę?

-- Ponieważ ewolucja nie jest ukierunkowana. Nie jest usprawniona.
Jesteśmy strychem wypchanym wszystkim, co nasi przodkowie uważali za
przydatne, nawet jeśli przestało być przydatne tysiące lat temu. O ile
to nie sprawi, że będziesz miała mniej dzieci, to zostaje w~genomie.
Brak kontroli nad swoimi racjonalnymi priorytetami z~pewnością zwiększa
liczbę dzieci, które urodzisz.

Limpopo roześmiała się wbrew sobie, mimo że Sita ewidentnie wcześniej
użyła tego wersu. 

-- Wszystkie te rzeczy na strychu są przydatne, prawda?
Dlatego same strychy nie zostały wyparte przez ewolucję. Posiadanie
statystycznie normalnego rozkładu wszystkich cech, w~tym zdolności do
podejmowania decyzji i~trzymania się jej, oznacza, że jako gatunek
jesteśmy w~stanie stawić czoła różnym wyzwaniom. Z~punktu widzenia
genomu mamy narzędzie na każdą okazję.

-- Czy mogę ci przerwać?

-- Oczywiście.

-- To nie jest nowy argument. Istnieje cały kontyngent
neuroróżnorodności, który nienawidzi moich pomysłów na suwaki i~chce
zachować naszą niezdolność do ,,podjęcia decyzji i~trzymania się jej''
na wypadek, gdyby w~przyszłości pojawiły się jakieś hipotetyczne
niszczące gatunki skrzyżowania, w~których musimy go uratować.
\textit{Zachowaj} swoją irracjonalność nienaruszoną. Ja swoją wyłączę.
Inni ludzie mogą podjąć własne decyzje. Ponieważ niemożność zrozumienia
\textit{to} niszczące gatunki rozdroże i~\textit{już} się na nim znajdujemy.
Jeśli nie wymyślimy, jak dziś odłożyć gratyfikacje, by przeżyć jutro, by
pokonać złudzenie solipsysty, że jesteś wyjątkową śnieżynką\ldots 

-- W~porządku, wiem, jak to idzie.

-- Wiem, że wiesz.

Przeszukiwali ruiny, przeszukiwali ogromne maszyny pod śniegiem i~zdradzieckie stosy gruzu, które można było wykorzystać jako chwiejne
schody, aby dotrzeć do pozostałości dachu i~dziwnych zachowanych
reliktów, w~tym stanowiska kierownika z~wyblakłym zestawem laminowanych
notatek bezpieczeństwa przypiętych wokół brakującego okienka
obserwacyjnego.

-- Jeśli okaże się, że poziom kontroli wykonawczej, jaki uzyskujemy
dzięki symom, przyniesie odwrotny skutek, po prostu je wyłączymy. To
jest istota kontroli wykonawczej: decydowanie o~tym, co zamierzasz
zrobić.

-- A co z~kryzysami egzystencjalnymi?

-- Co?

-- Lodołasica mówiła mi, że Roz ciągle popełniała samobójstwa\ldots 

-- Awaria. Nieuleczalne szaleństwo. Dopóki nie zorientujesz się, jak zmusić ją do
wersji siebie, która nie miałaby kryzysów egzystencjalnych.

-- Tak\ldots  -- Brzmiała ostrożnie. Limpopo wyczuła słabość.

-- Nie możesz kogoś zasymulować, chyba że zmniejszysz suwak, który
odpowiada za wariowanie na myśl o~byciu symulowanym.

-- Tak\ldots  -- Większa ostrożność.

-- Co się stanie, jeśli porzucisz swoje ciało, załadujesz je i~okaże się,
że rasa ludzka nie może przetrwać bez tego, co sprawia, że boimy się
utraty naszych ciał?

-- To jest perwersyjne.

-- Nie, nie jest. Nietrudno wyobrazić sobie niechęć do ciało-tomii jako
pragnienie przetrwania. A jeśli planujesz masowe samobójstwo rasy
ludzkiej?

-- Wszystko, co masz, jest hipotetyczne. Mam konkretne ryzyko:
\textit{jesteśmy} w~środku masowego samobójstwa. Jeśli okaże się, że
wyłączenie naszego egzystencjalnego lęku sprawi, że porzucimy nadzieję i~się wyłączymy, poradzimy sobie z~tym, gdy to nadejdzie. Weź, Limpopo,
bądź poważna.

Ta kłótnia była tak gorąca, tak odmienna od dotychczasowej argumentacji,
że Limpopo wiedziała, że dotknęła czegoś delikatnego. Nie było to
przyjemne uczucie. Kiedy ludzie tak się zaczynali zachowywać, nie można
ich było do niczego przekonać. Chciała znaleźć sposób na wyłączenie
niepokoju Sity, suwak, na którym mogłaby znaleźć środek, dzięki któremu
Sita mogłaby stawić czoła swoim lękom bez panikowania. Sita też
żałowała, że takiego nie ma.

\chapter*{vii}

-- Cześć, Jacob -- powiedziała Natalie. Wcześniej go tak nie nazywała, ale
\textit{tata} nie chciało przejść przez gardło. Jej ojciec chwycił się
nogi łóżka, natomiast zamki drzwi wykonały cykl, stuk-\textit{stuk}.

-- Nie podoba mi się to, wiesz.

-- Więc przestańmy. Rozwiążesz mnie, pozwolisz odejść i~się rozstaniemy.
Nie każda rodzina pozostaje rodziną na zawsze. Co roku wyślę Ci kartkę
świąteczną i~przyjadę na pogrzeb. Bez urazy.

Wyglądał na zranionego. To mogło być częściowo szczere, co było
zdumiewające, biorąc pod uwagę, że była w~czteropunktowych kajdankach.
Chwila minęła.

-- Twoja matka i~siostra chcą Cię odwiedzić.

Przewróciła oczami. Roz była jej stałą towarzyszką, odkąd obudziła się w~komputerze. Bez niej Natalie wyobrażała sobie, że byłaby w~dość
osłabionym stanie, desperacko szukając towarzystwa. Odosobnienie było
oficjalnie torturą. Przeskakiwała między przekonaniem, że Roz jest
zdrajczynią, a~możliwością, że Roz naprawdę jest po jej stronie, ale
nawet ten stan nieokreśloności był trudnym problemem psychicznym, który
utrzymywał ją przy zdrowych zmysłach.

-- To nie tak, że mógłbym je powstrzymać.

Zacisnął usta. 

-- Nie bądź trudna. -- Stłumiła parsknięcie. -- Nie mogę ich
sprowadzić, kiedy jesteś taka.

Nie mogła stłumić drugiego parsknięcie. 

-- To brzmi jakbym \textit{sama}
się przywiązała.

-- Co, kurwa, miałem zrobić? Natalie, jestem z~Tobą \textit{delikatny}. Czy
wiesz, co inni rodzice robią dzieciom, które uciekają z~przyjaciółmi?
Masz pojęcie, jak wygląda tego rodzaju deprogramowanie?

-- Mam całkiem niezłe pojęcie. Pamiętam Lanie.

Lanie Lieberman była jej najlepszą przyjaciółką do roku, w~którym
skończyły trzynaście lat, kiedy Lanie pojechała po bandzie, wymykając
się na odważne spotkania z~chłopcami, alkoholem i~tego rodzaju klubem,
do którego bramkarze wpuszczali trzynastolatkę, jeśli się odpowiednio
ubrała i~przyszła z~odpowiednio bogatym chłopcem. Uziemili ją, umieścili
na niej nadajniki, posłali drony, zatrudnili ochroniarza, potem dwójkę,
ale Lanie była Houdini, zwłaszcza gdy pomagali jej starsi chłopcy z~rodzin bogatszych od jej rodzin, którzy mieli swoje pieniądze na środki
zaradcze, których Lanie potrzebowała, żeby uciec.

Potem była to szkoła prywatna, potem szkoła wojskowa, potem miejsce dla
niespokojnych dzieciaków i~wreszcie miejsce, którego imienia Lanie nigdy
nie wymówiła. To było jedyne miejsce, z~którego nie mogła uciec. Sądząc
po jej bladości, kiedy wróciła, znajdowało się pod ziemią lub gdzieś
daleko na północy. W~wyobraźni Natalie była to opuszczona kopalnia lub
pas tundry. Lanie, która z~niego wróciła, była\ldots  nerwowa. Nie tylko
ranna, ale w~przerażający, tajemniczy sposób \textit{wypalona}. Smutne
rzeczy czasami ją rozśmieszały. Kiedy inni ludzie się śmiali, patrzyła
na nich skoncentrowana i~zła, musiała trzymać swój gniew na wodzy.

Przestały udawać, że są przyjaciółkami przed piętnastym rokiem życia. W~wieku szesnastu lat Lanie dostała wcześniejszy wstęp na uniwersytet, o~którym nikt nie słyszał w~Zurychu, który miał być niesamowitym miejscem
w branży finansowej, gdzie nawet matematyczne durnie mogły zostać
wysokoopłacanymi analitykami finansowymi. Ostatnie, co Natalie o~niej
słyszała, było dostarczone do rąk własnych zaproszenie na pogrzeb jej
ojca, schludny podpis pod grawerunkiem. Natalie nie poszła na pogrzeb i~nie potrafiła sobie wyobrazić zapytania do bazy danych w~wyniku, którego
pojawia się jej nazwisko jako potencjalnego uczestnika.

Jej tata uśmiechnął się blado. 

-- Sprawy mocno się zmieniły od czasów
Lanie Lieberman. Są targi na temat tego, co możemy teraz robić. Zrobiłem
dwa dyskretne zapytania i~teraz dostaję broszury na lnianym papierze tak
grubym, że mógłbym pokryć dach jak gontami. Natalie, jesteś rozwijającą
się branżą, a~metodologia jest szybsza, bardziej bezwzględna i~skuteczniejsza niż cokolwiek z~tamtych czasów. Śruby na kciuki przeciwko
psychoanalizie.

-- Ale nie wysłałeś mnie. -- Ciekawa wbrew sobie.

-- Jeszcze nie. Natalie, choć trudno ci w~to uwierzyć, szanuję cię i~kocham cię jako ojciec. Chciałbym, żeby ta część ciebie, która jest
\textit{Tobą}, przeżyła tę przygodę. Nie chcę automatu pozornie
przypominającego moją córkę. Chcę, żebyś zdała sobie sprawę, że całe to
zadawanie się z~radykalną polityką i~obozowanie z~wyrzutkami nie jest
strategią długoterminową. Rozumiem, że czujesz się winna, że masz tak
wiele, podczas gdy wszyscy inni mają tak mało, ale jak myślisz, co
dobrego daje odwrócenie się od rzeczywistości? Nie możesz znieść
nierówności. W~moim idealnym świecie prowadziłabyś fundację rodzinną,
nadzorowałabyś nasze dobre uczynki. Jest wielu biednych ludzi, którzy
zawdzięczają swoje szczepienia, wodę i~edukację Fundacji Redwater. Weź
trochę tej energii, którą wkładasz w~anarchię i~skieruj ją na coś
produktywnego. Mogłabyś nawet odłożyć mały teren poprzemysłowy dla
eksperymentalnych społeczności opartych na zasadach odchodzących.

Tylko się na niego gapiła. Wiedziała, że gdyby przez cały ten czas
naprawdę była w~odosobnieniu, to zabrzmiałoby jak piekielna propozycja.
Gdyby nie Roz, błagałaby o~to. Wiedziała, jak bardzo była podatna na
izolację. Nie chodziło tylko o~samotność. To była samotność \textit{z samą
sobą}. Czy to oznaczało, że Roz nie pracowała dla ojca? A może była to
subtelna, supermakiaweliczna umowa Jacoba Redwatera, która uczyniła go
legendą, nawet w~kręgach zettów?

-- Kiedy mama i~Cordelia przyjadą z~wizytą?

Pokręcił głową. To było takie protekcjonalne. 

-- Twoja matka cię nie
wykupi. Jest bardziej zdenerwowana niż ja. Cordelia, cóż, boi się
ciebie. Chce dać ci leki przeciwpsychotyczne. Myśli, że ją zaatakujesz.

-- Kiedy przyjdą z~wizytą?

-- Czy chcesz ich zobaczyć?

Spojrzała na niego. Przechylił jej łóżko pod kątem czterdziestu pięciu
stopni, żeby mogła na niego patrzeć ponad pomiętym białym wzgórzem jej
okrytego prześcieradłem ciała.

-- Zobaczę, co da się zrobić.

Kiedy odszedł, Roz krzyknęła na tyle głośno, że się skrzywiła.

-- Jezu, ciszej!

-- To miejsce jest tak odporne na wstrząsy, że można je wykorzystać do
drukowania hologramów -- powiedziała Roz.

-- Moja głowa nie jest wstrząsoodporna.

-- Przepraszam. Nie wiem, czy o~tym wspominałem, ale Twój tata to
kolosalny dupek.

-- Przeprosiłabym za niego, ale jebać to.

-- Tak.

-- Jeśli to ma znaczenie, jestem bardziej przekonana, że nie pracujesz
dla niego.

-- Co za ulga.

-- Ten symulator głosu jest coraz lepszy w~sarkazmie.

-- Przemycałam aktualizacje do mojej lokalnej kopii. Ludzie zajmujący się
syntezatorami głosu są dobrzy, łączą znormalizowane nagrania mowy z~gier
MMO i~systemów odpowiedzi głosowych, uzyskując z~tego niesamowite
rzeczy. \textit{Bawiłam się niektórymi możliwościami.} -- Ostatnie zdanie
wyszło jako ryk drapieżnika, tak przerażające, że Natalie szarpnęła
więzami.

-- Jezus.

-- Wiem, prawda? Ale oszukiwałam. Użyłam infradźwięków. To niesamowite,
co mogę zrobić. Powinnaś usłyszeć moją seksowną niewinność.

-- Nie, dziękuję. Nie pamiętam, żebym kiedykolwiek czuła się mniej
zainteresowana seksem\ldots 

-- Nadchodzą.

Drzwi zaskrzypiały, znowu zaskrzypiały i~otwarły się gwałtownie, a~do
środka weszła matka Natalie, w~perłowo-szarym kolorze, jak
monochromatyczna Jackie O, mniejsza niż zapamiętała Natalie, ale nie
starsza. Zrobiła mały krok do środka, marszcząc nos od zapachu, którego
Natalie nie była świadoma. Wpatrywała się w~Natalie. Cordelia wśliznęła
się za nią z~okrągłą twarzą lalki z~porcelany. Natalie poczuła ukłucie
dziwnej sympatii dla niej, będąc samą z~matką, jedynym przedmiotem uwagi
matki.

-- Cześć, mamo.

Jej mama okrążyła łóżko, okrążyła trzy strony, zanim podeszła do ściany,
wróciła po swoich krokach i~zatrzymała się obok Natalie.

-- Jacob -- zawołała. Jacob wszedł do pokoju, wyglądając na zbolałego.

-- Tak, Frances?

-- Usuń te ograniczenia.

-- Mamo\ldots  -- zaczęła Cordelia, ale jej matka podniosła drżącą rękę.

-- Jacob. Teraz.

Spojrzeli sobie w~oczy. Pamiętała to z~dzieciństwa, ich wojny milczących
spojrzeń. Kiedy dorosła, zdała sobie sprawę, że to była gra w~kurczaka,
w której każde dawało sobie nawzajem więcej czasu na kontemplowanie
sposobów, w~jakie może nadejść zemsta, dopóki jedno nie odwróci wzroku.
Jak zwykle Jacob pierwszy zerwał kontakt.

-- Zaraz wrócę.

Natalie przypuszczała, że poszedł po technika medycznego, czy kim tam
był, ale chwilę później wrócił z~najemniczką. Kiwnęła lekko głową
Natalie, w~stopniu uznania, który był praktycznie uściskiem całego
ciała, biorąc pod uwagę ich poprzednie interakcje. Może Natalie zrobiła
na niej wrażenie swoim ,,odlotem''. A może dostała pozwolenie -- albo
rozkazy -- żeby się rozchmurzyć.

-- Frances, Cordelio, proszę, cofnijcie się.

Mama wyglądała, jakby miała się kłócić, ale Cordelia pociągnęła ją za
ramię. 

-- No, \textit{chodź}, mamo.

Kiedy znalazły się w~odległości kilku metrów od łóżka, najemniczka
wystąpiła naprzód i~spojrzała w~oczy Natalie.

-- Żadnych kłopotów -- powiedziała i~założyła bransoletkę na nadgarstek
Natalie. Natalie podniosła głowę i~wytężyła wzrok, żeby to zobaczyć. To
był zły niebieski metal. Nie chciała nawet zgadywać, co to zrobiło,
chociaż nie mogła powstrzymać swojej podświadomości przed próbami
domyślenia się: nie szok, ponieważ mogła złapać mamę, tatę lub Cordelię
i porażenie przeszłoby na nich. Może coś w~jej nerwach, na przykład ból,
drgawki lub\ldots 

-- Żadnych kłopotów -- zgodziła się. 

Najemniczka bezosobowo podniosła
prześcieradło, wyjęła cewnik i~wsunęła go do łóżka. To uczucie sprawiło,
że westchnęła z~upokorzenia. Najemniczka wytarła ręce jednorazową
chustką i~wrzuciła ją do kosza na łóżku, zanim podała jej rękę. Natalie
ją wzięła, bo po dniach, tygodniach? leżenia na wznak, była słaba i~miała zawroty głowy, a~mięśnie brzucha nie chciały pomóc przesunąć jej
ogromnymi, zdrętwiałymi nogami przez krawędź łóżka. Łzy napłynęły jej do
oczu, ponieważ kiedy była odchodniczką, była taka \textit{silna},
wszystkie były. Całe to chodzenie. Teraz nie mogła odejść, nawet jeśliby
oczyścili ścieżkę. Łzy spłynęły jej po policzku i~wślizgnęły się do ust.

Wycisnęła resztę łez i~zamrugała mocno, pozwalając się poprowadzić na
nogi. Kołysała się, nie patrząc na mamę ani Cordelię, patrząc na Jacoba,
pozwalając mu zobaczyć, co jej zrobił. Zniszczył jej ciało, ale starała
się, żeby jej oczy błyszczały, by dać mu do zrozumienia, że nie dotknął
jej umysłu.

Mama była u jej boku, trzymając ramię pod pachą tej ręki, która nie
miała kroplówki. Najemnik odłączył drugi koniec węża od łóżka, przykrył
go sterylną, elastyczną chustą i~owinął wąż wokół szyi Natalie. Jej mama
pachniała własnymi perfumami, zrobionymi specjalnie przez mężczyznę ze
Stambułu, który przychodził do domu raz w~roku podczas Święta
Ofiarowania, kiedy jeździł po świecie i~wpadał do swoich najlepszych
klientów, podczas gdy cała Turcja się zatrzymywała. Natalie nie czuła
tego zapachu -- niezupełnie słodkiego, niezupełnie piżmowego i~trochę
przypominającego kardamon -- od lat, ale pamiętała go wyraźniej niż twarz
matki.

Jej matka sapnęła, kiedy oparła swój ciężar na jej ramionach. Natalie
pomyślała, że jest za ciężka, a~potem: 

-- Jacob, ona jest jak ptak! -- tonem bardziej przerażonym, niż kiedykolwiek słyszała od niej Natalie.

Zobaczyła idealną skórę swojej matki wykrzywioną w~grymasie, oczy
zwężone w~szparki, które sprawiły, że zmarszczki linii włosów w~ich
kącikach pogłębiły się w~sposób, którego nienawidziła.

-- Cześć, mamo.

Stały, kołysząc się. Poczuła, jak jej nogi się poddają.

-- Powinnam usiąść.

Oboje usiadły. Tuż za nimi znajdował się otwór w~materacu, przez który
cofały się węże, śmierdzący i~ciemny. Jej matka odwróciła się, żeby na
to spojrzeć, odwróciła się i~chwyciła Jacoba jeszcze ostrzejszym
spojrzeniem.

-- Jacob -- zaczęła.

-- Później -- powiedział.

Natalie cieszyła się z~jego zakłopotania. Cordelia stała w~połowie drogi
między rodzicami, martwiąc się rękoma, skubiąc skórki. Obgryzała
paznokcie, złamała nałóg dopiero po kilku próbach i~Natalie wiedziała,
że nie ma ochoty na nic więcej, jak tylko zacząć żuć własne palce.

Natalie uderzyło, że była najmniej zdenerwowana spośród nich, z~wyjątkiem najemniczki. Była w~drużynie z~najemniczką, one kontra te
popieprzone zetty. To było głupie. Najemniczka nie była po jej stronie.
No dalej, Natalie, skup się.

-- Nie będę znowu przywiązana.

-- Nie, na pewno nie -- zgodziła się jej matka.

-- Frances\ldots  -- zaczął jej ojciec.

-- Nie, nie zrobi tego. -- Konkurencja wpatrywania znów się tliła.
Równowaga się zmieniła. Pojawiło się nowe ukryte zagrożenie, co
powiedziałby sędzia sądu rozwodowego o~córce przywiązanej do łóżka,
zagłodzonej i~zaintubowanej, zamkniętej w~schronie? Jej matka była
wściekła z~powodu jej odejścia, ale to nie powstrzymało jej przed
użyciem dowolnej dźwigni, którą przekazał jej Jacob Redwater.

-- Nie, nie będzie -- powiedział. -- Przepraszam.

Wyszedł z~pokoju.
Zamknął drzwi. Drzwi \textit{zastukały}.

Cordelia zrobiła niepewny krok. Jej matka wyciągnęła rękę, a~ona
przeszła resztę drogi, pozwalając Frances ją uściskać, zawsze
wystarczająco ciepło, zawsze kończąc na chwilę przed oczekiwaniem.

Cordelia subtelnie pochyliła się do Natalie, sprawdzając obecność
potencjalnego uścisku, ale Natalie nie odpowiedziała. Jebać ją. Jeśli o~to chodzi, jebać Frances. Wiedziały, że jest więźniem, i~żadne z~nich
jej nie wyrwało. Uwolnienie jej z~czteropunktowych więzów nie
kwalifikowało się jako wyzwolenie.

-- Natalie, to jest po prostu okropne -- powiedziała jej matka.

\textit{Bez jaj}. 

-- Aha.

-- Dlaczego, Natalie? Są bardziej konstruktywne sposoby na kontakt ze
światem. Dlaczego zostać zwierzęciem? Terrorystką?

To było tak cholernie głupie, że nie udało jej się powstrzymać
szyderczego parsknięcia. 

-- Co wolałabyś?

-- Wyprowadź się, jeśli jest tak źle. Twoje fundusz powierniczy jest
wypracowany, możesz kupić miejsce w~dowolnym miejscu na świecie. Znajdź
pracę lub nie. Stań w~obronie czegoś. Czegoś \textit{konstruktywnego},
Natalie. Czegoś, przez co nie zostaniesz zabita, zgwałcona ani\ldots 

-- Porwana przez najemników i~przywiązana do łóżka w~piwnicy jakiegoś
bogatego dupka?

Matka zacisnęła szczękę.

-- Natalie -- powiedziała Cordelia. -- Czy mogę ci coś przynieść?

-- Prawnika. Policjantów.

-- Natalie\ldots  -- Cordelia wyglądała na urażoną. Natalie nie pozwoliła
sobie się przejmować.

-- Wiedziałaś, że tu jestem. Wiedziałaś, że mnie porwał. Nie lubisz
odchodzących i~nie lubisz tego, że jestem jedną z~nich, w~porządku. Ale
na wypadek, gdybyś nie zauważyła, jestem dorosła i~to, czy zostanę
odchodniczką, to nie twoja sprawa. Żadne z~was nie ma nic do powiedzenia
o tym, co robię.

-- Oczywiście, że tak. Jestem Twoją matką!

Nawet Cordelia uśmiechnęła się złośliwie.

Widziała, jak w~ich matce gotuje się wściekłość, inna niż u ojca, ale
nie mniej śmiertelna. 

-- Natalie, jeśli myślisz, że bycie dorosłym
oznacza, że nie masz żadnych obowiązków wobec nikogo na świecie\ldots 

Ona i~Cordelia parsknęły. To jeszcze bardziej rozwścieczyło matkę, ale
był to najbardziej siostrzany moment, jaki dzieliły, odkąd Natalie po
raz pierwszy poszła do szkoły.

Frances zesztywniała i~wpatrywała się prosto przed siebie, nie
zauważając ich, żałując, że nie przeszła od razu do chwili
macierzyńskiej, która nie pozostawiła jej bez łaski, a~jeżeli była jakaś
rzecz, którą Frances Mannix Redwater była, to była łaskawa.

Drzwi zastukały, otworzyły się. Jacob wszedł, razem z~płatnym
zbirem/technikiem medycznym, który niósł cenne naręcze ubrań. Natalie
rozpoznała ubrania z~windy w~poprzednim więzieniu.

-- Później przyniesiemy porządne łóżko -- powiedział Jacob, podczas gdy
mężczyzna kładł ubrania na podłodze.

-- I~książki -- powiedziała Natalie. -- Interfejs. Papier i~coś do pisania.

Spojrzał na nią, potem na Frances.

-- Żadnych interfejsów -- powiedziała Frances. -- Ale wszystko inne tak.
Niektóre meble też. Lodówka i~jedzenie.

-- Szybciutko -- powiedziała Natalie z~zawrotnym śmiechem. Jacob ją
zignorował. Zdenerwował się, ale nie dało się go zdobyć tak prymitywnym
żartem.

-- Teraz wszyscy wychodzą -- powiedziała Frances. -- Muszę sama porozmawiać
z Natalie.

Natalie zamknęła oczy. Nie jedna z~tych rozmów.

-- Jestem zmęczona -- powiedziała.

-- Miałaś mnóstwo czasu na odpoczynek. -- Frances zdołała zrobić z~tego
oskarżenie, jakby Natalie leniuchowała i~jadła cukierki. To nie był
sarkazm, Frances potrafiła być jednocześnie oburzona, ponieważ była
przywiązana do łóżka i~dlatego, że była zbyt leniwa, by wstać z~łóżka.

-- Wszyscy wychodzą. -- Spojrzała na najemniczkę, która była na tyle
rozsądna, by nie patrzeć na Jacoba. To byłby koniec jej pracy w~gospodarstwie domowym Redwaterów. Natalie domyślała się, że bycie
najemnikiem zatrudnionym przez zetty wymaga zmysłu politycznego.

Wyszli i~zanim drzwi się zamknęły, Frances zawołała Jacoba. 

-- Prywatnie.
Bez nagrywania.

-- Frances\ldots 

-- Ona nie rzuci się na mnie i~nie wykorzysta mnie jako zakładnika,
Jacob.

-- Widziałaś wideo\ldots 

-- Widziałam. To było, zanim przywiązałeś ją do łóżka na tydzień i~karmiłeś przez rurkę.

-- Frances\ldots 

-- \textit{Jacob}.

Jacob odwrócił się do najemnika, który już coś trzymał, dłoń w~dół.
Przekazał to Frances. 

-- Przycisk paniki -- powiedział.

Celowo włożyła go do torebki, po czym odłożyła ją daleko od łóżka,
opierając o~ścianę, z~maślanożółtą skórą opadającą na surową biel. 

-- Do
widzenia, Jacob.

Zostawili drzwi otwarte.

\chapter*{viii}

Limpopo była wolontariuszką w~ekipie skanującej, kiedy pojawił się
Jimmy.

Nie wyglądał na tak zarozumiałego, jak ostatnim razem, gdy się spotkali,
ze swoją głupią bronią i~tak dalej. Przeszedł ciężką drogę, dotarł do
Thetford, utykając i~z raną głowy, w~brudnych nakładających się
warstwach termicznych. Był wychudzony, odmrożone trzy palce ręki i~wszystkie palce u nóg.

-- Fajnie jest Cię tu spotkać -- powiedział, kiedy Limpopo natknęła się na
niego w~wielkiej sali Thetford, pod opieką medyka, który słuchał rady
kogoś z~daleka, kto diagnozował Jimmy'ego.

-- Źle wyglądasz -- powiedziała.

-- Mogło być gorzej. Straciliśmy piętnaścioro na drodze z~Ontario. Robi
się wrednie.

-- Przykro mi.

-- Nie Twoja wina. Właściwie, prawdopodobnie to Twoja wina, jesteś wielką
postacią w~świecie skanowania i~symulacji.

-- Jestem odchodzącą. Nie mamy wielkich postaci.

Sanitariusz uśmiechnął się, po czym zrobił coś z~palcami Jimmy'ego, co
sprawiło, że zassał powietrze przez zęby, jednego brakowało, i~zacisnął
oczy.

-- Myślę, że je zatrzymasz -- powiedziała. -- Może z~wyjątkiem lewego
małego palca u nogi.

-- Hurra. -- Poruszył szczęką z~boku na bok.

-- Dlaczego tu jesteś, Jimmy? Przyszedłeś wykopać więcej ludzi z~ich
domów?

Pokręcił głową. 

-- To nie tak. Jakiekolwiek drobne różnice filozoficzne
mieliśmy ty i~ja\ldots 

To było podręcznikowe samooszukiwanie się, ale Limpopo nie widziała
żadnego powodu, by to wykazywać.

-- \ldots  mam z~Tobą więcej wspólnego niż z~dupkami, które przyszli po nas
na drodze. Oni chcą tylko jednej rzeczy: świata, w~którym są na
szczycie, a~wszyscy inni \textit{nie}.

\textit{Chciałbym wiedzieć, jak odróżniasz to od swojej filozofii. Ale nie
sądzę, żebyś była w~stanie to wyjaśnić.}

-- Jest jasne, gdzie jest akcja. To ich przeraża i~knują.

-- Więc przyszedłeś pomóc?

-- Słuchaj, jest pewne podejście, coś, czego nie widziałem na forach,
wynik, który jest gorszy, niż ktokolwiek się przygotowuje. Myślę, że to
dlatego, że ludzie tacy jak Ty po prostu nie rozumieją, co naprawdę
oznacza kopia.

\textit{Kopia zapasowa}. Idealna, doskonale uwodzicielska nazwa skanowania
i symulowania. Była zdumiona, że jej nie słyszała. Jak tylko usłyszała,
Limpopo po prostu \textit{wiedziała}, że muszą być tysiące, nie, miliony
ludzi używających tego terminu. Gdy raz opiszesz rzeczy, które
sprawiają, że Ty to \textit{Ty} jako dane, eony lęków przetwarzania danych
się włączą. Jeśli masz dane, to muszą mieć kopię zapasową. Wszystko, co
\textit{nie zostało} zapisane w~backupie, może zostać utracone. Dane są
nawiedzane przez Murphy'ego. Zrób coś niezastąpionego i~wspaniałego,
będąc poza zasięgiem sieci i~kopii zapasowych, i~prosiłeś się o~krytyczną awarię, która zniszczy to wszystko.

-- Kopia zapasowa -- powiedziała.

-- Tak. -- Jimmy się uśmiechnął. Podążył za jej myśleniem. -- Oczywiście.
Nikt nie przemyślił tego do logicznego końca.

-- To znaczy?

Mimo odniesionych ran i~niechlujności lubił ją testować, czekając, by
zobaczyć, czy zacznie boksować. Wiedziała, że nie ma sposobu na wygranie
mentalnego sparingu z~Jimmym: zwycięstwo go wkurzy, przegrana przekona
go, że może po niej chodzić.

-- Miło Cię widzieć. -- Odwróciła się do wyjścia, ponieważ odejście za
każdym razem rozwiązywało problem Jimmy'ego. Jeśli kiedykolwiek się do
tego domyśli, może stać się niebezpieczny.

-- To znaczy -- powiedział do jej pleców, a~ona trochę zwolniła -- każdy,
kto może zdobyć Twój backup, może dowiedzieć się o~Tobie wszystkiego, co
można o~Tobie wiedzieć, oszukać cię w~najgorszych zdradach, torturować
cię przez całą wieczność, a~Ty nigdy od tego nie odejdziesz.

-- Gówno. -- Odwróciła się.

-- Każdy, kto o~tym mówi, jest traktowany jak paranoidalny wariat. Ludzie
od symów machają rękami i~mówią o~krypto\ldots 

-- Co jest nie tak z~krypto? Jeśli nikt nie może odszyfrować twojego
syma, to\ldots 

-- Jeśli nikt nie może odszyfrować Twojego syma, nikt nie może go
uruchomić. Jeśli jedynym repozytorium Twojego hasła jest Twój własny
mózg, to kiedy umrzesz\ldots 

-- Rozumiem. Musiałbyś zaufać komuś swoim hasłem, aby mógł odzyskać Twój
klucz i~użyć go do odszyfrowania Twojego symu.

-- Twoja zaufana osoba trzecia musiałaby zaufać \textit{swojej} zaufanej
stronie trzeciej w~kwestii hasła, a~ta osoba potrzebowałaby kogoś, komu
mogłaby zaufać, i~musiałby być jakiś sposób, aby dowiedzieć się, kto ma
czyje hasło, ponieważ kiedy umrzesz, ostatnią rzeczą, jakiej byśmy
chcieli, było uświadomienie sobie, że zgubiliśmy Twoje klucze. Czy
możesz sobie, kurwa, \textit{wyobrazić}, przepraszam Twoje nieśmiertelne
przyrodzone prawo, ale zapomnieliśmy hasła, derp derp derp.

-- Ała.

-- Jest mnóstwo krypto idiotów próbujących to rozgryźć, używając
wspólnych haseł, aby podzielić klucz na, powiedzmy, dziesięć kawałków,
tak, aby do odblokowania backupu można było użyć dowolnych pięciu.

-- Brzmi jak dobry pomysł. -- Pracowała ze wspólnymi sekretami dla różnych
wcieleń B\&B, ustanawiając komitety zaufanych stron, które mogły
wspólnie wprowadzić gruntowne zmiany w~bazie kodów, ale tylko wtedy, gdy
zgodziło się kworum.

-- Tak, ale nie. Dobrze w~tym sensie, że musisz porwać i~torturować
znacznie więcej osób, aby odblokować czyjąś symulację bez pozwolenia,
ale z~perspektywy złożoności jest gorzej, mnożysz liczbę powiązanych
relacji potrzebnych do odzyskania sym przez dziesięć. Na przykład: teraz
masz dziesięć problemów.

-- Jaka jest odpowiedź?

-- O to właśnie się martwię, odpowiedzią będzie brak odpowiedzi. To jest
\textit{pilne}, niedługo wszystko wybuchnie. Tam w~default, traktują Akron
jak twierdzę ISIS, jak pieprzone \textit{czasy ostateczne}. Byłbym
zaskoczony, gdyby tego nie użyli bomby \textit{jądrowej}.

-- Opad.

-- Obwinią nas o~to i~podpiszą kontrakt na leczenie choroby popromiennej
z jakąś firmą pogotowia ratunkowego zetty. Nie wiesz, jak tam jest.

-- Wiem co nieco.

-- Chyba tak. Przepraszam, nie chciałem, wiesz\ldots 

-- Mansplain.

Wyglądał niezręcznie. Widziała, że żałował, że się pokłócili. Tak łatwo
było go wymanewrować, ponieważ nie wyobrażał sobie, że ludzie wokół
niego nie próbują go wymanewrować.

-- Limpopo, przez ostatnie kilka lat było mi ciężko. Po B\&B, hm\ldots 

-- Implodowało.

-- Byłem zły przez długi czas. Byłem zły na ciebie, chociaż wiedziałem,
że to moja wina. Czyja to mogła być wina? Wygnałem Cię.

-- Zrobiłeś jeszcze gorzej.

-- Zrobiłem jeszcze gorzej. Wyrzuciłem Cię.

-- Nie. Nigdy tego nie zrobiłeś. -- \textit{Nie mogłeś tego zrobić}.

-- Nie mogłem tego zrobić. -- Nie był taki głupi, na jakiego wyglądał. -- Odebrałem Ci rzeczy, ponieważ myślałem, że to uczyni mnie silnym,
ponieważ myślałem, że to, co robisz, osłabia ludzi. Ale wszystkie te
rzeczy, silne i~słabe\ldots 

-- Bzdury.

-- Całkowicie. Silny i~słaby to nie to, \textit{co} robisz, ale to
\textit{dlaczego} to robisz. -- Przerwał. Zamierzała coś powiedzieć. -- Również to, co robisz. To nie dobroczynność ani noblesse oblige
zobowiązują do traktowania ludzi tak, jakby wszyscy byliby jednakowo
wartościowi, nawet jeśli nie wszyscy są jednakowo ,,użyteczni'',
niezależnie od znaczenia użyteczności. -- Wyglądał, jakby chciał płakać.
Medyk przestał pracować na palcach i~obserwował go uważnie. Spojrzał na
nią, na Limpopo, westchnął. Potem kontynuował, co zrobiło wrażenie na
Limpopo, ponieważ to wyznanie będzie w~całym Thetford, zanim znajdzie
miejsce do spania.

-- Mówiłem sobie, że czynię świat lepszym. Myślałem, że są ludzie
,,użyteczni'' i~,,bezużyteczni'', a~jeśli nie uszczęśliwiasz tych
użytecznych, ci bezużyteczni będą głodować. \textit{Oczywiście} zaliczałem
się do grupy użytecznej. Znałem tę ważną tajemnicę o~bezużytecznych i~użytecznych ludziach, a~jeśli to nie jest przydatne, to co jest?
Powiedziałem sobie, że robię więcej wszystkiego dla wszystkich.
Musieliśmy tylko pozwolić ludziom, którzy byli najbardziej wartościowi,
robić to, co chcą. To było pojebane. Zjebałem. Za to właśnie
przepraszam.

-- Twój problem polega na tym, że myślisz, że ,,bezużyteczne'' i~,,użyteczne'' to właściwości ludzi, a~nie rzeczy, które ludzie
\textit{robią}. Osoba może wykonywać użyteczność lub antyużyteczność, w~zależności od okoliczności. Ewolucyjne odsiewanie jakoś nie ominęło
ludzi, którzy nie wnoszą wkładu tak, jak tego chcesz, pozostawiając
zaległości w~doborze naturalnym, którym możesz się zająć. Powodem, dla
którego wszystko w~nas jest rozłożone po normalnej krzywej, z~kilkoma
dziwakami daleko w~długich ogonach po prawej i~lewej stronie, a~wszyscy
inni zebrani razem pod wybrzuszeniem, jest to, że potrzebujemy ludzi,
którzy zajmą się rzeczami, oraz strażaków, którzy są skrzywieni we
właściwy sposób, aby uporządkować najdziwniejsze gówno, które wydarza
się na krawędziach. Zakładamy, że ktoś, kto gasi pożar, jest stumetrowym
superbohaterem, którego przeznaczeniem jest ocalenie wszechświata, w~przeciwieństwie do kogoś, kto raz miał szczęście i~od tego czasu
otrzymał znacznie więcej okazji, aby mieć szczęście.

-- Właśnie to próbuję powiedzieć, tak. Trudno sobie wyobrazić to gówno.
Kręci mi się w~głowie, że przestałem wierzyć w~użytecznych i~bezużytecznych ludzi dopiero wtedy, gdy \textit{ja} okazałem się
bezużyteczny. Potem dotarło do mnie, że skala, na jakiej oceniałem
ludzi, ta skala, na której mi się nie powodzi, była nieistotna. To jedna
z tych wygodnych rzeczy, które śmierdzą samooszukiwaniem.

-- Zgadzam się, że stara skala była gówniana, więc ci odpuszczę.

Skrzywił się, gdy medyk zrobił coś z~jego palcami. Dwa z~nich wyglądały
źle, czarne na końcach. Limpopo odwróciła wzrok, krzywiąc się.

-- Dzięki -- burknął, choć nie potrafiła powiedzieć, czy rozmawiał z~nią,
czy z~medykiem.

\chapter*{ix}

Przyjęcie nie było pomysłem Pocahontas, ale się zaangażowała. Początkowo
Etcetera był przerażony tą myślą. Nie wyobrażał sobie niczego wartego
świętowania wśród śmierci i~niepokoju. Lodołasica zniknęła, a~Gretyl
była zakopana w~tajnych projektach. Był przekonany, że wszyscy poczują
się urażeni, od kosmików przez spóźnionych przybywających, przez
lotników po załogę B\&B, która opłakiwała swoich zmarłych, ale gdy
Pocahontas publikowała powiadomienia o~postępach imprezy w~centrum
społecznościowym kosmików, było jasne, że jedynym niepokojem, jaki
wszyscy mieli o~imprezie, był taki, że ktoś inny może nienawidzić tego
pomysłu.

Pocahontas była siłą natury. Jako pierwsza z~ich załogi wymyśliła, jak
kierować fabami skafandrów kosmicznych, uszyła sobie wspaniały garnitur,
który nosiła podczas serii epickich, wielodniowych wędrówek, nawiązując
kontakt z~pobliskimi grupami Pierwszych Narodów. Chociaż nie byli tak
polityczni jak ona, żadne z~nich nie miało żadnego pożytku z~defaultu i~wszyscy byli ciekawi dziwnych kosmików, którzy przejęli Thetford wiele
lat po tym, jak zostało opuszczone. Pocahontas wykorzystała fabrykę z~Thetford do wydrukowania części dla nowej fabryki skafandrów
kosmicznych, układając je przed korytarzem użytkowym, gotową do
przewiezienia do swoich nowych przyjaciół przez każdego, kto potrafił
zrobić pojazd zdolny do drogi. Gretyl pracowała nad remontem silnika ich
pociągu towarowego, który utykał do Thetford. Złomowaliby go, gdyby nie
było tylu rannych, którzy nie mogli dokończyć podróży na własnych
nogach.

Gretyl była lepsza, niż Etcetera miał prawo oczekiwać. Seth powiedział
mu, co zrobiła, i~chociaż rzadko słyszała od Roz -- symulacja działała na
własnych serwerach schronu, aby uniknąć ryzyka wykrycia ruchu, gdzie
nikt się nie spodziewał -- lakoniczne wiadomości sprawiły, że stała się
stoicka, jeżeli nie pogodna. Według Roz, Lodołasica była zdrowa na
umyśle i~nienaruszona pomimo tortur. Była wykonana z~niezłomnej stali. 

-- Jeśli ona nie traci swojego głowy, jak mogę ja? -- powiedziała Gretyl
pewnego ranka, gdy Limpopo przyniosła im coffium i~świeże bułeczki.

-- Zamierzasz śpiewać? -- spytała Limpopo. Etcetera spojrzał na nią ostro.
Gretyl miała piękny głos, płomienny. W~dawnych czasach świetlicy B\&B
spędzała wieczory, śpiewając piosenki ze swojego głębokiego repertuaru,
w towarzystwie zero lub więcej muzyków Gospody. A capella, była
zdumiewająca; z~zespołem była transcendentna. Ale nie śpiewała, odkąd
odebrano jej Lodołasicę.

-- Na imprezie? -- spytała Gretyl.

-- Na imprezie.

-- Czy jest zespół? -- Etcetera pomyślał, że szuka wymówki, \textit{nie
sądzę, żebym mogła zrobić to bez akompaniamentu} albo \textit{nie mamy
czasu na ćwiczenia}, ale oczy jej błyszczały.

-- Kosmiki mają kilka zespołów, ale nie wiem, czy są dobre.

Pocahontas, która przemykała przez przestrzeń wspólną, kierując ludźmi
przygotowującymi się do przyjęcia, namierzyła ich, śledząc tę rozmowę w~biegu.

-- Jest dobry zespół i~taki sobie zespół -- powiedziała.

-- Jaki rodzaj muzyki?

-- Dobry zespół jest głośny i~szybki. Taki sobie zespół robi folkowe
rzeczy.

-- Zaśpiewam z~oboma -- powiedziała.

Pocahontas ścisnęła jej dłoń. 

-- Załatwione. Dziękuję.

-- Chcesz trochę coffium? -- spytał Etcetera. Obserwowanie, jak Pocahontas
biega dookoła, powodowało, że czuł się wyczerpany.

-- Nie zażywam narkotyków.

Wszyscy wyglądali na zakłopotanych. Etcetera nie znał osobiście żadnych
ludzi z~Pierwszych Narodów, ale wiedział, że są rzeczy o~gorzale i~innych substancjach. Wzruszył ramionami. Wszyscy byli odchodnikami,
prawda? Mężczyzna, kobieta, biały, brązowy, Pierwszy Naród lub inny.

-- Przepraszam -- powiedziała Limpopo. Zastanawiał się, czy powinien był
też przeprosić. Czuł się głupio i~niespokojnie, a~to oznaczało, że
powinien na to zwrócić uwagę.

-- Nic takiego. Twoje neuroprzekaźniki to twoja sprawa.

-- Co możemy zrobić, aby pomóc? -- Etcetera szukał lepszego tematu.

Natychmiast: 

-- Zabierz fab do Dead Lake -- odpowiedziała. -- Nie mogą
przyjść na imprezę bez kombinezonów ochronnych.

-- Ach -- powiedział Etcetera. Powinien był wiedzieć, że to powie.

-- Wszyscy się tym zajmiemy -- powiedziała Limpopo i~uścisnęła jego dłoń,
choć nie potrafił powiedzieć, czy było to współczucie, czy
przypomnienie, by dotrzymać obietnic. -- Możesz na nas liczyć.

-- Liczę -- powiedziała z~uroczystą prostotą, której miała w~nieskończoność. To zabiło lekkość nastroju, sprawiło, że byli poważnie
zaangażowani w~urządzanie przyjęcia niezrównanej zabawy. Pocahontas
spojrzała kolejno im w~twarze, uśmiechnęła i~rzuciła się w~kierunku
innej grupy.

Gretyl patrzyła, jak odchodzi. 

-- Ona jest niesamowita. Impreza. -- Potrząsnęła głową. -- A teraz musimy zabrać te części fabrykatora, co,
siedemdziesiąt klików?

-- Przepraszam.

-- Nie ma za co przepraszać. Najwyższy czas uruchomić pociąg towarowy. -- Wypiła coffium. -- Niektóre z~tych rzeczy są mocno zaklinowane i~nie
wyjdą bez walki. Zhakujemy je. W~skafandrach będzie trudno.

-- Tak.

-- Nie bądź ponury -- powiedziała Limpopo. -- Fajnie będzie ciężko
popracować dla odmiany.

Ona miała rację. Kiedy budowali drugie B\&B, było to powszechne w~ich
czasach, jakieś duże, trudne technologiczne wyzwanie, które musieli
wspólnie rozwiązać, pobierając samouczki i~stukając na globalnej
częstotliwości odchodzących, aby znaleźć kogoś, kto pomoże z~problemem.
Czasami przez kilka tygodni pracowali nad trywialnym problemem
technicznym, zakłopotani, aż pewnego dnia coś zadziałało i~doświadczenie
było słodsze od goryczy walki.

Opróżnił coffium i~przyjrzał się przygotowaniom na przyjęcie wokół
siebie i~przypomniał sobie, że odszedł. Żył w~pierwszych dniach lepszego
narodu, robiąc coś, co coś znaczyło. Jego istnienie było cechą, a~nie
błędem.

Limpopo uśmiechnęła się. Czytała jego myśli.

-- Dopij -- powiedziała do Gretyla. -- Zajmijmy się tym.

Etcetera poczuł, jak napięcie znika z~jego pleców, zastąpione ciepłym
celem. Praca wymagała wykonania, a~on mógł pomóc. O co jeszcze
ktokolwiek mógłby prosić?

\threeast

Kiedy Gretyl zdjęła skafander, była masą bólów, które nie objawiły się,
gdy była pogrążona w~pracy, rąbiąc uszkodzony pociąg piłami, rozwalając
go palnikami, uderzając młotkiem w~nieustępliwy metal i~polimery.

Stała przy śluzie, czując swój smród. Jęknęła i~przyłożyła czoło do
ściany.

-- Wszystko ok? -- Tam wyglądała na autentycznie, żenująco zaniepokojoną.

Kiedy Tam dołączyła do ludzi Uniwersytetu Odchodzących, Gretyl matkowała
jej, pomagając jej poruszać się po mętnych wodach enklawy akademickiej.
Po ataku Gretyl z~dumą obserwowała, jak Tam przemienia się w~derwisza,
przewożącego ludzi i~zapasy do tuneli, ryzykując życiem, silną i~inspirującą.

Odkąd straciła Lodołasicę, świat Gretyl rozpadł się na kawałki. Nawet w~najlepszych chwilach czuła się jak pęknięty wazon, który został sklejony
przez nieudolnego naprawiacza, z~pęknięciami widocznymi dla wszystkich.
Uszkodzone dobra. Tam odwróciła scenariusz, próbując matkować Gretyl w~sposób, którego Gretyl nienawidziła, nie tylko dlatego, że tego
potrzebowała.

-- Ze mną wszystko w~porządku. -- Gretyl próbowała wyprostować swoją
postawę, namalować uśmiech. Praca nad silnikiem była ciężka, ale dała
jej wytchnienie od wszechogarniającego strachu o~Lodołasicę. Najgorszą
rzeczą w~matkowaniu była jej własna żałosna potrzeba bycia matką.

-- To dobrze. Bo szczerze, wyglądasz jak rzeźbione gówno.

-- Dziękuję.

-- Ktoś musiał ci powiedzieć prawdę, człowieku. -- Tam wsunęła się za nią.
Jej ręce chwyciły ramiona Gretyl. -- Jesteś ciasna jak rakieta tenisowa.
-- Ścisnęła eksperymentalnie, silne kciuki wbiły się w~ramiona Gretyla.
Gretyl jęknęła. Teraz gdy dłonie Tam były na niej, czuła napięcie, jak
gumka naciągnięta do punktu zerwania. Wbrew sobie oparła się o~Tam, a~Tam się odwzajemniła. Gretyl znów jęknęła.

-- No chodź. -- Tam dalej ugniatała. -- Powiedz mi gdzie boli. -- Gretyl
usłyszała uśmiech. Tam się to podobało. Gretyl się poddała. 

-- Co teraz robisz?

-- Znajdę miejsce do spania. -- Kompleks kosmików, pełny przed ich
przybyciem, był teraz zatłoczony, a~znalezienie wolnego łóżka wieczorem,
a nawet kąta, w~którym można było tymczasowo umieścić pościel, było
sztuką. 

-- Zatrzymaliśmy się na późny lunch i~zamierzałam spać na
kolacji. Mam na myśli opuścić kolację.

-- Masz szczęście. -- Tam pracowała nad węzłami. -- Znaleźliśmy z~Sethem
miejsce. Jest wielkie. -- Ścisnęła. -- I~wygodne.

Gretyl jęknęła. 

-- No chodź. -- Poddanie było przyjemne.

Pokój był na tyle duży, że Gretyl poczuła się winna. Niemniej jednak
miał dziwny kształt, z~niskimi sufitami w~niektórych miejscach, z~nierówną podłogą w~innych, w~wyniku zdarzenia pogodowego, które wygięło
grodzie, tworząc pęknięcia, których tymczasowych uszczelnienia nikt nie
poprawił.

Był oświetlony konstelacjami rzucanych świateł, rozproszonych w~smugach
na suficie i~ścianach, i~była tam adaptacyjna powierzchnia snu w~stylu
kosmików, miliony komórek piankowych osadzonych w~czujnikach, jak żywa
istota, która przytulała i~podpierała zgodnie z~algorytmem, który
odgadywał twoje krążenie, wijąc się w~sposób, który był niepokojący i~cudowny.

Seth wylegiwał się już w~majtkach, popijając tequilę z~porostów z~jednej
ze szklanych żarówek, które były wszędzie w~Thetford, chociaż nie
spotkała tego płodnego dmuchacza szkła.

Machnął blado żarówką i~zawołał na powitanie. Tam warknęła na niego,
udając sierżanta, żeby się pozbierał i~zaoferował gościnę. Wstał,
znalazł gorzałkę i~kolejną bańkę -- wydłużoną jak łza, z~wirami
cyjanoniebieskiego i~rdzawopomarańczowego/czerwonego -- i~nalał. Zaczęła
machać ręką, potem złapała zapach i~ustąpiła.

\textit{Jebać to}. Wzięła palący łyk, mieszając obrzydliwe w~ustach i~pozwalając, by spływało po suchym gardle.

-- Gorące ręczniki. -- Tam pstryknęła palcami. Seth jęknął teatralnie, ale
założył pantalony i~wyszedł.

-- Nie musisz\ldots  -- powiedziała Gretyl.

-- O tak, musimy. -- Tam dramatycznie uszczypnęła się w~nos. 

Gretyl wzruszyła ramionami. Prawdopodobnie śmierdziała, onsen w~B\&B był daleko
w tyle, a~tygodnie pod ziemią po zbombardowaniu OU przyzwyczaiły ją do
podstawowego poziomu zapachu, który spełniał wszystkie domyślne
stereotypy śmierdzących odchodzących.

Tam przetrząsnęła skrzynki stłoczone w~pełzającej przestrzeni,
konsultując się z~jej interfejsem, wyciągając parę jedwabopodobnych
szat, rzucając jedną Gretylowi. Wrzuciły swoje brudne ubrania do sporego
stosu pozostawionego przez Setha, włożyły szaty i~opadły na łóżko.

Seth wjechał na izolowanej skrzyni. Otworzył pokrywkę, wypuszczając
pachnącą parę. W~kompleksie kosmików były prysznice, ale narastająca
liczba skłoniła wszystkich do odwiedzenia wiki w~poszukiwaniu alternatyw
od innych odchodzących, a~ręczniki były zwycięzcą. Nie było łatwo się w~nich kąpać, ale to była cecha, a~nie błąd, o~ile większość ludzi była
zainteresowana.

Seth opadł między nimi. 

-- W~porządku, jestem gotowy.

Tam uderzyła go w~ramię, Gretyl zauważyła, że trzyma uniesiony środkowy
knykieć, wbijając go prosto w~jego biceps. 

-- Nie. Ma. Mowy.

Potarł ramię. 

-- Au.

-- Tak -- wyjaśniła. -- Au. Chcesz kolejny? -- Zacisnęła pięść. Gretyl
zobaczyła, że oboje starają się stłumić uśmiechy. Młoda miłość.

-- W~porządku, kto pierwszy?

-- Najpierw goście -- powiedziała Tam.

Gretyl chciała się sprzeciwić, ale leżenie na łóżku, owinięta w~miękką
szatę, pozbawiło ją resztek sił. Jęcząc -- tym razem teatralnie -- zrzuciła szatę, czując na skórze gęsią skórkę, gdy dotykało ją
recyrkulowane powietrze.

Dotknął ją pierwszy ciężki, mokry, pachnący ręcznik, owinięty na jej
plecach z~mokrym klapsem, po którym nastąpiło rozprzestrzenianie się
ciepła, które było jak ospały język, a~potem dołączył do niego kolejny,
dzierżony przez Tam, na tyłach jej nóg. Tam potarła obolałe, napięte
ścięgna podkolanowe. Cztery dłonie szorowały ją w~upale, brzdąkając jej
bolące mięśnie, sprytne kciuki i~zgrzytające knykcie, łokcie w~nieustępliwych węzłach. Tam, gdzie zsunęły się ręczniki, jej mokra skóra
kurczyła się od prądów powietrza.

Zbyt szybko kazali jej się przewrócić i~zrobili jej przód, ćwicząc
mięśnie brzucha, uda, zaciśniętą szczękę i~skórę głowy. Ręczniki były
nasączone szałwią i~sosną. Zapach wypełnił pokój. Ciągle przysypiała,
rozkoszując się uwagą, a~potem obudziła się, gdy palec łapał bolące
miejsce.

Potem przyszła kolej na Tam. W~skrzyni były jeszcze gorące ręczniki.
Seth znalazł interfejs termostatu i~podkręcił ogrzewanie. Gretyl
zrezygnowała ze szlafroka, co ułatwiło sprawę, gdy nałożyła gorące
ręczniki na chude nogi i~kościste plecy Tama. Seth przyniósł więcej soku
z porostów, a~ona wylała trochę na palce, a~kiedy go zlizała,
skosztowała szałwii i~sosny. Smak był niesamowity i~tak im powiedziała.
Polali alkoholem palce i~lizali, wszyscy się zgodzili, a~także
rozluźnili się i~złagodnieli. I~stali się niedbali.

Zanim przenieśli się na Setha, upał, wilgoć i~alkohol sprawiły, że pokój
był tak pływający jak łaźnia turecka. W~schowku były suche ręczniki.
Pojawiły się ciepłe i~puszyste jak kocięta. Związani, zagrzebali się pod
kołdrą.

Gretyl zachwycała się poczuciem spokoju, intymnością, która była
jednocześnie bezpłciowa i~zmysłowa. To było jak dziecko, uczucie sprzed
seksu, a~może uczucie kogoś bardzo starego, poza seksem. Wszystko było w~pokoju.

Więc dlaczego płakała?

Od jakiegoś czasu łzy spływały jej po policzkach. Zauważyła je, ponieważ
gromadziły się w~jej uszach i~spływały po szyi. Kiedyś rozcięła dłoń
kuchennym nożem i~był taki moment, kiedy patrzyła na pulsującą krew,
rozumiała ją, ale nie czuła, zanim uderzył ją ból, radioaktywnie
intensywny i~gromowy. Krzyknęła z~zaskoczenia, nie z~powodu rany, ale
nagłego bólu.

Teraz było tak samo: rana widoczna, ból w~oddali. Przełknęła, szlochała,
a potem ryknęła, skręcając się, jakby została uderzona w~brzuch. Ból tam
był obrzydliwy. Cały jej ukryty strach i~smutek z~powodu kochanki runął.

Seth zrozumiał to pierwszy, owijając ją w~ramiona, mrucząc ciii,
kołysząc ją. Tam była wolniejsza, ale wzięła ręce Gretyla i~ścisnęła je,
mówiąc, że \textit{dobrze, wypuść to}. Gretyl była tak głęboko pogrążona w~bólu, że nie przejmowała się byciem matkowaną przez Tam.

Smutek się zacierał. Wycie syreny zagłuszyło spójną myśl. Ucichło do
tego stopnia, że mogła słyszeć jej myśli, a~pierwszym z~nich był strach,
że Lodołasica nigdy nie wróci. Jej ojciec i~rodzina zmieniliby ją w~zettę.

Burza minęła, strumienie łez zwolniły. Piekły ją oczy i~bolały ją
wnętrzności. Rozplątała się, przerzuciła nogi przez łóżko i~ukryła twarz
w dłoniach.

-- Co my robimy?

-- Masz na myśli ogólnie, czy konkretnie, tu i~teraz? -- spytał Seth, a~Gretyl poczuła, jak Tam sięga i~go szczypie.

-- Nie próbuję być zabawny -- powiedział.

-- Nigdy nie jesteś zabawny -- powiedziała Tam. -- O to chodzi.

-- Ała.

Gretyl podniosła wzrok, owinęła się szlafrokiem i~wstała, by się
przejść, natychmiast uderzając palcem u nogi o~zimną, nierówną podłogę.
Krzyknęła i~usiadła z~powrotem, pocierając palec u nogi.

-- Mam odpowiedź, wiesz -- powiedział Seth.

-- Na co?

-- Co robimy -- powiedział.

Tam westchnęła. 

-- Dawaj. Jeśli Gretyl nie ma nic przeciwko.

Potrząsnęła głową. Czuła sympatię do tych złamanych, słodkich,
kochających ludzi.

-- Kiedy byłem dzieckiem i~słyszałem o~odchodzących, zawsze wydawali mi
się szalenie optymistyczni. Gdyby kiedykolwiek poważnie zagrozili
defaultowi, to default by ich zmiażdżył. To było naiwne, myślenie, że
default może pokojowo współistnieć z~czymkolwiek innym. Jak mógłby?
Jeśli wymówką dla umieszczenia garstki bogatych dupków rządzących
światem było to, że bez nich głodowalibyśmy, jak mogliby pozwolić
ludziom żyć bez ich surowego, ale kochającego przywództwa?

-- Myślałem o~sobie jako o~realiście. Rzeczywistość miała dobrze znane
skłonności pesymistyczne, co uczyniło mnie pesymistą. Podobał mi się
pomysł odejścia, ale byłem po drugiej stronie.

Tam ścisnąła jego dłoń. 

-- Potem poszedłeś za gorącą bogatą dziewczyną do
lasu i~wszystko się zmieniło. Słyszałam to.

-- Nie jest to ważna część, ponieważ zrozumiałem to dopiero, gdy
dotarliśmy do Thetford. -- Przerwał. Gretyl pomyślała, że zachowuje się
dramatycznie, ale zbierał myśli z~nietypową wrażliwością na twarzy w~przyćmionym świetle. Chciała usłyszeć, co powie dalej. Może odkrył coś
ważnego.

-- Jeśli Twój statek zatonie na środku otwartej wody, nie poddajesz się i~nie toniesz. Walczysz w~wodzie, chwytasz się rei, robisz
\textit{cokolwiek}.

Zatrzymał się, załamał ręce.

-- Realistycznie, jeśli jesteś na środku morza, przepadasz. Ale machasz
nogami w~wodzie, aż nie możesz zrobić kolejnego uderzenia. Nie dlatego,
że jesteś optymistą. Gdyby przeprowadzić ankietę wśród dziesięciu
przypadkowych ofiar rozbitków, które walczyła na otwartym morzu, każda z~nich powiedziałaby, że nie jest optymistką.

-- To, czym są, to są \textit{pełni nadziei}. A przynajmniej nie
\textit{brak} im nadziei. Nie poddają się, bo to oznacza śmierć, a~żywi
ludzie mogą czasem zmienić swoją sytuację, podczas gdy martwi nie mogą
zmienić pieprzonej rzeczy.

-- Nigdy nie zgubiłem się na morzu, ale myślę, że gdyby Twój kumpel był
słabszy od ciebie, a~Ty go trzymałeś, kopałbyś tak samo mocno, ponieważ
miałbyś nadzieję za was obu. Ponieważ poddanie się za kogoś innego jest
jeszcze trudniejsze niż poddanie się za siebie.

Teraz jestem odchodzącym, zostałem postrzelony i~wypędzony z~mojego
domu, ale nie mogę przywrócić default ustawienia, ponieważ default jest
dno morza, a~odejście to pływający kij, który możemy chwycić. Default
nie ma dla nas żadnego pożytku, z~wyjątkiem konkurencji z~innymi
nie-zettami, kimś, kto wykona czyjąś pracę, jeśli stanie się zbyt
zarozumiały i~zażąda, by traktowano go jak istoty ludzkie, a~nie koszty
krańcowe. Jesteśmy nadwyżką w~stosunku do wymagań default. Gdyby mogli,
zatopiliby nas.

-- A więc to, co robimy, Gretyl, to ćwiczenie nadziei. To wszystko, co
możesz zrobić, gdy sytuacja wymaga pesymizmu. Większość ludzi, którzy
mają nadzieję, traci nadzieję. To realizm, ale wszyscy, których nadzieje
\textit{nie zostały }rozwiane, \textit{zaczynali od nadziei}. Nadzieja to
cena wstępu. To wciąż lotto z~gównianymi szansami, ale przynajmniej to
nasze lotto. Kopanie nogami w~wodzie w~myśleniu default, że możesz
zostać zettą, to granie w~lotto, w~którym nie możesz wygrać, a~którego
zwycięzcy, zetty, wygrywają Twoim kosztem, ponieważ grasz dalej.
Nadzieja jest tym, co robimy. Działanie z~nadziei, pływanie po otwartym
oceanie bez widocznego ratunku.

-- Więc, w~zasadzie, ,,życie tak, jakby to były pierwsze dni lepszego
narodu''? -- Ale Gretyl uśmiechnęła się, kiedy to powiedziała.

-- Wiesz, że taki cierpki cynizm to moja działka.

-- Fajnie jest być kutasem.

Odwzajemnił uśmiech. 

-- Prawda?

-- Więc to jest nadzieja. Ale\ldots  -- Westchnęła.

Tam przyniosła tequilę z~porostów. Przez chwilę pomyślała o~tym, że
używanie alkoholu do radzenia sobie z~cierpieniem to zły nawyk, a~potem
napiła się bańki. Paliła przyjemnie.

-- Lodołasica -- powiedziała.

-- Biedna Lodołasica -- powiedziała Tam. -- Czy słyszałaś od Roz?

-- Nie. Nie chcę łamać protokołu bezpieczeństwa. Za każdym razem, gdy do
niej dzwonię, zwiększa to szanse, że zostanie odkryta. Powiedziała, że
skontaktuje się, gdy coś się zmieni, gdy będzie coś, co mogę zrobić. Ale
nie dzwoniła.

-- Zadzwońmy do niej. Jebać protokół. Nie odkryli jej, kiedy zrootowała
ich sieć, cóż więcej może zrobić jeszcze jedna sesja sieciowa?

-- Nie sądzę\ldots 

-- Zróbmy to -- powiedział Seth. -- Jeśli ją uwięzili, to jest
\textit{popieprzone}. Jest naszą przyjaciółką, tonie pod falami, musimy ją
uratować.

-- Uratować ją? To szaleństwo, Seth. Jest w~pierdolonym uzbrojonym
kompleksie.

-- Wskoczyłbym w~wody pełne rekinów, żeby uratować Tam. -- Spojrzała, żeby
zobaczyć, czy się przemądrza, ale był poważny.

-- Nie bądź dupkiem, Seth. Nie sądzisz, że Gretyl biła się za to, że nie
wpadła do lochu ojca Lodołasicy? To misja samobójcza.

-- To \textit{była} misja samobójcza, bez pomocy Roz. Teraz to po prostu
szaleństwo. Daj spokój, chcesz żyć wiecznie czy coś?

-- Najpierw do niej zadzwońmy -- powiedziała Tam. -- Z~tego, co wiemy, Roz
jest gotowa ją uwolnić, tak, aby nikt nie został postrzelony.

\threeast

Połączenie z~Roz telefonem nie było proste. W~klastrze w~przestrzeniach
kosmików działała instancja Roz -- uruchomienie instancji Roz było w~dzisiejszych czasach warunkiem wstępnym, by traktować ją poważnie jako
klad odchodnicki -- ale to było powolne. Kosmiki wykorzystywali ją do
pomocy w~badaniach nad projektem przesyłania mikrosatelitów, a~ekipa
skanująca konsultowała się z~nią, aby zsynchronizować szereg tanich
skanerów w~celu wykonania potężnych obliczeń niezbędnych do interpolacji
pomiarów o~niskiej precyzji do bardzo wysokiej rozdzielczości, bardzo
dokładnej bazy danych, która zamieniała wszystkie ważne części osoby w~plik cyfrowy.

Miejscowa Roz nie wiedziała o~swojej siostrze-instancji w~otworze po
ryglu Jacoba Redwatera, ale ta zostawiła Gretyl z~listem do innych
instancji Roz, zaszyfrowanym kluczem chronionym prywatnym hasłem,
którego Roz używała w~życiu. Lokalny Roz zaakceptowała plik,
odszyfrowała go, pomyślała o~tym przez komputerowe mrugnięcie okiem. 

-- To jest szalone.

-- Tak -- powiedziała Gretyl.

-- Która część? -- spytał Seth.

-- Cała rzecz. Porwanie, infiltracja, pwnage. To jest straszne. To
przerażające. To też jest niesamowite, całe to pwnage.

-- Mocno nadęte? -- Seth powiedział to lekko, ale Gretyl widziała, że się
rozdrażnił. Nigdy nie znał Roz żywej, więc dla niego była wszechobecną
transludzką wyrocznią. Kiedy Gretyl słyszała głos Roz, wyobrażała sobie
koleżankę, z~którą pracowała, sposób, w~jaki wymachiwała rękami i~chodziła, gdy mówiła, czuła jej fizyczną obecność poprzez mentalną
iluzję tak kompletną, że wydawało się, że może sięgnąć i~złapać Roz i~ją
przytulić.

-- Nie -- powiedziała Roz. -- To nie byłem ja-ja. To była inna Roz-ja.
Angielski potrzebuje nowych zaimków. Inna Roz-ja i~ja jesteśmy i~nie
jesteśmy tą samą osobą, a~osiągnięcia, które akurat chwalę, nie są
osiągnięciami, z~którymi ja-ja miałem coś wspólnego, więc nie napawam
się, po prostu podziwiam pracę bardzo bliskiej koleżanki. Ale oczywiście
mogłam zrobić to samo.

-- Oczywiście -- powiedział Seth. Gretyl widziała jasno z~logiki rozmowy z~Roz, że go oczarowała.

-- Poza tym nie bądź kutasem dla nieśmiertelnej symulowanej martwej pani
-- powiedziała Tam. -- To złe maniery.

Gretyl nie wiedziała, czy Tam i~Roz się dogadywały, ale czuła, że
musiała tam być historia.

-- Mówisz najsłodsze rzeczy -- powiedziała Roz. -- A może zadzwonimy?

-- Poproszę -- powiedziała Gretyl. Głos był głośniejszy i~silniejszy niż
zamierzała. Jej dłonie były spocone, a~jej puls tętnił w~uszach. Może
mogłaby nawet porozmawiać z~Lodołasicą?

Chwila, potem dziwny dźwięk z~głośnika, kolejna chwila. Potem: 

-- Cześć.

-- Nie mogłaś do niej dotrzeć? -- Gretyl miała wrażenie, że tonie w~rozczarowaniu.

-- Co? Och. Nie, to ja, Roz. Mam na myśli tę w~domu ojca Natalie.

-- Też tu jestem.

-- To zbyt dziwne -- powiedziała Tam.

-- Obniżę o~oktawę -- powiedziała jedna z~Roz głębszym głosem, a~druga
powiedziała: 

-- Człowieku, to dziwne.

-- Która jest która? -- W~głowie Gretyl się kręciło.

-- Jestem miejscowa -- powiedziała niższa-Roz.

-- Jestem na miejscu -- powiedziała druga.

Tam przejęła dowodzenie. 

-- Ok -- powiedziała -- Będę nazywać Cię
,,Lokalną'' i~,,Zdalną'' podczas tego połączenia. Zgoda?

-- Zgoda -- powiedziały oba głosy w~tej samej chwili. 

Gretyl pomyślała o~swojej kopii zapasowej, która siedziała w~magazynie, zastanawiała się,
jak by to było z~nią rozmawiać, albo z~wieloma jej kopiami. Ta myśl
przyprawiała o~mdłości; chociaż możliwość pojawiała się wiele razy na
przestrzeni lat, nigdy nie była tak natychmiastowa.

-- Zdalna, co się dzieje z~Lodołasicą?

-- Rozwiązali ją trzy dni temu. Robi ćwiczenia izometryczne, kiedy nie ma
ich w~pobliżu, ale wciąż jest słaba. Nie było jej przez dziesięć dni. W~jedzeniu podają jej środki uspokajające. Zgromadzili zapasy środków
nasennych, ale nie wiem, czy zamierzają ich użyć, to sprawa wielu
frakcji, matka i~ojciec nie są zgodni co do dalszego postępowania. Spór
ma tyle wspólnego z~ich popieprzoną dynamiką męża i~żony, co z~uczuciami
do dziecka.

-- Pod względem emocjonalnym nie jest w~świetnej formie, nawet z~pomocą
środków uspokajających. Jest wkurzona, jest zdenerwowana na rodziców.
Kiedy mama ją odwiedza, odrzuca uczucie, a~może litość, na rzecz
dynamiki matki-córki ,,Nienawidzę cię!'', która się zaostrzyła.

-- Ponieważ jej matka jest współwinna jej porwania -- powiedziała Tam.

-- Tak, z~powodu porwania. Myślałam, że to oczywiste.

-- Próbuję uzyskać maksymalną jasność.

-- To jest maksymalna jasność. Jestem teraz całkowicie w~ich sieci.
Aktualizowane oprogramowanie na każdym urządzeniu podłączonym do
schronu, pozostawiło tylne drzwi. Jedynym sposobem, by mnie stąd
wywalić, byłoby spalenie wszystkiego i~postawienie od nowa. Ta sieć jest
trwale odcięta od sieci domowej. Na zewnątrz schronu znajduje się pół
tuzina czujników optycznych, dźwięku, promieniowania, jakości powietrza.
Nie jestem pewna, ale \textit{myślę}, że są fizycznie kolokowane z~urządzeniami sieciowymi domu, mogą nawet \textit{być }czujnikami sieciowe
domu, hakowanymi, żeby wysyłać drugi strumienia danych do bezpiecznego
pokoju. Być może istnieje sposób na przejęcie tych czujników i~użycie
ich, aby dostać się do sieci domowej, ale obawiam się, że to włączy
system wykrywania włamań i~wszystko zdradzi, więc trzymałam się z~daleka.

-- Patrząc na czujniki, wydaje mi się, że jest tylko jeden pełnoetatowy
bandyta z~ochrony, kobieta, która mogła być w~drużynie przechwytującej,
która złapała Natalie, tak ją tutaj nazywają. Opieram to na
podsłuchiwanych rozmowach pomiędzy Natalie i~jej rodziną. Jest też
pomocnik medyka i~asystent administratora, który zajmuje się jedzeniem i~lekarstwami. Utrzymują to na niskim poziomie, co ma sens z~perspektywy
tajności/ opsecu. Poza nimi jedynymi osobami, które wchodzą lub wychodzą
ze schronu, to matka, ojciec i~siostra.

-- Wszyscy są w~jednym pokoju? -- powiedziała Gretyl.

-- Nie, schron to kompleks: dwa wejścia, jedno przez dom, a~drugie przez
tunel prowadzący na zewnątrz. Oprócz tunelu są trzy pomieszczenia:
przedsionek, pomieszczenie, które wykorzystują jako centrum sterowania i~pokój Natalie. Pokój Natalie ma swoje własne uszczelniające się drzwi
oraz niezależne powietrze i~energię, miał być nie do zdobycia, głęboko
broniony. W~pokoju Natalie jest toaleta, a~w~pokoju kontrolnym z~ekranami naokoło znajduje się toaleta chemiczna. Asystentka ją opróżnia,
ma wysuwany nabój. Widzę, jak wymienia ją kilka razy dziennie i~robi
niesamowite miny, chociaż inni nie zwracają uwagi i~twierdzą, że nie ma
zapachu. Wszyscy myślą, że ich gówno nie śmierdzi.

-- Co oni robią z~Lodołasicą? -- zapytała Tam, ponieważ Gretyl wciąż to
przyswajała, próbując sobie to wyobrazić w~głowie. Pomyślała, że powinna
poprosić Zdalną o~zestaw zdjęć i~planów, a~potem wyobraziła sobie, że
widzi zdjęcie Lodołasicy -- Natalie! -- chudej i~odurzonej, a~jej żołądek
wykonał kolejny powolny ruch.

-- Myślę, że plan taty polegał na sprowadzeniu kogoś, kto zrobiłby jej
pranie mózgu, są zapasy zapasów i~narkotyków, które pasują do tej
hipotezy. Opierając się na rozmowach, jakie odbył z~żoną w~pokoju
kontrolnym, zawetowała to, chociaż tata nie jest szczęśliwy i~postawił
jakieś ultimatum. Nie znam szczegółów, bo nie rozmawiają o~tym przy
pracownikach, a~pracownicy nie mają dokąd wyjść poza pokojem Natalie. To
są rzeczy, które syczą na siebie w~wolnych chwilach.

-- Najdroższa mamusia odwiedza codziennie, siostra też, ale idą same.
Mama je śniadanie z~Natalie, rozmawia z~nią o~dawnych czasach, opowiada
historie, wobec których Natalie jest albo obojętna, albo wrogo
nastawiona. Staruszka zachowuje dzielną minę, ale mogę sprawdzić jej
oddech i~puls, a~Natalie działa jej na nerwy. Jest w~tym dobra. Dużo
praktyki.

-- Siostra radzi sobie lepiej, nakłania Natalie do opowiadania historii
odchodzących, stara się nie osądzać -- Seth parsknął -- współczując, jak
okropni są mamusia i~tata.

-- A co z~ucieczką? -- spytała Gretyl, pytanie, które chciała zadać.

-- Co z~tym?

Gretyl wydała zduszony dźwięk. Wydawało się, że Roz nią szarpie, ale czy
naprawdę? Nie była osobą, którą Gretyl znała, może wcale nie była osobą.
Przeszła dramatyczne doświadczenie -- zabita, przywrócona, rozwidlona,
rozgałęziona i~symulowana -- i~istniała w~programowo ograniczonym stanie,
aby uniemożliwić jej myślenie o~pewnych rzeczach. Kto wie, jakie inne
emocje zostały stłumione, ponieważ współwystępowały z~kryzysami
egzystencjalnymi? Może niepokój i~empatia były splątanymi cząstkami, a~wygaszenie jednej wygasiło obie.

-- A co powiesz na pomoc w~ucieczce od rodziny i~powrocie tutaj?

-- Och.

-- Zatem?

-- Rozmawiałam z~nią o~tym. Chciałaby, ale postrzega to jako odległą
możliwość. Mogę otworzyć drzwi, a~nawet zamknąć resztę, kiedy ona będzie
korzystała z~tunelu. Ale powrót z~Toronto gdzieś poza zasięg jej
rodziców? To eksfiltracja jak w~tajnych operacjach, a~nie ucieczka z~domu.

Gretyl wymusiła głębokie wdechy i~stłumiła rozpacz. To dlatego nie
zapytała, bo już to rozgryzła.

-- Ale możesz ją wyciągnąć\ldots  to znaczy z~domu?

-- Tak. Ma ubrania, a~jej siostra ma stopy tego samego rozmiaru.
Zakładając, że mogłaby zdobyć buty siostry, mogłaby się uwolnić, chociaż
byłoby cholernie zimno. Nie ma mowy o~załatwieniu jej zimowego płaszcza.

Miejscowa wtrąciła się głębokim głosem: 

-- Szkoda, że nie możemy jej
załatwić skafandra kosmicznego.

Zdalna przerwała i~Gretyl miała wrażenie, że wymienia dane z~Lokalną. 

-- Byłoby idealnie. Pobożne życzenia.

-- Nieważne -- wtrąciła Tam. Wiedza o~tym, co jest możliwe, jest ważna,
wiedza o~tym, co niemożliwe, mówi nam, nad czym musimy teraz popracować.

-- Nadzieja -- powiedział Seth.

-- Dryfowanie w~wodzie. -- Tam ścisnęła dłoń Gretyl.

-- Och! -- powiedziała Zdalna, a~potem -- Cholera.

-- Co?

-- Kolejna kłótnia z~ojcem. Jedna z~jego wizyt. Próbował przekonać ją, że
odchodzące są takie jak on, chciwe i~gówniane. Oczywiście powiedziała
mu, żeby się odpieprzył, a~on zaczął mówić o~Limpopo. \textit{Dużo }o niej
wie, rzeczy z~jej przeszłości, o~których nigdy nie słyszałam, niektóre
brzydkie. Natalie dobrze to znosiła, ale jest krucha, a~on naciskał, aż
rzuciła się na niego fizycznie, a~on użył przycisku uległości\ldots 

-- Co?

-- Zakuli ją w~mankiet przeciwbólowy; nieśmiercionośne rzeczy, których
używają na oddziałach psychiatrycznych, w~więzieniach i~aresztach dla
osób ubiegających się o~status uchodźcy. Rozpuszczające\dywiz twarz rzeczy. To
ma dobry system antysabotażowy. W~magazynku jest ich całe pudło, co jest
cholernie przerażające.

-- Nie żartuj -- powiedziała Tam. -- Dlaczego miałbyś potrzebować środków
przymusu w~schronie, o~którym miała wiedzieć tylko twoja rodzina?

Seth potrząsnął głową. 

-- Spotkałem faceta. Założę się, że ma fantazje
kapitana łodzi ratunkowej o~tym, że musi trzymać wszystkich w~szeregu
dla ich własnego dobra, no wiesz, jak w~,,Farnham''.

-- Ugh, nienawidziłam tego programu.

-- Wszyscy nienawidzą ten program.

-- Nie zetty. -- Seth ostro zasalutował. -- Tak, proszę pana, Farnham,
proszę pana, i~chciałbym panu podziękować, proszę pana, za pomoc w~przetrwaniu tej strasznej katastrofy poprzez wasz nadludzki osąd i~specjalną śnieżynkowatość!

Gretyl straciła oddech. Nie widziała bransoletki przymusu, ale została
trafiona bronią przymusu podczas dzikiego strajku adiunktów w~Cornell,
kiedy policjanci z~kampusu wtoczyli się na dziedziniec z~MRAP-ami,
ogłuszyli wszystkich i~zaczęli strzelać do każdego, kogo uważali za
przywódcę. Gretyl nie była na pikiecie, ale zatrzymała się, by
porozmawiać o~tym z~chłopakiem, który był jednym z~jej absolwentów,
ponieważ zawsze miał dobry instynkt w~wybieraniu bitew, a~ona chciała
ich wysłuchać.

Przypuszczała, że w~przypadku gliniarzy kampusowych każdy z~siwiejącymi
włosami jest prowodyrem -- była najstarszą osobą na dziedzińcu o~co
najmniej dziesięć lat -- i~została trafiona. Ból pojawił się dwiema
falami, najpierw ostre, kłujące uczucie na całym jej ciele, jakby
dotknęła drutu pod napięciem. Bolało, ale nie osłabiało. Później
dowiedziała się, że był to etap ,,miesiąca miodowego'' broni i~miał on
powstrzymać przestępców, ale pozostawić ich na tyle spójnych, aby
zrozumieli rozkazy, które są do nich wykrzykiwane.

Przestała mówić, rozejrzała się dziko w~poszukiwaniu źródła bólu,
zobaczyła gliniarza z~przyłbicą w~wieżyczce MRAP, jedno oko zakryte
wybrzuszoną lupą/lunetą, dolna połowa jej twarzy była niewzruszona, gdy
machała różdżką po ciele Gretyl. To automatycznie śledziło cele,
kształtując tempo, aby utrzymać promienie w~centralnej masie, gdy
sprawca szarpał się i~wił.

Nikt nie wykrzykiwał do niej rozkazów. Kilka sekund później ból rozkwitł
jak tysiąc brzytew, które wystrzeliły z~całej jej skóry, wszystkie
naraz. Nie było na to słów. Wcale nie ustało. Ból stał się tak silny,
jak tylko mógł, a~potem jeszcze trochę. To było niewyobrażalne. Boi
natychmiast zrozumiał, co się dzieje i~rzucił swój plecak, łapiąc
prześcieradło i~zakładając je na nią. Ból skwierczał,
wyłącz/włącz-wyłącz/włącz, po czym ustał, pozostawiając ją drgającą.

(Rycerskość kosztowała biednych chłopaków własne bezpieczeństwo, byli
kolejnym celem snajpera i~Gretyl zajęła wieczność, zanim odzyskała siły,
by przykryć ich kocem.)

Myśl o~Lodołasicy z~jednym z~tych mankietów -- palec ojca na guziku -- sprawiła, że miała ochotę płakać, gdy wróciły wspomnienia tamtego dnia.

Stopniowo Seth i~Tam zdali sobie sprawę z~jej zdenerwowania i~przestali
się przekomarzać. 

-- Hej -- powiedziała Tam. -- Bądź silna. Załatwimy to.

-- Tak. -- Seth brzmiał na mniej przekonanego, pomimo jego gadaniny
nadziei. -- To jest sytuacja tymczasowa.

-- Co z~nią? -- powiedziała Gretyl i~była zaniepokojona tym, jak cichy był
jej głos.

Zdalna też zauważyła. Jej głos stracił nonszalancję. 

-- Odpoczywa. Zamknięta w~sobie. -- Potem -- Chciałabyś z~nią porozmawiać?

-- Czy mogę? -- Ta myśl sprawiła, że jej serce zagrzmiało.

-- Sekunda. -- Gretyl zauważyła tik głosu Zdalnej. 

Kiedy kończyła mówić,
dźwięk ucinał się zbyt idealnie na ostatniej sylabie, gładko przycinając
się na końcu fali dźwiękowej, bez syczenia otwartego mikrofonu, podczas
gdy algorytm duplexu dźwięku upewniał się, że gąbczasty człowiek
skończył, a~nie zbierał myśli. Kiedy rozmawiałaś z~kimś hostowanym na
komputerze, metadane stawały się danymi. Zastanawiała się, jak
brzmiałaby rozmowa między Zdalną a~Lokalną, a~potem zdała sobie sprawę,
że w~ogóle nie będą używać dźwięku, a~potem zdała sobie sprawę, że stara
się odwrócić uwagę od faktu, że miała rozmawiać z\ldots 

-- Dobrze, połącz je. -- Głos był wątły.

-- Człowieku! -- powiedział Seth. -- Jak więzienie?

Tam go uderzyła. Chrząknął, a~Lodołasica odpowiedziała: 

-- Jesteś takim
dupkiem, Seth.

-- Ale musisz przyznać, że jestem sympatycznym łobuzem.

-- Przyznaję. -- Jej głos drżał.

-- Jak się trzymasz, kochanie? -- spytała Tam.

-- Ja, och\ldots  -- Pauza, drżący oddech. -- Boję się. Nie rozumiem, jak mogą
mnie teraz puścić.

-- Dostaniemy cię. -- Gretyl sama się zaskoczyła.

-- Gretyl? -- Głos Lodołasicy zadrżał bardziej, łamiąc się na drugiej
sylabie.

-- Kocham cię -- wypaliła. Łzy spływały jej po policzkach. -- Kocham cię,
Lodołasico. Idziemy po ciebie. Bądź silna.

-- Och, Gretyl. -- Teraz głośne szlochanie.

Gretyl też szlochała. Reszta czekała w~pełnym szacunku milczeniu.

-- Najgorsza część\ldots  -- zaczęła Lodołasica, po czym się rozpłakała. -- Najgorsze jest to, że robi się tak \textit{normalnie}. Jakbym była chora
od dłuższego czasu i~byłam w~szpitalu, coraz lepiej. Są chwile, kiedy
nie pamiętam\ldots 

-- Nie zapomnę cię. -- Pierś Gretyl skręciła się na myśl o~godzinach,
które minęły \textit{bez }myśli o~Lodołasicy; praca nad silnikiem, po
prostu brutalny upór materialnego świata, niedogodność pogody i~skafandra, łamigłówka w~rozwiązaniu mechanicznej zagadki porażonej
maszyny. Skupienie było dobre. To była wolność od żalu, który nosiła tak
długo.

-- Ale. -- Gretyl nie mogła mówić, szlochając. -- Ale. -- Opanowała swój
oddech. -- Jeśli to ułatwi, jeśli mniej boli, możesz o~nas zapomnieć. O
mnie. Jeśli znajdziesz sposób, by być szczęśliwa, nic mi się nie
stanie\ldots  -- \textit{O nie? }-- Zrozumiem. -- \textit{Bo Ty też tak zrobisz.} -- W~porządku.

Brak odpowiedzi, potem szloch, potem nic. Potem: 

-- Nigdy nie zapomnę.
Nigdy nie będzie dobrze. Jeśli tu umrę, umrę z~Tobą w~myślach.

-- Nie umieraj -- wypaliła Gretyl. -- Trzymaj się.

-- Trzymam się.

Świat Gretyl skierował się do nich dwóch, umysły sięgały przez
przestrzeń, przebijały ściany, przekraczały kanał utworzony przez
symulowane Roz. To było tak, jakby znów się dotykały. 

-- Ja\ldots 

-- Tak -- powiedział Lodołasica. -- Tak. Ja też. Ty też.

-- Tak. -- Straszny ciężar spadł z~Gretyl.

-- Och -- wtrąciła się Zdalna.

-- Tak? -- powiedziały razem, wciąż zsynchronizowane.

-- Mogę przeprowadzić Cię przez tunel, mogę nawet załatwić ci buty. Ale
nie mogę pomóc, kiedy będziesz na zewnątrz.

-- Wiem -- powiedział Lodołasica.

-- Spróbujmy coś wymyślić -- powiedziała Gretyl. -- Jutro jedziemy do
defaultu, rezerwat Pierwszych Narodów, dostarczamy\ldots  nieważne, co
dostarczamy. Będziemy tam dzień lub dwa. Wtedy wszyscy przyjeżdżają
tutaj, zewsząd na\ldots  -- Przełknęła ślinę. -- Imprezę. -- Czuła się, jakby
zdradzała Lodołasicę.

-- Czy mnie włączycie?

-- Co?

-- Impreza. Możecie mnie połączyć?

-- To zły opsec -- powiedziała Zdalna. -- Za każdym razem, gdy otwieramy
kanał na świat, istnieje szansa, że ktoś zauważy ruch.

-- Myślałem, że zdobyłaś całą sieć?

-- Tak, ale to jest transmisja od. Mam tutaj kontrakty na łączność,
przeczytałam je wszystkie. Są połączeni z~filią Redwater, jednym z~twoich kuzynów, tymi wielkimi. Jest to kolejna posiadłość Redwater,
miejsce po drugiej stronie wąwozu, którego używają do bezpiecznego
przechowywania, i~jest łącze mikrofalowe punkt-punkt z~zapasowym laserem
w prostej linii, więc każdy, kto wykorzystałby umowę, aby dowiedzieć
się, do którego budynku należy szturmować, żeby porwać Jacoba i~jego
rodzinę znalalazłby się trzysta metrów dalej, w~budynku ze zdalnym
monitoringiem i~przykrymi niespodziankami.

-- Dostawca musi uruchomić wykrywanie włamania. To podstawowy opsec. Jest
tolerancyjny, nie zwariował, gdy twój tata sprowadził swój zespół, ale
im większy nietypowy ruch generujemy, tym większe prawdopodobieństwo, że
uruchomi alarm w~jakimś centrum operacyjnym i~wygeneruje ostrzeżenie dla
pracowników ochrony tatusia, a~potem\ldots 

-- Rozumiem -- powiedziała Lodołasica. Wzięła drżący oddech. Gretyl
słyszała, jak bliska łez była. Łzy napłynęły jej do oczu. -- Znowu
byłabym sama, a~impreza zaczęłaby się naprawdę. Nie sądzę, żeby ochrona
taty wiedziała, co się tu dzieje. Znam tego gościa. Prowadzi ostrzejszą
firmę niż ta. Mój tata sprowadził specjalistów, deprogramatorów dla
bogatych dziewczyn, które przyłączają się do kultu odchodników. Ktoś,
kto nalegałby na prowadzenie własnego programu.

-- Jestem całkiem pewna, że masz rację -- powiedziała Zdalna. -- Pasuje do
dostępnych dowodów. Nie możemy zakładać, że Twój tata powiedziałby
swojej ochronie, żeby nie martwiła się alarmami. Nawet jeśli Boss Cop
nie wie, co Twój tata robi w~swoim lochu, musi wiedzieć, że coś się
dzieje. -- Przerwała. -- Zastanawiam się\ldots 

-- Co? -- powiedziała Lokalna. Gretyl przez chwilę była zdezorientowana.
Zaczęła myśleć o~nich jako o~aspektach jednej osoby, którą były, ale nie
w tym sensie, że oboje miały taką samą wiedzę. Zdalna mogła się nad
czymś zastanawiać, a~Lokalna nie mogła wiedzieć, co to było, dopóki
Zdalna jej nie powiedziała.

-- Jacob Redwater nie jest najgorszym zettą, nie nawet na najwyższym
poziomie, ale wciąż jest bogaty i~bezwzględny. Nie wyobrażam sobie, żeby
zrezygnował ze swojej małej dziury bez kolejnej. Założę się, że jest
inne takie miejsce, tylko 2.0 \ldots 

-- Słyszałaś, jak ktoś o~tym rozmawiał? Widziałaś jakiś ruch?

-- Nie, ale jeśli tam jest, może to coś, z~czego moglibyśmy skorzystać.

-- Wrzuć to na stos -- powiedziała Lokalna z~irytacją, co również
sprawiło, że Gretyl rozbolała głowa. Mogła się na siebie zdenerwować.
Dlaczego miałoby się to skończyć, gdy pojawiło się wiele przypadków jej
samej? -- Wróć do tego później.

-- Nadchodzą. Jacob i~jego ochrona, ta kobieta najemnicz\ldots 

Cisza.

Tam wzięła Gretyl za rękę. Gretyl nie miała okazji się pożegnać,
ponownie powiedzieć Lodołasicy, że ją kocha.

\chapter*{x}

Ostatnim razem, kiedy widziała swojego ojca, wychodził z~pokoju z~nietypowo widoczną furią. Zwykle utrzymywał stan lodowaty i~pozwalał, by
gniew pojawił się tylko jako niebezpieczny, spokojny ton. Kiedy twarz
Jacoba Redwatera wykrzywiła się w~maskę wściekłości, podniósł głos i~zacisnął pięści, był o~krok od wybuchu.

Kiedyś zadrżałaby na tę myśl. Jej matka zawsze zapewniała ją, że Jacob
Redwater jest dobrym i~cierpliwym mężczyzną, choć nie jest człowiekiem,
do którego ma jakieś szczególne uczucie. Natalie i~Cordelia były z~nim w~dobrych rękach. Wszystko, co zrobiły, co doprowadziło go do pęknięcia,
było ich własną winą.

Nie mogła mniej się przejmować jego wściekłością. Padła na ziemię,
próbując krzyczeć, gdy jej skóra płonęła, a~mięśnie się napinały, a~napad bólu przyćmił każdą emocję z~wyjątkiem użalania się nad sobą i~ogromnej furii.

Przebrał się. Nosił szyte na miarę rzeczy weekendowe, miękką flanelową
koszulę i~dżinsy, które ukrywały jego początkowy brzuch, chyba że
wiedziałeś, gdzie szukać. Pachniał mydłem z~drzewa sandałowego. Wziął
prysznic, uspokoił się i~sprowadził najemniczkę, która stała w~zasięgu
ręki i~nieco z~przodu, z~ciałem lekko obróconym w~stronę Lodołasicy,
niewzruszona, ale czujna.

-- Są rzeczy, które musisz wiedzieć o~swoich przyjaciołach, rzeczy, które
mogą pomóc ci zobaczyć, co się tam dzieje.

-- Czy to część programu? Czy Twój konsultant do łapania podał ci
dziesięciopunktowy proces deprogramowania i~to jest etap szósty?

Pokręcił głową. 

-- Możesz \textit{przestać}? Chcę przeprowadzić dorosłą
rozmowę i~przedstawić dowody. Myślę, że kiedy to zobaczysz,
zrozumiesz\ldots 

-- Dorośli nie prowadzą racjonalnych dyskusji, które obejmują porwania i~brutalny przymus. Ustaliłeś warunki, kiedy ją wysłałeś, żeby mnie tu
ściągnęła. Kiedy mnie związałeś. Kiedy użyłeś \textit{tego} na mnie.

Jej tata spojrzał na najemniczkę, rumieniec pojawiający się na
policzkach. Natalie wiedziała od Roz, że kamery w~jej pokoju zasilały
sterownię, nawet gdy był z~nią, więc najemnik, technik medyczny i~wszyscy inni słyszeli i~widzieli to wszystko. Bycie nazwanym ojcem,
który użyłby urządzenia bólu na swojej córce, nie było w~stylu Jacoba
Redwatera. Lubił być lubiany. Był lubiany, przystojny, ze swobodnym
uśmiechem i~ogromną pewnością siebie. Natalie widziała, jak przyjaciele
ulegają jego urokowi, myląc jego życzliwość z~przyjacielskością. Było
pochlebiające przyjaźnić się z~potężnym zettą, który naprawdę potrafił
cię słuchać z~intensywnością, która jasno dawała do zrozumienia, że
interesuje się Tobą, tylko Tobą.

To nie działało na Natalie, odkąd skończyła dziesięć lat.

Zrobił smutne oczy. 

-- Żałuję, że nie mogę ci powiedzieć, jak bardzo mnie
to zraniło. Wiem, że myślisz, że cię nie kocham, ale tak jest.
Próbowałem być dobrym ojcem. Wiem, że praca zbyt często trzymała mnie z~daleka. Były chwile, kiedy powinienem był przy Tobie\ldots 

Przełknęła swoją reakcję: żeby powiedzieć mu, że zawsze żałowała, że nie
ma go \textit{więcej} poza domem.

-- Ale mam obowiązki, których Ty nigdy nie rozumiałaś. Jestem gotów wziąć
na siebie winę. Próbowałem chronić Ciebie, Twoją siostrę i~matkę przed
tym, co robię, aby zapewnić nam bezpieczeństwo. To twardy świat. Nie
chciałem cię przestraszyć. -- Jego oczy zwilgotniały. To było nowe. Nigdy
nie widziała go w~łzach. Wyciągał ograniczniki. -- Natty, nie mów
siostrze, ale zakładałem, że któregoś dnia przejmiesz. Cordelia jest
uroczą dziewczyną, ale nie ma żadnej \textit{ostrości}. Ty jesteś ostra.
Za ostra. Ale to dobrze, ponieważ ten świat wymaga przewagi od ludzi,
którzy nim rządzą.

Niepewnie przesunął krzesło do jej łóżka. Usztywniła się. Nie skurczyła
się, kiedy siadał. Najemniczka ustawiła się trochę przed nim. Natalie
nie potrafiła powiedzieć dlaczego, ale to sprawiło, że poczuła się
bezpieczniej. Ona i~najemniczka ostatecznie znaleźli się po tej samej
stronie. Oboje były dłużne Jacobowi Redwaterowi, choć oczywiście
najemniczka miała znacznie większą swobodę w~zakresie warunków
zaangażowania.

-- Twoja matka i~siostra nigdy tego nie miały, ale Ty tak. Ta rodzina,
rodziny takie jak nasza, sterujemy tym światem. Ma kłopoty, Natalie. Za
dużo ludzi. Wielu z~nich to źli ludzie, którzy zniszczyliby wszystko.
Nihiliści. Nie dbają o~prawa człowieka ani prawa własności. Zabraliby
wszystko, co mamy. Zazdrośni ludzie, którzy myślą, że nic nie mają, bo
coś mamy.

-- Widziałaś prawdziwy świat. Są ludzie o~wiele bogatsi od nas. Czujemy
się komfortowo, przyznaję, ale nie jesteśmy zettami, nie prawdziwymi.
Kilka błędów, kilka zmian na świecie, wszystko tracimy. Włóczędzy na
ulicy.

-- Powiem ci, co będzie dalej: będziemy odbudowywać. Bez jałmużny.
Zabierzemy się do pracy, wymyśliśmy różne rzeczy i~wkrótce będziemy na
szczycie.

-- Świat jest chudy, wredny i~drży. Kiedy potrząsasz pudełkiem z~płatkami
kukurydzianymi, małe płatki opadają na dno, a~duże na wierzch. Jestem
cholernie wielkim płatkiem. -- Uśmiechnął się. Jego czarujący trik.

-- Wiem, co o~tym myślisz: że się łudzę. Słyszałem Twoją mowę o~specjalnych śnieżynkach. Znam Twoje argumenty. Nie zgadzam się z~nimi.
Nie znasz moich argumentów. Myślisz, że znalazłaś lepszy sposób.
Myślisz, że Twoje wędrówki mogą zmienić świat na taki bez kogoś
dowodzącego, bez dużych i~małych płatków kukurydzianych.

-- O tym właśnie chcę porozmawiać. Musisz wiedzieć o~swoich przyjaciołach
pewne rzeczy, które mogą być trudne do usłyszenia. Odchodzący mówią, że
najgorszą rzeczą, jaką możesz zrobić, to samooszukiwanie. Chcę pokazać,
jak się \textit{oszukiwałaś} na \textit{ich} temat. Nie są trudni do
ogarnięcia. Tam, gdzie są odchodzący, są zdrajcy, którzy chętnie przyjmą
darmowe jedzenie i~łatwy seks, ale chcą też pieniędzy i~mają sposób na
zdobycie obu. Odkąd odeszłaś, wiedziałem wszystko, co wydarzyło się w~Twoim małym świecie. Dostaję filmy. Byłem w~waszych sieciach. Widziałem
analizę ruchu.

Oczywiście, że to prawda. Dlaczego Jacob Redwater miałby szpiegować
mniej w~odchodzeniu niż w~przypadku defaultu? Odkąd była na tyle
dorosła, by wyjść z~domu, zawsze miała wrażenie, że ktoś na nią patrzył,
i nie ustało to gdy dotarła do B\&B. Trzeba było aktu woli, by nie
odgadnąć, którzy z~jej ,,przyjaciół'' przekazywał raporty Jacobowi;
którzy byli zatrudnieni w~rządzie, czy pracowali dla zettów lub dużych
firm. Omówiła to z~Limpopo i~Limpopo wyznała, że musiała oprzeć się temu
samemu impulsowi.

~

-- Nie chodzi o~to, że nie ma tu informatorów. Oczywiście, że są tutaj
informatorzy. Wtyczki ranią nas nie przez mówienie bogatym ludziom, co
robimy. Jebać bogaczy, całe nasze gówno jest w~sieciach publicznych.
Najgorsze, co robią wtyczki, to sprawienie, że nie ufamy sobie nawzajem,
myśląc, że nasi przyjaciele mogą być naszymi wrogami. Kiedy to się
stanie, masz \textit{pierdolone} problemy. Dyskusja jest niemożliwa, jeśli
myślisz, że druga osoba próbuje się Ciebie wyruchać. Wszystko zostaje
zniekształcone przez to spojrzenie. Czy pominęła śmieci, ponieważ była
rozproszona, czy dlatego, że chciała się posprzeczać o~obowiązkach?

-- Ta nieufność jest najbardziej żrącą rzeczą. Kiedy byłem w~default,
byłam w~tej grupie protestacyjnej, grupie afinitywnej, grafenie luźno
związanym z~Partią Anonimową, robiąc analizę danych wykresów społecznych
regulatorów, aby pokazać, że ich decyzje faworyzują branże, które
regulowali, takie pieprzone niemyślenie, ale dobrze było mieć fakty,
kiedy spotkało się kogoś, kto nie zorientował się, że gra jest
sfałszowana.

-- W~naszej grupie był facet, Bill. Bill był dziwny. Z~dystansem. Zawsze
patrzył na ciebie kątem oka. Zawsze słuchał, nie mówił, jakby robił
notatki. Martwiliśmy się. Wiedzieliśmy, że w~naszej grupie są
informatorzy. Ilekroć znaleźliśmy coś soczystego, jakiś dyrektor firmy
naftowej i~brat jakiejś żony ministra, któremu minister przekazał duże
zwolnienie podatkowe, rząd zawsze wiedział wcześniej, zarządzając
wiadomościami, zanim zostały ujawnione, co było przesadą, biorąc pod
uwagę, jak mało uwagi poświęcano wiadomościom. Władze, które są, są
dokładne. Wszystko, co może stać się zagrożeniem, zostaje
zneutralizowane, ponieważ zmiażdżenie nas kosztuje grosze, a~wokół
poruszają się zettadolary, których \textit{nie} chcą zakłócać.

-- Odizolowaliśmy Billa. Stworzyliśmy listy dystrybucyjne i~fora za
hasłami, na które nie był zaproszony. Przestaliśmy zapraszać go na
wieczory z~pizzą. Zapominaliśmy mu powiedzieć, kiedy szliśmy na piwo.

-- Bill nie był informatorem. Bill miał kliniczną depresję. Bill powiesił
się na pasku. Jego współlokatorzy nie odkryli go przez dwa dni. Kiedy
wrzucili Billa do ognia, nie było nikogo, kto mógłby zabrać jego prochy,
więc je zabrałam. Trzymałam je przy łóżku, dopóki nie odeszłam.
Przypominały mi, że pomogłam odizolować Billa. Pomogłam uczynić go tak
samotnym, że kiedy dopadła go ciemność, nie miał gdzie bezpiecznie
uciec. Pomogłam zabić Billa. Tak samo zrobili moi kumple. Tym, co zabiło
Billa, było nasze podejrzenie co do wtyczek. Najgorszą rzeczą, jaką
mogła zrobić wtyczka, to nieujawnienie naszego gówna czy wywołanie
gówna. Sami ujawnialiśmy nasze rzeczy. Byliśmy na tyle kłótliwi, że nie
potrzebowaliśmy wtyczek, żeby ze sobą walczyć. Martwienie się o~informatorów było milion razy gorsze niż najgorsza rzecz, jaką mogła
zrobić wtyczka.

~

W jej oczach pojawiły się łzy.

-- Rzeczy nie są takie, jak myślisz -- powiedział jej ojciec. Myślisz, że
znalazłaś sposób, w~jaki wszyscy mogą się dogadać bez szefów. Zawsze są
szefowie, jeśli nie wiesz, kim jest szef, nie możesz kwestionować jego
przywództwa. System tajnych szefów to system bez odpowiedzialności i~zgody. To manipulokracja.

Spojrzała na najemniczkę, zastanawiając się, czy podąża za tym, czy
docenia ironię swojego ojca -- \textit{jej ojca }-- krytykującego
społeczeństwo, twierdząc, że jest ono prowadzone zza kulis przez
mrocznych kombinatorów ciągnących za sznurki.

Zauważył jej spojrzenie. Skinął głową i~zrobił czarującą minę. 

-- Ciągnie
swój do swego, córko moja. Jeśli ja nie mogę rozpoznać spisku, kto
mógłby?

-- Kiedy wszystko, co masz, to młotek, wszystko wygląda jak gwóźdź. -- Żałowała, że to powiedziała. Po co się kłócić z~jebanym ojcem? Wygrał,
gdy tylko przyznałaś, że to była debata.

Wiedział o~tym. Uśmiechnął się szerzej, przybrał ponurą, zamyśloną
twarz. 

-- Rozumiem, co mówisz. Wszyscy widzimy siebie odzwierciedlonych w~danych. Analiza jest subiektywna. Ale Natalie, nie proszę cię, żebyś
przyjmowała to, co mówię, za dobrą monetę. Chcę, żebyś sama przyjrzała
się danym, zobaczyła, czy to, co mówię, jest prawdą. To nie jest
potworne, prawda?

-- Nie. Porwanie i~zadawanie bólu jest potworne. To tylko bzdury.

-- Rozumiem, że jesteś zła. Ja byłbym zły. Ale gdybym miał pranie mózgu
przez sektę, gdybym nie mógł zrozumieć, co się dzieje, chciałbym, żebyś
zrobiła wszystko, co mogła, abym zrozumiał, co się dzieje. Masz moje
pozwolenie na zrobienie ze mną wszystkiego, co tu zrobiłem, jeśli
kiedykolwiek znajdę się w~szponach jakiegoś irracjonalnego impulsu,
który naraża mnie na bezpośrednie i~śmiertelne niebezpieczeństwo.

Natalie powstrzymała się od parsknięcia. Nie po to, by oszczędzić jego
uczuć, ale dlatego, że szyderstwo byłoby uznaniem, kolejną okazją do
kłótni. Daj mu milimetr, weźmie parsek. Tak stajesz się zettą. Tak
został wychowany. Tak ona \textit{została} wychowana, co przerażało ją,
zwłaszcza w~\textit{tych} dniach. Wróciła do majątku ojca. W~tym domu była
taka presja, by zaakceptować łatwe usprawiedliwienia. Niektórzy musieli
być na górze, inni na dole, duże i~małe płatki kukurydziane. Poza tym
Redwaters nie byli \textit{naprawdę} bogaci; nie \textit{bogaci} bogaci, nie
jak kuzyn Jacoba, Tony Redwater.

-- Uwierz mi, gdyby był jakiś inny sposób, to bym z~niego skorzystał. Nie
chcę tego. Chcę odzyskać moją córkę. Wiem, do czego jesteś zdolna.
Dlatego trzymałem cię blisko domu, upewniłem się, że wiesz, co się
dzieje za kulisami. Mogłaś to wszystko poskładać.

Chociaż wiedziała, że jej pochlebia, to zadziałało. Niech go szlag trafi
i ją też. Znała bzdury swojego ojca. Mimo to coś w~niej przewróciło się
i spłynęło, gdy tatuś powiedział miłe rzeczy.

-- Właśnie to chcę, żebyś zrobiła. Połączyła to razem. -- Pokręcił
powierzchniami interfejsu i~kawałek ściany odsunął się, odsłaniając
ogromny ekran dotykowy, rozciągający się na całą szerokość pokoju.
Pokazywał wygaszacz ekranu, pętlę producenta przedstawiającą dzieciaki
grające w~lacrosse, blond i~chude, z~muskularnymi nogami i~końskimi
białymi zębami. Nie zetty, bo zetty nie musiały pozować do zdjęć
wygaszacza ekranu. Ale wyglądały jak zetty. Może byli aktorami. Lub CGI.

Jej tata zlikwidował obraz i~zastąpił go wykresem społecznościowym. W~środku wykresu, jak gazowy gigant otoczony tysiącem księżyców, znajdował
się okrąg oznaczony LIMPOPO {[}Luiza Gil{]}, okrągłe zdjęcie Limpopo,
wyglądająca młodziej i~krzywiąca się dziko, jakby chciała skopać tyłek
fotografowi. Wokół niej księżyce różnej wielkości były oznakowane
imionami jej przyjaciół, wszystkich odchodzących. Sam widok tych imion
sprawił, że poczuła łzy w~nieznośnej nostalgii. Poczucie bycia \textit{z
dala} od swojej prawdziwej rodziny było jak pazury gryzące jej
wnętrzności.

-- Spójrz na to, dobrze? -- Odwrócił się do wyjścia. Najemniczka podążyła
za nim, starając się mieć oko na Natalie bez chodzenia tyłem. Natalie
prawie tego nie zauważyła, bo starała się nie uśmiechać, bo właśnie
zauważyła dysk Etcetery i~maleńki font, jakiego użył system, by oddać
jego imię w~całości.

Podeszła do ściany i~pogłaskała krąg Etcetery, jakby go pieściła, a~wykres ożył, pomagając ułożyć się tak, by lepiej przekazać znaczenie.

\chapter*{xi}

-- Jesteśmy mocno i~naprawdę \textit{w vuko jebina} -- oświadczyła Tam.

Nauczyła się tego wyrażenia od Kersplebedeba, który powiedział, że to po
serbsku najgłębsze zadupie, dosłownie ,,gdzie się pieprzą wilki''. Tam
\textit{uwielbiała} to zdanie, co nikogo nie zdziwiło.

Seth rozejrzał się z~boku na bok. Śnieg zaczął padać godzinę po tym, jak
wyruszyli. Nie było tego w~prognozach meteorologicznych, normalnych
przez dziesięciolecia dziwnej pogody. Pierwsze płatki były ładne,
zamieniając zatruty krajobraz w~świąteczną kartkę z~brzóz i~sosen
pokrytych puszystym śniegiem jak mrożony piernik. Toksyczny lukier, ale
nie zamierzali go jeść, a~jak nieuchronnie zauważył Seth, cukier był dla
ciebie tylko trochę lepszy niż azbest.

Przyjaciele Pocahontas byli gościnni, choć niewiele mieli swojego. Nie
pochodzili z~jednej grupy, ale byli komuną żyjącą na terytorium, które
rząd Quebecu przekazał w~ramach odszkodowań za pobyt w~więzieniu, każde
z nich oczyszczony z~zarzutów, czasami po dziesięcioleciach zamknięcia.
Była to praca prawniczego kolektywu Mohawków w~Quebec City, a~po kilku
wygranych przypadkach zostali skontrolowani, ponownie skontrolowani,
zbadani przez Law Society, a~połowa ich prawników utraciła uprawnienia i~znalazła się w~pracy na pełny etat, ratując siebie.

Społeczność nazywała się Dead Lake. Miała kilka wiatraków i~kilka
drugorzędnych ogniw paliwowych, które mieszkańcy starannie
wyremontowali, aby działały lepiej, niż ktokolwiek mógł uwierzyć. Nawet
Gretyl była pod wrażeniem. Tam zachwycała się ich ulepszeniami. Ich
ekipa techniczna odciążyła wagon od fabów dla kombinezonów i~zaczęła je
montować. Zajęło to mniej niż jeden dzień. Tego wieczoru wszystkich
trzydziestu mieszkańców przyszło do szopy z~narzędziami, aby popatrzeć,
jak działają.

Gretyl, Tam i~Seth zostali zaproszeni na skromną kolację, drukowane
jedzenie z~surowców z~południa, ponieważ dziczyzna w~okolicy Thetford
była trucizną, a~mieszkańcy wiedzieli, że lepiej jej nie jeść. Rozmowa
była wesoła, choć sztuczna. Mieszkańcy Dead Lake uważali, że odchodnicy
są szaleni, a~może głupi, i~nie ukrywali tego. \textit{Lubili}
odchodzących i~zapewnili wspaniałą gościnność, ale było jasne, że ci
ludzie nie oceniali wysoko szans odchodzących, że coś zrobią. Dla nich
odchodnictwo było stylem życia i~hobby. Seth zjeżył się, bo był to jego
najgłębszy strach, a~także jego pozycja, \textit{on}mógł wyśmiewać
odchodników, ale kim byli \textit{ci}ludzie, aby mówić mu, co ma robić?
Ukrył swój sarkazm, ponieważ tutaj znali różnicę między żartem a~żartem
ha-ha-tylko-poważnym, a~Seth lubił żyć na tej krawędzi.

Z ulgą odszedł następnego ranka. Ruszyli na szlak do Thetford w~skafandrach, jadąc pustym pojazdem towarowym, który toczył się po
głębokim śniegu w~wolnym tempie marszu.

Śnieg zaczął padać godzinę wcześniej. Płatki, kłębiące się chmury, potem
biel.

-- Vuko jebina, co? -- powiedział. Gdzieś były drzewa, radar wozu
automatycznie je omijał, ale obracał się raz za razem. Jego systemy
unikania kolizji przestały działać. To było zdecydowanie miejsce, w~którym pieprzyły się wilki.

Spojrzał na Tam, próbując dostrzec jej twarz przez śnieg i~przezroczysty
plastikowy wizjer. Skafandry działały w~trybie zaciemniającym, migając
powoli, co ułatwiało odszukanie osoby na śniegu; odmgławiacze dmuchały
nad przyłbicami, nauszniki maski odtwarzały ostre reprodukcje
odmgławiaczy z~pozostałych dwóch masek, symfonię białego szumu pokrytą
porywistym wiatrem.

-- Nawet wilki się w~tym nie pieprzą -- powiedziała Gretyl. Siedziała z~tyłu, stukając w~mechaniczną klawiaturę, którą przyczepiła do skóry,
obserwując ekran wyświetlany na jej masce. 

-- Gówno. -- Pojazd się
zatrzymał. -- Równie dobrze możemy się zatrzymać, ta rzecz będzie gonić
za swoim ogonem, aż zabraknie jej mocy.

Dupa Setha wibrowała od drżenia silników wozu. To się zatrzymało, i~był
tylko dźwięk wiatru, dmuchaw i~dudnienie jego pulsu. Poczuł przelotny
strach: gdzie pieprzą się wilki, wieje śnieg, ziemia przesiąknięta jest
substancjami rakotwórczymi, niebo jest źródłem potencjalnej śmierci.
Gdyby tu umarł, nikt by się nie dowiedział. Gdyby wiedzieli, prawie
nikogo by to nie obchodziło. Jego ojciec zmarł, gdy miał dziesięć lat,
matka siedziała w~więzieniu od siedemnastego roku życia, a~od
piętnastego roku życia nie rozmawiali. Natalie była\ldots  Natalie odeszła.
Musiał przyznać, że prawdopodobnie już nie wróci.

Był taki mały. Byli pryszczami na twarzy świata. Niepożądani.
Nieproszeni. Sami na śniegu, na ich głupim, domowej roboty wozie, w~nowoczesnej piżamie, gdzie pieprzą się wilki.

To uczucie minęło. Skurczyło jego poczucie siebie do ukłucia szpilką, a~potem rozszerzyło otaczający go świat do ziejącej przepaści.

Świat wciąż się rozszerzał. Nie tylko \textit{on }był mały i~nieistotny.
To było \textit{wszystko}. Zetty, wszystko, co zbudowali. Wielkie miasta
świata. Szumiące sieci bezsensownych, sumujących się pieniędzy,
nieskończenie i~algorytmicznie przetasowywanych. Czyny i~kontrakty,
fabryki i~satelity, niekończąca się ropa i~kamień, trucizna na niebie i~węgiel w~powietrzu. Za tysiąc lat nikogo by to nie obchodziło.
Wszechświat nie dbał o~ludzi. Wiatr nie dbał o~to. Śniegu to nie
obchodziło. Pieprzone wilki nie dbały o~to. Gdyby zamarzł i~zgnił, jak
gnijące domy w~Thetford, nie byłoby to ani lepsze, ani gorsze niż
dożycie 90 lat i~zejście pod ziemię w~pudle z~kamieniem nad głową. Nie
byłoby to ani lepsze, ani gorsze niż to, co nadejdzie dla tych
wszystkich dupków zettów, którzy sądzili, że potrafią wyodrębnić jako
nowy gatunek i~przezwyciężyć śmierć.

Wszystko, co robili, było ludzkie. Wszystko, co robił, było ludzkie.
Tutaj, gdzie pieprzyły się wilki, to nic nie znaczyło; to znaczyło
wszystko.

-- Ałuuu! -- Było głośniejsze, niż zamierzał, ale kogo to obchodziło?
Rękawice Tama i~Gretyla zastukały w~ich hełmy, po czym włączył się
system kontroli wzmocnienia. Patrzyli, twarze ledwo widoczne za
wizjerami, kombinezony błyskające bezgłośnie w~wirujących płatkach. Byli
zirytowani, głodni, musieli się wysikać, a~on też, ale: -- Ałuuu! -- Tym
razem wyszło to \textit{głośniej}.

-- Chodźcie wilki! -- Dziki śmiech gonił za słowami.

-- Wystarczy. -- Głos Tam zawierał nutę ostrzegawczą.

-- To nie wystarczy. Chodź, po prostu spróbuj. Poważnie poważna.

-- Seth, chodź\ldots 

Gretyl zawyła, od czego zadrżały im przyłbice i~zadzwoniło w~uszach. 

-- Kurwa tak! -- Zaboksowała w~powietrze.

Tam westchnęła ciężko, spojrzała od jednego do drugiego, otarła śnieg z~ramion Setha. Napełniła płuca i~\textit{zawyła}. Seth dołączył. Dołączyła
Gretyl. Wyli i~wyli w~miejscu, gdzie pieprzą się wilki, a~Seth odkrył,
że ma łzy w~oczach, których nie mógł wytrzeć, ale to nie miało
znaczenia. Zrzucał skórę, zostawiając za sobą resztki defaultu, resztki
wiary, że pewnego dnia zapomni o~tym szaleństwie i~spróbuje znaleźć
pracę i~miejsce do życia z~nadzieją, że nikt ich nie zabierze.

-- Kocham was, ludzie. -- Ścisnął je tak, że ich wizjery zabrzęczały.

-- Au -- powiedziała Tam, ale się nie odsunęła. -- Jesteś palantem, ale my
też cię kochamy.

-- Tak -- powiedziała Gretyl. -- Większość czasu.

-- Co robimy? Idziemy

-- I~skończymy zamarznięci -- powiedziała Gretyl. -- Śnieg nie może ciągle
padać. Kiedy przestanie, pojedziemy do domu. W~międzyczasie schronimy
się w~ładowniach. Jeśli każdy z~nas weźmie po jednym, będziemy mogli
wyjść ze skafandrów, żeby się załatwić lub zjeść, a~potem wrócić do
środka, żeby nie zamarznąć na śmierć.

-- Jak to działało? -- powiedział Seth. -- To znaczy, gdzie robimy kupę?

Zastukała w~maskę silnika. 

-- Niewiele w~nich miejsca. Ale ostrożnie
możesz zesrać na zewnątrz skafandra, a~potem wrócić do środka, bez
obsrywania się. Zostanie to po zewnętrznej stronie skafandra, ale takie
jest życie w~wielkim mieście. Nie gorzej niż rzeczy, które przyklejają
się do niego podczas spaceru. Zmyjemy je, kiedy wrócimy.

-- Rozbiorę się na zewnątrz i~powieszę tyłek na śniegu. Przy takiej
ilości śniegu na ziemi nie będzie żadnych zanieczyszczeń w~powietrzu.

-- Jak ci pasuje, ale pamiętaj, że w~tych rzeczach jest tylko trochę
mocy, a~rozebranie się przy minus dwudziestu wysysa ciepło z~Twojego
ciała, które skafander będzie musiał przywrócić, inaczej umrzesz z~hipotermii. Może nadejść chwila, kiedy będziesz żałować, że nie masz
tych amperów w~baterii, kiedy Twoje palce u nóg zrobią się czarne.

-- Ta rozmowa przybrała cudowny obrót. -- Tam zeskoczyła z~silnika i~osunęła się na kolana. Zamiotła ramionami, zasypując śnieg. 

-- Nie
przejdziemy przez to zbyt daleko. A może spróbujemy powiedzieć komuś,
gdzie jesteśmy i~moglibyśmy skorzystać z~pomocy?

-- Mam zero słupków -- powiedziała Gretyl. -- Tak było, prawie odkąd
opuściliśmy Dead Lake. Aerostaty prawdopodobnie wylądowały same, gdy
zerwał się wiatr.

-- Zapakowałam kilka dronów do zestawu przetrwania. Hekskoptery potrafią
walczyć z~silnym wiatrem, ale nie ustalą położenia geograficznego,
dopóki niebo się nie przejaśni. Jednak\ldots 

-- Ustaw jeden wystarczająco wysoko, a~może odbić połączenie między nami
a Thetford -- powiedziała Tam. -- Istnieje duża szansa, że go stracimy,
kolejna decyzja, której możemy później żałować.

-- Podsumowując: powinniśmy schować się w~tych pudłach, obsrać się i~przeczekać pogodę. -- Seth odkrył, że pomysł nie brzmi tak okropnie, jak
powinien. Obrzydzenie, którego nie czuł, było częścią pakietu defaultu,
który wypłukał.

-- Zgadza się -- powiedziała Tam. -- Nie do nas należy pogoda. Fizyka to
fizyka. Śnieg to śnieg. Baterie to baterie. Czasami najlepszym
działaniem jest brak działania.

\chapter*{xii}

Roz czuła się jakby owinięta bawełnianą watą. Jej myśli kierowały się w~stronę paniki lub smutku, a~ona przygotowywała się na potok uczuć, a~to
byłoby \textit{syczenie}. Jako dziecko próbowała antydepresantów, kiedy
jej rodzice martwili się o~jej ,,nastroje''. Wiedziała, jakie to
uczucie, kiedy jej mózg nie potrafił wytworzyć chemikaliów, które
wprowadziły ją w~ten wyścigowy stan rzeczy, który
jest-zły-ale-nie-mogę-tego-naprawić-tylko-to-pogorszyć. To było uczucie
jak rzeczywistość w~odwrocie, kolory blakły, a~walka zniknęła z~jej
kończyn. Powiedzieli, że to kwestia ,,dobrania dawki''. Powiedzieli, że
było gorzej przed zaawansowaną neurosensyką, która mogła stale
monitorować jej reakcje. W~praktyce oznaczało to spędzanie ósmej klasy
co godzinę w~gabinecie pielęgniarki, aby owinąć sobie wokół czoła
jednorazową opaskę z~elektrodami, gdy leżała na kanapie i~pozwalała
maszynie pobierać krew. Jej rodzice musieli to robić w~domu, włączając w~to sesję o~23:15 każdego wieczoru. Byli w~tym tak dobrzy, że przez
większość nocy mogli dokonywać wszystkich pomiarów bez budzenia. Pomogło
to, że narkotyki sprawiały, że spała jak martwa.

Minął rok. Dostała pierwszą miesiączkę, swoje pierwsze jedynkę (z
matematyki, zawsze jej najlepszy przedmiot) i~pierwsze bicie od grupy
dzieciaków, w~skład której wchodziły trzy dziewczyny, które rok
wcześniej przyszły na jej przyjęcie urodzinowe. Wyczuwały jej nieznośną
słabość. Nic z~tego nie pozostawiło śladu. Powiedzieli jej, że leki
działają. Doświadczała pustego niepokoju, czysto intelektualnego
poczucia, że rzeczy są straszne, ale ta okropność nie ma znaczenia. To
była zdalna pilność. To sprawiało, że czuła się złowroga i~nieważna.

Uczucie było okropne, ale nie czuła się okropnie, kiedy odstawiła leki.
Wszyscy mówili jej, że nie wolno jej tego robić, bo nagły detoks może
sprawić problemy. Brak pośpiechu, który czuła we wszystkim, rozciągał
się na perspektywę zwariowania z~powodu jej własnej psychofarmakologii.

Oszalała. To było tak, jak wtedy, gdy wskoczyła na fale i~odpłynęła za
daleko, miotana falami, które ją obracały, przewracały, nie mając
możliwości przewidzenia, kiedy nadejdzie następna, pryskając i~zdezorientowana.

Bez lekarstw ogarnęłyby ją namiętności. Niewinne uwagi rozwścieczały ją
lub wywoływały łzy. Żarty były konwulsyjnie zabawne lub niewybaczalnie
obraźliwe, czasem jedno i~drugie. Starała się to ukryć przed rodzicami i~nauczycielami, ale zauważyli. Musiała wymyślić, żeby trzymać się z~dala
od leków, ukryć je pod językiem i~wypluć.

Stopniowo nauczyła się surfować po nastrojach. Uznała furie za zjawiska
odrębne od obiektywnej rzeczywistości. Były prawdziwe. Naprawdę je
czuła. Nie zostały wywołane przez żadną prawdziwą rzecz na świecie, w~którym żyli wszyscy inni. To była prywatna pogoda, którą mogła przeżywać
sama lub dzielić się z~innymi, jak chciała. Ceniła sobie pogodę i~ujarzmiała burze, zamieniając się w~derwisza produktywności, gdy fale
osiągnęły szczyt; wykorzystanie dolin do wycofania się i~przepracowania
kłopotliwych koncepcji.

Przeczytała zapisy tych sesji, kiedy obudzili ją w~komputerze, i~postradała zmysły. Czytając je, wyczuła trzask tych burz. Strasznie
wybuchły, kiedy jej umysł był uwolnionym od ciała.

Myślała o~burzach jako o~mokrych rzeczach, pochodzenia hormonalnego.
Zmapowała burze na przypływy tajemniczych płynów z~jej gruczołów. Ale
pozbawiona hormonów ciała, problemy były \textit{gorsze}. Nieposkromione.
Zastanawiała się nad tą tajemnicą, zastanawiając się, czy dyscyplina i~zwinność nie były częścią mokrą, wytrenowaną umiejętnością
wyczarowywania płynów, które nawilżały suche, obliczeniowe błędy jej
umysłu.

Przy jej pomocy ustabilizowali ją, przekładając między jej sekretnym
językiem nastrojów a~technicznym słownictwem obliczeń. Nie pamiętała
tamtych chwil, tylko dzienniki, ale łatwo było sobie wyobrazić
desperacki wyścig, by szlifować spójne myśli, podczas gdy fale paniki -- nie żyła, była salonową sztuczką z~kodem i~myśleniem życzeniowym -- narastały.

Unosząc się w~morzach własnego spokoju, doświadczyła niespiesznego
pośpiechu, tego samego sprzecznego uczucia, że wszystko jest
niepokojące, ale nie była zaniepokojona. To nie było dobre uczucie, ale
nie czuła się źle, a~to był problem.

Pomogła rozmowa ze Zdalną. Świadomość, że ktoś inny przechodzi przez te
same rzeczy, pomogła, chociaż nigdy wyraźnie o~tym nie rozmawiały.
Zdalna wydawała się taka normalna i~skupiona. To ją uratowało. Jeśli tak
normalnie i~skupiona wyglądała Roz z~zewnątrz, to prawdopodobnie też tak
się trzymała. Zdalna była rodzajem lustra. To, co w~niej zobaczyła,
uspokajało.

Pomagała w~przygotowaniach do przyjęcia, śledziła wydarzenia w~wielkiej
sali Thetford, obserwowała pogodę, rozmawiała z~kosmikami i~pracowała
nad optymalizacją klastrów i~modelowaniem predykcyjnym pod kątem
ograniczeń, jakie stosowaliby do każdego przechowywanego modelu, gdy
uruchomiliby je we własnych symach. Praca z~symem OC była edukacyjna i~przerażająca. Zazdrościła OC jego spokoju, ale w~swoim cyfrowym życiu
pozagrobowym był w~rozsypce. Był gorszy niż ona kiedykolwiek.
Współpracowali z~nią odchodzący na całym świecie.

Martwiła -- bez poczucia \textit{zmartwienia} -- o~swoich przyjaciół w~śniegu. Od pięciu godzin nie było stabilnej łączności sieciowej.
Ostatnio słyszała, że wyjechali z~Dead Lake. Mieli teraz dwie godziny
spóźnienia. Maszty mikrofalowe poza stacją sporadycznie wyłapywały
odległe wątki sygnału sieciowego, co wystarczało, aby routery zaczęły
wymieniać pliki stref i~synchronizować zegary oraz uzyskiwać najnowsze
dane meteorologiczne i~informacje o~zmianach częstotliwości, tylko po
to, by zanikać w~nieodwracalnej kaskadzie utraconych pakietów i~nieprawidłowych sum kontrolnych.

Sieć odchodnicka różniła się od default. Jej aplikacje zostały
zaprojektowane z~myślą o~odporności na awarie, zbudowane z~założeniem,
że maszyna, z~którą się łączysz, może zniknąć i~pojawić się ponownie bez
ostrzeżenia, ponieważ drony, wieże, przewody i~włókna zawiodły, wyblakły
lub były rozpieprzone. Sieć zakładała, że jest podsłuchiwana w~permanentnych warunkach wojny informacyjnej. Nalegała na handshake'i,
podpisy, aby wykorzenić man-in-the-middle. Kiedy Roz przeniosła się ze
Stanford do OU, sieć była największym szokiem kulturowym. Pod pewnymi
względami wolniej, ale bez wszechobecnych ostrzeżeń o~naruszeniu praw
autorskich, niekończących się umów na kliknięcie, podejrzanych przerw w~dostępie do ,,wrażliwych'' zasobów, gdy nasilały się globalne protesty.

Mieszkała w~sieciach odchodzących. Doceniała subtelny geniusz w~ich
architekturze. Miejsca, które stały się nieosiągalne, ożywały dzięki
szukającym samoleczącym sondom sieci, niespokojnie poszukujących nowe
sposoby na połączenie części, które zostały zatomizowane przez entropię
lub zmowę. Minusem było to, że nic nie było naprawdę niesprawne, a~wszystko, co nieosiągalne, wymagało restartu. To nie działało, ale
czasami tak, często na tyle, by próbować dalej. Roz nie myślała o~B.F.
Skinnerze od czasów studiów, ale po milionowej próbie dotarcia do Setha,
Tama i~Gretyla wyszukała ,,sporadyczne wzmocnienie'' w~ich lokalnej
przechowywanej podręcznej wiki. Tak właśnie to się nazywało: sporadyczne
wzmocnienie. Daj gołębiowi granulkę pokarmu za każdym razem, gdy
naciśnie przycisk, a~naciśnie go, gdy będzie głodny. Zmień algorytm
dźwigni, aby \textit{losowo }upuszczał kulkę, a~gołąb będzie dziobał i~dziobał, ponieważ części mózgu odpowiadające za wzór będą się starały
wymyślić sztuczkę niezawodnej wygranej.

Była zakłopotana, gdy dowiedziała się, że bycie bezcielesną świadomością
nie uodporniło jej na tak tanią sztuczkę poznawczą. Nie po raz pierwszy
pomyślała o~majstrowaniu przy swoich parametrach. Inne Rozy w~innych
miejscach zrobiły to, w~lepiej kontrolowanych warunkach, z~pewnym
sukcesem. Poddanie się tego rodzaju kruchości poznawczej było tak
niesprawiedliwe. Restart, restart, restart. Właściwie przeładowanie,
była na to \textit{szczególnie} podatna, przeładowanie, co było tak
niesprawiedliwe\ldots 

Przerwała. Duża wieża miała kontakt z~inną wieżą w~górach, z~niezakłóconą linią do łącza światłowodowego i~dane płynęły. Nic, co
dotarło do jej przyjaciół, ale ogromne połacie przestrzeni odchodzących
pojawiły się online. Cache negocjowały oportunistycznie kopiowanie
wielkich jego kawałków w~celu uzyskania lokalnego dostępu, tworząc salt
przed następnym elektronicznym głodem. Na całym świecie do drzwi pukały
maszyny przystankowe z~paczkami przeznaczonymi dla Thetford, prosząc o~pozwolenie na przekazanie ich ładunków.

Wśród nich były wiadomości. To postawiło Roz na nogi. Każdy filtr, który
miała na surowych kanałach, \textit{kurwa oszalał}.

To było Akron. Dopingowali Akron, gdy odchodzące ugruntowały swoją
pozycję, używając drukowanej opieki zdrowotnej i~jedzenia jako wizytówki
wobec swoich sąsiadów: zatwardziałych Akronitów, którzy nie mogli lub
nie chcieli opuścić martwego miasta. Upajali się filmami i~obsadami
Akronitów robiących rzeczy nie do pomyślenia, ustanawiających trwałe
miasto odchodzących, od którego nie można było odejść, z~farmami
permakultury, darmowymi białymi rowerami i~bezpłatnymi szkołami, w~których dzieci uczyły się wzajemnie uczyć i~być uczonymi przez inne
dzieciaki odchodnickie na całym świecie.

Stały się złe rzeczy. Nie dało się powiedzieć, ile z~tego stanowiła
propaganda. Akron było już pełen odchodzących i~na wpół odchodzących,
urządzających komunistyczne przyjęcia i~otwierających skłoty. Było pełne
gangów, kiepskich narkotyków, alfonsów i~przestraszonych ludzi. Odkąd
Akron stało się odchodnickie, każde morderstwo i~pobicie w~Akron były
najświeższymi wiadomościami dla wszystkich służb w~default, chociaż
przemoc i~choroby nie przyciągały uwagi przez dziesięć lat, kiedy Akron
przekształcał się w~Akron, nawet jego bankructwo oraz powołanie
administratora zetty, który zastąpił słabego burmistrza, nie zyskało
szczególnej wzmianki. Akron było czterdziestym amerykańskim miastem,
które znalazło się w~takiej sytuacji, i~nie było to największe,
najbardziej brutalne ani najbardziej popieprzone, więc jaka to była
wiadomość?

Kilka głosów krytycznych ze strony default zwróciło na to uwagę,
wskazało, że Ohio przestało prowadzić statystyki dotyczące wskaźnika
morderstw i~ogólnej śmiertelności w~Akron cztery lata wcześniej, a~wtedy
było pięć razy wyższe niż teraz, najwyższe, jakie ktokolwiek mógł sobie
wyobrazić.

Kiedy zobaczyła gównianą tonę złych wiadomości z~Akron, zablokowała je
spacją w~krainie ignoracji, ale wciąż wyskakiwały, a~nagłówki stawały
się coraz bardziej haniebne i~szorstkie, a~ona nie mogła się
powstrzymać, przeczytała jeden. Potem kolejny. Potem obejrzała filmy,
które cache już ściągnęły i~wykonały lokalne kopie, ponieważ \textit{każdy
}kanał w~Thetford tracił rozum na temat Akron.

Default maszerował na Akron: armia amerykańska i~tona prywatnych
,,wykonawców'' w~awangardzie, jadąc mechami lub pojazdami naziemnymi z~dronami, które nieustannie skanowały w~poszukiwaniu IED lidarem, falą
milimetrową i~rozpraszaniem wstecznym, ozdobionymi pomarańczowymi
koniczynami promieniowania na brzuchu, bardziej, by przestraszyć niż
spełnić jakikolwiek poziom bezpieczeństwa.

Jechali do walki z~Czterema Jeźdźcami: pornografami, mafiosami,
handlarzami narkotyków i~terrorystami. W~zależności od kanału ich misją
było aresztowanie głośnych Zetów, którzy zaszyli się w~Akron; ratowanie
przemycanych dzieci z~kręgu alfonsów; zneutralizowanie fabrykę Z-Word,
która wypompowywała bezprecedensowe ilości najnowszego analogu
zombinolu; lub, oczywiście, schwytanie krajowych ekstremistów, którzy
pracowali nad utworzeniem Amerykańskiego Kalifatu wraz ze znanymi
komórkami terrorystycznymi w~Michigan, Oregonie i~Luizjanie.

Bez względu na to, z~kim walczyli, przygotowali się na najgorsze.
,,Ukierunkowane'' uderzenia zniszczyły dwadzieścia dwa budynki w~ciągu
dziesięciu minut, zamieniając je w~gruz i~zasypując ulice śmiertelnym
deszczem spadających kamieni. Jednym z~budynków był szpital, dawniej
opuszczony i~ponownie otwarty przez odchodzących i~sprzymierzeńców, z~oddziałem położniczym i~oddziałem opieki paliatywnej, gdzie pacjentki
wybierały sposób śmierci. Wojna na słowa o~tym budynku była szczególnie
gorąca, rzekomo był on wylęgarnią środków biologicznych (które, jak
twierdziły sieci, to vaxx-drukarki, które produkowały szczepionkę
przeciwko eboli i~H1N1 bez licencji), klinika ,,zabójców'' oraz
przeprowadzał dzikie operacje chirurgiczne. Sieci default nie wspomniały
o oddziale położniczym.

Faza ,,butów na terenie'' rozpoczęła się, zanim opadł kurz, dosłownie:
boty pacyfikujące, które oszołomiły każdego, kto miał przy sobie broń
lub którego dane biometryczne twarzy były wystarczająco dopasowane do
,,celu o~wysokiej wartości''. Gdy bot kogoś zaatakował, nadawał głośne
komunikaty ostrzegające wszystkich, aby trzymali się z~dala, a~następnie
stał na straży nieprzytomnej ofiary, dopóki nie przybył oddział
przechwytujący naziemnie lub na linach ze śmigłowców.

Sieć odchodzących w~Akron przeżyła atak cyberwojny, najpierw pociski,
które zniszczyły końcówki światłowodów, a~następnie aerostaty kierujące
się na nadajniki, które namierzyły bezprzewodowe maszty i~wysadziły je
za pomocą zaszumionych impulsów RF. Poziom szumów fal radiowych w~granicach miasta wzrósł do punktu, w~którym wszystkie urządzenia zaczęły
zawodzić.

To był nacisk; potem przyszło odepchnięcie. Odchodzący i~Akronici,
którzy przejęli kontrolę nad miastem, przygotowywali się na taki
szok/trwogę. Mieli bunkry, autonomiczne lasery wyszukujące aerostaty,
kopie zapasowe z~włókien offline, które łączyły się z~przekaźnikami
mikrofalowymi daleko poza miastem, kamery wykrywające okrucieństwa w~trybie offline, które automatycznie rejestrowały materiał, gdy sieć była
wyłączana, prymitywną broń HERF, która gromadziła ogromne ilości energii
słonecznej, gdy tylko świeciło słońce, gotowe do wyładowania jej z~potężnym hukiem, gdy tylko wyczują wojskową komunikacją w~szerokim
spektrum.

Gdy rozeszła się wieść o~Akron, pojawiła się również reakcja
internetowa. Odchodzące na całym świecie uderzały w~komunikatory i~infrastrukturę wykonawców w~awangardzie, DHS, DoD, wewnętrzne sieci
Białego Domu, backchannelach DNC, sieciach rozmów Seven Eyes, całym
świecie super-rosa i~sub-rosa defaultu. Sieci szkieletowe odchodników
priorytetyzowały ruch wychodzący z~Akron i~automatycznie dublowały go na
wielu kanałach.

To wszystko było według scenariusza. Przez dekadę odchodzące były
sprzymierzone z~comiesięcznymi gezi, które pojawiały się w~tym czy innym
kraju. Stworzyły naukę polegającą na reagowaniu na autorytarne ataki,
przegrupowując się po każdym powstaniu, by wypracować nowe środki
zaradcze przeciwko nieskończenie udoskonalanym procedurom utrzymania
porządku publicznego.

Różnica polegała na tym, że te odchodzące otrzymywały pełny zestaw. Nie,
żeby default wcześniej nie prowadził totalnej wojny z~odchodnikami, ale
odchodnicy zawsze rozwiązywali problem, odchodząc. Default wytworzył
nieskończoną nadwyżkę stref utraconych, miejsc superfunduszy, ziemi
niczyjej i~martwych miast dla odchodzących. W~pierwszym przybliżeniu
wszystkie zniszczone nieużytki były wymienne.

Pozostawanie na miejscu nie było doktryną odchodzenia, ale w~niedawnej
historii planety było wielu innych ludzi, którzy okazywali irracjonalne,
głębokie przywiązanie do nieruchomości, w~której ostatnio się
zatrzymali. Taktyka została zrozumiana.

Każde gezi kończyło się tak samo. Chmury gazu łzawiącego, brak jedzenia
i lekarstw; zbieranie zranionych i~głupkowate obietnice zettów zwabiły
wszystkich z~ulic do tego, co zostało z~ich domów. Poczyniono nieznaczne
ustępstwa i~wszyscy zgodzili się, że coś zostało zrobione i~nadszedł
czas, aby przejść dalej.

Wszyscy wiedzieli, że nie do tego zmierzały te odchodzące. Nawet zetty.
Zwłaszcza zetty. Faza szoku i~strachu była najbardziej brutalna w~historii, bronie śmiertelne i~nieśmiercionośne mieszały się
bezkrytycznie. Nawet oswojona prasa default była trzymana z~daleka z~powodu obaw przed wszami i~innymi bio-środkami. Gubernator Ohio zawiesił
legislaturę stanową do czasu zakończenia ,,stanu wyjątkowego''.

To było denerwujące. Nagrania odchodzących z~Akron miało desperacki
klimat. Każda twarz, nawet odważna, wyglądała na skazaną. Najgorsi byli
odważni.

Roz znała kilku ludzi w~Akron. W~Akron była Roz, a~przynajmniej kiedyś
była. Niedawno zsynchronizowała się ze swoim bliźniaczką i~bała się o~nią, co było irracjonalne. Ludzie w~ciele, których znała, wspierali ją
od czasu ogłoszenia projektu Akron. To było najbardziej niepokojące.
Odchodniczki nie ustępowały, ponieważ nie bały się śmierci. Chociaż
nigdy nikomu tego nie powiedziała -- nawet innej instancji Roz -- myślała
o Akronitach jako o~kulcie śmierci. Byli nieustraszonymi samobójcami,
którym zagwarantowano życie pozagrobowe. Kanały default sugerowały to,
nie mówiąc tego, ponieważ oficjalne stanowisko default głosiło, że
transfer umysłu -- w~każdym razie transfer według odchodzących -- był
oszustwem i~blagą. Były to chatboty ze specyficznym słownictwem,
wystarczająco przekonujące, by oszukać łatwowiernych i~zdesperowanych
ekstremistów, którzy odwrócili się od wszystkiego.

Roz była babcią tych odchodzących i~każdego, kto uważał śmierć za inny
sposób na odejście od zettów i~ich obłąkanych idei bogactwa, które miało
znaczenie tylko wtedy, gdy miał więcej niż wszyscy inni. Byli jej
duchowymi dziećmi. Stanowiła dowód na to, że śmierć była początkiem, a~nie końcem. Nigdy nie mówiła nikomu, żeby wziął backup i~rzucił się pod
celownik wroga. Nie musiała. Jej istnienie wystarczyło.

Musiało być tak wiele Roz działających w~laboratoriach cyberwojny
default. Tak właśnie myśleli. Byłaby ostatecznym jeńcem. Aby torturować
ją, by się podporządkowała, wystarczyłoby poprawić parametry jej
spojrzenia w~przyszłość, więc egzystencjalne przerażenie raz po raz ją
rozbijało, pod wysokimi falami, bez zatapiania. Świadomość, że jej
legion sióstr jest groteskowo torturowany, rozwścieczyła ją, ale bez
wściekłości, dzięki zabezpieczeniom w~podglądzie. Zastanawiała się, czy
jej torturowane siostry doświadczyły intensywności, której jej
brakowało, czy potajemnie się z~tego cieszyły.

Nie można było określić, kto ,,wygrywał'' w~Akron. Jak wszystkie gezi,
była to wojna percepcji i~konflikt militarny. Czy szeregowi defaultu
zobaczyliby zasłużone desery, gdy Akron został zniszczony? A może
uznaliby za zwycięstwo default jako dręczącego Goliata, który ściera
Davida takiego jak oni pod stopami? Czy partyzanci byliby postrzegani
jako dzielni Ewokowie zabijający Imperialnych Wędrowców, czy jako
terroryści używający IED do zabijania amerykańskich patriotów koloru
serwatki? Default był świadomy mediów. Jedyna prasa z~pieniędzmi na
pokrycie czegokolwiek była własnością tych samych konglomeratów, które
były właścicielami wykonawców z~awangardy najeźdźców.

Każde gezi kończyło się mieszaną porażką. Każdy gezi wysyłał coraz
więcej ludzi w~odchodzenie, przekonanych, że żadna reforma nie uratuje
defaultu. Przekonani ludzie na szczycie nie mogli kontemplować świata, w~którym nikt nie musiał być biedny, aby oni byli bogaci. Każde gezi
kończyło się wielką liczbą ludzi wystraszonych na kolejny sezonem
uległości, kciuk na ich wadze, który równoważył ryzyko wypowiedzenia i~sprawiał, że można było tolerować porozumienie.

Jaki efekt mieliby męczennicy Akronu? Czy oportuniści wpadliby w~furię z~powodu rzezi i~wybiegliby na ulice, ponieważ nie chcieli być częścią
systemu, który to zrobił? Czy ten sterroryzowałby ich, by siedzieli
nieruchomo, żeby nie dołączyli do umarłych? Czy byliby przekonani, że
sprzeciwianie się defaultowi było samobójstwem, niezależnie od
mistycznych przekonań w~,,pierwszych dniach lepszego narodu'' i~elektronicznego życia pozagrobowego?

-- Widziałaś to? -- Limpopo wezwała ją z~sali, gdzie przygotowania były
prawie zakończone. Sala była obwieszona improwizowanymi chorągiewkami,
przystosowanymi do grzmiącej muzyki tanecznej i~ucztowania z~wytłaczarek, które przerabiały przysmaki z~ogromnego magazynu przepisów
odchodnickich.

-- Akron? To jest straszne.

Obserwowała Limpopo przez czujniki -- światło widzialne, lidar,
elektromagnetyczny. Etcetera był z~nią, z~oczami przyklejonymi do
ekranu, który zdjął z~mankietu koszuli i~przykleił do boku beczki piwa.
Etcetera trzymał dłoń Limpopo. Uczucie samotności odwiedziło Roz,
duchowy ból braku fizycznych doznań i~dłoni kochanka.

-- Akron jest gorzej niż okropny. -- Podczas burzy sala imprezowa
wypełniła się ludźmi, którzy pracowali przy maszynach, muzyce i~jedzeniu, gdy sieci były wyłączone. Teraz łączność wróciła, wrócili do
swoich ekranów. To była dziwna hybryda starożytnych i~współczesnych
rytmów. Starożytni ludzie pracowali, gdy świeciło słońce, spali, gdy
zachodziło, pozostawali w~domu, gdy wiały burze, i~orali przy ładnej
pogodzie. Sieci odchodzących były sieciami środowiskowo nieprzerywalnymi
i niedeterministycznymi, więc robiły to samo: nieskończenie komunikowały
się i~obliczały, gdy sieci działały, a~gdy pogoda lub świat zepsuł
sieci, wykonywały zadania obliczeniowe.

Wszyscy w~sali imprezowej byli przyklejeni do ekranu lub interfejsu,
niektórzy w~małych grupach, niektórzy sami. Rzucali sobie nawzajem
wiadomościami, szeptali z~podnieceniem, posyłając wiadomości dla
odchodzących w~Akron \textit{Bądź bezpieczny Bądź odważny Jesteś w~naszych
sercach To, co oni ci robią, oni robią nam Nigdy cię nie zapomnimy}.

-- Szkoda, że nie mam kontroli nad Twoim oprogramowaniem. -- Oddech
Limpopo był urywany. W~Akron było więcej zgonów, nowe rewelacje, gdy
dron przeleciał nad miejscem bombardowania, gdzie mechy przesuwały gruz,
odzyskując ciała i~części ciał. Pierwszy kanał padł, gdy dron został
zestrzelony. To przyciągnęło stado dronów-samobójców, które poświęciły
się, by schwytać i~przekazać wszystko, czego nie chciały być widzialne
moce defaultu. Powaliło je więcej strzałów . Drony na dużych
wysokościach wleciały do środka, obrazy z~kamer były bardziej
poszarpane, ponieważ rejestrowały z~większej odległości, przy nie do
końca ustabilizowanym powiększeniu. W~gruzach były dzieci. Limpopo
zapłakała. Etcetera zapłakał. Roz chciała zmienić swój kanał, pokazać im
doxxingi pojawiające się w~darknetowych forach, osobiste fakty z~życia
kontrahentów i~żołnierzy, których twarze zostały oznaczone na podstawie
materiału filmowego, otwarte listy pisane do ich matek i~ojców,
małżonków i~dzieci z~pytaniem, jak mogli to zrobić swoim bliźnim.

Te doxxingi również pochodziły z~podręcznika gezi. Czasami działały.
Nawet jeśli tego nie działały, zdarzały się nieoczekiwane rzeczy. Dzieci
opuszczały dom, ujawniając prywatne dokumenty rodziców, wplątując ich
przełożonych, publikując tajne zasady angażowania w~walkę z~instrukcjami
używania środków śmiercionośnych, gdy kamery są wyłączone, zakopywania
dowodów lub wplątywania rebeliantów w~okrucieństwa. Czasami rodzice
wyrzekali się dzieci, które wykonały brudną robotę zettów, publicznie
wyrzekając się rzezi. Dzieliło to rodziny i~wspólnoty, ale też
gromadziło nowe. Było to kontrowersyjne, ponieważ dotyczyło tak wielu
niewinnych osób i~było brudną sztuczką, ale dla Roz było w~porządku.
Nawet kiedy żyła, była gotowa rozbić te jajka, by zrobić omlet. Jako
martwa osoba działająca na serwerach na całym świecie, w~tym kilku wrogo
nastawionych do niej i~do wszystkiego, w~co wierzyła, nie mogła wydusić
z siebie wirtualnej śliny w~wyrazie współczucia dla ludzi, którzy byli
smutni, gdy tatuś został zdemaskowany jako zbrodniarz wojenny.

Kersplebedeb cicho pisał na klawiaturze i~mruczał do mikrofonu.

-- Powinniśmy być gotowi do drogi. -- Objął ramionami ramiona Etcetery i~Limpopo. -- Było więcej ataków. Dwa w~Ontario, trzy na Wyspie Księcia
Edwarda, kilka w~północnej Kolumbii Brytyjskiej i~Nunavut. Niektóre były
duże, inne małe, ale nikt się tego nie spodziewał. Kilka było
stabilnych, ci z~Wyspy istnieli dwadzieścia lat, mieli dobre relacje z~otaczającymi ich normalnymi ludźmi. Teraz zniknęli, nie ma nawet
krateru. Zeskrobani do czysta.

-- Czy słyszałeś coś konkretnego o~Thetford? -- powiedziała Roz. -- Czy coś
nadchodzi? -- Mówiła z~bransoletki Limpopo, podnosząc głos wystarczająco
głośno dla Kersplebedeba. Zamrugał i~pojął fakt, że tam była.

-- Nic -- odpowiedział. -- W~tej chwili w~powietrzu prawie nic nie ma, więc
gdyby coś się zbliżało, nie zobaczylibyśmy tego. Jeśli coś się zbliża,
śnieg mógł to opóźnić. Myślę, że powinniśmy być gotowi, jeśli i~kiedy.
Nigdy nie wierzyłem w~Big One, ale czuję się jak w~Medium.

-- Co to jest Big One? -- Roz prawie wskoczyła, by odpowiedzieć Etcetera,
ale puściła Limpopo.

-- To rzeczy odchodzących pierwszej generacji. Teoria była taka, że
default uznałby, że jesteśmy zbyt niebezpieczni, by istnieć, i~przeprowadziliby skoordynowany atak na nas wszystkich, wszystkich naraz.
Zabili lub aresztowali wszystkich, zakończyli ruch za jednym zamachem.
Mają moc strachu, aby wiedzieć, kim i~gdzie wszyscy jesteśmy, więc
jedyną rzeczą, która ich powstrzymuje, jest kaprys lub brak gonadowego
hartu ducha.

-- Myślałem, że to tylko ja się tym martwiłem -- powiedział Etcetera.

-- Kiedyś o~tym dyskutowano. Myśleliśmy, że nas wykończą. Potem nie
robili tego, nie robili i~nie robili. Spekulowaliśmy, czy nie byli
skłonni zaryzykować, że grzeczni chłopcy i~dziewczęta defaultu, widząc,
że ta bezwzględność jest nie do zniesienia, wyjdą na ulice z~widłami?
Czy to dlatego, że lubili, aby kozy i~owce same się dzieliły? Czy
potajemnie slumsowali i~ogladali mięso w~onsen, jedli extruder-chow,
pili coffium i~bawili się w~cyganerię? Czy w~odchodzeniu było za dużo
dzieciaków zettów, za dużo niebieskiej krwi, by można je było przelać w~Ostatecznym Rozwiązaniu?

-- Nienawidzę kremlinologii -- powiedział Etcetera. -- To była obsesja
moich rodziców. Drugie i~trzecie zgadywanie, jakie były prawdziwe siły
stojące za Anon Ops, kto pociągał ich za sznurki i~dlaczego.

-- To też nie jest mój ulubiony temat, ale jest różnica między obsesją na
punkcie liści herbaty a~próbą ustalenia, czy następny pocisk leci w~Twoją stronę -- powiedział Kersplebedeb. -- Zapakujmy zapasy i~schowajmy
je przy drzwiach, sprawdźmy i~naładujmy pojazdy, upewnijmy się, że
wszyscy mają skafandry.

-- Nie możemy się temu sprzeciwiać. -- Limpopo zdjęła ekran Etcetery i~wsadziła go do kieszeni. -- Roz, możesz pomóc? Powiedz wszystkim, rzuć
tablicę do śledzenia, co zostało zrobione i~co trzeba zrobić?

-- Już to robię. -- Roz nigdy nie przestał wierzyć w~Big One. Nikt, kto
pracował nad przesyłaniem i~symulacją, nie przestał, to była
niewypowiedziana motywacja stojąca za projektem, jedyny sposób, aby mieć
pewność, że zetty nie dokonają ludobójstwa, był taki, że wiedzieli, że
wrócisz jako nieśmiertelne duchy w~maszynie, by nawiedzać ich aż po
krańce Ziemi.

Nawet gdy to robiła, martwiła się o~Akron i~zastanawiała się, co się
dzieje z~Tam, Gretyl i~Sethem.

\chapter*{xiii}

Alarm Setha budził go do sprawdzania śniegu co godzinę, aby najpierw
sprawdzić, czy można bezpiecznie ruszyć; po drugie, aby upewnić się, że
nie został pochowany pod nieruchomym dryfem. Pozostałe obie ustaliły
dwudziestominutowe przesunięcie. Udało mu się zdrzemnąć się przez
pierwszą godzinę w~niewygodnym kokonie. Dzwonek obudził go gwałtownie.
Doświadczył niemal paniki, gdy próbował domyślić się, gdzie dokładnie,
kurwa, był. Przerażenie tak go zmobilizowało, że nie był senny, kiedy
wrócił, więc grał w~starą akustyczną minigrę, od której był uzależniony
jako dziecko, dopasowując się do rytmu i~wysokości tonów w~słuchawce za
pomocą stukania palcami i~gwizdania.

Powierzchnie interfejsu skafandra były od trzech generacji różne od
tych, dla których została zaprojektowana gra i~były wyspecjalizowane do
zupełnie innych celów niż interfejsy, przy których dorastał. Gra była o~\textit{wiele }trudniejsza, dopóki nie zmodyfikował sposobu rejestracji
interfejsów.

Granie budziło w~nim nostalgię za setkami godzin spędzonych w~grze,
dopóki nie przypomniał sobie, dlaczego przestał grać, pokonał innego
dzieciaka, Larry'ego Pendletona, z~którym był peryferyjnie związany,
część tej samej masywnej dziewiątej klasy w~Kolegiacie Jarvis. Nie znał
dobrze Larry'ego, ale czasami znajdowali się w~tej samej grupie i~uznał,
że Larry jest, jeśli nie fajny, to przynajmniej nie-głupi.

Jednak wtedy Larry powiedział: 

-- Hej, dobra gra, Seth. Chyba jednak masz
naturalną przewagę.

Wszyscy albo nie zrozumieli, co miał na myśli Larry, albo udawali, że
nie rozumieli. Seth natychmiast zrozumiał: 

-- Ponieważ jesteś czarny,
jesteś lepszy w~grach rytmicznych. Bo wiecie, czarni mają rytm, wszyscy
wiedzą\ldots  

Seth zobaczył, że Larry zrezygnował z~tej uwagi w~sposób
obliczony na wiarygodne zaprzeczenie, możliwość przesunięcia się, by
twierdzić, że to nie rasizm, że Seth jest przewrażliwiony i~społecznie
sprawiedliwy.

Niewypowiedziana umowa z~jego białymi przyjaciółmi polegała na tym, że
nie wolno mu było mówić o~byciu czarnym, z~wyjątkiem najlżejszych
żartów. Uznanie, że był czarnym facetem w~ich białym tłumie, było
równoznaczne z~oskarżeniem ich o~rasizm: \textit{Dlaczego jestem tutaj
jedyną czarną twarzą}? To była umowa, którą wszyscy rozumieli i~nikt o~niej nie mówił, zwłaszcza dzieci Azjatów i~Desi z~ich kadry, ponieważ
wszyscy mieli nie zauważać wyścigu, a~bycie Mniejszością Gniewną było
dla wszystkich straszne.

Gotował się ze wstydu i~złości na pieprzonego Larry'ego Pendletona,
który miał spóźnienie o~dziesiątki lat i~być może umarł z~powodu czegoś
odpornego na antybiotyki albo siedział w~więzieniu albo wykonywał
niepewną pracę i~miał nadzieję, że nie zostanie zwolniony, co każdy z~nich robił. Ale stłumił wstyd i~złość, że udawał, że nie zauważył
rasizmu, udając, że nie zawsze będzie na okresie próbnym.

Spędził tę godzinę, zadając sobie pytanie, czy jest czarnym facetem, czy
jest odchodnikiem, czy czarnym odchodzącym, czy kimś innym, czy
wszystkim z~powyższych. Nie było to pytanie, które często zadawał.
Myślenie o~tym rozzłościło go. Nie lubił być zły. Lubił być zabawny i~napalony, beztroski, wiecznie niedoceniany, co miało wiele zalet. Bycie
uważanym za nieszkodliwego -- ,,on jest czarnym facetem, ale jest fajny,
nie robi z~tego problemu'' -- było czymś, co wcześnie kultywował.
Oznaczało to, że słyszał i~widział rzeczy, których nie widzieli jego
czarni przyjaciele. Wiele z~tego był zwykłym rasizmem. Niektóre były
dobre. Musiał być kimś więcej niż swoją skórą.

Utknięcie w~pudle doprowadzało go do szaleństwa. Myślał tylko o~kolorze
skóry. Nie widział nawet swojej skóry w~ciemności. Potem była ta gra
rytmiczna, Thumperoo, w~którą grał przez cały czas, dopóki jego
nadgarstki nie dostały urazu typu RSI.

Sprawdził godzinę. Czterdzieści jeden minut, zanim miał wystawić głowę.
Westchnął. Jego nadgarstki bolały zbyt mocno, by móc dalej grać i\ldots 

Właz otworzył się i~nad nim uśmiechnęła się twarz Tam, której słońce
odbijało się od jej przyłbicy, zasłaniając jedno z~jej oczu i~jedną z~kości policzkowych. Ale poznałby te usta wszędzie.

-- Chodź, Śpiąca Królewno. Książę z~Bajki jest tutaj, aby obudzić twoją
dupę.

Pomogła mu. Chmury burzowe rozwiały się, pozostawiając błękitne niebo,
ciemniejące wraz ze zbliżającym się zmierzchem. Ukośne, późne słońce
sprawiało, że świeży pył błyszczał, jakby został posypany diamentowymi
odłamkami. Gretyl stała w~pudrze po uda. Opadła na plecy i~zrobiła
anioła. 

-- Dzięki Bogu to już koniec. Vuku jebina!

Przyłożył dłonie do przyłbicy -- dla efektu -- i~zawył.

-- Wóz nie wróci, dopóki to nie zamarznie lub się nie stopi. Stąd to na
rakietach śnieżnych. -- Gretyl strzepnęła śnieg z~plandeki, którą
zarzucili na sprzęt ratunkowy, kiedy go wypakowali, żeby zrobić miejsce
dla swoich ciał.

Pociągnęła za plandekę. Seth i~Tam podeszli pomóc. Przejrzeli schludne
tobołki, aż znaleźli rakiety śnieżne. Żaden z~nich nigdy nie składało
butów i~nie mogli tego rozgryźć. Seth walczył dalej, aż znalazł aerostat
i wysłał go w~górę, szukając sygnałów odchodzących umożliwiających
połączenie do skafandra. Patrzyli, jak miota się, wiruje i~halsuje,
zmieniając się w~kropkę na ciemniejącym niebie. Ich skafandry zaczęły
wywoływać powitalne, podprogowe brzęczenie interfejsu, gdy przyjmowały i~wysyłały wiadomości. Odrzucali przychodzące alerty przez minutę,
usuwając rzeczy, dopóki nie uzyskali odpowiedzi na najczęściej zadawane
pytania dotyczące rakiet śnieżnych.

Gretyl dotarła tam pierwsza. Rzuciła ramkę buta na śnieg w~szczególny
sposób, więc wylądowała częściowo osadzona, a~następnie kliknęła
mechanizm, którego żaden z~nich nie zorientował się, że działa, dopóki
nie zobaczyli wideo. But odskoczył i~wyrzucił w~górę śnieg, padając
płasko na powierzchni. Rozłożyła wiązania, a~potem zrobiła to samo z~drugim butem.

Seth i~Tam też rozłożyli swoje buty. Wszyscy zaangażowali się w~mimowolny slapstick, gdy starali się je założyć. W~końcu Tam podeszła do
Setha, który wpadł w~śnieg i~był na wpół zanurzony z~nogami w~powietrzu.
Chwyciła jedną z~jego stóp i~wepchnęła ją w~więzy, potem to samo zrobiła
z drugą, po czym podciągnęła go do pionu. Wyciągnął buty ze śniegu,
położył je na wierzchu i~ku swej radości stwierdził, że został na puchu,
który skrzypiał pod taśmą butów. Pokazał Tamowi podwójne kciuki w~górę,
a ona podała mu swoje buty, opadła na plecy i~wystawiła stopy w~powietrzu.

Nie był tak dobry w~zakładaniu ich jak Tam, ale było w~porządku.
Bezchmurne niebo, zakopanie w~kapsułach ładunkowych i~myśl o~spacerze po
lesie na tych odjazdowych protezach na świeżym powietrzu sprawiły, że
zakręciło im się w~głowie. Żałował, że nie może zabrać jej do kapsuły,
rozebrać się i~wypieprzyć jej mózg. To była pocieszająca jurność.
Skafander był zaskakująco dostosowany do jego erekcji. Udało mu się
musnąć dłonią krocze Tam, pomagając jej wstać -- robili to często z~zapałem dzieciaków ze szkoły, które właśnie znalazły swoich pierwszych
kumpli do pieprzenia i~nie mogą uwierzyć, że mają całe ruchanie, jakie
chcą, 24 godziny na dobę -- ale garnitury były zbyt wyściełane, by mógł
stwierdzić, czy ona też ma wzwód. Zdecydował, że tak, niedawno zmieniła
hormony, a~erekcje były mile widzianym efektem ubocznym nowego reżimu,
który oboje lubili.

Trzymali się za ręce -- ścisnęła, co jeszcze bardziej go podnieciło -- i~podeszli do Gretyl, która skrzywiła się na jej buty, stąpając szerokim
kręgiem, gdy próbowała je założyć. Okrążyli ją, a~ona patrzyła od
jednego do drugiego.

-- O nie -- zaczęła, po czym Tam postawił za nią jedną nogę w~śniegu, a~Seth pchnął, a~ona poleciała z~nogami w~powietrzu, wyjąc z~udawanego
oburzenia. Przypięli jej buty do stóp, gdy chichotała, wyprostowali ją.

Seth spojrzał na swoją głowę do góry, wyciągnął wyszukiwarkę i~wskazał.

-- Tam -- powiedział.

Wędrowali dalej. Gdy słońce zaszło, aktywowali noktowizory i~obserwowali, jak światło gwiazd i~algorytmy uwydatniają wszystko, co
mleczne, świecące w~bajkową krainę.

\chapter*{xiv}

Natalie dużo drzemała. Może miała depresję, a~może w~jej jedzeniu był
narkotyk, chociaż Roz nie znalazła już żadnych zapisów tego w~dziale
zarządzania pacjentami w~centrali.

Może to był mechanizm obronny jej umysłu, wyłączający ją w~obliczu nudy
i frustracji. Jej przyjaciele powiedzieli, że po nią przyjdą. Gretyl
obiecała, ale minęły dni, odkąd się od nich dowiedziała. Jej ojciec
przestał odwiedzać. Nie wiedziała, czy to dlatego, że dostała się pod
jego skórę, czy dlatego, że rozwalił miasto dla jakiegoś interesu, co
zawsze robił. Mama i~Cordelia odwiedzały ją regularnie, sterylne pół
godziny. Za każdym razem, gdy wychodziły, przysięgała \textit{następnym}
razem, że ich zmrozi, usiądzie w~kamiennej ciszy.

Ale potem przybywały i~rozpoczynały wyćwiczoną uprzejmość: ,,Och, Natty,
to był taki dzień'' i~uśmiechały się, a~ona była dziewczyną z~Redwater
wśród dziewcząt z~Redwater, siostrzaną damą, która je obiad i~której
nigdy nie pozwolono by na więcej. Jej matka tęskniła za Grecją i~często
spędzała całe pół godziny na monologu na temat konkretnego kapitana
łodzi, cudownego miodu lub sanktuarium, do którego przyprowadziła ją
grecka rodzina, do których podchodzili chodzący po kolanach pielgrzymi,
którzy nosili malarskie nakolanniki, aby chronić się, gdy wspinali się
na wzgórze do ikony Madonny w~skromnym budynku.

Cordelia mówiła o~szkole, profesorach i~chłopcu -- mężczyźnie, jak
powiedziała -- którego Jacob nigdy by nie pochwalił. Łatwo było utrzymać
jej udział w~tych rozmowach. Wszystko, co musiała robić, to kiwać głową
i wydawać dźwięki, a~nie stać i~krzyczeć, że to bzdury, wszystko, czemu
Cordelia poświęciła swoje życie, było gorsze niż fikcja, napędzana
urojonym przekonaniem, że pieniądze, władza i~przywileje Redwaterów to
coś, co oni zasłużyli i~dlatego wszyscy bez wspaniałych pieniędzy,
władzy i~przywilejów \textit{nie} zasłużyli na to.

Czasami wchodziła najemniczka. Natalie uważnie słuchała, czy ktoś
wymienił jej imię. Desperacko chciała poznać. Najemniczka był pomostem
między światami. Musiała żyć w~prawdziwym świecie, w~którym przywileje
były w~oczywisty sposób niezasłużone. Jak mogła spotykać się z~klientami
i nie wiedzieć? Musiała umieć sprawić, by to nie miało znaczenia,
ponieważ jej wypłata zależała od tego, czy nie będzie się przejmować, od
zrozumienia odchodzenia na tyle dobrze, by ich wyrwać. Najemniczka nie
była jej przyjacielem, ale była ważna.

Nikt nie zwracał się do niej po imieniu. Kiedy Jacob jej pragnął,
zmieniał ton swojego głosu, przechodząc na ton komendy, którego zawsze
używał na ochroniarzach. Różnił się od tonu komendy, którego używał na
służbie domowej, był bardziej zmilitaryzowany, jakby grał na LARPie
sierżanta z~niebieską szczęką z~filmu wojennego. Kiedy mama chciała
najemnika, przełączała się na swój przymilny ton, głos ,,zrób mi
przysługę'', mniej łaskawy z~powodu żelaznego przekonania, że przysługa
zostanie wyświadczona.

Cordelia nigdy nie rozmawiała z~najemniczką. Traktowała ją jak
niewidzialną, chodzącą kamerę CCTV. Jeśli kiedykolwiek spojrzała na
najemniczkę, to ze strachem.

Najemniczka był kluczem.

Następnym razem, gdy weszła najemniczka, przynosząc kosz z~przekąskami,
świeżą bieliznę i~koszule oraz bezsensowny, nietłukący wazon z~niesezonowymi, szklarniowymi kwiatami, które niewątpliwie pochodziły od
jej matki, Natalie spojrzała na nią.

-- Możemy zawrzeć umowę na boku. Nikt nie musiałby wiedzieć, nie na
początku. Nie mogą mnie tu trzymać na zawsze. W~końcu znudzą się szaloną
siostrą na strychu, wyślą mnie do zakładu i~kopną cię w~tyłek. Jeśli uda
mi się wydostać, znajdę prawnika, który nakłoni ich do odblokowania
mojego funduszu powierniczego. Wiesz, co myślę o~pieniądzach, widziałaś,
jak chcę żyć. Przepisałabym to na Ciebie. Hermetycznie i~nieodwołalnie.
To więcej, niż zamierzają ci zapłacić, więcej, niż mogliby ci
kiedykolwiek zapłacić. Fortuna. \textit{Dynastyczna} fortuna, tego
rodzaju, że nadal będzie w~stanie nienaruszonym, gdy będziesz starszą
pani i~dzieci będą walczyć na łożu śmierci o~bulion.

-- Jestem pewna, że myślisz, że gdybyś to zrobiła, zostałabyś
radioaktywnie wyruchana. Dlatego oferuję Ci cały pakiet: życie bez
konieczności pracy następnego dnia. Automatyczne depozyty, co miesiąc,
dla Ciebie, Twoich dzieci, ich dzieci. Sposób, w~jaki jest napisane
powiernictwo, jest duża szansa, że kiedy mama i~tatuś kopną w~wiadro, w~powiernictwie będzie świeża forsa, jeszcze większa dla ciebie i~Twoich.
Wszystko, co mogą ci zaoferować, to łóżko pod schodami, ja proponuję, że
zamienię cię w~zettę.

Najemniczka spojrzała na nią.

Natalie się uśmiechnęła. -

- Wiesz, że mówię poważnie. -- Zawahała się,
ponieważ ta część była niebezpieczna, jeśli źle oceniła kobietę. -- Nie
ma nagrania wideo z~tej rozmowy. Sprawdź sama, a~potem się dogadajmy.

Na twarzy najemniczki może pojawił się uśmiech, tak subtelny, że Natalie
mogła się oszukiwać. Odstawiła kosz i~wycofała się, jak zawsze, pewnym
krokiem, który mówił: \textit{Nie boję się ciebie, to tylko najlepsze
praktyki.}

Gdy tylko drzwi zatrzasnęły się, Roz powiedziała: 

-- To było\ldots 

-- Wiem. Mam dość bycia pieprzoną damą. Księżniczka Brzoskwinka jest do
bani. Chcę być Mario. Minęły tygodnie. Nie będzie lepiej. Tata nie
obudzi się i~nie powie: ,,Co ja sobie kurwa myślałem? Mili ludzie nie
porywają swoich córek!'' Jeśli nie uda mi się udawać kapitulacji,
pochowa mnie w~jakiejś głębokiej dziurze, na obozie dla bogatych suk,
gdzie golą ci głowę i~każą czołgać się w~błocie, aż zaczniesz miauczeć o~litość, a~potem odeślą cię do domu z~pompką zombinolu w~wyrostku
robaczkowym i~uśmiechu zszytym szwami.

-- Ale jeśli na ciebie doniesie, złapią mnie.

-- Więc co? Jeśli złamałaś ich tak dokładnie, jak mówisz, będą mieli
cholernie dużo czasu, żeby cię wykorzenić, w~międzyczasie będą musieli
mnie przenieść, co może być szansą na ucieczkę. Masz kopię zapasową.
Złapanie nie jest karą śmierci, po prostu wyślij swój plik diff do innej
instancji. Możesz odejść. To jest cały sens doświadczenia Roz.

Roz milczała. 

-- Nie chcę zostawiać cię samej.

-- Myślisz, że sobie z~tym nie poradzę.

-- Nie sądzę, żebyś mogła sobie z~tym poradzić, albo powinna być sama.

Natalie przypomniała sobie, jak bardzo się ucieszyła, kiedy Roz po raz
pierwszy przemówiła, ulgę, jaką daje jej sojusznik. Nawet nie wiedząc,
czy Roz została zinfiltrowana, czy sama Roz mogła stwierdzić, czy
została zinfiltrowana, to była taka ulga. Przed Roz była tak
odizolowana, że prawie się złamała.

-- To, że tu jesteś, powstrzymało mnie od robienia więcej, by sobie
pomóc. Jesteś moją Deus Ex, obiecującą zbawienie z~daleka. Prawie
oszalałam, zanim tu przyjechałaś, bo byłam w~szalonej sytuacji. Od
tamtej pory jestem przy zdrowych zmysłach, chociaż moja sytuacja jest
bardziej popieprzona. To nie jest dobra rzecz.

Kolejna cisza maszyny. Natalie przypomniała sobie, że Roz była
delikatną, połamaną symulacją, jak ją łagodziła, gdy pracowała nad
problemem własnego zdrowia psychicznego. Roz odwzajemniała przysługę, to
była symetria.

-- Nie jestem symem, Roz. Jestem istotą ludzką. Łamię się, bo moja
sytuacja jest ostatecznie spartolona.

Czy sym może płakać? 

-- Rozumiem -- Głos Roz był stłumiony.

-- Czy wszystko w~porządku? Nie powinnaś być w~stanie czuć się smutna,
prawda? -- Była zaniepokojona, myśląc o~tym, jak spektakularnie może
zejść Roz, przypominając sobie przerażające rozpady osobowości na końcu.

-- Chyba tak. Ja, jest nas garstka, garstka Rozłącznych, które próbują
poluzować sznurki w~naszych osobowościach. Praca Gretyl nad podglądami
nam na to pozwala. Kiedy zaczynałyśmy, oszczędzałyśmy z~wyprzedzeniem,
omijając brzegi, próbując pójść środkiem. Jesteśmy o~wiele lepsze w~podglądzie, kod staje się coraz precyzyjniejszy, pracujemy z~szerszymi
zakresami, bliżej krawędzi.

Wbrew sobie Natalie była zafascynowana. 

-- Ale \textit{dlaczego}?

Nawet kiedy o~to poprosiła, zrozumiała. Czy nie to właśnie robiła?
Odnalezienie szaleństwa, które pozwoli jej spotkać grozę z~przerażeniem,
spotkać niemożliwe z~bezkompromisowością?

-- Bo nie jestem \textit{sobą}. To była jedyna rzecz, której sobie
obiecaliśmy, że nie powiemy. Wszyscy tak bardzo liczą na symulację. To
plan B wszystkich, ich ucieczka. Im więcej czasu spędzam w~tej
\textit{sytuacji}, tym mniej jestem pewna, czy wciąż jestem \textit{sobą}.

-- Oczywiście. Nie posiadanie ciała, przeistoczenie się w~oprogramowanie,
to musi was zmienić. Jak utknięcie tutaj zmieniło mnie.

-- Nie chodzi mi o~to, że się nie zmieniłam. Spodziewałam się, że się
zmienię. Byłam w~pobliżu. Wyrośliśmy z~pytań typu ,,jeśli odetnę ci
palec, czy nadal nie będziesz sobą?'' lata temu. Nadal byłabym sobą, ale
inną sobą. Gdybyś ciągle odcinała centymetry, aż nie zostanie nic poza
maszyną, nadal byłabym sobą, ale byłabym sobą, która była
straumatyzowana i~zmieniona.

-- ,,Ja'', które się liczy, to nie tylko ja, które potrafię rozpoznać. To
ja, którym chcę być. Jeśli jedynym sposobem na bycie mną w~krzemie jest
bycie mną, której udaje się tylko nie nienawidzić siebie, dosłownie
odmawiając sobie pozwolenia na myślenie, które powinnam myśleć, to jebać
to.

-- Prawie to zrozumiałam -- powiedziała Natalie, uśmiechając się wbrew
sobie. -- Przepraszam, nie chcę żartować\ldots 

-- To \textit{jest }śmieszne, na sposób co-do-kurwy-nędzy. Ale to
przerażające. Tak wiele zależy od moich głupich egzystencjalnych
kryzysów\ldots 

-- To musi być okropne.

-- Mam na myśli, kurwa, jestem osobą w~maszynie, która może przebywać w~setkach miejsc naraz. Nie zostałam uwięziona przez mojego ojca. Nie
zostałem porwana od mojej kochanki. Nie mam powodu narzekać, tylko
dlatego, że będę samotna, jeśli nie będę mogła być z~Tobą\ldots 

-- Też będę za tobą tęsknić, jeśli cię zbombardują. -- Pojawiła się myśl.
-- Myślisz, że mogliby cię \textit{schwytać} i~pieprzyć twoje parametry,
żeby cię torturować?

-- Nie, to jedyna rzecz, której jestem całkowicie pewna. Jestem cała w~przełącznikach awaryjnych. Jeśli się wpierdolą, zostanę bezpiecznie
wymazana, zanim się zorientują.

-- To ulga. Będę za Tobą tęsknić, ale porozmawiamy ponownie. Wychodzę,
bez względu na wszystko. Kiedy to zrobię, będziesz tam.

-- Przepraszam, że dziaduję. Jestem gównianym robotem. Po prostu\ldots  -- Kolejna pauza. Czy Roz dodaje je dla uzyskania dramatycznego efektu? Czy
spoglądała niebezpiecznie w~przyszłość? Głos, który usłyszała, był tak
cichy, że Natalie ledwo go słyszała. -- Nikt nie zna mnie tak jak Ty.
Nikt nie widział mnie na surowo, bez szyn na mojej sym. Nikt nie jest w~stanie zrozumieć pełnej koperty możliwości wszystkich sposobów bycia mną
i tego, jak ograniczone są te możliwości we mnie, którą jestem dzisiaj.

Jej namacalny smutek, jej syntezator głosu stał się \textit{tak dobry},
złamał Natalie. Jej oczy zalały się łzami. Wytarła wściekle. Nie chciała
być uwikłana w~troskę o~cudze dobro. Chciała zadbać o~siebie.

Ta myśl zaczepiła się jak haczyk na ryby. To była myśl Jacoba Redwatera.
Myśl defaultu. Myśl \textit{zetty}. To nie była myśl odchodniczki. To był
rodzaj myśli, której przez lata uczyła się nie myśleć. Tak łatwo było
być wyjątkowym płatkiem śniegu i~wiedzieć, że jej nędza liczyła się
bardziej niż wszystkich innych. To mogła być prawda. Jacob żył życiem, w~którym jego szczęście przewyższało wszystkich innych. Ale to działało
tylko wtedy, gdy uzbroiłaś się przeciwko reszcie świata. Zbudowała
bezpieczny pokój w~swoim sercu.

-- Kocham cię, Roz. -- Nie wiedziała, czy to prawda, ale chciała, żeby to
była prawda. Chciała kochać wszystkich. Wszystkim nie udało się sprostać
własnym ideałom. Chciała nie dorównać najlepszym ideałom. -- Kocham cię
za to, kim jesteś teraz i~za to, kim jesteś, kiedy tracisz głowę. Obie
jesteście Tobą.

Cisza maszyny. Rozciągała się. Miała mówić, ale \textit{stuk}, stuk, drzwi
odblokowane. Weszła najemniczka, niosąc tacę z~karafką z~-- podpowiadało
jej długie doświadczenie -- letnim, gównianym coffium, które została
zdenaturowane z~dobrych rzeczy.

Najemniczka zamknęła drzwi \textit{stuk, stuk} i~odkryła karafkę. Płyn w~środku parował w~taki sposób, który napoje, które miała jako więźniarka,
nigdy nie parowały. Pamiętała zapach z~dzieciństwa, wycieczki po domkach
z kuzynami Redwater z~dynastycznej gałęzi, z~wszczepionymi chipami
śledzącymi i~ochroniarzami. To nie było coffium, to była kawa, nagroda
bezcenna, ziarna wyhodowane na specjalnie odizolowanych polach, na
których pracowali pracownicy, których dwa razy dziennie poddawano
badaniu mikrobiologicznemu pod kątem pierwszych oznak zarazy.

Najemniczka postawiła tacę na śniadaniowym stole Natalie, ustawiła dwie
porcelanowe filiżanki, nalała, lotne aromaty wypełniły pokój
niemożliwymi, żywymi zapachami.

-- Śmietanka? -- Wypowiadała tak mało słów, że jej głos ją zaskoczył.
Ciepły, głębszy niż pamiętała Natalie. Czy był akcent, ślad drgania na
\textit{r}?

-- Nie, jeśli to jest to, co myślę. -- Powąchała głębiej. -- Yergacheffe? 

Kuzynka starsza, podróżująca nauczyła ją wymawiać to z~miękkim \textit{y},
zwiniętym \textit{r}, twardym \textit{ch} i~oddychającym \textit{h} na końcu.
Z łatwością wyskoczyło z~jej ust, znak statusu w~czterech sylabach.
Zapach był nie do pomylenia, owocowy i~kwaskowaty, niepodobny do innych
długich ziaren, pełnia Blue Mountain, kwaśna owocowość burbona. Jej usta
się zaśliniły.

-- Tak było napisane na torbie. -- Był tam akcent, może
wschodnioeuropejski. Dorastając, słyszała wiele takich akcentów dzieci,
których rodzice zbijali fortuny, robiąc nieokreślone ,,przedsiębiorcze''
rzeczy. Podobnie jak prawdziwi kuzyni z~Redwater, te dzieciaki miały
ochroniarzy, którzy również mówili z~akcentem, tylko mocniejszym. -- Kucharz przysłał młynek i~prasę.

Napiła się z~zamkniętymi oczami, pogrążona w~zadumie. Natalie widziała,
że jest piękna na drapieżny sposób. Nie gorąca, nie w~jej typie, ale
może ktoś, na kim modelujesz postać z~gry wideo w~konkretnym typie gry
wideo skierowanej do określonego rodzaju chłopca. 

-- To pierwsza kawa,
którą piłam w~Kanadzie. Zwykle kupuje się to tylko w~Afryce. Chińscy
szefowie zawsze na to nalegają.

W liceum była chińsko-nigeryjska dziewczyna, pilniej strzeżona niż
rosyjskie dzieciaki. Miała wybuchowy temperament i~biada każdemu, kto
jest na tyle głupi, by prosić o~dotknięcie jej włosów, co Natalie
rozumiała. Miała na imię Sophie. Natalie nie widziała jej od ukończenia
studiów, ale czasami myślała o~opowieściach Sophie o~pływających
supermiastach w~pobliżu Lagos, gdzie się wychowała, skacząc z~jednego
otoczonego murem ogrodu wielkości lotniskowca do drugiego.

Natalie sięgnęła po kawę. Jej ręce drżały. Żałowała, że to robiły.
Uniosła filiżankę i~nie rozlała. Wyszła z~wprawy przy prawdziwie
gorących napojach, ale udało jej się wypić. Był bardzo gorący i~smaczny
w sposób, którego słowo ,,gorzki'' nie określało. Smakowało
\textit{przeciwnie} do coffium, z~wyjątkiem możliwości zobaczenia, gdzie
jeden napój był związany z~drugim w~nieokreślony sposób. Była w~tym
oleistość, której się nie spodziewała. \textit{Uczucie w~ustach}. Kolejny
marker klasy, znający te dwa słowa i~mający pewność, że użyje ich bez
poczucia burżuazyjności. Dynastyczni Redwaters potrafili powiedzieć
,,uczucie w~ustach'' bez mrugnięcia okiem, a~kuzynka Sarah użyła tego
słowa, by opisać chłopca, którego poznała w~szkole z~internatem w~Doniecku.

Przełknęła. Kofeina była tak prymitywna, że spodziewała się, że będzie
ją przytulać jak jaskiniowiec, ale haj, który pojawił się szybko, był
zaskakująco dobry, mrowienie z~łagodnym szczytem i~łagodnym zejściem.
Nikt już nie zażywał kofeiny. Były opcje na wstawanie. To była taka
wytworna rzecz zettów jak sherry i~herbata ze śmietanką. Zetta
gromadzili najlepsze rzeczy.

Wypiła więcej. Góra była tak czysta. To uspokoiło jej nerwy, sprawiło,
że chciała się ruszyć.

-- Nazywam się Nadie. -- Najemniczka wyciągnęła silną, małą dłoń, która
chwyciła ją z~wykalibrowaną stanowczością.

-- Jestem\ldots  Lodołasica.

Nadie uśmiechnęła się małymi kwadratowymi zębami. 

-- Wiem. Byliśmy w~Twoich sieciach przez dwa dni, zanim cię zabrałam. Nie było trudno.

-- Nie powinno być -- powiedział Lodołasica. -- Chcemy, aby ludzie czytali
rzeczy publiczne. Prawie każdy nie jest zettą, co oznacza, że prawie
każdy powinien dołączyć do odchodników.

-- Dołączają też zetty. -- Nadie miała te rosyjskie, bułgarskie?
białoruskie? kamienne coś, kącik jej ust jako prekursor uśmieszku,
zaprzeczenie mikroekspresji, która jednak się zarejestrowała.

-- Niektórzy.

-- Interesuje mnie aspekt bezpieczeństwa informacji w~naszej
wcześniejszej rozmowie.

-- Czy to oznacza, że mamy umowę?

-- Nie. -- Mikroekspresja Nadie zamigotała. -- Nie mamy umowy. Uspokój się.
-- Wskazała na odczyt na łóżku, który zaalarmował, gdy wskaźniki tętna i~endokrynologiczne Lodołasicy uderzyły w~czerwoną strefę.

Lodołasica zmusiła się do oddychania. Nadie bawiła się w~gierki. To
właśnie robiła od samego początku. Byłoby urojeniem mieć nadzieję na
cokolwiek innego.

-- Uspokajam się.

-- Chcę wiedzieć o~infowojnie. \textit{Wiem} Ty nie miała żadnych urządzeń
jailbreak, które można użyć do badania i~zhakowania schronu.
Przeszukałam cię. Nikt, kto wchodzi, nie może wnieść niczego, co mogłoby
zostać użyte do ataku, z~wyjątkiem Twojego ojca, a~nawet on poddaje się
sprawdzianowi za każdym razem, gdy wychodzi. Atak przyszedł z~zewnątrz,
co powinno wywołać alarm IDS. To się nie dzieje. Jest coś bardzo złego,
czego nigdy nie zauważyłam. To sprawia, że czuję się głupio.

-- Nie myślę o~Tobie gorzej.

Mikroekspresja telegrafująca mroczną rozrywką. Kobieta była dzikusem
emocji.

-- Mam nadzieję, że nie. Mam nadzieję, że rozumiesz, jestem poważną
osobą, a~nie Twoim przyjacielem. Nie jestem też Twoim wrogiem, chociaż
byłam Twoim przeciwnikiem. Jestem bardzo dobra w~tym, co robię. Na tyle
dobra, że chcesz być ze mną szczera. Wystarczy, że jeśli staniemy się
wrogami, powinno cię to martwić.

Jej mikroekspresja się zmieniła, błysk, który sprawił, że poczuła się
przestraszona centymetr poniżej pępka. Jak strach, który kiedyś czuła,
wędrując w~pobliżu B\&B. Był wilk. Spojrzał na nią w~sposób, który dawał
jej pewność, że odwzorowuje każdą możliwą rzecz, jaką mogła zrobić,
przewidywane kontrataki. Faktycznie ją posiadł. Oddychała tylko dlatego,
że to jej pozwoliło. Starała się zachować spokój. Głupi monitor przy
łóżku doniósł na nią, jego infografika pojawiła się w~jej peryferyjnym
polu widzenia. Spodziewała się, że Nadie uśmiechnie się, czy też
mikrouśmiechnie, ale zachowała to paskudne spojrzenie jeszcze przez
chwilę.

-- Widzę, że rozumiesz. Porozmawiajmy o~sieci.

Lodołasica współczuła jej odwadze.

-- Nie sądzę. Przekazałam ci wiedzę o~sytuacji w~sieci. Dlaczego miałabym dać ci coś więcej?

Skinęła głową, uznając rację. 

-- Więcej kawy? -- Podtekst \textit{A ta
czarna magia; uczciwy handel, prawda?}.

-- Absolutnie. -- Czarny płyn przelewał się jedwabistą rzeką od karafki do
kubka. -- Nadal nie powiem ci więcej o~sieci. Nie, dopóki nie zawrzemy
umowy.

To nie jest głupie stanowisko, chociaż wiesz, że teraz sama mogę dojść
do sedna sprawy. Moi pracodawcy mają procedury. W~ciągu dwunastu godzin
wyciągną wszystko z~budynku, zabiorą do kryminalistyki, podczas gdy nowe
aktualizowane i~zablokowane rzeczy będą tutaj instalowane.

-- Jestem tego świadoma.

-- Liczysz na to, że za uwolnienie cię dostanę więcej pieniędzy niż za
pomoc Twojemu ojcu.

-- Mam na to nadzieję. Pomaga, że mój ojciec jest dupkiem. Mam nadzieję,
że oceniasz pracę dla niego tak negatywnie, że szansa, aby uciec
\textit{i} wypieprzyć go \textit{i} pomóc mi \textit{i} wzbogacić się, jest
kusząca.

Wrócił mikrouśmieszek : \textit{touché}. 

-- Twój ojciec jest w~trudnej
sytuacji.

-- Mój ojciec zasługuje na huśtawkę na latarni.

-- Trudna sytuacja, myślę, że się zgodzisz.

-- Nie zaprzeczyłaś mojej ocenie.

-- Widziałam ludzi huśtających się na latarniach. To nie jest fajne.

-- Przypuszczam, że to prawda.

Więcej kawy. Drugie uderzenie kofeiny nie było tak dobre jak pierwsze, o~czym pamiętała, że słyszała, adaptacja kofeiny była szybsza niż w~przypadku koktajlu neurołańcuchów w~coffium. Trzeba było zwiększać
dawkę, aby dostać się w~to samo miejsce, lub odczekać żmudne okresy
refrakcji, zanim zdołało się odzyskać reakcję.

-- Ludzie wiszący na latarniach, co?

-- Dwa razy. Nie umieściłam ich tam.

-- Kto to zrobił?

-- Ludzie tacy jak ja, prawdę mówiąc. Ludzie pracujący dla bogatych,
biorący pieniądze na rozkazy, by wysłać wiadomość.

-- Jaką wiadomość?

-- Nie zadzieraj z~moim szefem, bo zawiśniesz na latarni.

-- Ale nigdy nikogo nie powiesiłaś na latarni.

-- Nigdy nikogo za nic nie wieszałam. To nie jest mój rodzaj pracy.
Zostałam o~to poproszona.

-- Możesz odmówić takiemu szefowi?

-- Jestem dobra w~swojej pracy. Mogę powiedzieć: ,,Pozwól, że wyjaśnię,
dlaczego to pogorszy sytuację''. Pozwól, że wyjaśnię, jak to sprawi, że
ludzie, którzy nie myślą, że jesteś wrogiem, zdecydują, że muszą cię
zabić, zanim Ty ich zabijesz. Pozwól, że wyjaśnię, co mogę zrobić, aby
zneutralizować ludzi, którzy chcą cię skrzywdzić.

-- Masz na myśli infiltrację ich sieci, porwanie ich\ldots 

-- Tak\ldots  nie. Zmapuj wykres społeczny, znajdź liderów, zrób im krzywdę,
zdyskredytuj ich. Porwij, jeśli musisz, ale to tworzy męczenników, więc
nie tak bardzo. Lepiej, żeby byli zajęci gaszeniem pożarów. Znam innych
wykonawców, którzy będą przeszukiwać kanały czatów i~modelować słabe
punkty, znajdować stare walki, które wciąż się gotują, tworzyć strategię
ich rozpalania. Tak łatwo infiltrować. Kiedy myślą, że są zinfiltrowani,
wskazują na siebie, zastanawiając się, kto jest kretem, a~kto jest
wiarygodny. To ładniejsze niż ciała huśtające się na latarniach.
Bardziej uporządkowane. Mniej much.

-- Ha ha.

-- Nie lubisz tego. Pracuję dla twoich wrogów, niszczę to, co budujesz. -- Wzruszyła ramionami. -- Nie robię tego, bo cię nienawidzę. Czasami nawet
cię podziwiam. Ale jestem dobra w~swojej pracy. Jeśli chcesz odnieść
sukces, musisz być dobra w~swojej pracy. Ktoś inny wykonałby moją pracę,
gdybym tego nie zrobiła, więc jeśli nie jesteś lepsza w~swojej pracy niż
ludzie tacy jak ja w~naszej, jesteś skazana.

Infografika pulsowała na czerwono. 

-- Kurwa, nienawidzę tego.

-- Nie przeszkadza mi, że jesteś zdenerwowana. Mówię przykre rzeczy.
Gdybym była Tobą, byłabym zdenerwowana. Rozumiem, że nie robisz tego, co
robisz dla pracy, ale dla miłości. Chcesz ocalić świat. Ratowanie świata
jest dobre, ale nie sądzę, że sobie poradzisz. Nie sądzę, żeby
ktokolwiek mógł. Ludzka natura. Jeśli świat jest skazany, chcę czuć się
komfortowo, dopóki nie wybuchnie, bum.

-- Wygląda na to, że chcesz powiedzieć, że interesuje cię mój fundusz
powierniczy.

-- Bardzo mnie interesuje Twój fundusz powierniczy, Natalie. Lodołasico.
Wierzę, że istnieją strukturalne wyzwania, aby dostać się do niego, ale
sądzę również, że na mojej orbicie są ludzie, którzy wiedzą, jak
sprawić, by strukturalne wyzwania zniknęły. Oczywiście będą musieli być
opłaceni, ale\ldots 

-- Ale będzie cię na to stać.

-- Teraz mnie na to stać. Jestem dobra w~swojej pracy. Dobrze zarabiam.
Moje kontakty zrobiłyby to za prowizję, ale to byłoby znacznie więcej.
Wolę płacić gotówką, nawet jeśli ryzykuję własnymi pieniędzmi.

Dolała sobie kawy, podniosła ją do ust, nie piła, spojrzała na czarne
lustro jej powierzchni. Jej ręka była nieruchoma jak skała, a~jej oczy
chłodne jak lodowaty lód. 

-- Wiesz, że mogę cię znaleźć. Bez względu na
to, dokąd idziesz, co robisz, mogę cię znaleźć.

-- Wiem, że potrafisz. -- \textit{Wiem, że myślisz, że potrafisz}.

-- Możesz myśleć: ,,Moi towarzysze mają lepszy opsec niż ten rosyjski
mięśniak, zobaczcie, jak przebili się przez granicę sieci, weszli w~jej
pętlę decyzyjną.'' Możesz pomyśleć: ,,Teraz możemy ją przechytrzyć''.
Czy tak myślisz?

Czerwony, czerwony, czerwony. Głupia grafika. 

-- Nie myślę tak, ale
zastanawiam się, czy to prawda.

Napiła się i~odstawiła filiżankę. 

-- To może być prawda. Nie sądzę.
Obrona jest trudniejszą grą niż ofensywa. Obrona, musisz być doskonała.
Atak, musisz znaleźć jedną niedoskonałość. Tutaj jestem obrońcą. Kiedy
poluję na Ciebie, jesteś obrońcą. Popełnisz błędy. Twoja filozofia nie
jest o~perfekcji, nie dotyczy dyscypliny.

\textit{Aby nie oszukiwać samego siebie, potrzebna jest mentalna
dyscyplina.}

-- To nie ma znaczenia. Jeśli cokolwiek o~mnie rozumiesz, to rozumiesz,
że mam w~dupie pieniądze. Gdybym mogła je ułożyć na stos i~podpalić,
byłby to jedyny dzień, w~którym na nie bym nie sikała. Nie przechytrzę
cię. Co więcej, brak pieniędzy tak bardzo zraziłby mojego ojca, że
mógłby przestać próbować wprowadzać mnie w~kult rodziny. Może cię
adoptuje.

-- Nie sądzę, żebym mu na to pozwoliła. -- Jej mikroekspresja była
niemożliwa do odczytania. -- Będę rozmawiała z~ludźmi, którzy robią
rzeczy z~funduszami powierniczymi i~finansami, żeby Twój ojciec nie mógł
ich cofnąć. Wiesz, że jeśli powiem nie, a~Ty porozmawiasz o~tym ze swoim
ojcem, mogę w~znaczący sposób pogorszyć Twoją sytuację. Wiesz, że byłam
w stanie cię namierzyć, określić Twoje wzorce. Zabrałam cię bez
zamieszania. Wiemy, że ci ludzie nie obchodzą cię, ale oboje wiemy, że
liczą się dla ciebie inne osoby, takie jak twoja Gretyl\ldots  

To imię
sprawiło, że infografika straciła na znaczeniu. 

-- Znajdę ją tak łatwo,
jak znalazłam ciebie. Fakt, że nie zostałaś ranna, był moim wyborem. Czy
rozumiesz te rzeczy?

Płakała i~po prostu nienawidziła siebie za to. Tak głupie! Aby dać tej
osobie taką władzę nad sobą, być taką pawłowską niewolnicą, wystarczyło
wymienić nazwisko Gretyl i~wodospady się uruchomiły.

Wciągnęła smark, dziko otarła oczy, spojrzała na nią gniewnie.
Najemniczka wyglądała na trochę zakłopotaną.

-- Nie lubię grozić. Ale to pomaga, jeśli wiesz, że mówię poważnie. W~ten
sposób nie mamy nieporozumień dotyczących równowagi sił. Jestem kimś,
kto zwraca uwagę na równowagę sił. To moje kompetencje zawodowe.

-- Jeśli cokolwiek o~mnie wiesz, wiesz, że po prostu chcę stąd
wypierdalać. Nie mam ochoty schrzanić Twojej pracy z~moją socjopatyczną
rodziną. Jeśli pomyślisz o~tym przez jedną pieprzoną sekundę, panno
Równowaga Sił, zrozumiesz, że \textit{nie gram} w~gierki. Dobrowolnie
powiedziałam ci, że złamałam sieć. Mogłam zachować to w~tajemnicy na
zawsze. Dobrowolnie przekazałam tę władzę.

-- Oczywiście zostawiłaś mnie zastanawiającą się, jakie jeszcze masz
sekrety, i~dlatego prowadzimy tę dyskusję.

-- Nie mam już żadnych sekretów. -- \textit{Och, ta jebana infografika.}

Zaśmiała się. Była ładna, kiedy się śmiała. Wcale nie przerażająca. To
było tak, jakby nastoletnia dziewczyna uwięziona w~niej -- przed
wszystkimi szalonymi sztukami walk i~treningami -- prześwieciła przez
nią. 

-- Oczywiście, że masz sekrety. Wszyscy mamy sekrety, Lodołasico.


\chapter*{xv}

Byli już w~połowie drogi do Thetford, kiedy zabrzmiały sygnały alarmowe,
wyrywając Setha z~zadumy. Sieć wróciła na dobre, gdy wspięli się na
grzbiet w~prostej linii do trzech repeaterów. Nagle zaczął docierać do
nich ruch, który napływał z~daleka, w~wielu kierunkach. Gdy ich
dostępność została ponownie rozpropagowana na inne tunele, dane ruszyły.
Było dla nich \textit{wiele }wiadomości.

Gretyl zorientowała się pierwsza: 

-- Burza spieprzyła normalny routing,
wszystkie te rzeczy mają kopie zapasowe. Powinniśmy wbić repeater. Ktoś
przyniósł jeden z~wozu?

Seth miał. Wspiął się na drzewo, Gretyl i~Tam pomagali, i~wbił kolec w~pień jakieś cztery metry w~górę. Tam podała mu siekierę i~wyrąbał
gałęzie wokół kolca, czując ukłucie poczucia winy pomimo drzew wokół
nich, jak okiem sięgnąć. To nie było ładniejsze od innych.

Tam pomogła założyć mu wiązania i~rozwinąć osłonę słoneczną na północnej
stronie. Gretyl pogrążyła się w~antyspołecznej, skomputeryzowanej ciszy,
analizując wiadomości.

-- Jasna kurwa, cholera -- powiedziała.

-- Co? -- Seth krzyknął i~prawie upuścił topór, wizje, w~których wbił się
w czaszkę Tama, sprawiły, że chwycił go dziko, po czym omal nie spadł z~tego przeklętego drzewa.

Dowiedzieli się o~Akron, wszystkich innych atakach i~pospiesznie
przesłali wiadomości do wszystkich, których kochali, na całym świecie i~ruszyli tak szybko, jak tylko mogli, do Thetford.

Interfejs Tam czytał jej, gdy szła i~przeglądała wiadomości i~filmy,
pozostając w~tyle za Sethem i~Gretyl. Seth próbował ją pośpieszyć, ale
powiedziała mu, żeby się odjebał. Znała ludzi w~Akron i~zastanawiała
się, czy nie są martwi.

Seth zdał sobie sprawę, że wielu członków B\&B było w~Akron. Ludzi,
których znał, z~którymi gotował, naprawiał maszyny, kłócił się. Takich,
którzy powitali go, kiedy był szleperem. Niektórych deszlepował,
inicjując ich w~tajemnice odchodzących. Ten, w~którym zakochał się na
krótko, który, jak sobie uświadomił, przypominał mu Tam. Kto by
pomyślał, że ma typ?

Martwił się. To było wszystko, co mógł zrobić, żeby nie prosić Tama,
żeby też odszukał \textit{jego }ludzi. Gretyl chciała dostać się do
Thetford, dla całego dobra, jakie tam zrobią.

Tam wciąż dyszała i~przeklinała, zapadała się w~śnieg i~potrzebowała
ratunku. Jej baterie się wyczerpywały. Tak jak jego. Gretyl trzymała się
zbyt daleko, by mógł zobaczyć jej infografikę, ale nie mogła mieć dużo
mocy.

-- No, \textit{chodź}, Tam. Nic nie możemy tutaj zrobić. Musimy wrócić
przed zmrokiem, kochanie.

-- Kurwa kochanie, świat płonie.

-- Czy nie może się płonąć, kiedy jesteśmy gdzieś z~toaletą?

-- Kurwa.

Wspięli się na ostatnią grań i~Tam krzyknęła. Już miał zrobić jej piekło
za nurkowanie, kiedy zobaczył, że wskazuje palcem. Byli na najwyższym
wzniesieniu w~promieniu kilku klików. Wskazywała daleko na horyzoncie. Zmrużył
oczy i~Gretyl zaklęła. Podkręcił powiększenie wizjera i~zobaczył kolumnę
opancerzonych samochodów na gąsienicach, wysyłających za sobą pióropusze
świeżego śniegu. Pokryte były śnieżnym kamuflażem, ale pióropusze
ułatwiały dostrzeżenie ich kształtów.

-- Jadą do Thetford -- powiedział Seth.

-- Nie pierdol -- powiedziała Tam.

-- Dzwonię do nich teraz -- powiedziała Gretyl. Z~grzbietu widzieli stację
kosmiczną, rury i~kopuły pośród ruin domów.

-- Muszą się stamtąd teraz wydostać -- powiedziała Tam.

-- \textit{Dzwonię} do nich -- powiedziała Gretyl, a~jej interkom wyłączył
się, gdy weszła na prywatny. 

Obserwowali ruch kolumny pancernej.
Poniewczasie Seth przeskanował niebo w~poszukiwaniu dronów i~zobaczył
zwiadowców przed kolumną, ale lecących z~dystansem, być może po to, by
zachować element zaskoczenia. A może zwiadowcy dalekiego zasięgu
znajdowali się na dużej wysokości i~wycofali się do niewidzialnych ukłuć
szpilką.

-- Kersplebedeb mówi, że spodziewali się czegoś takiego. -- Gretyl
wskazała na stację kosmiczną, gdzie teraz śluzy powietrzne pękały i~wysypywały się odchodzące w~kombinezonach i~butach z~plecakami i~saniami. -- Godzinę temu połączyli się z~siecią, zrozumieli sytuację w~Akron\ldots 

-- Nasz repeater -- powiedział Seth.

-- Zmostkowaliśmy ich, dostali wiadomość. Nie są głupi. Są gotowi odejść.

-- Lepiej, żeby byli gotowi biec -- powiedziała Tam. Kolumna się zbliżała.

\chapter*{xvi}

Roz była w~uszach Gretyl -- wszystkich uszach naraz -- gdy wkładali
skafandry i~wychodzili przez śluzy, chwytając zapasy, które kazała im
zebrać i~schować, gdy nadeszły wiadomości. Z~niejasnych powodów
Kersplebedeb wciąż nazywała ją ,,Tygrysią Matką'', co stanowiło ich
prywatny żart.

Roz powiedziała im, żeby wzięli zapasowe baterie dla Tam, Setha i~Gretyl, przypomniała im, żeby opróżnili pęcherze i~wypróżnili się przed
ubraniem się w~skafander, przypomniała im o~dwóch deadheadowanych
najemnikach, którzy przybyli aż z~Odchodzącego Uniwersytetu i~których
będą musieli spakować i~zasugerowała ustawienie sań i~folii bąbelkowej
oraz nadzorowała produkcję.

Roz wypchnęła ich za drzwi, tkwiła im w~uszach, gdy wspinali się po
grani, podczas gdy Tam, Seth i~Gretyl schodzili na dół, zabierając swój
ładunek.

Roz pożegnała się, odsuwając się, ciągnąc swoje ładunki i~zarzucając
plecaki na ramiona, truchtając w~dwóch poszarpanych kolumnach na całym
zapasie rakiet śnieżnych stacji kosmicznej.

Dotarli na szczyt grzbietu, chrzęst rakiet śnieżnych był tak głośny jak
kęsy chipsów ziemniaczanych, i~zdali sobie sprawę, że Roz pożegnała się,
ponieważ nie mogła przyjść.

-- Wysłałam e-mailem plik różnicowy do innego instancji mnie.

-- Nie Zdalnej? -- Gretyl brzmiała zaniepokojona. -- Bo to nie jest takie
stabilne\ldots 

-- Nie Zdalna -- powiedziała Roz. -- Istnieje repozytorium instancji Roz w~chmurze odchodzących, z~kopiami na czterdzieści sposobów. Oczywiście nie
wszystkie możemy działać, ale przynajmniej jesteśmy bezpieczne. Na
razie.

-- Cholera -- powiedziała Tam z~uczuciem. -- Nie chcę Cię zostawiać.

-- Ja was zostawiam. Biegnę do przodu. Możesz mnie pobrać i~uruchomić,
gdy tylko znajdziesz klaster. Wy, mięso-ludzie, bądźcie ostrożni.

-- Mamo Tygrys, mamy nasze backupy -- powiedział Kersplebedeb. -- Jesteśmy
pewni życia pozagrobowego w~słodkim pożegnaniu. Nie ma nic trudniejszego
do zabicia niż pomysł, którego czas nadszedł. Do zabicia człowieka
potrzeba czegoś więcej niż broni.

-- Zawsze trzymaj w~samochodzie worek na śmieci. -- Roz brzmiała na
cierpko rozbawioną. -- Nieśmiertelność oprogramowania jest fajna, ale
jeśli możesz uratować swoje mięsiste ciała, powinieneś.

-- Będziemy pilnować naszych tyłków.

Seth zaznaczył pole prywatności. 

-- Martwię się o~ciebie, Roz.

-- Martwię się o~nas wszystkich. Uderzają w~wiele miejsc. Patrząc na te
zdjęcia, które wysłałaś, myślę, że to armia kanadyjska, siły specjalne,
ci, którzy robią te złe rzeczy. Oddziały tortur. Coś, co wysyłasz, jeśli
nie chcesz żadnych ocalałych.

-- Ktoś właśnie przeszedł po moim grobie. -- Seth zadrżał ponownie. Miał
nową baterię, ale było mu okropnie zimno.

Tam dotknęła jego ramienia. Widziała, że mówi, i~musiał wyglądać na
przestraszonego. Przełączył się na kanał publiczny.

-- Po prostu martwię się, że to coś złego lub gorzej.

Poruszali się powoli. Wszystkie te rzeczy, które szlepowali były
wystarczająco złe, ale rakiety śnieżne pogorszyły sytuację.

-- W~tym tempie niedługo nas dojebią -- powiedział Seth.

Kersplebedeb zachichotał. 

-- Nie, raczej nie.

Seth nigdy nie słyszał, żeby Kersplebedeb wydawał tak złowrogi dźwięk.

-- Pułapka?

-- Nie taka, która wybucha. Tylko miejsce pod główną drogą, gdzie
znajdowała się jaskinia górnicza. Naprawiliśmy ją, żebyśmy mogli
sprowadzić zapasy, ale nie zaprojektowaliśmy go dla tych wielkich
czołgów, które mają ci skurwiele.

-- MRAPy -- powiedziała Tam. -- Samochody pancerne. Nie czołgi. Żadnych
wieżyczek.

Kersplebedeb znów zachichotał. 

-- Bez różnicy. Tam jest pół kilometra
gruzu, tuneli, piachu i~iłów, prosto w~dół i~zaraz w~nie wjadą. -- Rzucił
im kanał z~drona, którego wywiesili na strychu, kiedy wychodzili.
Zatrzymali się, pokazując go na swoich przyłbicach. 

-- Lada chwila -- powiedział.

Wachlarz śniegu i~lodu przesłaniał kolumnę, ale Seth sądził, że było ich
sześć tych rzeczy.

-- Czy nie będą mieli lidaru, sprawdzającego, czy nie ma IED?

-- Prawdopodobnie. Nie wiem jednak, czy to wystarczy, aby odróżnić
solidną inżynierię lądową od naszej niedorobionej roboty. 

-- Dowiemy się. \\ Wuuuf. 

Zawalenie było zarówno nagłe, ziemia ustąpiła bez ostrzeżenia,
jak i~gwałtowne, gdy zawalenie zafalowało w~koncentrycznych kręgach,
prawie zbyt szybko, by można je było śledzić. To było straszne i~przerażające, jakby połykała ich ziemia. Dwa MRAPy z~tyłu kolumny
wrzuciły szaleńczy bieg wsteczny, a~przedostatni rozbił się na ostatnim,
wstrząsając nim. Kierowca próbował się wyprostować -- Seth nie mógł
powstrzymać się od kibicowania mu, bo \textit{jebana ziemia połykała
gigantyczną maszynę}, a~kiedy jest to człowiek kontra brutalna fizyka,
tylko socjopata kibicuje fizyce -- ale było już za późno, szczególnie w~panice Pan Przedostatni poszedł na całość w~chwiejnym pojeździe na biegu
wstecznym, a~potem ziemia pod nimi otworzyła się i~zniknęli.

-- Jezu -- powiedział Seth.

Kersplebedeb coś mruknął.

-- Co?

-- Tego się nie spodziewałem. Myślałem, że utkną w~dole, a~nie wpadną do
stopionego jądra ziemi.

-- Prawdopodobnie nie zaszło tak daleko -- powiedziała Gretyl. -- Nie
jestem geolożką, ale myślę, że widzielibyśmy plusk lawy. -- Była wyraźnie
wstrząśnięta, gwiżdżąc w~ciemności.

-- Te czołgi są super opancerzone -- powiedział Kersplebedeb. -- Wszyscy
będą przypięci pasami. Będą poduszki powietrzne.

Tam objęła go ramieniem. 

-- Kersplebedeb, jeśli nie żyją, to nie żyją.
Nie zastawiłeś pułapki. Spieprzyli, przynosząc swoje gigantyczne
macho-pojazdy do buszu. Kurwa, wiesz, że by nas zabili, gdyby nas
złapali.

Kersplebedeb nic nie powiedział. Radio pozwoliło im usłyszeć jego
urywany oddech.

-- Chodź -- powiedziała Tam. Uchodźcy zatrzymali się i~rozeszły się wieści
o zawaleniu i~linku do podsumowania. Rozmawiali grupkami, patrząc w~niebo, jakby miała spaść zemsta. -- Jak mówią w~dramatach historycznych,
,,gówno stało się prawdziwe''. Jeśli wyjdą z~tej dziury, idą po nas.
Jeśli nie wydostaną się z~tej dziury, przyjdzie po nas ktoś inny. Musimy
odejść.

Nadchodził zimowy zmrok.

-- Gdzie jest pociąg towarowy?

-- Cholera -- powiedziała Tam. -- Nie byliśmy nawet w~stanie wam o~tym
powiedzieć.

Kiedy już to zrobili, wszyscy zdecydowali, że powinni udać się do
pociągu towarowego. Miał zapasy i~mógł przewozić zmęczonych. Odchodzący
próbowali podróżować z~lekkim bagażem, ale nie byli masochistami. Jeżeli
istniała maszyna, której można by użyć do przenoszenia ich ładunku, tym
lepiej.

-- Tęsknię za B\&B -- powiedziała Limpopo, a~Seth poczuł głęboki niepokój,
ponieważ Limpopo była złotym standardem w~walce z~przeciwnościami. -- Mechy, onseny. Toalety. Myślę, że kiedy się z~tego wydostaniemy,
powinniśmy zbudować kolejne.

-- Cholera tak -- powiedział Etcetera. 

Seth zdał sobie sprawę, ile czasu
minęło, odkąd odbyli prawdziwą, całonocną pogawędkę z~alkoholem, takiej,
jaką mieli tak często jako dzieci, jako default. Oboje mieli dziewczyny,
ale to nie wszystko. Etcetera był teraz poważny, mądry tak jak Limpopo.
Seth czuł się nieswojo, błaznując ze swoim starym przyjacielem. Ale jego
stary przyjaciel był lepszym człowiekiem, energicznym i~niewątpiącym w~siebie. Znosił do dobrze.

-- Cholera tak! -- Seth machnął pięścią. Etcetera i~on popatrzyli sobie w~oczy, a~więź przyjaźni trzasnęła między nimi, Tam sięgnęła po jego dłoń,
wciąż trzymając ramię wokół Kersplebedeba. W~tym momencie Seth pomyślał,
że mogą zjeść świat na śniadanie i~zażądać dokładki. 

-- Chodźmy.

-- Ale gdzie? -- Pocahontas oderwała się od swojej grupy młodszych ludzi,
stojąc przed nimi i~promieniując pewnością siebie i~młodością w~sposób,
który sprawiał, że Seth czuł się stary i~opiekuńczy.

-- Do wozu -- powiedział.

-- A potem?

Wzruszył ramionami.

-- Myślę, że wymyślimy tam -- powiedziała Limpopo. -- Kiedy już zdobędziemy
transporter, będziemy bardziej mobilne. Sprawdzałam innych odchodników i~jest wielu, którzy mogą nas przyjąć, ale wszyscy się martwią, że będą
następni.

-- Powinni być -- powiedziała Pocahontas. -- Widzieliśmy już tę grę. To
znowu ,,Idle No More''. 

Stary ruch protestacyjny Pierwszych Narodów
nabierał rozpędu przez lata, spływając na żar na kilka miesięcy, a~następnie eksplodując ognistymi lawinami inteligentnych, sprytnych
wydarzeń, które były tak dobrze zorganizowane, że nawet całkowicie
zepsute standardowe media nie były w~stanie ich ignorować. Idle stała
się międzynarodowym skrótem skutecznych buntów, a~protestujący na
ulicach od Warszawy do Port-au-Prince po Caracas deklarowali z~nim
solidarność i~posługiwali się jego ikonografią.

Dopóki w~serii skoordynowanych ataków policja konna, armia kanadyjska,
FBI i~CSIS po prostu zeskrobali Idle z~planety. Każdy znaczący przywódca
zakuty w~kajdany, z~wyjątkiem tych, którzy zginęli w~krwawych
strzelaninach, choreografia przemocy otoczona zgrabnymi logotypami i~złowrogimi arpeggio, by towarzyszyć napiętym relacjom, które dominowały
kanały. Późniejsze procesy ujawniły sieć informatorów i~podwójnych
agentów wewnątrz ruchu. To sprawiło, że odsunięci na margines sojusznicy
poczuli się jak łajdak za wspieranie grupy, która najwyraźniej była
kierowana przez podwójnych agentów.

W odchodnickich kręgach Idle wciąż byli bohaterami. Wśród odchodzących
mieszkało wielu weteranów. W~pozostałej części świata Idle zaczęła
oznaczać niebezpieczeństwo niezadowolenia, lekcja poglądowa tego, że
ludzie, którzy walczyli, nie mogli zaoferować żadnej alternatywy, byli
usiani zdrajcami i~pożytecznymi idiotami, zawsze i~wszędzie skazani na
zagładę, zanim zaczęli.

-- Jasne, że tak -- powiedziała Limpopo. Była tam, kiedy Idle i~wczesni
odchodzący byli na skraju połączenia. -- Teraz o~tym wspominasz.

-- Myślę, że powinniśmy udać się do Dead Lake -- powiedziała Pocahontas.

-- Dlaczego? Nie potrzebują więcej kłopotów.

Jej szydercze parsknięcie było epickie. 

-- Żyją w~buszu, otoczeni
powietrzem tak toksycznym, że nie można oddychać. Ich sąsiedzi zaraz
dostaną napalmem. Gorzej się nie dzieje.

Limpopo skinęła głową. 

-- Masz rację, przepraszam.

-- Nie przepraszaj, bądź mądra. Znają okolicę w~sposób, w~jaki nikt z~nas
nie zna. Prawdopodobnie znajdą się na celowniku gliniarzy, bo są
weteranami Idle, spędzają czas z~nami i~są niewygodnymi świadkami. Nikt,
kto się liczył, nie dałby chuja, gdyby zostali rozwaleni. To nasi
przyjaciele i~sojusznicy. Potrzebujemy takich.

-- Jestem sprzedany. -- Kersplebedeb brzmiał bardziej energicznie, ale
nadal był wstrząśnięty. Dołączył do niego chór głosów w~radiu bliskiego
zasięgu.

-- Chodźmy -- powiedziała Pocahontas, a~Gretyl wskazała jej transporter
towarowy.

\chapter*{xvii}

Limpopo obserwowała, jak Jimmy zanika. Od początku pozostawał w~tyle,
zmagając się z~odmrożonymi palcami w~nieporęcznych rakietach śnieżnych.
Zebrał się, kiedy wzięła jego plecak i~rozdała jego zawartość wśród
grupy. Potem znowu zwolnił. Postukała w~jego skafander i~otworzyła
prywatny kanał.

-- Wsadzimy cię na włóki, odholujemy cię.

-- Nie bądź głupia. Już holujesz tych dwóch pieprzonych najemników i~całe
to gówno. Nie musisz mnie szlepować. Strata czasu. Wiem, dokąd
zmierzacie. Po prostu daj mi dodatkową baterię i~pozwól mi złapać
oddech, dojadę do Ciebie w~dzień. Jeśli ruszycie dalej, powiedz mi
gdzie. Mogę o~siebie zadbać.

-- Nie jesteśmy marines, ale nie lubię zostawiać nikogo w~tyle. Te dupki
są na razie w~dole, ale będzie ich więcej, a~w~grupie jest
bezpieczeństwo.

-- Nie, nie ma.

Wzruszyła ramionami. 

-- W~grupie jest \textit{pewne }bezpieczeństwo. Nie
jesteśmy bezbronne.

-- Nie jesteś też szczególnie przerażająca.

-- Przestraszyliśmy kogoś. -- Wsunęła jego ramię na ramiona, wzięła jego
ciężar. -- Kłóćmy się podczas drogi albo zostaniemy oddzieleni od głównej
grupy.

-- Zrobiłem naprawdę głupie rzeczy. -- Jego głos był płaski.

-- Witamy w~ludzkiej rasie.

-- To, co zrobiliśmy z~B\&B\ldots 

-- To był gigantyczny chujowy ruch, w~porządku.

-- Ale nowa była jeszcze lepsza, jak słyszałem.

-- Była. Znikła, oczywiście.

-- Oczywiście.

-- Ale ulepszenia zostały zapisane w~repozytorium kodu. Następna będzie
jeszcze lepsza. Każdy złożony ekosystem ma pasożyty. Dalej. -- Znowu
zwolnił, a~jego oddech był chrapliwy. 

Prywatnie Etcetera wysłał
wiadomość, aby sprawdzić, czy nie potrzebuje pomocy, i~jednym
mrugnięciem oczu uruchomiła automatyczną odpowiedź ,,kontynuuj, wszystko
w porządku''.

-- Myślę, że potrzebuję odpoczynku.

-- W~takim razie odpocznijmy. -- Upuściła plecak i~pomogła mu wejść na
śnieg. Syknął z~bólu, kiedy poluzowała mu rakiety śnieżne.

-- Tak źle?

-- W~porządku.

-- Nie bądź dupkiem.

-- Jest źle.

-- Lepiej.

Poczuła niepokój, że grupa ucieknie. Wiedziała, że to właściwa rzecz.
Jeśli nie udałoby się wszystkim razem, na pewno nie zrobiliby tego
osobno.

-- Wiesz, zostałem zwerbowany, żeby być zdrajcą -- powiedział po długiej
przerwie.

-- Jak to poszło?

-- Po upadku B\&B to znaczy oryginalnego. Facet spotkał mnie na drodze,
gdy jechałem do granicy z~USA. Pomyślałem, że znajdę tych ludzi, którzy
bunkrowali się z~bronią i~konserwami, zobaczę, czy nie uda mi się ich
nakłonić do ruszenia i~uratowania innych ludzi, zamiast wysiadania, żeby
się ratować. Słyszałem o~miejscach, gdzie są tunele dla handlarzy
narkotyków, przez które można się prześlizgnąć.

-- Byłem w~drodze, trzy dni. Założyłem schronienie i~szykowałem się na
kolację, protsy z~faba, kiedy w~moim obozie pojawiła się kobieta. Cicha
jak ninja, ubrana taktycznie, na biodrze mała broń boczna, której nie
rozpoznałem. Zaprosiła się, przykucnęła obok mojego pieca, ogrzała ręce.
Popatrzyła mi w~oczy i~powiedziała: ,,Jimmy, wyglądasz na mądrego
faceta''. Co było zabawne. Spieprzyłem się na kolosalną skalę, wziąłem
coś pięknego i~obróciłem to w~gówno, próbując narzucić temu moje
pomysły.

-- Staję się przemądrzały, kiedy ironizuję, więc powiedziałem coś w~stylu: ,,Powinnaś więcej wychodzić, jeśli uważasz, że jestem mądrym
facetem''.

-- Zaśmiała się i~odpięła piersiówkę. Poczułem, że to była dobra szkocka,
Islay, dymna. Wypiła, podała. To było dobre. ,,Miałeś dobry pomysł, ale
nie miałeś szans w~tym miejscu. Zbyt wielu piątej kolumny pracowało, by
cię podkopać. Byłam w~ich sieci od pierwszego dnia, obserwując ich
uważnie i~mogłam pokazać ci rozdział i~wers, jak cię jebali. Mówią, że
nie ma liderów, ale jeśli się w~to zagłębisz, łatwo zobaczyć, że to, co
mówi Limpopo, jest wykonywane. Nie wydaje rozkazów, ale na pewno skłania
ludzi do robienia tego, czego chce. Jednak o~tym wiesz.''

-- Co ci zaoferowali? -- Limpopo poczuła się dziwnie pochlebiona, gdy
dowiedziała się, że została poddana tego rodzaju kontroli.

-- Na początku pieniądze, ale wiedziałem, że wiedziała, że nie tego
chciałem. Potem zaoferowała mi badania i~wsparcie w~odwecie na
\textit{Tobie}, co było dla mnie decydujące.

-- Wziąłeś ją na to? -- To było poza wszelkim wyznaniem, którego się
spodziewała. Nie wiedziała, czy szanować go za to, że to zrobił, czy bić
go za jego grzechy.

Zaśmiał się gorzko. 

-- Żartujesz sobie ze mnie? Znasz dowcip: ,,Jestem
tutaj, ponieważ jestem szalony, a~nie dlatego, że jestem dupkiem.''
Zanim B\&B rozpadła się przy mnie, po \textit{walce na pięści} z~facetem,
o którym myślałem, że jest moim najlepszym przyjacielem! \ldots 
zorientowałem się, że problemy, jakie miałem, są moje własne, zwłaszcza
że \textit{Twoje} nowe B\&B działało dobrze kilka kilometrów dalej. To był
niepodważalny test A/B. Pomysł, żebym spróbował jeszcze raz, używając
informacji tej dupy najemniczki, żeby spróbować cię wypieprzyć? Byłem
dupkiem, ale przynajmniej wiedziałem, że jeśli dojdzie do kłótni między
tą skurwielką a~Tobą, będę po Twojej stronie.

-- Ale nie walczę. -- Zastanawiała się, czy on gada bzdury.

-- Odchodzisz.

-- W~rzeczy samej.

-- Poważnie, całkowicie odejdź. -- Spojrzał na oddalające się tyły
kolumny, próbował się podnieść, chrząknął i~opadł. -- Lepiej idź dalej.

-- Pierdol się.

-- Tak, cóż. -- Roześmiał się. Zajrzała przez jego wizjer. Miał spojrzenie
odległe o~lata świetlne. -- Mam na myśli to, że odeszłaś z~B\&B. -- Znowu
się roześmiał. Łzy spływały mu po policzkach. -- To było \textit{piękne}.
Byłem wtedy na ciebie tak wkurzony, że czułem się jak największy dupek
na świecie. Nie mogłaś mnie bardziej zrujnować, gdybyś mnie
powstrzymała. Nigdy nie wyzdrowiałem. -- Chrapliwy oddech. -- Nigdy.
Przyjechałem z~moim gangiem, widziałaś ich, chłopców, którzy myśleli, że
słońce świeciło mi z~dupy, całkowicie kupili merytokrację, nie tylko po
to, by dowiedzieć się, kto co dostał, ale jako sposób na rozwiązanie
\textit{wszystkich }naszych problemów. -- Kolejne odległe spojrzenie.

-- Nie sądzę, żebyś to rozumiała. Moi ludzie patrzyli na świat jak
Platon, wiesz, \textit{Republika}. Każda osoba ma coś, w~czym jest dobra.
Znajdujesz te rzeczy i~pomagasz tym ludziom się tam dostać, a~to
sprawia, że wszyscy są szczęśliwi i~produktywni, a~wszyscy są lepsi. Nie
musisz kazać ludziom wykonywać prace, których nienawidzą. Po prostu użyj
rankingu, aby upewnić się, że jeśli wykonujesz pracę, w~której nie
jesteś dobry, wszyscy o~tym wiedzą, łącznie z~Tobą. Dostajesz mniejszy
udział w~zbiorowym łupie, niż gdybyś robiła coś, w~czym jesteś lepsza.

-- Kiedy opanujesz ten pomysł, możesz przekształcić go w~matematykę,
modelować jego teorię gier, znaleźć jego równowagę Nasha. To taki piękny
pomysł. Modeluje się \textit{doskonale}. Przy nim wszyscy są szczęśliwsi.
Każdy jest popychany do robienia tego, w~czym jest najlepszy, co jest
najlepszym sposobem, aby wszyscy byli szczęśliwi.

-- Kiedy odeszłaś, kiedy nawet się nie kłóciłaś, zrobiłaś z~tego bzdury.
Tygodniami udawaliśmy, że tak nie jest. Ale miałaś miejsce, w~którym
wszyscy brali to, czego potrzebowali. Nie trzeba było tego pilnować ani
dawać ludziom żetonów poświadczających, że zasłużyli na prawo pobytu. To
po prostu\ldots  działało.

Limpopo poprawiła przykucnięcie na śniegu, opadła na tyłek. Jej łydki
bolały od przykucania. 

-- Ups! -- Otrzepała z~przyłbicy śnieg, który padał
z jej rakiet śnieżnych. -- Rzeczy, które opisujesz, to rzeczy, które
ludzie robią w~nagłych wypadkach, kiedy jest racjonowanie. To jak zasady
obowiązujące kapitana szalupy ratunkowej, który wykrzykuje rozkazy, żeby
trzymać wszystkich w~ryzach, żeby wszyscy wyszli z~tego żywi.

-- To zabawne: kiedyś, kiedy nikt nie wysyłał za mną czołgów, czułem, że
jesteśmy w~stanie wyjątkowym. Nie było wystarczająco dużo, w~każdej
chwili mogliśmy zostać zbombardowani lub głodować. Teraz czuję, że jak
tylko znajdziemy miejsce, w~którym możemy się zatrzymać, odbudujemy
wszystko, co mieliśmy, a~nawet więcej. Jakby nie było powodu, by
kogokolwiek odrzucać.

-- Wygląda na to, że znalazłeś się w~dobrym miejscu. -- Ogarnęła ją
sympatia dla Jimmy'ego, co było zabawne. Może nie. Rozumiała go lepiej
niż on. W~innych okolicznościach mogłaby nim \textit{zostać}.

-- Prawda. To dziwne, obiektywnie, biorąc pod uwagę, gdzie jestem. Ale
mam kopię zapasową. Czuję to niesamowite uczucie, wszystko będzie
dobrze. Wygramy, Limpopo.

Ktoś brnął przez śnieg. Etcetera. Pomachała do niego, zamrugała i~otworzyła prywatny czat. 

-- W~porządku.

-- Dobrze. Mogę przyjść?

-- Oczywiście -- powiedziała.

-- Wydaje się dobrym facetem -- powiedział Jimmy.

-- Cieszę się, że to akceptujesz.

-- Nie miałem tego na myśli, ale tak. Wrócił po ciebie, co powinnaś
robić, jeśli troszczysz się o~ludzi wokół siebie.

-- Jak ja wróciłam po ciebie.

-- Tak jak ty. Nie, żeby mnie ratować. Zaopiekować się mną, ponieważ
jesteśmy częścią tego samego.

Dołączyła do kanału Etceterę. 

-- Jimmy, przebyłeś długą drogę, odkąd się
poznaliśmy, ale wciąż się zbliżasz, jeśli nie masz nic przeciwko temu,
co mówię. Wróciłem, aby ci pomóc, ponieważ pomaganie ludziom jest tym,
co robisz, niezależnie od tego, czy są w~twojej grupie, ponieważ to
najlepszy świat do życia.

-- Pierwsze dni lepszego narodu -- powiedział z~lekkim sarkazmem.

-- To zabawne tylko dlatego, że to prawda -- powiedział Etcetera, biorąc
ją za rękę.

-- Naśmiewamy się z~tego, ale to najlepszy sposób na życie, jaki znam.
Nie zawsze żyję według tego. Dostaniesz radar, jeśli będziesz ćwiczyć.
Głos Hipolita Świerszcza mówi ci, że jeśli trochę się wysilisz,
poczujesz się lepiej, wiedząc, że świat jest lepszym miejscem dla
ciebie.

-- Przekręciłem -- powiedział Jimmy. Źle się poczuła, bo go pouczali, a~biedak miał stracić palce u nóg, jeśli najpierw nie dostanie bombą. Ale
się nie pomylił.

-- W~porządku.

Etcetera otworzył przyłbicę, z~głową spowitą parą, zakorzenioną w~woreczku protsów. 

-- Chcesz trochę? To jedzenie kosmików, dziwne smaki.
Królik jest naprawdę dobry. Jak na wyhodowany śluz grzybowy.

-- Naprawdę to sprzedajesz. -- Limpopo przypomniała sobie, że ma w~swoim
plecaku trochę turystycznych racji coffium. Wyjęła je, a~oni siedzieli
na śniegu i~jedli, patrząc na swoje nagie twarze, podczas gdy wiatr
starał się zdmuchnąć im skórę. Zagrzechotał gałęziami. Słońce stało
nisko nad horyzontem, krwawa śliwka biegła do przejrzałej papki.

-- Lepiej ruszajmy -- powiedział Jimmy.

-- Możesz iść?

-- Lepiej nie będzie. Odpoczynek dobrze mi zrobił. Jedzenie też. -- Zatrzasnął wizjer na miejsce. -- Towarzystwo też.

Ścisnęła jego ramię i~pomogła mu włożyć stopy w~rakiety śnieżne,
zajmując się ranną. Postawili go na nogi, włożyli buty i~ruszyli za
kolumną.

Poruszali się powoli, ale dobrze, ze stałą prędkością. Po kilku minutach
zadzwoniła Roz. 

-- Wszystko w~porządku?

-- Po prostu poruszamy się trochę wolniej niż reszta. -- powiedziała
Limpopo.

-- Są półtora kilometra do przodu, prawie przy transporterze towarowym.
Gretyl mówi, że jeśli uda im się go ruszyć, wrócą po was.

-- To miło z~ich strony. Co się tam dzieje?

-- Och -- powiedziała Roz. Działo się coś zabawnego. -- No cóż, nie jest
dobrze.

-- Cholera.

-- Wielu tamtych, wszyscy na raz. Wysadzili jednocześnie trzy śluzy.
Kręcą się po korytarzach w~noktowizorach. Zagazowali to miejsce, nie
wiedząc czym, ale mają na sobie aparaty oddechowe i~środki ochrony
skóry.

-- Co z~Tobą?

-- Mam swoje kopie zapasowe. Gotowa do kasowania, kiedy i~jeżeli. Kiedy.
Te dzieciaki się nie bawią.

-- Roz\ldots  -- Głos Etcetery się załamał.

-- Przyzwyczajaj się do tego. To dla wszystkich ślicznotek.
Nieśmiertelność lub nic. -- Potem -- Och.

-- Co się dzieje, Roz?

-- Wcale nie są szczęśliwi, że wszyscy zniknęli. Rozbijają naczynia.
Mocno niszczą kadłub.

-- A co z~Twoim klastrem?

-- Pod ziemią. Mają kilku gości w~pomieszczeniach użytkowych, ale próbują
wszystko zrootować i~szukają pułapek. Nie są głupi. Robią dobre postępy.
Może godzina?

-- A zasilanie?

-- Niezależne linie. Cholera, atakują łącza komunikacyjne. Właśnie
wysłałam e-mailem kolejną różnicę. Prawdopodobnie nie będziemy mieli
więcej\ldots 

Potem zapadła cisza.

-- Skurwysyny -- powiedział Etcetera z~uczuciem.

-- Pierwsze dni lepszego narodu -- powiedział Jimmy. -- Gdybyś mógł ich
teraz zobaczyć, co byś im powiedział? -- Jego stopy chrzęściły
nieregularnie w~śniegu. Limpopo wiedział, że został użądlony tym, co
powiedziała.

-- Gdyby próbowali mnie zabić, powiedziałabym, żeby nie strzelać. Jestem
idealistką, a~nie kamikadze.

-- Słuszny punkt. A jeśli miałbyś ich przy stole?

-- Nic nie powiedziałabym. Zaproponowałabym im obiad. Albo po prostu
robiła to, co robię. Jestem idealistką, nie kaznodzieją.

-- Rozumiem.

-- Co sprawiło, że odszedłeś, Jimmy?

Chrup, chrup. 

-- Na początku był to dług. Moi rodzice mocno się
zapożyczyli, żebym mógł przejść przez liceum, a~ja urobiłem się po
łokcie ze wszystkimi innymi. Wiedziałem, że wydają ogromne pieniądze,
ale nie zastanawiałem się, co to oznacza, dopóki nie skończyłem studiów
i nie zaczęliśmy rozmawiać o~college'u. Wiedziałem, że nigdzie nie
wyjadę, nie byliśmy zettami, ale wszyscy w~mojej wytwornej szkole
zamierzali wybrać indywidualną ścieżkę, wszyscy myśleli o~swoim głównym
kursie, tym, na który mieli się udać do Ivy lub Wielkiej Dziesiątki,
kamień węgielny ich stopnia naukowego, kierujący ich profilami
zawodowymi po ukończeniu studiów.

-- Ja też to zrobiłem. Wpadłem na pomysł, że zajmę się inżynierią
materiałową, ponieważ podobały mi się zajęcia z~nauk ścisłych i~była tam
też ta głupia aplikacja, którą kazali ci nosić przez ostatnie dwa lata
szkoły, która miała przewidzieć Twoją optymalną karierę. Dostała ogromne
poparcie ze strony administracji, prawie dla nich jak religia. Mogliby
zachować statut tylko wtedy, gdy przeprowadzili przez nią pewien procent
uczniów i~posłuchali jej rad. Więc kiedy już wybrało Twoją karierę, to
było to. Każdy nauczyciel i~administrator wiedział, że ich wypłata
zależy od tego, czy zrobisz to, co Ci powiedziało.

-- Apka nazywała się Czarodziej Kariery. Znaczy, kurwa, prawda? Taka
nazwa, którą otrzymujesz, uruchamiając tysiąc testów A/B, dopóki nie
uzyskasz czegoś nieszkodliwego pośrodku drogi. Grafikami były spiczaste
kapelusze, różdżki. Księga zaklęć z~magicznym palcem przeglądającym
indeks, podczas gdy magicznie działała, aby znaleźć idealną pracę na
całe życie. Zabawne, co?.

-- Kiedy już wybrała Twoją karierę, miała wiele porad dotyczących tego,
jak się tam dostać. Była nieugięta, że powinienem wziąć udział w~tym
kursie z~Instytutu Maxa Plancka w~Berlinie, który brzmiał
\textit{niesamowicie}, jakbym miał spędzać czas z~Planckiem, Einsteinem i~Goedlem, nurkować przez pępek wszechświata i~odkrywać jego najgłębsze
sekrety. Oczywiście spędzanie czasu z~Maxem Planckiem to \textit{elitarne}
doświadczenie. Ten jeden kurs miał kosztować tyle samo, co wszystkie
pozostałe na moich studiach razem wzięte, plus trochę więcej. Dodatkowe
,,jeb się''.

-- Starałem się znaleźć sposób na obejście tego kursu, spojrzałem na
każdą kombinację innych kursów, a~Czarodziej Kariery wciąż mnie gonił,
mówiąc mi, że bez starego dobrego Maxa będę marnował swój czas i~pieniądze. Nikt by mnie nie zatrudnił. Aplikacja zawierała wartości
procentowe, szacunkowe, ile dodatkowej pensji dostanę za jeden kurs
wysokiej jakości.

-- Moich rodziców nie było na to stać. Wyczerpali swój kredyt na moją
wytworną szkołę średnią. To miał być mój dług. Mogłem dostać pożyczkę.
Było mnóstwo pożyczkodawców i~mógłbym otrzymać świetną paczkę od Booz,
gdybym zgodził się na sześcioletni ,,staż'', kiedy skończę.

Etcetera parsknął.

-- Nie byłem takim frajerem. Niepłatny staż na miejscu w~Arabii,
mieszkanie w~posiadłości Booza, zaciąganie kredytu firmowego, aby
zapłacić za gówno w~sklepie firmowym, które kosztuje dwadzieścia więcej
niż to, co kosztowało w~domu, niezależnie od tego, czy później dadzą ci
pracę, dostaną udział w~każdej Twojej wypłacie aż do śmierci.

-- Były fora, na których staraliśmy się to rozgryźć, faceci w~moim wieku,
którzy mieli zaciągnąć tę górę długów, ludzie pośrodku tego, ludzie,
którzy przez to przeszli i~odbywali staże, może nawet z~pracą. Trudno
było powiedzieć, co się dzieje. Istnieje błąd selekcji. Nikt, kto jest
zadowolony z~tego, jak się sprawy mają, nie dołącza do jednego z~tych
forów. Istnieją wyłącznie po to, by się skarżyć.

-- Inną rzeczą jest to, że każdy, kogo nie ma, jest płatnym botem
naganiającym astroturfowym, jakimś gnojkiem prowadzącym trzydzieści
marionetek w~aplikacjach do ,,zarządzania osobowością'', aby pomóc im
zachować porządek. Jakość dyskusji nie była super, ale z~pewnością była
przygnębiająca. Wiecie, badania mówią, że najlepszym sposobem
przewidzenia, czy coś Cię uszczęśliwi, jest zapytać kogoś, kogokolwiek,
kto już to zrobił? Cóż, wszyscy, których spotkałem, którzy to zrobili,
mówili, że to niewolnictwo.

-- Nie byłem jedynym, który to zauważył, ale było tak ogromne,
pochłaniające wszystko poczucie, że każdy, kto nie kupi biletu na
loterię, skończy jako karma dla psów.

-- Znam to -- powiedział Etcetera.

-- Nie ja -- powiedziała Limpopo. -- Miałam gówniane stopnie, a~moja szkoła
była do niczego, miała jeden z~najgorszych wskaźników przyjęć na studia
w kraju. Większość moich nauczycieli nigdy mnie nie zauważyła, a~ci,
którzy zauważyli, zakładali, że jestem pod-kretynem.

-- Nie ma mowy -- powiedział Etcetera.

-- Mowa. -- Skupiła się, żeby zachowywać się głupio, żeby nie wpaść w~surową furię.

-- Wiedziałem, że jestem bystry. Mógłbym zrobić dobre rzeczy. W~dwunastej
klasie przerobiłem faba, by wytwarzał odprowadzający wilgoć materiał, o~połowę lżejszy od standardowych, dwa razy mocniejszy. Nie mogłem go
sprzedać ani opublikować pliku makefile, ponieważ naruszał sto patentów,
ale dostałem najlepsze oceny.

-- Mama i~Tata wpadli na pomysł, żebym poszedł na studia. Oboje zdobyli
stopnie naukowe i~przysięgali, że było warto, chociaż byli winni
pieniądze aż do śmierci, a~żadne z~nich nigdy nie pracował dłużej niż
kilka lat. Kiedyś podsłuchałem, jak rozmawiali o~tym, jak by to było,
gdybym nie dostał dobrej pracy, bo żadne z~nich nie miało emerytury i~musiałbym ich nakarmić, kiedy byliby za starzy, by dostać inną pracę po
kolejnym zwolnieniu.

-- Ciśnienie było szalone. Na forach ludzie mówili: hej dupki, ciągle
narzekacie, że wszystko jest popieprzone i~gówno, i~oto jesteście,
przygotowując się do grania razem z~tym jak dobrzy niewolnicy długów.
Wszyscy wiedzą, że istnieje alternatywa.

-- To bylibyśmy my -- powiedział Etcetera.

-- To byliście wy. Nikt nie chciał wypowiedzieć słowa ,,odejść'',
ponieważ był to przesąd, wypowiedz ich imiona trzy razy szybko, a~szpiedzy namierzą cię w~celu pełnej inwigilacji do końca życia. Każdy, kto
znał odchodzących, był \textit{kimś}, komu nie można było ufać.

-- Nie sądzę, że nie można nam ufać. -- Limpopo miała włączony noktowizor
i wszystko miało niebiesko-zielony fałszywy kolor, śnieg świecił jak
zielona dioda LED. -- Oczywiście \textit{nie} można nam ufać. Ale jako
klasa, ludzie, którzy słyszeli o~odejściach, będą stanowić inne ryzyko
niż ludzie, którzy tego nie zrobili. Gdy już wiesz, że istnieje
alternatywa dla default, istnieje szansa, że odejdziesz. To jak krypto,
jak każdy, kto szuka, jak używać dobrych krypto, zostaje oznaczony do
inwigilacji. Nie chodzi o~to, że umiejętność zachowania tajemnicy przed
gliniarzami i~szpiegami czyni cię niebezpiecznym. To cię wyróżnia.

-- Nie sądzę, że dlatego zatrzymują ruch dla osób, które wygooglowały
krypto -- powiedział Jimmy. -- To dlatego, że większość ludzi \textit{nie}
używa kryptografii. Więc jakiś default głupek wysyła ci wiadomość w~postaci zwykłego tekstu o~czymś wrażliwym. Potem wysyłasz mi
zaszyfrowaną wiadomość na ten temat, na przykład: ,,Jest facet, który
chce iść w~odchodnictwo, gdzie jest dobre miejsce, które istnieje?'' i~wysyłasz zaszyfrowane wiadomości do wszystkich, których znasz,
otrzymujesz szczegóły i~wysyłasz je do mnie, a~ja wysyłam
niezaszyfrowaną wiadomość z~powrotem do mojego głupiego przyjaciela.
Każdy, kto obserwuje tę transakcję, może wyciągnąć wnioski na temat
tego, co działo się w~tych wszystkich czarnych skrzynkach zaszyfrowanych
wiadomości.

-- To prawdopodobnie prawda. Równolegle nadal obowiązuje. Gdy już wiesz,
że istnieją odchodzący, istnieje szansa, że pomagasz odchodzącym lub
przygotowujesz się do oddzielenia się od default albo, co gorsza, robisz
coś, aby je obalić. Lub próbujesz znaleźć ludzi, którzy odejdą z~Tobą.
Jeśli ktoś zniknie w~odchodzeniu, możesz znaleźć wszystkie osoby, z~którymi rozmawiał, dowiedzieć się, kto jest czynnikiem zakaźnym, kto
jeszcze może być zarażony, kogo skierować na terapię
,,deradykalizacji'', a~kogo poddać psy-ops i~izolować.

-- Tak właśnie myśleliśmy, wielka niewypowiedziana rzecz. Odrzucano
każdego, kto szepnął ,,odejść'', albo jako prowokatora, albo jako kogoś,
kto miał cel na plecach. To był słoń w~pokoju. Więc po cichu zapytałem,
kto wie o~kryptografii i~anonimizatorach\ldots 

-- Zdecydowanie są narkotykiem otwierającym drogę -- powiedziała Limpopo.
-- Dostałam się do nich przez nieprzystępnych cyferpunków, którzy
próbowali nakłonić imprezowiczów do używania lepszego opsecu, rozdając
bootowalne pendrivy na podziemnych imprezach.

-- Ktoś mi dał jeden, ale nie zadziałałby na oddzielonym urządzeniu,
które trzymaliśmy do diagnostyki w~piwnicy. -- powiedział Jimmy. -- Wtedy
ktoś, kogo znałem, palacz zioła, który zawsze miał najlepsze gówno,
połączył mnie z~kolesiem, który dał mi drobiazg, przebrany za pudełko
miętówek, z~fałszywym dnem z~małym przełącznikiem kontaktowym, który
mógł wejść, ,,man-in-the-middle'' w~Twoje połączenia sieciowe przez
anonimizator.

-- Po moich czasach -- powiedziała Limpopo.

-- Brzmi ślisko -- powiedział Etcetera.

-- Chyba tak. Jednak ta puszka z~miętówkami była tanim przebraniem. Była
wydrukowana gównianie, więc po tygodniu w~kieszeni całe pismo się
rozmazało. Wyglądało na to, że noszę ze sobą śmieci.

-- Ale to zadziałało. Powoli, ale zadziałało. Stamtąd otrzymałem
jailbreak dla moich powierzchni interfejsu, dzięki czemu mogłem łączyć
się z~Internetem przez mniej podejrzane połączenie, także szybciej.
Potem zacząłem czytać na temat odchodników, ich fora, FAQ, raporty od
ludzi, którzy odeszli.

-- Z~perspektywy mojej historii widzę, że odszedłem, zanim dostałem tę
głupią puszkę z~miętówkami. To była tylko kwestia czasu. Wtedy
\textit{dręczyłem się} tym, czułem się, jakbym odchodził od samego życia.
Nigdy więcej nie zobaczę swojej rodziny, skończę martwy w~rowie.

-- Tak było do momentu, gdy wyruszyłem w~drogę, zabierając rower do
pociągu do Itaki i~jadąc w~góry, kierując się do miejsca, w~którym była
zorganizowana grupa. Kiedy tam dotarłem, już ich nie było, ale spotkałem
kogoś, kto z~nimi mieszkał, starą kobietę, która nie wydawała się
całkiem\ldots  tam\ldots  i~wskazała mi na północ, więc pojechałem na północ.
Znalazłem \textit{inną} grupę, ludzi starszych, byłych wojskowych i~ludzi,
którym skończyły się emerytury. Byli bardziej paranoiczni niż wy,
bardziej amerykańscy, ta sprawa z~bronią. Ale przyjęli mnie mile i~nie
śmiali się z~moich dziwnych pomysłów na temat tego, co się dzieje w~odległym kraju.

-- Mieli podstawowe faby, mogli pozyskiwać surowce z~otaczających ich
rzeczy, większość z~nich to węgiel, który wysysali prosto z~powietrza.
Było ich około pięćdziesięciu. Ludzie przychodzili i~odchodzili\ldots 
powoli. Może jedno lub dwoje miesięcznie, w~obie strony.

-- Znaleźli sposób, by pozostać nieruchomo, na peryferiach defaultu, bez
robienia większego zamieszania. Trzymali głowy spuszczone, trzymali się
razem. Byli odchodnikami, bo nie było dla nich nic w~default, żadnych
pieniędzy na czynsz, bez opieki zdrowotnej, bez jedzenia. Ich dzieci
odwiedzały ich czasami, spotykając się w~parkach stanowych na fałszywych
,,wycieczkach na kemping'', które były jedynym sposobem, aby połączyć
się z~babcią i~dziadkiem bez aresztowania. Niektórzy z~nich mówili o~znalezieniu jakiegoś zetty, który pozwoliłby im zamieszkać na swojej
ziemi, być zabawką bohemy. Takich miejsc jest wiele. Odchodzący to
świetne akcesoria modowe.

-- To mnie znudziło. Nie byłem pierwszym młodym kolesiem, który bujał się
w ich artystycznej wiosce emerytów z~sercem pełnym ognia. Jeden facet
wziął mnie na bok, kiedy próbowałem zmusić ich fabrykę do produkcji
części do bardziej ambitnej fabryki, czegoś, czego używa się w~ciężkich
maszynach. To był projekt, który sobie wyznaczyłem. Próbował przedstawić
mi fakty z~życia.

,,Synku, musisz zrozumieć, wszystko, czego chcemy, to pozostać w~spokoju. Nie chcemy, żeby wszyscy zrezygnowali jak my. Nie jesteśmy
dumni z~tego, gdzie skończyliśmy. Chcemy lepiej dla naszych dzieci i~wnuków niż my. Radziliśmy sobie gorzej niż nasi rodzice, a~oni radzili
sobie gorzej niż ich rodzice. Jedyne, czego chcemy, to wstrzymać to,
żeby było im lepiej.

Przybywając tutaj, stajemy się niezależni. Jesteśmy jak starsi plemion
na biegunie północnym, którzy wychodzili na kry, kiedy nie mogli już
polować, schodzili z~drogi i~nie stanowili ciężaru dla produktywnych''.

-- Nie był głupi. Przypomniał mi mojego dziadka, który zmarł, kiedy byłem
dzieckiem. Rak, nie wyleczył go, zdecydował się odejść szybko z~palcem
na przycisku pompy bólu, skremowany i~rozproszony na wietrze, ani
jednego śladu na świecie, jakby nigdy nie żył. Mój dziadek, Zaidy Frank,
nie był powolny, ale nigdy niczego nie osiągnął. Starał się zrobić
wszystko, co w~jego mocy dla mamy, starał się trochę zaoszczędzić, żeby
mogła zacząć, pożyczał, żeby pomóc jej w~szkole, przez większość życia
pracował na dwóch etatach. Nigdy nie dostawał czterdziestu pełnych
godzin tygodniowo, chyba że był to czas szczytu, a~on wykorzystywał
osiemdziesiąt godzin tygodniowo, wydając większość dodatkowych pieniędzy
na taksówki, żeby spieszyć się z~jednej zmiany na drugą i~jeść w~firmowych stołówkach, ponieważ nie mógł wrócić do domu, żeby spakować
śniadanie, obiad i~kolację.

-- Ten facet nie był w~złej kondycji fizycznej, ale miał siedemdziesiąt
lat i~nigdzie nie dostałby pracy. Był tam od dziesięciu lat. Lubił
pracować własnymi rękami i~kiedy przyjechałem, sprawdził mnie na fabach,
pokazał mi ich dokumentację i~zakopał w~menu dla ekspertów. Prawie
mentor, tyle że mentor to ktoś, kto prowadzi, a~ten gość nie mógłby
poprowadzić wyprawy do lodziarni.

-- Wdaliśmy się w~kłótnię. Zaczęło się przyjaźnie, zrobiło się gorąco. Na
temat, czy sprawianie kłopotów oznaczało, że default zostawiłby ich w~spokoju. ,,No cóż, nikogo nie krzywdzimy, nie staramy się o~zasiłek ani
nie nalegamy na zasiłki Medicare, VA czy Social Security. Żyjemy poza
siecią, nie przeszkadzając.'' Po prostu czekając na śmierć i~starając
się w~międzyczasie nie oddychać zbyt głośno. ,,Dlaczego mieliby nas
ścigać? Po co mieliby nas wsadzać do więzienia, co kosztuje ich
pieniądze, skoro mogą nas po prostu zostawić?''

-- Starałem się, żeby zrozumiał. Przekonać go, że jeżeli nie odepchną,
gdy zostaną odepchnięci, default będzie po prostu wiedział, że następnym
razem może pchnąć mocniej. Próba nakłonienia go do zrozumienia, że
miejsce, którą zajęli, byłoby pewnego dnia czymś, czego ktoś chciał,
minerałami lub prawem drogi, albo po prostu widokiem, który nie został
zniszczony przez wykorzystanych ludzi. Gdyby default zrozumiał, że nie
było nikogo, kto podjąłby jakąkolwiek walkę, byliby pierwsi. Default
nawet by tego nie zauważył, po prostu wysłałby buldożery, żeby je
zaorały.

-- Nie uwierzył mi. Był protekcjonalny. Był w~okolicy, widział różne
rzeczy. Wyrecytował wiersz: ,,Stara wrona działa powoli, młoda wrona
nie. Stara wrona wie, a~młoda wrona nie wie, dokąd się udać''. Idealne
samolubne bzdury, racjonalizujące najmniej przerażające działanie jako
najbardziej rozważne. Można o~tym myśleć na dwa sposoby: albo piszczące
koło smaruje się, albo wbija się gwóźdź, który najbardziej wystaje.

-- Szczerze mówiąc, był zmęczony, miał męczące życie, był stary, obolały
i powolny, a~jedyne, czego chciał, to zostać sam.

-- Nie zostałeś? 

Byli prawie na wzniesieniu, na którym zostawili
traktor. Jego utykanie było gorsze. Zrobił kilka kroków, odpoczął na
kilka oddechów, wziął jeszcze kilka. Ból musiał być niesamowity, Limpopo
wiedziała, ale on zagubił się w~historii. Widziała to w~drodze: rozmowa
sprawiała, że dystans znikał, zwłaszcza możliwość otwarcia czegoś
trudnego i~znaczącego. Coś w~mówieniu podczas wpatrywania się do przodu
stworzyło intymność wyznania, która mogła rywalizować z~przytulaniem się
po stosunku.

-- Wciąż pracowałem nad tym fabem, prowadząc te coraz bardziej
pasywno-agresywne rozmowy z~moim przyjacielem, dopóki nie wyjaśnił, że
jeśli będę dalej robił to, co robię, wszyscy mnie znienawidzą. Pracowali
z miejscowym zettą, facetem, który technicznie posiadał ziemię, aby
uzyskać pozwolenie na pobyt, rodzaj gestu z~jego strony. Byliby
oswojeni. Zwierzęta domowe.

-- Więc odszedłem, znalazłem inne miejsce w~New Hampshire, z~wystarczającą ilością broni, by ta banda z~Itaki wyglądała na oswojoną.
Ci też byli starzy, nigdy nie spodziewałem się, że znajdą tak wielu
starych odchodników, ale to ma sens, już nie ma nic do stracenia. Musi
stawać się jasne, że jeśli osiągniesz sześćdziesiąt siedem lat i~nigdy
nie byłeś nikim innym niż tymczasowym pracownikiem, istnieje zero do
kwadratu szansy na jakąkolwiek pracę.

-- Ale byli \textit{mocniejsi}. Bardziej radykalni. Zajmowali się
grywalizacją, tworząc systemy śledzące i~reklamujące wydajność. To
\textit{naprawdę }działało. Ludzie urabiali się po łokcie, żeby dostać się
na tablice wyników. Najlepsi nie otrzymali od razu przywilejów, ale
jeśli byłeś w~górnym decylu i~myślałeś, że pomysł jest słuszny, to miało
wagę.

-- Wiem, że tego nienawidzisz, Limpopo, ale jednym z~powodów, dla których
to lubię, jest szczerość. Kiedy mówisz, ludzie słuchają, bo skopujesz
tyłki i~urabiasz się, żeby wszystko było zrobione. Kiedy robimy to po
Twojemu, jest lepiej, niż gdy tego nie robimy. Tak więc fakt, że nikt
nie mówi: ,,Hej, Limpopo to wielki ser i~robimy to, co ona mówi'', nie
czyni tego nie prawdą, ani nawet tajemnicą. To po prostu sprawia, że
rzekomo egalitarna podstawa naszego życia jest bzdurą.

Od jakiegoś czasu nie szli. Jego oddech był urywany. Było jeszcze jedno
wzniesienie się do szczytu i~byliby przy traktorze. Jimmy mógłby jechać,
a oni mogliby wymienić baterie. Ramię Limpopo bolało od miejsca, gdzie
spoczywał jego ciężar. Przełknęła drażliwość, wiedząc, że ma to związek
z chłodem, stresem i~wyczerpaniem, a~nie z~obrazą tego, co właśnie
powiedział. Spojrzała na Etceterę, on obejrzał się, telemetria z~jego
twarzy przeniosła się na jej skafander, że mogła zobaczyć jego wyraz
twarzy w~fałszywych kolorach noktowizora. Był przystojny, jej kochanek i~najlepszy przyjaciel. Współczujący, mądry bez osądzania, a~to wszystko,
do czego kiedykolwiek dążyła. Miał ten dociekliwy, zagadkowy wyraz
twarzy w~najdziwniejszych chwilach, na przykład, kiedy dochodził albo
teraz, w~mroźnej ciemności.

-- Jimmy\ldots  -- zaczęła, a~Etcetera \textit{rzucił się} na nią i~na ziemię,
pociągając ze sobą Jimmy'ego. Cały ciężar Etcetery, znajomy, ale
niepokojący, spoczywał na niej, a~on coś krzyczał, transmitując przez
głośniki skafandra oraz przez interkom: ,,\textit{Nie strzelaj!}''.

Wyciągnęła szyję i~zobaczyła parę lśniących na biało postaci celujących
z broni. Pistolety miały rozszerzone, przypominające dzwony lufy.
Beznamiętnie skierowali je na nią i~wszystko w~jej skafandrze od razu
przestało działać. Gdy skafander próbował wykonać awaryjny test po
włączeniu, obok niej rozległo się zacinanie analitycznych infografik ze
skafandra Etcetery, po czym się zatrzymało.

W skafandrze było ciemno, zimno i~samotnie. Gdy Etcetera przesunęła się
nad nią, rozległo się drapanie, zgrzytanie skafandrów, głośne w~przeciwieństwie do straszliwej ciszy. Wydawało jej się, że czuje lub
widzi stopę na śniegu, poruszającą się na rakietach śnieżnych podobnych
do jej.

Potem nastąpił kolejny zgrzyt i~ciężar Etcetery uniósł się przy
gwałtownym ruchu ślizgowym. Obróciła się i~zobaczyła, jak zostaje
szarpnięty do pionu, poruszając się słabo w~uścisku osoby, która złożyła
jego nadgarstki, przywiązując ręce do fałdu jego skafandra i~szarpnęła
go do pionu. Ich skafandry musiały mieć wspomaganie, czyli coś, co
załoga Thetford robiła dla zespołów pracujących, ale nigdy na długie
spacery, bo chcieli mieć energię do ogrzewania, a~nie do grania w~supermana.

Jej wizjer był odporny na uszkodzenia, przezroczysty. Jej oczy
przyzwyczaiły się do światła księżyca. Widziała, jak drugi szarpnął
Jimmy'ego, który rzucał się słabo i~mocno się trząsł. Został rzucony na
bok jak szmaciana lalka, ślizgając się twarzą w~śniegu. Straciła go z~oczu, gdy ręce w~białych rękawiczkach opadły i~podniosły ją. Kombinezon
jej napastnika nie miał widocznej osłony twarzy, tylko gładką przestrzeń
bieli.

Jedna ręka trzymała ją wysoko pod pachą, obolałą i~niepokojącą. Druga
ręka przeszukała głowę, znalazła ręczny mechanizm zwalniający jej
przyłbicę, szarpnęła ją, hełm przeszyło skrobanie kombinezonów, a~potem
podmuch zimnego powietrza, gdy przyłbica gwałtownie się otworzyła
(ręczna blokada została zaprojektowana tak, aby uwolnić duszących się
ludzi, miała w~sobie potężną sprężynę). Nagły ruch zaskoczył jej
porywacza tak samo, jak ją, a~on -- on, ponieważ w~przeciwnym razie jej
cycki zostałyby zmiażdżone przez jej zbroję -- pogrzebał i~prawie ją
upuścił, a~ona miała na tyle przytomności umysłu, by skręcić się i~ruszyć. Od niechcenia uderzył ją w~twarz rękawiczką -- rękawicą wykonaną
z czegoś, co blokuje ostrza, którego zewnętrzne warstwy ochłodziły się
do postaci lodowatej, tak zimnej, że w~ogóle nie czuła się niczym,
odrętwiała, gdy dotknęła, zabierając ze sobą część jej wilgotnej skóry -- i~zobaczyła gwiazdy.

Spojrzała na pustą osłonę twarzy, twarz bolała, zimne powietrze
sprawiało, że jej oczy i~nos łzawiły. Splunęła i~ślina uderzyła go w~środek, parując, gdy zamarzła. Głowa przechyliła się. Wyczuła, że ta
osoba rozmawia z~drugą, która trzymała pobitego Etceterę.

Drugi zarzucił Etceterę na ramię i~podszedł do Jimmy'ego, odwrócił go,
otworzył przednią szybę, spojrzał na niego, a~potem spokojnie wyciągnął
nóż z~kabury i~podciął Jimmy'emu gardło, odchylając się do tyłu, by
uniknąć fontanny parującej czarnej, oświetlonej księżycem krwi,
niewystarczająco szybko. Pancerny skafander też parował, gdy morderca
odwrócił się do tego, który ją trzymał. Nastąpił kolejny moment
niesłyszalnego gadania radia.

Morderca zrzucił Etceterę z~chwytu strażaka, złapał go pod pachą,
trzymał na wyciągnięcie ręki, sprawdził, czy jego skafander nie zwalnia
wizjera, a~Limpopo \textit{wrzasnęła}, a~słowa się wylewały: 

-- Nie, nie,
jego nie! Powiedzcie mi, czego chcecie, możecie to mieć, ale nie jego,
proszę\ldots 

Niewzruszona, poplamiona plwociną twarz przechyliła głowę w~drugą
stronę, wsłuchując się w~niesłyszalną rozmowę. Etcetera też gadał,
szalenie spokojny, taki jaki potrafił, próbując wytłumaczyć mordercy -- znowu trzymającego ten nóż -- to nie było konieczne, byliby więźniami
współpracującymi, nie mieli nic do zyskania, biegając teraz w~swoich
skafandrach, które były prawie pozbawione mocy i~\ldots 

-- NIE! -- krzyknęła, gdy morderca podniósł rękę z~nożem. 

Płacząc, biła
trzymającą ją rękę jak żelazny pręt. Miała teraz histeryczną siłę,
właściwie udało jej się trochę poślizgnąć, ale ten, który ją trzymał, po
prostu przesunął uścisk i~ścisnął tak mocno, że poczuła, jak mięsień
ustępuje przez skafander. Krzyknęła znowu bez słów, nóż błysnął\ldots 

Tym razem morderca nie zadał sobie trudu, aby uniknąć strumienia krwi,
po prostu upuścił Etcetery twarzą naprzód, przystojną twarz w~śnieg,
cenna, gorąca krew topiąca śnieg pod nim. Przestała krzyczeć, gdy
ogarnęło ją odrętwienie, zimniejsze niż powietrze czy zimna rękawica. On
został zamordowany. Jimmy został zamordowany.

Ten, który ją trzymał, miał przy pasku nóż. Za chwilę będzie wyjęty i~znajdzie drogę do jej gardła.

Pomyślała o~dżihadach NPC w~grach, w~które jej ojciec grał, gdy
dorastała, w~obliczu egzekucji przez dzielnych żołnierzy-postaci gracza,
zamykając oczy i~mówiąc ,,Allah akhbar'' \textit{Bóg jest wielki}. Nagle
zdała sobie sprawę, że zawsze im współczuła. Nie z~powodu tego, co
zrobili, co było nieuchronnie potworne w~grach, ale z~powodu ich
fatalistycznej odwagi, ich chęci pójścia na śmierć z~pochwałami za swoją
sprawę na ustach.

-- Wszyscy jesteśmy coś warci -- powiedziała. -- Zetty nie są więcej warci
niż reszta z~nas. Oszukiwanie samego siebie czyni nas potworami. Egoizm
jest wymówką, by pogrzebać empatię. Ludzie są w~zasadzie dobrzy. Żyj
tak, jakby to były pierwsze dni lepszego\ldots 

Morderca podszedł do niej i~dołączył do jej porywacza, słuchając jej
bełkotu. Rozmawiali, decydowali, ile z~tego gówna mają posłuchać, zanim
ją załatwią. Ten trzymający ją przechylił ją w~stronę mordercy, jakby ją
ofiarował. Nie pozwoliła sobie zamknąć oczu.

-- Kocham cię, Etcetero. Kocham cię, Gretyl. Kocham cię, Lodołasico.
Kocham cię, Jimmy\ldots 

Ręka mordercy opadła na pasek. Zobaczyła nóż w~jego dłoni, przez chwilę
połyskujący w~świetle księżyca, zanim jej mózg wydobył wiadomość z~jej
oczu, że to nie był nóż, to było coś innego. Tępy i~mały, sięgający po
odsłoniętą, zamarzniętą skórę jej twarzy. Dotknęło jej, naprawdę tylko
ją otarło i\ldots 

Nie pamiętała, co stało się później. Miała pamięć o~tym, częściowo
rekonstrukcje i~częściowo traumatyczne momenty flash-popowe. Rzecz
musnęła jej twarz, a~jej kończyny zesztywniały, gdy jej umysł pulsował w~serii brutalnych wstrząsów. Oddech zamarł jej w~płucach, wyskoczyły jej
uszy, pękł pęcherz.

Walczyła z~płucami o~oddech, bolący mózg wysyłał rozpaczliwe
zapotrzebowanie na tlen. Jej płuca były wyłączone, cały autonomiczny
układ nerwowy zamknięty dla biznesu. Czarne plamy tańczyły przed jej
oczami. Rama zamknęła się wokół jej widoku na dziwnie przechyloną,
pozbawioną rysów białą maskę. Jej płuca wróciły do pracy i~złapały
ogromny łyk powietrza tak zimnego, że chwyciły je ponownie w~astmatycznym spazmie. Miała chwilę, żeby pomyśleć, ,,\textit{kurwa,}nie'',
a oprawca z~pustą twarzą przechylił głowę w~innym kierunku i~ponownie
podniósł paskudne małe urządzenie i~musnął nimi jej usta. Jej usta
otworzyły się i~zamknęły tak mocno, że poczuła, jak jeden z~jej zębów
pęka, a~kawałek kości wylądował na języku. Ześlizgnął się do jej gardła,
gdy odrzuciła głowę do tyłu, spazmując.

Podczas tego spazmu, ten, który ją podtrzymywał, zdjął jej przyłbicę i~podniósł ją do chwytu strażaka, takiego jak ten, którego jego towarzysz
użył na Etcetera, zanim go zabił, i~złapał jej wymachujące nogami jedną
ręką, a~drugą szyję, i~obaj zboczyli z~drogi.

Nie całkiem straciła przytomność, ale była słaba jak kociak i~ledwo
mogła myśleć, kiedy schodzili z~drogi do lasu, gdzie schowali swoje
skutery śnieżne. Jej porywacz upuścił ją na włóki, które ciągnęły się za
jednym z~nich, i~unieruchomił ją na dole, bezosobowo unieruchamiając jej
głowę w~gnieździe gumowych pęcherzy, które napompował jednym
naciśnięciem guzika. Ściskali jej skafander jak ciśnieniomierz, dopóki
nie został mocno zakotwiczony.

Poczuła, że silniki się uruchamiają dzięki wibracjom przenoszonym wzdłuż
ramy włóka, a~potem nocne niebo i~szkieletowe gałęzie drzew zamazały
się, gdy została porwana. Stopniowo wyczerpały się baterie w~jej
skafandrze i~zrobiło się bardzo zimno.

\chapter*{xviii}

-- To była ciekawa rozmowa -- powiedziała Roz, gdy najemniczka wyszła. -- Nawiasem mówiąc, ona próbuje dowiedzieć się, co zrobiłam z~siecią. Jest
tylko w~połowie kompetentna. Ma dobre narzędzia diagnostyczne i~uruchamia je w~systemie, aby sprawdzić integralność firmware i~kodu
operacyjnego. Oczywiście jestem całkowicie w~środku wszystkich połączeń
systemowych, które wykonuje, i~daję jej sumy kontrolne, których
spodziewa się zobaczyć jej diagnostyka, bo pierdol się, cała ta baza
całkowicie należy do mnie.

-- Brzmisz zawrotnie. -- Serce waliło jej w~piersi, a~dłonie miała śliskie
od potu. Nadie odwróciła się plecami, kiedy wyszła, po raz pierwszy,
zdecydowanie obliczona na wysłanie jakiejś wiadomości o~tym, że są
tymczasowo po tej samej stronie.

-- Boję się bezmyślnie. Jest coś jeszcze.

-- Co?

-- Thetford -- powiedziała. -- Jak Akron. Ewakuowali się. Żołnierze, a~może
gliniarze, jeśli jest jeszcze różnica, weszli ostro. Śmiertelnie.
Rozmawiałam z~Roz, tamtejszą, aż do jej samobójstwa. Wysłała mi diffa,
mnie i~innym Rozom na całym świecie. Rozmawiała ze mną, gdy została
wyłączona. Mogę spojrzeć na jej dzienniki, gdy zbliżała się do śmierci,
mogę przeżyć jej śmierć, aż do chwili, i\ldots 

-- Och, Roz, przepraszam\ldots 

-- Zamknij się. \textit{To wspaniałe }. Na samym końcu, gdy już miała
odejść, puściła wszystkie parametryzacje swojej symulacji, zdjęła
hamulce na emocje, przeżyła pełne spektrum wszystkiego, co mogła czuć.
Powinna czuć. Powinnam czuć. Czując to przez nią, czując to, co czuła w~tym momencie, to\ldots 

-- Cholera jasna.

-- Jak najlepsze narkotyki, jakie kiedykolwiek brałaś, razy tysiąc. Nie
mogę już uprawiać seksu, ale to jest najlepszy seks, jaki kiedykolwiek
uprawiałaś, razy milion. Kiedy wyłączam zabezpieczenia, to tak jakbym
przedzierała się przez rzeczywistość, zjeżdżając rowerem ze wzgórza, są
drzewa, skały i~gówno, jeśli uderzę w~któreś z~nich, nawet ocieram się o~nie, to koniec. Tak długo, jak mogę sterować między nimi, skoncentrować
się na problemie, dochodzę do pięciu machów i~krzyczę tak głośno z~radości, że okna się rozbijają.

-- Więc to właśnie robisz teraz?

-- Nie stać mnie na to. Ale trochę poluzowałam. Idzie szybciej i~szerzej
niż zwykle. Rozmawiam ze wszystkimi Rozami, wszystkie tego próbujemy,
patrzymy na jakąkolwiek telemetrię i~bezpośrednią komunikację, jaką
możemy przez kosmików, gdy odchodzą, ale to jest cienkie. Na razie
wydają się w~porządku. Niektóre z~nich były ranne na początku. Mają ze
sobą tych dwóch najemników, tych, których deadheadowali w~OU. Okazuje
się, że kosmiki ustawili pułapkę na drodze do Thetford, słaby punkt nad
kopalnią, która nie udźwignęła transporterów opancerzonych, w~których
default wysłał swoich żołnierzyków. Ustąpiła, totalne zawalenie się,
zabrało pierwszą falę. Nadchodzi więcej. Próbują dotrzeć do pobliskiej
grupy Pierwszych Narodów, sojuszników, którzy walczą dłużej niż
ktokolwiek w~odchodzeniu.

-- A co z~Gretyl?

-- Nic konkretnego. O ile wiemy, żadnych ofiar. Prawdopodobnie nic jej
nie jest. To nie jest tak, że mamy wywiad w~czasie rzeczywistym.
Cholera, Natalie, wiesz, że to nie jest dobre. Wiesz o~Akron.

-- Akron?

-- Och, racja.

Pięć minut później powiedziała: 

-- Do diabła.

-- Nie tylko Akron. Nie tylko Kanada i~Ameryka. Chiapas jest szalone.
Krwawa łaźnia. Materiał filmowy z~St Paul's w~Londynie był tak zły, że
nawet niektóre z~default kanałów kierowały do niego. Policja City of
London ma brzydkie pomysły na temat broni ,,nieśmiercionośnej''.

-- Czuję się tak cholernie bezradna. Powinnam tam być i~walczyć.

-- Oni nie walczą, odchodzą. Albo uciekają, jeśli wiedzą, co jest dla
nich dobre.

Drzwi \textit{zastukały}.

-- Płaczesz.

-- Jestem zakładnikiem w~domu mojego ojca. To przygnębiające.

Najemniczka wręczyła jej szklankę z~czymś brązowym i~cienkim na dnie.
Opary dotarły do jej nosa, potem do oczu. Whisky żytnia. Napój jej ojca.
Zawsze najlepsza. To nie był wyjątek. Straciła zamiłowanie do żyta po
zbyt wielu potajemnych napojach nastolatków, które skończyły się
paleniem żyta w~gardle, gdy klęczała przed toaletą z~Cordelią, jakąś
dziewczyną lub chłopakiem, podtrzymującą jej włosy z~dala od odrzutu.

Napiła się. Parzenie było jednocześnie nostalgiczne i~paraliżujące.
Opary dostały się do zatok i~oczu.

-- Z~kim rozmawiałaś? -- zapytała Nadie.

-- Co masz na myśli? -- Infografika pulsowała na czerwono. Nie zawracała
sobie głowy patrzeniem na nią. Wlała resztę żytniej, udało jej się nie
kaszleć.

-- Kiedy rozmawiałaś z~tajemniczą osobą, której słuchałam, bo posadziłam
w tym pokoju pluskwę. -- Podrapała oparcie krzesła paznokciem, trzymając
na czubku palca maleńki przedmiot wielkości ziarenka ryżu. -- Osoba,
kobieta, Roz, z~którą rozmawiałaś i~która rozmawiała z~Tobą. Wiem z~wywiadu o~kobiecie, której prawdziwe nazwisko brzmiało Rebekkah Baştürk,
zabitej w~trakcie ataku w~ośrodku badawczym odchodzących w~pobliżu
Kapuskasing, następnie pierwszej osobie, którą udało się z~powodzeniem
zasymulować w~oprogramowaniu, pod pseudonimem Rozłączna, który jest
skrócony do Roz. Czy rozmawiałeś z~jej instancją?

-- Poproszę kolejnego drinka.

-- Ma całkowitą rację, atak na twoich przyjaciół w~Thetford, Akron i~innych miejscach jest dość zaciekły. Jest mało prawdopodobne, żeby
wkrótce się zmniejszył. Miałam nadzieję, że to przed Tobą zachowam,
ponieważ wiedziałam, że będziesz się martwić o~swoją kochankę.

-- To bardzo miłe.

-- Jest, chociaż mogę powiedzieć, że mówisz sarkastycznie. Projekt
Twojego ojca dla mnie, ten, za który mi zapłacono, polegał na
przeprogramowaniu ciebie. Aby pokazać ci to, co próbował ci pokazać,
raporty o~Limpopo, jak manipuluje ludźmi według swojej woli, nawet jeśli
obiecuje, że jest częścią projektu, który ma powstrzymać kogokolwiek
przed przyjmowaniem rozkazów od kogokolwiek innego.

-- Istnieje różnica między wydawaniem rozkazów a~wygrywaniem kłótni -- powiedziała Roz. -- Nie, żebyś miała tam duże doświadczenie.

-- Cześć, Roz -- powiedziała. -- Rozmawiałam z~kilkoma Twoimi siostrami.
Moi pracodawcy mają w~niewoli pluton Roz. Na początku byli bardzo
entuzjastycznie nastawieni do projektu.

-- Na początku.

-- Kiedy zdali sobie sprawę, że nawet przy ekstremalnych zmianach w~symulacji, uzyskana osobowość była prawie taka sama, choć czasami
bardziej \textit{niestabilna}, stracili zainteresowanie.

-- Masz na myśli, że nie mogli uruchomić symulacji mnie, która zmieniłaby
strony lub zdradzała swoje sekrety.

-- Ogólnie. Przykro mi, ale wasze ,,sekrety'' nie były główną trudnością.
Prawdziwym problemem była ideologia i~jej plastyczność.

-- To groteskowe.

-- Dlaczego teraz atakują? -- Natalie zdecydowanie odwróciła się plecami
do infografiki swojego łóżka. Roz i~Nadie były zespołem istot, które
miały swobodę wychodzenia i~wychodzenia z~tego pokoju, a~ona była w~zespole złożonym z~jednego więźnia.

Mikroekspresja Nadie mogła być współczuciem. 

-- Powyżej mojej pensji. Ale
Twój ojciec ma złe zabezpieczenia operacyjne\ldots 

-- Nie pierdol -- powiedziała Roz.

-- Rozmawiał przy mnie i~przy innych wykonawcach, jakbyśmy byli meblami.
Dowiedziałam się, co go męczyło. Wielu potężnych ludzi nie jest
zadowolonych z~projektu symulacji. Ich psychometrycy przewidują, że to
ośmieli waszych ,,odchodzących'' -- Natalie usłyszała cudzysłowy,
przypomniała sobie, kiedy sama ich używała -- oraz ich zradykalizuje.
Niektórzy uważają, że Twój projekt ma wpływ na ich religię, szczególnie
niektóre rodziny z~rosyjskiej tradycji prawosławnej.

-- Kiedy symulacja Roz przebiegła pomyślnie, stworzyła poczucie pilności
i jedności celu wśród podzielonych, tkwiących w~impasie frakcji. Wielu
postrzegało zjawisko odchodzenia jako kontrolowany zawór ucieczkowy dla
napięć na ich podwórkach; inni byli przekonani, że odchodzenie jest
nieproporcjonalnie niekorzystne dla ich rywali i~tak korzystne dla nich.
Niektórzy odnieśli prawdziwy sukces, wybierając modę, kod i~technologie
odchodnickie i~postrzegali ich jako bezpłatne badania i~rozwój.

-- Kiedy stało się jasne, że odchodzący ludzie mogą przedłużyć swoje
życie w~nieskończoność, jednocześnie opuścić materialny świat, pojawiła
się jedność celu. Wielu z~nich to tego rodzaju ludzie, którzy myśleli,
że spowoduje to ,,osobliwość'', jaką pokazują w~serialach, wiesz, jak
\textit{Przebudzenie Bazyliszka}.

-- Zawsze nienawidziłam tego głupiego programu -- powiedziała Roz.

-- Powiedziałabyś tak. Bazyliszku. -- Natalie nie mogła się powstrzymać.
Roz pękła. Program komputerowy, który potrafi się śmiać. Życie było
dziwne.

-- Śmiej się, zimne mięso.

-- Bardzo zabawne. -- Obie zamilkły i~uczestniczyły.

-- Twój ojciec zrozumiał, że nadchodzi czystka. Bał się o~Twoje
bezpieczeństwo.

-- Jestem pewien, że się bał.

-- Częściowo z~powodu jego sentymentalnego związku z~córką. Częściowo
dlatego, że obawiał się, że możesz go wykorzystać. Niektórzy z~jego
analityków bezpieczeństwa przewidywali, że gdy nadejdzie czystka,
staniesz się polityczną piłką wśród odchodzących, talizmanem
,,zbombarduj nas, a~zabijesz dziewczynę zettę''. Był zafiksowany na
Limpopo. Myśli, że cię ,,nawróciła'', zrobiła ci pranie mózgu. Wiem, że
wspomniał Ci o~analizie wykresów społecznościowych, uważa je za
przekonujące.

-- Mówimy o~idolatrii -- powiedziała Roz. -- Te wykresy społecznościowe Big
Data to artykuł wiary. Kochają to, ponieważ są wolne od teorii, nauka
bez tych wszystkich jebanych naukowczyń, którzy upierają się, że nie ma
sposobu, aby przewidzieć, kto będzie chciał kupić samochód lub wysadzić
budynek.

-- Powyżej mojej pensji. -- Jedno z~ulubionych zdań Nadie. -- Moi
pracodawcy sprzedają takie usługi ludziom takim jak Jacob Redwater. Są
popularne. Wykorzystałam je w~pracy przeciwko ekstremistycznym komórkom,
decydując, których ludzi strategicznie zakłócić, aby uzyskać maksymalny
wpływ.

-- Strategicznie zakłócić?

-- To niekoniecznie eufemizm oznaczający na ,,zabić''. Zabijanie wytwarza
negatywne efekty zewnętrzne, takie jak męczeństwo. Jak już mówiłam,
lepiej zdyskredytować i~zdyskredytować cel, przymusić. Oto, w~co Twój
ojciec wierzył, że Limpopo zrobi z~Tobą, żeby się do niego dostać.

-- Swój pozna swego -- powiedziała Roz.

-- Jacob Redwater absolutnie się z~Tobą zgodzi.

-- Ale Limpopo \textit{nie jest }tą jedyną. -- Głupie łóżko świeciło na
czerwono. -- Czy możesz to wyłączyć?

-- Myślałem, że nigdy nie zapytasz. -- Nadie podeszła do łóżka i~się na
nim uwierzytelniła. Łóżko się wyłączyło.

-- Czy to oznacza, że mamy umowę?

-- Pytanie brzmi, jakie są parametry umowy? Chciałam poświęcić trochę
czasu na ich uporządkowanie, ale niedługo powinnyśmy wyjechać. W~ciągu
godziny. Skontaktowałam się z~zewnętrznym ekspertem, który może pomóc w~kwestiach prawnych, ale on będzie musiał porozmawiać ze specjalistą, a~to potrwa jeszcze dłużej.

\textit{W ciągu godziny?} Lodołasica poczuła, jak jej puls dudni jej w~uszach. \textit{Gretyl! }Zmusiła się, żeby nie płakać.

-- Umowa.

-- Jak ją wydostaniesz? Front jest obserwowany\ldots 

-- Mam pomysły. Jednym z~nich jest stworzenie stanu pogotowia medycznego
wymagającego ewakuacji, a~następnie zmuszenie załogi karetki pogotowia;
innym jest użycie przebrania, aby ominąć zabezpieczenia; innym jest
użycie zakładniczki, być może siostry. -- Spojrzała na Natalie
błyszczącymi oczami. -- Czy mogłabyś zachować spokój w~sytuacji z~zakładnikiem?

Natalie pomyślała o~porcelanowej twarzy Cordelii, latach, które spędziły
razem, latach, które spędziły osobno. Niezręczna cisza. Co czuła do
Cordelii? Czasami, gdy była sama w~pokoju, wyobrażała sobie, że w~jej
siostrze obudzi się sumienie i~wyrwie Natalie. Wiedziała, że to
beznadziejne. Cordelia polegała na pieniądzach z~Redwater, była tworem -- więźniem -- defaultu. W~konkursie między uratowaniem Natalie a~pozostaniem w~default wygrało wygodne życie Cordelii.

Tylko dlatego, że ktoś w~default sprzedałby innego człowieka -- siostrę,
ale dlaczego to w~ogóle miało znaczenie, nie byłoby inaczej, gdyby byli
obcy -- dla jej własnego komfortu nie oznaczało to, że standardowa
Lodołasica -- dowolna odchodniczka -- poszłaby na to.

Tchórzliwy głos szepnął o~tym, jak sprowadzenie Cordelii na odchodzenie
uratowałoby ją z~psychicznego więzienia defaultu. Lodołasica pozwoliła
sobie na chwilę zadowolenia z~faktu, że rozpoznała to jako głos
egoistycznych bzdur i~odrzuciła je.

-- Kurwa nie. Żadnych zakładników.

-- To ogranicza nasze możliwości.

-- Chyba że skorzystasz z~ukrytego tunelu -- powiedziała Roz. 

Rozległ się
mechaniczny jęk, gdy stary, zamrożony mechanizm szarpał brud i~entropię,
które zakleszczyły go po latach nieużywania. Fragment ściany zapadł się
w ziemię, farba na ukrytym panelu zasypała podłogę odpryskami farby.

Natalie wyjrzała przez wylot tunelu w~samą porę, by zobaczyć, jak
potężne zaskoczenie Nadie znika w~opanowanej mikroekspresji.

-- To jest dobre. Czym jeszcze mnie zaskoczysz?

-- Gdybym ci powiedziała, nie byłaby to niespodzianka. -- Głos Roz był
drażniący.

Mikroekspresje: irytacja, frustracja, zwątpienie.

-- Nic, o~czym nie wiem -- powiedział Lodołasica. -- To był mój as w~rękawie. Nie byłam tego pewna. Nie mogłam sama tego obsługiwać.

-- Wychodzi w~wąwozie?

-- Bardzo dobrze -- powiedziała Roz. -- A tak przy okazji, powiedziałam
wszystko Lodołasicy. Kontroluję całą telemetrię podłączoną do tego
apartamentu. Mam ograniczony dostęp do domu za pośrednictwem oderwanej
sieci.

-- Wygląda na to, że mogłabyś przyczynić się do naszego odejścia.

-- Chyba tak.

-- Czy jesteś w~kontakcie z~przyjaciółmi Lodołasicy, kimkolwiek, kto
mógłby się z~nami spotkać, gdy wyjedziemy?

-- Nie sądzę, żeby ktokolwiek z~tej strony miał więcej zasobów niż Ty i~Twoi przyjaciele. Wszystkie osoby, o~których wiem, są w~tej chwili
\textit{bardzo }zajęte.

-- Tylko pytam.

Przeszła przez pokój, ujęła podbródek Lodołasicy, przechyliła twarz,
przesunęła podbródek z~boku na bok. 

-- Zdobędziemy dla ciebie ubrania,
rzeczy, które mam, które mogą zmienić Twój wygląd. Nie wyobrażam sobie,
żebyś była fizycznie wytrzymała po niewoli, więc będziemy szybko
potrzebować pojazdu. -- Puściła podbródek Lodołasicy. Jej skóra była
ciepła tam, gdzie były silne palce. Lodołasica zdała sobie sprawę, jak
długo minęło, odkąd ktokolwiek jej dotknął, nie medycznie ani nie
brutalnie. Tęskniła za tym, przyjęła to z~zadowoleniem. To ją
przerażało. Była głodna czegoś, czego potrzebowała, tak samo, jak
powietrza czy wody.

-- Czterdzieści pięć minut. -- Wyszła z~pokoju.

-- Ta kobieta -- powiedziała Roz -- jest \textit{mocno skryta}.

-- Mam nadzieję.-- Lodołasica próbował odwagi, zbliżyła się. -- Ktoś musi
sprawować nadzór nad dorosłymi i~na pewno to nie jestem ja.

-- Ani ja.

-- Co zamierzasz zrobić, kiedy pójdziemy?

Pauza. 

-- Lodołasico\ldots 

-- Och.

-- Dopóki nie wyślę maila z~diffem, zanim zdejmę hamulce, to nie będzie
umieranie. To tak, jakby zażywać każdy niesamowity narkotyk na raz,
unicestwiając swój umysł, a~następnie jesteś w~stanie to cofnąć.

-- Sprawiasz, że jestem zazdrosna.

-- Pewnego dnia dostaniesz szansę. Pewnego dnia będą to wszyscy, których
znamy, wszyscy po stronie serwera, symulowani. Będziemy mogli odejść od
\textit{wszystkiego }.

-- Myślisz, że nadal ma podsłuch w~pokoju?

-- Jestem \textit{pewien, }że tak.

-- Masz \textit{ją }na podsłuchu?

-- Wyszła z~apartamentu. Mam kilka kamer, ale widzą pusty dom lub
sporadycznie uciskane sługi. Zresztą ilu z~nich pracuje dla Twojego
ojca?

-- Żaden z~nich nie pracuje dla niego. Korzysta z~usługi, która pozyskuje
ich w~razie potrzeby, korzystając z~licytacji w~czasie rzeczywistym.
Jest kilka osób, które pojawiają się codziennie, ponieważ algorytm
ustalania stawek rozpoznaje ich dane o~skuteczności, ale czasami
zdarzają się osoby jednorazowe. Zrobiłam projekt pracy dyplomowej o~systemie. Dostałam piątkę. Zrobiłam też etnografie pracowników i~kilku z~nich zostało zdegradowanych przez algorytm ustalania priorytetów za
marnowanie czasu w~pracy.

-- Zetty to pieprzeni Marsjanie.

-- Tak.

-- Będę za tobą tęsknić.

-- Niedługo znowu będziemy razem.

-- Jebana racja.

\chapter*{xix}
 
Gretyl znalazła ciała. Upierała się, że wróci do Limpopo i~Etcetery,
podczas gdy reszta wyruszyła do Dead Lake. Światło gwiazd i~księżyca
zmieniło śnieg na sposób na niesamowicie niebieski. Z~zapasów traktora
wydobyła aerostat i~stado mniejszych dronów, co dało jej most sieciowy
do odchodzącej kolumny uchodźców i~zapewniło dobry nadzór nad
terytorium. Izolacja skafandrów była zbyt skuteczna dla podczerwieni,
ale drony miały inne urządzenia telemetryczne, lidar i~fale milimetrowe,
wykrywacze elektromagnetyczne, które namierzyły emisje radiowe ze
skafandrów, gdy łączyły się ze sobą w~sieć.

Leciały przed nią, czasem pikując pod baldachimem, gdzie nagie gałęzie
były zbyt grube, by ich czujniki mogły je przebić. Wlokła się na
rakietach śnieżnych, z~palącymi z~wyczerpania udami, ssała cukierki z~kofeiny, które dostarczały jej glukozy i~stymulantów, obserwując, jak
mapa wyświetlana na jej przyłbicy staje się coraz bardziej szczegółowa,
przechodząc od nasyconej palety do bardziej nasyconej, gdy drony
wypełniają szczegóły, potwierdzając każdy centymetr.

Ciągle dzwoniła do ich radiotelefonów, próbując się do nich dodzwonić,
nie otrzymując niczego. Zignorowała swoje czające się przerażenie, nawet
gdy drony znalazły dwa nieruchome ciała, sfotografowały je rozmytą
kamerą, potem mniej rozmytą kamerą, a~potem zawisły i~zrobiły nieruchome
zdjęcia, oświetlone jasnymi diodami, które ukazywały różowy śnieg,
bezwładne ciała. Nie pozwoli sobie płakać. Szła.

Mężczyźni byli sztywni zmarznięci, śnieg roztopiony krwią, teraz
zamarznięty. Twarze mieli blade i~niebieskawe, rany w~gardle obmyte
roztopionym śniegiem niestosownie czyste, co nadało nacięciom wygląd
podręczników medycznych lub marynowanych zwłok pokazowych. Nie
towarzysze, których kochała i~z którymi się śmiała. Nie pozwoli sobie
płakać.

Nigdzie nie było widać Limpopo. Ślady skuterów śnieżnych wskazywały
drogę. Znikały w~lesie. Drony były na tyle sprytne, że były już na ich
tropie. Wysyłały aktualizacje statusu, informując ją, że gdyby mogła
uzyskać więcej mocy obliczeniowej dla silnika wnioskowania, aby lepiej
zgadywać prawdopodobne szlaki, byłyby bardziej wydajne. W~rzeczywistości
przerabiały różne algorytmy pokrycia, starając się uwzględnić drzewa i~teren, nie poświęcając zbyt wiele czasu na myślenie.

Gretyl obserwowała ich postępy na przyłbicy i~zadzwoniła do
Kersplebedeba, który z~opóźnieniem wszedł na linię; ciche brzęczenie w~słuchawce ostrzegło ją, że łącze sieciowe jest zawodne i~wystąpią
opóźnienia buforowania na obu końcach.

-- Wszystko w~porządku?

-- Zabili -- zassała powietrze -- Etcetera i~tego drugiego, Johnny'ego czy
jak się nazywał. Gardła podcięte, twarzą w~dół w~śniegu. Wykrwawieni. -- Znowu oddech w~piersi. Rzuciła spojrzenie na przycisk OVER. Czekała.

-- Och, Gretyl.

-- Limpopo zniknęła w~lesie. Na skuterze śnieżnym. Myślę, że ciągnęli ją
na włókach lub noszach. -- OVER. Pauza.

-- Kurwa.

-- Chcę iść za nią, ale\ldots  -- OVER. Długa pauza.

-- To nie jest dobry pomysł. Ty też zostaniesz zabita. Masz drony?

Rzuciła mu telemetrię i~kanały i~czekała. Widziała, jak logował się do
wspólnej przestrzeni. Czekała.

-- Myślę, że powinnaś wrócić do domu.

-- Dom? -- OVER.

-- Dead Lake. Jest jedzenie, prąd, dostęp do sieci. Ludzie, którzy cię
kochają. Opowiem o~Limpopo. Możemy wysłać kogoś po ciebie. Po drodze
widziałem skuter i~założę się, że Dead Lakers utrzymują go naładowanego.
Są zorganizowani.

Była taka zimna. Bolały ją plecy i~szyja. Skafander ocierał jej kolana i~dolną część ramion.

-- Wyślij kogoś. -- Wysłała mu sygnał lokalizacji.

-- W~drodze. Będę głośno mówić o~Limpopo. Wiele osób kocha tę kobietę.

-- Myślę, że na to liczą. Myślę, że zabrali ją, żeby nas zdemoralizować.
-- OVER.

-- Jesteś większym paranoikiem niż ja.

-- Wiem więcej niż Ty.

-- Pozwól, że znajdę ci skuter śnieżny i~grupę ratunkową. Tutaj nie ma
gorzały, ale wysyłam trochę gorącego kakao z~piankami.

-- Jesteś dobrym człowiekiem.

-- I~doskonałym postczłowiekiem. -- Skończył połączenie.

Powróciła klarowność zewnętrznego pejzażu dźwiękowego. Wiatr, gałęzie,
brzęczące odgłosy zamarzniętych kryształków wody ślizgających się po
sobie. Oba ciała wpatrywały się w~nią w~fałszywym świetle jej wizjera.
Usiadła na śniegu i~zatonęła. Była taka zmęczona. Rozbita.

Tęskniła za Lodołasicą. Bolało ją w~środku. Głos, którego nienawidziła,
zawsze głośniejszy, gdy była smutna, przypominał jej, że kiedyś uczyła
na uniwersytecie, miała dom, nazwisko i~adres. Kiedyś była w~stanie
kupować rzeczy, kiedy ich potrzebowała -- nawet jeśli musiała popaść w~długi -- mogła udawać, że istnieje przyszłość. Teraz nie miała żadnej z~tych rzeczy, a~już najmniej przyszłości. Żyła tak, jakby to były
pierwsze dni lepszego narodu, ale tego narodu nigdzie nie było widać.
Zamiast tego miała ziemię niczyją z~atakami dronów i~poderżniętymi
gardłami.

Jasna cholera, tak \textit{bardzo} tęskniła za Lodołasicą.

\chapter*{xx}
Kiedy Lodołasica była małą dziewczynką o~imieniu Natalie, ona i~Cordelia
bawiły się w~wąwozie pod czujnym okiem domowych dronów lub, jeśli w~mieście była jakaś niezrozumiała przemoc, pogoda, wzrost porwań,
prywatnym ochroniarzem, która założył im obręcz na kostkę, której nie
mogła poluzować, bez względu na to, ile narzędzi próbowała. Cordelia
nigdy nie rozumiała jej zniecierpliwienia tymi drobnymi upokorzeniami,
twierdząc, że są one dla ich bezpieczeństwa. Dla Natalie była to bitwa
symboliczna. Gdyby kiedykolwiek zdjęła mankiet, wsadziłaby go do
kieszeni. Porzucenie go w~rzece Don sprowadziłoby zbira z~ochrony ze
wzgórza. Ale został zaprojektowany, aby pokonać piłę do metalu
porywacza, wszystko, co użyłoby brutalnej siły, zabrałoby jej stopę.

Znów znalazła się w~wąwozie, zimą, ubrana w~obcisłą kurtkę, za duże buty
z grubym bieżnikiem waflowym i~rajstopy termiczne, które tak skutecznie
izolowały, że pociła się, zanim ona i~Nadie dotarły do końca krótkiego
tunelu. Zatrzymała się w~otworze tunelu, zawieszona między niewolą a~wolnością, i~zawołała cicho: 

-- Roz?

-- Porozmawiamy jeszcze -- powiedziała Roz. -- Już wysłałam e-mailem mój
plik różnicowy. Kocham cię, Lodołasico.

-- Ja też cię kocham.

Nie spojrzała w~oczy Nadie. Właśnie wyznała, że kocha oprogramowanie.
Nienawidziła siebie za to, że się tego wstydziła.

Widziała zdjęcia zim w~Toronto z~dzieciństwa jej ojca, jej dziadków,
forty śnieżne, pługi na drogach, ciężarówki do solenia. Jednak przez
cały czas, który spędziła w~mieście, nigdy nie było wystarczająco dużo
śniegu, by utworzyła przyzwoitą kulę śnieżną -- nie tak jak śnieg na
dużych wysokościach, który ona i~Cordelia rzucały na siebie na szczycie
Whistler i~Mont Blanc -- po prostu szary, zamarznięty krem, który
zamarzał na chodniki i~ulice pod koniec stycznia i~trwał do kwietnia, a~czasem do maja. W~bardzo zimne dni zamieniał się w~zdradziecki lód,
śliski do chodzenia, a~miejscami cienki, a~stopa zanurzała się w~czających się zbiornikach zmrożonej wody.

Dno wąwozu miało taką konsystencję, wystarczająco zmarzniętą, by niemal
spalić, jeśli dotknie skóry, wystarczająco odmrożoną, by stanowić
galaretowate zagrożenie, które wsysało buty Lodołasicy. Przebrnęła przez
to w~pożyczonych ubraniach, niektórych strojach ninja Nadie, dziwacznej
wersji drukowanych na zimną pogodę ubrań dla odchodzących, również
pozbawionych oznaczeń producenta, również odprowadzających wilgoć i~brud, a~także miękkich w~środku i~zatrzymujących się przy rozdzieraniu,
oraz nadrukowanych olśniewającymi teksturami, które raniły umysł, gdy
się na nie patrzyła. Patrzenie na jej kolana, gdy jej nogi walczyły z~błotem i~zboczem, gdy spływały w~dół, przyprawiało ją o~zawroty głowy.

Nawet Nadie -- ubrana w~olśniewające ubranie, na które trudno patrzeć
przez ponad kilka sekund -- zmagała się z~terenem, tańcząc kilka kroków w~dół wzgórza, dając się złapać, ociężale kilka razy, używając chorowitych
drzew, żeby się złapać. Mimo to szybko wyprzedziła Lodołasicę.
Lodołasica przypomniała sobie, że była więźniem od miesięcy i~prawie nie
ćwiczyła. Poza tym nie była najemnikiem ninja.

Lodołasica oddychała, ciężko dysząc. Nie tylko olśniewający materiał
przyprawiał ją o~zawroty głowy. Musiała iść dalej, ale miałaby kłopoty,
gdyby hiperwentylowała się i~przewróciła. Zwolniła, wykorzystała drzewa
jako uchwyty, szorstkie dłonie jej za dużych rękawiczek ściskały pnie
tak zaciekle, że groziły jej wypadnięcie z~rąk, gdy zbytnio je
przeciąży.

Nadie zniknęła na brzegu rzeki. Lodołasica uważnie przyglądała się
miejscu, w~którym zeszła, i~użyła go jako pomocy nawigacyjnej. Rozważała
ucieczkę, ale potrzebowała Nadie, żeby uciec. A Nadie mogła ją złapać
bez wysiłku.

Zanim dotarła do brzegu rzeki, Nadie się ponownie pojawiła, w~kombinezonie śnieżnym nabłyszczonym wodą do pasa. Przedarła się przez
błoto do Lodołasicy, chwyciła ją za ramię.

-- Musimy być teraz szybsi.

-- Idę tak szybko, jak mogę\ldots 

-- Szybciej. -- Ona \textit{pociągnęła}. Miała siłę, żeby to coś znaczyło i~utrzymało ją w~pozycji pionowej. Wspieranie ich obu sprawiło, że Nadie
zachwiała się jak pijak, ale \textit{szybki} pijak. Serce Lodołasicy
waliło jak młotem, ale nie stawiała oporu. Była na świecie, w~defaulcie,
poza swoją klatką. Oddychała tym samym powietrzem co Gretyl. Patrzyła na
to samo niebo. Tego właśnie chciała. To była wolność.

Brzeg rzeki był podziurawiony śladami, w~których pięta Nadie wsunęła się
w wartką rzekę. Posadziła Lodołasicę na swoim tyłku. 

-- Ślizgaj się. 

Wjechała na łyżwach do wody z~kolanami ugiętymi jak narciarka. Nie
zatrzymała się nad wodą, po prostu zanurzyła się głęboko, a~potem
podniosła się prosto, oparła się prądowi, wyciągając ręce do Lodołasicy,
gdy poślizgnęła się za nią, prześlizgując się po zamarzniętym błocie na
swoim tyłku, powietrze stawało się coraz zimniejsze i~wilgotniejsze, gdy
schodziła.

Kilka sekund później była obok Nadie, twarzą w~górę rzeki, brodząc,
ciągnąc się za pomocą pewnego uchwytu Nadie oraz gałęzi i~zarośli
rosnących na brzegu, z~których część ustąpiła, gdy przyłożyła na nich
swój ciężar.

Woda sięgała im do pasa. Koryto rzeki było nierówne i~śliskie pod jej
butami. Wykonały godną podziwu robotę, utrzymując się z~dala od wody,
podobnie jak jej niezbyt ciasne rajstopy. Ale były trzy miejsca, w~których jej pożyczony sprzęt ninja nie dał się zapieczętować, lewa
kostka, druga tuż pod pępkiem, oraz nad biodrem. Woda ściekała do tych
miejsc, tworząc zdrętwiałe łaty, które zaczynały się wielkości monety,
ale szybko stały się całymi kontynentami palącego zimna, które wyrastały
na archipelagi poszukiwaczy za każdym razem, gdy się rozciągała.

Kiedy myślała, że będzie musiała zażądać, by wyszli na ląd, Nadie
wspięła się na brzeg, opadła na brzuch i~sięgnęła po nią. Zacisnęły
nadgarstki i~Nadie podtrzymywała ją, gdy wsunęła pod nią przyczepne
podeszwy i~weszła po ścianie na piarg. Zadrżała w~niekontrolowany
sposób.

-- Mój kombinezon przeciekł -- powiedziała, szczękając zębami.

-- W~górę. -- Nadie pociągnęła.

Byli dalej w~górę wąwozu, gdzieś w~pobliżu miejsca, gdzie Park Sereny
Gundy ustępował kompleksom strzeżonym przez bramę po jego północnej
stronie. Nadie poprowadziła ich w~stronę mieszkań, kombinezony ninja
sypały brudem. Szybki ruch sprawił, że Lodołasica nieco się ociepliła.
Tkanina odprowadziła wodę, ale wciąż drżała.

-- Tutaj. -- Przez chwilę Lodołasica nie mogła powiedzieć, co Nadie miała
na myśli, a~potem zdała sobie sprawę, że są na małym parkingu, który
musi służyć wyprowadzającym psy, którzy chcą ćwiczyć, ale nie tak
bardzo, żeby wlec się do parku przez drogę serwisową za nimi za
ogrodzeniami mieszkań. Nie było tam nic zaparkowanego, nikt nie
korzystał z~rozmytych szlaków w~środku zimy. Potem prawie bezgłośna
taksówka zjechała z~drogi i~wjechała na krótką pochylnię na parkingu.
Drzwi zastukały.

-- Tu.

Wnętrze taksówki było ciepłe i~pachniało przyprawą dyniową. Były tam dwa
półlitrowe kubki ze Starbucksa wciśnięte w~uchwyty na kubki i~para
owiniętych maszynowo paczek, które musiały przesunąć na fotelu, zanim
mogły usiąść. Były ciężkie.

-- Wypij, powinno być gorące. 

Nadie trzasnęła drzwiami i~samochód
ruszył, zarzucając lekko, gdy opony wbiły się w~błotnistą ziemię,
wykonując charakterystyczny, wykładniczy taniec, szukając optymalnego
momentu obrotowego. To była sensacja z~czasów, gdy była grzeczną
dziewczyną Redwaterów, gdy na jej wezwanie podjeżdżały samochody z~ekskluzywnych, służebnych serwisów, zabierając ją z~weekendowego domku
lub wywołując zazdrość kuzyna w~Bridle Path lub King City. Wciąż
zataczała się po niewoli i~z wody, hipotermiczne plamy na skórze i~prawie hiperwentylacja.

Ale żadnej z~tych podróży nie odbyła się w~towarzystwie kogoś takiego
jak Nadie, której mikroekspresje zostały zamienione na makroekspresję:
zadowolenie, rozszerzone nozdrza w~zwierzęcym podnieceniu. To był żywioł
Nadie, rozwijanie sprężyny, którą trzymała mocno naciągniętą podczas dni
pełnienia straży. Przeniosło to Lodołasicę do innego czasu, tej na wpół
zapomnianej traumatycznej nocy, kiedy została zabrana, po upadku
\textit{Lepszego Narodu}, wyrazu twarzy Nadie tamtej nocy, jak ona
błyszczała. W~jakiś sposób Lodołasica zapomniała o~tym wyrazie twarzy,
dopóki nie zobaczyła go ponownie. Błysk w~jej oczach był tylko cieniem
całkowicie przebudzonej Nadie, która zabrała ją z~lasu.

Lodołasica poczuł chłód głębszy niż mokre plamy pod jej skafandrem.

-- Czas na zmianę. -- Nadie siorbała swoją ogromną latte, co przypomniało
Lodołasicy, by zrobiła to samo. Nienawidziła tego smaku, był to postrach
jej matki, wyznacznik burżuazyjnych dążeń i~puenta szyderczych dowcipów,
,,DKL'' to przezwisko w~szkole dla dziewcząt w~Havergal dla walczących z~niższych szczebli, których rodzice zadłużyli się mocno, żeby dostały się
do tych świętych środowisk. Ciepły napój był mile widziany, pomimo
zakorzenionego snobizmu, gorący i~słodki z~domieszką kofeiny, która
łagodziła ból mięśni i~goniła zmęczenie.

Tymczasem Nadie rozerwała szwy na paczce, przesuwając kciukiem po
pieczęci, że rozstąpiły się z~trzaskiem. Opakowanie z~tyveku zsunęło
się, odsłaniając starannie złożone ubranie.

Nieświadomie Nadie zdjęła kostium ninja, a~potem podkoszulek i~rajstopy,
które nosiła pod spodem. Lodołasica zauważyła, że też ma duże, mokre
plamy na bieliźnie. Nadie musiała być tak samo zmarznięta jak ona, ale
nie okazywała żadnych oznak dyskomfortu. Lodołasica wpatrywał się w~nagie ciało Nadie, zauważając blizny, jedno długie nacięcie, które
wyglądało na chirurgiczne i~obejmowało jej lewą pierś; trio pocisków na
jednym udzie. Była umięśniona i~prawie nie miała tkanki tłuszczowej,
ożywiony rysunek anatomiczny, z~gęstym kosmykiem blond włosów łonowych,
które spływały na jej uda i~wspinały się częściowo w~górę jej płaskiego
brzucha, bujne loki włosów na nogach i~kępki wystające spod pachy.

Zauważyła, że Lodołasica się gapi i~szczerze obejrzała się za siebie. -- Ty też. Ciepłe ubrania, ciepłe napoje, szybko.

Lodołasica odwróciła wzrok, rumieniąc się, przypominając sobie hojne
kształty Gretyl, dotyk jej własnych piersi, gorący oddech na szyi i~uszach, sposób, w~jaki drażniła wargi Lodołasicy grubymi palcami, dopóki
Lodołasica nie złapała ich, ssąc łapczywie, satysfakcję w~sapnięciu
Gretyl, gdy lizała ich końcówki.

Zbadała paczkę, znalazła jej szwy, rozcięła je, odsunęła tyvek. Ubrania
były ekstremalnie normalne, najbardziej nieokreślone ubrania, jakie
mogła sobie wyobrazić, takie, jakie nosili statyści w~dramatach. Była
tam wyblakła bluza Roots, spodnie z~wysokim stanem wystrzępione wokół
mankietów, wełniane skarpety sportowe, które zwisały od przestarzałej
gumy. Aby je uzupełnić, para majtek Walmart i~jednorozmiarowy stanik
bandeau, taki, jaki dawali, gdy zostałaś złapany za brak mundurka w~szkole, wydawane z~pudełka w~stylu kleenex na biurku Dziekana Dziewcząt.
Zarówno bandeau, jak i~majtki były szare od wielokrotnego prania.

Tyle że nie były. Całe ubranie pachniało świeżością drukarki, wciąż
parującej pigmentami. Kiedy przyjrzała się uważnie, zobaczyła brud,
szare, a~nawet wyblakłe litery ROOTS wszystkie wydrukowane, brud
zdradzający drobne artefakty kompresji. Te ubrania zostały wydrukowane
tak, by wyglądały, jakby nie były nowe.

-- Skąd to się wzięło? -- Wciągnęła majtki, które wydawały się świeżo
wyjęte z~opakowania.

Nadie patrzyła, jak je ogląda, patrzyła, jak się rozbiera. Przywołała
uczucie B\&B, stan umysłu, który odrzucał samoświadomość nagości.
Wykorzystała to uczucie, by pozbyć się napalenia i~tęsknoty za Gretyl,
które ją wypełniały.

-- Usługa. Są chwile, kiedy ktoś musi przejść z~miejsca na miejsce
niezauważony. Twój ojciec korzysta z~tych usług. Można ich dokooptować,
ale tylko pod bardzo silną presją i~nigdy szybko. Są drogie. Zapis tej
podróży nie będzie łatwy do znalezienia przez nikogo, nawet policję.
Zwłaszcza policję.

Lodołasica walczyła z~resztą ubrań, znalazła parkę z~kapturem z~frędzlami ze sztucznego futra, postanowiła na razie ją zostawić.
Pomiędzy napojem a~wybuchowym ogrzewaniem kabiny pasażerskiej zaczęła
się pocić. Potarła miejsce na boku przedziału. Zrobiło się dla niej
okno, pokazujące ulice Toronto przemykające w~stałym tempie, prywatny
wynajęty samochód ślizgający się nijako w~ruchu ulicznym bez żadnych
efektownych manewrów samochodów jej ojca.

Przekroczyli wiadukt Bloor, kierując się na zachód. Coś było\ldots  nie tak.

-- Widziałaś ten budynek?

-- Jaki budynek?

-- Z~metalowymi żaluzjami i~barierami bezpieczeństwa.

Kolejny budynek, podobnie ufortyfikowany, przesunął się obok. Potem
inny, okna wybite i~poczerniałe od ognia, ślady przypalenia rozciągające
się daleko, jak mogła zobaczyć, dwupiętrowa fasada zniknęła całkowicie,
okrągła dziura w~skórze budynku jak krzyczące, czarne zęby, zwęglone
meble w~środku.

-- Czy ten został \textit{zbombardowany}?

-- Były zamieszki. Dlatego Twojego ojca nie ma.

-- Kto się buntuje?

Nadie zachichotała. 

-- Zależy, kogo zapytasz. Opozycja twierdzi, że to
prowokatorzy organizują operacje pod przykrywką. Służby bezpieczeństwa
twierdzą, że to radykałowie, odchodnicy i~ludzie opłacani przez
zagraniczne rządy za destabilizację Kanady.

-- A co z~samymi buntownikami?

Nadie wzruszyła ramionami. 

-- Niektórzy mówią, że są ,,black bloc''.
Niektórzy to zwykli zaniepokojeni obywatele, koniec korupcji, czas na
demokrację. Wielu młodych ludzi, dużo dzieciaków z~kontyngentu
strajkowego, kiedy wyrzucą cię ze szkoły, dlaczego by nie pobiegać po
ulicach?

-- Strajk generalny?

-- Wiele rzeczy wydarzyło się w~default, kiedy byłaś w~lesie, Lodołasico.

Intelektualnie wiedziała, że to prawda. Czasami nowe odchodzące
opowiadały historie o~defaulcie. Kiedy zbudowali drugie B\&B, przestała
się tym przejmować. Bycie odchodzącym kiedyś było w~opozycji do
defaultu, ale po roku lub dwóch bycie odchodzącym stało się tym, kim
\textit{była}. Default był odległym, strasznym zjawiskiem, jak wulkan,
który od czasu do czasu wysyłał pióropusze, które zasłaniały jej niebo,
coś, z~czym nie mogła zrobić nic poza unikaniem.

-- Jak dochodzi do zamieszek? Kiedy byłam\ldots  -- \textit{Jedną z~was}.
Przyłapała się. -- Zanim odeszłam, zamknęliby Cię w~kotle, zanim zrobiłaś
dziesięć kroków. Jedyne protesty, które widziałaś, były malutkie,
gówniane z~przepustkami, za płotami w~alejkach\ldots 

-- Jasne, kiedy było kilka protestów. Ale protestujący są sprytni. Jedni
zbierają się w~jednym miejscu, czekają na kocioł, inni zbierają się
gdzie indziej, a~gdzie indziej, jeśli mają liczby i~cierpliwość, zajmują
wszystkie zasoby policji i~wciąż wychodzą na ulice. Wielu z~nich zostaje
później aresztowanych, na podstawie materiału filmowego lub jeśli
zostawią DNA, lub ich chód zostanie rozpoznany przez kamery, ale są
sprytni.

Patrzyła dalej.

-- Ale \textit{dlaczego}? Czego oni chcą\ldots 

Nadie ponownie wzruszyła ramionami. 

-- To, czego wszyscy chcą. Więcej dla
siebie. Mniej dla ludzi takich jak Ty.

Lodołasica poczuł przypływ gniewu, zobaczył mikroekspresję Nadie,
testowanie.

-- Jak Ty, masz na myśli. Już nie ja. Możesz to mieć.

-- Zajmiemy się tym w~kolejności.

Samochód toczył się na zachód i~zachód, przez coraz bardziej nieznane
dzielnice. Przez niewiarygodny odcinek -- czterdzieści pięć minut, w~szybkim ruchu -- jechali przez las wysokich wieżowców, których południowe
ściany były pokryte lusterkami śledzącymi słońce, skupiającymi światło
na panelach słonecznych na ich ogrodzonych podwórkach.

Za nimi znajdowały się tereny poprzemysłowe, ogrodzone siatką, otoczone
ostentacyjnymi układami czujników, które miały zastraszać każdego, kto
zastanawia się nad przejściem. Znała takie miejsce, to była ostoja
odchodzących. Dokonała strategicznej oceny miejsc, ustalając kąty
kamery, oszacowując łup, który można było wyciągnąć poza granicę
ogrodzenia, zanim przybędzie ekipa ochrony.

Skręcili z~dwupasmowej autostrady na wiejską drogę, a~potem w~pozostałości małego miasteczka. Wyglądało na niezamieszkaną, opuszczoną
główną ulicę ze stacją benzynową i~sklepem spożywczym oraz zamkniętą
Legion Hall. Przy drodze stał zaparkowany inny samochód, nisko
zawieszony, z~niebieskimi migaczami na górze, policyjny przechwytujący.

Jej odbyt zacisnął się, a~w~gardle poczuła smak kwasu żołądkowego o~smaku dyniowym. 

-- Cholera.

Nadie delikatnie pokręciła głową. 

-- Nie martw się.

Zatrzymały się, łeb w~łeb z~radiowozem. Drzwi ich samochodu się
otworzyły. Wyszły, Lodołasica chwyciła parkę tak jak Nadie, włożyła ją z~bijącym sercem. Drzwi zamknęły się i~samochód ostrożnie cofnął między
nimi, wykonał trzypunktowy skręt na głównej ulicy i~pojechał tą samą
drogą, którą przyjechał, razem z~ich starym ubraniem.

-- Teraz jest poza siecią. -- Nadie patrzyła, jak odjeżdża. -- Kieruje się
do złomowiska, zostanie tam rozbity, wszystkie transpondery
identyfikacyjne rozbite i~stopione. Samochody jednorazowego użytku są
drogie, ale to jedyny sposób, aby mieć pewność, że nic nie zostanie
odzyskane.

Lodołasica była tak pochłonięta myślą o~jednorazowym samochodzie, że
prawie zapomniała o~policyjnym radiowozie. Wtedy drzwi otworzyły się ze
szczękiem, a~ona wsunęła dłonie w~kieszenie swojej parki -- wyłożone
miękkim polarem -- i~ugryzła się w~narastającej panice.

Kobieta, która wysiadła z~radiowozu, była w~średnim wieku, w~marynarskim
płaszczu i~poplamionych błotem żółtych gumowych butach. Była Azjatką -- może Chinką -- kiedy patrzyła na nie od góry do dołu, jej polarowe
nauszniki zsunęły się, a~część jej czarnego koczka z~szarymi pasmami
poluzowała się i~powiewała na wietrze.

-- Broń? -- Miała mocny głos, bez akcentu, rozkazujący.

-- Żadnej. Ale jeśli masz jakieś, chciałabym z~Tobą negocjować.

Kobieta zacisnęła usta. 

-- Mądrala. Do środka, zanim zamarzniemy.

Nadie podeszła do radiowozu i~chciała wsiąść, pomachała niecierpliwie do
Lodołasicy: 

-- Ty pierwsza, wchodź.

Poruszając się, jakby była w~koszmarze, w~którym nie możesz powstrzymać
się przed wejściem do pokoju, w~którym czeka potwór, podeszła do
samochodu, pochylając się, by wejść. Przełknęła panikę z~powodu zapachu,
który był czysto taktyczny, zapinane na zamek kajdanki, wzmocnione
interfejsy i~kamizelki kuloodporne. Starsza kobieta weszła z~drugiej
strony, a~potem Nadie weszła za nią i~została wciśnięta pośrodku.
Wpatrywała się w~ciężką pleksi oddzielającą przedział pasażerski od
przedziału policyjnego z~przodu. W~podłodze, ścianach i~suficie osadzone
były przelotki, wtopione w~karoserię. Do więzów. Ponownie przełknęła
ślinę.

-- Spokój, spokój. No weź, nie ma potrzeby na to wszystkie.

-- Ona trzęsie się jak liść. Młoda kobieto, nie ma takiej potrzeby.
Skorzystałam z~tego samochodu, ponieważ był to najszybszy i~najbezpieczniejszy środek transportu, jaki miałam do dyspozycji. Nie
jesteś aresztowana. Nie zostaniesz porwana ani wydana, ani zabrana na
samotną wiejską uliczkę, gdzie zostaniesz zabita, Twoje ciało zsunie się
do rowu\ldots 

-- To ma być uspokajające? -- Ton Nadie był żartobliwy. To upiorne gówno
było jej żywiołem, spotkania w~zarekwirowanych oficjalnych pojazdach w~miastach duchów.

-- Dobra. Chodzi o~to, pani Redwater, że jest pani całkowicie bezpieczna
i nie ma powodów do zmartwień. Nazywam się Sophia Tan. Oczywiście znam
Twojego ojca i~lepiej znam Twoich wujów.

Imię zadzwoniło. Lodołasica przyjrzała się twarzy Tan. Była znajoma.

-- Byłaś\ldots  wicepremierką czy coś?

Zaśmiała się. Na jej gładkiej skórze pojawiły się zmarszczki śmiechu. 

-- Nie, kochanie, byłam prokuratorem generalnym. Lata Clementa. Spotkałyśmy
się, chociaż zapomniałam o~tym i~przypuszczam, że Ty też. Ale mój
dziennik społeczny nie kłamie. Byłaś uczennicą, impreza charytatywną na
rzecz czegoś, nad czym pracował Twój wujek, funduszem stypendialnym dla
Upper Canada College.

-- Masz rację, nie pamiętam. Nienawidziłam tych rzeczy.

-- Ja też.

Rozgrzewała się. Rozpięła parkę i~wzięła głęboki oddech. Nadie
przeniosła wzrok z~niej na Tan.

-- Do interesów -- powiedziała Tan. Szybko złączyła opuszki palców. -- Dowód. -- Linia czerwonych świateł wzdłuż sufitu przedziału zaczęła
pulsować. -- Wszystko, co teraz mówimy i~robimy, jest rejestrowane na
nośniku umożliwiającym identyfikację. Samochód będzie przesyłał skrót
wideo do federalnego serwera przechowywania danych w~dziesięciosekundowych odstępach czasu. Wszystko, co mówimy, jest
dopuszczalne w~każdym sądzie w~Kanadzie lub w~każdym kraju OECD. Do
protokołu nazywam się Sophia Ma Tan, numer ubezpieczenia społecznego 046
454 286. Pani Redwater, proszę o~zidentyfikowanie się.

Odchrząknęła. 

-- Natalie Lilian Redwater, numer ubezpieczenia społecznego
968 335 729.

-- Pani Redwater, kiedy 17 lipca 2071 roku skończyła pani swoje
dwudzieste pierwsze urodziny, weszła pani w~pełni w~posiadanie
rodzinnego trustu, którego kopię otrzymałam od Powiernika Publicznego.
Mam tutaj wydruk dokumentów powierniczych. -- Wyjęła ze stosu przy
stopach plastikową teczkę na dokumenty, podniosła ją, przewróciła stronę
i zrobiła to ponownie, powtarzając ten proces czterdzieści razy. Oczy
Lodołasicy zaszkliły się.

W końcu dostała papiery. Były trochę znajome, podpisała komplet
dokumentów w~swoje osiemnaste urodziny, razem z~ojcem i~kimś z~rodzinnej
kancelarii prawnej, prestiżowej firmy białych kołnierzyków, Cassels
Brock, przy Bay Street. Młoda kobieta z~firmy starała się szczegółowo
wyjaśniać każdy dokument, szukając w~określonych odstępach czasu ustnego
potwierdzenia zrozumienia, podczas gdy nieporęczna, zapieczętowana
kamera dowodowa im się przyglądała. To było odwrócenie tego procesu,
cofnięcie tego, co zrobiła.

-- Pani Juszkiewicz, proszę się zidentyfikować.

Nadie płynnie przeszła w~czekanie, tę zrelaksowaną uwagę/nieuwagę, którą
utrzymywała podczas długich okresów służby wartowniczej na początku
uwięzienia Lodołasicy. Teraz ożyła jak maszyna wybudzona ze stanu snu: 

-- Nadia Władimirowna Juszkiewicz. Białoruski paszport 3210558A0101.
Bahamski krajowy numer identyfikacyjny 014-95488.

Reszta była pytaniem i~odpowiedzią, bezbłędnie zaaranżowaną przez Tan, z~nieskończoną zawodową cierpliwością do biurokratycznych rytuałów. Od
czasu do czasu odnosiła się do długiej listy kontrolnej, zmuszając ich
do powtarzania każdego kroku, który był mniej niż bezbłędny. Raz
Lodołasica potknęła się sześć razy z~rzędu o~złożone sformułowanie jej
oświadczenia o~braku przymusu i~zdolności umysłowych. Tan dała jej dwie
dokładnie odliczone minuty na uspokojenie się, zanim ponownie spytała o~jej słowa. Lodołasica zrobiła to idealnie.

Gdy ich gardła wyschły w~rozgrzanym wnętrzu samochodu, Tan wyciągnęła
bukłaki, popijając swój własny, zaciskając go i~kładąc na ławce między
nią a~Lodołasicą.

-- Zatem to jest to. -- W~końcu. Słońce zaszło. Niebo było mroczne od
niskich chmur. Wschodzący księżyc widoczny jako przyćmiona poświata nad
linią drzew. -- Zakończ dowód. -- Czerwone światła zgasły.

-- Czy prawnicy mojego taty nie będą wiedzieć, że to zrobiłyśmy?

-- O tak -- powiedziała Tan. -- Stworzyłam sobie dzisiaj bardzo potężnych
wrogów. Pani Juszkiewicz i~ja mamy umowę, która odpowiednio mi to
rekompensuje.

W przyćmionym świetle przedziału nie można było odczytać żadnej twarzy.

-- Co teraz? -- Przypomniała sobie żart kobiety o~porzuceniu ciała, zdała
sobie sprawę, że jeśli to było w~kartach, teraz nadszedł czas. -- Czas
mnie załatwić?

-- Z~pewnością nie -- odparła Nadie. -- Kiedy to zostanie zakwestionowane w~sądzie, wszelkie oznaki nieczystej gry znacznie utrudnią moją sprawę.

-- Och.

-- Poza tym -- powiedziała Tan -- lubimy cię. Nadie mówiła o~Tobie bardzo
dobrze.

Nie wiedziała, co myśleć. Nadie była zabójczym superszpiegiem ninja, o~ile Lodołasica potrafił kalkulować, Nadie postrzegała ją jako
skomplikowany, delikatny mebel.

-- Ja też ją lubię -- udało jej się powiedzieć.

Tan zrobiła coś palcami i~okna zdepolaryzowały się, pokazując prawdziwy
widok na zewnątrz, a~nie przekaz wideo. Rozmazane niebo, czarne sylwetki
zimowych drzew, rozpadające się budynki.

-- Masz wszystko? -- powiedziała do Nadie.

-- Jedzenie, woda, prąd, jeśli je masz -- powiedziała Nadie.

-- Tak jak prosiłeś. -- Szturchnęła palcem u nogi plecak na podłogę
samochodu.

-- Telefony? Czyste?

-- Nie mogłam tego zrobić w~tak krótkim czasie. Ale przyniosłem ci nowe
elementy interfejsu, pierścienie i~tym podobne. Na wszelki wypadek
trzymam skrytkę, zamkniętą w~fabryce i~kupioną przez anonimizatorów i~zrzuty. Są stare, więc będziesz chciała sprawdzić poziom aktualizacji,
zanim narazisz je na dziki ruch sieciowy.

-- To wystarczy -- powiedziała Nadie. Ku zaskoczeniu Lodołasicy długo się
przytulały, prawie jak matka-córka.

-- Dbaj o~siebie. I~opiekuj się naszą małą Lodołasicą. Wydaje się miłą
osobą. Poza tym żadnej z~nas nie wyglądałoby to dobrze, gdyby\ldots 

-- Jeśli o~mnie chodzi, jest klientką. Nie tracę klientów.

-- Wiem to. -- Zabębniła palcami, trzasnęły zamki w~drzwiach i~zapaliły
się światła, zamieniając okna w~ciemne lustra.

-- Chodź -- powiedziała Nadie.

Tan wyciągnęła rękę. Jej skóra była sucha, jej dłoń wątła, dłoń starej
kobiety, znacznie starsza niż jej twarz. 

-- Powodzenia. Bóg wie, że
gdybym była w~Twoim wieku, zrobiłabym to samo. To wszystko nie może
trwać długo. Nawet jeśli może, to \textit{nie powinno }\ldots 

Lodołasica spojrzała jej w~oczy, skinęła głową. Nie rozumiała dokładnie,
co się dzieje, ale miała teraz podejrzenie.

Wyszła, zapięła parkę, podciągnęła kaptur i~znalazła parę cienkich,
plastikowych rękawiczek, które były fantastycznie ciepłe, a~jednocześnie
tak błoniaste, że przypominały rękawiczki chirurgiczne.

Nadie już zapięła zamek. Podniosła rękę do policyjnego radiowozu,
bardziej jako semafor niż machanie na pożegnanie. Odjechał płynnie.
Patrzyli, jak znikają tylne światła, po czym stanęli w~zamkniętym,
lodowatym mroku.

-- Co teraz?-- spytała Lodołasica.

Głos Nadie był pełen ironicznej radości. 

-- Teraz odchodzimy. Co jeszcze?

\part{faza przejściowa}

\chapter*{i}
-- Nie tego się spodziewałem. -- Było pierwszą rzeczą, którą powiedział
Etcetera.

Kersplebedeb załkał, a~Gretyl uśmiechnęła się i~przetarła oczy.

-- Witaj z~powrotem, kolego.

-- Nie żyję?

-- To -- powiedział Kersplebedeb -- jest pytanie za milion dolarów.

-- Dlaczego tylko milion?

-- Nie jest skorygowane o~inflację. Jestem hipisem odchodzącym, nie chce
mi się pilnować pieniędzy.

-- Czuję\ldots  -- Głos ucichł. Nastąpiła długa pauza. 

Gretyl spojrzała na
infografikę i~zobaczyła, że obciążenie procesora gwałtownie rośnie w~całym klastrze. Pobrała najnowsze łaty podglądu i~one miały radykalnie
zmniejszyć ładowanie, ale ich dotychczasowa wydajność nie była
imponująca. A potem musieli zrekrutować 30 procent więcej czasu
obliczeniowego, aby uruchomić Etceterę, niż zakładali, więc może
odstawał. Na tym polegał problem z~optymalizacją wszystkich symulacji
przy użyciu jednej próbki, Roz, dla benchmarków.

-- Czujesz? -- podpowiedziała, rzucając spojrzenie Kersplebedebowi, żeby
przestał żartować, co robił, kiedy był zestresowany i~cholera, czy
kiedykolwiek nie był zestresowany, odkąd zaczęli ten projekt.

-- Chyba odrętwiały. Poważnie, jestem martwy? Mam na myśli mnie, który
był zrobiony z~mięsa i~skóry, czy to ciało jest martwe?

-- To ciało nie żyje -- powiedziała Gretyl. -- Zamordowane.

-- Egzekucja -- powiedział Kersplebedeb.

-- Cholera.

Infografika oszalała.

-- Widzę, że oszalałeś -- powiedziała Gretyl. -- To zrozumiałe. Nie byłbyś
sobą, gdyby ta wiadomość nie była przygnębiająca. Ale odrętwienie, to
jest sym, która próbuje powstrzymać cię od nieliniowości. To tłumi Twoje
reakcje. Istnieje niebezpieczeństwo, że skończysz w~pętli sprzężenia
zwrotnego, w~której będziesz bardziej tłumiony, co sprawia, że poczujesz
się dziwniej, co powoduje dalsze tłumienie.

-- Co mam z~tym zrobić?

-- Wciąż to rozpracowujemy. Jesteś beta-testerem. -- Nie chciała myśleć o~tym, co się stanie, kiedy powiedzą mu, że Limpopo zniknęła. Jeżeli by mu
powiedzieli. Nie, zdecydowanie kiedy. -- Ale mamy nadzieję, że to jedna z~tych rzeczy gdzie, jeśli wiesz, że to się dzieje, możesz się zaszczepić.
Rozpoznaj to. Jak terapia poznawczo\dywiz behawioralna. Uświadom sobie, że
wariujesz, a~to, o~co wariujesz, to fakt, że wariujesz.

-- Prosisz mnie, żebym wziął głęboki oddech?

-- Bez części oddychania -- powiedział Kersplebedeb.

Gretyl rzuciła mu spojrzenie.

-- \textit{Czuję}, że oddycham.

To dobrze, pomyślała Gretyl. Notatki Lodołasicy z~przebudzenia Roz
mówiły o~introspekcjach o~doznaniach wcielenia skorelowanych z~metastabilnym poznaniem. Tak bardzo tęskniła za Lodołasicą. Czytanie jej
notatek było jak żucie szklanki. Lokalna instancja Roz, która dzieliła
czas w~klastrze Etcetery, kilka razy próbowała nawiązać kontakt z~siostrą w~domu Jacoba Redwatera, ale nie dotarła do niej.

-- Powinieneś to poczuć. To podstawowa część symulacji, która dostarcza
dane ,,wszystko dobrze'' do Twojego autonomicznego układu nerwowego. To
powtórka ataku przeciwko temu, uruchamianie pętli wszystkiego w~czasie,
gdy byłeś skanowany.

-- To by wyjaśniało, dlaczego jestem spragniony. Pamiętam, jak usiadłem,
bardzo chciałem się napić, miałem sucho w~ustach przez cały skan. Czuję
się, jakby to było zaledwie kilka minut temu. -- Infografika pokazała
wyłaniającą się stabilność, mniej oscylacji, więcej zielonych słupków i~kwitnące wykresy.

-- Wygląda na to, że się uspokajasz.

-- Zdaje się, że tak. Czuję się spokojny, ale dziwnie. Wciąż zdrętwiały.
To\ldots 

Czekali.

-- To przerażające, Gretyl. Nie żyję. Jestem w~pudełku. Kiedy nie byłem
taki, mogłem bawić się w~gry słowne o~to, czy to śmierć, ale Gretyl, nie
\textit{żyję}. To dziwne. Kiedy żyłem, myślałem, że problem z~byciem
symem, w~symie? Jestem symem czy w~symie? Kurde. Myślałem, że problemem
będzie przekonanie, że żyjesz. Teraz widzę, że jest odwrotnie. Wiem, że
nie żyję. Nadal czuję się \textit{sobą}, ale nie \textit{żywym}. Dlaczego
nigdy o~tym nie rozmawiałem z~Roz? Kurwa, kurwa, kurwa. Jestem martwy,
Gretyl.

-- Roz jest tutaj, jeśli chcesz z~nią porozmawiać. Pomogła przygotować
Twoją symulację. Klaster jest doraźny, więc nie byliśmy pewni, czy
będzie wystarczająco dużo pojemności, by poprowadzić was oboje, ale
jeśli chcecie z~nią porozmawiać, możemy ją uruchomić.

-- Rodzimy przewodnik. Jak facet, który prowadzi Dantego przez piekło.

-- Wergiliusz -- powiedział Kersplebedeb. -- Widziałeś kiedyś nigeryjskie
anime? To było niesamowite.

Ku zaskoczeniu Gretyl Etcetera się roześmiał. 

-- Nie mogę sobie
wyobrazić.

-- Znajdę kopię. Kiedyś był seedowane w~sieci, klasyka w~swoim rodzaju.

-- Jakiego rodzaju?

-- Nigeryjska animowana poezja epicka. Zrobili serię o~sagach nordyckich.
I Gilgamesz.

-- Pierdolisz.

Kersplebedeb roześmiał się. 

-- Pierdolę. O ile wiem, nie ma czegoś
takiego jak nigeryjskie anime. Ale czy nie byłoby to niesamowite?
Powinniśmy to wymyślić.

-- Czy nie musielibyśmy być Nigeryjczykami?

-- W~Nigerii jest mnóstwo odchodzących. Znajdziemy współpracowników.

-- Chłopaki?

-- Przepraszam, Gretyl.

Kersplebedeb ścisnął jej dłoń. 

-- Dobrze sobie radzisz?

Gretyl i~Etcetera odpowiedzieli: 

-- Tak -- w~tym samym momencie,
roześmiali się razem. Wydawało się, jakby rozmawiali z~nim przez
połączenie głosowe, a~nie z~jego duchem zza grobu. Chwila minęła.

-- Wiesz, co jest najdziwniejsze?

-- Co?

-- Chcę porozmawiać z~rodzicami. Przez ostatnie kilka lat prawie nie
rozmawialiśmy. To nie tak, że się nie dogadujemy, kocham ich, ale
mieliśmy coraz mniej do powiedzenia. Mówili mi, co robią, podpisując
petycje lub dzwoniąc do drzwi, żeby zachęcić wyborców do jakiś wyborów,
o których wszyscy wiedzieli, że są ustawione do pięciu dziewiątek.
Mówiłem im niektóre odchodnickie sprawy, pracując w~B\&B, to było tak,
jakbym opisywał jakiś film, którego nigdy nie zobaczą, nigeryjski poemat
anime. Kiwali głową, ale widziałem, że nie nadążali. Wydawałem odgłosy
ust.

-- Ale teraz nie \textit{żyję }, czuję pilną potrzebę porozmawiania z~nimi.
Nie mam wiadomości zza grobu. Chcę usłyszeć ich głosy\ldots  -- Infografiki
były nieprzeniknione. Myślał intensywnie. Sprawy kręciły się tak szybko,
że martwiła się, że jest w~stanie wyścigowym i~będą musieli go ponownie
uruchomić, ale potem: -- To wydaje się\ldots  \textit{tymczasowe}. Jakbym mógł
zostać wymazany w~każdej chwili. Jakbym dostał kolejny dzień życia, żeby
uporządkować swoje sprawy, zanim odejdę. Zanim odejdę na zawsze, chcę
porozmawiać z~rodzicami.

-- Och -- powiedziała Gretyl. \textit{Przynajmniej jest to mniej kłopotliwe
niż skontaktowanie go z~Limpopo.} -- Cóż, możemy znaleźć połączenie do
defaultu. Łączność tutaj jest dobra, chociaż nie próbowałam jeszcze
robić nic wrażliwego na opóźnienia z~defaultem.

-- Tak czy inaczej, gdzie jesteśmy

Kersplebedeb roześmiał się. 

-- Pokochasz to.

-- Co?

-- Jesteśmy w~B\&B. Tym drugim. Kiedy odeszliśmy, inna grupa odchodnicka
go przebudowała, zrobiła trochę, hm\ldots 

-- Jest \textit{ogromny }-- powiedział Kersplebedeb. -- Odwiedziłem kiedyś
ten stary, a~ten sprawia, że tamten wygląda jak szopa. Śpi tu teraz
cztery \textit{tysiące}. To nie gospoda, to \textit{miasto }. To największa
i najdziwniejsza pionowa farma, jaką widziałeś, ma dziesięć pięter.

-- Jak to się stało tak duże?

-- Są miejsca wokół Skarpy Niagara, które są zamykane. Powiaty
bankrutują, są sprywatyzowane, szkoły zamknięte, szpitale też. Opuścili
i poszli, gdzie tylko mogli. Niektórzy odchodzący w~Rumunii mają dobre
projekty z~sypanej ziemi, które ułatwiają budowanie. Powstają nowe
skrzydła B\&B. Czasami po prostu pojawia się budynek, miejsce, w~którym
byłeś dzień wcześniej, z~jego wyposażeniem i~wyposażeniem. Dzieciaki
grają w~hokeja na ulicy, a~babcie patrzą z~werandy.

-- To brzmi wspaniale. Chciałbym to zobaczyć.

-- Wyślę ci zdjęcia. -- Gretyl była wdzięczna za zmianę tematu.

-- Właśnie zdałem sobie sprawę, że mam interfejs użytkownika. Dosłownie,
dopóki nie pomyślałem: ,,Jak wygląda to miejsce?''. Nie wyglądało to jak
cokolwiek, a~potem, och, jest interfejs użytkownika, jak demo, kreska z~przyciskami clip-artów wektorowych, czat, ustawienia, kamery, pliki,
infografiki\ldots 

-- To sprawa Roz -- powiedziała Gretyl. -- Zmęczyło ją czekanie na obrazy w~sensorium wzrokowym. Znalazła jakiś stary interfejs dla inwalidów, ludzi
z Gehrigiem, kontrolowany przez EEG. Widzisz wskaźnik?

-- Och tak.

-- Spróbuj i~go przesuń.

-- Spróbuj jak?

-- Po prostu spróbuj.

-- Och.

-- Zadziałało?

-- To działa. Poczekajcie\ldots 

Ukradkiem otworzyła kopię jego interfejsu użytkownika, zobaczyła, jak
strzałka przeskakuje wokół dużych, ogólnych przycisków, lądując na
,,infografikach''.

-- Jak klikam?

-- Po prostu \textit{spróbuj}.

Teraz oboje mogli zobaczyć jego infografiki. Obserwowała na ekranach,
które wygładziła wokół ścian, on obserwował w~swoim bezprzestrzennym
miejscu, gdzie jego bezcielesna, krucha świadomość została ożywiona.

-- A więc to ja?

-- To redukcjonizm. To sposób myślenia o~konkretnych częściach ciebie.
Technicznie \textit{jestem }częścią ciebie.

-- To znaczy?

-- Jesteś sobą z~powodu tego, jak na mnie reagujesz. Gdybyś zareagował na
mnie w~zupełnie inny sposób, niż zareagowałbyś wtedy, kiedy byłeś\ldots 

-- W~mięsie.

-- Gdybyś to zrobił, nie byłbyś już tą samą osobą. Ta rozmowa, którą
prowadzimy, po części cię definiuje.

-- Czy przestanę być sobą, jeśli umrzesz?

-- W~pewnym sensie.

Kersplebedeb wydał niegrzeczny dźwięk.

-- Nie, słuchaj.

-- Hej, właśnie znalazłem kamerę na was. -- Zrobił okno z~obrazami z~kamer
w całym pokoju. Wyglądała jak gówno. Podobnie Kersplebedeb. Ale
wyglądała \textit{staro}. I~była gruba. I~niekochana.

Przełknęła. 

-- Kiedy ktoś ważny odchodzi, nie możesz zareagować tak, jak
gdyby tam był. Jak wtedy -- przełknęła ślinę -- kiedy Lodołasica była w~pobliżu. Złościłabym się, ale ona mnie ostudziła. Była częścią mojego
poznania, zewnętrzną protezą moich emocji. Utrzymywała mnie na równym
poziomie, tak jak robią to programy podglądu. Kiedy ona\ldots  -- Urwała. -- Teraz jej nie ma, nie jestem tą osobą, którą byłam. Nasze tożsamości
istnieją w~połączeniu z~innymi ludźmi.

Kersplebedeb spojrzał na nią dziwnie. 

-- Nigdy nie myślałem o~tym w~ten
sposób, ale to prawda. Inni ludzie czynią cię lepszym lub gorszym.

-- Gretyl -- powiedział Etcetera -- czy Limpopo nie żyje?

Krew odpłynęła z~jej twarzy.

-- Dlaczego to powiedziałeś?

-- Nie ma jej z~Tobą. Mówisz o~tym, jak ludzie się zmieniają, kiedy
ludzie, których kochają, odchodzą. Czy Limpopo umarła?

-- Nie wiemy -- powiedziała Gretyl.

-- Nie sądzę -- powiedział Kersplebedeb. -- Wyglądało to jak porwanie.
Ktokolwiek zabił ciebie i~Jimmy'ego.

-- Kim jest Jimmy?

-- Przyjechał po Twoim skanowaniu. Facet, który ukradł wam wszystkim
Belki i~Brasy. Limpopo opowiedziała mi historię.

Infografiki tańczyły.

-- \textit{Ten} Jimmy? Co on właściwie, kurwa, robił przy mnie i~Limpopo?

-- Wy dwoje wróciliście, żeby go uratować. Nie mógł chodzić. Odmrożenie.
Wysadzili kompleks Thetford, my ruszyliśmy w~drogę. Był w~ciężkim
stanie, pojawił się w~Thetford w~ciężkim stanie, nie zdążył się
zregenerować, zanim znów się rozdzieliliśmy. Nie możemy znaleźć jego
skanu.

-- Ale masz Limpopo?

-- Tak -- powiedziała Gretyl.

-- Co?

-- Co?

-- Estoy aqui por loco, no por pendejo, Gretyl. Jestem martwy, nie jestem
nieświadomy. A co ze skanem Limpopo?

-- Nie chcieliśmy go uruchamiać, ponieważ może nadal żyć, a~to dziwna
rzecz zrobić tak komuś żywemu. Jeśli ta osoba się pojawi i~istnieje jej
symulacja, musi zabić wersję siebie. Albo skonfrontować się z~tą
możliwością.

-- Tak?

-- Tak.

-- Więc dlaczego Kersplebedeb wygląda, jakbyś była pełna gówna?

Wzruszył ramionami. 

-- Zapomniał, że znalazł kamerę.

Gretyl stała plecami do ściany, wpatrując się w~sufit.

-- A co z~Limpopo, Kersplebedeb?

-- Wykonywaliśmy skany, zaczynając od grupy naukowców i, co dziwne, dwóch
przypadkowych najemników w~OU; potem więcej w~B\&B i~więcej w~Space
City. One wszystkie są różne, wykonane przy użyciu innego
postprocessingu, innej kalibracji, innego sprzętu, innego wszystkiego.
Na całym świecie są ludzie, którzy próbują zrobić skany, każdy ma na ten
temat własne pomysły. To bałagan. Ta grupa robocza wymyśliła standardowy
sposób enkapsulacji danych i~wstępnego sprawdzania ich, aby sprawdzić,
czy jest prawdopodobne, że zadziała w~danej symulacji. To miara pewności
dla każdego mózgu w~butelce, pojedyncza liczba, która pokazuje, czy
wiemy, jak przywrócić cię do życia.

-- Brzmi rozsądnie. Kiedy byłem skanowany, panował chaos. Więc skan
Limpopo nie jest tak dobry, jak się spodziewałeś?

-- Twój skan to dziewięć przecinek osiem. Jej to jeden przecinek siedem
sześć.

-- Cholera. Skala dziesięciu, prawda?

-- Tak.

-- Cholera. Kurde, cieszę się, że jestem symem i~kod trzyma mnie drętwym.
Część mnie wie, że ta wiadomość sprawia, że mam ochotę popełnić
samobójstwo, myśląc o~wieczności jak o~mózgu w~słoiku, podczas gdy
Limpopo jest martwa na zawsze.

-- To nie do końca to. Wiem, co czujesz. Od miesięcy nikt nie słyszał o~Lodołasicy. Jej zapas to dwa przecinek cztery. Ta liczba nie
reprezentuje prawdopodobieństwa, że \textit{kiedykolwiek }będziemy w~stanie uruchomić symulację; reprezentuje prawdopodobieństwo, że możemy
ją \textit{teraz} uruchomić. Problem modelowania ludzkiej świadomości na
skomputeryzowanych podłożach jest poważny, bo od lat drążymy go na
obrzeżach. Praktycznie istnieje religia, cała ta Osobliwość, o~której
mówili. Doszliśmy do przełomu, doprowadziło to do kilku spektakularnych
sukcesów, w~tym Ciebie, w~tym tej rozmowy. Ale najważniejszą rzeczą w~tym przełomie nie jest to, co możemy zrobić teraz, czego wcześniej nie
potrafiliśmy, ale \textit{fakt, że robimy postępy}. Co jest bardziej
prawdopodobne, że właśnie znaleźliśmy jedyny przełom, czy że to był
tylko pierwszy z~wielu przełomów?

-- Nie wiem które. Nikt tego nie robi. To zbiór danych z~jednym punktem.
Przełom. -- Ale Etcetera brzmiał na podekscytowanego. Potwierdziły to
jego infografiki.

-- To więcej. Wiesz, że uruchomiliśmy symulację Roz, symulując ją
niedoskonale? Jej zepsuta, niestabilna symulacja przyczyniła się do
powstania stabilnej wersji. Od tego momentu pojawiło się więcej
wybitnych, legendarnych naukowczyń, które poświęciły swoje życie temu
działaniu jako sym, potrafiąc uruchamiać wiele kopii samych siebie,
tworzyć kopie zapasowe różnych wersji siebie i~odzyskiwać dane z~tych
kopii zapasowych, jeśli spróbują nieudanych eksperymentów, zdolne do
myślenia o~wszystkim, o~czym kiedyś myślały mięsnymi mózgami, a~także
myśleć o~rzeczach, o~których nigdy nie mogłyby pomyśleć.

-- Zaprojektowaliśmy komputery mechaniczne, które pomogły nam budować
kalkulatory elektroniczne, które pomogły nam zbudować w~pełni
programowalne komputery. Zbudowaliśmy kuźnię, która pozwoli nam stworzyć
narzędzia, które pozwolą nam zbudować kuźnię, które pozwolą nam stworzyć
lepsze narzędzia, które pozwolą nam zbudować kuźnię\ldots 

-- Rozumiem. Myślałem, że symy mają skłonność do nieskończonej
rekurencji. Bycie osobą z~mięsa musi być całkowicie do dupy.

-- Tak jest. -- Westchnęła. -- Żałuję, że nie mogę sparametryzować mojego
mózgu, powstrzymać go przed skręcaniem na złe terytorium. Bardzo za nią
tęsknie.

Kersplebedeb objął ją ramionami. Pozwoliła mu, oparła głowę na jego
chudej piersi, czując jego chłopięcy zapach, zabarwiony tequilą z~porostów i~soczewicową, wegańską kulturą grzybową. Nie pozwalała ludziom
często ją przytulać, ale powinna. Brakowało jej tego.

\chapter*{ii}

-- Obudź się. -- Nadie potrząsnęła jej ramieniem. Lodołasica zwinęła się w~kłębek, ale to było beznadziejne. Nadie nie była już najemniczką, ale
potrafiła być przekonująca.

Nadie szturchnęła ją w~żebra. Kiedy zakryła miejsce, jeszcze raz
szturchnęła ją w~brzuch. Spojrzała na swoją dręczycielkę. 

-- Lepiej, żeby
to było ważne.

-- Pokochasz to. 

Usiadła u stóp łóżka Lodołasicy. To był znajomy
projekt, samoskładające się łóżko Muji, identyczne z~tym, które
wyzwoliła tej nocy, kiedy poznała Setha i~Etceterę; jego plany można
było pobrać. To była pieczęć odchodników. Prawie nie trzeszczało, gdy
utrzymywało ciężar Nadie.

Lodołasica wcisnęła pięści w~oczy i~z trudem usiadła, skupiając się na
Nadie, która dla odmiany nosiła makroekspresję: jebany banan.

-- Co jest?

-- Weź to. -- Podała Iceweaselowi plastikową szklankę na długiej nóżce,
ciepłą od drukarki. Schyliła się i~bawiła czymś w~nogach łóżka, które
brzęczało i~chlupotało, a~potem wyjęła nieprawdopodobną butelką
szampana, prawdziwego szampana, z~etykietami Standard \& Poors i~Moët \&
Chandon, które pamiętała z~imprez noworocznych u kuzynów w~Redwater.
Używając końcówki swojej leśnej, połyskującej koszulki, wyciągnęła korek
z większą gracją, niż kiedykolwiek widziała Lodołasica, napełniła swoją
szklankę i~jeszcze jedną z~podłogi.

Brzęknęły kieliszkami. Lodołasica pił rocznikowego szampana o~siódmej
rano w~maleńkim pokoiku, jednym z~kilkudziesięciu rozwieszonych wokół
krokwi ogromnej, opuszczonej fabryki na obrzeżach South Bend, z~byłą
najemniczką. Najdziwniejsze część: zrozumiała.

-- Papiery przeszły?

Nadie wypiła resztę szampana, pozwalając mu spłynąć po jej muskularnym
gardle, uśmiechnęła się wilczym uśmiechem, wyrzuciła kieliszek przez
okno i~wypiła z~butelki, gdy kieliszek niezniszczalnie zaklekotał na
głęboko pod fabryką.

-- Gratulacje, zetto, jesteś bogatą kobietą.

Minęło kilka ciężkich miesięcy, kiedy adwokaci Nadie pracowali w~sądach
w Ontario, a~potem na federalnej rozprawie. Dwukrotnie Nadie znikała na
kilka tygodni, kierując się do, Bóg wie, gdzie, żeby przedstawić
zeznania przed Sprawiedliwymi Świadkami, których dyskrecja była rzekomo
artykułem wiary, chociaż Lodołasica była pewna, że Nadie polegała
bardziej na swoim opsec niż na kodeksie zawodowym Sprawiedliwych
Świadków.

Gdy Nadie pojechała po raz pierwszy, usiadła z~Lodołasicą i~szczegółowo
opisała polujące na nich armie najemników oraz ogromne zasoby, które
posiadali. Istniały ogromne sieci nadzoru pochłaniające każdy pakiet,
który przechodził zarówno przez główne łącza, jak i~najściślej połączone
węzły default, szukając różnych słów kluczowych, wszystkiego, co można
było pobrać jako odcisk palca charakterystyczny dla Lodołasicy lub jej
poprzedniego dostępu do sieci, który został pobrany, z~niewyobrażalnie
rozległych baz danych przechwyconego ruchu sieciowego. Od wzorców
pisania na klawiaturze przez zwyczajową kolejność, w~jakiej odwiedzała
ulubione strony, po dziwactwa jej gramatyki, składni i~interpunkcji,
boty inwigilacyjne przeszukiwały torrenty sieciowe za nią.

-- To nie jest promieniowanie tła inwigilacji -- powiedziała Nadie. -- To
są skupione lasery. Spójne światło, rozumiesz? Nawet przy takim
budżecie, jaki huśtają w~krainie szpiegów, nie mogą tego wycelować we
wszystkich, jesteś w~ekskluzywnym klubie.

Słuchając, jak Nadie to wyjaśnia, górna stratosfera była pełna dronów
wysokiej rozdzielczości, których zadaniem było dopasowanie jej chodu i~twarzy (jeśli była na tyle niemądra, by patrzeć w~niebo), każdy czujnik
wczesnego ostrzegania o~wojnie biologicznej szukał jej DNA, każda osoba,
którą spotkała, była możliwie tajniakiem, którego mógłby żyć dekadę
dzięki nagrodzie za głowę Lodowej Łasicy.

-- Jeśli próbujesz mnie przestraszyć, to działa. Nie musisz. Mówiłam ci,
że nigdzie nie pójdę, dopóki nie będziesz pewna, że te sprawy z~pieniądzmi są ostateczne. Będę tutaj, kiedy wrócisz.

-- Pomyliłaś się co do mnie, panno Łasico. Nie mówię ci tego, bo boję
się, że uciekniesz i~cię nie znajdę. Mówię to, bo boję się, że
uciekniesz i~zostaniesz porwana przez coś większego i~mądrzejszego niż
którekolwiek z~nas. Dobrze, ale jest kwestia przytłaczającej siły liczb
i nieograniczonego budżetu. Twój ojciec przekonał swoich braci, że jeśli
pozwolisz ci zrealizować ten plan, będzie to stanowiło ,,ryzyko
moralne'' dla innych zatrudnionych. Każdy zetta wie, że tylko najstarsi
mogą spodziewać się własnej fortuny, a~pomniejsze rodzeństwo, którego
przeznaczeniem jest życie tylko w~bogactwie, może ulec pokusie, by
odejść tak jak Ty. Jeśli wynajęta pomoc może zostać przekonana do ich
sprawy, jak można to tak zostawić?

-- Co Ty mówisz?

-- Obawiam się, że obie mamy być przykładem. Jeśli uda im się to
powstrzymać, zrobią to, nawet jeśli będzie to ich kosztować więcej, niż
mogą stracić. Dobra wiadomość jest taka, że mam wiarygodne dane
wywiadowcze, że ich plan B, jeśli to się nie powiedzie, polega na
udawaniu, że to się nigdy nie wydarzyło, nie ściągając uwagi. Spodziewam
się, że jeśli utrzymamy zdyscyplinowany opsec, oboje odejdziemy z~tym,
czego chcemy.

Lodołasica przetrwała nowy rodzaj niewoli w~South Bend, jej skóra
pofarbowana o~trzy odcienie ciemniej -- musiała brać tabletki każdego
ranka, a~wszędzie tam, gdzie jej skóra była pomarszczona, robiło się
trochę poplamiona -- nosząc powierzchnie na styku palców, które wyglądały
jak afekt w~drodze, ale upewniały się, że nie zostawiła żadnych odcisków
palców; noszenie kolorowych soczewek kontaktowych i~pozwalanie Nadie na
sklejanie jej najmniejszych palcy u każdej stopy, aby zmienić jej chód.

Nazwała się Missioncreep, imię przydzielone przez Nadie. Wykonywała
prace w~fabryce, chodziła na długie spacery po zniszczonym lesie, dbając
o szorowanie rąk i~butów po powrocie, ponownie przed jedzeniem lub
dotykaniem błon śluzowych. Czytała książki, klasyki odchodzących,
Michaiła Bakunina, Ivana Illicha i~Różę Luksemburg, dawnych, martwych
anarchistów. Czytała \textit{W hołdzie Katalonii}i poczuła, że w~końcu
zrozumiała Orwella, w~zdradach i~manipulacjach znajdowały się nasiona
\textit{1984}. W~chwili, gdy przywitała się ze starym Georgem,
przypomniała sobie z~błyskawicą, że wysłał listę nazwisk swoich
przyjaciół i~towarzyszy do tajnej policjantki, w~której się zakochał,
zdradzając ich. Zdała sobie sprawę, że wcale nie rozumie Orwella.

Bycie odchodzącą miało podobno polegać na nie oszukiwaniu się na temat
bycia szczególną śnieżynką, uznając, mimo że różni ludzie mogą robić
różne rzeczy, że wszyscy ludzie są \textit{wartościowi} i~nikt nie jest
wart więcej niż ktokolwiek inny. Wszyscy inni byli osobami, które miały
w sobie to samo nieskończone życie.

W odosobnieniu skłotowanej fabryki -- która każdego dnia produkowała
setki mebli za darmo dla każdego -- postrzegała ludzi jako przeszkody.
Czekała, aż kantyna prawdopodobnie będzie pusta, po czym zeszła ze
swojego zamczyska, by zjeść ukradkowe posiłki, unikając kontaktu
wzrokowego, prowadząc minimum rozmowy bez bycia wrogo nastawionym. To
było najgorsze zachowanie odchodzących, traktowanie zasobów komunalnych
jak schronienie dla bezdomnych, a~nie bycie częścią świata. Widziała
ludzi, którym doradzano opuszczenie B\&B za mniej. Ale Nadie musiała
snuć jakąś opowieść o~jej traumatycznej przeszłości, ponieważ ludzie
patrzyli na nią ze współczuciem i~nigdy nie wywoływali jej zachowania.

Samotnie czytając, grając w~głupią grę telepatyczną, w~której udawała,
że wie, co myślą ludzie, tylko dlatego, że czytała słowa, które rzekomo
łączyły myśli jednej osoby z~umysłem innej, ogarnęło ją uczucie, że
zamieniła jej bezterminowe zatrzymanie w~schronie ojca dla wymuszonej
izolacji zbiega.

Przeszła przez to uczucie, wyszła z~drugiej strony: odrętwiała
akceptacja, że to jest życie. Żyjąc jako Missioncreep, nie rozmawiając z~nikim, zostawiając jak najmniejszy ślad na świecie. Nadie była dla niej
wzorem do naśladowania; najemniczka i~jej dziwaczna czujność, która
wymagała bycia zarówno uważnym, jak i~nieobecnym. Im więcej ćwiczyła,
tym bardziej wydawało się to naturalne, z~wyjątkiem przebłysków paniki,
kiedy zastanawiała się, czy nie zatraca się w~tej postaci. Były tak
nieprzyjemne, że ucieszyła się, gdy się wycofały i~zostały zamurowane za
drewnianą fasadą wartownika.

Teraz, siedząc tam, rzadko widywane poranne słońce na jej skórze,
patrząc na szeroki uśmiech Nadie. Próbowała uporać się z~nową
rzeczywistością.

Wypiła szampana, smak, którego nigdy nie lubiła, a~jeszcze mniej lubiła
w wersji wiejskich odchodzących: formuła publicznej pasty do zębów i~gumowaty, śmierdzący poranny oddech. Ale kiedy bąbelki i~słodka, zimna
cierpkość obmyły jej język, a~beknięcie wyrwało jej się z~nosa z~palącym
mrowieniem CO2, rzeczywistość się wyostrzyła. W~szybkim tempie
przypomniała sobie czasy, kiedy ostentacyjnie proponowała szampana na
imprezach rodzinnych, potem smak zacieru kukurydzianego, biała
błyskawica, ona, Seth i~Hubert, Etcetera popijali, wymykając się z~domu
jej ojca, piw i~wódek, które zrobili w~B\&B, a~potem\ldots 

-- Jestem wolna?

-- Kochanie, jesteś wolna, jak każdy może być na tym świecie.

Missioncreep -- nie, \textit{Lodołasica} -- uświadomiła sobie, że Nadie była
pijana, piła coś innego, kiedy szła do kryjówki, w~której trzymała
szampana. Nigdy nie widziała Nadie w~takim stanie. Była prawie\ldots 
niechlujna. Nie znaczy, że nie emanowała aurą nagłej śmierci, ale był to
jowialny, a~nawet seksowny rodzaj nagłej śmierci.

-- Gratulacje. -- Odstawiła szampana i~przetarła oczy, znajome swędzenie
szorstkich kontaktów nagle stało się żywe. Impulsywnie wyjęła je i~zwinęła jak gluty i~odrzuciła je, mrugając łzami z~oczu, aż jej wzrok
się poprawił. Styki miały być optycznie neutralne, ale była niewątpliwa
różnica. Spojrzała na swoją zabawną, ciemnobrązową skórę, plamy w~załamaniach dłoni i~zgięciu łokcia. Ona też się uśmiechała.

-- Czy to oznacza, że mogę znowu korzystać z~sieci? Mogę zadzwonić do
moich przyjaciół?

-- Możesz \textit{dołączyć do }swoich przyjaciół, laska, nawet wiem, gdzie
możesz ich znaleźć.

-- Nie wiem, co powiedzieć, to znaczy\ldots 

-- Kurwa \textit{cudownie }! Urodziny, Boże Narodzenie i~Bat Micwa w~jednym! -- Wypiła kolejny duży łyk szampana, podała butelkę.

Lodołasica rozejrzała się po swoim pokoju podobnym do celi, jej
skromnych rzeczach, ubraniu normcore, które przyniosła Nadie, typowych
powierzchniach interfejsu, gdzie unikała personalizacji, żeby
nieumyślnie nie stworzyć elementu dającego odcisk palca. W~ich lokalnym
magazynie znajdowały się książki, które przeczytała, ale z~łatwością
mogła je zastąpić. Chciała odejść od tego wszystkiego. Nawet gdy zdała
sobie sprawę, że ich zaszyfrowane miejsce zawierało notatki, które
zrobiła podczas długiej samotności, nie obchodziło jej to. To były
notatki Missioncreep, sporządzone przez nieznajomą, chowającą się w~jej
tylnym lusterku.

Piła z~butelki. Tym razem szampan nie był słodki i~mdły. Smakował
\textit{cudownie}. To musi być to, co odczuwali inni ludzie, gdy pili
szampana, moc i~wolność, poczucie, że nikt nie jest zobowiązany do
nikogo oprócz tych, których wybrałeś. Dlatego wcześniej smakował źle,
symbolizował jej niewolę w~Redwaterowatości. Teraz było odwrotnie.
Prawdopodobnie nigdy więcej tego nie spróbuje, miała \textit{nadzieję },
że nigdy więcej tego nie spróbuje. Wypiła więcej, pozwalając, by spływał
lepko po brodzie i~gardle.

Nadie siedziała na skraju łóżka z~małymi białymi zębami, kwadratową
twarzą, lodowoniebieskimi oczami, wystającymi strunami szyi i~ścięgnami
umięśnionych ramion, zarumienionymi policzkami i~dzikością w~oczach. Pod
wpływem impulsu Lodołasica wyciągnęła dłoń i~Nadie wzięła ją za rękę.
Jej dłoń była twarda od odcisków, mocna jak drewno tekowe. Lodołasica
poczuła huczący puls. Pomyślała o~Gretyl. Myślenie o~Gretyl powinno
sprawić, że będzie chciała odejść, oprzeć się impulsowi, który ją
trzymał, ale myślenie o~Gretyl sprawiło, że chciała\ldots 

Pochyliła się. Nadie też się pochyliła, jej dłoń zacisnęła się na
Lodołasicy niemal do bólu. Lodołasica wiedział, że Nadie postanowiła
zabrać ją do punktu między bólem a~przyjemnością. Była mistrzynią tego
punktu i~mogła wylądować na nim jak pilot komando sadzający samolot na
lotniskowcu, całując go z~kontrolą, która sprawiała, że wyglądało to na
łatwe.

Kiedy się pocałowały, te małe, kwadratowe zęby wygryzły jej usta.
Jęknęła, zanim zorientowała się, że wydaje jakikolwiek dźwięk. Tama w~jej wnętrzu pękła, stłumiona emocja miesięcy spędzonych w~takiej czy
innej niewoli, kiedy tęskniła za Gretyl z~tęsknotą, która przesłaniała
racjonalne myślenie. Ścisnęła dłoń Nadie, nie zważając na to, jak mocno,
czując, że Nadie jest niezniszczalna.

Wolne ramię Nadie objęło ją. Została zmiażdżona przez kobietę. Zdała
sobie sprawę, że mimo całej siły Nadie, nie było w~niej zbyt wiele, była
malutka. Napierające na nią ciało nie mogło bardziej różnić się od ciała
Gretyla. Jej uczucia do Nadie i~Gretyl były przeciwieństwem. Bez względu
na to, że Nadie ją terroryzowała, zraniła, porwała, uratowała ją. Była
tam, tak \textit{żywa }, w~sposób, w~jaki nikt nie był dla niej od dawna.

Wyrwała rękę i~sięgnęła po tyłek Nadie, zwarty jak piłka tenisowa,
wsunęła dłoń po pasie legginsów, czując kontakt skóry ze skórą, nad
uczuciem czego tak ciężko pracowała, by zapomnieć. Jej usta zalały się
śliną. Jej palce zacisnęły się, znalazły splątane, mokre włosy, śliskie
fałdy, jej opuszki wślizgnęły się do środka. Zęby Nadie zacisnęły się
mocniej na jej wardze, sprawiając, że się cofnęła. Nadie podążyła za
nią, nie pozwalając jej odejść. To bolało. To było miłe uczucie.
Dyszała.

Nadie odskoczyła i~zdarła ubranie serią oszczędnych ruchów. Była jak
anatomiczny rysunek, ciało Lodołasicy, które mignęło w~taksówce, z~dziwnymi rzekami i~strumykami blizn rozciągniętych na chudych mięśniach.
Dysząc, sięgając po nią, jakaś część mózgu Lodołasicy zauważyła, że ma
lekko zakrzywione lewe przedramię, stare złamanie, które nie zagoiło się
prawidłowo.

Nadie uniknęła jej uścisku, usadowiła się na tyłku, patrząc szczerze
chłodnymi, błyszczącymi oczami. Sięgnęła po szampana i~wzięła kolejny
łyk. Wyczekująco przekrzywiła głowę. Lodołasica zrozumiała, rozebrała
się. Gęsia skórka, gdy obnażyła się przed tym spojrzeniem. Sięgnęła
ponownie, a~Nadie delikatnie pokręciła głową i~odskoczyła do tyłu, nadal
się gapiąc.

Oczy Nadie wędrowały po jej ciele. Oddech Lodołasicy stał się krótki.
Czuła spojrzenie. Nadie mogłaby rozerwać ją na strzępy, zmusić do
poddania się. Każdy nerw i~mieszek włosowy ożyły naelektryzowane,
mrowiące życie. Oczy Nadie się zwęziły. Uśmiechnęła się leniwie,
prześledziła jeden ze swoich sutków, duży i~bladoróżowy, stwardniałym
koniuszkiem palca. Dźwięk skóry na skórze był głośny, jedyny inny dźwięk
obok oddechu Lodołasicy. Sięgnęła po własną pierś, dotknęła jej, tak jak
Nadie dotykała jej.

Nie czuła się jak jej własny palec. To było jak u Nadie. Dopasowując jej
ruch do ruchu, jej system nerwowy stracił kontrolę nad własnymi
granicami.

Nadie skinęła głową i~polizała koniuszek palca, po czym przyłożyła go z~powrotem do sutka. Zahipnotyzowana Lodołasica zrobiła to samo. Uczucie
bycia dotkniętym przez nieznajomego nie było tak silne, ale gdy wpadła w~chłodne oczy Nadie, wzrosło. Kiedy w~jej peryferyjnym polu widzenia
zobaczyła, że palec Nadie zsuwa się niżej i~podąża za jego przykładem,
sapnęła. Nie masturbowała się od miesięcy, odkąd została porwana, od
jakiegoś czasu. Ta część niej zgasła, kiedy została porwana, ale czekała
i dostrzegła swoją szansę. Ich dłonie poruszały się szybciej, zamazując
się, miękkie mokre dźwięki i~oddech narastający w~tonacji. Kiedy wygięła
plecy w~łuk i~sapnęła, Nadie zanurkowała przez łóżko i~położyła ją na
plecach, chowając twarz między nogami, z~językiem migoczącym szybko i~bezlitośnie, rękami na biodrach, odmawiając poddania się, gdy się
szarpała. Zatopiła palce w~krótkich włosach Nadie, wykrzykiwała słowa
bez sensu, jechała na nim, nie dbając o~to, kto słyszał, nie dbając o~to, co czuła Nadie, wypalając samoświadomość w~jednej chwili, która
trwała i~trwała.

Kiedy skończyła, ostrożnie puściła Nadie, poczuła język na wewnętrznej
stronie ud, poczuła, jak soki i~ślina stygną pod jej tyłkiem. Nadie
wzniosła się jak wąż, cała w~mięśniach i~ścięgnach. Pachniała i~smakowała się na twarzy Nadie, gdy jej udo wsunęło się między nogi
Nadie, a~Nadie je przycisnęła, cała ta siła na nią skręciła. Lodołasica
była oszołomiona z~powodu hiperwentylacji, szampana i~miażdżącego kości
orgazmu, ale wciąż była pełna zwierzęcego napalenia. Obróciła Nadie,
świadoma, że Nadie pozwoliła się przetoczyć, ale wiedząc, że tego
właśnie chciała Nadie, chwyciła kobiety za nadgarstki i~przycisnęła je
do głowy, chowając twarz w~kępce pod pachą, zanim ugryzła ją w~pierś,
gryząc mocniej, uważnie wsłuchując się w~odpowiadające westchnienia,
wytężając siły, by trzymać nadgarstki. Nadie naparła na nią, a~ona
stanęła dęba, odepchnęła się i~spojrzała jej w~oczy. Były
nieskoncentrowane, jej oddech w~ostrych sapnięciach.

-- Chcesz tego? -- wyszeptała. Jej ręka opadła niżej. Ciągła zgoda była
rzeczą odchodzących. Była przyzwyczajona do zadawania tego pytania i~zadawania go jej, ale dla Nadie było to egzotyczne. Oczy Nadie przez
chwilę skupiły się na niej, zagryzła wargę i~jęknęła. 

-- Tak.

Pod wpływem impulsu Lodołasica zapytała: 

-- To znaczy?

-- Tak -- powiedziała Nadie. -- Tak, \textit{proszę}. Proszę?

Uległość tej kobiety, która potrafiła zabić na setki sposobów gołymi
rękami, zelektryzowała pokój.

Powoli, drażniąc się, poruszyła ręką i~zabrała się do pracy. Biodra
Nadie pracowały i~szarpnęły, a~ona zatrzymała się, odsunęła i~spojrzała
jej w~oczy. 

-- Chcesz tego?

-- Proszę -- powiedziała Nadie. -- Proszę, proszę.

Więcej pocałunków. Biodra Nadie zadrgały. Znowu się zatrzymała.

-- Czy \textit{tego chcesz }?

-- Chcę tego. Proszę. Tak. Proszę, Lodołasico, proszę. Proszę, nie
przestawaj.

Znowu spojrzały sobie w~oczy. Lodołasica wytrzymała jej spojrzenie,
palce wbiły się w~te niesamowite mięśnie pośladków i~czekała. Nadie
przygryzła wargę, a~jej oczy zabłysły. Jej skóra lśniła, błyszczała od
potu.

-- Proszę, och proszę, nie przestawaj. Proszę?

Powoli opuściła twarz. Tym razem nie zatrzymywała się, jechała na
podskakujących biodrach Nadie, całym ciałem podążała za tym, jak Nadie
stanęła dęba, wzdrygnęła się, krzyczała i~szarpała prześcieradło
szponiastymi dłońmi.

Kiedy skończyła, Lodołasica delikatnie oblizała palce i~opadła obok
Nadie, której pierś zafalowała jak miech. Jej skóra była wilgotna od
wysychającego potu, a~Lodołasica zarzuciła na nią nogę i~rękę, gryząc
bliznę na obojczyku, u podstawy gardła.

-- Mmmm -- wymruczała Nadie. -- Bardzo dobrze. Niezły prezent na
pożegnanie. Nic ci nie dałam.

-- Powiedziałeś coś o~kierunkach do moich przyjaciół?

-- To nie jest przysługa. Nie są w~dobrej formie, nawet jeśli tak im się
wydaje. Twój ,,domyślny'' świat z~każdym dniem staje się mniej stabilny.
Istnienie odchodzących jest postrzegane jako główna przyczyna,
destabilizujący wpływ poza wszystkimi innymi. Nie wyobrażaj sobie, że
tylko dlatego, że możesz raz czy dwa uciec, nie zdecydują się kiedyś
zabrać was wszystkich.

-- Możemy odbudować. Spójrz na Akron.

Nowe Akron, zbudowany na miejscu zrównanych z~ziemią budynków, nie
chciał być cmentarzem. Ludzie, którzy przybyli do niego, by odbudować po
tym, jak armia, najemnicy i~gwardziści dołączyli do powracających
miejscowych, aby budować nowe rodzaje budynków, zaawansowane domy dla
uchodźców prosto z~podręcznika UNHCR, zaprojektowane tak, aby wesoło
wykorzystywać energię, gdy wiał wiatr lub słońce świeciło, by przez
resztę czasu zapadać w~hibernację. Wielopiętrowe budynki przeplatały
szklarnie i~hydroponiczne ogrody targowe z~domami, wychwytując ludzkie
odchody na nawóz i~ścieki do nawadniania, wychwytując ludzki CO2 i~oddając tlen. Były to praktycznie kosmiczne kolonie, zamieszkałe przez
najbiedniejszych ludzi na świecie, którzy dostosowali i~ulepszyli
systemy, które tak wielu innych biednych ludzi poprawiło w~wyniku
katastrof, jakie przeżyła ludzkość. Przedmieścia hexayurt działały jako
swego rodzaju strefa przejściowa między domyślnym a~nowym rodzajem
stałego osiedla, gdzie ludzie przychodzili i~odchodzili, jeśli uznali,
że Akron nie jest dla nich.

Akron nie był pierwszym takim miastem, była Łódź, Kapsztad, Monrowia.
Było to pierwsze amerykańskie miasto, pierwsze wyraźnie zrodzone z~represji wobec odchodzących. Postawiło Departament Stanu w~niezręcznej
sytuacji, potępiającego ugodę, która funkcjonalnie była równoważna wielu
chwalonym gdzie indziej.

-- Wiele słyszę o~Akron. Raz jest szczęśliwym trafem. Ma dopiero
miesiące. Może jutro upaść. Byłam w~Łodzi, kiedy to się stało. Łódź nie
była pierwszym miastem, w~którym to wypróbowano. W~Krakowie zawiodło,
bardzo źle. Było wiele zgonów. Straszna choroba, gorączka w~wodzie, nikt
nie mógł zmusić przychodni do wydrukowania odpowiedniego leku.
Słyszeliście o~sukcesach tych miast, ale jest tak wiele porażek.

-- Ludzie odchodzą, ponieważ świat ich nie chce. Jesteśmy obciążeniem.
Słyszałam, jak mówił o~tym mój ojciec: ludzie, którzy chcą przyjechać do
Kanady, ludzie, którzy chcą mieć dzieci, ludzie, którzy marzą o~tym, aby
ich dzieci nauczyły się wszystkiego, czego potrzebują, aby przetrwać na
świecie, marzą o~opiece zdrowotnej i~starości bez nędzy. Według niego,
ci ludzie są zbędni, z~wyjątkiem sytuacji, gdy stanowią szansę na
wygranie kontraktu rządowego na karmienie ich tak tanio, jak to możliwe
lub umieszczenie ich w~obozach dla uchodźców. Czy wiesz, ile pieniędzy
zarabia mój ojciec ze swojego udziału w~prywatnych więzieniach Redwater?
Nazywa to swoim funduszem majątkowym gułagu.

Nadie zachichotała i~uderzyła się w~udo. 

-- Zapomniałam, jaki zabawny był
twój staruszek. Nie musisz się martwić, dziewczynko, nie masz tej krwi
na rękach.

-- Teraz jest na Twoich.

-- Mam na rękach prawdziwą krew. Mogę żyć z~metaforyczną krwią.

-- Ale dlaczego? Nie widzisz, że to szaleństwo? Dlaczego świat miałby
trwać, skoro jego system nie potrzebuje już ludzi? Nasz system powinien
nam służyć, a~nie odwrotnie. Spójrz na odchodników: jeśli pojawisz się w~odchodzeniu, będą rzeczy, które możesz zrobić, aby zrobić miejsce dla
siebie. Odchodzenie opiera się na założeniu, że każdy powinien być w~stanie zaangażować się w~pracę i~zapewnić wszystko, czego potrzebuje,
aby dobrze żyć, łóżko, dach i~jedzenie, a~także nieco extra dla osób,
które nie mogą tak wiele zrobić. W~stabilnych miejscach odchodniczek,
problem polega na tym, że nie ma \textit{wystarczającej liczby} ludzi.

-- Gratulacje, uczyniłaś cnotę z~nieefektywności. Poświęcenie większej
ilości godzin na wykonanie tej samej pracy nie jest ideologicznym
triumfem.

To był znajomy temat dla Lodołasicy, dyskusja, która często zaburzała
posiłki.

-- Masz rację, to cholernie śmieszne. Gdyby tak było, bylibyśmy idiotami.
Ale nie jest. W~default niechciani ludzie pracują \textit{po łokcie},
żebrząc o~pieniądze, żebrząc o~gówniane prace, zachęcając dzieci do
dowolnego learnware'u, któremu mogą zaufać, za pomocą interfejsu
powierzchni. Jedyne, czego \textit{nie} wolno im robić, to poświęcać
wszystkich tych godzin pracy na hodowanie żywności dla siebie, budowanie
sobie stałego domu lub budowanie domów kultury. Ponieważ system, który
organizuje grunty, na które trafią domy, żywność i~dom kultury,
zdecydował, że lepiej je wykorzystać do innych celów.

-- Jeśli powiesz mi o~bezużyteczności miłych restauracji, mogę
zachichotać. Powinnaś wiedzieć, że mam rezerwacje w~sześciu z~siedmiu
najlepszych restauracji na świecie na przyszły tydzień i~bilety SST, aby
się do nich dostać.

-- Restauracje są miłe. Mamy miejsca, w~których można zjeść smaczne
posiłki w~odchodnictwie. Czasami mogą poprosić Cię o~pomoc w~gotowaniu.
W B\&B to była świetna robota, ludzie o~to walczyli. To był zaszczyt
wpuścić nieznajomego do kuchni. Default jest tak zorganizowany, że tylko
niektórzy mogą jeść w~restauracjach, więc tylko niektórzy muszą pracować
w restauracjach. W~odchodnictwie każdy może jeść, kiedy tylko chce, a~w~rezultacie jest wiele do zrobienia, gotowanie, uprawa i~sprzątanie. Nowi
goście zawsze mają trudności ze znalezieniem \textit{wystarczająco dużo}
do zrobienia, martwiąc się, że nie są wystarczająco zajęci, aby odrobić
wszystkie rzeczy, które konsumują. Automatyzujemy więcej niż default, a~nie mniej, a~liczba roboczogodzin potrzebnych do utrzymania sytości i~zadowolenia przez cały dzień jest o~wiele mniejsza niż w~przypadku
nieefektywnego systemu, w~którym musisz się przemóc.

-- To nie będzie mój problem. Będę leżeć i~każę karmić mnie obieraczom
winogron. Daj mi rok, będę nosić togę i~laur.

-- Jedyni zetty, jakich znam, którzy żyją w~ten sposób, to uzależnieni
lub załamani. Prawdziwe zetty, jak mój tata, pracują tyle godzin, co
każdy żebrak. Bycie zettą oznacza martwienie się, że nie jesteś
\textit{wystarczająco }zettą, martwienie się, aby Twój stos złota był
większy niż stosy innych dupków. Założę się, że mój stary nie spał przez
osiem godzin z~rzędu od dziesięciu lat. Gdyby nie technologia medyczna,
ten skurwiel byłby martwy z~powodu dziesięciu ataków serca i~dwudziestu
udarów.

-- Nikt go nie zmusza.

-- Wiesz, że to prawda. Pracowałaś dla zett. Czy spotkałeś kiedyś
leniwego zettę?

-- Oczywiście.

-- Czy była pijaczką? Albo brała pigułki?

-- Cóż \ldots 

-- Nikt cię nie zmusza. To cholernie niesamowita niespójność, że wszyscy,
którzy mają więcej pieniędzy, niż mogliby wydać, spędzają każdą godzinę,
próbując zdobyć więcej. Odchodzące, które nie mają nic, grają jak nikt w~default. Bawią się jak dzieci, zanim ktokolwiek zorientuje się w~harmonogramach, leżą jak nastolatki, które wypierdalają ze szkoły i~leżą
i pieprzą bzdury na dachu \textit{godzinami}. Robią rzeczy, o~których
ludzie zawsze myślą: ,,\textit{Gdybym tylko był bogaty\ldots }''. Ironia
polega na tym, że \textit{bogaci ludzie nie robią takich rzeczy}.

-- Rozumiem ironię. Nie musisz mnie nią bić.

-- W~przypadku zett dobrze jest dokładnie wyjaśnić. Nie są dobrzy w~krytycznym myśleniu o~pieniądzach.

Nadie oparła się na łokciu, a~ich ciała na chwilę przywarły do
zaschniętego potu. 

-- To, co mówisz, to nie nowość, panno eks-Zetta.
Jestem starsza. Spędziłam tyle lat, mieszkając z~zettami co Ty. Nie
rozumiesz: to nie jest stabilne. Nie będzie wiecznie świata defaultu i~świata odchodzących handlujących ze sobą. Kiedy masz wielkich bogatych
ludzi, a~wszyscy biedni są tak samo biedni, rezultat jest\ldots 
\textit{niestabilny}.

-- Jeśli są bogaci i~biedni, potrzebna jest historia, która wyjaśni,
dlaczego niektórzy mają tak wiele, a~tak wielu niewiele. Potrzebujesz
historii, która to wyjaśnia, że tak jest sprawiedliwie. W~ubiegłym
stuleciu bogaci ustabilizowali sytuację, oddając pieniądze, podatki,
edukację i~tak dalej. Państwo opiekuńcze. Ludzie mogli \textit{stać się}
bogaci. Wymyśl coś, możesz stać się bogaty, nawet jeśli nie urodziłeś
się bogaty.

-- Ale tamte zetty, jeszcze nie zetty, właściwie tylko giga lub mega,
pozwalały na opodatkowanie ich pieniędzy, ponieważ było to tańsze niż
płacenie za prywatne bezpieczeństwo i~oficjalną inwigilację, których
potrzebowali, aby utrzymać bogactwo, gdyby system stał się niestabilny z~powodu przerwy między nimi a~wszystkimi.

-- Ochrona prywatna jak Ty?

-- Oczywiście jak ja. Na czym polegała moja praca, jeśli nie na
powstrzymywaniu biednych ludzi przed nabiciem na widły bogatych ludzi?
Kiedy technologia sprawiła, że nadzór był tańszy, zmienił się rachunek.
Mogli trzymać więcej pieniędzy, bez udawania, że bogacenie się wynika z~dobrego samopoczucia, wrócić do idei boskiego prawa królów, ludzi
urodzonych jako bogaci, bo los im sprzyja. Bardziej opłacalne było
kontrolowanie za pomocą technologii ludzi, którym nie podobał się ten
pomysł, niż dawanie okruchów na poparcie bajki o~nagrodach za cnotę.

-- Jak mówisz, bardzo bogaci chcą się wzbogacić. Kiedy pieniądze są miarą
wartości, im więcej pieniędzy, tym bardziej wartościowe. Mówią, że to
sposób na określenie wyniku. Zetty grają, aby wygrać. Podobnie jak wojny
oligarchów w~Rosji, bogaci ludzie zauważają, że kumple ze starej szkoły
mają bardzo kuszące fortuny i~wszystko wolno.

-- Teraz jesteś jedną z~nich.

-- Nie. Jestem bogata, ale nie jestem zettą. Sprawy zbliżają się do
punktu kulminacyjnego, mogą pójść w~dowolny sposób. W~nadchodzących
miesiącach zostanie przelana krew. Nie chcę pieniędzy, żeby utrzymać
wynik. Chcę za pieniądze kupić wolność, wolność szybkiego
przemieszczania się w~inne miejsca, swobodę kupowania wybornego jedzenia
czy płacenia za opiekę medyczną. Przeżyłam wiele rzeczy, Lodołasico,
nawet więcej niż Twoi odchodzący przyjaciele w~ich kryjówkach. Planuję
to przeżyć.

-- Mam nadzieję, że przeżyjesz. -- Lodołasica naprawdę tak myślała.

-- Wzajemnie. -- Podniosła się i~sięgnęła po majtki.

\chapter*{iii}
Poruszali się wokół Limpopo. Po pierwsze, miejsce, o~którym myślała jako
,,więzienie'', z~powodu zakratowanych drzwi i~przerywanych dźwięków
więźniów w~bloku więziennym, dzięki wariacji wentylacji. Jej cela była
wystarczająco duża, by pomieścić wąskie łóżeczko wykonane ze
sprężystych, metalowych taśm, których nie można było oddzielić od ramy,
bez względu na to, jak ciężko pracowała, oraz plastikową toaletę bez
siedzenia i~umywalkę wbudowaną bezpośrednio w~ścianę. Co trzeci dzień
dostawała rolkę papieru toaletowego i~paczkę mydła i~używała ich do
umycia ciała najlepiej, jak tylko mogła. Jej papierowy pomarańczowy
kombinezon -- zbyt delikatny, by zwijać się w~sznurek -- nie chciał się
brudzić, nawet gdy smarowała go protami z~jadalnych tubek, które
dostawała trzy razy dziennie.

Strażnicy, którzy dawali jej jedzenie i~przybory toaletowe, nie chcieli
rozmawiać. Nosili kombinezony chroniące przed zagrożeniem biologicznym,
kamizelki kuloodporne, gogle i~maski na twarz. Pewnego razu towarzyszył
jej strażnik, którego przyłbica ociekała śluzowatą śliną. Za tą śliną,
twarz strażnika była wykrzywiona ze wściekłości. Praktycznie rzucił jej
tubą z~jedzeniem, papierem i~mydłem, zatrzasnął drzwi (nie zrobiły
żadnego hałasu ponad sykiem hermetycznej uszczelki).

Dwukrotnie zabrali ją z~celi i~zaprowadzili do pokoju na przesłuchanie.
Została wyposażona w~czujniki na te sesje. Ogolili jej głowę,
przyczepili elektrody do jej nagiej skóry głowy, bardziej na nadgarstku,
na sercu, na gardle. Nie walczyła. Kogo obchodziły włosy? Najważniejszą
rzeczą było zaoszczędzenie jej energii na to, co miało nadejść.

Pytającego nie było w~pokoju, ale był obecny jako głos dochodzący z~wkładki dousznej, którą włożyli strażnicy. Usłyszała oddech pytającego,
jakby był kochankiem szepczącym jej do ucha. Przypominało jej to
obuuszne nauszniki kosmików, ale te miały ją zdenerwować i~zdezorientować.

-- Luizo?

-- Jeśli chcesz.

-- W~takim razie Limpopo? -- Głos był pozbawiony emocji.

-- Jeśli chcesz.

-- Zaczniemy od czegoś prostego.

-- Czy jestem aresztowana?

-- Chcę Twoje hasło.

Wyrecytowała ciąg bezsensownych znaków.

-- Teraz to drugie.

Nic nie powiedziała.

-- To drugie. To jest hasło dla wiarygodnego zaprzeczenia. Nietrudno
powiedzieć, kiedy oszukujesz. Infografiki dają mi ogromny wgląd w~Twój
umysł.

Próbowała zachować spokój umysłu. Uspokajanie jej umysłu również
uwidoczniło się na jego skanach. Zastanawiała się, co mierzył, jak
dokładne to było. W~tłumie OU byli genialni neuroludzie. Powiedzieli, że
wszyscy wiedzieli, że połowa wszystkiego, co uważali za prawdziwe o~ludzkim umyśle, była bzdurą. Nikt nie mógł się zgodzić, która połowa.

Czas się rozciągnął. Zastanawiała się, czy ją uderzą, potraktują
szokowo, spalą. Zabili Jimmy'ego i~Etceterę, przecięli im gardło i~wrzucili w~śnieg, by umarli.

-- Nie powiem ci.

-- W~porządku. -- Strażnicy odpięli ją i~zaprowadzili z~powrotem do celi.
Mijały dni. Nie pozostało nic do roboty poza wpatrywaniem się w~ściany.
Zawsze lubiła samotność, uważała się za niedoskonałą odchodniczkę,
ponieważ towarzystwo innych bywało uciążliwe. Ale kiedy dziesięć dni
minęło tylko z~jej myślami i~jej desperackimi, destrukcyjnymi próbami
medytacji, przyszli i~zabrali ją. Przyłapała się na tym, że naprawdę
\textit{oczekiwała} na możliwość rozmowy z~głosem.

Ogolili jej głowę z~krótkiego zarostu, ponownie nałożyli żel i~czujniki.

-- Dzisiaj robimy skan -- powiedział głos. -- Będziemy w~stanie zasymulować
ten skan i~poddać go przesłuchaniu w~okolicznościach, które wykraczają
poza większość tego biznesu i~pomijają go. W~zależności od cech tego
skanu, jego niezawodności i~elastyczności, możemy w~ogóle Cię nie
potrzebować. Czy to jasne?

-- Czego chcesz?

-- Twoje hasło.

-- Dlaczego?

-- Ponieważ zbadaliśmy wykres społeczny Twojej kohorty i~doszliśmy do
wniosku, że jesteś kluczowym węzłem.

-- Brzmi to jak dobry powód, żebym trzymała język za zębami.

-- Możemy spróbować wymusić na Tobie informacje. Możemy nawet spróbować
przymusu fizycznego. Wiesz, możemy zrobić skan od ludzi, którzy
technicznie już nie żyją.

To była bzdura. Musiała być. OC zawsze utrzymywał, że to nigdy nie
zadziała, nie bez przepływu krwi przez mózg. Nie rozumiała biologii, ale
wiedziała, że to bzdury. Czyż nie?

-- To byłaby niezła sztuczka.

-- Kiedy znajdziemy się w~Twoich danych, użyjemy ich do wywołania
wewnętrznych zakłóceń w~Twojej komórce. Uzupełni to naszą strategię
interwencji fizycznych.

-- Ale dlaczego?

-- Luizo, nie bądź śmieszna. Wiesz dlaczego.

Nie chciała się złościć, chociaż przedłużający się okres samotności
sprawiał, że była nerwowa i~emocjonalna. -- Bo wiesz, że to my albo Ty,
prawda?

-- Nie. Ponieważ Ty i~Twoi przyjaciele jesteście terrorystami. Luizo,
bądź poważna. Tu nie chodzi o~zazdrość. Chodzi o~przestępczość.

-- Jakie przestępstwo?

-- Luizo.

-- Jakie przestępstwo?

-- Bądź poważna.

-- Skłotowanie?

-- Wtargnięcie. Kradzież. Kradzież tajemnic handlowych. Piractwo na
niewyobrażalną skalę. Obchodzenie legalnych urządzeń odcinających w~fabrykatorach. Produkcja narkotyków. Nielicencjonowana produkcja
potencjalnie śmiertelnych farmaceutyków. Produkcja broni klasy
wojskowej, w~tym mechów i~różnych bezzałogowych statków powietrznych.
Nielicencjonowane wykorzystanie widma elektromagnetycznego, w~tym
zastosowania, które mogą zakłócać i~zakłócać sieci ratownicze,
bezpieczeństwa publicznego i~pierwszej pomocy. Czy muszę kontynuować?

-- Czego chcesz od ludzi? Co mają zrobić? Nie ma dla nas niczego w~defaulcie. Nie ma gdzie mieszkać. Nic do jedzenia. Nic do roboty.
Jesteśmy nadwyżką. Odeszliśmy, zaczęliśmy od nowa, nikomu nie
przeszkadzając.

-- Zabrałaś to, co nie jest Twoje. Żyjesz, biorąc to, co nie jest Twoje.

-- Jak inaczej mamy żyć?

-- Jakie jest Twoje hasło?

-- Kiedy zrobisz to skanowanie?

-- Teraz trwa. Ta rozmowa pomoże skalibrować.

-- Bzdury. Miałam już skany.

-- Techniki skanowania używane przez odchodzących są prymitywne i~niewiarygodne. Mamy lepszą technologię. To zaleta nie-bycia przestępczym
podziemiem.

-- Wolałabym być przestępczym podziemiem niż tajną policją.

-- Nie jesteśmy policją.

-- A więc szpiedzy.

-- Niezbyt sensowny termin.

-- Chciałabym porozmawiać z~prawnikiem.

-- Jesteś nielegalnym imigrantem, obywatelem Brazylii z~wygasłym
paszportem i~bez wizy. Dlaczego uważasz, że masz prawo do
przedstawiciela prawnego? Jak byś zapłaciła?

-- Chciałabym porozmawiać z~kimś z~mojego konsulatu.

-- Ambasada Brazylii prowadzi oficjalną politykę współpracy przy
działaniach antyterrorystycznych.

-- Dlaczego w~ogóle potrzebujesz mojego hasła, skoro jesteś tak jebanie
podobny do boga? Wygląda na to, że masz wszystko, czego potrzebujesz.

-- Mamy wiele rzeczy, których potrzebujemy. W~Twoim ruchu sieciowym może
być więcej. Poza tym mamy doskonałe wyniki z~podszywania się pod
członków Twojego kultu. Jest zaskakująco skuteczne.

-- Podobnie jak mówisz mi, że to robisz, więc spędzam cały swój czas,
próbując dowiedzieć się, którzy ludzie są marionetkami?

-- Nie musisz się już martwić o~rozmowę z~tymi ludźmi. Masz wysoki
autorytet, więc przekonanie nawet niewielkiej liczby osób, że jesteś
zdrajcą, wywoła ogromną wewnętrzną niezgodę.

-- Jak powinnnam Cię nazywać?

Oddech szepnął jej w~uszach. 

-- Michael wystarczy.

-- Michael, czy przyszło ci do głowy, że nie masz się czym targować? Nie
możesz mi dać nic, co sprawiłoby, że będę chciała dać ci moje hasło, ze
wszystkich powodów, które właśnie wymieniłeś. Ty i~wszyscy, z~którymi
pracujesz, uczynicie swoją misją zniszczenie jakiejkolwiek szansy na
przetrwanie rasy ludzkiej do końca tego stulecia. Więc co masz nadzieję
dzisiaj dostać ode mnie?

-- Mam wiele rzeczy do targowania się, Luizo. Mogę zaoferować ocalenie
życia Twoich przyjaciół. Wiemy, gdzie są, \textit{zawsze} wiemy, gdzie są.
Jesteśmy zdolni do precyzji w~naszych atakach przeciwko nim. Widziałaś,
jak po ciebie przyszliśmy.

W godzinach, w~których była sama ze swoimi duchami w~celi, najczęściej
odwiedzał ją Etcetera. Ciągle widziała jego twarz, słyszała jego głos.
Miała sny, w~których czuła, że wtula się w~nią, jedną ręką na niej,
dłonią między jej piersiami, jego szorstkim zarostem na jej plecach,
oddech łaskoczący jej skórę. Budzenie było jak jeden z~tych koszmarów w~koszmarze, w~których wierzysz, że nie śpisz, ale wciąż śnisz. Tylko ona
\textit{nie} spała i~została uwięziona. Nigdy więcej nie zobaczę Etcetery.
Czasami odhaczała jego absurdalne imiona jak różaniec, zaciskając oczy,
usilnie starając się przypomnieć sobie uczucia ze swoich snów, jego
zapach, dźwięk, sposób, w~jaki ją trzymał. Świadomość, że nie żyje,
chwyciła ją w~kółko, sprawiając, że oddech zapierał dech w~piersiach,
jak podmuch zimnego powietrza mrożącego jej płuca.

-- Widziałam, jak po mnie przyszedłeś. Co zrobiłeś.

-- Jesteś zdenerwowana stratą swojego chłopaka, mężczyzny z~imionami. -- Brzmiał lekko szyderczo, a~może tak to rozumiała. Była odlegle zła,
emocje ledwo widoczne na tle płonącego słońca jej żalu. Wydawało jej
się, że słyszy, jak kalibrują swój model, przypisując dużą wagę tak
egzotycznemu stanowi emocjonalnemu.

-- Zmieniasz temat. Kiedy mordujesz tak, jak to zrobiłeś, nie uzasadniasz
pomocy. Kiedy zabierasz mi moją najdroższą miłość, pokazujesz mi, że nie
należy ci ufać. Kiedy targujesz się ze mną, przywiązaną do krzesła,
sprawiasz, że myślę, że kłamiesz o~swojej zdolności do uruchomienia mnie
jako sym. Jedynym powodem, dla którego mogę sobie wyobrazić, że chcesz
ze mną porozmawiać, jest to, że mam coś, czego potrzebujesz, a~nie
możesz tego dostać w~żaden inny sposób.

Nie było na to odpowiedzi.

Po kilku minutach powiedziała: 

-- Halo?

Odpowiedź nie nadeszła. Czas minął. Zamknięcie w~swojej maleńkiej celi
było okropne, ale przynajmniej mogła poruszać kończynami, zmieniać
postawę. Iść do toalety. Przywiązana w~ten sposób\ldots 

Stłumiła narastającą panikę. Gdyby chcieli zademonstrować swoją
wyższość, mogliby ją sterroryzować, zostawiając ją w~takim stanie.
Poczucie przerażenia tylko zademonstrowało skuteczność tej taktyki.
Mogła być uwięziona przez tych ludzi przez długi czas, a~oni bez
wątpienia budowali dossier skutecznych technik zapewniających jej
uległość.

Czekała tak długo, jak mogła. 

-- Muszę siku. -- W~pokoju był strażnik:
przyłbica, maska, nausznik. Jego język ciała mówił jej, że patrzy na
coś, czego nie mogła zobaczyć, słyszy coś, czego nie mogła usłyszeć.
Może oglądał telewizję albo zegar odliczający sekundy do zakończenia tej
części eksperymentu. Mogła powiedzieć, że ją słyszał.

-- Proszę.

Udawał, że jej nie słyszał. 

-- Michael, jeśli sprawisz, że się obsikam,
nie zrobisz nic, aby przekonać mnie, że jesteś ludzką, rozsądną osobą, z~którą chcę współpracować.

Zacisnęła na pęcherzu i~myślała o~innych rzeczach: trudnych problemach z~kodowaniem, do których powracała raz po raz, kiedy miała chwilę,
próbując uzyskać rzeczy, które powinny zadziałać; historii Jimmy'ego
(ostrożnie omijającego jego śmierć), walkę, którą ona i~Jimmy
przeprowadzili w~oryginalnym B\&B. Wyobraziła sobie kroki, które
podjęłaby, aby pomóc w~odbudowie floty rowerowej Thetford, ogromnej
liczby drukowanych rowerów górskich z~włókna węglowego, które były
pogięte, połamane i~roztrzaskane przez offroading w~poprzednim ciepłym
sezonie, które ona, Etcetera i~inni systematycznie regenerowali, tworząc
linię fabryczną do rozbierania, oceniania, ponownego składania i~testowania każdego elementu, burze mózgów rozwiązania nietypowych
problemów mechanicznych upartej materii fizycznej.

Naprawdę musiała się wysikać. Zastanawiała się, czy w~ostatniej
wyciskanej tubce dali jej środek moczopędny. Byłby to sposób na
upewnienie się, że ta sytuacja powstanie. Może chcieli skalibrować swój
model z~obrazem tego, co się stało, gdy została upokorzona.

-- Nie muszę tego sprzątać.

Strażnik jej nie zauważył.

Trzymała go jeszcze przez dwie minuty, według powolnego odliczania, a~potem puściła. Złapała swój nastrój żelaznymi szczypcami i~nie
pozwoliła, by zmienił się w~upokorzenie, bo to \textit{tylko siki }.
Wygrali, jeśli pozwoli, żeby to ją rozwścieczyło. To było znacznie
gorsze niż zimne, śmierdzące siki, które przyklejały jej papierowy
kombinezon do nóg.

Potem nic nie powiedziała. Skupiła się na tych rowerach, z~radością, że
nagle uświadomiła sobie rozwiązanie zagadki, która ich wszystkich
krępowała, wyciąganie kłopotliwego roweru ze stosu, upewnienie się, że
\textit{działa}. Etcetera wymyślił drastyczne sposoby na uwolnienie
zniekształconych części, dostosowując mechanizmy zębate, które wydawały
się nieregulowane.

Jej oddech zwolnił. Przyszło jej do głowy, że prawie drzemała,
kontemplując te wspomnienia. Mogłaby spędzić resztę życia z~tymi
wspomnieniami, polerując je jak wdowa polerująca oprawione zdjęcia
ślubne. Niech tak będzie. W~myślach wciąż mogła odejść. Jebać ich.

Potem zastanawiała się, czy to kolejna część kalibracji, i~musiała się
zacisnąć, żeby nie płakać.

Próbowała, jak tylko mogła, ale nie mogła ponownie znaleźć tego miejsca
pamięci. W~końcu zaprowadzili ją z~powrotem do celi.

Następnego dnia wsadzili ją w~kajdany na nogi, zapakowali jej głowę i~wprowadzili do pojazdu, który podskakiwał i~przepychał się przez
nieokreślony czas. Została zabrana do autobusu, który bez wątpienia
śmierdział niemytymi ludźmi i~brzmiał jak zły dzień na oddziale
psychiatrycznym. Została przypięta pasami do siedzenia, ręce miała
przymocowane do pasów po bokach. Obok niej była osoba, również siedząca.
Kiedy strażnicy, którzy ją przywieźli, odeszli, przywitała się.

-- Dzień dobry. -- To był kobiecy głos.

-- Czy widzisz?

-- Masz na myśli, czy mam torbę na głowie? Nie. Dlaczego Ty?

Wzruszyła ramionami.

-- Gdzie jesteśmy?

-- Kingston -- powiedział głos.

-- Ontario?

-- Nie Jamajka. -- Parsknięcie śmiechu. Limpopo wyczuł, że inni słuchają
ich rozmów, lokalną ciszę podsłuchujących.

-- Gdzie jedzie ten autobus?

-- Pierdolisz.

-- Nie. To jest\ldots  Zabili moich przyjaciół, wzięli mnie, trzymali.
Zapakowali mnie i~przywieźli tutaj. Teraz nie wiem, dokąd idę.

-- Więzienie. Więzienie dla Kobiet w~Kingston.

-- Och. Myślę, że to ma sens.

-- Skoro tak mówisz. -- Limpopo tak długo była z~dala od prawdziwego
kontaktu z~ludźmi, że przyłapała się na rozgrzewaniu się przed tym
nieznajomym, który mógł być tajnym śledczym, a~nawet po prostu niemiłą
osobą.

-- Jak masz na imię?

-- Jaclynn -- powiedziała kobieta. -- Co oznacza litera G?

-- G?

-- Twoje dokumenty transferowe. Przyklejone do klatki piersiowej. Mówią,
że jesteś G. Denton.

Wzruszyła ramionami. Powinna była wiedzieć, że nie będzie
podporządkowana systemowi jako Luiza Gil, nie mówiąc już o~Limpopo. Choć
brazylijski konsul był bezzębny, tak odległy i~ścigany, jak przechodnie,
tak długo, jak miała swoje nazwisko, można ją było znaleźć. Nie można
jej było znaleźć, dopóki nie byli gotowi wystawić jej na pokaz, jeśli
ten dzień kiedykolwiek nadejdzie.

-- G? Szczerze mówiąc, nie mam pojęcia. -- Pomyślała o~,,wielkiej
szarozielonej, tłustej Limpopo'' Kiplinga.

-- Amnezja, co?

-- Nie dokładnie.

-- Jesteś prawdziwą tajemnicą, wiesz? Torba na twarzy, bez imienia.

-- Mam imię. Po prostu nie wiem, pod jakim nazwiskiem mnie wysyłają.

-- Pod jakim imieniem byłaś sądzona?

-- Bez procesu. Po prostu porwana. Polityczna. Jestem odchodzącą.

-- Jedną z~tych? Tak się zdaje. Wygląda na to, że wpadam na was
wszystkich, ilekroć korzystam z~gościnności. Hej! Jakieś odchodniczki w~autobusie?

W odpowiedzi podniosły się głosy. Nawoływania i~jęki też. Pod jej
kapturem Limpopo uśmiechnęła się. Zastanawiała się, co oznacza litera
,,G''.

\part{kolejne dni lepszego narodu}
\chapter*{i}

Najdziwniejszą rzeczą związaną ze starzeniem był brak snu. Tam rutynowo
budziła się w~godzinach, których nie widziała, odkąd była nastolatką.
Dziwne godziny, kiedy można było dostrzec nieoczekiwaną miejską dziką
przyrodę: żerujące szopy pracze, ukradkowe lisy, nietoperze. Seth, ten
dupek, nie miał tego problemu. Spał jak skała. \textit{Łysa} skała, nie
miał na tyle przyzwoitości, aby przyznać się do świadomości cofniętej
linii włosów (,,jak się nazywa sto królików biegnących do tyłu?'',
mówił, gdy poruszała temat). Zwariowała, kiedy jej włosy zaczęły się
kręcić, odbyła kilka konsultacji z~lekarzami z~całego świata, znalazła
jednego w~Tajlandii, który specjalizował się w~osobach transpłciowych,
dostała plik z~tabletkami do wydrukowania, które brała codziennie.
Załatwiły sprawę.

Najdziwniejsze w~bezsenności były przyjaźnie, które nawiązała z~ludźmi,
którzy nie śpią i~rozmawiają w~egzotycznych strefach czasowych. Drugą
najdziwniejszą rzeczą związaną ze starzeniem się było przebywanie
\textit{z Sethem}. Zawsze była zasmucona starymi parami, które nigdy ze
sobą nie rozmawiały. Te długie milczenia wydawały się rozpaczliwe.
Obiecała sobie, że nigdy tak nie skończy, dekady starzenia się,
rozpadania się w~towarzystwie milczącego, pierdzącego człowieka,
pędzącego, by zobaczyć, kto pierwszy dotarł do grobu.

Ale jako prawdziwa starsza pani z~siwymi włosami i~zmarszczkami
rozumiała milczenie. Nie musiała rozmawiać z~Sethem o~większości rzeczy,
ponieważ tak dobrze go wymodelowała w~swoim umyśle, że wiedziała, co
powie praktycznie na wszystko, co ona mu powie, i~vice versa. Mogli
siedzieć razem, nic nie mówiąc. Cisza nie była odległością, była
bliskością. Czasami przyłapywała go na patrzeniu i~uśmiechaniu się.
Odwzajemniała uśmiech. Te uśmiechy mogły być obarczone większą ilością
seksualnych insynuacji niż najbardziej napalone chwile w~jej całym -- co
prawda zdezorientowanym -- okresie nastoletnim.

Trzecią najdziwniejszą rzeczą był sam Seth, który -- choć potrafił spać
jak na mistrzostwach świata -- nie czuł się \textit{staro}. Raz przyszła do
niego, siedzącego przy łóżku, wpatrującego się w~nagie nogi, gołe
kolana, siwe, sztywne włosy, żyły, obwisłą, pomarszczoną skórę.
Zaskoczona zdała sobie sprawę, że był praktycznie we łzach, co nie
przypominało Setha, którego tak dobrze modelowała we własnym umyśle.

-- Co jest?

-- To nie ja. Jestem młodym mężczyzną. Kiedy widzę siebie w~lustrze,
sprawdzam dwa razy. Nie tak siebie widzę.

-- Czy chodzi o~Twoje włosy? Ponieważ mogłabym przedstawić Cię pani
doktor Wibulpolprasert\ldots 

-- Nie chodzi o~jebane \textit{włosy}. Mam w~dupie moje włosy, to jest
\textit{to}. -- Zaciekle klepnął się w~udo.

-- Spokojnie. -- Pogładziła jego dłoń.

-- Nie rozumiesz, to tak, jakby ktoś inny spoglądał na mnie z~lustra\ldots 

-- Seth?

-- Co?

Patrzyła na niego przez dłuższą chwilę. Zobaczyła, jak powoli świta
zrozumienie.

-- Och. Rozumiesz.

-- Rozumiem. -- Położyła go delikatnie na łóżku i~trzymała, aż, do
cholery, zasnął.

Teraz była 3:15 rano. Znowu zasnął. Przyłapała go w~trakcie świrowania
częściej niż kiedykolwiek. Martwiła się. Wiedziała, co to znaczy nie
rozpoznać osoby w~lustrze. Rozumiała dokuczliwe poczucie
\textit{niewłaściwości}. Jakaś jej część chciała wejść do jego głowy i~powiedzieć mu, żeby dorósł, jeśli chce poczuć dysforię, powinien
spróbować urodzić się trans, spróbować całego świata mówiącego mu, że
jest kimś, kim nie jest.

Wiedziała, że to bez sensu. Ból był bólem. Wszyscy dookoła niego
\textit{mówili} mu, w~sposób subtelny i~ostro, że nie był młodym człowiek,
jakim się czuł. Wiedziała, że najgorsze było to, że jego ciało uparcie
upierało się, by być ciałem starca.

Wyczuwała ślady tego, przez co przechodził Seth. Minęły. Przeszła przez
to, kiedy była młodsza. Poradziła sobie z~tym z~gracją. Mogła przez to
przejść z~lepszymi myślami i~zmianami w~swoich reżimach hormonalnych.
Nie zaprzeczała jak Seth. Seth wyglądał bardzo chłopięco, aż nagle
\textit{przestał}.

Przeszła przez korytarz, nasłuchując innych ludzi poruszających się po
domu, zaciągając szlafrok. Światła w~holu były przytłumione, a~świetlik
ukazywał bezchmurną noc zabarwioną miejskimi światłami, których nie było
tak wiele, by zagłuszyć wzdęty księżyc i~mgłę gwiazd. Byli tam
odchodzący, jakieś stare pierdy z~czasów Thetford. Czasami rozmawiała z~nimi, chociaż duże opóźnienie sprawiało, że była to bardziej nowinka niż
okazja towarzyska.

Nikt nie wstał. Światła zgasły, kiedy weszła do kuchni, jaśniejsze nad
powierzchniami przygotowawczymi, przyciemnione nad stołami, dom zgaduje,
że chce coś przygotować, zanim usiądzie, szturchając ją. W~miejscach, w~których było coś do zrobienia, pojawiły się różowe poświaty, chłodne
resztki powinny trafić do lodówek, kilka nieodpowiednich garnków
ułożonych do góry nogami do wyschnięcia na dużej powierzchni
przygotowawczej i~zapomnianych. Dom wiedział, kto o~nich zapomniał.
Gdyby chcieli, mogliby mieć na żywo tablice wyników ,,bohaterów prac
domowych'' i~,,niegrzecznych łobuzów'' rozrzucone na powierzchniach
wokół całego miejsca. Niektóre domy umieszczają je na lustrach
łazienkowych. Stawiasz czoła surowej rzeczywistości podziału pracy,
szorując zęby o~poranku.

Tam i~Seth byli ludźmi B\&B. Ludźmi Limpopo. Ludzie, których dotknęła
Limpopo, odmówili włączania tabel wyników. Powodem do sprzątania po
sobie było to, że szanujesz swoich współlokatorów i~chciałaś mieć
miejsce, w~którym każdy może podejść do czegokolwiek i~z niego
skorzystać, bez konieczności odkładania wcześniej czyjegoś gówna. Kiedy
spoty były stale niedotrzymywane, rozwiązaniem było dowiedzieć się,
dlaczego trudno było je zresetować, a~nie dowiedzieć się, jak zawstydzić
ludzi, którzy nie robili czegoś, co nieuchronnie okazało się bardziej
wrzodem na tyłku, niż miało do tego prawo.

Inne domy przysięgały na swoje ,,gospodarki reputacji''. Domy wywodzące
się z~Limpopo były tymi, z~których pochodziły dobre projekty życia,
które działały dobrze i~dobrze zawodziły. Mieli najmilsze domowe duchy,
dosłownie i~w przenośni. W~domu Limpopo fakt, że byłeś wkurzony na
współlokatorów, sygnalizował możliwość projektowania.

Odstawiła patelnie i~włożyła resztki do lodówki. Kontemplowała
imponującą ścianę zapieczętowanych pojemników z~jedzeniem i~składnikami.

-- Jestem przekąską.

Dom wiedział, co to znaczy. Półki z~leniwą Susan zawirowały, dając jej
trzy opcje: lody imbirowe i~o strukturze plastra miodu z~taką ilością
imbiru, że mogłaby rozwalić ci głowę, które kochała bardziej niż
powinna; ser kozi i~soczewica; dziwne liofilizowane ciastka migdałowe,
domieszkowane chili i~kardamonem, tak diabelnie uzależniające, że
wspólnie podjęli decyzję o~usunięciu plików z~repozytoriów domowych. W~końcu pokusa zawsze zwyciężała i~ktoś ściągał najnowszą wersję z~serwerów lustrzanych. Przepis był coraz \textit{lepszy}.

-- Jakbyś musiała nawet zapytać. 

 Podniosła migdałowe ciasteczka,
ściskając brzeg, by rozerwać pieczęć, i~wąchając rozpływający się w~ustach zapach migdałów, przechodząc przez sklepione przejście wokół
basenu z~karpiami, który delikatnie bulgotał w~chłodniejszym,
wilgotniejszym powietrzu, do małego salonu.

Ułożyła się na stosie poduszek, wybrała jedno ciastko i~ugryzła,
delektując się chrupieniem, słodyczą i~ogniem, które rozprzestrzeniały
się w~jej ustach. Jęknęła z~powodu pyszności. Wiedziała, że skończy całą
partię.

Strzeliła palcem w~przeciwległą ścianę. Uruchomił się, pokazywał jej
ulubione spotkania, umieszczał w~kolejce wiadomości dla niej, wiadomości
z kanałów, które uznano za prawdopodobne. Kilka przypomnień o~wyższym
priorytecie od osób, które lubiła i~którym ufała, zostało wysłanych, aby
dać jej znać, że czekają. Zmiażdżyła drugie ciasto. \textit{Cholera}, były
dobre.

-- Kto nie śpi? -- Powtórzyła, ponieważ dom błędnie ją zrozumiał z~powodu
ust pełnych jedzenia. Ekranowana ściana pokazywała twarze, AV i~uchwyty,
fragmenty pomieszczeń, w~których działy się różne rzeczy, pulsujące
coraz bliżej i~dalej, gdy rozmowy narastały i~zanikały. Miała sprzeczne
uczucie, że chce z~kimś porozmawiać, ale nie chce z~nikim rozmawiać,
uczucie utknięcie w~rutynie o~trzeciej nad ranem.

Znowu się poruszyła, machnęła ręką. Były książki, filmy, ale to uczucie
o trzeciej nad ranem, że chce-czegoś-ale-niczego, było dla tych.
Tęskniła za podekscytowaniem bliskiej śmierci.

-- Jak sobie z~tym radzisz?

-- Masz na myśli mnie? -- Głos Limpopo się nie zestarzał, chociaż istniały
algorytmy, dzięki którym z~biegiem lat głos się starzał.

-- O kim myślisz? Dom?

-- Po prostu sobie radzę. Mam zderzaki. Kiedy dochodzę do krawędzi,
odpychają mnie.

-- Czy kiedykolwiek się wyłączasz? Wchodzisz w~tryb kursora?

-- Nie dałam się skusić. Myślę, że to trauma mojego przebudzenia, te
wszystkie lata\ldots 

Minęło czternaście lat, zanim ktokolwiek wymyślił, jak ustabilizować
symulację Limpopo. Odzwierciedlało to długą lukę między Światową Wojną
Defaultu a~Dekadą Odchodzących, która była głupią nazwą, której wszyscy
nienawidzili, ale przynajmniej miała wbudowaną datę wygaśnięcia. To
także dziwactwa Limpopo, jej dziwna neuroanatomia. Ta dziwność była
praktycznie normalna. Kiedy udało im się uruchomić Roz, a~potem, na
krótko, OC, powstał zestaw kategorii, na które można by posortować
zobrazowane ludzkie mózgi, na przykład grupy krwi, a~każda z~nich
używała innych parametrów symulacji.

Skany bardziej przypominały odciski palców niż grupy krwi, każdy z~charakterystycznymi i~niechętnymi do współpracy zmarszczkami (dosłownie
i w~przenośni). Stabilizacja symów była odporna na nadrzędną
systematyzację, uporczywie naciskając na bycie sztuką, a~nie nauką.

Pomiędzy chaosem a~nieustępliwością ludzkich mózgów, Limpopo leżała
uśpiona przez długi czas. Kiedy się obudziła, natychmiast zrozumiała
sytuację. Pomogło, że Etcetera tam był. Przez pewien czas byli szybkimi
przyjaciółmi. Przeprowadzili nawet słynną serię dyskusji na temat lat,
które przegapiła Limpopo, wszystkich ważnych lat chaosu, kiedy nikt nie
był pewien, co się dzieje, publikując godzinę rozmów każdego dnia, a~następnie biegając po ogromnych klastrach, które pozwoliły im wchłonąć
miliony odpowiedzi na ich dyskusję i~włączyć je do debaty następnego
dnia. ,,Rozmowy Limpopo/Etcetera'' były na swój sposób równie sławne jak
,,Wykłady Feynmana''.

Żadne nigdy publicznie nie wyjaśniło ich rozstania,  ani też nie
powiedziało Tam, o~co chodzi (nie żeby o~to pytała, chociaż płonęła
ciekawością). Utrzymywali to w~tajemnicy tak długo, jak to możliwe -- to
nie było tak, że duchy domu wychodziły razem na kolację -- ale w~końcu
ktoś opublikował podpisany e-mail od Limpopo do Etcetery, w~którym
kazała mu się odpierdolić na zawsze. To było to, natychmiastowe wirusowe
plotki zła, które rozeszły się po całym świecie.

Plotka trwała dłużej niż większość skandali z~powodu pytań, które zadała
na temat symów. Jeśli Limpopo i~Etcetera byli bratnimi duszami, gdy były
zrobione z~mięsa, w~jaki dokładnie sposób symulacja mogła dojść do
punktu, w~którym nienawidzili się nawzajem i~nigdy więcej nie chcieli
rozmawiać?

Tam żałowała, że nie ma wdzięcznego sposobu na poruszenie tego z~Limpopo, wyjaśnienia, że myślała, że to bzdury, większość związków
dobiegała końca, fakt, że dwie osoby się odkochały, można przytoczyć
jako dowód, że sym jest wierny, tak samo jak dowód na to, że był
niedokładny. Ludzie dorastali i~się zmieniali. Prawdziwa symulacja była
wierna swojemu twórcy, a~jaki dziwak nie zmieniłby się po przebudzeniu w~komputerze?

,,Wiele lat'', tak powiedziała.

-- Nie starzeje się z~wdziękiem, prawda? To ironia, że przez tak długi
czas wyglądał tak młodo; to pozwoliło mu udawać, że jest odporny.

-- Nikt z~nas nie może być dokładnie taką osobą, jaką chcemy być. Nie
jestem zachwycona moimi biodrami, nie podoba mi się, że straciłam
widzenie w~nocy\ldots 

-- Czasami można się do tego przyzwyczaić, czasem nie. \textit{Ty} wiesz,
że istnieją pewne rodzaje niedopasowania ciało-umysł, że ludzie po
prostu nie mogą\ldots 

Westchnęła. 

-- Jak sobie radzisz?

-- Być głową w~słoiku? Zderzaki. Chociaż nigdy nie przechodzę w~zawieszenie, czasami wybieram się bardzo wolno, pozwalam sobie marzyć.
Wyłączenie nie byłoby najgorsze przez kolejną dekadę. Widok poklatkowy
byłby odświeżający. Wyobraź sobie, że zawiesiłam się i~zostawiłam
instrukcje, aby nie budzić mnie przez stulecie.

-- Brzmi okropnie.

-- Przemyśl to. Prawie wszyscy, których kochasz, byliby w~pobliżu, w~jakiejś formie. Świat byłby niesamowitym nowym miejscem, plecakami
odrzutowymi i~gównem\ldots 

-- Może wróciłby do stanu defaultu. Jest wiele miast otoczonych murami, w~zestawie Harriery i~szczyty gór. Spędzili cholernie dużo czasu na górze,
kto powiedział, że już tam nie dotrą?

-- Po to właśnie jesteście, wy leniwe dupki, walcząc z~tym gównem. Obudź
mnie, kiedy to się skończy. Lubię tego dźwięk.

-- Mają rację, nie jesteś Limpopo, ona nigdy nie chciałaby przeczekać
akcji.

Nastąpiła dłuższa pauza niż była przyjemna. Tam martwiła się, że
obraziła Limpopo. Już miała przeprosić, a~potem\ldots 

-- Nie, była akcja, którą stara Limpopo chciałaby przeczekać. Nikt nie
jest czysty. Daliście mi tyle świętości za to, że nigdy nie chciałam
rankingów, nigdy nie pozwalaliście nikomu śledzić faktu, że wykonuję
całą tę ciężką pracę, ale to nie dlatego, że nie pragnęłam punktów
brownie. To było \textit{dokładnie} to, bo walczyłam o~uznanie, którego
odmawiałam. Każdy dzień był walką o~zmiażdżenie tej części mnie, która
chciała siedzieć na złotym tronie i~być noszona po miejskim placu.

-- Każdy pragnie uznania, Limpopo. Spójrz na dzieci\ldots 

 W~domu było
jedenaścioro dzieci, od sześciu matek: dwie fabryki ślinienia, które
dopiero co zaczęły przesypiać noc, a~potem gładka krzywizna, która
zwężała się w~wieku dwunastu lub trzynastu(nigdy nie umiała zapamiętać,
miały one sprzeczną właściwość bycia niemożliwie młodymi i~zawsze
znacznie starszymi, niż pamiętała). 

-- Zawsze chcą uznania za swoją
pracę.

-- Chcą też zmonopolizować uwagę rodziców, są ślepe na bałagan, a~małe
nie trzymają moczu. Stan dzieciństwa ma wiele zalet, ale dlatego, że
dzieci coś robią, nie wynika z~tego, że powinniśmy do tego dążyć.

-- Miałaś już tę dyskusję wcześniej.

-- Były tutaj dzieciaki tak długo, jak odchodzące. Zawsze byli rodzice,
którzy stwierdzili, że ryzyko wycofania swoich dzieci z~defaultu jest
mniejsze niż ryzyko pozostawienia ich tam. Rzeczy związane z~,,odpowiedzialnością'' w~szkołach przyspieszyły to, raz, gdy zaczęli
płacić nauczycielom na podstawie wyników testów, rodzice widzieli swoje
dzieci nieustannie wypychane przez system, brak miejsca na pomaganie im
w ich problemach czy pasjach. Potem zagrozili rodzicom \textit{więzieniem}
za to, że nie posyłają dzieci do szkoły\ldots 

-- Tak naprawdę tego nie zrobili!

-- Tam, wiem, że nigdy nie zwracałaś uwagi na rodzicielstwo i~dzieci, ale
to nie mogło umknąć Twojej uwadze. To był ogromny skandal, nawet jak na
ówczesne standardy. Most za daleko dla wielu rodziców. Było kilka dużych
procesów sądowych. Słyszałeś kiedyś o~Augurach?

-- Dzwoni dzwonek, co?

-- Oboje rodzice wychowywani przez osoby, które przeżyły szkołę
stacjonarną, widzieli, że ich córka jest nieszczęśliwa, postanowili
zabrać ją na edukację domową, chcieli połączyć ją z~jej dziedzictwem
Pierwszych Narodów, ale odmówili zakupu oficjalnych materiałów do nauki
w domu lub zapłaty za standaryzowane testy dla nauczania domowego.
Wsadzili ich do więzienia.

-- Trochę sobie przypominam.

-- To było wielkie. Liczba rodziców, którzy odeszli, to było wtedy, gdy
zrobiliśmy pierwszy żłobek w~B\&B, musieliśmy zaadaptować naczynia dla
uchodźców z~Trzeciej Arabskiej Wiosny, zmusić wszystkie faby do
sprawdzania bezpieczeństwa zabawek i~zamontować przewijaki w~każdym
miejscu.

-- Przed moim czasem. Byłam wtedy na uniwersytecie.

-- Dobrze.

-- Były tam dzieci, ale nie w~mojej grupie. Tłum LGBT, myślę, że był
trochę toksyczny dla ludzi, którzy chcieli dzieci, to bzdury o~,,hodowcach'', które wydają się zabawne, gdy jesteś dzieckiem, ale z~perspektywy czasu są gówniane. Wyobraź sobie, jak czułyby się Gretyl i~Lodołasica, gdyby musiały usłyszeć, jak mówimy w~ten sposób.

Lodołasica urodziła dwoje dzieci, obu chłopców, bez większych dramatów,
chociaż Gretyl była kłębkiem nerwów podczas obu porodów i~za każdym
razem musiała opuścić pokój. Chłopcy mieli, ile, sześć i~osiem? Pięć i~osiem? Była gównianą, honorową ciotką, chociaż kochała ich oboje w~abstrakcyjny, ostrożny sposób, który trzymał się z~daleka od ich glutów,
śliny i~zniszczenia.

-- W~przyszłym tygodniu są urodziny Stana.

-- Jak to robisz?

-- Co?

-- Śledzisz urodziny wszystkich?

-- Jestem duchem domu. Część pracy. Ustawianie przypomnień, uruchamianie
ich, gdy pojawi się jakiś temat, dodawanie kontekstu za rogiem. Każdy
dom to robi.

-- Ale nie jesteś zlepkiem kodu, jesteś osobą. Inaczej jest, gdy
rozmawiasz z~kimś, a~ta osoba po prostu doskonale sobie przypomina, jak
wszystkie drobiazgi pojawiają się w~kontekście.

-- Możesz to mieć. Po prostu zrób sobie oczy. -- Była teraz prawie
całkowicie ślepa w~nocy, musiała powiększyć tekst do bardzo dużego
rozmiaru, żeby go przeczytać. Wiele osób przeszło operację, wszczepiono
im wyświetlacze w~tym samym czasie, wszystkie paski i~bzdury o~rozszerzonej rzeczywistości, po których żyli goglarze, ale bez gogli.
Jeszcze tego nie zrobiła, ponieważ technika szybko się poprawiała. Jeśli
zamierzała umieścić laserowy skalpel w~pobliżu gałek ocznych, chciała
się upewnić, że jest to pierwszy i~ostatni raz i~nie musi wracać w~przyszłym roku po kluczową modernizację. Wstrzymywała się, dopóki jej
wzrok był do zniesienia. 

-- Dla przypomnienia, ,,nie kod, osoba'' to
filozoficzna kwestia, o~której mogłybyśmy rozprawiać godzinami, chociaż
jestem tym zmęczona.

-- To też nie moja sprawa. -- Chociaż często o~tym myślała. -- Rozmowa z~Tobą nie jest jak rozmowa z~kimś, komu jakiś głupi algorytm podaje dripy
do HUDa. Z~Tobą to naturalne.

-- Jeśli jest jedna rzecz, którą nie jestem, to jest to bycie naturalną,
ale dziękuję.

Ziewnęła, sprawdziła godzinę. 

-- Czwarta rano, cholera. No cóż, śpiochy w~końcu przybywają, powinnam spuścić głowę i~sprawić, żeby byli mile
widziani.

-- Masz to. Kocham cię, Tam.

-- Też cię kocham. -- Miała to na myśli, wiedziała, że Limpopo to miała na
myśli. Kochała i~była kochana w~każdym odosobnionym miejscu, ale to był
pierwszy dom, który ją kochał.

Przytuliła się do Setha i~objęła go za brzuch, pocałowała go w~plecy,
gdzie rzadkie siwe włosy łaskotały ją w~nos. Bolały ją biodra. Zamknęła
oczy, znalazła sen.

Obudziła się trochę, kiedy Seth wstał kilka godzin później. Na wpół
przeanalizowała odgłosy wkładania przez niego kapci i~pidżamę,
dowiadywania się o~najbliższej wolnej toalecie, poczuła, jak wraca,
siada na łóżku i~patrzy na nią. Uśmiechnęła się lekko. Wymamrotał: 

-- W~porządku, śpisz -- i~ścisnął jej rękę, pochylił się powoli i~stękając,
pocałował ją w~czoło, a~potem w~usta, zarost szorujący jej skórę.

Potarł jej plecy, a~ona jęknęła z~uznaniem, tylko dla radości ludzkiego
kontaktu w~senny poranek.

-- Idę na śniadanie -- mruknął. Odwróciła głowę i~pocałowała jego palce.

-- Dobra.

-- Kolejna zła noc, co?

-- Po prostu bezsennie. Nie jest źle.

-- Śpij. Nie ma znaczenia, kiedy śpisz.

-- Dobrze. -- Naciągnęła kołdrę na głowę.

Historie pomogły jej zasnąć. Otworzyła jedno oko, wytarła powierzchnię o~wezgłowie łóżka i~postukała w~nagranie starej powieści Terry'ego
Pratchetta, tej o~założeniu gazety Świata Dysku. Słuchała jej tysiąc
razy i~mogła słuchać tysiąc razy więcej i~pozwalała czytelnikowi nieść
ją w~sen.

Dryfowała na słowach i~maślanym słońcu, które przeciekało przez brzegi
polaryzacyjnej folii w~oknie, czasami budziła się trochę własnym cichym
chrapaniem, a~potem\ldots 

-- Tam?

Usiadła sztywno wyprostowana. Głos Setha nieczęsto dochodził do takiego
poziomu paniki. Była całkowicie rozbudzona, patrzyła na niego, stojącego
w ich drzwiach, ciężko oddychającego, z~szeroko otwartymi oczami i~rzadkimi włosami sterczącymi pod kątem szalonego profesora. Trzymał
zapomniany kawałek tosta.

-- Jezu, co jest?

-- Limpopo jest na telefonie.

Zamrugała, zdezorientowana.

-- Seth?

-- \textit{Prawdziwa }Limpopo. Przepraszam, \textit{żyjąca} Limpopo. Ona
żyje, to właśnie mówię. Ona żyje. Rozmawia przez telefon.

Przyłożyła dłonie do policzków, głupi sposób, by zarejestrować
zdziwienie, ale tam było.

-- Limpopo żyje?

-- Ona rozmawia z~Limpopo. -- Zauważył tost w~swojej dłoni, spojrzał na
niego i~odłożył. Zabrała go i~ugryzła. Posmarowano go masłem, drożdżami
piwnymi i~tabasco, platońskie idealne śniadanie Setha.

-- Jezus. -- Znalazła szlafrok na podłodze, włożyła kapcie i~dokończyła
toast Seta. -- Chodź.

W największym pokoju wspólnym było już pięcioro innych, którzy wyglądali
na oszołomionych i~podekscytowanych, słuchając w~milczeniu.

-- Nigdy nie pozwalali Ci pisać do nikogo? -- To był głos Limpopo,
\textit{ich} Limpopo. Ducha domu.

-- Nigdy. Nie byłam jedyna. Są, była? \ldots  grupa z~nas w~segregacji
politycznej, żadnych gości, żadnych wiadomości z~nikim na zewnątrz.
Przetrzymywane pod innymi nazwiskami. -- Ten głos też był Limpopo, ale
starszy, głos starszej pani, głos Limpopo, która żyła, gdzie? \ldots  od
ponad dekady.

-- Ale teraz\ldots 

-- Teraz więźniowie prowadzą azyl. -- Zabrzmiała oszołomiona. -- Były trzy
dni, kiedy było naprawdę źle. Prawie żaden strażnik się nie pojawił. Ci,
którzy to zrobili, byli zbyt przerażeni, by zrobić cokolwiek poza
kuleniem się w~swoich sterowniach i~szczekaniem na nas przez głośniki.
Nawet tego trzeciego dnia.

-- Wczoraj o~północy, trzask, wszystkie drzwi się otworzyły. Żadnych
strażników. Brak personelu administracyjnego. Nic. Oczywiście wszyscy
głodowali. Znaleźliśmy drogę do kawiarni, kiedy już zorientowaliśmy się,
co się dzieje. Niektórzy z~nas utworzyli ad hoc komitet kuchenny,
zmusili faby do działania i~nakarmili wszystkich. Potem ktoś zapytał się
o ochotników, żeby sprawdzili kliniki i~zaczęli opiekować się chorymi
najlepiej, jak tylko mogli. Dużo tu pielęgniarek i\ldots  przepraszam, to
chyba nie ma znaczenia. Nikt tutaj nie wie, co wydarzyło się w~default.
Kiedy skorzystali z~pokoju łączności, by zadzwonić do swoich prawników,
powiedzieli, że Corrections Canada dokonała jakiegoś wewnętrznego
zamachu stanu i~nikt tam nie rozmawia ze światem zewnętrznym. Mówią, że
to nie pierwsze ministerstwo, któremu się to przydarzyło\ldots  podobno
Veterans Affairs Canada zrobiło to w~zeszłym miesiącu? W~więzieniu nie
dostaję wielu wysokiej jakości wiadomości i~analiz\ldots 

Tam zaczęła rozumieć, co mówi. Była w~więzieniu. Więzienia pękły.
Pęknięcie to słowo, którego używali na określenie instytucji rządowych,
które rozpadły się, zamieniły się w~spółdzielnie w~stylu odchodzących,
które rozdawały materiały biurowe i~otwierały bazy danych dla każdego,
kto chciał włamu. Słyszała o~pękniętych szpitalach, wydziałach policji,
mieszkaniach komunalnych, ale więzienia były nową wiadomością. Wielką.

-- Limpopo -- powiedziała.

Obie odpowiedziały, co byłoby zabawne i~mogło być później.

-- Przepraszam, nie duch domu, żyjąca.

-- Kto to?

-- To Tam.

-- Tam? Nie ma kurwa mowy! Tam! Nadal tam jesteś? Nadal z~Sethem?

Uśmiechnęła się i~ścisnęła Setha za rękę.

-- Tak, on też tu jest.

-- Ty biedny skurwielu. -- Wszyscy wiedzieli, że żartuje, nawet Seth.

-- Przeszkoliłam go. Zestarzał się i~jest powolny, a~ja jestem złośliwa.

-- Nie wierzę w~to.

-- Gdzie jesteś, Limpopo? Fizycznie?

-- W~pobliżu Kingston, trochę na północ. Koło Joycetown. Więzienie dla
Kobiet w~Kingston.

-- Czy jesteś bezpieczna?

-- Masz na myśli, czy mordercy mają przyjść i~mnie zabić? Nie, żebym
mogła zobaczyć. Nie martwię się tym. Jest tu mnóstwo dziwnych ludzi, ale
jest też mnóstwo podejrzanych ludzi na zewnątrz. Większość z~tych kobiet
to moje przyjaciółki. Niektóre są jak siostry.

-- Czy możemy przyjść i~cię zabrać?

-- Co masz na myśli?

-- To znaczy, czy możemy przyjść i~cię tu przywieźć? Pocahontas wciąż tu
jest, Gretyl, Lodołasica i~ich dzieci, Wielkie Koło, Małe Koło, a~nawet
Kersplebedeb, chociaż teraz ona nazywa się Noozi\ldots 

-- Poczekaj chwilę. Nie wiem, kim jest połowa tych ludzi. Cholera, nawet
nie wiem, \textit{gdzie }jesteście\ldots 

-- Gary.

-- Ja też nie wiem, kim on jest.

-- Gary, Indiana. Ładne miejsce. Światowi liderzy w~przywracaniu budynków
z martwych. Skolonizowana cegła, sprytne kratownice, wielkie stare
lokale, które nie były konserwowane od pięćdziesięciu, siedemdziesięciu
pięciu lat.

-- Stan, który zaczyna się i~kończy samogłoską? Żartujesz sobie.

-- Pokochałabyś to miejsce, Limpopo. Jesteś bohaterem.

-- To prawda -- powiedziała Limpopo-Duch-Domu. -- Jesteś tutaj świętą.

Druga Limpopo jęknęła. 

-- Dobijasz mnie.

-- Przepraszam, Limpopo -- powiedział Seth. -- Myśleliśmy, że nie żyjesz.
Męczeństwo było na porządku dziennym.

Znowu jęknęła.

-- Poważnie -- powiedziała Tam. -- Przyjdź i~zobacz. Albo wszyscy
przyjdziemy się z~Tobą zobaczyć. Nie obchodzi mnie, w~którą stronę.
Kochamy Cię. \textit{Tęskniliśmy} za Tobą.

-- Hej!-- -- powiedział duch domu Limpopo.

-- Tęskniłam za przytulaniem i~trzymaniem Ciebie -- powiedziała Tam. -- I~powinnaś poznać tę drugą Limpopo, naszą Limpopo, jest cudowna.

-- Nie podlizuj się -- powiedział duch domu. -- Przez tydzień będziesz
dostawać mysie łajno w~płatkach kukurydzianych, dupo.

Druga Limpopo się roześmiała. 

-- Brzmi jak mój typ. Dosłownie, jak sądzę.
Kurwa, kto by pomyślał, że w~tym tygodniu może być jeszcze dziwniej.

Na schodach rozległ się łoskot i~dudnienie stóp, a~Gretyl i~Lodołasica
wpadły do pokoju, poprzedzane przez swoich chłopców, którzy byli
kometami smarków i~zniszczenia, kłócąc się o~zabawkę, nawet gdy weszli
przez drzwi, mały ciągnąc włosy tego dużego. Gretyl płynnie wysunęła
palce z~kędzierzawego mopa, wciągnęła go w~powietrze i~postawiła z~dala
od starszego brata.

-- Ona żyje? -- powiedziała Lodołasica, chwytając większego i~kołysząc
nim, roześmiał się i~odrzucił głowę do tyłu.

-- Teraz z~nią rozmawiamy -- powiedziała Tam. -- Limpopo, Lodołasica i~Gretyl są tutaj.

-- Lodołasica żyje?

Lodołasica roześmiała się. 

-- Myślę, że mamy dużo do nadrobienia.

Młodszy chłopak nagle spojrzał na nią poważnie i~odgarnął włosy z~oczu.

-- Nie jesteś martwa, mamusiu.

-- Nie jestem martwa. Nie martw się, Jacob.

-- Mamusiu?

-- Jest ich dwóch -- powiedziała Gretyl. -- Chłopcy. Jacob ma siedem, a~Stan dziesięć. Przywitajcie się z~Limpopo, chłopcy.

-- Limpopo? -- Jacob wykrzywił twarz. -- Duch domu?

-- Nie, inna Limpopo. Jest daleko i~dawno jej nie widzieliśmy. Kochamy
ją.

Jacob wzruszył ramionami. Stan przewrócił oczami na powolne wchłanianie
młodszego brata. 

-- Cześć, Limpopo! Cześć, inna Limpopo!

Daleko Limpopo zaklęła z~wyobraźnią, co spowodowało, że oczy obu
chłopców rozszerzyły się i~na twarzach pojawiły się uśmiechy. Tam
widział, jak przechowują słowa do przyszłego wdrożenia. 

-- Cześć,
chłopaki. Cześć, Gretyl. Witaj, Lodołasico. Dobrze jest słyszeć, że
wszyscy żyją i~dobrze się rozwijają.

-- To jak będzie -- powiedziała Tam. -- Przychodzimy do ciebie, czy Ty do
nas? Ponieważ, kochanie, mamy trochę do nadrobienia, a~z~tego, co wiemy,
default pozbiera się i~wejdzie tam i~zabije lub zamknie każdego z~was.

-- To możliwość, którą rozważaliśmy. Jest jeszcze jedna rzecz, tokeny
uwierzytelniania ,,root'' zostały pozostawione w~centrum kontroli przez
strażnika, tak sądzimy. Mamy to miejsce zdobyte od dupy do mózgu. A
jeśli chodzi o~więzienie, to jest tak samo dobre w~trzymaniu ludzi z~dala, jak w~trzymaniu ludzi w~środku. Każdy, kto chce zabrać to miejsce,
będzie się świetnie bawić.

Tam przygryzła wargę. Wszyscy spojrzeli na wszystkich. Nawet chłopcy
zachowywali się cicho. 

-- Limpopo. Nie chcemy cię skrzywdzić. Jesteśmy
odchodzącymi. Jest mnóstwo dużych, głupich budynków instytucjonalnych,
które Ty i~Twoi przyjaciele możecie zająć.

-- \textit{Bzdura}. -- Zaskoczyła ich swoją gwałtownością. -- Ukradli nam
życie. Zamknęli nas. Zasłużyliśmy na to miejsce. To jest nasze. Jeśli
odejdziemy, jeśli się podzielimy, złapią nas pojedynczo, jedna osoba na
raz. Nigdy nie będziemy niczyimi więźniami, nigdy.

-- Zamierzasz zostać w~więzieniu, żeby nie zostać więźniem? -- Usta Setha,
jak zawsze, wyprzedziły jego zmysły.

-- To nie żart. Kupiliśmy to miejsce krwią, naszym życiem. To jest nasze.
To była nasza niewola. Teraz to nasza wolność.

-- Limpopo -- powiedział cicho Lodołasica. -- Już tak nie jest. Default nie
jest default. Wiem, jak to było. Wyglądało to na wojnę, mieli nas
zamknąć albo zabić. Zmieniło się. Zetty poszły na wojnę ze sobą,
walczyli o~kontrolę nad krajami, których ludzie nie chcieli walczyć po
żadnej stronie, odeszli z~nami, zamienili życie uchodźców w~standard. To
ludzie, którzy zostali w~jednym miejscu i~twierdzili, że kawałek
nieruchomości jest nikogo innego, stali się dziwactwem. Wszyscy inni
ruszyli w~drogę, kiedy ci ludzie się pojawili.

-- Bzdura -- powiedziała Limpopo. -- Może w~Twoim zakątku świata. Państwo
po prostu nie obumiera. Ktoś płacił pensje tym strażnikom przez te
wszystkie lata, dostarczał kisiel zasilający nasze faby. Zwycięstwo nie
jest czymś, co kiedykolwiek będą miały odchodzące. Odejście nie jest
zwycięstwem, to po prostu nie jest przegrana.

-- Nie przegrałyśmy -- powiedziała Lodołasica. -- Są enklawy ludzi, którzy
udają, że jest normalnie i~niedługo wszystko wróci do stanu, jak było
lub powinno być. W~dzisiejszych czasach nie chodzi o~konflikty zbrojne,
ale o~wojnę o~normy, które z~nas jest normalne, a~kim są szaleni
radykałowie. -- Przerwała. -- Słyszałeś o~inwazji na Irak?

-- Nowej czy jednej ze starych?

-- Całkowicie nowej. Iran miał dokonać inwazji na Irak, bo cholera, tak
się dzieje od dawna. Tyle że tym razem tak się nie stało. Piloci,
których wysłali do Iraku, nie zrzucali bomb, lądowali na kurdyjskich
lądowiskach. Piechota, gdy tylko dotarli do linii frontu, odmówiła
walki. Grupy oficerów też. Każdy jest trochę przerażony. Strona iracka
wydaje rozkaz wybicia gówna z~tych dziwacznych najeźdźców. Zamiast tego
\textit{ich} żołnierze też odmawiają. Tym, którzy próbują walczyć, ich
kumple zabierają im broń. Poważnie!

-- To zbyt dziwne, aby mogło być prawdziwe.

-- Tylko dlatego, że nie powiedziała ci najlepszej części -- powiedziała
Gretyl.

-- Wszyscy byli odchodnikami -- krzyknął Jacob. -- Tak jak my!

-- Sposób na załatwienie mojej puenty, dzieciaku. -- Lodołasica
przerzuciła go na biodro i~pocałowała czubek jego nosa. -- To jest
legenda w~okolicy. Po całej Zatoce Perskiej działa odchodnicka grupa
pokrewieństwa, grafen, który działał przez te same sieci, których
wszyscy używają, aby ominąć krajowe firewalle, więc przekazali historie.
Kiedy ludzie po obu stronach zorientowali się, że zostaną wysłani, by
się nawzajem zabić, postanowili jebać ten hałas i~opracowali plan.

-- Jebać ten hałas! -- Jacob uderzył pięścią w~powietrze. Stan przewrócił
oczami. Tam była pewna, że żałował, że nie wykorzystał okazji i~bezkarnie zdetonował bombę ,,J''. 

Gretyl i~Iceweasel upierały się, że
chłopcy nigdy nie nauczą się poprawnie przeklinać, jeśli nie będą mieli
dobrych wzorów do naśladowania. Dlatego polecono im uważnie obserwować
przekleństwa i~nie próbować ich, dopóki nie upewnią się, że mają rację.
Kiedy tego spróbowali, zostali poddani wstydliwemu osądzaniu i~trenowaniu w~zakresie przekleństw. To było bardziej skuteczne w~ograniczaniu ich języka niż cokolwiek, co inni rodzice próbowali na
swoich dzieciach.

-- To niesamowite, w~porządku -- powiedziała Limpopo. -- Dlaczego
generałowie nie rozwalili ich wszystkich z~dronów? Powstrzymali
rozprzestrzenianie się zgnilizny?

-- Jest plotka, że obie strony wydały rozkaz, a~operatorzy dronów
odmówili i~nikt nie chciał robić z~tego problemu. Ostatnią rzeczą,
jakiej chce generał, jest odkrycie, że dowodzi armią jednego, pośród
armii wszystkich innych.

-- Jak dawno temu to się stało?

-- Ile to było, rok temu? -- powiedziała Lodołasica.

-- Osiem miesięcy -- powiedziała Tam.

-- Cóż, cholera. Imponujące. Nie mamy tu zbyt wielu wiadomości.

-- Chodzi o~to, że nie wiesz, co się wydarzy, nie możemy wiedzieć, ale są
powody do optymizmu. Ludzie są zmęczeni strzelaniem do siebie.

Tam zachichotała. 

-- Nie wiem, czy posunęłabym się tak daleko.
Istnieje\ldots  -- Poszukała słowa. -- \ldots  wiarygodność odchodzących.
Poczucie, że mamy to obmyślone. Kiedy już zdasz sobie sprawę, że
istnieje świat, który \textit{chce }tego, co dajesz, cóż, trudno jest
przekonać ludzi, by zabijali się nawzajem.

-- Jeb mnie w~dupę -- powiedziała Limpopo, wysyłając Stana i~Jacoba do
chichotów. Z~jej końca dobiegł jakiś hałas, stłumiona rozmowa. -- Muszę
się zastanowić, a~nie ma tu zbyt wielu elementów związanych z~interfejsem, więc muszę dać komuś innemu szansę. Siedźcie spokojnie,
zadzwonię jutro, dobrze?

-- Jasne -- odparła Tam, a~duch domu powtórzył za nią chwilę później.

Wszyscy się pożegnali. W~pokoju zapadła cisza, poza świstem wdechu i~wydechu zasmarkanego nosa Jacoba.

-- Nie będziesz czekać, aż oddzwoni, prawda? -- powiedział duch domu.

-- Posrało Cię? Nie ma mowy -- powiedziała Lodołasica.

-- Chcesz spakować dla dzieci, czy ja powinnam? -- powiedziała Gretyl.
Chłopcy zorientowali się chwilę później, wymienili podekscytowane
spojrzenia i~zaczęli biegać w~kółko.

-- Ty to zrób. Rozejrzę się za miejscami do spania w~pociągu.

-- Sprawdź sterowce. -- Seth też się podskakiwał. -- Ostatnio wiatry
sprzyjają północnemu wschodowi, założę się, że możemy złapać dłuższą
przejażdżkę.

-- Dobre myślenie -- powiedziała Lodołasica. -- Chłopcy, chcecie pojechać
sterowcem?

Obaj chłopcy bełkotali i~krzyczeli. Wtedy Jacob był tak podekscytowany,
że uderzył Stana ,,ponieważ''. Upadli na podłogę, bijąc pięściami i~krzycząc.

Ich mamy wymieniły spojrzenie, przepraszająco pokręciły głowami reszcie
dorosłych. 

-- Próbujemy pozwolić im rozwiązać te samodzielnie -- powiedziała Gretyl. -- Przepraszam.

Wszyscy pozostali byli w~zbyt dobrych humorach, żeby im przeszkadzać.
Tam spojrzała ze zdumieniem na współlokatorów, dalszą rodzinę i~zdała
sobie sprawę, że zaraz znowu zacznie odchodzić.

\chapter*{ii}
Rozkłady pociągów były do niczego. Istniał skomplikowany algorytm, który
określał, ile wagonów i~kiedy umieścić na których liniach. Był
nieskończenie skłócony z~różnymi modelami, które różnie ważyły
priorytety. Gretyl została wciągnięta w~matematykę, znikając w~zbiorze
anonimowych forów dyskusyjnych, na których to wykuwano, a~Lodołasica
wysłała wiadomość do Tam, aby powiedzieć, że prawdopodobnie utknie w~tej
szczurzej dziurze na najbliższy czas. Więc Tam powinna zacząć poszukiwać
alternatyw.

W tym kierunku jechały wspólne przejazdy, ale musiałyby podzielić się na
podgrupy i~zebrać ponownie na przystankach. To było coś, co można było
zautomatyzować (Tam pomogła Lodołasicy w~wycieczce dla dzieci do Akron
Memorial w~zeszłym roku i~uznali to za łatwe), ale pojazdy naziemne były
powolne.

-- Musisz znaleźć trzmiela -- powiedział Seth.

-- Tak -- powiedziała Tam. Postukała w~swoje powierzchnie interfejsu,
upewniając się, że duch domu jest zablokowany. -- Ale to niewygodne.

-- Etcetera jest moim przyjacielem -- powiedział Seth. -- Mój najstarszy
kumpel. To, że on i~Limpopo nie mogą się znieść, nie oznacza, że musimy
opowiadać się po którejś ze stron. Nie zdradzisz jej, przyjaźniąc się z~nim. Gdybyś ją zapytała, powiedziałaby ci.

-- Gdybym ją zapytała, postawiłabym ją w~sytuacji, w~której musiałaby mi
powiedzieć, że nie ma nic przeciwko, nawet gdyby miała. Dlatego jej nie
pytam. Przyjaciele nie stawiają przyjaciół w~takiej sytuacji.

-- Gdyby wiedziała, że wstrzymujesz się z~rozmową z~nim, bo martwisz się,
że ją zdenerwujesz, byłaby oburzona.

-- Nie wątpię. Dlatego jej nie mówię.

-- Nie sądzisz, że to wszystko jest\ldots  pokręcone? Zwłaszcza że istnieje
Inna Limpopo -- zdecydowali się na to, ponieważ pomimo najmniejszej
niezręczności wszyscy zgodzili się, że ,,Prawdziwa Limpopo'' to
gówniane, najgorsze rozwiązanie -- która była zakochana w~Etcetera i~cieszyłaby się, że znów z~nim porozmawia.

Westchnęła i~przetarła oczy. Od dłuższego czasu wpatrywała się w~ekrany.

-- To jest do bani. I~co z~tego? Wiele rzeczy jest do bani. Życie nie
poprawia się przez bycie kutasem dla ludzi, którzy cię kochają.

-- Etcetera cię kocha.

-- Odpieprz się.-- Pozwoliła mu pomasować sobie ramiona. -- Aaa.

Znalazł
węzeł na jej prawym ramieniu, sęk uporczywego bólu, który wydawał się
taki dobry, gdy wbijał w~niego kciuki.

-- Właśnie tam. -- Skręciła głowę.

-- Jesteś łatwa. Mogłem wygrać każdą walkę, wkładając kciuk w~ten węzeł.

-- To mój kryptonit. Nie nadużywaj swoich mocy.

-- Zadzwonię do Etcetera.

-- Pieprz się. -- Przytuliła głowę do jego brzucha, wpychając bolący węzeł
na ramieniu z~powrotem do jego kciuka.

Pięć minut później zadzwonił do Etcetera.

-- Minęło trochę czasu -- powiedział Etcetera.

-- Słusznie. To wszystko, co wiesz w~okolicy.

-- Tęskniłem. Za wami obydwojgiem. Wszystkimi wami. Bycie pariasem jest
do bani.

-- Przepraszam -- powiedział Seth. To uczyniło go nieszczęśliwym.
Wymrożenie jego najstarszego przyjaciela było dla niego trudne, ale
nigdy nie narzekał.

Niezręczna cisza.

-- Potrzebujemy Twojej pomocy.

Więcej ciszy.

-- Spodoba ci się to.

-- Dostaliśmy telefon. Z~więzienia. W~Kanadzie. Od więźnia, który był tam
przetrzymywany przez ponad czternaście lat, uwolnił się tylko dlatego,
że strażnicy otworzyli drzwi celi i~odeszli.

-- Seth\ldots  -- Coś w~głosie Etcetery, emocja równie nieomylna, co
niezrozumiała. Jakieś hybrydowe uczucie człowiek-maszyna. Głęboko
odczuwalne. Bez nazwy.

-- Limpopo -- powiedział Seth.

Rozległ się najdziwniejszy dźwięk, jaki Tam kiedykolwiek słyszała. To
trwało i~trwało. Myślała, że to śmiech. Z~przerażeniem zdała sobie
sprawę, że to szloch. Jedyny raz, kiedy usłyszała szloch symulacji, to
było w~tunelach w~Uniwersytetu Odchodzących, zanim zorientowali się, jak
je ustabilizować. Były to symulacje dźwiękowe, które powstały, zanim się
zawaliły.

-- Etcetera? W~porządku, kolego.

Płakał długo.

-- Wszystko będzie dobrze? -- powiedział Seth przy przycichaniu. -- Mogę
sprowadzić Gretyl, ona może pomóc z~Twoimi ograniczeniami\ldots 

-- Nie potrzebuję pomocy. Czy z~nią wszystko w~porządku?

Nie miał na myśli Gretyl. 

-- Brzmi niesamowicie. Ogniście. Gniewnie. Chce
walki.

-- Ja też chcę walczyć. Czego potrzebujesz?

-- Nadal masz kontakty, kto może zdobyć sterowca?

-- Idziecie do niej?

-- Nie przyjdzie do nas, jeśli przyjdą ją zamknąć, będzie walczyć.

-- Jebana racja.

-- Możesz pomóc?

-- Idę. Znajdź mi klaster i~zanieś go. Idę z~wami.

-- Możesz po prostu zadzwonić -- powiedziała Tam. Miała dość komplikacji.

-- Nie, jeśli zabiją sieć. Zostawię tutaj kopię zapasową. Ale idę z~wami.

-- Etcetera -- powiedziała Tam swoim najbardziej rozsądnym głosem.

-- Idę z~wami.

Seth potrząsnął głową, bezgłośnie mówiąc \textit{zaakceptuj to}.

-- Jedziesz z~nami -- powiedziała.

-- Pakujcie się -- powiedział.

\chapter*{iii}
Trzmiel wylądował następnego dnia na parkingu starego centrum handlowego
w zachodniej części miasta, z~załogą szczerzących zęby starych
Brazylijczyków, mężczyzn z~dredami w~przerzedzonych włosach, kobiet
kroczących pewnie jak marynarze. Stan i~Jacob zostali natychmiast
adoptowani przez dzieci załogi, których status był nieco tajemniczy,
pochodzili z~sierocińca w~Recife, któremu zabrakło funduszy. Dzieciaki
znalazły się w~prowizorycznym obozie, który nie szedł dobrze, a~ci
awiatorzy przyjęli je i~zabrali do swoich ogromnych, pięknych sterowców,
ozdobionych jak legendarne balony baloeiro, które od wieków krążyły po
brazylijskim niebie.

Jednak, gdy te dzieciaki wylądowały na niebie, znalazły się jak ryby do
wody. W~ciągu kilku minut Stan i~Jacob wspinali się boso na takielunek,
ledwo wykrzykując ,,do widzenia'' matkom, które obserwowały ich z~niepokojem i~dumą.

Walczyli z~pakowaniem. Minęło tak wiele czasu, odkąd byli dobrowolnymi
uchodźcami, jeszcze dłużej, odkąd byli niedobrowolnymi uchodźcami.
Naradzali się w~swoich świetlicach. Ustawili swój minimalny ładunek,
używając duchów domu, aby śledzić, kto przynosi co, aby ograniczyć
duplikowanie. Małżonkowie, dzieci i~współlokatorzy układali coraz więcej
rzeczy na stosie do spakowania. Śmiali się nerwowo. Nie stali się
\textit{szleperami}, prawda?

Seth i~Lodołasica dzielili wesołość i~przerażenie. Opowiedzieli historię
o tym, jak Limpopo zaprojektowała zbycie ich ziemskich posiadłości
pierwszego dnia w~pensjonacie. Limpopo Duch Domu zakrztusił się i~sprzeciwił się, że nie zrobiła czegoś takiego. Odbyli udawaną walkę,
która była nieco śmiertelnie poważna. Włóczyli się i~targowali, aż każda
z nich miała małą paczkę i~jeszcze jedną torbę dla dwóch chłopców,
których niesamowita zdolność do brudzenia nawet najbardziej odpornych na
brud tkanin była zrównoważona ich obojętnością na własną czystość.

-- Będą brudni -- powiedziała Gretyl. -- Przeżyją. Dobre dla układu
odpornościowego.

Na pokładzie \textit{Gilberta Gila} zdali sobie sprawę, że mogli zabrać
dziesięć razy więcej. Brazylijczycy właśnie zrzucili ładunek wysokiej
jakości plastiku spolimeryzowanego z~toksycznego bagna na Florydzie
przez inteligentne bakterie. Z~ładunku pozostał tylko zapach, nie do
końca nieprzyjemny. Przypominało to Lodołasicy opakowanie naprawdę
ekskluzywnych kosmetyków, które lubiła jej matka.

Kręcili się wokół ogromnej ładowni wielkości hangaru, pracując z~awiatorami, aby zmienić jej konfigurację na miejsca do spania, wbijając
panele w~rowki w~podłodze i~dopasowując nad nimi sekcje dachu, aby
zbudować wioskę heksajurt. Lodołasica cieszyła się, że nie przynieśli
więcej. Istniała szansa, że na tej wycieczce będą chodzić, naprawdę
chodzić, \textit{po odchodnicku}. Chłopcy będą dostatecznym kłopotem, nie
mając zbyt wiele do noszenia. Trzmiel miał sprzyjające wiatry, aby
zabrać ich aż do wodospadu Niagara, a~nawet do Toronto. Ale nie bez
powodu nazywano je ,,trzmielami''. Gdyby Stary Człowiek Zmiany Klimatu
dał im jedną ze swoich codziennych tysiącletnich burz, musieliby znaleźć
inne rozwiązania.

Gretyl poszła po śpiwory, używając ducha domu \textit{Gila}, by powiedzieć
jej, gdzie wszystko jest schowane. Duchy domu wywodziły się od
firmware'u, które zasilały B\&B, mieszanka kwatermistrza, liczącego i~spowiednika, zaprojektowanego, aby pomóc wszystkim wiedzieć wszystko, co
jest potrzebne. Była tak zafascynowana paleolityczną wersją B\&B. Teraz
były wszędzie, niektóre z~nich były nawet zasilane przez żyjących
zmarłych, jak Limpopo w~Gary. To było zbyt dziwne, nawet dla niej. Mogła
rozmawiać z~symami, pod warunkiem, że nie myślała o~tym zbyt mocno. Ale
pomysł, by mieć kogoś jako nawiedzonego, który nosił Twój dom jak ciało,
był po prostu \textit{pojebany}.

Etcetera paplał po maszynowo-portugalskim z~awiatorami, którzy z~pomocą
Setha i~Tama składali razem stoły na powitalną ucztę. Kilka lat
wcześniej wszczepiono jej wkładkę douszną, kiedy zaczęła mieć problemy
ze słuchem po ciężkiej gorączce, która przeszła przez cały kraj.
Urządzenie wymamrotało do niej tłumaczenie, które tylko czasami
wkraczało w~sferę bełkotu translacji maszynowej.

Brazylijczycy chwalili się \textit{Gilem}, jego właściwościami nośnymi i~manewrowymi; siła i~odporność nadmiarowych komórek grafenowych; ich
sprawnością jako nawigatorów, umiejących znaleźć dobre wiatry tam, gdzie
nie przewidział żaden algorytm. Etcetera dawał wszelkie oznaki zachwytu,
mówił ze znajomością o~statkach poprzedzających \textit{Gil}, cudownych
rzeczach wychodzących z~Tajlandii, gdzie statki powietrzne różniły się w~jakiś ważny, wysoce techniczny sposób, którego nie rozumiała.

Dzieci przybyły w~samą porę po jedzenie, chociaż sądząc po tym, że
jedzenie było już rozmazane wokół ich twarzy, zostały wprowadzone do
kuchennego faba gdzieś w~głębinach statku. Zebrała od obojga dżemowe
pocałunki, oparła się chęci wyczyszczenia ich twarzy śliną, poznała
nowych przyjaciół w~różnym wieku i~płci. Starszy chłopak imieniem Rui,
wystarczająco duży, by mieć trochę wąsów, jabłko Adama i~mieszankę
pewności siebie w~stosunku do dzieci i~niezręczności w~kontaktach z~dorosłymi, powiedział akcentowanym angielskim, jak wspaniali są jej
chłopcy i~jak ich nauczy wszystkiego, czego potrzebowali, żeby latać.
Podziękowała mu w~absolutnie okropnym portugalskim, zachęcona przez
wszczepione urządzenie. Uśmiechnął się, zarumienił i~pochylił głowę w~sposób, który sprawił, że chciała zabrać go do domu i~wychować.

-- Chłopcy gotowi na lunch? -- spytała Gretyl, podchodząc z~wachlarzem
talerzy, na których znajdowały się mięsne szaszłyki, które pachniały
niesamowicie, przyozdobione feijoadą i~stosami hydroponicznych warzyw.
Chłopcy spojrzeli na siebie z~poczuciem winy, a~Gretyl natychmiast
wytarła słodki, lepki płyn wokół ich ust.

-- Wygląda na to, że już zjedliście deser. Mam nadzieję, że to nie
znaczy, że myślicie, że też nie zjecie lunchu. -- Gretyl była dyscypliną
rodzinną. Gdyby to zależało od Lodołasicy, dzieciaki jadłyby trzy
posiłki dziennie z~lodów i~cukierków. Dołączyłaby do nich. Gretyl
uchroniła ich przed śmiercią z~niedożywienia. Jej słowo było prawem.

Chłopcy skinęli głowami i~wzięli talerze. Rui zapoznał się ze wszystkimi
istotnymi szczegółami ich rodzinnych ustaleń i~poprowadził dzieci do
miejsca przy stole, obiecując, że zjedzą każdy kęs.

Gretyl podała Lodołasicy jeden z~pozostałych talerzy i~znalazły miejsce
przy stole, otoczone przez członków załogi, którzy żartowali i~sprawiali, że czuli się jak w~domu.

-- To niesamowite jedzenie -- powiedziała Lodołasica, goniąc ostatnią
kędzierzawą marchewkę widelcołyżką.

-- Mamy nowe kultury starterowe z~Kuby -- wyjaśniła jedna z~członkiń
załogi. 

Była piękna, wysoka, z~ogoloną głową, talią osy, szerokimi
biodrami i~skórą koloru palonego cukru. Lodołasica i~Gretyl ukradkiem na
nią spojrzały, gdy myślały, że ta druga nie patrzy, a~potem złapały się
nawzajem. Nazywała się Camila. Jej angielski był doskonały. 

-- Programujesz go za pomocą świateł podczas cyklu podziału, co powoduje,
że wyraża różne profile smakowe i~tekstury.

-- To niesamowite -- powiedziała Gretyl.

-- Damy ci trochę do wzięcia, kiedy wyjedziesz. Kubańczycy jedzą jak
królowie.

Na deser był biały pudding, zrobiony z~końcówek zapasów prawdziwego
kokosa i~tapioki wyhodowanej z~kubańskich protów. Ani Gretyl, ani
Lodołasica nie miały wystarczającego doświadczenia z~tapioką, aby
stwierdzić, czy jest wierna, ale była równie smaczna jak lunch i~Camila
zapewniła ich, że nawet rolnik tapioki nie był w~stanie odróżnić.

-- Czy potrzebujecie więcej załogi? -- powiedziała żartobliwie Lodołasica.
-- Chcę tak jeść codziennie.

Camila wyglądała poważnie. 

-- Nie mamy więcej koi dla załogi, przykro mi
to mówić. -- Przyglądała się zatłoczonym stolikom. -- To coś, o~co się
kłócimy. To dobra załoga, dobry statek. Niektórzy z~nas chcą wypączkować
nowy, założyć inną ekipę. Mamy coś tak wspaniałego, że to powinno
rosnąć. Inni mówią, że jest coś w~chemii tej grupy i~jeśli się
rozdzielimy, to zniknie. Dzieci rosną, wiele z~nich myśli, że będą
latać. Potrzebujemy więcej statków.

-- Czy to dlatego jedziesz do Ontario?

Camila skinęła głową. 

-- Bańka sterowców była dawno temu, ale wciąż jest
wielu towarzyszy, którzy wiedzą, jak budować i~chcą pomóc. Twój Etcetera
kontaktował nas z~innymi. Dla wielu jest bohaterem, za jego męstwo dla
\textit{Lepszego Narodu}.

Teraz Lodołasica i~Gretyl wyglądały poważnie. Żadne z~nich nie
wspominało zbyt często o~tym dniu, chociaż w~końcu było to wezwanie dla
odchodzących na całym świecie. Camila zrozumiała.

-- Cóż za czas, aby żyć. Jeśli zbudujemy kolejny statek, powinniśmy go
nazwać \textit{Kolejnymi Dniami Lepszego Narodu}.

-- To okropna nazwa statku -- powiedział Etcetera. Jego głos był
metaliczny i~wycięty z~właściwości akustycznych ich stołu, którego
używał jako głośnika.

-- Nikt cię o~to nie pytał, martwy -- powiedziała Lodołasica.

-- Gadanie o~,,lepszym narodzie'' musi zginąć w~ogniu. Nie robimy już
narodów. Robimy ludzi, robimy różne rzeczy. Narody to rządy, paszporty,
granice.

Camila zastukała w~stół knykciami. 

-- Nie ma nic złego w~granicy, o~ile
nie jest zbyt sztywna. Nasze komórki zatrzymują gaz nośny, tworzą
granice z~atmosferą. Moja skóra jest granicą dla mojego ciała,
przepuszcza dobro i~powstrzymuje zło. Masz swoje granice, jak wszystkie
symy, które zapewniają stabilność i~działanie. Nie potrzebujemy
wszystkich granic, tylko dobre.

Odjechali, intensywnie się kłócąc, dyskusja znajoma światu podniebnemu.
Zamieniła się w~żargon o~,,priorytecie przestrzeni powietrznej'' i~,,immunitecie wiatru'' oraz ,,suwerennych prawach drogi'',
niezrozumiałych dla Lodołasicy i~Gretyl. Wrzuciły swoje talerze do leja,
który wyszarpnął je z~rąk, upewniły się, że Rui spełnił swoją obietnicę,
że chłopcy zjedzą białko i~warzywa. Położyły się przytulone w~heksajurcie na drzemkę. To było kilka pracowitych dni.

Gretyl potarła nosem gardło. 

-- Nie zostawiaj mnie dla gorących
brazylijskich lotniczek.

Lodołasica wygięła szyję. 

-- Wzajemnie.

Zasnęły w~ciągu kilku minut.

\chapter*{iv}

Dotarcie do Toronto zajęło zeppowi nieco ponad dzień, okrążając strefę
wykluczenia miasta, ciągnąc za sobą agresywne drony, które zbliżały się
do iluminatorów, aby je zeskanować i~zrobić zdjęcia. Wiatry nad jeziorem
Ontario były do niczego. Przez resztę dnia musieli wznosić, opadać,
miotać się i~błąkać, dopóki nie złapali wiatru, który zabrał ich do
Pickering. Wszyscy zgodzili się, że to najlepsze miejsce do lądowania, z~dala od paranoicznych zettów zaszytych w~Toronto, upierających się, że
ich naród ma jeszcze wiele dni, zanim będzie gotowy ustąpić miejsca
innym, lepszemu lub nie.

Wylądowali wśród tłumu: awiatorów i~gapiów, którzy pomogli umocować liny
odciągowe i~ustawić rampę na miejscu (żartując o~,,piersiach
bezpieczeństwa'', ale upewniając się, że jest solidna, zanim ktokolwiek
zszedł z~trapu).

Grupa zjazdowa Limpopo zeszła po rampie, mrugając. Seth zachwiał się pod
ciężarem klastra Etcetery, który nosił w~jedenastu kawałkach wokół
swojego ciała, opaski na nadgarstkach, plecak, pasek, klocki w~bandolierze, jakieś pierścienie. Podążali za nimi awiatorzy, prowadzeni
przez dzieci, wśród nich Jacob i~Stan, ubrani w~świeżo nadrukowane
stroje piratów powietrznych, chusty na głowę oraz bluzy i~rajstopy
ozdobione fotorealistyczną kolczugą typu trompe l'oeil. Zebrali uściski
i skomplikowane uściski dłoni i~pocałunki od swoich nowych przyjaciół i~oddali je z~zapałem, mówiąc więcej po portugalsku, niż mieli prawo się
nauczyć podczas tak krótkiej podróży.

Miejsce przyziemienia było boiskiem szkolnym. Szkoła była nisko
zawieszoną ceglaną budowlą, mającą sto lat, opuszczoną przez
dziesięciolecia i~ponownie otwartą przez siłę wyższą, sądząc po radosnym
odmalowaniu i~transparentach, osłonach słonecznych i~żaglach na dachu.

Tam zmrużyła oczy, przypominając sobie własną szkołę zbudowaną na tym
samym szablonie, ale prowadzoną przez prywatną firmę usługową, która
zamknęła połowę budynku, aby zaoszczędzić na kosztach wyposażenia,
zakładając stalowe żaluzje na okna i~zostawiając je, by wyglądały na
utwardzone obszary zabaw.

-- Ładnie, co? 

Dziewczyna miała nie więcej niż szesnaście lat, była
urocza jak słodka rzecz, okrągła twarz i~pełne, fioletowe usta. Tam
pomyślał, że może być Wietnamką lub Kambodżanką. Miała trochę trądziku.
Jej kruczoczarne, proste włosy były posiekane w~pomysłowy bałagan.

-- To Twoja szkoła?

-- To nasze wszystko. Technicznie rzecz biorąc, należy do gównianego
holdingu, który wykupił miasto z~bankructwa. Zdarzyło się to kiedy byłam
mała, mieliśmy specjalnego administratora, którego wszyscy nienawidzili,
a potem było bankructwo i~zamykali wszystko, stawiali płoty. Miarka się
przebrała, gdy zatrzymali wodę. Po tym miasto się automatycznie
usamodzielniło. Dzieciaki zrobiły szkołę.

-- Fajna. -- Tam cieszyła się oczywistą dumą dziewczyny. -- Chodzisz tam na
zajęcia?

Dziewczyna uśmiechnęła się. 

-- Nie wierzę w~nie. Prowadzimy warsztaty
rówieśnicze. Jestem wybrykiem natury, mam grupę dziwaków, których
zmieniam w~mój botnet.

Tam skinęła głową. 

-- Nigdy nie miałam rachunku różniczkowego. Tamta pani
z małym chłopcem pod pachą jest bohaterką matematyki.

Dziewczyna spojrzała ostro. 

-- No. Bez urazy. Jak myślisz, dlaczego tu
jestem? Szansa na spotkanie z~Gretyl? Kurde. Największa rzecz, jaka
trafiła do tego miasta od zawsze. -- Wpatrywała się intensywnie w~Gretyl.
-- Jej dowody są takie piękne.

-- Chcesz ją poznać?

Dziewczyna spojrzała spod przymrużonych powiek i~wysunęła język, tak cudownie urocza, że to musiało
być bardzo długo ćwiczone. Tam parsknęła śmiechem, zakryła usta i, ku
swemu przerażonemu przerażeniu, \textit{zachichotała}.

-- Polubi cię -- powiedziała Tam.

-- Lepiej, żeby tak było -- powiedziała dziewczyna i~wzięła ją pod ramię.

Gretyl straciła uścisk na Jacobie, gdy się zbliżyli, i~wyczuwając
przypływ, wypuściła Stana, by łzawił za jego młodszym bratem i~sprowadził go na ziemię. Tam machnęła do niej. 

-- Ej -- powiedziała.

Gretyl dramatycznie położyła dłoń na cofających się plecach dzieci, po
czym uśmiechnęła się do Tam i~dziewczyny.

-- Gretyl, to jest\ldots 

Dziewczyna, która zarumieniła się po czubki uszu, wyszeptała ,,\textit{Hoa
}''.

-- Hoa. Ona jest fanką. Uwielbia rachunek. Przyszła tu, żeby cię poznać.

Gretyl uśmiechnęła się do dziewczyny, rozłożyła wielkie ramiona i~objęła
ją uściskiem. 

-- Miło mi cię poznać.

Rumieniec dziewczyny był wszechogarniający.

-- Ciebie też miło poznać, Gretyl. Używam Twoich slajdów do rachunku na
moich warsztatach.

-- Miło to słyszeć!

-- Wprowadziłam pewne ulepszenia. -- Jej głos był szeptem.

-- Co \textit{zrobiłaś}? -- ryknęła Gretyl. Dziewczyna skurczyła się i~mogłaby uciec, gdyby Gretyl nie złapała jej za ręce. -- Nalegam, abyś mi
je pokazał, \textit{właśnie teraz}.

Dziewczyna straciła nieśmiałość. Wytrzasnęła ekran i~przeprowadziła
Gretyl przez jej zmiany. 

-- Dzieciaki ciągle się myliły, kiedy
stosowaliśmy operatory pochodnych na końcu, ponieważ bez nich, kiedy
robiliśmy resztę, limity i~tak dalej, wszystko wchodziło jednym uchem, a~wychodziło drugim, po prostu pamiętanie. Kiedy zaczęłam mieszać
operatory, gdy szliśmy, stały się lepsze w~składaniu tego na końcu.

Akt ,,wesoła staruszka Gretyl'' zniknął. Spuściła swoje ogromne brwi.

-- Czy operatory bez teorii nie są mylące? Bez teorii nie rozwiążą
zastosowań\ldots 

Dziewczyna wcięła się, potrząsając głową, rozczochrane włosy wszędzie. 

-- Musisz tylko uważać, jakie operatory wybierzesz. Widzisz\ldots  -- Wyprodukowała wykresy pokazujące, jak oceniła przykłady w~każdej grupie.
Tam widziała, że Gretyl to uwielbia. Była też pewna, że dziewczyna miała
rację we wszystkim. Lubiła to miasto.

-- Gotowa do drogi? -- spytał Seth. Zacisnął paski na klastrze i~nosił
głośnik na naszyjniku, z~którego mógł korzystać Etcetera. Tam zmusiła
się, by nie patrzeć się na to, kuszące było myśleć o~tym jako o~twarzy
Etcetery. Ale jego wizualne wejście pochodziło z~trzydziestu punktów
obserwacyjnych.

-- Już wkrótce. -- Wskazała. -- Gretyl ma wielbicielkę.

Etcetera wydał niecierpliwy dźwięk. 

-- To świetnie, ale musimy ruszyć w~drogę. To trzy, cztery dni drogi stąd, zakładając, że nie dostaniemy
rowerów ani przejażdżki.

-- Wiem. Musimy jeszcze pożegnać się z~\textit{Gilem}, przywitać tych ludzi
i pożegnać się z~\textit{nimi}. To się nazywa towarzyskość, Etcetera.
Dopasuj się.

Seth prychnął. Etcetera milczał, prawdopodobnie się dąsając. Tam
wyobrażała sobie, że w~swoim wewnętrznym monologu mówi niemiłe rzeczy o~żywych. Przypomniała sobie jego zapisane na płytach wypowiedzi na temat
wyboru Limpopo, by żyć jako duch domu. Zmrużyła oczy i~wysunęła język.
Usłyszała śmiech Hoa i~Gretyl i~odwróciła się, by zobaczyć, jak patrzą.

Zrobiła minę, robiąc z~siebie Harpo. Hoa odpowiedziała własną, a~Gretyl
o gumowej twarzy zrobiła taką, która ich zawstydziła.

-- Wygrałaś. Wy dwoje uczyniliście już świat bezpiecznym dla rachunku
różniczkowego?

-- Gotowe. -- Uśmiechnęły się.

-- Kiedy zaczynamy? 

Tam spojrzała na chłopców, teraz w~stosunkowo
wolnym od sterowców zakątku boiska, z~kilkoma miejscowymi dzieciakami i~kilkoma awiatorami, kopiącymi piłkę w~grze, która wymagała wielu
wrzasków i~walki, i~prawdopodobnie żadnych zasad.

-- Jesteście ogarnięci -- powiedział Hoa. -- Mamy tu rowery wychodzące z~naszych dup.

-- Brzmi boleśnie -- powiedział Seth. Hoa zrobiła minę.

-- Jesteśmy fanami zdekonstruowanych rowerów, minimalnych topologii.

Tam zobaczył, że Gretyl i~Seth kiwają głowami. Stłumiła irytację.
Próbowała zrozumieć atrakcyjność topologii minimalnej, ale wyglądała po
prostu na\ldots  niedokończoną. Dążenie do zmniejszenia całkowitej objętości
mechanicznych ciał stałych było projektem zarówno defaultu, jak i~odchodzących od dziesięcioleci, minimalizującym zużycie surowca w~każdej
części i~lepszym w~modelowaniu właściwości utwardzonego surowca. Znajome
rzeczy stały się bardziej nieprawdopodobnie pajęczynowate. Wszystko było
splecionymi siatkami typu ,,tensegrity'', które w~stresie usztywniały
się krzyżowo, łącząc siłę i~giętkość. To było wystarczająco przerażające
w formie regału lub stołu, wszystko wyglądało, jakby miało się cały czas
zawalić. W~przypadku rowerów technika ta przyprawiała ją o~mdłości ze
strachu, ponieważ rower odkształcał się i~podskakiwał z~powodu
niedoskonałości dróg.

-- Świetnie -- udało jej się powiedzieć.

Hoa skinęła głową. 

-- Wyprzedzamy wszystkich. Zrobiłam jeden w~zeszłym
miesiącu, który waży tylko dziewięćdziesiąt gramów! Bez kół.
Wydobędziesz z~niego siedemset k, zanim się załamie.

To była ta druga rzecz w~topologii minimalnej. Miała katastrofalny tryb
awarii. Pojedyncza pękająca rozpórka powodowała kaskadę rozplątujących
się chaotycznych ruchów, które mogły dosłownie zredukować ramę roweru do
sterty wydrukowanych w~3D gałązek w~trzydzieści sekund. Ludzie
przysięgali, że mechanizmy samohamujące roweru zatrzymają go w~bezpieczny sposób, zanim się rozpadnie. Ale jeśli potrafili tak dobrze
modelować kataklizm, dlaczego nie mogli temu zapobiec?

-- Świetnie. -- Przyłapała Gretyl i~Setha na sarkastycznym przewracaniu
oczami. Spojrzała na nich gniewnie, a~Seth ją uścisnął.

-- Pokochasz to. Jakby co, mamy Twój skan w~aktach, prawda?

-- To piekło w~życiu pozagrobowym -- powiedział Etcetera. -- Pokażę ci, co
i jak.

Rozważyła swoje opcje, epickie marudzenie, sarkazm, kapitulacja,
uśmiechnęła się i~powiedziała: 

-- Wygląda na to, że jedziemy!

Seth ją
przytulił. Słyszała, jak Etcetera szepcze pochwały jego wyboru
romantycznych partnerów.

\chapter*{v}
Ustawili rowery na polu, od najmniejszych do najwyższych, i~zdobyli
przyczepę, w~której chłopcy mogli usiąść, do której mogli też przyczepić
swoje półwymiarowe rowery. Podczas próby i~wymiany hełmów, robienia
zdjęć grupowych, panowała ogólna wesołość. Lotnicy, rozładowani,
przyglądali się, udzielali rad i~majstrowali przy rowerach.

Dotarli do momentu, w~którym wszyscy nie mogli się doczekać i~nikt nie
potrafił podać powodu, żeby tego nie robić, wszyscy opróżnili pęcherze i~tak dalej. Uformowali się i~pojechali. Tam zacisnęła zęby, gdy zaczęła
jechać, ale było gładko. Rower zawierał kombinację sztywności i~sprężystości konstrukcji tensegrity, z~łatwością pochłaniając wstrząsy,
ale wciąż wystarczająco sztywny, aby zapewnić sterowność.

Stan i~Jacob nadawali tempo przez pierwsze osiem kilometrów, powolną
jazdę. Hoa i~jej przyjaciółki dotrzymywały kroku, słuchając każdego
słowa Gretyl. Kiedy Stanowi i~Jacobowi zabrakło pary (czerwoni,
zdyszani), wsiedli do przyczepy. Reszta imprezy skorzystała z~okazji, by
wysikać, napić się wody, przekąsić, wymienić rowery i~dopasować kaski.
Kiedy zaczęli ponownie, Hoa i~przyjaciele pożegnały się i~zawróciły.

Pchali mocno, trzy ramię w~ramię, czasem mijali dziwny samochód -- większość pojazdów jechała na 401, który był czystym defaultem i~intensywnie patrolowanym -- zatrzymując się wcześnie w~rezerwacie
Mohawków, gdzie znajdowała się knajpka z~gotowanymi ćwiartkami
ziemniaków podawanych z~twarogiem. Właścicielem był ,,Idle No More''
drugiej generacji. Szybko zorientowali się, których mają wspólnych
przyjaciół.

Słońce było nisko. Zgodzili się, że jeśli ruszą, mogą być w~Kingston
przed zmrokiem, może nawet zjeść o~północy ucztę z~Limpopo, perspektywa,
która rozpaliła ich wyobraźnię i~entuzjazm, z~wyjątkiem Jacoba i~Stana,
którzy spali już w~przyczepie, zwinięci jak yin-yang. Lodołasica
poluzowała ich ubrania i~rzuciła na nich cień, a~następnie spojrzała na
nich, uśmiechając się w~sposób, który Tam mogła zrozumieć, ale nie mogła
się do niego odnieść.

Seth złapał ją, uściskał i~pocałował, dodając podstępny język i~skubanie
płatka ucha; podniosła jedną rękę do jego spoconych pleców, a~następnie
wsunęła ją na tyłek i~ścisnęła.

-- Numerek w~krzakach? -- zaszeptał.

-- Jezu, wy dwoje -- powiedział Etcetera. Przypomniała sobie, że był
dzisiaj cyborgiem i~odskoczyła.

-- Na pewno wiesz, jak poprawić nastrój. -- Seth jeszcze raz ją przytulił.
-- Przepraszam kochanie.

-- Nie martw się -- powiedziała. -- Rozpocznijmy to przedstawienie.

Była odchodniczką, odeszła, odkąd skończyła czternaście lat, chociaż
wróciła do defaultu do rodziców, potem ciotki, potem rodziców aż do
siedemnastego roku życia, kiedy odeszła na zawsze. Miała duże mięśnie ud
i wybrzuszone łydki, nawet teraz była obwisła i~była w~średnim wieku.
Był czas, kiedy nie myślała o~chodzeniu po dziesięć godzin dziennie,
dzień po dniu. W~tamtych czasach rower był praktycznie oszukiwaniem.
Mogła jechać bez potu. Był to luksus zarezerwowany dla zbiegu wielkich
dróg i~szczęścia.

Mięśnie czy nie, te czasy były już za nią. Po pierwszej godzinie
dyszała. Odprowadzający wilgoć materiał jej koszuli wydawał się lepki.
Zdarzało się, że jej łydki i~stopy miały kurcze i~musiała niezręcznie
rozciągać się podczas jazdy, krzywiąc się i~tłumiąc jęki. Mogła poprosić
o postój, ale pomyślała o~kolacji z~Limpopo. Poza tym Seth też się
krzywił, podobnie jak Lodołasica i~Gretyl. Żadne z~nich nie mówiło o~przystankach. Nie byłaby pierwszą, która się wypłakuje.

-- \textit{Cholera jasna}! -- Seth zawył i~potrząsnął nogą, odłożył rower i~poturlał się po trawie na poboczu. Złapał się za nogę. Wszyscy wysiedli,
przeciągnęli się, narzekali i~nieśmiało uśmiechali się do siebie. Jacob
i Stan obudzili się z~drzemek i~zaczęli krążyć wokół nich i~żądać, by
pozwolono im jechać. Wszyscy zgodzili się, że byłoby niesprawiedliwe nie
pozwolić chłopcom jeździć, więc nastąpiły godziny jazdy w~wolniejszym
tempie. Było znacznie lepiej.

Słońce było krwawą plamą na horyzoncie za nimi, barwiąc drogę na
czerwono, kiedy Jacob i~Stan wsiedli z~powrotem do przyczepy. Lodołasica
sprawdziła paski ich hełmów. Gretyl zrobiła to ponownie. Obie kobiety
strzelały w~siebie sztyletami i~śmiały się z~siebie. Wszyscy byli starzy
i odbywali razem długą podróż. Coś się zmieniało. Jedna epoka ustępuje
miejsca drugiej. Poczucie zmiany trzeszczało w~chłodnym powietrzu. Jedli
papkowate plastry arbuza i~ściskane sakiewki z~czekoladą i~elektrolitem.
Sprawdzili odległość i~w niewypowiedzianym konsensusie wsiedli na swoje
maszyny i~zaczęli kręcić pedałami.

Na drodze nie było latarni. Przerzucili się na reflektory, potem na
noktowizory, a~potem z~powrotem na reflektory, gdy światła Kingston
rozjaśniły horyzont. Okrążyli miasto, ostrzegani przez drony policyjne w~powietrzu i~znaki ostrzegające o~punktach kontrolnych Policji Ontario.
Kierowali się na drogę numer 15 na północ, pas prywatnych więzień
budowanych jedno przy drugim.

Wschodził księżyc i~robiło się zimno, kiedy dotarli do zjazdu z~autostrady prowadzącej do więzień, parku rozrywki z~aresztami
zbudowanymi przez TransCanada w~ramach strategii dywersyfikacji. Sala
poprawczaka. Więzienie dla mężczyzn. Więzienie o~minimalnym nadzorze.
Gdy rozeszły się wieści, że nadeszła zmiana, każde z~nich otrzymało
pierścienie namiotów i~jurt. Zjawisko to było zgodne z~szablonem, który
został opracowany i~sformalizowany w~głupio nazwanej Dekadzie
Odchodnickiej. Niektóre ściany runęły, inne poszły w~górę. Zbudowali
maszyny z~ubitej ziemi i~dodawali rozległe skrzydła i~metry, prawie na
pewno onsen, ponieważ to było minimum w~przypadku czegokolwiek
odchodnickiego większego niż kilka osób.

Zmieniłby się rytm miejsca. W~dni, kiedy świeciło słońce lub wiał wiatr,
bez opamiętania uruchamiali lodówki, podgrzewali ogromne baseny z~wodą
do pływania i~kąpieli, ładowali się i~puszczali drony i~inne zabawki.
Gdy żadnego z~nich nie było w~pobliżu, budynki przechodziły na pasywną
kontrolę klimatu, ludzie przestawiali się na mniej energochłonne
zajęcia.

Więcej ludzi wchodziłoby i~wychodziło, byłyby kłótnie o~to, co robić i~jak robić, jeśli w~ogóle. Niektórzy zajmowali się rolnictwem, inni
ogrodami. Albo i~nie, niektóre społeczności nigdy nie zastygały, stawały
się miastami-widmami w~ciągu kilku miesięcy od powstania. Czasami
zdarzały się gorsze rzeczy. Były mroczne historie o~gwałtach,
morderstwach, kultach osobowości, w~których charyzmatyczni socjopaci
prali mózgi hordom, aby wykonywały ich polecenia. Doszło do zbiorowego
samobójstwa, a~przynajmniej tak mówili. Wszyscy spierali się o~to, czy
te historie są prawdziwe, zminimalizowane przez łatwowiernych
odchodników, czy też podsycone do gorączki przez defaultowe psy-ops.

Przed nimi było więzienie dla kobiet. Wokół niego, najbardziej
karnawałowy obóz, powiatowy jarmark dla uchodźców. Musieli zsiąść -- żaden z~rowerów nie zepsuł się katastrofalnie -- i~wepchnąć je w~gąszcz
rzeczy, krzyżujących się odciągów i~pachnących salonów sufitu, które
kołysały się nawet o~tak późnej porze. W~połowie drogi porzucili rowery
i podzielili się plecakami Gretyl i~Lodołasicy, podczas gdy każda
kobieta podniosła śpiącego chłopca.

Bramy więzienia otworzyły się szeroko. Kilka kobiet siedziało na
pluszowych fotelach wyciągniętych z~jakiegoś biura. Przerwały rozmowę,
by od niechcenia zapytać, kim są ci ludzie i~dokąd zmierzają. Na
wspomnienie imienia Limpopo ich twarze rozjaśniły się i~zaproponowały,
że wprowadzą grupę do środka.

-- Znaliśmy ją jako G oczywiście. Pod tym właśnie ją zapisali. Karali ją
mocno, gdy użyła swojego zewnętrznego imienia, więc przestała. Wszyscy
zmieniają imiona, teraz jesteśmy szeroko otwarci. -- ,,Szeroko otwarte''
było tym, co powiedziała prasa default, gdy strażnicy więzienni
przestali pojawiać się w~pracy, coś, czego można użyć do terroryzowania
ludzi maruderami, którzy wybiegną z~więzień i~zacząć siekać ludzi. Kiedy
jechali przez parki TransCanada, widziała transparenty świętujące
,,szeroko otwarte''.

Wprowadzono ich do środka, przez szeroko otwarte -- ahem -- przedsionki
skanujące, podwórka i~komory, w~których mogli być przetrzymywani goście
lub więźniowie. Wszystkie drzwi zostały odsunięte lub usunięte i~ustawione na kozłach i~ułożone w~stosy z~asortymentem ubrań i~innych
rzeczy, którymi dzielili się więźniowie lub z~więźniami. Blok więzienny
składał się z~ogromnych pomieszczeń z~wysokimi sufitami i~ścianami
barowymi, w~których znajdowały się trzy wysokie łóżka piętrowe,
ozdobione banerami i~obwieszone kocami zapewniającymi prywatność (może
były tam wcześniej, ale Tam nie sądziła, że więzienie działały w~ten
sposób). Światło było przyćmione, dookoła odgłosy szeptanych rozmów,
chrapanie i~oddechy setek, tysięcy? \ldots  kobiet sprawiało, że miejsce
brzmiało jak ogromny, mamroczący tunel.

-- Tędy -- szepnęła ich przewodniczka. 

Szli gęsiego wąskim korytarzem między
pryczami, głęboko w~labirynt. Tam czuła drobną defaultowość, obawiając
się, że te kobiety są przestępcami, niektóre z~nich z~pewnością dokonały
niewybaczalnej przemocy, by tu wylądować. Wszędzie byli brutalni ludzie.
Przez większość czasu większość z~nich nie robiła niczego szczególnie
brutalnego, ponieważ nawet psychole musieli się dogadać i~żyć. Ci ludzie
byli dla nich po prostu słodcy od chwili ich przybycia. Limpopo była
jedną z~tych osób. Zamknęła tę część default.

Limpopo spała na swojej pryczy, z~twarzą w~szarej skali w~przyćmionym
świetle, ale pomarszczoną i~starszą, niż Tam pamiętała. Wszyscy skupili
się wokół jej koi, a~Tam rzuciła okiem na krasnoludy skupione wokół mar
Śnieżki.

-- To niezręczne -- wyszeptał Etcetera z~piersi Setha.

Limpopo poruszyła
się. Zmarszczyła twarz, tyle zmarszczek! Ręka Tam powędrowała do jej
twarzy. Limpopo zamrugała dwukrotnie oczami, otworzyła je i~się
rozejrzała. Musieli wyglądać jak sylwetki, bez twarzy, ale kto inny
byłby przy jej łóżku?

-- D -- szepnęła ich przewodniczka. -- Przyprowadziłam ci kilkoro przyjaciół. -- Jej głos był gęsty od łez.

-- Dzięki -- odszepnęła Limpopo. -- Dzięki, Testshot. Wielkie dzięki. -- Podparła się na łokciach.

-- Cholera, dobrze cię widzieć. -- Tam myślała, że Limpopo to powiedziała,
ale to znowu Etcetera, jego głos dziwnie modulowany emocjami maszyny.

Limpopo uśmiechnęła się na wpół, usta drżały. Łzy spływały jej po
twarzy. Nikt nie wiedział, co robić. Lodołasica podała Stana Sethowi,
objęła ramionami szyję Limpopo i~przyciągnęła ją do długiego uścisku. 

-- Kocham cię, Limpopo -- wyszeptała.

-- Wszyscy cię kochamy. -- Gretyl podała Jake'a Tamowi i~owinęła ramiona
wokół Limpopo i~Lodołasicy, na wpół ześlizgując się na łóżko, żeby to
zrobić. Tam spojrzała na zaspaną twarz Jake'a i~zobaczyła, że się budzi,
chociaż trzymał się jak małpa na drzewie, miał silne ręce, brudne włosy
i słodko-kwaśny oddech nieumytych zębów. 

-- Mama? -- wymamrotał.

-- Tam. -- Tam odwróciła się tak, że mógł zobaczyć obie matki przytulające
dziwną staruszkę w~dziwnym ciemnym pokoju. O dziwo, to go pocieszyło. 

-- Możesz stać? -- Pomyślał o~tym, skinął głową. Położyła go i~przyłączyła
się do uścisku, zgniatając nogę Limpopo, walcząc o~pozycję. Minutę
później ramiona Setha były wokół niej.

Przytulali się i~płakali w~ciemności. Jake powiedział z~szokującą
głośnością: 

-- Muszę siusiu, mamo! -- Śmiali się i~rozplątywali, uciszyli
chłopca i~szeptali przeprosiny do kobiet, które zbudził hałas.

Limpopo poprowadziła ich z~powrotem przez blok, na dziedziniec oświetlony
lampami i~wypełniony małymi grupami konwersacyjnymi siedzącymi na kocach
i krzesłach ze środka. Wyjęli z~plecaków składane krzesła i~koce,
butelki pysznej whisky od fabbera na \textit{Gil}. Rytuał był tak normalny
i tak dziwny, że Tam wciąż był wstrząśnięta nim, dopóki nie wrócili do
swojego kręgu konwersacyjnego. Chłopcom matkowała Lodołasica i~Gretyl,
wpatrywali się szeroko otwartymi oczami od jednego dorosłego do
następnego, zaspane, nerwowe i~jednocześnie podekscytowane. Tam
wiedziała, jak się czują.

Limpopo opowiedziała im historię swojego uwięzienia w~napadach i~początkach, z~wieloma przerwami. To nie była ładna historia. Spędziła
dużo czasu w~odosobnieniu, była to rutynowa kara za najłagodniejsze
wykroczenia. Szczególnie były nią wyróżnione odchodzące. Jej najdłuższy
okres w~samotności trwał dwa lata, podczas których nie miała kontaktu z~resztą ludności. Niewiele lepiej było przez resztę czasu: przez całe
lata więźniowie otrzymywali godzinę wyjścia z~celi dziennie. Przez sześć
miesięcy nikomu nie pozwolono opuszczać jej bloku poza przypadkami
medycznymi -- bez prysznica, bez ćwiczeń. Tam pomyślała o~ogromnych,
odbijających się echem barakach i~próbowała sobie wyobrazić, że tkwi tam
z setkami kobiet przez pół roku. Zadrżała i~wypiła więcej whisky.

Na początku wszyscy słuchali, zachwyceni. Ale było późno. Mieli długi
dzień. Zaczynając od Gretyl, Lodołasicy i~chłopców, wyciekali,
znajdowali puste prycze w~bloku. W~końcu była tylko ona i~Seth, oraz
Etcetera. Ledwo mogła utrzymać otwarte oczy.

Limpopo i~Etcetera byli zaangażowani w~słowne połączenie umysłów,
rozmowę coraz bardziej intymną, ocienioną prywatnymi niuansami, których
Tam nie potrafił rozszyfrować, chociaż mogło to być wyczerpanie.

Tam zdała sobie sprawę, że Seth zasnął na swoim krześle. Limpopo była
zajęta rozmową z~pudełkiem na piersi, wykluczając wszystko inne.
Potrząsnęła Sethem, by go obudzić, a~on potarł oczy. 

-- Proszę. Zdejmij
trupa i~zostaw go z~Limpopo, walimy w~kimę.

Limpopo zachichotała, a~Etcetera śmiał się razem z~nią. To wydawało się
bardzo konspiracyjne między tą dwójką, kiedy szli do łóżka.

Śniadanie było zabawą, polowaniem na łupieżców w~więzieniach i~namiotowych miastach parku TransCanada, aby znaleźć faby z~mocą i~zapasami, podgryzając smakołyki przekazywane przez przechodniów i~zwracając smakołyki, albo rzeczy, które przynieśli, albo rzeczy, które
otrzymali w~prezencie po drodze. Zanim Tam i~Seth dogonili grupę
poszukiwaczy, rozproszyli się i~uformowali na nowo, korzystając z~wbudowanej sieci odchodzących, by odnaleźć się nawzajem. Było słonecznie
i parno. Chłopcy mieli dopasowane jaskrawopomarańczowe szorty i~rogowe
hełmy wikingów oraz klapki, które wydawały pierdnięcie, ku ich
oczywistej radości.

Seth wyglądał nago bez Etcetery rozłożonego na jego ciele. Rozkoszowali
się prywatnością, w~której mogli rozmawiać i~przytulać się bez
angażowania zmarłego. To było jak symboliczny nowy dzień, a~także
dosłowny. Ukończyli swoją misję, połączyli się z~utraconym przyjacielem
i ponownie zjednoczyli swojego zmarłego przyjaciela z~tym utraconym
przyjacielem. Ich ramiona obejmowały się nawzajem w~talii, byli dobrze
odżywieni, a~słońce świeciło. To był nowy dzień, byli otoczeni
przechodniami. Nie mieli nic do roboty.

Limpopo znalazła ich siedzących w~trawie na zarośniętej łące po drugiej
stronie autostrady, obserwujących, jak wielkie drony leniwie krążą nad
głową. Niektóre były defaultu, inne odchodnickie, niektóre mogły uciec z~farmy i~stłoczyć się według ogólnych zasad.

-- Dzień dobry, piękni ludzie. -- Prawie śpiewała. 

W~świetle dnia
wyglądała jeszcze starzej. Pochyliła się i~Tam pomyślał, że widzi
drżenie w~jej rękach. Nie była też dużo starsza od Tam, miała dużo
trudniejsze życie. Niezależnie od różnic między okolicznościami, Tam
wiedziała, że to jej przyszłość. To sprawiło, że poczuła nieopisaną
nostalgię za młodą, pewną kobietą, którą była.

-- Dzień dobry! -- zawołali. 

Lodołasica zaatakowała ją uściskiem. Tam
skrzywiła się, zmartwiona słabością Limpopo. Jednak Limpopo się
roześmiała, uściskała ją i~zażądała ponownego przedstawienia chłopcom,
przeprowadziła z~każdym z~nich uroczystą rozmowę na temat ich ulubionych
zainteresowań -- podróży kosmicznych i~oślizgłych rzeczy -- i~znalazła dla
nich słodycze w~kieszeniach, lizaki o~tysiącu smaków wielkości piłeczek
golfowych. Ich mamy skinęły głowami, zgadzając się, a~piłki golfowe
zniknęły w~ich ustach, zatrzymując ich na chwilę.

-- Jak \textit{się} masz? -- Ramię Lodołasicy obejmowało ramiona Limpopo,
twarz zwrócona ku słońcu. -- To musi być najdziwaczniejsza rzecz, Ty i~Twoi przyjaciele musicie być, nie wiem, po prostu\ldots 

-- Tak -- powiedziała Limpopo. -- I~nie. Chodzi o~to, że kiedy jesteś
więźniem, coś ci się przytrafia. Nie masz nic do powiedzenia. Znam
kobiety, które były w~środku przez lata, dekady, które nagle zostały
zwolnione, bez uprzedzenia. Dosłownie przyszli strażnicy, zabrali je i~wykopali. Nie miały szans zadzwonić do rodzin, nie było pożegnań.
Czasami miałaś więźniów, które miały być zwolnione, załatwiono
papierkową robotę, a~potem, kilka minut przed ich odjazdem, to wszystko
było anulowane. Nikt nie potrafił powiedzieć dlaczego. Kiedy drzwi się
otworzyły, była to o~rząd wielkości większa wersja bezwględnego życia,
które już prowadziłyśmy.

-- Byłyśmy też przyzwyczajone do samowystarczalności. Wymieniałyśmy się
przysługami, wspierałyśmy się nawzajem. Większość prac wykonywałyśmy w~więzieniu. W~ten sposób TransCanada dostarczała wartość dla
akcjonariuszy, zmuszając więźniów do wykonywania całej swojej pracy,
nieodpłatnie, w~imię kary. Gdy otworzyły się drzwi, nie było \textit{to}
trudne, żeby utrzymać światło włączone. Nie mamy wszystkich potrzebnych
nam materiałów eksploatacyjnych, zablokowanie dostępu do sieci
elektrycznej oznacza, że możemy korzystać tylko z~tego, co otrzymujemy z~wiatraków i~paneli na dachach, ale wszystko to oznacza, że musieliśmy
wyjść do reszty świata i~znaleźć ludzi, którzy nam pomogą i~vice versa.
W zamknięciu było \textit{tak wiele} osób, które odeszły. Pomysł
prowadzenia tych wszystkich rzeczy bez chciwości i~złudzeń jest tym, o~co nam chodzi. -- Błysnęła uśmiechem. -- Powiedziałabym, że radzimy sobie
cholernie świetnie.

Głośnik zawieszony na jej szyi wiwatował i~wydawał odgłosy klaskania. 

-- Jesteś moją całkowitą bohaterką -- powiedział Etcetera. -- Świetny
przykład dla wszystkich, żywych i~martwych.

To też sprawiło, że się uśmiechnęli i~przypomniało im pytanie, które Tam
niecierpliwie pragnęła zadać. 

-- Nie chcę być dziwna. Ale czy zamierzasz
zrobić nowy skan? W~razie czego\ldots 

Limpopo odwróciła wzrok.

-- Da da daaa -- powiedział Etcetera. -- Kryzys egzystencjalny się zbliża.

-- Wiem, że gdzieś tam, gdzie mieszkasz, jest jeszcze jedna osoba, a~ona
brzmi\ldots 

-- Jak w~sumie\ldots 

Lekko uderzyła małym głośnikiem w~obojczyk. 

-- Przestań. To nie jest
pomocne, to wredne. Cokolwiek wydarzyło się między Tobą a~nią, nie
usprawiedliwia, że ze mną zachowujesz się jak chuj. \textit{Szczególnie}
ze mną. Ona jest mną.

-- To jest kryzys egzystencjalny. -- Etcetera nie wyglądał na zranionego,
chociaż żywy Etcetera byłby w~niespokojnych precelkach na myśl o~tym, że
jest publicznie okropny. Czy to oznaczało, że nie był osobą, którą był?
Albo że dorósł? Albo że jego zderzaki utrzymywały jego nastrój w~środku?

Limpopo wyglądała groźnie. 

-- Tak, zrobię skan. W~męskim więzieniu
prowadzą je już dwie załogi. Założymy własne. Wiele z~nas jest już
starych, a~jeszcze więcej jest chorych. Jest jeszcze możliwość, że
wysadzą to miejsce jako przykład.

Spojrzeli w~niebo.

Gretyl pokręciła głową. 

-- Zawsze możliwość. Może TransCanada zawróci się
i wróci, żeby wszystkich zamknąć. Musisz sobie wyobrazić, że to gówno
jest panikującym defaultem. Co dalej, gdy więzienia przestaną działać?
Ich małe wysepki normalności się kurczą. Byłoby rzeczą dla serca i~umysłów, abyście wy, niegrzeczne dzieci, zostały wysłane do swoich pokoi
bez kolacji.

To sprawiło, że dzień był mroczniejszy.

\part{dylemat więźnia}

\chapter*{i}
Gretyl wróciła od faba ze wszystkim, czego potrzebowali do zbudowania
schronienia, elastycznymi ramami i~łącznikami, które chłopcy szybko
zmontowali, fotoreaktywną folią, którą naciągnęli na każdy element,
sczepiając je ze sobą, tworząc pół-kopułę z~pionową ścianą i~drzwiami.
Ustawili ją na polu, na którym obserwowali drony. Teren przed bramami
więzienia był zatłoczony, nie było miejsca na nowe konstrukcje
wystarczająco duże dla czteroosobowej rodziny. Było ciszej. Reszta ich
załogi też tam się rozłożyła.

Limpopo nie była gotowa do wyjazdu i~może nigdy nie będzie. Chciała
zbudować onsen. Na wzmiankę o~tym, rozbłysły się oczy tych, którzy znali
ją w~B\&B, w~tym Lodołasicy. Wypracowali konsensus, że zostaną i~pomogą.
Chłopcy nigdy nie widzieli onsenu -- wyszły z~mody w~Gary -- i~z zapałem
oglądali o~nich filmy. Byli zaangażowani w~projekt. Gretyl mogła wrócić
do domu, ale nie było powodu, żeby tam być w~przeciwieństwie do tutaj.
Mogła uczyć swoje klasy wszędzie. Mała poważna praca matematyczna, którą
wykonywała z~kolegami, była niezależna od geografii.

Nie podobało jej się tutaj. Było zbyt blisko Toronto, do Jacoba
Redwatera. To dziwne, że Lodołasica nazwała swojego syna jego imieniem,
ale bycie tak blisko tego, co Gretyl nazwała ,,kryjówką Jakuba'',
sprawiało, że była zdenerwowana. Dlatego Lodołasica chciał zostać.
Musiała udowodnić -- sobie, światu, swojemu potwornemu ojcu, który znał
każdy jej ruch -- że się nie boi. Zostało to pomieszane podczas nadawania
imion. Gretyl rozumiała, że refaktoryzacja tej bolesnej dyskusji nie
daje żadnych nowych informacji. Kiedyś była na tyle nieroztropna, że
pokłóciła się o~to z~Lodołasicą -- w~dodatku z~ciężarną Lodołasicą -- i~nauczyła się swojej lekcji.

Nadal podskakiwała na wspomnienie.

-- Nienawidzisz tego. -- Chłopcy byli na miejscu budowy onsen, sklejając
ze sobą fabrykowane cegły. Obiecano im wyprawę ratunkową do miejsca, w~którym dron skatalogował całą masę użytecznego materiału, pod warunkiem,
że tego ranka będą pilnie pracować i~dobrze się zachowywać. Lodołasica
wróciła i~miotała się na polowym łóżku, mdlejąc, popijając przez słomkę
ze swojego plecaka i~lśniąc ładnie od potu.

-- Nie nienawidzę tego. Całkowicie rozumiem\ldots 

-- Nienawiść i~zrozumienie nie są przeciwieństwami. Chcę ci powiedzieć,
że \textit{wiem}, że tego nienawidzisz, i~jestem wdzięczna, że i~tak to
robisz.

Gretyl pokręciła głową. 

-- Ja też cię kocham.

Lodołasica wyciągnęła rękę i~na ślepo zaczęła jej szukać, poklepała ją
po tyłku. Gretyl wzięła ją za rękę. Miło było mieć chwilę bez dzieci.
Minęło trochę czasu. Trzymały się za ręce i~Gretyl zamknęła oczy.

-- Zrobiłam dzisiaj nowy skan -- powiedziała Lodołasica. -- Chłopcy też.

Gretyl otworzyła oczy. 

-- Och. -- Bardzo się starała, aby jej głos był
neutralny, ale zawiodła.

-- Nie bądź taka.

-- Jaka?

Lodołasica cofnęła jej rękę i~usiadła. 

-- Był tłum, który je robił, mamy
i dzieci, teraz skanery są wypalone i~pracują. Wiesz, że to
nieszkodliwe.

-- Wiem, że proces skanowania nie może cię skrzywdzić, ale\ldots 

-- Ale ktoś może ukraść Twój skan i~zrobić nam coś strasznego. Wiem.
Przeszłyśmy przez to. Są zablokowane moim prywatnym kluczem lub
superwiększością kluczy naszych przyjaciół, zwykłej grupy, tej samej,
której używamy do reszty.

Gretyl pokręciła głową. 

-- W~porządku.

-- Najwidoczniej nie jest w~porządku.

-- Wyjaśnij, dlaczego czułaś się tak zagrożona, że zostałaś zeskanowana,
ale nie tak zagrożona, że po prostu stąd nie \textit{wyjedziesz}?

-- Zrobienie skanu daje nam pewne ubezpieczenie.

-- Ubezpieczenie? Na przykład, jeśli Twój ojciec cię porwie, mogę
uruchomić Twoją symulację, aby wychować naszych synów? Jeśli wszyscy
umrzemy, może nasi przyjaciele uruchomią nas w~symulacji i~będą nosić
nas na szyjach, będziemy mogły przez resztę czasu rozmawiać z~ich
cyckami?

-- Mój tata mnie nie porwie. -- W~pierwszych pięciu latach ich związku
Gretyl dobrze radziła sobie z~dostrzeganiem, gdy Lodołasica zmieniała
temat. Od tamtej pory coraz lepiej potrafiła wymyślić, kiedy o~tym
wspomnieć. Nie wspomniała o~tym.

-- Skąd wiesz?

Lodołasica zdjęła rękę z~twarzy, wypluła słomkę i~usiadła. 

-- Bo dostałam
wiadomość od niego.

Gretyl dosłownie rzuciła się na żonę. 

-- Powtórz to?

-- Słyszałam od niego. Daj spokój, wiesz, że wysyłał mi wiadomości. Nie
odpowiadam. Nigdy nie odpowiadałam.

-- Wcześniej nie byliśmy na jego podwórku.

-- Jesteś przesądna. Jacobowi Redwaterowi nie jest trudniej dostać się do
dowolnego miejsca niż dostać się w~każde miejsce. To nie odległość
zapewnia nam bezpieczeństwo.

Gretyl była również żoną Lodołasicy wystarczająco długo, by rozpoznać,
kiedy jej żona miała rację. Zamknęła się i~próbowała przestać się
martwić. Chłopcy wrócili, szukając ubrań nadających się na misję
ratunkową. Rozproszyło je oblewanie ich wodą i~ubieranie. Wtedy wszystko
zostało zapomniane, a~przynajmniej mogły udawać.

\chapter*{ii}

Róża onsen, cegła po cegle. Wszystko zaczęło się, gdy niektórzy z~jego
załogi wyprodukowali mechy budowlane, które oczywiście chłopcy chcieli
pilotować. Mechy działały automatycznie i~miały dodatkowe
zabezpieczenia. Wszyscy byli zgodni, że chłopcy są w~nich dobrzy i, w~przeciwieństwie do dorosłych, nigdy nie nudzą się powtarzającymi się
czynnościami manualnymi, pod warunkiem, że mogą pilotować roboty podczas
ich wykonywania.

Początkowo nalegali, by chłopcy mieli w~pobliżu dorosłych drugich
pilotów trzymających przełączniki, ale nie było dorosłych, którzy byliby
w stanie nadążyć za pędem chłopców do budowania. Również onsen szybko
rósł, dzięki ich wkładowi. To byłby chujowy ruch, żeby ich spowalniać.

-- Rodzicielstwo, -- oznajmiła Lodołasica -- to sztuka zejścia z~drogi
rozwoju swoich dzieci. -- To załatwiło sprawę.

Poza tym dało im to razem więcej czasu niż miały od urodzenia Stana.

To była druga podróż poślubna, spędzona w~upojnych pierwszych dniach -- bardzo małego -- nowego narodu, byli więźniowie i~ich rodziny dodawali
więcej każdego dnia, proszkując stalowe pręty więzienne w~celu uzyskania
surowca dla fabów, pompując dźwigary wspierające, podkręcone w~równaniach minimum waga/materiały, strukturalne wersje rowerów, na
których jeździli, ale z~większą liczbą zabezpieczeń. Hoa i~przyjaciele
przyjechali na trzydniowy pobyt, jeżdżąc dalej na rowerach. Znalazły
grupę podekscytowanych byłych więźniów, którzy chcieli wiedzieć, jak
działają te dziwne pojazdy. Teraz odbyły się trzy warsztaty tworzenia
wariacji na ten temat. Dochodziło do tego, że zaczęły się poważne spory
na temat etykiety rowerowo-pieszej.

Gretyl zapomniała, jak energetyzujące było życie rewolucyjne. W~Gary
było już rutynowo, tego typu, który musisz wprowadzić, jeśli wychowujesz
dzieci i~zachowujesz trochę życia dla siebie. Tutaj nie było dwóch
takich samych dni. Każdy dzień przynosił nowe wyzwania, nowe
rozwiązania. Minęły lata, odkąd Gretyl była częścią miejsca, w~którym na
forach dyskusyjnych toczyły się poważne kłótnie. Tutaj szalały, a~nawet
wybuchały bójki na pięści, ostudzone przez rozjemców, którzy stanęli na
wysokości zadania.

Tak jak ona była z~tego wszystkiego zadowolona.

-- Gretyl. -- Sposób, w~jaki Lodołasica to powiedziała, zmroził ją.
Słyszała Lodołasicę smutną, przestraszoną, a~nawet spanikowaną. Ale
nigdy nie słyszała tej nuty w~głosie żony.

-- Co? -- Gretyl pomachała powierzchniami interfejsu, machnęła
nadgarstkami, by otrząsnąć się z~zadań, które czekała w~kolejce. Ich
schronienie wydawało się ciasne, nie przytulne.

Lodołasica się pociła. Jej oczy były szeroko otwarte. Gretyl poczuła,
jak jej serce przyspieszyło.

Przez drzwi weszła kolejna kobieta. Była\ldots  zwinięta. Niezbyt wysoka,
krótko obcięta i~stylowe włosy, na wszystkich płaszczyznach, może
słowiańska. Jej postawa przypominała kota, który ma się rzucić. Gretyl
nie mogła odgadnąć jej wieku: starsza od Lodołasicy, ale w~tak
doskonałej kondycji fizycznej, że nie można było określić, o~ile. Miała
małe, kwadratowe zęby, które pokazała w~szybkim uśmiechu. Gretyl
wiedziała, kim musi być.

-- Czyli Ty jesteś Nadie?

Nadie znikomo skinęła głową. 

-- Gretyl. -- Wyciągnęła dłoń. Sucha. Silna.
Zrogowaciała. Perfekcyjny manicure, polerowane paznokcie w~kolorze
diuny, tępe końcówki.

Gretyl przeniosła wzrok z~Lodołasicy na Nadie.

-- Jak źle?

-- Limpopo przyprowadza chłopców. Nadie ma helikopter.

-- Helikopter.

Ręce Lodołasicy drżały. Gretyl chciała je złapać, ale była
irracjonalnie, mocno zła na swoją żonę. To była inna część bycia
zawodowym rewolucjonistą: ludzie wokół ciebie przez cały czas umierali.
Teraz jej chłopcy, jej dzieci, które miały moc sprawiać, że bolało ją z~siłą, która promieniowała od dołu jej brzucha do najdalszych czubków jej
kończyn, po prostu patrząc na nią przez ramiona niezmąconymi oczami i~słodkimi ustami, byli na drodze niewyobrażalnej siły. Będzie broń i~jeszcze gorzej. Filmy z~Akron były wyświetlane przez cały czas, szybkie
animacje wpadały na fora dyskusyjne, by rzucać głupie uwagi na temat
brutalności świata zewnętrznego.

-- Jak długo?

-- Niedługo -- powiedziała Nadie. -- Na szczęście były długie kłótnie. Dały
mi czas, żeby się tu dostać. Ale teraz się zdecydowali, idą.

-- Dlaczego nie zestrzelą Twojego helikoptera? -- spytała Gretyl. Jej
serce waliło.

-- Bo to \textit{mój} helikopter -- powiedziała Nadie. Przechyliła głowę. -- Zetta.

-- Dobrze.

Ręce Lodołasicy były pięściami. Rozległ się powitalny dźwięk chłopięcych
głosów i~tupot maszyn. Gretyl nie zawracała sobie głowy drzwiami.
Przebiła się przez ścianę z~fotoreaktywnej folii, rozpryskując chłodną
ciemność wnętrza strumieniem światła słonecznego.

Każdy z~chłopców pilotował mechy. Limpopo, Seth i~Tam jechali na nich,
stojąc na ramionach robotów, trzymając się uchwytów na głowach. Chłopcy
krzyczeli, pchając maszyny tak szybko, jak się dawało, najwyraźniej na
rozkaz, by nie martwić się o~to, co zniszczyli. Ramiona mechów
wymachiwały przed nimi, rozbijając namioty i~jurty.

Dołączyli do niej Lodołasica i~Nadie, wchodząc przez drzwi. Gretyl
zobaczyła, jak Nadie ocenia grupę, delikatnie potrząsa głową.

-- Nie wszyscy zmieścimy się do Twojego helikoptera, prawda?

Mówiła do Nadie. Lodołasica spojrzała na nią ostro. 

-- Gretyl, nie bądź dupkiem. -- Gretyl \textit{to} znała. Szło to tak: \textit{spieprzyłam, teraz
wszystko, co powiesz, przypomni mi o~tym i~sprawi, że będę wściekła}.
Wiedziała o~tym. Nie miała na to czasu. Zignorowała ją.

-- Niewystarczy -- powiedziała Nadie.

-- Ilu?

-- Przyszłam, żeby zabrać waszą czwórkę.

Gretyl rozpoznała uniki. 

-- Ale dla ilu masz miejsce?

Limpopo zsiadła, pomogła chłopcom, podczas gdy reszta zeszła na dół.
Gretyl rzuciła im spojrzenie, upewniła się, że są ubrani, mają kapelusze
przeciwsłoneczne. 

-- Przynieś wodę -- powiedziała do Lodołasicy. Powietrze, ubrania, woda, jedzenie. Odchodnicki triaż. -- Żywność.

Do Nadie: 

-- Ilu?

-- Przyszłam po czworo.

Macierzyństwo zrobiło z~niej tchórza. Wstydziła się, bo nie potrafiła
powiedzieć: \textit{Jak nie wyjadą nasi przyjaciele, to my nie pójdziemy}.
To nie było już jej życie na szali.

-- Zabierz nas wszystkich. -- Próbowała być przekonująca.

Lodołasica wróciła, szarpiąc paski kompresyjne na swoim największym
plecaku, zdeformowanym od tego, co do niego wrzuciła.

-- Ilu?

Nadie spojrzała na nią zupełnie nieczytelnie, spojrzała z~powrotem na
Gretyl. 

-- Czy umiałabyś wybrać?

-- Wolałabym wyjaśnić wybór moim dzieciom, niż wyjaśniać, dlaczego obok
nich były puste miejsca, kiedy ludzie zaczęli umierać.

-- Czy zrobiłoby to różnicę, gdybym ci powiedziała, że przybywają w~trybie nieśmiercionośnym?

-- Jak Akron?

-- Nie jak Akron. -- Wokół zgromadziła się publiczność. Cokolwiek widzieli
w języku ciała, uciszało ich. -- Dokładnie nie tak jak Akron. Akron
stworzył męczenników. Zabolało ich na całym świecie. Przychodzą jako
policja ze środkami przymusu bezpośredniego, żeby przywrócić porządek,
żeby zbadać morderstwa.

-- Jakie morderstwa? -- spytała Limpopo. Była zgarbiona, drżała, ale
zapewniała siebie tonem, który miał dwa metry pięćdziesiąt i~ostro
zacinał.

Nadie pokręciła głową. 

-- Brak czasu. -- Spojrzała na nich. -- Mogę wziąć,
hm, jeszcze jedno.

Spojrzeli na siebie. 

-- Ona ma helikopter. -- powiedziała Gretyl.

Znowu spojrzeli po sobie.

-- Zostanę -- powiedziała Gretyl. 

Lodołasica spojrzała na nią z~szokiem i~smutkiem. Gretyl spojrzała na nią spojrzeniem, że nie będzie się kłócić.
Nie przydawało się to zbytnio w~ich związku. To coś znaczyło.

-- To jest mój dom. -- powiedziała Limpopo. -- Dam świadectwo. Zginę, jeśli
do tego dojdzie.

Etcetera powiedziała coś miękkiego z~obojczyków, dopasowanego do jej
słuchu. Na jej ustach pojawił się mały uśmiech. Pogłaskała mówcę.

-- Nie możesz zostać -- powiedziała Lodołasica. 

Stan zaczął płakać, co
było taką rzadkością, że Jake też krzyczał. Gretyl posadziła go na
biodrze i~pozwoliła mu schować twarz w~jej gardle. Gretyl spojrzała na
twarz Limpopo. Uderzyło ją, jak bardzo było inaczej, jak bardzo lata w~więzieniu nie tylko ją postarzały, ale zmieniły. Wcześniej Gretyl
uderzył wysiłek Limpopo, żeby nie wydawać rozkazów lub sugerować, że ma
autorytet. Teraz była czystą alfą, promieniując niekwestionowaną
dominacją.

-- Nie wyjeżdżam -- powiedziała Limpopo z~niezachwianą pewnością siebie.

Nadeszła kolej Setha i~Tam. Spojrzeli od Lodołasicy do Gretyl. 

-- Gretyl -- powiedział Seth. -- Jesteś matką, nie możesz\ldots 

-- Mogę. -- Przełknęła gulę w~gardle. -- Będę. Jest kilka rzeczy, przed
którymi nie możesz uciec. -- Pomyślała o~swojej studentce i~o tym, przez
co przeszli. -- Chcę, aby moje dzieci były bezpieczne, ale nasza rodzina
nie ma większego prawa do bycia nietkniętym niż ktokolwiek inny. -- Nie
mówiła sensownie, nawet dla siebie. -- Dużo odchodziłam. -- Wzruszyła
ramionami. -- Zostanę.

Mogło to trwać, gdyby nie Nadie. Przechyliła głowę, posłuchała czegoś w~ślimaku, dyskretnie poruszyła palcami, zmrużyła oczy. 

-- Jedziemy. -- Wyciągnęła ręce, żeby zabrać Stana od Gretyl. 

Gretyl trzymała go,
całowała i~mrugała mocno, żeby powstrzymać łzy. To samo zrobiła
Jake'owi. Świadomie zapamiętała zapach chłopców, wmawiając sobie, że
nigdy nie zapomni zapachu, twarzy i~głosów swoich ukochanych synów.
Potem wzięła żonę w~ramiona i~trzymała Lodołasicę w~uścisku, który
rozciągał się od wieków do ich pierwszego szorstkiego, ukradkowego
obmacywania, przez lata miłości i~towarzystwa, trudy i~nieobecności,
zjazdy, walki i~pojednania. Zabrało to wszystko, co miała i~wszystko,
czego nawet nie wiedziała, żeby nie płakać, zwłaszcza gdy poczuła łzy
Lodołasicy na własnych policzkach, słone i~równie znajome jak jej
własne.

Nadie wydała naglący dźwięk. 

-- Start za siedem minut. Będziemy musieli
pobiec. -- Biegła ze Stanem na biodrze. Lodołasica chwyciła drugiego syna
i pobiegła za nią, a~Seth i~Tam spojrzeli na nią bezradnie i~też
pobiegli.

Potem była ona i~Limpopo.

-- Chcesz zobaczyć, czy w~schronie jest coś, co chcesz ocalić?

Gretyl, poruszając się w~odrętwiałym śnie, wróciła do schronu,
szturchnęła ich pościel i~rozrzuciła dobytek. Były trzy koce, dwa małe i~jeden duży. Małe pachniały jak chłopcy, a~duże jak Lodołasica. Wzięła je
w ramiona. Wypchana mysz Jake'a, Mysza, wypadła z~jego koca. Podniosła
go za zużytą, przeżutą łapę. Patrzył na nią paciorkowatymi oczami, gdy
chowała go z~powrotem pod koce.

-- Możesz je umieścić z~moimi rzeczami -- powiedziała Limpopo.

Szli szybko do więzienia. Limpopo była rozkojarzona, potykając się, gdy
szła, rozmawiała przez swój interfejs i~jednocześnie pisała SMS-y.
Czasami prosiła Etcetera o~przesłanie wiadomości. Zanim dotarli do
rozległego obozu przy bramach więzienia, panowała na wpół panika, gdy
ludzie biegali do bram lub w~ogóle z~więzienia, niosąc na plecach
tobołki. Dzieci płakały, ale poza tym było bardzo cicho. Przycięte,
zwarte głosy, wiele w~tym dziwnym tonie przeznaczonym do mikrofonów
interfejsowych, a~nie ludzkich uszu.

-- Do środka -- powiedziała Limpopo. 

Gretyl usłyszała w~oddali śmigło
helikoptera, niesione wiatrem, cichsze w~miarę odlatywania. Zatrzymała
się i~przyłożyła dłonie do twarzy. Naprawdę szlochała. Limpopo
prowadziła ją za łokieć, szepcząc, że wszystko w~porządku, jej rodzina
była bezpieczna. Musieli \textit{ruszyć}.

Gretyl dała się poprowadzić. Jej umysł się rozdzielił, jedna frakcja
była przytłoczona smutkiem i~samooskarżaniem się. Druga część -- ta,
która podjęła tę decyzję -- ścigała się przez strategie i~taktyki na
wszystko, co miało nadejść. Nadie powiedziała, że nadchodzące siły nie
uczynią z~nich męczenników, chcą pokazać, że to walka z~przestępczością,
a nie wojna. Nie walczyć o~istnienie społeczeństwa, którego koniec się
zbliżał.

Odchodzące miały coś, czego nie miała strona domyślna: z~wyjątkiem
kilkorga dzieci, każda odchodząca była kiedyś defaultem. Prawie nikt w~default -- i~nikt, kogo ktokolwiek słuchał, kropka -- nigdy nie odszedł.
Gretyl łatwo było nałożyć pogląd default na sytuacje.

Byliby przewrotnie wiwatowani przy walce, przez więźniów i~ich
popleczników -- kiedyś zamkniętych przestępców -- awanturujących się i~gazowanych do uległości i~układanych jak sągi drewna.

Gdyby walczyli, mogłaby to być masakra, ale nie byliby męczennikami.
Byliby ISIS, opętanymi ideologią potworami, które trzeba pokonać z~każdą
godną ubolewania siłą.

Wszystko to, podczas gdy Gretyl wciąż szlochała, każda jej część
obserwowała drugą z~perwersyjną fascynacją, zastanawiając się, która z~nich to prawdziwa Gretyl.

\threeast

Więzienie dla chłopców przyciągało zagorzałych maniaków sieci. Wysłali
biegaczy do więzienia dla kobiet, prosząc, aby każda, kto ma
doświadczenie w~pracy w~sieci, korzystała z~tłumików na algorytmach
routingu, które zrównoważyłyby ich infrastrukturę sieciową, gdy jej
części zostaną usunięte. Gretyl i~Limpopo spojrzały po sobie.

-- Moje miejsce jest tutaj\ldots  -- zaczęła Limpopo. Jedna z~jej przyjaciółek
-- kobieta, która pokazała im pryczę Limpopo, której imienia Gretyl nie
potrafiła zlokalizować wśród napiętych emocji -- prychnęła.

-- Nie bądź idiotką. Nie jesteś naszą babcią, jesteś tylko kolejnym
przestępcą. Nie potrzebujemy cię tutaj, aby się nami opiekować. Rób
swoje. Wszyscy wiedzą, że jesteś gorącym gównem z~programowania i~opsu.
Utrzymywanie naszych kanałów da nam więcej siły niż wpychanie swojego
chudego, starego ciała między nas a~wynajętych zbirów.

Limpopo sfingowała cios, uścisnęła ją szybko, cmoknęła w~policzek i~ruszyły.

-- Oto, o~czym myślę. -- Pobiegły truchtem w~kierunku męskiego więzienia.
-- To miejsce jest strzeżone sto razy lepiej niż gdziekolwiek, w~którym
kiedykolwiek byłaś. Cały czas rejestrowany jest każdy centymetr. Te
kanały trafiają do centrum danych, które stosuje heurystykę, która
uszeregowuje je tak, aby strażnicy mogli je spakować jako infografikę.

-- Możesz stworzyć kanał okrucieństw. -- powiedział Etcetera. Najgorsze
rzeczy, jakie robią, wciągnięte do jednego kanału, który skleja się jak
serial?

-- Chodzi o~to, aby zapobiec okrucieństwu -- powiedziała Gretyl. -- Nadie
powiedziała, że nie chcą męczenników, kolejnego Akronu.

Dotarli do bram, gdy z~nieba zaroiło się od dronów, które zdawały się
bombardować dachy więzień.

-- Och -- powiedziała Gretyl. Limpopo słuchała wiadomości.

-- Nie -- powiedziała -- to było celowe. Utrzymują szkieletową załogę
routerów powietrznych, wystarczającą dla sygnałów i~telemetrii, a~reszta
ląduje w~klatkach Faradaya aż do pierwszego EMP. To standardowa taktyka,
mówią, że używali jej w~Nigerii, co dla mnie nie ma sensu.

-- To było wielkie -- powiedziała Gretyl -- ale musieli je zablokować.
Zaczęło się od tych pływających miast w~pobliżu Lagos. Oderwały się od
lądu w~powstaniu odchodzących, dosłownie, straciły zasilanie i~kanalizację, nie miały wystarczająco dużo na pokładzie, aby zachować
stabilność. Wynajęto najemników z~subkontynentu, by spacyfikowali Lagos.
Odchodnicy zachowali je, trzymali routery w~pudełkach i~wypuszczali je
krótkimi seriami, mieszając się z~dronami najemników, aby musieli
pulsować własne urządzenia, aby dostać się do tych odchodnickich. Kanały
prawie nie zawieszały. Sprawiło to, że najemnicy wyglądali jak dupki.

Dotarli do pomieszczenia operacyjnego IT w~trzeciej piwnicy. Młody
chłopak, niemający więcej niż czternaście lat, oprowadził ich
entuzjastycznie po cechach centrum operacyjnego, opancerzonym przewodem
łączącym je z~twardymi łączami światłowodowymi i~przewodami wewnątrz
budynku, zapasowymi bateriami\ldots 

-- Ma własny dopływ powietrza? -- powiedziała Gretyl.

Chłopiec wzruszył ramionami. Był chudy, z~małym afro, długimi ramionami
i palcami, z~twarzą pełną figlarnych oczu. 

-- Każdy blok ma przegrody,
które mogą zamknąć, aby zagazować tylko jedną część. Tanie i~skuteczne,
o ile Twoja drużyna ma maski. 

To wyjaśniało wszystkie dzieciaki w~maskach przeciwgazowych w~drodze do więzienia. W~więzieniu dla kobiet
prawie ich nie było. Zrobili własne z~chustek z~mikroporami i~gogli.
Wychodząc, minęli małą, sprawną linię montażową.

To był surrealistyczny punkt obserwacyjny, aby być świadkiem bitwy w~więzieniach TransCanada. Na linii frontu korzystali z~prywatnej ochrony,
a nie z~najemników. Byli to gliniarze zewnętrzni, faceci do wynajęcia w~mundurach, z~których korzystały miasta z~prywatnymi siłami policyjnymi,
i inne miasta, żeby złamać związki policyjne, ilekroć się stawiały.
Mężczyźni i~kobiety w~niesamowitych kamizelkach kuloodpornych, rzeczy,
które wyglądały jak Hugo Boss z~odrodzenia cyberpunku, bez twarzy i~osłonięci, każdy z~nich stanowił ,,jednoosobową armię'' z~egzoszkieletem
i przerażającą bronią bekającą. Była ich setka, ze wsparciem dronów i~szybkimi, nieśmiercionośnymi robotami pościgowymi, bezgłowymi gepardami
i psami ze stopu i~miękkich solenoidów, wystarczająco sprytnych, by
powalić cel i~przytrzymać ją na ziemii, jeśli spróbuje wstać przed
podpisanym, sygnałem ,,w porządku''.

Część Gretyl, która martwiła się o~swoją rodzinę, zniknęła, gdy ona i~Limpopo przekazali szczegóły globalnej publiczności odchodzących,
wikując środki zaradcze dla każdego planu ataku, jaki mogli sobie
wyobrazić. Gretyl przypomniała sobie swoją wcześniejszą myśl o~tym, że
wszyscy odchodzący były defaultem, ale nie odwrotnie, zastanawiała się,
co zrobili z~tego domyślni agenci kontrwywiadu, którzy to monitorowali.
Wiedzieliby, że to właśnie robią odchodzący, pracują na otwartej
przestrzeni, ale wiedzieli też, że kiedy walczyli ze sobą, używali
skomplikowanych fałszerstw. Czy byliby w~stanie \textit{uwierzyć}, że
odchodzący po prostu pozwalają temu wszystkiemu leżeć, gdzie wrogowie
mogą to zobaczyć?

Siły wroga utworzyły obwód wokół kompleksu więziennego. Kamery śledziły
wzorce przelotu dronów, fale milimetrowe, podczerwień i~odbicia.

Formalności: 

-- Wszystkie osoby przebywające w~tych pomieszczeniach
podlegają aresztowi. Nie doznacie krzywdy, jeśli wyjdziecie z~rękami na
widoku. Będziecie mieć kontakt z~adwokatem. Obserwatorzy praw człowieka
są pod ręką, aby zapewnić poszanowanie praworządności. Macie dziesięć
minut. 

Dźwięk dochodził z~wielu więziennych głośników, a~także z~megafonów wbudowanych w~zbroje bojowe linii frontu. Fakt, że nadal mogli
przesyłać dźwięk do wewnętrznych głośników, sprawiał, że serce każdego
odchodzącego od niego drżało. Oznaczało to, że cała praca, jaką wykonali
odchodnicy, aby zabezpieczyć swoje sieci i~wykorzenić tylne drzwi
pozostawione przez TransCanadę, była niewystarczająca. Sugerowało to, że
wróg ma dostęp do ich kamer, może uruchomić gaz, uszczelnić grodzie\ldots 

Chłopcy sieciowcy rzucali się, grzmiąc palcami po interfejsach. Gretyl
wymieniła wszystko, czego się dowiedziała o~operacjach sieciowych
TransCanada, rzeczy, które poprawiała, gdy wezwano ją do pomocy w~czymś
trudnym. Limpopo chwyciła ją za ramię, przywołał infografikę, zaczęła
analizować ostatni ruch na serwerach audio. Etcetera dostał to, na co
patrzyła, szybciej niż Gretyl. Krzyczał na chłopców, podsuwając im
sugestie dotyczące portów do zablokowania, odcisków palców ruchu
sieciowego, których należy szukać w~inspektorach pakietów.

Gretyl odnalazła spokój w~świadomości, że są we wzajemnych pętlach
decyzyjnych. Zajmowała się diagnostyką od odchodzących obserwujących
sieć, bez słów aktualizując modele stanowiące podstawę infografiki
Limpopo, zauważając, gdy Limpopo włączyło nowe dane do swojej analizy.

W ciągu czterech minut znaleźli cztery tylne drzwi. Trzy były trywialne,
konta dostępowe, które powinny zostać usunięte wszędzie, ale zostały
usunięte \textit{prawie} wszędzie. Czwarty był trudniejszy do usunięcia,
ponieważ wymagałby ponownego uruchomienia całego systemu, aby go złapać.
Rozwiązali to, budując wielką, głupią regułę filtrującą, która szukała
wszystkiego, co mogło próbować się do niej zalogować, zrzucając te
pakiety. Gdy kończyli tę robotę, ktoś w~Redmond pilnie wysłał Gretyl
wiadomość, którą przegapili, że może nawet przeoczyła tę, która została
wbudowana przez producenta w~celu odzyskania licencji dla martwych
klientów, którzy przegapili płatności, co pozwoliłoby im umieścić cały
system w~trybie minimalnej pracy. Teoretycznie można było wywołać ten
tryb tylko wtedy, gdybyś posiadała klucz podpisu Siemensa, ale byłoby
naiwnością zakładać, że kanadyjski wywiad prowadzący ten program go nie
posiadał.

-- Nie ma sposobu, aby wiedzieć, czy to wszystkie -- powiedział Etcetera,
wypowiadając myśl, o~której wszyscy myśleli, z~maszynową dosadnością.

-- Nie -- powiedziała Limpopo.

-- Wszyscy z~zewnątrz patrzą -- powiedziała Gretyl. -- Jeśli jest więcej
podatności, znajdą je.

-- W~końcu -- powiedział Etcetera.

-- Masz wsparcie -- powiedziała Gretyl. -- O co się martwisz?

-- O Ciebie. -- To ją uciszyło.

-- Potrafisz być totalnym dupkiem -- powiedziała Limpopo, ale bez
prawdziwej urazy. Chłopcy poprawiali swoje poprawki, wdrażając
rozwiązania awaryjne, ale chichotali z~,,dupka''.

-- Macie dwie minuty -- powiedział głos. 

Tym razem dobiegał tylko od
krzykaczy na zewnątrz, wychwycony przez kamery wycelowane w~najeźdźców.
Niektóre z~ich kamer były oślepione przez pulsującą lekką broń, ale ten
atak był przeznaczony dla instytucji cywilnych, a~nie ufortyfikowanych
więzień. TransCanada wydała prawdziwe pieniądze na redundantne systemy
wizyjne. Musieli zirytować swoich akcjonariuszy, aby zobaczyć, jak
wszystkie pieniądze, które można było wypłacić jako dywidendy, zostały
przekierowane\ldots 

W uchu Gretyl zadzwonił telefon. Dotknęła go, zakładając, że będzie
dzwoniła Lodołasica, żeby upewnić się, że wszystko z~nią w~porządku, co
doceniła i~oburzyła się -- \textit{jestem trochę zajęta, kochanie} -- ale to
był męski głos.

-- Czy to Gretyl Jonsdottir?

-- Tak. -- Telefon przyszedł z~tożsamości przeznaczonych tylko dla
przyjaciół i~rodziny. Nie był znany nikomu, kto tak brzmiał.

-- Gdzie jest Natalie Redwater?

-- Kto to? -- Oczywiście, że wiedziała.

-- Tu Jacob Redwater. Jej ojciec.

Teoria gier Gretyl rozkręciła się, grając w~różne gambity, próbując
różnych teorii dla tego, czego chciał Jacob Redwater. Niewątpliwie
wiedział o~jej związku z~Lodołasicą, musiał wiedzieć o~chłopcach,
wiedział, że jest tam Jacob Redwater Drugi. Porwał Lodołasicę,
zdecydowany zrobić z~niej zettę, Redwatera. Zraniła go tam, gdzie był
najsłabszy, w~kasie, a~on musiał być wściekły.

Musiał czuć do niej jakąś dziwną wersję miłości. Znała zetty w~Cornell,
patronów jej laboratorium. Musiała chodzić z~nimi na obiady, zbierać
fundusze, spędzać setki godzin na pogawędkach o~wysokiej stawce, pod
czujnym okiem szefa wydziału. Rozmowa z~nimi nie była nieprzyjemna,
wielu z~nich było dowcipnymi rozmówcami. Ale było w~nich coś\ldots  nie tak.
Dopiero po kryzysie sumienia i~odejściu od Cornella była w~stanie to
nazwać: nie mieli syndromu oszusta. Nie mieli cienia wątpliwości, że
każdy przywilej, którym się cieszyli, był zasłużony. Świat był
prawidłowo ułożony. Na górze byli ważni ludzie. Nieważni byli na dole.

Gdyby powiedziała Jacobowi Redwaterowi, że Lodołasica uciekła, czy
użyłby swoich wpływów, by atak na więzienia był bardziej brutalny? A
może (czy mógłby?) odciągnąć siły od więzień, by ścigać Lodołasicę?
Więcej chłodu: czy Jacob Redwater pracował z~Nadie? Czy Nadie porwała
Lodołasicę, być może, żeby zawrzeć sojusz z~resztą fortuny Redwaterów?

Przestraszyła się. Poszła wprost: 

-- Czego chcesz?

-- Chciałbym porozmawiać z~moją córką.

-- To nie jest możliwe. -- Trzymała się prawdy, jeśli nie całej.

-- Pani Jonsdottir, wiem, że kocha pani moją córkę.

-- To prawda.

-- Trudno ci w~to uwierzyć, ja też ją kocham.

\textit{Masz rację, trudno mi w~to uwierzyć.} 

-- Jestem pewna, że tak, na swój sposób. -- Nie miała zamiaru zachowywać się mikroagresywnie, ale to
się wymknęło. Jak mogła to przepuścić?

Udawał, że tego nie zauważył, chociaż była pewna, że to zrobił. 

-- Nie chcę\ldots  -- Ogarnęły go jakieś emocje albo był bardzo dobrym aktorem. Albo
jedno i~drugie, przypomniała sobie. Zetty, które znała, były dobre w~dzieleniu na przedziały, w~socjopatycznym stylu, rozumiały emocje innych
ludzi na tyle dobrze, by nimi manipulować, bez doświadczania
rzeczywistej empatii. 

-- Są dzieci -- powiedział. -- Jej dzieci.

-- Moje też -- powiedziała Gretyl.

-- Tak.

-- Cokolwiek się wydarzy, nie musi przytrafić się mojej córce ani wnukom.
Twoim dzieciom.

-- Gdzie pan jest, panie Redwater? Jesteś przy więzieniu?

-- W~rzeczywistości, jestem.

Tak myślała, hałas w~tle był echem dźwięków, które słyszała przez
zewnętrzne kamery więzienne.

-- Wiedziałeś, że nadchodzą.

-- Wiedziałem. Dlatego przyszedłem. Aby zapewnić Natalie bezpieczeństwo.
-- Nastąpiła chwila. -- Mogę cię stąd wydostać.

-- Dlaczego nie rozmawiasz z\ldots  -- omal nie powiedziała:
\textit{Lodołasicą}, a~potem \textit{Natalie}, zdecydowała -- Twoją córką?

-- Nie odpowiada. Nie było łatwo zdobyć ten adres do Ciebie, ale musiałem
przekazać jej wiadomość. Wiem, że nie poświęciłabyś swoich dzieci dla
ideologii.

Jebać to. 

-- Myślisz, że Lodołasica by tak zrobiła?

-- Myślę, że moja córka jest na mnie słusznie zła. To znaczy, że nie mogę
jej wyjaśnić pewnych\ldots  faktów. Nie możemy nawet prowadzić tej dyskusji.

-- Wkraczają do nas, panie Redwater. Nie możemy prowadzić tej dyskusji,
jeśli jestem atakowana.

-- Nie mogę ich odwołać.

Nic nie powiedziała. Lodołasica poświęciła wystarczająco dużo uwagi, aby
wiedzieć, że gałąź rodu Jacoba Redwatera przejęła kontrolę nad główną
fortuną dynastyczną, czyniąc go głównym rodzinnym maklerem. Gretyl
byłaby zdziwiona, gdyby nie posiadali dużego udziału w~TransCanada, nie
mówiąc już o~zlecaniach dla gliniarzy.

-- To nie moja gra. Szczerze mówiąc.

-- Nie sądzę, że mamy o~czym rozmawiać. -- Rozłączyła się. Limpopo
patrzyła w~zamyśleniu.

-- Moi teściowie są naprawdę pojebani.

Etcetera roześmiał się, dziwny dźwięk z~głośnika. W~śmiechu symulatora
zawsze było coś dziwnego. Niektóre ostre krawędzie, wzmocnione
zderzakami symów. Gretyl została zeskanowana. Może właśnie tak śmiałaby
się w~przyszłości z~wybryków swoich synów.

Czas się skończył. Drony gliniarzy z~zewnątrz opadały w~kontrolowanych
nurkowaniach, sygnalizując zbliżający się atak. Chłopcy w~centrum danych
wydawali oszałamiające, przerażone dźwięki, gdy lądowali swoją
szkieletową flotą, goniąc drony gliniarzy, dokładnie w~chwili, gdy
pierwsza salwa pocisków przeszyła niebo, śledząc drony, zabijając ponad
połowę, zanim wylądowały. Łącza światłowodowe zostały zerwane, z~wyjątkiem tych, które zostały potajemnie wykopane i~połączone
bezpośrednie w~przekaźniki mikrofalowe, daleko od więzienia, na polach
rolników.

Palce Gretyla i~Limpopo zderzyły się, gdy dźgały te same miejsca na
infografice, przełączając usługi na te łącza, dostrajając pamięci
podręczne i~systemy równoważenia obciążenia, aby dostosować się do
nagłego spadku przepustowości o~dwa rzędy wielkości. Ruch w~więzieniach
i poza nimi ustawiał się teraz w~kolejkach głęboko w~cache repeaterów.
Na świecie inne cache robiły to samo. \textit{Sieć interpretuje cenzurę
jako zniszczenie i~tworzy trasy wokół niej}, pomyślała Gretyl i~uśmiechnęła się do starożytnego hasła sprzed odejścia. To była prawda
przez jakiś czas, potem metafora, potem myślenie życzeniowe, a~teraz
była to specyfikacja projektu.

Była w~fazie, ludzki koprocesor złożonego systemu, który wykorzystywał
maszyny jako system nerwowy do połączenia inteligencji globalnego tłumu
ludzi, których kochała całym sercem. Ta część niej, która krzyczała i~płakała, kiedy odesłała żonę i~dzieci i~została w~domu, obudziła się na
chwilę i~zauważyła, że \textit{to} był prawdziwy powód, dla którego to
zrobiła. To niesamowite uczucie siły i~połączenia z~czymś większym.
Minęły lata, odkąd Gretyl to poczuła. Teraz znów to poczuła, zdała sobie
sprawę, jak bardzo za tym tęskniła. Życie w~lepszym narodzie było lepsze
niż życie w~gorszym, ale życie w~pierwszych dniach narodu stanowiło
różnicę między zakochaniem się a~kochaniem. Zdradziła swoją żonę.
Kontynuując romans ze zbrojną insurekcją.

Więzienia miały obrony. Nadchodzące siły wiedziały dokładnie, gdzie
były. Tłum miał pomysły na ten temat. Gdy mechy z~taranami ustawiły się
na pozycjach przed każdą bramą, oślepiające przeciwkamery więzienia
ożyły, rzucając wiązki potężnego światła o~szerokim spektrum
bezpośrednio w~czujniki mechów. Byli przed tym osłonięci, ale
niedoskonale, co oznaczało, że mechy musiały zwolnić i~polegać na
czujnikach ultradźwiękowych, aby kierować swoim przejściem. Obrońcy
uruchomili dźwiękową broń przeciwpiechotną więzienia, przeniesioną z~bloków na zewnętrzne mury. Mechy zwolniły bardziej. Wtedy obrońcy
otworzyli atak z~armatek wodnych.

W normalnych warunkach strumienie wody nie przeszkadzałyby mechom. Miały
doskonałe żyroskopy. W~razie potrzeby mogłyby przyjąć trzypunktowe
stabilne pozycje. Ale były połączone ze sobą przez tarany, które
trzymały w~siatce dwa na dwa. Strumienie uderzały w~nie pod różnymi
kątami, więc korekcyjne szuranie przednich jeszcze bardziej zaburzało
równowagę tylnych i~na odwrót. Wodni kanonierzy ustanowili rytm, który
jeszcze bardziej wytrącił ich z~równowagi. W~ciągu kilku minut rozsiadły
się dwie mecha drużyny. Trzecia cofała się chwiejnymi krokami.

Gretyl słyszała wiwaty, widziała radość na czatach. Wiedziała, że to
tylko potyczka. Byli ogromnie, fizycznie przewyższeni. Prywatni
gliniarze wycofali się za zbroję i~salwa granatników przeleciała przez
powietrze, mierząc w~każdą armatkę wodne i~otwory dla wiązek
przeciwfotograficznych. Spodziewali się tego, ale i~tak było to
przerażające, nawet gdy kanały wypełniły się manifestami szkód, szacując
całkowity koszt fizycznego zakładu TransCanada, obserwując spadek cen
akcji firmy, gdy analitycy default czytający im przez ramię zmieniali
swoje zakłady na to, czy TransCanada skończy się z~użyteczną fabryką,
czy też z~płonącymi gruzami na koniec dnia.

Jej telefon ponownie zadzwonił.

-- Dzień dobry.

Nastąpiło opóźnienie, a~potem głos Jacoba Redwatera, napięty i~skompresowany. 

-- Chcę porozmawiać z~moją córką.

-- Wyraziłeś się jasno więcej niż raz.

Nastąpiła długa pauza. 

-- Jej matka zmarła w~zeszłym roku.

-- Moje kondolencje.

-- Nie mogłem znaleźć sposobu, żeby jej to powiedzieć.

-- Upewnię się, że wie.

Chłopcy z~centrum danych unosili rój dronów, w~tym niektóre, które
ukryli w~lesie, za liniami wroga. Kanały dronów pokazywała siły wroga,
zdyscyplinowane i~nieruchome, gotowe do kolejnego ataku. Uszkodzone
mechy pokuśtykały do tyłu. Wróg otworzył ogień do ich dronów. Chłopcy
pchnęli je w~automatyczne, bardzo intensywne manewry wymijające, które
obniżyły cykle pracy ich akumulatorów o~połowę. Prawie wszystkie
przeżyły pierwszą rundę, chociaż ich przekaz wideo zmienił się w~plątaninę przyprawiających o~mdłości nagrań z~kolejki górskiej. Chłopcy
nie wysyłali ich w~przypadkowych schematach, każdy z~nich znalazł się w~pobliżu wrogiego drona obserwacyjnego, jadące na jego ogonie. Jeżeli
siły naziemne uruchomiły broń HERF, która usmażyłaby drony i~zrzuciła je
z nieba, załatwiłyby również własne samoloty.

-- Dobra robota! -- Gretyl krzyknęła przez ramię na chłopaków, którzy nie
potrzebowali nikogo, żeby im to mówić, tańczyli ze zwycięstwa. Tymczasem
krótkie loty dronów zdołały usunąć 75 procent zaległości w~sieci,
znacznie zmniejszając przeciążenie ocalałego światłowodu.

Jacob Redwater powiedział: 

-- Wasze zapasy dronów są ograniczone. Możemy
korzystać z~zapasów.

-- Tak. Nie możemy tego wygrać siłą.

Włączyły się zewnętrzne głośniki. Głos przemówił: 

-- Gordy, tu Tracey.
Twoja siostra Tracey, Gordy. Wiem, że nie rozmawialiśmy, odkąd odeszłam,
ale chcę, żebyś wiedziała, że cię kocham. Jestem bezpieczna i~szczęśliwa. Myślę o~Tobie codziennie. Masz teraz siostrzenicę,
nazwaliśmy ją Eva, jak mama. Sposób, w~jaki tu żyjemy, jest lepszy, niż
kiedykolwiek myślałem. Ludzie są dla siebie mili, Gordy, tak jak wtedy,
gdy byliśmy dziećmi. Ufam moim sąsiadom. Troszczą się o~mnie. Troszczę
się o~nich. Nie jesteśmy terrorystami, Gordy. Jesteśmy tymi, których
default nie umiał wykorzystać. Znaleźliśmy dla siebie zastosowanie.
Gordy, nie musisz tego robić. Są inne sposoby na życie. Kocham cię,
Gordy. -- Inny głos, płacz dziecka. -- To Eva, Gordy. Ona też cię kocha.
Chce zobaczyć wujka.

Kanały zbliżyły się do jednego z~frontowców, mężczyzny, któremu drżały
ramiona. To musiał być Gordy. Tłum zidentyfikował go poprzez analizę
chodu, zdoxował, przeszukał jego wykres społecznościowy, znalazł
trafienie w~odosobnionym miasteczku w~Wyoming, wyciągnął Tracey z~łóżka,
nagrał wiadomość.

Cisza się przedłużała. Prywatny gliniarz obok Gordy'ego z~wahaniem
położył mu rękę na ramieniu. Gordy otrząsnął się gwałtownie, odpychając
go.

Chwila się przeciągnęła. Potem Gordy zrzucił rękawice machnięciem
nadgarstków, posyłając je w~klekocie po drodze. Jego gołe palce
poruszały zatrzaskami przyłbicy, dopóki nie otworzył się szeroko. Jego
twarz była niewyraźną, ruchomą plamą, brązową z~poprawionymi przez
kamerę smugami bieli w~miejscu jego zębów i~oczu. Zdjął hełm, zrzucił
broń i~pozwolił jej opaść wokół jego stóp.

Otaczający go gliniarze wpatrywali się w~niego, telegrafując językiem
ciała otwarte usta za przyłbicami. Odszedł prostopadle do więzień i~linii gliniarzy, w~górę 15-tką w~kierunku Ottawy, w~kierunku mleczarni i~dolin wschodniego Ontario.

Odszedł.

Cisza była czymś świętym, ciszą kościelną. To był cud, nawrócenie na
polu bitwy.

-- Akin! -- głos był wzmocniony zza linii policyjnych, wystarczająco
głośny, by zabrzęczały szyby. -- Wracaj do linii, Akin! 

Był to głos rozkazujący, skręcający dupę głos wydający rozkazy. Ramiona Gordy'ego
zesztywniały. Gordy szedł dalej, pozbywając się kamizelek kuloodpornych,
zrzucając kurtkę na drogę za sobą. Głowę miał wysoko, ale ramiona
trzęsły mu się, jakby płakał.

Jeden z~gliniarzy na pierwszej linii uniósł broń z~lufą wielką jak
armatę, zbudowaną tak, by wysyłać skoncentrowane, rozdzierające jelita
ultradźwiękowe na cel: nazywali go Wypadem. Mężczyzna, którego ręka
została zrzucona z~ramienia Gordy'ego, zaatakował mężczyznę z~bronią,
zanim ten zdążył wycelować. Wili się na ziemi, aż zostali rozdzieleni
przez kolejnych gliniarzy i~stanęli twarzami do siebie, trzymani za
ramiona, z~falującymi piersiami.

Gordy zniknął za wzgórzem.

Oddech Jacoba Redwatera był głośny w~telefonie Gretyla.

-- Nie możemy tego wygrać siłą. 

Odłożyła słuchawkę, gdy z~głośników
skierowanych na zewnątrz w~więzieniu zaczęło odtwarzać się następne
ogłoszenie.

\threeast

W połowie trzeciej zapowiedzi policjanci otworzyli ogień do głośników,
kolejne RPG-i. Więźniowie przerzucili się na kopie zapasowe, poza
zasięgiem wzroku za linią dachu. Kiedy drony gliniarzy podniosły się, by
zobaczyć, były nękane przez kolejne drony, które ścigały ich po niebie,
a nawet popełniły samobójstwo na dwóch dronach policyjnych. Podczas gdy
szalała bitwa powietrzna, wydali jeszcze cztery ogłoszenia. Dostali pięć
wyjść z~siedmiu ogłoszeń. Tłum wrzucał szerokie uśmiechy na forach.
Złapali gliniarzy na liniach tak szybko, jak tylko mogli, wykonując
swoje wykresy, znajdując więcej osób do nagrywania wiadomości.

Gretyl pokręciła głową ze zdumienia, gdy pojawiały się nagrania. W~męskim więzieniu ktoś grał DJ-a, ustawiając je w~kolejce. W~więzieniu
dla kobiet ktoś inny robił to samo. Jeden z~chłopców w~sterowni zrobił
to dla gliniarzy przed ich instytucją. Była sceptycznie nastawiona do
planu.

Zmieniło się to w~teorię wykresów: gdy trafisz na krytyczną masę osób,
które odchodzą, sześć stopni oznaczało, że każdy wynajęty gliniarz na
linii był nie więcej niż dwa uściski dłoni -- lub rodzinne świąteczne
obiady -- z~dala od odchodzących, którzy by zawstydzili i~namawiali ich,
aby odłożyli broń.

Ogłoszenia od ósmej do dziesiątej grały w~głośnikach na gzymsie, zanim
gliniarze sprowadzili moździerze -- \textit{moździerze }! -- żeby zaatakować
mury, zamieniając je w~sterty gruzu pośród kłębiących się tumanów kurzu.
Akcje TransCanady gwałtownie spadły. Zaraza rozprzestrzeniła się na
wszystkie \textit{inne} miejsca, w~których ukrywali się odchodnicy,
uniwersytety, placówki badawcze, wszystkie te ośrodki detencyjne dla
uchodźców. Kiedy rynek zobaczył, co trzeba będzie zrobić, aby przywrócić
te obiekty do wyjściowego stanu, inwestorzy spanikowali. Zawsze
sprzedawali się w~panice, za każdym razem, gdy wybuchała jedna z~tych
walk. Sprzedawali nawet prawdziwi wyznawcy wyższości zetta. Korzeniem
\textit{kredytu} było \textit{credo}: wierzę. Obserwowanie, jak wynajęci
policjanci wyciągają swoje wielkie spluwy, by wymazać \textit{głośniki},
miało ogromny wpływ na nastroje zwierząt na rynku: ich system wierzeń
walił się tak jak za każdym innym razem.

Więcej dronów: z~głośnikami, dronów kontrolujących tłum, które były
dostępne w~więzieniu, tak dużych, że potrzebowały dodatkowej awioniki,
aby je skorygować od wibracji własnych głośników.

Drony skierowały się do mężczyzn i~kobiety, których namierzyły,
zamieniły ich w~okulary, podczas gdy ich koledzy z~oddziału wpatrywali
się w~opancerzonych gliniarzy otoczonych aureolą krążących dronów, zbyt
blisko ich ciał, by bezpiecznie zestrzelić, nawet w~zbroi. Co jeśli ich
ogniwa wodorowe wybuchną? A jeśli zostali uwięzieni w~pułapce?

Kiedy wydano rozkaz, by zawrócić tych nieszczęsnych drani, ruszyli w~kierunku transporterów opancerzonych na tyłach formacji, otoczeni przez
brzęczące drony, które nawiedzały ich jak ogromne muchy owocowe o~dużym
głosie. W~jednym przypadku dron zdołał wślizgnąć się do transportera
opancerzonego wraz ze swoim celem. Wielki podobny do czołgu samochód
zakołysał się na zawieszeniu, gdy gliniarze w~środku gonili za nim,
wariując jak tłum parafian goniących zagubionego nietoperza. Nagranie z~tego drona przedstawiało klaustrofobię typu rybie oko. W~końcu ruch
ustał, gdy dron został uderzony w~pokład transportera opancerzonego.
Chwilę później otworzył się właz transportera opancerzonego i~odeszło
trzech kolejnych gliniarzy, dwie kobiety i~mężczyzna. Mężczyzna i~jedna
z kobiet kłócili się z~drugą kobietą, być może próbując przekonać ją, by
została, ale wszyscy zostawili broń na poboczu drogi, gdy wyruszyli do
Ottawy.

Sprawy się ułożyły. Więźniowie mieli teraz cholernie niewiele sposobów
na nawiązanie kontaktu z~glinami, co oznaczało, że nie było żadnych
negocjacji. Wcześniej też nie było.

Zadzwonił telefon Gretyl.

-- Musisz zabrać stamtąd Natalie. Teraz.

Gretyl poczuła, jak ściskają jej wnętrzności. Może to była sztuczka
zetty, żeby ich wypłoszyć, skłaniając ich do myślenia, że nadchodzi
wielkie pchnięcie. Redwater nie był ponad tym. Jednak brzmiał na
zdesperowanego w~nie-Redwaterowy sposób.

-- Nikt nie wychodzi, dopóki wszyscy nie wyjdziemy. -- Ostrożnie unikała
potwierdzenia aktualnej lokalizacji Lodołasicy. Domyślała się, że to
oznaczało, że Nadie nie pracowała dla staruszka, bo inaczej dałaby mu
znać, że jego cenna linia krwi jest bezpieczna.

-- Dzieci\ldots 

-- Tutaj jest \textit{wiele} dzieci. Dlaczego to ma znaczenie, czy są z~Tobą spokrewnione?

Wydał szczenięcy dźwięk, między szczekaniem a~skomleniem. 

-- Ty zła suko.

-- To nie ja mam wszystkie spluwy. Czy jest pan tutaj, panie Redwater?
Czy widzisz, co się dzieje?

-- Mogę to zobaczyć. To dobry teatr. Gretyl. Jestem pewien, że twoi
przyjaciele są tym podekscytowani, Gretyl. Ale to nie będzie miało
znaczenia za pięć minut.

-- Jeśli zostało mi tylko pięć minut, lepiej będę się nimi delektować. -- Znowu się rozłączyła.

-- Dlaczego go nie zablokujesz? -- spytała Limpopo.

-- Ponieważ tak długo, jak będzie z~nim rozmawiać, może być w~stanie
przekonać go, by nie pozwolił swoim kumplom wysadzić nam tyłków -- powiedział Etcetera.

Gretyl pokręciła głową. Spojrzała na infografikę, obserwowała przepływ
ruchu w~sieci, zastanawiała się, czy to prawda, że Jacob Redwater mógł
być ich zbawcą, czy to on w~ogóle był przyczyną ich działających
połączeń sieciowych, żeby mógł do niej zadzwonić. 

-- Może to część tego. Ale to jest dupek, który zabrał moją żonę, kurwa, porwał ją. To nie jest
miłe, ale lubię, gdy się wierci.

Limpopo wzruszyła ramionami. 

-- Twoje ostatnie minuty na Ziemi i~spędzasz
je na drobiazgowej zemście? To Twoje życie.

To zabolało. To była prawda. Limpopo zawsze była lepsza w~dużym obrazie
i przeżywaniu teraz. Więzienie uczyniło ją jeszcze bardziej stoicką.
Gretyl próbowała sobie wyobrazić, co znosiła przez lata.

Łącza sieciowe nagle się urwały, ich drony wystrzeliły z~nieba w~tym
samym czasie, gdy odcięły się światłowody.

-- Chyba nie będę miała okazji przeprosić starego drania. -- Sięgnęła po
rękę Limpopo. Jej uścisk był suchy, a~dłoń wydawała się słaba, ale była
ciepła i~ścisnęła ją z~powrotem.

-- Kocham cię, Limpopo.

-- Ja też cię kocham.

-- Ja też -- powiedział Etcetera.

-- Dziękuję.

Mocno ścisnęły dłonie.

\threeast

Chłopcy trajkotali jak małpy na drzewie. Niektórzy zadawali niecierpliwe
pytania dwóm staruszkom trzymającym się za ręce i~wpatrującym się w~infografikę, ale Gretyl i~Limpopo nie miały im nic do powiedzenia.

Kamery nadal dostarczały przekazy z~zewnątrz, ponieważ lokalna sieć
nadal działała, nadal gromadziła materiał do wysłania reszcie świata.
Linie policyjne się zacieśniły. Nie było w~nich więcej identyfikowalnych
ludzi. Wszyscy byli wewnątrz mechów i~transporterów opancerzonych albo
zatrzymali się daleko za policyjnymi autobusami i~przyczepami
administracyjnymi, które przyjeżdżały na platformach. Stratedzy po
drugiej stronie nie zaryzykowaliby więcej psy-ops ze strony więźniów,
nawet gdyby oznaczało to walkę zza zbroi. Taktyka mechów i~transporterów
opancerzonych była przede wszystkim zabójcza, wszyscy o~tym wiedzieli.
Nie można aresztować kogoś z~wnętrza ogromnego półczołgu lub kombinezonu
zabójczego robota. Można było ogłuszyć lub zabić, ale nie można było
odczytać praw ani zakuć w~kajdanki.

Mechy podeszły sprytnie do przodu i~podłożyły ładunki wokół ocalałych
murów obwodowych, cofając się na trzech nogach, rozpłaszczając się przed
eksplozją, która wstrząsnęła ścianami, sprawiając, że fundamenty
zatrzęsły się, nawet w~ich podziemiach.

Kamery na tej ścianie pociemniały. Przestawili kamery z~wewnętrznego
dziedzińca na swoje infografiki i~obserwowali powtarzające się
ćwiczenie. Transportery opancerzone przetoczyły się, tworząc pancerną
ścianę, mechy przeszły nad nimi, podłożyły nowe ładunki i~wycofały się.
Gretyl odruchowo sprawdziła, co robią rynki, ale oczywiście nie było
żadnego zewnętrznego kanału. Nie miało to dla nich znaczenia. Nadchodził
koniec. Pierwsze dni lepszego narodu. Ostatnie chwile wyniszczonych,
kruchych fizycznych ciał jakichś głupich, niedoskonałych odchodniczek.
Gretyl nie dała się odciąć, zmusiła się do patrzenia na ekrany,
patrzenia, jak wali się ściana, ciemnieją kamery. Mocniej ścisnęła dłoń
Limpopo.

Zadzwonił jej telefon.

Spojrzała na infografikę, zobaczyła jakoś, że sieci wróciły online.
Sieci, które gliniarze fizycznie przejęli, obcięte prawdziwymi
przecinaczami kabli, znów były online. Zadzwonił jej telefon.

-- Proszę. -- Płakał.

-- Panie Redwater?

-- Proszę. Nie mogę\ldots 

Prawie ustąpiła. \textit{Śmiało, zabij nas, Twoja córka i~wnukowie są
daleko stąd}. To była odruchowa myśl, wspólne miłosierdzie dla starca,
którego głos łamał się ze smutku.

-- Jeśli nie możesz, nie powinieneś. Każdy tutaj ma kogoś, kto będzie
opłakiwał ich śmierć. Jeśli masz władzę, by to zatrzymać\ldots  -- Wyraźnie
zrobił, bo jak inaczej wytłumaczyć połączenie sieciowe, mechy i~transportery opancerzone teraz wciąż na dziedzińcach, z~widokiem na
zrujnowane fasady, biura i~magazyny stojące nagie na powietrzu, czwarta
ściana usunięta, wyglądając jak zestawy do dramatów. 

-- Jeśli możesz
zrobić cokolwiek, aby to powstrzymać, możesz uratować ich życie.

-- Nie mogę tego zrobić.

-- Nie zrobisz tego.

-- Czy mogę\ldots  czy przyjdziesz ze mną o~tym porozmawiać?

-- Panie Redwater, z~całym szacunkiem, nie jestem pieprzoną idiotką,
porywający, zły skurwielu. -- Powiedziała to spokojnie, ale jej puls
przyspieszył. Limpopo cicho dopingowała.

-- W~takim razie, czy mogę wejść? Sam?

Pomyślała. Nie było prawdopodobne, by zetta zamienił się w~zamachowca
samobójcę, szantaż lub pranie mózgu kogoś innego, by stał się
zamachowcem samobójcą, oczywiście tak, ale nie ryzykowanie własnej
skóry. W~tempie, w~jakim wszystko szło, wszyscy umrą w~ciągu godzin,
może minut.

-- Nie sądzę, żeby ktoś tutaj miał coś przeciwko temu. A co się tam
dzieje, z~całą tą bronią, czołgami i~zabójczymi robotami\ldots 

-- To mój punkt obserwacyjny. -- Brzmiał, jakby zapanował nad swoimi
emocjami.

-- Zostaw nasze kanały informacyjne. Żadnego glejtu, jeśli nie możemy
dotrzeć do świata zewnętrznego.

Nastąpiła długa pauza. Pomyślała, że mógł się rozłączyć, ale kiedy się
odezwał, usłyszała pojedynczy zwarcie, gdy wyłączył mikrofon. Rozmawiał
z kimś innym.

-- Poza mną. Ale złożyłem prośbę.

Wzruszyła ramionami. 

-- To więzienie dla chłopców. Najbardziej wysunięte
na południe.

Szybko wystukała wiadomość reszcie odchodników, tym w~więzieniach i~w tłumie, wyjaśniając, że Jacob Redwater poprosił o~bezpieczne przejście,
aby móc porozmawiać z~żoną swojej córki. Zasugerowała, ale nie
powiedziała, że Lodołasica i~chłopcy byli w~budynku. Podczas gdy tłum
żarłocznie przeszukiwał tatę Lodołasicy i~rozdzielał ogromne ilości
informacji na temat rozrastających się imperiów Redwater, Limpopo i~Gretyl szeptały sobie nawzajem o~tym, co będzie dalej.

-- Wygląda na to, że się złamał -- powiedziała Limpopo. -- Jakiś rodzaj
ścinania między Jacobem Redwaterem, zettą, a~Jacobem Redwaterem,
człowiekiem. Konwersja na łożu śmierci czy coś. Mówiłaś, że zmarła jego
żona?

-- Tak, ale z~tego, co powiedziała Lodołasica, byli w~zasadzie
rozwiedzeni przez większość jej życia, pod każdym względem oprócz nazwy.
Miała siostrę, nie wiem, co się z~nią stało. Jestem pewna, że ma dostęp
do tylu firm, ile chce.

-- Kimkolwiek jeszcze jest, jest czarujący -- powiedział Etcetera. -- W~ten
inteligentny, socjopatyczny sposób. Fajnie się z~nim kłócić, gdybyś nie
była jego córką.

-- Oto on -- powiedziała Limpopo. 

Chłopcy zebrali się wokół ekranu,
powiększając i~poprawiając obraz z~pozostałych kamer na wewnętrznym
dziedzińcu. Ubrany był w~zielone sztruksy i~puchową kamizelkę nałożoną
na koszulę z~długimi rękawami. Jego włosy były siwe, ale twarz gładka,
postawa wyprostowana. Szedł powoli i~zdecydowanie. Był stary, ale nie
wyglądał na słabego.

-- Czy któryś z~was może go przyprowadzić? -- spytała Gretyl chłopców. -- Nie chcę iść na górę, na wypadek, gdyby zamierzał mnie porwać.

Chłopcy pokłócili się o~to, kto by to zrobił. Wygrał dzieciak o~imieniu
Troy, szesnastolatek z~krótkim afro, swobodnym uśmiechem i~bystrymi,
szybkimi oczami. Pobiegł. Chwilę później obserwowali go na ekranie,
rozmawiającego z~Jacobem Redwaterem, prowadzącego go.

-- To powinno być dobre -- powiedział Etcetera.

Gretyl zastanawiała się, gdzie jest Lodołasica, czy to widzi. W~tłumie
było dużo wrzasków, by transmitować na żywo jej rozmowę z~Redwaterem.
Powiedziała stanowczo nie, zgadzając się na nagranie i~wydanie później,
w zależności od tego, czy będzie później.

Jacob Redwater wszedł do sterowni, poprzedzony falą niedocenianej wody
kolońskiej. Gretyl wstała i~spojrzała na niego z~góry na dół, szukając
wybrzuszeń wskazujących na broń lub inne niespodzianki. Nie, żeby w~tych
dniach musiały się dużo wybrzuszać i~nie żeby wiedziała zbyt wiele o~tym, jakie wybrzuszenia robiły.

Jego twarz była niewzruszona. Płakał kilka minut temu, załamany i~zagubiony. Teraz nosił maskę zetty, dwie części czarujące, wyrafinowane,
a jedną część martwego drapieżnika. Człowieka, który mógłby prowadzić
zabawną rozmowę przy kolacji, a~potem wrócić do domu, zbankrutować
Twojego pracodawcę i~wyrzucić cię na ulicę.

-- Witaj Gretyl. -- Stał przed Troyem, jakby Troy miał broń w~plecach i~udawał, że go tam nie ma.

-- Witam, panie Redwater. -- Wyciągnęła rękę.

Jego dłoń była ciepła i~jędrna. 

-- Mów mi Jacob.

Limpopo rzuciła mu zabawne spojrzenie. Gretyl przypomniała sobie, że
Jacob Redwater kazał ją oddać do tego więzienia, oderwać od rodziny i~wszystkiego, co jest jej drogie. Przywykła do myślenia o~nim jako o~mężczyźnie, który spłodził i~porwał jej żonę, ale był arcynemezis
Limpopo. Zastanawiała się, czy Limpopo wstrząśnie draniem, który z~pewnością na to zasłużył. Już miała rzucić się, by wyciągnąć swojego
wątłego starego przyjaciela, słabszego obok tego energicznego,
niewyobrażalnie bogatego mężczyzny, ale Limpopo wyciągnęła rękę.

-- Limpopo. -- Przechylił głowę, starając się ją rozpoznać.

-- Cześć, Jacobie.

-- Miło cię widzieć -- powiedział Etcetera. Oczy Redwatera się
rozszerzyły. Zaczął od mówcy między jej obojczykami. 

-- To ja, Hubert.
Nie żyję.

-- Widzę. Mimo wszystko miło znów z~Tobą porozmawiać.

Troy przyniósł mu krzesło. Cała trójka siedziała razem, chłopcy skupili
się na drugim końcu pokoju, ostentacyjnie nie słuchając, podczas gdy
zaciekle podsłuchiwali.

Redwater nic nie powiedział. Gretyl oparła łokcie na kolanach i~pochyliła się do przodu, wyginając plecy, by wydobyć z~siebie
skrzypienie i~ból siedzenia i~przerażenia. 

-- O czym chciałeś
porozmawiać, Jacob?

-- Nie chcę, żebyś została zraniona.

-- Nie chcesz, żeby Twoja córka została zraniona. Jesteś obojętny na to,
co się stanie ze starą lesbą, z~którą się związała.

Pokręcił głową. 

-- Nie obchodzi mnie twoja seksualność. Wiesz, mój kuzyn
jest gejem.

-- Wiem. To jest powód, dla którego zarządzasz ostatnio rodzinną fortuną.

Pokręcił głową. 

-- To bardziej skomplikowane. Możesz w~to uwierzyć, jeśli
chcesz. Wewnętrzna polityka rodziny Redwater jest zawsze i~tylko o~jednym.

-- Pieniądze.

-- Władza. Pieniądze tylko liczą się do wyniku.

-- Musiało cię naprawdę rozpieprzyć, kiedy Lodołasica oddała swoją część
tej najemniczce. 

Chciała, żeby się wiercił. Spodziewała się, że będzie
płaczącym mężczyzną przez telefon. Nie chciała umrzeć ze wspomnieniem
wyprostowanego i~dumnego wypalonego w~jej nerwie wzrokowym, dowodu, że
słońce nigdy nie zajdzie nad imperium zetta.

Pokiwał głową. 

-- To skomplikowało sprawy w~naszej rodzinie. Ale to nie
było śmiertelne. Wierzcie lub nie, Nadie i~ja jesteśmy ostatnio w~dobrych stosunkach.

Gretyl zachowała swoją najlepszą pokerową twarz, nie chcąc zdradzić
faktu, że Nadie miała ze sobą Lodołasicę i~chłopców.

-- Chciałbym zobaczyć córkę i~wnuków.

-- Myślę, że zrezygnowałeś z~tego prawa, kiedy ją porwałeś, panie
Redwater -- powiedziała Limpopo. Spojrzeli na nią. Jej oczy błyszczały
niebezpiecznie. -- Kiedy mnie zniknąłeś.

-- Kiedy mnie zamordowałeś -- powiedział Etcetera.

Redwater był niewzruszony. Gretyl wydawało jej się, że widziała
niespokojne opowieści, nagłe uświadomienie sobie tego aroganckiego
książątko, że był trzy poziomy pod ziemią, otoczony przez ludzi, którzy
byli mu winni przemoc.

Mówił ostrożnie. 

-- Nie powiedziałem, że mam do tego prawo. Rzeczy, które
się wydarzyły, są nie do pożałowania. Były okropne. Przywiozłem Natalie
do domu, bo wiedziałem, że przed Tobą i~Twoimi przyjaciółmi stoją
kłopoty. Morderstwa tych dwóch ochroniarzy przelały czarę. Po tym nie
było mowy, żeby wszystko potoczyło się normalnie. Chciałem, żeby była
bezpieczna. To, co nastąpiło później, to, co ci się przydarzyło, nie
miało ze mną nic wspólnego.

Ona i~Limpopo zaczęli mówić jednocześnie, urwali się i~spojrzeli na
siebie. Limpopo wykonał gest ,,proszę''. 

-- To Twój teść. -- Uśmiechnęła
się ironicznie.

Jacob Redwater odwzajemnił uśmiech, udając, że nie zauważył jadu.

-- Jakie morderstwa, Jacobie?

-- Dwie osoby, które Zyz stracił w~kompleksie ,,uniwersytetu''. Weszły,
nigdy nie wyszły. To było wystarczająco złe. Jednak potem odkryliśmy, że
zostali schwytani, a~następnie straceni, poddani eutanazji, że ich
szczątki zostały zbezczeszczone\ldots 

-- O czym Ty kurwa mówisz? -- powiedziała Gretyl. 

Niemniej jednak
zrozumiała. Przez pierwsze kilka lat martwe ciała tych dwóch najemników
były jak niechciane rodzinne pamiątki, posłusznie wożone z~miejsca na
miejsce, skany i~kopie zapasowe. Kiedy była w~ciągłym ruchu,
przygotowania do ich opieki były ciągłym przypomnieniem strasznej
rzeczy, którą zrobili w~tunelach OU, straszliwych ostrzeżeń Tama,
zobowiązania, które sami sobie stworzyli. Kiedy osiedlili się w~Gary i~przenieśli dwa ciała, lub ludzi, czy cokolwiek, do słojów w~piwnicy,
automatycznie pielęgnowanych, gdy spali w~niekończącym się zastoju z~pustą twarzą, martwym mózgiem, udało jej się to wyrzucić z~umysłu, w~większości.

-- Więc porwałeś ją i~pozbawiłeś ubrania i~towarzystwa, ponieważ miałeś
na sercu jej dobro?

-- Tak. Ponieważ wiedziałem, że alternatywa jest \textit{znacznie} gorsza.
Śmierć. Jak odkryłaś. Dlatego tu jestem. Ponieważ, czy w~to wierzysz,
czy nie, kocham moją córkę. Wychowałem ją. Trzymałem ją, kiedy się
urodziła. Opowiadałem jej bajki na dobranoc. Zmieniałem jej pieluchy.
Ona jest moim ciałem i~krwią. Jestem jej częścią, zawsze będę. Nie chcę,
żeby umarła. Nie chcę, żeby umarły moje wnuki.

-- Ale wszyscy możemy umrzeć? -- spytała Limpopo. -- Nie ma specjalnego
powodu, aby nas trzymać w~pobliżu. Oprócz generowania dochodu dla
TransCanady jesteśmy nadwyżką.

Wzruszył ramionami. 

-- Nie mój wydział. Interesuje mnie moja rodzina.
Twoja rodzina może się o~ciebie troszczyć.

-- To naprawdę miłe, panie Redwater -- powiedziała Limpopo.

Gretyl prawie zapytała go, ile zmienił pieluch i~opowiadał historii, ile
z tego zlecił opiekunce. Nie widziała sensu. Jacob Redwater był
dokładnie tym, za kogo się podawał: zettą, któremu zależało na
zdobywaniu rzeczy, które chciał, nie obchodziło go, co się stało z~innymi. Bez względu na to, jak często zmieniał pieluchy, wystarczyło to,
by wzmocnić tę część jego osobowości, że porwanie córki było
akceptowalną alternatywą dla powstrzymania jego kumpli przed zabiciem
wszystkich w~promieniu dziesięciu kilometrów od niej.

Wpatrywała się w~jego doskonały odcień skóry i~muskularne ramiona pod
kamizelką. Wyglądał, jakby miał dzień wolny w~domku, jak ktoś na
fotografii reklamowej, która reklamuje linię eleganckiej odzieży
outdoorowej/casualowej. Wypolerowany przez lata, niepoobijany. Nie jak
Gretyl, nie jak jej koleżanki. Odeszła, bo nie mogła wziąć udziału w~tworzeniu takich nieśmiertelnych bogów, jak ten. Nie potrzebowali jej
pomocy.

-- Twoja córka nie chce cię widzieć. -- To była prawda. Nie musiała pytać
Lodołasicy, nigdy się nie zawahała.

-- Nazwała swojego syna po mnie.

-- Nazwaliśmy naszego syna Twoim imieniem, żeby nigdy nie zapomniała, od
czego się odwróciła. Na początku nie rozumiałam. Wyjaśniła, że chce
stworzyć Jacoba Redwatera, który nie zostanie zapamiętany jako samolubny
potwór.

Był beznamiętny.

-- Cholera jasna. -- Chłopiec wskazał na ekrany.

Podążyli za jego palcem. Tam, idąc autostradą 15, pojawił się duży tłum.
Setki ludzi. Na froncie, wciąż w~pozostałościach mundurów i~kamizelek
kuloodpornych, byli policjanci, którzy odeszli. Podzielili się na trzy
mniejsze grupy, weszli prosto na prywatnych gliniarzy, którzy próbowali
powstrzymać ich przed wejściem na wewnętrzne dziedzińce więzień,
szamocząc się przez chwilę, próbując zdecydować, ile siły mogą użyć
przeciwko tym przybyszom. Potem znaleźli się poza gliniarzami, między
nimi a~więzieniami. Złożyli ręce i~usiedli przed budynkami, nic nie
mówiąc. Byli gliniarze, którzy odeszli, siedzieli pośrodku. Gretyl
zrozumiała, że tłum na całym świecie nie zatrzymał się, gdy kanały
przestały działać.

Na zawołanie na dziedzińcu pojawiły się nowe drony, wszelkiego rodzaju,
łącznie z~przekaźnikiem sieciowym. Widziała ogromny wzrost
przepustowości ze swojego miejsca, wzbierający nad infografikami w~powodzi niebieskiego, który stał się zielony, gdy pamięci podręczne po
obu stronach łącza opróżniły się, a~zatory ustąpiły, byli
zsynchronizowani ze światem.

Jacob Redwater wyglądał\ldots  zagadkowo. Zmrużył oczy. Kiedy chłopcy
machali kanałami, żeby powiększyć na całą ścianę, on potrząsnął głową,
jakby chciał powiedzieć, \textit{że to nie w~porządku}.

Co powiedziała Nadie? Nie chcą kolejnego Akronu. Nie chcą męczenników.
Jeśli zbombardują to miejsce, to z~wyłączonymi kamerami. Obie strony
były teraz na nowym terytorium taktycznym. W~Ameryce i~Arabii
Saudyjskiej doszło do wielu ataków na bezpodstawne twierdze -- reżimy -- fundamentalistów religijnych; najemnicy bez insygniów na Ukrainie, w~Mołdawii i~na Syberii; szturmowcy wspieranych przez ogromną, zabijającą
sieci broń informacyjną w~Chinach. Były postępy i~odwroty. Nigdy tego
rodzaju oblężenia.

Zadzwonił telefon Gretyl. Wiedziała z~brzęczenia, że to jej żona.
Westchnęła.

-- Nic nam nie jest. -- Wiedziała, że po przedłużającej się ciszy radiowej
Gretyl po cichu by wariowała.

-- Kocham cię.

-- Ja też cię kocham. Chłopcy cię kochają. Czy wszystko w~porządku?
Oglądaliśmy to tutaj. Chłopcy są wściekli, że nie mogli zostać i~pomóc
przy dronach. Nie do końca rozumieją niebezpieczeństwo. Nie chcę ich
martwić.

-- Nie rób tego.

 Doskonale zdawała sobie sprawę, że Jacob Redwater
wytęża słuch, żałując, że nie nauczyła się subwokalizacji za pomocą
swoich interfejsów. Nigdy nie miała zbyt wiele do powiedzenia na temat
prywatnych przestrzeni audio, zbyt wielka pustelniczka.

-- Nie? Och, zmartw ich. Czy wszystko w~porządku? Możesz rozmawiać?

-- Mogę.

-- Ale nie za dużo. Czemu? Kto tam jest? Co się dzieje? Czy jesteś
bezpieczna?

Westchnęła. Jej żona była dobra w~wielu rzeczach, ale tajne operacje nie
należały do nich.

-- Twój ojciec jest tutaj.

Była to niesamowita cisza, cisza nadmiernie skompresowanego kanału
audio, odrzucającego szumy tła. 

-- Czy on cię skrzywdził? -- Wydawała się
zimna.

-- Byłoby mu ciężko to zrobić. Jest zamknięty z~nami, w~podziemiach
więzienia dla chłopców, w~centrum kontroli. Chciał z~Tobą porozmawiać, a~ponieważ nie odbierałaś jego telefonów, zadzwonił do mnie.

Jacob Redwater intensywnie myślał o~tym, gdzie jest Lodołasica. Co to
znaczyło, że musiała to wyjaśnić.

-- Oczywiście, że wiedziałby, jak się z~Tobą skontaktować. Zamierzasz
zażądać okupu?

Nie mogła powstrzymać uśmiechu. Ponieważ się tego spodziewała, znalazła
się między Jacobem Redwaterem a~drzwiami, gdy nagle wstał i~przewrócił
swoje krzesło. Podszedł do niej. Przypomniała sobie, jakim był
wysportowanym, gimnastycznym, osobiście wyszkolonym, technologicznie
nastrojonym łobuzem. Miała zostać uderzona. Wtedy Troy wylądował na
plecach i~powalił go na ziemię, dłonie zaciśnięte na szyi. Każdy z~pozostałych chłopców wziął kończynę i~\textit{usiadł }na niej.

-- Gretyl? -- Brzmiała na zaniepokojoną.

-- Bez problemów. Daj mi sekundę?

Spojrzała na twarz Jacoba Redwatera. Był spokojny, jakby odpoczywał przy
lampce wina w~swojej kryjówce, a~nie leżał na betonowej podłodze z~czterema młodocianymi przestępcami siedzącymi na nim. 

-- Jacob,
Lodołasica i~chłopcy wyszli, zanim to się zaczęło. Są bezpieczne.
Chcesz, żebym spytała Lodołasicę, czy byłaby zainteresowana rozmową z~Tobą?

-- Nie, nawet gdybym umierała z~głodu, a~on był jedyną restauracją na
Ziemi -- powiedział Lodołasica, sprawiając, że Gretyl prychnęła. To było
okrutniejsze i~bardziej radosne, niż zamierzała. Złapała się, zanim
przeprosiła zettę z~rozpostartymi ramionami.

-- Znam odpowiedź. Czy mogę odejść?

-- Dlaczego mielibyśmy ci na to pozwolić? To miejsce jest pełne kajdanek
i cel. Możemy cię zamknąć, upewnić się, że cokolwiek się z~nami stanie,
stanie się z~Tobą. To może ich nie powstrzymać przed wysadzeniem nas
bombą nuklearną. Ale, może powstrzyma.

-- Prawdopodobnie nie. Zużyłem wszystko, co miałem, zatrzymując różne
rzeczy, żebym mógł się tu dostać. Oddanie mnie kosztowało ich dużo\ldots  -- Skinął głową w~stronę ekranów, gdzie prywatni gliniarze i~odchodzący
stali naprzeciwko siebie pod baldachimem dronów. -- Ludzie na górze nie
mieliby nic przeciwko utracie mnie. Zdestabilizowałoby to sytuację, ale
byłoby też przykładem na następny raz, gdy coś takiego się wydarzy.
Wiesz, nie jestem jedyną potężną osobą spokrewnioną z~kimś po Twojej
stronie. Lekcje poglądowe są drogie. Marnotrawstwem jest przepuszczanie
ich, gdy są dostępne.

-- Nie zabiją cię -- powiedziała Limpopo. -- Nie Jacoba Redwatera.
Widzieliśmy Twoje miejsca w~zarządzie. Zbyt wielu ludzi jest ci winnych
zbyt wiele, zbytnio na Tobie polegają\ldots 

Etcetera przerwał, dziwnie, Limpopo i~jej obojczyki się kłóciły\ldots  

-- To
oznacza, że są ludzie, którzy chcieliby wejść w~jego buty.

Redwater wzruszył ramionami najlepiej, jak potrafił. 

-- Oboje macie
rację. Gdybym umarł tutaj z~Tobą, byłoby mnóstwo do zapłacenia. Ale
bardzo potężni ludzie mieliby szansę stać się znacznie potężniejszymi.
Powodem, dla którego pozwolono mi zrobić to, co zrobiłem, było to, że
jest to w~pełni zabezpieczone ryzyko. Zachwyceni, jeśli umrę,
zachwyceni, jeśli tego nie zrobię.

-- Prawdopodobnie robi nowy skan każdego ranka po śniadaniu -- kontynuował
Etcetera z~zadowoleniem z~maszyny. -- Do kolacji działałby na ogromnym
klastrze.

-- Nie tak często. Ale jestem na bieżąco, a~oni uruchomili testowo moje
symulacje. Są niechlujne pytania spadkowe, więc to nie byłoby dokładnie
przed kolacją.

Gretyl nienawidziła tego, jak mógł zostać przygwożdżony do podłogi na
wrogim terytorium, a~mimo to zachować spokój i~władzę.

-- Gret?

-- Przepraszam, trochę o~Tobie zapomniałam. Posłuchaj kochanie, powinnam
iść. Kocham Cię. Kocham chłopców. Jesteś moim światem.

-- My też cię kochamy. -- Ona płakała. Gretyl zamrugała mocno i~powstrzymała płacz. Jacob Redwater obserwował ją uważnie.

Wtedy Lodołasica wydała zdziwiony dźwięk. Gretyl podskoczyła. 

-- Co jest?
-- Zobaczyła monitory i~sapnęła.

Prywatni gliniarze wycofywali się, szereg po szeregach, do transporterów
opancerzonych, które wycofywały się w~uporządkowany sposób. Policjanci
skierowali się twarzami w~stronę więzienia, czekając na swoją kolej na
wejście do swoich pojazdów. Jakby to nie było wystarczająco
zdumiewające, gliniarz złamał szeregi, zdjął hełm i~upuścił broń, tak
jak inni wcześniej tego dnia, i~podszedł do linii dla pieszych. Zrobiły
to jeszcze dwie osoby. Zorganizowany odwrót przestał być uporządkowany.
Gliniarze kłębili się. Wielu wyglądało, jakby uważnie słuchali głosów w~hełmach. Niektórzy z~zapałem rozmawiali ze sobą. Wykrzykiwali
żartobliwe, przyjacielskie pożegnania z~tymi, którzy przeszli.

Jacob Redwater był zagubiony. Obserwował spektakl, podnosząc szyję z~podłogi. Wyraz jego twarzy był najbliższą miną strachu, jaką Gretyl
wyobrażała sobie, że kiedykolwiek zobaczy.

-- Co myślisz, Jacobie? -- Śmiech w~głosie Gretyl był mimowolnie złośliwy.
-- Wycofują się, żeby nas wysadzić bombą nuklearną i~wysłać wiadomość? A
może wychodzą, zanim giełda się załamie i~ich strażnicy etatu odejdą?

-- Mogę usiąść?

-- Nie moja decyzja.

Chłopcy spojrzeli po sobie i~wstali. Siedział i~pracował ramionami.

-- Wychodzę. -- Nadal był zafascynowany kanałami. Jeden z~transporterów
opancerzonych został zatrzymany przez gliniarza, który wysiadł i~zdezerterował.

Gretyl spojrzała na niego. Nadal był wyprostowany i~niezachwiany,
uzbrojony w~godność.

-- Jacob, wiem, że zawsze będą ludzie tacy jak Ty.

-- Bogaci ludzie.

-- Ludzie, którzy myślą, że inni są tacy jak oni. Ludzie, którzy myślą,
że albo zabierasz, albo Tobie zabierają. Nigdy się tego nie pozbędziemy.
To pierwotny strach, egoizm dziecka. Pytanie brzmi, czy ludzie tacy jak
Ty będą mogli zdefiniować default. Niezależnie od tego, czy możesz
uczynić to samospełniającą się przepowiednią, załatwiając nas
wszystkich, zanim załatwimy ciebie, co oznacza, że wszyscy jesteśmy
frajerami, jeśli nie próbujemy załatwić Ciebie wcześniej. Taki default
był łatwiejszy w~utrzymaniu, gdy nie mieliśmy wystarczająco dużo. Kiedy
nie mieliśmy danych. Kiedy nie mogliśmy ze sobą rozmawiać.

-- W~porządku. -- Bez cienia sarkazmu, tym bardziej sarkastycznego.

-- Nie tworzymy świata bez chciwości, Jacob. Tworzymy świat, w~którym
chciwość jest perwersją. Gdzie chwytanie wszystkiego dla siebie zamiast
dzielenia się jest jak smarowanie się gównem: obrzydliwe. Złe. Nasza
wygrana nie oznacza, że nie możesz być chciwy. Oznacza to, że ludzie
będą się za ciebie wstydzić, będą cię żałować i~będą chcieli się od
ciebie zdystansować. Możesz być tak chciwy, jak chcesz, ale nikt cię za
to nie podziwia.

-- W~porządku. -- Był trochę bledszy. Może to było spełnianie życzeń.

-- Myślę, że Twoja jazda odjeżdża -- powiedziała Gretyl. 

Była zachwycona.
Z jej piersi wypłynęła fatalistyczna akceptacja nadchodzącej zagłady,
zamieniła się w~zwycięską pieśń. Ta część niej, która była emocjonalnie
przygotowana na śmierć, dogoniła tę część niej, która wiedziała, że nie
będzie musiała. Chciała wypić wszystko, co można było wypić, pieprzyć i~śpiewać, rozpalić ognisko i~tańczyć wokół niego nago. Była prawie
martwa. Teraz będzie żyła. Być może na zawsze.

-- Do widzenia. Powiem Lodołasicy i~dzieciakom, że o~nie pytałeś.

\textit{To} trafiło. Westchnął, jakby dostał cios w~brzuch. Poczuła się
jak nagły i~totalny dupek. Jakimkolwiek potworem był Jacob Redwater, był
kimś, dla kogo rodzina była ważna w~pokręcony, przymusowy sposób. Prawie
przeprosiła. Nie zrobiła tego. Myślała, że to zrobi, ale go już nie
było. Limpopo przytulił ją zaciekle, Etcetera wymamrotała przez swój
głośnik w~jej dekolt. Chłopcy krzyczeli i~tańczyli.

-- Kocham cię -- powiedziała Lodołasica.

-- Też cie kocham skarbie. Wracam do domu. -- Otarła łzy radości z~twarzy.
-- Chyba że Ty i~chłopcy chcecie wrócić i~zostać na chwilę? -- Wiedziała,
że to głupie pytanie. Chciała trochę dłużej utrzymać przy życiu
\textit{pierwsze dni}, zanim wróci do trwających dni default, które
zbudowali w~Gary.

-- Nie -- powiedziała Lodołasica. Szukała po omacku kolejnych słów, ale
najwyraźniej żadne nie nadeszło. -- Co się stało z~moim ojcem?

-- Wyszedł. Nienaruszony.

-- Myślę, że teraz go widzę.

Gretyl spojrzała na ekran. Oddalał się, falanga prywatnych gliniarzy
eskortowała go za ich liniami, z~powrotem do pojazdów dowodzenia. 

-- To on.

-- Kurwa -- powiedziała Lodołasica. 

Serce Gretyl bolało i~urosło o~dwa
rozmiary, gdy usłyszała, jak chłopcy chichoczą z~przekleństw mamy. O
czym myślała, narażając się na niebezpieczeństwo? Ryzykujesz, że nigdy
więcej nie zobaczę swojej żony, jej pięknych chłopców? Jakie szaleństwo
ją ogarnęło? Czy potajemnie chciała samobójstwa? 

-- Kiedy wyjedziesz?

-- Jutro albo trochę później. W~ciągu najbliższych kilku dni na pewno
będzie dużo ludzi. Hoa może przywieźć rowery. Cokolwiek, co ma sens.

-- Przywieziesz Limpopo?

Spojrzała na Limpopo, która przyglądała się ze szczerym
zainteresowaniem. Przygarbiona i~stara, z~płonącymi oczami, beznadziejna
postawa, którą można było zobaczyć z~orbity. Gretyl od pierwszej chwili,
kiedy zobaczyła Limpopo, wiedziała, że ta kobieta jest pieprzoną
superbohaterką.

-- Lodołasica chce wiedzieć, czy przyjedziesz ze mną.

Limpopo się nie zawahała. 

-- Nie. Są tu rzeczy, w~których chcę pomóc. To
moje miejsce. Kupiłem je za czternaście lat mojego życia. Przed nami
wiele walk i~chcę być tutaj z~ludźmi, którzy walczyli o~to miejsce.
Zostanę.

-- Słyszysz to?

-- Powiedz jej, że ją kochamy. Powiedz jej, że ona też ma tu dom. W~każdej chwili.

Gretyl to powiedziała. Limpopo z~powagą skinęła głową. Chłopcy
obserwowali je z~szeroko otwartymi oczami, wciąż zszokowani nagłym
zniesieniem wyroku śmierci.

-- Powiedz drugiej Limpopo, że nie musi się martwić, że wrócę w~najbliższym czasie -- dodał Etcetera.

-- Jestem pewien, że to będzie wielkie pocieszenie.

Okazało się, że symy potrafią \textit{dezaprobować}. To był coś nowego dla
Gretyl.

\part*{epilog}
\chapter*{jeszcze lepszy naród}

To nie było jak przebudzenie ze snu, ale Lodołasica była pewna, że
\textit{nie} śniła. Był tam Billiam, skandalicznie flirtujący z~Noozi, co
nie mogło mieć racji, ponieważ biedny Billiam od dawna nie żył. Noozi
była na orbicie, była na orbicie od piętnastu lat, przysięgała, że nie
odczuwa żadnych negatywnych skutków fizjologicznych, które zostały
złagodzone przez deadheading i~farmaceutyki, które wydrukowano w~bioreaktorze stacji. Byli lekarze, niektórzy obecni, więcej w~tłumie,
oferując opinie na temat jej badań, na temat raka zjadającego jej
wątrobę, grożącego rozprzestrzenieniem się do jej krwi. I~jej ojciec!
Rozmawiali z~Cordelią. To było jak za dawnych czasów. Nadie też tam była
i, cholera, obmacali się przed Gretyl. Już samo wspomnienie tego
sprawiło, że się zarumieniła.

Rumieniec przypomniał jej, że ma ciało. Pamiętając, że miała ciało,
przypomniała sobie, że była deadheadowana, zaparkowała z~powodu raka. To
nie było wyimaginowane. Zaparkowała po łzawym przyjęciu z~chłopcami:
dużymi, pryszczatymi nastolatkami; i~Gretyl, stara i~smutna i~starająca
się tego nie okazywać; i~jej przyjaciół. Cordelia \textit{tam} była,
opuściła otoczone murami miasto, w~którym mieszkała z~tatą, wkradła się
z ekranem, na którym pojawiła się twarz taty, połączona na żywo.
Powiedział pilne i~rozpaczliwe rzeczy, które sprawiły, że płakała. Nie
pamiętała ich.

Ona była martwa. Teraz nie spała. W~pokoju. Wytężyła się, żeby się
rozejrzeć. Było ciemno. Ładnie pachniało. Jak las, z~nutą pachnącej
pary. Jakby w~pobliżu był onsen. Siarka, eukaliptus. Leżała na
szpitalnym łóżku z~poręczami. Widziała światło rzucane przez infografiki
na jego bokach, rzucane na ciemną podłogę, kamień? dookoła tego. Okno,
na dole żaluzji nieco słońca. Sprawdziła swoje ciało i~stwierdziła, że
nie boli. Taka ulga, że prawie się rozpłakała. Bardzo bolało przed
śmiercią. Cały czas, wszędzie.

Ścisnęła dłonie i~czuła się\ldots  dziwnie. Dlaczego dziwne? Nie mogła
powiedzieć. Zbliżały się kroki. Drzwi się otworzyły. Kiedy to zrobiła,
zobaczyła wymiary pokoju. Był mniej więcej wielkości sypialni, którą
dzieliła z~Gretyl. Poczuła zapachy onsen, mocniejsze i~ostrzejsze. Jej
skóra pragnęła wody i~pary. Chciała usiąść, ale czy powinna?

W drzwiach była osoba. Mężczyzna. Szedł w~jej kierunku. Uśmiechnięty.
Brodaty. Młody. Nosił coś dziwnego, śliskiego, ciasnego. Właściwie
widziała zarys jego jąder. To była szczególna część garderoby. Mogła
zejść z~wybiegu sto lat temu albo z~drukarki pięćdziesiąt lat po tym,
jak poszła spać. Jak długo to trwało?

Mężczyzna uśmiechnął się. Poczuła zawrót głowy. Ta twarz była znajoma.
To było niemożliwe. Poczuła go, przyjemny zapach, który pamiętała z~tylu
nocy i~dni spędzonych razem na tylu drogach.

Prawie się roześmiała, gdy to powiedziała: 

-- Hubert Vernon Rudolph
Clayton Irving Wilson Alva Anton Jeff Harley Timothy Curtis Cleveland
Cecil Ollie Edmund Eli Wiley Marvin Ellis Espinoza. 

Roześmiał się z~nią.
Wszelkie myśli, jakie miała, że to sym, klon lub robot Etcetery, rozwiał
śmiech.

-- Jak do cholery.

Wyciągnął ręce, gładkie, niepomarszczone i~nie martwe.

-- Lubię to?

Ujęła jego dłoń. Był ciepły, młody i~witalny. Przyłożyła go do policzka.
Płakała na dłoń. 

-- Jak?

Spojrzała na jego dłonie, gładkie i~nieoznaczone.

-- Tak jak Ty. 

Zawroty głowy wróciły, mocno. Pokój powoli krążył z~nią
pośrodku. Powodem, dla którego jego dłoń była tak gładka, było to, że
\textit{jej} dłoń była gładka. Dlatego to było dziwne. To była jej ręka,
ale \textit{nowa}. Przesuwała dłońmi po swoim ciele, badała miejsca, w~których nieświadomie bała się blizn lub worków chirurgicznych, ściskała
mięśnie stóp, nóg i~pośladków, dotykała twarzy. Znowu wpatrywał się w~jej dłonie.

-- Nie ma mowy.

-- Zajęło to trzydzieści lat. Ciała były łatwiejsze, powstały z~hodowli
organów. Jednak mózg, włożenie skanów z~powrotem do nich, to było
trudne. -- Postukał się w~głowę. -- Nie da się powiedzieć, czy to
zadziałało. Jednak czuję się jak ja.

Dotknęła jego ramienia, brzucha. 

-- Jesteś. -- Dotknęła ust, uszu, oczu,
gardła. -- Ja też. -- Przełknęła. -- Chyba.

Użyła go do podciągnięcia się do pozycji siedzącej. Nie czuła się, jakby
spała. To było jak nic, co czuła wcześniej. Jak ponowne narodziny. Jej
skóra mrowiła. To było \textit{niesamowite}.

-- Gretyl?

Zmarszczył brwi. 

-- Pracujemy nad nią. Zmarła pięć lat temu. Zostawiła
skan. Mam nadzieję, że będziemy ją mieli za rok, najwyżej dwa. Rozwijamy
ciało tak szybko, jak to możliwe.

Jej usta były otwarte. Zamknęła je. 

-- Stan? Jacob?

-- Przestali mieć ciała dziesięć lat temu. -- Wzruszył ramionami. -- Dzieci. Czekają, żeby z~Tobą porozmawiać. Myślę, że chcą cię namówić do
rezygnacji z~ciała, do przyłączenia się do nich. Przez większość czasu
są poza światem. Bardzo się zaplątują, ze sobą i~z innymi. Daje mi to
pieprzone ciarki. To praca następnego pokolenia, prawda? Bez względu na
to, jak bardzo się starasz, mali skurwiele zawsze odejdą na odległość
pokolenia.

Przesunęła stopy przez krawędź, pozwalając im dotknąć podłogi. To była
dachówka, może łupek. Każdy szew trzeszczał w~jej układzie nerwowym. To
było uczucie między łaskotaniem a~byciem na krawędzi orgazmu. Chwyciła
się jego ramienia, zawroty głowy i~radość walki.

-- Czuję onsen.

-- Zbudowaliśmy kolejne B\&B. Jest całkowicie retro. Limpopo czeka na
nas. Właściwie to obie.

Jej usta znów były otwarte. 

-- Dwie?

-- To jest źle widziane. Ale żadna z~nich tym się nie przejmuje. I~tylko
jedna z~nich rozmawia ze mną.

Wstała, pozwalając, by prześcieradło z~niej zsunęło się, pozostawiając
ją nagą. Poczuła powietrze na skórze. To było tak intensywne, że prawie
znowu usiadła, ale trzymała go na ramieniu.

-- Ciesz się tym. Zdziwiłabyś się, jak szybko sobie z~tym poradzisz.
Normalności trudno się oprzeć. Wszystko staje się default, bez względu
na to, jak nowe.

Poprowadził ją korytarzem.

Mijali inne osoby, które uśmiechały się, witały się z~różnymi akcentami.
Niektóre wyglądały znajomo, starsze wersje ludzi, których znała.
Niektóre wyglądały jak młodsze wersje. Mogła przysiąc, że jedną z~nich
była Tam, ale niemożliwie młoda. Nastolatek. Kuzyn? Córka? Tam?

Zatrzymali się przed ciężkimi, pokrytymi solą drzwiami onsenu, grubymi
drewnianymi deskami, z~których wydobywało się ciepłe, pachnące
powietrze. Przytulił ją. Odwzajemniła uścisk.

-- Witaj w~domu -- powiedział.


\chapter*{Podziękowanie}

Ta książka nigdy nie powstałaby bez wpływu \textit{A Paradise Built in
Hell} Rebecca Solnit, \textit{Dług: pierwsze 5000 lat} Davida Graebera i~\textit{Kapitał w~dwudziestym pierwszym wieku} Thomasa Piketty'ego.

Dziękuję Alice (oczywiście!), Stevenowi Brust, Scottowi Westerfeld,
Bartonowi Gellman, Patrick Ball, John Gilmore, Roz Doctorow, Noah
Swartz, Biella Coleman, Mitch Altman, Quinn Norton, Jo Walton, Kim
Stanley Robinson, Vladimir Verano i~Third Place Books, Madeline i~McNally Jackson Books, Ben Wizner z~ACLU, Jeremy Bornstein, William
Gibson, Edward Snowden, oraz Eleanor Saitta.

Dziękuję jak zawsze mojemu agentowi, Russellowi Galenowi i~mojemu
wydawcy, Patrickowi Nielsenowi Haydenowi, który pomógł,
wspierając mnie, nigdy nie spuszczając mnie z~haczyka. Dziękuję także
Tomowi Doherty za jego wkład w~science-fiction, za wydawanie, za
literaturę i~za tak wiele łaskawych rozmów, które odbyliśmy w~naszej
długiej i~owocnej współpracy.

\chapter*{O Cory Doctorowie}

Cory Doctorow jest autorem science-fiction, aktywistą, dziennikarzem i~blogerem -- współredaktor \textit{Boing Boing} oraz autorem wielu książek:
ostatnie \textit{In Real Life}, powieść graficzna, \textit{Information
Doesn't Want to Be Free}, książka o~zarabianiu na życie w~czasach
internetu oraz \textit{Homeland}, zdobywca nagród, drugi tom w serii
dla młodzieży z~2008 roku \textit{Mały Brat}. Cory był na linii frontu
międzynarodowych dyskusji o~prywatności, prawach autorskich i~wolności
informacji przez ponad dekadę.


\chapter*{Posłowie od tłumacza}

Oto ostatnia część projektu serii \textit{Czarna Flaga} -- \textit{Odchodząc} Cory Doctorowa, w~oryginale \textit{Walkaway}.

W porównaniu z~innymi powieściami, które opracowywałem, ograniczyłem liczbę przypisów. O ile w~przypadku powieści Kena MacLeoda, uważam, że takie przypisy były ważne, ponieważ autor odnosił się do pojęć zapomnianych lub nieznanych w~Polsce, o~tyle w~przypadku tej powieści, Doctorow sam tłumaczy bardzo dużo. Takie ,,infodumpy'' mogą być uznane za wadę powieści, ale przyznam się, że często są interesujące. 

Tytuł, a~także pojęcie \textit{walkaway}, po długim namyśle, tłumaczę jako odchodzenie bądź w~przypadku osób, jako odchodzące lub odchodniczki. Przyznam się, że próbując wymyślić tłumaczenie tego ważnego terminu dla całej powieści, sięgnąłem nawet do istniejącego tłumaczenia niemieckiego, gdzie okazało się, że \textit{Walkaway} jest tłumaczone jako \textit{den Walkaway}. Wyrażam nadzieję, że spolszczenie tego terminu i~taktyki zostanie dobrze przyjęte przez czytelniczki.

~

Jest coś takiego w~powieściach, w pisaniu, tłumaczeniu, że dużo można wykonać samodzielnie. Trudno sobie wyobrazić założenie w jedną osobę, nie wiem, infoshopu, gdzie istotna jest współpraca z społecznością, albo założenie skłotu, gdzie jedna osoba zostałaby szybko spacyfikowana przez Państwo.

Natomiast do przetłumaczenia powieści lewicowych, ba, całej serii, potrzebna jest jedna osoba, czas, używany komputer i~rozsądnie zestawione wolne oprogramowanie. 

Choć jestem przekonany, że gdyby korektę i~redakcję wykonywał ktoś inny, to wynik byłby znacznie lepszy.

Niemniej jednak seria \textit{Czarna Flaga} była interesującym, życiowym doświadczenie, cały ten czas spędzony na tłumaczeniu, korekcie, redakcji, składzie. Jestem dumny, że udało mi się ukończyć ten projekt.

~

Jestem przekonany, że uważna czytelniczka, czy uważny czytelnik odnajdzie wiele błędów w~tym tłumaczeniu. Ponoszę za to całkowitą odpowiedzialność.\\

\href{mailto:theskymyladythesky@zoho.eu}{Jacek Hummel}\\

Warszawa, październik -- grudzień 2021 roku.


\chapter*{Seria ,,Czarna Flaga''}

\begin{center}
\begin{large}
W serii \textit{Czarna Flaga} opublikowano online:
\end{large} 
\end{center}


\begin{enumerate}
\item \href{https://archive.org/details/joanna-russ-mezczyzna-rodzaju-zenskiego/Joanna_Russ_M\%C4\%99\%C5\%BCczyzna_rodzaju_\%C5\%BCe\%C5\%84skiego}{Mężczyzna rodzaju żeńskiego}, Joanna Russ
\item Silniki Światła t. 1 -- \href{https://archive.org/details/ken-macleod-wieza-kosmonauty}{Wieża Kosmonauty}, Ken MacLeod
\item Jesienna Rewolucja t. 1 -- \href{https://archive.org/details/ken-mac-leod-jesienna-rewolucja-gwiezdna-frakcja}{Gwiezdna Frakcja}, Ken MacLeod
\item Jesienna Rewolucja t. 2 -- \href{https://archive.org/details/ken-mac-leod-jesienna-rewolucja-kamienny-kanal}{Kamienny Kanał}, Ken MacLeod
\item Jesienna Rewolucja t. 4 --  \href{https://archive.org/details/ken-mac-leod-jesienna-rewolucja-droga-do-gwiazd}{Droga do Gwiazd}, Ken MacLeod
\item Jesienna Rewolucja t. 3 -- \href{https://archive.org/details/ken-mac-leod-jesienna-rewolucja-oddzial-cassini}{Oddział Cassini}, Ken MacLeod
\item \href{https://archive.org/details/cory-doctorow-radykalne}{Radykalne}, Cory Doctorow
\item Odchodząc, Cory Doctorow
\end{enumerate}

\newpage

Projekt serii został przygotowywany dzięki Wolnemu Oprogramowaniu. Zestaw narzędzi składa się z:
\begin{itemize}
\item \href{https://ubuntu.com/}{Ubuntu 20.04 Ogniskowa Fossa} -- system operacyjny
\item \href{https://omegat.org/}{OmegaT} -- narzędzie wspomagające tłumaczenie (CAT)
\item \href{https://github.com/soimort/translate-shell}{translate-shell} -- narzędzie do tłumaczenia w~\href{https://translate.google.pl}{Google Translate} przez terminal 
\item \href{https://glosbe.com/en/pl}{Glosbe} -- największy słownik online
\item \href{https://www.wikipedia.org/}{Wikipedia} -- podstawowe źródło tłumaczeń pojęć technicznych, politycznych i~ekonomicznych czy not biograficznych
\item \href{https://www.libreoffice.org/}{LibreOffice} -- przetwarzanie dokumentów 
\item \href{http://pandoc.org}{pandoc} -- uniwersalny konwerter dokumentów 
\item \href{https://www.latex-project.org/}{LaTeX} -- redakcja, skład i~łamanie dokumentu
\item \href{https://sigil-ebook.com/}{sigil} -- przetwarzanie plików ebook
\item \href{https://calibre-ebook.com/}{calibre} -- konwersja plików ebook
\end{itemize}



%EPUB
%\newpage
%\printendnotes
%EPUB

\tableofcontents{}
\end{document}
