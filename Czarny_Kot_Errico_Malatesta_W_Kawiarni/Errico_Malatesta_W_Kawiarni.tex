\documentclass[oneside,polish,11pt,sfheadings]{mwbk}
%polonizacja
\usepackage[T1]{fontenc}
%\usepackage{times}
%\usepackage{palatino}
%\usepackage{bookman}
\usepackage{utopia}
\usepackage[polish]{babel}
\usepackage[utf8]{inputenc}
\usepackage{polski} 
\frenchspacing 
\usepackage{indentfirst} 
%koniec polonizacja
%grafika
\usepackage{graphicx}
%pakiet czcionki
\usepackage{times}
\usepackage[a5paper]{geometry} %wielkość papieru (148x210-book w~PL)
%gwiazdki
\newcommand{\threeast}{\par\centerline{*\,*\,*}\medskip\par}

%EPUB
\usepackage[hyperfootnotes=true]{hyperref} 
%move footnotes to endnotes
%\usepackage{enotez}
%\let\footnote=\endnote
%\setenotez{
%  list-name = Przypisy,
%  backref = true
%}

%pdf anonimize
%dla EPUB wykomentować
%\pdfsuppressptexinfo=-1 %Suppress PTEX.Fullbanner and info of imported PDFs

%pakiet odnośników i~pdf metadata
%\usepackage[unicode, pdftex]{hyperref}
%\hypersetup{pdfauthor={Errico Malatesta},
%            pdftitle={W Kawiarni},
%            pdfsubject={At the Cafe},
%            pdfkeywords={tłum. Jacek Hummel, Creative Commons, tłumaczenie CC BY %4.0},
%            pdfcreator={pdfLaTeX}}
%dla EPUB koniec wykomentowania


\begin{document}

\title{W Kawiarni}
\author{Errico Malatesta}


%--titlepage start
\DeclareRobustCommand{\cs}[1]{\texttt{\char`\\#1}}
\newlength{\tpheight}\setlength{\tpheight}{0.9\textheight}
\newlength{\txtheight}\setlength{\txtheight}{0.9\tpheight}
\newlength{\tpwidth}\setlength{\tpwidth}{0.9\textwidth}
\newlength{\txtwidth}\setlength{\txtwidth}{0.9\tpwidth}
\newlength{\drop}
\newcommand*{\titleSI}{\begingroup% Sagas
\drop = 0.13\txtheight
\centering
{\Huge \textsf{~}}\\[1\baselineskip]
{\huge \textsf{~}}\\[1\baselineskip]
%{\LARGE  \textsf{~}}\\[4\baselineskip]
{\Huge \textsc{W Kawiarni}}\\[1\baselineskip]
{\LARGE \textsc{Dialogi o anarchizmie}}\\[2\baselineskip]
{\huge \textsc{Errico Malatesta}}\\[4\baselineskip]
{\large Na podstawie \href{libcom.org}{libcom.org} \\ przetłumaczył i~opracował:}\\
{\Large \href{mailto:theskymyladythesky@zoho.eu}{Jacek Hummel}}\\[1\baselineskip]
{\normalsize \textit{Tłumaczenie jest dostępne na licencji\\
\href{https://creativecommons.org/licenses/by/4.0/deed.pl}{Creative Commons Uznanie autorstwa 4.0 Międzynarodowe}} \\ [1\baselineskip] \par}
\includegraphics[scale=0.3]{CC.png}

~

\vfill
{\Large {Warszawa, 2023}}\\
%\vspace*{\drop}
\endgroup}
\titleSI
\thispagestyle{empty}
%--titlepage end

\chapter*{Wstęp}
 



Malatesta zaczął pisać serię dialogów, które się składają \begin{itshape}W kawiarni: dialogi o anarchizmie \end{itshape} w~marcu 1897 roku, kiedy ukrywał się w~Ankonie i~redagował czasopismo
\textit{Agitacja}. Luigi Fabbri w~swojej relacji z~tego okresu, napisanej dla pełnego wydania zestawu dialogów z~1922 roku (\begin{itshape}Bologna, Edizioni di Volontà \end{itshape}), pod redakcją Malatesty
(\begin{itshape}Przedruk, Torino, Sargraf,
1961\end{itshape}), daje nam zwodniczy obraz Malatesty, gładko ogolonego dla przebrania, wychodzącego na
miasto z~fajką w~ustach, uśmiechającego się bezczelnie do swoich przyjaciół, którzy dla jego bezpieczeństwa życzyli mu,
żeby przebywał gdzie indziej. 

 
Idea dialogów została mu zasugerowana przez fakt, że często bywał w~kawiarni, która zwykle nie była miejscem spotkań
wywrotowców takich jak on. Rzeczywiście, jeden ze stałych bywalców, który był członkiem policji, zwykł angażować
Malatestę w~rozmowę, oczywiście, jak zauważa Fabbri, nie mając pojęcia, że w~jego zasięgu znajduje się
prawdziwa nagroda. Anarchizm prawie na pewno byłby jednym z~tematów rozmów, ponieważ anarchiści z~miasta nieustannie
bombardowali swoich współmieszkańców propagandą, która wywoływała częste procesy sądowe. 

 
Forma, jaką miały przybrać dialogi, została więc zaczerpnięta z~rzeczywistego miejsca i~z własnego doświadczenia
Malatesty. Zaowocowało to narzędziem literackim doskonale pasującym do jego szczególnego geniuszu, jakim jest jego
umiejętność przekładania złożonych idei na prosty język i~udostępniania ich bezpośrednio. Forma dialogu pozwoliła
również Malatescie na debatę nad ideami jego przeciwników, poddając jednocześnie krytycznej analizie własne
anarchistyczne poglądy, mając  na celu przekazanie czytelnikom ich politycznego znaczenia i~praktycznego zastosowania.
Rzeczywiście, jedną z~mocnych stron dialogów jest brak słomianych ludzi. Badanie anarchizmu jest dociekliwe i~autentyczne, często podkreślające to, co przeciwnicy uznaliby za słabe i~wrażliwe punkty. To sprawia, że
 zadziorna obrona Malatesty jest jeszcze bardziej imponująca. 

 
Pod koniec 1897 roku Malatesta został zidentyfikowany i~odkryty przez policję w~Ankonie. Aresztowany, a następnie
zwolniony. Natychmiast rozpoczął rundę wykładów, porzucając zarówno swój dziennik, jak i~niedokończone dialogi. W~1898
r. został umieszczony w~areszcie domowym, a w~marcu 1899 r. uciekł za granicę, ponownie stając się uchodźcą. Dialogi
pozostawały niedokończone od numeru dziesiątego i~w~tej formie zostały opublikowane, zarówno w~czasopismach i~jako
broszura. 

 
Głównymi propagandystami pierwszych dziesięciu dialogów są alter ego Malatesty, Giorgio, anarchista, Prospero, bogaty
burżuj, Cesare, sklepikarz i~Ambrogio, sędzia. Malatesta jest zatem w~stanie odzwierciedlić szereg stanowisk
politycznych i~poglądów zaczerpniętych z~szerokiego spektrum społeczeństwa. Jeśli Prospero opowiada się za bogactwem i~przywilejami, Cesare przemawia za mniejszymi właścicielami i~klasami średnimi. Wykazuje się świadomością problemów
społecznych i~wydaje się podatny na perswazję Giorgia, ale też troskę o to, by żadne rozwiązanie nie mogło zakłócić
istniejącego porządku społecznego. Ambrogio jest głosem prawa i~liberalnego państwa oraz akceptowanych idei dotyczących
praw i~sprawiedliwości. Jest także, jako główny przeciwnik Giorgio, tym, który wyraża zdroworozsądkowe poglądy na temat
ludzkiej natury i~ludzkich zachowań. Jego poglądy zawierają liberalny wyraz teorii praw, złagodzony przez to, co
nazwałby uznaniem ograniczeń nałożonych na wolność przez nieuniknione dyktaty rzeczywistości. Rezultatem jest szerokie
płótno, na którym Malatesta jest w~stanie, odpowiadając na różne punkty widzenia i~odpowiadając na liczne krytyki,
jakie wywołują poglądy Giorgio, namalować umiejętnie narysowany i~szczegółowy anarchistyczny obraz świata. 

 
Na stosunkowo krótkiej przestrzeni Malatesta przedstawia nam wszystkie podstawowe doktryny komunistycznego anarchizmu i~rozważa jeden po drugim wiele głównych zarzutów wobec jego stanowiska. Po zarysowaniu sceny, są to własność prywatna i~własność prawa, które stają się przedmiotem zainteresowania. W~Dialogach Drugim, Trzecim i~Czwartym argumentuje się, że
przyczyny ubóstwa tkwią w~naturze systemu własności i~związanej z~nim struktury klasowej, przy czym następuje
zdecydowany atak na prawo własności prywatnej i~system kapitalistyczny, z~incydentalnymi dyskusjami na temat Malthusa i~wolnego handlu. Jednocześnie wprowadzane są idee całkowitej zmiany ustroju własnościowego i~stworzenia społeczeństwa
bez rządu. Pochodzenie własności i~praw własności jest rozważane w~Dialogu piątym, a Giorgio utrzymuje, że prawa
własności muszą zostać zniesione, jeśli ma się uniknąć wyzysku. W~Dialogu szóstym przedstawiana jest sprawa wspólnej
własności i~wprowadzana jest idea komunizmu. Dyskusja komunizmu trwa w~Dialogu Siódmym, sprzeciwiając się mu jako
tyrańskiemu i~opresyjnemu systemowi, silnie podtrzymywanemu przez Ambrogio w~imię abstrakcyjnej wolności. Giorgio
przeciwstawia się poprzez przedstawienie społeczeństwa anarchistycznego jako dobrowolnej, złożonej federacji
stowarzyszeń, a przy okazji przeciwstawia anarchistyczną formę wolnego komunizmu tej ze szkoły autorytarnej. Dialog
Ósmy przenosi uwagę na kwestię rządu i~państwa oraz tego, jak społeczeństwo może funkcjonować pod ich nieobecność.
Zostaje także przedstawiona rozszerzona krytyka parlamentaryzmu i~reprezentacji oraz obrona anarchizmu jako porządku
społecznego utrzymywanego przez dobrowolne porozumienie i~dobrowolną delegację. Spór jest kontynuowany w~następnym
Dialogu (Dialog dziewiąty), gdzie ponownie wymienione zostają zarzuty wobec społeczeństwa bez rządu, a Giorgio dalej
rozwija formę argumentu Kropotkina na temat uniwersalności wzajemnej pomocy, idei wprowadzonej po raz pierwszy w~Dialogu szóstym. Dyskurs Dziesięć zwraca się w~nowym kierunku, skupiając się na seksie, miłości i~rodzinie. Omawiając
wiele kwestii związanych z~feminizmem, przekonująco odrzuca się wszelkie nieodłączne podstawy nierówności płci. 

 
Dopiero 15 lat później, w~1913 roku, Malatesta powrócił do dialogów. W~tym czasie ponownie osiedlił się w~Ankonie i~rozpoczął publikację swojego nowego dziennika \begin{itshape}Volonta \end{itshape}. W~tej
nowej publikacji ponownie wydał oryginalne dziesięć dialogów w~zredagowanej i~poprawionej formie oraz dodał cztery
kolejne. Początkowo w~Dialogach Jedenastym i~Dwunastym to ponownie Cesare, Prospero i~Ambrogio są rozmówcami Giorgia.
Kwestia przestępczości zostaje poruszona w~Dialogu Jedenastym. Jak radzimy sobie z~przestępcami pod nieobecność rządu,
prawa, sądów czy więzień? Giorgio odpowiada, że sprawa ta musi być rozwiązywana wspólnie. Od tego momentu dyskusja
przechodzi do kontrastu między pracą umysłową i~fizyczną oraz starym wątkiem, kto ma wykonywać prace, których nikt nie
chce wykonywać. Czyż nie każda osoba  będzie chciała zostać poetą? Podana jest typowa odpowiedź, czyli rotacji  dobrowolna zadań
i rozwijania wielu umiejętności przez członków społeczności. Dialog Dwunasty bada potrzebę rewolucji i~przedstawia
argumenty za smutną koniecznością gwałtownej rewolucji, ponieważ istniejący porządek jest utrzymywany przez przemoc, a
uprzywilejowane klasy nie zrezygnują z~władzy, dopóki nie zostanie ona wstrząśnięta. 

 
W Dialogu trzynastym poznajemy nową postać, Vincenzo, młodego republikanina, i~wywiązuje się dyskusja na temat zalet i~ograniczeń republikańskiego podejścia do zmiany. Główna wada zostaje zidentyfikowana, mianowicie poleganie na rządzie i~systemach demokratycznej reprezentacji. Stwierdza się, że republikanizm nie jest tak radykalny, jak uważają jego
zwolennicy, ponieważ pozostaje ofiarą zła istniejącego systemu politycznego. Ostatni dialog tej nowej serii (Dialog
czternasty) powraca do tematu rewolucji. Giorgio podkreśla, że anarchizm w~swoim dążeniu do usunięcia państwa i~rządu
jest nowym czynnikiem w~historii i~proponuje zmiany zupełnie inne i~głębsze niż poprzednie rewolucje, których celem
była po prostu zmiana ustroju politycznego. 

 
Po raz kolejny dialogi miały zostać przerwane przez wydarzenia polityczne. W~czerwcu 1914 roku, gdy zbierały się burzowe
chmury i~wojny światowej, w~Marchii i~Romagna, w~tak zwanym Czerwonym Tygodniu. Malatesta był zaangażowany w~te
popularne walki, w~wyniku czego został zmuszony do schronienia się w~Londynie. Minęło sześć lat i~Malatesta wrócił do
Włoch, osiedlając się w~Mediolanie, gdzie redagował gazetę \begin{itshape}Umanita
Nova \end{itshape}. Fabbri zauważa, że zbyt był zajęty, aby poświęcać uwagę starym dialogom i~nie
zamierzał ich uzupełniać. Fabbri informuje nas jednak, że ktoś, kto spędził u niego dwa tygodnie jako gość, namówił go
do kontynuowania projektu. Można by pomyśleć, że tajemniczym gościem był sam Fabbri. Rezultatem były kolejne trzy
dialogi, raczej kontynuacja niż zakończenie, ponieważ nie ma oczywistego punktu zamknięcia. 

 
W tych trzech ostatnich esejach Malatesta powraca się do niektórych starych tematów i~zwraca się uwagę na niektóre nowe
tematy o współczesnym znaczeniu. Dialogue Piętnasty przedstawia Gino, robotnika, i~analizuje obawy zwykłych ludzi
dotyczące braku porządku obywatelskiego w~proponowanym społeczeństwie bezpaństwowym i~postrzeganej potrzeby policji.
Policja, argumentuje Malatesta za pośrednictwem Giorgio, hoduje przestępców, tak jak argumentował wcześniej w~\begin{itshape}Anarchia\end{itshape}, że \textit{louvreterie} (łapacze wilków) hodują
wilki, ponieważ bez wilków lub przestępców przetrwanie odpowiednich ciał urzędników byłoby zagrożone (Londyn, 1974:
33–34). Twierdzi, że obrona społeczna jest obowiązkiem społeczności. Fakt, że ta kwestia była już omawiana w~Dialogu
Jedenastym, wskazuje na jej wagę dla Malatesty. W~Dialogue Szesnastym poznajemy Pippo, kalekiego weterana wojennego,
który porusza kwestie nacjonalizmu i~patriotyzmu. Uwagi Malatesty są echem wezwania Lenina do solidarności klasowej w~obliczu dzielącego i~destrukcyjnego nacjonalizmu pierwszej wojny światowej. Giorgio wyjaśnia, że jego zdaniem
patriotyzm jest po prostu narzędziem, za pomocą którego burżuazja rekrutuje poparcie klasy robotniczej dla istniejącego
reżimu własności i~terytorialnych ambicji tych, którzy z~niego korzystają. Wreszcie, w~Dialogu Siedemnastym, Luigi,
socjalista, wchodzi i~rozpoczyna dyskusję. Wynika z~niej, że ma na celu odróżnienie anarchizmu zarówno od socjalizmu
parlamentarnego, jak i~autorytarnego, ale z~głównym naciskiem na nieuchronną porażkę ścieżki parlamentarnej i~jakiejkolwiek formy tego, co Eduard Bernstein nazwał socjalizmem ewolucyjnym. Podkreśla się potrzebę rewolucyjnej
zmiany. 

 
Prace nad dialogami w~ich obecnym kształcie zakończono do października 1920 roku.  16 października Malatesta został
aresztowany i~osadzony w~więzieniu San Vittore. Policja przeprowadziła zakrojone na szeroką skalę przeszukanie jego
mieszkania za bronią i~materiałami wybuchowymi, ale rękopis dialogów pozostał nieodkryty lub zignorowany. Zostały one
opublikowane jako zestaw, ze wstępem Fabbriego, w~1922 roku. 

 
Te dialogi Malatesty stanowią nie tylko znaczący wkład w~anarchistyczną teorię polityczną, ale także znaczący dokument
historyczny. Napisane w~ciągu 23 lat są komentarzem do burzliwych czasów i~ważnych wydarzeń historycznych, obejmujących
epokę wyróżniającą się w~szczególności lewicową agitacją i~organizacją w~całej Europie. W~czasie, który obejmowały te
rozważania na temat anarchizmu, świat był świadkiem II Międzynarodówki, powstania bolszewizmu, pierwszej wojny
światowej, narodzin faszyzmu i~rosyjskich rewolucji, zarówno w~1904, jak i~1917 r.  Bez żadnej bezpośredniej aluzji do
każdego z~tych wydarzeń dialogi prowadzą ożywioną debatę dotyczącą wielu kwestii, które poruszają. W~pewnym sensie
Malatesta przekształcił anarchistyczną teorię w~bieżący komentarz do swoich czasów. To dzieło inteligencji, stylu i~prawdziwego artyzmu. 

Paul Nursey-Bray\ 


\chapter*{Jeden}



 
\noindent PROSPERO [\textit{Pulchny mieszczanin, pełen ekonomii politycznej i~innych nauk}]: Ale oczywiście\ldots
oczywiście\ldots my o tym wszystko wiemy. Są ludzie cierpiący głód, kobiety prostytuujące się, dzieci umierające z~braku
opieki. Zawsze mówisz to samo\ldots w~końcu stajesz się nudny. Pozwól mi delektować się moimi gelati w~spokoju\ldots
Oczywiście, w~naszym społeczeństwie jest tysiące zła, głód, ignorancja, wojna, zbrodnia, zaraza, straszne
nieszczęścia\ldots i~co z~tego? Dlaczego jest to twoje zmartwienie? 




 
\noindent MICHELE [\textit{Student, który dotrzymuje towarzystwa socjalistom i~anarchistom}]: Przepraszam? Dlaczego to moje
zmartwienie? Masz wygodny dom, dobrze zaopatrzony stół, służbę na twoje rozkazy; dla ciebie wszystko jest w~porządku. I~dopóki ty i~twoi macie się dobrze, nawet jeśli świat wokół was się zawali, nic nie ma znaczenia. Naprawdę, gdybyś tylko
miał trochę serca\ldots 




 
\noindent PROSPERO: Dosyć, dość\ldots nie każ\ldots Przestań się wściekać, młodzieńcze. Myślisz, że jestem niewrażliwy, obojętny na
nieszczęścia innych. Wręcz przeciwnie, moje serce krwawi (kelner, przynieś mi koniak i~cygaro), moje serce krwawi; ale
wielkich problemów społecznych nie rozwiązuje się sentymentami. Prawa natury są niezmienne i~ani wielkie przemowy, ani
ckliwe sentymentalizmy nie mogą tego zdziałać cokolwiek na ten temat. Mądry człowiek akceptuje los i~czerpie z~życia
to, co najlepsze, bez gonienia za bezsensownymi marzeniami. 




 
\noindent MICHELE: Ach? A więc mamy do czynienia z~prawami natury?\ldots A co jeśli biedni wbili sobie do głowy, żeby te\ldots prawa
natury poprawić. Słyszałem przemówienia z~trudem popierające te nadrzędne prawa. 




 
\noindent PROSPERO: Oczywiście, oczywiście. Dobrze znamy ludzi, z~którymi się zadajesz. W~moim imieniu powiedz tym łajdakom
socjalistom i~anarchistom, których wybrałeś na swoje preferowane towarzystwo, że dla nich i~dla tych, którzy będą
próbowali wprowadzić w~życie ich nikczemne teorie, mamy dobrych żołnierzy i~doskonałych
\begin{itshape}karabinierów \end{itshape}. 




 
\noindent MICHELE: Och! Jeśli zamierzasz sprowadzić żołnierzy i~\begin{itshape}karabinierów \end{itshape}, nie będę już mówić. To tak, jakby proponować walkę
na pięści, aby pokazać, że moje opinie są błędne. Nie polegaj jednak na brutalnej sile, jeśli nie masz innych
argumentów. Jutro możesz znaleźć się w~słabszej pozycji; co wtedy? 




 
\noindent PROSPERO: A więc co? Cóż, gdyby doszło do tego nieszczęścia, nastąpiłby wielki nieład, eksplozja złych namiętności,
masakry, grabieże\ldots a potem wszystko wróciłoby do tego, co było przedtem. Może kilku biednych by się wzbogaciło, kilku
bogatych popadłoby w~biedę, ale ogólnie nic by się nie zmieniło, bo świat się nie może zmienić. Przyprowadź mi, po
prostu przyprowadź mi jednego z~tych twoich anarchistycznych agitatorów, a zobaczysz, jak wygarbuję mu skórę. Są dobrzy
w napełnianiu głów ludzi takich jak ty nieprawdopodobnymi historiami, ponieważ wasze głowy są puste; ale zobaczysz, czy
ze mną będą w~stanie utrzymać swoje absurdy. 




 
\noindent MICHELE: W~porządku. Przyprowadzę mojego przyjaciela, który wyznaje socjalistyczne i~anarchistyczne zasady i~z
przyjemnością poprowadzę z~nim waszą dyskusję. W~międzyczasie przedyskutuj ze mną sprawy, bo choć nie mam jeszcze
dobrze wykształconych poglądów, to wyraźnie widzę, że społeczeństwo w~jego dzisiejszym kształcie jest czymś sprzecznym
ze zdrowym rozsądkiem i~przyzwoitością. No już, jesteś taki tłusty i~kwitnący, że odrobina podniecenia nie zaszkodzi.
Pomoże ci to w~trawieniu. 




 
\noindent PROSPERO: Chodź więc; porozmawiajmy. Jednak powinieneś wiedzieć, że byłoby lepiej, gdybyś się uczył, zamiast wypluwać
opinie na tematy, które go dotyczącą domeną innych, bardziej uczonych i~mądrzejszych. Wydaje mi się, że wyglądasz na 20
lat? 




 
\noindent MICHELE: To nie dowodzi, że uczyłeś się więcej i~jeśli mam cię oceniać na podstawie tego, co mówisz, wątpię, nawet jeśli
dużo się uczyłeś, czy wiele z~tego zyskałeś. 




 
\noindent PROSPERO: Młody człowieku, młody człowieku, naprawdę! Trochę szacunku. 




 
\noindent MICHELE: W~porządku, szanuję cię. Ale nie rzucaj mi w~twarz moim wiekiem, gdy w~istocie sprzeciwiasz się przy pomocy
policji. Argumenty nie są stare ani młode, są dobre lub złe; to wszystko. 




 
\noindent PROSPERO: No, no, przejdźmy do tego, co masz do powiedzenia? 




 
\noindent MICHELE: Muszę powiedzieć, że nie mogę zrozumieć, dlaczego chłopi, którzy sieją, sieją i~żną, nie mają wystarczającej
ilości chleba, wina ani mięsa; dlaczego murarze budujący domy nie mają dachu jako schronienia, dlaczego szewcy noszą
znoszone buty. Innymi słowy, dlaczego tym, którzy pracują, którzy wszystko produkują, brakuje podstawowych artykułów
pierwszej potrzeby; podczas gdy ci, którzy nic nie robią, rozkoszują się obfitością. Nie mogę zrozumieć, dlaczego
istnieją ludzie, którym brakuje chleba, skoro jest tyle nieuprawianej ziemi i~tak wielu ludzi, którzy byliby
niezmiernie szczęśliwi, gdyby mogli ją uprawiać; dlaczego jest tak wielu murarzy bez pracy, podczas gdy wielu ludzi
potrzebuje domów; dlaczego wielu szewców, krawców itp\ldots jest bez pracy, podczas gdy większości społeczeństwa brakuje
butów, ubrań i~wszelkich artykułów niezbędnych do życia cywilnego. Czy mógłbyś powiedzieć, które prawa naturalne
wyjaśnia i~uzasadnia takie absurdy? 




 
\noindent PROSPERO: Nic nie mogłoby być prostsze i~jaśniejsze. 

 
Do produkcji nie wystarczy praca ludzka, potrzeba ziemi, materiałów, narzędzi, pomieszczeń, maszyn, a także środków do
przeżycia w~oczekiwaniu na wytworzenie produktu i~dostarczenie go na rynek: jednym słowem potrzebny jest kapitał. Twoi
chłopi, twoi robotnicy, mają tylko swoją pracę fizyczną; w~konsekwencji nie mogą pracować, jeśli nie jest to życzeniem
tych, którzy posiadają ziemię i~kapitał. A ponieważ jest nas niewielu i~mamy dość, nawet jeśli przez jakiś czas
pozostawiamy naszą ziemię nieuprawianą, a nasz kapitał nieczynny, podczas gdy robotników jest wielu i~zawsze
ograniczają ich pilne potrzeby, wynika z~tego, że muszą oni pracować kiedykolwiek i~jakkolwiek, gdy tak sobie życzymy,
i na warunkach, które nam odpowiadają. A kiedy już nie będziemy potrzebować ich pracy i~uznamy, że nie ma żadnego zysku
z ich pracy, muszą trwać na bezrobociu, nawet gdy naprawdę potrzebują tych rzeczy, które potrafią wyprodukować. 

 
Czy jesteś teraz zadowolony? Czy mam to wyjaśnić jaśniej? 




 
\noindent MICHELE: Oczywiście, to się nazywa szczerość, co do tego nie ma wątpliwości. 

 
Jednak jakim prawem ziemia należy tylko do nielicznych? Jak to się dzieje, że kapitał znajduje się w~kilku rękach,
szczególnie w~rękach tych, którzy nie pracują? 




 
\noindent PROSPERO: Tak, tak, wiem, co o czym mówisz, a nawet znam mniej lub bardziej kulawe argumenty, którymi inni by ci się
sprzeciwili; prawo właścicieli wywodzi się z~ulepszeń, jakie wnoszą do ziemi, z~oszczędności, za pomocą których odbywa
się przekształcenie pracy w~kapitał itd. Ale pozwól mi być jeszcze bardziej szczerym. Rzeczy są takie, jakie są, w~wyniku faktów historycznych, wytworu setek lat historii ludzkości. Cała ludzka egzystencja była, jest i~zawsze będzie
ciągłą walką. Są tacy, którym poszło dobrze i~tacy, którym poszło źle. Co mogę z~tym zrobić? Im gorzej dla jednych, tym
lepiej dla innych. Biada pokonanym! To jest wielkie prawo natury, przeciwko któremu żaden bunt nie jest możliwy. 

 
Co byś wolał? Czy powinienem pozbawić się wszystkiego, co mam, aby gnić w~biedzie, podczas gdy ktoś inny żywi się moimi
pieniędzmi? 




 
\noindent MICHELE: Nie do końca tego chce. Jednak myślę: co by było, gdyby robotnicy, korzystając ze swojej liczebności i~opierając się na waszej teorii, że życie jest walką, a prawa wynikają z~faktów, wbili sobie do głowy pomysł stworzenia
nowego ,,faktu historycznego'', zabierając swoją ziemię i~kapitał i~inaugurację nowych praw? 




 
\noindent PROSPERO: Ach! z~pewnością skomplikowałoby to sprawę. 

 
Ale\ldots będziemy kontynuować przy innej okazji. Teraz muszę iść do teatru. 

 
Dobry wieczór wam wszystkim. 










\chapter*{Dwa}



 
\noindent AMBROGIO [\textit{Sędzia}]: Słuchaj, Signor Prospero, teraz, kiedy to tylko między nami, wszyscy dobrzy
konserwatyści. Tamtego wieczoru, kiedy rozmawiałeś z~tą pustą głową, Michele, nie chciałem interweniować; ale czy sądzi
pan, że to był sposób na obronę naszych instytucji? 

 
Prawie wydawało się, że jesteś anarchistą! 




 
\noindent PROSPERO: Cóż, ja nigdy! Dlaczego? 




 
\noindent AMBROGIO: Bo w~istocie twierdziłeś, że cała obecna organizacja społeczna opiera się na sile, dostarczając w~ten sposób
argumentów tym, którzy chcieliby ją zniszczyć siłą. A co z~najwyższymi zasadami rządzącymi społeczeństwami
obywatelskimi, prawami, moralnością, religią, czy one się na nic nie liczą? 




 
\noindent PROSPERO: Oczywiście, zawsze masz usta pełne praw.  To zły nawyk wywodzący się z~twojego zawodu. 

 
Jeśli jutro rządy zadekretują, załóżmy, kolektywizm, potępilibyście zwolenników własności prywatnej z~taką samą
beznamiętnością, z~jaką potępiacie dzisiaj anarchistów\ldots i~zawsze w~imię najwyższych zasad wiecznych i~niezmiennych
praw! 

 
Widzisz, to tylko kwestia nazw. Ty mówisz prawa, ja mówię siła; ale tak naprawdę liczą się błogosławieni
\begin{itshape}karabinierzy \end{itshape}, a kto ma ich po swojej stronie, ma rację. 




 
\noindent AMBROGIO: No nie, signor Prospero! Wydaje się niemożliwe, aby twoja miłość do sofizmu zawsze tłumiła twoje konserwatywne
instynkty. 

 
Nie rozumiesz, jak wiele złych skutków wynika z~widoku osoby takiej jak ty, jednego ze starszych miasta, dostarczającej
argumentów najgorszym wrogom porządku. 

 
Wierz mi, że powinniśmy położyć kres temu złemu zwyczajowi sprzeczek między sobą, przynajmniej publicznie; wszyscy się
jednoczmy, by bronić naszych instytucji, które z~powodu niegodziwości czasów otrzymują brutalne ciosy\ldots i~aby dbać o
nasze zagrożone interesy. 




 
\noindent PROSPERO: Zjednoczmy się za wszelką cenę; ale jeśli nie zostaną podjęte zdecydowane środki, jeśli nie przestaniecie
używać liberalnych doktryn, niczego nie rozwiążemy. 




 
\noindent AMBROGIO: O! Tak na pewno. Potrzebujemy surowego prawa, które będzie ściśle stosowane. 

 
Ale to nie wystarczy. Sam przymus nie może długo utrzymywać ludzi w~niewoli, szczególnie w~dzisiejszych czasach. Trzeba
propagandzie przeciwstawiać propagandę, trzeba przekonywać ludzi, że mamy rację. 




 
\noindent PROSPERO: Naprawdę żartujesz sobie! Mój biedny przyjacielu, w~naszym wspólnym interesie, błagam cię, uważaj na
propagandę. To wywrotowa rzecz, nawet jeśli jest prowadzona przez konserwatystów; a wasza propaganda zawsze obróciłaby
się na korzyść socjalistów, anarchistów czy jakkolwiek oni się nazywają. 

 
Idź i~przekonaj głodnego, że on po prostu nie je, tym bardziej że to on produkuje jedzenie! Dopóki o tym nie myślą i~nadal błogosławią Boga i~szefa za to, co otrzymują, jest w~porządku. Ale od chwili, gdy zaczną się zastanawiać nad
swoją pozycją to koniec: staną się wrogiem, z~którym nigdy się nie pogodzicie. Nigdy w~życiu! Musimy za wszelką cenę
unikać propagandy, zdusić prasę drukarską, z~lub bez, a może nawet wbrew prawu. 




 
\noindent AMBROGIO: To prawda, to prawda. 




 
\noindent PROSPERO: Zapobiegaj wszelkim zgromadzeniom, rozbijaj wszelkie stowarzyszenia, wsadzaj do więzienia wszystkich, którzy
myślą\ldots 




 
CESAR [\textit{sklepikarz}]: Spokojnie, spokojnie, nie daj się ponieść pasji. Pamiętaj, że inne rządy, w~bardziej
sprzyjających czasach, przyjęły środki, które sugerujesz\ldots i~przyspieszyło to ich własny upadek. 




 
\noindent AMBROGIO: Cicho, cicho! Nadchodzi Michele z~anarchistą, którego skazałem w~zeszłym roku na sześć miesięcy więzienia za
wywrotowy manifest. Właściwie, między nami, manifest został sporządzony w~taki sposób, że prawo nie mogło go tknąć, ale
co można zrobić? Był w~tym zbrodniczy zamiar\ldots a przecież społeczeństwa trzeba bronić! 




 
\noindent MICHELE: Dobry wieczór panom. Czy mogę przedstawić wam mojego przyjaciela anarchistę, który przyjął wyzwanie rzucone
któregoś wieczoru przez signora Prospero. 




 
\noindent PROSPERO: Ale jakie wyzwanie, jakie wyzwanie?! Rozmawialiśmy tylko w~gronie przyjaciół dla zabicia czasu. 

 
Jednak tłumaczyłeś nam, czym jest anarchizm, czyli czym jest coś, czego nigdy nie byliśmy w~stanie zrozumieć. 




 
\noindent GIORGIO [\textit{Anarchista}]: Nie jestem nauczycielem anarchizmu i~nie przyszedłem wygłaszać wykładów na ten
temat; ale w~razie potrzeby potrafię bronić swoich idei. Poza tym jest tu pewien dżentelmen (z ironicznym tonem
nawiązując do sędziego Ambrogio), który powinien wiedzieć o tym więcej ode mnie. Potępił wielu ludzi za anarchizm; a
ponieważ z~całą pewnością jest człowiekiem sumienia, nie zrobiłby tego bez uprzedniego dogłębnego przestudiowania
związanych z~tym argumentów. 




 
\noindent CESARE: Chodź, chodźmy, nie wchodźmy w~sprawy osobiste\ldots a ponieważ musimy mówić o anarchizmie, zacznijmy od razu ten
temat. 

 
Widzisz, ja również zdaję sobie sprawę, że sprawy idą źle i~że należy znaleźć środki zaradcze. Ale nie musimy stawać się
utopistami, a przede wszystkim musimy unikać przemocy. Z~pewnością rząd powinien bardziej wziąć sobie do serca sprawę
robotniczą: powinien zapewnić pracę bezrobotnym; chronić przemysł krajowy, zachęcać do handlu. Ale\ldots 




 
\noindent GIORGIO: Ileż rzeczy chciałbyś, żeby zrobił ten biedny rząd! Rząd nie chce interesować się interesami robotników i~to
jest zrozumiałe. 




 
\noindent CESARE: Jak to może być zrozumiałe? Do tej pory rzeczywiście rząd wykazywał brak zdolności i~być może niewielkie
pragnienie zaradzenia bolączkom kraju; ale jutro oświeceni i~sumienni ministrowie mogą zrobić to, czego nie zrobiono do
tej pory. 




 
\noindent GIORGIO: Nie, drogi panie, to nie jest kwestia tej czy innej posługi. Jest to ogólnie kwestia rządu; wszystkich rządów,
dzisiejszych, wczorajszych i~jutrzejszych. Rząd pochodzi od właścicieli, potrzebuje wsparcia właścicieli, aby się
utrzymać, jego członkowie sami są właścicielami; jak zatem może służyć interesom pracowników? 

 
Z drugiej strony rząd, nawet gdyby chciał, nie mógłby rozwiązać kwestii społecznej, ponieważ jest to wypadkowa czynników
ogólnych, których rząd nie może usunąć, a które de facto same określają charakter i~kierunek rządzenia. Aby rozwiązać
kwestię społeczną, musimy radykalnie zmienić cały system, którego rząd ma misję bronić. 

 
Mówisz o dawaniu pracy bezrobotnym. Ale co może zrobić rząd, jeśli nie ma pracy? Czy ma zmuszać ludzi do wykonywania
bezużytecznej pracy, ale wtedy kto by im płacił? Czy powinien ukierunkować produkcję na zaspokajanie niezaspokojonych
potrzeb ludności? Ale wtedy właściciele nie byliby w~stanie sprzedawać produktów, które wywłaszczają od robotników, w~rzeczywistości musieliby przestać być właścicielami, ponieważ rząd, aby zapewnić ludziom pracę, odebrałby im ziemię i~kapitał, które zmonopolizowali. 

 
To byłaby rewolucja społeczna, likwidacja całej przeszłości, a dobrze wiecie, że jeśli nie dokonają tego robotnicy,
chłopi i~nieuprzywilejowani, rząd na pewno nigdy tego nie zrobi. 

 
Chroń przemysł i~handel, powiadasz: ale rząd może co najwyżej faworyzować jedną klasę przemysłową ze szkodą dla innej,
faworyzować kupców z~jednego regionu kosztem kupców z~innego, więc w~sumie nic by się nie zyskało, tylko trochę
faworyzowania, trochę niesprawiedliwości i~więcej nieproduktywnych wydatków. Jeśli chodzi o rząd, który chroni
wszystkich, jest to pomysł absurdalny, ponieważ rządy niczego nie produkują i~dlatego mogą jedynie przekazywać bogactwo
wyprodukowane przez innych. 




 
\noindent CESARE: Ale co wtedy? Jeśli rząd nie chce i~nie jest w~stanie nic zrobić, to jakie jest lekarstwo? Nawet jeśli dokonasz
rewolucji, będziesz musiał stworzyć inny rząd; a ponieważ mówisz, że wszystkie rządy są takie same, po rewolucji
wszystko będzie takie samo jak przedtem. 




 
\noindent GIORGIO: Miałbyś rację, gdyby nasza rewolucja doprowadziła po prostu do zmiany rządu. Jednak my chcemy całkowitego
przekształcenia ustroju własności, systemu produkcji i~wymiany; a jeśli chodzi o rząd, bezużyteczny, szkodliwy i~pasożytniczy organ, w~ogóle go nie chcemy. Wierzymy, że dopóki istnieje rząd, czyli organ nałożony na społeczeństwo i~wyposażony w~środki do narzucania siłą własnej woli, nie będzie prawdziwej emancypacji, nie będzie pokoju między
ludźmi. 

 
Wiesz, że jestem anarchistą, a anarchia oznacza społeczeństwo bez rządu. 




 
\noindent CESARE: Ale co masz na myśli? Społeczeństwo bez rządu! Jak mógłbyś żyć? Kto miałby stanowić prawo? Kto miałby je
realizować? 




 
\noindent GIORGIO: Widzę, że nie masz pojęcia, czego chcemy. Aby uniknąć marnowania czasu na dygresje, proszę pozwolić mi krótko,
ale metodycznie wyjaśnić nasz program; a potem możemy omawiać sprawy dla naszej obopólnej korzyści. 

 
Ale teraz jest późno; będziemy kontynuować następnym razem. 










\chapter*{Trzy}



 
\noindent CESARE: Więc dziś wieczorem wyjaśnisz, jak możemy żyć bez rządu? 




 
\noindent GIORGIO: Zrobię, co w~mojej mocy. Ale przede wszystkim musimy wziąć pod uwagę, jak sprawy mają się w~społeczeństwie i~czy rzeczywiście konieczna jest zmiana jego składu. 

 
Patrząc na społeczeństwo, w~którym żyjemy, pierwsze zjawiska, które nas uderzają, to bieda dotykająca mas, niepewność
jutra, która bardziej lub mniej, ciąży na wszystkich, nieustająca walka wszystkich walczących ze wszystkimi, aby
przezwyciężyć głód\ldots 




 
\noindent AMBROGIO: Ale, drogi panie, mógłbyś jeszcze długo mówić o tych społecznych nieszczęściach; niestety przykładów jest
mnóstwo. Ale to nie służy żadnemu celowi i~nie pokazuje, że byłoby lepiej, gdybyśmy wszystko wywrócili do góry nogami.
Nie tylko bieda dotyka ludzkość; są też zarazy, cholera, trzęsienia ziemi\ldots i~byłoby dziwne, gdybyście chcieli
skierować rewolucję przeciwko tym plagom. 

 
Zło tkwi w~naturze rzeczy\ldots 




 
\noindent GIORGIO: Ale tak naprawdę chcę ci pokazać, że bieda zależy od obecnego sposobu organizacji społecznej i~że w~bardziej
egalitarnym i~racjonalnie zorganizowanym społeczeństwie musi zniknąć. 

 
Kiedy nie znamy przyczyn zła i~nie mamy rozwiązań, cóż, niewiele możemy z~tym zrobić; ale gdy tylko rozwiązanie zostanie
znalezione, troską i~obowiązkiem każdego staje się wprowadzenie go w~życie. 




 
\noindent AMBROGIO: Oto twój błąd: nędza wynika z~przyczyn nadrzędnych w~stosunku do woli ludzkiej i~ludzkiego prawa. Ubóstwo
wynika z~podłości natury, która nie dostarcza wystarczającej ilości produktów do zaspokojenia ludzkich pragnień. 

 
Przyjrzyj się zwierzętom, których nie można winić o \begin{itshape}hańbę
kapitalizmu \end{itshape} ani o \textit{rząd tyrański}; muszą walczyć o jedzenie i~często umierają
z głodu. 

 
Kiedy szafa jest pusta, szafka jest pusta. Prawda jest taka, że  na świecie jest zbyt wielu ludzi. Gdyby
ludzie potrafili panować nad sobą i~nie mieli dzieci, póki nie mogliby ich utrzymać\ldots Czytałeś Malthusa? 




 
\noindent GIORGIO: Tak, trochę; ale wszystko jedno, gdybym nie czytał jego pracy. To, co wiem, bez konieczności czytania
jakiejkolwiek części, to to, że trzeba mieć trochę odwagi, muszę powiedzieć, żeby utrzymywać takie rzeczy! 

 
Bieda wynika z~podłości natury, mówisz, chociaż jesteś świadom, że istnieje nieuprawiana ziemia\ldots 




 
\noindent AMBROGIO: Jeśli ziemia jest nieuprawiana, oznacza to, że nie można jej uprawiać, że nie może wyprodukować wystarczającej
ilości, aby pokryć związane z~tym koszty. 




 
\noindent GIORGIO: Wierzysz w~to? 

 
Wypróbuj eksperyment i~daj ją chłopom, a zobaczysz, jakie ogrody stworzą. Ale ty nie mówisz poważnie? Przecież znaczna
część tej ziemi była uprawiana w~czasach, gdy sztuka rolnictwa była w~powijakach, a chemia i~technika rolnicza prawie
nie istniały! Czy nie wiesz, że dzisiaj nawet kamienie mogą zamienić się w~żyzną ziemię? Czy nie wiecie, że
agronomowie, nawet ci mniej wizjonerscy, obliczyli, że terytorium takie jak Włochy, jeśli będzie racjonalnie uprawiane,
z łatwością utrzyma w~obfitości stumilionową populację? 

 
Prawdziwym powodem, dla którego ziemia pozostaje nieuprawiana i~dlaczego uprawiana ziemia daje tylko niewielką część
swojego pełnego potencjału, biorąc pod uwagę przyjęcie mniej prymitywnych metod uprawy, jest to, że właściciele nie
mają żadnego interesu w~zwiększaniu jej produkcji. 

 
Nie troszczą się o dobro ludzi; produkują, aby sprzedawać, i~wiedzą, że gdy jest dużo towarów, ceny są obniżane, a zyski
maleją i~w~sumie mogą być mniejsze niż wtedy, gdy towarów jest mało i~można je sprzedać po cenach, które im
odpowiadają. 

 
Nie chodzi o to, że dzieje się tak tylko w~odniesieniu do produktów rolnych. W~każdej gałęzi działalności człowieka jest
tak samo. Na przykład: W~każdym mieście biedni są zmuszani do życia w~zarażonych ruderach, stłoczonych bez względu na
higienę i~moralność, w~warunkach, w~których nie można utrzymać czystości i~osiągnąć ludzkich warunków. Dlaczego tak się
dzieje? Może dlatego, że nie ma domów? Ale dlaczego nie buduje się solidnych, wygodnych i~pięknych domów dla
wszystkich? 

 
Kamieni, cegieł, wapna, stali, drewna, wszystkich materiałów potrzebnych do budowy istnieje pod dostatkiem; podobnie jak
istnieją bezrobotni murarze, stolarze i~architekci, którzy nie pragną niczego więcej poza pracą; dlaczego więc jest tak
dużo niewykorzystanych możliwości, skoro można je wykorzystać dla korzyści wszystkich. 

 
Powód jest prosty i~polega na tym, że gdyby było dużo domów, czynsze by spadły. Właściciele już wybudowanych domów,
czyli ci sami ludzie, którzy mają środki na budowę innych, tak naprawdę nie mają chęci patrzenia, jak ich czynsze
spadają tylko po to, by zyskać aprobatę biednych. 




 
\noindent CESARE: Jest trochę prawdy w~tym, co mówisz; ale łudzisz się co do wyjaśnienia bolesnych rzeczy, które nękają nasz
kraj. 

 
Przyczyną źle uprawianej lub pozostawionej bezczynności ziemi, ugrzęźnięcia interesów i~ogólnie nędzy jest brak rozmachu
burżuazji. Kapitaliści albo się boją, albo są ignorantami i~nie chcą lub nie wiedzą, jak rozwijać przemysł; właściciele
ziemscy nie wiedzą, jak zerwać z~metodami swoich dziadków i~nie chcą, aby im przeszkadzano; handlowcy nie wiedzą, jak
znaleźć nowe rynki zbytu, a rząd ze swoją polityką fiskalną i~głupią polityką celną,  zamiast zachęcać do prywatnych
inicjatyw, przeszkadza im i~dusi ich w~powijakach. Spójrz na Francję, Anglię i~Niemcy. 




 
\noindent GIORGIO: Nie wątpię, że nasza burżuazja jest gnuśna i~ignorancka, ale jej niższość tylko wyjaśnia, dlaczego zostaje
pokonana przez burżuazję innych krajów w~walce o podbój rynku światowego: W~żaden sposób nie wyjaśnia to przyczyny
biedy ludzi. Wyraźnym dowodem jest to, że bieda, brak pracy i~cała reszta społecznego zła istnieje w~krajach, w~których
burżuazja jest bardziej aktywna i~inteligentniejsza, tak samo jak we Włoszech; w~rzeczywistości zło to jest na ogół
bardziej intensywne w~krajach, w~których przemysł jest bardziej rozwinięty, chyba że robotnicy byli w~stanie, dzięki
organizacji, oporowi lub buntowi, uzyskać lepsze warunki życia. 

 
Kapitalizm jest wszędzie taki sam.  Aby przetrwać i~prosperować, potrzebuje stałej sytuacji częściowego niedoboru:
potrzebuje jej do utrzymania swoich cen i~stworzenia głodnych mas do pracy w~każdych warunkach. 

 
Widzisz, w~rzeczywistości, kiedy produkcja w~kraju jest w~pełnym rozkwicie, nigdy nie ma to na celu zapewnienia
producentom środków na zwiększenie konsumpcji, ale zawsze na sprzedaż na rynek zewnętrzny. Jeśli konsumpcja krajowa
wzrasta, dzieje się to tylko wtedy, gdy robotnicy mogą skorzystać z~tych okoliczności, aby zażądać wzrostu ich płac, a
w konsekwencji mogą kupić więcej towarów. Ale potem, kiedy z~tego czy innego powodu rynek zewnętrzny, dla którego
produkują, już nie kupuje, przychodzi kryzys, praca się zatrzymuje, płace spadają, a straszne ubóstwo znów zaczyna siać
spustoszenie. A jednak w~tym samym kraju, gdzie ogromnej większości brakuje wszystkiego, o wiele rozsądniej byłoby
pracować na własne potrzeby! Ale, w~takim razie, co zyskaliby na tym kapitaliści! 




 
\noindent AMBROGIO: Więc myślisz, że to wszystko wina kapitalizmu? 




 
\noindent GIORGIO: Tak, oczywiście; lub bardziej ogólnie, wynika to z~faktu, że kilka jednostek zgromadziło ziemię i~wszystkie
środki produkcji i~może narzucić robotnikom swoją wolę w~taki sposób, że zamiast produkować dla zaspokojenia potrzeb
ludzi i~mając na uwadze te potrzeby, produkcja jest nastawiona na generowanie zysku dla pracodawców. 

 
Wszystkie usprawiedliwienia, jakie wymyślacie, aby zachować burżuazyjne przywileje, są całkowicie błędne, lub raczej są
kłamstwami. Przed chwilą mówiłeś, że przyczyną ubóstwa jest niedobór produktów. A innym razem, mierząc się z~problemami
bezrobotnych, powiedzielibyście, że magazyny są pełne, że towarów nie można sprzedać, a właściciele nie mogą tworzyć
miejsc pracy, żeby nie wyrzucać towarów. 

 
W rzeczywistości jest to typowy przykład absurdalności systemu: umieramy z~głodu, ponieważ magazyny są pełne i~nie ma
potrzeby uprawiać ziemi, a raczej właściciele ziemscy nie potrzebują uprawiać swojej ziemi; szewcy nie pracują, więc
chodzą w~znoszonych butach, bo butów jest za dużo\ldots i~tak to idzie\ldots 




 
\noindent AMBROGIO: Więc to kapitaliści powinni umrzeć z~głodu? 




 
\noindent GIORGIO: O! Zdecydowanie nie. Powinni po prostu pracować jak wszyscy inni. Może ci się to wydawać surowe, ale nie
rozumiesz: kiedy ktoś się dobrze odżywia, praca nie jest groźna. W~rzeczywistości mogę ci wykazać, że jest potrzebą i~spełnieniem ludzkiej natury. Ale bądź uczciwy, jutro muszę iść do pracy i~jest już bardzo późno. 

 
Do następnego razu. 










\chapter*{Cztery}



 
\noindent CESARE: Lubię się z~tobą kłócić. Masz pewien sposób przedstawiania rzeczy, który sprawia, że  wydajesz się
mieć rację\ldots i~rzeczywiście, nie twierdzę, że całkowicie się mylisz. 

 
Istnieją z~pewnością pewne absurdy, rzeczywiste lub pozorne, w~obecnym porządku społecznym. Na przykład trudno mi
zrozumieć politykę celną. Podczas gdy tutaj ludzie umierają z~głodu lub związanych z~tym chorób, ponieważ brakuje im
wystarczającej ilości chleba dobrej jakości, rząd utrudnia import zboża z~Ameryki, gdzie mają więcej, niż potrzebują i~niczego bardziej nie pragną, niż sprzedać je nam. To tak, jakby być głodnym, ale nie chcieć jeść! 

 
Jednakże\ldots 




 
\noindent GIORGIO: Tak, rzeczywiście, ale rząd nie jest głodny; podobnie jak wielcy hodowcy pszenicy we Włoszech, w~interesie
których rząd nakłada cła na pszenicę. Gdyby ci, którzy są głodni, mieli swobodę działania, zobaczylibyście, że nie
odrzuciliby pszenicy! 




 
\noindent CESARE: Wiem o tym i~rozumiem, że tego rodzaju argumentami zniechęcasz zwykłych ludzi, którzy widzą rzeczy tylko w~ogólnych kategoriach i~z jednego punktu widzenia. Ale aby uniknąć błędów, musimy patrzeć na wszystkie strony problemu,
co właśnie miałem zrobić, kiedy mi przerwałeś. 




 
Prawdą jest, że interesy właścicieli mają ogromny wpływ na nałożenie podatku importowego. Ale z~drugiej strony, gdyby
granice były otwarte, Amerykanie, którzy mogą produkować pszenicę i~mięso na korzystniejszych warunkach niż u nas,
zaopatrzyliby w~końcu cały nasz rynek: I~co wtedy zrobiliby nasi rolnicy? Właściciele byliby zrujnowani, ale robotnikom
byłoby jeszcze gorzej. Chleb sprzedawano by za niewielkie kwoty. Ale gdyby nie było sposobu na zarobienie tych
pieniędzy, i~tak umarłbyś z~głodu. A potem Amerykanie, niezależnie od tego, czy towary są drogie, czy tanie, chcą
zarabiać, a jeśli we Włoszech nie produkujemy, to czym zapłacimy? 

 
Możesz powiedzieć, że we Włoszech moglibyśmy uprawiać te produkty, które są odpowiednie dla naszej gleby i~klimatu, a
następnie wymieniać je za granicą: na przykład wino, pomarańcze, kwiaty i~tym podobne. Ale co, jeśli rzeczy, które
jesteśmy w~stanie wyprodukować na korzystnych warunkach, nie są pożądane przez innych, albo ponieważ nie mają z~nich
pożytku, czy też dlatego, że sami je wytwarzają? Nie mówiąc już o tym, że do zmiany reżimu produkcyjnego potrzebny jest
kapitał, wiedza i~przede wszystkim czas: co byśmy w~międzyczasie jedli? 




 
\noindent GIORGIO: Doskonale! Trafiłeś w~punkt. Wolny handel nie może rozwiązać problemu ubóstwa bardziej niż protekcjonizm. Wolny
handel jest dobry dla konsumentów i~szkodzi producentom, i~odwrotnie, protekcjonizm jest dobry dla chronionych
producentów, ale szkodzi konsumentom; a ponieważ pracownicy są w~tym samym czasie zarówno konsumentami, jak i~producentami, w~końcu zawsze jest tak samo. 

 
I zawsze będzie tak samo, dopóki system kapitalistyczny nie zostanie obalony. 

 
Gdyby pracownicy pracowali dla siebie, a nie dla zysków właściciela, to każdy kraj byłby w~stanie wyprodukować
wystarczająco dużo na własne potrzeby i~musieliby jedynie dojść do porozumienia z~innymi krajami w~sprawie podziału
produktywnej pracy zgodnie z~jakością gleby, klimatem, dostępnością zasobów, skłonnościami mieszkańców itp., aby
wszystkie ludzie mogli cieszyć się wszystkim, co najlepsze, przy minimalnym możliwym wysiłku. 




 
\noindent CESARE: Tak, ale to tylko mrzonki. 




 
\noindent GIORGIO: Dziś mogą być snami; ale kiedy ludzie zrozumieją, w~jaki sposób mogą poprawić życie, sen wkrótce zostanie
przekształcony w~rzeczywistość. Jedynymi przeszkodami są egoizm jednych i~ignorancja innych. 




 
\noindent CESARE: Są inne przeszkody, przyjacielu. Myślisz, że po wyrzuceniu właścicieli będziesz pławić się w~złocie\ldots 




 
\noindent GIORGIO: Nie o to mi chodzi. Wręcz przeciwnie, myślę, że przezwyciężyć ten stan niedostatku, w~którym utrzymuje nas
kapitalizm, i~zorganizować produkcję w~dużej mierze w~celu zaspokojenia potrzeb wszystkich, trzeba wykonać dużo pracy;
ale ludziom nie brakuje nawet chęci do pracy, ale możliwości. Na obecny system narzekamy nie tyle dlatego, że musimy
utrzymywać jakichś próżniaków: chociaż to nam się na pewno nie podoba, ale dlatego, że to właśnie ci próżniacy regulują
pracę i~uniemożliwiają nam pracę w~dobrych warunkach i~produkcję obfitości dla wszystkich. 




 
\noindent CESARE: Przesadzasz. To prawda, że często właściciele nie zatrudniają ludzi, aby spekulować na braku produktów, ale
częściej dzieje się tak dlatego, że sami nie mają kapitału. 

 
Ziemia i~surowce nie wystarczą do produkcji. Potrzebujecie, jak wiecie, narzędzi, maszyn, pomieszczeń, środków na
opłacenie pracujących robotników, jednym słowem kapitału; a ten kumuluje się powoli. Ileż przedsięwzięć nie udaje się
ruszyć z~miejsca lub, które już ruszyły, upada z~powodu braku kapitału! Czy możecie sobie wyobrazić skutek, gdyby tak,
jak sobie życzycie, nastąpiła rewolucja społeczna? Wraz ze zniszczeniem kapitału i~wielkim chaosem, który by po nim
nastąpił, nastąpiłoby ogólne zubożenie. 




 
\noindent GIORGIO: To kolejny błąd lub kolejne kłamstwo obrońców obecnego porządku: brak kapitału. 

 
Kapitału może brakować w~tym czy innym przedsięwzięciu, ponieważ został zabrany przez innych; ale jeśli weźmiemy pod
uwagę społeczeństwo jako całość, okaże się, że istnieje wielka ilość nieaktywnego kapitału, podobnie jak wielka jest
ilość nieuprawianej ziemi. 

 
Czy nie widzicie, ile maszyn rdzewieje, ile fabryk pozostaje zamkniętych, ile domów jest bez lokatorów. 

 
Istnieje zapotrzebowanie na żywność, która żywiłaby pracowników podczas ich pracy; ale tak naprawdę pracownicy muszą
jeść, nawet jeśli są bezrobotni. Jedzą mało i~źle, ale pozostają przy życiu i~są gotowi do pracy, gdy tylko pracodawca
ich potrzebuje. Tak więc to nie dlatego, że brakuje środków do życia, pracownicy nie pracują; a gdyby mogli pracować na
własny rachunek, przystosowaliby się tam, gdzie było to naprawdę konieczne, do pracy żyjąc tak samo, jak gdy są
bezrobotni, bowiem wiedzieliby, że dzięki temu tymczasowemu poświęceniu mogliby w~końcu uciec od społecznego stanu
ubóstwa i~poddaństwa. 

 
Wyobraź sobie, a to jest coś co, było wielokrotnie obserwowane, że trzęsienie ziemi niszczy miasto, rujnując całą
dzielnicę. W~krótkim czasie miasto zostaje zrekonstruowane w~formie piękniejszej niż wcześniej i~po katastrofie nie
pozostaje ani śladu. Ponieważ w~takim przypadku w~interesie właścicieli i~kapitalistów jest zatrudnianie ludzi, szybko
znajdują się środki i~w~mgnieniu oka odbudowuje się całe miasto, gdzie przedtem ciągle twierdzili, że brakuje im
środków na budowę kilku ,,domów robotniczych''. 

 
Jeśli chodzi o zniszczenie kapitału, które miałoby miejsce w~czasie rewolucji, to należy mieć nadzieję, że w~ramach
świadomego ruchu, mającego na celu wspólne posiadanie bogactwa społecznego, ludzie nie będą chcieli zniszczyć tego, co
ma stać się ich własnością. W~każdym razie nie byłoby to tak złe jak trzęsienie ziemi! 

 
Nie, z~pewnością będą trudności, zanim wszystko ułoży się jak najlepiej; ale widzę tylko dwie poważne przeszkody, które
należy pokonać, zanim zaczniemy: brak świadomości ludzi
i\ldots\begin{itshape}karabinierzy \end{itshape}. 




 
\noindent AMBROGIO: Ale powiedz mi coś więcej; mówisz o kapitale, pracy, produkcji, konsumpcji itp.; ale nigdy nie mówisz o
prawach, sprawiedliwości, moralności i~religii? 

 
Kwestie najlepszego wykorzystania ziemi i~kapitału są bardzo ważne; ale jeszcze ważniejsze są kwestie moralne. Chciałbym
też, aby wszystkim dobrze się żyło, ale jeśli w~celu osiągnięcia tej utopii musimy złamać prawa moralne, jeśli musimy
odrzucić odwieczne zasady prawa, na których powinno opierać się każde społeczeństwo obywatelskie, to wolałbym
nieskończenie, aby dzisiejsze cierpienia trwały wiecznie! 

 
Pomyśl tylko, że musi istnieć przecież także najwyższa wola, która reguluje świat. Świat nie powstał sam z~siebie i~musi
istnieć \begin{itshape}coś poza nim \end{itshape}. Nie mówię Bóg, raj, piekło, bo bylibyście
zdolni nie wierzyć w~nie, musi istnieć\textit{coś poza} ten świat, co wszystko wyjaśnia i~gdzie można znaleźć
rekompensatę za pozorne niesprawiedliwości tutaj na dole. 

 
Myślisz, że możesz naruszyć tę z~góry ustaloną harmonię wszechświata? Nie jesteś w~stanie tego zrobić. Nie możemy zrobić
nic innego, jak tylko się jej poddać. 

 
Choć raz przestańcie podburzać masy, przestańcie budzić urojone nadzieje w~duszach najmniej szczęśliwych, przestańcie
dmuchać w~ogień, który niestety tli się pod popiołem. 

 
Czy wy lub inni współcześni barbarzyńcy chcielibyście zniszczyć w~straszliwym społecznym kataklizmie cywilizację, która
jest chwałą naszych przodków i~nas samych? Jeśli chcesz zrobić coś wartościowego, jeśli chcesz w~jak największym
stopniu ulżyć cierpieniu ubogich, powiedz im, aby pogodzili się ze swoim losem, ponieważ prawdziwe szczęście polega na
zadowoleniu. W~końcu każdy niesie swój własny krzyż; każda klasa ma swoje własne udręki i~obowiązki, i~nie zawsze ci,
którzy żyją wśród bogactw, są najbardziej szczęśliwi. 




 
\noindent GIORGIO: Chodź, mój drogi sędzio, odłóż na bok deklaracje o ,,wielkich zasadach'' i~konwencjonalne oburzenie; nie jesteśmy
tutaj w~sądzie i~na razie nie wydajesz na mnie żadnego wyroku. 

 
Jak można by się domyślić, słysząc, jak mówisz, że nie jesteś jednym z~upośledzonych! i~jak użyteczna jest rezygnacja
ubogich\ldots dla tych, którzy z~nich żyją. 

 
Przede wszystkim błagam was, odłóżcie na bok argumenty transcendentalne i~religijne, w~które nawet wy nie wierzycie. Nic
nie wiem o tajemnicach Wszechświata i~ty nic więcej nie wiesz; więc nie ma sensu wprowadzać ich do dyskusji. Co do
reszty, bądźcie świadomi, że wiara w~najwyższego stwórcę, w~Boga stwórcę i~ojca ludzkości, nie byłaby dla was
bezpieczną bronią. Jeśli kapłani, którzy zawsze służyli i~służą bogatym, wywnioskują z~tego, że obowiązkiem biednych
jest poddanie się swemu losowi, to inni mogą wydedukować (i w~toku historii tak wywnioskowali) prawo do sprawiedliwości
i równości. Jeśli Bóg jest naszym wspólnym ojcem, to wszyscy jesteśmy spokrewnieni. Bóg nie może chcieć, aby niektóre z~jego dzieci wykorzystywały i~męczyły inne; a bogaci, władcy, byłoby tak wielu Kainów przeklętych przez Ojca. 

 
Ale zostawmy to. 




 
\noindent AMBROGIO: W~takim razie zapomnijmy o religii, jeśli chcesz, bo tak wiele z~niej byłoby dla ciebie bezcelowe. Ale
uznalibyście prawa, moralność, wyższą sprawiedliwość! 




 
\noindent GIORGIO: Posłuchaj, jeśli to prawda, że prawa, sprawiedliwość i~moralność mogą wymagać i~sankcjonować ucisk i~nieszczęście choćby jednej istoty ludzkiej, od razu powiedziałbym ci, że prawa, sprawiedliwość i~moralność to tylko
kłamstwa, haniebna broń wykuta do obrony uprzywilejowanych; i~tacy są, kiedy mają na myśli to, co ty przez nie
rozumiesz. 

 
Prawa, sprawiedliwość, moralność powinny dążyć do jak największego dobra dla wszystkich, w~przeciwnym razie są
synonimami aroganckiego zachowania i~niesprawiedliwości. I~z pewnością prawdą jest, że ta ich koncepcja odpowiada
potrzebom istnienia i~rozwoju ludzkiej współpracy społecznej, która ukształtowała się i~przetrwała w~ludzkiej
świadomości i~stale zyskuje na sile, pomimo wszelkiego sprzeciwu ze strony tych, którzy teraz zdominowali świat. Ty sam
nie mógłbyś obronić, inaczej niż żałosnym sofizmem, obecnych instytucji społecznych swoją interpretacją abstrakcyjnych
zasad moralności i~sprawiedliwości. 




 
\noindent AMBROGIO: Naprawdę jesteś bardzo zarozumiały. Nie wystarczy zaprzeczać, jak mi się wydaje, prawu własności, ale
utrzymujesz, że nie jesteśmy w~stanie bronić go własnymi zasadami\ldots 




 
Giorgio: Tak, dokładnie. Jeśli chcesz, zademonstruję ci to następnym razem. 










\chapter*{Pięć}



 
\noindent GIORGIO: Zatem, mój drogi sędzio, jeśli się nie mylę, mówiliśmy o prawie własności. 




 
\noindent AMBROGIO: Rzeczywiście. Jestem bardzo ciekawy, jak broniłbyś w~imię sprawiedliwości i~moralności swoich propozycji
plądrowania i~grabieży. 

 
Społeczeństwo, w~którym nikt nie jest pewny swojej własności, nie byłoby już społeczeństwem, ale hordą dzikich bestii
gotowych pożreć się nawzajem. 




 
\noindent GIORGIO: Czy nie wydaje ci się, że tak właśnie jest w~przypadku dzisiejszego społeczeństwa? 

 
Zarzucasz nam grabież i~rabunek; ale przeciwnie, czy to nie właściciele nieustannie plądrują robotników i~okradają ich z~owoców ich pracy? 




 
\noindent AMBROGIO: Właściciele wykorzystują swoje towary w~sposób, który uważają za najlepszy i~mają do tego prawo, tak samo
robotnicy swobodnie rozporządzają swoją pracą. Właściciele i~pracownicy swobodnie zawierają umowy za cenę pracy, a
kiedy umowa jest przestrzegana, nikt nie może narzekać. 

 
Dobroczynność może złagodzić dotkliwe kłopoty, niezasłużone kłopoty, ale prawa muszą pozostać nietykalne. 




 
\noindent GIORGIO: Ale ty mówisz o wolnej umowie! Robotnik, który nie pracuje, nie może jeść, a jego wolność przypomina wolność
napadniętego przez złodziei podróżnika, który w~obawie przed utratą życia oddaje sakiewkę. 




 
\noindent AMBROGIO: W~porządku; ale nie można tego wykorzystać do zanegowania prawa każdej osoby do rozporządzania swoją
własnością według własnego uznania. 




 
\noindent GIORGIO: Ich własność, ich własność! Ale czy to nie dzieje się dlatego, że właściciele ziemscy mogą twierdzić, że ziemia
i jej produkty są ich własnością, a kapitaliści mogą uważać za swoje narzędzia pracy i~inny kapitał stworzony przez
ludzką działalność? 




 
\noindent AMBROGIO: Prawo uznaje ich prawo do tego. 




 
\noindent GIORGIO: Ach! Jeśli to tylko prawo, to nawet uliczny zabójca mógłby domagać się prawa do zabójstwa i~rabunku: musiałby
tylko sformułować kilka artykułów prawnych uznających te prawa. Z~drugiej strony, tego właśnie dokonała klasa
dominująca: stworzyła prawa legitymizujące uzurpacje, których już dokonała, i~uczyniła z~nich narzędzie nowych
zawłaszczeń. 

 
Jeśli wszystkie wasze ,,najwyższe zasady'' opierają się na kodeksach prawnych, wystarczy, że jutro powstanie ustawa o
zniesieniu własności prywatnej, a to, co dzisiaj nazywacie rabunkiem i~grabieżą, stanie się natychmiast ,,najwyższą
zasadą''. 




 
\noindent AMBROGIO: O! Ale prawo musi być sprawiedliwe! Musi być zgodne z~zasadami praw i~moralności i~nie powinno być wynikiem
nieokiełznanych kaprysów, bo inaczej\ldots 




 
\noindent GIORGIO: Tak więc to nie ustawa tworzy prawa, ale prawa, które uzasadniają ustawę. Jakim więc prawem całe istniejące
bogactwo, zarówno bogactwo naturalne, jak i~to, które zostało stworzone pracą ludzkości, należy do kilku jednostek i~daje im prawo życia i~śmierci nad masami nieuprzywilejowanych? 




 
\noindent AMBROGIO: Jest to prawo, które każda osoba ma i~musi mieć, aby swobodnie rozporządzać produktem swojej działalności.
Jest naturalne dla ludzkości, bez niej cywilizacja nie byłaby możliwa. 




 
\noindent GIORGIO: Cóż, ja nigdy! Oto mamy teraz obrońcę praw pracowniczych. Brawo, naprawdę! Ale powiedz mi, dlaczego ci, którzy
pracują, są tymi, którzy nic nie mają, podczas gdy własność faktycznie należy do tych, którzy nie pracują? 

 
Czy nie przyszło ci do głowy, że logiczną konsekwencją twojej teorii jest to, że obecni właściciele są złodziejami i~że,
sprawiedliwie, musimy ich wywłaszczyć, aby przekazać bogactwo, które przywłaszczyli, prawowitym właścicielom,
robotnikom ? 




 
\noindent AMBROGIO: Jeśli istnieją właściciele, którzy nie pracują, to dlatego, że jako pierwsi podjęli pracę, oni lub ich
przodkowie, i~mieli zasługi, by oszczędzać, oraz geniusz, który sprawił, że ich oszczędności przyniosły owoce. 




 
\noindent GIORGIO: Rzeczywiście, czy możesz sobie wyobrazić robotnika, który z~reguły zarabia ledwie na utrzymanie się przy życiu,
oszczędzającego i~gromadzącego bogactwa! 

 
Dobrze wiesz, że źródłem własności jest przemoc, rabunek i~kradzież, legalna lub nielegalna. Ale załóżmy, że jeśli
podoba ci się, że ktoś osiągnął pewne oszczędności produkcji w~swojej pracy, swojej osobistej pracy: jeśli chce cieszyć
się nimi później, kiedy i~jak chce, to jest w~porządku. Ale ten pogląd na sprawy zmienia się całkowicie, gdy zaczyna
się proces oszczędzania, jak to nazywacie,\begin{itshape}niedźwiedzi owoc \end{itshape}.
Oznacza to zmuszanie innych do pracy i~okradanie ich z~części tego, co oni produkują; oznacza gromadzenie niektórych
towarów i~sprzedawanie ich po cenie wyższej niż ich koszt; oznacza sztuczne tworzenie niedoboru w~celu spekulacji;
oznacza odbieranie innym środków do życia pochodzących z~wolnej pracy, aby zmusić ich do pracy za marne wynagrodzenie;
i wiele innych podobnych rzeczy, które nie odpowiadają poczuciu sprawiedliwości i~dowodzą, że własność, jeśli nie
pochodzi z~jawnego i~otwartego rabunku, pochodzi z~pracy innych, którą właściciele w~taki czy inny sposób obrócili dla
swoich własną korzyść. 

 
Czy wydaje ci się sprawiedliwe, że osoba, która (przyznajmy) swoją pracą i~swoim geniuszem zgromadziła trochę kapitału,
może z~tego powodu okradać innych z~produktów swojej pracy, a ponadto pozostawić wszystkim pokoleniom swoich potomków
prawo do życia w~bezczynności na koszt robotników? 

 
Czy wydaje ci się sprawiedliwe, że ponieważ było kilku pracowitych i~oszczędnych ludzi — mówię to, aby uwypuklić twoje
stanowisko — którzy zgromadzili trochę kapitału, wielkie masy ludzkości muszą być skazane na wieczną nędzę i~brutalizację? 

 
A z~drugiej strony, nawet gdyby ktoś pracował na siebie, swoimi mięśniami i~mózgiem, nie wyzyskując nikogo; choćby,
wbrew wszelkim przeciwnościom taki był w~stanie wyprodukować znacznie więcej, niż potrzebował, bez bezpośredniej lub
pośredniej współpracy całego społeczeństwa, nie oznacza to, że z~tego powodu powinien być uprawniony do wyrządzania
krzywdy innym, do odbierać innym środki do życia. Gdyby ktoś zbudował drogę wzdłuż brzegu, nie mógłby z~tego powodu
domagać się prawa do odmowy innym dostępu do morza. Gdyby ktoś mógł samodzielnie uprawiać i~uprawiał całą ziemię
prowincji, nie mógłby z~tego powodu zagłodzić wszystkich mieszkańców tej prowincji. Gdyby ktoś stworzył jakieś nowe i~potężne środki produkcji, nie miałby prawa używać swojego wynalazku w~taki sposób, aby podporządkować sobie ludzi, a
tym bardziej przekazać niezliczonym potomkom prawa do dominacji i~wykorzystywania przyszłych pokoleń. 

 
Ale gubię się w~myśleniu przez chwilę, że właściciele są robotnikami lub potomkami robotników! Czy chcesz, abym ci
powiedział, skąd pochodzi majątek wszystkich panów w~naszej gminie, zarówno szlachciców starożytnego rodu, jak
i\begin{itshape}nowobogackich \end{itshape}? 




 
\noindent AMBROGIO: Nie, nie, na litość boską zostawmy sprawy osobiste. 

 
Jeśli istnieją bogactwa zdobyte wątpliwymi środkami, nie stanowi to powodu do odmowy prawa własności. Przeszłość to
przeszłość i~nie ma sensu ponownie rozkopywać starych problemów. 




 
\noindent GIORGIO: Zostawimy je zakopane, jeśli tego chcesz. Na tyle na ile wiem, nie jest to ważne. Własność indywidualna powinna
zostać zniesiona nie tyle dlatego, że została nabyta w~mniej lub bardziej wątpliwy sposób, ile dlatego, że daje prawo i~środki do wykorzystywania pracy innych, a jej rozwój zawsze doprowadzi do tego, że wielkie masy ludzi są zależne od
kilku. 

 
Ale, nawiasem mówiąc, jak możesz uzasadnić indywidualną własność ziemską swoją teorią oszczędności? Nie możesz mi
powiedzieć, że to powstało dzięki pracy właścicieli lub ich przodków? 




 
\noindent AMBROGIO: Widzisz. Nieuprawiana, sterylna ziemia nie ma żadnej wartości. Ludzie ją okupują, odzyskują, sprawiają, że
plonuje i~naturalnie mają prawo do jej plonów, które nie zostałyby wyprodukowane bez ich pracy na tej ziemi. 




 
\noindent GIORGIO: W~porządku: to jest prawo robotnika do owoców jego własnej pracy; ale to prawo ustaje, gdy przestaje uprawiać
ziemię. Nie sądzisz? 

 
Jak to się dzieje, że obecni właściciele posiadają terytoria, często ogromne, choć nie pracują, nigdy nie pracowali i~najczęściej nie pozwalają pracować innym? 

 
Jak to jest, że ziemie, które nigdy nie były uprawiane, są własnością prywatną? Co to za dzieło, co to za ulepszenie,
które mogło dać początek, w~tym przypadku, prawu własności? 

 
Prawda jest taka, że  dla ziemii, a tym bardziej reszty, źródłem własności prywatnej jest przemoc. I~nie
możesz tego skutecznie usprawiedliwić, jeśli nie zaakceptujesz zasady, że prawo równa się sile, na takim razie\ldots
Niebiosa dopomóż, jeśli któregoś dnia staniesz się osłabiony. 




 
\noindent AMBROGIO: Krótko mówiąc, tracisz z~oczu użyteczność społeczną, nieodłączną konieczność społeczeństwa obywatelskiego. Bez
prawa własności nie byłoby bezpieczeństwa, uporządkowanej pracy, a społeczeństwo pogrążyłoby się w~chaosie. 




 
\noindent GIORGIO: Co! Teraz mówisz o społecznym pożytku? Ale kiedy w~naszych wcześniejszych rozmowach zajmowałem się tylko
szkodami powodowanymi przez własność prywatną, wezwaliście mnie z~powrotem do sporów o prawa abstrakcyjne! 

 
Wystarczy na ten wieczór. Przepraszam, ale muszę iść. Zajmiemy się tym innym razem. 










\chapter*{Sześć}



 
\noindent GIORGIO: Cóż, słyszałeś, co się stało. Ktoś powiedział gazecie o rozmowie, którą odbyliśmy ostatnim razem, a za jej
opublikowanie gazeta została ocenzurowana. 




 
\noindent AMBROGIO: Ach! 




 
\noindent GIORGIO: Oczywiście, to oczywiste, że nic nie wiesz\ldots! Nie rozumiem, jak możesz twierdzić, że jesteś tak pewny swoich
pomysłów, skoro tak bardzo boisz się publicznej dyskusji na ich temat. Gazeta wiernie przedstawiała zarówno twoje, jak
i moje argumenty. Powinieneś się cieszyć, że opinia publiczna jest w~stanie docenić racjonalne podstawy, na których
opiera się obecna konstytucja społeczna, i~oddaje sprawiedliwość daremnej krytyce jej przeciwników. Zamiast tego
zamykasz ludzi, uciszasz ich. 




 
\noindent AMBROGIO: Nie jestem w~to w~ogóle zaangażowany; należę do sądowniczej części, a nie do ministerstwa publicznego. 




 
\noindent GIORGIO: Tak, wiem! Ale wszyscy jesteście takimi samymi kolegami i~ten sam duch ożywia was wszystkich. 

 
Jeśli denerwuje cię moja paplanina, powiedz mi\ldots a pójdę sobie pogadać gdzie indziej. 




 
\noindent AMBROGIO: Nie, nie, wręcz przeciwnie, przyznaję, że jestem zainteresowany. Kontynuujmy; co do zakazu zbliżania się,
jeśli sobie życzysz, dogadam się z~prokuratorem. W~końcu przy obowiązującym prawie nikomu nie odmawia się prawa do
dyskusji. 




 
\noindent GIORGIO: Kontynuujmy zatem. Ostatnim razem, jeśli dobrze pamiętam, w~obronie prawa własności wziąłeś za podstawę prawo
pozytywne, czyli kodeks cywilny, potem poczucie sprawiedliwości, potem użyteczność społeczną. Pozwól, że podsumuję w~kilku słowach moje poglądy na ten temat. 

 
Z mojego punktu widzenia własność indywidualna jest niesprawiedliwa i~niemoralna, ponieważ opiera się albo na jawnej
przemocy, na oszustwie, albo na legalnym wyzysku pracy innych; a to jest szkodliwe, bo przeszkadza produkcji i~uniemożliwia zaspokojenie potrzeb wszystkich przez to, co można uzyskać z~ziemi i~pracy, a także ponieważ tworzy biedę
dla mas i~generuje nienawiść, zbrodnie i~większość zła, które dotyka współczesne społeczeństwo. 

 
Z tych powodów chciałbym ją znieść i~zastąpić ustrojem własności opartym na wspólnej własności, w~którym wszyscy ludzie,
wnosząc odpowiednią ilość pracy, otrzymają maksymalny możliwy poziom dobrobytu. 




 
\noindent AMBROGIO: Doprawdy, nie rozumiem, z~jaką logiką doszedłeś do wspólnej własności. Walczyliście przeciwko własności,
ponieważ według was wywodzi się ona z~przemocy i~wyzysku pracy innych; powiedziałeś, że kapitaliści regulują produkcję,
mając na uwadze swoje zyski, a nie lepsze zaspokojenie potrzeb społecznych przy jak najmniejszym wysiłku pracowników;
odmówiłeś sobie prawa do czerpania dochodów z~ziemi, której sam nie uprawiasz, czerpania zysków z~własnych pieniędzy
lub uzyskiwania odsetek poprzez inwestowanie w~budowę domów i~inne gałęzie przemysłu; ale uznałeś jednak prawo
robotników do produktów ich własnej pracy, właściwie broniłeś tego. W~konsekwencji, zgodnie ze ścisłą logiką, na
podstawie tych kryteriów można kwestionować weryfikację tytułów własności i~domagać się zniesienia odsetek od pieniędzy
i dochodów prywatnych; możesz nawet prosić o likwidację obecnego społeczeństwa i~podział ziemi i~narzędzi pracy między
tych, którzy chcą z~nich korzystać\ldots ale nie możesz mówić o komunizmie. Zawsze musi istnieć indywidualna własność
produktów własnej pracy; a jeśli chcesz, aby twój wyemancypowany robotnik miał w~przyszłości to zabezpieczenie, bez
którego nie będzie wykonywana żadna praca, która nie przynosi natychmiastowego zysku, musisz uznać indywidualną
własność ziemi i~narzędzi produkcji w~zakresie, w~jakim są one używane. 




 
\noindent GIORGIO: Doskonale, proszę kontynuuj; można by powiedzieć, że nawet ty jesteś naznaczony smołą socjalizmu. Jesteś ze
szkoły socjalistycznej różnej się od mojej, ale to wciąż socjalizm. Socjalistyczny sędzia to ciekawe zjawisko. 




 
\noindent AMBROGIO: Nie, nie, nie jestem socjalistą. Pokazałem tylko twoje sprzeczności i~pokazałem ci, że logicznie powinieneś
być \begin{itshape}mutualista \end{itshape}, a nie komunistą, zwolennikiem podziału
własności. 

 
A potem muszę ci powiedzieć, że podział własności na małe części uniemożliwiłby jakiekolwiek duże przedsięwzięcie i~doprowadziłby do powszechnej nędzy. 




 
\noindent GIORGIO: Ale nie jestem mutualistą, zwolennikiem podziału własności, ani, o ile mi wiadomo, żadnym innym współczesnym
socjalistą. 

 
Nie sądzę, aby podział własności był gorszy niż pozostawienie jej w~całości w~rękach kapitalistów; ale wiem, że ten
podział, tam, gdzie to możliwe, spowodowałby poważne szkody w~produkcji. Przede wszystkim nie mógłby przetrwać i~doprowadziłby ponownie do powstania wielkich fortun i~do proletaryzacji mas, aw ostatecznym rozrachunku do nędzy i~wyzysku. 

 
Mówię, że \textit{pracownik ma prawo do całego produktu pracy}: ale uznaję, że to prawo jest tylko formułą
abstrakcyjnej sprawiedliwości; i~oznacza w~praktyce, że nie powinno być wyzysku, że każdy musi pracować i~korzystać z~owoców swojej pracy, zgodnie z~uzgodnionym między sobą zwyczajem. 

 
Pracownicy nie są odizolowanymi istotami, które żyją samodzielnie i~dla siebie, ale istotami społecznymi, które żyją w~ciągłej wymianie usług z~innymi pracownikami i~muszą skoordynować swoje prawa z~prawami innych. Co więcej, nie jest
możliwe, tym bardziej przy nowoczesnych metodach produkcji, dokładne określenie pracy, jaką włożył każdy robotnik, tak
jak niemożliwe jest określenie różnic w~produktywności każdego robotnika lub każdej grupy pracowników, ile wynika z~płodności gleby, jakości używanych narzędzi, korzyści lub trudności wynikające z~położenia geograficznego lub
środowiska społecznego. Dlatego rozwiązania nie można znaleźć w~odniesieniu do ścisłych praw każdej osoby, ale należy
go szukać w~braterskiej zgodzie, w~solidarności. 




 
\noindent AMBROGIO: Ale w~takim razie nie ma już wolności. 




 
\noindent GIORGIO: Wręcz przeciwnie, dopiero wtedy będzie wolność. Wy, tak zwani liberałowie, nazywacie wolność teoretycznym,
abstrakcyjnym prawem do robienia czegoś; i~byłbyś w~stanie powiedzieć bez uśmiechu lub rumieńca, że osoba, która
umarła z~głodu, ponieważ nie była w~stanie zdobyć dla siebie pożywienia, mogła jeść. My przeciwnie, nazywamy wolność
możliwością zrobienia czegoś, i~ta wolność, jedyna prawdziwa, staje się tym większa, im większa jest zgoda między
ludźmi i~wsparcie, jakiego sobie wzajemnie udzielają. 




 
\noindent AMBROGIO: Powiedziałeś, że jeśli majątek zostanie podzielony, to wkrótce wielkie fortuny zostaną przywrócone i~nastąpi
powrót do pierwotnej sytuacji. Dlaczego? 




 
\noindent GIORGIO: Ponieważ na początku osiągnięcie idealnej równości byłoby celem niemożliwym do osiągnięcia. Istnieją różne
rodzaje gruntów, niektóre dają dużo przy niewielkim nakładzie pracy, a inne mało przy dużym nakładzie pracy; różne
miejsca mają wiele zalet i~wad; istnieją również wielkie różnice w~sile fizycznej i~intelektualnej między jedną osobą a
drugą. Teraz z~tych podziałów wyłoniłaby się naturalnie rywalizacja i~walka: najlepsza ziemia, najlepsze narzędzia,
najlepsze miejsca trafiłyby do najsilniejszych, najbardziej inteligentnych lub najbardziej przebiegłych. Stąd, będąc w~rękach najlepszych środków materialnych, ci ludzie szybko znaleźliby się w~pozycji wyższej od innych, a korzystając z~tych wczesnych przewag, łatwo by urośli w~siłę, rozpoczynając w~ten sposób nowy proces wyzysku i~wywłaszczania słabych,
co doprowadziłoby do przywrócenia społeczeństwa burżuazyjnego. 




 
\noindent AMBROGIO: Zatem, naprawdę poważnie, jesteś komunistą? Chcesz praw, które uznałyby udział każdej osoby za nieprzenoszalny
i otaczałyby słabych poważnymi gwarancjami prawnymi. 




 
\noindent GIORGIO: O! Zawsze myślisz, że można wszystko naprawić prawami. Nie na darmo jesteś sędzią. Prawa są ustanawiane i~uchylane, aby zadowolić najsilniejszych. 

 
Ci, którzy są trochę silniejsi od przeciętnej, łamią je; ci, którzy są o wiele silniejsi, uchylają je i~zmuszają innych,
aby dopasowali się do ich interesów. 




 
\noindent AMBROGIO: A więc? 




 
\noindent GIORGIO: No więc, już ci mówiłem, konieczne jest zastąpienie walki między ludźmi porozumieniem i~solidarnością, a żeby
to osiągnąć, trzeba przede wszystkim znieść własność indywidualną. 




 
\noindent AMBROGIO: Ale nie byłoby żadnych problemów ze wszystkimi dostępnymi dobrami. Wszystko należy do wszystkich, kto chce,
może pracować, a kto nie, może się kochać; jedz, pij, raduj się! Och, co za Kraina Obfitości! Co za dobre życie! Co za
piękny dom wariatów! Ha! Ha! Ha! 




 
\noindent GIORGIO: Zważywszy na to, co robisz, chcąc racjonalnie bronić społeczeństwa, które utrzymuje się przy pomocy brutalnej
siły, nie sądzę, żebyś miał z~czego się śmiać! 

 
Tak, mój dobry panie, jestem komunistą. Ale wydaje mi się, że masz dziwne pojęcie komunizmu. Następnym razem postaram
się, żebyś zrozumiał. Na razie dobry wieczór. 










\chapter*{Siedem}



 
\noindent AMBROGIO: No to może mi wyjaśnisz, o co chodzi z~tym twoim komunizmem. 




 
\noindent GIORGIO: Z~przyjemnością. 

 
Komunizm jest metodą organizacji społecznej, w~której ludzie, zamiast walczyć między sobą o zmonopolizowanie naturalnych
korzyści i~alternatywnie wyzyskiwanie i~uciskanie się nawzajem, jak to ma miejsce w~dzisiejszym społeczeństwie,
zrzeszają się i~zgadzają się współpracować w~najlepszym interesie wszystkich. Wychodząc z~zasady, że ziemia, kopalnie i~wszystkie siły natury należą do wszystkich i~że całe zgromadzone bogactwo i~zdobycze poprzednich pokoleń również należą
do wszystkich, ludzie w~komunizmie chcieliby współpracować, aby wyprodukować wszystko, co niezbędne. 




\noindent AMBROGIO: Rozumiem. Chcecie, jak stwierdzono w~ulotce, która trafiła w~ręce podczas procesu anarchistycznego, aby każdy
mógł produkować zgodnie ze swoimi możliwościami i~konsumować zgodnie ze swoimi potrzebami\begin{itshape};
lub,\end{itshape}aby każdy dawał, co może i~brał to, czego potrzebuje\textit{Czy tak nie
jest?}




 
\noindent GIORGIO: W~rzeczywistości są to zasady, które często powtarzamy; ale aby poprawnie reprezentowały naszą koncepcję
społeczeństwa komunistycznego, konieczne jest zrozumienie, co mają na myśli. Nie chodzi oczywiście o absolutne prawo do
zaspokojenia \begin{itshape}wszystkich \end{itshape} potrzeb, ponieważ potrzeby są
nieskończone, rosną szybciej niż środki do ich zaspokojenia, a więc ich zaspokojenie jest zawsze ograniczone
zdolnościami produkcyjnymi; nie byłoby też pożyteczne ani sprawiedliwe, aby społeczność w~celu zaspokojenia nadmiernych
potrzeb, zwanych inaczej kaprysami, kilku jednostek, podejmowała pracę niewspółmierną do wytwarzanej użyteczności. Nie
mówimy też o zatrudnieniu \textit{wszystkich } swoich sił w~wytwarzaniu rzeczy, bo dosłownie oznaczałoby to
pracę do wyczerpania, co oznaczałoby, że maksymalizując zaspokojenie ludzkich potrzeb, niszczymy ludzkość. 

 
Chcielibyśmy, aby wszyscy żyli jak najlepiej: aby każdy przy minimalnym wysiłku osiągnął maksimum satysfakcji. Nie wiem,
jaką podać ci teoretyczną formułę, która poprawnie obrazuje taki układ spraw; ale kiedy pozbędziemy się społecznego
środowiska szefa i~policji, a ludzie uznają się za rodzinę i~pomyślą o pomaganiu sobie nawzajem zamiast o wyzysku,
wkrótce zostanie odkryta praktyczna formuła życia społecznego. W~każdym razie wykorzystamy w~pełni to, co wiemy i~co
możemy zrobić, wprowadzając częściowe modyfikacje, gdy nauczymy się robić rzeczy lepiej. 




 
\noindent AMBROGIO: Rozumiem: jesteś zwolennikiem tzw \begin{itshape}prise au tas \end{itshape}, jak
powiedzieliby twoi towarzysze z~Francji, to znaczy każdy produkuje to, co mu się podoba i~na czym mu zależy,
\textit{wrzuca na stos} lub, jeśli wolisz, przynosi do wspólnego magazynu to, co wyprodukował; i~każda osoba
 \begin{itshape}bierze ze sterty\end{itshape} jak mu się podoba i~czego potrzebuje. Czy tak
nie jest? 




 
\noindent GIORGIO: Zauważyłem, że postanowiłeś się trochę doinformować w~tej sprawie i~chyba przeczytałeś dokumenty procesu
dokładniej niż zwykle, kiedy wysyłasz nas do więzienia. Gdyby wszyscy sędziowie i~policjanci tak postępowali, to
rzeczy, które nam kradną podczas rewizji, przynajmniej na coś by się przydały! 

 
Wróćmy jednak do naszej dyskusji. Nawet ta formuła \begin{itshape}wziąć ze
sterty \end{itshape} jest tylko formą słów, która wyraża skłonność do zastąpienia dzisiejszego ducha
rynku duchem braterstwa i~solidarności, ale nie wskazuje z~całą pewnością określonej metody organizacji społecznej. Być
może znalazłbyś wśród nas takich, którzy tę formułę traktują dosłownie, bo zakładają, że spontanicznie podejmowanej
pracy zawsze będzie dużo, a produktów będzie się kumulować w~takiej ilości i~różnorodności, że zasady dotyczące pracy
czy konsumpcji będą bezcelowe. Ale ja tak nie myślę: wierzę, jak już mówiłem, że ludzie zawsze mają więcej potrzeb niż
możliwości ich zaspokojenia i~cieszę się z~tego, bo to bodziec do postępu; i~myślę, że nawet gdybyśmy mogli, absurdalną
stratą energii byłoby produkowanie na ślepo w~celu zaspokojenia wszystkich możliwych potrzeb, zamiast kalkulować
rzeczywiste potrzeby i~organizować je tak, aby jak najmniejszym wysiłkiem je zaspokoić. Tak więc raz jeszcze
rozwiązanie leży w~zgodzie między ludźmi i~w~porozumieniach, wyrażonych lub milczących, które nastąpią, gdy osiągną
równość warunków i~będą zainspirowani poczuciem solidarności. 

 
Postaraj się wejść w~ducha naszego programu i~nie przejmuj się zbytnio formułami, które, w~naszej partii, jak każdej
innej, nie są zwięzłe i~uderzające, ale zawsze są niejasnym i~niedokładnym sposobem wyrażenia szerokiego kierunku. 




 
\noindent AMBROGIO: Ale czy nie rozumiesz, że komunizm jest zaprzeczeniem wolności i~osobowości człowieka? Być może istniało u
początków ludzkości, kiedy istoty ludzkie, słabo rozwinięte intelektualnie i~moralnie, były szczęśliwe, kiedy mogły
zaspokoić swoje materialne apetyty jako członkowie hordy. Być może jest to możliwe w~stowarzyszeniu religijnym lub
zakonie, które dąży do stłumienia ludzkich namiętności i~szczyci się włączaniem jednostki do wspólnoty religijnej i~uważa posłuszeństwo za główny obowiązek. Ale w~nowoczesnym społeczeństwie, w~którym następuje wielki rozkwit
cywilizacji, wytworzony przez swobodną działalność jednostek, z~potrzebą niezależności i~wolności, która dręczy i~uszlachetnia współczesnego człowieka, komunizm jest nie tylko niemożliwym marzeniem, jest powrotem do barbarzyństwa.
Każda aktywność byłaby sparaliżowana; każdy obiecujący konkurs, w~którym można było się wyróżnić, zamanifestować swoją
indywidualność, zgaszony\ldots 




 
\noindent GIORGIO: I~tak dalej, i~tak dalej. 

 
Pospiesz się. Nie marnuj swojej elokwencji. Są to dobrze znane zwroty\ldots i~są niczym więcej niż mnóstwem bezczelnych i~nieodpowiedzialnych kłamstw. Wolność, indywidualność tych, którzy umierają z~głodu! Co za prymitywna ironia! Cóż za
głęboka hipokryzja! 

 
Bronicie społeczeństwa, w~którym ogromna większość żyje w~bestialskich warunkach, społeczeństwa, w~którym robotnicy
umierają z~niedostatku i~głodu, w~którym tysiące i~miliony dzieci umierają z~powodu braku opieki, w~którym kobiety
prostytuują się z~głodu, w~którym ignorancja zaćmiewa umysł, w~którym nawet wykształceni muszą sprzedawać swój talent i~kłamać, aby się najeść, w~którym nikt nie jest pewien jutra, a ty śmiesz mówić o wolności i~indywidualności? 

 
Być może wolność i~możliwość rozwoju własnej indywidualności istnieją dla ciebie, dla małej kasty uprzywilejowanych\ldots a
może nawet dla nich nie. Te same uprzywilejowane osoby są ofiarami walki między jednym człowiekiem a drugim, która
zanieczyszcza całe życie społeczne, i~wiele by zyskały, gdyby mogły żyć w~społeczeństwie wzajemnego zaufania, wolnym
wśród wolnych, równym wśród równych. 

 
Czy jednak można podtrzymać pogląd, że solidarność szkodzi wolności i~rozwojowi jednostki? Gdybyśmy rozmawiali o
rodzinie, a będziemy o tym rozmawiać, kiedy tylko zechcesz, wyśpiewywałbyś jeden ze zwykłych konwencjonalnych hymnów na
cześć tej świętej instytucji, tego kamienia węgielnego itd. itd.  Cóż, co w~rodzinie wychwalamy, jeśli nie to, co
powszechnie istnieje, miłość i~solidarność panujące wśród jej członkowie. Czy uważałbyś, że członkowie rodziny byliby
bardziej wolni, a ich indywidualność bardziej rozwinięta, gdyby zamiast się kochać i~wspólnie pracować dla wspólnego
dobra, kradli, nienawidzili się i~bili? 




 
\noindent AMBROGIO: Ale żeby uregulować społeczeństwo jak rodzinę, zorganizować i~sprawić, by społeczeństwo komunistyczne
funkcjonowało, potrzebna jest ogromna centralizacja, żelazny despotyzm i~wszechobecne państwo. Wyobraź sobie, jaką
opresyjną władzę miałby rząd, który mógłby rozporządzać całym społecznym bogactwem i~przydzielać wszystkim pracę, którą
muszą wykonać, i~dobra, które mogliby skonsumować! 




 
\noindent GIORGIO: Z~pewnością, jeśli komunizm miałby być tym, czym go sobie wyobrażasz i~jak jest postrzegany przez kilka
autorytarnych szkół, to byłoby to niemożliwe do osiągnięcia, a jeśli możliwe, skończyłoby się kolosalną i~bardzo
skomplikowaną tyranią, która nieuchronnie wywołałaby wielką reakcję. 

 
Ale w~komunizmie, który chcemy, nie ma nic z~tych rzeczy. Chcemy wolnego komunizmu,
\begin{itshape}anarchizmu \end{itshape}, jeśli to słowo cię nie obraża. Innymi słowy, chcemy
komunizmu, który jest swobodnie zorganizowany, od dołu do góry, począwszy od jednostek, które jednoczą się w~stowarzyszeniach, które powoli rosną stopniowo w~coraz bardziej złożone federacje stowarzyszeń, ostatecznie obejmując
całą ludzkość w~ogólnym porozumieniu współpracy i~solidarności. I~tak jak ten komunizm będzie swobodnie ukonstytuowany,
musi dobrowolnie utrzymywać się dzięki woli zaangażowanych. 




 
\noindent AMBROGIO: Ale aby stało się to możliwe, potrzebowalibyście ludzi, aby byli aniołami, aby wszyscy byli altruistami!
Zamiast tego ludzie są z~natury egoistyczni, niegodziwi, obłudni i~leniwi. 




 
\noindent GIORGIO: Z~pewnością dlatego, że aby komunizm stał się możliwy, konieczne jest, aby istoty ludzkie, częściowo z~powodu
impulsu do towarzyskości, a częściowo z~wyraźnego zrozumienia swoich interesów, nie życzyły sobie nawzajem złej woli,
ale chciały się dogadać i~praktykować wzajemną pomoc. Ale ten stan nie wydaje się niemożliwy, nawet teraz jest normalny
i powszechny. Obecna organizacja społeczna jest permanentną przyczyną antagonizmów i~konfliktów między klasami i~jednostkami: I~jeśli pomimo tego społeczeństwo wciąż jest w~stanie się utrzymać i~nie przeradza się dosłownie w~stado
pożerających się nawzajem wilków, to właśnie dzięki głębokiemu ludzkiemu instynktowi do socjalizacji, który wytwarza
tysiące aktów solidarności, sympatii, oddania, poświęcenia, które są dokonywane w~każdej chwili, nawet o nich nie
myśląc, który umożliwia przetrwanie społeczeństwa, niezależnie od przyczyn rozpadu, które niesie ze sobą w~sobie. 

 
Istoty ludzkie są z~natury zarówno egoistyczne, jak i~altruistyczne, biologicznie zdeterminowane, powiedziałbym, do
społeczności. Gdyby ludzie nie byli egoistami, to znaczy, gdyby nie mieli instynktu samozachowawczego, nie mogliby
istnieć jako jednostki; i~gdyby nie byli altruistami, innymi słowy, gdyby nie mieli instynktu poświęcenia się dla
innych, którego pierwszym przejawem jest miłość do własnych dzieci, nie mogliby istnieć jako gatunek i~najprawdopodobniej nie rozwinęliby życia społecznego. 

 
Współistnienie uczucia egoistycznego i~altruistycznego oraz niemożność zaspokojenia ich w~istniejącym społeczeństwie
sprawiają, że dzisiaj nikt nie jest usatysfakcjonowany, nawet ci, którzy zajmują uprzywilejowane pozycje. Z~drugiej
strony komunizm jest formą społeczną, w~której mieszają się egoizm i~altruizm, i~wszyscy to zaakceptują, ponieważ
przynosi korzyści wszystkim. 




 
\noindent AMBROGIO: Może być tak, jak mówisz: ale czy myślisz, że wszyscy by chcieli i~wiedzieliby, jak przystosować się do
obowiązków, jakie narzuca społeczeństwo komunistyczne, jeśli na przykład ludzie nie chcą pracować? Oczywiście na
wszystko masz w~teorii odpowiedź, która najlepiej pasuje do twojej argumentacji, i~powiesz mi, że praca jest organiczną
potrzebą, przyjemnością i~że każdy będzie walczył o jak najwięcej takiej przyjemności! 




 
\noindent GIORGIO: Nie mówię tego, chociaż wiem, że wielu moich przyjaciół by tak powiedziało. Według mnie potrzebą organiczną i~przyjemnością jest ruch, aktywność nerwowa i~mięśniowa; ale praca jest zdyscyplinowaną działalnością skierowaną na
obiektywny cel, zewnętrzny w~stosunku do organizmu. I~dobrze rozumiem, jak to się dzieje, że ktoś woli jazdę konną,
kiedy zamiast tego trzeba sadzić kapustę. Wierzę jednak, że istoty ludzkie, kiedy mają jakiś cel na widoku, mogą się
przystosować i~dostosowują się do warunków niezbędnych do jego osiągnięcia. 

 
Ponieważ produkty, które uzyskuje się dzięki pracy, są niezbędne do przeżycia i~nikt nie będzie miał środków, by zmusić
innych do pracy za nie, wszyscy uznają konieczność pracy i~będą faworyzować taką strukturę, w~której praca będzie mniej
męcząca i~bardziej produktywna, a to jest, moim zdaniem, organizacja komunistyczna. 

 
Weź również pod uwagę, że w~komunizmie ci sami pracownicy zorganizują i~pokierują pracą, a zatem mają interes w~uczynieniu jej lekką i~przyjemną; rozważ to, że w~komunizmie w~naturalny sposób rozwinie się opinia publiczna, która
potępi bezczynność jako szkodliwą dla wszystkich, a jeśli znajdą się jacyś próżniacy, będą oni tylko znikomą
mniejszością, którą można tolerować bez zauważalnej szkody. 




 
\noindent AMBROGIO: Ale przypuśćmy, że wbrew twoim optymistycznym prognozom próżniaków będzie bardzo dużo, co byś zrobił?
Poparłbyś ich? Jeśli tak, to równie dobrze możecie wspierać tych, których nazywacie burżuazją! 




 
\noindent GIORGIO: Naprawdę jest wielka różnica; ponieważ burżuazja nie tylko bierze udział w~tym, co produkujemy, ale
uniemożliwia nam produkowanie tego, czego chcemy i~jak chcemy to produkować. Niemniej jednak wcale nie mówię, że
powinniśmy utrzymywać próżniaków, gdy jest ich tak dużo, że powodują szkody: bardzo się boję, aby bezczynność i~przyzwyczajenie do życia z~innych nie prowadziły do chęci dowodzenia. Komunizm to wolna umowa: kto jej nie akceptuje
lub nie przestrzega, pozostaje poza nią. 




 
\noindent AMBROGIO: Ale wtedy będzie nowa klasa nieuprzywilejowana? 




 
\noindent GIORGIO: Wcale nie. Każda osoba ma prawo do ziemi, do narzędzi produkcji i~wszelkich korzyści, z~jakich człowiek może
korzystać w~stanie cywilizacyjnym, jaki osiągnął. Jeśli ktoś nie chce zaakceptować komunistycznego życia i~obowiązku,
to ich sprawa. Oni i~podobnie myślący dojdą do porozumienia, a jeśli znajdą się w~gorszym stanie niż inni, udowodni im
to wyższość komunizmu i~zmusi ich do zjednoczenia się z~komunistami. 




 
\noindent AMBROGIO: A więc będzie można nie akceptować komunizmu? 




 
\noindent GIORGIO: Oczywiście: I~ktokolwiek to będzie, będzie miał takie same prawa jak komuniści do bogactwa naturalnego i~nagromadzonych produktów poprzednich pokoleń. Na miłość boską!! Zawsze mówiłem o wolnym porozumieniu, o wolnym
komunizmie. Jak może istnieć wolność bez możliwej alternatywy? 




 
\noindent AMBROGIO: Czyli nie chcesz na siłę narzucać swoich pomysłów? 




 
\noindent GIORGIO: O! Oszalałeś? Bierzecie nas za policjantów czy sędziów? 




 
\noindent AMBROGIO:Cóż, w~takim razie nie ma nic złego. Każdy może realizować swoje marzenia! 




 
\noindent GIORGIO: Uważaj, żeby nie popełnić błędu: narzucanie idei to jedno, a obrona przed złodziejami i~przemocą, a odzyskanie
swoich praw to co innego. 




 
\noindent AMBROGIO: Ach! Ach! Więc aby \begin{itshape}odzyskać swoje prawa \end{itshape} użyłbyś siły,
prawda? 




 
\noindent GIORGIO Na to ci nie odpowiem: może ci się to przydać do sporządzenia aktu oskarżenia w~jakimś procesie. Powiem wam, że
z pewnością, kiedy ludzie staną się świadomi swoich praw i~będą chcieli położyć kres\ldots, istnieje ryzyko, że zostaniesz
potraktowany dość szorstko. Ale będzie to zależeć od Twojego oporu. Jeśli zrezygnujesz w~dobrej woli, wszystko będzie
spokojne i~przyjazne; jeśli wręcz przeciwnie, jesteś uparty, a jestem pewien, że taki będziesz, tym gorzej dla ciebie.
Dobrego wieczoru. 










\chapter*{Osiem}



 
\noindent AMBROGIO: Wiesz! Im więcej myślę o waszym wolnym komunizmie, tym bardziej utwierdzam się w~przekonaniu, że jesteś\ldots
prawdziwym oryginałem. 




 
\noindent GIORGIO: A to dlaczego? 




 
\noindent AMBROGIO: Bo zawsze mówisz o pracy, przyjemnościach, porozumieniach, umowach, ale nigdy nie mówisz o władzy społecznej,
o rządzie. Kto będzie regulował życie społeczne? Jaki będzie rząd? Jak zostanie ukonstytuowany? Kto go wybierze? w~jaki
sposób zapewni poszanowanie prawa i~karanie przestępców? Jak będą ukonstytuowane różne władze, ustawodawcza, wykonawcza
czy sądownicza? 




 
\noindent GIORGIO: Nie wiemy, co zrobić z~tymi wszystkimi władzami. Nie chcemy rządu. Czy nadal nie rozumiesz, że jestem
anarchistą? 




 
\noindent AMBROGIO: Cóż, mówiłem ci, że jesteś oryginałem. Jeszcze mogłem zrozumieć komunizm i~przyznać, że mógłby on przynieść
wielkie korzyści, gdyby wszystko jeszcze regulował światły rząd, który miałby siłę, by wszyscy szanowali prawo. Ale
tak, bez rządu, bez prawa! Jaki byłby bałagan? 




 
\noindent GIORGIO: Przewidziałem to: najpierw byłeś przeciwko komunizmowi, ponieważ powiedziałeś, że potrzebuje silnego i~scentralizowanego rządu; teraz, kiedy usłyszałeś o społeczeństwie bez rządu, zgodziłbyś się nawet na komunizm, o ile
istniałby rząd z~żelazną ręką. Krótko mówiąc, najbardziej przeraża cię wolność! 




 
\noindent AMBROGIO: Ale to jak wpaść z~deszczu pod rynnę! Pewne jest, że społeczeństwo bez rządu nie może istnieć. Jak można
oczekiwać, że wszystko będzie działać bez zasad, bez jakichkolwiek przepisów? Będzie tak, że ktoś skręci w~prawo, ktoś
inny w~lewo, a statek pozostanie nieruchomy lub, co bardziej prawdopodobne, opadnie na dno. 




 
\noindent GIORGIO: Nie powiedziałem, że nie chcę zasad i~reguł. Powiedziałem ci, że nie chcę
\begin{itshape}Rządu \end{itshape}, a przez rząd rozumiem władzę, która stanowi prawa i~narzuca je wszystkim. 




 
\noindent AMBROGIO: Ale jeśli ten rząd jest wybierany przez ludzi, czyż nie jest to reprezentacja woli tych samych ludzi? Na co
mógłbyś narzekać? 




 
\noindent GIORGIO: To po prostu kłamstwo. Ogólna, abstrakcyjna, popularna wola jest niczym więcej niż metafizyczną fantazją.
Społeczeństwo składa się z~ludzi, a ludzie mają tysiące różnych i~zróżnicowanych woli, w~zależności od różnic
temperamentu i~okoliczności, i~oczekiwanie, że uda im się wydobyć z~nich, poprzez magiczne działanie urny wyborczej,
ogólną wolę wspólną dla wszystkich, to po prostu absurdalność. Nawet jedna osoba nie byłaby w~stanie powierzyć komuś
innemu wykonania swojej woli we wszystkich kwestiach, które mogą się pojawić w~danym okresie; bo oni sami nie potrafią
z góry powiedzieć, jaka będzie ich wola przy tych różnych okazjach. Jak można mówić w~imieniu zbiorowości, ludzi,
których członkowie już w~chwili wydawania mandatu byli już między sobą w~niezgodzie? 

 
Pomyśl tylko przez chwilę, jak przebiegają wybory, i~zauważ, że zamierzam mówić o tym, jak by przebiegały, gdyby wszyscy
ludzie byli wykształceni i~niezależni, a więc głosowanie w~pełni świadome i~wolne. Ty, na przykład, zagłosowałbyś na
tego, kogo uważasz za najlepiej nadającego się do służenia twoim interesom i~wdrożenia twoich pomysłów. To już dużo
ustępuje, bo masz tyle pomysłów i~tyle różnych zainteresowań, że nie wiedziałbyś, jak znaleźć osobę, która myśli zawsze
tak jak ty we wszystkich sprawach: ale czy wtedy takiej osobie dasz twój głos i~kto będzie tobą rządził? w~żaden
sposób. Twój kandydat może nie odnieść sukcesu, więc twoja wola nie stanowi części tak zwanej woli powszechnej: ale
załóżmy, że odniesie sukces. 

 
Na tej podstawie, czy ta osoba byłaby twoim władcą? Nawet w~snach. Byłaby tylko jednym z~wielu (na przykład we włoskim
parlamencie jednym z~535), a w~rzeczywistości będziesz rządzony przez większość, którym nigdy nie daliście mandatu. I~ta większość (której członkowie otrzymali wiele różnych lub sprzecznych ze sobą mandatów, a jeszcze lepiej otrzymali
tylko ogólne delegowanie uprawnień, bez żadnego konkretnego mandatu) nie jest w~stanie, nawet gdyby chciała, ustalić
nieistniejącej woli ogólnej i~uszczęśliwić wszystkich, tylko zrobi to, czego chce, lub będzie postępować zgodnie z~życzeniami tych, którzy w~danej chwili nad nimi panują. 

 
Daj spokój, lepiej odłożyć na bok to staroświeckie udawanie rządu, który reprezentuje wolę ludu. 

 
Z pewnością istnieją pewne kwestie ogólnego porządku, co do których w~danym momencie wszyscy ludzie się zgodzą. Ale w~takim razie, jaki jest sens rządzenia? Kiedy wszyscy czegoś chcą, wystarczy, że to wcielą w~życie. 




 
\noindent AMBROGIO: Krótko mówiąc, przyznałeś, że potrzebne są zasady, jakieś normy życia. Kto powinien je stworzyć? 




 
\noindent GIORGIO: Sami zainteresowani, ci, którzy muszą przestrzegać tych przepisów. 




 
\noindent AMBROGIO: Kto by narzucał przestrzeganie? 




 
\noindent GIORGIO: Nikt, bo mówimy o normach, które są swobodnie akceptowane i~swobodnie przestrzegane. Nie mylcie norm, o których
mówię, a które są praktycznymi konwencjami opartymi na poczuciu solidarności i~trosce wszystkich o interes zbiorowy, z~prawem, które jest regułą napisaną przez nielicznych i~narzuconą siłą wszystkim wszyscy. Nie chcemy praw, ale wolnych
umów. 




 
\noindent AMBROGIO: A jeśli ktoś złamie umowę? 




 
\noindent GIORGIO: A dlaczego ktoś miałby naruszać umowę, na którą się zgodził? z~drugiej strony, gdyby doszło do naruszeń, byłoby
to ostrzeżeniem, że umowa nie zadowoliła wszystkich i~musi zostać zmodyfikowana. I~każdy będzie szukał lepszego układu,
bo w~interesie wszystkich leży, aby nikt nie był nieszczęśliwy. 




 
\noindent AMBROGIO: Ale wydaje się, że tęsknisz za prymitywnym społeczeństwem, w~którym każdy jest samowystarczalny, a relacje
między ludźmi są nieliczne, podstawowe i~ograniczone. 




 
\noindent GIORGIO: Wcale nie. Ponieważ od chwili, gdy stosunki społeczne się zwielokrotniają i~stają się bardziej złożone,
ludzkość doświadcza większej satysfakcji moralnej i~materialnej, będziemy dążyć do jak najliczniejszych i~bardziej
złożonych relacji. 




 
\noindent AMBROGIO: Ale wtedy będziesz musiał delegować funkcje, zlecać zadania, nominować przedstawicieli w~celu zawierania
umów. 




 
\noindent GIORGIO: Oczywiście. Jednak nie myśl, że jest to równoznaczne z~powołaniem rządu. Rząd tworzy prawa i~je egzekwuje,
podczas gdy w~wolnym społeczeństwie istnieje tylko delegacja władzy do określonych, tymczasowych zadań, do określonych
prac i~nie daje żadnych uprawnień ani żadnej specjalnej nagrody. A uchwały delegatów zawsze podlegają zatwierdzeniu
przez tych, których reprezentują. 




 
\noindent AMBROGIO: Ale nie wyobrażasz sobie, że wszyscy zawsze się zgodzą. Jeśli istnieją ludzie, którym nie odpowiada twój
porządek społeczny, co zrobisz? 




 
\noindent GIORGIO: Ci ludzie zrobią wszystko, co im najbardziej odpowiada, a my i~oni dojdziemy do porozumienia, żeby nie
przeszkadzać sobie nawzajem. 




 
\noindent AMBROGIO: A jeśli inni będą robić kłopoty? 




 
\noindent GIORGIO: W~takim razie\ldots będziemy się bronić. 




 
\noindent AMBROGIO: Ach! Ale czy nie widzisz, że z~tej potrzeby obrony może powstać nowy rząd? 




 
\noindent GIORGIO: Oczywiście, że to widzę: I~właśnie z~tego powodu zawsze mówiłem, że anarchizm nie jest możliwy, dopóki nie
zostaną wyeliminowane najpoważniejsze przyczyny konfliktów, porozumienie społeczne nie będzie służyło interesom
wszystkich, a duch solidarności nie zostanie dobrze rozwinięty wśród ludzkości. 

 
Jeśli chcesz stworzyć anarchizm dzisiaj, pozostawiając nienaruszoną własność indywidualną i~inne instytucje społeczne,
które z~niej się wywodzą, natychmiast wybuchłaby taka wojna domowa, że rząd, nawet tyrania, byłby mile widziany jako
błogosławieństwo. 

 
Ale jeśli w~tym samym czasie, gdy ustanowisz anarchizm, zniesiesz własność indywidualną, przyczyny konfliktów, które
przetrwają, nie będą nie do pokonania i~osiągniemy porozumienie, ponieważ dzięki porozumieniu wszyscy będą
uprzywilejowani. 

 
W końcu jest zrozumiałe, że instytucje są tyle warte, ile ludzie, którzy sprawiają, że funkcjonują, a anarchizm w~szczególności, czyli panowanie wolnej umowy, nie może istnieć, jeśli ludzie nie rozumieją korzyści płynących z~solidarności i~nie chcą się ugodzić. 

 
Dlatego angażujemy się w~szerzenie propagandy. 










\chapter*{Dziewięć}



 
\noindent AMBROGIO: Pozwólcie, że wrócę do Twojego anarchistycznego komunizmu. Szczerze mówiąc, nie mogę go znieść\ldots 




 
\noindent GIORGIO: Ach! Wierzę ci. Po tym, jak przeżyłeś swoje życie między kodeksami i~księgami prawa, broniąc praw państwa i~praw właścicieli, społeczeństwo bez państwa i~właścicieli, w~którym nie będzie już buntowników i~głodujących ludzi do
wysłania na galery, musi ci się wydawać czymś z~innego świata. 

 
Ale jeśli chcesz odłożyć na bok tę postawę, jeśli masz siłę przezwyciężyć swoje nawyki myślowe i~zechcesz zastanowić się
nad tą sprawą bez uprzedzeń, z~łatwością zrozumiesz, że pozwalając, aby celem społeczeństwa było jak największe dobro
dla wszystkich, koniecznym rozwiązaniem jest anarchistyczny komunizm. Jeśli myślisz przeciwnie, że społeczeństwo jest
stworzone dla interesów kilku osób kochających przyjemności kosztem reszty, cóż\ldots 




 
\noindent AMBROGIO: Nie, nie, przyznaję, że społeczeństwo musi mieć za cel dobro wszystkich, ale z~tego powodu nie mogę
zaakceptować waszego systemu. Usilnie próbuję poznać twój punkt widzenia, a ponieważ zainteresowałem się dyskusją,
chciałbym przynajmniej dla siebie mieć jasne pojęcie, czego chcesz: ale twoje wnioski wydają się tak utopijne, więc\ldots 




 
\noindent GIORGIO: Krótko mówiąc, co jest dla ciebie niejasne lub nie do przyjęcia w~wyjaśnieniu, które ci dałem. 




 
\noindent AMBROGIO: Jest \ldots nie wiem\ldots cały system. 

 
Zostawmy na boku kwestię prawa, co do której nie zgodzimy się; ale załóżmy, że, jak twierdzisz, wszyscy mamy równe prawo
do korzystania z~istniejącego bogactwa, przyznaję, że komunizm wydawałby się rozwiązaniem najszybszym i~być może
najlepszym. Ale, to, co wydaje mi się absolutnie niemożliwe, to społeczeństwo bez rządu. 

 
Cały swój gmach budujecie na wolnej woli członków stowarzyszeń\ldots 




 
\noindent GIORGIO: Dokładnie. 




 
\noindent AMBROGIO: I~to jest twój błąd. Społeczeństwo oznacza hierarchię, dyscyplinę, podporządkowanie jednostki kolektywowi. Bez
władzy nie jest możliwe żadne społeczeństwo. 




 
\noindent GIORGIO:  Dokładnie odwrotnie. Społeczeństwo w~ścisłym tego słowa znaczeniu może istnieć tylko wśród równych; a ci równi
sobie zawierają porozumienia między sobą, jeśli znajdują w~nich przyjemność i~wygodę, ale nie chcą się sobie
podporządkowywać. 

 
Te stosunki hierarchii i~uległości, które wydają ci się istotą społeczeństwa, są stosunkami między niewolnikami a
panami: I~mam nadzieję, że zgodzisz się, że niewolnik nie jest w~rzeczywistości partnerem pana, tak jak zwierzę domowe
nie jest partner osoby, która go posiada. 




 
\noindent AMBROGIO: Ale czy ty naprawdę wierzysz w~społeczeństwo, w~którym każdy robi, co chce! 




 
\noindent GIORGIO: Pod warunkiem, że rozumie się, że ludzie chcą żyć w~społeczeństwie i~dlatego dostosują się do potrzeb życia
społecznego. 




 
\noindent AMBROGIO: A jeśli nie chcą? 




 
\noindent GIORGIO: Wtedy społeczeństwo nie byłoby możliwe. Ale skoro tylko w~społeczeństwie ludzkość, przynajmniej w~swojej
nowoczesnej formie, może zaspokoić swoje potrzeby materialne i~moralne, dziwne jest przypuszczenie, że chcielibyśmy
wyrzec się tego, co jest warunkiem wstępnym życia i~dobrobytu. 

 
Ludzie mają trudności z~osiągnięciem porozumienia, gdy omawiają sprawy w~sposób abstrakcyjny; ale skoro tylko jest coś
do zrobienia, to musi być zrobione i~co leży w~interesie wszystkich, dopóki nikt nie dysponuje środkami, by narzucić
swoją wolę innym i~zmusić ich do robienia rzeczy po swojemu, upór i~zaciekłość wkrótce ustają, ludzie stają się zgodni
i sprawa jest załatwiona z~maksymalną możliwą satysfakcją dla wszystkich. 

 
Musisz zrozumieć: nic ludzkiego nie jest możliwe bez woli ludzkości. Cały problem dla nas polega na zmianie tej woli, to
znaczy na uświadomieniu ludziom, że aby walczyć ze sobą, nienawidzić się, wykorzystywać, to stracić wszystko, oraz na
przekonaniu innych, by życzyli sobie ładu społecznego opartego na wzajemnym wsparciu i~solidarności. 




 
\noindent AMBROGIO: Tak więc, aby wprowadzić swój anarchistyczny komunizm, musisz poczekać, aż wszyscy będą do tego przekonani i~będą chcieli, aby to zadziałało. 




 
\noindent GIORGIO: O nie! Żartowalibyśmy! Wola zależy głównie od środowiska społecznego i~jest prawdopodobne, że dopóki obecne
warunki będą trwać, zdecydowana większość będzie nadal wierzyć, że społeczeństwo nie może być zorganizowane w~inny
sposób niż ten, który istnieje obecnie. 




 
\noindent AMBROGIO: A więc?! 




 
\noindent GIORGIO: Stworzymy więc między sobą komunizm i~anarchizm\ldots kiedy będzie nas wystarczająco dużo, by to zrobić, przekonani,
że jeśli inni zobaczą, że dobrze sobie radzimy, wkrótce pójdą w~nasze ślady, lub przynajmniej, nie mogąc osiągnąć
komunizmu i~anarchizmu, będziemy pracować nad zmianą warunków społecznych w~taki sposób, aby wywołać zmianę woli w~pożądanym kierunku. 

 
Musisz zrozumieć; chodzi o wzajemną interakcję między wolą a otaczającymi nas warunkami społecznymi\ldots Robimy i~będziemy
robić wszystko, co w~naszej mocy, aby dążyć do naszego ideału. 

 
To, co musisz jasno zrozumieć, jest następujące. Nie chcemy nikogo zmuszać; ale nie chcemy, aby inni narzucali swoją wolę
ani wolę społeczeństwa. Buntujemy się przeciwko tej mniejszości, która przemocą wyzyskuje i~uciska lud. Kiedy wolność
zostanie zdobyta dla nas i~dla wszystkich, i, co oczywiste, środki do bycia wolnymi, innymi słowy, prawo do użytkowania
ziemi i~środki produkcji, będziemy polegać wyłącznie na sile słów i~przykładów, aby nasze idee zatriumfowały. 




 
\noindent AMBROGIO: W~porządku; i~myślisz, że w~ten sposób dojdziemy do społeczeństwa, które rządzi się po prostu dzięki
dobrowolnej zgodzie swoich członków? Jeśli tak jest, to byłoby sprawą bez
\begin{itshape}precedensu \end{itshape}! 




 
\noindent GIORGIO: Nie tak bardzo, jak myślisz. Właściwie zawsze tak było\ldots to znaczy, jeśli uznać pokonanych, zdominowanych,
uciskanych pochodzących z~niższych warstw ludzkości, za tak naprawdę nienależących do społeczeństwa. 

 
W sumie, nawet dziś zasadnicza część życia społecznego, zarówno w~klasie dominującej, jak i~w~klasie zdominowanej,
dokonuje się poprzez spontaniczne, często nieświadome porozumienia między jednostkami: na mocy zwyczaju, punktów
honoru, poszanowania obietnic, obawy przed opinią publiczną, poczucie uczciwości, miłości, sympatii, zasad dobrego
wychowania, bez żadnej interwencji prawa i~rządu. Prawo i~rządy stają się konieczne tylko wtedy, gdy mamy do czynienia
z relacjami między dominującymi a zdominowanymi. Wśród równych każdy wstydzi się wezwać policjanta lub zwrócić się do
sędziego! 

 
W państwach despotycznych, gdzie wszyscy mieszkańcy są traktowani jak stado w~służbie jedynego władcy, nikt nie ma
władzy oprócz suwerena\ldots i~tych, których suweren potrzebuje, aby utrzymać masy w~uległości. Ale stopniowo, w~miarę jak
przybywają inni, osiągają emancypację i~wchodzą do klasy panującej, czyli społeczeństwa w~ścisłym tego słowa znaczeniu,
czy to przez bezpośredni udział w~rządzeniu, czy też przez posiadanie bogactwa, społeczeństwo kształtuje się w~sposób,
który zaspokaja wolę wszystkich dominatorów. Cały aparat ustawodawczy i~wykonawczy, cały rząd ze swoimi prawami,
żołnierzami, policjantami, sędziami itp. służy jedynie regulowaniu i~zapewnianiu wyzysku ludu. W~przeciwnym razie
właściciele uznaliby za prostsze i~bardziej ekonomiczne uzgodnienia między sobą i~znieśliby państwo. Sami burżujowie
wyrazili taką opinię\ldots kiedy na chwilę zapomną, że bez żołnierzy i~policjantów, ludzie zepsuliby imprezę. 

 
Zniszcz podziały klasowe, upewnij się, że nie ma już niewolników do trzymania w~ryzach, a państwo natychmiast nie będzie
miało powodu do istnienia. 




 
\noindent AMBROGIO: Ale nie przesadzaj. Państwo również czyni dobro dla wszystkich. Edukuje, czuwa nad zdrowiem publicznym, broni
życia obywateli, organizuje usługi publiczne\ldots nie mów mi, że to rzeczy bezwartościowe lub szkodliwe! 




 
\noindent GIORGIO: Fuj! Zrobione tak, jak zwykle robi to państwo, czyli prawie wcale. Prawda jest taka, że to robotnicy naprawdę
robią te rzeczy, a państwo, ustanawiając się ich regulatorem, przekształca takie usługi w~instrumenty dominacji,
obracając je w~szczególną korzyść dla władców i~właścicieli. 

 
Edukacja szerzy się, jeśli w~społeczeństwie jest pragnienie nauczania i~jeśli są nauczyciele zdolni do wychowania;
zdrowie publiczne rozwija się, gdy społeczeństwo wie, docenia i~może je zastosować w~praktyce przepisy dotyczące
zdrowia publicznego i~kiedy są lekarze, którzy mogą udzielać porad; życie obywateli jest bezpieczne, kiedy ludzie są
przyzwyczajeni do uważania życia i~wolności ludzkich za święte i~kiedy\ldots nie ma sędziów ani policji, którzy dawaliby
przykłady brutalności; usługi publiczne są organizowane wtedy, gdy społeczeństwo poczuje taką potrzebę. 

 
Państwo niczego nie tworzy: W~najlepszym przypadku jest to tylko inny nadmiar, bezwartościowe marnowanie energii. Ale
gdyby tylko to było bezużyteczne! 




 
\noindent AMBROGIO: Zostaw to tutaj. W~każdym razie myślę, że powiedziałeś wystarczająco dużo. Chcę się nad tym zastanowić. 

 
Dopóki znów się nie spotkamy\ldots 










\chapter*{Dziesięć}



 
\noindent AMBROGIO: Zastanowiłem się nad tym, co mi mówiłeś podczas tych naszych rozmów\ldots i~poddaję się. Nie dlatego, że
przyznaję się do porażki; ale\ldots jednym słowem masz swoje argumenty i~przyszłość może być po twojej stronie. 

 
Tymczasem jestem sędzią i~dopóki istnieje prawo, muszę je szanować i~czuwać nad jego przestrzeganiem. Rozumiesz\ldots 




 
\noindent GIORGIO: Och, rozumiem bardzo dobrze. Idź, idź, jeśli chcesz. Do nas będzie należało zniesienie prawa, a tym samym
uwolnienie was od obowiązku działania wbrew sumieniu. 




 
\noindent AMBROGIO: Spokojnie, spokojnie, tego nie powiedziałem\ldots ale nieważne. 

 
Prosiłbym o kilka innych wyjaśnień. 

 
Być może moglibyśmy dojść do porozumienia w~kwestiach dotyczących ustroju własnościowego i~politycznej organizacji
społeczeństwa; w~końcu są to formacje historyczne, które wielokrotnie się zmieniały i~prawdopodobnie jeszcze się
zmienią. Ale są pewne święte instytucje, pewne głębokie uczucia ludzkiego serca, które nieustannie obrażacie: rodzina,
ojczyzna! 

 
Na przykład, chcesz złożyć to wszystko razem. Naturalnie, nawet kobiety połączycie razem i~stworzycie w~ten sposób
wielki seraj; czyż nie tak? 




 
\noindent GIORGIO: Słuchaj; jeśli chcesz ze mną porozmawiać, proszę, nie mów głupich rzeczy i~nie żartuj w~złym guście. Sprawa, z~którą mamy do czynienia, jest zbyt poważna, by wtrącać wulgarne żarty! 




 
\noindent AMBROGIO: Ale\ldots  mówiłem poważnie. Co byś zrobił z~kobietami? 




 
\noindent GIORGIO: Tym gorzej dla ciebie, bo to naprawdę dziwne, że nie rozumiesz absurdalności tego, co właśnie powiedziałeś. 

 
Złącz kobiety! Dlaczego nie powiesz, że chcemy łączyć mężczyzn? Jedynym wyjaśnieniem tego twojego pomysłu jest to, że
przez zakorzeniony przyzwyczajenie uważasz kobietę za gorszą istotę stworzoną i~umieszczoną na tym świecie, by służyła
jako zwierzę domowe i jako narzędzie przyjemności dla płci męskiej, więc mówisz o niej tak, jakby była rzeczą, i~wyobrażasz sobie, że musimy jej przypisać to samo przeznaczenie, jakie przypisujemy rzeczom. 

 
Ale my, którzy uważamy kobietę za istotę ludzką równą sobie, która powinna cieszyć się wszystkimi prawami i~wszystkimi
zasobami, którymi cieszy się lub powinna cieszyć się płeć męska, uważamy pytanie: ,,Co zrobisz z~kobietami?{\textquotedbl} za pozbawione znaczenia. Zamiast tego zapytaj: ,,Co same kobiety zrobią?'' i~na to odpowiem
\begin{itshape}zrobią co zechcą zrobić \end{itshape}, a ponieważ mają taką samą jak mężczyźni
potrzebę życia w~społeczeństwie, jest to pewne\textit{Będą chciały} zgodzić się z~innymi stworzeniami,
mężczyznami i~kobietami, w~celu zaspokojenia ich potrzeb z~jak najlepszą korzyścią dla nich samych i~dla wszystkich
innych. 




 
\noindent AMBROGIO: Rozumiem; uważasz kobiety za równe mężczyznom. Jednak wielu naukowców, badając budowę anatomiczną i~fizjologiczne funkcje kobiecego ciała, utrzymuje, że kobieta jest z~natury gorsza od mężczyzny. 




 
\noindent GIORGIO: Tak, oczywiście. Cokolwiek jest potrzebne do utrzymania, zawsze znajdzie się naukowiec, który będzie chciał to
utrzymać. Są naukowcy, którzy utrzymują niższość kobiet, ale są też tacy, którzy wręcz przeciwnie, utrzymują, że
rozumienie kobiet i~ich możliwości rozwoju są takie same jak mężczyzn, i~jeśli obecnie kobiety ogólnie wydają się mieć
mniejsze zdolności niż mężczyźni, to wynika z~otrzymanego wykształcenia i~środowiska, w~którym żyją. Jeśli dobrze
poszukasz, znajdziesz nawet niektórych naukowców, a przynajmniej kobiety-naukowców, którzy twierdzą, że mężczyzna jest
istotą gorszą, przeznaczoną do wyzwolenia kobiet z~materialnego trudu i~pozostawienia im swobody rozwijania swoich
talentów w~nieograniczony sposób. Sądzę, że pogląd ten został przyjęty w~Ameryce. 

 
Ale kogo to obchodzi. Tu nie chodzi o rozwiązanie problemu naukowego, ale o realizację przysięgi, ideału człowieka. 

 
Daj kobietom wszelkie środki i~swobodę rozwoju, a co ma nadejść, to nadejdzie. Jeśli kobiety są równe mężczyznom lub są
mniej lub bardziej inteligentne, okaże się to w~praktyce, a nawet nauka będzie miała przewagę, ponieważ będzie
dysponować pewnymi pozytywnymi danymi, na których może oprzeć swoje wnioski. 




 
\noindent AMBROGIO: A więc nie bierzecie pod uwagę zdolności, jakimi obdarzone są jednostki? 




 
\noindent GIORGIO: Nie w~tym sensie, że te zdolności powinny tworzyć specjalne prawa. W~naturze nie znajdziesz dwóch równych
osobników; domagamy się jednak równości społecznej dla wszystkich, innymi słowy tych samych zasobów, tych samych
możliwości, i~uważamy, że ta równość nie tylko odpowiada poczuciu sprawiedliwości i~braterstwa, które rozwinęło się w~ludzkości, ale działa na korzyść wszystkich, niezależnie od tego, czy są silni lub słabi. 

 
Nawet wśród mężczyzn, wśród mężczyzn, są tacy, którzy są bardziej, a inni mniej inteligentni, ale to nie znaczy, że
jeden powinien mieć więcej praw niż drugi. Są tacy, którzy utrzymują, że blondynki są bardziej utalentowane niż
brunetki i~vice versa, że rasy o podłużnych czaszkach są lepsze od tych o szerokich czaszkach i~vice versa; a
zagadnienie, jeśli jest oparte na faktach, jest z~pewnością interesujące dla nauki. Jednak biorąc pod uwagę obecny stan
uczuć i~ludzkie ideały, absurdem byłoby udawanie, że blondynki i~dolichocefalicy powinni dominować nad brunatnymi i~rozgałęzionymi lub odwrotnie. 

 
Nie sądzisz? 




 
\noindent AMBROGIO: W~porządku; ale spójrzmy na kwestię rodziny. Chcesz ją zlikwidować czy zorganizować na innej podstawie? 




 
\noindent GIORGIO: Spójrz. Jeśli chodzi o rodzinę, musimy wziąć pod uwagę stosunki ekonomiczne, stosunki seksualne oraz relacje
między rodzicami a dziećmi. 

 
Na tyle na ile rodzina jest instytucją ekonomiczną, jasne jest, że po zniesieniu własności indywidualnej, a w~konsekwencji dziedziczenia, nie ma ona już racji bytu i~\begin{itshape}de facto \end{itshape}
zaniknie. W~tym sensie rodzina dawno jest już zniesiona dla ogromnej większości ludności, która składa się z~pracowników. 




 
\noindent AMBROGIO: A jeśli chodzi o stosunki płciowe? Czy chcesz wolnej miłości, \ldots 




 
\noindent GIORGIO: Och, daj spokój! Myślisz, że zniewolona miłość naprawdę może istnieć? Wymuszone współżycie istnieje, podobnie
jak udawana i~wymuszona miłość, ze względu na interes lub wygodę społeczną; prawdopodobnie znajdą się mężczyźni i~kobiety, którzy będą szanować węzeł małżeński ze względu na przekonania religijne lub moralne; ale prawdziwa miłość nie
może istnieć, nie może być poczęta, jeśli nie jest doskonale wolna. 




 
\noindent AMBROGIO: To prawda, ale jeśli wszyscy podążą za fantazjami natchnionymi przez boga miłości, nie będzie już moralności,
a świat stanie się burdelem. 




 
\noindent GIORGIO: Jeśli chodzi o moralność, to naprawdę możesz się chwalić wynikami swoich instytucji! Cudzołóstwo, wszelkiego
rodzaju kłamstwa, długo żywiona nienawiść, mężowie zabijający żony, żony trujące mężów, dzieciobójstwo, dzieci
dorastające pośród skandali i~rodzinnych bójek\ldots i~to jest moralność, o której obawiasz, że jest zagrożona przez wolną
miłość? 

 
Dziś świat jest burdelem, ponieważ kobiety są często zmuszane do prostytucji z~powodu głodu; i~dlatego, że małżeństwo,
często zawierane z~czystego rachunku zysków, jest przez cały czas trwania związkiem, do którego miłość albo wcale nie
wchodzi, albo wchodzi tylko jako dodatek. 

 
Zapewnić wszystkim środki do właściwego i~niezależnego życia, dać kobietom pełną swobodę rozporządzania własnym ciałem,
zniszczyć uprzedzenia religijne i~inne, które wiążą mężczyzn i~kobiety z~mnóstwem konwencji wywodzących się z~niewolnictwa i~które je utrwalają, i~związki seksualne będą tworzone z~miłości i~dadzą początek szczęściu jednostek i~dobru gatunku. 




 
\noindent AMBROGIO: Ale w~skrócie, czy jesteś za związkami trwałymi czy tymczasowymi? Czy chcesz oddzielnych par, czy wielości i~różnorodności stosunków seksualnych, a nawet rozwiązłości? 




 
\noindent GIORGIO: Chcemy wolności. 

 
Do tej pory stosunki seksualne bardzo cierpiały pod presją brutalnej przemocy, konieczności ekonomicznej, przesądów
religijnych i~przepisów prawnych, że nie udało się ustalić, jaka forma stosunków seksualnych najlepiej odpowiada
fizycznemu i~moralnemu dobru jednostek i~gatunku. 

 
Z pewnością, gdy usuniemy warunki, które dziś czynią stosunki między kobietami i~mężczyznami sztucznymi i~wymuszonymi,
zostanie ustanowiona higiena seksualna i~moralność seksualna, które będą szanowane nie ze względu na prawo, ale poprzez
przekonanie oparte na doświadczeniu, że zaspokajają one dobro nasze i~gatunku. Może to nastąpić tylko jako efekt
wolności. 




 
\noindent AMBROGIO: A dzieci? 




 
\noindent GIORGIO: Musisz zrozumieć, że kiedy będziemy mieli wspólny majątek i~ustalimy solidarność społeczną na solidnych
podstawach moralnych i~materialnych, utrzymanie dzieci będzie przedmiotem troski wspólnoty, a ich wychowanie troską i~odpowiedzialnością wszystkich. 

 
Prawdopodobnie wszyscy mężczyźni i~wszystkie kobiety pokochają wszystkie dzieci; a jeśli, jak sądzę, jest pewne, rodzice
darzą własne dzieci szczególnym uczuciem, mogą być tylko zachwyceni wiedząc, że przyszłość ich dzieci jest bezpieczna,
gdy dla ich utrzymania i~wychowania współpracuje całe społeczeństwo. 




 
\noindent AMBROGIO: Ale szanujesz przynajmniej prawa rodziców do dzieci? 




 
\noindent GIORGIO: Prawa nad dziećmi składają się z~obowiązków. Człowiek ma wiele praw, to znaczy wiele praw, aby im przewodzić i~troszczyć się o nie, kochać je i~martwić się o nie: a ponieważ rodzice na ogół kochają swoje dzieci bardziej niż
ktokolwiek inny, zwykle ich obowiązkiem i~prawem jest zaspokojenie ich potrzeb. Nie trzeba obawiać się jakichkolwiek
wyzwań, ponieważ jeśli kilku nienaturalnych rodziców da swoim dzieciom niewielką miłość i~nie będzie się nimi
opiekować, będą zadowoleni, że inni zajmą się dziećmi i~uwolnią je od tego zadania. 

 
Jeśli przez prawa rodzica nad dziećmi rozumiesz prawo do ich maltretowania, korumpowania i~wykorzystywania, to ja
kategorycznie odrzucam te prawa i~myślę, że żadne społeczeństwo godne tego miana by ich nie uznało i~nie zniosło. 




 
\noindent AMBROGIO: Ale czy nie sądzisz, że powierzając społeczności odpowiedzialność za utrzymanie dzieci, sprowokujesz taki
wzrost liczby ludności, że nie starczy rzeczy dla wszystkich. Ale oczywiście nie będziecie chcieli słuchać żadnych
rozmów o maltuzjanizmie i~powiecie, że to absurd. 




 
\noindent GIORGIO: Powiedziałem ci przy innej okazji, że absurdalne jest udawanie, że obecna bieda zależy od przeludnienia, i~absurdalne jest pragnienie proponowania środków zaradczych opartych na praktykach maltuzjańskich. Jestem bardzo skłonny
uznać powagę kwestii ludnościowej i~przyznaję, że w~przyszłości, gdy każde nowo narodzone dziecko będzie miało
zapewnione wsparcie, bieda może odrodzić się z~powodu rzeczywistego nadmiaru ludności. Wyemancypowani i~wykształceni
ludzie, kiedy uznają to za konieczne, rozważą stworzenie ograniczenia do zbyt szybkiego namnażania się gatunku; ale
dodam, że zaczną się nad tym poważnie zastanawiać dopiero wtedy, gdy zniknie gromadzenie i~przywileje, przeszkody,
jakie chciwość właścicieli stawiała produkcji, czy wszelkie społeczne przyczyny ubóstwa, wtedy dopiero konieczność
osiągnięcia równowagi między liczbą istot żywych, zdolności produkcyjnych i~dostępnej przestrzeni wydadzą się każdemu
jasne i~proste. 




 
\noindent AMBROGIO: A jeśli ludzie nie będą chcieli o tym myśleć? 




 
\noindent GIORGIO: Cóż w~takim razie tym gorzej dla nich! 

 
Nie chcecie zrozumieć: nie ma opatrzności, czy to boskiej, czy naturalnej, która troszczyłaby się o dobro ludzkości.
Ludzie muszą sami zadbać o swoje dobro, robiąc to, co uważają za przydatne i~konieczne do osiągnięcia tego celu. 

 
Zawsze mówisz: ale co, jeśli nie chcą? w~takim przypadku nic nie osiągną i~zawsze pozostaną na łasce ślepych sił, które
ich otaczają. 

 
Tak jest dzisiaj: ludzie nie wiedzą, co zrobić, aby się uwolnić, a jeśli wiedzą, nie chcą robić tego, co trzeba, aby się
wyzwolić. I~w~ten sposób pozostają niewolnikami. 

 
Jednak mamy nadzieję, że szybciej, niż myślisz, będą wiedzieć, co robić i~będą w~stanie to zrobić. 

 
Wtedy będą wolni. 










\chapter*{Jedenaście}



 
\noindent AMBROGIO: Któregoś dnia doszedłeś do wniosku, że wszystko zależy od woli. Mówiłeś, że jeśli ludzie chcą być wolni, jeśli
chcą robić to, co trzeba, aby żyć w~społeczeństwie równych sobie, wszystko będzie dobrze, a jeśli nie, to tym gorzej
dla nich. Byłoby w~porządku, gdyby wszyscy chcieli tego samego; ale jeśli jedni chcą żyć w~anarchii, a inni wolą
kuratelę rządu, jeśli jedni są gotowi uwzględniać potrzeby społeczności, a inni chcą korzystać z~dobrodziejstw życia
społecznego, ale nie chcą się dostosowywać do związane z~tym potrzeby i~chcą robić to, co im się podoba, nie biorąc pod
uwagę szkód, jakie może to wyrządzić innym, co się stanie, jeśli nie będzie rządu, który określa i~nakłada obowiązki
społeczne? 




 
\noindent GIORGIO: Jeśli istnieje rząd, zatriumfuje wola rządzących i~ich partii oraz związanych z~nimi interesów, a problem, jak
zaspokoić wolę wszystkich, nie zostanie rozwiązany. Wręcz przeciwnie, trudności się pogłębią. Frakcja rządząca może nie
tylko wykorzystywać własne zasoby do ignorowania lub naruszania woli innych, ale ma do dyspozycji siłę całego
społeczeństwa do narzucenia jej woli. Tak jest w~przypadku naszego obecnego społeczeństwa, w~którym klasa robotnicza
dostarcza rządowi żołnierzy i~bogactwo, aby utrzymać robotników jako niewolników. 

 
Myślę, że już wam powiedziałem: chcemy społeczeństwa, w~którym każdy ma środki do życia tak, jak chce, w~którym nikt nie
może zmusić innych do pracy dla niego, w~którym nikt nie może zmusić drugiego do poddania się ich woli. Kiedy dwie
zasady zostaną wprowadzone w~życie, wolność dla wszystkich i~narzędzia produkcji dla wszystkich, wszystko inne nastąpi
naturalnie, siłą okoliczności, a nowe społeczeństwo zorganizuje się w~sposób, który najlepiej odpowiada interesom
wszystkich. 




 
\noindent AMBROGIO: A jeśli niektórzy chcą się narzucić brutalną siłą? 




 
\noindent GIORGIO: Wtedy będą rządem; lub kandydatami na rząd, a my przeciwstawimy się im siłą. Musisz zrozumieć, że jeśli dziś
chcemy zrobić rewolucję przeciwko rządowi, to nie po to, by podporządkować się nowym ciemiężcom. Jeśli tacy wygrają,
rewolucja zostanie pokonana i~trzeba będzie ją powtórzyć. 




 
\noindent AMBROGIO: Ale z~pewnością pozwoliłbyś na trochę etycznych zasad, wyższych od woli i~kaprysów ludzkości, do których każdy
jest zobowiązany się dostosować\ldots przynajmniej moralnie? 




 
\noindent GIORGIO: Cóż to za moralność, która przewyższa wolę ludzką? Kto to przepisał? Skąd się bierze? 

 
Moralność zmienia się w~zależności od czasów, krajów, klas, okoliczności. Wyrażają to, co ludzie w~danych momentach i~okolicznościach uważają za najlepsze zachowanie. Krótko mówiąc, dla każdej osoby dobre obyczaje są zgodne z~tym, co
lubi lub co sprawia jej przyjemność, z~powodów materialnych lub emocjonalnych. 

 
Dla was moralność nakazuje poszanowanie prawa, to znaczy poddanie się przywilejom, którymi cieszy się wasza klasa; dla
nas wymaga buntu przeciwko uciskowi i~poszukiwania dobra dla wszystkich. Dla nas wszystkie przepisy moralne są pojęte
przez miłość między ludźmi. 




 
\noindent AMBROGIO: A przestępcy? Czy uszanujesz ich wolność? 




 
\noindent GIORGIO: Uważamy, że działanie przestępcze oznacza naruszenie wolności innych. Kiedy przestępcy są liczni, potężni i~zorganizowali swoją dominację na stabilnych podstawach, tak jak to ma miejsce dzisiaj z~właścicielami i~władcami, aby
się wyzwolić, potrzebna jest rewolucja. 

 
Kiedy wręcz przeciwnie, przestępczość sprowadza się do pojedynczych przypadków niewłaściwego zachowania lub choroby,
podejmiemy próbę znalezienia przyczyn i~wprowadzenia odpowiednich środków zaradczych. 




 
\noindent AMBROGIO: W~międzyczasie? Będziesz potrzebował policji, magistratu, kodeksu karnego, kilku strażników więziennych
itp. 




 
\noindent GIORGIO: A zatem, powiedzielibyście, rekonstytucja rządu, powrót do stanu ucisku, w~którym żyjemy dzisiaj. 

 
W rzeczywistości największa szkoda wyrządzona przez przestępstwo to nie tyle jednorazowy i~przejściowy przypadek
naruszenia praw kilku jednostek, ile niebezpieczeństwo, że stanie się ono okazją i~pretekstem do ukonstytuowania się
władzy, która, pozornie broniąc społeczeństwa, ujarzmi je i~pognębi. 

 
Znamy już cel policji i~magistratu oraz to, że są raczej przyczyną niż lekarstwem na niezliczone zbrodnie. 

 
Musimy zatem spróbować zniszczyć przestępczość poprzez wyeliminowanie przyczyn; a kiedy pozostaną pozostali przestępcy,
kolektyw bezpośrednio zainteresowany powinien pomyśleć o umieszczeniu ich w~sytuacji, w~której nie będą mogli wyrządzić
krzywdy, bez powierzania komukolwiek szczególnej funkcji ścigania przestępców. 

 
Znasz historię o koniu, który poprosił człowieka o ochronę i~pozwolił mu wsiąść na grzbiet? 




 
\noindent AMBROGIO: W~porządku. W~tym momencie szukam tylko informacji, a nie dyskusji. 

 
Inna sprawa. Widząc, że w~waszym społeczeństwie wszyscy są równi społecznie, wszyscy mają taki sam dostęp do edukacji i~rozwoju, wszyscy mają pełną swobodę wyboru własnego życia, jak zamierzacie wykonać konieczne prace. Są prace przyjemne
i pracochłonne, zdrowe i~niezdrowe. Naturalnie każda osoba wybierze lepszą pracę, kto zajmie się innymi, często
najbardziej potrzebnymi? 

 
A do tego jest wielki podział na pracę umysłową i~fizyczną. Czy nie sądzisz, że każdy chciałby być lekarzem,
\begin{itshape}literatem \end{itshape}, poetą, i~żeby nikomu nie chciało się uprawiać roli,
robić butów itd. itd.  No ? 




 
\noindent GIORGIO: Chcesz patrzeć w~przyszłość społeczeństwa, społeczeństwa równości, wolności, a przede wszystkim solidarności i~wolnego porozumienia, zakładającego kontynuację dzisiejszych warunków moralnych i~materialnych. Oczywiście coś takiego
się pojawia i~jest niemożliwe. 

 
Kiedy każdy będzie miał środki, każdy osiągnie maksymalny rozwój materialny i~intelektualny, na jaki pozwolą jego
naturalne zdolności: każdy zostanie wtajemniczony w~intelektualne radości i~produktywną pracę; ciało i~mózg będą się
harmonijnie rozwijać; na różnych poziomach, w~zależności od zdolności i~skłonności, wszyscy będą naukowcami i~\begin{itshape}literatami \end{itshape} biegłymi w~literaturze i~wszyscy będą robotnikami. 

 
Co by się wtedy stało? 

 
Wyobraź sobie, że kilka tysięcy lekarzy, inżynierów, pisarzy i~artystów ma być przetransportowanych na rozległą i~żyzną
wyspę, zaopatrzonych w~narzędzia pracy i~pozostawionych samym sobie. 

 
Myślisz, że pozwolą sobie umrzeć z~głodu, zamiast pracować własnymi rękami, czy raczej popełnić samobójstwo, niż dojść
do porozumienia i~podzielić pracę według swoich upodobań i~możliwości? Gdyby istniały prace, których nikt nie chciał
wykonywać, wszyscy wykonywaliby je po kolei i~każdy szukałby sposobów, aby niezdrowe i~nieprzyjemne prace uczynić
bezpiecznymi i~przyjemnymi. 




 
\noindent AMBROGIO: Dosyć, dość, muszę ci zadać jeszcze tysiąc pytań, ale ty błąkasz się po totalnej utopii i~znajdujesz
wyimaginowane sposoby rozwiązania na wszystkie problemy. 

 
Wolałbym, abyś porozmawiał ze mną o sposobach i~środkach, za pomocą których proponujesz zrealizować swoje marzenia. 




 
\noindent GIORGIO: Z~przyjemnością, o tyle, że jeśli chodzi o mnie, to chociaż ideał jest użyteczny i~konieczny jako sposób
wskazania ostatecznego celu, najpilniejszym pytaniem jest to, co trzeba zrobić dzisiaj iw najbliższej przyszłości. 

 
Porozmawiamy o tym następnym razem. 










\chapter*{Dwanaście}



 
\noindent AMBROGIO: Więc dziś wieczorem porozmawiasz z~nami o środkach, za pomocą których zamierzasz osiągnąć swoje ideały\ldots
stworzyć anarchizm. 

 
Już sobie to wyobrażam. Będą bomby, masakry, doraźne egzekucje; a potem grabieże, podpalenia i~podobne subtelności. 




 
\noindent GIORGIO: Po prostu trafiłeś, mój drogi panie, na niewłaściwą osobę, musiałeś pomyśleć, że rozmawiasz z~jakimś
urzędnikiem, który dowodzi żołnierzami europejskimi, kiedy jadą
\begin{itshape}cywilizować \end{itshape} Afrykę czy Azję, albo kiedy
\textit{cywilizują } samych siebie w~kraju. 

 
To nie w~moim stylu, proszę mi wierzyć. 




 
\noindent CESARE: Myślę, drogi panie, że nasz przyjaciel, który w~końcu pokazał, że jest rozsądnym młodym człowiekiem, choć zbyt
wielkim marzycielem, oczekuje triumfu idei poprzez naturalną ewolucję społeczeństwa, rozprzestrzenianie się edukacji,
postęp nauki, rozwój produkcji. 

 
I w~końcu nie ma w~tym nic złego. Jeśli anarchizm ma nadejść, to nadejdzie i~nie ma sensu męczyć się, by uniknąć
nieuniknionego. 

 
Ale w~takim razie\ldots to jest tak daleko! Żyjmy w~pokoju. 




 
\noindent GIORGIO: Rzeczywiście, czy to nie byłby dobry powód, żebyś sobie pofolgować! 

 
Ale nie, Signor Cesare, nie opieram się na ewolucji, nauce i~całej reszcie. Trzeba by było czekać zbyt długo! I, co
gorsza, każdy czekałby na próżno! 

 
Ewolucja człowieka zmierza w~kierunku, w~którym jest napędzana wolą ludzkości, i~nie ma żadnego prawa naturalnego, które
mówiłoby, że ewolucja musi nieuchronnie dać pierwszeństwo wolności, a nie permanentnemu podziałowi społeczeństwa na
dwie kasty, mógłbym prawie powiedzieć, że na dwie rasy, dominujących i~zdominowanych. 

 
Każdy stan społeczeństwa, gdy znalazł wystarczające powody do istnienia, może również trwać w~nieskończoność, dopóki
dominujący nie napotkają świadomego, aktywnego, agresywnego sprzeciwu ze strony zdominowanych. Czynniki dezintegracji i~spontanicznej śmierci, które istnieją w~każdym reżimie, nawet jeśli istnieją kompensacyjne czynniki odbudowy i~witalności działające jako antidotum, zawsze mogą zostać zneutralizowane dzięki umiejętnościom tego, kto rozporządza
siłą społeczeństwa i~kieruje nią tak, jak sobie tego życzy. 

 
Mógłbym ci zademonstrować, gdybym nie bał się zająć zbyt wiele czasu, jak burżuazja chroni się przed tymi
\begin{itshape}naturalnymi \end{itshape} tendencjami, od których niektórzy socjaliści
spodziewali się rychłej śmierci. 

 
Nauka to potężna broń, której można używać zarówno w~dobrych, jak i~złych celach. A ponieważ w~obecnych warunkach
nierówności jest bardziej dostępna dla uprzywilejowanych niż uciskanych, jest bardziej użyteczna dla tych pierwszych
niż dla drugich. 

 
Edukacja, przynajmniej ta wykraczająca poza powierzchowny pryzmat, jest prawie bezużyteczna i~niedostępna dla mas
upośledzonych, a nawet wtedy może być kierowana w~sposób wybrany przez wychowawców, a raczej przez tych, którzy płacą i~wybierają wychowawców. 




 
\noindent AMBROGIO: Ale wtedy zostaje tylko przemoc! 




 
\noindent GIORGIO: Mianowicie rewolucja. 




 
\noindent AMBROGIO: Gwałtowna rewolucja? Zbrojna rewolucja? 




 
\noindent GIORGIO: Dokładnie. 




 
\noindent AMBROGIO: Dlatego bomby\ldots 




 
\noindent GIORGIO: Nieważne, signor Ambrogio. Jest pan sędzią pokoju, ale nie lubię powtarzać, że to nie jest trybunał i~przynajmniej na razie nie jestem oskarżonym, z~którego ust byłoby w~pana interesie wyciągnąć jakąś nierozważną
uwagę. 

 
Rewolucja będzie gwałtowna, ponieważ wy, klasa dominująca, utrzymujecie się przemocą i~nie wykazujecie żadnej skłonności
do pokojowego poddania. Więc będą strzały, bomby, fale radiowe, które eksplodują wasze zapasy materiałów wybuchowych i~naboje w~skrzyniach waszych żołnierzy z~daleka\ldots to wszystko może się zdarzyć. Są to pytania techniczne, które, jeśli
chcesz, zostawimy technikom. 

 
Mogę was zapewnić, że o ile to od nas zależy, przemoc, która została na nas narzucona przez waszą przemoc, nie wyjdzie
poza wąskie granice wskazane przez konieczność walki, to znaczy, że będzie ona przede wszystkim zdeterminowana przez
stawiany przez was opór. Jeśli stanie się najgorsze, będzie to spowodowane waszym uporem i~krwiożerczą edukacją, którą
na przykład udostępniasz publicznie. 




 
\noindent CESARE: Ale jak dokonacie tej rewolucji, skoro jest was tak mało? 




 
\noindent GIORGIO: Możliwe, że jest nas tylko ograniczona liczba. Masz taką nadzieję, a ja nie chcę odbierać ci tej słodkiej
iluzji. Oznacza to, że będziemy zmuszeni podwoić, a następnie ponownie podwoić nasze liczby\ldots 

 
Z pewnością naszym zadaniem, gdy nie ma możliwości zrobienia więcej, polega na wykorzystaniu propagandy do zgromadzenia
mniejszości świadomych jednostek, które będą wiedziały, co mają robić i~zobowiążą się to robić. Naszym zadaniem jest
przygotowanie mas, lub jak największej ich części, do działania we właściwym kierunku, gdy nadarzy się okazja. A przez
właściwy kierunek rozumiemy: wywłaszczenie obecnych posiadaczy bogactwa społecznego, obalenie rządu, zapobieżenie
powstaniu nowych przywilejów i~nowych form rządów oraz bezpośrednia reorganizacja poprzez działalność robotniczą,
produkcji, dystrybucji i~całego życia społecznego. 




 
\noindent CESARE: A jeśli okazja się nie nadarzy? 




 
\noindent GIORGIO: Cóż, będziemy szukać sposobów, aby tak się stało. 




 
\noindent PROSPERO: Ile ty masz złudzeń, mój chłopcze!!! 

 
Myślisz, że żyjemy jeszcze w~czasach broni z~epoki kamienia. 

 
Dzięki nowoczesnej broni i~taktyce zostałbyś zmasakrowany, zanim mógłbyś się ruszyć. 




 
\noindent GIORGIO: Niekoniecznie. Nowemu uzbrojeniu i~taktyce można przeciwstawić odpowiednie reakcje. 

 
I znowu, ta broń jest w~rzeczywistości w~rękach synów ludu, a wy, zmuszając wszystkich do podjęcia służby wojskowej,
uczycie wszystkich, jak się nią posługiwać. 

 
Och! Nie możecie sobie wyobrazić, jak naprawdę będziecie bezradni w~dniu, w~którym zbuntuje się wystarczająca liczba. 

 
To my, proletariat, klasa uciskana, jesteśmy elektrykami i~gazownikami, to my kierujemy lokomotywami, to my wytwarzamy
materiały wybuchowe i~budujemy kopalnie, to my kierujemy samochodami i~samolotami, to my jesteśmy żołnierzami\ldots to
niestety my bronimy was przed samymi sobą. Przeżyjecie tylko dzięki nieświadomej zgodzie twoich ofiar. Uważaj, aby nie
obudzić ich świadomości\ldots 

 
No i~wiesz, wśród anarchistów każdy rządzi swoimi działaniami, a twoja policja tak jest przyzwyczajona do szukania
wszędzie, z~wyjątkiem miejsc, w~których występuje prawdziwe niebezpieczeństwo. 

 
Ale nie mam zamiaru uczyć cię techniki powstańczej. To jest sprawa, która\ldots was nie dotyczy. 

 
Dobrego wieczoru. 










\chapter*{Trzynaście}



 
\noindent VINCENZO [\textit{Młody republikanin}]: Pozwolisz, że włączę się do twojej rozmowy, abym mógł zadać kilka pytań i~poczynić kilka obserwacji?\ldots Nasz przyjaciel Giorgio mówi o anarchizmie, ale mówi, że anarchizm musi przyjść
swobodnie, bez narzucania się, z~woli ludu. I~on mówi też, że aby dać wolne ujście woli ludu, trzeba zburzyć przez
powstanie monarchiczny i~militarystyczny reżim, który dziś tę wolę dusi i~fałszuje. Tego chcą republikanie,
przynajmniej rewolucyjni republikanie, innymi słowy ci, którzy naprawdę chcą stworzyć republikę. Dlaczego więc nie
ogłaszasz się republikaninem? 

 
W republice lud jest suwerenny, a jeśli ktoś robi to, czego chce lud, a oni chcą anarchizmu, to będzie anarchizm. 




 
\noindent GIORGIO: Naprawdę wierzę, że zawsze mówiłem o \begin{itshape}wola ludzkości \end{itshape} a
nie\textit{wola ludu}, a jeśli powiedziałem ostatnie, to była to forma słów, niedokładne użycie języka, któremu
cała moja rozmowa służy przecież poprawie. 




 
\noindent VINCENZO: Ale po co to całe zainteresowanie słowami?!! Czy społeczeństwo nie składa się z~ludzi? 




 
\noindent GIORGIO: To nie jest kwestia słów. Jest to kwestia istoty: jest to cała różnica między
\begin{itshape}demokracją \end{itshape}, co oznacza rząd ludu, i\textit{anarchią}, co
nie oznacza rządu, ale wolność dla każdego. 

 
Lud z~pewnością składa się z~ludzkości, to znaczy ze świadomej jedności, współzależnej według własnego uznania, ale
każda osoba ma swoją własną wrażliwość i~własne zainteresowania, pasje, szczególne pragnienia, które w~zależności od
sytuacji zwiększają lub wzajemnie się znoszą, wzmacniają lub neutralizują. Najsilniejsza, najlepiej uzbrojona wola
jednostki, partii, klasy zdolnej do panowania, narzuca się i~odnosi sukces, podając się za wolę wszystkich; w~rzeczywistości to, co nazywa siebie \begin{itshape}wolą ludu \end{itshape} jest wolą tych,
którzy dominują, lub jest to hybrydowy produkt obliczeń numerycznych, który nie do końca odpowiada woli kogokolwiek i~który nikogo nie satysfakcjonuje. 

 
Już własnymi wypowiedziami demokraci, czyli republikanie (bo tylko oni są prawdziwymi demokratami) przyznają, że tzw.
rząd ludu jest jedynie rządem większości, który wyraża i~wdraża własną wolę poprzez swoich przedstawicieli. Dlatego
,,suwerenność'' mniejszości jest po prostu prawem nominalnym, które nie przekłada się na działanie; i~zauważmy, że ta
,,mniejszość'' oprócz tego, że jest często najbardziej zaawansowaną i~postępową częścią populacji, może być również
większością liczbową, gdy mniejszość zjednoczona przez wspólnotę interesów lub idei, lub przez poddanie się przywódcy,
znajdzie się w~obliczu wielu niezgodnych frakcji. 

 
Ale czy partia, której kandydaci odnoszą sukces, a zatem rządzi w~imieniu większości, jest naprawdę rządem, który wyraża
wolę większości? Funkcjonowanie systemu parlamentarnego (niezbędnego w~każdej republice, która nie jest małą i~odizolowaną, niezależną gminą) sprawia, że każdy przedstawiciel jest jednostką organu wyborczego, jednym z~wielu i~liczy się tylko jako setna lub tysięczna powstająca ustaw, które w~ostatecznym rozrachunku winny być wyrazem woli
większości elektorów. 

 
A teraz zostawmy na boku kwestię, czy reżim republikański potrafi zrealizować wolę wszystkich i~powiedz mi przynajmniej,
czego wy chcecie, co chcielibyście, żeby ta republika robiła, jakie instytucje społeczne powinna powołać. 




 
\noindent VINCENZO: Ale to oczywiste. 

 
To, czego chcę, czego chcą wszyscy prawdziwi republikanie, to sprawiedliwość społeczna, emancypacja robotników, równość,
wolność i~braterstwo. 




 
\noindent GŁOS: Jak we Francji, w~Szwajcarii i~w~Ameryce. 




 
\noindent VINCENZO: To nie są prawdziwe republiki. Powinieneś kierować swoją krytykę do prawdziwej republiki, której szukamy, a
nie różnych rządów, burżuazyjnych, wojskowych i~duchownych, które w~różnych częściach świata noszą miano republiki. W~przeciwnym razie, przeciwstawiając się socjalizmowi i~anarchizmowi, mógłbym przytoczyć tak zwanych anarchistów, którzy
są czymś zupełnie innym. 




 
\noindent GIORGIO: Dobrze powiedziane. Ale dlaczego, u licha, istniejące republiki nie okazały się prawdziwymi republikami?
Dlaczego tak naprawdę wszyscy lub prawie wszyscy, wychodząc od ideałów równości, wolności i~braterstwa, które są
waszymi ideałami, a powiedziałbym, że także naszymi, stały się systemami przywilejów, które się umacniają, w~których
robotnicy są wyzyskiwani w~skrajnościach, kapitaliści są bardzo potężni, ludzie bardzo uciskani, a rząd tak całkowicie
nieuczciwy jak w jakimkolwiek ustroju monarchicznym? 

 
Instytucje polityczne, organy regulujące społeczeństwo, prawa indywidualne i~zbiorowe uznane w~konstytucji są takie
same, jak będą w~waszej republice. 

 
Dlaczego konsekwencje były tak złe, a przynajmniej tak negatywne i~dlaczego miałyby być inne, gdyby to była wasza
republika. 




 
\noindent VINCENZO: Ponieważ\ldots ponieważ\ldots 




 
\noindent GIORGIO: Powiem ci dlaczego, a mianowicie, że w~tych republikach warunki ekonomiczne ludności pozostały zasadniczo takie
same; podział społeczeństwa na klasę posiadającą i~klasę proletariacką pozostał niezmieniony, a więc prawdziwa władza
pozostała w~rękach tych, którzy posiadając monopol na środki produkcji, dzierżyli w~swej władzy wielkie masy
nieuprzywilejowanych. Naturalnie klasa uprzywilejowana zrobiła wszystko, co w~jej mocy, aby umocnić swoją pozycję,
która zostałaby zachwiana przez rewolucyjny zapał, z~którego zrodziła się republika, i~wkrótce wszystko wróciło do
tego, co było wcześniej \ldots z~wyjątkiem, być może, w~odniesieniu do tych różnic, tych postępy, które nie zależą od
formy rządów, lecz od wzrostu świadomości robotników, od wzrostu wiary we własne siły, których masy nabywają za każdym
razem, gdy uda im się obalić rząd. 




 
\noindent VINCENZO: Ale całkowicie zdajemy sobie sprawę ze znaczenia kwestii ekonomicznych. Wprowadzimy podatek progresywny, który
sprawi, że bogaci będą ponosić większą część wydatków publicznych, zniesiemy cła ochronne, nałożymy podatek na
nieuprawiane ziemie, ustalimy płacę minimalną, pułap cen, ustanowimy prawa, które będą chronić pracowników\ldots 




 
\noindent GIORGIO: Nawet jeśli uda ci się to wszystko zrobić, kapitaliści jeszcze raz znajdą sposób, aby uczynić to bezużytecznym
lub obrócić na swoją korzyść. 




 
\noindent VINCENZO: W~takim razie oczywiście wywłaszczymy ich być może bez odszkodowania i~stworzymy komunizm. 

 
Czy jesteś zadowolony? 




 
\noindent GIORGIO: Nie, nie\ldots komunizm stworzony wolą rządu, a nie bezpośrednią i~dobrowolną pracą grup robotniczych, jakoś
specjalnie do mnie nie przemawia. Gdyby to było możliwe, byłaby to najbardziej dusząca tyrania, jakiej społeczeństwo
ludzkie kiedykolwiek zostało poddane. 

 
Ale wy mówicie: zrobimy to czy tamto, jakby dlatego z~faktu bycia republikanami w~przededniu republiki, kiedy republika
zostanie proklamowana, będziecie rządzić. 

 
Ponieważ republika jest systemem tego, co nazywacie suwerennością ludu, a ta suwerenność wyraża się poprzez powszechne
prawo wyborcze, rząd republikański będzie się składał z~mężczyzn wyznaczonych w~głosowaniu powszechnym. 

 
A ponieważ nie złamałeś w~akcie rewolucji republikańskiej władzy kapitalistów przez wywłaszczenie ich w~sposób
rewolucyjny, pierwszy parlament republikański będzie odpowiedni dla kapitalistów\ldots a jeśli nie pierwszy, który może
nadal odczuwać skutki do rozmiarów rewolucyjnej burzy, z~pewnością kolejne parlamenty będą tym, czego pragną
kapitaliści, i~będą zmuszone zniszczyć wszystko, co dobre rewolucja przypadkowo mogła zrobić. 




 
\noindent VINCENZO: Ale w~takim razie, skoro anarchizm nie jest dziś możliwy, czy musimy spokojnie wspierać monarchię nie wiadomo
jak długo? 




 
\noindent GIORGIO: W~żadnym wypadku. Możecie liczyć na naszą współpracę, tak jak my będziemy prosić o waszą, byle tylko
okoliczności sprzyjały ruchowi powstańczemu. Oczywiście zakres wkładów, które będziemy starali się wnieść do tego
ruchu, będzie znacznie szerszy niż wasz, ale to nie unieważnia wspólnego interesu, jaki mamy w~zrzuceniu jarzma, które
dziś ciąży na nas obu. Później zobaczymy. 

 
W międzyczasie szerzmy razem propagandę i~spróbujmy przygotować masy, aby następny ruch rewolucyjny zapoczątkował
najgłębszą możliwą przemianę społeczną i~pozostawił szeroko i~łatwo otwartą drogę do dalszego postępu. 










\chapter*{Czternaście}



 
\noindent CESARE: Wznówmy naszą zwykłą rozmowę. 

 
Najwyraźniej tym, co najbardziej cię interesuje, jest powstanie; i~przyznaję, że jakkolwiek wydaje się to trudne,
prędzej czy później można je zorganizować i~wygrać. Zasadniczo rządy polegają na żołnierzach; a poborowi niechętni
żołnierze, którzy są wpychani do koszar wojskowych, są zawodną bronią. W~obliczu powszechnego powstania ludu żołnierze,
którzy sami należą do tzw ludu, nie wytrzymają długo; a gdy tylko urok i~strach przed dyscypliną zostaną przełamane,
albo się rozwiążą, albo dołączą do ludu. 

 
Przyznaję więc, że szerząc dużo propagandy wśród robotników i~żołnierzy, czy też wśród młodzieży, która jutro będzie
żołnierzami, stawiacie się w~sytuacji, w~której możecie wykorzystać sprzyjającą sytuację, takie jak kryzysy
gospodarcze, nieudaną wojnę, strajk generalny, głód itp. itd., do obalenia rządu. 

 
Ale wtedy? 

 
Powiedz mi: ludzie sami zdecydują, zorganizują się itd.  Ale to są słowa. Prawdopodobnie nastąpi to, że po krótszym lub
dłuższym okresie nieporządku, rozproszenia i~prawdopodobnie masakr, nowy rząd zajmie miejsce drugiego, przywróci
porządek\ldots i~wszystko potoczy się po staremu. 

 
W jakim więc celu było takie marnowanie energii? 




 
\noindent GIORGIO: Gdyby tak się stało, jak sugerujesz, to nie znaczy, że powstanie byłoby bezużyteczne. Po rewolucji rzeczy nie
wracają do tego, co było wcześniej, ponieważ ludzie cieszyli się okresem wolności i~wypróbowali własne siły, i~nie jest
łatwo zmusić ich do ponownego zaakceptowania poprzednich warunków. Nowy rząd, jeśli rząd musi istnieć, będzie czuł, że
nie może bezpiecznie utrzymać się u władzy, jeśli nie da mu to jakiejś satysfakcji, i~zwykle stara się usprawiedliwić
swoje dojście do władzy, nadając sobie tytuł interpretatora i~następcy rewolucji. 

 
Rzecz jasna, prawdziwym zadaniem, jakie postawi przed sobą rząd, będzie niedopuszczenie do dalszego rozwoju rewolucji
oraz ograniczenie i~zmiana zdobyczy rewolucji w~celu dominacji; ale nie będzie mógł przywrócić rzeczy do stanu
poprzedniego. 

 
Tak było we wszystkich poprzednich rewolucjach. 

 
Mamy jednak powody, by mieć nadzieję, że w~następnej rewolucji wypadniemy dużo lepiej. 




 
\noindent CESARE: Dlaczego? 




 
\noindent GIORGIO: Ponieważ w~poprzednich rewolucjach wszyscy rewolucjoniści, wszyscy inicjatorzy i~główni aktorzy rewolucji
chcieli zmienić społeczeństwo za pomocą praw i~chcieli rządu, który by te prawa ustanawiał i~narzucał. Było to zatem
nieuniknione, że to stworzy nowy rząd, i~było rzeczą naturalną, że nowy rząd myślał przede wszystkim o tym
\begin{itshape}rządzeniu \end{itshape}, to znaczy skonsolidowaniu swojej władzę i~w~tym celu
stworzeniu wokół siebie partii i~klasy uprzywilejowanej, której wspólnym interesem było trwałe utrzymanie się u
władzy. 

 
Ale teraz w~historii pojawił się nowy czynnik, który reprezentują anarchiści. Teraz są rewolucjoniści, którzy chcą
zrobić rewolucję z~wyraźnie antyrządowymi celami, dlatego ustanowienie nowego rządu napotkałoby przeszkodę, jakiej
nigdy wcześniej nie znaleziono. 

 
Co więcej, dawni rewolucjoniści, chcąc dokonać upragnionej transformacji społecznej za pomocą praw, zwracali się do mas
wyłącznie z~prośbą o podstawową współpracę, jaką mogli zapewnić, i~nie zadali sobie trudu, aby uświadomić im, czego
można sobie życzyć i~jaką drogę mogliby spełnić swoje aspiracje. Naturalnie więc lud, podatny na samozagładę, sam
prosił o rząd, gdy zachodziła potrzeba reorganizacji codziennego życia społecznego. 

 
Z drugiej strony naszą propagandą i~organizacjami robotniczymi dążymy do ukształtowania świadomej mniejszości, która
wie, co chce robić i~która wmieszana w~masy mogłaby zaspokoić najpilniejsze potrzeby i~podjąć te inicjatywy, które z~innej strony od rządu oczekiwano okazji. 




 
\noindent CESARE: Bardzo dobrze; ale ponieważ będziecie tylko mniejszością i~prawdopodobnie w~wielu częściach kraju nie będziecie
mieli żadnego wpływu, tak samo powstanie rząd i~będziecie musieli to znosić. 




 
\noindent GIORGIO: Jest więcej niż prawdopodobne, że rząd ustanowi sam siebie; ale czy będziemy musieli to znosić\ldots to się
okaże. 

 
Zanotuj to dobrze. W~poprzednich rewolucjach głównym celem było utworzenie nowego rządu i~oczekiwanie na rozkazy tego
rządu.  A tymczasem sytuacja zasadniczo się nie zmieniała, a raczej sytuacja ekonomiczna mas pogorszała się z~powodu
przerwy w~przemyśle i~handlu. Dlatego ludzie szybko męczyli się tym wszystkim;  śpieszono się, aby to skończyć i~przy
wrogości opinii publicznej wobec tych, którzy zbyt długo chcieli przedłużać stan powstańczy. I~tak każdy, kto wykazał
się zdolnością do przywracania porządku, czy to był najemnik, czy też sprytny i~odważny polityk, a może jakiś wypędzony
władca, był witany powszechnym aplauzem jako rozjemca i~wyzwoliciel.  

 
Wręcz przeciwnie, rozumiemy rewolucję zupełnie inaczej. Chcemy, aby przemiana społeczna, do której zmierza rewolucja,
zaczęła się urzeczywistniać od pierwszego aktu powstańczego.  Chcemy, aby ludzie natychmiast przejęli w~posiadanie
istniejące bogactwo; ogłaszali rezydencje dżentelmenów własnością publiczną i~zapewniali poprzez dobrowolne i~aktywne
inicjatywy minimum mieszkania dla całej ludności, a także natychmiast wprowadzać w~życie dzięki pracy stowarzyszenia
budowniczych budowę tylu nowych domów, ile uzna się za konieczne. Chcemy, aby wszystkie dostępne produkty żywnościowe
stały się własnością społeczności i~abyśmy zorganizowali, zawsze poprzez dobrowolne działania i~pod prawdziwą kontrolą
społeczeństwa, równą dystrybucję produktów dla wszystkich. Chcemy, aby robotnicy rolni zawładnęli nieuprawianą ziemią i~ziemią właścicieli ziemskich i~w~ten sposób przekonali tych ostatnich, że teraz ziemia należy do robotników. Chcemy,
aby robotnicy odsunęli się od kierownictwa właścicieli i~kontynuowali produkcję na własny rachunek i~dla społeczeństwa.
Chcielibyśmy nawiązać od razu stosunki wymiany między różnymi stowarzyszeniami wytwórczymi i~różnymi gminami; a
jednocześnie chcemy spalić, zniszczyć wszystkie tytuły i~wszelkie materialne oznaki indywidualnej własności i~dominacji
państwowej. Krótko mówiąc, chcemy, aby od pierwszej chwili masy odczuły korzyści płynące z~rewolucji i~aby tak zakłócić
sytuację, aby przywrócenie dawnego porządku było niemożliwe. 




 
\noindent CESARE: I~myślisz, że to wszystko jest łatwe do przeprowadzenia?  




 
\noindent GIORGIO: Nie, doskonale zdaję sobie sprawę ze wszystkich trudności, z~jakimi będziemy musieli się zmierzyć; Jasno
przewiduję, że nasz program nie może być zastosowany wszędzie na raz, a zastosowany spowoduje tysiące nieporozumień i~tysiące błędów. Ale sam fakt, że są ludzie, którzy chcą ją zastosować i~będą próbowali ją zastosować wszędzie tam,
gdzie to możliwe, jest już gwarancją, że w~tym momencie rewolucja nie może już być prostą transformacją polityczną i~musi zapoczątkować głęboką zmianę w~całość życia społecznego. 

 
Co więcej, burżuazja zrobiła coś podobnego podczas wielkiej rewolucji francuskiej pod koniec XVIII wieku, choć w~mniejszym stopniu, a ancien régime nie mógł się odbudować pomimo Cesarstwa i~Restauracji. 




 
\noindent CESARE: Ale jeśli, pomimo wszystkich twoich dobrych lub złych intencji, powstanie rząd, wszystkie twoje projekty wylecą
w powietrze, i~musiałbyś podporządkować się prawu jak wszyscy. 




 
\noindent GIORGIO: A to dlaczego? 

 
To, że powstanie rząd lub rządy, jest z~pewnością bardzo prawdopodobne. Jest wielu ludzi, którzy lubią rozkazywać, a
jeszcze więcej jest skłonnych do posłuszeństwa! 

 
Ale bardzo trudno jest sobie wyobrazić, jak ten rząd mógłby się narzucić, dać się zaakceptować i~stać się regularnym
rządem, jeśli w~kraju jest wystarczająco dużo rewolucjonistów i~nauczyli się wystarczająco angażować masy w~zapobieganie znajdowaniu przez nowy rząd sposobu na umocnienie się i~stabilizację. 

 
Rząd potrzebuje żołnierzy i~zrobimy wszystko, co w~naszej mocy, aby odmówić im żołnierzy; rząd potrzebuje pieniędzy i~zrobimy wszystko, co w~naszej mocy, aby nikt nie płacił podatków i~nie udzielał mu kredytu. 

 
Istnieją gminy i~być może niektóre regiony we Włoszech, gdzie rewolucjoniści są dość liczni, a robotnicy całkiem gotowi
ogłosić autonomię i~zająć się własnymi sprawami, odmawiając uznania rządu i~przyjęcia jego agentów lub wysyłania do
niego przedstawicieli. 

 
Te regiony, te gminy staną się ośrodkami wpływów rewolucyjnych, wobec których żaden rząd będzie bezsilny, jeśli będziemy
działać szybko i~nie damy mu czasu na uzbrojenie się i~umocnienie. 




 
\noindent CESARE: Ale to jest wojna domowa! 




 
\noindent GIORGIO: Bardzo możliwe. Jesteśmy za pokojem, tęsknimy za pokojem\ldots ale nie poświęcimy rewolucji dla naszego pragnienia
pokoju. Nie poświęcimy go, ponieważ tylko tą drogą możemy osiągnąć prawdziwy i~trwały pokój. 










\chapter*{Piętnaście}



 
GINO [\textit{Pracownik}]: Słyszałem, że wieczorami omawiacie kwestie społeczne i~przyszedłem zadać, za
pozwoleniem tych panów, pytanie mojemu przyjacielowi Giorgio. 

 
Powiedzcie mi, czy to prawda, że wy anarchiści chcecie usunąć policję? 




 
\noindent GIORGIO: Oczywiście. Co! Nie zgadzasz się? Od kiedy jesteś przyjacielem policji i~\begin{itshape}karabinierów \end{itshape}? 




 
\noindent \noindent GINO: Nie jestem ich przyjacielem i~dobrze o tym wiesz. Ale nie jestem też przyjacielem morderców i~złodziei i~chciałbym, aby mój majątek i~moje życie były dobrze strzeżone i~strzeżone. 




 
\noindent GIORGIO: A kto cię strzeże przed strażnikami?\ldots 

 
Czy uważasz, że ludzie bez powodu stają się złodziejami i~mordercami? 

 
Czy uważacie, że najlepszym sposobem na zapewnienie sobie bezpieczeństwa jest nadstawianie karku bandzie ludzi, którzy
pod pretekstem naszej obrony gnębią nas i~dopuszczają się szantażu i~wyrządzają tysiąc razy większe szkody niż wszyscy
złodzieje i~wszyscy mordercy? Czy nie lepiej byłoby zniszczyć przyczyny zła, czyniąc to tak, aby każdy mógł dobrze żyć,
nie odbierając chleba z~ust innym, i~organizując to tak, aby każdy mógł się kształcić i~rozwijać, oraz móc wypędzić z~serc złe namiętności zazdrości, nienawiści i~zemsty? 




 
\noindent \noindent GINO: Daj spokój! Istoty ludzkie są z~natury złe i~gdyby nie było praw, sędziów, żołnierzy i~innych
\begin{itshape}karabinierów \end{itshape} trzymających nas w~ryzach, pożeralibyśmy się
nawzajem jak wilki. 




 
\noindent GIORGIO: Gdyby tak było, byłby to jeszcze jeden powód, by nie dawać nikomu władzy dowodzenia i~rozporządzania wolnością
innych. Zmuszeni do walki ze wszystkimi, z~każdą osobą o przeciętnej sile, narazilibyśmy się na to samo ryzyko i~moglibyśmy na przemian być zwycięzcami i~przegranymi: bylibyśmy dzikusami, ale przynajmniej moglibyśmy cieszyć się
względną swobodą dżungli i~zaciekłych emocje drapieżników. Ale gdybyśmy dobrowolnie dali nielicznym prawo i~władzę
narzucania ich woli, to skoro według ciebie sam fakt bycia człowiekiem predysponuje nas do wzajemnego pożerania się,
będzie to równoznaczne z~oddaniem się w~niewolę i~ubóstwo. 

 
Jednak oszukujesz sam siebie, mój drogi przyjacielu. Ludzkość jest dobra lub zła w~zależności od okoliczności. To, co
jest wspólne dla ludzi, to instynkt samozachowawczy i~dążenie do dobrego samopoczucia i~pełnego rozwoju własnych sił.
Jeśli aby dobrze żyć, trzeba surowo traktować innych, tylko nieliczni będą mieli siłę niezbędną do oparcia się pokusie.
Ale umieść ludzi w~społeczeństwie ich bliźnich, w~warunkach sprzyjających dobrobytowi i~rozwojowi, a stanie się złym
będzie wymagało wielkiego wysiłku, tak jak dzisiaj potrzeba wielkiego wysiłku, aby być dobrym. 




 
\noindent \noindent GINO: W~porządku, może być tak, jak mówisz. Tymczasem policja, czekając na przemiany społeczne, zapobiega popełnianiu
przestępstw. 




 
\noindent GIORGIO: Zapobiegać?! 




 
\noindent \noindent GINO: W~każdym razie, policjanci zapobiegają ogromnej liczbie przestępstw i~stawiają przed sądem sprawców tych
przestępstw, którym nie byli w~stanie zapobiec. 




 
\noindent GIORGIO: Nawet to nie jest prawdą. Wpływ policji na liczbę i~znaczenie przestępstw jest prawie zerowy. W~rzeczywistości,
niezależnie od tego, jak bardzo zreformowana zostanie organizacja magistratu, policji i~więzień, lub liczba policjantów
zmniejszy się lub zwiększy, podczas gdy warunki ekonomiczne i~moralne ludności pozostaną niezmienione, przestępczość
pozostanie mniej więcej stała. 

 
Z drugiej strony, wymaga to jedynie najmniejszej zmiany w~stosunkach między właścicielami a robotnikami, czyli zmiany
cen pszenicy i~innych żywotnie niezbędnych artykułów spożywczych, czy kryzysu pozbawiającego robotników pracy, czy
szerzenia się naszych idei, które otwierają ludziom nowe horyzonty, wywołując u nich uśmiech nowej nadziei, i~natychmiastowy wpływ na wzrost lub spadek liczba przestępstw zostanie odnotowana. 

 
Policja, to prawda, wysyła przestępców do więzienia, kiedy może ich złapać; ale to, ponieważ nie zapobiega nowym
wykroczeniom, jest złem dodanym do zła, dalszym niepotrzebnym cierpieniem zadawanym istotom ludzkim. 

 
A nawet jeśli policja odniesie sukces w~uniknięciu kilku przestępstw, to zdecydowanie nie wystarczy, aby zrekompensować
przestępstwa, które prowokuje, i~nękanie, któremu poddaje społeczeństwo. 

 
Sama funkcja, którą pełnią, budzi podejrzliwość wobec policji i~powoduje konflikt z~całym społeczeństwem; czyni ich
łowcami ludzkości; prowadzi ich do ambicji odkrywania ,,wielkich'' przypadków przestępczości i~tworzy w~nich szczególną
mentalność, co bardzo często prowadzi ich do rozwinięcia pewnych wyraźnie antyspołecznych instynktów. Nierzadko zdarza
się, że policjant, który powinien zapobiegać przestępstwom lub je wykrywać, zamiast tego je prowokuje lub wymyśla, aby
promować swoją karierę lub po prostu stać się ważnym i~potrzebnym. 




 
\noindent \noindent GINO: Ale wtedy sami policjanci byliby tacy sami jak przestępcy! Takie rzeczy zdarzają się sporadycznie, tym bardziej,
że policjanci nie zawsze rekrutują się z~najlepszej części społeczeństwa, ale generalnie\ldots 




 
\noindent GIORGIO: Generalnie środowisko ma nieubłagany wpływ, a profesjonalne zniekształcenia uderzają nawet w~tych, którzy
domagają się poprawy. 

 
Powiedz mi: co może być lub co może stać się z~moralnością tych, którzy są zobowiązani przez swoje pensje do
prześladowania, aresztowania, dręczenia kogokolwiek wskazanego przez przełożonych, nie martwiąc się, czy jest winny czy
niewinny, przestępca czy anioł? 




 
\noindent \noindent GINO: Tak\ldots ale\ldots 




 
\noindent GIORGIO: Pozwól, że powiem kilka słów o najważniejszej części pytania; innymi słowy o tzw. przestępstwach, które policja
zobowiązuje się powstrzymać lub im zapobiegać. 

 
Z pewnością wśród czynów, za które prawo karze, są takie, które są i~zawsze będą złymi czynami; ale są wyjątki, które
wynikają ze stanu bestialstwa i~rozpaczy, do którego doprowadza ludzi bieda. 

 
Ogólnie jednak karalne są czyny naruszające przywileje klasy wyższej i~te, które atakują rząd w~wykonywaniu jego władzy.
W ten sposób policja, skutecznie lub nie, służy ochronie nie całego społeczeństwa, ale klasy wyższej i~utrzymywaniu
ludzi w~uległości. 

 
Mówiłeś o złodziejach. Kto jest większym złodziejem niż właściciel, którzy bogaci się kradnąc produkty pracy
robotników? 

 
Mówiłeś o mordercach. Kto jest większym mordercą niż kapitaliści, którzy nie wyrzekając się przywileju kierowania i~życia bez pracy, są przyczyną strasznych niedostatków i~przedwczesnej śmierci milionów robotników, nie mówiąc już o
ciągłej rzezi dzieci? 

 
Ci złodzieje i~mordercy, o wiele bardziej winni i~o wiele bardziej niebezpieczni niż ci biedni ludzie, których popychają
do przestępstwa nędzne warunki, w~jakich się znajdują, nie są przedmiotem troski policji: wręcz przeciwnie!\ldots 




 
\noindent \noindent GINO: Krótko mówiąc, myślisz, że po dokonaniu rewolucji ludzkość stanie się niespodziewanie wieloma małymi aniołkami.
Wszyscy będą szanować prawa innych; wszyscy będą życzyć sobie nawzajem jak najlepiej i~pomagać sobie nawzajem; nie
będzie więcej nienawiści ani zazdrości\ldots raj ziemski, co za bzdury?! 




 
\noindent GIORGIO: Wcale nie. Nie wierzę, że przemiana moralna nadejdzie nagle, niespodziewanie. Oczywiście wielka, niezmierna
zmiana nastąpi przez prosty fakt, że zapewniony jest chleb i~zdobyta wolność; ale wszystkie złe namiętności, które
ucieleśniły się w~nas pod wpływem odwiecznego niewolnictwa i~walki między ludźmi, nie znikną od razu. Jeszcze długo
będą tacy, którzy będą odczuwać pokusę narzucania swojej woli przemocą innym, którzy będą chcieli wykorzystać
sprzyjające okoliczności do stworzenia sobie przywilejów, którzy zachowają niechęć do pracy zainspirowani warunkami
niewolnictwa, w~jakich są dziś zmuszani do pracy, i~tak dalej. 




 
\noindent \noindent GINO: Czyli nawet po rewolucji będziemy musieli bronić się przed przestępcami? 




 
\noindent GIORGIO: Bardzo prawdopodobne. Pod warunkiem że ci, których wtedy uważa się za przestępców, to nie ci, którzy raczej się
buntują niż umierają z~głodu, a tym bardziej ci, którzy atakują istniejącą organizację społeczeństwa i~starają się
zastąpić ją lepszą; ale ci, którzy wyrządziliby krzywdę wszystkim, ci, którzy naruszyliby integralność osobistą,
wolność i~dobro innych. 




 
\noindent \noindent GINO: W~porządku, więc zawsze będziesz potrzebować policji. 




 
\noindent GIORGIO: Ależ wcale nie. Byłoby naprawdę wielką głupotą chronić się przed kilkoma brutalami, kilkoma próżniakami i~niektórymi degeneratami, otwierając szkołę dla próżniactwa i~przemocy oraz tworząc grupę rzezimieszków, którzy nauczą
się traktować obywateli za więzienną przynętę i~którzy zrobią z~polowań swoje główne i~jedyne zajęcie. 




 
\noindent \noindent GINO: A więc co! 




 
\noindent GIORGIO: Cóż, będziemy się bronić. 




 
\noindent \noindent GINO: I~myślisz, że to możliwe? 




 
\noindent GIORGIO: Nie tylko uważam, że jest możliwe, że ludzie będą się bronić bez powierzania komukolwiek szczególnej funkcji
obrony społeczeństwa, ale jestem pewien, że jest to jedyna skuteczna metoda. 

 
Powiedz mi! Jeśli jutro przyjdzie do ciebie ktoś poszukiwany przez policję, wydasz go? 




 
\noindent \noindent GINO: Co, zwariowałeś? Nawet gdyby byli najgorszymi ze wszystkich morderców. Co ty mnie masz za policjanta?! 




 
\noindent GIORGIO: Ach! Ach! Zawód policjanta musi być straszny, jeśli ktokolwiek szanujący się uważa, że  się go
hańbi, podejmując go, nawet jeśli uważa, że jest pożyteczny i~potrzebny społeczeństwu. 

 
A teraz powiedz mi coś jeszcze. Gdybyś spotkał chorego z~chorobą zakaźną lub niebezpiecznego szaleńca, zabrałbyś go do
szpitala? 




 
\noindent \noindent GINO: Oczywiście. 




 
\noindent GIORGIO: Nawet siłą? 




 
\noindent \noindent GINO: Ale\ldots Musisz zrozumieć! Pozostawienie ich na wolności może zaszkodzić wielu ludziom! 




 
\noindent GIORGIO: Wyjaśnij mi teraz, dlaczego bardzo pilnujesz się, aby nie wydać mordercy, podczas gdy szaleńca lub chorego na
dżumę zabierasz do szpitala, jeśli to konieczne, nawet siłą? 




 
\noindent \noindent GINO: Cóż\ldots przede wszystkim uważam, że bycie policjantem jest odrażające, a opieka nad chorymi jest rzeczą honorową i~humanitarną. 




 
\noindent GIORGIO: Cóż, już widać, że pierwszym efektem działania policji jest sprawienie, że obywatele umywają ręce od obrony
społecznej, a właściwie stają po stronie tych, których policja słusznie lub niesłusznie prześladuje. 




 
\noindent \noindent GINO: Chodzi też o to, że kiedy zabieram kogoś do szpitala, wiem, że zostawiam go w~rękach lekarzy, którzy starają się
go wyleczyć, żeby mógł wyjść na wolność, kiedy już nie będzie zagrożeniem dla innych ludzie. W~każdym przypadku, nawet
nieuleczalnym, lekarze będą starali się ulżyć cierpieniu i~nigdy nie będą stosować surowszego traktowania, niż jest to
bezwzględnie konieczne. Gdyby lekarze nie wykonywali swoich obowiązków, społeczeństwo by ich do tego zmuszało, ponieważ
dobrze wiadomo, że ludzie są przetrzymywani w~szpitalach w~celu wyleczenia i~żeby się nie męczyć. 

 
Natomiast, jeśli ktoś oddaje kogoś w~ręce policji, policja z~ambicji stara się go skazać, mało troszcząc się o to, czy
jest winny, czy niewinny; następnie umieszcza go w~więzieniu, gdzie zamiast szukać ich poprawy poprzez troskliwą
opiekę, robią wszystko, aby cierpieli, uczynili bardziej zgorzkniałymi, a następnie uwalniają go jako jeszcze bardziej
niebezpiecznego wroga społeczeństwa, niż byli przed pójściem do więzienia. 

 
Można to jednak zmienić poprzez radykalną reformę. 




 
\noindent GIORGIO: Aby zreformować, mój drogi kolego, lub zniszczyć instytucję, pierwszą rzeczą nie jest tworzenie korporacji
zainteresowanej jej utrzymaniem. 

 
Policja (a to, co mówię o policji, odnosi się również do sędziów), wykonując swój zawód polegający na wysyłaniu ludzi do
więzienia i~biciu ich, gdy nadarzy się okazja, zawsze będzie uważała się za sprzeciwiającą się społeczeństwu. Wściekle
ścigają prawdziwego lub domniemanego przestępcę z~taką samą pasją, z~jaką myśliwy goni zwierzynę, ale jednocześnie w~interesie policji jest, aby przestępców było więcej, bo to oni są powodem ich istnienia, a im większa liczba i~szkodliwość przestępców, tym większa siła i~społeczne znaczenie policji! 

 
Aby przestępstwo było traktowane racjonalnie, aby szukać jego przyczyn i~rzeczywiście czynić wszystko, co możliwe, aby
je wyeliminować, konieczne jest powierzenie tego zadania tym, którzy są narażeni na przestępstwo i~ponoszą jego skutki,
czyli innymi słowy, całemu społeczeństwu, a nie tym, dla których istnienie przestępczości jest źródłem władzy i~zarobków. 




 
\noindent GINO: O! Możliwe, że masz rację. Do następnego razu. 










\chapter*{Szesnaście}



 
\noindent PIPPO [\textit{Kaleka wojenny}]: Mam dość! Proszę pozwólcie mi opowiedzieć wam, że jestem zdumiony, powiedziałbym
prawie oburzony, że chociaż macie najróżniejsze opinie, wydajecie się zgadzać w~ignorowaniu zasadniczej kwestii, jaką
jest ojczyzna, zapewnienia wielkości i~chwały naszym Włochom. 




 
Prospero, Cesare, Vincenzo, \begin{itshape}i wszyscy obecni oprócz  \end{itshape}Giorgio i~Luigi (młody socjalista), \textit{hałaśliwie zaprotestowali i} Ambrogio \begin{itshape} powiedział w~imieniu wszystkich \end{itshape}: W~tych dyskusjach nie mówiliśmy o Włoszech, tak jak nie mówiliśmy o
naszych matkach. Nie trzeba było mówić o tym, co już zostało zrozumiane, o tym, co jest lepsze od jakiejkolwiek opinii,
od jakiejkolwiek dyskusji. Proszę, Pippo, nie wątp w~nasz patriotyzm, nawet w~patriotyzm Giorgio. 




 
\noindent GIORGIO: Ale nie; w~mój patriotyzm z~pewnością można wątpić, bo nie jestem patriotą. 




 
\noindent PIPPO: Już się domyśliłem: jesteś z~tych, co krzyczą \begin{itshape}koniec z~Włochami \end{itshape} i~chciałbyś zobaczyć nasz kraj upokorzony, pokonany, zdominowany przez
cudzoziemców. 




 
\noindent GIORGIO: Ależ wcale. Są to zwykłe oszczerstwa, którymi nasi przeciwnicy próbują oszukać ludzi, aby uprzedzić ich do nas.
Nie wykluczam, że są ludzie, którzy w~dobrej wierze wierzą w~ten bełkot, ale jest to wynikiem ignorancji i~braku
zrozumienia. 

 
Nie chcemy dominacji wszelkiego rodzaju i~dlatego nie możemy chcieć, aby Włochy były zdominowane przez inne kraje, tak
jak nie chcemy, aby Włochy dominowały nad innymi. 

 
Cały świat uważamy za naszą ojczyznę, całą ludzkość za naszych braci i~siostry; dlatego dla nas byłoby to po prostu
absurdem chcieć szkodzić i~upokarzać kraj, w~którym żyjemy; w~którym mamy naszych bliskich, którego językiem mówimy
najlepiej, kraj, który daje nam najwięcej i~któremu dajemy najwięcej w~zakresie wymiany pracy, idei i~uczuć. 




 
\noindent AMBROGIO: Ale ten kraj to ojczyzna, którą nieustannie przeklinacie. 




 
\noindent GIORGIO: Nie przeklinamy naszej ojczyzny ani niczyjego kraju. Przeklinamy patriotyzm, który wy nazywacie patriotyzmem,
który jest arogancją narodową, czyli głoszeniem nienawiści do innych krajów, pretekstem do nastawiania ludzi przeciwko
ludziom w~śmiertelnych wojnach, aby służyć złowrogim interesom kapitalizmu i~nieumiarkowanym ambicjom władców i~drobnych polityków. 




 
\noindent VINCENZO: Spokojnie, spokojnie. 

 
Masz rację mówiąc o tzw \begin{itshape}patriotyzmie  \end{itshape}bardzo wielu kapitalistów i~bardzo wielu monarchistów, dla których miłość ojczyzny jest naprawdę pretekstem. I~tak jak ty gardzę i~brzydzę się
tymi, którzy nie ryzykują nic dla kraju i~bogacą się w~imię ojczyzny na pot i~krew robotników i~uczciwych ludzi ze
wszystkich klas. Ale są ludzie, którzy są prawdziwymi patriotami, którzy poświęcili i~są gotowi poświęcić wszystko,
swój majątek, wolność i~życie dla swojego kraju. 

 
Wiecie, że republikanami zawsze kierował najwyższy patriotyzm i~zawsze rzetelnie wywiązywali się ze swoich
obowiązków. 




 
\noindent GIORGIO: Zawsze podziwiam tych, którzy poświęcają się dla swoich idei, ale to nie przeszkadza mi dostrzegać, że ideały
republikanów i~szczerych patriotów, których z~pewnością można znaleźć we wszystkich partiach, stały się w~tym momencie
przestarzałe i~służą jedynie rządom i~kapitalistom jako sposób na maskowanie ich prawdziwych celów ideałami i~wpływanie
na nieświadome masy i~entuzjastyczną młodzież. 




 
\noindent VINCENZO: Co masz na myśli mówiąc nieaktualne?! Miłość ojczyzny jest naturalnym uczuciem ludzkiego serca i~nigdy się nie
zdezaktualizuje. 




 
\noindent GIORGIO: To, co nazywasz miłością do własnego kraju, jest przywiązaniem do tego kraju, z~którym łączy cię najsilniejszy
związek moralny i~który daje największą pewność materialnego dobrobytu; i~z pewnością jest to naturalne i~zawsze takie
pozostanie, przynajmniej do czasu, gdy cywilizacja rozwinie się do punktu, w~którym każdy człowiek będzie odnajdywał
\begin{itshape}de facto \end{itshape} swój kraj w~dowolnej części świata. Ale to nie ma nic
wspólnego z~tzw mitem ,,ojczyzny'', który każe uważać innych ludzi za gorszych, który każe pragnąć dominacji własnego
kraju nad innymi, co uniemożliwia docenienie pracy tzw. obcych, a który sprawia, że uważasz, że pracownicy mają więcej
wspólnego ze swoimi szefami i~policją w~swoim kraju niż z~pracownikami z~innych krajów, z~którymi mają wspólne
zainteresowania i~aspiracje. 

 
W końcu nasze międzynarodowe, kosmopolityczne uczucia wciąż się rozwijają, jako kontynuacja dokonanego już postępu.
Możesz czuć się bardziej przywiązany do swojej rodzinnej wioski lub regionu z~tysiąca powodów sentymentalnych i~materialnych, ale to nie znaczy, że jesteś zaściankowy lub przywiązany do swojego regionu: szczycisz się będąc Włochem,
i jeśli zajdzie taka konieczność, stawiałbyś ogólne interesy Włoch ponad interesami regionalnymi lub lokalnymi. Jeśli
uważacie, że poszerzenie pojęcia własnego kraju od gminy do narodu jest postępem, dlaczego poprzestać na tym i~nie
objąć całego świata ogólną miłością do rodzaju ludzkiego i~braterską współpracą między wszystkimi ludźmi? 

 
Dziś stosunki między krajami, wymiana surowców oraz produktów rolnych i~przemysłowych są już takie, że kraj, który
chciałby się odizolować od innych lub, co gorsza, wejść w~konflikt z~innymi, skazałby się na wycieńczoną egzystencję i~zupełną i~totalna porażka. Już teraz jest mnóstwo mężczyzn, którzy ze względu na swoje związki, rodzaj studiów i~pracy,
ze względu na swoją pozycję ekonomiczną uważają się za obywateli świata i~są nimi naprawdę. 

 
Poza tym, czy nie widzicie, że wszystko, co na świecie wielkie i~piękne, ma charakter globalny i~ponadnarodowy. Nauka
jest międzynarodowa, tak samo jak sztuka, tak samo jest z~religią, która wbrew swoim kłamstwom jest wielką demonstracją
duchowej aktywności ludzkości. Jak powiedziałby signor Ambrogio, prawa i~moralność są uniwersalne, ponieważ starają się
rozszerzyć swoje własne koncepcje na każdego człowieka. Każda nowa prawda odkryta w~jakiejkolwiek części świata, każdy
nowy wynalazek, każdy pomysłowy produkt ludzkiego mózgu jest lub powinien być użyteczny dla całej ludzkości. 

 
Aby powrócić do izolacji, do rywalizacji i~nienawiści między narodami, trwanie w~ciasnym i~mizantropijnym patriotyzmie
oznaczałoby stawianie się poza wielkimi nurtami postępu, które pchają ludzkość ku przyszłości pokoju i~braterstwa,
oznaczałoby to stawianie się na zewnątrz i~przeciw cywilizacji. 




 
\noindent CESARE: Zawsze mówisz o pokoju i~braterstwie; ale pozwól, że zadam ci praktyczne pytanie. Gdyby na przykład Niemcy lub
Francuzi przybyli do Mediolanu, Rzymu lub Neapolu, aby zniszczyć nasze pomniki sztuki i~mordować lub ciemiężyć naszych
rodaków, co byście zrobili? Byłbyś niewzruszony? 




 
\noindent GIORGIO: Co ty mówisz? z~pewnością byłbym bardzo przygnębiony i~zrobiłbym wszystko, co w~mojej mocy, aby temu zapobiec.
Ale zauważ to dobrze, byłbym równie zmartwiony i~gdybym mógł, zrobiłbym wszystko, aby uniemożliwić Włochom niszczenie,
uciskanie i~zabijanie w~Paryżu, Wiedniu, Berlinie\ldots lub w~Libii. 




 
\noindent CESARE: Naprawdę \begin{itshape}na równi \end{itshape} zakłopotany? 




 
\noindent GIORGIO: Być może nie w~praktyce. Czułbym się gorzej z~powodu złych uczynków popełnionych we Włoszech, ponieważ we
Włoszech mam więcej przyjaciół, znam Włochy lepiej, więc moje uczucia byłyby głębsze i~bardziej bezpośrednie. Ale to
nie znaczy, że wykroczenia popełnione w~Berlinie byłyby mniej złe niż te popełnione w~Mediolanie. 

 
To tak, jakby mieli zabić brata, przyjaciela. Z~pewnością cierpiałbym bardziej, niż gdyby zabili kogoś, kogo nie znam:
ale to nie znaczy, że zabicie nieznanej mi osoby jest mniej zbrodnicze niż zabicie przyjaciela. 




 
\noindent PIPPO: W~porządku. Ale co zrobiłeś, aby powstrzymać ewentualną inwazję Niemców na Mediolan? 




 
\noindent GIORGIO: Nic nie zrobiłem. Właściwie moi przyjaciele i~ja zrobiliśmy wszystko, co w~naszej mocy, aby uniknąć walki;
ponieważ nie byliśmy w~stanie zrobić tego, co byłoby pożyteczne i~konieczne. 




 
\noindent PIPPO: Co masz na myśli? 




 
\noindent GIORGIO: To oczywiste. Znaleźliśmy się w~sytuacji, w~której musieliśmy bronić interesów naszych szefów, naszych
ciemiężców, zabijając niektórych naszych braci, robotników z~innych krajów, pędzonych do rzeźni tak jak my, przez ich
szefów i~prześladowcy. I~odmówiliśmy bycia narzędziem tych, którzy są naszym prawdziwym wrogiem, czyli naszymi
szefami. 

 
Gdybyśmy, po pierwsze, potrafili uwolnić się od naszych wewnętrznych wrogów, to bylibyśmy w~stanie bronić naszego kraju,
a nie kraju kapitalistów. Mogliśmy podać braterską dłoń wysłanym przeciwko nam zagranicznym robotnikom, a gdyby nie
zrozumieli i~zechcieli nadal służyć swoim panom, sprzeciwiając się nam, bronilibyśmy się. 




 
\noindent AMBROGIO: Wy troszczycie się tylko o interesy robotników, o interesy waszej klasy, nie rozumiejąc, że naród jest ponad
interesami klasowymi. Istnieją pewne uczucia, pewne tradycje, pewne interesy, które jednoczą wszystkich ludzi tego
samego narodu, pomimo różnic w~ich warunkach i~wszystkich antagonizmów klasowych. 

 
I znowu, istnieje duma z~własnych korzeni. Czy nie jesteś dumny z~bycia Włochem, z~przynależności do kraju, który dał
światu cywilizację i~nawet dzisiaj, pomimo wszystko, jest w~czołówce postępu? 

 
Jak to się dzieje, że nie czujesz potrzeby obrony cywilizacji łacińskiej przed krzyżackim barbarzyństwem? 




 
\noindent GIORGIO: Proszę, nie rozmawiajmy o cywilizacji i~barbarzyństwie tego czy innego kraju. 

 
Mogłem od razu powiedzieć wam, że jeśli robotnicy nie są w~stanie docenić waszej ,,cywilizacji łacińskiej'', to wasza
wina, wina burżuazji, która odebrała robotnikom środki do samokształcenia. Jak możesz oczekiwać, że ktoś będzie
pasjonatem czegoś, co zabraniałeś mu poznać? 

 
Przestań nas wprowadzać w~błąd. Czy chcesz, żebyśmy uwierzyli, że Niemcy są bardziej barbarzyńscy niż ktokolwiek inny,
kiedy sam przez lata podziwiałeś wszystko, co pochodzi z~Niemiec? Gdyby jutro zmieniły się warunki polityczne i~inaczej
ustawiły się kapitalistyczne interesy, po raz kolejny powiedziałbyś, że Niemcy stoją na czele cywilizacji, a Francuzi
czy Anglicy to barbarzyńcy. 

 
Co to znaczy? Jeśli czyjś kraj okaże się bardziej zaawansowany niż inny, ma obowiązek szerzyć swoją cywilizację, pomagać
swoim rodakom, którzy są zacofani i~nie czerpią korzyści z~jego wyższości, aby uciskać i~wyzyskiwać\ldots ponieważ każde
nadużycie władzy prowadzi do korupcji i~dekadencji. 




 
\noindent AMBROGIO: Ale w~każdym razie przynajmniej szanujesz solidarność narodową, która musi przewyższać wszelką konkurencję
klasową. 




 
\noindent GIORGIO: Rozumiem. To właśnie ten pozór narodowej solidarności interesuje was szczególnie i~z tym właśnie walczymy.
Solidarność narodowa oznacza solidarność między kapitalistami a robotnikami, między ciemiężycielami a ciemiężonymi,
innymi słowy przyzwolenie ciemiężonych na ich stan podporządkowania. 

 
Interesy robotników są sprzeczne z~interesami pracodawców, a kiedy w~szczególnych okolicznościach znajdują się
tymczasowo w~zgodzie, staramy się zrobić z~nich antagonistów, biorąc pod uwagę, że emancypacja człowieka i~cały
przyszły postęp zależą od walki między robotnikami a właścicielami, musi to doprowadzić do całkowitego zniknięcia
wyzysku i~ucisku jednej osoby przez drugą. 

 
Nadal próbujecie oszukać robotników kłamstwami nacjonalizmu: ale na próżno. Robotnicy zrozumieli już, że robotnicy
wszystkich krajów są ich towarzyszami, a wszyscy kapitaliści i~wszystkie rządy, krajowe czy zagraniczne, są ich
wrogami. 

 
I tym miłym akcentem życzę wam dobrego wieczoru. Wiem, że nie przekonałem ani sędziów, ani właścicieli, którzy mnie
posłuchali. Ale może nie na próżno opowiadałem się za Pippo, Vincenzo i~Luigim, którzy są proletariuszami jak ja. 


\chapter*{Siedemnaście}



 
\noindent LUIGI [\textit{socjalista}]: Skoro wszyscy tutaj wyrazili swoją opinię, pozwólcie, że ja wyrażę swoją? 

 
To tylko niektóre z~moich własnych pomysłów i~nie chcę narażać się na połączoną nietolerancję burżuazji i~anarchistów. 




 
\noindent GIORGIO: Jestem zdumiony, że tak mówisz. 

 
Ponieważ obaj jesteśmy robotnikami, możemy i~musimy uważać się za przyjaciół i~towarzyszy, ale wydaje się, że sądzisz,
że anarchiści są wrogami socjalistów. Wręcz przeciwnie, jesteśmy ich przyjaciółmi, współpracownikami. 

 
Nawet jeśli wielu wybitnych socjalistów próbowało i~nadal próbuje przeciwstawić socjalizm anarchizmowi, prawda jest
taka, że  jeśli socjalizm oznacza społeczeństwo lub aspiracje do społeczeństwa, w~którym ludzie żyją we
wspólnocie, w~którym dobro wszystkich jest warunkiem dobrobytu każdego, w~którym nikt nie jest niewolnikiem ani
wyzyskiwany i~każdy ma środki do maksymalnego rozwoju i~korzystania w~pokoju ze wszystkich dobrodziejstw cywilizacji i~pracy społecznej, nie tylko jesteśmy socjalistami, ale mamy prawo uważać się za najbardziej radykalnych i~konsekwentnych socjalistów. 

 
W końcu nawet signor Ambrogio, który wysłał tak wielu z~nas do więzienia, wie, że jako pierwsi wprowadziliśmy,
wyjaśniliśmy i~propagowaliśmy socjalizm; i~jeśli stopniowo porzuciliśmy tę nazwę i~nazywaliśmy siebie po prostu
anarchistami, to dlatego, że obok nas powstała inna szkoła, dyktatorska i~parlamentarna, której udało się zwyciężyć i~uczynić socjalizm czymś tak hybrydycznym i~przystosowawczym, że niemożliwe było pogodzić z~naszymi ideałami i~metodami
doktrynę, która była odrażająca dla naszej natury. 




 
\noindent LUIGI: W~rzeczywistości zrozumiałem twoje argumenty i~z pewnością zgadzamy się w~wielu sprawach, zwłaszcza w~krytyce
kapitalizmu. 

 
Ale nie we wszystkim się zgadzamy, po pierwsze dlatego, że anarchiści wierzą tylko w~rewolucję i~wyrzekają się bardziej
cywilizowanych środków walki, które zastąpiły te gwałtowne metody, które być może kiedyś były konieczne, a po drugie,
ponieważ nawet gdybyśmy finalizowali gwałtowną rewolucją, konieczne byłoby wprowadzenie do władzy nowego rządu, aby
działał w~sposób uporządkowany i~nie pozostawiał wszystkiego arbitralnym działaniom i~wściekłości mas. 




 
\noindent GIORGIO: Cóż, przedyskutujmy to dalej. Czy naprawdę wierzycie, że możliwe jest radykalne przekształcenie społeczeństwa,
zniesienie przywilejów, obalenie rządu, wywłaszczenie burżuazji bez użycia siły? 

 
Mam nadzieję, że nie łudzicie się, że właściciele i~władcy poddadzą się bez oporu, bez użycia sił, którymi dysponują, i~da się ich jakoś nakłonić do odgrywania roli ofiarnych ofiar. W~przeciwnym razie zapytaj tych tutaj panów, którzy,
gdyby mogli, pozbyliby się ciebie i~mnie z~wielką przyjemnością i~bardzo szybko. 




 
\noindent LUIGI: Nie, nie mam takich złudzeń. 

 
Dzisiaj robotnicy stanowią zdecydowaną większość elektoratu i~mają prawo głosu w~sprawach administracyjnych i~w~wyborach
politycznych, wydaje mi się, że gdyby byli świadomi i~chętni, mogliby bez większego wysiłku umieścić u władzy ludzi,
którym mogliby zaufać, socjalistów i, jeśli kto chce, nawet niektórych anarchistów, którzy mogliby stanowić dobre
prawa, znacjonalizować ziemię i~warsztaty oraz wprowadzić socjalizm. 




 
\noindent GIORGIO: Oczywiście, gdyby robotnicy byli świadomi i~zaangażowani! 

 
Ale gdyby byli wystarczająco rozwinięci, aby móc zrozumieć przyczyny swoich problemów i~lekarstwa na nie, gdyby naprawdę
byli zdeterminowani, by się wyzwolić, wówczas rewolucja mogłaby zostać przeprowadzona z~niewielką lub żadną przemocą, a
sami robotnicy mogliby robić, co chcą i~nie trzeba by było posyłać do parlamentu i~do rządu ludzi, którzy, nawet jeśli
nie dali się upić i~zepsuć pokusami władzy, jak to niestety bywa, nie są w~stanie zaspokajać potrzeb społecznych i~robić to, czego oczekują od nich wyborcy. 

 
Jednak niestety pracownicy, a przynajmniej zdecydowana większość z~nich, są nieświadomi lub oddani; żyją w~warunkach,
które nie dopuszczają możliwości wyzwolenia się moralnego, chyba że najpierw nastąpi poprawa ich sytuacji materialnej.
Tak więc transformacja społeczeństwa musi nastąpić dzięki inicjatywie i~pracy tych grup mniejszościowych, które dzięki
szczęśliwym okolicznościom były w~stanie wznieść się ponad wspólny poziom, mniejszości liczbowych, które ostatecznie
stają się dominującą siłą zdolną do pociągnięcia ich za sobą zacofane masy. 

 
Przyjrzyjcie się faktom, a zobaczycie wkrótce, że właśnie z~powodu warunków moralnych i~materialnych, w~jakich znajduje
się proletariat, burżuazja i~rząd zawsze uzyskują od parlamentu to, co im odpowiada. Dlatego przyznają powszechne prawo
wyborcze i~pozwalają mu funkcjonować. Gdyby dostrzegli jakiekolwiek niebezpieczeństwo legalnego wywłaszczenia, jako
pierwsi odstąpiliby od legalności i~pogwałcili to, co nazywają wolą ludu. Już tak robią przy każdej okazji, gdy prawa
przez pomyłkę działają przeciwko nim. 




 
\noindent LUIGI: Mówisz tak, ale w~międzyczasie widzimy, że liczba socjalistycznych posłów stale rośnie. Pewnego dnia będą
większością i\ldots 




 
\noindent GIORGIO: Ale czy nie widzicie, że kiedy socjaliści wchodzą do parlamentu, od razu zostają oswojeni i~z zagrożenia stają
się kolaborantami i~zwolennikami panującego porządku? w~końcu wysyłając socjalistów do parlamentu, oddajemy burżuazji
przysługę, ponieważ najbardziej aktywni, zdolni i~popularni ludzie są usuwani z~serc mas i~transportowani do środowiska
burżuazyjnego. 

 
Ponadto, jak już panu powiedziałem, kiedy socjalistyczni posłowie do parlamentu naprawdę staną się zagrożeniem, rząd
wypędzi ich z~parlamentu bagnetem i~zniesie powszechne prawo wyborcze. 




 
\noindent LUIGI: Może ci się tak wydawać, ponieważ zawsze postrzegasz rzeczy w~kategoriach świata w~skrajnym kryzysie. 

 
Przeciwieństwo jest prawdą. Świat porusza się stopniowo dzięki stopniowej ewolucji. 

 
Dla proletariatu jest to konieczne, by przygotować się do przejęcia władzy od burżuazji, kształcąc się, organizując się,
wysyłając swoich przedstawicieli do organów stanowiących i~tworzących prawa; a kiedy dojrzeje, weźmie wszystko w~swoje
ręce i~powstanie nowe społeczeństwo, do którego aspirujemy. 

 
We wszystkich krajach cywilizowanych wzrasta liczba posłów socjalistycznych i~naturalnie rośnie też ich poparcie wśród
mas. 

 
Pewnego dnia z~pewnością staną się większością, a jeśli wtedy burżuazja i~jej rząd nie ustąpią pokojowo i~nie będą
próbowały przemocą stłumić woli ludu, to na przemoc odpowiemy przemocą. 

 
Konieczne jest poświęcenie czasu. Próba narzucenia praw natury i~historii jest bezużyteczna i~szkodliwa. 




 
\noindent GIORGIO: Drogi Luigi, prawa natury nie potrzebują obrońców: same budzą szacunek. Ludzie mozolnie je odkrywają i~wykorzystują swoje odkrycie do czynienia dobra lub zła; ale wystrzegaj się przyjmowania jako praw naturalnych faktów
społecznych, które zainteresowane strony (w naszym przypadku ekonomiści i~socjologowie broniący burżuazji) opisują jako
takie. 

 
Jeśli chodzi o ,,prawa historii'', są one formułowane po historii. Przede wszystkim twórzmy historię. 

 
Świat porusza się wolno lub szybko, idzie do przodu lub do tyłu, w~wyniku nieokreślonej liczby czynników naturalnych i~ludzkich, i~błędem jest wierzyć w~ciągłą ewolucję, która zawsze porusza się w~tym samym kierunku. 

 
Obecnie z~pewnością prawdą jest, że społeczeństwo znajduje się w~ciągłej, powolnej ewolucji; ale ewolucja w~istocie
oznacza zmianę, a jeśli niektóre zmiany prowadzą nas we właściwym kierunku, które sprzyjają wyniesieniu ludzkości ku
nadrzędnemu ideałowi wspólnoty i~wolności, to inne zamiast tego wzmacniają istniejące instytucje lub cofają i~anulują
postęp już osiągnięty. 

 
Podczas gdy ludzie pozostają w~opozycji do siebie, żadne zyski nie są pewne, żaden postęp w~organizacji społecznej nie
może być uznany za ostatecznie osiągnięty. 

 
Musimy wykorzystać i~wspierać wszystkie elementy postępu i~walki, utrudniać i~próbować zneutralizować regresywne i~konserwatywne siły. 

 
Dziś los ludzkości zależy od walki między robotnikami a wyzyskiwaczami i~od pojednania między dwiema wrogimi klasami, od
współpracy między kapitalistami i~robotnikami, między rządem a ludem, prowadzonej z~zamiarem lub pod pretekstem
stonowania sporów społecznych, służy jedynie faworyzowaniu klasy ciemiężców, wzmocnieniu chwiejnych instytucji i, co
gorsza, oddzieleniu od mas najbardziej rozwiniętych elementów proletariackich i~przekształceniu ich w~nową klasę
uprzywilejowaną, mającą interes wspólny z~baronami przemysł, finanse i~politykę, w~utrzymywaniu ogromnej większości
ludzi w~stanie niższości i~podporządkowania. 

 
Mówisz o ewolucji i~wydajesz się myśleć, że koniecznie i~nieuchronnie, czy ludzie tego chcą, czy nie, ludzkość dotrze do
socjalizmu, czyli społeczeństwa stworzonego dla równego interesu wszystkich, w~którym środki produkcji należą do
wszystkich, w~którym każdy będzie robotnikiem, w~którym każdy będzie korzystał na równych prawach ze wszystkich
dobrodziejstw cywilizacji. 

 
Ale tak nie jest prawdziwe. Socjalizm nadejdzie, jeśli ludzie tego zechcą i~zrobią wszystko, co konieczne, aby go
osiągnąć. W~przeciwnym razie możliwe jest, że zamiast socjalizmu powstałaby sytuacja społeczna, w~której różnice między
ludźmi byłyby większe i~trwalsze, w~której ludzkość zostałaby podzielona na dwie różne rasy, szlachciców i~służących, z~klasą pośrednią, która służyć zapewnieniu, poprzez połączenie inteligencji i~brutalnej siły, dominacji jednego nad
drugim, albo po prostu może istnieć kontynuacja obecnego stanu ciągłej walki, przemian ulepszeń i~pogorszenia, kryzysów
i okresowych wojen. 

 
Właściwie powiedziałbym, że gdybyśmy zostawili rzeczy ich naturalnemu biegowi, ewolucja prawdopodobnie potoczyłaby się w~kierunku przeciwnym do tego, którego pragniemy, zmierzałaby w~kierunku konsolidacji przywilejów, w~kierunku stabilnej
równowagi ustanowionej na korzyść teraźniejszości władców, bo to naturalne, że siła należy do silnych, i~kto rozpoczyna
walkę z~pewną przewagą nad przeciwnikiem, zawsze uzyska większą przewagę w~toku zmagań. 




 
\noindent LUIGI: Być może masz rację; właśnie dlatego musimy wykorzystać wszystkie dostępne nam środki: edukację, organizację i~walkę polityczną\ldots 




 
\noindent GIORGIO: Wszystkie środki, tak, ale wszystkie środki, które prowadzą do naszego celu. 

 
Edukacja, na pewno. Jest to pierwsza rzecz, która jest potrzebna, ponieważ jeśli nie oddziałujemy na umysły jednostek,
jeśli nie budzimy ich sumień, jeśli nie pobudzamy zmysłów, jeśli nie pobudzimy ich woli, postęp nie będzie możliwy. A
przez wykształcenie mam na myśli nie tyle naukę książkową, choć też potrzebną, ale mało dostępną dla proletariuszy,
raczej wykształcenie, które zdobywa się poprzez świadomy kontakt ze społeczeństwem, propagandę, dyskusje,
zainteresowanie sprawami publicznymi, udział w~walce o doskonalenie swoje i~innych. 

 
Ta edukacja jednostki jest konieczna i~wystarczyłaby do przekształcenia świata, gdyby można było ją rozszerzyć na
wszystkich. 

 
Jednak niestety nie jest to możliwe. Ludzie są pod wpływami, zdominowani, można powiedzieć, kształtowani przez
środowisko, w~którym żyją; a kiedy środowisko nie jest odpowiednie, postęp można osiągnąć tylko poprzez walkę z~nim. W~danym momencie tylko ograniczona liczba jednostek jest zdolna, czy to dzięki odziedziczonym zdolnościom, czy też dzięki
szczególnie sprzyjającym okolicznościom, wznieść się ponad środowisko, zareagować przeciwko niemu i~przyczynić się do
jego przemiany. 

 
Dlatego to świadoma mniejszość musi przełamać lody i~gwałtownie zmienić zewnętrzne okoliczności. 

 
Organizacja: Świetna i~potrzebna rzecz, pod warunkiem, że służy do walki z~bossami, a nie do dogadywania się z~nimi. 

 
Walka polityczna: Oczywiście, pod warunkiem, że mamy na myśli walkę z~rządem, a nie współpracę z~rządem. 

 
Zwróć uwagę. Jeśli chcesz ulepszyć system kapitalistyczny i~sprawić, by był znośny, a co za tym idzie utrwalić go,
wówczas pewne dostosowania, pewne zakresy współpracy mogą być do zaakceptowania; ale jeśli naprawdę chcesz obalić
system, musisz wyraźnie postawić się na zewnątrz i~przeciwko samemu systemowi. 

 
A ponieważ rewolucja jest konieczna i~skoro jakkolwiek by na to nie spojrzeć, problem zostanie rozwiązany tylko przez
rewolucję, czy nie sądzisz, że powinniśmy od teraz przygotowywać się duchowo i~materialnie, zamiast oszukiwać masy i~dawać im nadzieję? móc wyzwolić się bez ofiar i~krwawych walk. 




 
\noindent LUIGI: W~porządku. Załóżmy, że masz rację i~że rewolucja jest nieunikniona. Jest wielu socjalistów, którzy mówią to
samo. Ale zawsze będzie konieczne ustanowienie nowego rządu do kierowania i~organizowania rewolucji. 




 
\noindent GIORGIO: Dlaczego? Jeśli wśród mas nie będzie wystarczającej liczby rewolucjonistów, pracowników fizycznych i~niefizycznych, zdolnych do zaspokojenia potrzeb walki i~życia, rewolucja nie zostanie dokonana, a jeśli zostanie
dokonana, nie zatriumfuje. A jeśli istnieje wystarczająca liczba, do czego służy rząd, jeśli nie do sparaliżowania
powszechnej inicjatywy iw istocie do zdławienia samej rewolucji. 

 
Co właściwie może zrobić rząd parlamentarny lub dyktatorski? 

 
Musi przede wszystkim obmyślać i~zapewnić sobie własne istnienie jako rząd, innymi słowy stworzyć siłę zbrojną, aby
bronić się przed swoimi przeciwnikami i~narzucać własną wolę krnąbrnym; wtedy musiałby się dokształcać, studiować,
próbować godzić sprzeczne wole i~interesy, a więc ustanawiać prawa\ldots które najprawdopodobniej nikomu się nie
spodobają. 

 
Tymczasem trzeba żyć dalej. Każda nieruchomość \begin{itshape} \end{itshape}\textit{de
facto} przeszłaby w~ręce robotników, a następnie, ponieważ konieczne jest zaspokojenie codziennych potrzeb, ci sami
robotnicy musieliby rozwiązywać problemy życia codziennego bez czekania na decyzje rządzących, ci ostatni więc\ldots mogą
tylko zadeklarować swoją bezużyteczność jako władcy i~wtapiają się w~tłum jako robotnicy. 

 
Albo własność pozostałaby w~rękach właścicieli, wtedy oni, trzymając i~rozporządzając bogactwami według własnego
uznania, pozostaliby prawdziwymi arbitrami życia społecznego,i upewniliby się, że nowy rząd złożony z~socjalistów (nie
anarchistów, ponieważ anarchiści nie chcą rządzić ani być rządzonymi) albo podporządkuje się życzeniom burżuazji, albo
zostanie szybko zmieciony. 

 
Nie będę się nad tym rozwodzić, bo muszę iść i~nie wiem, kiedy wrócę. Minie trochę czasu, zanim się zobaczymy. 

 
Pomyśl o tym, co powiedziałem, mam nadzieję, że kiedy wrócę, znajdę nowego towarzysza. 

 
Do widzenia wszystkim. 


\tableofcontents
\end{document}
